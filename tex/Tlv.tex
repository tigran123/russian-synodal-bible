\bibbookdescr{Tlv}{
  inline={Завещание Левия,\\третьего сына Иакова и Лии\fns{В греч. тексте $+$ ``о священстве и о гордыне''.}},
  toc={Завещание Левия},
  bookmark={Завещание Левия},
  header={Завещание Левия},
  abbr={Лви}
}
\vs Tlv 1:1
Список слов Левия, которые говорил он сыновьям своим обо всём,
что они совершат и что случится с ними вплоть до дней Суда.
\vs Tlv 1:2
Он был здоров, когда призвал их к себе;
открылось же ему, что должен он вскоре умереть.
И когда собрались они, сказал к ним:

\vs Tlv 2:1
Я, Левий, родился в Хевроне и пришёл с отцом моим в Сиким.
\vs Tlv 2:2
Не было мне ещё двадцати лет,
когда сотворил я возмездие Еммору вместе с Симеоном за сестру нашу Дину.
\vs Tlv 2:3
И когда я пас стадо в Авелмехоле,
дух познания Господа сошел на меня,
и узрел я всех людей, уклонившихся от пути своего,
и грех выстроил себе дом на стенах, а неправедность восседала на башнях.
\vs Tlv 2:4
И был я в скорби о роде сынов человеческих и молил Господа,
чтобы спас меня.
\vs Tlv 2:5
Тогда снизошел на меня сон, и увидел я гору высокую и сам был на ней.

\vs Tlv 2:6
И вот, разверзлись небеса, и ангел Господень сказал мне:
Левий, Левий, войди!
\vs Tlv 2:7
И взошёл я на первое небо и узрел там великую воду висящую.
\vs Tlv 2:8
И ещё увидел я второе небо, много более светлое и сияющее,
высота же его была бесконечной.
\vs Tlv 2:9
И сказал я ангелу: что это такое?
И отвечал мне ангел: не удивляйся тому, ибо иное небо узришь,
более светлое и несравненное.
\vs Tlv 2:10
И поднявшись туда, встанешь ты рядом с Господом, и слугой ему будешь,
и тайны его возвестишь людям, и о грядущем избавлении Израиля возгласишь.
\vs Tlv 2:11
И через тебя и через Иуду явится Господь людям,
чтобы спасти собою весь род человеческий.
\vs Tlv 2:12
И жизнь твоя~--- удел Господа,
и будет он тебе полем и виноградником, и плодом, и золотом, и серебром.

\vs Tlv 3:1
Так услышь о показанных тебе небесах.
Нижнее оттого мрачно на вид, что зрит оно нечестия людские.
\vs Tlv 3:2
И имеет оно. огонь, снег и лед,
уготовленные на день Суда Божией справедливостью.
В нём~---  все духи воздаяний для возмездия людям.
\vs Tlv 3:3
На втором же небе~--- силы войск, построенных на день Суда,
дабы воздать духам соблазна и Велиара, а на них~--- святые.
\vs Tlv 3:4
В высшем же из всех пребывает великая слава,
превосходящая всякую святость.
\vs Tlv 3:5
В следующем же небе~--- архангелы,
служащие Господу и умилостивляющие его
ко всякому неведению праведных.
\vs Tlv 3:6
Подносят они Господу ароматы благоуханные,
жертву мысленную и незапятнанную кровью.
\vs Tlv 3:7
В том же небе, что за ним,~--- ангелы,
несущие молитвы ангелам о лице Божием.
\vs Tlv 3:8
В следующем за ним~--- престолы и власти,
коими хвалебная песнь Богу воспевается.

\vs Tlv 3:9
Когда же смотрит на нас Господь,
все мы дрожим, а небо и земля
и бездна от лица величия его сотрясаются.
\vs Tlv 3:10
Сыны же человеческие, не чувствующие того,
согрешают и гневят Всевышнего.
\vs Tlv 4:1
Познай же ныне, что сотворит Господь Суд над сынами человеческими.
Когда скалы рухнут, и солнце погаснет, и воды высохнут,
и огонь затаится, и всякое творение смутится,
и незримые духи истощатся, и Ад лишится защиты своей
[от страдания Всевышнего],
тогда люди утратят веру и будут упорствовать в неправедности своей,
и за то будут судимы и примут кару.

\vs Tlv 4:2
И услышал Всевышний молитву твою, да избавит тебя от неправедности
и сделает сыном своим, и рабом, и слугою пред лицом его.
\vs Tlv 4:3
Светом знания просияешь ты в Иакове,
и будешь как солнце всему семени Израиля.
\vs Tlv 4:4
И дастся тебе благословение и всему семени твоему дотоле,
пока не посетит Господь все народы по милосердию своему,
во веки веков.
[Но только сыновья твои возложат руки на него, дабы распять его.]

\vs Tlv 4:5
И для того даны тебе совет и знание,
чтобы наставил ты сыновей своих в этом.
\vs Tlv 4:6
Ибо благословляющие тебя благословенны будут,
а проклинающие тебя погибнут.

\vs Tlv 5:1
И вслед за тем открыл мне ангел врата небесные,
и увидел я Святого Всевышнего, восседающего на престоле.
\vs Tlv 5:2
И сказал он мне: Левий, тебе дал я благословение на священство,
доколе не приду и не поселюсь среди Израиля.
\vs Tlv 5:3
И тогда свёл меня ангел на землю и дал мне оружие и меч и сказал мне:
сотвори месть Сихему за Дину, сестру твою, и я буду с тобой,
ибо Господь послал меня.
\vs Tlv 5:4
И погубил я в то время сынов Еммора, как написано на скрижалях отцов.
\vs Tlv 5:5
И сказал ему: прошу тебя, господи, научи меня имени твоему,
дабы призывать мне его в день скорби.
\vs Tlv 5:6
И отвечал он: я ангел, просящий за народ Израилев,
да знает, что не сокрушится он.
\vs Tlv 5:7
И я, проснувшись, благословил Всевышнего.

\vs Tlv 6:1
И тогда пошел я к отцу моему, обрел бронзовый щит, отчего и имя
горы~--- Щит, что близ Гевала одесную Авимы.
\vs Tlv 6:2
И сохранил я слова те в сердце моём.
\vs Tlv 6:3
И совещался я с отцом моим и Рувимом, дабы сказать сынам Еммора,
чтобы приняли они обрезание, ибо пылал я рвением из-за мерзости,
которую сотворили они над сестрою моей.
\vs Tlv 6:4
И я убил первым Сихема, а Симеон~--- Еммора.
\vs Tlv 6:5
А вслед за тем пришли братья мои и перебили город тот остриём меча.
\vs Tlv 6:6
И услышал о том отец мой и, разгневавшись, огорчился,
что приняли они обрезание и умерли,
и в благословениях своих обошел нас.

\vs Tlv 6:7
В том согрешили мы, что сотворили это против воли его, а он болен
был в тот день.
\vs Tlv 6:8
Но я видел, что воля Божия была во зло Сикимам,
так как они хотели и Сарре и Ревекке сделать то,
что сделали Дине, сестре нашей,
но воспрепятствовал им Господь.
\vs Tlv 6:9
И преследовали они Авраама, отца нашего, бывшего чужеземцем,
и изнуряли скот беременный,
и Евлаю, родившуюся в доме Авраама, жестоко оскорбляли.
\vs Tlv 6:10
И так делали они всем чужеземцам,
силою похищая жен их и принуждая их.
\vs Tlv 6:11
И настиг их, наконец, гнев Божий.

\vs Tlv 7:1
И сказал я отцу моему Иакову:
тобою уничтожит Господь Хананеев
и даст землю их тебе и семени твоему после тебя.
\vs Tlv 7:2
Отныне назовутся Сикимы городом глупцов.
Ибо как смеются над глупцами, так посмеёмся и мы над ними.
\vs Tlv 7:3
Безумие сотворили они в Израиле, осквернив сестру мою.
И, встав, пошли мы в Вефиль.

\vs Tlv 8:1
И снова узрел я видение, подобное прежнему,
после того как были мы здесь 70 дней.
\vs Tlv 8:2
И узрел я семерых мужей в белых одеждах, говорящих мне:
восстав, облачись в одеяния священства,
и венец праведности,
и наперсник знания,
и подир правды,
и дощечку веры,
и митру главы,
и ефод пророчества.
\vs Tlv 8:3
И каждый из них нёс нечто, вручал мне и говорил мне:
отныне стань священником, и ты, и всё семя твое.
\vs Tlv 8:4
И первый помазал меня елеем святым и дал мне жезл.
\vs Tlv 8:5
А второй омыл меня водою чистой, и дал мне вкусить хлеба и вина,
и облачил меня в одеяние святое и славное.
\vs Tlv 8:6
Третий же облачил меня в виссон, подобный ефоду.
\vs Tlv 8:7
Четвёртый же надел на меня пояс, подобный порфире.
\vs Tlv 8:8
Пятый же дал мне ветвь тучной оливы.
\vs Tlv 8:9
Шестой надел мне на голову венец.
\vs Tlv 8:10
Седьмой надел мне диадему священства и наполнил руки мои фимиамом,
дабы служил я священником Господу Богу.
\vs Tlv 8:11
И говорят мне: Левий, разделится семя твоё на три чина в знак славы
Господа грядущего.
\vs Tlv 8:12
И будет первый жребий велик, и над ним не явится другого.
\vs Tlv 8:13
Второй будет жребий священства.
\vs Tlv 8:14
Третьему наречено будет новое имя, ибо восстанет
царь от Иуды и сотворит новое священство
по образу народов для всех народов.
\vs Tlv 8:15
И обретёт любовь явление его, ибо он будет
пророком Всевышнего от семени Авраама, отца вашего.
\vs Tlv 8:16
И всё желанное в Израиле твоё будет и семени твоего, и семя твоё
вкушать будет всё прекрасное видом, и трапезу Господа разделит.
\vs Tlv 8:17
И будут из них священники и судьи, и книжники, и на устах у них святое будет.
\vs Tlv 8:18
И очнувшись от сна, понял я, что этот сон подобен первому.
\vs Tlv 8:19
И скрыл это в сердце моём, и не возвестил о том ни одному человеку на земле.

\vs Tlv 9:1
Спустя же два дня пришли я, Иуда и отец наш Иаков к Исааку, праотцу нашему.
\vs Tlv 9:2
И благословил меня отец отца моего по видениям, которые видел я.
И не пожелал он отправиться с нами в Вефиль.
\vs Tlv 9:3
Когда же пришли мы в Вефиль, увидел отец мой Иаков видение обо мне,
что буду я у них священником.
\vs Tlv 9:4
И восстав наутро, принёс через меня Господу десятину от всего.
\vs Tlv 9:5

И так пришли мы в Хеврон, чтобы пребывать там.
\vs Tlv 9:6
И постоянно призывал меня Исаак, дабы наставлять меня в законе Господа, как и
явил мне ангел.
\vs Tlv 9:7
И учил он меня закону священства, жертвоприношений, всесожжении,
первенцев от плодов, жертв доброхотных и искупительных.
\vs Tlv 9:8
И каждый день наставлял он меня, и занят был со мною, и говорил мне:
\vs Tlv 9:9
удерживай себя от духа блуда, ибо он продолжителен и осквернит
святое через семя твое.
\vs Tlv 9:10
Потому возьми себе жену, ещё будучи молод,
чтобы не было на ней позора и скверны,
и не из рода иноплеменных народов.
\vs Tlv 9:11
И прежде чем войти в святое место, соверши омовение;
и когда приносишь жертву, омойся;
и закончив жертвоприношение, также омойся.
\vs Tlv 9:12
И 12 деревьев, имеющих листья, принеси Господу,
как учил и меня Авраам.
\vs Tlv 9:13
И от всякого животного чистого и пернатого принеси жертву Господу.
\vs Tlv 9:14
И от всех первых плодов и вина принеси первины в жертву Господу Богу.
И осоли всякую жертву солью.

\vs Tlv 10:1
Ныне, сохраните то, что завещаю вам, дети,
ибо услышанное мною от отцов наших возвестил вам.
\vs Tlv 10:2
И вот, неповинен я в нечестии вашем и в преступлениях,
которые совершите вы в конце веков [против Спасителя мира Христа],
соблазняя Израиль и навлекая на него беды всякие от Бога.
\vs Tlv 10:3
И сотворите вы беззакония в Израиле, так что не вынесет
Иерусалим злых дел ваших,
но порвётся завеса в Храме и не скроет непристойности вашей.
\vs Tlv 10:4
И рассеетесь вы пленниками среди народов и будет
там позор и проклятие на вас.
\vs Tlv 10:5
Ибо дом, который изберёт Господь,
Иерусалимом наречётся, как сказано в книге Еноха праведного.

\vs Tlv 11:1
Когда же я взял себе жену, было мне 28 лет;
ей было имя Мелха. 
\vs Tlv 11:2
И зачала она, и родила сына, и нарекли ему имя Гирсон,
ведь были мы в чужой земле.
\vs Tlv 11:3
И увидел я, что не быть ему среди первых.
\vs Tlv 11:4
Кааф же родился в 35-ый год жизни моей,
и было то при восходе солнца.
\vs Tlv 11:5
И узрел я в видении:
стоял он в вышних посреди собрания.
\vs Tlv 11:6
Оттого нарек я ему имя Кааф,
[что значит начало великих дел и наставление].
\vs Tlv 11:7
И 3-го сына родила мне на 40-ом году жизни моей, и оттого,
что страдала в родах его, нарек я его Мерари, что значит огорчение.
\vs Tlv 11:8
Иохаведа же родилась в Египте на 64-ом году моем:
ибо был я во славе между братьев моих.

\vs Tlv 12:1
И взял Гирсон жену и родил от нее Ливни и Шимеи.
\vs Tlv 12:2
Сыновья же Каафовы суть Амрам, Ицгар, Хеврон и Узиил.
\vs Tlv 12:3
А сыновья Мерари суть Махли и Муши.
\vs Tlv 12:4
На 94-ом же году моём взял Амрам Иохаведу,
дочь мою, себе в жёны, ибо в один день родились он и дочь моя.
\vs Tlv 12:5
8-и лет был я, когда вошли мы в землю Ханаанскую,
18-и лет, когда убил я Сихема;
с 19-и лет был я священником,
в 28 лет взял я жену,
и 48-и вошёл я в Египет.
\vs Tlv 12:6
И вот, дети мои, вы~--- третье поколение.
\vs Tlv 12:7
Иосиф умер, когда было мне 118 лет.

\vs Tlv 13:1
Ныне, дети мои, завещаю вам:
бойтесь Господа Бога вашего всем сердцем вашим,
и живите в простоте по всем законам его.
\vs Tlv 13:2
И учите детей ваших грамоте,
дабы имели они знание во всю жизнь свою,
читая постоянно закон Божий.
\vs Tlv 13:3
Ибо всякий, кто познает закон Господа,
почитаем будет, и не примут его как чужого,
куда бы ни пришел он.
\vs Tlv 13:4
И многих друзей, б\acc{о}льших, нежели родители, обретет он,
и возжелают многие из людей служить ему и слушать закон из уст его.

\vs Tlv 13:5
Творите же справедливость, дети мои, на земле, да обретёте её на небесах.
\vs Tlv 13:6
И сейте в душах ваших доброе, и обретёте его в жизни вашей;
если же посеете злое, всякую смуту и скорбь пожнёте.
\vs Tlv 13:7
Мудрость обретёте вы в страхе Божием, ибо если придёт пленение
и уничтожатся города, и земли, и золото, и серебро,
и всякое имущество погибнут,
то мудрости у мудрого никто не сможет отнять,
разве только ослепление нечестия и ожесточение греха.

\vs Tlv 13:8
Если кто убережет себя от злых этих дел,
то будет у него мудрость, и для неприятелей~--- сияющая,
и в чужой земле родина,
и среди врагов друга даст ему.
\vs Tlv 13:9
Всякий, кто учит добру и творит добро,
воссядет на престоле рядом с царями, как Иосиф, брат мой.

\vs Tlv 14:1
Познал я, дети мои, из писаний Еноха,
что в конце веков согрешите вы против Господа,
наложив руки [на Него] и у всех народов будете посмешищем.
\vs Tlv 14:2
А ведь отец наш Израиль чист от нечестия первосвященников
[которые возложат руки свои на Спасителя мира].
\vs Tlv 14:3
Как чисто солнце над землёй пред лицом Господа, так и вы
будьте светочами Израиля надо всеми народами.
\vs Tlv 14:4
И если вы помрачитесь нечестием, что тогда делать народам,
в слепоте пребывающим?
И навлечёте вы проклятие на род ваш за то,
что свет закона, данный вам для просвещения всякого человека,
его захотите вы убить, уча заповедям, которые противны законам Божиим.

\vs Tlv 14:5
Приношения Господу расхитите, и от частей его украдёте отборные,
и пожрёте их дерзко с блудницами.
\vs Tlv 14:6
И заповедям Господним учить станете из алчности,
и замужних женщин оскверните,
и с блудницами и с прелюбодейками осквернитесь,
дочерей же язычников возьмёте в жёны,
и будет смешение ваше подобно Содому и Гоморре.
\vs Tlv 14:7
И возгордитесь вы в священстве вашем, вознесясь над людьми,
и не только над ними, но и над заповедями Божиими.
\vs Tlv 14:8
Ибо презрите вы святое, ругаясь и насмехаясь.
 
\vs Tlv 15:1
Оттого Храм, избранный Господом, запустеет в нечистоте вашей,
а вы пленниками будете у всех народов.
\vs Tlv 15:2
И мерзостью будете для них, и срам стяжаете и позор вечный
от правосудия Божия.
\vs Tlv 15:3
И все ненавидящие вас возрадуются погибели вашей.
\vs Tlv 15:4
И если не обретёте милости через Авраама, Исаака и Иакова,
отцов ваших, ни единого из семени вашего не останется на земле.

\vs Tlv 16:1
И ныне познал я, что 70 седмин пребудете вы в заблуждении
и станете осквернять священство и жертвенники пятнать.
\vs Tlv 16:2
И закон отвергнете, и речи пророков уничтожите в совращении злом.
Преследовать будете вы мужей справедливых, и благочестивых возненавидите,
а словами правдивыми гнушаться станете.
\vs Tlv 16:3
А человека, обновляющего закон силою Всевышнего,
в обмане обвините, и затем и подниметесь, чтобы убить его, не зная,
что восстанет он, и во злобе вашей примете кровь его невинную на головы ваши.
\vs Tlv 16:4
Говорю же вам, что из-за того запустеют святыни ваши до основания.
\vs Tlv 16:5
И не будет чисто место ваше, но будете прокляты
и рассеяны среди народов дотоле, пока не явится он вновь,
и не смилуется, и не примет вас к себе.

\vs Tlv 17:1
И как услышали вы о семидесяти седминах, услышьте и о
священстве.
\vs Tlv 17:2
Ибо каждый юбилей будет священство.
И в первый юбилей первый помазанный
на священство велик будет и станет говорить с Богом как с отцом,
и священство его наполнится Господом, и во дни радости его для спасения мира
он воскреснет.
\vs Tlv 17:3
Во второй же юбилей помазанный взят будет в печали возлюбленного,
и будет священство его почтено и превыше всего прославится.
\vs Tlv 17:4
Третий же священник скорбью объят будет.
\vs Tlv 17:5
Четвёртый же в страданиях будет,
ибо множество несправедливости поднимется против него,
и во всём Израиле возненавидит каждый ближнего своего.
\vs Tlv 17:6
Пятый тьмою будет объят.
\vs Tlv 17:7
Так же~--- и шестой, и седьмой.
\vs Tlv 17:8
В седьмой же юбилей будет мерзость,
которой не могу высказать пред лицом людей,
ибо тогда узнают, как творить её.
\vs Tlv 17:9
Оттого пленены будут и ограблены, и исчезнет земля их и само бытие их.
\vs Tlv 17:10
В пятую же седмину вернутся они в землю опустошения
их и возобновят Дом Господень.
\vs Tlv 17:11
В седьмую же седмину обретут они священников,
которые будут идолопоклонники, любостяжатели, гордецы,
беззаконники, нечестивцы, растлители детей и скотоложцы.

\vs Tlv 18:1
И когда придёт отмщение им от Господа,
исчезнет священство.
\vs Tlv 18:2
Тогда восставит Господь священника нового,
которому все слова Господа откроются,
и сам будет вершить он суд правды на земле множество дней.
\vs Tlv 18:3
И взойдёт на небесах звезда его, словно царская,
свет знания несущая, словно свет солнца, и возвеличится во вселенной.
\vs Tlv 18:4
Озарит она землю, словно солнце, и истребит всякий мрак из
поднебесной, и настанет мир на всей земле.
\vs Tlv 18:5
Небеса возвеселятся во дни его,
и земля возрадуется, и облака возликуют,
[и знание Господне прольется на землю, как вода морская,]
и ангелы славы лика Господня возрадуются ему.
\vs Tlv 18:6
Небеса разверзнутся, и из Храма Славы сойдёт
на него святость с голосом Отцовым, словно голос Авраама к Исааку.
\vs Tlv 18:7
И прольётся на него слава Всевышнего,
и дух знания и святости почиет на нём [в воде].
\vs Tlv 18:8
Ибо он даст величие Господа сынам своим воистину навеки;
и не унаследует ему никто в поколениях и поколениях до века.
\vs Tlv 18:9
И в священство его народы исполнятся знанием на земле и освящены
будут благодатью Господней.
[Израиль же умалится в незнании и помрачится в скорби.]
В священство его исчезнет грех, и беззаконники перестанут творить зло.
\vs Tlv 18:10
И отверзнет он врата Рая и отвратит меч, угрожающий Адаму.
\vs Tlv 18:11
И даст он святым вкусить от Древа Жизни, и дух святости пребудет на них.
\vs Tlv 18:12
И Велиара он свяжет и даст власть детям своим попрать злых духов.
\vs Tlv 18:13
И возрадуется Господь детям своим, и благоволить будет возлюбленным
его до века.
\vs Tlv 18:14
Тогда возвеселятся Авраам, Исаак и Иаков, и я возрадуюсь,
и все святые облекутся радостью.

\vs Tlv 19:1
Ныне же, дети мои, всё вы слышали.
Изберите себе либо свет, либо тьму;
либо закон Господа, либо дела Велиара.
\vs Tlv 19:2
И отвечали ему сыновья его, говоря:
пред лицом Господа будем жить мы, и по закону его.
\vs Tlv 19:3
И сказал им отец их: свидетель Господь, и свидетели ангелы его,
и свидетели вы, и свидетель я речам уст ваших.
И сказали ему сыновья его: свидетели.

\vs Tlv 19:4
Так окончил Левий завещание сыновьям своим, и вытянул ноги свои на
ложе, и приложился к отцам своим, прожив 137 лет.
\vs Tlv 19:5
И положили его во гроб и после погребли его в Хевроне с
Авраамом, Исааком и Иаковом.

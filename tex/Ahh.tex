\bibbookdescr{Ahh}{
  inline={\LARGE Книга\\\Huge Ахиахара премудрого},
  toc={Книга Ахиахара},
  bookmark={Книга Ахиахара},
  header={Книга Ахиахара премудрого},
  abbr={Ахх}
}
\vs Ahh 1:1
Сказал Ахиахар: Когда жил я во дни Сеннахериба, царя Ниневийского, и когда был я, Ахиахар, хранителем сокровищ его и писцом, и когда был я молод,
\vs Ahh 1:2
прорицатели, волхвы и мудрецы сказали мне: Не будет у тебя дитяти.
\vs Ahh 1:3
И стяжал я богатство великое, и блага имел я в избытке, и взял себе шестьдесят жен,
\vs Ahh 1:4
и построил им шестьдесят дворцов, пространных, чудных и удивительных, и дома многие;
\vs Ahh 1:5
и достиг я шестидесяти лет, и не родилось дитя у меня.
\vs Ahh 1:6
Тогда я, Ахиахар, стал приносить богам жертвы и приношения, возжигал пред ними курения и ароматы
\vs Ahh 1:7
и говорил к ним: о, боги, дайте мне сына, в котором будет благоволение мое до того дня, когда я умру, и он наследует мне, и закроет очи мои, и похоронит меня.
\vs Ahh 1:8
И от дня смерти моей до смерти его, если будет он брать на всякий день от золота моего вволю и расточать его непрестанно, богатство мое не кончится.
\vs Ahh 1:9
Идолы не отвечали ему, и он оставил их и преисполнился мукою и тоскою великою.
\vs Ahh 1:10
И изменил он речь свою, и помолился Богу, и уверовал, и призвал Его в горении сердца своего,
\vs Ahh 1:11
и сказал: Боже небес и земли, Создатель всех тварей, я молю Тебя даровать мне сына, в котором будет благоволение мое, который утешит меня в час мой смертный, и закроет очи мои, и предаст меня погребению.
\vs Ahh 1:12
И пришел голос, и сказал ему:
\vs Ahh 1:13
За то, что ты надеялся на богов, и возложил на них упование твое, и приносил им дары, ты умрешь, ни сынов не имея, ни дочерей;
\vs Ahh 1:14
однако же говорю тебе: вот, у тебя есть Надав, сын сестры твоей; возьми его, научи его всей науке твоей, и он примет наследие твое.

\vs Ahh 2:1
И взял я Надава, сына сестры моей, и пестовал его, и взращивал его, и приставил к нему восемь кормилиц, чтобы питать его.
\vs Ahh 2:2
Я давал ему елея и меда, облачал его в пурпур и багрянец, покоил его на ложах мягких и на коврах.
\vs Ahh 2:3
И преуспевал Надав, сын сестры моей, и возрастал, подобно благородному кедру.
\vs Ahh 2:4
И учил я его письму, и мудрости, и философии.
\vs Ahh 2:5
Когда же вернулся царь Ассур-Аддин от празднеств своих и от странствий своих, он однажды призвал меня, Ахиахара, писца своего и хилиарха своего,
\vs Ahh 2:6
и сказал мне: о, друг мой достославный, дорогой, почитаемый, мудрый и разумный, хранитель печати моей и поверенный тайн моих! ты состарился и одряхлел, и смерть твоя приблизилась; скажи, кто будет мне служить после смерти твоей и погребения твоего?
\vs Ahh 2:7
И сказал я ему: о, владыка мой и царь, вечно живи в роды родов! у меня есть сын сестры моей, который мне как сын.
\vs Ahh 2:8
Вот, я наставил его во всей мудрости моей, и он мудр и рассудителен.
\vs Ahh 2:9
И повелел мне владыка мой: ступай, приведи его, чтобы мне видеть его, и если мне будет угодно, он будет служить мне и ходить предо мною.
\vs Ahh 2:10
Что до тебя, продолжай путь твой; он упокоит тебя от трудов твоих и окружит старость твою почетом и славою.
\vs Ahh 2:11
И я, Ахиахар, взял Надава, сына сестры моей, и представил его пред лице царя Ассур-Аддина, и отдал его в руку цареву;
\vs Ahh 2:12
и когда увидел его царь, он имел в нем свое благоволение и возрадовался ему,
\vs Ahh 2:13
и сказал: да сохранит Господь сына твоего!
\vs Ahh 2:14
Как ты служил мне и отцу моему Сеннахерибу и как ты вел дела наши со тщанием, так будет делать и Надав, сын сестры твоей;
\vs Ahh 2:15
он послужит мне и устроит дела мои, а я воздам ему честь, и возвышу его ради тебя, и позабочусь о нем.
\vs Ahh 2:16
И склонился я пред царем, и сказал ему: владыка мой царь, вовеки живи!
\vs Ahh 2:17
Прошу тебя позаботиться о нем и помогать ему; пусть обитает он в доме твоем, как и я служил тебе и служил отцу твоему.
\vs Ahh 2:18
И подал царь ему руку, и поклялся держать его при себе в почете и чести.
\vs Ahh 2:19
И поднялся я, и сказал: да будет так, о царь!
\vs Ahh 2:20
И наставлял я сына моего Надава, и передавал ему премудрость мою, и обильно уделял ему поучение, пока он не стал писцом, как я.
\vs Ahh 2:21
Вот как наставлял я его, и вот как говорил я, Ахиахар Премудрый.

\vs Ahh 3:1
О, Надав, сын мой, послушай слова мои, последуй советам моим и помни о речах моих.
\vs Ahh 3:2
Ей, Надав, сын мой! Если будешь ты внимать словам моим, и замкнешь их в сердце твоем, и никому не откроешь их,
\vs Ahh 3:3
из страха, чтобы пещь огненная не попалила языка твоего, и чтобы ты не причинил муки телу твоему и урона разуму твоему, и чтобы не посрамиться тебе пред Богом и перед людьми.
\vs Ahh 3:4
О, сын мой, если услышишь слово, никому не открывай его и не говори ничего из того, что увидишь.
\vs Ahh 3:5
Сын мой, не развязывай узла сокровенного и не запечатывай узла развязанного.
\vs Ahh 3:6
Сын мой, направляй стопу свою и слово свое, слушай и не спеши давать ответ.
\vs Ahh 3:7
Сын мой, не желай красоты внешней, ибо красота проходит и минует, но добрая память и доброе имя пребывают вовеки.
\vs Ahh 3:8
Сын мой, не бери жену с речью сварливою, ибо в речах ее горечь, и в нити ее яд, и ты попадешь в западню ее.
\vs Ahh 3:9
Сын мой, если увидишь, что женщина украшена нарядами и умащена благовониями, но нрав ее дурной, сварливый и бесстыжий, пусть сердце твое не желает ее;
\vs Ahh 3:10
если отдашь ей все, что имеешь ты, найдешь, что это не обратится к славе твоей, но ты прогневишь Бога, и ярость Его постигнет тебя.
\vs Ahh 3:11
Сын мой, не спеши говорить и не влагай в ответы и речи твои похвальбы, словно миндальное дерево, пускающее листья свои и зелень свою прежде всех деревьев и дающее плоды свои после всех;
\vs Ahh 3:12
будь, как древо приятное, хвалимое, сладкое, полное утехи, как смоковница, которая склоняет ветви, зеленеет и пускает листья последней, но плод ее бывает вкушаем первым.
\vs Ahh 3:13
Сын мой, склони главу твою, устреми взор твой долу и приготовься быть внимателен.
\vs Ahh 3:14
Будь разумен, покорен, сдержан, невозмутим.
\vs Ahh 3:15
Не будь бесстыден и сварлив.
\vs Ahh 3:16
Не возвышай голоса твоего с похвальбою и буйством,
\vs Ahh 3:17
ибо если бы громкого голоса было довольно, чтобы воздвигнуть дом, осел строил бы по два дома в день;
\vs Ahh 3:18
и если бы плуг направлялся силою, верблюд направлял бы его лучше всех.
\vs Ahh 3:19
Сын мой, лучше таскать камни с мудрым, нежели пить вино с глупцом.
\vs Ahh 3:20
Сын мой, пролей вино твое и окропи им могилы праведных.
\vs Ahh 3:21
Сын мой, иди босыми ногами по терниям и по колючкам, чтобы проторить тропу к детям твоим и детям детей твоих.
\vs Ahh 3:22
Сын мой, когда дует ветер, а море еще не возмутилось, веди ладью твою и корабль твой к гавани, пока море не возмутилось, и не пришло в движение, и не умножило валов своих, и не потопило корабля.
\vs Ahh 3:23
Сын мой, не забывайся с глупцом и не имей общения с нецеломудренным.
\vs Ahh 3:24
Сын мой, не приближайся к женщине сварливой и говорящей заносчиво, не желай красоты женщины словоохотливой и нечистой,
\vs Ahh 3:25
ибо красота женщины есть позор ее, и блеском одежды своей и красотой внешней она пленит тебя и обманет тебя.
\vs Ahh 3:26
Сын мой, как нет пользы от колец в ушах дикого осла, так нет пользы от женщины с пышною осанкою, если она лукава в словах своих и делах своих, лишена мудрости, словоохотлива и многоречива.
\vs Ahh 3:27
Сын мой, если мудрый недужен, врач сможет уврачевать и вылечить его, но нет врачевания для недугов и ран неразумного.
\vs Ahh 3:28
Сын мой, прими того, кто ниже тебя и беднее тебя; если он не воздаст тебе, Бог воздаст тебе.
\vs Ahh 3:29
Сын мой, не уставай наказывать дитя твое; наказание дитяти как удобрение сада, как завязывание кошеля, как обуздание скотины и как затвор на воротах.
\vs Ahh 3:30
Сын мой, оторви сына твоего от зла, чтобы уготовать себе покой в старости твоей;
\vs Ahh 3:31
поучай его и наказывай его, пока он юн, понудь его слушать повелений твоих, чтобы немного после он не стал вопить и восставать на тебя,
\vs Ahh 3:32
чтобы он не навлек на тебя бесчестия пред товарищами твоими,
\vs Ahh 3:33
чтобы не пришлось тебе опустить голову в людных местах и на площадях,
\vs Ahh 3:34
чтобы не краснел ты по причине лукавства дел его и не был ты уничижен по причине порочного бесстыдства его.
\vs Ahh 3:35
Сын мой, не доводи детей твоих до крайности, чтобы они не прокляли тебя и Бог не прогневался на них,
\vs Ahh 3:36
ибо написано: кто злословит отца своего и матерь свою, смертию умрет; это грех, прогневляющий Бога;
\vs Ahh 3:37
и еще: кто чтит отца своего и матерь свою, будет долголетен и будут ему блага обильные.
\vs Ahh 3:38
Сын мой, не пускайся в путь без меча и не переставай помнить о Боге в сердце твоем,
\vs Ahh 3:39
ибо ты не знаешь, когда враги лютые встретятся тебе; будь готов на пути твоем, ибо враги твои многочисленны.
\vs Ahh 3:40
Сын мой, каково древо, изобилующее плодами, листами и ветвями, таков муж с женою доброю, и плоды их дети их и родственники.
\vs Ahh 3:41
У кого нет ни жены, ни детей, ни родственников, презрен и пренебрегаем от врагов своих, как древо, стоящее вдали от дороги, которое прохожие пинают ногами и едят от плодов его, и дикий зверь стряхивает листы его и ест их.
\vs Ahh 3:42
Сын мой, если есть у тебя слуги, не предпочитай одного и не отвергай другого, ибо ты не знаешь, какого выберешь в конце.
\vs Ahh 3:43
Сын мой, коза блуждающая и умножающая шаги свои станет добычею волка.
\vs Ahh 3:44
Сын мой, услади язык твой словами Божьими и усовершенствуй слова уст твоих;
\vs Ahh 3:45
говори к любому с добротою и изяществом, ибо пасть пса промышляет ему хлеб и гортань его навлекает на него удары и камни.
\vs Ahh 3:46
Сын мой, не давай ближнему твоему наступать на ногу твою, чтобы он не наступил на грудь твою.
\vs Ahh 3:47
Сын мой, если зовешь мудрого делать работу твою, не говори ему долгих поучений и вразумлений, ибо он сделает работу твою, как желает сердце твое;
\vs Ahh 3:48
но если зовешь неразумного, не говори с ним перед другим, но лучше ступай и не зови его, ибо не сделает он работу по сердцу твоему, сколь бы долгие советы ни давал ты ему.
\vs Ahh 3:49
Сын мой, поспешно уходи со свадеб и с празднеств, не дожидаясь, чтобы главу твою умастили елеем и благовониями, дабы не навлечь на главу твою ударов и рубцов.
\vs Ahh 3:50
Сын мой, того, чья рука полна, именуют мудрым и досточтимым, а того, чья рука пуста, именуют злым, убогим, бедным и неимущим, и никто не воздает ему чести.
\vs Ahh 3:51
Сын мой, я вкушал полынь и пробовал мирру, но не нашел ничего горше бедности и нужды.
\vs Ahh 3:52
Сын мой, я поднимал железо и свинец, но не нашел ничего тяжелее хулы и клеветы.
\vs Ahh 3:53
Сын мой, я ворочал камни, но не нашел ничего столь тяжкого, как зять, живущий в доме тестя своего.
\vs Ahh 3:54
Сын мой, если нога твоя оступится и ты упадешь, это лучше, чем если ты оступишься языком твоим.
\vs Ahh 3:55
Сын мой, друг близкий лучше, чем брат далекий,
\vs Ahh 3:56
и доброе имя лучше, чем богатства мира, ибо богатства прейдут и развеются, но доброе имя пребудет вечно.
\vs Ahh 3:57
Сын мой, красота гибнет, разрушается и пропадает, и мир преходит, и все престает и прекращается, но доброе имя не преходит, не престает и не разрушается.
\vs Ahh 3:58
Сын мой, шум плача и рыдания лучше, нежели шум веселия и празднества,
\vs Ahh 3:59
ибо внимать шуму плача учит человека постигнуть грех свой и дать за него удовлетворение.
\vs Ahh 3:60
Сын мой, не восставай в суждении твоем на мужей славных и превосходных величием и властью, ибо от шуток и слов глумливых происходят гнев и раздор.
\vs Ahh 3:61
Слово гневное пробуждает и возбуждает ярость, и от ярости этой происходит раздор, а после раздора приходит и убийство.
\vs Ahh 3:62
\ldots если ты окажешься в месте таком, тебя могут убить или тебя могут позвать в свидетели;
\vs Ahh 3:63
и когда от тебя будут требовать и вымогать свидетельство твое, ты претерпишь страдание и от стыда или страха дашь свидетельство ложное и будешь посрамлен.
\vs Ahh 3:64
И я повелеваю тебе: спеши бежать из того места, где спорят, и душа твоя будет умиротворена.
\vs Ahh 3:65
Сын мой, стяжи сердце чистое и неоскверненное, разумение ясное и непомраченное,
\vs Ahh 3:66
доставь себе дух смиренный и найди себе стезю прямую, и не будет на свете человека достойнее тебя, и жизнь твоя будет блаженна.
\vs Ahh 3:67
Сын мой, не входи в сад судей, страшись судилища и не бери в жены дочь судьи.
\vs Ahh 3:68
Сын мой, защищай друга твоего перед начальником словами добрыми и исторгай немощь его из пасти льва.
\vs Ahh 3:69
Сын мой, не радуйся смерти врага твоего.
\vs Ahh 3:70
Сын мой, когда увидишь, что вошел человек старше тебя, встань перед ним.
\vs Ahh 3:71
Сын мой, око человеческое подобно источнику: оно не насытится, пока не наполнится прахом.
\vs Ahh 3:72
Сын мой, если хочешь быть мудр, воспрети устам твоим ложь и руке твоей хищение, и будешь мудр.
\vs Ahh 3:73
Сын мой, не входи в устройство брака женщины, ибо, если она будет недовольна, она проклянет тебя, и, если она будет счастлива, она не вспомнит о тебе.
\vs Ahh 3:74
Сын мой, если ты украл, извести власть имущего и предложи ему долю, и тогда ты можешь получить прощение, в противном же случае приключится тебе зло.
\vs Ahh 3:75
Сын мой, пусть лучше мудрый побьет тебя многими ударами жезла, чем неразумный помажет тебя елеем благовонным.
\vs Ahh 3:76
Сын мой, пусть нога твоя не бежит к другу твоему, чтобы он не пресытился тобою и не возненавидел тебя.
\vs Ahh 3:77
Сын мой, не возлагай кольца золотого на руку твою, если ты небогат, чтобы неразумные не глумились над тобою.

\vs Ahh 4:1
И когда я, Ахиахар, преподал мудрость эту Надаву, сыну сестры моей, я полагал, что он сохранит ее в сердце своем, пребывая при дворе, и не ведал того, что он не слушал слов моих, но бросал их словно на ветер.
\vs Ahh 4:2
Он усвоил обыкновение говорить: Ахиахар, отец мой, стар и утратил дух свой.
\vs Ahh 4:3
И Надав, сын мой, присвоил стада мои, и расточил добро мое, и не пощадил лучших слуг моих, и бил их пред лицем моим, и не пожалел скотов моих и мулов моих, и умерщвлял их.
\vs Ahh 4:4
Когда увидел я, что творил он, я сказал ему:
\vs Ahh 4:5
Сын мой, не тронь добра моего, ибо сказано в изречениях: чего рука твоя не стяжала око твое не видело.
\vs Ahh 4:6
И я известил обо всем этом владыку моего царя,
\vs Ahh 4:7
и повелел царь: пусть никто не дерзает приближаться к добру Ахиахара, писца; пока живет Ахиахар, да не приближается никто ни к достоянию его, ни к дому его.

\vs Ahh 5:1
Когда увидел Надав, что я взял брата его младшего и стал его воспитывать, это было ему неприятно; и позавидовал он, и возымел в уме своем помыслы злые по этой причине,
\vs Ahh 5:2
и сказал: Ахиахар, отец мой, стар, и мудрость его пропала, и слова его достойны презрения; ужели он отдаст добро свое брату моему, а меня изгонит из дома своего?
\vs Ahh 5:3
И когда я, Ахиахар, услышал слова Надавовы, я сказал:
\vs Ahh 5:4
Увы тебе, премудрость моя! Надав, сын мой, лишил тебя вкуса твоего и презрел мудрые слова мои.
\vs Ahh 5:5
Когда сказал я это, сын мой весьма раздражился и приготовил в сердце своем зло мне.
\vs Ahh 5:6
И пошел он ко двору царскому, чтобы сотворить зло, которое было в сердце его, как будто написал Ахиахар от лица своего письма лукавые, а он отправился ко двору объявить о них.
\vs Ahh 5:7
И вот письма от имени моего к царям, враждебным царю Сеннахерибу.
\vs Ahh 5:8
Одно было к царю Персидскому и Еламитскому, и он написал его так:
\vs Ahh 5:9
От Ахиахара, писца и хранителя печати царя Сеннахериба, мир тебе! Когда получишь ты это письмо, выступай немедля, и приходи в Ассирию, и возьмешь ты всю землю сию без войны и без боя.
\vs Ahh 5:10
Другое было от имени моего к фараону, царю Мицрейскому, и он составил его так:
\vs Ahh 5:11
Когда придет к тебе письмо это, выходи ко мне на долину южную двадцать пятого числа месяца Ава; и приведу тебя к Ниневии, и овладеешь ты царством без боя.
\vs Ahh 5:12
Он переписал письма эти по подобию руки моей и запечатал их печатью моей, а после подбросил их в один из покоев царских.

\vs Ahh 6:1
И написал он еще другое письмо от лица владыки моего царя:
\vs Ahh 6:2
От Ассур-Аддина Ахиахару, писцу моему и хранителю печати моей, мир!
\vs Ahh 6:3
Когда получишь ты это письмо, собери все воинство у горы и выступай к долине Нешрин двадцать пятого числа месяца Ава;
\vs Ahh 6:4
и когда увидишь ты, что я приближаюсь к тебе, построй воинства твои пред лицем моим, как бы ты приготовлялся к сражению,
\vs Ahh 6:5
ибо посланцы фараона, царя Мицрейского, придут со мною, и они увидят, каковы мои силы.
\vs Ahh 6:6
И сын мой Надав передал мне письмо через двух письмоносцев.
\vs Ahh 6:7
И взял сын мой Надав одно из писем, так, словно бы нашел его, и прочитал его пред царем.
\vs Ahh 6:8
И тогда царь уязвился весьма, и прогневался на Ахиахара, и говорил:
\vs Ahh 6:9
Боже! В чем погрешил я против Тебя и против Ахиахара, что он решился так поступить со мною?

\vs Ahh 7:1
И тогда ответил Надав и сказал царю:
\vs Ahh 7:2
Не печалься, о, владыка мой царь, но пойдем на долину Нешрин, как написано в этом письме; мы узнаем истину, и все будет так, как ты повелишь.
\vs Ahh 7:3
И повелел царь приготовляться выступать на долину, чтобы узнать, какова истина в этом деле,
\vs Ahh 7:4
и Надав, сын мой, сопровождал царя, и пришли они, и обрели меня с воинством, сопутствовавшим мне, в долине Нешрин.
\vs Ahh 7:5
И когда увидел я, что царь приближается ко мне, я построил воинство в боевой порядок пред лицем его, словно для сражения, доверясь письму, которое послал ко мне сын мой.
\vs Ahh 7:6
И сказал сын мой царю: ступай к себе со всяким спокойствием, о, владыка мой!
\vs Ahh 7:7
Я же приведу пред очи твои Ахиахара, отца моего.
\vs Ahh 7:8
И царь отошел в место свое.

\vs Ahh 8:1
И пришел ко мне Надав, сын мой, и взял слово, и сказал:
\vs Ahh 8:2
Владыка царь послал меня к тебе, чтобы сказать тебе: все, что ты сделал, хорошо, царь весьма хвалит тебя.
\vs Ahh 8:3
Ныне же отпусти воинства; пусть все расходятся к себе, ты же иди к царю один.
\vs Ahh 8:4
И пошел я тогда к царю, и когда увидел он меня, то сказал мне:
\vs Ahh 8:5
Ты пришел, Ахиахар, писец мой, отец и кормилец Ассура и Ниневии!
\vs Ahh 8:6
Я всегда почитал тебя и покоил тебя, ты же отпал от меня и стал один из врагов моих.
\vs Ahh 8:7
После он дал мне письмо, написанное от имени моего и запечатленное печатью моей.
\vs Ahh 8:8
И сказал мне царь: прочти письмо это.
\vs Ahh 8:9
И когда я прочел его, сотряслись члены мои, и язык мой перестал повиноваться мне; искал я слово мудрое и не находил его.
\vs Ahh 8:10
И взял слово Надав, сын мой, и сказал:
\vs Ahh 8:11
Убирайся с глаз царя, старик неразумный, и простри руки твои в узы и ноги твои в железа!
\vs Ahh 8:12
Тогда царь Ассур-Аддин отвратил лице свое от меня и сказал Навусемаку, палачу, который был со мною в дружбе:
\vs Ahh 8:13
Пойди, убей Ахиахара и удали голову его на сто локтей от тела его.
\vs Ahh 8:14
Тогда пал я лицем на землю, поклонился царю и сказал ему:
\vs Ahh 8:15
Владыка царь, вовеки живи! Ты хочешь умертвить меня; да будет по воле твоей.
\vs Ahh 8:16
Я знаю, что не погрешил пред тобою; но повели, владыка царь, чтобы меня умертвили перед дверью дома моего и чтобы тело мое отдали для погребения.
\vs Ahh 8:17
И повелел царь, чтобы было так.

\vs Ahh 9:1
И я, Ахиахар, послал сказать жене моей:
\vs Ahh 9:2
Приходи ко мне и приведи с собою тысячу девиц, одетых в виссон, в пурпур и в шафран, которые будут плясать предо мною и оплакивать меня до самой смерти моей.
\vs Ahh 9:3
И приготовь еды палачу Навусемаку, другу моему, и Парфянам, которые придут с ним;
\vs Ahh 9:4
выйди к ним навстречу и пригласи их войти ко мне, дабы и я смог войти в дом мой, как гость и чужак.
\vs Ahh 9:5
И когда жена моя приняла вестников, исполнилась она премудрости великой и выполнила все, что я велел ей.
\vs Ahh 9:6
И вышла она навстречу Навусемаку и Парфянам, и пригласила их войти в дом ее.
\vs Ahh 9:7
И принесла Эшфагни еды Навусемаку и Парфянам хлеба; и достала она им также вина, и разлила для них;
\vs Ahh 9:8
и служила им Эшфагни на пире их, пока они не опьянели и не уснули.
\vs Ahh 9:9
Когда опьянели Парфяне от вина, уснули они сном глубоким, и каждый из них уснул на месте своем.
\vs Ahh 9:10
И восхвалил я Господа небес и земли за все, что произошло, и сказал я:
\vs Ahh 9:11
Боже, Спаситель мира, ведающий все, что было, и все, что будет, призри на меня оком милостивым пред Навусемаком.

\vs Ahh 10:1
И когда я, Ахиахар, увидел все это, я заговорил и сказал Навусемаку:
\vs Ahh 10:2
Подними глаза твои к небу, Навусемак, и помысли о Боге; вспомни о хлебе и соли, которые мы съели с тобою, и не замышляй моей смерти.
\vs Ahh 10:3
Вспомни, что отец владыки моего царя также предал мне тебя, чтобы я умертвил тебя, а я не умертвил тебя, ибо знал, что ты не согрешил, и я оставил тебе жизнь до того дня, когда царь пожелал видеть тебя и дал мне дары многие. И ты спаси меня ныне.
\vs Ahh 10:4
Чтобы не распространилась молва и не сказали, что он не предан смерти, вот, у меня есть в темнице моей человек, заслуживший смерть; возьми одежды мои, надень на него и после повели Парфянам умертвить его.
\vs Ahh 10:5
Когда я сказал это, Навусемак, палач, друг мой, исполнился печали обо мне;
\vs Ahh 10:6
и взял он одежды мои, и облачил в них раба, который был в темнице, а после разбудил Парфян, которые поднялись под действием вина, и умертвили раба и отдалили голову его на сто локтей от туловища его, и отдали тело его для погребения.
\vs Ahh 10:7
И распространилась молва по Ассирии и по Ниневии, что Ахиахар умерщвлен.

\vs Ahh 11:1
И тогда Навусемак совместно с женою моею Эшфагни устроил мне в земле укром в три локтя ширины, и четыре локтя длины, и пять локтей высоты;
\vs Ahh 11:2
они дали мне есть и пить и послали сказать владыке моему царю, что Ахиахар предан смерти.
\vs Ahh 11:3
И сказал царь: страдание Ахиахарово пало на главу мою; писец мой и мудрец, защищавший пролом в стене града, я отправил тебя на гибель по слову отрока!
\vs Ahh 11:4
И призвал царь Надава, сына моего и сказал ему: ступай, оплакивай отца твоего!
\vs Ahh 11:5
Надав, сын мой, пошел в дом мой, и он не оплакивал меня и не творил память обо мне, но собрал женщин блудных и посадил их есть и пить среди песен и веселия.
\vs Ahh 11:6
Он убивал, и обнажал, и избивал слуг моих и служанок моих, он не постыдился даже женщины, воспитавшей его, но велел ей совершить с ним блуд и распутство.
\vs Ahh 11:7
Я же в недрах рва темного слышал вопль поваров моих, и пирожников моих, и хлебников моих, которые творили плач и стенание.
\vs Ahh 11:8
И тотчас обратил я молитву мою ко Всевидящему.
\vs Ahh 11:9
Прошли дни, и пришел Навусемак, и отворил затвор мой, и дал мне есть и пить,
\vs Ahh 11:10
и я сказал ему: поминай меня пред лицем Бога, и после всего, что ты увидел, скажи Ему:
\vs Ahh 11:11
Яхве, Праведный и Благий на небесах и на земле, доныне Ахиахар был ограждаем Тобою, и он приносил Тебе в жертву тельцов тучных; и вот, лежит он во рву мрачном, и свет не доходит к нему.
\vs Ahh 11:12
Услышь, Яхве, вопль раба Твоего и смилуйся над ним!

\vs Ahh 12:1
И когда узнал царь Мицрейский, что я, Ахиахар, умерщвлен, он пришел в радость великую и послал Ассур-Аддину письмо:
\vs Ahh 12:2
Царь Мицрейский Ассур-Аддину, царю Ассирийскому и Ниневийскому, мир!
\vs Ahh 12:3
Мне нужно построить крепость между небом и землею; пошли мне зодчего мудрого, которому я мог бы поручить все дело, чтобы я его вопрошал, а он мне отвечал.
\vs Ahh 12:4
Если человек, которого ты пошлешь ко мне, сделает все, что я скажу, я соберу и пошлю к тебе через него подать с Мицры за три года.
\vs Ahh 12:5
Если же ты не пошлешь мне человека, который смог бы сделать то, о чем я говорю, собери и пошли ко мне через моего посланника подать с Ассирии и Ниневии за три года.
\vs Ahh 12:6
Когда письмо это было прочитано пред лицем царя, он повелел собрать всех своих вельмож, и мудрецов, и волшебников, и книжников царства своего и сказал им:
\vs Ahh 12:7
Кто из вас пойдет в Мицру и даст ответ фараону?
\vs Ahh 12:8
И ответили вельможи царю, и сказали ему все:
\vs Ahh 12:9
Ты знаешь, владыка царь, что во дни твои и во дни отца твоего все вопросы такого рода разрешал Ахиахар, писец,
\vs Ahh 12:10
ныне же Надав, сын его, наследовавший ремесло писца и наученный им мудрости его, должен заняться делом этим.

\vs Ahh 13:1
И когда услышал Надав слова эти, возопил он пред лицем царя воплем великим и сказал царю:
\vs Ahh 13:2
И боги не могут сделать таких дел, как же смогут это люди?
\vs Ahh 13:3
При словах этих царь опечалился и удручился, и сошел с престола своего, и облачился во вретище, и сел на землю, и возрыдал, и говорил с плачем:
\vs Ahh 13:4
Увы тебе, Ахиахар, писец мой, что я велел погубить тебя по слову отрока, и не осталось никого, кто был бы тебе подобен и равен.
\vs Ahh 13:5
А ныне кто вернет тебя мне? Я заплатил бы за тебя, оценив тебя на вес золота.
\vs Ahh 13:6
И когда услышал Навусемак от царя таковые слова, он простерся ниц, и поклонился царю, и сказал:
\vs Ahh 13:7
Царь, вовеки живи! Презирающий слово владыки своего повинен смерти; повели же распять меня на древе, ибо ослушался я слова твоего;
\vs Ahh 13:8
ведь Ахиахар, которого повелел ты мне умертвить, жив доселе.
\vs Ahh 13:9
И ответил царь Набусемаку: говори, о Навусемак, ибо ты человек добрый и справедливый и не можешь совершить зла.
\vs Ahh 13:10
Если дело и вправду так, как ты сказал, и если ты мне представишь Ахиахара живым, я дам тебе дары великие: серебра мириаду талантов и пурпура сотню риз.
\vs Ahh 13:11
Когда Навусемак услышал, что царь говорит это, он начал говорить:
\vs Ahh 13:12
Я молю владыку моего царя сказать мне одно лишь, что он забывает за мною эту вину и не держит гнева на меня.
\vs Ahh 13:13
И царь поклялся ему в этом с радостию.

\vs Ahh 14:1
И взошел тогда Навусемак на колесницу, и примчался так быстро, как ветер могучий.
\vs Ahh 14:2
И отворил он мне, и я вышел на свет.
\vs Ahh 14:3
И не была посрамлена надежда моя на Бога.
\vs Ahh 14:4
И отвел меня Навусемак к царю, и повергся я на землю.
\vs Ahh 14:5
Волосы мои спадали на плечи мои, и борода моя доходила до груди моей, и тело мое было засыпано землею, и ногти мои отросли, как когти орла.
\vs Ahh 14:6
Когда царь увидел меня, он много плакал и сказал мне:
\vs Ahh 14:7
О, Ахиахар, я не погрешил против тебя, но это сын твой, воспитанный тобою, погрешил против тебя.
\vs Ahh 14:8
И отвечал я, и сказал царю: владыка мой, ныне вижу я лице твое, и скорбь моя отнята у меня.
\vs Ahh 14:10
И отвечал царь, и сказал мне: иди в дом твой, и обрежь власы твои, и омой тело твое в водах, и давай себе покой сорок дней, а после приходи пред очи мои.
\vs Ahh 14:11
И пошел я в дом мой, и делал все, как повелел мне владыка мой царь;
\vs Ahh 14:12
но лишь двадцать дней оставался я в доме моем, а когда возвратились ко мне силы мои, пошел я пред очи царя.
\vs Ahh 14:13
И показал мне царь письмо, пришедшее от царя Мицрейского.
\vs Ahh 14:14
И заговорил царь, и сказал: ты только посмотри, Ахиахар, на Мицрейцев! Что написали они мне, и какую подать наложили они на Ассур и на Ниневию!
\vs Ahh 14:15
И отвечал я ему, и сказал: владыка мой царь, вовеки живи!
\vs Ahh 14:16
Об этом деле не пекись и не печалуйся; я пойду в Мицру, и я дам ответ, и я представлю всем недругам твоим загадку и ее решение, и я принесу подать с Мицры за три года.
\vs Ahh 14:17
И при словах таковых царь возрадовался весьма и устроил день веселия, и оставила печаль лице его, и принес он в жертву тельцов и овнов, и дал мне дары великие.
\vs Ahh 14:18
И поставил он Навусемака надо всеми, и дал ему чин высокий.

\vs Ahh 15:1
И написал я письмо Эшфагни, жене моей:
\vs Ahh 15:2
О, супруга моя, когда письмо это придет к тебе, прикажи, чтобы ловчие изловили для меня двух орлов юных;
\vs Ahh 15:3
и повели; чтобы слуги мои принесли для меня нити льняной и сделали из нее для меня две веревки в палец толщиною и в тысячу локтей длиною; и скажи, чтобы кузнецы сковали для меня две клетки.
\vs Ahh 15:4
И отдай Навухаила и Тевшалома, слуг моих, семи кормилицам первородившим, чтобы те питали их млеком своим, пока они не вырастут;
\vs Ahh 15:5
и помести с ними орлов юных, чтобы они возрастали вместе, и давай орлам в корм по два овна на каждый день.
\vs Ahh 15:6
И пусть отроки выучатся говорить: принесите глины и кирпичей! Зодчие, гости царя, не имеют себе дела.
\vs Ahh 15:7
Жена моя была весьма смышлена и сделала все, что я велел ей;
\vs Ahh 15:8
и получил я приказ от царя отправляться в Мицру.
\vs Ahh 15:9
При вести этой ассирияне и ниневитяне обрадовались весьма и удалились к себе.
\vs Ahh 15:10
И ответил я царю: владыка мой царь, дозволь мне отправляться в Мицру.
\vs Ahh 15:11
И когда повелел он мне отправляться, я взял с собою отряд многочисленный и выступил в путь.
\vs Ahh 15:12
И когда прибыл я к вечернему отдыху, я прежде всего отпустил войско,
\vs Ahh 15:13
после взял из клеток орлов юных, привязал веревки к ногам их и велел отрокам моим влезть на веревки, а после отпустил орлов,
\vs Ahh 15:14
и поднялись они в воздух; а отроки кричали, как были научены:
\vs Ahh 15:15
Принесите кирпичей, глины и строительного раствора! это нужно для гостей и зодчих царя!
\vs Ahh 15:16
И после этого я вернул их к себе на землю.

\vs Ahh 16:1
И когда прибыл я в Мицру, слуги царя доложили обо мне, и царь повелел, чтобы я пришел к нему.
\vs Ahh 16:2
Я вошел к царю и приветствовал его, и после он спросил меня:
\vs Ahh 16:3
Каково твое имя? И ответил я:
\vs Ahh 16:4
Абикам, один из муравьев царя Ниневийского.
\vs Ahh 16:5
И когда фараон услыхал это, он был недоволен и сказал:
\vs Ahh 16:6
Ужели владыка твой настолько презирает меня, что он отрядил ко мне муравья, чтобы отвечать мне?
\vs Ahh 16:7
И ответил я, и сказал ему: владыка, пчела есть малейшая среди птиц и насекомых, и посмотри, какое дивное дело творит она.
\vs Ahh 16:8
С почетом допускают ее к столу государей великих;
\vs Ahh 16:9
а пред Сеннахерибом и малые как великие, и он судит их по величию и по назначению, им определенному.
\vs Ahh 16:10
И после он сказал мне: ступай, Абикам, в отведенное тебе место, а поутру вставай и приходи ко мне.
\vs Ahh 16:11
И повелел царь, чтобы вельможи его заутра облачились в одежды цвета красного, а сам он облачился с утра в одежды из виссона и пурпура;
\vs Ahh 16:12
и воссел он на престоле своем, а вельможи его заняли места вокруг него и перед ним. И я был введен в присутствие царя, и затем он спросил меня:
\vs Ahh 16:13
С кем, можно сравнить меня, о Абикам, и с кем можно сравнить вельмож моих?
\vs Ahh 16:14
И ответил я ему: тебя можно сравнить с Вилом, о владыка мой царь, а вельмож твоих со жрецами его.
\vs Ahh 16:15
И сказал мне царь: ступай, Абикам, и поутру приходи.
\vs Ahh 16:16
И повелел царь вельможам своим сменить одежды свои и облечься в одежды из льна белого,
\vs Ahh 16:17
и сам он также облекся в белое, а после воссел на престоле своем, и вельможи его заняли места перед ним и вокруг него.
\vs Ahh 16:18
И ввели меня пред очи его, и спросил он меня:
\vs Ahh 16:19
С кем можно сравнить меня, о Абикам, и с кем можно сравнить вельмож моих?
\vs Ahh 16:20
И ответил я ему, и сказал ему: тебя можно сравнить с Солнцем, а вельмож твоих с лучами его.
\vs Ahh 16:21
И снова сказал мне царь: ступай, Абикам, а поутру возвращайся ко мне.
\vs Ahh 16:22
И повелел он вельможам своим заутра переоблачиться в одежды черные;
\vs Ahh 16:23
врата дворца были покрыты черным и алым; царь же облекся в одежды алые.
\vs Ahh 16:24
После фараон велел меня ввести; я вошел, и он спросил меня:
\vs Ahh 16:25
С кем можно сравнить меня, о Абикам, и с кем можно сравнить вельмож моих?
\vs Ahh 16:26
И сказал я ему: тебя можно сравнить с Месяцем, о царь, а вельмож твоих со звездами.
\vs Ahh 16:27
И сказал он мне: ступай, Абикам, а поутру приходи ко мне.
\vs Ahh 16:28
И повелел фараон вельможам своим облечься заутра в другие одежды разных цветов;
\vs Ahh 16:29
и врата дворца были затянуты красным разных оттенков; и царь облачился в одежды разноцветные.
\vs Ahh 16:30
После фараон велел меня ввести: я вошел, и он спросил меня:
\vs Ahh 16:31
С кем можно сравнить меня и с кем можно сравнить вельмож моих?
\vs Ahh 16:32
И ответил я ему: тебя можно сравнить с Нисаном, а вельмож твоих с цветами его.
\vs Ahh 16:33
Когда царь услыхал это, он возрадовался весьма и был исполнен веселия. Он сказал мне:
\vs Ahh 16:34
Абикам, ты сравнил меня один раз с Вилом, а вельмож моих со жрецами его,
\vs Ahh 16:35
другой раз ты сравнил меня с Солнцем, а вельмож моих с лучами его,
\vs Ahh 16:36
в третий раз ты сравнил меня с Месяцем, а вельмож моих со звездами,
\vs Ahh 16:37
в четвертый раз ты сравнил меня с Нисаном, а вельмож моих с цветами его;
\vs Ahh 16:38
с кем же ты сравнишь Ассур-Аддина, владыку твоего?
\vs Ahh 16:39
И ответил я, и сказал ему: сохрани меня Бог, о царь, говорить о владыке моем Ассур-Аддине, когда ты сидишь,
\vs Ahh 16:40
ибо владыка мой царь Ассур-Аддин подобен Вилсамину, а вельможи его подобны молниям;
\vs Ahh 16:41
стоит ему пожелать, и он обращает росу и дождь в твердый град, он заставляет дыму восходить к небесам владычества своего, он издает гром и рыкание и возбраняет Солнцу вставать и лучам его показываться;
\vs Ahh 16:42
он возбраняет Вилу и жрецам его уходить и приходить местами людными;
\vs Ahh 16:43
он возбраняет Месяцу восходить и звездам блистать.
\vs Ahh 16:44
И если он захочет повелеть ветру северному, ветер соделает дождь и град, и побьет Нисан, и погубит цветы его.
\vs Ahh 16:45
И возмутился царь, слыша это.
\vs Ahh 16:46
И спросил фараон: заклинаю тебя жизнью владыки твоего Ассур-Аддина, каково имя твое?
\vs Ahh 16:47
И ответил я: Ахиахар, писец! И печать царя Ассур-Аддина вручена мне.
\vs Ahh 16:48
И спросил Фараон: так ты жив?
\vs Ahh 16:49
И ответил я: так, я жив, о владыка мой царь, и я видел Ассур-Аддина, и он продлил дни мои, и Бог избавил меня от смерти, и от казни лютой, и от греха, которого не творили руки мои.
\vs Ahh 16:50
И сказал мне царь: ступай, писец, а поутру приходи к мне и скажи мне слово, которого никто не слыхал и которого не слыхали вельможи мои ни в едином из городов Мицрейских.

\vs Ahh 17:1
И тогда я, Ахиахар, отошел в уединение и написал письмо такое:
\vs Ahh 17:2
От фараона, царя Мицрейского, Ассур-Аддину, царю Ассирийскому, мир!
\vs Ahh 17:3
Цари имеют нужду в царях, и судьи имеют нужду в судьях, а в наше время они имеют нужду в дарах, ибо средства их умалены.
\vs Ahh 17:4
Сокровищнице моей недостает денег, но позволь мне занять в твоей сокровищнице девятьсот талантов серебра, и я вскоре верну их тебе.
\vs Ahh 17:5
Я свернул письмо это и взял его с собою, и я сказал царю:
\vs Ahh 17:6
Слова, написанного в письме этом, не слышал ни ты, ни другой человек.
\vs Ahh 17:7
Все возопили: мы слышали его, и нет в том никакого сомнения!
\vs Ahh 17:8
И тогда ответил я им: итак, вы слышали, что Мицра должна Ассирии девятьсот талантов! И все были исполнены изумления.
\vs Ahh 17:9
И сказал мне тогда царь: Ахиахар!
\vs Ahh 17:10
И я ответил: вот я!
\vs Ahh 17:11
И сказал он мне: построй мне дворец между небом и землею, превыше земли локтей на тысячу.
\vs Ahh 17:12
И тотчас взял я из клеток орлов моих юных, и привязал к ногам их веревку должной длины, и велел посадить на нее мальчиков, которые принялись кричать:
\vs Ahh 17:13
Глины сюда, кирпичей сюда! Вот идут зодчие! Дайте им, с чем работать, что нужно зодчим царевым, и смешайте для зодчих вина.
\vs Ahh 17:14
Вельможи увидели, услышали и были в изумлении.
\vs Ahh 17:15
Тогда я, Ахиахар, взял жезл и бил зодчих, пока они не побежали доставать потребное для стройки.
\vs Ahh 17:16
Тогда сказал царь: ты обезумел, Ахиахар! кто сможет подавать им наверх то, что они требуют?
\vs Ahh 17:17
Я сказал им: зачем же вы поминаете понапрасну имя Ассур-Аддина? Если бы он был здесь и если бы он пожелал построить два дворца в один день, он бы их построил.
\vs Ahh 17:18
Царь сказал мне: оставь ты этот дворец. Приходи ко мне поутру.
\vs Ahh 17:19
И когда поутру я вошел к нему, он посмотрел на меня, и увидел меня, и сказал:
\vs Ahh 17:20
Ахиахар, изъясни мне, что это у нас приключилось? Жеребец твоего владыки заржал в Ассуре, в Ниневии, а наши кобылицы услышали его и выкинули плод.
\vs Ahh 17:21
И тогда я, Ахиахар, вышел от царя; и я велел слугам взять кота, бога Мицрейцев, и бить его до тех пор, покуда Мицрейцы не услышали воплей его.
\vs Ahh 17:22
Мицрейцы пошли и донесли царю: этот Ахиахар взял кота, который есть бог, и бил его.
\vs Ahh 17:23
Царь внял им и спросил меня: о, Ахиахар, зачем ты учиняешь богам нашим бесчестие?
\vs Ahh 17:24
И я сказал ему: царь, вовеки живи! Этот кот учинил мне урон великий и отнюдь не малый;
\vs Ahh 17:25
ибо царь подарил мне петуха, имевшего голос весьма прекрасный, который пел тогда, когда мне надо было идти ко двору и когда царь меня требовал, и будил меня от сна моего.
\vs Ahh 17:26
И вот урон, учиненный мне котом этим: он побывал ночью этой в Ассуре, в Ниневии, и откусил голову петуху тому, и вернулся сюда.
\vs Ahh 17:27
Тогда царь сказал мне: будучи стар, ты заблуждаешься. Между Ассуром и Мицрой триста парасангов: как же кот мог за эту ночь дойти туда, откусить голову этому петуху и вернуться обратно?
\vs Ahh 17:28
Я сказал ему: пусть между Ассуром и Мицрой триста парасангов, разве не слышали мы, что кобылицы ваши услышали ржание жеребца нашего и выкинули плод? Так и с котом.
\vs Ahh 17:29
При этих словах царь смутился и в изумлении сказал мне: о, Ахиахар, изъясни мне то, что я скажу тебе:
\vs Ahh 17:30
есть у меня столп великий, сложенный из восьми тысяч семисот шестидесяти трех кирпичей, на верху которого насаждено двенадцать кедров;
\vs Ahh 17:31
на верху каждого из этих кедров по тридцати колес, и по каждому колесу бегут две нити, одна белая, а другая черная.
\vs Ahh 17:32
Я ответил царю о предмете, о котором он спрашивал меня: умы баранов и быков знают то, что ты спрашиваешь у меня, царь.
\vs Ahh 17:33
Столп, о котором говорил владыка мой царь, это год;
\vs Ahh 17:34
столп этот сложен из восьми тысяч семисот шестидесяти трех кирпичей, каковы суть восемь тысяч семьсот шестьдесят три часа;
\vs Ahh 17:35
двенадцать кедров суть двенадцать месяцев года;
\vs Ahh 17:36
тридцать колес суть тридцать дней месяца;
\vs Ahh 17:37
две нити, одна черная и другая белая, это ночь и день.
\vs Ahh 17:38
Царь сказал мне еще: перестань.
\vs Ahh 17:39
Однако я требую от тебя, о Ахиахар, чтобы ты свил две длинные веревки из песка, по пятидесяти локтей в длину и по пальцу в ширину.
\vs Ahh 17:40
Я отвечал ему: прикажи, владыка мой царь, чтобы мне принесли такую веревку из твоей сокровищницы, чтобы мне свить подобную ей.
\vs Ahh 17:41
Он сказал мне: ты не понял слов моих: если не совьешь ты мне веревки, как я сказал тебе, не получишь ты подати Мицрейской.
\vs Ahh 17:42
И тогда я, Ахиахар, покинул царя и провел ночь ту в размышлении великом, и поутру пришел мне помысл некий.
\vs Ahh 17:43
И стал я позади дворца, в котором обитал царь, и сделал в стене напротив солнца дыру, и прошло солнце сквозь стену дворца.
\vs Ahh 17:44
И сделал я другую дыру в той же стене; после взял я пригоршню пыли и вложил в дыры, и пыль явилась в луче и была увлечена.
\vs Ahh 17:45
И заговорил я, и сказал я царю: повели, владыка мой царь, чтобы эти лучи связали в пучок, и я сделаю подобный пучок, если ты пожелаешь.
\vs Ahh 17:46
Увидев это, царь и вельможи его были объяты изумлением и недоумением и были весьма унижены.
\vs Ahh 17:47
Тогда царь велел принести мне верхний камень от разбитого жернова, а после заговорил и сказал:
\vs Ahh 17:48
Прошей мне этот камень, Ахиахар!
\vs Ahh 17:49
И я тотчас взял пест из того же камня, что и жернов, бросил его и сказал царю:
\vs Ahh 17:50
Владыка мой царь, у меня нет с собой шильев сапожника, и я не нахожу того, что мне потребно;
\vs Ahh 17:51
вели, однако, сапожникам твоим продеть нить в этот пест, который одного естества с жерновом, и я тотчас прошью его.
\vs Ahh 17:52
На эти слова царь засмеялся и сказал: добро же, Ахиахар! День, в который ты рожден, да будет благословен пред лицем богов Мицрейских!
\vs Ahh 17:53
Поелику я вижу тебя живым и здравствующим, я сотворю этот день великим празднеством и временем веселия.
\vs Ahh 17:54
Когда царь фараон был побит во всем, когда я оказал отпор его хитростям, когда я разрешил и упразднил все измышления его и все загадки его, он отдал мне подать с Мицры за три года,
\vs Ahh 17:55
и сверх того я получил те девятьсот талантов, о которых шла речь в изготовленном мною письме, как о ссуженных моим государем, и о которых все будто бы слышали, по собственному их признанию.
\vs Ahh 17:56
Я был осыпан дарами от царя и почестями от вельмож его.

\vs Ahh 18:1
И тотчас царь Ассур-Аддин поспешил мне навстречу. И начал царь говорить мне слова мудрые:
\vs Ahh 18:2
Проси и требуй от меня, чего хочешь.
\vs Ahh 18:3
И сказал я: о, владыка мой царь, вовеки живи!
\vs Ahh 18:4
И царь сошел ко мне навстречу и радовался радостию великою.
\vs Ahh 18:5
Он почтил меня, и посадил подле себя на престоле своем и на твердыне своей, и сказал мне:
\vs Ahh 18:6
Проси у меня, Ахиахар, всего, чего пожелаешь. Если пожелаешь, отдам тебе все царство мое. И сказал ему:
\vs Ahh 18:7
О, владыка мой царь, вовеки живи, в роды и роды! Все, чего прошу я у величия твоего, если нашел я благоволение в очах твоих, это дать хорошее место Навусемаку, копьеносцу, ибо ему обязан я тем, что доселе живу.
\vs Ahh 18:8
И выказал мне тогда царь приязнь свою милостями многими, особенно же дарами и подарками, которые принял я от руки его.
\vs Ahh 18:9
И осыпал меня царь дарами многими, и дарил подарки Навусемаку.
\vs Ahh 18:10
И стал царь расспрашивать меня обо всем, что было со мною пред лицом фараоновым, и о загадках фараоновых;
\vs Ahh 18:11
и рассказывал я ему все от начала и до конца, по порядку и по отдельности; он же, слушая, дивился.
\vs Ahh 18:12
И затем вынул я сокровища, и сребро, и золото, и дары, и подарки, что дал мне царь Мицрейский, чтобы доставил я их ему из Мицры; и радовался он радостию несказанною.
\vs Ahh 18:13
И сказал он мне: сколько желаешь ты получить от меня?
\vs Ahh 18:14
Я же сказал ему: я не прошу ничего, кроме как видеть тебя счастливым и благоденственным.
\vs Ahh 18:15
Что бы делал я с этими богатствами и с прочим? Однако прошу у блаженства твоего, чтобы ты дал мне власть делать все, что я пожелаю, Надаву, дабы отомстить ему, и чтобы не взыскивал ты с меня кровь его.
\vs Ahh 18:16
И тотчас дозволил мне царь делать все, что я пожелаю.
\vs Ahh 18:17
Я взял Надава и пошел в дом мой; и связал я его узами и цепями железными, и возложил я оковы железные на руки его и на ноги его и железо на плечи его,
\vs Ahh 18:18
а после стал я бичевать его розгами и бить его ударами лютыми и припоминал ему поучение, которое преподал ему в премудрости, и в знании, и в философии.

\vs Ahh 19:1
И сказал я: сын мой, того, кто не слушал ушами своими, понуждают слушать спиною его.
\vs Ahh 19:2
Надав, сын мой, заговорил и сказал:
\vs Ahh 19:3
Зачем гневаешься ты на сына твоего? Я отвечал ему:
\vs Ahh 19:4
Я, сын мой, посадил тебя на престоле славы, ты же сбросил меня с престола моего; и правда моя спасла меня.
\vs Ahh 19:5
Ты был для меня, сын мой, как скорпион, который ужалил скалу, и та сказала ему: ужалил ты сердце неуязвимое.
\vs Ahh 19:6
Он ужалил иглу, и та сказал ему: ужалил ты жало, которое сильнее, чем твое.
\vs Ahh 19:7
Ты был для меня, сын мой, как тот, кто бросает камень в небо; до неба он не дометнет, однако, бросив, согрешит.
\vs Ahh 19:8
Ты был для меня, сын мой, как тот, кто увидел ближнего своего дрожащим от стужи, и взял сосуд с водою, и метнул в него.
\vs Ahh 19:9
Сын мой, отвечай мне! Ты напал на меня, как голодный лев на осла, блуждавшего поутру.
\vs Ahh 19:10
Сказал лев ослу: подойди в мире, брат мой и друг мой!
\vs Ahh 19:11
Осел ответил: такого мира пожелаю тому, кто не привязал меня и не помешал мне выйти навстречу тебе.
\vs Ahh 19:12
Сын мой, ты был для меня как западня, укрытая под навозом. И пришел воробей, и увидела его западня, и сказала: брат мой, что делаешь ты здесь?
\vs Ahh 19:13
И ответил воробей: смотрю на тебя.
\vs Ahh 19:14
И сказала западня: помолись Богу, слава Ему!
\vs Ahh 19:15
И спросил воробей: что у тебя за палка?
\vs Ahh 19:16
Ответила западня: это посох мой и опора моя, я подпираюсь им, когда стою на молитве.
\vs Ahh 19:17
Спросил воробей: что за зерна во устах твоих?
\vs Ahh 19:18
Ответила западня: это пища, и это хлеб, восстанавливающий силы тех, кто мучим голодом.
\vs Ahh 19:19
Я поместила его во рту моем, чтобы он служил для пропитания голодных, ищущих у меня прибежища своего.
\vs Ahh 19:20
Воробей сказал: вот, я весьма изнурен голодом, и я прихожу, чтобы есть зерна.
\vs Ahh 19:21
Западня ответила ему и сказала: приблизься, о брат мой, и не страшись!
\vs Ahh 19:22
Когда же приблизился воробей, чтобы взять зерен, она тотчас схватила его за голову; и сказал воробей западне:
\vs Ahh 19:23
Если таков пост твой, и такова молитва твоя, и для такой цели зерна те, Бог не примет ни поста твоего, ни молитвы твоей и не подаст тебе никакого блага.
\vs Ahh 19:24
Сын мой, ты был для меня как жук-долгоносик, обретающийся в хлебах, что не годен ни на что доброе, но губит хлеба.
\vs Ahh 19:25
Сын мой, ты был для меня как котел, к которому приладили золотые ручки, не отчистив его дна от черноты.
\vs Ahh 19:26
Сын мой, ты был для меня как птица, которая замкнута в западне и не может убежать от ловца;
\vs Ahh 19:27
и тогда она поднимает голос приятный и сладостный и собирает вокруг себя многих птиц, малых или больших, дабы они также были уловлены.
\vs Ahh 19:28
Сын мой, ты был для меня как козел, который ведет товарищей своих на живодерню и не может спасти себя же самого.
\vs Ahh 19:29
Сын мой, ты был для меня как пес, которого проняла стужа и который пошел греться к гончарам, а когда согрелся, норовил облаять их и искусать.
\vs Ahh 19:30
Они пытались ударить его, он же залаял, а они, страшась быть искусанными, убили его.
\vs Ahh 19:31
Сын мой, ты был для меня как та свинья, которая пошла в баню вместе с вельможами;
\vs Ahh 19:32
и прошла она в баню, и омылась, а как вышла, увидала грязь и принялась в ней валяться.
\vs Ahh 19:33
Сын мой, рука, которая не трудится, и не утомляется, и не совершает работ, будет отсечена по причине лености своей.
\vs Ahh 19:34
Сын мой, это я показал тебе лице царево, и привел тебя к милостям великим, и научил тебя, и воспитал тебя, и доставил тебе всякое благо; и чем ты воздал мне, и чем отплатил мне?
\vs Ahh 19:35
Увы, и ах, и горе! Если бы ты ничего не получил от меня и ничего не принял от меня, ты не имел бы никакой власти надо мною во все дни жизни твоей.
\vs Ahh 19:36
Сын мой, сказало дерево дровосекам: если бы в руках ваших не было части меня, вы не напали бы на меня.
\vs Ahh 19:37
Ты был для меня как кот, которому сказали: перестань воровать, а тогда входи и выходи, как захочешь.
\vs Ahh 19:38
Он же ответил им: это естество мое, и если бы имел я глаза из серебра, руки из золота и ноги из берилла, я отнюдь не отстал бы от воровских дел моих.
\vs Ahh 19:39
Ты был для меня, сын мой, как змея, что забралась на ветку терновника и плыла по реке;
\vs Ahh 19:40
волк увидел ее и сказал: зло забралось на зло, и зло злейшее несет их.
\vs Ahh 19:41
Ответила змея волку тому: а ты отводишь ли коз к хозяину их?
\vs Ahh 19:42
Сын мой, я видел козу, которую пригнали на живодерню, но еще не пришло время ее, и потому она вернулась к себе, и увидала она детей своих и отпрысков детей своих.
\vs Ahh 19:43
Сын мой, я давал тебе вкушать от всякой снеди доброй, а ты не насытил меня и хлебом, смешанным с прахом;
\vs Ahh 19:44
я помазывал тебя благовониями усладительными, а ты осквернил тело мое прахом;
\vs Ahh 19:45
я упоевал тебя вином старым, а ты не напоил меня даже водою.
\vs Ahh 19:46
Сын мой, я возрастил тебя высоким, как кедр, а ты согнул меня при жизни моей и упоил меня лукавством.
\vs Ahh 19:47
Сын мой, я возвысил тебя как башню, и я говорил:
\vs Ahh 19:48
Когда враг мой придет на меня, я поднимусь и найду прибежище.
\vs Ahh 19:49
Ты же увидел врага моего и склонился к нему.
\vs Ahh 19:50
Ты был для меня, сын мой, как крот, который выходит на лице земли, чтобы обвинять Бога, не давшего ему зрения; и прилетает орел, и уносит его.

\vs Ahh 20:1
И ответил Надав, сын мой, и сказал мне:
\vs Ahh 20:2
Далече да будет от тебя, владыка мой, обычай немилосердных, но поступи со мною по милости твоей!
\vs Ahh 20:3
Даже когда человек погрешает против Бога, Бог отпускает ему грехи; так и ты ныне прости меня, и я буду печься обо всех твоих скотах или буду пасти овец твоих и свиней твоих, и меня будут называть злым, а тебя добрым.
\vs Ahh 20:4
Я ответил ему и сказал ему: сын мой, ты был для меня как пальма, которая обреталась вдали от дороги и не давала плодов; и пришел хозяин ее, и хотел удалить ее.
\vs Ahh 20:5
И сказала ему пальма эта: дай мне год, и я принесу плод сафлора.
\vs Ahh 20:6
И сказал ей хозяин ее: злосчастная, тебя недостало на то, чтобы принести твой собственный плод, как достанет тебя на то, чтобы принести чужой?
\vs Ahh 20:7
Сын мой, старость орла лучше, чем юность грифа.
\vs Ahh 20:8
Сын мой, если бы волку велели держаться вдали от овец, он ответит, что ему мил прах, ими подымаемый.
\vs Ahh 20:9
Сказали волку: учись: буква элэп, буква бит.
\vs Ahh 20:10
Он ответил: баранина, козлятина.
\vs Ahh 20:11
Сын мой, ты оправдал пословицу, которая гласит: кого ты породил, называй сыном твоим, а кого ты воспитал, называй рабом твоим.
\vs Ahh 20:13
Сын мой, больше всякого иного слова оправдал ты это: возьми сына сестры твоей на руки твои и разбей его о камень.
\vs Ahh 20:14
Бог, сохранивший мне жизнь, знающий все и воздающий каждому по делам его, сын мой, да судит и да рассудит между мною и тобою.
\vs Ahh 20:15
Ничего больше не скажу тебе. Бог да воздаст тебе по делам твоим.

\vs Ahh 21:1
И когда юный Надав услышал слово это, тело его тотчас раздулось и стало как мех и бурдюк полный, и внутренности его вышли из чресл его.
\vs Ahh 21:2
Злое его деяние воспламенило его, палило его; иссушало его, обессиливало его, губило его, умертвило его.
\vs Ahh 21:3
Конец его привел его к погибели, и ниспал он в геенну вместе с завистливыми и горделивыми,
\vs Ahh 21:4
как сказано: Сын выроет ров, и согрешит, и падет в яму, которую сам выкопал;
\vs Ahh 21:5
и еще: Кто творит лукавое, впадет в погибель;
\vs Ahh 21:6
и еще: Кто строит кову брату своему, сам падет в нее.
\vs Ahh 21:7
Здесь кончается повесть об Ахиахаре, мудреце и философе достославном, который разумел тайны и толковал загадки.

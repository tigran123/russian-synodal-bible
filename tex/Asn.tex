\bibbookdescr{Asn}{
  inline={Книга Иосифа и Асенефи},
  toc={Иосиф и Асенефь},
  bookmark={Иосиф и Асенефь},
  header={Иосиф и Асенефь},
  abbr={Асн}
}
\vs Asn 1:1
И было в 1-ый год 7-ми лет изобилия, в 3-ий месяц, в 5-ый день месяца.
\vs Asn 1:2
И послал фараон Иосифа обойти всю страну Египетскую.
\vs Asn 1:3
И в 4-ом месяце 1-го года, в 18-ый день месяца
он прибыл в пределы Илиополя, и он собрал
пшеницы полей края того, как песок морской.
\vs Asn 1:4
И был муж в том городе, сатрап фараона;
и был он поставлен над всеми сатрапами, и превосходил разумом
всех вельмож фараоновых.
\vs Asn 1:5
И был муж тот весьма богат,
и был он советником фараона,
и было имя ему Потифер, жрец илиопольский.
\vs Asn 1:6
У него была дочь, около
18-ти лет от роду, дева высокого роста и прекрасная лицом,
превосходившая бывших на земле.
\vs Asn 1:7
В ней не было никакого
сходства с дщерями египтян; она во всём походила на дочерей Еврейских:
была она высока, как Сарра, и благообразна,
как Ревекка, и прекрасна, как Рахиль.
И было имя девы той Асенефь.
\vs Asn 1:8
И слава о красоте её прошла по всей земле той,
и даже до пределов земли той; искали её руки и сыновья всех
сатрапов, и сыновья вельмож, и все царственные юноши, и военачальники;
\vs Asn 1:9
и разделяла их всех ревность и вражда из-за Асенефи,
и они готовы были из-за неё воевать между собою.
\vs Asn 1:10
И первородный сын фараона,
услышав о ней, стал просить отца своего дать её ему в жёны,
и говорил отцу:
дай мне в жёны Асенефь, дочь Потифера, жреца илиопольского.
\vs Asn 1:11
И ответил ему отец его, фараон:
зачем домогаешься ты жены ниже тебя?
не ты ли царь всей вселенной?
Ведь обручена уже с тобою дочь моавитского царя,
царевна красоты отменной; её и бери в жёны.

\vs Asn 2:1
И Асенефь уничижала и презирала всякого мужа
и была очень горда и надменна в отношении всех.
Никакой муж никогда не видел её.
\vs Asn 2:2
При доме Потифера была башня,
весьма великая и высокая,
и в ней горница, имевшая 10 комнат,
где она и жила, никем не видимая.
\vs Asn 2:3
И была 1-ая комната велика и благолепна:
пол её был выложен каменьями порфировыми;
\vs Asn 2:4
и была та горница убрана мрамором;
её стены были унизаны драгоценными блестящими камнями;
под кровом её поставлены были боги египетские,
золотые и серебряные, без числа.
\vs Asn 2:5
И всех их почитала Асенефь,
боялась их,
всегда приносила им жертвы всесожжения и фимиам.
\vs Asn 2:6
И 2-ая комната была хранилищем всего убранства
Асенефи и всех ларцов её с золотом, серебром,
златоткаными ризами,
превосходными дорогими камнями,
всем девичьим её убранством.
\vs Asn 2:7
И 3-ья комната содержала все блага земные
и служила Асенефи кладовой.
\vs Asn 2:8
Остальные же 7 комнат отданы были 7-ми девам,
по одной каждой.
И девы эти служили Асенефи, одного года,
родившиеся в одну ночь с нею.
\vs Asn 2:9
И все они были прелестны, как звёзды небесные.
С ними никогда не говорил муж, ни даже дитя мужского пола.
\vs Asn 2:10
В комнате, где охранялось девство Асенефи,
были 3 больших окна.
И 1-ое, самое большое, выходившее на двор,
было обращено на восток;
2-ое глядело на юг,
а 3-ье на север, где прямая дорога.
\vs Asn 2:11
В комнате, выходившей на восток,
утверждено было золотое ложе,
убранное золотой пурпуровой тканью,
украшенное иакинфом и виссоном.
\vs Asn 2:12
На ложе том почивала Асенефь,
и не сидел на ложе том ни один муж с женой,
кроме одной Асенефи.
\vs Asn 2:13
Обширный двор окружал комнаты,
а двор высокие четырёхугольные стены из больших камней.
\vs Asn 2:14
Входили во двор 3-мя железными воротами,
которые охранялись 8-ью сильными вооруженными мужами.
\vs Asn 2:15
На дворе, вдоль стены,
росли различные красивые
плодовые деревья со спелыми на них плодами,
ибо наступила пора урожая.
\vs Asn 2:16
На правой стороне двора был большой источник,
в\acc{о}ды которого, стекались в водоём; от водоёма исходил ручей,
бегущий посреди двора, орошая находившиеся там деревья.

\vs Asn 3:1
И было на 1-ом году 7-ми лет изобилия,
в 4-ый месяц, в 18-ый день месяца,
когда Иосиф вступил в пределы илиопольские
для собирания хлеба во время изобилия.
\vs Asn 3:2
Приблизившись к тому городу,
Иосиф послал перед лицом своим 12 мужей к жрецу Потиферу сказать:
\vs Asn 3:3
Сегодня я остановлюсь у тебя,
ибо вот полдень, час трапезы,
и солнечный жар усиливается,
и отдохну под сенью твоего дома.
\vs Asn 3:4
Потифер, услышав это,
возрадовался радостью великой, и сказал:
Да будет благословен Бог Иосифов,
внушивший ему посетить нас!
\vs Asn 3:5
И Потифер, призвав
домоправителя своего, сказал ему:
Поспеши, и устрой дом мой, и приготовить
большой обед, ибо Иосиф, сильный бог, ныне придёт к нам,.
\vs Asn 3:6
Асенефь, услышав, что отец
её и мать возвратились с поля наследия её, обрадовалась и сказала:
\vs Asn 3:7
пойду и увижу отца моего и
матерь, возвратившихся с поля наследия моего; то было время жатвы.
\vs Asn 3:8
И Асенефь поспешно надела на себя виссонную ризу златотканную,
шитую нитями иакинфовыми; опоясалась золотым поясом,
надела обручи на руки и обручи на ноги, дорогое ожерелье на шею и
усыпанную различными камнями обувь на ноги.
\vs Asn 3:9
И на всём её убранстве были начертаны
имена богов египетских, на ожерелье же её и на драгоценных камнях
вырезаны были лица идолов.
\vs Asn 3:10
И возложила она на голову венец,
и замкнула повязку вокруг висков своих,
а сверху покрылась летним покрывалом.

\vs Asn 4:1
И поспешила она, и спустилась
по лестнице из своей горницы навстречу отцу и матери
и поклонилась им с приветствием.
\vs Asn 4:2
И возрадовались Потифер и жена его радостью великой,
глядя на дочь свою; ибо видели её родители
её нарядившейся как невесту бога.
\vs Asn 4:3
И вынесли они всё добро,
что принесли они с поля наследия их, и дали дочери своей.
\vs Asn 4:4
И возрадовалась Асенефь о добре том,
при виде всех плодов винограда, смоквы и финика, и о гранатовых
яблоках, ибо всё было в поре той.
\vs Asn 4:5
И сказал Потифер дочери своей Асенефи: Дитя моё!
\vs Asn 4:6
---~Вот я, господин мой!
\vs Asn 4:7
И он сказал: Пойди, сядь между нами и скажу тебе слова мои.
\vs Asn 4:8
И села Асенефь между отцом своим и матерью.
\vs Asn 4:9
И взял Потифер правую руку дочери и, поцеловав её, сказал: Дитя моё!
\vs Asn 4:10
---~Да говорит господин мой и отец мой!
\vs Asn 4:11
И сказал Потифер:
Вот, Иосиф, сильный бог, сегодня придёт к нам: он повелитель всей страны
Египетской, ибо фараон поставил его над всеми своими владениями,
\vs Asn 4:12
и он спаситель всей нашей земли,
ибо доставляет хлеб всей стране нашей,
чем и избавит людей от предстоящего голода.
\vs Asn 4:13
Иосиф муж благочестивый, целомудренный,
скромный, и девственник, как ты ныне, муж, сильный в премудрости
и знании, ибо с ним дух Божий и благодать Господня.
\vs Asn 4:14
Итак, дитя моё,
приди и я отдам тебя ему в жёны и он будет тебе мужем навсегда.
\vs Asn 4:15
Асенефь, услышав слова отца
своего, побледнела и разлился по ней пот кровавый, обильный.
\vs Asn 4:16
С гневом посмотрев на отца, она сказала:
отец, господин мой!
Неужели по этим словам ты, как рабу,
отдашь меня человеку чужому, беглому, проданному в рабство?
\vs Asn 4:17
Не сын ли он пастуха из земли ханаанской?
Не он ли был уличён в том, что лёг с госпожою своею,
за что господин его бросил его в мрачную темницу,
откуда вывел его царь,
потому что тот истолковал его сон,
как толкуют старицы египетские?
\vs Asn 4:18
Нет, но я сочетаюсь с первородным сыном фараона,
ибо он царь всей земли.
\vs Asn 4:19
Услышав это, не стал
Потифер продолжать разговор с своею дочерью об Иосифе,
так как она ответила ему дерзко и гневно.

\vs Asn 5:1
И пришёл к Потиферу один из
отроков его, и говорит: вот, Иосиф у ворот двора нашего!
\vs Asn 5:2
И убежала Асенефь от лица
отца своего и матери, как только услышала, что они хотят отдать её за Иосифа,
взошла в горницу и вступила в свою комнату.
\vs Asn 5:3
И стала она у большого своего окна,
выходящего на восток, чтобы видеть Иосифа, входящего в дом отца её.
\vs Asn 5:4
И вышли Потифер, и жена его,
и все рабы его, и все слуги дома его Иосифу навстречу, и отверзли восточные
ворота двора.
\vs Asn 5:5
И въехал Иосиф, восседая на 2-ой колеснице фараоновой,
запряжённой 4-мя белоснежными конями,
все в золотых удилах; и вся колесница была из цельного золота.
\vs Asn 5:6
И Иосиф был облачён в белую прекрасную одежду
с пурпуровой накидкой из златотканого виссона,
с золотым венцом на главе.
\vs Asn 5:7
Вокруг венца вделаны были 12 драгоценных камней,
и на камнях 12 блестящих лучей из золота.
\vs Asn 5:8
В левой руке у Иосифа был жезл,
а в правой масленичные ветви с тучными плодами.
\vs Asn 5:9
И он вступил во двор,
и затворены были за ним все ворота.
\vs Asn 5:10
И муж\acc{и} и жёны остались за
воротами, ибо привратники заложили их и никому не дали входить.
\vs Asn 5:11
И пришли Потифер, и жена его,
и все сродники его, кроме дочери его Асенефи,
и пали на лицо своё и поклонились Иосифу.
\vs Asn 5:12
И Иосиф сошел с колесницы
своей, и они приняли его в свои объятия.
\vs Asn 6:1
И увидела Асенефь Иосифа и полюбила его сильною любовью:
и сокрушилось сердце её, и подкосились колени её,
и дрожь напала на всё тело её,
и великий страх напал на Асенефь,
и ужас овладел ею, и она сказала со вздохом:
\vs Asn 6:2
куда пойду я и куда сокроюсь от лица его?
или как взглянет на меня Иосиф, сын Божий?
ибо худое говорила я о нём.
куда бегу и укроюсь?
\vs Asn 6:3
Ибо всё сокрытое видит он
и ничто тайное не утаится от него
по причине великого света, пребывающего в нём.
\vs Asn 6:4
И ныне милостив будь ко мне, Бог Иосифа,
ибо в неведении говорила я слова лукавые.
\vs Asn 6:5
Что сделаю теперь я, несчастная?
Давно ли с презрением говорили о нём со мною отец мой и мать,
что идёт к нам сын пастуха из земли ханаанской~--- так
они отзывались об Иосифе!
\vs Asn 6:6
Ныне же само солнце с неба приходит
к нам в колеснице его, и вступает в наш дом.
\vs Asn 6:7
И я, неразумная, дерзкая,
негодная, с презрением дурно говорила о нём,
не ведая, что Иосиф сын богов;
\vs Asn 6:8
ибо невозможно родиться человеку с такой красотой,
и какая утроба произведёт такого светозарного человека!
\vs Asn 6:9
Я же, злополучная и неразумная,
худое говорила о нём со своим отцом!
\vs Asn 6:10
И теперь господин мой удалил меня от него;
ибо я по неведению худо отозвалась о нём;
пусть теперь мой отец отдаст меня
к нему в рабы в вечное услужение.

\vs Asn 7:1
И вступил Иосиф в дом Потифера и сел на седалище.
\vs Asn 7:2
И омыли ноги его, и приготовили ему трапезу особо:
ибо Иосиф не ел с египтянами,
считая осквернением вкушать с ними.
\vs Asn 7:3
И говорит Иосиф Потиферу и всем его сродникам:
кто эта женщина, которая стоит в горнице у окна?
пусть она удалится отсюда, из этого дома.
\vs Asn 7:4
Ибо Иосиф опасался беспокойства от неё;
ибо досаждали ему все жёны и дочери вельмож египетских,
желавшие возлечь с ним.
\vs Asn 7:5
При виде его они воспламенялись страстью к нему;
но Иосиф презирал их; и посланцев,
которых жёны египетские посылали
к нему с золотом и серебром и богатыми дарами,
он отсылал с бранью и угрозой.
\vs Asn 7:6
И говорил он перед Господом:
нет, не сотворю греха перед лицом Бога Израилева.
\vs Asn 7:7
И он всегда имел перед глазами образ отца своего,
Иакова, и не забывал заповедей отца своего,
который говорил Иосифу и всем сыновьям своим:
\vs Asn 7:8
берегитесь, сыны мои, жён иноплемённых,
не имейте с ними общения;
ибо общение с ними гибель для вас и осквернение.
\vs Asn 7:9
Вот почему Иосиф сказал:
Пусть та женщина удалится из этого дома.
\vs Asn 7:10
---~Господин! Та, которую ты видел в горнице,
не чужая женщина, но дочь наша и раба твоя:
\vs Asn 7:11
она дева, не видевшая мужа,
и никто из мужей ещё не видел её, кроме тебя сегодня.
\vs Asn 7:12
Если желаешь, она придёт поклониться тебе,
ибо дочь наша тебе сестра.
\vs Asn 7:13
И возрадовался Иосиф радостью великой,
когда Потифер сказал, что она дева
и что она ещё не видела мужа.
\vs Asn 7:14
Он подумал в мыслях своих, сказав сам себе:
если она дева, то должна ненавидеть всякого мужа
и не будет обременять меня.
\vs Asn 7:15
И говорит Иосиф Потиферу и всем сродникам его:
если дочь твоя дева, пусть она придёт, и так как она
сестра мне, то отныне я готов любить её как сестру свою.
\vs Asn 8:1
И взошла мать её в горницу,
и привела Асенефь, и поставила её перед Иосифа.
\vs Asn 8:2
И сказал Потифер Асенефи:
Дочь моя!
Приветствуй брата твоего; ибо он подобно тебе целомудрен по сей день
и ненавидит всякую жену чужую, как и ты всякого чужого мужа.
\vs Asn 8:3
И Асенефь сказала Иосифу:
радуйся, господин, благословенный Всевышнего Бога!
\vs Asn 8:4
И говорит Иосиф Асенефи:
да благословит тебя Господь, дающий жизнь всему!
\vs Asn 8:5
И сказал Потифер: Дочь моя!
Подойди и поцелуй брата своего.
\vs Asn 8:6
И когда подошла Асенефь поцеловать Иосифа,
простёр Иосиф десницу свою и, положив её на грудь её, сказал:
\vs Asn 8:7
Не подобает мужу богобоязненному,
который благословляет Бога живого устами своими,
который вкушает хлеб благословенный и животворящий,
который пьёт благословенную чашу бессмертия,
помазуется помазанием нетления,
\vs Asn 8:8
лобызать жену иноплемённую,
благословляющую своими устами мёртвых и немых идолов,
вкушающую с жертвенников их удавленину,
и пьющую на возлияниях их из чаши вино обмана,
и помазующуюся помазанием погибели.
\vs Asn 8:9
Но мужу богобоязненному надлежит лобызать
своих благочестивых, возлюбленных мать и сестру,
и всех из своего племени и народа,
и жену, делящую с ним ложе,
устами своими благословляющих Бога Живого.
\vs Asn 8:10
Так же и жене богобоязненной не подобает
лобызать чужого мужа, ибо это скверна перед Богом.
\vs Asn 8:11
И когда услышала Асенефь слова Иосифа,
сильно опечалилась; и стала она воздыхать,
и смотрела на Иосифа со страхом, и глаза её наполнились слезами.
\vs Asn 8:12
При виде этого Иосиф сжалился над нею,
ибо был Иосиф кроток и милостив, и боялся Бога.
\vs Asn 8:13
И он поднял десницу свою и
возложил её на голову её, и сказал:
\vs Asn 8:14
Господь, Бог отца моего
Израиля, сильный и Вышний Бог Иакова!
\vs Asn 8:15
Ты, который из мрака вызвал всё существующее к свету!
\vs Asn 8:16
Ты, который вывел из заблуждения к истине, из смерти к жизни,
\vs Asn 8:17
Господи, животвори и благослови деву сию, и обнови её духом твоим,
\vs Asn 8:18
и воссоздай её невидимой твоею рукою, и сообщи ей новую жизнь.
\vs Asn 8:19
И да вкушает она хлеб жизни, и да пьёт она от чаши благословения:
\vs Asn 8:20
приобщи её к народу твоему,
избранному тобою прежде мироздания,
\vs Asn 8:21
и да войдёт она в покой твой, уготованный тобою твоим возлюбленным,
\vs Asn 8:22
и да живёт она жизнью вечною!

\vs Asn 9:1
И возрадовалась Асенефь радостью великой
при благословении Иосифа,
и поспешно возвратилась в уединённую свою горницу,
и пала на своё ложе с воздыханиями.
\vs Asn 9:2
Ибо нашли на неё и радость,
и печаль, и страх, и трепет,
и сильный пот, когда услышала она те слова Иосифа,
что говорил он ей во имя Бога Всевышнего,
\vs Asn 9:3
и плакала она плачем великим и горьким:
раскаяние объяло её сердце при мысли о своих богах,
которым она служила; и она возненавидела всех своих идолов.
\vs Asn 9:4
Так она пробыла до наступления вечера.
\vs Asn 9:5
И Иосиф ел и пил, и по окончании трапезы приказал своим отрокам:
\vs Asn 9:6
Запрягите коней в колесницу:
вот, отхожу в путь и обойду город и страну эту.
\vs Asn 9:7
И сказал Потифер Иосифу:
Отдохни здесь, господин мой,
под кровом сим этот день, завтра поедешь в путь свой.
\vs Asn 9:8
И отвечал Иосиф:
Нет, отойду сегодня же,
ибо в сей день Бог начал творить свои создания.
\vs Asn 9:9
В 7-ой же день, когда снова наступит этот день,
возвращусь и я к вам и отдохну под кровом этим.

\vs Asn 9:10
И Иосиф отправился в путь,
а Потифер со всем своим семейством отправился в поле наследия своего.
\vs Asn 10:1
И в доме осталась одна Асенефь с 7-ью девицами:
тосковала она и плакала до заката солнца, не ела хлеба, и воды не пила.
\vs Asn 10:2
И когда наступила ночь, и все бывшие в доме заснули,
не спала одна только Асенефь.
\vs Asn 10:3
И вспоминая Иосифа, она плакала и сильно била себя в перси;
великий страх напал на неё, и начала она сильно дрожать.

\vs Asn 10:4
И когда всюду водворилась тишина,
Асенефь открыла дверь свою и спустилась с ложа своего, и сошла из
горницы тихонько по лестнице,
\vs Asn 10:5
и пришла к мельнице, и нашла
мельника спящим вместе с своими сыновьями, и поспешно сняла с дверей шерстяную
завесу,
\vs Asn 10:6
и насыпала в неё пепел из печи,
и понесла в горницу, и положила на пол,
и заперла дверь железным запором.
\vs Asn 10:7
И она начала громко рыдать и плакать.
\vs Asn 10:8
И услышали кормилица и сверстница её,
которую она любила больше всех дев, стенание госпожи своей,
\vs Asn 10:9
и пробудились от сна прочие девы,
и подошли к дверям Асенефи и нашли их запертыми.
\vs Asn 10:10
До них доходили плач и рыдания,
и они спросили: Что с тобою, госпожа наша Асенефь?
Чем ты огорчена?
 Отвори нам, чтоб мы увидели, что с тобою случилось.
\vs Asn 10:11
И Асенефь, не отпирая дверей, отвечала им изнутри, говоря:
Голова моя отяжелела и не нахожу покоя на ложе своём,
\vs Asn 10:12
нет силы отворить вам,
ослабели все члены мои, ни встать не могу, ни отпереть;
\vs Asn 10:13
разойдитесь по своим комнатам,
успокойтесь и дайте мне также успокоиться
и отдохнуть немного.
\vs Asn 10:14
И девы по слову её разошлись по своим комнатам.
\vs Asn 10:15
И встала Асенефь, тихонько отперла дверь
и пошла в другую комнату, где хранились ларцы с убранством её;
\vs Asn 10:16
и открыла ковчежец, и вынула из него
чёрное траурное платье,
которое она надевала,
оплакивая смерть первородного брата своего.
\vs Asn 10:17
И принесла Асенефь то траурное платье
в свою комнату и, положив его, заперла дверь запором.
\vs Asn 10:18
И поспешно совлекла Асенефь с себя
царственное своё одеяние и виссон,
и златотканную порфиру, и облеклась в чёрное;
\vs Asn 10:19
и развязала золотой пояс, и препоясалась вервием;
\vs Asn 10:20
и сложила венец и повязку с головы своей,
и запястья с рук и ног.
\vs Asn 10:21
И взяла она всё это и выбросила в окно,
выходившее на север.
\vs Asn 10:22
И поспешила Асенефь,
и взяла также золотых и серебряных богов,
которым не было числа, и разбила их намелко,
и выбросила их в окно из горницы нищим.
\vs Asn 10:23
И взяла Асенефь царственный свой ужин,
хлеб и рыбу, и мясо тельца,
и жертвы для богов своих, и чаши для вина,
в которых она совершала возлияния, и выбросила в окно.
\vs Asn 10:24
И бросила она всю пищу чужим собакам на съедение,
дабы ужин её, мясо агнцев, приготовленный для
идолов, не сделался пищею её собственных собак.
\vs Asn 10:25
Тогда Асенефь распорола шерстяную завесу,
наполненную пеплом, и посыпала им пол;
\vs Asn 10:26
и взяла вретище, и препоясала чресла свои;
и сняла покрывало с главы своей и расплела свои волосы,
и посыпала главу пеплом,
лежавшим на полу, и накрылась им, и пала ниц в пепел.
\vs Asn 10:27
И начала часто бить себя в перси руками,
рыдать и проливать горькие слёзы всю ночь до утра.

\vs Asn 10:28
И когда рассвело, восстала Асенефь и увидела,
и вот пепел под нею стал от слёз её как грязь болотная.
\vs Asn 10:29
И снова она пала на лицо своё
в пепел и пролежала до вечера, до заката солнца.
\vs Asn 10:30
И так делала Асенефь 7 дней,
в продолжение которых она
не переставала мучить и терзать себя:
7 дней она не вкусила хлеба и не пила воды.

\vs Asn 11:1
И было в 8-ой день, на рассвете,
когда начали кричать петухи
и собаки лаять на проходящих,
она подняла голову свою от пола
\vs Asn 11:2
(ибо члены её расслабели от
непринятия пищи в продолжение 7-ми дней),
и пала на колена и, опершись рукой о пол,
поникла головой.
\vs Asn 11:3
Волосы на голове у неё были
распущены, взъерошены, покрыты густым пеплом.
\vs Asn 11:4
Сложив руки, Асенефь оплакивала свою голову,
била себя в грудь, издавала глубокие вздохи, рвала себе
волосы, посыпая их пеплом.
\vs Asn 11:5
Таким-то образом Асенефь,
утруждая себя, изнемогла, лишилась сил
\vs Asn 11:6
и, обратившись к стене, села у окна,
выходящего на восток, и наклонила голову на грудь,
и положила руки на колена и оставалась безмолвною,
\vs Asn 11:7
ибо не нашлось слова на устах её:
в тесноте своей она в продолжение 7-ми дней не раскрывала уст.
\vs Asn 11:8
И сказала Асенефь в сердце своём:
Что мне делать?
Кто будет моим прибежищем?
К кому я обращусь?
\vs Asn 11:9
Я дева и сирота, всеми покинутая:
отец и мать меня возненавидели, потому что я возненавидела их богов,
я уничтожила, я бросила их на попрание людям;
за это возненавидели меня отец, и мать, и все мои сродники.
\vs Asn 11:10
Отец мой сказал:
Отныне Асенефь не назовётся нашей дочерью,
потому что она уничтожила золотых и серебряных богов наших.
\vs Asn 11:11
И вот, я стала ненавистной
в глазах людей, ибо надмевалась над всеми,
за коих сватали меня.
И теперь все обрадовались моему горю.
\vs Asn 11:12
Об этом только она думала и говорила:
\vs Asn 11:13
Господь Всевышний, Бог Иосифа!
Ты ненавидишь чествующих идолов мёртвых, немых и бездыханных;
ибо ты Бог мстительный и страшный богам чуждым.
\vs Asn 11:14
За это и меня возненавидел Бог,
что я чествовала идолов немых, бездыханных,
за то, что я восхваляла их, что я ела от жертвенного их мяса,
\vs Asn 11:15
уста мои осквернены их трапезой,
и я не имею права взывать к Господу, Богу неба и земли,
к Всевышнему Избавителю Иосифа.
\vs Asn 11:16
Ибо душа моя осквернена
жертвоприношениями и всесожжениями идолам.
\vs Asn 11:17
Слышала я, как говорили,
что Бог евреев Бог истинный, Бог живой,
Бог милостивый, долготерпеливый, многомилостивый,
не вменяющий человеку грехи, терпеливый к кающемуся,
не обличающий человека в тесноте его.
\vs Asn 11:18
Итак, дерзну, обращусь к нему,
сделаю его своим прибежищем,
исповедую ему все грехи мои,
изолью мольбы свои пред ним,
и он помилует меня.
\vs Asn 11:19
Быть может, он взглянет на горе
моё и сжалится надо мною, покинутою;
быть может, он, видя мои рыдания,
поможет мне,
\vs Asn 11:20
ибо он отец сирот и помощник угнетённых,
дерзну и я воззвать к нему~--- быть может, простит меня.

\vs Asn 11:21
И отвернулась Асенефь от стены
и обратилась к окну,
выходящему на восток, и стала на колена свои,
и подняла руки к небу;
\vs Asn 11:22
но страх напал на Асенефь,
и она не могла раскрыть уста свои и произнести имя Бога.
\vs Asn 11:23
И снова обратилась к стене
и села, и стала бить себя руками
в грудь и в голову неоднократно.
\vs Asn 11:24
И говорила она в сердце своём,
не раскрывая уст:
Несчастная я сирота уста мои осквернены жертвенным
мясом идолов и хвалением богов египетских.
\vs Asn 11:25
И хотя проливаю теперь слёзы
и покрываю голову пеплом,
но не могу устами своими хвалить
святое и страшное имя Бога,
боясь гнева его за призывание его имени.
\vs Asn 11:26
Итак, что делать мне, злосчастной?
Дерзну, обращусь к нему:
\vs Asn 11:27
если он в гневе своём низвергнет меня,
то он властен восстановить; если накажет, то может утешить;
при наказании может возобновить меня своею милостью;
\vs Asn 11:28
если грехи мои огорчат его,
то примириться со мною и отпустить все мои грехи.
\vs Asn 11:29
Итак, дерзну, открою уста свои,
обращусь к нему, может быть, сжалится и простит мои прегрешения.

\vs Asn 12:1
И встала Асенефь, отвернулась от стены,
стала на колена, воздела руки свои к востоку,
взглянула на небо и произнесла:
\vs Asn 12:2
Господь, Бог веков!
Ты создал всё, ты оживил всех тварей,
\vs Asn 12:3
ты вывел всё из небытия,
всё видимое из невидимого на свет,
\vs Asn 12:4
ты поднял небо и основал его
на ветрах, и землю утвердил на водах;
\vs Asn 12:5
ты поставил над бездною великие горы,
которые не тонут, но держатся на водах, как дубовый лист:
\vs Asn 12:6
горы те живые, ибо внимают гласу твоему,
Господи, ибо ты сообщаешь жизнь всем созданиям твоим.
\vs Asn 12:7
Господь, Бог мой!
На тебя уповаю, к тебе простираю мольбы мои,
тебе исповедаю грехи мои
и пред тобою открою беззакония мои:
\vs Asn 12:8
пощади меня, Господи,
ибо я во всём согрешила,
совершила преступления перед тобою, Господи!
\vs Asn 12:9
Я произносила недостойные речи,
оскверняла уста свои жертвенным мясом
и от трапезы богов египетских.
\vs Asn 12:10
Согрешила я, Господи,
согрешила и лукавое сотворила,
почитая идолов глухих и мёртвых по неведению;
\vs Asn 12:10
поэтому я недостойна к тебе обратиться с мольбами,
по причине прегрешений моих.
\vs Asn 12:11
Согрешила я, Господи, перед лицом твоим,
я, Асенефь, дочь жреца Потифера,
некогда гордая, надменная,
стоявшая выше всех богатством,
теперь стою как сирота, покинутая всеми.
\vs Asn 12:12
Тебе приношу, Господи, моление моё,
и к тебе взываю: спаси меня от гонящих меня
\vs Asn 12:14
Как испуганное дитя бежит к отцу,
тянется к нему, чтобы тот поднял его с пола,
и, раз уже в его объятиях,
оно крепко обхватывает руками
его шею и тут успокаивается;
\vs Asn 12:15
так и я, преследуемая со всех сторон,
к тебе, Господи, прибегаю:
\vs Asn 12:16
простри руку твою надо мною,
как отец чадолюбивый и милостивый,
и возьми меня с лица земли.
\vs Asn 12:17
Ибо вот старый свирепый лев преследует меня,
ибо он отец богов египетских, а идолы народов~--- дети львов;
\vs Asn 12:18
я же выбросила всех богов,
уничтожила их, а лев отец их,
разгневанный, хочет поглотить меня.
\vs Asn 12:19
Избавь меня, Господи, от когтей его
и спаси из пасти его, дабы он не схватил меня как волк,
\vs Asn 12:20
и не растерзал, и не бросил в огонь печи,
из огня в вихрь, который, охватив,
лишит меня зрения и низвергнет в бездну морскую,
\vs Asn 12:21
где поглотит меня великое чудовище морское,
существующее изначала, и где я погибну на вечные времена.
\vs Asn 12:22
Господи!
Спаси меня, прежде чем постигнет меня всё это;
\vs Asn 12:23
спаси и укрепи меня покинутую,
ибо отец и мать отреклись от меня, сказав:
Асенефь не дочь нам, она уничтожила богов наших,
отвергла их.
\vs Asn 12:24
Ты один, Господи, надежда моя, на тебя уповаю,
ибо ты отец сирот и защитник гонимых, и притесняемых
покровитель.
\vs Asn 12:25
Помилуй меня, деву.
Ты милостив, как отец;
ты жалостлив, как мать;
ты долготерпелив как никто.
\vs Asn 12:26
Ибо вот, всё наследие,
данное мне отцом моим Потифером,
тленно и скоротечно;
твои же дары непреходящи, вечны.
\vs Asn 12:27
Теперь я отреклась от всего
и ото всех, покинула все блага земные.
\vs Asn 12:28
Тебя одного сделала своей надеждой.
\vs Asn 12:29
Одевшись во вретище,
покрывшись пеплом, оплакиваю грехи мои.
\vs Asn 12:30
Призри на сиротство моё,
Господи, ибо к тебе прибегла я.
\vs Asn 12:31
Вот, бросила я царственную
ризу свою златотканную,
виссон и серьги драгоценные и облеклась в хитон чёрный.
\vs Asn 12:32
Вот, сняла с себя золотой пояс
и препоясалась вервием и вретищем.
\vs Asn 12:33
Вот, бросила с головы венец
и покрыла главу мою пеплом.
\vs Asn 12:34
Вот, мраморные полы моего жилища,
прежде убранные разноцветными каменьями и пурпуром,
блестящие чистотой,
теперь омочены моими слезами и омрачены пеплом.
\vs Asn 12:35
Вот, Господь мой, от пепла и слёз плача моего
грязь великая соделалась в чертоге моём,
как на пути проезжем.
\vs Asn 12:36
Вот, я отказалась от царского моего ужина
и бросила его чужим собакам на съедение.
\vs Asn 12:37
И вот, 7 дней и 7 ночей не ела я хлеба, не пила воды;
\vs Asn 12:38
уста у меня высохли,
как кожа тимпана;
язык мой стал как рог;
губы мои сделались как черепица,
лицо моё осунулось;
очи опухли от беспрерывных слёз,
все силы оставили меня.
\vs Asn 12:39
Теперь, узнав, что боги,
чествуемые мною по неведению,
были немые и глухие идолы,
я отдала их на попрание,
серебро и золото растаскали воры и унесли с глаз моих;
\vs Asn 12:40
потому что надежду свою я положила на тебя, Господь, Бог Иосифа!
\vs Asn 12:41
Прости меня, ибо всё сделала я по неведению,
я порицала господина моего Иосифа, не зная,
что он сын возлюбленный у тебя.
\vs Asn 12:42
Мне сказали легкомысленные люди про него,
что он сын пастуха земли ханаанской, и я поверила им.
\vs Asn 12:43
И заблуждалась, и с презрением отнеслась
к избраннику твоему и стала говорить о нём непочтительно,
не зная, что он сын твой.
\vs Asn 12:45
Ибо кто из людей породил такую красоту,
и кто есть другой столь мудрый и сильный, как Иосиф?
\vs Asn 12:46
Ты одарил его дивной красотой, великой мудростью и добродетелью.
\vs Asn 12:45
Но, Господь мой, тебе
поручаю его, ибо возлюбила его я больше души моей.
\vs Asn 12:46
Сохрани его премудростью и благодатью твоею
и предай меня в рабы ему, и буду я омывать ноги его и оправлять
ему ложе, и служить ему во все дни жизни моей.

\chhdr{Исповедь Асенефи перед Богом.}
\vs Asn 13:1
Согрешила я перед тобою, согрешила, Господи!
Совершила беззаконие я, Асенефь, дочь Потифера,
жреца илиопольского, главного смотрителя над всеми богами.
\vs Asn 13:2
Согрешила я перед тобой, Господи, согрешила,
совершила беззаконие, почитала богов, коим нет числа,
и ела от жертвенного их мяса.
\vs Asn 13:3
Согрешила я, согрешила, Господи, перед тобой,
совершила беззаконие; ибо я была дева гордая, надменная.
\vs Asn 13:4
Согрешила я, Господи, согрешила перед тобой,
совершила беззаконие: я ела хлеб удушающий,
пила чашу сетей, вкушая от стола смерти.
\vs Asn 13:5
Согрешила я, Господи, согрешила перед тобой,
совершила беззаконие: не знала я Господа, Бога небес, не
надеялась на Всевышнего живого Бога веков.
\vs Asn 13:6
Согрешила я, Господи, согрешила перед тобой,
совершила беззаконие: я надеялась на величие своей
славы, на красоту свою; была я горда и надменна.
\vs Asn 13:7
Согрешила я, Господи, согрешила перед тобой,
совершила беззаконие, презирая всех людей, из которых ни
одного не считала я человеком.
\vs Asn 13:8
Согрешила я, Господи, согрешила перед тобой,
совершила беззаконие; много раз говорила, что нет на земле
князя, достойного развязать девственный мой пояс.
\vs Asn 13:9
Согрешила я, Господи, согрешила перед тобой,
совершила беззаконие: я ненавидела всех желавших взять
меня в жёны, я презирала их и порицала.
\vs Asn 13:10
Согрешила я, Господи, согрешила перед тобой,
совершила беззаконие, но по твоему милосердию я
сделаюсь невестой сына великого государя~--- отпусти мне грехи мои.
\vs Asn 13:11
Когда пришёл Божий воин Иосиф,
он низложил мою гордость, он усмирил меня,
и уловил красотою своею,
и мудростью своею поймал меня, как рыбку в сети,
\vs Asn 13:12
душою своею предложил мне лекарство жизни,
силою своею утвердил меня и посвятил меня Богу веков.
\vs Asn 13:13
Он дал мне есть хлеб жизни и пить чашу бессмертия.
\vs Asn 13:14
Он напоил меня, и я стала его невестой вовеки.

\vs Asn 14:1
И когда Асенефь прекратила беседу с Господом,
вот, взошла денница на небеса от страны восточной.
\vs Asn 14:2
И увидела её Асенефь, и возрадовавшись, сказала:
Внял Бог молитве моей; ибо вот, светило,
предвестник великого дня, явилось.
\vs Asn 14:3
И вот видит Асенефь подле денницы разверзлось небо,
и показался свет неизреченный.
\vs Asn 14:4
Видя это, она пала лицом на пепел.
\vs Asn 14:5
И сошло с неба подобие мужа,
и стал он перед головой Асенефи, и стал звать её: Асенефь!
\vs Asn 14:6
И сказала она: Кто это звал меня?
Двери моей горницы заперты, башня моя высока,
кто это осмелился войти в мой чертог!
\vs Asn 14:7
И тот муж вторично позвал её, и сказал: Асенефь, Асенефь!
\vs Asn 14:8
И спросила Асенефь: Возвести мне, кто ты?
\vs Asn 14:9
---~Я князь Израилев и архистратиг
всего воинства Всевышнего.
Встань, становись на ноги, и я поведаю слово.
\vs Asn 14:10
И Асенефь, подняв голову, увидела мужа,
и вот, во всём подобен он Иосифу:
и одеждою, и венцом, и царским жезлом;
\vs Asn 14:11
лицо же его словно молния,
глаза как солнечные лучи,
волосы на голове как пламя,
а руки как раскалённое железо,
от рук и ног его сыпались искры,
как от пламенеющего огня.
\vs Asn 14:12
При виде этого Асенефь пала на лицо своё на землю,
и ужас объял её, и дрожь проникла до костей членов её.
\vs Asn 14:13
И тот муж сказал ей:
Мужайся, Асенефь, не бойся,
но встань на ноги свои, и я обращу к тебе мои слова.
\vs Asn 14:14
И сказал тот муж:
Пойди, сними с себя хитон чёрный,
выражение печали, и отложи вретище с чресл твоих,
и отряхни прах с головы твоей;
\vs Asn 14:15
и омой лицо своё живою водою,
и облекись в ризу новую, нетронутую,
и опояшь чресла твои двойным
золотым поясом девства твоего,
и приди, и тогда я скажу тебе слово.

\vs Asn 14:16
И Асенефь поспешно
удалилась во 2-ую свою комнату,
где были корзины с её убранствами.
\vs Asn 14:17
И открыла она ковчежец свой,
и вынула полотняное дорогое платье,
до которого никто ещё не касался,
и сняла с себя чёрное траурное платье
и надела платье новое.
\vs Asn 14:18
И сняла она вервие и вретище с чресл своих,
и опоясала 2-мя поясами своего девства:
одним стан свой, а другим грудь свою с сосцами.
\vs Asn 14:19
И отряхнула прах с головы своей,
и умыла руки и лицо своё водою чистою;
и взяла она чистый полотняный покров
и покрыла им свою голову.

\vs Asn 15:1
И пришла она к мужу тому в 1-ую комнату,
и муж тот, увидев её, сказал:
\vs Asn 15:2
Сними с головы своей покрывало,
зачем ты надела его сегодня?
ибо ты до сей поры чистая и скромная дева,
и голова твоя как голова юноши.
\vs Asn 15:3
И сняла Асенефь покрывало с головы своей.
И сказал ей тот муж:
\vs Asn 15:4
Мужайся, чистая дева Асенефь!
Вот я внял исповеди твоей и молитвам,
вот я увидел 7-дневные тяжкие твои лишения;
\vs Asn 15:5
теперь я своими глазами вижу тебя,
стоящую на пепле и проливающую слёзы:
мужайся, чистая дева Асенефь!
\vs Asn 15:6
Ибо имя твоё вот уже написано
на небесах Божьими перстами
в книги живых с именами изначала вписанных
и пребудет оно там неизгладимо во веки веков.
\vs Asn 15:7
Отныне ты обновишься:
ты вкусишь от животворящего,
благословенного хлеба
и пить будешь от чаши благословения
и бессмертия;
и умастишь себя чистым елеем.
\vs Asn 15:8
Мужайся, чистая дева Асенефь!
Вот я отдаю ныне тебя Иосифу в невесты навсегда,
и отныне имя тебе будет не Асенефь, но Город убежища;
\vs Asn 15:9
ибо через тебя многие народы прибегнут к Господу,
Богу небес, и под сенью крыл твоих укроются
уповающие на Господа Бога;
\vs Asn 15:10
и за стенами твоими будут защищены,
притекающие к Всевышнему через покаяние,
ибо покаяние есть дщерь Всевышнего,
и она предстательствует на всякий час
пред Всевышним за тебя и за всех кающихся.
\vs Asn 15:11
Ибо он есть Отец покаяния,
она же есть матерь дев, и на всякий час молит его о кающихся;
\vs Asn 15:12
ибо возлюбленных своих
вознесет Бог, который есть податель даров, подкрепитель всех дев.
\vs Asn 15:13
Ему угодно девство, он ищет его,
он заботится о нём всегда.
\vs Asn 15:14
Кающихся он принимает под сень свою,
готовит для них на небесах место покоя, и они навеки будут под
покровом его.
\vs Asn 15:15
И есть покаяние весьма прекрасная дева,
чистая, и непорочная, и кроткая, и Бог Всевышний любит её, и
все ангелы почитают её.
\vs Asn 15:16
И вот, я иду к Иосифу,
и буду говорить с ним о тебе, и он войдёт к тебе сегодня, и увидит тебя, и
возлюбит тебя, и будет женихом твоим, ты же будешь ему невестой.
\vs Asn 15:17
Итак, внимай, о ты, дева Асенефь!
Облекись в брачную ризу, в ризу древнюю, приготовленную для тебя из
начала в покое твоем;
\vs Asn 15:18
и надень на себя всякий убор твой избранный,
и укрась себя нарядами, как добрую невесту, и ты пойдёшь навстречу Иосифу.
\vs Asn 15:19
Ибо вот, он сегодня приблизится к тебе,
и увидит тебя, и возрадуется.
\vs Asn 15:20
И когда тот муж кончил речь свою,
Асенефь возрадовалась радостью великою,
и пала она на лицо своё,
поклонилась ему и сказала:
\vs Asn 15:21
Благословен Бог Всевышний,
пославший тебя избавить меня от мрака и возвести к свету, и
извлёкший меня из глубины бездны.
\vs Asn 15:22
Скажи имя твое, господин!
Поведай мне, дабы я могла благословлять его навеки.
\vs Asn 15:23
Отвечает ей муж тот:
Имя моё написано на небесах в книге
Всевышнего Божьими перстами прежде,
чем были написаны все имена.
\vs Asn 15:24
Я Князь Всевышнего,
и имена, вписанные в книгу Всевышнего,
не подлежат ни исследованию, ни слуху, ни
зрению человека в этом мире.
\vs Asn 15:25
Асенефь сказала: Если я
обрела благодать пред тобою и разумею все слова, сказанные мне тобою, то
позволь рабе твоей сказать пред тобою слово.
\vs Asn 15:26
И сказал тот муж: Говори.
\vs Asn 15:27
И говорит Асенефь: Прошу тебя, господин.
\vs Asn 15:28
Произнося эти слова, она приблизилась к его руке
и с такой мольбой: Сядь на малое время на этом ложе:
оно чисто и не осквернено, ибо на нём не сидели ещё муж и жена.
\vs Asn 15:29
Я поставлю пред тобою стол
и принесу из моей кладовой хлеб, и вкусишь ты от него;
\vs Asn 15:30
и старое вино, благовоние
которого до небес, и будешь ты пить от него,
и отправишься в путь свой.
\vs Asn 16:1
И сказал ей муж тот: Иди и принеси скорее.
\vs Asn 16:2
И Асенефь поспешно принесла
и поставила перед ним пустой стол,
и выйдя от него, она готова была уже войти в
кладовую за хлебом,
\vs Asn 16:3
когда он сказал: Принеси мне мёд сотовый.
\vs Asn 16:4
Асенефь остановилась, и она опечалилась,
потому что не было у неё сотового мёда в кладовой.
\vs Asn 16:5
И спросил её тот муж: Что же ты остановилась?
\vs Asn 16:6
И отвечала Асенефь: Пошлю отрока за город,
недалеко отсюда поле наследия нашего, он тотчас принесет оттуда
медовые соты, и я поставлю их перед тобою, господин.
\vs Asn 16:7
И сказал ей тот муж: Войди в свою кладовую
и ты найдешь медовые соты на столе, возьми их и принеси.
\vs Asn 16:8
И взошла Асенефь в свою кладовую и нашла на столе соты.
\vs Asn 16:9
И были ячейки тех сотов большие,
и белые как снег, и полные мёдом.
И ячейки эти подобны были небесной
росе, и запах от них благоухание жизни.
\vs Asn 16:10
Асенефь изумилась и подумала:
уж не из уст ли этого мужа вышли эти соты,
так как запах от них, как от этого мужа?
\vs Asn 16:11
И взяла Асенефь медовые соты,
принесла и поставила их на пустой стол перед тем мужем.
\vs Asn 16:12
И спросил тот муж: Как же это ты сказала,
что нет у меня в кладовой медовых сотов, а между тем,
вот, принесены оттуда эти соты?
\vs Asn 16:13
И, смутившись, сказала Асенефь:
У меня, господин, не было сотов медовых в кладовой; но в то время,
как ты заговорил, быть может, из уст твоих они изошли;
ибо запах от них как запах от уст твоих.
\vs Asn 16:14
И улыбнулся тот муж, видя разумность Асенефи.
\vs Asn 16:15
И подозвав её к себе, он простёр правую руку свою к голове её.
\vs Asn 16:16
И Асенефь испугалась руки того мужа,
ибо искры сыпались из уст его, как от раскалённого железа,
\vs Asn 16:17
она, устремив глаза,
смотрела на его руку, а он при виде этого, улыбнувшись, сказал:
\vs Asn 16:18
Блаженна ты, Асенефь;
ибо неизречённые тайны Всевышнего Бога открылись тебе!
Блаженны и те, кои предстанут пред Господа с покаянием;
\vs Asn 16:19
они вкусят от медовых этих
сотов, дающих жизнь; ибо их приготовили пчёлы рая места сладости;
\vs Asn 16:20
они приготовили их из росы
живых райских роз, и ангелы Божьи вкушают от него, и все сыны Всевышнего, и
вкусивший от этих сотов не умрёт вовеки.
\vs Asn 16:21
И простёр тот муж правую руку,
отломил частицу от сотов,
и вкусил сам, и частицу рукою своею вложил ей в уста,
\vs Asn 16:22
говоря: Вот ты, Асенефь,
вкусила хлеб жизни, и пила чашу бессмертия,
и умастилась елеем непорочности.
\vs Asn 16:23
Отныне тело твоё
распускаться будет подобно цветку, выросшему на земле Всевышнего; кости твои
утучнятся подобно кедрам, растущим в раю сладости,
\vs Asn 16:24
и сила проникнет всю тебя,
и молодость твоя не увидит старости, и красота не покинет тебя вовеки,
\vs Asn 16:25
и будешь ты как город,
окружённый бойницами во имя Господа Бога, царя веков.
\vs Asn 16:26
И простёр тот муж руку к
отломанной части сотов, и соты сделались целы, как прежде.
\vs Asn 16:27
И он снова протянул правую
руку свою и перстом коснулся края сотов,
обращенного на восток, и обратил его на сторону,
выходящую на запад; и путь перста его принял кровавый вид.
\vs Asn 16:28
И он, простерши руку в другой раз,
коснулся ею края сотов, обращенного на север,
и обратил его на сторону, выходящую на юг,
и путь перста его имел вид кровавый.
\vs Asn 16:29
И стояла Асенефь слева, и
смотрела и видела всё, что делал он.
\vs Asn 16:30
И сказал тот муж медовым сотам: Приблизьтесь сюда.
\vs Asn 16:31
И вот из твердых сотовых
ячеек поднялись тысячи и тьмы белоснежных
пчёл с длинными пурпуровыми крыльями,
\vs Asn 16:32
у иных же крылья были как
виссон, унизанный дорогими камнями,
и как гиацинт, и как нити златые; их головки
были украшены золотыми венцами.
\vs Asn 16:33
Пчёлы эти были прекрасны на
вид и жала их были изострены.
\vs Asn 16:34
И покрыли Асенефь все пчёлы
роем с головы до ног; и иные пчёлы,
большие, как бы царицы роя, присели к устам Асенефь.
\vs Asn 16:35
Поднялись они из ячеек
своих, облепили всё лицо Асенефи
и стали работать на её лице, и отверстия ячеек
приходились к устам Асенефи.
\vs Asn 16:36
И тот муж сказал пчёлам:
Ступайте по своим местам.
\vs Asn 16:37
И поднялись все пчёлы и
улетели по направлению к небу.
\vs Asn 16:38
И те из них, которые жалить
хотели Асенефь, падали мёртвые.
\vs Asn 16:39
И тот муж, жезлом своим
прикоснувшись к мёртвым пчелам, сказал:
И вы восстаньте и ступайте на свои места.
\vs Asn 16:40
И встрепенулись они, и
полетели перед домом Асенефи,
и уселись на плодовых деревьях.
\vs Asn 17:1
И сказал тот муж Асенефи:
Видела ли слово сие?
\vs Asn 17:2
И отвечала она: Так, господин: всё сие я видела.
\vs Asn 17:3
И он сказал: Таковы будут слова мои ныне.
\vs Asn 17:4
И он в 3-ий раз простёр правую руку свою
к части медовых сотов и съел её, не повредив.
\vs Asn 17:5
И поднялся огонь от стола, и пожрал соты,
и запах от сжигаемых сотов наполнил собою горницу,
и запах был весьма приятен.
\vs Asn 17:6
И сказала Асенефь мужу:
Есть у меня 7 девиц однолеток, служащих мне, воспитанных со мною от
младенчества моего, родившихся в одну ночь со мною: я их люблю как сестёр;
позову их сюда, чтобы ты благословил их, как ты благословил меня.
\vs Asn 17:7
И сказал муж: Зови.
И, будучи позваны, они стали перед ним.
\vs Asn 17:8
И тот муж сказал:
Да благословит вас Бог Всевышний;
да будете вы 7-ью столпами для этого города,
и пусть почиет на вас Господне благословение вовеки.
\vs Asn 17:9
И сказал он Асенефи:
Переставь отсюда этот стол.
\vs Asn 17:10
И когда обратилась Асенефь,
чтобы поставить стол на его прежнее место,
муж тот сделался невидим.
\vs Asn 17:11
И увидела Асенефь подобие
колесницы, несущейся на восток; и колесница подобна была огню, кони её как
молнии, и на колеснице стоял муж тот.
\vs Asn 17:12
И сказала Асенефь:
Я, неразумная и дерзкая, позволила себе сказать смело,
что человек пришёл в мою горницу, не ведая,
что пришедший сегодня ко мне был
Господь небесный, который вот возвращается на своё место,
\vs Asn 17:13
и она присовокупила:
Помилуй, Господи, и сжалься надо мною рабою твоею, ибо в неведении я говорила
перед лицом твоим дерзновенно и неразумно.

\vs Asn 18:1
И между тем как Асенефь
погружена была в эти размышления, прибежал отрок из числа рабов Потифера и
сказал:
\vs Asn 18:2
Вот Иосиф, бог сильный,
едет к нам: его колесница уже перед нашим двором.
\vs Asn 18:3
Асенефь поспешно позвала
кормилицу свою, заведовавшую всем её имуществом, и сказала:
\vs Asn 18:4
Пойди, займись поскорее
убранством нашего дома и приготовь лучшую вечерю для бога сильного Иосифа, ныне
едущего к нам.
\vs Asn 18:5
Тут кормилица заметила, что
у Асенефи ланиты впали по причине 7-дневного воздержания от пищи, ей грустно
стало, и она заплакала, и, взяв её за правую руку, поцеловала,
\vs Asn 18:6
и спросила: Что с тобой, дитя?
Отчего такие впалые у тебя ланиты?
\vs Asn 18:7
Отвечала Асенефь:
Сильная головная боль посетила меня,
ночь провела без сна, вот отчего изменилась в лице.
\vs Asn 18:8
И пошла её кормилица убирать дом и готовить вечерю;
Асенефь же вспомнила слова того мужа, и поспешила
во 2-ую свою комнату, где в хранилищах лежали её наряды.
\vs Asn 18:9
И открыв большой ковчежец,
она вынула из него брачные свои одежды, превосходные, блестящие, и надела их.
\vs Asn 18:10
И опоясалась она золотым царским поясом,
украшенным различными камнями многоценными;
\vs Asn 18:11
и надела на руки и ноги
золотые запястья, на шею дорогие ожерелья, унизанные бесчисленными дорогими
каменьями;
\vs Asn 18:12
и надела на голову золотой
венец, унизанный с обеих сторон у чела 12-ью большими камнями;
\vs Asn 18:13
и набросила на голову
лёгкое покрывало, как подобает невесте; и взяла царский жезл в руку.
\vs Asn 18:14
И вспомнила Асенефь слова
своей кормилицы, что печально выражение лица твоего, воздохнула и с грустью
сказала: лицо моё горит, если Иосиф заметит это, ему не понравится.
\vs Asn 18:15
И, обратившись к подругам
своим, сказала: Принесите мне чистой, ключевой воды, умою лицо своё.
Принесли и налили воды в рукомойницу.
\vs Asn 18:16
И она наклонилась и
увидела в воде лицо своё, подобное солнцу, глаза свои как восходящую утреннюю
звезду, прелестные ланиты свои как части граната, уста свои как
распустившуюся розу и зубы, блестящие белизной.
\vs Asn 18:17
И Асенефь, созерцая себя в
воде в таком виде, возрадовалась радостью великой и стала умывать лицо своё.
\vs Asn 18:18
И когда пришла кормилица с
донесением об исполнении данных ей приказаний, взглянув на Асенефь, удивилась
так, что не могла опомниться в продолжение 2-ух часов:
так велико было её изумление!
\vs Asn 18:19
Она, став на колени, спросила:
Откуда эта великая, дивная красота, госпожа моя?
Вижу сам Господь, Бог небесный, избрал тебя быть невестой Иосифа.

\vs Asn 19:1
Между тем как они
беседовали, пришёл отрок из рабов и сказал Асенефи: Вот, Иосиф стоит у врат
двора нашего.
\vs Asn 19:2
Асенефь поспешно спустилась
по лестнице в сопровождении 7-ми дев навстречу Иосифу и стала в проходе дома.
\vs Asn 19:3
Иосиф вступил на двор;
ворота затворились, и чужой народ остался за воротами.
\vs Asn 19:4
Тогда Асенефь вышла из
прохода навстречу Иосифу.
\vs Asn 19:5
И увидев её, Иосиф был
поражен великой её красотой и спросил: Скажи, кто ты?
\vs Asn 19:6
И она ответила: Я раба
твоя, Асенефь, которая по повелению твоему выбросила всех своих идолов и
уничтожила.
\vs Asn 19:7
Сегодня приходил ко мне
некий муж, который дал мне хлеб жизни и вино благословения, сказав: вот Я отдаю
тебя как вечную невесту Иосифу, который будет твоим женихом навсегда;
\vs Asn 19:8
к этому он прибавил: отныне
ты будешь называться не Асенефь, а Городом прибежища; ибо через тебя
прибегнут многие народы к Богу Всевышнему.
\vs Asn 19:9
Он прибавил: я пойду к
Иосифу и поведаю в слух его слова мои о тебе.
\vs Asn 19:10
Ты знаешь уже, господин
мой; ибо тот муж приходил к тебе и говорил обо мне.
\vs Asn 19:11
И сказал Иосиф Асенефи:
Всевышний Бог благословил тебя; ибо Господь Бог утвердил стены твои на высоте,
стены же твои адамантовые, они стены жизни;
\vs Asn 19:12
ибо многие сыны
человеческие жить будут в твоем Городе прибежища, и Господь Бог воцарится в
нём вовеки.
\vs Asn 19:13
Итак, приди ко мне, дева
чистая; зачем так далеко стоишь от меня! Ибо благую весть о тебе принёс мне от
небес муж, поведавший мне всё, что было с тобою.
\vs Asn 19:14
И он, подняв руку, подозвал
Асенефь. И она подошла к Иосифу и пала ему в объятия.
\vs Asn 19:15
И оживились души у них и
исполнились радостью; и Иосиф, дав Асенефи лобзание, сообщил ей дух жизни, дух
премудрости, и дух истины.
\vs Asn 19:16
И, обнявшись, они долго
лобызали друг друга.
\vs Asn 20:1
Наконец, Асенефь сказала:
Пойди сюда, господин мой, взойди в наш дом; ибо я убрала наш дом и приготовила
великолепную вечерю.
\vs Asn 20:2
Она взяла его за руку правую
и ввела в дом свой, посадила на седалище отца своего и принесла воды, чтобы
омыть ноги его.
\vs Asn 20:3
И Иосиф сказал ей: Пусть
придёт одна из дев и умоет мои ноги.
\vs Asn 20:4
И отвечала Асенефь: Нет,
господин мой, отныне я раба твоя. С чего ты взял, что другая будет умывать
ноги твои? Ноги твои мои ноги, тело твоё моё тело.
\vs Asn 20:5
И она настояла на своём и
омыла ему ноги.
\vs Asn 20:6
И посмотрел Иосиф на её руки
и не мог наглядеться на их жизненность: пальцы у неё ходили, как у скорописца.
\vs Asn 20:7
Затем Асенефь, взяв его за
правую руку, облобызала её; а он поцеловал её в голову. И она села по правую
руку его.
\vs Asn 20:8
И пришли отец её и мать с
поля наследия своего, пришли и все сродники её и увидели Асенефь, как бы
окружённую светом (красота её была словно небесная), сидящую с Иосифом и
одетую в ризу брачную.
\vs Asn 20:9
При виде этого они
ужаснулись, поражённые её красотой, и они воздали славу Богу, который
животворит все.
\vs Asn 20:10
После этого они ели, и пили, и веселились.
\vs Asn 20:11
И сказал Потифер Иосифу:
Завтра ты пригласишь сановников и вельмож египетских, и я устрою свадьбу
вашу, и ты возьмёшь дочь мою Асенефь себе в жёны.
\vs Asn 20:12
И ответил Иосиф: Нет,
прежде я отправлюсь к фараону; ибо он для меня как отец, он меня поставил
князем над этой страной. Поведаю его слуху об Асенефи,
и он отдаст мне Асенефь в жёны.
\vs Asn 20:13
На это Потифер ответил:
Ступай с миром.
\vs Asn 20:14
Иосиф остался тот день у
Потифера и не вошёл к Асенефи;
\vs Asn 20:15
ибо говорил: Не подобает
мужу богобоязненному прежде брака почивать с женою своею.

\vs Asn 21:1
На утро Иосиф отправился к
фараону и сказал ему:
Отдай мне Асенефь, дочь Потифера,
илиопольского жреца, в жёны.
\vs Asn 21:2
Фараон сказал: Ведь она
твоя невеста и с давних пор обручена.
\vs Asn 21:3
И он послал вестников за
Потифером, который пришёл с Асенефью, и представил её перед фараона.
\vs Asn 21:4
И изумился фараон при виде
красоты её и сказал:
Да благословит тебя, дитя моё,
Бог Иосифа, избравшего тебя в невесту себе!
Да не покинет тебя красота твоя!
\vs Asn 21:5
Справедлив Господь,
избравший тебя для Иосифа, и как говорится, отныне наречёшься ты дочерью
Всевышнего, и будет тебе Иосиф женихом навеки.
\vs Asn 21:6
И фараон возложил на Иосифа
и Асенефь золотые венцы, взятые из царской сокровищницы.
\vs Asn 21:7
И фараон, поставив Асенефь
по правую руку Иосифа, возложил на их головы свои руки, правую руку на голову
Асенефь,
\vs Asn 21:8
и сказал: Да благословит
вас Всевышний Бог, и да прославит на вечные времена.
\vs Asn 21:9
И фараон повернул их лицом к
лицу, подвинул их близко и принудил лобызаться.
\vs Asn 21:10
После сего фараон сотворил брак,
устроил роскошную вечерю и пир, и винопитие великое в продолжение 7-ми
дней; и были приглашены все князья египетские, вельможи,
все цари соседних народов.
\vs Asn 21:11
И приказал царь возвестить
по всей земле Египетской, что если кто в продолжение 7-ми дней бракосочетания
Иосифа и Асенефи будет работать, тот умрёт горькою смертью.
\vs Asn 21:12
И было после сего, когда
совершилось торжество брачное и закончилось пиршество, Иосиф вошёл к Асенефи,
она зачала и родила Манассию в доме Иосифа.

\vs Asn 22:1
После этого прошло 7 лет изобилия, и настало 7 лет голода.
\vs Asn 22:2
И услышал Иаков о сыне своём Иосифе,
и прибыл в Египет со всеми сродниками своими на 2-ом году голода, в
21-ый день месяца нисана, и поселился в земле Гесем.
\vs Asn 22:3
И сказала Асенефь Иосифу:
Пойду я посетить отца твоего, ибо отец твой Израиль для меня как Бог.
\vs Asn 22:4
И сказал Иосиф: Ты пойдёшь со мною и увидишь отца моего.
\vs Asn 22:5
И пришли Иосиф и Асенефь в
страну Гесем; и встретились ему братья его,
и пали на лицо своё и поклонились
ему, в особенности же Асенефи.
\vs Asn 22:6
И вошли они к Иакову,
который сидел на ложе своём: и был он сед и очень стар.
\vs Asn 22:7
И Асенефь сильно поразил вид
его; ибо Иаков, несмотря на седину, был благовиден, как прекрасный юноша;
\vs Asn 22:8
глава его была бела как
снег, кудрявые волосы его были густы, как у хушитянина;
белая красивая борода его покрывала всю грудь его;
\vs Asn 22:9
глаза его веселы,
блестящие и красивые; грудь его, и плечи, и мышцы, и пальцы на руках как у
сильного ангела; бёдра и голени его как у нефилима.
\vs Asn 22:10
И представлялся Иаков как
муж боговидный.
\vs Asn 22:11
Асенефь, видя его, пришла в
ужас и пала пред ним на землю на лицо своё.
\vs Asn 22:12
И спросил Иаков: Эта ли
моя невестка, жена твоя? Да благословит её Всевышний Бог!
\vs Asn 22:13
И, подозвав её к себе,
облобызал и благословил её; также и Асенефь простёрла руки свои, и обвила ими
шею Иакова и повисла на раменах отца мужа своего, как бы возвратившегося с войны
целым и невредимым, и лобызала его.
\vs Asn 22:14
После того ели они и пили.
\vs Asn 22:15
И отправились Иосиф и
Асенефь в дом свой, и взяли с собою Симеона и Левия.
\vs Asn 22:16
Но не все сыновья Зелфы и
Валлы, Лии и Рахили провожали их, потому что завидовали им и были их врагами.
\vs Asn 22:17
С правой стороны Асенефи
шел Левий, а с левой Симеон. И Асенефь держала за руку Левия,
\vs Asn 22:18
ибо более всех братьев
Иосифа Асенефь возлюбила Левия, как мужа пророчествующего, и благочестивого, и
богобоязненного.
\vs Asn 22:19
Потому что он читал
письмена, написанные на небесах, и разумел их, и знал тайны Всевышнего, которые
он открывал ей в словах таинственных.
\vs Asn 22:20
И Левий сильно любил
Асенефь. Он видел место её упокоения в вышних, и окружавшие его стены были
словно адамантовые, и основания его как каменные основы 3-го неба.
\vs Asn 23:1
И было, когда возвращались
Иосиф и Асенефь путем своим, первородный сын фараона увидел её с высоты стены и
сильно возмутился духом при виде великой красоты её.
\vs Asn 23:2
И сказал он: Нет, не быть этому!
\vs Asn 23:3
И сын фараона отправил
гонцов призвать к себе Симеона и Левия, которые, пришедши, предстали
пред лицом его.
\vs Asn 23:4
И сказал им первородный сын
фараона: Я знаю, что вы силою превосходите всех людей на земле: вашею десницею
был сокрушён город Сихем, и 2-мя мечами вашими были истреблены 30000
мужей воинственных.
\vs Asn 23:5
И вот призываю я вас,
придите на помощь мне! Вот я приму вас в товарищи себе: дам я вам много золота
и серебра, и рабов, и рабынь, и дома, и большие уделы, и богатства;
только помогите мне, исполните это моё слово.
\vs Asn 23:6
Окажите мне милость, ибо я
поруган братом вашим Иосифом: он отнял у меня Асенефь, которая первоначально
была обручена со мною.
\vs Asn 23:7
И ныне будьте со мною,
воздвигните брань на брата вашего Иосифа, тогда я убью его мечом своим и возьму
Асенефь себе в жену;
\vs Asn 23:8
этим вы докажете верность
вашу и будете мне братьями и друзьями даже до конца, исполните только это моё
слово немедленно.
\vs Asn 23:9
Но если, выслушав моё
предложение, вы пренебрежёте им~--- знайте, что вас ожидает этот меч.
\vs Asn 23:10
Говоря это, обнажил он меч свой и показал им.
\vs Asn 23:11
Услышав такую надменную
речь из уст сына фараонова, Симеон и Левий пришли в негодование.
\vs Asn 23:12
Симеон же был муж смелый и
решительный; он готов был схватиться за рукоятку своего меча, вынуть его из
ножен и поразить им сына фараонова за оскорбительные слова;
\vs Asn 23:13
но Левий провидел его
намерение и, как муж, одарённый даром пророчества, духовным оком прозрел, что
было изображено у него на сердце, и ногою своею наступил ему на правую ногу и
тем дал ему знак, чтобы тот укротил гнев свой.
\vs Asn 23:14
И сказал Левий сыну
фараона, и сказал смело, и без гнева, и с сердцем кротким и ликом ясным:
\vs Asn 23:15
Зачем ты, господин наш,
произносишь такие речи? Мы мужи богобоязненные, отец наш раб Бога
Всевышнего, и брат наш Иосиф возлюблен Богом;
\vs Asn 23:16
как же мы можем творить
злое дело и грешить перед Богом, перед отцом нашим Иаковом и братом нашим
Иосифом?
\vs Asn 23:17
Итак, слушай: мужу
богобоязненному ни в каком случае не подобает творить беззаконие потому только,
что имеет он в руках меч, поэтому воздержись говорить недоброе о нашем брате
Иосифе;
\vs Asn 23:18
ибо если ты будешь
упорствовать в злом твоём намерении, то вот готовы обнаженные мечи наши в
правых руках наших перед лицом твоим.
\vs Asn 23:19
С этими словами Симеон и
Левий извлекли мечи свои из ножен, и сказали: Видишь ли ты в наших руках эти
мечи?
\vs Asn 23:20
посредством этих мечей
отомстил Господь сихемлянам за оскорбление, нанесённое сынам израилевым в лице
сестры нашей Дины, которую обесчестил Сихем, сын Еммора.
\vs Asn 23:21
При виде меча обоюдоострого
сын фараонов испугался, и задрожали кости его; ибо мечи те блистали, как пламя
огня.
\vs Asn 23:22
Потемнело в глазах у сына
фараонова, и упал он на лицо своё под ноги их.
\vs Asn 23:23
Тогда Левий, протянув руки,
ухватил его и сказал: Встань, не бойся; смотри за собой; впредь остерегайся
говорить злое слово о брате нашем.
\vs Asn 23:24
И сказав это, вышли от сына
фараонова Симеон и Левий. Ужас и печаль овладели сыном фараона; ибо он боялся
Симеона.

\vs Asn 24:1
Тяжело было ему от любви к Асенефи; великая, безмерная тоска
напала на него.
\vs Asn 24:2
Тогда рабы его стали нашептывать ему, говоря: Вот, сыны Валлы
и сыны Зелфы, рабынь Лии и Рахили, жен иаковлевых, враги Иосифу и Асенефи,
которым они завидуют: они-то будут тебе покорны и исполнят твою волю.
\vs Asn 24:3
И сын фараонов послал вестников призвать их к себе. И пришли
они ночью и стали перед ним.
\vs Asn 24:4
И сказал им сын фараонов: Речь свою обращаю к вам, как к
мужам сильным.
\vs Asn 24:5
И говорят ему старшие братья Дан и Гад: Пусть говорит теперь
господин наш, что хочет, и мы, рабы твои, услышим и исполним волю твою.
\vs Asn 24:6
И сын фараонов возрадовался радостью великой и сказал своим
рабам: Отойдите от меня немного, ибо этим мужам я имею сообщить нечто втайне,
и все они отошли.
\vs Asn 24:7
И сказал сын фараонов Дану и Гаду: Перед лицом вашим
благословение и смерть; выбирайте скорее благословение, нежели смерть; вы
мужи сильные, вы не должны умереть как женщина, мужайтесь и отмстите врагам
вашим.
\vs Asn 24:8
Ибо я сам слышал, как брат ваш Иосиф говорил о вас отцу моему
фараону, что вы сыновья рабыни его матери, а не братья его: не дождусь смерти
моего отца, чтобы их истребить вместе с их родом, дабы они, дети служанки, не
могли участвовать в наследии.
\vs Asn 24:9
Это они продали меня измаильтянам, и я отплачу им за зло, мне
причинённое: пусть только умрёт отец мой.
\vs Asn 24:10
И похвалил его отец мой, фараон, говоря: Ты хорошо сказал,
возьми у меня 1000 человек воинов, и выйди на них тайно, и сотвори им, как
сотворили они тебе, и я буду твоим помощником.
\vs Asn 24:11
Когда же те мужи услышали
слова сына фараонова, возмутились духом, и опечалились, и сказали сыну
фараонову: Просим тебя, господин, помоги нам.
\vs Asn 24:12
И тот в ответ: Я помогу
вам, если вы послушаетесь меня.
\vs Asn 24:13
И они сказали: Стоим перед
тобою мы, рабы твои; прикажи, и мы исполним твою волю.
\vs Asn 24:14
И говорит им фараонов сын:
Нынешнею ночью я убью отца моего, потому что фараон стал отцом Иосифа;
\vs Asn 24:15
вы же убейте Иосифа, и я
возьму себе в жёны Асенефь, а вы и братья ваши будете моими сонаследниками
исполните только моё слово.
\vs Asn 24:16
И сказали ему Дан и Гад:
мы рабы твои: сегодня же исполним твоё приказание;
\vs Asn 24:17
ибо ныне мы слышали, как
Иосиф говорил Асенефи: пойди завтра в поле наследия нашего, ибо настало время
сбора винограда.
\vs Asn 24:18
600 сильных воинов
будут сопровождать её и 60 ей предшествовать. Теперь послушай, что мы
тебе скажем. И открыли они ему свои мысли.
\vs Asn 24:19
И сын фараонов дал каждому
из 4-ёх мужей по 500 воинов, назначив их князьями и начальниками.
\vs Asn 24:20
И говорят ему Дан и Гад:
Нынешнею ночью мы пойдем и сядем в засаде у потока в зарослях тростника;
\vs Asn 24:21
ты же возьми с собой 50 стрелков всадников и поезжай вперед.
\vs Asn 24:22
Как только покажется
Асенефь, мы предадим мечу воинов, сопровождающих её, тогда бросится она на
колеснице вперёд, и попадёт тебе в руки Асенефь, и сотворишь с ней, как хочет
душа твоя.
\vs Asn 24:23
После этого мы умертвим
Иосифа, погружённого в печаль, и сыновей его пред глазами его.
\vs Asn 24:24
И обрадовался сын фараонов,
услышав слова сии, и дал им 2000 воинов.
\vs Asn 24:25
И пришли они к потоку, и
укрылись в зарослях тростника, и 500 засели впереди, и заняли широкую
переправу с той и с другой стороны потока.

\vs Asn 25:1
И сын фараонов встал и отправился в ту ночь в дом отца своего.
\vs Asn 25:2
И пришёл сын фараонов к ложу
отца своего, чтобы убить его мечом; но телохранители не позволили ему доступ к
отцу его и спросили: Какая тебе надобность, господин?
\vs Asn 25:3
И отвечал сын фараона: Хочу
видеть отца моего, так как отправляюсь на сбор винограда в новый виноградник.
\vs Asn 25:4
И сказали ему телохранители:
Страданием страдает отец твой, всю ночь не мог заснуть, и ныне немного
успокоился. Приказал он никого не впускать.
\vs Asn 25:5
И отошёл он в ярости, и
отправился сын фараонов к своим воинам и при рассвете засел в засаде, как
посоветовали ему Дан и Гад.
\vs Asn 25:6
Услышав об этом, сыновья
Иакова, Неффалим и Асир, младшие братья, сказали Дану и Гаду: За что вы
задумываете ещё злые козни против нашего отца, Израиля, и брата нашего, Иосифа?
\vs Asn 25:7
Разве Господь не бережёт его
как зеницу ока? Не вы ли некогда продали его?
\vs Asn 25:8
А ныне он царствует над
страной, он раздаёт по доброй воле пшеницу, которой питается народ, он спасает
многим жизнь.
\vs Asn 25:9
Если вы сегодня попытаетесь
причинить ему зло, то умолит он Бога Израилева и появится на небе и настигнет
вас огонь, который пожрёт вас, и ангелы Божьи будут сражаться за него и явятся к
нему на помощь.
\vs Asn 25:10
Дан и Гад разгневались на
своих братьев и сказали им: В противном случае мы умрём как женщины!
Да не будет того.
\vs Asn 25:11
И вышли навстречу Иосифу и Асенефь.

\vs Asn 26:1
На рассвете Асенефь встала и
сказала Иосифу: Как ты сказал, я отправлюсь в поле наследия нашего для сбора
винограда; но я боюсь, как бы кто-нибудь, придя, не похитил бы меня у тебя.
\vs Asn 26:2
И сказал ей Иосиф: Мужайся,
не бойся ничего, но спеши идти; Господь будет с тобою, и он убережёт тебя как
зеницу ока и сохранит тебя от всякого зла.
\vs Asn 26:3
Я же пойду на труд свой и
раздавать буду хлебные припасы в городе, чтобы кормить народ и принять меры,
дабы от голода никто не погиб в стране.
\vs Asn 26:4
И отошла Асенефь своим путем;
и Иосиф отошёл на труд свой и раздавал хлеб.
\vs Asn 26:5
И приблизилась Асенефь в
сопровождении 600 воинов к месту, где была ложбина.
\vs Asn 26:6
И внезапно вышли из засады воины Дана и Гада
и напали на воинов Асенефи, и завязался бой с сильными Асенефи,
\vs Asn 26:7
и убили из них около 50 всадников,
ехавших впереди, а Асенефь бежала на своей колеснице.
\vs Asn 26:8
И Левий узнал всё сие, и
известил своих братьев, сынов Лии, об измене.
\vs Asn 26:9
И каждый из них обнажил меч
свой при бедре своём, и щит свой на плече своём, и взял копье в правую руку. И
побежали они поспешно вслед Асенефи.

\vs Asn 26:10
И Асенефь бежала, и вот, сын фараона, сопровождаемый
50-ью всадниками, навстречу ей.
\vs Asn 26:11
И увидела его Асенефь, и испугалась, и вострепетала. Тогда
призвала она имя Господа, Бога Всевышнего.
\vs Asn 27:1
И Вениамин был с нею в её колеснице; и был он отрок сильный,
красивый, богобоязненный и весьма храбрый.
\vs Asn 27:2
И он сошел с колесницы, и
набрал у потока полные руки гладких камней, и бросил их в сына фараонова, и
поразил его в левый висок, и причинил ему жестокую рану, и сын фараона упал с
коня своего и лежал на земле.
\vs Asn 27:3
И после того Вениамин
поднялся поспешно на высокую скалу и сказал вознице Асенефи: Достань мне
гладких камней из потока.
\vs Asn 27:4
И тот достал ему 48 гладких камней.
И бросил те камни Вениамин, и убил 48 мужей,
сопровождавших сына фараонова.

\vs Asn 27:5
И сыны Лии Рувим, Симеон,
Левий, Иуда, Иссахар и Завулон погнались за мужами, сидевшими в засаде в
кустах, и напали на них неожиданно; и шестеро их убили их всех.
\vs Asn 27:6
И братья их, Дан и Гад,
сыновья Валлы и Зелфы, убежали при виде их, говоря:
\vs Asn 27:7
Мы не устояли перед нашими
братьями, и сын фараона побеждён и ранен смертельно Вениамином,
и все, бывшие с ним, погибли от руки его.
Пойдём же, убьём Асенефь, и скроемся в зарослях тростника.
\vs Asn 27:8
И пришли они, держа в руке
мечи свои обнаженными и полными крови.
\vs Asn 27:9
И увидела их Асенефь и
сказала: Господь, Бог мой, ты, который спас меня от смерти и который сказал
мне: живи вовеки! избавь меня от меча этих нечестивых мужей.
\vs Asn 27:10
И внял Бог гласу её, и
тотчас мечи их выпали из рук их на землю и рассыпались в прах.
\vs Asn 28:1
Видя это, сыны Валлы и Зелфы
устрашились и сказали: Истинно Господь воюет против нас за Асенефь.
\vs Asn 28:2
И пали на лица свои на землю
и бросились к ногам Асенефи и сказали ей:
\vs Asn 28:3
Ты наша госпожа и царица;
мы согрешили пред тобою, и Бог воздал нам по делам нашим.
\vs Asn 28:4
Мы, рабы твои, умоляем тебя,
помилуй нас и спаси нас от рук братьев наших, ибо они идут отмстить нам и мечи
их изострены на нас.
\vs Asn 28:5
И сказала Асенефь:
Мужайтесь и не страшитесь братьев ваших, ибо они мужи богобоязненные; идите в
эти тростниковые заросли, пока не умолю я за вас и не усмирю гнева их; ибо
велика дерзость ваша против них.
\vs Asn 28:6
Мужайтесь и не бойтесь, и да
рассудит Господь между мною и вами! И убежали в заросли тростника Дан и Гад.
\vs Asn 28:7
И вот, прибежали сыновья
Левия, подобно стаду оленей, и сошла Асенефь с закрытой колесницы своей и
встретила их со слезами.
\vs Asn 28:8
Они же, пав на землю,
поклонились ей, и плакали громко, и искали братьев своих.
\vs Asn 28:9
И сказала Асенефь: Пощадите
братьев ваших и не делайте им зла, ибо Господь явился мне защитником против них
и сокрушил мечи их, и, как воск от огня, растаяли они на земле.
\vs Asn 28:10
И этого будет с них, ибо
Господь воюет против них, а вы пощадите их, ведь они братья ваши и кровь отца
вашего Израиля.
\vs Asn 28:11
И сказал ей Симеон: Зачем
госпожа наша говорит доброе о врагах наших? Нет, мы перебьём их мечами нашими,
так как они замышляли об отце нашем Израиле,
и о брате нашем Иосифе, уже 2-жды;
а ныне и на тебя
\vs Asn 28:12
И, простёрши руку свою,
Асенефь коснулась бороды его и поцеловала её, говоря: Ты никогда этого не
сделаешь, брат мой, и не отплатишь злом за зло ближнему своему; ибо Господь
судит обиду сию; а ведь они братья ваши и чада отца вашего и убежали от лица
вашего.
\vs Asn 28:13
И преклонился Симеон пред
Асенефь. И подошёл к ней Левий, и облобызал ей руки, и благословил её; и понял
он, что она желает спасти братьев его.
\vs Asn 28:14
И находились они в зарослях
тростника; и узнал он это от братьев их, но не дал знать им о том, ибо опасался,
как бы они в пылу не перебили их.
\vs Asn 29:1
И сын фараонов поднялся от
земли и сел, выплевывая кровь из уст своих, ибо кровь текла из виска его в уста
его.
\vs Asn 29:2
И подбежал к нему Вениамин,
и взял меч его, и вынул его из ножен (ибо не носил Вениамин меча при бедре
своём) и хотел убить его и поразить сына фараона в грудь.
\vs Asn 29:3
И подошел к нему Левий, и
взял его за руку, и сказал: Брат мой, не делай этого, ведь мы мужи
богобоязненные и не подобает мужу богобоязненному воздавать злом за зло, ни
попирать поверженного или добивать до смерти попавшего в руки врага.
\vs Asn 29:4
И теперь вложи свой меч в
ножны и помоги мне обвязать раны его, и если жив будет, сделается нашим другом;
как и фараон нам как отец.
\vs Asn 29:5
И поднял Левий сына
фараонова, и отер кровь с лица его, и наложил повязку на рану его, и принял его
на коня своего, и повёз его к отцу его, и рассказал ему обо всем случившемся.

\vs Asn 29:6
И поднялся фараон с престола своего и поклонился Левию.
\vs Asn 29:7
И на 3-ий день умер сын
фараонов от раны, причинённой камнем Вениаминовым.
\vs Asn 29:8
И оплакивал фараон сына
своего первородного, и впал от печали в недуг.
\vs Asn 29:9
И умер фараон, имея 109 лет от роду, и оставил диадему свою Иосифу.

\vs Asn 29:10
И царствовал Иосиф в земле
египетской 48 лет, а после того предал Иосиф диадему внуку фараона.
\vs Asn 29:11
И был Иосиф в земле
египетской, как отец его.
\vs Asn 29:12
И так хранил его Бог от
нежной юности даже до конца жизни его, ибо был он семенем избранных мужей
праведных, Авраама, Исаака и Иакова, и молитвы их шли перед ним;
\vs Asn 29:13
и солнце и звёзды
преклонились пред Иосифом, знаменуя, что быть ему царем.

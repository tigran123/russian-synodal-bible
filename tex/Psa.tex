\bibbookdescr{Psa}{
  inline={Псалтирь\fns{У Евреев: <<Книга Хвалений>>.}},
  toc={Псалтирь},
  bookmark={Псалтирь},
  header={Псалтирь},
  %headerleft={},
  %headerright={},
  abbr={Пс}
}
\vs Psa 1:0 Псалом Давида.
\rsbpar\vs Psa 1:1 Блажен муж, который не ходит на совет нечестивых и не стоит на пути грешных и не сидит в собрании развратителей,
\vs Psa 1:2 но в законе Господа воля его, и о законе Его размышляет он день и ночь!
\vs Psa 1:3 И будет он как дерево, посаженное при потоках вод, которое приносит плод свой во время свое, и лист которого не вянет; и во всем, что он ни делает, успеет.
\vs Psa 1:4 Не так~--- нечестивые, [не так]: но они~--- как прах, возметаемый ветром [с лица земли].
\vs Psa 1:5 Потому не устоят\fns{В славянском переводе: Сего ради не воскреснут\dots} нечестивые на суде, и грешники~--- в собрании праведных.
\vs Psa 1:6 Ибо знает Господь путь праведных, а путь нечестивых погибнет.
\vs Psa 2:0 Псалом Давида.
\rsbpar\vs Psa 2:1 Зачем мятутся народы, и племена замышляют тщетное?
\vs Psa 2:2 Восстают цари земли, и князья совещаются вместе против Господа и против Помазанника Его.
\vs Psa 2:3 <<Расторгнем узы их, и свергнем с себя оковы их>>.
\vs Psa 2:4 Живущий на небесах посмеется, Господь поругается им.
\vs Psa 2:5 Тогда скажет им во гневе Своем и яростью Своею приведет их в смятение:
\vs Psa 2:6 <<Я помазал Царя Моего над Сионом, святою горою Моею\fns{6-й стих по переводу 70-ти: Я поставлен от Него Царем над Сионом, святою горою Его.};
\vs Psa 2:7 возвещу определение: Господь сказал Мне: Ты Сын Мой; Я ныне родил Тебя;
\vs Psa 2:8 проси у Меня, и дам народы в наследие Тебе и пределы земли во владение Тебе;
\vs Psa 2:9 Ты поразишь их жезлом железным; сокрушишь их, как сосуд горшечника>>.
\vs Psa 2:10 Итак вразумитесь, цари; научитесь, судьи земли!
\vs Psa 2:11 Служите Господу со страхом и радуйтесь [пред Ним] с трепетом.
\vs Psa 2:12 Почтите Сына, чтобы Он не прогневался, и чтобы вам не погибнуть в пути \bibemph{вашем}, ибо гнев Его возгорится вскоре. Блаженны все, уповающие на Него.
\vs Psa 3:1 Псалом Давида, когда он бежал от Авессалома, сына своего.
\rsbpar\vs Psa 3:2 Господи! как умножились враги мои! Многие восстают на меня;
\vs Psa 3:3 многие говорят душе моей: <<нет ему спасения в Боге>>.
\vs Psa 3:4 Но Ты, Господи, щит предо мною, слава моя, и Ты возносишь голову мою.
\vs Psa 3:5 Гласом моим взываю к Господу, и Он слышит меня со святой горы Своей.
\vs Psa 3:6 Ложусь я, сплю и встаю, ибо Господь защищает меня.
\vs Psa 3:7 Не убоюсь тем народа, которые со всех сторон ополчились на меня.
\vs Psa 3:8 Восстань, Господи! спаси меня, Боже мой! ибо Ты поражаешь в ланиту всех врагов моих; сокрушаешь зубы нечестивых.
\vs Psa 3:9 От Господа спасение. Над народом Твоим благословение Твое.
\vs Psa 4:1 Начальнику хора. На струнных \bibemph{орудиях}. Псалом Давида.
\rsbpar\vs Psa 4:2 Когда я взываю, услышь меня, Боже правды моей! В тесноте Ты давал мне простор. Помилуй меня и услышь молитву мою.
\vs Psa 4:3 Сыны мужей! доколе слава моя будет в поругании? доколе будете любить суету и искать лжи?
\vs Psa 4:4 Знайте, что Господь отделил для Себя святаго Своего; Господь слышит, когда я призываю Его.
\vs Psa 4:5 Гневаясь, не согрешайте: размыслите в сердцах ваших на ложах ваших, и утишитесь;
\vs Psa 4:6 приносите жертвы правды и уповайте на Господа.
\vs Psa 4:7 Многие говорят: <<кто покажет нам благо?>> Яви нам свет лица Твоего, Господи!
\vs Psa 4:8 Ты исполнил сердце мое веселием с того времени, как у них хлеб и вино [и елей] умножились.
\vs Psa 4:9 Спокойно ложусь я и сплю, ибо Ты, Господи, един даешь мне жить в безопасности.
\vs Psa 5:1 Начальнику хора. На духовых \bibemph{орудиях}. Псалом Давида.
\rsbpar\vs Psa 5:2 Услышь, Господи, слова мои, уразумей помышления мои.
\vs Psa 5:3 Внемли гласу вопля моего, Царь мой и Бог мой! ибо я к Тебе молюсь.
\vs Psa 5:4 Господи! рано услышь голос мой,~--- рано предстану пред Тобою, и буду ожидать,
\vs Psa 5:5 ибо Ты Бог, не любящий беззакония; у Тебя не водворится злой;
\vs Psa 5:6 нечестивые не пребудут пред очами Твоими: Ты ненавидишь всех, делающих беззаконие.
\vs Psa 5:7 Ты погубишь говорящих ложь; кровожадного и коварного гнушается Господь.
\vs Psa 5:8 А я, по множеству милости Твоей, войду в дом Твой, поклонюсь святому храму Твоему в страхе Твоем.
\vs Psa 5:9 Господи! путеводи меня в правде Твоей, ради врагов моих; уровняй предо мною путь Твой.
\vs Psa 5:10 Ибо нет в устах их истины: сердце их~--- пагуба, гортань их~--- открытый гроб, языком своим льстят.
\vs Psa 5:11 Осуди их, Боже, да падут они от замыслов своих; по множеству нечестия их, отвергни их, ибо они возмутились против Тебя.
\vs Psa 5:12 И возрадуются все уповающие на Тебя, вечно будут ликовать, и Ты будешь покровительствовать им; и будут хвалиться Тобою любящие имя Твое.
\vs Psa 5:13 Ибо Ты благословляешь праведника, Господи; благоволением, как щитом, венчаешь его.
\vs Psa 6:1 Начальнику хора. На восьмиструнном. Псалом Давида.
\rsbpar\vs Psa 6:2 Господи! не в ярости Твоей обличай меня и не во гневе Твоем наказывай меня.
\vs Psa 6:3 Помилуй меня, Господи, ибо я немощен; исцели меня, Господи, ибо кости мои потрясены;
\vs Psa 6:4 и душа моя сильно потрясена; Ты же, Господи, доколе?
\vs Psa 6:5 Обратись, Господи, избавь душу мою, спаси меня ради милости Твоей,
\vs Psa 6:6 ибо в смерти нет памятования о Тебе: во гробе кто будет славить Тебя?
\vs Psa 6:7 Утомлен я воздыханиями моими: каждую ночь омываю ложе мое, слезами моими омочаю постель мою.
\vs Psa 6:8 Иссохло от печали око мое, обветшало от всех врагов моих.
\vs Psa 6:9 Удалитесь от меня все, делающие беззаконие, ибо услышал Господь голос плача моего,
\vs Psa 6:10 услышал Господь моление мое; Господь примет молитву мою.
\vs Psa 6:11 Да будут постыжены и жестоко поражены все враги мои; да возвратятся и постыдятся мгновенно.
\vs Psa 7:1 Плачевная песнь, которую Давид воспел Господу по делу Хуса, из племени Вениаминова.
\rsbpar\vs Psa 7:2 Господи, Боже мой! на Тебя я уповаю; спаси меня от всех гонителей моих и избавь меня;
\vs Psa 7:3 да не исторгнет он, подобно льву, души моей, терзая, когда нет избавляющего [и спасающего].
\vs Psa 7:4 Господи, Боже мой! если я что сделал, если есть неправда в руках моих,
\vs Psa 7:5 если я платил злом тому, кто был со мною в мире,~--- я, который спасал даже того, кто без причины стал моим врагом,~---
\vs Psa 7:6 то пусть враг преследует душу мою и настигнет, пусть втопчет в землю жизнь мою, и славу мою повергнет в прах.
\vs Psa 7:7 Восстань, Господи, во гневе Твоем; подвигнись против неистовства врагов моих, пробудись для меня на суд, который Ты заповедал,~---
\vs Psa 7:8 сонм людей станет вокруг Тебя; над ним поднимись на высоту.
\vs Psa 7:9 Господь судит народы. Суди меня, Господи, по правде моей и по непорочности моей во мне.
\vs Psa 7:10 Да прекратится злоба нечестивых, а праведника подкрепи, ибо Ты испытуешь сердца и утробы, праведный Боже!
\vs Psa 7:11 Щит мой в Боге, спасающем правых сердцем.
\vs Psa 7:12 Бог~--- судия праведный, [крепкий и долготерпеливый,] и Бог, всякий день строго взыскивающий,
\vs Psa 7:13 если \bibemph{кто} не обращается. Он изощряет Свой меч, напрягает лук Свой и направляет его,
\vs Psa 7:14 приготовляет для него сосуды смерти, стрелы Свои делает палящими.
\vs Psa 7:15 Вот, \bibemph{нечестивый} зачал неправду, был чреват злобою и родил себе ложь;
\vs Psa 7:16 рыл ров, и выкопал его, и упал в яму, которую приготовил:
\vs Psa 7:17 злоба его обратится на его голову, и злодейство его упадет на его темя.
\vs Psa 7:18 Славлю Господа по правде Его и пою имени Господа Всевышнего.
\vs Psa 8:1 Начальнику хора. На Гефском \bibemph{орудии}. Псалом Давида.
\rsbpar\vs Psa 8:2 Господи, Боже наш! как величественно имя Твое по всей земле! Слава Твоя простирается превыше небес!
\vs Psa 8:3 Из уст младенцев и грудных детей Ты устроил хвалу, ради врагов Твоих, дабы сделать безмолвным врага и мстителя.
\vs Psa 8:4 Когда взираю я на небеса Твои~--- дело Твоих перстов, на луну и звезды, которые Ты поставил,
\vs Psa 8:5 то чт\acc{о} \bibemph{есть} человек, что Ты помнишь его, и сын человеческий, что Ты посещаешь его?
\vs Psa 8:6 Не много Ты умалил его пред Ангелами: славою и честью увенчал его;
\vs Psa 8:7 поставил его владыкою над делами рук Твоих; всё положил под ноги его:
\vs Psa 8:8 овец и волов всех, и также полевых зверей,
\vs Psa 8:9 птиц небесных и рыб морских, все, преходящее морскими стезями.
\vs Psa 8:10 Господи, Боже наш! Как величественно имя Твое по всей земле!
\vs Psa 9:1 Начальнику хора. По смерти Лабена. Псалом Давида.
\rsbpar\vs Psa 9:2 Буду славить [Тебя], Господи, всем сердцем моим, возвещать все чудеса Твои.
\vs Psa 9:3 Буду радоваться и торжествовать о Тебе, петь имени Твоему, Всевышний.
\vs Psa 9:4 Когда враги мои обращены назад, то преткнутся и погибнут пред лицем Твоим,
\vs Psa 9:5 ибо Ты производил мой суд и мою тяжбу; Ты воссел на престоле, Судия праведный.
\vs Psa 9:6 Ты вознегодовал на народы, погубил нечестивого, имя их изгладил на веки и веки.
\vs Psa 9:7 У врага совсем не стало оружия, и город\acc{а} Ты разрушил; погибла память их с ними.
\vs Psa 9:8 Но Господь пребывает вовек; Он приготовил для суда престол Свой,
\vs Psa 9:9 и Он будет судить вселенную по правде, совершит суд над народами по правоте.
\vs Psa 9:10 И будет Господь прибежищем угнетенному, прибежищем во времена скорби;
\vs Psa 9:11 и будут уповать на Тебя знающие имя Твое, потому что Ты не оставляешь ищущих Тебя, Господи.
\vs Psa 9:12 Пойте Господу, живущему на Сионе, возвещайте между народами дела Его,
\vs Psa 9:13 ибо Он взыскивает за кровь; помнит их, не забывает вопля угнетенных.
\vs Psa 9:14 Помилуй меня, Господи; воззри на страдание мое от ненавидящих меня,~--- Ты, Который возносишь меня от врат смерти,
\vs Psa 9:15 чтобы я возвещал все хвалы Твои во вратах дщери Сионовой: буду радоваться о спасении Твоем.
\vs Psa 9:16 Обрушились народы в яму, которую выкопали; в сети, которую скрыли они, запуталась нога их.
\vs Psa 9:17 Познан был Господь по суду, который Он совершил; нечестивый уловлен делами рук своих.
\vs Psa 9:18 Да обратятся нечестивые в ад,~--- все народы, забывающие Бога.
\vs Psa 9:19 Ибо не навсегда забыт будет нищий, и надежда бедных не до конца погибнет.
\vs Psa 9:20 Восстань, Господи, да не преобладает человек, да судятся народы пред лицем Твоим.
\vs Psa 9:21 Наведи, Господи, страх на них; да знают народы, что человеки они.
\vs Psa 9:22 Для чего, Господи, стоишь вдали, скрываешь Себя во время скорби?
\vs Psa 9:23 По гордости своей нечестивый преследует бедного: да уловятся они ухищрениями, которые сами вымышляют.
\vs Psa 9:24 Ибо нечестивый хвалится похотью души своей; корыстолюбец ублажает себя.
\vs Psa 9:25 В надмении своем нечестивый пренебрегает Господа: <<не взыщет>>; во всех помыслах его: <<нет Бога!>>
\vs Psa 9:26 Во всякое время пути его гибельны; суды Твои далеки для него; на всех врагов своих он смотрит с пренебрежением;
\vs Psa 9:27 говорит в сердце своем: <<не поколеблюсь; в род и род не приключится \bibemph{мне} зла>>;
\vs Psa 9:28 уста его полны проклятия, коварства и лжи; под языком~--- его мучение и пагуба;
\vs Psa 9:29 сидит в засаде за двором, в потаенных местах убивает невинного; глаза его подсматривают за бедным;
\vs Psa 9:30 подстерегает в потаенном месте, как лев в логовище; подстерегает в засаде, чтобы схватить бедного; хватает бедного, увлекая в сети свои;
\vs Psa 9:31 сгибается, прилегает,~--- и бедные падают в сильные когти его;
\vs Psa 9:32 говорит в сердце своем: <<забыл Бог, закрыл лице Свое, не увидит никогда>>.
\vs Psa 9:33 Восстань, Господи, Боже [мой], вознеси руку Твою, не забудь угнетенных [Твоих до конца].
\vs Psa 9:34 Зачем нечестивый пренебрегает Бога, говоря в сердце своем: <<Ты не взыщешь>>?
\vs Psa 9:35 Ты видишь, ибо Ты взираешь на обиды и притеснения, чтобы воздать Твоею рукою. Тебе предает себя бедный; сироте Ты помощник.
\vs Psa 9:36 Сокруши мышцу нечестивому и злому, так чтобы искать и не найти его нечестия.
\vs Psa 9:37 Господь~--- царь на веки, навсегда; исчезнут язычники с земли Его.
\vs Psa 9:38 Господи! Ты слышишь желания смиренных; укрепи сердце их; открой ухо Твое,
\vs Psa 9:39 чтобы дать суд сироте и угнетенному, да не устрашает более человек на земле.
\vs Psa 10:0 Начальнику хора. Псалом Давида.
\rsbpar\vs Psa 10:1 На Господа уповаю; как же вы говорите душе моей: <<улетай на гору вашу, \bibemph{как} птица>>?
\vs Psa 10:2 Ибо вот, нечестивые натянули лук, стрелу свою приложили к тетиве, чтобы во тьме стрелять в правых сердцем.
\vs Psa 10:3 Когда разрушены основания, что сделает праведник?
\vs Psa 10:4 Господь во святом храме Своем, Господь,~--- престол Его на небесах, очи Его зрят [на нищего]; вежды Его испытывают сынов человеческих.
\vs Psa 10:5 Господь испытывает праведного, а нечестивого и любящего насилие ненавидит душа Его.
\vs Psa 10:6 Дождем прольет Он на нечестивых горящие угли, огонь и серу; и палящий ветер~--- их доля из чаши;
\vs Psa 10:7 ибо Господь праведен, любит правду; лице Его видит праведника.
\vs Psa 11:1 Начальнику хора. На восьмиструнном. Псалом Давида.
\rsbpar\vs Psa 11:2 Спаси [меня], Господи, ибо не стало праведного, ибо нет верных между сынами человеческими.
\vs Psa 11:3 Ложь говорит каждый своему ближнему; уста льстивы, говорят от сердца притворного.
\vs Psa 11:4 Истребит Господь все уста льстивые, язык велеречивый,
\vs Psa 11:5 \bibemph{тех}, которые говорят: <<языком нашим пересилим, уста наши с нами; кто нам господин>>?
\vs Psa 11:6 Ради страдания нищих и воздыхания бедных ныне восстану, говорит Господь, поставлю в безопасности того, кого уловить хотят.
\vs Psa 11:7 Слова Господни~--- слова чистые, серебро, очищенное от земли в горниле, семь раз переплавленное.
\vs Psa 11:8 Ты, Господи, сохранишь их, соблюдешь от рода сего вовек.
\vs Psa 11:9 Повсюду ходят нечестивые, когда ничтожные из сынов человеческих возвысились.
\vs Psa 12:1 Начальнику хора. Псалом Давида.
\rsbpar\vs Psa 12:2 Доколе, Господи, будешь забывать меня вконец, доколе будешь скрывать лице Твое от меня?
\vs Psa 12:3 Доколе мне слагать советы в душе моей, скорбь в сердце моем день [и ночь]? Доколе врагу моему возноситься надо мною?
\vs Psa 12:4 Призри, услышь меня, Господи Боже мой! Просвети очи мои, да не усну я \bibemph{сном} смертным;
\vs Psa 12:5 да не скажет враг мой: <<я одолел его>>. Да не возрадуются гонители мои, если я поколеблюсь.
\vs Psa 12:6 Я же уповаю на милость Твою; сердце мое возрадуется о спасении Твоем; воспою Господу, облагодетельствовавшему меня, [и буду петь имени Господа Всевышнего].
\vs Psa 13:0 Начальнику хора. Псалом Давида.
\rsbpar\vs Psa 13:1 Сказал безумец в сердце своем: <<нет Бога>>. Они развратились, совершили гнусные дела; нет делающего добро.
\vs Psa 13:2 Господь с небес призрел на сынов человеческих, чтобы видеть, есть ли разумеющий, ищущий Бога.
\vs Psa 13:3 Все уклонились, сделались равно непотребными; нет делающего добро, нет ни одного.
\vs Psa 13:4 Неужели не вразумятся все, делающие беззаконие, съедающие народ мой, \bibemph{как} едят хлеб, и не призывающие Господа?
\vs Psa 13:5 Там убоятся они страха, [где нет страха,] ибо Бог в роде праведных.
\vs Psa 13:6 Вы посмеялись над мыслью нищего, что Господь упование его.
\vs Psa 13:7 <<Кто даст с Сиона спасение Израилю!>> Когда Господь возвратит пленение народа Своего, тогда возрадуется Иаков и возвеселится Израиль.
\vs Psa 14:0 Псалом Давида.
\rsbpar\vs Psa 14:1 Господи! кто может пребывать в жилище Твоем? кто может обитать на святой горе Твоей?
\vs Psa 14:2 Тот, кто ходит непорочно и делает правду, и говорит истину в сердце своем;
\vs Psa 14:3 кто не клевещет языком своим, не делает искреннему своему зла и не принимает поношения на ближнего своего;
\vs Psa 14:4 тот, в глазах которого презрен отверженный, но который боящихся Господа славит; кто клянется, \bibemph{хотя бы} злому, и не изменяет;
\vs Psa 14:5 кто серебра своего не отдает в рост и не принимает даров против невинного. Поступающий так не поколеблется вовек.
\vs Psa 15:0 Песнь Давида.
\rsbpar\vs Psa 15:1 Храни меня, Боже, ибо я на Тебя уповаю.
\vs Psa 15:2 Я сказал Господу: Ты~--- Господь мой; блага мои Тебе не нужны.
\vs Psa 15:3 К святым, которые на земле, и к дивным \bibemph{Твоим}~--- к ним все желание мое.
\vs Psa 15:4 Пусть умножаются скорби у тех, которые текут к \bibemph{богу} чужому; я не возлию кровавых возлияний их и не помяну имен их устами моими.
\vs Psa 15:5 Господь есть часть наследия моего и чаши моей. Ты держишь жребий мой.
\vs Psa 15:6 Межи мои прошли по прекрасным \bibemph{местам}, и наследие мое приятно для меня.
\vs Psa 15:7 Благословлю Господа, вразумившего меня; даже и ночью учит меня внутренность моя.
\vs Psa 15:8 Всегда видел я пред собою Господа, ибо Он одесную меня; не поколеблюсь.
\vs Psa 15:9 Оттого возрадовалось сердце мое и возвеселился язык мой; даже и плоть моя успокоится в уповании,
\vs Psa 15:10 ибо Ты не оставишь души моей в аде и не дашь святому Твоему увидеть тление,
\vs Psa 15:11 Ты укажешь мне путь жизни: полнота радостей пред лицем Твоим, блаженство в деснице Твоей вовек.
\vs Psa 16:0 Молитва Давида.
\rsbpar\vs Psa 16:1 Услышь, Господи, правду [мою], внемли воплю моему, прими мольбу из уст нелживых.
\vs Psa 16:2 От Твоего лица суд мне да изыдет; да воззрят очи Твои на правоту.
\vs Psa 16:3 Ты испытал сердце мое, посетил меня ночью, искусил меня и ничего не нашел; от мыслей моих не отступают уста мои.
\vs Psa 16:4 В делах человеческих, по слову уст Твоих, я охранял себя от путей притеснителя.
\vs Psa 16:5 Утверди шаги мои на путях Твоих, да не колеблются стопы мои.
\vs Psa 16:6 К Тебе взываю я, ибо Ты услышишь меня, Боже; приклони ухо Твое ко мне, услышь слова мои.
\vs Psa 16:7 Яви дивную милость Твою, Спаситель уповающих [на Тебя] от противящихся деснице Твоей.
\vs Psa 16:8 Храни меня, как зеницу ока; в тени крыл Твоих укрой меня
\vs Psa 16:9 от лица нечестивых, нападающих на меня,~--- от врагов души моей, окружающих меня:
\vs Psa 16:10 они заключились в туке своем, надменно говорят устами своими.
\vs Psa 16:11 На всяком шагу нашем ныне окружают нас; они устремили глаза свои, чтобы низложить \bibemph{меня} на землю;
\vs Psa 16:12 они подобны льву, жаждущему добычи, подобны скимну, сидящему в местах скрытных.
\vs Psa 16:13 Восстань, Господи, предупреди их, низложи их. Избавь душу мою от нечестивого мечом Твоим,
\vs Psa 16:14 от людей~--- рукою Твоею, Господи, от людей мира, которых удел в \bibemph{этой} жизни, которых чрево Ты наполняешь из сокровищниц Твоих; сыновья их сыты и оставят остаток детям своим.
\vs Psa 16:15 А я в правде буду взирать на лице Твое; пробудившись, буду насыщаться образом Твоим.
\vs Psa 17:1 Начальнику хора. Раба Господня Давида, который произнес слова песни сей к Господу, когда Господь избавил его от рук всех врагов его и от руки Саула. И он сказал:
\rsbpar\vs Psa 17:2 Возлюблю тебя, Господи, крепость моя!
\vs Psa 17:3 Господь~--- твердыня моя и прибежище мое, Избавитель мой, Бог мой,~--- скала моя; на Него я уповаю; щит мой, рог спасения моего и убежище мое.
\vs Psa 17:4 Призову достопоклоняемого Господа и от врагов моих спасусь.
\vs Psa 17:5 Объяли меня муки смертные, и потоки беззакония устрашили меня;
\vs Psa 17:6 цепи ада облегли меня, и сети смерти опутали меня.
\vs Psa 17:7 В тесноте моей я призвал Господа и к Богу моему воззвал. И Он услышал от [святаго] чертога Своего голос мой, и вопль мой дошел до слуха Его.
\vs Psa 17:8 Потряслась и всколебалась земля, дрогнули и подвиглись основания гор, ибо разгневался [Бог];
\vs Psa 17:9 поднялся дым от гнева Его и из уст Его огонь поядающий; горячие угли \bibemph{сыпались} от Него.
\vs Psa 17:10 Наклонил Он небеса и сошел,~--- и мрак под ногами Его.
\vs Psa 17:11 И воссел на Херувимов и полетел, и понесся на крыльях ветра.
\vs Psa 17:12 И мрак сделал покровом Своим, сению вокруг Себя мрак вод, облаков воздушных.
\vs Psa 17:13 От блистания пред Ним бежали облака Его, град и угли огненные.
\vs Psa 17:14 Возгремел на небесах Господь, и Всевышний дал глас Свой, град и угли огненные.
\vs Psa 17:15 Пустил стрелы Свои и рассеял их, множество молний, и рассыпал их.
\vs Psa 17:16 И явились источники вод, и открылись основания вселенной от грозного \bibemph{гласа} Твоего, Господи, от дуновения духа гнева Твоего.
\vs Psa 17:17 Он простер \bibemph{руку} с высоты и взял меня, и извлек меня из вод многих;
\vs Psa 17:18 избавил меня от врага моего сильного и от ненавидящих меня, которые были сильнее меня.
\vs Psa 17:19 Они восстали на меня в день бедствия моего, но Господь был мне опорою.
\vs Psa 17:20 Он вывел меня на пространное место и избавил меня, ибо Он благоволит ко мне.
\vs Psa 17:21 Воздал мне Господь по правде моей, по чистоте рук моих вознаградил меня,
\vs Psa 17:22 ибо я хранил пути Господни и не был нечестивым пред Богом моим;
\vs Psa 17:23 ибо все заповеди Его предо мною, и от уставов Его я не отступал.
\vs Psa 17:24 Я был непорочен пред Ним и остерегался, чтобы не согрешить мне;
\vs Psa 17:25 и воздал мне Господь по правде моей, по чистоте рук моих пред очами Его.
\vs Psa 17:26 С милостивым Ты поступаешь милостиво, с мужем искренним~--- искренно,
\vs Psa 17:27 с чистым~--- чисто, а с лукавым~--- по лукавству его,
\vs Psa 17:28 ибо Ты людей угнетенных спасаешь, а очи надменные унижаешь.
\vs Psa 17:29 Ты возжигаешь светильник мой, Господи; Бог мой просвещает тьму мою.
\vs Psa 17:30 С Тобою я поражаю войско, с Богом моим восхожу на стену.
\vs Psa 17:31 Бог!~--- Непорочен путь Его, чисто слово Господа; щит Он для всех, уповающих на Него.
\vs Psa 17:32 Ибо кто Бог, кроме Господа, и кто защита, кроме Бога нашего?
\vs Psa 17:33 Бог препоясывает меня силою и устрояет мне верный путь;
\vs Psa 17:34 делает ноги мои, как оленьи, и на высотах моих поставляет меня;
\vs Psa 17:35 научает руки мои брани, и мышцы мои сокрушают медный лук.
\vs Psa 17:36 Ты дал мне щит спасения Твоего, и десница Твоя поддерживает меня, и милость Твоя возвеличивает меня.
\vs Psa 17:37 Ты расширяешь шаг мой подо мною, и не колеблются ноги мои.
\vs Psa 17:38 Я преследую врагов моих и настигаю их, и не возвращаюсь, доколе не истреблю их;
\vs Psa 17:39 поражаю их, и они не могут встать, падают под ноги мои,
\vs Psa 17:40 ибо Ты препоясал меня силою для войны и низложил под ноги мои восставших на меня;
\vs Psa 17:41 Ты обратил ко мне тыл врагов моих, и я истребляю ненавидящих меня:
\vs Psa 17:42 они вопиют, но нет спасающего; ко Господу,~--- но Он не внемлет им;
\vs Psa 17:43 я рассеваю их, как прах пред лицем ветра, как уличную грязь попираю их.
\vs Psa 17:44 Ты избавил меня от мятежа народа, поставил меня главою иноплеменников; народ, которого я не знал, служит мне;
\vs Psa 17:45 по одному слуху о мне повинуются мне; иноплеменники ласкательствуют предо мною;
\vs Psa 17:46 иноплеменники бледнеют и трепещут в укреплениях своих.
\vs Psa 17:47 Жив Господь и благословен защитник мой! Да будет превознесен Бог спасения моего,
\vs Psa 17:48 Бог, мстящий за меня и покоряющий мне народы,
\vs Psa 17:49 и избавляющий меня от врагов моих! Ты вознес меня над восстающими против меня и от человека жестокого избавил меня.
\vs Psa 17:50 За то буду славить Тебя, Господи, между иноплеменниками и буду петь имени Твоему,
\vs Psa 17:51 величественно спасающий царя и творящий милость помазаннику Твоему Давиду и потомству его во веки.
\vs Psa 18:1 Начальнику хора. Псалом Давида.
\rsbpar\vs Psa 18:2 Небеса проповедуют славу Божию, и о делах рук Его вещает твердь.
\vs Psa 18:3 День дню передает речь, и ночь ночи открывает знание.
\vs Psa 18:4 Нет языка, и нет наречия, где не слышался бы голос их.
\vs Psa 18:5 По всей земле проходит звук их, и до пределов вселенной слов\acc{а} их. Он поставил в них жилище солнцу,
\vs Psa 18:6 и оно выходит, как жених из брачного чертога своего, радуется, как исполин, пробежать поприще:
\vs Psa 18:7 от края небес исход его, и шествие его до края их, и ничто не укрыто от теплоты его.
\vs Psa 18:8 Закон Господа совершен, укрепляет душу; откровение Господа верно, умудряет простых.
\vs Psa 18:9 Повеления Господа праведны, веселят сердце; заповедь Господа светла, просвещает очи.
\vs Psa 18:10 Страх Господень чист, пребывает вовек. Суды Господни истина, все праведны;
\vs Psa 18:11 они вожделеннее золота и даже множества золота чистого, слаще меда и капель сота;
\vs Psa 18:12 и раб Твой охраняется ими, в соблюдении их великая награда.
\vs Psa 18:13 Кто усмотрит погрешности свои? От тайных \bibemph{моих} очисти меня
\vs Psa 18:14 и от умышленных удержи раба Твоего, чтобы не возобладали мною. Тогда я буду непорочен и чист от великого развращения.
\vs Psa 18:15 Да будут слова уст моих и помышление сердца моего благоугодны пред Тобою, Господи, твердыня моя и Избавитель мой!
\vs Psa 19:1 Начальнику хора. Псалом Давида.
\rsbpar\vs Psa 19:2 Да услышит тебя Господь в день печали, да защитит тебя имя Бога Иаковлева.
\vs Psa 19:3 Да пошлет тебе помощь из Святилища и с Сиона да подкрепит тебя.
\vs Psa 19:4 Да воспомянет все жертвоприношения твои и всесожжение твое да соделает тучным.
\vs Psa 19:5 Да даст тебе [Господь] по сердцу твоему и все намерения твои да исполнит.
\vs Psa 19:6 Мы возрадуемся о спасении твоем и во имя Бога нашего поднимем знамя. Да исполнит Господь все прошения твои.
\vs Psa 19:7 Ныне познал я, что Господь спасает помазанника Своего, отвечает ему со святых небес Своих могуществом спасающей десницы Своей.
\vs Psa 19:8 Иные колесницами, иные конями, а мы именем Господа Бога нашего хвалимся:
\vs Psa 19:9 они поколебались и пали, а мы встали и стоим прямо.
\vs Psa 19:10 Господи! спаси царя и услышь нас, когда будем взывать [к Тебе].
\vs Psa 20:1 Начальнику хора. Псалом Давида.
\rsbpar\vs Psa 20:2 Господи! силою Твоею веселится царь и о спасении Твоем безмерно радуется.
\vs Psa 20:3 Ты дал ему, чего желало сердце его, и прошения уст его не отринул,
\vs Psa 20:4 ибо Ты встретил его благословениями благости, возложил на голову его венец из чистого золота.
\vs Psa 20:5 Он просил у Тебя жизни; Ты дал ему долгоденствие на век и век.
\vs Psa 20:6 Велика слава его в спасении Твоем; Ты возложил на него честь и величие.
\vs Psa 20:7 Ты положил на него благословения на веки, возвеселил его радостью лица Твоего,
\vs Psa 20:8 ибо царь уповает на Господа, и по благости Всевышнего не поколеблется.
\vs Psa 20:9 Рука Твоя найдет всех врагов Твоих, десница Твоя найдет [всех] ненавидящих Тебя.
\vs Psa 20:10 Во время гнева Твоего Ты сделаешь их, как печь огненную; во гневе Своем Господь погубит их, и пожрет их огонь.
\vs Psa 20:11 Ты истребишь плод их с земли и семя их~--- из среды сынов человеческих,
\vs Psa 20:12 ибо они предприняли против Тебя злое, составили замыслы, но не могли [выполнить их].
\vs Psa 20:13 Ты поставишь их целью, из луков Твоих пустишь стрелы в лице их.
\vs Psa 20:14 Вознесись, Господи, силою Твоею: мы будем воспевать и прославлять Твое могущество.
\vs Psa 21:1 Начальнику хора. При появлении зари. Псалом Давида.
\rsbpar\vs Psa 21:2 Боже мой! Боже мой! [внемли мне] для чего Ты оставил меня? Далеки от спасения моего слова вопля моего.
\vs Psa 21:3 Боже мой! я вопию днем,~--- и Ты не внемлешь мне, ночью,~--- и нет мне успокоения.
\vs Psa 21:4 Но Ты, Святый, живешь среди славословий Израиля.
\vs Psa 21:5 На Тебя уповали отцы наши; уповали, и Ты избавлял их;
\vs Psa 21:6 к Тебе взывали они, и были спасаемы; на Тебя уповали, и не оставались в стыде.
\vs Psa 21:7 Я же червь, а не человек, поношение у людей и презрение в народе.
\vs Psa 21:8 Все, видящие меня, ругаются надо мною, говорят устами, кивая головою:
\vs Psa 21:9 <<он уповал на Господа; пусть избавит его, пусть спасет, если он угоден Ему>>.
\vs Psa 21:10 Но Ты извел меня из чрева, вложил в меня упование у грудей матери моей.
\vs Psa 21:11 На Тебя оставлен я от утробы; от чрева матери моей Ты~--- Бог мой.
\vs Psa 21:12 Не удаляйся от меня, ибо скорбь близка, а помощника нет.
\vs Psa 21:13 Множество тельцов обступили меня; тучные Васанские окружили меня,
\vs Psa 21:14 раскрыли на меня пасть свою, \bibemph{как} лев, алчущий добычи и рыкающий.
\vs Psa 21:15 Я пролился, как вода; все кости мои рассыпались; сердце мое сделалось, как воск, растаяло посреди внутренности моей.
\vs Psa 21:16 Сила моя иссохла, как черепок; язык мой прильпнул к гортани моей, и Ты свел меня к персти смертной.
\vs Psa 21:17 Ибо псы окружили меня, скопище злых обступило меня, пронзили руки мои и ноги мои.
\vs Psa 21:18 Можно было бы перечесть все кости мои; а они смотрят и делают из меня зрелище;
\vs Psa 21:19 делят ризы мои между собою и об одежде моей бросают жребий.
\vs Psa 21:20 Но Ты, Господи, не удаляйся от меня; сила моя! поспеши на помощь мне;
\vs Psa 21:21 избавь от меча душу мою и от псов одинокую мою;
\vs Psa 21:22 спаси меня от пасти льва и от рогов единорогов, услышав, \bibemph{избавь} меня.
\vs Psa 21:23 Буду возвещать имя Твое братьям моим, посреди собрания восхвалять Тебя.
\vs Psa 21:24 Боящиеся Господа! восхвалите Его. Все семя Иакова! прославь Его. Да благоговеет пред Ним все семя Израиля,
\vs Psa 21:25 ибо Он не презрел и не пренебрег скорби страждущего, не скрыл от него лица Своего, но услышал его, когда сей воззвал к Нему.
\vs Psa 21:26 О Тебе хвала моя в собрании великом; воздам обеты мои пред боящимися Его.
\vs Psa 21:27 Да едят бедные и насыщаются; да восхвалят Господа ищущие Его; да живут сердца ваши во веки!
\vs Psa 21:28 Вспомнят, и обратятся к Господу все концы земли, и поклонятся пред Тобою все племена язычников,
\vs Psa 21:29 ибо Господне есть царство, и Он~--- Владыка над народами.
\vs Psa 21:30 Будут есть и поклоняться все тучные земли; преклонятся пред Ним все нисходящие в персть и не могущие сохранить жизни своей.
\vs Psa 21:31 Потомство [мое] будет служить Ему, и будет называться Господним вовек:
\vs Psa 21:32 придут и будут возвещать правду Его людям, которые родятся, чт\acc{о} сотворил Господь.
\vs Psa 22:0 Псалом Давида.
\rsbpar\vs Psa 22:1 Господь~--- Пастырь мой; я ни в чем не буду нуждаться:
\vs Psa 22:2 Он покоит меня на злачных пажитях и водит меня к водам тихим,
\vs Psa 22:3 подкрепляет душу мою, направляет меня на стези правды ради имени Своего.
\vs Psa 22:4 Если я пойду и долиною смертной тени, не убоюсь зла, потому что Ты со мной; Твой жезл и Твой посох~--- они успокаивают меня.
\vs Psa 22:5 Ты приготовил предо мною трапезу в виду врагов моих; умастил елеем голову мою; чаша моя преисполнена.
\vs Psa 22:6 Так, благость и милость [Твоя] да сопровождают меня во все дни жизни моей, и я пребуду в доме Господнем многие дни.
\vs Psa 23:0 Псалом Давида. [В первый день недели.]
\rsbpar\vs Psa 23:1 Господня земля и что наполняет ее, вселенная и все живущее в ней,
\vs Psa 23:2 ибо Он основал ее на морях и на реках утвердил ее.
\vs Psa 23:3 Кто взойдет на гору Господню, или кто станет на святом месте Его?
\vs Psa 23:4 Тот, у которого руки неповинны и сердце чисто, кто не клялся душею своею напрасно и не божился ложно [ближнему своему],~---
\vs Psa 23:5 \bibemph{тот} получит благословение от Господа и милость от Бога, Спасителя своего.
\vs Psa 23:6 Таков род ищущих Его, ищущих лица Твоего, Боже Иакова!
\vs Psa 23:7 Поднимите, врата, верхи ваши, и поднимитесь, двери вечные, и войдет Царь славы!
\vs Psa 23:8 Кто сей Царь славы?~--- Господь крепкий и сильный, Господь, сильный в брани.
\vs Psa 23:9 Поднимите, врата, верхи ваши, и поднимитесь, двери вечные, и войдет Царь славы!
\vs Psa 23:10 Кто сей Царь славы?~--- Господь сил, Он~--- Царь славы.
\vs Psa 24:0 Псалом Давида.
\rsbpar\vs Psa 24:1 К Тебе, Господи, возношу душу мою.
\vs Psa 24:2 Боже мой! на Тебя уповаю, да не постыжусь [вовек], да не восторжествуют надо мною враги мои,
\vs Psa 24:3 да не постыдятся и все надеющиеся на Тебя: да постыдятся беззаконнующие втуне.
\vs Psa 24:4 Укажи мне, Господи, пути Твои и научи меня стезям Твоим.
\vs Psa 24:5 Направь меня на истину Твою и научи меня, ибо Ты Бог спасения моего; на Тебя надеюсь всякий день.
\vs Psa 24:6 Вспомни щедроты Твои, Господи, и милости Твои, ибо они от века.
\vs Psa 24:7 Грехов юности моей и преступлений моих не вспоминай; по милости Твоей вспомни меня Ты, ради благости Твоей, Господи!
\vs Psa 24:8 Благ и праведен Господь, посему наставляет грешников на путь,
\vs Psa 24:9 направляет кротких к правде, и научает кротких путям Своим.
\vs Psa 24:10 Все пути Господни~--- милость и истина к хранящим завет Его и откровения Его.
\vs Psa 24:11 Ради имени Твоего, Господи, прости согрешение мое, ибо велико оно.
\vs Psa 24:12 Кто есть человек, боящийся Господа? Ему укажет Он путь, который избрать.
\vs Psa 24:13 Душа его пребудет во благе, и семя его наследует землю.
\vs Psa 24:14 Тайна Господня~--- боящимся Его, и завет Свой Он открывает им.
\vs Psa 24:15 Очи мои всегда к Господу, ибо Он извлекает из сети ноги мои.
\vs Psa 24:16 Призри на меня и помилуй меня, ибо я одинок и угнетен.
\vs Psa 24:17 Скорби сердца моего умножились; выведи меня из бед моих,
\vs Psa 24:18 призри на страдание мое и на изнеможение мое и прости все грехи мои.
\vs Psa 24:19 Посмотри на врагов моих, как много их, и \bibemph{какою} лютою ненавистью они ненавидят меня.
\vs Psa 24:20 Сохрани душу мою и избавь меня, да не постыжусь, что я на Тебя уповаю.
\vs Psa 24:21 Непорочность и правота да охраняют меня, ибо я на Тебя надеюсь.
\vs Psa 24:22 Избавь, Боже, Израиля от всех скорбей его.
\vs Psa 25:0 Псалом Давида.
\rsbpar\vs Psa 25:1 Рассуди меня, Господи, ибо я ходил в непорочности моей, и, уповая на Господа, не поколеблюсь.
\vs Psa 25:2 Искуси меня, Господи, и испытай меня; расплавь внутренности мои и сердце мое,
\vs Psa 25:3 ибо милость Твоя пред моими очами, и я ходил в истине Твоей,
\vs Psa 25:4 не сидел я с людьми лживыми, и с коварными не пойду;
\vs Psa 25:5 возненавидел я сборище злонамеренных, и с нечестивыми не сяду;
\vs Psa 25:6 буду омывать в невинности руки мои и обходить жертвенник Твой, Господи,
\vs Psa 25:7 чтобы возвещать гласом хвалы и поведать все чудеса Твои.
\vs Psa 25:8 Господи! возлюбил я обитель дома Твоего и место жилища славы Твоей.
\vs Psa 25:9 Не погуби души моей с грешниками и жизни моей с кровожадными,
\vs Psa 25:10 у которых в руках злодейство, и которых правая рука полна мздоимства.
\vs Psa 25:11 А я хожу в моей непорочности; избавь меня, [Господи,] и помилуй меня.
\vs Psa 25:12 Моя нога стоит на прямом \bibemph{пути}; в собраниях благословлю Господа.
\vs Psa 26:0 Псалом Давида. [Прежде помазания.]
\rsbpar\vs Psa 26:1 Господь~--- свет мой и спасение мое: кого мне бояться? Господь крепость жизни моей: кого мне страшиться?
\vs Psa 26:2 Если будут наступать на меня злодеи, противники и враги мои, чтобы пожрать плоть мою, то они сами преткнутся и падут.
\vs Psa 26:3 Если ополчится против меня полк, не убоится сердце мое; если восстанет на меня война, и тогда буду надеяться.
\vs Psa 26:4 Одного просил я у Господа, того только ищу, чтобы пребывать мне в доме Господнем во все дни жизни моей, созерцать красоту Господню и посещать [святый] храм Его,
\vs Psa 26:5 ибо Он укрыл бы меня в скинии Своей в день бедствия, скрыл бы меня в потаенном месте селения Своего, вознес бы меня на скалу.
\vs Psa 26:6 Тогда вознеслась бы голова моя над врагами, окружающими меня; и я принес бы в Его скинии жертвы славословия, стал бы петь и воспевать пред Господом.
\vs Psa 26:7 Услышь, Господи, голос мой, которым я взываю, помилуй меня и внемли мне.
\vs Psa 26:8 Сердце мое говорит от Тебя: <<ищите лица Моего>>; и я буду искать лица Твоего, Господи.
\vs Psa 26:9 Не скрой от меня лица Твоего; не отринь во гневе раба Твоего. Ты был помощником моим; не отвергни меня и не оставь меня, Боже, Спаситель мой!
\vs Psa 26:10 ибо отец мой и мать моя оставили меня, но Господь примет меня.
\vs Psa 26:11 Научи меня, Господи, пути Твоему и наставь меня на стезю правды, ради врагов моих;
\vs Psa 26:12 не предавай меня на произвол врагам моим, ибо восстали на меня свидетели лживые и дышат злобою.
\vs Psa 26:13 Но я верую, что увижу благость Господа на земле живых.
\vs Psa 26:14 Надейся на Господа, мужайся, и да укрепляется сердце твое, и надейся на Господа.
\vs Psa 27:0 Псалом Давида.
\rsbpar\vs Psa 27:1 К тебе, Господи, взываю: твердыня моя! не будь безмолвен для меня, чтобы при безмолвии Твоем я не уподобился нисходящим в могилу.
\vs Psa 27:2 Услышь голос молений моих, когда я взываю к Тебе, когда поднимаю руки мои к святому храму Твоему.
\vs Psa 27:3 Не погуби меня с нечестивыми и с делающими неправду, которые с ближними своими говорят о мире, а в сердце у них зло.
\vs Psa 27:4 Воздай им по делам их, по злым поступкам их; по делам рук их воздай им, отдай им заслуженное ими.
\vs Psa 27:5 За то, что они невнимательны к действиям Господа и к делу рук Его, Он разрушит их и не созиждет их.
\vs Psa 27:6 Благословен Господь, ибо Он услышал голос молений моих.
\vs Psa 27:7 Господь~--- крепость моя и щит мой; на Него уповало сердце мое, и Он помог мне, и возрадовалось сердце мое; и я прославлю Его песнью моею.
\vs Psa 27:8 Господь~--- крепость народа Своего и спасительная защита помазанника Своего.
\vs Psa 27:9 Спаси народ Твой и благослови наследие Твое; паси их и возвышай их во веки!
\vs Psa 28:0 Псалом Давида. [При окончании праздника кущей.]
\rsbpar\vs Psa 28:1 Воздайте Господу, сыны Божии, воздайте Господу славу и честь,
\vs Psa 28:2 воздайте Господу славу имени Его; поклонитесь Господу в благолепном святилище \bibemph{Его}.
\vs Psa 28:3 Глас Господень над водами; Бог славы возгремел, Господь над водами многими.
\vs Psa 28:4 Глас Господа силен, глас Господа величествен.
\vs Psa 28:5 Глас Господа сокрушает кедры; Господь сокрушает кедры Ливанские
\vs Psa 28:6 и заставляет их скакать подобно тельцу, Ливан и Сирион, подобно молодому единорогу.
\vs Psa 28:7 Глас Господа высекает пламень огня.
\vs Psa 28:8 Глас Господа потрясает пустыню; потрясает Господь пустыню Кадес.
\vs Psa 28:9 Глас Господа разрешает от бремени ланей и обнажает леса; и во храме Его все возвещает о \bibemph{Его} славе.
\vs Psa 28:10 Господь восседал над потопом, и будет восседать Господь царем вовек.
\vs Psa 28:11 Господь даст силу народу Своему, Господь благословит народ Свой миром.
\vs Psa 29:1 Псалом Давида; песнь при обновлении дома.
\rsbpar\vs Psa 29:2 Превознесу Тебя, Господи, что Ты поднял меня и не дал моим врагам восторжествовать надо мною.
\vs Psa 29:3 Господи, Боже мой! я воззвал к Тебе, и Ты исцелил меня.
\vs Psa 29:4 Господи! Ты вывел из ада душу мою и оживил меня, чтобы я не сошел в могилу.
\vs Psa 29:5 Пойте Господу, святые Его, славьте память святыни Его,
\vs Psa 29:6 ибо на мгновение гнев Его, на \bibemph{всю} жизнь благоволение Его: вечером водворяется плач, а на утро радость.
\vs Psa 29:7 И я говорил в благоденствии моем: <<не поколеблюсь вовек>>.
\vs Psa 29:8 По благоволению Твоему, Господи, Ты укрепил гору мою; но Ты сокрыл лице Твое, \bibemph{и} я смутился.
\vs Psa 29:9 \bibemph{Тогда} к Тебе, Господи, взывал я, и Господа [моего] умолял:
\vs Psa 29:10 <<что пользы в крови моей, когда я сойду в могилу? будет ли прах славить Тебя? будет ли возвещать истину Твою?
\vs Psa 29:11 услышь, Господи, и помилуй меня; Господи! будь мне помощником>>.
\vs Psa 29:12 И Ты обратил сетование мое в ликование, снял с меня вретище и препоясал меня веселием,
\vs Psa 29:13 да славит Тебя душа моя и да не умолкает. Господи, Боже мой! буду славить Тебя вечно.
\vs Psa 30:1 Начальнику хора. Псалом Давида. [Во время смятения.]
\rsbpar\vs Psa 30:2 На Тебя, Господи, уповаю, да не постыжусь вовек; по правде Твоей избавь меня;
\vs Psa 30:3 приклони ко мне ухо Твое, поспеши избавить меня. Будь мне каменною твердынею, домом прибежища, чтобы спасти меня,
\vs Psa 30:4 ибо Ты каменная гора моя и ограда моя; ради имени Твоего води меня и управляй мною.
\vs Psa 30:5 Выведи меня из сети, которую тайно поставили мне, ибо Ты крепость моя.
\vs Psa 30:6 В Твою руку предаю дух мой; Ты избавлял меня, Господи, Боже истины.
\vs Psa 30:7 Ненавижу почитателей суетных идолов, но на Господа уповаю.
\vs Psa 30:8 Буду радоваться и веселиться о милости Твоей, потому что Ты призрел на бедствие мое, узнал горесть души моей
\vs Psa 30:9 и не предал меня в руки врага; поставил ноги мои на пространном месте.
\vs Psa 30:10 Помилуй меня, Господи, ибо тесно мне; иссохло от горести око мое, душа моя и утроба моя.
\vs Psa 30:11 Истощилась в печали жизнь моя и лета мои в стенаниях; изнемогла от грехов моих сила моя, и кости мои иссохли.
\vs Psa 30:12 От всех врагов моих я сделался поношением даже у соседей моих и страшилищем для знакомых моих; видящие меня на улице бегут от меня.
\vs Psa 30:13 Я забыт в сердцах, как мертвый; я~--- как сосуд разбитый,
\vs Psa 30:14 ибо слышу злоречие многих; отвсюду ужас, когда они сговариваются против меня, умышляют исторгнуть душу мою.
\vs Psa 30:15 А я на Тебя, Господи, уповаю; я говорю: Ты~--- мой Бог.
\vs Psa 30:16 В Твоей руке дни мои; избавь меня от руки врагов моих и от гонителей моих.
\vs Psa 30:17 Яви светлое лице Твое рабу Твоему; спаси меня милостью Твоею.
\vs Psa 30:18 Господи! да не постыжусь, что я к Тебе взываю; нечестивые же да посрамятся, да умолкнут в аде.
\vs Psa 30:19 Да онемеют уста лживые, которые против праведника говорят злое с гордостью и презреньем.
\vs Psa 30:20 Как много у Тебя благ, которые Ты хранишь для боящихся Тебя и которые приготовил уповающим на Тебя пред сынами человеческими!
\vs Psa 30:21 Ты укрываешь их под покровом лица Твоего от мятежей людских, скрываешь их под сенью от пререкания языков.
\vs Psa 30:22 Благословен Господь, что явил мне дивную милость Свою в укрепленном городе!
\vs Psa 30:23 В смятении моем я думал: <<отвержен я от очей Твоих>>; но Ты услышал голос молитвы моей, когда я воззвал к Тебе.
\vs Psa 30:24 Любите Господа, все праведные Его; Господь хранит верных и поступающим надменно воздает с избытком.
\vs Psa 30:25 Мужайтесь, и да укрепляется сердце ваше, все надеющиеся на Господа!
\vs Psa 31:0 Псалом Давида. Учение.
\rsbpar\vs Psa 31:1 Блажен, кому отпущены беззакония, и чьи грехи покрыты!
\vs Psa 31:2 Блажен человек, которому Господь не вменит греха, и в чьем духе нет лукавства!
\vs Psa 31:3 Когда я молчал, обветшали кости мои от вседневного стенания моего,
\vs Psa 31:4 ибо день и ночь тяготела надо мною рука Твоя; свежесть моя исчезла, как в летнюю засуху.
\vs Psa 31:5 Но я открыл Тебе грех мой и не скрыл беззакония моего; я сказал: <<исповедаю Господу преступления мои>>, и Ты снял с меня вину греха моего.
\vs Psa 31:6 За то помолится Тебе каждый праведник во время благопотребное, и тогда разлитие многих вод не достигнет его.
\vs Psa 31:7 Ты покров мой: Ты охраняешь меня от скорби, окружаешь меня радостями избавления.
\vs Psa 31:8 <<Вразумлю тебя, наставлю тебя на путь, по которому тебе идти; буду руководить тебя, око Мое над тобою>>.
\vs Psa 31:9 <<Не будьте как конь, как лошак несмысленный, которых челюсти нужно обуздывать уздою и удилами, чтобы они покорялись тебе>>.
\vs Psa 31:10 Много скорбей нечестивому, а уповающего на Господа окружает милость.
\vs Psa 31:11 Веселитесь о Господе и радуйтесь, праведные; торжествуйте, все правые сердцем.
\vs Psa 32:0 [Псалом Давида.]
\rsbpar\vs Psa 32:1 Радуйтесь, праведные, о Господе: правым прилично славословить.
\vs Psa 32:2 Славьте Господа на гуслях, пойте Ему на десятиструнной псалтири;
\vs Psa 32:3 пойте Ему новую песнь; пойте Ему стройно, с восклицанием,
\vs Psa 32:4 ибо слово Господне право и все дела Его верны.
\vs Psa 32:5 Он любит правду и суд; милости Господней полна земля.
\vs Psa 32:6 Словом Господа сотворены небеса, и духом уст Его~--- все воинство их:
\vs Psa 32:7 Он собрал, будто груды, морские воды, положил бездны в хранилищах.
\vs Psa 32:8 Да боится Господа вся земля; да трепещут пред Ним все живущие во вселенной,
\vs Psa 32:9 ибо Он сказал,~--- и сделалось; Он повелел,~--- и явилось.
\vs Psa 32:10 Господь разрушает советы язычников, уничтожает замыслы народов, [уничтожает советы князей].
\vs Psa 32:11 Совет же Господень стоит вовек; помышления сердца Его~--- в род и род.
\vs Psa 32:12 Блажен народ, у которого Господь есть Бог,~--- племя, которое Он избрал в наследие Себе.
\vs Psa 32:13 С небес призирает Господь, видит всех сынов человеческих;
\vs Psa 32:14 с престола, на котором восседает, Он призирает на всех, живущих на земле:
\vs Psa 32:15 Он создал сердца всех их и вникает во все дела их.
\vs Psa 32:16 Не спасется царь множеством воинства; исполина не защитит великая сила.
\vs Psa 32:17 Ненадежен конь для спасения, не избавит великою силою своею.
\vs Psa 32:18 Вот, око Господне над боящимися Его и уповающими на милость Его,
\vs Psa 32:19 что Он душу их спасет от смерти и во время голода пропитает их.
\vs Psa 32:20 Душа наша уповает на Господа: Он~--- помощь наша и защита наша;
\vs Psa 32:21 о Нем веселится сердце наше, ибо на святое имя Его мы уповали.
\vs Psa 32:22 Да будет милость Твоя, Господи, над нами, как мы уповаем на Тебя.
\vs Psa 33:1 Псалом Давида, когда он притворился безумным пред Авимелехом и был изгнан от него и удалился.
\rsbpar\vs Psa 33:2 Благословлю Господа во всякое время; хвала Ему непрестанно в устах моих.
\vs Psa 33:3 Господом будет хвалиться душа моя; услышат кроткие и возвеселятся.
\vs Psa 33:4 Величайте Господа со мною, и превознесем имя Его вместе.
\vs Psa 33:5 Я взыскал Господа, и Он услышал меня, и от всех опасностей моих избавил меня.
\vs Psa 33:6 Кто обращал взор к Нему, те просвещались, и лица их не постыдятся.
\vs Psa 33:7 Сей нищий воззвал,~--- и Господь услышал и спас его от всех бед его.
\vs Psa 33:8 Ангел Господень ополчается вокруг боящихся Его и избавляет их.
\vs Psa 33:9 Вкусите, и увидите, как благ Господь! Блажен человек, который уповает на Него!
\vs Psa 33:10 Бойтесь Господа, [все] святые Его, ибо нет скудости у боящихся Его.
\vs Psa 33:11 Скимны бедствуют и терпят голод, а ищущие Господа не терпят нужды ни в каком благе.
\vs Psa 33:12 Придите, дети, послушайте меня: страху Господню научу вас.
\vs Psa 33:13 Хочет ли человек жить и любит ли долгоденствие, чтобы видеть благо?
\vs Psa 33:14 Удерживай язык свой от зла и уста свои от коварных слов.
\vs Psa 33:15 Уклоняйся от зла и делай добро; ищи мира и следуй за ним.
\vs Psa 33:16 Очи Господни \bibemph{обращены} на праведников, и уши Его~--- к воплю их.
\vs Psa 33:17 Но лице Господне против делающих зло, чтобы истребить с земли память о них.
\vs Psa 33:18 Взывают [праведные], и Господь слышит, и от всех скорбей их избавляет их.
\vs Psa 33:19 Близок Господь к сокрушенным сердцем и смиренных духом спасет.
\vs Psa 33:20 Много скорбей у праведного, и от всех их избавит его Господь.
\vs Psa 33:21 Он хранит все кости его; ни одна из них не сокрушится.
\vs Psa 33:22 Убьет грешника зло, и ненавидящие праведного погибнут.
\vs Psa 33:23 Избавит Господь душу рабов Своих, и никто из уповающих на Него не погибнет.
\vs Psa 34:0 Псалом Давида.
\rsbpar\vs Psa 34:1 Вступись, Господи, в тяжбу с тяжущимися со мною, побори борющихся со мною;
\vs Psa 34:2 возьми щит и латы и восстань на помощь мне;
\vs Psa 34:3 обнажи меч и прегради \bibemph{путь} преследующим меня; скажи душе моей: <<Я~--- спасение твое!>>
\vs Psa 34:4 Да постыдятся и посрамятся ищущие души моей; да обратятся назад и покроются бесчестием умышляющие мне зло;
\vs Psa 34:5 да будут они, как прах пред лицем ветра, и Ангел Господень да прогоняет \bibemph{их};
\vs Psa 34:6 да будет путь их темен и скользок, и Ангел Господень да преследует их,
\vs Psa 34:7 ибо они без вины скрыли для меня яму~--- сеть свою, без вины выкопали \bibemph{ее} для души моей.
\vs Psa 34:8 Да придет на него гибель неожиданная, и сеть его, которую он скрыл \bibemph{для меня}, да уловит его самого; да впадет в нее на погибель.
\vs Psa 34:9 А моя душа будет радоваться о Господе, будет веселиться о спасении от Него.
\vs Psa 34:10 Все кости мои скажут: <<Господи! кто подобен Тебе, избавляющему слабого от сильного, бедного и нищего от грабителя его?>>
\vs Psa 34:11 Восстали на меня свидетели неправедные: чего я не знаю, о том допрашивают меня;
\vs Psa 34:12 воздают мне злом за добро, сиротством душе моей.
\vs Psa 34:13 Я во время болезни их одевался во вретище, изнурял постом душу мою, и молитва моя возвращалась в недро мое.
\vs Psa 34:14 Я поступал, как бы это был друг мой, брат мой; я ходил скорбный, с поникшею головою, как бы оплакивающий мать.
\vs Psa 34:15 А когда я претыкался, они радовались и собирались; собирались ругатели против меня, не знаю за что, поносили и не переставали;
\vs Psa 34:16 с лицемерными насмешниками скрежетали на меня зубами своими.
\vs Psa 34:17 Господи! долго ли будешь смотреть \bibemph{на это}? Отведи душу мою от злодейств их, от львов~--- одинокую мою.
\vs Psa 34:18 Я прославлю Тебя в собрании великом, среди народа многочисленного восхвалю Тебя,
\vs Psa 34:19 чтобы не торжествовали надо мною враждующие против меня неправедно, и не перемигивались глазами ненавидящие меня безвинно;
\vs Psa 34:20 ибо не о мире говорят они, но против мирных земли составляют лукавые замыслы;
\vs Psa 34:21 расширяют на меня уста свои; говорят: <<хорошо! хорошо! видел глаз наш>>.
\vs Psa 34:22 Ты видел, Господи, не умолчи; Господи! не удаляйся от меня.
\vs Psa 34:23 Подвигнись, пробудись для суда моего, для тяжбы моей, Боже мой и Господи мой!
\vs Psa 34:24 Суди меня по правде Твоей, Господи, Боже мой, и да не торжествуют они надо мною;
\vs Psa 34:25 да не говорят в сердце своем: <<хорошо! [хорошо!] по душе нашей!>> Да не говорят: <<мы поглотили его>>.
\vs Psa 34:26 Да постыдятся и посрамятся все, радующиеся моему несчастью; да облекутся в стыд и позор величающиеся надо мною.
\vs Psa 34:27 Да радуются и веселятся желающие правоты моей и говорят непрестанно: <<да возвеличится Господь, желающий мира рабу Своему!>>
\vs Psa 34:28 И язык мой будет проповедовать правду Твою и хвалу Твою всякий день.
\vs Psa 35:1 Начальнику хора. Раба Господня Давида.
\rsbpar\vs Psa 35:2 Нечестие беззаконного говорит в сердце моем: нет страха Божия пред глазами его,
\vs Psa 35:3 ибо он льстит себе в глазах своих, будто отыскивает беззаконие свое, чтобы возненавидеть его;
\vs Psa 35:4 слова уст его~--- неправда и лукавство; не хочет он вразумиться, чтобы делать добро;
\vs Psa 35:5 на ложе своем замышляет беззаконие, становится на путь недобрый, не гнушается злом.
\vs Psa 35:6 Господи! милость Твоя до небес, истина Твоя до облаков!
\vs Psa 35:7 Правда Твоя, как горы Божии, и судьбы Твои~--- бездна великая! Человеков и скотов хранишь Ты, Господи!
\vs Psa 35:8 Как драгоценна милость Твоя, Боже! Сыны человеческие в тени крыл Твоих покойны:
\vs Psa 35:9 насыщаются от тука дома Твоего, и из потока сладостей Твоих Ты напояешь их,
\vs Psa 35:10 ибо у Тебя источник жизни; во свете Твоем мы видим свет.
\vs Psa 35:11 Продли милость Твою к знающим Тебя и правду Твою к правым сердцем,
\vs Psa 35:12 да не наступит на меня нога гордыни, и рука грешника да не изгонит меня:
\vs Psa 35:13 там пали делающие беззаконие, низринуты и не могут встать.
\vs Psa 36:0 Псалом Давида.
\rsbpar\vs Psa 36:1 Не ревнуй злодеям, не завидуй делающим беззаконие,
\vs Psa 36:2 ибо они, как трава, скоро будут подкошены и, как зеленеющий злак, увянут.
\vs Psa 36:3 Уповай на Господа и делай добро; живи на земле и храни истину.
\vs Psa 36:4 Утешайся Господом, и Он исполнит желания сердца твоего.
\vs Psa 36:5 Предай Господу путь твой и уповай на Него, и Он совершит,
\vs Psa 36:6 и выведет, как свет, правду твою и справедливость твою, как полдень.
\vs Psa 36:7 Покорись Господу и надейся на Него. Не ревнуй успевающему в пути своем, человеку лукавствующему.
\vs Psa 36:8 Перестань гневаться и оставь ярость; не ревнуй до того, чтобы делать зло,
\vs Psa 36:9 ибо делающие зло истребятся, уповающие же на Господа наследуют землю.
\vs Psa 36:10 Еще немного, и не станет нечестивого; посмотришь на его место, и нет его.
\vs Psa 36:11 А кроткие наследуют землю и насладятся множеством мира.
\vs Psa 36:12 Нечестивый злоумышляет против праведника и скрежещет на него зубами своими:
\vs Psa 36:13 Господь же посмевается над ним, ибо видит, что приходит день его.
\vs Psa 36:14 Нечестивые обнажают меч и натягивают лук свой, чтобы низложить бедного и нищего, чтобы пронзить \bibemph{идущих} прямым путем:
\vs Psa 36:15 меч их войдет в их же сердце, и луки их сокрушатся.
\vs Psa 36:16 Малое у праведника~--- лучше богатства многих нечестивых,
\vs Psa 36:17 ибо мышцы нечестивых сокрушатся, а праведников подкрепляет Господь.
\vs Psa 36:18 Господь знает дни непорочных, и достояние их пребудет вовек:
\vs Psa 36:19 не будут они постыжены во время лютое и во дни голода будут сыты;
\vs Psa 36:20 а нечестивые погибнут, и враги Господни, как тук агнцев, исчезнут, в дыме исчезнут.
\vs Psa 36:21 Нечестивый берет взаймы и не отдает, а праведник милует и дает,
\vs Psa 36:22 ибо благословенные Им наследуют землю, а проклятые Им истребятся.
\vs Psa 36:23 Господом утверждаются стопы \bibemph{такого} человека, и Он благоволит к пути его:
\vs Psa 36:24 когда он будет падать, не упадет, ибо Господь поддерживает его за руку.
\vs Psa 36:25 Я был молод и состарился, и не видал праведника оставленным и потомков его просящими хлеба:
\vs Psa 36:26 он всякий день милует и взаймы дает, и потомство его в благословение будет.
\vs Psa 36:27 Уклоняйся от зла, и делай добро, и будешь жить вовек:
\vs Psa 36:28 ибо Господь любит правду и не оставляет святых Своих; вовек сохранятся они; [а беззаконные будут извержены] и потомство нечестивых истребится.
\vs Psa 36:29 Праведники наследуют землю и будут жить на ней вовек.
\vs Psa 36:30 Уста праведника изрекают премудрость, и язык его произносит правду.
\vs Psa 36:31 Закон Бога его в сердце у него; не поколеблются стопы его.
\vs Psa 36:32 Нечестивый подсматривает за праведником и ищет умертвить его;
\vs Psa 36:33 но Господь не отдаст его в руки его и не допустит обвинить его, когда он будет судим.
\vs Psa 36:34 Уповай на Господа и держись пути Его: и Он вознесет тебя, чтобы ты наследовал землю; и когда будут истребляемы нечестивые, ты увидишь.
\vs Psa 36:35 Видел я нечестивца грозного, расширявшегося, подобно укоренившемуся многоветвистому дереву;
\vs Psa 36:36 но он прошел, и вот нет его; ищу его и не нахожу.
\vs Psa 36:37 Наблюдай за непорочным и смотри на праведного, ибо будущность \bibemph{такого} человека есть мир;
\vs Psa 36:38 а беззаконники все истребятся; будущность нечестивых погибнет.
\vs Psa 36:39 От Господа спасение праведникам, Он~--- защита их во время скорби;
\vs Psa 36:40 и поможет им Господь и избавит их; избавит их от нечестивых и спасет их, ибо они на Него уповают.
\vs Psa 37:1 Псалом Давида. В воспоминание [о субботе].
\rsbpar\vs Psa 37:2 Господи! не в ярости Твоей обличай меня и не во гневе Твоем наказывай меня,
\vs Psa 37:3 ибо стрелы Твои вонзились в меня, и рука Твоя тяготеет на мне.
\vs Psa 37:4 Нет целого места в плоти моей от гнева Твоего; нет мира в костях моих от грехов моих,
\vs Psa 37:5 ибо беззакония мои превысили голову мою, как тяжелое бремя отяготели на мне,
\vs Psa 37:6 смердят, гноятся раны мои от безумия моего.
\vs Psa 37:7 Я согбен и совсем поник, весь день сетуя хожу,
\vs Psa 37:8 ибо чресла мои полны воспалениями, и нет целого места в плоти моей.
\vs Psa 37:9 Я изнемог и сокрушен чрезмерно; кричу от терзания сердца моего.
\vs Psa 37:10 Господи! пред Тобою все желания мои, и воздыхание мое не сокрыто от Тебя.
\vs Psa 37:11 Сердце мое трепещет; оставила меня сила моя, и свет очей моих,~--- и того нет у меня.
\vs Psa 37:12 Друзья мои и искренние отступили от язвы моей, и ближние мои стоят вдали.
\vs Psa 37:13 Ищущие же души моей ставят сети, и желающие мне зла говорят о погибели \bibemph{моей} и замышляют всякий день козни;
\vs Psa 37:14 а я, как глухой, не слышу, и как немой, который не открывает уст своих;
\vs Psa 37:15 и стал я, как человек, который не слышит и не имеет в устах своих ответа,
\vs Psa 37:16 ибо на Тебя, Господи, уповаю я; Ты услышишь, Господи, Боже мой.
\vs Psa 37:17 И я сказал: да не восторжествуют надо мною [враги мои]; когда колеблется нога моя, они величаются надо мною.
\vs Psa 37:18 Я близок к падению, и скорбь моя всегда предо мною.
\vs Psa 37:19 Беззаконие мое я сознаю, сокрушаюсь о грехе моем.
\vs Psa 37:20 А враги мои живут и укрепляются, и умножаются ненавидящие меня безвинно;
\vs Psa 37:21 и воздающие мне злом за добро враждуют против меня за то, что я следую добру.
\vs Psa 37:22 Не оставь меня, Господи, Боже мой! Не удаляйся от меня;
\vs Psa 37:23 поспеши на помощь мне, Господи, Спаситель мой!
\vs Psa 38:1 Начальнику хора, Идифуму. Псалом Давида.
\rsbpar\vs Psa 38:2 Я сказал: буду я наблюдать за путями моими, чтобы не согрешать мне языком моим; буду обуздывать уста мои, доколе нечестивый предо мною.
\vs Psa 38:3 Я был нем и безгласен, и молчал \bibemph{даже} о добром; и скорбь моя подвиглась.
\vs Psa 38:4 Воспламенилось сердце мое во мне; в мыслях моих возгорелся огонь; я стал говорить языком моим:
\vs Psa 38:5 скажи мне, Господи, кончину мою и число дней моих, какое оно, дабы я знал, какой век мой.
\vs Psa 38:6 Вот, Ты дал мне дни, \bibemph{как} пяди, и век мой как ничто пред Тобою. Подлинно, совершенная суета~--- всякий человек живущий.
\vs Psa 38:7 Подлинно, человек ходит подобно призраку; напрасно он суетится, собирает и не знает, кому достанется то.
\vs Psa 38:8 И ныне чего ожидать мне, Господи? надежда моя~--- на Тебя.
\vs Psa 38:9 От всех беззаконий моих избавь меня, не предавай меня на поругание безумному.
\vs Psa 38:10 Я стал нем, не открываю уст моих; потому что Ты соделал это.
\vs Psa 38:11 Отклони от меня удары Твои; я исчезаю от поражающей руки Твоей.
\vs Psa 38:12 Если Ты обличениями будешь наказывать человека за преступления, то рассыплется, как от моли, краса его. Так, суетен всякий человек!
\vs Psa 38:13 Услышь, Господи, молитву мою и внемли воплю моему; не будь безмолвен к слезам моим, ибо странник я у Тебя \bibemph{и} пришлец, как и все отцы мои.
\vs Psa 38:14 Отступи от меня, чтобы я мог подкрепиться, прежде нежели отойду и не будет меня.
\vs Psa 39:1 Начальнику хора. Псалом Давида.
\rsbpar\vs Psa 39:2 Твердо уповал я на Господа, и Он приклонился ко мне и услышал вопль мой;
\vs Psa 39:3 извлек меня из страшного рва, из тинистого болота, и поставил на камне ноги мои и утвердил стопы мои;
\vs Psa 39:4 и вложил в уста мои новую песнь~--- хвалу Богу нашему. Увидят многие и убоятся и будут уповать на Господа.
\vs Psa 39:5 Блажен человек, который на Господа возлагает надежду свою и не обращается к гордым и к уклоняющимся ко лжи.
\vs Psa 39:6 Много соделал Ты, Господи, Боже мой: о чудесах и помышлениях Твоих о нас~--- кто уподобится Тебе!~--- хотел бы я проповедовать и говорить, но они превышают число.
\vs Psa 39:7 Жертвы и приношения Ты не восхотел; Ты открыл мне уши\fns{Открыл мне уши~--- по переводу 70-ти: уготовил мне тело.}; всесожжения и жертвы за грех Ты не потребовал.
\vs Psa 39:8 Тогда я сказал: вот, иду; в свитке книжном написано о мне:
\vs Psa 39:9 я желаю исполнить волю Твою, Боже мой, и закон Твой у меня в сердце.
\vs Psa 39:10 Я возвещал правду Твою в собрании великом; я не возбранял устам моим: Ты, Господи, знаешь.
\vs Psa 39:11 Правды Твоей не скрывал в сердце моем, возвещал верность Твою и спасение Твое, не утаивал милости Твоей и истины Твоей пред собранием великим.
\vs Psa 39:12 Не удерживай, Господи, щедрот Твоих от меня; милость Твоя и истина Твоя да охраняют меня непрестанно,
\vs Psa 39:13 ибо окружили меня беды неисчислимые; постигли меня беззакония мои, так что видеть не могу: их более, нежели волос на голове моей; сердце мое оставило меня.
\vs Psa 39:14 Благоволи, Господи, избавить меня; Господи! поспеши на помощь мне.
\vs Psa 39:15 Да постыдятся и посрамятся все, ищущие погибели душе моей! Да будут обращены назад и преданы посмеянию желающие мне зла!
\vs Psa 39:16 Да смятутся от посрамления своего говорящие мне: <<хорошо! хорошо!>>
\vs Psa 39:17 Да радуются и веселятся Тобою все ищущие Тебя, и любящие спасение Твое да говорят непрестанно: <<велик Господь!>>
\vs Psa 39:18 Я же беден и нищ, но Господь печется о мне. Ты~--- помощь моя и избавитель мой, Боже мой! не замедли.
\vs Psa 40:1 Начальнику хора. Псалом Давида.
\rsbpar\vs Psa 40:2 Блажен, кто помышляет о бедном [и нищем]! В день бедствия избавит его Господь.
\vs Psa 40:3 Господь сохранит его и сбережет ему жизнь; блажен будет он на земле. И Ты не отдашь его на волю врагов его.
\vs Psa 40:4 Господь укрепит его на одре болезни его. Ты изменишь все ложе его в болезни его.
\vs Psa 40:5 Я сказал: Господи! помилуй меня, исцели душу мою, ибо согрешил я пред Тобою.
\vs Psa 40:6 Враги мои говорят обо мне злое: <<когда он умрет и погибнет имя его?>>
\vs Psa 40:7 И если приходит кто видеть меня, говорит ложь; сердце его слагает в себе неправду, и он, выйдя вон, толкует.
\vs Psa 40:8 Все ненавидящие меня шепчут между собою против меня, замышляют на меня зло:
\vs Psa 40:9 <<слово велиала пришло на него; он слег; не встать ему более>>.
\vs Psa 40:10 Даже человек мирный со мною, на которого я полагался, который ел хлеб мой, поднял на меня пяту.
\vs Psa 40:11 Ты же, Господи, помилуй меня и восставь меня, и я воздам им.
\vs Psa 40:12 Из того узнаю, что Ты благоволишь ко мне, если враг мой не восторжествует надо мною,
\vs Psa 40:13 а меня сохранишь в целости моей и поставишь пред лицем Твоим на веки.
\vs Psa 40:14 Благословен Господь Бог Израилев от века и до века! Аминь, аминь!
\vs Psa 41:1 Начальнику хора. Учение. Сынов Кореевых.
\rsbpar\vs Psa 41:2 Как лань желает к потокам воды, так желает душа моя к Тебе, Боже!
\vs Psa 41:3 Жаждет душа моя к Богу крепкому, живому: когда приду и явлюсь пред лице Божие!
\vs Psa 41:4 Слезы мои были для меня хлебом день и ночь, когда говорили мне всякий день: <<где Бог твой?>>
\vs Psa 41:5 Вспоминая об этом, изливаю душу мою, потому что я ходил в многолюдстве, вступал с ними в дом Божий со гласом радости и славословия празднующего сонма.
\vs Psa 41:6 Что унываешь ты, душа моя, и что смущаешься? Уповай на Бога, ибо я буду еще славить Его, Спасителя моего и Бога моего.
\vs Psa 41:7 Унывает во мне душа моя; посему я воспоминаю о Тебе с земли Иорданской, с Ермона, с горы Цоар.
\vs Psa 41:8 Бездна бездну призывает голосом водопадов Твоих; все воды Твои и волны Твои прошли надо мною.
\vs Psa 41:9 Днем явит Господь милость Свою, и ночью песнь Ему у меня, молитва к Богу жизни моей.
\vs Psa 41:10 Скажу Богу, заступнику моему: для чего Ты забыл меня? Для чего я сетуя хожу от оскорблений врага?
\vs Psa 41:11 Как бы поражая кости мои, ругаются надо мною враги мои, когда говорят мне всякий день: <<где Бог твой?>>
\vs Psa 41:12 Что унываешь ты, душа моя, и что смущаешься? Уповай на Бога, ибо я буду еще славить Его, Спасителя моего и Бога моего.
\vs Psa 42:1 Суди меня, Боже, и вступись в тяжбу мою с народом недобрым. От человека лукавого и несправедливого избавь меня,
\vs Psa 42:2 ибо Ты Бог крепости моей. Для чего Ты отринул меня? для чего я сетуя хожу от оскорблений врага?
\vs Psa 42:3 Пошли свет Твой и истину Твою; да ведут они меня и приведут на святую гору Твою и в обители Твои.
\vs Psa 42:4 И подойду я к жертвеннику Божию, к Богу радости и веселия моего, и на гуслях буду славить Тебя, Боже, Боже мой!
\vs Psa 42:5 Что унываешь ты, душа моя, и что смущаешься? Уповай на Бога; ибо я буду еще славить Его, Спасителя моего и Бога моего.
\vs Psa 43:1 Начальнику хора. Учение. Сынов Кореевых.
\rsbpar\vs Psa 43:2 Боже, мы слышали ушами своими, отцы наши рассказывали нам о деле, какое Ты соделал во дни их, во дни древние:
\vs Psa 43:3 Ты рукою Твоею истребил народы, а их насадил; поразил племена и изгнал их;
\vs Psa 43:4 ибо они не мечом своим приобрели землю, и не их мышца спасла их, но Твоя десница и Твоя мышца и свет лица Твоего, ибо Ты благоволил к ним.
\vs Psa 43:5 Боже, Царь мой! Ты~--- тот же; даруй спасение Иакову.
\vs Psa 43:6 С Тобою избодаем рогами врагов наших; во имя Твое попрем ногами восстающих на нас:
\vs Psa 43:7 ибо не на лук мой уповаю, и не меч мой спасет меня;
\vs Psa 43:8 но Ты спасешь нас от врагов наших, и посрамишь ненавидящих нас.
\vs Psa 43:9 О Боге похвалимся всякий день, и имя Твое будем прославлять вовек.
\vs Psa 43:10 Но ныне Ты отринул и посрамил нас, и не выходишь с войсками нашими;
\vs Psa 43:11 обратил нас в бегство от врага, и ненавидящие нас грабят нас;
\vs Psa 43:12 Ты отдал нас, как овец, на съедение и рассеял нас между народами;
\vs Psa 43:13 без выгоды Ты продал народ Твой и не возвысил цены его;
\vs Psa 43:14 отдал нас на поношение соседям нашим, на посмеяние и поругание живущим вокруг нас;
\vs Psa 43:15 Ты сделал нас притчею между народами, покиванием головы между иноплеменниками.
\vs Psa 43:16 Всякий день посрамление мое предо мною, и стыд покрывает лице мое
\vs Psa 43:17 от голоса поносителя и клеветника, от взоров врага и мстителя:
\vs Psa 43:18 все это пришло на нас, но мы не забыли Тебя и не нарушили завета Твоего.
\vs Psa 43:19 Не отступило назад сердце наше, и стопы наши не уклонились от пути Твоего,
\vs Psa 43:20 когда Ты сокрушил нас в земле драконов и покрыл нас тенью смертною.
\vs Psa 43:21 Если бы мы забыли имя Бога нашего и простерли руки наши к богу чужому,
\vs Psa 43:22 то не взыскал ли бы сего Бог? Ибо Он знает тайны сердца.
\vs Psa 43:23 Но за Тебя умерщвляют нас всякий день, считают нас за овец, \bibemph{обреченных} на заклание.
\vs Psa 43:24 Восстань, что спишь, Господи! пробудись, не отринь навсегда.
\vs Psa 43:25 Для чего скрываешь лице Твое, забываешь скорбь нашу и угнетение наше?
\vs Psa 43:26 ибо душа наша унижена до праха, утроба наша прильнула к земле.
\vs Psa 43:27 Восстань на помощь нам и избавь нас ради милости Твоей.
\vs Psa 44:1 Начальнику хора. На \bibemph{музыкальном орудии} Шошан. Учение. Сынов Кореевых. Песнь любви.
\rsbpar\vs Psa 44:2 Излилось из сердца моего слово благое; я говорю: песнь моя о Царе; язык мой~--- трость скорописца.
\vs Psa 44:3 Ты прекраснее сынов человеческих; благодать излилась из уст Твоих; посему благословил Тебя Бог на веки.
\vs Psa 44:4 Препояшь Себя по бедру мечом Твоим, Сильный, славою Твоею и красотою Твоею,
\vs Psa 44:5 и в сем украшении Твоем поспеши, воссядь на колесницу ради истины и кротости и правды, и десница Твоя покажет Тебе дивные дела.
\vs Psa 44:6 Остры стрелы Твои, [Сильный],~--- народы падут пред Тобою,~--- они~--- в сердце врагов Царя.
\vs Psa 44:7 Престол Твой, Боже, вовек; жезл правоты~--- жезл царства Твоего.
\vs Psa 44:8 Ты возлюбил правду и возненавидел беззаконие, посему помазал Тебя, Боже, Бог Твой елеем радости более соучастников Твоих.
\vs Psa 44:9 Все одежды Твои, как смирна и алой и касия; из чертогов слоновой кости увеселяют Тебя.
\vs Psa 44:10 Дочери царей между почетными у Тебя; стала царица одесную Тебя в Офирском золоте.
\vs Psa 44:11 Слыши, дщерь, и смотри, и приклони ухо твое, и забудь народ твой и дом отца твоего.
\vs Psa 44:12 И возжелает Царь красоты твоей; ибо Он Господь твой, и ты поклонись Ему.
\vs Psa 44:13 И дочь Тира с дарами, и богатейшие из народа будут умолять лице Твое.
\vs Psa 44:14 Вся слава дщери Царя внутри; одежда ее шита золотом;
\vs Psa 44:15 в испещренной одежде ведется она к Царю; за нею ведутся к Тебе девы, подруги ее,
\vs Psa 44:16 приводятся с весельем и ликованьем, входят в чертог Царя.
\vs Psa 44:17 Вместо отцов Твоих, будут сыновья Твои; Ты поставишь их князьями по всей земле.
\vs Psa 44:18 Сделаю имя Твое памятным в род и род; посему народы будут славить Тебя во веки и веки.
\vs Psa 45:1 Начальнику хора. Сынов Кореевых. На \bibemph{музыкальном орудии} Аламоф. Песнь.
\rsbpar\vs Psa 45:2 Бог нам прибежище и сила, скорый помощник в бедах,
\vs Psa 45:3 посему не убоимся, хотя бы поколебалась земля, и горы двинулись в сердце морей.
\vs Psa 45:4 Пусть шумят, вздымаются воды их, трясутся горы от волнения их.
\vs Psa 45:5 Речные потоки веселят град Божий, святое жилище Всевышнего.
\vs Psa 45:6 Бог посреди его; он не поколеблется: Бог поможет ему с раннего утра.
\vs Psa 45:7 Восшумели народы; двинулись царства: [Всевышний] дал глас Свой, и растаяла земля.
\vs Psa 45:8 Господь сил с нами, Бог Иакова заступник наш.
\vs Psa 45:9 Придите и видите дела Господа,~--- какие произвел Он опустошения на земле:
\vs Psa 45:10 прекращая брани до края земли, сокрушил лук и переломил копье, колесницы сжег огнем.
\vs Psa 45:11 Остановитесь и познайте, что Я~--- Бог: буду превознесен в народах, превознесен на земле.
\vs Psa 45:12 Господь сил с нами, заступник наш Бог Иакова.
\vs Psa 46:1 Начальнику хора. Сынов Кореевых. Псалом.
\rsbpar\vs Psa 46:2 Восплещите руками все народы, воскликните Богу гласом радости;
\vs Psa 46:3 ибо Господь Всевышний страшен,~--- великий Царь над всею землею;
\vs Psa 46:4 покорил нам народы и племена под ноги наши;
\vs Psa 46:5 избрал нам наследие наше, красу Иакова, которого возлюбил.
\vs Psa 46:6 Восшел Бог при восклицаниях, Господь при звуке трубном.
\vs Psa 46:7 Пойте Богу нашему, пойте; пойте Царю нашему, пойте,
\vs Psa 46:8 ибо Бог~--- Царь всей земли; пойте все разумно.
\vs Psa 46:9 Бог воцарился над народами, Бог воссел на святом престоле Своем;
\vs Psa 46:10 князья народов собрались к народу Бога Авраамова, ибо щиты земли~--- Божии; Он превознесен \bibemph{над ними}.
\vs Psa 47:1 Песнь. Псалом. Сынов Кореевых.
\rsbpar\vs Psa 47:2 Велик Господь и всехвален во граде Бога нашего, на святой горе Его.
\vs Psa 47:3 Прекрасная возвышенность, радость всей земли гора Сион; на северной стороне \bibemph{ее} город великого Царя.
\vs Psa 47:4 Бог в жилищах его ведом, как заступник:
\vs Psa 47:5 ибо вот, сошлись цари и прошли все мимо;
\vs Psa 47:6 увидели и изумились, смутились и обратились в бегство;
\vs Psa 47:7 страх объял их там и мука, как у женщин в родах;
\vs Psa 47:8 восточным ветром Ты сокрушил Фарсийские корабли.
\vs Psa 47:9 Как слышали мы, так и увидели во граде Господа сил, во граде Бога нашего: Бог утвердит его на веки.
\vs Psa 47:10 Мы размышляли, Боже, о благости Твоей посреди храма Твоего.
\vs Psa 47:11 Как имя Твое, Боже, так и хвала Твоя до концов земли; десница Твоя полна правды.
\vs Psa 47:12 Да веселится гора Сион, [и] да радуются дщери Иудейские ради судов Твоих, [Господи].
\vs Psa 47:13 Пойдите вокруг Сиона и обойдите его, пересчитайте башни его;
\vs Psa 47:14 обратите сердце ваше к укреплениям его, рассмотрите домы его, чтобы пересказать грядущему роду,
\vs Psa 47:15 ибо сей Бог есть Бог наш на веки и веки: Он будет вождем нашим до самой смерти.
\vs Psa 48:1 Начальнику хора. Сынов Кореевых. Псалом.
\rsbpar\vs Psa 48:2 Слушайте сие, все народы; внимайте сему, все живущие во вселенной,~---
\vs Psa 48:3 и простые и знатные, богатый, равно как бедный.
\vs Psa 48:4 Уста мои изрекут премудрость, и размышления сердца моего~--- знание.
\vs Psa 48:5 Приклоню ухо мое к притче, на гуслях открою загадку мою:
\vs Psa 48:6 <<для чего бояться мне во дни бедствия, \bibemph{когда} беззаконие путей моих окружит меня?>>
\vs Psa 48:7 Надеющиеся на силы свои и хвалящиеся множеством богатства своего!
\vs Psa 48:8 человек никак не искупит брата своего и не даст Богу выкупа за него:
\vs Psa 48:9 дорог\acc{а} цена искупления души их, и не будет того вовек,
\vs Psa 48:10 чтобы остался \bibemph{кто} жить навсегда и не увидел могилы.
\vs Psa 48:11 Каждый видит, что и мудрые умирают, равно как и невежды и бессмысленные погибают и оставляют имущество свое другим.
\vs Psa 48:12 В мыслях у них, что домы их вечны, и что жилища их в род и род, и земли свои они называют своими именами.
\vs Psa 48:13 Но человек в чести не пребудет; он уподобится животным, которые погибают.
\vs Psa 48:14 Этот путь их есть безумие их, хотя последующие за ними одобряют мнение их.
\vs Psa 48:15 Как овец, заключат их в преисподнюю; смерть будет пасти их, и наутро праведники будут владычествовать над ними; сила их истощится; могила~--- жилище их.
\vs Psa 48:16 Но Бог избавит душу мою от власти преисподней, когда примет меня.
\vs Psa 48:17 Не бойся, когда богатеет человек, когда слава дома его умножается:
\vs Psa 48:18 ибо умирая не возьмет ничего; не пойдет за ним слава его;
\vs Psa 48:19 хотя при жизни он ублажает душу свою, и прославляют тебя, что ты удовлетворяешь себе,
\vs Psa 48:20 но он пойдет к роду отцов своих, которые никогда не увидят света.
\vs Psa 48:21 Человек, который в чести и неразумен, подобен животным, которые погибают.
\vs Psa 49:0 Псалом Асафа.
\rsbpar\vs Psa 49:1 Бог богов, Господь возглаголал и призывает землю, от восхода солнца до запада.
\vs Psa 49:2 С Сиона, который есть верх красоты, является Бог,
\vs Psa 49:3 грядет Бог наш, и не в безмолвии: пред Ним огонь поядающий, и вокруг Его сильная буря.
\vs Psa 49:4 Он призывает свыше небо и землю, судить народ Свой:
\vs Psa 49:5 <<соберите ко Мне святых Моих, вступивших в завет со Мною при жертве>>.
\vs Psa 49:6 И небеса провозгласят правду Его, ибо судия сей есть Бог.
\vs Psa 49:7 <<Слушай, народ Мой, Я буду говорить; Израиль! Я буду свидетельствовать против тебя: Я Бог, твой Бог.
\vs Psa 49:8 Не за жертвы твои Я буду укорять тебя; всесожжения твои всегда предо Мною;
\vs Psa 49:9 не приму тельца из дома твоего, ни козлов из дворов твоих,
\vs Psa 49:10 ибо Мои все звери в лесу, и скот на тысяче гор,
\vs Psa 49:11 знаю всех птиц на горах, и животные на полях предо Мною.
\vs Psa 49:12 Если бы Я взалкал, то не сказал бы тебе, ибо Моя вселенная и все, что наполняет ее.
\vs Psa 49:13 Ем ли Я мясо волов и пью ли кровь козлов?
\vs Psa 49:14 Принеси в жертву Богу хвалу и воздай Всевышнему обеты твои,
\vs Psa 49:15 и призови Меня в день скорби; Я избавлю тебя, и ты прославишь Меня>>.
\vs Psa 49:16 Грешнику же говорит Бог: <<что ты проповедуешь уставы Мои и берешь завет Мой в уста твои,
\vs Psa 49:17 а сам ненавидишь наставление Мое и слова Мои бросаешь за себя?
\vs Psa 49:18 когда видишь вора, сходишься с ним, и с прелюбодеями сообщаешься;
\vs Psa 49:19 уста твои открываешь на злословие, и язык твой сплетает коварство;
\vs Psa 49:20 сидишь и говоришь на брата твоего, на сына матери твоей клевещешь;
\vs Psa 49:21 ты это делал, и Я молчал; ты подумал, что Я такой же, как ты. Изобличу тебя и представлю пред глаза твои [грехи твои].
\vs Psa 49:22 Уразумейте это, забывающие Бога, дабы Я не восхитил,~--- и не будет избавляющего.
\vs Psa 49:23 Кто приносит в жертву хвалу, тот чтит Меня, и кто наблюдает за путем своим, тому явлю Я спасение Божие>>.
\vs Psa 50:1 Начальнику хора. Псалом Давида,
\vs Psa 50:2 когда приходил к нему пророк Нафан, после того, как Давид вошел к Вирсавии.
\rsbpar\vs Psa 50:3 Помилуй меня, Боже, по великой милости Твоей, и по множеству щедрот Твоих изгладь беззакония мои.
\vs Psa 50:4 Многократно омой меня от беззакония моего, и от греха моего очисти меня,
\vs Psa 50:5 ибо беззакония мои я сознаю, и грех мой всегда предо мною.
\vs Psa 50:6 Тебе, Тебе единому согрешил я и лукавое пред очами Твоими сделал, так что Ты праведен в приговоре Твоем и чист в суде Твоем.
\vs Psa 50:7 Вот, я в беззаконии зачат, и во грехе родила меня мать моя.
\vs Psa 50:8 Вот, Ты возлюбил истину в сердце и внутрь меня явил мне мудрость [Твою].
\vs Psa 50:9 Окропи меня иссопом, и буду чист; омой меня, и буду белее снега.
\vs Psa 50:10 Дай мне услышать радость и веселие, и возрадуются кости, Тобою сокрушенные.
\vs Psa 50:11 Отврати лице Твое от грехов моих и изгладь все беззакония мои.
\vs Psa 50:12 Сердце чистое сотвори во мне, Боже, и дух правый обнови внутри меня.
\vs Psa 50:13 Не отвергни меня от лица Твоего и Духа Твоего Святаго не отними от меня.
\vs Psa 50:14 Возврати мне радость спасения Твоего и Духом владычественным утверди меня.
\vs Psa 50:15 Научу беззаконных путям Твоим, и нечестивые к Тебе обратятся.
\vs Psa 50:16 Избавь меня от кровей, Боже, Боже спасения моего, и язык мой восхвалит правду Твою.
\vs Psa 50:17 Господи! отверзи уста мои, и уста мои возвестят хвалу Твою:
\vs Psa 50:18 ибо жертвы Ты не желаешь,~--- я дал бы ее; к всесожжению не благоволишь.
\vs Psa 50:19 Жертва Богу~--- дух сокрушенный; сердца сокрушенного и смиренного Ты не презришь, Боже.
\vs Psa 50:20 Облагодетельствуй, [Господи,] по благоволению Твоему Сион; воздвигни стены Иерусалима:
\vs Psa 50:21 тогда благоугодны будут Тебе жертвы правды, возношение и всесожжение; тогда возложат на алтарь Твой тельцов.
\vs Psa 51:1 Начальнику хора. Учение Давида,
\vs Psa 51:2 после того, как приходил Доик Идумеянин и донес Саулу и сказал ему, что Давид пришел в дом Ахимелеха.
\rsbpar\vs Psa 51:3 Что хвалишься злодейством, сильный? милость Божия всегда \bibemph{со мною};
\vs Psa 51:4 гибель вымышляет язык твой; как изощренная бритва, он \bibemph{у тебя}, коварный!
\vs Psa 51:5 ты любишь больше зло, нежели добро, больше ложь, нежели говорить правду;
\vs Psa 51:6 ты любишь всякие гибельные речи, язык коварный:
\vs Psa 51:7 за то Бог сокрушит тебя вконец, изринет тебя и исторгнет тебя из жилища [твоего] и корень твой из земли живых.
\vs Psa 51:8 Увидят праведники и убоятся, посмеются над ним [и скажут]:
\vs Psa 51:9 <<вот человек, который не в Боге полагал крепость свою, а надеялся на множество богатства своего, укреплялся в злодействе своем>>.
\vs Psa 51:10 А я, как зеленеющая маслина, в доме Божием, и уповаю на милость Божию во веки веков,
\vs Psa 51:11 вечно буду славить Тебя за то, что Ты соделал, и уповать на имя Твое, ибо оно благо пред святыми Твоими.
\vs Psa 52:1 Начальнику хора. На духовом \bibemph{орудии}. Учение Давида.
\rsbpar\vs Psa 52:2 Сказал безумец в сердце своем: <<нет Бога>>. Развратились они и совершили гнусные преступления; нет делающего добро.
\vs Psa 52:3 Бог с небес призрел на сынов человеческих, чтобы видеть, есть ли разумеющий, ищущий Бога.
\vs Psa 52:4 Все уклонились, сделались равно непотребными; нет делающего добро, нет ни одного.
\vs Psa 52:5 Неужели не вразумятся делающие беззаконие, съедающие народ мой, \bibemph{как} едят хлеб, и не призывающие Бога?
\vs Psa 52:6 Там убоятся они страха, где нет страха, ибо рассыплет Бог кости ополчающихся против тебя. Ты постыдишь их, потому что Бог отверг их.
\vs Psa 52:7 Кто даст с Сиона спасение Израилю! Когда Бог возвратит пленение народа Своего, тогда возрадуется Иаков и возвеселится Израиль.
\vs Psa 53:1 Начальнику хора. На струнных \bibemph{орудиях}. Учение Давида,
\vs Psa 53:2 когда пришли Зифеи и сказали Саулу: <<не у нас ли скрывается Давид?>>
\rsbpar\vs Psa 53:3 Боже! именем Твоим спаси меня, и силою Твоею суди меня.
\vs Psa 53:4 Боже! услышь молитву мою, внемли словам уст моих,
\vs Psa 53:5 ибо чужие восстали на меня, и сильные ищут души моей; они не имеют Бога пред собою.
\vs Psa 53:6 Вот, Бог помощник мой; Господь подкрепляет душу мою.
\vs Psa 53:7 Он воздаст за зло врагам моим; истиною Твоею истреби их.
\vs Psa 53:8 Я усердно принесу Тебе жертву, прославлю имя Твое, Господи, ибо оно благо,
\vs Psa 53:9 ибо Ты избавил меня от всех бед, и на врагов моих смотрело око мое.
\vs Psa 54:1 Начальнику хора. На струнных \bibemph{орудиях}. Учение Давида.
\rsbpar\vs Psa 54:2 Услышь, Боже, молитву мою и не скрывайся от моления моего;
\vs Psa 54:3 внемли мне и услышь меня; я стенаю в горести моей, и смущаюсь
\vs Psa 54:4 от голоса врага, от притеснения нечестивого, ибо они возводят на меня беззаконие и в гневе враждуют против меня.
\vs Psa 54:5 Сердце мое трепещет во мне, и смертные ужасы напали на меня;
\vs Psa 54:6 страх и трепет нашел на меня, и ужас объял меня.
\vs Psa 54:7 И я сказал: <<кто дал бы мне крылья, как у голубя? я улетел бы и успокоился бы;
\vs Psa 54:8 далеко удалился бы я, и оставался бы в пустыне;
\vs Psa 54:9 поспешил бы укрыться от вихря, от бури>>.
\vs Psa 54:10 Расстрой, Господи, и раздели языки их, ибо я вижу насилие и распри в городе;
\vs Psa 54:11 днем и ночью ходят они кругом по стенам его; злодеяния и бедствие посреди его;
\vs Psa 54:12 посреди его пагуба; обман и коварство не сходят с улиц его:
\vs Psa 54:13 ибо не враг поносит меня,~--- это я перенес бы; не ненавистник мой величается надо мною,~--- от него я укрылся бы;
\vs Psa 54:14 но ты, который был для меня то же, что я, друг мой и близкий мой,
\vs Psa 54:15 с которым мы разделяли искренние беседы и ходили вместе в дом Божий.
\vs Psa 54:16 Да найдет на них смерть; да сойдут они живыми в ад, ибо злодейство в жилищах их, посреди их.
\vs Psa 54:17 Я же воззову к Богу, и Господь спасет меня.
\vs Psa 54:18 Вечером и утром и в полдень буду умолять и вопиять, и Он услышит голос мой,
\vs Psa 54:19 избавит в мире душу мою от восстающих на меня, ибо их много у меня;
\vs Psa 54:20 услышит Бог, и смирит их от века Живущий, потому что нет в них перемены; они не боятся Бога,
\vs Psa 54:21 простерли руки свои на тех, которые с ними в мире, нарушили союз свой;
\vs Psa 54:22 уста их мягче масла, а в сердце их вражда; слова их нежнее елея, но они суть обнаженные мечи.
\vs Psa 54:23 Возложи на Господа заботы твои, и Он поддержит тебя. Никогда не даст Он поколебаться праведнику.
\vs Psa 54:24 Ты, Боже, низведешь их в ров погибели; кровожадные и коварные не доживут и до половины дней своих. А я на Тебя, [Господи,] уповаю.
\vs Psa 55:1 Начальнику хора. О голубице, безмолвствующей в удалении. Писание Давида, когда Филистимляне захватили его в Гефе.
\rsbpar\vs Psa 55:2 Помилуй меня, Боже! ибо человек хочет поглотить меня; нападая всякий день, теснит меня.
\vs Psa 55:3 Враги мои всякий день ищут поглотить меня, ибо много восстающих на меня, о, Всевышний!
\vs Psa 55:4 Когда я в страхе, на Тебя я уповаю.
\vs Psa 55:5 В Боге восхвалю я слово Его; на Бога уповаю, не боюсь; что сделает мне плоть?
\vs Psa 55:6 Всякий день извращают слова мои; все помышления их обо мне~--- на зло:
\vs Psa 55:7 собираются, притаиваются, наблюдают за моими пятами, чтобы уловить душу мою.
\vs Psa 55:8 Неужели они избегнут воздаяния за неправду \bibemph{свою}? Во гневе низложи, Боже, народы.
\vs Psa 55:9 У Тебя исчислены мои скитания; положи слезы мои в сосуд у Тебя,~--- не в книге ли они Твоей?
\vs Psa 55:10 Враги мои обращаются назад, когда я взываю к Тебе, из этого я узна\acc{ю}, что Бог за меня.
\vs Psa 55:11 В Боге восхвалю я слово \bibemph{Его}, в Господе восхвалю слово \bibemph{Его}.
\vs Psa 55:12 На Бога уповаю, не боюсь; что сделает мне человек?
\vs Psa 55:13 На мне, Боже, обеты Тебе; Тебе воздам хвалы,
\vs Psa 55:14 ибо Ты избавил душу мою от смерти, [очи мои от слез,] да и ноги мои от преткновения, чтобы я ходил пред лицем Божиим во свете живых.
\vs Psa 56:1 Начальнику хора. Не погуби. Писание Давида, когда он убежал от Саула в пещеру.
\rsbpar\vs Psa 56:2 Помилуй меня, Боже, помилуй меня, ибо на Тебя уповает душа моя, и в тени крыл Твоих я укроюсь, доколе не пройдут беды.
\vs Psa 56:3 Воззову к Богу Всевышнему, Богу, благодетельствующему мне;
\vs Psa 56:4 Он пошлет с небес и спасет меня; посрамит ищущего поглотить меня; пошлет Бог милость Свою и истину Свою.
\vs Psa 56:5 Душа моя среди львов; я лежу среди дышущих пламенем, среди сынов человеческих, у которых зубы~--- копья и стрелы, и у которых язык~--- острый меч.
\vs Psa 56:6 Будь превознесен выше небес, Боже, и над всею землею да будет слава Твоя!
\vs Psa 56:7 Приготовили сеть ногам моим; душа моя поникла; выкопали предо мною яму, и \bibemph{сами} упали в нее.
\vs Psa 56:8 Готово сердце мое, Боже, готово сердце мое: буду петь и славить.
\vs Psa 56:9 Воспрянь, слава моя, воспрянь, псалтирь и гусли! Я встану рано.
\vs Psa 56:10 Буду славить Тебя, Господи, между народами; буду воспевать Тебя среди племен,
\vs Psa 56:11 ибо до небес велика милость Твоя и до облаков истина Твоя.
\vs Psa 56:12 Будь превознесен выше небес, Боже, и над всею землею да будет слава Твоя!
\vs Psa 57:1 Начальнику хора. Не погуби. Писание Давида.
\rsbpar\vs Psa 57:2 Подлинно ли правду говорите вы, судьи, и справедливо судите, сыны человеческие?
\vs Psa 57:3 Беззаконие составляете в сердце, кладете на весы злодеяния рук ваших на земле.
\vs Psa 57:4 С самого рождения отступили нечестивые, от утробы \bibemph{матери} заблуждаются, говоря ложь.
\vs Psa 57:5 Яд у них~--- как яд змеи, как глухого аспида, который затыкает уши свои
\vs Psa 57:6 и не слышит голоса заклинателя, самого искусного в заклинаниях.
\vs Psa 57:7 Боже! сокруши зубы их в устах их; разбей, Господи, челюсти львов!
\vs Psa 57:8 Да исчезнут, как вода протекающая; когда напрягут стрелы, пусть они будут как переломленные.
\vs Psa 57:9 Да исчезнут, как распускающаяся улитка; да не видят солнца, как выкидыш женщины.
\vs Psa 57:10 Прежде нежели котлы ваши ощутят горящий терн, и свежее и обгоревшее да разнесет вихрь.
\vs Psa 57:11 Возрадуется праведник, когда увидит отмщение; омоет стопы свои в крови нечестивого.
\vs Psa 57:12 И скажет человек: <<подлинно есть плод праведнику! итак есть Бог, судящий на земле!>>
\vs Psa 58:1 Начальнику хора. Не погуби. Писание Давида, когда Саул послал стеречь дом его, чтобы умертвить его.
\rsbpar\vs Psa 58:2 Избавь меня от врагов моих, Боже мой! защити меня от восстающих на меня;
\vs Psa 58:3 избавь меня от делающих беззаконие; спаси от кровожадных,
\vs Psa 58:4 ибо вот, они подстерегают душу мою; собираются на меня сильные не за преступление мое и не за грех мой, Господи;
\vs Psa 58:5 без вины \bibemph{моей} сбегаются и вооружаются; подвигнись на помощь мне и воззри.
\vs Psa 58:6 Ты, Господи, Боже сил, Боже Израилев, восстань посетить все народы, не пощади ни одного из нечестивых беззаконников:
\vs Psa 58:7 вечером возвращаются они, воют, как псы, и ходят вокруг города;
\vs Psa 58:8 вот они изрыгают хулу языком своим; в устах их мечи: <<ибо>>, \bibemph{думают они}, <<кто слышит?>>
\vs Psa 58:9 Но Ты, Господи, посмеешься над ними; Ты посрамишь все народы.
\vs Psa 58:10 Сила~--- у них, но я к Тебе прибегаю, ибо Бог~--- заступник мой.
\vs Psa 58:11 Бог мой, милующий меня, предварит меня; Бог даст мне смотреть на врагов моих.
\vs Psa 58:12 Не умерщвляй их, чтобы не забыл народ мой; расточи их силою Твоею и низложи их, Господи, защитник наш.
\vs Psa 58:13 Слово языка их есть грех уст их, да уловятся они в гордости своей за клятву и ложь, которую произносят.
\vs Psa 58:14 Расточи их во гневе, расточи, чтобы их не было; и да познают, что Бог владычествует над Иаковом до пределов земли.
\vs Psa 58:15 Пусть возвращаются вечером, воют, как псы, и ходят вокруг города;
\vs Psa 58:16 пусть бродят, чтобы найти пищу, и несытые проводят ночи.
\vs Psa 58:17 А я буду воспевать силу Твою и с раннего утра провозглашать милость Твою, ибо Ты был мне защитою и убежищем в день бедствия моего.
\vs Psa 58:18 Сила моя! Тебя буду воспевать я, ибо Бог~--- заступник мой, Бог мой, милующий меня.
\vs Psa 59:1 Начальнику хора. На \bibemph{музыкальном орудии} Шушан-Эдуф. Писание Давида для изучения,
\vs Psa 59:2 когда он воевал с Сириею Месопотамскою и с Сириею Цованскою, и когда Иоав, возвращаясь, поразил двенадцать тысяч Идумеев в долине Соляной.
\rsbpar\vs Psa 59:3 Боже! Ты отринул нас, Ты сокрушил нас, Ты прогневался: обратись к нам.
\vs Psa 59:4 Ты потряс землю, разбил ее: исцели повреждения ее, ибо она колеблется.
\vs Psa 59:5 Ты дал испытать народу твоему жестокое, напоил нас вином изумления.
\vs Psa 59:6 Даруй боящимся Тебя знамя, чтобы они подняли его ради истины,
\vs Psa 59:7 чтобы избавились возлюбленные Твои; спаси десницею Твоею и услышь меня.
\vs Psa 59:8 Бог сказал во святилище Своем: <<восторжествую, разделю Сихем и долину Сокхоф размерю:
\vs Psa 59:9 Мой Галаад, Мой Манассия, Ефрем крепость главы Моей, Иуда скипетр Мой,
\vs Psa 59:10 Моав умывальная чаша Моя; на Едома простру сапог Мой. Восклицай Мне, земля Филистимская!>>
\vs Psa 59:11 Кто введет меня в укрепленный город? Кто доведет меня до Едома?
\vs Psa 59:12 Не Ты ли, Боже, \bibemph{Который} отринул нас, и не выходишь, Боже, с войсками нашими?
\vs Psa 59:13 Подай нам помощь в тесноте, ибо защита человеческая суетна.
\vs Psa 59:14 С Богом мы окажем силу, Он низложит врагов наших.
\vs Psa 60:1 Начальнику хора. На струнном \bibemph{орудии}. Псалом Давида.
\rsbpar\vs Psa 60:2 Услышь, Боже, вопль мой, внемли молитве моей!
\vs Psa 60:3 От конца земли взываю к Тебе в унынии сердца моего; возведи меня на скалу, для меня недосягаемую,
\vs Psa 60:4 ибо Ты прибежище мое, Ты крепкая защита от врага.
\vs Psa 60:5 Да живу я вечно в жилище Твоем и покоюсь под кровом крыл Твоих,
\vs Psa 60:6 ибо Ты, Боже, услышал обеты мои и дал \bibemph{мне} наследие боящихся имени Твоего.
\vs Psa 60:7 Приложи дни ко дням царя, лета его \bibemph{продли} в род и род,
\vs Psa 60:8 да пребудет он вечно пред Богом; заповедуй милости и истине охранять его.
\vs Psa 60:9 И я буду петь имени Твоему вовек, исполняя обеты мои всякий день.
\vs Psa 61:1 Начальнику хора Идифумова. Псалом Давида.
\rsbpar\vs Psa 61:2 Только в Боге успокаивается душа моя: от Него спасение мое.
\vs Psa 61:3 Только Он~--- твердыня моя, спасение мое, убежище мое: не поколеблюсь более.
\vs Psa 61:4 Доколе вы будете налегать на человека? Вы будете низринуты, все вы, как наклонившаяся стена, как ограда пошатнувшаяся.
\vs Psa 61:5 Они задумали свергнуть его с высоты, прибегли ко лжи; устами благословляют, а в сердце своем клянут.
\vs Psa 61:6 Только в Боге успокаивайся, душа моя! ибо на Него надежда моя.
\vs Psa 61:7 Только Он~--- твердыня моя и спасение мое, убежище мое: не поколеблюсь.
\vs Psa 61:8 В Боге спасение мое и слава моя; крепость силы моей и упование мое в Боге.
\vs Psa 61:9 Народ! надейтесь на Него во всякое время; изливайте пред Ним сердце ваше: Бог нам прибежище.
\vs Psa 61:10 Сыны человеческие~--- только суета; сыны мужей~--- ложь; если положить их на весы, все они вместе легче пустоты.
\vs Psa 61:11 Не надейтесь на грабительство и не тщеславьтесь хищением; когда богатство умножается, не прилагайте \bibemph{к нему} сердца.
\vs Psa 61:12 Однажды сказал Бог, и дважды слышал я это, что сила у Бога,
\vs Psa 61:13 и у Тебя, Господи, милость, ибо Ты воздаешь каждому по делам его.
\vs Psa 62:1 Псалом Давида, когда он был в пустыне Иудейской.
\rsbpar\vs Psa 62:2 Боже! Ты Бог мой, Тебя от ранней зари ищу я; Тебя жаждет душа моя, по Тебе томится плоть моя в земле пустой, иссохшей и безводной,
\vs Psa 62:3 чтобы видеть силу Твою и славу Твою, как я видел Тебя во святилище:
\vs Psa 62:4 ибо милость Твоя лучше, нежели жизнь. Уста мои восхвалят Тебя.
\vs Psa 62:5 Так благословлю Тебя в жизни моей; во имя Твое вознесу руки мои.
\vs Psa 62:6 Как туком и елеем насыщается душа моя, и радостным гласом восхваляют Тебя уста мои,
\vs Psa 62:7 когда я вспоминаю о Тебе на постели моей, размышляю о Тебе в \bibemph{ночные} стражи,
\vs Psa 62:8 ибо Ты помощь моя, и в тени крыл Твоих я возрадуюсь;
\vs Psa 62:9 к Тебе прилепилась душа моя; десница Твоя поддерживает меня.
\vs Psa 62:10 А те, которые ищут погибели душе моей, сойдут в преисподнюю земли;
\vs Psa 62:11 сразят их силою меча; достанутся они в добычу лисицам.
\vs Psa 62:12 Царь же возвеселится о Боге, восхвален будет всякий, клянущийся Им, ибо заградятся уста говорящих неправду.
\vs Psa 63:1 Начальнику хора. Псалом Давида.
\rsbpar\vs Psa 63:2 Услышь, Боже, голос мой в молитве моей, сохрани жизнь мою от страха врага;
\vs Psa 63:3 укрой меня от замысла коварных, от мятежа злодеев,
\vs Psa 63:4 которые изострили язык свой, как меч; напрягли лук свой~--- язвительное слово,
\vs Psa 63:5 чтобы втайне стрелять в непорочного; они внезапно стреляют в него и не боятся.
\vs Psa 63:6 Они утвердились в злом намерении, совещались скрыть сеть, говорили: кто их увидит?
\vs Psa 63:7 Изыскивают неправду, делают расследование за расследованием даже до внутренней жизни человека и до глубины сердца.
\vs Psa 63:8 Но поразит их Бог стрелою: внезапно будут они уязвлены;
\vs Psa 63:9 языком своим они поразят самих себя; все, видящие их, удалятся \bibemph{от них}.
\vs Psa 63:10 И убоятся все человеки, и возвестят дело Божие, и уразумеют, что это Его дело.
\vs Psa 63:11 А праведник возвеселится о Господе и будет уповать на Него; и похвалятся все правые сердцем.
\vs Psa 64:1 Начальнику хора. Псалом Давида для пения.
\rsbpar\vs Psa 64:2 Тебе, Боже, принадлежит хвала на Сионе, и Тебе воздастся обет [в Иерусалиме].
\vs Psa 64:3 Ты слышишь молитву; к Тебе прибегает всякая плоть.
\vs Psa 64:4 Дела беззаконий превозмогают меня; Ты очистишь преступления наши.
\vs Psa 64:5 Блажен, кого Ты избрал и приблизил, чтобы он жил во дворах Твоих. Насытимся благами дома Твоего, святаго храма Твоего.
\vs Psa 64:6 Страшный в правосудии, услышь нас, Боже, Спаситель наш, упование всех концов земли и находящихся в море далеко,
\vs Psa 64:7 поставивший горы силою Своею, препоясанный могуществом,
\vs Psa 64:8 укрощающий шум морей, шум волн их и мятеж народов!
\vs Psa 64:9 И убоятся знамений Твоих живущие на пределах \bibemph{земли}. Утро и вечер возбудишь к славе \bibemph{Твоей}.
\vs Psa 64:10 Ты посещаешь землю и утоляешь жажду ее, обильно обогащаешь ее: поток Божий полон воды; Ты приготовляешь хлеб, ибо так устроил ее;
\vs Psa 64:11 напояешь борозды ее, уравниваешь глыбы ее, размягчаешь ее каплями дождя, благословляешь произрастания ее;
\vs Psa 64:12 венчаешь лето благости Твоей, и стези Твои источают тук,
\vs Psa 64:13 источают на пустынные пажити, и холмы препоясываются радостью;
\vs Psa 64:14 луга одеваются стадами, и долины покрываются хлебом, восклицают и поют.
\vs Psa 65:0 Начальнику хора. Песнь.
\rsbpar\vs Psa 65:1 Воскликните Богу, вся земля.
\vs Psa 65:2 Пойте славу имени Его, воздайте славу, хвалу Ему.
\vs Psa 65:3 Скажите Богу: как страшен Ты в делах Твоих! По множеству силы Твоей, покорятся Тебе враги Твои.
\vs Psa 65:4 Вся земля да поклонится Тебе и поет Тебе, да поет имени Твоему, [Вышний]!
\vs Psa 65:5 Придите и воззрите на дела Бога, страшного в делах над сынами человеческими.
\vs Psa 65:6 Он превратил море в сушу; через реку перешли стопами, там веселились мы о Нем.
\vs Psa 65:7 Могуществом Своим владычествует Он вечно; очи Его зрят на народы, да не возносятся мятежники.
\vs Psa 65:8 Благословите, народы, Бога нашего и провозгласите хвалу Ему.
\vs Psa 65:9 Он сохранил душе нашей жизнь и ноге нашей не дал поколебаться.
\vs Psa 65:10 Ты испытал нас, Боже, переплавил нас, как переплавляют серебро.
\vs Psa 65:11 Ты ввел нас в сеть, положил оковы на чресла наши,
\vs Psa 65:12 посадил человека на главу нашу. Мы вошли в огонь и в воду, и Ты вывел нас на свободу.
\vs Psa 65:13 Войду в дом Твой со всесожжениями, воздам Тебе обеты мои,
\vs Psa 65:14 которые произнесли уста мои и изрек язык мой в скорби моей.
\vs Psa 65:15 Всесожжения тучные вознесу Тебе с воскурением тука овнов, принесу в жертву волов и козлов.
\vs Psa 65:16 Придите, послушайте, все боящиеся Бога, и я возвещу \bibemph{вам}, что сотворил Он для души моей.
\vs Psa 65:17 Я воззвал к Нему устами моими и превознес Его языком моим.
\vs Psa 65:18 Если бы я видел беззаконие в сердце моем, то не услышал бы меня Господь.
\vs Psa 65:19 Но Бог услышал, внял гласу моления моего.
\vs Psa 65:20 Благословен Бог, Который не отверг молитвы моей и не отвратил от меня милости Своей.
\vs Psa 66:1 Начальнику хора. На струнных \bibemph{орудиях}. Псалом. Песнь.
\rsbpar\vs Psa 66:2 Боже! будь милостив к нам и благослови нас, освети нас лицем Твоим,
\vs Psa 66:3 дабы познали на земле путь Твой, во всех народах спасение Твое.
\vs Psa 66:4 Да восхвалят Тебя народы, Боже; да восхвалят Тебя народы все.
\vs Psa 66:5 Да веселятся и радуются племена, ибо Ты судишь народы праведно и управляешь на земле племенами.
\vs Psa 66:6 Да восхвалят Тебя народы, Боже, да восхвалят Тебя народы все.
\vs Psa 66:7 Земля дала плод свой; да благословит нас Бог, Бог наш.
\vs Psa 66:8 Да благословит нас Бог, и да убоятся Его все пределы земли.
\vs Psa 67:1 Начальнику хора. Псалом Давида. Песнь.
\rsbpar\vs Psa 67:2 Да восстанет Бог\fns{В славянском переводе: Да воскреснет Бог\dots}, и расточатся враги Его, и да бегут от лица Его ненавидящие Его.
\vs Psa 67:3 Как рассеивается дым, Ты рассей их; как тает воск от огня, так нечестивые да погибнут от лица Божия.
\vs Psa 67:4 А праведники да возвеселятся, да возрадуются пред Богом и восторжествуют в радости.
\vs Psa 67:5 Пойте Богу нашему, пойте имени Его, превозносите Шествующего на небесах; имя Ему: Господь, и радуйтесь пред лицем Его.
\vs Psa 67:6 Отец сирот и судья вдов Бог во святом Своем жилище.
\vs Psa 67:7 Бог одиноких вводит в дом, освобождает узников от оков, а непокорные остаются в знойной пустыне.
\vs Psa 67:8 Боже! когда Ты выходил пред народом Твоим, когда Ты шествовал пустынею,
\vs Psa 67:9 земля тряслась, даже небеса таяли от лица Божия, и этот Синай~--- от лица Бога, Бога Израилева.
\vs Psa 67:10 Обильный дождь проливал Ты, Боже, на наследие Твое, и когда оно изнемогало от труда, Ты подкреплял его.
\vs Psa 67:11 Народ Твой обитал там; по благости Твоей, Боже, Ты готовил \bibemph{необходимое} для бедного.
\vs Psa 67:12 Господь даст слово: провозвестниц великое множество.
\vs Psa 67:13 Цари воинств бегут, бегут, а сидящая дома делит добычу.
\vs Psa 67:14 Расположившись в уделах [своих], вы стали, как голубица, которой крылья покрыты серебром, а перья чистым золотом:
\vs Psa 67:15 когда Всемогущий рассеял царей на сей \bibemph{земле}, она забелела, как снег на Селмоне.
\vs Psa 67:16 Гора Божия~--- гора Васанская! гора высокая~--- гора Васанская!
\vs Psa 67:17 что вы завистливо смотрите, горы высокие, на гору, на которой Бог благоволит обитать и будет Господь обитать вечно?
\vs Psa 67:18 Колесниц Божиих тьмы, тысячи тысяч; среди их Господь на Синае, во святилище.
\vs Psa 67:19 Ты восшел на высоту, пленил плен, принял дары для человеков, так чтоб и из противящихся могли обитать у Господа Бога.
\vs Psa 67:20 Благословен Господь всякий день. Бог возлагает на нас бремя, но Он же и спасает нас.
\vs Psa 67:21 Бог для нас~--- Бог во спасение; во власти Господа Вседержителя врата смерти.
\vs Psa 67:22 Но Бог сокрушит голову врагов Своих, волосатое темя закоснелого в своих беззакониях.
\vs Psa 67:23 Господь сказал: <<от Васана возвращу, выведу из глубины морской,
\vs Psa 67:24 чтобы ты погрузил ногу твою, как и псы твои язык свой, в крови врагов>>.
\vs Psa 67:25 Видели шествие Твое, Боже, шествие Бога моего, Царя моего во святыне:
\vs Psa 67:26 впереди шли поющие, позади играющие на орудиях, в средине девы с тимпанами:
\vs Psa 67:27 <<в собраниях благословите \bibemph{Бога Господа}, вы~--- от семени Израилева!>>
\vs Psa 67:28 Там Вениамин младший~--- князь их; князья Иудины~--- владыки их, князья Завулоновы, князья Неффалимовы.
\vs Psa 67:29 Бог твой предназначил тебе силу. Утверди, Боже, то, что Ты соделал для нас!
\vs Psa 67:30 Ради храма Твоего в Иерусалиме цари принесут Тебе дары.
\vs Psa 67:31 Укроти зверя в тростнике, стадо волов среди тельцов народов, хвалящихся слитками серебра; рассыпь народы, желающие браней.
\vs Psa 67:32 Придут вельможи из Египта; Ефиопия прострет руки свои к Богу.
\vs Psa 67:33 Царства земные! пойте Богу, воспевайте Господа,
\vs Psa 67:34 шествующего на небесах небес от века. Вот, Он дает гласу Своему глас силы.
\vs Psa 67:35 Воздайте славу Богу! величие Его~--- над Израилем, и могущество Его~--- на облаках.
\vs Psa 67:36 Страшен Ты, Боже, во святилище Твоем. Бог Израилев~--- Он дает силу и крепость народу [Своему]. Благословен Бог!
\vs Psa 68:1 Начальнику хора. На Шошанниме. Псалом Давида.
\rsbpar\vs Psa 68:2 Спаси меня, Боже, ибо воды дошли до души [моей].
\vs Psa 68:3 Я погряз в глубоком болоте, и не на чем стать; вошел во глубину вод, и быстрое течение их увлекает меня.
\vs Psa 68:4 Я изнемог от вопля, засохла гортань моя, истомились глаза мои от ожидания Бога [моего].
\vs Psa 68:5 Ненавидящих меня без вины больше, нежели волос на голове моей; враги мои, преследующие меня несправедливо, усилились; чего я не отнимал, то должен отдать.
\vs Psa 68:6 Боже! Ты знаешь безумие мое, и грехи мои не сокрыты от Тебя.
\vs Psa 68:7 Да не постыдятся во мне все, надеющиеся на Тебя, Господи, Боже сил. Да не посрамятся во мне ищущие Тебя, Боже Израилев,
\vs Psa 68:8 ибо ради Тебя несу я поношение, и бесчестием покрывают лице мое.
\vs Psa 68:9 Чужим стал я для братьев моих и посторонним для сынов матери моей,
\vs Psa 68:10 ибо ревность по доме Твоем снедает меня, и злословия злословящих Тебя падают на меня;
\vs Psa 68:11 и пл\acc{а}чу, постясь душею моею, и это ставят в поношение мне;
\vs Psa 68:12 и возлагаю на себя вместо одежды вретище,~--- и делаюсь для них притчею;
\vs Psa 68:13 о мне толкуют сидящие у ворот, и поют в песнях пьющие вино.
\vs Psa 68:14 А я с молитвою моею к Тебе, Господи; во время благоугодное, Боже, по великой благости Твоей услышь меня в истине спасения Твоего;
\vs Psa 68:15 извлеки меня из тины, чтобы не погрязнуть мне; да избавлюсь от ненавидящих меня и от глубоких вод;
\vs Psa 68:16 да не увлечет меня стремление вод, да не поглотит меня пучина, да не затворит надо мною пропасть зева своего.
\vs Psa 68:17 Услышь меня, Господи, ибо блага милость Твоя; по множеству щедрот Твоих призри на меня;
\vs Psa 68:18 не скрывай лица Твоего от раба Твоего, ибо я скорблю; скоро услышь меня;
\vs Psa 68:19 приблизься к душе моей, избавь ее; ради врагов моих спаси меня.
\vs Psa 68:20 Ты знаешь поношение мое, стыд мой и посрамление мое: враги мои все пред Тобою.
\vs Psa 68:21 Поношение сокрушило сердце мое, и я изнемог, ждал сострадания, но нет его,~--- утешителей, но не нахожу.
\vs Psa 68:22 И дали мне в пищу желчь, и в жажде моей напоили меня уксусом.
\vs Psa 68:23 Да будет трапеза их сетью им, и мирное пиршество их~--- западнею;
\vs Psa 68:24 да помрачатся глаза их, чтоб им не видеть, и чресла их расслабь навсегда;
\vs Psa 68:25 излей на них ярость Твою, и пламень гнева Твоего да обымет их;
\vs Psa 68:26 жилище их да будет пусто, и в шатрах их да не будет живущих,
\vs Psa 68:27 ибо, кого Ты поразил, они \bibemph{еще} преследуют, и страдания уязвленных Тобою умножают.
\vs Psa 68:28 Приложи беззаконие к беззаконию их, и да не войдут они в правду Твою;
\vs Psa 68:29 да изгладятся они из книги живых и с праведниками да не напишутся.
\vs Psa 68:30 А я беден и страдаю; помощь Твоя, Боже, да восставит меня.
\vs Psa 68:31 Я буду славить имя Бога [моего] в песни, буду превозносить Его в славословии,
\vs Psa 68:32 и будет это благоугоднее Господу, нежели вол, нежели телец с рогами и с копытами.
\vs Psa 68:33 Увидят \bibemph{это} страждущие и возрадуются. И оживет сердце ваше, ищущие Бога,
\vs Psa 68:34 ибо Господь внемлет нищим и не пренебрегает узников Своих.
\vs Psa 68:35 Да восхвалят Его небеса и земля, моря и все движущееся в них;
\vs Psa 68:36 ибо спасет Бог Сион, создаст города Иудины, и поселятся там и наследуют его,
\vs Psa 68:37 и потомство рабов Его утвердится в нем, и любящие имя Его будут поселяться на нем.
\vs Psa 69:1 Начальнику хора. Псалом Давида. В воспоминание.
\rsbpar\vs Psa 69:2 Поспеши, Боже, избавить меня, \bibemph{поспеши}, Господи, на помощь мне.
\vs Psa 69:3 Да постыдятся и посрамятся ищущие души моей! Да будут обращены назад и преданы посмеянию желающие мне зла!
\vs Psa 69:4 Да будут обращены назад за поношение меня говорящие [мне]: <<хорошо! хорошо!>>
\vs Psa 69:5 Да возрадуются и возвеселятся о Тебе все, ищущие Тебя, и любящие спасение Твое да говорят непрестанно: <<велик Бог!>>
\vs Psa 69:6 Я же беден и нищ; Боже, поспеши ко мне! Ты помощь моя и Избавитель мой; Господи! не замедли.
\vs Psa 70:1 На Тебя, Господи, уповаю, да не постыжусь вовек.
\vs Psa 70:2 По правде Твоей избавь меня и освободи меня; приклони ухо Твое ко мне и спаси меня.
\vs Psa 70:3 Будь мне твердым прибежищем, куда я всегда мог бы укрываться; Ты заповедал спасти меня, ибо твердыня моя и крепость моя~--- Ты.
\vs Psa 70:4 Боже мой! избавь меня из руки нечестивого, из руки беззаконника и притеснителя,
\vs Psa 70:5 ибо Ты~--- надежда моя, Господи Боже, упование мое от юности моей.
\vs Psa 70:6 На Тебе утверждался я от утробы; Ты извел меня из чрева матери моей; Тебе хвала моя не престанет.
\vs Psa 70:7 Для многих я был как бы дивом, но Ты твердая моя надежда.
\vs Psa 70:8 Да наполнятся уста мои хвалою, [чтобы мне воспевать славу Твою,] всякий день великолепие Твое.
\vs Psa 70:9 Не отвергни меня во время старости; когда будет оскудевать сила моя, не оставь меня,
\vs Psa 70:10 ибо враги мои говорят против меня, и подстерегающие душу мою советуются между собою,
\vs Psa 70:11 говоря: <<Бог оставил его; преследуйте и схватите его, ибо нет избавляющего>>.
\vs Psa 70:12 Боже! не удаляйся от меня; Боже мой! поспеши на помощь мне.
\vs Psa 70:13 Да постыдятся и исчезнут враждующие против души моей, да покроются стыдом и бесчестием ищущие мне зла!
\vs Psa 70:14 А я всегда буду уповать [на Тебя] и умножать всякую хвалу Тебе.
\vs Psa 70:15 Уста мои будут возвещать правду Твою, всякий день благодеяния Твои; ибо я не знаю им числа.
\vs Psa 70:16 Войду в \bibemph{размышление} о силах Господа Бога; воспомяну правду Твою~--- единственно Твою.
\vs Psa 70:17 Боже! Ты наставлял меня от юности моей, и доныне я возвещаю чудеса Твои.
\vs Psa 70:18 И до старости, и до седины не оставь меня, Боже, доколе не возвещу силы Твоей роду сему и всем грядущим могущества Твоего.
\vs Psa 70:19 Правда Твоя, Боже, до превыспренних; великие дела соделал Ты; Боже, кто подобен Тебе?
\vs Psa 70:20 Ты посылал на меня многие и лютые беды, но и опять оживлял меня и из бездн земли опять выводил меня.
\vs Psa 70:21 Ты возвышал меня и утешал меня, [и из бездн земли выводил меня].
\vs Psa 70:22 И я буду славить Тебя на псалтири, Твою истину, Боже мой; буду воспевать Тебя на гуслях, Святый Израилев!
\vs Psa 70:23 Радуются уста мои, когда я пою Тебе, и душа моя, которую Ты избавил;
\vs Psa 70:24 и язык мой всякий день будет возвещать правду Твою, ибо постыжены и посрамлены ищущие мне зла.
\vs Psa 71:0 О Соломоне. [Псалом Давида.]
\rsbpar\vs Psa 71:1 Боже! даруй царю Твой суд и сыну царя Твою правду,
\vs Psa 71:2 да судит праведно людей Твоих и нищих Твоих на суде;
\vs Psa 71:3 да принесут горы мир людям и холмы правду;
\vs Psa 71:4 да судит нищих народа, да спасет сынов убогого и смирит притеснителя,~---
\vs Psa 71:5 и будут бояться Тебя, доколе пребудут солнце и луна, в роды родов.
\vs Psa 71:6 Он сойдет, как дождь на скошенный луг, как капли, орошающие землю;
\vs Psa 71:7 во дни его процветет праведник, и будет обилие мира, доколе не престанет луна;
\vs Psa 71:8 он будет обладать от моря до моря и от реки\fns{Евфрат.} до концов земли;
\vs Psa 71:9 падут пред ним жители пустынь, и враги его будут лизать прах;
\vs Psa 71:10 цари Фарсиса и островов поднесут ему дань; цари Аравии и Савы принесут дары;
\vs Psa 71:11 и поклонятся ему все цари; все народы будут служить ему;
\vs Psa 71:12 ибо он избавит нищего, вопиющего и угнетенного, у которого нет помощника.
\vs Psa 71:13 Будет милосерд к нищему и убогому, и души убогих спасет;
\vs Psa 71:14 от коварства и насилия избавит души их, и драгоценна будет кровь их пред очами его;
\vs Psa 71:15 и будет жить, и будут давать ему от золота Аравии, и будут молиться о нем непрестанно, всякий день благословлять его;
\vs Psa 71:16 будет обилие хлеба на земле, наверху гор; плоды его будут волноваться, как \bibemph{лес} на Ливане, и в городах размножатся люди, как трава на земле;
\vs Psa 71:17 будет имя его [благословенно] вовек; доколе пребывает солнце, будет передаваться имя его\fns{В славянском переводе: Прежде солнца пребывает имя его.}; и благословятся в нем [все племена земные], все народы ублажат его.
\vs Psa 71:18 Благословен Господь Бог, Бог Израилев, един творящий чудеса,
\vs Psa 71:19 и благословенно имя славы Его вовек, и наполнится славою Его вся земля. Аминь и аминь.
\vs Psa 71:20 Кончились молитвы Давида, сына Иесеева.
\vs Psa 72:0 Псалом Асафа.
\rsbpar\vs Psa 72:1 Как благ Бог к Израилю, к чистым сердцем!
\vs Psa 72:2 А я~--- едва не пошатнулись ноги мои, едва не поскользнулись стопы мои,~---
\vs Psa 72:3 я позавидовал безумным, видя благоденствие нечестивых,
\vs Psa 72:4 ибо им нет страданий до смерти их, и крепки силы их;
\vs Psa 72:5 на работе человеческой нет их, и с \bibemph{прочими} людьми не подвергаются ударам.
\vs Psa 72:6 Оттого гордость, как ожерелье, обложила их, и дерзость, \bibemph{как} наряд, одевает их;
\vs Psa 72:7 выкатились от жира глаза их, бродят помыслы в сердце;
\vs Psa 72:8 над всем издеваются, злобно разглашают клевету, говорят свысока;
\vs Psa 72:9 поднимают к небесам уста свои, и язык их расхаживает по земле.
\vs Psa 72:10 Потому туда же обращается народ Его, и пьют воду полною чашею,
\vs Psa 72:11 и говорят: <<как узнает Бог? и есть ли ведение у Вышнего?>>
\vs Psa 72:12 И вот, эти нечестивые благоденствуют в веке сем, умножают богатство.
\vs Psa 72:13 [И я сказал:] так не напрасно ли я очищал сердце мое и омывал в невинности руки мои,
\vs Psa 72:14 и подвергал себя ранам всякий день и обличениям всякое утро?
\vs Psa 72:15 \bibemph{Но} если бы я сказал: <<буду рассуждать так>>,~--- то я виновен был бы пред родом сынов Твоих.
\vs Psa 72:16 И думал я, как бы уразуметь это, но это трудно было в глазах моих,
\vs Psa 72:17 доколе не вошел я во святилище Божие и не уразумел конца их.
\vs Psa 72:18 Так! на скользких путях поставил Ты их и низвергаешь их в пропасти.
\vs Psa 72:19 Как нечаянно пришли они в разорение, исчезли, погибли от ужасов!
\vs Psa 72:20 Как сновидение по пробуждении, так Ты, Господи, пробудив \bibemph{их}, уничтожишь мечты их.
\vs Psa 72:21 Когда кипело сердце мое, и терзалась внутренность моя,
\vs Psa 72:22 тогда я был невежда и не разумел; как скот был я пред Тобою.
\vs Psa 72:23 Но я всегда с Тобою: Ты держишь меня за правую руку;
\vs Psa 72:24 Ты руководишь меня советом Твоим и потом примешь меня в славу.
\vs Psa 72:25 Кто мне на небе? и с Тобою ничего не хочу на земле.
\vs Psa 72:26 Изнемогает плоть моя и сердце мое: Бог твердыня сердца моего и часть моя вовек.
\vs Psa 72:27 Ибо вот, удаляющие себя от Тебя гибнут; Ты истребляешь всякого отступающего от Тебя.
\vs Psa 72:28 А мне благо приближаться к Богу! На Господа Бога я возложил упование мое, чтобы возвещать все дела Твои [во вратах дщери Сионовой].
\vs Psa 73:0 Учение Асафа.
\rsbpar\vs Psa 73:1 Для чего, Боже, отринул нас навсегда? возгорелся гнев Твой на овец пажити Твоей?
\vs Psa 73:2 Вспомни сонм Твой, \bibemph{который} Ты стяжал издревле, искупил в жезл достояния Твоего,~--- эту гору Сион, на которой Ты вселился.
\vs Psa 73:3 Подвигни стопы Твои к вековым развалинам: все разрушил враг во святилище.
\vs Psa 73:4 Рыкают враги Твои среди собраний Твоих; поставили знаки свои вместо знамений \bibemph{наших};
\vs Psa 73:5 показывали себя подобными поднимающему вверх секиру на сплетшиеся ветви дерева;
\vs Psa 73:6 и ныне все резьбы в нем в один раз разрушили секирами и бердышами;
\vs Psa 73:7 предали огню святилище Твое; совсем осквернили жилище имени Твоего;
\vs Psa 73:8 сказали в сердце своем: <<разорим их совсем>>,~--- и сожгли все места собраний Божиих на земле.
\vs Psa 73:9 Знамений наших мы не видим, нет уже пророка, и нет с нами, кто знал бы, доколе \bibemph{это будет}.
\vs Psa 73:10 Доколе, Боже, будет поносить враг? вечно ли будет хулить противник имя Твое?
\vs Psa 73:11 Для чего отклоняешь руку Твою и десницу Твою? Из среды недра Твоего порази \bibemph{их}.
\vs Psa 73:12 Боже, Царь мой от века, устрояющий спасение посреди земли!
\vs Psa 73:13 Ты расторг силою Твоею море, Ты сокрушил головы змиев в воде;
\vs Psa 73:14 Ты сокрушил голову левиафана, отдал его в пищу людям пустыни, [Ефиопским];
\vs Psa 73:15 Ты иссек источник и поток, Ты иссушил сильные реки.
\vs Psa 73:16 Твой день и Твоя ночь: Ты уготовал светила и солнце;
\vs Psa 73:17 Ты установил все пределы земли, лето и зиму Ты учредил.
\vs Psa 73:18 Вспомни же: враг поносит Господа, и люди безумные хулят имя Твое.
\vs Psa 73:19 Не предай зверям душу горлицы Твоей; собрания убогих Твоих не забудь навсегда.
\vs Psa 73:20 Призри на завет Твой; ибо наполнились все мрачные места земли жилищами насилия.
\vs Psa 73:21 Да не возвратится угнетенный посрамленным; нищий и убогий да восхвалят имя Твое.
\vs Psa 73:22 Восстань, Боже, защити дело Твое, вспомни вседневное поношение Твое от безумного;
\vs Psa 73:23 не забудь крика врагов Твоих; шум восстающих против Тебя непрестанно поднимается.
\vs Psa 74:1 Начальнику хора. Не погуби. Псалом Асафа. Песнь.
\rsbpar\vs Psa 74:2 Славим Тебя, Боже, славим, ибо близко имя Твое; возвещают чудеса Твои.
\vs Psa 74:3 <<Когда изберу время, Я произведу суд по правде.
\vs Psa 74:4 Колеблется земля и все живущие на ней: Я утвержу столпы ее>>.
\vs Psa 74:5 Говорю безумствующим: <<не безумствуйте>>, и нечестивым: <<не поднимайте р\acc{о}га,
\vs Psa 74:6 не поднимайте высоко р\acc{о}га вашего, [не] говорите [на Бога] жестоковыйно>>,
\vs Psa 74:7 ибо не от востока и не от запада и не от пустыни возвышение,
\vs Psa 74:8 но Бог есть судия: одного унижает, а другого возносит;
\vs Psa 74:9 ибо чаша в руке Господа, вино кипит в ней, полное смешения, и Он наливает из нее. Даже дрожжи ее будут выжимать и пить все нечестивые земли.
\vs Psa 74:10 А я буду возвещать вечно, буду воспевать Бога Иаковлева,
\vs Psa 74:11 все роги нечестивых сломлю, и вознесутся роги праведника.
\vs Psa 75:1 Начальнику хора. На струнных \bibemph{орудиях}. Псалом Асафа. Песнь.
\rsbpar\vs Psa 75:2 Ведом в Иудее Бог; у Израиля велико имя Его.
\vs Psa 75:3 И было в Салиме жилище Его и пребывание Его на Сионе.
\vs Psa 75:4 Там сокрушил Он стрелы лука, щит и меч и брань.
\vs Psa 75:5 Ты славен, могущественнее гор хищнических.
\vs Psa 75:6 Крепкие сердцем стали добычею, уснули сном своим, и не нашли все мужи силы рук своих.
\vs Psa 75:7 От прещения Твоего, Боже Иакова, вздремали и колесница и конь.
\vs Psa 75:8 Ты страшен, и кто устоит пред лицем Твоим во время гнева Твоего?
\vs Psa 75:9 С небес Ты возвестил суд; земля убоялась и утихла,
\vs Psa 75:10 когда восстал Бог на суд, чтобы спасти всех угнетенных земли.
\vs Psa 75:11 И гнев человеческий обратится во славу Тебе: остаток гнева Ты укротишь.
\vs Psa 75:12 Делайте и воздавайте обеты Господу, Богу вашему; все, которые вокруг Него, да принесут дары Страшному:
\vs Psa 75:13 Он укрощает дух князей, Он страшен для царей земных.
\vs Psa 76:1 Начальнику хора Идифумова. Псалом Асафа.
\rsbpar\vs Psa 76:2 Глас мой к Богу, и я буду взывать; глас мой к Богу, и Он услышит меня.
\vs Psa 76:3 В день скорби моей ищу Господа; рука моя простерта ночью и не опускается; душа моя отказывается от утешения.
\vs Psa 76:4 Вспоминаю о Боге и трепещу; помышляю, и изнемогает дух мой.
\vs Psa 76:5 Ты не даешь мне сомкнуть очей моих; я потрясен и не могу говорить.
\vs Psa 76:6 Размышляю о днях древних, о летах веков \bibemph{минувших};
\vs Psa 76:7 припоминаю песни мои в ночи, беседую с сердцем моим, и дух мой испытывает:
\vs Psa 76:8 неужели навсегда отринул Господь, и не будет более благоволить?
\vs Psa 76:9 неужели навсегда престала милость Его, и пресеклось слово Его в род и род?
\vs Psa 76:10 неужели Бог забыл миловать? Неужели во гневе затворил щедроты Свои?
\vs Psa 76:11 И сказал я: <<вот мое горе~--- изменение десницы Всевышнего>>.
\vs Psa 76:12 Буду вспоминать о делах Господа; буду вспоминать о чудесах Твоих древних;
\vs Psa 76:13 буду вникать во все дела Твои, размышлять о великих Твоих деяниях.
\vs Psa 76:14 Боже! свят путь Твой. Кто Бог так великий, как Бог [наш]!
\vs Psa 76:15 Ты~--- Бог, творящий чудеса; Ты явил могущество Свое среди народов;
\vs Psa 76:16 Ты избавил мышцею народ Твой, сынов Иакова и Иосифа.
\vs Psa 76:17 Видели Тебя, Боже, воды, видели Тебя воды и убоялись, и вострепетали бездны.
\vs Psa 76:18 Облака изливали воды, тучи издавали гром, и стрелы Твои летали.
\vs Psa 76:19 Глас грома Твоего в круге небесном; молнии освещали вселенную; земля содрогалась и тряслась.
\vs Psa 76:20 Путь Твой в море, и стезя Твоя в водах великих, и следы Твои неведомы.
\vs Psa 76:21 Как стадо, вел Ты народ Твой рукою Моисея и Аарона.
\vs Psa 77:0 Учение Асафа.
\rsbpar\vs Psa 77:1 Внимай, народ мой, закону моему, приклоните ухо ваше к словам уст моих.
\vs Psa 77:2 Открою уста мои в притче и произнесу гадания из древности.
\vs Psa 77:3 Что слышали мы и узнали, и отцы наши рассказали нам,
\vs Psa 77:4 не скроем от детей их, возвещая роду грядущему славу Господа, и силу Его, и чудеса Его, которые Он сотворил.
\vs Psa 77:5 Он постановил устав в Иакове и положил закон в Израиле, который заповедал отцам нашим возвещать детям их,
\vs Psa 77:6 чтобы знал грядущий род, дети, которые родятся, и чтобы они в свое время возвещали своим детям,~---
\vs Psa 77:7 возлагать надежду свою на Бога и не забывать дел Божиих, и хранить заповеди Его,
\vs Psa 77:8 и не быть подобными отцам их, роду упорному и мятежному, неустроенному сердцем и неверному Богу духом своим.
\vs Psa 77:9 Сыны Ефремовы, вооруженные, стреляющие из луков, обратились назад в день брани:
\vs Psa 77:10 они не сохранили завета Божия и отреклись ходить в законе Его;
\vs Psa 77:11 забыли дела Его и чудеса, которые Он явил им.
\vs Psa 77:12 Он пред глазами отцов их сотворил чудеса в земле Египетской, на поле Цоан:
\vs Psa 77:13 разделил море, и провел их чрез него, и поставил воды стеною;
\vs Psa 77:14 и днем вел их облаком, а во всю ночь светом огня;
\vs Psa 77:15 рассек камень в пустыне и напоил их, как из великой бездны;
\vs Psa 77:16 из скалы извел потоки, и воды потекли, как реки.
\vs Psa 77:17 Но они продолжали грешить пред Ним и раздражать Всевышнего в пустыне:
\vs Psa 77:18 искушали Бога в сердце своем, требуя пищи по душе своей,
\vs Psa 77:19 и говорили против Бога и сказали: <<может ли Бог приготовить трапезу в пустыне?>>
\vs Psa 77:20 Вот, Он ударил в камень, и потекли воды, и полились ручьи. <<Может ли Он дать и хлеб, может ли приготовлять мясо народу Своему?>>
\vs Psa 77:21 Господь услышал и воспламенился гневом, и огонь возгорелся на Иакова, и гнев подвигнулся на Израиля
\vs Psa 77:22 за то, что не веровали в Бога и не уповали на спасение Его.
\vs Psa 77:23 Он повелел облакам свыше и отверз двери неба,
\vs Psa 77:24 и одождил на них манну в пищу, и хлеб небесный дал им.
\vs Psa 77:25 Хлеб ангельский ел человек; послал Он им пищу до сытости.
\vs Psa 77:26 Он возбудил на небе восточный ветер и навел южный силою Своею
\vs Psa 77:27 и, как пыль, одождил на них мясо и, как песок морской, птиц пернатых:
\vs Psa 77:28 поверг их среди стана их, около жилищ их,~---
\vs Psa 77:29 и они ели и пресытились; и желаемое ими дал им.
\vs Psa 77:30 Но еще не прошла прихоть их, еще пища была в устах их,
\vs Psa 77:31 гнев Божий пришел на них, убил тучных их и юношей Израилевых низложил.
\vs Psa 77:32 При всем этом они продолжали грешить и не верили чудесам Его.
\vs Psa 77:33 И погубил дни их в суете и лета их в смятении.
\vs Psa 77:34 Когда Он убивал их, они искали Его и обращались, и с раннего утра прибегали к Богу,
\vs Psa 77:35 и вспоминали, что Бог~--- их прибежище, и Бог Всевышний~--- Избавитель их,
\vs Psa 77:36 и льстили Ему устами своими и языком своим лгали пред Ним;
\vs Psa 77:37 сердце же их было неправо пред Ним, и они не были верны завету Его.
\vs Psa 77:38 Но Он, Милостивый, прощал грех и не истреблял их, многократно отвращал гнев Свой и не возбуждал всей ярости Своей:
\vs Psa 77:39 Он помнил, что они плоть, дыхание, которое уходит и не возвращается.
\vs Psa 77:40 Сколько раз они раздражали Его в пустыне и прогневляли Его в \bibemph{стране} необитаемой!
\vs Psa 77:41 и снова искушали Бога и оскорбляли Святаго Израилева,
\vs Psa 77:42 не помнили рук\acc{и} Его, дня, когда Он избавил их от угнетения,
\vs Psa 77:43 когда сотворил в Египте знамения Свои и чудеса Свои на поле Цоан;
\vs Psa 77:44 и превратил реки их и потоки их в кровь, чтобы они не могли пить;
\vs Psa 77:45 послал на них насекомых, чтобы жалили их, и жаб, чтобы губили их;
\vs Psa 77:46 земные произрастения их отдал гусенице и труд их~--- саранче;
\vs Psa 77:47 виноград их побил градом и сикоморы их~--- льдом;
\vs Psa 77:48 скот их предал граду и стада их~--- молниям;
\vs Psa 77:49 послал на них пламень гнева Своего, и негодование, и ярость и бедствие, посольство злых ангелов;
\vs Psa 77:50 уравнял стезю гневу Своему, не охранял души их от смерти, и скот их предал моровой язве;
\vs Psa 77:51 поразил всякого первенца в Египте, начатки сил в шатрах Хамовых;
\vs Psa 77:52 и повел народ Свой, как овец, и вел их, как стадо, пустынею;
\vs Psa 77:53 вел их безопасно, и они не страшились, а врагов их покрыло море;
\vs Psa 77:54 и привел их в область святую Свою, на гору сию, которую стяжала десница Его;
\vs Psa 77:55 прогнал от лица их народы и землю их разделил в наследие им, и колена Израилевы поселил в шатрах их.
\vs Psa 77:56 Но они еще искушали и огорчали Бога Всевышнего, и уставов Его не сохраняли;
\vs Psa 77:57 отступали и изменяли, как отцы их, обращались назад, как неверный лук;
\vs Psa 77:58 огорчали Его высотами своими и истуканами своими возбуждали ревность Его.
\vs Psa 77:59 Услышал Бог и воспламенился гневом и сильно вознегодовал на Израиля;
\vs Psa 77:60 отринул жилище в Силоме, скинию, в которой обитал Он между человеками;
\vs Psa 77:61 и отдал в плен крепость Свою и славу Свою в руки врага,
\vs Psa 77:62 и предал мечу народ Свой и прогневался на наследие Свое.
\vs Psa 77:63 Юношей его поедал огонь, и девицам его не пели брачных песен;
\vs Psa 77:64 священники его падали от меча, и вдовы его не плакали.
\vs Psa 77:65 Но, как бы от сна, воспрянул Господь, как бы исполин, побежденный вином,
\vs Psa 77:66 и поразил врагов его в тыл, вечному сраму предал их;
\vs Psa 77:67 и отверг шатер Иосифов и колена Ефремова не избрал,
\vs Psa 77:68 а избрал колено Иудино, гору Сион, которую возлюбил.
\vs Psa 77:69 И устроил, как небо, святилище Свое и, как землю, утвердил его навек,
\vs Psa 77:70 и избрал Давида, раба Своего, и взял его от дворов овчих
\vs Psa 77:71 и от доящих привел его пасти народ Свой, Иакова, и наследие Свое, Израиля.
\vs Psa 77:72 И он пас их в чистоте сердца своего и руками мудрыми водил их.
\vs Psa 78:0 Псалом Асафа.
\rsbpar\vs Psa 78:1 Боже! язычники пришли в наследие Твое, осквернили святый храм Твой, Иерусалим превратили в развалины;
\vs Psa 78:2 трупы рабов Твоих отдали на съедение птицам небесным, тела святых Твоих~--- зверям земным;
\vs Psa 78:3 пролили кровь их, как воду, вокруг Иерусалима, и некому было похоронить их.
\vs Psa 78:4 Мы сделались посмешищем у соседей наших, поруганием и посрамлением у окружающих нас.
\vs Psa 78:5 Доколе, Господи, будешь гневаться непрестанно, будет пылать ревность Твоя, как огонь?
\vs Psa 78:6 Пролей гнев Твой на народы, которые не знают Тебя, и на царства, которые имени Твоего не призывают,
\vs Psa 78:7 ибо они пожрали Иакова и жилище его опустошили.
\vs Psa 78:8 Не помяни нам грехов \bibemph{наших} предков; скоро да предварят нас щедроты Твои, ибо мы весьма истощены.
\vs Psa 78:9 Помоги нам, Боже, Спаситель наш, ради славы имени Твоего; избавь нас и прости нам грехи наши ради имени Твоего.
\vs Psa 78:10 Для чего язычникам говорить: <<где Бог их?>> Да сделается известным между язычниками пред глазами нашими отмщение за пролитую кровь рабов Твоих.
\vs Psa 78:11 Да придет пред лице Твое стенание узника; могуществом мышцы Твоей сохрани обреченных на смерть.
\vs Psa 78:12 Семикратно возврати соседям нашим в недро их поношение, которым они Тебя, Господи, поносили.
\vs Psa 78:13 А мы, народ Твой и Твоей пажити овцы, вечно будем славить Тебя и в род и род возвещать хвалу Тебе.
\vs Psa 79:1 Начальнику хора. На музыкальном \bibemph{орудии} Шошанним-Эдуф. Псалом Асафа.
\rsbpar\vs Psa 79:2 Пастырь Израиля! внемли; водящий, как овец, Иосифа, восседающий на Херувимах, яви Себя.
\vs Psa 79:3 Пред Ефремом и Вениамином и Манассиею воздвигни силу Твою, и приди спасти нас.
\vs Psa 79:4 Боже! восстанови нас; да воссияет лице Твое, и спасемся!
\vs Psa 79:5 Господи, Боже сил! доколе будешь гневен к молитвам народа Твоего?
\vs Psa 79:6 Ты напитал их хлебом слезным, и напоил их слезами в большой мере,
\vs Psa 79:7 положил нас в пререкание соседям нашим, и враги наши издеваются \bibemph{над нами}.
\vs Psa 79:8 Боже сил! восстанови нас; да воссияет лице Твое, и спасемся!
\vs Psa 79:9 Из Египта перенес Ты виноградную лозу, выгнал народы и посадил ее;
\vs Psa 79:10 очистил для нее место, и утвердил корни ее, и она наполнила землю.
\vs Psa 79:11 Горы покрылись тенью ее, и ветви ее как кедры Божии;
\vs Psa 79:12 она пустила ветви свои до моря и отрасли свои до реки.
\vs Psa 79:13 Для чего разрушил Ты ограды ее, так что обрывают ее все, проходящие по пути?
\vs Psa 79:14 Лесной вепрь подрывает ее, и полевой зверь объедает ее.
\vs Psa 79:15 Боже сил! обратись же, призри с неба, и воззри, и посети виноград сей;
\vs Psa 79:16 охрани то, что насадила десница Твоя, и отрасли, которые Ты укрепил Себе.
\vs Psa 79:17 Он пожжен огнем, обсечен; от прещения лица Твоего погибнут.
\vs Psa 79:18 Да будет рука Твоя над мужем десницы Твоей, над сыном человеческим, которого Ты укрепил Себе,
\vs Psa 79:19 и мы не отступим от Тебя; оживи нас, и мы будем призывать имя Твое.
\vs Psa 79:20 Господи, Боже сил! восстанови нас; да воссияет лице Твое, и спасемся!
\vs Psa 80:1 Начальнику хора. На Гефском орудии. Псалом Асафа.
\rsbpar\vs Psa 80:2 Радостно пойте Богу, твердыне нашей; восклицайте Богу Иакова;
\vs Psa 80:3 возьмите псалом, дайте тимпан, сладкозвучные гусли с псалтирью;
\vs Psa 80:4 трубите в новомесячие трубою, в определенное время, в день праздника нашего;
\vs Psa 80:5 ибо это закон для Израиля, устав от Бога Иаковлева.
\vs Psa 80:6 Он установил это во свидетельство для Иосифа, когда он вышел из земли Египетской, где услышал звуки языка, которого не знал:
\vs Psa 80:7 <<Я снял с рамен его тяжести, и руки его освободились от корзин.
\vs Psa 80:8 В бедствии ты призвал Меня, и Я избавил тебя; из среды грома Я услышал тебя, при водах Меривы испытал тебя.
\vs Psa 80:9 Слушай, народ Мой, и Я буду свидетельствовать тебе: Израиль! о, если бы ты послушал Меня!
\vs Psa 80:10 Да не будет у тебя иного бога, и не поклоняйся богу чужеземному.
\vs Psa 80:11 Я Господь, Бог твой, изведший тебя из земли Египетской; открой уста твои, и Я наполню их>>.
\vs Psa 80:12 Но народ Мой не слушал гласа Моего, и Израиль не покорялся Мне;
\vs Psa 80:13 потому Я оставил их упорству сердца их, пусть ходят по своим помыслам.
\vs Psa 80:14 О, если бы народ Мой слушал Меня и Израиль ходил Моими путями!
\vs Psa 80:15 Я скоро смирил бы врагов их и обратил бы руку Мою на притеснителей их:
\vs Psa 80:16 ненавидящие Господа раболепствовали бы им, а их благоденствие продолжалось бы навсегда;
\vs Psa 80:17 Я питал бы их туком пшеницы и насыщал бы их медом из скалы.
\vs Psa 81:0 Псалом Асафа.
\rsbpar\vs Psa 81:1 Бог стал в сонме богов; среди богов произнес суд:
\vs Psa 81:2 доколе будете вы судить неправедно и оказывать лицеприятие нечестивым?
\vs Psa 81:3 Давайте суд бедному и сироте; угнетенному и нищему оказывайте справедливость;
\vs Psa 81:4 избавляйте бедного и нищего; исторгайте \bibemph{его} из руки нечестивых.
\vs Psa 81:5 Не знают, не разумеют, во тьме ходят; все основания земли колеблются.
\vs Psa 81:6 Я сказал: вы~--- боги, и сыны Всевышнего~--- все вы;
\vs Psa 81:7 но вы умрете, как человеки, и падете, как всякий из князей.
\vs Psa 81:8 Восстань\fns{В славянском переводе: Воскресни\dots}, Боже, суди землю, ибо Ты наследуешь все народы.
\vs Psa 82:1 Песнь. Псалом Асафа.
\rsbpar\vs Psa 82:2 Боже! Не премолчи, не безмолвствуй и не оставайся в покое, Боже,
\vs Psa 82:3 ибо вот, враги Твои шумят, и ненавидящие Тебя подняли голову;
\vs Psa 82:4 против народа Твоего составили коварный умысел и совещаются против хранимых Тобою;
\vs Psa 82:5 сказали: <<пойдем и истребим их из народов, чтобы не вспоминалось более имя Израиля>>.
\vs Psa 82:6 Сговорились единодушно, заключили против Тебя союз:
\vs Psa 82:7 селения Едомовы и Измаильтяне, Моав и Агаряне,
\vs Psa 82:8 Гевал и Аммон и Амалик, Филистимляне с жителями Тира.
\vs Psa 82:9 И Ассур пристал к ним: они стали мышцею для сынов Лотовых.
\vs Psa 82:10 Сделай им то же, что Мадиаму, что Сисаре, что Иавину у потока Киссона,
\vs Psa 82:11 которые истреблены в Аендоре, сделались навозом для земли.
\vs Psa 82:12 Поступи с ними, с князьями их, как с Оривом и Зивом и со всеми вождями их, как с Зевеем и Салманом,
\vs Psa 82:13 которые говорили: <<возьмем себе во владение селения Божии>>.
\vs Psa 82:14 Боже мой! Да будут они, как пыль в вихре, как солома перед ветром.
\vs Psa 82:15 Как огонь сжигает лес, и как пламя опаляет горы,
\vs Psa 82:16 так погони их бурею Твоею и вихрем Твоим приведи их в смятение;
\vs Psa 82:17 исполни лица их бесчестием, чтобы они взыскали имя Твое, Господи!
\vs Psa 82:18 Да постыдятся и смятутся на веки, да посрамятся и погибнут,
\vs Psa 82:19 и да познают, что Ты, Которого одного имя Господь, Всевышний над всею землею.
\vs Psa 83:1 Начальнику хора. На Гефском \bibemph{орудии}. Кореевых сынов. Псалом.
\rsbpar\vs Psa 83:2 Как вожделенны жилища Твои, Господи сил!
\vs Psa 83:3 Истомилась душа моя, желая во дворы Господни; сердце мое и плоть моя восторгаются к Богу живому.
\vs Psa 83:4 И птичка находит себе жилье, и ласточка гнездо себе, где положить птенцов своих, у алтарей Твоих, Господи сил, Царь мой и Бог мой!
\vs Psa 83:5 Блаженны живущие в доме Твоем: они непрестанно будут восхвалять Тебя.
\vs Psa 83:6 Блажен человек, которого сила в Тебе и у которого в сердце стези направлены \bibemph{к Тебе}.
\vs Psa 83:7 Проходя долиною плача, они открывают в ней источники, и дождь покрывает ее благословением;
\vs Psa 83:8 приходят от силы в силу, являются пред Богом на Сионе.
\vs Psa 83:9 Господи, Боже сил! Услышь молитву мою, внемли, Боже Иаковлев!
\vs Psa 83:10 Боже, защитник наш! Приникни и призри на лице помазанника Твоего.
\vs Psa 83:11 Ибо один день во дворах Твоих лучше тысячи. Желаю лучше быть у порога в доме Божием, нежели жить в шатрах нечестия.
\vs Psa 83:12 Ибо Господь Бог есть солнце и щит, Господь дает благодать и славу; ходящих в непорочности Он не лишает благ.
\vs Psa 83:13 Господи сил! Блажен человек, уповающий на Тебя!
\vs Psa 84:1 Начальнику хора. Кореевых сынов. Псалом.
\rsbpar\vs Psa 84:2 Господи! Ты умилосердился к земле Твоей, возвратил плен Иакова;
\vs Psa 84:3 простил беззаконие народа Твоего, покрыл все грехи его,
\vs Psa 84:4 отъял всю ярость Твою, отвратил лютость гнева Твоего.
\vs Psa 84:5 Восстанови нас, Боже спасения нашего, и прекрати негодование Твое на нас.
\vs Psa 84:6 Неужели вечно будешь гневаться на нас, прострешь гнев Твой от рода в род?
\vs Psa 84:7 Неужели снова не оживишь нас, чтобы народ Твой возрадовался о Тебе?
\vs Psa 84:8 Яви нам, Господи, милость Твою, и спасение Твое даруй нам.
\vs Psa 84:9 Послушаю, что скажет Господь Бог. Он скажет мир народу Своему и избранным Своим, но да не впадут они снова в безрассудство.
\vs Psa 84:10 Так, близко к боящимся Его спасение Его, чтобы обитала слава в земле нашей!
\vs Psa 84:11 Милость и истина сретятся, правда и мир облобызаются;
\vs Psa 84:12 истина возникнет из земли, и правда приникнет с небес;
\vs Psa 84:13 и Господь даст благо, и земля наша даст плод свой;
\vs Psa 84:14 правда пойдет пред Ним и поставит на путь стопы свои.
\vs Psa 85:0 Молитва Давида.
\rsbpar\vs Psa 85:1 Приклони, Господи, ухо Твое и услышь меня, ибо я беден и нищ.
\vs Psa 85:2 Сохрани душу мою, ибо я благоговею пред Тобою; спаси, Боже мой, раба Твоего, уповающего на Тебя.
\vs Psa 85:3 Помилуй меня, Господи, ибо к Тебе взываю каждый день.
\vs Psa 85:4 Возвесели душу раба Твоего, ибо к Тебе, Господи, возношу душу мою,
\vs Psa 85:5 ибо Ты, Господи, благ и милосерд и многомилостив ко всем, призывающим Тебя.
\vs Psa 85:6 Услышь, Господи, молитву мою и внемли гласу моления моего.
\vs Psa 85:7 В день скорби моей взываю к Тебе, потому что Ты услышишь меня.
\vs Psa 85:8 Нет между богами, как Ты, Господи, и нет дел, как Твои.
\vs Psa 85:9 Все народы, Тобою сотворенные, приидут и поклонятся пред Тобою, Господи, и прославят имя Твое,
\vs Psa 85:10 ибо Ты велик и творишь чудеса,~--- Ты, Боже, един Ты.
\vs Psa 85:11 Наставь меня, Господи, на путь Твой, и буду ходить в истине Твоей; утверди сердце мое в страхе имени Твоего.
\vs Psa 85:12 Буду восхвалять Тебя, Господи, Боже мой, всем сердцем моим и славить имя Твое вечно,
\vs Psa 85:13 ибо велика милость Твоя ко мне: Ты избавил душу мою от ада преисподнего.
\vs Psa 85:14 Боже! гордые восстали на меня, и скопище мятежников ищет души моей: не представляют они Тебя пред собою.
\vs Psa 85:15 Но Ты, Господи, Боже щедрый и благосердный, долготерпеливый и многомилостивый и истинный,
\vs Psa 85:16 призри на меня и помилуй меня; даруй крепость Твою рабу Твоему, и спаси сына рабы Твоей;
\vs Psa 85:17 покажи на мне знамение во благо, да видят ненавидящие меня и устыдятся, потому что Ты, Господи, помог мне и утешил меня.
\vs Psa 86:1 Сынов Кореевых. Псалом. Песнь.
\rsbpar\vs Psa 86:2 Основание его\fns{Иерусалима.} на горах святых. Господь любит врата Сиона более всех селений Иакова.
\vs Psa 86:3 Славное возвещается о тебе, град Божий!
\vs Psa 86:4 Упомяну знающим меня о Рааве\fns{О Египте.} и Вавилоне; вот Филистимляне и Тир с Ефиопиею,~--- \bibemph{скажут}: <<такой-то родился там>>.
\vs Psa 86:5 О Сионе же будут говорить: <<такой-то и такой-то муж родился в нем, и Сам Всевышний укрепил его>>.
\vs Psa 86:6 Господь в переписи народов напишет: <<такой-то родился там>>.
\vs Psa 86:7 И поющие и играющие,~--- все источники мои в тебе.
\vs Psa 87:1 Песнь. Псалом, Сынов Кореевых. Начальнику хора на Махалаф, для пения. Учение Емана Езрахита.
\rsbpar\vs Psa 87:2 Господи, Боже спасения моего! днем вопию и ночью пред Тобою:
\vs Psa 87:3 да внидет пред лице Твое молитва моя; приклони ухо Твое к молению моему,
\vs Psa 87:4 ибо душа моя насытилась бедствиями, и жизнь моя приблизилась к преисподней.
\vs Psa 87:5 Я сравнялся с нисходящими в могилу; я стал, как человек без силы,
\vs Psa 87:6 между мертвыми брошенный,~--- как убитые, лежащие во гробе, о которых Ты уже не вспоминаешь и которые от руки Твоей отринуты.
\vs Psa 87:7 Ты положил меня в ров преисподний, во мрак, в бездну.
\vs Psa 87:8 Отяготела на мне ярость Твоя, и всеми волнами Твоими Ты поразил [меня].
\vs Psa 87:9 Ты удалил от меня знакомых моих, сделал меня отвратительным для них; я заключен, и не могу выйти.
\vs Psa 87:10 Око мое истомилось от горести: весь день я взывал к Тебе, Господи, простирал к Тебе руки мои.
\vs Psa 87:11 Разве над мертвыми Ты сотворишь чудо? Разве мертвые встанут и будут славить Тебя?
\vs Psa 87:12 или во гробе будет возвещаема милость Твоя, и истина Твоя~--- в месте тления?
\vs Psa 87:13 разве во мраке позн\acc{а}ют чудеса Твои, и в земле забвения~--- правду Твою?
\vs Psa 87:14 Но я к Тебе, Господи, взываю, и рано утром молитва моя предваряет Тебя.
\vs Psa 87:15 Для чего, Господи, отреваешь душу мою, скрываешь лице Твое от меня?
\vs Psa 87:16 Я несчастен и истаеваю с юности; несу ужасы Твои и изнемогаю.
\vs Psa 87:17 Надо мною прошла ярость Твоя, устрашения Твои сокрушили меня,
\vs Psa 87:18 всякий день окружают меня, как вода: облегают меня все вместе.
\vs Psa 87:19 Ты удалил от меня друга и искреннего; знакомых моих не видно.
\vs Psa 88:1 Учение Ефама Езрахита.
\rsbpar\vs Psa 88:2 Милости [Твои], Господи, буду петь вечно, в род и род возвещать истину Твою устами моими.
\vs Psa 88:3 Ибо говорю: навек основана милость, на небесах утвердил Ты истину Твою, \bibemph{когда сказал}:
\vs Psa 88:4 <<Я поставил завет с избранным Моим, клялся Давиду, рабу Моему:
\vs Psa 88:5 навек утвержу семя твое, в род и род устрою престол твой>>.
\vs Psa 88:6 И небеса прославят чудные дела Твои, Господи, и истину Твою в собрании святых.
\vs Psa 88:7 Ибо кто на небесах сравнится с Господом? кто между сынами Божиими уподобится Господу?
\vs Psa 88:8 Страшен Бог в великом сонме святых, страшен Он для всех окружающих Его.
\vs Psa 88:9 Господи, Боже сил! кто силен, как Ты, Господи? И истина Твоя окрест Тебя.
\vs Psa 88:10 Ты владычествуешь над яростью моря: когда воздымаются волны его, Ты укрощаешь их.
\vs Psa 88:11 Ты низложил Раава, как пораженного; крепкою мышцею Твоею рассеял врагов Твоих.
\vs Psa 88:12 Твои небеса и Твоя земля; вселенную и что наполняет ее, Ты основал.
\vs Psa 88:13 Север и юг Ты сотворил; Фавор и Ермон о имени Твоем радуются.
\vs Psa 88:14 Крепка мышца Твоя, сильна рука Твоя, высока десница Твоя!
\vs Psa 88:15 Правосудие и правота~--- основание престола Твоего; милость и истина предходят пред лицем Твоим.
\vs Psa 88:16 Блажен народ, знающий трубный зов! Они ходят во свете лица Твоего, Господи,
\vs Psa 88:17 о имени Твоем радуются весь день и правдою Твоею возносятся,
\vs Psa 88:18 ибо Ты украшение силы их, и благоволением Твоим возвышается рог наш.
\vs Psa 88:19 От Господа~--- щит наш, и от Святаго Израилева~--- царь наш.
\vs Psa 88:20 Некогда говорил Ты в видении святому Твоему, и сказал: <<Я оказал помощь мужественному, вознес избранного из народа.
\vs Psa 88:21 Я обрел Давида, раба Моего, святым елеем Моим помазал его.
\vs Psa 88:22 Рука Моя пребудет с ним, и мышца Моя укрепит его.
\vs Psa 88:23 Враг не превозможет его, и сын беззакония не притеснит его.
\vs Psa 88:24 Сокрушу пред ним врагов его и поражу ненавидящих его.
\vs Psa 88:25 И истина Моя и милость Моя с ним, и Моим именем возвысится рог его.
\vs Psa 88:26 И положу на море руку его, и на реки~--- десницу его.
\vs Psa 88:27 Он будет звать Меня: Ты отец мой, Бог мой и твердыня спасения моего.
\vs Psa 88:28 И Я сделаю его первенцем, превыше царей земли,
\vs Psa 88:29 вовек сохраню ему милость Мою, и завет Мой с ним будет верен.
\vs Psa 88:30 И продолжу вовек семя его, и престол его~--- как дни неба.
\vs Psa 88:31 Если сыновья его оставят закон Мой и не будут ходить по заповедям Моим;
\vs Psa 88:32 если нарушат уставы Мои и повелений Моих не сохранят:
\vs Psa 88:33 посещу жезлом беззаконие их, и ударами~--- неправду их;
\vs Psa 88:34 милости же Моей не отниму от него, и не изменю истины Моей.
\vs Psa 88:35 Не нарушу завета Моего, и не переменю того, что вышло из уст Моих.
\vs Psa 88:36 Однажды Я поклялся святостью Моею: солгу ли Давиду?
\vs Psa 88:37 Семя его пребудет вечно, и престол его, как солнце, предо Мною,
\vs Psa 88:38 вовек будет тверд, как луна, и верный свидетель на небесах>>.
\vs Psa 88:39 Но \bibemph{ныне} Ты отринул и презрел, прогневался на помазанника Твоего;
\vs Psa 88:40 пренебрег завет с рабом Твоим, поверг на землю венец его;
\vs Psa 88:41 разрушил все ограды его, превратил в развалины крепости его.
\vs Psa 88:42 Расхищают его все проходящие путем; он сделался посмешищем у соседей своих.
\vs Psa 88:43 Ты возвысил десницу противников его, обрадовал всех врагов его;
\vs Psa 88:44 Ты обратил назад острие меча его и не укрепил его на брани;
\vs Psa 88:45 отнял у него блеск и престол его поверг на землю;
\vs Psa 88:46 сократил дни юности его и покрыл его стыдом.
\vs Psa 88:47 Доколе, Господи, будешь скрываться непрестанно, будет пылать ярость Твоя, как огонь?
\vs Psa 88:48 Вспомни, какой мой век: на какую суету сотворил Ты всех сынов человеческих?
\vs Psa 88:49 Кто из людей жил~--- и не видел смерти, избавил душу свою от руки преисподней?
\vs Psa 88:50 Где прежние милости Твои, Господи? Ты клялся Давиду истиною Твоею.
\vs Psa 88:51 Вспомни, Господи, поругание рабов Твоих, которое я ношу в недре моем от всех сильных народов;
\vs Psa 88:52 как поносят враги Твои, Господи, как бесславят следы помазанника Твоего.
\vs Psa 88:53 Благословен Господь вовек! Аминь, аминь.
\vs Psa 89:1 Молитва Моисея, человека Божия.
\rsbpar\vs Psa 89:2 Господи! Ты нам прибежище в род и род.
\vs Psa 89:3 Прежде нежели родились горы, и Ты образовал землю и вселенную, и от века и до века Ты~--- Бог.
\vs Psa 89:4 Ты возвращаешь человека в тление и говоришь: <<возвратитесь, сыны человеческие!>>
\vs Psa 89:5 Ибо пред очами Твоими тысяча лет, как день вчерашний, когда он прошел, и \bibemph{как} стража в ночи.
\vs Psa 89:6 Ты \bibemph{как} наводнением уносишь их; они~--- \bibemph{как} сон, как трава, которая утром вырастает, утром цветет и зеленеет, вечером подсекается и засыхает;
\vs Psa 89:7 ибо мы исчезаем от гнева Твоего и от ярости Твоей мы в смятении.
\vs Psa 89:8 Ты положил беззакония наши пред Тобою и тайное наше пред светом лица Твоего.
\vs Psa 89:9 Все дни наши прошли во гневе Твоем; мы теряем лета наши, как звук.
\vs Psa 89:10 Дней лет наших~--- семьдесят лет, а при большей крепости~--- восемьдесят лет; и самая лучшая пора их~--- труд и болезнь, ибо проходят быстро, и мы летим.
\vs Psa 89:11 Кто знает силу гнева Твоего, и ярость Твою по мере страха Твоего?
\vs Psa 89:12 Научи нас так счислять дни наши, чтобы нам приобрести сердце мудрое.
\vs Psa 89:13 Обратись, Господи! Доколе? Умилосердись над рабами Твоими.
\vs Psa 89:14 Рано насыти нас милостью Твоею, и мы будем радоваться и веселиться во все дни наши.
\vs Psa 89:15 Возвесели нас за дни, \bibemph{в которые} Ты поражал нас, за лета, \bibemph{в которые} мы видели бедствие.
\vs Psa 89:16 Да явится на рабах Твоих дело Твое и на сынах их слава Твоя;
\vs Psa 89:17 и да будет благоволение Господа Бога нашего на нас, и в деле рук наших споспешествуй нам, в деле рук наших споспешествуй.
\vs Psa 90:0 [Хвалебная песнь Давида.]
\rsbpar\vs Psa 90:1 Живущий под кровом Всевышнего под сенью Всемогущего покоится,
\vs Psa 90:2 говорит Господу: <<прибежище мое и защита моя, Бог мой, на Которого я уповаю!>>
\vs Psa 90:3 Он избавит тебя от сети ловца, от гибельной язвы,
\vs Psa 90:4 перьями Своими осенит тебя, и под крыльями Его будешь безопасен; щит и ограждение~--- истина Его.
\vs Psa 90:5 Не убоишься ужасов в ночи, стрелы, летящей днем,
\vs Psa 90:6 язвы, ходящей во мраке, заразы, опустошающей в полдень.
\vs Psa 90:7 Падут подле тебя тысяча и десять тысяч одесную тебя; но к тебе не приблизится:
\vs Psa 90:8 только смотреть будешь очами твоими и видеть возмездие нечестивым.
\vs Psa 90:9 Ибо ты \bibemph{сказал}: <<Господь~--- упование мое>>; Всевышнего избрал ты прибежищем твоим;
\vs Psa 90:10 не приключится тебе зло, и язва не приблизится к жилищу твоему;
\vs Psa 90:11 ибо Ангелам Своим заповедает о тебе~--- охранять тебя на всех путях твоих:
\vs Psa 90:12 на руках понесут тебя, да не преткнешься о камень ногою твоею;
\vs Psa 90:13 на аспида и василиска наступишь; попирать будешь льва и дракона.
\vs Psa 90:14 <<За то, что он возлюбил Меня, избавлю его; защищу его, потому что он познал имя Мое.
\vs Psa 90:15 Воззовет ко Мне, и услышу его; с ним Я в скорби; избавлю его и прославлю его,
\vs Psa 90:16 долготою дней насыщу его, и явлю ему спасение Мое>>.
\vs Psa 91:1 Псалом. Песнь на день субботний.
\rsbpar\vs Psa 91:2 Благо есть славить Господа и петь имени Твоему, Всевышний,
\vs Psa 91:3 возвещать утром милость Твою и истину Твою в ночи,
\vs Psa 91:4 на десятиструнном и псалтири, с песнью на гуслях.
\vs Psa 91:5 Ибо Ты возвеселил меня, Господи, творением Твоим: я восхищаюсь делами рук Твоих.
\vs Psa 91:6 Как велики дела Твои, Господи! дивно глубоки помышления Твои!
\vs Psa 91:7 Человек несмысленный не знает, и невежда не разумеет того.
\vs Psa 91:8 Тогда как нечестивые возникают, как трава, и делающие беззаконие цветут, чтобы исчезнуть на веки,~---
\vs Psa 91:9 Ты, Господи, высок во веки!
\vs Psa 91:10 Ибо вот, враги Твои, Господи,~--- вот, враги Твои гибнут, и рассыпаются все делающие беззаконие;
\vs Psa 91:11 а мой рог Ты возносишь, как рог единорога, и я умащен свежим елеем;
\vs Psa 91:12 и око мое смотрит на врагов моих, и уши мои слышат о восстающих на меня злодеях.
\vs Psa 91:13 Праведник цветет, как пальма, возвышается подобно кедру на Ливане.
\vs Psa 91:14 Насажденные в доме Господнем, они цветут во дворах Бога нашего;
\vs Psa 91:15 они и в старости плодовиты, сочны и свежи,
\vs Psa 91:16 чтобы возвещать, что праведен Господь, твердыня моя, и нет неправды в Нем.
\vs Psa 92:0 [Хвалебная песнь Давида. В день предсубботний, когда населена земля.]
\rsbpar\vs Psa 92:1 Господь царствует; Он облечен величием, облечен Господь могуществом [и] препоясан: потому вселенная тверда, не подвигнется.
\vs Psa 92:2 Престол Твой утвержден искони: Ты~--- от века.
\vs Psa 92:3 Возвышают реки, Господи, возвышают реки голос свой, возвышают реки волны свои.
\vs Psa 92:4 Но паче шума вод многих, сильных волн морских, силен в вышних Господь.
\vs Psa 92:5 Откровения Твои несомненно верны. Дому Твоему, Господи, принадлежит святость на долгие дни.
\vs Psa 93:0 [Псалом Давида в четвертый день недели.]
\rsbpar\vs Psa 93:1 Боже отмщений, Господи, Боже отмщений, яви Себя!
\vs Psa 93:2 Восстань, Судия земли, воздай возмездие гордым.
\vs Psa 93:3 Доколе, Господи, нечестивые, доколе нечестивые торжествовать будут?
\vs Psa 93:4 Они изрыгают дерзкие речи; величаются все делающие беззаконие;
\vs Psa 93:5 попирают народ Твой, Господи, угнетают наследие Твое;
\vs Psa 93:6 вдову и пришельца убивают, и сирот умерщвляют
\vs Psa 93:7 и говорят: <<не увидит Господь, и не узнает Бог Иаковлев>>.
\vs Psa 93:8 Образумьтесь, бессмысленные люди! когда вы будете умны, невежды?
\vs Psa 93:9 Насадивший ухо не услышит ли? и образовавший глаз не увидит ли?
\vs Psa 93:10 Вразумляющий народы неужели не обличит,~--- Тот, Кто учит человека разумению?
\vs Psa 93:11 Господь знает мысли человеческие, что они суетны.
\vs Psa 93:12 Блажен человек, которого вразумляешь Ты, Господи, и наставляешь законом Твоим,
\vs Psa 93:13 чтобы дать ему покой в бедственные дни, доколе нечестивому выроется яма!
\vs Psa 93:14 Ибо не отринет Господь народа Своего и не оставит наследия Своего.
\vs Psa 93:15 Ибо суд возвратится к правде, и за ним \bibemph{последуют} все правые сердцем.
\vs Psa 93:16 Кто восстанет за меня против злодеев? кто станет за меня против делающих беззаконие?
\vs Psa 93:17 Если бы не Господь был мне помощником, вскоре вселилась бы душа моя в \bibemph{страну} молчания.
\vs Psa 93:18 Когда я говорил: <<колеблется нога моя>>,~--- милость Твоя, Господи, поддерживала меня.
\vs Psa 93:19 При умножении скорбей моих в сердце моем, утешения Твои услаждают душу мою.
\vs Psa 93:20 Станет ли близ Тебя седалище губителей, умышляющих насилие вопреки закону?
\vs Psa 93:21 Толпою устремляются они на душу праведника и осуждают кровь неповинную.
\vs Psa 93:22 Но Господь~--- защита моя, и Бог мой~--- твердыня убежища моего,
\vs Psa 93:23 и обратит на них беззаконие их, и злодейством их истребит их, истребит их Господь Бог наш.
\vs Psa 94:0 [Хвалебная песнь Давида.]
\rsbpar\vs Psa 94:1 Приидите, воспоем Господу, воскликнем [Богу], твердыне спасения нашего;
\vs Psa 94:2 предстанем лицу Его со славословием, в песнях воскликнем Ему,
\vs Psa 94:3 ибо Господь есть Бог великий и Царь великий над всеми богами.
\vs Psa 94:4 В Его руке глубины земли, и вершины гор~--- Его же;
\vs Psa 94:5 Его~--- море, и Он создал его, и сушу образовали руки Его.
\vs Psa 94:6 Приидите, поклонимся и припадем, преклоним колени пред лицем Господа, Творца нашего;
\vs Psa 94:7 ибо Он есть Бог наш, и мы~--- народ паствы Его и овцы руки Его. О, если бы вы ныне послушали гласа Его:
\vs Psa 94:8 <<не ожесточите сердца вашего, как в Мериве, как в день искушения в пустыне,
\vs Psa 94:9 где искушали Меня отцы ваши, испытывали Меня, и видели дело Мое.
\vs Psa 94:10 Сорок лет Я был раздражаем родом сим, и сказал: это народ, заблуждающийся сердцем; они не познали путей Моих,
\vs Psa 94:11 и потому Я поклялся во гневе Моем, что они не войдут в покой Мой>>.
\vs Psa 95:0 [Хвалебная песнь Давида. На построение дома.]
\rsbpar\vs Psa 95:1 Воспойте Господу песнь новую; воспойте Господу, вся земля;
\vs Psa 95:2 пойте Господу, благословляйте имя Его, благовествуйте со дня на день спасение Его;
\vs Psa 95:3 возвещайте в народах славу Его, во всех племенах чудеса Его;
\vs Psa 95:4 ибо велик Господь и достохвален, страшен Он паче всех богов.
\vs Psa 95:5 Ибо все боги народов~--- идолы, а Господь небеса сотворил.
\vs Psa 95:6 Слава и величие пред лицем Его, сила и великолепие во святилище Его.
\vs Psa 95:7 Воздайте Господу, племена народов, воздайте Господу славу и честь;
\vs Psa 95:8 воздайте Господу славу имени Его, несите дары и идите во дворы Его;
\vs Psa 95:9 поклонитесь Господу во благолепии святыни. Трепещи пред лицем Его, вся земля!
\vs Psa 95:10 Скажите народам: Господь царствует! потому тверда вселенная, не поколеблется. Он будет судить народы по правде.
\vs Psa 95:11 Да веселятся небеса и да торжествует земля; да шумит море и что наполняет его;
\vs Psa 95:12 да радуется поле и все, что на нем, и да ликуют все дерева дубравные
\vs Psa 95:13 пред лицем Господа; ибо идет, ибо идет судить землю. Он будет судить вселенную по правде, и народы~--- по истине Своей.
\vs Psa 96:0 [Псалом Давида, когда устроялась земля его.]
\rsbpar\vs Psa 96:1 Господь царствует: да радуется земля; да веселятся многочисленные острова.
\vs Psa 96:2 Облако и мрак окрест Его; правда и суд~--- основание престола Его.
\vs Psa 96:3 Пред Ним идет огонь и вокруг попаляет врагов Его.
\vs Psa 96:4 Молнии Его освещают вселенную; земля видит и трепещет.
\vs Psa 96:5 Горы, как воск, тают от лица Господа, от лица Господа всей земли.
\vs Psa 96:6 Небеса возвещают правду Его, и все народы видят славу Его.
\vs Psa 96:7 Да постыдятся все служащие истуканам, хвалящиеся идолами. Поклонитесь пред Ним, все боги\fns{По переводу 70-ти: все Ангелы Его.}.
\vs Psa 96:8 Слышит Сион и радуется, и веселятся дщери Иудины ради судов Твоих, Господи,
\vs Psa 96:9 ибо Ты, Господи, высок над всею землею, превознесен над всеми богами.
\vs Psa 96:10 Любящие Господа, ненавидьте зло! Он хранит души святых Своих; из руки нечестивых избавляет их.
\vs Psa 96:11 Свет сияет на праведника, и на правых сердцем~--- веселие.
\vs Psa 96:12 Радуйтесь, праведные, о Господе и славьте память святыни Его.
\vs Psa 97:0 Псалом [Давида].
\rsbpar\vs Psa 97:1 Воспойте Господу новую песнь, ибо Он сотворил чудеса. Его десница и святая мышца Его доставили Ему победу.
\vs Psa 97:2 Явил Господь спасение Свое, открыл пред очами народов правду Свою.
\vs Psa 97:3 Вспомнил Он милость Свою [к Иакову] и верность Свою к дому Израилеву. Все концы земли увидели спасение Бога нашего.
\vs Psa 97:4 Восклицайте Господу, вся земля; торжествуйте, веселитесь и пойте;
\vs Psa 97:5 пойте Господу с гуслями, с гуслями и с гласом псалмопения;
\vs Psa 97:6 при звуке труб и рога торжествуйте пред Царем Господом.
\vs Psa 97:7 Да шумит море и что наполняет его, вселенная и живущие в ней;
\vs Psa 97:8 да рукоплещут реки, да ликуют вместе горы
\vs Psa 97:9 пред лицем Господа, ибо Он идет судить землю. Он будет судить вселенную праведно и народы~--- верно.
\vs Psa 98:0 [Псалом Давида.]
\rsbpar\vs Psa 98:1 Господь царствует: да трепещут народы! Он восседает на Херувимах: да трясется земля!
\vs Psa 98:2 Господь на Сионе велик, и высок Он над всеми народами.
\vs Psa 98:3 Да славят великое и страшное имя Твое: свято оно!
\vs Psa 98:4 И могущество царя любит суд. Ты утвердил справедливость; суд и правду Ты совершил в Иакове.
\vs Psa 98:5 Превозносите Господа, Бога нашего, и поклоняйтесь подножию Его: свято оно!
\vs Psa 98:6 Моисей и Аарон между священниками и Самуил между призывающими имя Его взывали к Господу, и Он внимал им.
\vs Psa 98:7 В столпе облачном говорил Он к ним; они хранили Его заповеди и устав, который Он дал им.
\vs Psa 98:8 Господи, Боже наш! Ты внимал им; Ты был для них Богом прощающим и наказывающим за дела их.
\vs Psa 98:9 Превозносите Господа, Бога нашего, и поклоняйтесь на святой горе Его, ибо свят Господь, Бог наш.
\vs Psa 99:0 Псалом [Давида] хвалебный.
\rsbpar\vs Psa 99:1 Воскликните Господу, вся земля!
\vs Psa 99:2 Слу\-ж\acc{и}\-те Господу с веселием; идите пред лице Его с восклицанием!
\vs Psa 99:3 Познайте, что Господь есть Бог, что Он сотворил нас, и мы~--- Его, Его народ и овцы паствы Его.
\vs Psa 99:4 Входите во врата Его со славословием, во дворы Его~--- с хвалою. Славьте Его, благословляйте имя Его,
\vs Psa 99:5 ибо благ Господь: милость Его вовек, и истина Его в род и род.
\vs Psa 100:0 Псалом Давида.
\rsbpar\vs Psa 100:1 Милость и суд буду петь; Тебе, Господи, буду петь.
\vs Psa 100:2 Буду размышлять о пути непорочном: <<когда ты придешь ко мне?>> Буду ходить в непорочности моего сердца посреди дома моего.
\vs Psa 100:3 Не положу пред очами моими вещи непотребной; дело преступное я ненавижу: не прилепится оно ко мне.
\vs Psa 100:4 Сердце развращенное будет удалено от меня; злого я не буду знать.
\vs Psa 100:5 Тайно клевещущего на ближнего своего изгоню; гордого очами и надменного сердцем не потерплю.
\vs Psa 100:6 Глаза мои на верных земли, чтобы они пребывали при мне; кто ходит путем непорочности, тот будет служить мне.
\vs Psa 100:7 Не будет жить в доме моем поступающий коварно; говорящий ложь не останется пред глазами моими.
\vs Psa 100:8 С раннего утра буду истреблять всех нечестивцев земли, дабы искоренить из града Господня всех делающих беззаконие.
\vs Psa 101:1 Молитва страждущего, когда он унывает и изливает пред Господом печаль свою.
\rsbpar\vs Psa 101:2 Господи! услышь молитву мою, и вопль мой да придет к Тебе.
\vs Psa 101:3 Не скрывай лица Твоего от меня; в день скорби моей приклони ко мне ухо Твое; в день, [когда] воззову [к Тебе], скоро услышь меня;
\vs Psa 101:4 ибо исчезли, как дым, дни мои, и кости мои обожжены, как головня;
\vs Psa 101:5 сердце мое поражено, и иссохло, как трава, так что я забываю есть хлеб мой;
\vs Psa 101:6 от голоса стенания моего кости мои прильпнули к плоти моей.
\vs Psa 101:7 Я уподобился пеликану в пустыне; я стал как филин на развалинах;
\vs Psa 101:8 не сплю и сижу, как одинокая птица на кровле.
\vs Psa 101:9 Всякий день поносят меня враги мои, и злобствующие на меня клянут мною.
\vs Psa 101:10 Я ем пепел, как хлеб, и питье мое растворяю слезами,
\vs Psa 101:11 от гнева Твоего и негодования Твоего, ибо Ты вознес меня и низверг меня.
\vs Psa 101:12 Дни мои~--- как уклоняющаяся тень, и я иссох, как трава.
\vs Psa 101:13 Ты же, Господи, вовек пребываешь, и память о Тебе в род и род.
\vs Psa 101:14 Ты восстанешь, умилосердишься над Сионом, ибо время помиловать его,~--- ибо пришло время;
\vs Psa 101:15 ибо рабы Твои возлюбили и камни его, и о прахе его жалеют.
\vs Psa 101:16 И убоятся народы имени Господня, и все цари земные~--- славы Твоей.
\vs Psa 101:17 Ибо созиждет Господь Сион и явится во славе Своей;
\vs Psa 101:18 призрит на молитву беспомощных и не презрит моления их.
\vs Psa 101:19 Напишется о сем для рода последующего, и поколение грядущее восхвалит Господа,
\vs Psa 101:20 ибо Он приникнул со святой высоты Своей, с небес призрел Господь на землю,
\vs Psa 101:21 чтобы услышать стон узников, разрешить сынов смерти,
\vs Psa 101:22 дабы возвещали на Сионе имя Господне и хвалу Его~--- в Иерусалиме,
\vs Psa 101:23 когда соберутся народы вместе и царства для служения Господу.
\vs Psa 101:24 Изнурил Он на пути силы мои, сократил дни мои.
\vs Psa 101:25 Я сказал: Боже мой! не восхити меня в половине дней моих. Твои лета в роды родов.
\vs Psa 101:26 В начале Ты, [Господи,] основал землю, и небеса~--- дело Твоих рук;
\vs Psa 101:27 они погибнут, а Ты пребудешь; и все они, как риза, обветшают, и, как одежду, Ты переменишь их, и изменятся;
\vs Psa 101:28 но Ты~--- тот же, и лета Твои не кончатся.
\vs Psa 101:29 Сыны рабов Твоих будут жить, и семя их утвердится пред лицем Твоим.
\vs Psa 102:0 Псалом Давида.
\rsbpar\vs Psa 102:1 Благослови, душа моя, Господа, и вся внутренность моя~--- святое имя Его.
\vs Psa 102:2 Благослови, душа моя, Господа и не забывай всех благодеяний Его.
\vs Psa 102:3 Он прощает все беззакония твои, исцеляет все недуги твои;
\vs Psa 102:4 избавляет от могилы жизнь твою, венчает тебя милостью и щедротами;
\vs Psa 102:5 насыщает благами желание твое: обновляется, подобно орлу, юность твоя.
\vs Psa 102:6 Господь творит правду и суд всем обиженным.
\vs Psa 102:7 Он показал пути Свои Моисею, сынам Израилевым~--- дела Свои.
\vs Psa 102:8 Щедр и милостив Господь, долготерпелив и многомилостив:
\vs Psa 102:9 не до конца гневается, и не вовек негодует.
\vs Psa 102:10 Не по беззакониям нашим сотворил нам, и не по грехам нашим воздал нам:
\vs Psa 102:11 ибо как высоко небо над землею, так велика милость [Господа] к боящимся Его;
\vs Psa 102:12 как далеко восток от запада, так удалил Он от нас беззакония наши;
\vs Psa 102:13 как отец милует сынов, так милует Господь боящихся Его.
\vs Psa 102:14 Ибо Он знает состав наш, помнит, что мы~--- персть.
\vs Psa 102:15 Дни человека~--- как трава; как цвет полевой, так он цветет.
\vs Psa 102:16 Пройдет над ним ветер, и нет его, и место его уже не узнает его.
\vs Psa 102:17 Милость же Господня от века и до века к боящимся Его,
\vs Psa 102:18 и правда Его на сынах сынов, хранящих завет Его и помнящих заповеди Его, чтобы исполнять их.
\vs Psa 102:19 Господь на небесах поставил престол Свой, и царство Его всем обладает.
\vs Psa 102:20 Благословите Господа, [все] Ангелы Его, крепкие силою, исполняющие слово Его, повинуясь гласу слова Его;
\vs Psa 102:21 благословите Господа, все воинства Его, служители Его, исполняющие волю Его;
\vs Psa 102:22 благословите Господа, все дела Его, во всех местах владычества Его. Благослови, душа моя, Господа!
\vs Psa 103:0 [Псалом Давида о сотворении мира.]
\rsbpar\vs Psa 103:1 Благослови, душа моя, Господа! Господи, Боже мой! Ты дивно велик, Ты облечен славою и величием;
\vs Psa 103:2 Ты одеваешься светом, как ризою, простираешь небеса, как шатер;
\vs Psa 103:3 устрояешь над водами горние чертоги Твои, делаешь облака Твоею колесницею, шествуешь на крыльях ветра.
\vs Psa 103:4 Ты творишь ангелами Твоими духов, служителями Твоими~--- огонь пылающий.
\vs Psa 103:5 Ты поставил землю на твердых основах: не поколеблется она во веки и веки.
\vs Psa 103:6 Бездною, как одеянием, покрыл Ты ее, на горах стоят воды.
\vs Psa 103:7 От прещения Твоего бегут они, от гласа грома Твоего быстро уходят;
\vs Psa 103:8 восходят на горы, нисходят в долины, на место, которое Ты назначил для них.
\vs Psa 103:9 Ты положил предел, которого не перейдут, и не возвратятся покрыть землю.
\vs Psa 103:10 Ты послал источники в долины: между горами текут [воды],
\vs Psa 103:11 поят всех полевых зверей; дикие ослы утоляют жажду свою.
\vs Psa 103:12 При них обитают птицы небесные, из среды ветвей издают голос.
\vs Psa 103:13 Ты напояешь горы с высот Твоих, плодами дел Твоих насыщается земля.
\vs Psa 103:14 Ты произращаешь траву для скота, и зелень на пользу человека, чтобы произвести из земли пищу,
\vs Psa 103:15 и вино, которое веселит сердце человека, и елей, от которого блистает лице его, и хлеб, который укрепляет сердце человека.
\vs Psa 103:16 Насыщаются древа Господа, кедры Ливанские, которые Он насадил;
\vs Psa 103:17 на них гнездятся птицы: ели~--- жилище аисту,
\vs Psa 103:18 высокие горы~--- сернам; каменные утесы~--- убежище зайцам.
\vs Psa 103:19 Он сотворил луну для \bibemph{указания} времен, солнце знает свой запад.
\vs Psa 103:20 Ты простираешь тьму и бывает ночь: во время нее бродят все лесные звери;
\vs Psa 103:21 львы рыкают о добыче и просят у Бога пищу себе.
\vs Psa 103:22 Восходит солнце, [и] они собираются и ложатся в свои логовища;
\vs Psa 103:23 выходит человек на дело свое и на работу свою до вечера.
\vs Psa 103:24 Как многочисленны дела Твои, Господи! Все соделал Ты премудро; земля полна произведений Твоих.
\vs Psa 103:25 Это~--- море великое и пространное: там пресмыкающиеся, которым нет числа, животные малые с большими;
\vs Psa 103:26 там плавают корабли, там этот левиафан, которого Ты сотворил играть в нем.
\vs Psa 103:27 Все они от Тебя ожидают, чтобы Ты дал им пищу их в свое время.
\vs Psa 103:28 Даешь им~--- принимают, отверзаешь руку Твою~--- насыщаются благом;
\vs Psa 103:29 скроешь лице Твое~--- мятутся, отнимешь дух их~--- умирают и в персть свою возвращаются;
\vs Psa 103:30 пошлешь дух Твой~--- созидаются, и Ты обновляешь лице земли.
\vs Psa 103:31 Да будет Господу слава во веки; да веселится Господь о делах Своих!
\vs Psa 103:32 Призирает на землю, и она трясется; прикасается к горам, и дымятся.
\vs Psa 103:33 Буду петь Господу во \bibemph{всю} жизнь мою, буду петь Богу моему, доколе есмь.
\vs Psa 103:34 Да будет благоприятна Ему песнь моя; буду веселиться о Господе.
\vs Psa 103:35 Да исчезнут грешники с земли, и беззаконных да не будет более. Благослови, душа моя, Господа! Аллилуия!
\vs Psa 104:1 Славьте Господа; призывайте имя Его; возвещайте в народах дела Его;
\vs Psa 104:2 воспойте Ему и пойте Ему; поведайте о всех чудесах Его.
\vs Psa 104:3 Хвалитесь именем Его святым; да веселится сердце ищущих Господа.
\vs Psa 104:4 Ищите Господа и силы Его, ищите лица Его всегда.
\vs Psa 104:5 Воспоминайте чудеса Его, которые сотворил, знамения Его и суды уст Его,
\vs Psa 104:6 вы, семя Авраамово, рабы Его, сыны Иакова, избранные Его.
\vs Psa 104:7 Он Господь Бог наш: по всей земле суды Его.
\vs Psa 104:8 Вечно помнит завет Свой, слово, [которое] заповедал в тысячу родов,
\vs Psa 104:9 которое завещал Аврааму, и клятву Свою Исааку,
\vs Psa 104:10 и поставил то Иакову в закон и Израилю в завет вечный,
\vs Psa 104:11 говоря: <<тебе дам землю Ханаанскую в удел наследия вашего>>.
\vs Psa 104:12 Когда их было еще мало числом, очень мало, и они были пришельцами в ней
\vs Psa 104:13 и переходили от народа к народу, из царства к иному племени,
\vs Psa 104:14 никому не позволял обижать их и возбранял о них царям:
\vs Psa 104:15 <<не прикасайтесь к помазанным Моим, и пророкам Моим не делайте зла>>.
\vs Psa 104:16 И призвал голод на землю; всякий стебель хлебный истребил.
\vs Psa 104:17 Послал пред ними человека: в рабы продан был Иосиф.
\vs Psa 104:18 Стеснили оковами ноги его; в железо вошла душа его,
\vs Psa 104:19 доколе исполнилось слово Его: слово Господне испытало его.
\vs Psa 104:20 Послал царь, и разрешил его владетель народов и освободил его;
\vs Psa 104:21 поставил его господином над домом своим и правителем над всем владением своим,
\vs Psa 104:22 чтобы он наставлял вельмож его по своей душе и старейшин его учил мудрости.
\vs Psa 104:23 Тогда пришел Израиль в Египет, и переселился Иаков в землю Хамову.
\vs Psa 104:24 И весьма размножил \bibemph{Бог} народ Свой и сделал его сильнее врагов его.
\vs Psa 104:25 Возбудил в сердце их ненависть против народа Его и ухищрение против рабов Его.
\vs Psa 104:26 Послал Моисея, раба Своего, Аарона, которого избрал.
\vs Psa 104:27 Они показали между ними слова знамений Его и чудеса [Его] в земле Хамовой.
\vs Psa 104:28 Послал тьму и сделал мрак, и не воспротивились слову Его.
\vs Psa 104:29 Преложил воду их в кровь, и уморил рыбу их.
\vs Psa 104:30 Земля их произвела множество жаб \bibemph{даже} в спальне царей их.
\vs Psa 104:31 Он сказал, и пришли разные насекомые, скнипы во все пределы их.
\vs Psa 104:32 Вместо дождя послал на них град, палящий огонь на землю их,
\vs Psa 104:33 и побил виноград их и смоковницы их, и сокрушил дерева в пределах их.
\vs Psa 104:34 Сказал, и пришла саранча и гусеницы без числа;
\vs Psa 104:35 и съели всю траву на земле их, и съели плоды на полях их.
\vs Psa 104:36 И поразил всякого первенца в земле их, начатки всей силы их.
\vs Psa 104:37 И вывел \bibemph{Израильтян} с серебром и золотом, и не было в коленах их болящего.
\vs Psa 104:38 Обрадовался Египет исшествию их; ибо страх от них напал на него.
\vs Psa 104:39 Простер облако в покров [им] и огонь, чтобы светить [им] ночью.
\vs Psa 104:40 Просили, и Он послал перепелов, и хлебом небесным насыщал их.
\vs Psa 104:41 Разверз камень, и потекли воды, потекли рекою по местам сухим,
\vs Psa 104:42 ибо вспомнил Он святое слово Свое к Аврааму, рабу Своему,
\vs Psa 104:43 и вывел народ Свой в радости, избранных Своих в веселии,
\vs Psa 104:44 и дал им земли народов, и они наследовали труд иноплеменных,
\vs Psa 104:45 чтобы соблюдали уставы Его и хранили законы Его. Аллилуия!
\vs Psa 105:0 Аллилуия.
\rsbpar\vs Psa 105:1 Славьте Господа, ибо Он благ, ибо вовек милость Его.
\vs Psa 105:2 Кто изречет могущество Господа, возвестит все хвалы Его?
\vs Psa 105:3 Блаженны хранящие суд и творящие правду во всякое время!
\vs Psa 105:4 Вспомни о мне, Господи, в благоволении к народу Твоему; посети меня спасением Твоим,
\vs Psa 105:5 дабы мне видеть благоденствие избранных Твоих, веселиться веселием народа Твоего, хвалиться с наследием Твоим.
\vs Psa 105:6 Согрешили мы с отцами нашими, совершили беззаконие, соделали неправду.
\vs Psa 105:7 Отцы наши в Египте не уразумели чудес Твоих, не помнили множества милостей Твоих, и возмутились у моря, у Чермного моря.
\vs Psa 105:8 Но Он спас их ради имени Своего, дабы показать могущество Свое.
\vs Psa 105:9 Грозно рек морю Чермному, и оно иссохло; и провел их по безднам, как по суше;
\vs Psa 105:10 и спас их от руки ненавидящего и избавил их от руки врага.
\vs Psa 105:11 Воды покрыли врагов их, ни одного из них не осталось.
\vs Psa 105:12 И поверили они словам Его, [и] воспели хвалу Ему.
\vs Psa 105:13 \bibemph{Но} скоро забыли дела Его, не дождались Его изволения;
\vs Psa 105:14 увлеклись похотением в пустыне, и искусили Бога в необитаемой.
\vs Psa 105:15 И Он исполнил прошение их, \bibemph{но} послал язву на души их.
\vs Psa 105:16 И позавидовали в стане Моисею \bibemph{и} Аарону, святому Господню.
\vs Psa 105:17 Разверзлась земля, и поглотила Дафана и покрыла скопище Авирона.
\vs Psa 105:18 И возгорелся огонь в скопище их, пламень попалил нечестивых.
\vs Psa 105:19 Сделали тельца у Хорива и поклонились истукану;
\vs Psa 105:20 и променяли славу свою на изображение вола, ядущего траву.
\vs Psa 105:21 Забыли Бога, Спасителя своего, совершившего великое в Египте,
\vs Psa 105:22 дивное в земле Хамовой, страшное у Чермного моря.
\vs Psa 105:23 И хотел истребить их, если бы Моисей, избранный Его, не стал пред Ним в расселине, чтобы отвратить ярость Его, да не погубит [их].
\vs Psa 105:24 И презрели они землю желанную, не верили слову Его;
\vs Psa 105:25 и роптали в шатрах своих, не слушались гласа Господня.
\vs Psa 105:26 И поднял Он руку Свою на них, чтобы низложить их в пустыне,
\vs Psa 105:27 низложить племя их в народах и рассеять их по землям.
\vs Psa 105:28 Они прилепились к Ваалфегору и ели жертвы бездушным,
\vs Psa 105:29 и раздражали \bibemph{Бога} делами своими, и вторглась к ним язва.
\vs Psa 105:30 И восстал Финеес и произвел суд,~--- и остановилась язва.
\vs Psa 105:31 И \bibemph{это} вменено ему в праведность в роды и роды во веки.
\vs Psa 105:32 И прогневали \bibemph{Бога} у вод Меривы, и Моисей потерпел за них,
\vs Psa 105:33 ибо они огорчили дух его, и он погрешил устами своими.
\vs Psa 105:34 Не истребили народов, о которых сказал им Господь,
\vs Psa 105:35 но смешались с язычниками и научились делам их;
\vs Psa 105:36 служили истуканам их, \bibemph{которые} были для них сетью,
\vs Psa 105:37 и приносили сыновей своих и дочерей своих в жертву бесам;
\vs Psa 105:38 проливали кровь невинную, кровь сыновей своих и дочерей своих, которых приносили в жертву идолам Ханаанским,~--- и осквернилась земля кровью;
\vs Psa 105:39 оскверняли себя делами своими, блудодействовали поступками своими.
\vs Psa 105:40 И воспылал гнев Господа на народ Его, и возгнушался Он наследием Своим
\vs Psa 105:41 и предал их в руки язычников, и ненавидящие их стали обладать ими.
\vs Psa 105:42 Враги их утесняли их, и они смирялись под рукою их.
\vs Psa 105:43 Много раз Он избавлял их; они же раздражали [Его] упорством своим, и были уничижаемы за беззаконие свое.
\vs Psa 105:44 Но Он призирал на скорбь их, когда слышал вопль их,
\vs Psa 105:45 и вспоминал завет Свой с ними и раскаивался по множеству милости Своей;
\vs Psa 105:46 и возбуждал к ним сострадание во всех, пленявших их.
\vs Psa 105:47 Спаси нас, Господи, Боже наш, и собери нас от народов, дабы славить святое имя Твое, хвалиться Твоею славою.
\vs Psa 105:48 Благословен Господь, Бог Израилев, от века и до века! И да скажет весь народ: аминь! Аллилуия!
\vs Psa 106:0 [Аллилуия.]
\rsbpar\vs Psa 106:1 Славьте Господа, ибо Он благ, ибо вовек милость Его!
\vs Psa 106:2 Так да скажут избавленные Господом, которых избавил Он от руки врага,
\vs Psa 106:3 и собрал от стран, от востока и запада, от севера и моря.
\vs Psa 106:4 Они блуждали в пустыне по безлюдному пути и не находили населенного города;
\vs Psa 106:5 терпели голод и жажду, душа их истаевала в них.
\vs Psa 106:6 Но воззвали к Господу в скорби своей, и Он избавил их от бедствий их,
\vs Psa 106:7 и повел их прямым путем, чтобы они шли к населенному городу.
\vs Psa 106:8 Да славят Господа за милость Его и за чудные дела Его для сынов человеческих:
\vs Psa 106:9 ибо Он насытил душу жаждущую и душу алчущую исполнил благами.
\vs Psa 106:10 Они сидели во тьме и тени смертной, окованные скорбью и железом;
\vs Psa 106:11 ибо не покорялись словам Божиим и небрегли о воле Всевышнего.
\vs Psa 106:12 Он смирил сердце их работами; они преткнулись, и не было помогающего.
\vs Psa 106:13 Но воззвали к Господу в скорби своей, и Он спас их от бедствий их;
\vs Psa 106:14 вывел их из тьмы и тени смертной, и расторгнул узы их.
\vs Psa 106:15 Да славят Господа за милость Его и за чудные дела Его для сынов человеческих:
\vs Psa 106:16 ибо Он сокрушил врата медные и вереи железные сломил.
\vs Psa 106:17 Безрассудные страдали за беззаконные пути свои и за неправды свои;
\vs Psa 106:18 от всякой пищи отвращалась душа их, и они приближались ко вратам смерти.
\vs Psa 106:19 Но воззвали к Господу в скорби своей, и Он спас их от бедствий их;
\vs Psa 106:20 послал слово Свое и исцелил их, и избавил их от могил их.
\vs Psa 106:21 Да славят Господа за милость Его и за чудные дела Его для сынов человеческих!
\vs Psa 106:22 Да приносят Ему жертву хвалы и да возвещают о делах Его с пением!
\vs Psa 106:23 Отправляющиеся на кораблях в море, производящие дела на больших водах,
\vs Psa 106:24 видят дела Господа и чудеса Его в пучине:
\vs Psa 106:25 Он речет,~--- и восстает бурный ветер и высоко поднимает волны его:
\vs Psa 106:26 восходят до небес, нисходят до бездны; душа их истаевает в бедствии;
\vs Psa 106:27 они кружатся и шатаются, как пьяные, и вся мудрость их исчезает.
\vs Psa 106:28 Но воззвали к Господу в скорби своей, и Он вывел их из бедствия их.
\vs Psa 106:29 Он превращает бурю в тишину, и волны умолкают.
\vs Psa 106:30 И веселятся, что они утихли, и Он приводит их к желаемой пристани.
\vs Psa 106:31 Да славят Господа за милость Его и за чудные дела Его для сынов человеческих!
\vs Psa 106:32 Да превозносят Его в собрании народном и да славят Его в сонме старейшин!
\vs Psa 106:33 Он превращает реки в пустыню и источники вод~--- в сушу,
\vs Psa 106:34 землю плодородную~--- в солончатую, за нечестие живущих на ней.
\vs Psa 106:35 Он превращает пустыню в озеро, и землю иссохшую~--- в источники вод;
\vs Psa 106:36 и поселяет там алчущих, и они строят город для обитания;
\vs Psa 106:37 засевают поля, насаждают виноградники, которые приносят им обильные плоды.
\vs Psa 106:38 Он благословляет их, и они весьма размножаются, и скота их не умаляет.
\vs Psa 106:39 Уменьшились они и упали от угнетения, бедствия и скорби,~---
\vs Psa 106:40 Он изливает бесчестие на князей и оставляет их блуждать в пустыне, где нет путей.
\vs Psa 106:41 Бедного же извлекает из бедствия и умножает род его, как стада овец.
\vs Psa 106:42 Праведники видят сие и радуются, а всякое нечестие заграждает уста свои.
\vs Psa 106:43 Кто мудр, тот заметит сие и уразумеет милость Господа.
\vs Psa 107:1 Песнь. Псалом Давида.
\rsbpar\vs Psa 107:2 Готово сердце мое, Боже, [готово сердце мое]; буду петь и воспевать во славе моей.
\vs Psa 107:3 Воспрянь, псалтирь и гусли! Я встану рано.
\vs Psa 107:4 Буду славить Тебя, Господи, между народами; буду воспевать Тебя среди племен,
\vs Psa 107:5 ибо превыше небес милость Твоя и до облаков истина Твоя.
\vs Psa 107:6 Будь превознесен выше небес, Боже; над всею землею \bibemph{да будет} слава Твоя,
\vs Psa 107:7 дабы избавились возлюбленные Твои: спаси десницею Твоею и услышь меня.
\vs Psa 107:8 Бог сказал во святилище Своем: <<восторжествую, разделю Сихем и долину Сокхоф размерю;
\vs Psa 107:9 Мой Галаад, Мой Манассия, Ефрем~--- крепость главы Моей, Иуда~--- скипетр Мой,
\vs Psa 107:10 Моав~--- умывальная чаша Моя, на Едома простру сапог Мой, над землею Филистимскою восклицать буду>>.
\vs Psa 107:11 Кто введет меня в укрепленный город? Кто доведет меня до Едома?
\vs Psa 107:12 Не Ты ли, Боже, \bibemph{Который} отринул нас и не выходишь, Боже, с войсками нашими?
\vs Psa 107:13 Подай нам помощь в тесноте, ибо защита человеческая суетна.
\vs Psa 107:14 С Богом мы окажем силу: Он низложит врагов наших.
\vs Psa 108:0 Начальнику хора. Псалом Давида.
\rsbpar\vs Psa 108:1 Боже хвалы моей! не премолчи,
\vs Psa 108:2 ибо отверзлись на меня уста нечестивые и уста коварные; говорят со мною языком лживым;
\vs Psa 108:3 отвсюду окружают меня словами ненависти, вооружаются против меня без причины;
\vs Psa 108:4 за любовь мою они враждуют на меня, а я молюсь;
\vs Psa 108:5 воздают мне за добро злом, за любовь мою~--- ненавистью.
\vs Psa 108:6 Поставь над ним нечестивого, и диавол да станет одесную его.
\vs Psa 108:7 Когда будет судиться, да выйдет виновным, и молитва его да будет в грех;
\vs Psa 108:8 да будут дни его кратки, и достоинство его да возьмет другой;
\vs Psa 108:9 дети его да будут сиротами, и жена его~--- вдовою;
\vs Psa 108:10 да скитаются дети его и нищенствуют, и просят \bibemph{хлеба} из развалин своих;
\vs Psa 108:11 да захватит заимодавец все, что есть у него, и чужие да расхитят труд его;
\vs Psa 108:12 да не будет сострадающего ему, да не будет милующего сирот его;
\vs Psa 108:13 да будет потомство его на погибель, и да изгладится имя их в следующем роде;
\vs Psa 108:14 да будет воспомянуто пред Господом беззаконие отцов его, и грех матери его да не изгладится;
\vs Psa 108:15 да будут они всегда в очах Господа, и да истребит Он память их на земле,
\vs Psa 108:16 за то, что он не думал оказывать милость, но преследовал человека бедного и нищего и сокрушенного сердцем, чтобы умертвить его;
\vs Psa 108:17 возлюбил проклятие,~--- оно и придет на него; не восхотел благословения,~--- оно и удалится от него;
\vs Psa 108:18 да облечется проклятием, как ризою, и да войдет оно, как вода, во внутренность его и, как елей, в кости его;
\vs Psa 108:19 да будет оно ему, как одежда, в которую он одевается, и как пояс, которым всегда опоясывается.
\vs Psa 108:20 Таково воздаяние от Господа врагам моим и говорящим злое на душу мою!
\vs Psa 108:21 Со мною же, Господи, Господи, твори ради имени Твоего, ибо блага милость Твоя; спаси меня,
\vs Psa 108:22 ибо я беден и нищ, и сердце мое уязвлено во мне.
\vs Psa 108:23 Я исчезаю, как уклоняющаяся тень; гонят меня, как саранчу.
\vs Psa 108:24 Колени мои изнемогли от поста, и тело мое лишилось тука.
\vs Psa 108:25 Я стал для них посмешищем: увидев меня, кивают головами [своими].
\vs Psa 108:26 Помоги мне, Господи, Боже мой, спаси меня по милости Твоей,
\vs Psa 108:27 да познают, что это~--- Твоя рука, и что Ты, Господи, соделал это.
\vs Psa 108:28 Они проклинают, а Ты благослови; они восстают, но да будут постыжены; раб же Твой да возрадуется.
\vs Psa 108:29 Да облекутся противники мои бесчестьем и, как одеждою, покроются стыдом своим.
\vs Psa 108:30 И я громко буду устами моими славить Господа и среди множества прославлять Его,
\vs Psa 108:31 ибо Он стоит одесную бедного, чтобы спасти его от судящих душу его.
\vs Psa 109:0 Псалом Давида.
\rsbpar\vs Psa 109:1 Сказал Господь Господу моему: седи одесную Меня, доколе положу врагов Твоих в подножие ног Твоих.
\vs Psa 109:2 Жезл силы Твоей пошлет Господь с Сиона: господствуй среди врагов Твоих.
\vs Psa 109:3 В день силы Твоей народ Твой готов во благолепии святыни; из чрева прежде денницы подобно росе рождение Твое\fns{По переводу 70-ти: из чрева прежде денницы Я родил Тебя.}.
\vs Psa 109:4 Клялся Господь и не раскается: Ты священник вовек по чину Мелхиседека.
\vs Psa 109:5 Господь одесную Тебя. Он в день гнева Своего поразит царей;
\vs Psa 109:6 совершит суд над народами, наполнит \bibemph{землю} трупами, сокрушит голову в земле обширной.
\vs Psa 109:7 Из потока на пути будет пить, и потому вознесет главу.
\vs Psa 110:0 Аллилуия.
\rsbpar\vs Psa 110:1 Славлю [Тебя], Господи, всем сердцем [моим] в совете праведных и в собрании.
\vs Psa 110:2 Велики дела Господни, вожделенны для всех, любящих оные.
\vs Psa 110:3 Дело Его~--- слава и красота, и правда Его пребывает вовек.
\vs Psa 110:4 Памятными соделал Он чудеса Свои; милостив и щедр Господь.
\vs Psa 110:5 Пищу дает боящимся Его; вечно помнит завет Свой.
\vs Psa 110:6 Силу дел Своих явил Он народу Своему, чтобы дать ему наследие язычников.
\vs Psa 110:7 Дела рук Его~--- истина и суд; все заповеди Его верны,
\vs Psa 110:8 тверды на веки и веки, основаны на истине и правоте.
\vs Psa 110:9 Избавление послал Он народу Своему; заповедал на веки завет Свой. Свято и страшно имя Его!
\vs Psa 110:10 Начало мудрости~--- страх Господень; разум верный у всех, исполняющих \bibemph{заповеди Его}. Хвала Ему пребудет вовек.
\vs Psa 111:0 Аллилуия.
\rsbpar\vs Psa 111:1 Блажен муж, боящийся Господа и крепко любящий заповеди Его.
\vs Psa 111:2 Сильно будет на земле семя его; род правых благословится.
\vs Psa 111:3 Обилие и богатство в доме его, и правда его пребывает вовек.
\vs Psa 111:4 Во тьме восходит свет правым; благ он и милосерд и праведен.
\vs Psa 111:5 Добрый человек милует и взаймы дает; он даст твердость словам своим на суде.
\vs Psa 111:6 Он вовек не поколеблется; в вечной памяти будет праведник.
\vs Psa 111:7 Не убоится худой молвы: сердце его твердо, уповая на Господа.
\vs Psa 111:8 Утверждено сердце его: он не убоится, когда посмотрит на врагов своих.
\vs Psa 111:9 Он расточил, раздал нищим; правда его пребывает во веки; рог его вознесется во славе.
\vs Psa 111:10 Нечестивый увидит \bibemph{это} и будет досадовать, заскрежещет зубами своими и истает. Желание нечестивых погибнет.
\vs Psa 112:0 Аллилуия.
\rsbpar\vs Psa 112:1 Хвалите, рабы Господни, хвалите имя Господне.
\vs Psa 112:2 Да будет имя Господне благословенно отныне и вовек.
\vs Psa 112:3 От восхода солнца до запада \bibemph{да будет} прославляемо имя Господне.
\vs Psa 112:4 Высок над всеми народами Господь; над небесами слава Его.
\vs Psa 112:5 Кто, как Господь, Бог наш, Который, обитая на высоте,
\vs Psa 112:6 приклоняется, чтобы призирать на небо и на землю;
\vs Psa 112:7 из праха поднимает бедного, из брения возвышает нищего,
\vs Psa 112:8 чтобы посадить его с князьями, с князьями народа его;
\vs Psa 112:9 неплодную вселяет в дом матерью, радующеюся о детях? Аллилуия!
\vs Psa 113:0 [Аллилуия.]
\rsbpar\vs Psa 113:1 Когда вышел Израиль из Египта, дом Иакова~--- из народа иноплеменного,
\vs Psa 113:2 Иуда сделался святынею Его, Израиль~--- владением Его.
\vs Psa 113:3 Море увидело и побежало; Иордан обратился назад.
\vs Psa 113:4 Горы прыгали, как овны, и холмы, как агнцы.
\vs Psa 113:5 Что с тобою, море, что ты побежало, и [с тобою], Иордан, что ты обратился назад?
\vs Psa 113:6 Что вы прыгаете, горы, как овны, и вы, холмы, как агнцы?
\vs Psa 113:7 Пред лицем Господа трепещи, земля, пред лицем Бога Иаковлева,
\vs Psa 113:8 превращающего скалу в озеро воды и камень в источник вод.
\vs Psa 113:9 Не нам, Господи, не нам, но имени Твоему дай славу, ради милости Твоей, ради истины Твоей.
\vs Psa 113:10 Для чего язычникам говорить: <<где же Бог их>>?
\vs Psa 113:11 Бог наш на небесах [и на земле]; творит все, что хочет.
\vs Psa 113:12 А их идолы~--- серебро и золото, дело рук человеческих.
\vs Psa 113:13 Есть у них уста, но не говорят; есть у них глаза, но не видят;
\vs Psa 113:14 есть у них уши, но не слышат; есть у них ноздри, но не обоняют;
\vs Psa 113:15 есть у них руки, но не осязают; есть у них ноги, но не ходят; и они не издают голоса гортанью своею.
\vs Psa 113:16 Подобны им да будут делающие их и все, надеющиеся на них.
\vs Psa 113:17 [Дом] Израилев! уповай на Господа: Он наша помощь и щит.
\vs Psa 113:18 Дом Ааронов! уповай на Господа: Он наша помощь и щит.
\vs Psa 113:19 Боящиеся Господа! уповайте на Господа: Он наша помощь и щит.
\vs Psa 113:20 Господь помнит нас, благословляет [нас], благословляет дом Израилев, благословляет дом Ааронов;
\vs Psa 113:21 благословляет боящихся Господа, малых с великими.
\vs Psa 113:22 Да приложит вам Господь более и более, вам и детям вашим.
\vs Psa 113:23 Благословенны вы Господом, сотворившим небо и землю.
\vs Psa 113:24 Небо~--- небо Господу, а землю Он дал сынам человеческим.
\vs Psa 113:25 Ни мертвые восхвалят Господа, ни все нисходящие в могилу;
\vs Psa 113:26 но мы [живые] будем благословлять Господа отныне и вовек. Аллилуия.
\vs Psa 114:0 [Аллилуия.]
\rsbpar\vs Psa 114:1 Я радуюсь, что Господь услышал голос мой, моление мое;
\vs Psa 114:2 приклонил ко мне ухо Свое, и потому буду призывать Его во \bibemph{все} дни мои.
\vs Psa 114:3 Объяли меня болезни смертные, муки адские постигли меня; я встретил тесноту и скорбь.
\vs Psa 114:4 Тогда призвал я имя Господне: Господи! избавь душу мою.
\vs Psa 114:5 Милостив Господь и праведен, и милосерд Бог наш.
\vs Psa 114:6 Хранит Господь простодушных: я изнемог, и Он помог мне.
\vs Psa 114:7 Возвратись, душа моя, в покой твой, ибо Господь облагодетельствовал тебя.
\vs Psa 114:8 Ты избавил душу мою от смерти, очи мои от слез и ноги мои от преткновения. Буду ходить пред лицем Господним на земле живых.
\vs Psa 115:0 [Аллилуия.]
\rsbpar\vs Psa 115:1 Я веровал, и потому говорил: я сильно сокрушен.
\vs Psa 115:2 Я сказал в опрометчивости моей: всякий человек ложь.
\vs Psa 115:3 Что воздам Господу за все благодеяния Его ко мне?
\vs Psa 115:4 Чашу спасения прииму и имя Господне призову.
\vs Psa 115:5 Обеты мои воздам Господу пред всем народом Его.
\vs Psa 115:6 Дорог\acc{а} в очах Господних смерть святых Его!
\vs Psa 115:7 О, Господи! я раб Твой, я раб Твой и сын рабы Твоей; Ты разрешил узы мои.
\vs Psa 115:8 Тебе принесу жертву хвалы, и имя Господне призову.
\vs Psa 115:9 Обеты мои воздам Господу пред всем народом Его,
\vs Psa 115:10 во дворах дома Господня, посреди тебя, Иерусалим! Аллилуия.
\vs Psa 116:0 [Аллилуия.]
\rsbpar\vs Psa 116:1 Хвалите Господа, все народы, прославляйте Его, все племена;
\vs Psa 116:2 ибо велика милость Его к нам, и истина Господня [пребывает] вовек. Аллилуия.
\vs Psa 117:0 [Аллилуия.]
\rsbpar\vs Psa 117:1 Славьте Господа, ибо Он благ, ибо вовек милость Его.
\vs Psa 117:2 Да скажет ныне [дом] Израилев: [Он благ,] ибо вовек милость Его.
\vs Psa 117:3 Да скажет ныне дом Ааронов: [Он благ,] ибо вовек милость Его.
\vs Psa 117:4 Да скажут ныне боящиеся Господа: [Он благ,] ибо вовек милость Его.
\vs Psa 117:5 Из тесноты воззвал я к Господу,~--- и услышал меня, и на пространное место \bibemph{вывел меня} Господь.
\vs Psa 117:6 Господь за меня~--- не устрашусь: что сделает мне человек?
\vs Psa 117:7 Господь мне помощник: буду смотреть на врагов моих.
\vs Psa 117:8 Лучше уповать на Господа, нежели надеяться на человека.
\vs Psa 117:9 Лучше уповать на Господа, нежели надеяться на князей.
\vs Psa 117:10 Все народы окружили меня, но именем Господним я низложил их;
\vs Psa 117:11 обступили меня, окружили меня, но именем Господним я низложил их;
\vs Psa 117:12 окружили меня, как пчелы [сот], и угасли, как огонь в терне: именем Господним я низложил их.
\vs Psa 117:13 Сильно толкнули меня, чтобы я упал, но Господь поддержал меня.
\vs Psa 117:14 Господь~--- сила моя и песнь; Он соделался моим спасением.
\vs Psa 117:15 Глас радости и спасения в жилищах праведников: десница Господня творит силу!
\vs Psa 117:16 Десница Господня высока, десница Господня творит силу!
\vs Psa 117:17 Не умру, но буду жить и возвещать дела Господни.
\vs Psa 117:18 Строго наказал меня Господь, но смерти не предал меня.
\vs Psa 117:19 Отворите мне врата правды; войду в них, прославлю Господа.
\vs Psa 117:20 Вот врата Господа; праведные войдут в них.
\vs Psa 117:21 Славлю Тебя, что Ты услышал меня и соделался моим спасением.
\vs Psa 117:22 Камень, который отвергли строители, соделался главою угла:
\vs Psa 117:23 это~--- от Господа, и есть дивно в очах наших.
\vs Psa 117:24 Сей день сотворил Господь: возрадуемся и возвеселимся в оный!
\vs Psa 117:25 О, Господи, спаси же! О, Господи, споспешествуй же!
\vs Psa 117:26 Благословен грядущий во имя Господне! Благословляем вас из дома Господня.
\vs Psa 117:27 Бог~--- Господь, и осиял нас; вяжите вервями жертву, \bibemph{ведите} к рогам жертвенника.
\vs Psa 117:28 Ты Бог мой: буду славить Тебя; Ты Бог мой: буду превозносить Тебя, [буду славить Тебя, ибо Ты услышал меня и соделался моим спасением].
\vs Psa 117:29 Славьте Господа, ибо Он благ, ибо вовек милость Его.
\vs Psa 118:0 [Аллилуия.]
\rsbpar\vs Psa 118:1 Блаженны непорочные в пути, ходящие в законе Господнем.
\vs Psa 118:2 Блаженны хранящие откровения Его, всем сердцем ищущие Его.
\vs Psa 118:3 Они не делают беззакония, ходят путями Его.
\vs Psa 118:4 Ты заповедал повеления Твои хранить твердо.
\vs Psa 118:5 О, если бы направлялись пути мои к соблюдению уставов Твоих!
\vs Psa 118:6 Тогда я не постыдился бы, взирая на все заповеди Твои:
\vs Psa 118:7 я славил бы Тебя в правоте сердца, поучаясь судам правды Твоей.
\vs Psa 118:8 Буду хранить уставы Твои; не оставляй меня совсем.
\vs Psa 118:9 Как юноше содержать в чистоте путь свой?~--- Хранением себя по слову Твоему.
\vs Psa 118:10 Всем сердцем моим ищу Тебя; не дай мне уклониться от заповедей Твоих.
\vs Psa 118:11 В сердце моем сокрыл я слово Твое, чтобы не грешить пред Тобою.
\vs Psa 118:12 Благословен Ты, Господи! научи меня уставам Твоим.
\vs Psa 118:13 Устами моими возвещал я все суды уст Твоих.
\vs Psa 118:14 На пути откровений Твоих я радуюсь, как во всяком богатстве.
\vs Psa 118:15 О заповедях Твоих размышляю, и взираю на пути Твои.
\vs Psa 118:16 Уставами Твоими утешаюсь, не забываю слова Твоего.
\vs Psa 118:17 Яви милость рабу Твоему, и буду жить и хранить слово Твое.
\vs Psa 118:18 Открой очи мои, и увижу чудеса закона Твоего.
\vs Psa 118:19 Странник я на земле; не скрывай от меня заповедей Твоих.
\vs Psa 118:20 Истомилась душа моя желанием судов Твоих во всякое время.
\vs Psa 118:21 Ты укротил гордых, проклятых, уклоняющихся от заповедей Твоих.
\vs Psa 118:22 Сними с меня поношение и посрамление, ибо я храню откровения Твои.
\vs Psa 118:23 Князья сидят и сговариваются против меня, а раб Твой размышляет об уставах Твоих.
\vs Psa 118:24 Откровения Твои~--- утешение мое, [и уставы Твои]~--- советники мои.
\vs Psa 118:25 Душа моя повержена в прах; оживи меня по слову Твоему.
\vs Psa 118:26 Объявил я пути мои, и Ты услышал меня; научи меня уставам Твоим.
\vs Psa 118:27 Дай мне уразуметь путь повелений Твоих, и буду размышлять о чудесах Твоих.
\vs Psa 118:28 Душа моя истаевает от скорби: укрепи меня по слову Твоему.
\vs Psa 118:29 Удали от меня путь лжи, и закон Твой даруй мне.
\vs Psa 118:30 Я избрал путь истины, поставил пред собою суды Твои.
\vs Psa 118:31 Я прилепился к откровениям Твоим, Господи; не постыди меня.
\vs Psa 118:32 Потеку путем заповедей Твоих, когда Ты расширишь сердце мое.
\vs Psa 118:33 Укажи мне, Господи, путь уставов Твоих, и я буду держаться его до конца.
\vs Psa 118:34 Вразуми меня, и буду соблюдать закон Твой и хранить его всем сердцем.
\vs Psa 118:35 Поставь меня на стезю заповедей Твоих, ибо я возжелал ее.
\vs Psa 118:36 Приклони сердце мое к откровениям Твоим, а не к корысти.
\vs Psa 118:37 Отврати очи мои, чтобы не видеть суеты; животвори меня на пути Твоем.
\vs Psa 118:38 Утверди слово Твое рабу Твоему, ради благоговения пред Тобою.
\vs Psa 118:39 Отврати поношение мое, которого я страшусь, ибо суды Твои благи.
\vs Psa 118:40 Вот, я возжелал повелений Твоих; животвори меня правдою Твоею.
\vs Psa 118:41 Да придут ко мне милости Твои, Господи, спасение Твое по слову Твоему,~---
\vs Psa 118:42 и я дам ответ поносящему меня, ибо уповаю на слово Твое.
\vs Psa 118:43 Не отнимай совсем от уст моих слова истины, ибо я уповаю на суды Твои
\vs Psa 118:44 и буду хранить закон Твой всегда, во веки и веки;
\vs Psa 118:45 буду ходить свободно, ибо я взыскал повелений Твоих;
\vs Psa 118:46 буду говорить об откровениях Твоих пред царями и не постыжусь;
\vs Psa 118:47 буду утешаться заповедями Твоими, которые возлюбил;
\vs Psa 118:48 руки мои буду простирать к заповедям Твоим, которые возлюбил, и размышлять об уставах Твоих.
\vs Psa 118:49 Вспомни слово [Твое] к рабу Твоему, на которое Ты повелел мне уповать:
\vs Psa 118:50 это~--- утешение в бедствии моем, что слово Твое оживляет меня.
\vs Psa 118:51 Гордые крайне ругались надо мною, но я не уклонился от закона Твоего.
\vs Psa 118:52 Вспоминал суды Твои, Господи, от века, и утешался.
\vs Psa 118:53 Ужас овладевает мною при виде нечестивых, оставляющих закон Твой.
\vs Psa 118:54 Уставы Твои были песнями моими на месте странствований моих.
\vs Psa 118:55 Ночью вспоминал я имя Твое, Господи, и хранил закон Твой.
\vs Psa 118:56 Он стал моим, ибо повеления Твои храню.
\vs Psa 118:57 Удел мой, Господи, сказал я, соблюдать слова Твои.
\vs Psa 118:58 Молился я Тебе всем сердцем: помилуй меня по слову Твоему.
\vs Psa 118:59 Размышлял о путях моих и обращал стопы мои к откровениям Твоим.
\vs Psa 118:60 Спешил и не медлил соблюдать заповеди Твои.
\vs Psa 118:61 Сети нечестивых окружили меня, но я не забывал закона Твоего.
\vs Psa 118:62 В полночь вставал славословить Тебя за праведные суды Твои.
\vs Psa 118:63 Общник я всем боящимся Тебя и хранящим повеления Твои.
\vs Psa 118:64 Милости Твоей, Господи, полна земля; научи меня уставам Твоим.
\vs Psa 118:65 Благо сотворил Ты рабу Твоему, Господи, по слову Твоему.
\vs Psa 118:66 Доброму разумению и ведению научи меня, ибо заповедям Твоим я верую.
\vs Psa 118:67 Прежде страдания моего я заблуждался; а ныне слово Твое храню.
\vs Psa 118:68 Благ и благодетелен Ты, [Господи]; научи меня уставам Твоим.
\vs Psa 118:69 Гордые сплетают на меня ложь; я же всем сердцем буду хранить повеления Твои.
\vs Psa 118:70 Ожирело сердце их, как тук; я же законом Твоим утешаюсь.
\vs Psa 118:71 Благо мне, что я пострадал, дабы научиться уставам Твоим.
\vs Psa 118:72 Закон уст Твоих для меня лучше тысяч золота и серебра.
\rsbpar\vs Psa 118:73 Руки Твои сотворили меня и устроили меня; вразуми меня, и научусь заповедям Твоим.
\vs Psa 118:74 Боящиеся Тебя увидят меня~--- и возрадуются, что я уповаю на слово Твое.
\vs Psa 118:75 Знаю, Господи, что суды Твои праведны и по справедливости Ты наказал меня.
\vs Psa 118:76 Да будет же милость Твоя утешением моим, по слову Твоему к рабу Твоему.
\vs Psa 118:77 Да придет ко мне милосердие Твое, и я буду жить; ибо закон Твой~--- утешение мое.
\vs Psa 118:78 Да будут постыжены гордые, ибо безвинно угнетают меня; я размышляю о повелениях Твоих.
\vs Psa 118:79 Да обратятся ко мне боящиеся Тебя и знающие откровения Твои.
\vs Psa 118:80 Да будет сердце мое непорочно в уставах Твоих, чтобы я не посрамился.
\vs Psa 118:81 Истаевает душа моя о спасении Твоем; уповаю на слово Твое.
\vs Psa 118:82 Истаевают очи мои о слове Твоем; я говорю: когда Ты утешишь меня?
\vs Psa 118:83 Я стал, как мех в дыму, \bibemph{но} уставов Твоих не забыл.
\vs Psa 118:84 Сколько дней раба Твоего? Когда произведешь суд над гонителями моими?
\vs Psa 118:85 Яму вырыли мне гордые, вопреки закону Твоему.
\vs Psa 118:86 Все заповеди Твои~--- истина; несправедливо преследуют меня: помоги мне;
\vs Psa 118:87 едва не погубили меня на земле, но я не оставил повелений Твоих.
\vs Psa 118:88 По милости Твоей оживляй меня, и буду хранить откровения уст Твоих.
\vs Psa 118:89 На веки, Господи, слово Твое утверждено на небесах;
\vs Psa 118:90 истина Твоя в род и род. Ты поставил землю, и она стоит.
\vs Psa 118:91 По определениям Твоим все стоит доныне, ибо все служит Тебе.
\vs Psa 118:92 Если бы не закон Твой был утешением моим, погиб бы я в бедствии моем.
\vs Psa 118:93 Вовек не забуду повелений Твоих, ибо ими Ты оживляешь меня.
\vs Psa 118:94 Твой я, спаси меня; ибо я взыскал повелений Твоих.
\vs Psa 118:95 Нечестивые подстерегают меня, чтобы погубить; \bibemph{а} я углубляюсь в откровения Твои.
\vs Psa 118:96 Я видел предел всякого совершенства, \bibemph{но} Твоя заповедь безмерно обширна.
\vs Psa 118:97 Как люблю я закон Твой! весь день размышляю о нем.
\vs Psa 118:98 Заповедью Твоею Ты соделал меня мудрее врагов моих, ибо она всегда со мною.
\vs Psa 118:99 Я стал разумнее всех учителей моих, ибо размышляю об откровениях Твоих.
\vs Psa 118:100 Я сведущ более старцев, ибо повеления Твои храню.
\vs Psa 118:101 От всякого злого пути удерживаю ноги мои, чтобы хранить слово Твое;
\vs Psa 118:102 от судов Твоих не уклоняюсь, ибо Ты научаешь меня.
\vs Psa 118:103 Как сладки гортани моей слова Твои! лучше меда устам моим.
\vs Psa 118:104 Повелениями Твоими я вразумлен; потому ненавижу всякий путь лжи.
\vs Psa 118:105 Слово Твое~--- светильник ноге моей и свет стезе моей.
\vs Psa 118:106 Я клялся хранить праведные суды Твои, и исполню.
\vs Psa 118:107 Сильно угнетен я, Господи; оживи меня по слову Твоему.
\vs Psa 118:108 Благоволи же, Господи, принять добровольную жертву уст моих, и судам Твоим научи меня.
\vs Psa 118:109 Душа моя непрестанно в руке моей, но закона Твоего не забываю.
\vs Psa 118:110 Нечестивые поставили для меня сеть, но я не уклонился от повелений Твоих.
\vs Psa 118:111 Откровения Твои я принял, как наследие на веки, ибо они веселие сердца моего.
\vs Psa 118:112 Я приклонил сердце мое к исполнению уставов Твоих навек, до конца.
\vs Psa 118:113 Вымыслы \bibemph{человеческие} ненавижу, а закон Твой люблю.
\vs Psa 118:114 Ты покров мой и щит мой; на слово Твое уповаю.
\vs Psa 118:115 Удалитесь от меня, беззаконные, и буду хранить заповеди Бога моего.
\vs Psa 118:116 Укрепи меня по слову Твоему, и буду жить; не посрами меня в надежде моей;
\vs Psa 118:117 поддержи меня, и спасусь; и в уставы Твои буду вникать непрестанно.
\vs Psa 118:118 Всех, отступающих от уставов Твоих, Ты низлагаешь, ибо ухищрения их~--- ложь.
\vs Psa 118:119 \bibemph{Как} изгарь, отметаешь Ты всех нечестивых земли; потому я возлюбил откровения Твои.
\vs Psa 118:120 Трепещет от страха Твоего плоть моя, и судов Твоих я боюсь.
\vs Psa 118:121 Я совершал суд и правду; не предай меня гонителям моим.
\vs Psa 118:122 Заступи раба Твоего ко благу \bibemph{его}, чтобы не угнетали меня гордые.
\vs Psa 118:123 Истаевают очи мои, ожидая спасения Твоего и слова правды Твоей.
\vs Psa 118:124 Сотвори с рабом Твоим по милости Твоей, и уставам Твоим научи меня.
\vs Psa 118:125 Я раб Твой: вразуми меня, и познаю откровения Твои.
\vs Psa 118:126 Время Господу действовать: закон Твой разорили.
\vs Psa 118:127 А я люблю заповеди Твои более золота, и золота чистого.
\vs Psa 118:128 Все повеления Твои~--- все призна\acc{ю} справедливыми; всякий путь лжи ненавижу.
\vs Psa 118:129 Дивны откровения Твои; потому хранит их душа моя.
\vs Psa 118:130 Откровение слов Твоих просвещает, вразумляет простых.
\vs Psa 118:131 Открываю уста мои и вздыхаю, ибо заповедей Твоих жажду.
\rsbpar\vs Psa 118:132 Призри на меня и помилуй меня, как поступаешь с любящими имя Твое.
\vs Psa 118:133 Утверди стопы мои в слове Твоем и не дай овладеть мною никакому беззаконию;
\vs Psa 118:134 избавь меня от угнетения человеческого, и буду хранить повеления Твои;
\vs Psa 118:135 осияй раба Твоего светом лица Твоего и научи меня уставам Твоим.
\vs Psa 118:136 Из глаз моих текут потоки вод оттого, что не хранят закона Твоего.
\vs Psa 118:137 Праведен Ты, Господи, и справедливы суды Твои.
\vs Psa 118:138 Откровения Твои, которые Ты заповедал,~--- правда и совершенная истина.
\vs Psa 118:139 Ревность моя снедает меня, потому что мои враги забыли слова Твои.
\vs Psa 118:140 Слово Твое весьма чисто, и раб Твой возлюбил его.
\vs Psa 118:141 Мал я и презрен, \bibemph{но} повелений Твоих не забываю.
\vs Psa 118:142 Правда Твоя~--- правда вечная, и закон Твой~--- истина.
\vs Psa 118:143 Скорбь и горесть постигли меня; заповеди Твои~--- утешение мое.
\vs Psa 118:144 Правда откровений Твоих вечна: вразуми меня, и буду жить.
\vs Psa 118:145 Взываю всем сердцем [моим]: услышь меня, Господи,~--- и сохраню уставы Твои.
\vs Psa 118:146 Призываю Тебя: спаси меня, и буду хранить откровения Твои.
\vs Psa 118:147 Предваряю рассвет и взываю; на слово Твое уповаю.
\vs Psa 118:148 Очи мои предваряют \bibemph{утреннюю} стражу, чтобы мне углубляться в слово Твое.
\vs Psa 118:149 Услышь голос мой по милости Твоей, Господи; по суду Твоему оживи меня.
\vs Psa 118:150 Приблизились замышляющие лукавство; далеки они от закона Твоего.
\vs Psa 118:151 Близок Ты, Господи, и все заповеди Твои~--- истина.
\vs Psa 118:152 Издавна узнал я об откровениях Твоих, что Ты утвердил их на веки.
\vs Psa 118:153 Воззри на бедствие мое и избавь меня, ибо я не забываю закона Твоего.
\vs Psa 118:154 Вступись в дело мое и защити меня; по слову Твоему оживи меня.
\vs Psa 118:155 Далеко от нечестивых спасение, ибо они уставов Твоих не ищут.
\vs Psa 118:156 Много щедрот Твоих, Господи; по суду Твоему оживи меня.
\vs Psa 118:157 Много у меня гонителей и врагов, \bibemph{но} от откровений Твоих я не удаляюсь.
\vs Psa 118:158 Вижу отступников, и сокрушаюсь, ибо они не хранят слова Твоего.
\vs Psa 118:159 Зри, как я люблю повеления Твои; по милости Твоей, Господи, оживи меня.
\vs Psa 118:160 Основание слова Твоего истинно, и вечен всякий суд правды Твоей.
\vs Psa 118:161 Князья гонят меня безвинно, но сердце мое боится слова Твоего.
\vs Psa 118:162 Радуюсь я слову Твоему, как получивший великую прибыль.
\vs Psa 118:163 Ненавижу ложь и гнушаюсь ею; закон же Твой люблю.
\vs Psa 118:164 Семикратно в день прославляю Тебя за суды правды Твоей.
\vs Psa 118:165 Велик мир у любящих закон Твой, и нет им преткновения.
\vs Psa 118:166 Уповаю на спасение Твое, Господи, и заповеди Твои исполняю.
\vs Psa 118:167 Душа моя хранит откровения Твои, и я люблю их крепко.
\vs Psa 118:168 Храню повеления Твои и откровения Твои, ибо все пути мои пред Тобою.
\vs Psa 118:169 Да приблизится вопль мой пред лице Твое, Господи; по слову Твоему вразуми меня.
\vs Psa 118:170 Да придет моление мое пред лице Твое; по слову Твоему избавь меня.
\vs Psa 118:171 Уста мои произнесут хвалу, когда Ты научишь меня уставам Твоим.
\vs Psa 118:172 Язык мой возгласит слово Твое, ибо все заповеди Твои праведны.
\vs Psa 118:173 Да будет рука Твоя в помощь мне, ибо я повеления Твои избрал.
\vs Psa 118:174 Жажду спасения Твоего, Господи, и закон Твой~--- утешение мое.
\vs Psa 118:175 Да живет душа моя и славит Тебя, и суды Твои да помогут мне.
\vs Psa 118:176 Я заблудился, как овца потерянная: взыщи раба Твоего, ибо я заповедей Твоих не забыл.
\vs Psa 119:0 Песнь восхождения.
\rsbpar\vs Psa 119:1 К Господу воззвал я в скорби моей, и Он услышал меня.
\vs Psa 119:2 Господи! избавь душу мою от уст лживых, от языка лукавого.
\vs Psa 119:3 Что даст тебе и что прибавит тебе язык лукавый?
\vs Psa 119:4 Изощренные стрелы сильного, с горящими углями дроковыми.
\vs Psa 119:5 Горе мне, что я пребываю у Мосоха, живу у шатров Кидарских.
\vs Psa 119:6 Долго жила душа моя с ненавидящими мир.
\vs Psa 119:7 Я мирен: но только заговорю, они~--- к войне.
\vs Psa 120:0 Песнь восхождения.
\rsbpar\vs Psa 120:1 Возвожу очи мои к горам, откуда придет помощь моя.
\vs Psa 120:2 Помощь моя от Господа, сотворившего небо и землю.
\vs Psa 120:3 Не даст Он поколебаться ноге твоей, не воздремлет хранящий тебя;
\vs Psa 120:4 не дремлет и не спит хранящий Израиля.
\vs Psa 120:5 Господь~--- хранитель твой; Господь~--- сень твоя с правой руки твоей.
\vs Psa 120:6 Днем солнце не поразит тебя, ни луна ночью.
\vs Psa 120:7 Господь сохранит тебя от всякого зла; сохранит душу твою [Господь].
\vs Psa 120:8 Господь будет охранять выхождение твое и вхождение твое отныне и вовек.
\vs Psa 121:0 Песнь восхождения. Давида.
\rsbpar\vs Psa 121:1 Возрадовался я, когда сказали мне: <<пойдем в дом Господень>>.
\vs Psa 121:2 Вот, стоят ноги наши во вратах твоих, Иерусалим,~---
\vs Psa 121:3 Иерусалим, устроенный как город, слитый в одно,
\vs Psa 121:4 куда восходят колена, колена Господни, по закону Израилеву, славить имя Господне.
\vs Psa 121:5 Там стоят престолы суда, престолы дома Давидова.
\vs Psa 121:6 Просите мира Иерусалиму: да благоденствуют любящие тебя!
\vs Psa 121:7 Да будет мир в стенах твоих, благоденствие~--- в чертогах твоих!
\vs Psa 121:8 Ради братьев моих и ближних моих говорю я: <<мир тебе!>>
\vs Psa 121:9 Ради дома Господа, Бога нашего, желаю блага тебе.
\vs Psa 122:0 Песнь восхождения.
\rsbpar\vs Psa 122:1 К Тебе возвожу очи мои, Живущий на небесах!
\vs Psa 122:2 Вот, как очи рабов \bibemph{обращены} на руку господ их, как очи рабы~--- на руку госпожи ее, так очи наши~--- к Господу, Богу нашему, доколе Он помилует нас.
\vs Psa 122:3 Помилуй нас, Господи, помилуй нас, ибо довольно мы насыщены презрением;
\vs Psa 122:4 довольно насыщена душа наша поношением от надменных и уничижением от гордых.
\vs Psa 123:0 Песнь восхождения. Давида.
\rsbpar\vs Psa 123:1 Если бы не Господь был с нами,~--- да скажет Израиль,~---
\vs Psa 123:2 если бы не Господь был с нами, когда восстали на нас люди,
\vs Psa 123:3 то живых они поглотили бы нас, когда возгорелась ярость их на нас;
\vs Psa 123:4 воды потопили бы нас, поток прошел бы над душею нашею;
\vs Psa 123:5 прошли бы над душею нашею воды бурные.
\vs Psa 123:6 Благословен Господь, Который не дал нас в добычу зубам их!
\vs Psa 123:7 Душа наша избавилась, как птица, из сети ловящих: сеть расторгнута, и мы избавились.
\vs Psa 123:8 Помощь наша~--- в имени Господа, сотворившего небо и землю.
\vs Psa 124:0 Песнь восхождения.
\rsbpar\vs Psa 124:1 Надеющийся на Господа, как гора Сион, не подвигнется: пребывает вовек.
\vs Psa 124:2 Горы окрест Иерусалима, а Господь окрест народа Своего отныне и вовек.
\vs Psa 124:3 Ибо не оставит [Господь] жезла нечестивых над жребием праведных, дабы праведные не простерли рук своих к беззаконию.
\vs Psa 124:4 Благотвори, Господи, добрым и правым в сердцах своих;
\vs Psa 124:5 а совращающихся на кривые пути свои да оставит Господь ходить с делающими беззаконие. Мир на Израиля!
\vs Psa 125:0 Песнь восхождения.
\rsbpar\vs Psa 125:1 Когда возвращал Господь плен Сиона, мы были как бы видящие во сне:
\vs Psa 125:2 тогда уста наши были полны веселья, и язык наш~--- пения; тогда между народами говорили: <<великое сотворил Господь над ними!>>
\vs Psa 125:3 Великое сотворил Господь над нами: мы радовались.
\vs Psa 125:4 Возврати, Господи, пленников наших, как потоки на полдень.
\vs Psa 125:5 Сеявшие со слезами будут пожинать с радостью.
\vs Psa 125:6 С плачем несущий семена возвратится с радостью, неся снопы свои.
\vs Psa 126:0 Песнь восхождения. Соломона.
\rsbpar\vs Psa 126:1 Если Господь не созиждет дома, напрасно трудятся строящие его; если Господь не охранит города, напрасно бодрствует страж.
\vs Psa 126:2 Напрасно вы рано встаете, поздно просиживаете, едите хлеб печали, тогда как возлюбленному Своему Он дает сон.
\vs Psa 126:3 Вот наследие от Господа: дети; награда от Него~--- плод чрева.
\vs Psa 126:4 Что стрелы в руке сильного, то сыновья молодые.
\vs Psa 126:5 Блажен человек, который наполнил ими колчан свой! Не останутся они в стыде, когда будут говорить с врагами в воротах.
\vs Psa 127:0 Песнь восхождения.
\rsbpar\vs Psa 127:1 Блажен всякий боящийся Господа, ходящий путями Его!
\vs Psa 127:2 Ты будешь есть от трудов рук твоих: блажен ты, и благо тебе!
\vs Psa 127:3 Жена твоя, как плодовитая лоза, в доме твоем; сыновья твои, как масличные ветви, вокруг трапезы твоей:
\vs Psa 127:4 так благословится человек, боящийся Господа!
\vs Psa 127:5 Благословит тебя Господь с Сиона, и увидишь благоденствие Иерусалима во все дни жизни твоей;
\vs Psa 127:6 увидишь сыновей у сыновей твоих. Мир на Израиля!
\vs Psa 128:0 Песнь восхождения.
\rsbpar\vs Psa 128:1 Много теснили меня от юности моей, да скажет Израиль:
\vs Psa 128:2 много теснили меня от юности моей, но не одолели меня.
\vs Psa 128:3 На хребте моем орали оратаи, проводили длинные борозды свои.
\vs Psa 128:4 Но Господь праведен: Он рассек узы нечестивых.
\vs Psa 128:5 Да постыдятся и обратятся назад все ненавидящие Сион!
\vs Psa 128:6 Да будут, как трава на кровлях, которая прежде, нежели будет исторгнута, засыхает,
\vs Psa 128:7 которою жнец не наполнит руки своей, и вяжущий снопы~--- горсти своей;
\vs Psa 128:8 и проходящие мимо не скажут: <<благословение Господне на вас; благословляем вас именем Господним!>>
\vs Psa 129:0 Песнь восхождения.
\rsbpar\vs Psa 129:1 Из глубины взываю к Тебе, Господи.
\vs Psa 129:2 Господи! услышь голос мой. Да будут уши Твои внимательны к голосу молений моих.
\vs Psa 129:3 Если Ты, Господи, будешь замечать беззакония,~--- Господи! кто устоит?
\vs Psa 129:4 Но у Тебя прощение, да благоговеют пред Тобою.
\vs Psa 129:5 Надеюсь на Господа, надеется душа моя; на слово Его уповаю.
\vs Psa 129:6 Душа моя ожидает Господа более, нежели стражи~--- утра, более, нежели стражи~--- утра.
\vs Psa 129:7 Да уповает Израиль на Господа, ибо у Господа милость и многое у Него избавление,
\vs Psa 129:8 и Он избавит Израиля от всех беззаконий его.
\vs Psa 130:0 Песнь восхождения. Давида.
\rsbpar\vs Psa 130:1 Господи! не надмевалось сердце мое и не возносились очи мои, и я не входил в великое и для меня недосягаемое.
\vs Psa 130:2 Не смирял ли я и не успокаивал ли души моей, как дитяти, отнятого от груди матери? душа моя была во мне, как дитя, отнятое от груди.
\vs Psa 130:3 Да уповает Израиль на Господа отныне и вовек.
\vs Psa 131:0 Песнь восхождения.
\rsbpar\vs Psa 131:1 Вспомни, Господи, Давида и все сокрушение его:
\vs Psa 131:2 как он клялся Господу, давал обет Сильному Иакова:
\vs Psa 131:3 <<не войду в шатер дома моего, не взойду на ложе мое;
\vs Psa 131:4 не дам сна очам моим и веждам моим~--- дремания,
\vs Psa 131:5 доколе не найду места Господу, жилища~--- Сильному Иакова>>.
\vs Psa 131:6 Вот, мы слышали о нем в Ефрафе, нашли его на полях Иарима.
\vs Psa 131:7 Пойдем к жилищу Его, поклонимся подножию ног Его.
\vs Psa 131:8 Стань, Господи, на \bibemph{место} покоя Твоего,~--- Ты и ковчег могущества Твоего.
\vs Psa 131:9 Священники Твои облекутся правдою, и святые Твои возрадуются.
\vs Psa 131:10 Ради Давида, раба Твоего, не отврати лица помазанника Твоего.
\vs Psa 131:11 Клялся Господь Давиду в истине, и не отречется ее: <<от плода чрева твоего посажу на престоле твоем.
\vs Psa 131:12 Если сыновья твои будут сохранять завет Мой и откровения Мои, которым Я научу их, то и их сыновья во веки будут сидеть на престоле твоем>>.
\vs Psa 131:13 Ибо избрал Господь Сион, возжелал [его] в жилище Себе.
\vs Psa 131:14 <<Это покой Мой на веки: здесь вселюсь, ибо Я возжелал его.
\vs Psa 131:15 Пищу его благословляя благословлю, нищих его насыщу хлебом;
\vs Psa 131:16 священников его облеку во спасение, и святые его радостью возрадуются.
\vs Psa 131:17 Там возращу рог Давиду, поставлю светильник помазаннику Моему.
\vs Psa 131:18 Врагов его облеку стыдом, а на нем будет сиять венец его>>.
\vs Psa 132:0 Песнь восхождения. Давида.
\rsbpar\vs Psa 132:1 Как хорошо и как приятно жить братьям вместе!
\vs Psa 132:2 \bibemph{Это}~--- как драгоценный елей на голове, стекающий на бороду, бороду Ааронову, стекающий на края одежды его;
\vs Psa 132:3 как роса Ермонская, сходящая на горы Сионские, ибо там заповедал Господь благословение и жизнь на веки.
\vs Psa 133:0 Песнь восхождения.
\rsbpar\vs Psa 133:1 Благословите ныне Господа, все рабы Господни, стоящие в доме Господнем, [во дворах дома Бога нашего,] во время ночи.
\vs Psa 133:2 Воздвигните руки ваши к святилищу, и благословите Господа.
\vs Psa 133:3 Благословит тебя Господь с Сиона, сотворивший небо и землю.
\vs Psa 134:0 Аллилуия.
\rsbpar\vs Psa 134:1 Хвалите имя Господне, хвалите, рабы Господни,
\vs Psa 134:2 стоящие в доме Господнем, во дворах дома Бога нашего.
\vs Psa 134:3 Хвалите Господа, ибо Господь благ; пойте имени Его, ибо это сладостно,
\vs Psa 134:4 ибо Господь избрал Себе Иакова, Израиля в собственность Свою.
\vs Psa 134:5 Я познал, что велик Господь, и Господь наш превыше всех богов.
\vs Psa 134:6 Господь творит все, что хочет, на небесах и на земле, на морях и во всех безднах;
\vs Psa 134:7 возводит облака от края земли, творит молнии при дожде, изводит ветер из хранилищ Своих.
\vs Psa 134:8 Он поразил первенцев Египта, от человека до скота,
\vs Psa 134:9 послал знамения и чудеса среди тебя, Египет, на фараона и на всех рабов его,
\vs Psa 134:10 поразил народы многие и истребил царей сильных:
\vs Psa 134:11 Сигона, царя Аморрейского, и Ога, царя Васанского, и все царства Ханаанские;
\vs Psa 134:12 и отдал землю их в наследие, в наследие Израилю, народу Своему.
\vs Psa 134:13 Господи! имя Твое вовек; Господи! память о Тебе в род и род.
\vs Psa 134:14 Ибо Господь будет судить народ Свой и над рабами Своими умилосердится.
\vs Psa 134:15 Идолы язычников~--- серебро и золото, дело рук человеческих:
\vs Psa 134:16 есть у них уста, но не говорят; есть у них глаза, но не видят;
\vs Psa 134:17 есть у них уши, но не слышат, и нет дыхания в устах их.
\vs Psa 134:18 Подобны им будут делающие их и всякий, кто надеется на них.
\vs Psa 134:19 Дом Израилев! благословите Господа. Дом Ааронов! благословите Господа.
\vs Psa 134:20 Дом Левиин! благословите Господа. Боящиеся Господа! благословите Господа.
\vs Psa 134:21 Благословен Господь от Сиона, живущий в Иерусалиме! Аллилуия!
\vs Psa 135:0 [Аллилуия.]
\rsbpar\vs Psa 135:1 Славьте Господа, ибо Он благ, ибо вовек милость Его.
\vs Psa 135:2 Славьте Бога богов, ибо вовек милость Его.
\vs Psa 135:3 Славьте Господа господствующих, ибо вовек милость Его;
\vs Psa 135:4 Того, Который один творит чудеса великие, ибо вовек милость Его;
\vs Psa 135:5 Который сотворил небеса премудро, ибо вовек милость Его;
\vs Psa 135:6 утвердил землю на водах, ибо вовек милость Его;
\vs Psa 135:7 сотворил светила великие, ибо вовек милость Его;
\vs Psa 135:8 солнце~--- для управления днем, ибо вовек милость Его;
\vs Psa 135:9 луну и звезды~--- для управления ночью, ибо вовек милость Его;
\vs Psa 135:10 поразил Египет в первенцах его, ибо вовек милость Его;
\vs Psa 135:11 и вывел Израиля из среды его, ибо вовек милость Его;
\vs Psa 135:12 рукою крепкою и мышцею простертою, ибо вовек милость Его;
\vs Psa 135:13 разделил Чермное море, ибо вовек милость Его;
\vs Psa 135:14 и провел Израиля посреди его, ибо вовек милость Его;
\vs Psa 135:15 и низверг фараона и войско его в море Чермное, ибо вовек милость Его;
\vs Psa 135:16 провел народ Свой чрез пустыню, ибо вовек милость Его;
\vs Psa 135:17 поразил царей великих, ибо вовек милость Его;
\vs Psa 135:18 и убил царей сильных, ибо вовек милость Его;
\vs Psa 135:19 Сигона, царя Аморрейского, ибо вовек милость Его;
\vs Psa 135:20 и Ога, царя Васанского, ибо вовек милость Его;
\vs Psa 135:21 и отдал землю их в наследие, ибо вовек милость Его;
\vs Psa 135:22 в наследие Израилю, рабу Своему, ибо вовек милость Его;
\vs Psa 135:23 вспомнил нас в унижении нашем, ибо вовек милость Его;
\vs Psa 135:24 и избавил нас от врагов наших, ибо вовек милость Его;
\vs Psa 135:25 дает пищу всякой плоти, ибо вовек милость Его.
\vs Psa 135:26 Славьте Бога небес, ибо вовек милость Его.
\vs Psa 136:0 [Давида.]
\rsbpar\vs Psa 136:1 При реках Вавилона, там сидели мы и плакали, когда вспоминали о Сионе;
\vs Psa 136:2 на вербах, посреди его, повесили мы наши арфы.
\vs Psa 136:3 Там пленившие нас требовали от нас слов песней, и притеснители наши~--- веселья: <<пропойте нам из песней Сионских>>.
\vs Psa 136:4 Как нам петь песнь Господню на земле чужой?
\vs Psa 136:5 Если я забуду тебя, Иерусалим,~--- забудь меня десница моя;
\vs Psa 136:6 прилипни язык мой к гортани моей, если не буду помнить тебя, если не поставлю Иерусалима во главе веселия моего.
\vs Psa 136:7 Припомни, Господи, сынам Едомовым день Иерусалима, когда они говорили: <<разрушайте, разрушайте до основания его>>.
\vs Psa 136:8 Дочь Вавилона, опустошительница! блажен, кто воздаст тебе за то, что ты сделала нам!
\vs Psa 136:9 Блажен, кто возьмет и разобьет младенцев твоих о камень!
\vs Psa 137:0 Давида.
\rsbpar\vs Psa 137:1 Славлю Тебя всем сердцем моим, пред богами\fns{В переводе 70-ти: пред Ангелами.} пою Тебе, [что Ты услышал все слова уст моих].
\vs Psa 137:2 Поклоняюсь пред святым храмом Твоим и славлю имя Твое за милость Твою и за истину Твою, ибо Ты возвеличил слово Твое превыше всякого имени Твоего.
\vs Psa 137:3 В день, когда я воззвал, Ты услышал меня, вселил в душу мою бодрость.
\vs Psa 137:4 Прославят Тебя, Господи, все цари земные, когда услышат слова уст Твоих
\vs Psa 137:5 и воспоют пути Господни, ибо велика слава Господня.
\vs Psa 137:6 Высок Господь: и смиренного видит, и гордого узнает издали.
\vs Psa 137:7 Если я пойду посреди напастей, Ты оживишь меня, прострешь на ярость врагов моих руку Твою, и спасет меня десница Твоя.
\vs Psa 137:8 Господь совершит за меня! Милость Твоя, Господи, вовек: дело рук Твоих не оставляй.
\vs Psa 138:0 Начальнику хора. Псалом Давида.
\rsbpar\vs Psa 138:1 Господи! Ты испытал меня и знаешь.
\vs Psa 138:2 Ты знаешь, когда я сажусь и когда встаю; Ты разумеешь помышления мои издали.
\vs Psa 138:3 Иду ли я, отдыхаю ли~--- Ты окружаешь меня, и все пути мои известны Тебе.
\vs Psa 138:4 Еще нет слова на языке моем,~--- Ты, Господи, уже знаешь его совершенно.
\vs Psa 138:5 Сзади и спереди Ты объемлешь меня, и полагаешь на мне руку Твою.
\vs Psa 138:6 Дивно для меня ведение [Твое],~--- высоко, не могу постигнуть его!
\vs Psa 138:7 Куда пойду от Духа Твоего, и от лица Твоего куда убегу?
\vs Psa 138:8 Взойду ли на небо~--- Ты там; сойду ли в преисподнюю~--- и там Ты.
\vs Psa 138:9 Возьму ли крылья зари и переселюсь на край моря,~---
\vs Psa 138:10 и там рука Твоя поведет меня, и удержит меня десница Твоя.
\vs Psa 138:11 Скажу ли: <<может быть, тьма скроет меня, и свет вокруг меня \bibemph{сделается} ночью>>;
\vs Psa 138:12 но и тьма не затмит от Тебя, и ночь светла, как день: как тьма, так и свет.
\vs Psa 138:13 Ибо Ты устроил внутренности мои и соткал меня во чреве матери моей.
\vs Psa 138:14 Славлю Тебя, потому что я дивно устроен. Дивны дела Твои, и душа моя вполне сознает это.
\vs Psa 138:15 Не сокрыты были от Тебя кости мои, когда я созидаем был в тайне, образуем был во глубине утробы.
\vs Psa 138:16 Зародыш мой видели очи Твои; в Твоей книге записаны все дни, для меня назначенные, когда ни одного из них еще не было.
\vs Psa 138:17 Как возвышенны для меня помышления Твои, Боже, и как велико число их!
\vs Psa 138:18 Стану ли исчислять их, но они многочисленнее песка; когда я пробуждаюсь, я все еще с Тобою.
\vs Psa 138:19 О, если бы Ты, Боже, поразил нечестивого! Удалитесь от меня, кровожадные!
\vs Psa 138:20 Они говорят против Тебя нечестиво; суетное замышляют враги Твои.
\vs Psa 138:21 Мне ли не возненавидеть ненавидящих Тебя, Господи, и не возгнушаться восстающими на Тебя?
\vs Psa 138:22 Полною ненавистью ненавижу их: враги они мне.
\vs Psa 138:23 Испытай меня, Боже, и узнай сердце мое; испытай меня и узнай помышления мои;
\vs Psa 138:24 и зри, не на опасном ли я пути, и направь меня на путь вечный.
\vs Psa 139:0 Псалом.
\vs Psa 139:1 Начальнику хора. Псалом Давида.
\rsbpar\vs Psa 139:2 Избавь меня, Господи, от человека злого; сохрани меня от притеснителя:
\vs Psa 139:3 они злое мыслят в сердце, всякий день ополчаются на брань,
\vs Psa 139:4 изощряют язык свой, как змея; яд аспида под устами их.
\vs Psa 139:5 Соблюди меня, Господи, от рук нечестивого, сохрани меня от притеснителей, которые замыслили поколебать стопы мои.
\vs Psa 139:6 Гордые скрыли силки для меня и петли, раскинули сеть по дороге, тенета разложили для меня.
\vs Psa 139:7 Я сказал Господу: Ты Бог мой; услышь, Господи, голос молений моих!
\vs Psa 139:8 Господи, Господи, сила спасения моего! Ты покрыл голову мою в день брани.
\vs Psa 139:9 Не дай, Господи, желаемого нечестивому; не дай успеха злому замыслу его: они возгордятся.
\vs Psa 139:10 Да покроет головы окружающих меня зло собственных уст их.
\vs Psa 139:11 Да падут на них горящие угли; да будут они повержены в огонь, в пропасти, так, чтобы не встали.
\vs Psa 139:12 Человек злоязычный не утвердится на земле; зло увлечет притеснителя в погибель.
\vs Psa 139:13 Знаю, что Господь сотворит суд угнетенным и справедливость бедным.
\vs Psa 139:14 Так! праведные будут славить имя Твое; непорочные будут обитать пред лицем Твоим.
\vs Psa 140:0 Псалом Давида.
\rsbpar\vs Psa 140:1 Господи! к Тебе взываю: поспеши ко мне, внемли голосу моления моего, когда взываю к Тебе.
\vs Psa 140:2 Да направится молитва моя, как фимиам, пред лице Твое, воздеяние рук моих~--- как жертва вечерняя.
\vs Psa 140:3 Положи, Господи, охрану устам моим, и огради двери уст моих;
\vs Psa 140:4 не дай уклониться сердцу моему к словам лукавым для извинения дел греховных вместе с людьми, делающими беззаконие, и да не вкушу я от сластей их.
\vs Psa 140:5 Пусть наказывает меня праведник: это милость; пусть обличает меня: это лучший елей, который не повредит голове моей; но мольбы мои~--- против злодейств их.
\vs Psa 140:6 Вожди их рассыпались по утесам и слышат слова мои, что они кротки.
\vs Psa 140:7 Как будто землю рассекают и дробят нас; сыплются кости наши в челюсти преисподней.
\vs Psa 140:8 Но к Тебе, Господи, Господи, очи мои; на Тебя уповаю, не отринь души моей!
\vs Psa 140:9 Сохрани меня от силков, поставленных для меня, от тенет беззаконников.
\vs Psa 140:10 Падут нечестивые в сети свои, а я перейду.
\vs Psa 141:0 Учение Давида. Молитва его, когда он был в пещере.
\rsbpar\vs Psa 141:1 Голосом моим к Господу воззвал я, голосом моим к Господу помолился;
\vs Psa 141:2 излил пред Ним моление мое; печаль мою открыл Ему.
\vs Psa 141:3 Когда изнемогал во мне дух мой, Ты знал стезю мою. На пути, которым я ходил, они скрытно поставили сети для меня.
\vs Psa 141:4 Смотрю на правую сторону, и вижу, что никто не признаёт меня: не стало для меня убежища, никто не заботится о душе моей.
\vs Psa 141:5 Я воззвал к Тебе, Господи, я сказал: Ты прибежище мое и часть моя на земле живых.
\vs Psa 141:6 Внемли воплю моему, ибо я очень изнемог; избавь меня от гонителей моих, ибо они сильнее меня.
\vs Psa 141:7 Выведи из темницы душу мою, чтобы мне славить имя Твое. Вокруг меня соберутся праведные, когда Ты явишь мне благодеяние.
\vs Psa 142:0 Псалом Давида, [когда он преследуем был сыном своим Авессаломом].
\rsbpar\vs Psa 142:1 Господи! услышь молитву мою, внемли молению моему по истине Твоей; услышь меня по правде Твоей
\vs Psa 142:2 и не входи в суд с рабом Твоим, потому что не оправдается пред Тобой ни один из живущих.
\vs Psa 142:3 Враг преследует душу мою, втоптал в землю жизнь мою, принудил меня жить во тьме, как давно умерших,~---
\vs Psa 142:4 и уныл во мне дух мой, онемело во мне сердце мое.
\vs Psa 142:5 Вспоминаю дни древние, размышляю о всех делах Твоих, рассуждаю о делах рук Твоих.
\vs Psa 142:6 Простираю к Тебе руки мои; душа моя~--- к Тебе, как жаждущая земля.
\vs Psa 142:7 Скоро услышь меня, Господи: дух мой изнемогает; не скрывай лица Твоего от меня, чтобы я не уподобился нисходящим в могилу.
\vs Psa 142:8 Даруй мне рано услышать милость Твою, ибо я на Тебя уповаю. Укажи мне, [Господи,] путь, по которому мне идти, ибо к Тебе возношу я душу мою.
\vs Psa 142:9 Избавь меня, Господи, от врагов моих; к Тебе прибегаю.
\vs Psa 142:10 Научи меня исполнять волю Твою, потому что Ты Бог мой; Дух Твой благий да ведет меня в землю правды.
\vs Psa 142:11 Ради имени Твоего, Господи, оживи меня; ради правды Твоей выведи из напасти душу мою.
\vs Psa 142:12 И по милости Твоей истреби врагов моих и погуби всех, угнетающих душу мою, ибо я Твой раб.
\vs Psa 143:0 Давида. [Против Голиафа.]
\rsbpar\vs Psa 143:1 Благословен Господь, твердыня моя, научающий руки мои битве и персты мои брани,
\vs Psa 143:2 милость моя и ограждение мое, прибежище мое и Избавитель мой, щит мой,~--- и я на Него уповаю; Он подчиняет мне народ мой.
\vs Psa 143:3 Господи! что есть человек, что Ты знаешь о нем, и сын человеческий, что обращаешь на него внимание?
\vs Psa 143:4 Человек подобен дуновению; дни его~--- как уклоняющаяся тень.
\vs Psa 143:5 Господи! Приклони небеса Твои и сойди; коснись гор, и воздымятся;
\vs Psa 143:6 блесни молниею и рассей их; пусти стрелы Твои и расстрой их;
\vs Psa 143:7 простри с высоты руку Твою, избавь меня и спаси меня от вод многих, от руки сынов иноплеменных,
\vs Psa 143:8 которых уста говорят суетное и которых десница~--- десница лжи.
\vs Psa 143:9 Боже! новую песнь воспою Тебе, на десятиструнной псалтири воспою Тебе,
\vs Psa 143:10 дарующему спасение царям и избавляющему Давида, раба Твоего, от лютого меча.
\vs Psa 143:11 Избавь меня и спаси меня от руки сынов иноплеменных, которых уста говорят суетное и которых десница~--- десница лжи.
\vs Psa 143:12 Да будут сыновья наши, как разросшиеся растения в их молодости; дочери наши~--- как искусно изваянные столпы в чертогах.
\vs Psa 143:13 Да будут житницы наши полны, обильны всяким хлебом; да плодятся овцы наши тысячами и тьмами на пажитях наших;
\vs Psa 143:14 \bibemph{да будут} волы наши тучны; да не будет ни расхищения, ни пропажи, ни воплей на улицах наших.
\vs Psa 143:15 Блажен народ, у которого это есть. Блажен народ, у которого Господь есть Бог.
\vs Psa 144:0 Хвала Давида.
\rsbpar\vs Psa 144:1 Буду превозносить Тебя, Боже мой, Царь [мой], и благословлять имя Твое во веки и веки.
\vs Psa 144:2 Всякий день буду благословлять Тебя и восхвалять имя Твое во веки и веки.
\vs Psa 144:3 Велик Господь и достохвален, и величие Его неисследимо.
\vs Psa 144:4 Род роду будет восхвалять дела Твои и возвещать о могуществе Твоем.
\vs Psa 144:5 А я буду размышлять о высокой славе величия Твоего и о дивных делах Твоих.
\vs Psa 144:6 Будут говорить о могуществе страшных дел Твоих, и я буду возвещать о величии Твоем.
\vs Psa 144:7 Будут провозглашать память великой благости Твоей и воспевать правду Твою.
\vs Psa 144:8 Щедр и милостив Господь, долготерпелив и многомилостив.
\vs Psa 144:9 Благ Господь ко всем, и щедроты Его на всех делах Его.
\vs Psa 144:10 Да славят Тебя, Господи, все дела Твои, и да благословляют Тебя святые Твои;
\vs Psa 144:11 да проповедуют славу царства Твоего, и да повествуют о могуществе Твоем,
\vs Psa 144:12 чтобы дать знать сынам человеческим о могуществе Твоем и о славном величии царства Твоего.
\vs Psa 144:13 Царство Твое~--- царство всех веков, и владычество Твое во все роды. [Верен Господь во всех словах Своих и свят во всех делах Своих.]
\vs Psa 144:14 Господь поддерживает всех падающих и восставляет всех низверженных.
\vs Psa 144:15 Очи всех уповают на Тебя, и Ты даешь им пищу их в свое время;
\vs Psa 144:16 открываешь руку Твою и насыщаешь все живущее по благоволению.
\vs Psa 144:17 Праведен Господь во всех путях Своих и благ во всех делах Своих.
\vs Psa 144:18 Близок Господь ко всем призывающим Его, ко всем призывающим Его в истине.
\vs Psa 144:19 Желание боящихся Его Он исполняет, вопль их слышит и спасает их.
\vs Psa 144:20 Хранит Господь всех любящих Его, а всех нечестивых истребит.
\vs Psa 144:21 Уста мои изрекут хвалу Господню, и да благословляет всякая плоть святое имя Его во веки и веки.
\vs Psa 145:0 [Аллилуия. \bibemph{Аггея и Захарии}.]
\rsbpar\vs Psa 145:1 Хвали, душа моя, Господа.
\vs Psa 145:2 Буду восхвалять Господа, доколе жив; буду петь Богу моему, доколе есмь.
\vs Psa 145:3 Не надейтесь на князей, на сына человеческого, в котором нет спасения.
\vs Psa 145:4 Выходит дух его, и он возвращается в землю свою: в тот день исчезают [все] помышления его.
\vs Psa 145:5 Блажен, кому помощник Бог Иаковлев, у кого надежда на Господа Бога его,
\vs Psa 145:6 сотворившего небо и землю, море и все, что в них, вечно хранящего верность,
\vs Psa 145:7 творящего суд обиженным, дающего хлеб алчущим. Господь разрешает узников,
\vs Psa 145:8 Господь отверзает очи слепым, Господь восставляет согбенных, Господь любит праведных.
\vs Psa 145:9 Господь хранит пришельцев, поддерживает сироту и вдову, а путь нечестивых извращает.
\vs Psa 145:10 Господь будет царствовать во веки, Бог твой, Сион, в род и род. Аллилуия.
\vs Psa 146:0 [Аллилуия.]
\rsbpar\vs Psa 146:1 Хвалите Господа, ибо благо петь Богу нашему, ибо это сладостно,~--- хвала подобающая.
\vs Psa 146:2 Господь созидает Иерусалим, собирает изгнанников Израиля.
\vs Psa 146:3 Он исцеляет сокрушенных сердцем и врачует скорби их;
\vs Psa 146:4 исчисляет количество звезд; всех их называет именами их.
\vs Psa 146:5 Велик Господь наш и велика крепость [Его], и разум Его неизмерим.
\vs Psa 146:6 Смиренных возвышает Господь, а нечестивых унижает до земли.
\vs Psa 146:7 Пойте поочередно славословие Господу; пойте Богу нашему на гуслях.
\vs Psa 146:8 Он покрывает небо облаками, приготовляет для земли дождь, произращает на горах траву [и злак на пользу человеку];
\vs Psa 146:9 дает скоту пищу его и птенцам ворона, взывающим \bibemph{к Нему}.
\vs Psa 146:10 Не на силу коня смотрит Он, не к \bibemph{быстроте} ног человеческих благоволит,~---
\vs Psa 146:11 благоволит Господь к боящимся Его, к уповающим на милость Его.
\vs Psa 147:0 [Аллилуия.]
\rsbpar\vs Psa 147:1 Хвали, Иерусалим, Господа; хвали, Сион, Бога твоего,
\vs Psa 147:2 ибо Он укрепляет вереи ворот твоих, благословляет сынов твоих среди тебя;
\vs Psa 147:3 утверждает в пределах твоих мир; туком пшеницы насыщает тебя;
\vs Psa 147:4 посылает слово Свое на землю; быстро течет слово Его;
\vs Psa 147:5 дает снег, как в\acc{о}лну; сыплет иней, как пепел;
\vs Psa 147:6 бросает град Свой кусками; перед морозом Его кто устоит?
\vs Psa 147:7 Пошлет слово Свое, и все растает; подует ветром Своим, и потекут воды.
\vs Psa 147:8 Он возвестил слово Свое Иакову, уставы Свои и суды Свои Израилю.
\vs Psa 147:9 Не сделал Он того никакому \bibemph{другому} народу, и судов Его они не знают. Аллилуия.
\vs Psa 148:0 [Аллилуия.]
\rsbpar\vs Psa 148:1 Хвалите Господа с небес, хвалите Его в вышних.
\vs Psa 148:2 Хвалите Его, все Ангелы Его, хвалите Его, все воинства Его.
\vs Psa 148:3 Хвалите Его, солнце и луна, хвалите Его, все звезды света.
\vs Psa 148:4 Хвалите Его, небеса небес и воды, которые превыше небес.
\vs Psa 148:5 Да хвалят имя Господа, ибо Он [сказал, и они сделались,] повелел, и сотворились;
\vs Psa 148:6 поставил их на веки и веки; дал устав, который не прейдет.
\vs Psa 148:7 Хвалите Господа от земли, великие рыбы и все бездны,
\vs Psa 148:8 огонь и град, снег и туман, бурный ветер, исполняющий слово Его,
\vs Psa 148:9 горы и все холмы, дерева плодоносные и все кедры,
\vs Psa 148:10 звери и всякий скот, пресмыкающиеся и птицы крылатые,
\vs Psa 148:11 цари земные и все народы, князья и все судьи земные,
\vs Psa 148:12 юноши и девицы, старцы и отроки
\vs Psa 148:13 да хвалят имя Господа, ибо имя Его единого превознесенно, слава Его на земле и на небесах.
\vs Psa 148:14 Он возвысил рог народа Своего, славу всех святых Своих, сынов Израилевых, народа, близкого к Нему. Аллилуия.
\vs Psa 149:0 [Аллилуия.]
\rsbpar\vs Psa 149:1 Пойте Господу песнь новую; хвала Ему в собрании святых.
\vs Psa 149:2 Да веселится Израиль о Создателе своем; сыны Сиона да радуются о Царе своем.
\vs Psa 149:3 да хвалят имя Его с ликами, на тимпане и гуслях да поют Ему,
\vs Psa 149:4 ибо благоволит Господь к народу Своему, прославляет смиренных спасением.
\vs Psa 149:5 Да торжествуют святые во славе, да радуются на ложах своих.
\vs Psa 149:6 Да будут славословия Богу в устах их, и меч обоюдоострый в руке их,
\vs Psa 149:7 для того, чтобы совершать мщение над народами, наказание над племенами,
\vs Psa 149:8 заключать царей их в узы и вельмож их в оковы железные,
\vs Psa 149:9 производить над ними суд писанный. Честь сия~--- всем святым Его. Аллилуия.
\vs Psa 150:0 [Аллилуия.]
\rsbpar\vs Psa 150:1 Хвалите Бога во святыне Его, хвалите Его на тверди силы Его.
\vs Psa 150:2 Хвалите Его по могуществу Его, хвалите Его по множеству величия Его.
\vs Psa 150:3 Хвалите Его со звуком трубным, хвалите Его на псалтири и гуслях.
\vs Psa 150:4 Хвалите Его с тимпаном и ликами, хвалите Его на струнах и органе.
\vs Psa 150:5 Хвалите Его на звучных кимвалах, хвалите Его на кимвалах громогласных.
\vs Psa 150:6 Все дышащее да хвалит Господа! Аллилуия.
\vs Psa 151:0 [\bibemph{Псалом Давида на единоборство с Голиафом}\fns{У Евреев этого псалма нет: он переведен с греческого.}.
\rsbpar\vs Psa 151:1 Я был меньший между братьями моими и юнейший в доме отца моего; пас овец отца моего.
\vs Psa 151:2 Руки мои сделали орган, персты мои настраивали псалтирь.
\vs Psa 151:3 И кто возвестил бы Господу моему?~--- Сам Господь, Сам услышал меня.
\vs Psa 151:4 Он послал вестника Своего и взял меня от овец отца моего, и помазал меня елеем помазания Своего.
\vs Psa 151:5 Братья мои прекрасны и велики, но Господь не благоволил избрать из них.
\vs Psa 151:6 Я вышел навстречу иноплеменнику, и он проклял меня идолами своими.
\vs Psa 151:7 Но я, исторгнув у него меч, обезглавил его и избавил сынов Израилевых от поношения.]

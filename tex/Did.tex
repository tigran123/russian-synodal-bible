\bibbookdescr{Did}{
  inline={Учение двенадцати апостолов},
  toc={Учение 12-и апостолов},
  bookmark={Учение 12-и апостолов},
  header={Учение 12-и апостолов},
  abbr={Дид}
}
\chhdr{Учение Господа народам через двенадцать апостолов}
\vs Did 1:1
Есть два пути: один~--- жизни и один~--- смерти,
но между обоими путями большое различие.
\vs Did 1:2
Путь жизни таков.
Во-первых, ты должен любить Бога, создавшего тебя;
во-вторых~--- ближнего своего, как себя самого;
и всего того, чего не хочешь, чтобы было с тобою,
и ты не делай другому.

\vs Did 1:3
Слов же сих учение таково: благословляйте проклинающих вас и
молитесь за врагов ваших, поститесь за гонящих вас, ибо какая
вам за то благодарность, если вы любите любящих вас?
Не то же ли делают и язычники?
Вы же любите ненавидящих вас и не будете иметь врага.
\vs Did 1:4
Удаляйся от плотских и мирских похотей.
Если кто ударит тебя в правую щёку, обрати к нему и другую и будешь совершен.
Если кто наймёт тебя на одну милю, иди с ним две.
Если кто отнимет у тебя верхнюю одежду, отдай и хитон.
Если кто возьмет у тебя твоё, не требуй назад, да и не сможешь.
\vs Did 1:5
Всякому, просящему у тебя, давай и не требуй назад, ибо Отец
хочет чтобы всё подаваемо было из его даров.
Блажен дающий по заповеди, ибо он неповинен.
Горе принимающему, ибо если кто, имея нужду,
принимает, тот будет неповинен, если же кто принимает,
не имея нужды, тот даст отчёт, почему принял и на что:
подвергшись же заключению, испытан будет относительно того,
что сделал, и не выйдет оттуда, пока не отдаст последнего кодранта.
\vs Did 1:6
Но и о сём также сказано: пусть милостыня твоя запотеет
в руках твоих, пока ты не узнаешь, кому дать.

\vs Did 2:1
Вторая же заповедь учения:
\vs Did 2:2
Не убивай,
не прелюбодействуй,
не совершай деторастления,
не будь блудником,
не кради,
не занимайся магией,
не изготавливай волшебных снадобий,
не умерщвляй дитя в зародыше и рожденного не убивай,
не пожелай достояния ближнего твоего.
\vs Did 2:3
Не клянись,
не лжесвидетельствуй,
не злословь,
не злопамятствуй.
\vs Did 2:4
Не будь двоедушным и двуязычным,
ибо двуязычие есть сеть смерти.
\vs Did 2:5
Да не будет слово твоё лживым и пустым, но преисполненным дела.
\vs Did 2:6
Не будь
ни корыстолюбивым,
ни хищником,
ни лицемером,
ни злобным,
ни надменным,
не принимай лукавого умысла на ближнего своего.
\vs Did 2:7
Не имей ненависти ни к одному человеку, но одних обличай, за
других молись, а иных люби более души своей.

\vs Did 3:1
Чадо моё!
Беги от всякого зла и от
всего\fnote{всего}{\vsep\ дела.}
подобного ему.
\vs Did 3:2
Не будь
ни гневливым, ибо гнев ведет к убийству,
ни ревнивым,
ни сварливым,
ни запальчивым, ибо от всего этого рождаются убийства.

\vs Did 3:3
Чадо моё!
Не будь ни похотником, ибо похоть ведет к блуду,
ни срамословом,
ни бесстыжеглазым, ибо от всего этого рождаются прелюбодеяния.

\vs Did 3:4
Чадо моё!
Не гадай по полёту птиц, ибо птицегадание ведет к идолослужению,
не заговаривай,
не занимайся математикой,
ни очищениями,
не желай смотреть на это, ибо от всего этого рождается идолослужение.

\vs Did 3:5
Чадо моё!
Не будь ни лживым, поелику ложь доводит до воровства,
ни сребролюбцем,
ни тщеславным, ибо от всего этого рождается воровство.

\vs Did 3:6
Чадо моё!
Не будь ни ропотником, ибо ропот доводит до богохульства,
ни своенравным,
ни лукавомыслящим, ибо от всего этого рождаются богохульства.
\vs Did 3:7
Но будь кротким, ибо кроткие наследуют землю.
\vs Did 3:8
Будь долготерпеливым, и милостивым, и незлобивым, и смиренным,
и благим, и всегда трепещущим от слов, которые услышал.
\vs Did 3:9
Не превозносись и не допускай в душе своей дерзости.
Да не прилепится душа твоя к гордым,
но обращайся с праведными и смиренными.
\vs Did 3:10
То, что случается с тобой, принимай как благо,
зная, что без Бога ничего не пргоисходит.

\vs Did 4:1
Чадо моё!
Возвещающего тебе Слово Божие помни день и ночь,
почитай же его, как Господа, ибо где возвещается господство,
там Господь есть.
\vs Did 4:2
Даже ищи каждый день иметь личное общение со святыми,
чтобы ты почивал на словах учения их.
\vs Did 4:3
Не производи разделения, а примиряй спорящих; суди праведно, не
будь лицеприятен при обличении преступлений.
\vs Did 4:4
Не думай двоедушно, так или нет.

\vs Did 4:5
Не будь протягивающим руки для принятия подаяний,
но сжимающим её для подаяния.
\vs Did 4:6
Если ты имеешь, что подать от труда рук твоих,
то дай выкуп за грехи твои.
\vs Did 4:7
Не колеблись подать и, подавая, не ропщи,
ибо ты должен знать, кто добрый Мздовоздаятель.
\vs Did 4:8
Не отвращайся от нуждающегося (ср. Сир.4:5), но во всем будь
общником с братом твоим и \bibemph{ничего} не называй своим,
ибо если вы соучастники в нетленном, то насколько более в тленном!
\vs Did 4:9
Не отнимай руки твоей от сына твоего или от дочери твоей,
но от юности учи их страху Божию.

\vs Did 4:10
В гневе твоём не отдавай приказаний рабу твоему или служанке твоей,
надеющимся на того же Бога, дабы они никогда не перестали бояться Бога,
сущего над обоими вами, ибо он пришел призвать ко спасению,
не по лицу судя, а тех коих уготовал дух.
\vs Did 4:11
Вы же, рабы, подчиняйтесь господам своим, как образу Божию,
по совести и со страхом.

\vs Did 4:12
Ненавидь всякое лицемерие и всё, что неугодно Господу.
\vs Did 4:13
Не оставляй заповедей Господа, но храни то, что принял,
не прибавляя и не убавляя.
\vs Did 4:14
В церкви исповедуй преступления свои и не приступай к молитве
своей в лукавой совести.
Этот путь есть путь жизни.

\vs Did 5:1
Путь же смерти таков.
Прежде всего он лукав и исполнен проклятия.
Убийства, прелюбодеяния, вожделения, блуд, кражи, идолослужение,
магия, изготовления снадобий, ограбления, лжесвидетельства, лицемерия,
двоедушие, хитрость, гордыня, злоба, самодовольство, любостяжание,
сквернословие, ревнование, дерзость, высокомерие, бахвальство, бесстрашие.
\vs Did 5:2
гонители благих, ненавистники истины, любители лжи,
не признающие воздаяния за праведность,
не привязывающиеся к благому, ни к суду праведному,
бдящие не во благо, но в зло; от которых далеки кротость и
терпение, любящие суетное, гоняющиеся за мздовоздаянием,
не милующие нищего, не болеззнующие об удрученном,
не вещающие Создавшего их, убийцы детей, губители создания Божия,
отвращающиеся от нуждающегося, обременяющие угнетенного,
заступники богатых, беззаконные судьи бедных, всегрешные.
О если бы вы, чада, избавились от всех таких!

\vs Did 6:1
Смотри, чтобы кто не совратил тебя с этого пути учения, ибо
таковой учит тебя вне Бога.
\vs Did 6:2
Ибо если ты сможешь понести всё иго Господне, то будешь
совершен, а если не можешь, то делай то, что можешь.
\vs Did 6:3
Относительно пищи понеси то, что можешь, но крепко
воздерживайся от идоложертвенного, ибо это есть служение
мёртвым богам.

\vs Did 7:1
Что же \bibemph{касается} крещения,
крестите так: преподав наперед всё это,
крестите во имя Отца и Сына и Святого Духа
в проточной воде.
\vs Did 7:2
Если же нет проточной воды, окрести в иной воде,
а если не можешь в холодной~--- в теплой.
\vs Did 7:3
Если же нет ни той, ни другой, то возлей воду на голову трижды
во имя Отца и Сына и Святого Духа.
\vs Did 7:4
А пред крещением пусть постятся крещающий и крещаемый и, если
могут, некоторые другие; вели же \bibemph{обязательно} поститься
крещаемому день или два до \bibemph{крещения}.

\vs Did 8:1
Посты же ваши да не будут с лицемерами:
они постятся во второй и пятый день недели,
вы же поститесь в четвертый и шестой.
\vs Did 8:2
И не молитесь, как лицемеры, но как повелел Господь в Евангелии своём,
так молитесь:
Отче наш, Cущий на Небе!
Да святится Имя твоё;
да приидет Царствие Твоё;
да будет Воля твоя и на земле, как на Небе;
хлеб наш насущный дай нам на сей день,
и оставь нам долг наш, как и мы оставляем должникам нашим,
и не введи нас в искушение,
но избавь нас от лукавого,
потому что твоя есть сила и слава во веки.
\vs Did 8:3
Трижды в день молитесь так.

\vs Did 9:1
Что же касается евхаристии, совершайте ее так.
\vs Did 9:2
Сперва о чаше:
Благодарим тебя, Отче наш, за святой виноград Давида,
отрока твоего, который виноград ты открыл нам чрез Иисуса, отрока твоего.
Тебе слава во веки!
\vs Did 9:3
О хлебе же ломимом: Благодарим Тебя, Отче наш, за жизнь и знание,
которые ты открыл нам чрез Иисуса, отрока твоего.
Тебе слава во веки.
\vs Did 9:4
Как сей преломляемый хлеб был рассеян по холмам и собранный
вместе стал единым, так и экклесия твоя от концов земли
да соберется в Царствие твоё, ибо твоя есть слава и сила
чрез Иисуса Христа во веки.
\vs Did 9:5
И от евхаристии вашей никто да не вкушает и не пьет, кроме
крещенных во имя Господне, ибо и о сем сказал Господь:
не давайте святыни псам.

\vs Did 10:1
По исполнении же вкушения так благодарите:
\vs Did 10:2
Благодарим тебя, Отче святый, за имя твоё святое,
которое ты вселил в сердцах наших,
и за знание, и веру, и бессмертие,
которые ты открыл нам чрез Иисуса, отрока твоего.
Тебе слава во веки!
\vs Did 10:3
Ты, Владыко Вседержитель, сотворил всё ради имени твоего,
пищу же и питие дал людям в наслаждение,
чтобы они благодарили тебя, а нам даровал духовную пищу и питие,
и жизнь вечную чрез отрока твоего.
\vs Did 10:4
Более всего благодарим тебя потому, что ты всемогущ.
Тебе слава во веки!
\vs Did 10:5
Помяни, Господи, экклесию твою, да избавишь её от всякого зла
и усовершишь её в любви твоей, и от четырёх ветров собери её,
освящённую, в царство твоё, которое ты уготовал ей,
потому что Твоя есть сила и слава во веки.
\vs Did 10:6
Да приидет благодать и да прейдёт мир сей.
Осанна Богу Давидову!
Кто свят, да приступает, кто нет, пусть покается.
Маранат. Аминь.
\vs Did 10:7
Пророкам же позволяйте совершать евхаристию когда захотят.

\vs Did 11:1
Кто, пришедши, будет учить вас всему этому,
пред сим сказанному, примите его.
\vs Did 11:2
Если же сам учащий, совратившись, будет преподавать иное
учение, к ниспровержению, не слушайте его; но если для
преумножения правды и познания Господа, примите его, как Господа.
\vs Did 11:3
Относительно же апостолов и пророков поступайте
согласно учению евангельскому.
\vs Did 11:4
Всякий апостол, приходящий к вам, пусть будет принят, как Господь.
\vs Did 11:5
Но пусть он не остаётся более одного дня, а если будет нужда,
то и другой \bibemph{день}, но если он пробудет три дня, то лжепророк.
\vs Did 11:6
Уходя же, апостол пусть ничего не принимает, кроме хлеба,
до \bibemph{следуюшего} места ночлега;
а если он будет требовать серебра, то он лжепророк.
\vs Did 11:7
И всякого пророка, говорящего в духе, не испытывайте и не
судите, ибо всякий грех отпустится,
а этот грех не отпустится.
\vs Did 11:8
Но не всякий, говорящий в духе,~--- пророк,
но \bibemph{только} тот, кто хранит пути Господни;
так что по путям распознаётся и лжепророк и пророк.
\vs Did 11:9
И никакой пророк, в духе определяющий быть трапезе,
не вкушает от неё, а если не так, то он лжепророк.
\vs Did 11:10
Всякий пророк, учащий истине, если он сам не делает того, чему
учит, есть лжепророк.
\vs Did 11:11
Но всякий пророк, признанный истинным, вступающий в мирское
таинство экклесии, но не учащий делать то, что сам делает,
не должен быть судим вами, ибо он суд имеет у Бога,
ибо так поступали и древние пророки.
\vs Did 11:12
Если же кто в духе скажет: дай мне серебра или чего другого,
вы не должны слушать того; но если он назначит подаяние для других, неимущих,
то никто да не осуждает его.

\vs Did 12:1
Всякий, приходящий во имя Господне, да будет принят, а потом,
уже испытав его,
вы узнаете,~--- ибо вы будете иметь разумение,~--- правого и ложного.
\vs Did 12:2
Если приходящий~--- странник, помогите ему, сколько можете,
но он не должен оставаться у вас более двух или трёх дней,
и то если бы нужда оказалась.
\vs Did 12:3
Если же он желает поселиться у вас, то, если он ремесленник,
пусть трудится и ест.
\vs Did 12:4
А если он не знает ремесла, то вы по своему усмотрению
позаботьтесь о нём, но так, чтобы христианин не жил среди вас праздным.
\vs Did 12:5
Если же он не желает так поступать, то он христоторговец.
Остерегайтесь таковых!

\vs Did 13:1
А всякий истинный пророк, желающий поселиться у вас,
достоин своего пропитания.
\vs Did 13:2
Точно так же и истинный учитель, и он достоин, как трудящийся,
своего пропитания.
\vs Did 13:3
Поэтому всякий начаток от произведений точила и гумна, от волов
и овец возьми и отдай пророкам, ибо они ваши архиереи.
\vs Did 13:4
Если же вы не имеете пророка, то отдайте начаток нищим.
\vs Did 13:5
Если ты приготовишь пищу, то, взявши начаток, отдай его по
заповеди.
\vs Did 13:6
Точно так же если ты открыл сосуд вина или елея, то возьми
начаток и отдай пророкам.
\vs Did 13:7
И от серебра, и от одежды, и от всякого имения возьми начаток,
сколько тебе угодно, и отдай его по заповеди.

\vs Did 14:1
Каждый день\fnote{Каждый день}{\vsep\
В день Господень, \bibemph{т.е. в субботу}.},
собравшись, преломите хлеб и благодарите,
исповедав прежде согрешения ваши дабы жертва ваша была чиста.
\vs Did 14:2
Всякий же, имеющий распрю с другом своим, да не приходит вместе
с вами, пока они не примирятся, чтобы не осквернилась жертва ваша.
\vs Did 14:3
Ибо о ней сказал Господь: на всяком месте и во всякое время
должно приносить мне жертву чистую, потому что я царь великий,
говорит Господь, и имя мое чудно в народах.

\vs Did 15:1
Рукополагайте себе старейшин и прислужников, достойных Господа,
мужей кротких и несребролюбивых, и истинных, и испытанных,
ибо и они исполняют для вас служение пророков и учителей.
\vs Did 15:2
Поэтому не презирайте их, ибо они~--- почтенные ваши
наравне с пророками и учителями.

\vs Did 15:3
Обличайте друг друга, но не во гневе, а в мире, как имеете в
евангелии, и со всяким, поступающим оскорбительно по отношению к другому,
пусть никто не говорит и никто у вас не слушает его, пока не покается.
\vs Did 15:4
Молитва же ваша и милостыня, и все вообще добрые дела творите
так, как имеете в евангелии Господа нашего.

\vs Did 16:1
Бодрствуйте относительно жизни вашей;
светильники ваши да не будут погашены,
и чресла ваши не препоясаны,
но будьте готовыми, ибо вы не знаете часа,
в который Господь ваш приидет.
\vs Did 16:2
Вы должны часто собираться вместе, исследуя, что потребно душам
вашим, ибо не принесёт вам пользы всё время вашей веры,
если не сделаетесь совершенными в последний час.
\vs Did 16:3
Ибо в последние дни умножатся лжепророки и губители, и овцы
превратятся в волков, и любовь превратится в ненависть.
\vs Did 16:4
Ибо, когда возрастёт беззаконие, люди будут ненавидеть друг
друга и преследовать, и тогда явится мирообольститель,
как бы Сын Божий, и совершит знамения и чудеса, и земля
предана будет в руки его, и сотворит беззакония,
каких никогда не было от века.
\vs Did 16:5
Тогда тварь человеческая пойдет в огонь испытания и многие
соблазнятся и погибнут, а устоявшие в вере своей спасутся
от проклятия его\fnote{от проклятия его}{\vsep\ этим самим проклятием.}.
\vs Did 16:6
И тогда явится знамение истины:
во-первых, знамение отверстия на небе,
потом знамение звука трубного
и третье~--- воскресение мертвых.
\vs Did 16:7
Но не всех вместе, а как сказано: приидет Господь и все святые с ним.
\vs Did 16:8
Тогда увидит мир Господа, грядущего на облаках небесных.

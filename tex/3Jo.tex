\bibbookdescr{3Jo}{
  inline={Третье Соборное Послание\\\LARGE Святого Апостола Иоанна Богослова},
  toc={3-е Иоанна},
  bookmark={3-е Иоанна},
  header={3-е Иоанна},
  %headerleft={},
  %headerright={},
  abbr={3~Ин}
}
\vs 3Jo 1:1 Старец~--- возлюбленному Гаию, которого я люблю по истине.
\rsbpar\vs 3Jo 1:2 Возлюбленный! молюсь, чтобы ты здравствовал и преуспевал во всем, как преуспевает душа твоя.
\vs 3Jo 1:3 Ибо я весьма обрадовался, когда пришли братия и засвидетельствовали о твоей верности, как ты ходишь в истине.
\vs 3Jo 1:4 Для меня нет б\acc{о}льшей радости, как слышать, что дети мои ходят в истине.
\rsbpar\vs 3Jo 1:5 Возлюбленный! ты как верный поступаешь в том, что делаешь для братьев и для странников.
\vs 3Jo 1:6 Они засвидетельствовали перед церковью о твоей любви. Ты хорошо поступишь, если отпустишь их, как должно ради Бога,
\vs 3Jo 1:7 ибо они ради имени Его пошли, не взяв ничего от язычников.
\vs 3Jo 1:8 Итак мы должны принимать таковых, чтобы сделаться споспешниками истине.
\rsbpar\vs 3Jo 1:9 Я писал церкви; но любящий первенствовать у них Диотреф не принимает нас.
\vs 3Jo 1:10 Посему, если я приду, то напомню о делах, которые он делает, понося нас злыми словами, и не довольствуясь тем, и сам не принимает братьев, и запрещает желающим, и изгоняет из церкви.
\vs 3Jo 1:11 Возлюбленный! не подражай злу, но добру. Кто делает добро, тот от Бога; а делающий зло не видел Бога.
\vs 3Jo 1:12 О Димитрии засвидетельствовано всеми и самою истиною; свидетельствуем также и мы, и вы знаете, что свидетельство наше истинно.
\rsbpar\vs 3Jo 1:13 Многое имел я писать; но не хочу писать к тебе чернилами и тростью,
\vs 3Jo 1:14 а надеюсь скоро увидеть тебя и поговорить устами к устам.
\vs 3Jo 1:15 Мир тебе. Приветствуют тебя друзья; приветствуй друзей поименно. Аминь.

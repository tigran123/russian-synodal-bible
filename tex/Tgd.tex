\bibbookdescr{Tgd}{
  inline={Завещание Гада,\\девятого сына Иакова и Зелфы},
  toc={Завещание Гада},
  bookmark={Завещание Гада},
  header={Завещание Гада},
  abbr={Гад}
}
\vs Tgd 1:1
Список завещания Гада,
речённого им к сыновьям своим в 125-ый год жизни его.
Сказал он им:
\vs Tgd 1:2
послушайте, дети мои:
родился я 9-ым сыном Иакова и смелым был на пастбищах.
\vs Tgd 1:3
Стерёг я по ночам стадо, и когда приходил лев,
или волк, или другой зверь на пастбище,
преследовал я его, и настигал, и ловил за ногу его рукою моей,
и бросал его словно камень, и убивал его.
\vs Tgd 1:4
Иосиф же, брат мой, пас стадо вместе с нами около 30-ти дней и,
будучи молод, занемог от зноя.
\vs Tgd 1:5
И возвратился он в Хеврон, к отцу нашему;
и положил его тот рядом с собой,
ибо весьма любил его.

\vs Tgd 1:6
И сказал Иосиф отцу нашему:
сыновья Зелфы и Баллы приносят жертвы
из добрых животных и поедают их, а Рувим и Иуда того не ведают.
\vs Tgd 1:7
Видел же он, как вытащил я барана из пасти медведицы,
и её убил, а барана принес в жертву:
горевали мы, что не может он жить, и съели его.
\vs Tgd 1:8
И оттого гневался я на Иосифа вплоть до дня, когда был он продан.
\vs Tgd 1:9
И был во мне дух ненависти, и не желал я ни слышать об Иосифе,
ни видеть его, ибо в лицо укорял он нас,
говоря, что без Иуды едим мы животных.
Ибо всему, что говорил он, верил отец.

\vs Tgd 2:1
Исповедуюсь ныне в грехе моём, дети,
ибо многократно желал я убить Иосифа, возненавидев его душою.
\vs Tgd 2:2
И за сон его обратил я на него ненависть,
и хотел его извести с земли (живого),
как телец изводит траву на поле.
\vs Tgd 2:3
Но Иуда и я продали его Измаильтянам за 30 золотых, спрятав 10 и показав
только 20 своим братьям.
\vs Tgd 2:4
И так из-за жадности наш замысел был приведён в исполнение.
\vs Tgd 2:5
Так Бог отцов наших избавил его от рук моих,
дабы не сотворил я беззакония великого в Израиле.

\vs Tgd 3:1
А ныне услышьте слово правды:
творите справедливость,
и делайте всё по закону Всевышнего, 
и не соблазняйтесь духом ненависти,
ибо зло это во всех делах человеческих.
\vs Tgd 3:2
Всё, что творит человек, мерзостно для ненавидящего:
если делает по закону Господа, не хвалит его,
если боится Господа и желает справедливости, не любит его.
\vs Tgd 3:3
Правду порицает, счастливому завидует,
злословию радуется, гордыню любит,
ибо ненависть ослепляет душу его, как и меня,
когда смотрел я на Иосифа.

\vs Tgd 4:1
Берегитесь же ненависти, дети мои,
ибо ненавидящий и против Господа беззаконие творит.
\vs Tgd 4:2
Ибо не хочет он слышать заповедей его о любви к ближнему,
и тем грешит против Бога.
\vs Tgd 4:3
Если падёт брат его,
стремится сразу возвестить всем и желает,
чтобы осуждённый и покаранный умер он.
\vs Tgd 4:4
Если же раб какой, клевещет на него перед господином его,
и радуется, если в мучениях умрёт он.
\vs Tgd 4:5
Ибо зависти содействует ненависть также и против счастливых:
видящий успех чей-то или слышащий о нём всегда изнемогает.
\vs Tgd 4:6
Как любовь мёртвых желает оживить и на смерть обречённых
воззвать к жизни хочет, так ненависть живых желает убить
и не хочет, чтобы лишь немного согрешившие живы были.
\vs Tgd 4:7
Ибо дух ненависти через малодушие содействует Сатане
во всём на погибель людей, а дух любви через
долготерпение закону Божию содействует во спасение людей.

\vs Tgd 5:1
И потому ненависть~--- зло,
что постоянно содействует она лжи, говоря против правды,
и малое великим делает, и свет тьмою представляет,
и о сладком говорит, что оно горько,
и клевете научает, и гнев возбуждает,
и войну поднимает, и гордыню, и всякую алчность,
а сердц\acc{а} злом и ядом дьявольским наполняет.
\vs Tgd 5:2
По опыту своему говорю вам это, дети мои, дабы изгнали вы ненависть
дьявольскую и Бога возлюбили.
\vs Tgd 5:3
Праведность изгоняет ненависть,
смирение убивает зависть,
ибо праведный и смиренный стыдится творить неправедное,
и не от того, что другой осудит его, а своё же сердце,
ибо видит Господь душу его.
\vs Tgd 5:4
Не станет говорить он против человека благочестивого,
ибо страх Божий живёт в нём.
\vs Tgd 5:5
Ибо страшась оскорбить Господа, вовек не пожелает
он обидеть и человека, даже и в мыслях своих.
\vs Tgd 5:6
Узнал в конце о том и я, когда раскаялся об Иосифе.
\vs Tgd 5:7
Ибо истинное обращение к Богу [убивает незнание и]
прогоняет тьму, и освещает очи, и знание дает душе,
и помыслы ведёт ко спасению.
\vs Tgd 5:8
И не от людей научился я этому, а в покаянии познал.
\vs Tgd 5:9
Навёл же на меня Бог болезнь печени,
и если бы не помогли мольбы отца моего,
испустил бы я, верно, дух мой.
\vs Tgd 5:10
Ибо чем человек грешит, тем он и карается.
\vs Tgd 5:11
Оттого, что была печень моя безжалостна к Иосифу,
был я осужден на страдание печени немилосердное
в течение 11 месяцев, по времени, что гневался я на Иосифа.

\vs Tgd 6:1
И ныне, дети мои, даю вам совет:
любите каждый ближнего своего, прогоняйте ненависть из сердец ваших.
Возлюбите друг друга делом, словом и помыслом душевным.
\vs Tgd 6:2
Ибо я пред лицом отца моего мирно говорил с Иосифом;
когда же вышел от него, дух ненависти помрачил мой разум
и смутил рассуждение мое, так что захотел я убить Иосифа.
\vs Tgd 6:3
Возлюбите друг друга от сердца, и если кто согрешит против тебя,
говори ему: мир тебе; и не затаи коварства в душе своей.
Если же, раскаявшись, признает он вину свою, отпусти ему.
\vs Tgd 6:4
Если же станет отрицать он, не вступай в спор с ним,
дабы не согрешить дважды, когда он начнёт ругаться.
\vs Tgd 6:5
Да не услышит во время тяжбы чужой человек тайн твоих,
дабы не возненавидел он тебя и не сделался врагом тебе,
и великий грех не сотворил тебе, ибо часто будет
он замышлять коварство и зло творить, вникая в дела твои.
\vs Tgd 6:6
Если же будет он отрицать и, уличённый во грехе,
устыдится, успокойся и не обличай его;
ибо раскается он, что согрешил против тебя,
и, устрашившись, пожелает жить в мире с тобой.
\vs Tgd 6:7
Если же нет в нём стыда и упорствует он во зле,
и тогда отпусти ему от сердца, а возмездие оставь Богу.

\vs Tgd 7:1
И если кто-либо счастливее вас, не огорчайтесь,
но молитесь за него, дабы и в конце был он счастлив,
ибо это полезно вам будет.
\vs Tgd 7:2
И если и далее он возвышается, не завидуйте ему,
помня, что всякая плоть умрет.
Хвалы же возносите Господу,
добро и счастье дающему людям.
\vs Tgd 7:3
Исследуйте суды Господа, и просветит он,
и успокоит помыслы ваши.
\vs Tgd 7:4
Если же кто злом богатеет, как Исав, брат отца моего,
не ревнуйте: ожидайте, что Господь положит предел.
\vs Tgd 7:5
Если отнимется злое богатство, и раскается человек,
простит Господь, а не раскается~--- предан будет на вечные муки.
\vs Tgd 7:6
А бедный, если без зависти радуется он всему,
что дает Господь, превыше всех богатеет,
ибо не ведает он суеты праздных людей.
\vs Tgd 7:7
Удалите же зависть от душ ваших и возлюбите друг друга в прямоте сердца.

\vs Tgd 8:1
Скажите это и вы детям вашим, дабы чтили они Левия и Иуду,
ибо от них восставит Господь спасение Израилю.
\vs Tgd 8:2
Ибо познал я, что отступятся дети ваши от них,
и во всяком зле, вреде и порче будут пред Богом.

\vs Tgd 8:3
И отдохнув немного, сказал ещё:
дети мои, послушайте отца вашего,
и похороните меня рядом с отцами моими.
\vs Tgd 8:4
И вытянув ноги, почил в мире.
\vs Tgd 8:5
И спустя 5 лет отнесли его в Хеврон и положили рядом с отцами его.

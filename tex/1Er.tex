\bibbookdescr{1Er}{
  inline={Пастырь Ермы. Книга 1. Видения},
  toc={1-я Ермы},
  bookmark={1-я Ермы},
  header={1-я Ермы},
  abbr={1~Ермы}
}
\chhdr{Видение 1-е.}
\vs 1Er 1:1
Воспитатель мой продал в Риме одну отроковицу.
По прошествии многих лет я увидел её, узнал и полюбил как сестру.
\vs 1Er 1:2
Через некоторое время, увидев, что она купается в реке Тибр,
я подал ей руку и вывел из реки.
\vs 1Er 1:3
Глядя на ее красоту, я думал:
<<Счастлив бы я был, если бы имел жену такую же и лицом и нравом.>>
Только это, и ничего более я не подумал.
\vs 1Er 1:4
Позже шёл я с такими мыслями и прославлял творение Божье,
раздумывая, сколь величественно оно и прекрасно.
\vs 1Er 1:5
Во время прогулки я заснул, и дух подхватил меня
и понёс куда-то, через местность,
по которой человек не мог пройти.
Была она скалиста, крута и непроходима из-за вод.
\vs 1Er 1:6
Миновав её, я достиг равнины и, преклонив колена,
начал молиться Господу и исповедовать грехи свои.
\vs 1Er 1:7
И во время моей молитвы отверзлось небо
и увидел я ту женщину, которую пожелал себе.
\vs 1Er 1:8
Она приветствовала меня с неба:
<<Здравствуй, Ерма.>>
\vs 1Er 1:9
Взглянув на неё, я спросил:
<<Госпожа, что ты здесь делаешь?>>
\vs 1Er 1:10
Я взята сюда, чтобы обличить пред Господом грехи твои,~--- она ответила.
\vs 1Er 1:11
Госпожа, ужели ты меня будешь обвинять?
\vs 1Er 1:12
Нет, но выслушай слова, которые хочу сказать тебе.
Бог, живущий на небесах, сотворивший из ничего всё
сущее и умноживший ради святой Церкви своей,
гневается на тебя за то, что ты согрешил против меня.
\vs 1Er 1:13
Госпожа, если я согрешил против тебя,
то каким образом?~--- спросил я.~--- Где или когда я сказал тебе
какое-нибудь дурное слово?
\vs 1Er 1:14
Не всегда ли я уважал тебя как госпожу;
не всегда ли я почитал тебя как сестру?
Что же наговариваешь на меня столь дурное?
\vs 1Er 1:15
Тогда она, улыбаясь, ответила мне:
<<В сердце твоём возникло нечистое пожелание.
Ужели не думаешь, что для человека праведного и то порочно,
если в сердце его возникает худое пожелание?
Это~--- грех для него, и притом тяжкий.
\vs 1Er 1:16
Ибо человек праведный и помышляет праведное.
И когда он помышляет праведное и неуклонно к тому стремится,
то имеет на небесах благоволение Господа во всяком деле.
\vs 1Er 1:17
Те же, которые затаили нечистое в сердцах своих,
навлекают на себя смерть и тлен; особенно те,
которые любят настоящий век, роскошествуют
в богатстве своём и не ожидают благ будущих,~--- гибнут души их.
\vs 1Er 1:18
А это делают двоедушные, которые не имеют надежды
в Господе, не радеют о своей жизни.
\vs 1Er 1:19
Но ты молись Господу, и исцелит он грехи твои,
и всего дома твоего, и всех святых.>>

\vs 1Er 2:1
После того как произнесла она эти слова, небеса заключились.
\vs 1Er 2:2
И я, весь в скорби и страхе, сказал себе:
<<Если это вменяется мне в грех, то как могу спастись или
каким образом умолю Господа о бесчисленных грехах моих?
Какими словами упрошу Господа быть ко мне милостивым?>>
\vs 1Er 2:3
Размышляя так, увидел я вдруг перед собой большую кафедру,
словно сотворённую из в\acc{о}лны, белой как снег.
\vs 1Er 2:4
И пришла старая женщина в блестящей одежде с книгою в руке,
села одна и приветствовала меня:
<<Здравствуй, Ерма.>>
\vs 1Er 2:5
И я, в печали и слезах, ответил:
<<Здравствуй, госпожа.>>
\vs 1Er 2:6
Она спросила:
<<Что печален, Ерма, ты, который был терпелив, умерен и всегда весел?>>
\vs 1Er 2:7
Госпожа, одна прекрасная женщина, укорила меня,
будто я согрешил против неё,~--- ответил я.
\vs 1Er 2:8
И она сказала мне:
<<В сердце твоё м возникло вожделение к ней.
Это должно быть чуждо рабу Господню,
ведь для рабов Божьих даже и такой помысел составляет грех.
\vs 1Er 2:9
И сердце чистое не должно желать дурного~--- особенно твоё, Ерма;
ты избегаешь всякого преступного пожелания
и исполнен простоты и великого незлобия.

\vs 1Er 3:1
Впрочем, не ради тебя гневается на тебя Господь,
но за дом твой, который впал в нечестие перед
Господом и своими родителями.
\vs 1Er 3:2
И ты, любя детей, не вразумлял своего семейства,
но позволил им сильно развратиться.
\vs 1Er 3:3
За это и гневается на тебя Господь,
но он исправит всё, что прежде сделано худого в доме твоём.
\vs 1Er 3:4
За их грехи и беззакония ты подавлен мирскими делами.
\vs 1Er 3:5
Но милосердие Божье сжалилось над тобою и семейством твоим
и сохранило тебя в славе.
\vs 1Er 3:6
Ты только не колеблись, но будь благодушен и укрепляй свое семейство.
\vs 1Er 3:7
Как кузнец, усердно работая молотом,
совершает свой труд, так и праведное слово
ежедневное победит всякое зло.
\vs 1Er 3:8
Поэтому не переставай вразумлять детей своих,
ибо Господь знает, что они покаются от всего сердца
своего и будут написаны в Книге жизни.>>
\vs 1Er 3:9
Сказав это, она спросила меня:
<<Хочешь послушать, что я буду читать?>>
\vs 1Er 3:10
Хочу, госпожа,~--- ответил я.
\vs 1Er 3:11
Итак, слушай.
И, раскрыв книгу, она читала величественные и дивные слова,
которых не мог я удержать в памяти, ибо были они страшны,
человек не мог вынести их.
\vs 1Er 3:12
Впрочем, самые последние слова я запомнил,
так как были они краткими и отрадными для нас:
\vs 1Er 3:13
<<Вот Бог Саваоф, который невидимою силою
и великим своим разумом сотворил мир,
и славным светом своим благоукрасил тварь,
\vs 1Er 3:14
и всесильным словом своим утвердил небо,
и землю основал на водах, и всемощной силой своею создал свою
святую Церковь, которую и благословил.
\vs 1Er 3:15
Вот, он изменит небеса и горы, холмы и моря,
и всё уравняется для избранных его,
\vs 1Er 3:16
чтобы исполнить обещание, которое он дал,
с великою славою и торжеством, если они соблюдут заповеди
Божьи, полученные ими с великою верою.>>

\vs 1Er 4:1
Окончив чтение, она встала с кафедры;
и пришли четверо юношей и понесли кафедру на восток.
\vs 1Er 4:2
А она подозвала меня к себе и, коснувшись груди моей, спросила:
<<Понравилось ли тебе мое чтение?>>
\vs 1Er 4:3
Госпожа, самое последнее мне нравится,
но предыдущее страшно и жестоко.
\vs 1Er 4:4
И она сказала:
<<Эти последние слова относятся к праведным,
а первые~--- к отступникам и народам.>>
\vs 1Er 4:5
В это время явились 2 каких-то мужа,
подняли её на плечи и отправились вслед за кафедрой, на восток.
\vs 1Er 4:6
Она удалилась весёлая и на прощание произнесла:
<<Мужайся, Ерма!>>

\chhdr{Видение 2-е.}
\vs 1Er 5:1
Гуляя в окрестностях Кумских в то же примерно время,
что и в прошлом году, вспомнил я о прежнем видении,
и снова вознёс меня дух туда же, где прежде.
\vs 1Er 5:2
Достигнув того места,
я преклонил колена и начал молиться Господу
и прославлять имя его за то, что он удостоил меня
и открыл мне прежние грехи мои.
\vs 1Er 5:3
И когда восстал я от молитвы,
увидел пред собою ту старицу,
которую видел прежде: она гуляла и читала какую-то книгу.
\vs 1Er 5:4
Можешь ли возвестить это избранникам Божьим?~--- спросила она меня.
\vs 1Er 5:5
Я ответил:
<<Госпожа, так много я не могу запомнить, но дай мне книгу; я перепишу.>>
\vs 1Er 5:6
Возьми,~--- сказала она,~--- а потом возврати её мне.
\vs 1Er 5:7
Взяв книгу, я удалился в поле и списал всё буква в букву,
не понимая смысла.
\vs 1Er 5:8
И когда окончил я списывание книги,
вдруг забрали её из рук моих, но кто это был~--- не увидел я.

\vs 1Er 6:1
Спустя 15 дней, в которые я постился
и много молился Господу открылся мне смысл написанного.
\vs 1Er 6:2
Написано было следующее:
<<Дети твои, Ерма, отступили от Господа, хулили его и в великом нечестии
предали своих родителей; и прослыли они предателями родителей;
\vs 1Er 6:3
предавши их, они не исправились,
но присоединили к грехам своим распутство и нечестие скверны и
таким образом исполнили неправды свои.
\vs 1Er 6:4
Объяви эти слова всем детям своим и жене своей,
так как и она не воздержана в речах своих и тем согрешает.
\vs 1Er 6:5
Услышав же эти слова, она обуздает свой язык
и заслужит помилование.
\vs 1Er 6:6
Она образумится после того, как передашь ей слова,
которые Господь повелел открыть тебе.
\vs 1Er 6:7
Тогда отпустятся грехи, совершённые прежде,
как им, так и всем святым, если от всего сердца покаются
они и удалят сомнения из сердец своих.
\vs 1Er 6:8
Ибо славою своею поклялся Господь,
что тот из избранных его, кто и в этот предопределённый день будет
продолжать грешить, не получит спасения.
\vs 1Er 6:9
Ибо покаянию праведных положены сроки,
и определены дни покаяния для всех святых,
но народам позволено каяться до самого последнего дня.
\vs 1Er 6:10
Поэтому скажи настоятелям Церкви,
чтобы они совершали пути свои в истине,
дабы могли получить обетования со многою славою.
\vs 1Er 6:11
И вы, праведники, стойте твердо и не будьте двоедушны,
чтобы переселение ваше было со святыми ангелами.
\vs 1Er 6:12
Блаженны те, кто претерпит наступающее великое гонение
и не отречётся от своей жизни,
\vs 1Er 6:13
ибо сыном своим поклялся Господь,
что отрекающиеся от Господа губят свою жизнь.
\vs 1Er 6:14
Это относится к тем, которые отрекутся в предстоящие дни;
\vs 1Er 6:15
к тем же, которые прежде отрекались,
по великому милосердию он сделался милостивым.

\vs 1Er 7:1
А ты, Ерма, не помни неправды детей своих
и не оставляй жены своей, но позаботься о том, чтобы они
освободились от прежних грехов.
\vs 1Er 7:2
Они образумятся правым
учением, если ты не будешь держать зла на них.
\vs 1Er 7:3
Ибо злопамятство приводит
к смерти, забвение зла~--- к жизни вечной.
\vs 1Er 7:4
А ты, Ерма, потерпел большие мирские бедствия
за преступления дома твоего, поскольку не обращал на
них внимания как на не касающиеся тебя нисколько
и предался неправедным своим занятиям.
\vs 1Er 7:5
Но то, что не отступил ты от живого Бога, спасёт тебя;
простота твоя и великое воздержание спасут тебя,
если ты пребудешь в них; и всех спасут они,
кто поступает так же.
\vs 1Er 7:6
Пребывающие в невинности и простоте будут
сильны против всякого зла и обретут жизнь вечную.
\vs 1Er 7:7
Блаженны все делающие правду: они не погибнут вовек.
\vs 1Er 7:8
Но скажешь: вот приходит великое гонение.
Если тебе кажется, то опять отрекись.
\vs 1Er 7:9
Господь близок к обращающимся,
как написали в книгах Елдада и Модада,
которые в пустыне пророчествовали народу.>>

\vs 1Er 8:1
Во время сна моего,
братия, один красивый юноша явился мне и спросил:
<<Кто, ты думаешь, та старица, от которой получил ты книгу?>>
\vs 1Er 8:2
Сивилла,~--- ответил я.
\vs 1Er 8:3
Ошибаешься,~--- сказал он,~--- она не сивилла.
\vs 1Er 8:4
Кто же она, господин?
\vs 1Er 8:5
Она есть Церковь Божья.
\vs 1Er 8:6
Я спросил его, почему же она стара.
\vs 1Er 8:7
Так как,~--- объяснил он,~--- сотворена она прежде всего,
и для неё сотворён мир.
\vs 1Er 8:8
После того было мне видение в доме моём,
и пришла та старица и спросила меня,
отдал ли я уже книгу предстоятелям Церкви.
\vs 1Er 8:9
Я отвечал, что нет ещё, и она сказала:
<<Хорошо, потому что я добавлю ещё несколько слов.
\vs 1Er 8:10
Когда же исчерпаю все слова,
тогда пусть через тебя они дойдут до избранных.
\vs 1Er 8:11
Для этого ты напишешь 2 книги
и одну отдашь Клименту; а другую~--- Гранте.>>
\vs 1Er 8:12
Климент отошлёт во внешние города, ибо ему это предоставлено;
Гранта же будет назидать вдов и сирот.
\vs 1Er 8:13
А ты прочтёшь её в этом городе вместе с пресвитерами,
предстоятелями Церкви.

\chhdr{Видение 3-е.}
\vs 1Er 9:1
Было мне, братья, следующее видение.
После того как я много раз постился и молил
Господа об откровении, которое было обещано мне чрез ту старицу,
\vs 1Er 9:2
ночью явилась старица и сказала:
<<Так как ты очень просишь и желаешь знать всё,
то приходи в поле и около 6-и часов я явлюсь тебе и покажу то,
что нужно тебе видеть.>>
\vs 1Er 9:3
Я спросил её:
<<На каком месте поля?>>
\vs 1Er 9:4
Она говорит:
<<Где хочешь; место же выбери сам.>>
\vs 1Er 9:5
И я избрал место прекрасное, уединенное.
Но прежде, нежели начал я говорить и сказал ей о месте,
она говорит мне:
<<Приду, куда пожелаешь.>>
\vs 1Er 9:6
Итак, братья, заметил я часы и явился на поле,
к месту куда назначил ей прийти.
\vs 1Er 9:7
И вижу я поставленную скамью, на ней льняная подушка,
а над скамьей простёрта парусина.
\vs 1Er 9:8
Видя такие приготовления,
между тем как никого нет на месте,
я изумился, волосы у меня поднялись,
и ужас объял меня оттого, что я был один.
\vs 1Er 9:9
Но придя в себя и вспомнив славу Божью,
я ободрился и, преклонив колена,
исповедал Богу свои грехи, как всегда.
\vs 1Er 9:10
Вот, пришла старица с 6-ю юношами,
которых я прежде видел, и, ставши позади меня,
слушала, как я молился и исповедовался перед Богом.
\vs 1Er 9:11
Коснувшись меня, она сказала:
<<Перестань молиться только о грехах своих,
молись и о правде, чтобы часть из неё получил ты для дома своего.>>
\vs 1Er 9:12
Взяв меня за руку, она привела меня к скамейке
и велела тем юношам:
<<Идите и стройте.>>
\vs 1Er 9:13
Когда мы остались одни, она сказала мне:
<<Садись здесь.>>
\vs 1Er 9:14
Госпожа, пусть прежде сядут пресвитеры.
\vs 1Er 9:15
Я тебе говорю, настаивала она,~--- садись.
\vs 1Er 9:16
Я хотел было сесть по правую сторону,
но она рукою показала,
чтобы садился я по левую сторону.
\vs 1Er 9:17
Когда опечалился я, что не позволила сесть мне
по правую сторону, она проговорила:
<<Не печалься, Ерма.
Место по правую сторону принадлежит тому
кто уже угодил Богу и пострадал за имя его.>>
\vs 1Er 9:18
У тебя много недостает для того, чтобы сидеть с ними.
Но оставайся в простоте своей, как прежде, и будешь сидеть с ними,
\vs 1Er 9:19
равно как и все, кто будет творить дела их и претерпит то,
что они претерпели.>>

\vs 1Er 10:1
Я сказал ей:
<<Госпожа, я желал бы узнать, что они претерпели.>>
\vs 1Er 10:2
Слушай:
<<Лютых зверей, бичевание, темницы, кресты ради имени его.
За это принадлежит правая сторона святыни им и всякому,
кто пострадает за имя Божье, а остальным~--- левая сторона.
\vs 1Er 10:3
Но для тех и других, и для сидящих по правую сторону
и для сидящих по левую,~--- одни и те же дары обетования;
только сидящие по правую сторону имеют некоторую честь.
\vs 1Er 10:4
Ты желаешь сидеть по правую сторону с ними,
но у тебя много слабостей.
Очисти себя от своих слабостей,
и все недвоедушные должны очиститься к тому дню от своих слабостей.>>
\vs 1Er 10:5
Сказав это, она хотела удалиться, но я бросился к ногам её
и умолял её Господом, чтобы явила мне обещанное видение.
\vs 1Er 10:6
И она опять взяла меня за руку,
подняла и посадила на скамейку по левую сторону и,
поднимая какой-то блестящий жезл, спросила:
<<Видишь ли большую работу?>>
\vs 1Er 10:7
Госпожа, ничего не вижу.
\vs 1Er 10:8
Неужели не видишь против себя великой башни,
которая на водах строится из блестящих квадратных камней?
\vs 1Er 10:9
Действительно, строилась квадратная башня теми 6-ю юношами,
которые пришли с нею.
Многие тысячи других мужей приносили камни.
\vs 1Er 10:10
Некоторые доставали камни со дна,
другие из земли и подавали тем 6-и юношам,
они же принимали их и строили.
\vs 1Er 10:11
Камни, извлечённые со дна, сразу клали в здание,
потому что они были гладкие и ровные и так примыкали один к другому,
что соединения их нельзя было заметить,
и башня казалась возведенной из одного камня.
\vs 1Er 10:12
Камни же, принесённые из земли,
не все использовались для строительства.
Некоторые из них строители откладывали,
потому что были они шероховаты,
или с трещинами, или светлы и круглы
и не годились для здания башни.
\vs 1Er 10:13
А некоторые камни они раскалывали и отбрасывали далеко в сторону.
И отброшенные камни, видел я, падали на дорогу и,
не оставаясь на ней, скатывались:
\vs 1Er 10:14
одни~--- в место пустынное,
другие попадали в огонь и горели,
иные падали близ воды и не могли скатиться в воду;
хотя и стремились попасть в неё.

\vs 1Er 11:1
Показав мне это, старица хотела удалиться, но я сказал:
<<Госпожа, какая польза мне видеть, но не понимать,
что значит это строение?>>
\vs 1Er 11:2
Она отвечала мне:
<<Любопытный ты человек, если желаешь понять значение башни.>>
\vs 1Er 11:3
Действительно, госпожа,
говорю я,~--- желаю узнать и возвестить братьям, чтобы и они возрадовались,
услышав это, и прославили Господа.
\vs 1Er 11:4
Услышат многие.
И, услышавши, некоторые возрадуются, а другие восплачут;
\vs 1Er 11:5
впрочем, и последние, если, услышавши, принесут покаяние,
также будут радоваться.
\vs 1Er 11:6
Выслушай теперь объяснение башни, я открою всё,
и не докучай мне более об откровении.
\vs 1Er 11:7
Откровения эти закончились, ибо имеют предел.
А ты не перестаёшь требовать откровений, потому что настойчив.
\vs 1Er 11:8
Итак, башня, которую видишь строящейся,~--- это я, Церковь,
которая явилась тебе теперь и прежде.
\vs 1Er 11:9
Спрашивай же что хочешь о башне, и я открою тебе,
чтобы возрадовался ты со святыми.
\vs 1Er 11:10
Госпожа, если однажды сочла ты меня достойным того,
чтобы всё открыть мне, то открой,~--- просил я старицу.
\vs 1Er 11:11
Всё, что следует открыть тебе, откроется,
только бы сердце твоё было с Господом
и ты не сомневался, что бы ни увидел.
\vs 1Er 11:12
Госпожа,~--- спросил я её, почему башня построена на водах?
\vs 1Er 11:13
И прежде я говорила тебе, отвечала она,~--- что ты любопытен
и усердно изыскиваешь; ища~--- найдёшь истину:
\vs 1Er 11:14
слушай же, почему башня строится на водах:
жизнь ваша через воду спасена и спасётся.
А башня основана словом всемогущего и преславного имени
и держится невидимою силою Господа.

\vs 1Er 12:1
Я на это сказал ей:
<<Величественное и дивное дело!
А кто, госпожа, те 6 юношей, которые строят?>>
\vs 1Er 12:2
Это~--- первозданные ангелы Божии,
которым Господь вверил всё своё творение для того,
чтобы они умножали, благоустраивали и управляли его творением:
их силами и будет окончено строительство башни.>>
\vs 1Er 12:3
А кто те остальные, которые приносят камни?
\vs 1Er 12:4
И это~--- святые ангелы Господа, но первые выше.
Когда окончится строительство башни, они все вместе
будут ликовать около башни и прославлять Господа за то,
что совершилось строительство башни.
\vs 1Er 12:5
Желал бы я знать,~--- сказал я,~--- какое значение
и в чём различие камней.
\vs 1Er 12:6
И она отвечала мне:
<<Разве ты лучше всех, чтобы тебе это было открыто?
Есть более достойные, которым следовало бы открыть эти видения.
\vs 1Er 12:7
Но, чтобы прославлялось имя Божье,
тебе это открыто и ещё откроется ради тех, кто имеет сомнение в
сердце своем, будет ли это или нет.
\vs 1Er 12:8
Скажи им, что всё это истинно и что ничего нет ложного,
но всё твердо и крепко основано.

\vs 1Er 13:1
Выслушай теперь и о камнях, на которых возведено здание.
\vs 1Er 13:2
Камни квадратные и белые,
хорошо приходящиеся один к другому своими соединениями,
это суть апостолы, епископы, учителя и дьяконы,
\vs 1Er 13:3
которые ходили в святом учении Божьем,
надзирали и свято и непорочно служили
избранникам Божьим как почившие, так и живущие еще доселе,
\vs 1Er 13:4
которые всегда пребывали в мире и согласии
и слушали взаимно друг друга: потому-то они и в здании башни
хорошо примыкают один к другому.
\vs 1Er 13:5
А камни, извлекаемые из
глубины и закладываемые в здание и соприкасающиеся с прочими камнями,
вошедшими в здание, это суть те,
которые уже умерли и пострадали за имя Господа.>>
\vs 1Er 13:6
Госпожа, я желаю знать, кого означают другие камни,
которые достали из земли.
\vs 1Er 13:7
Те, которые неотделанными
кладутся в основание, означают людей, которых Бог одобрил за то, что они жили
праведно пред Господом и исполняли его заповеди.
\vs 1Er 13:8
А которые приносятся и
кладутся в само здание башни, это суть новообращенные к вере и верные.
\vs 1Er 13:9
Ангелами призываются они к
совершению добра, и потому не нашлось в них зла.
\vs 1Er 13:10
А те камни, которые откладываются в сторону возле башни?
\vs 1Er 13:11
Она ответила:
<<Это те, которые согрешили и желают покаяться;
потому они брошены невдалеке от башни,
что будут пригодны, если покаются.
\vs 1Er 13:12
Посему желающие покаяться
будут тверды в вере, если только принесут покаяние теперь,
пока строится башня.
\vs 1Er 13:13
Ибо когда строительство окончится,
то им уже не найдётся места, и они, отверженные,
только останутся лежать при башне.

\vs 1Er 14:1
Желаешь знать, кто те,
которые раскалывают и отбрасывают далеко от башни?
\vs 1Er 14:2
Желаю, госпожа.
\vs 1Er 14:3
Это суть сыны беззакония,
которые уверовали притворно и от которых не отступила неправда всякого рода;
\vs 1Er 14:4
потому они не имеют спасения,
что не годны в здание по неправедности своей,~--- они расколоты и
отброшены далеко по гневу Господа за то, что оскорбили его.
\vs 1Er 14:5
А значение прошлых камней,
которые во множестве видел ты сложенными
и не использованными в строительстве, таково.
\vs 1Er 14:6
Шероховатые суть те,
которые познали истину; но не остались в ней и не находятся в общении со
святыми, потому они и не годны.
\vs 1Er 14:7
Камни с трещинами~--- это
суть те, которые держат в сердцах вражду друг к другу; будучи вместе, они
миролюбивы, но, разойдясь, обретают в сердцах злобу.
И эта злоба~--- трещины в камнях.
\vs 1Er 14:8
Камни меньшего размера
это те люди, которые, хоть и уверовали, но имеют еще много неправды, поэтому
они коротки.
\vs 1Er 14:9
Кто же, госпожа, белые и
круглые, что тоже не идут в здание башни?
\vs 1Er 14:10
Она отвечала мне:
<<Доколе ты будешь глуп и неразумен?
Ты обо всём спрашиваешь и ничего не понимаешь.>>
\vs 1Er 14:11
Белые и круглые камни
это те, которые имеют веру, но имеют и богатства века сего; и когда придёт
гонение, то ради богатств своих и попечений они отрекутся от Господа.
\vs 1Er 14:12
Когда же будут они угодны Господу?
\vs 1Er 14:13
Когда отсечены будут богатства их, которые их утешают,
тогда они будут полезны Господу для здания.
\vs 1Er 14:14
Ибо как круглый камень,
пока не будет обсечен и не лишится некоторых своих частей, не сможет стать
квадратным, так и богатые в нынешнем веке, если не лишатся своих богатств, не
смогут быть угодными Господу.
\vs 1Er 14:15
Прежде всего ты должен знать это по себе самому:
когда ты был богат, был бесполезен; а теперь ты
полезен и годен для жизни; ты и сам был из тех камней.

\vs 1Er 15:1
Прочие же камни, которые
ты видел, были отброшены далеко от башни, катились по дороге и с дороги
скатывались в места пустынные, означают тех, которые, хотя уверовали, но, по
сомнению своему, оставили истинный путь, думая, что они могут найти лучший.
\vs 1Er 15:2
Но они обольщаются и бедствуют, ходя по путям пустынным.
\vs 1Er 15:3
Камни, упавшие в огонь и
горевшие, означают тех, которые навсегда отказались от живого Бога и которым,
по причине преступных похотей, ими творимых, уже не приходит мысль покаяться.
\vs 1Er 15:4
Кого же означают камни,
которые падали близ воды и не могли скатиться в неё?
\vs 1Er 15:5
Тех, которые слышали Слово
и желают креститься во имя Господа, когда приходит им на память святость
истины, но потом они уклоняются и опять предаются своим порочным пожеланиям.
\vs 1Er 15:6
Так она окончила объяснение башни.
\vs 1Er 15:7
Но я, будучи настойчив, спросил её:
<<Есть ли покаяние для тех камней, которые отброшены,
и будет ли им место в этой башне?>>
\vs 1Er 15:8
Она сказала:
<<Есть для них покаяние; но в этой башне не найдут они места,
а попадут в иное, низшее место,
причем когда они пострадают и исполнятся дни грехов их.
\vs 1Er 15:9
И за то они будут
переведены, что приняли Слово истинное.
\vs 1Er 15:10
И тогда избавятся они от
наказаний своих, когда содрогнутся сердцем от порочных дел,
ими сотворенных, и они покаются.
\vs 1Er 15:11
Если же они не опомнятся,
то не спасутся из-за упорства своего сердца.>>

\vs 1Er 16:1
Когда я перестал спрашивать старицу обо всём этом,
она предложила:
<<Хочешь увидеть ещё что-то?>>
\vs 1Er 16:2
И так как я очень желал
увидеть, то радость отразилась на лице моём.
\vs 1Er 16:3
Взглянув на меня, она улыбнулась
и спросила:
<<Видишь 7 женщин вокруг башни?>>
\vs 1Er 16:4
Вижу, госпожа.
\vs 1Er 16:5
Башня эта по распоряжению Господа ими поддерживается.
\vs 1Er 16:6
Слушай теперь об их действиях.
Первая из них, которая держит башню руками, называется Верою;
посредством неё спасаются избранники Божьи.
\vs 1Er 16:7
Другая же, которая препоясана и ведет себя мужественно,
называется Воздержанием, она~--- дочь Веры.
\vs 1Er 16:8
Кто последует за нею, будет блажен в своей жизни,
ибо удержится от всех худых дел и всякой злой
похоти и станет наследником вечной жизни.
\vs 1Er 16:9
Кто же другие 5, госпожа?
\vs 1Er 16:10
Дочери одна другой.
Одна называется Простотою,
другая~--- Невинностью,
3-я~--- Скромностью,
4-я~--- Знанием,
5-я~--- Любовью.
\vs 1Er 16:11
Поэтому, когда исполнишь дела матери их,
тогда сможешь и всё соблюсти.
\vs 1Er 16:12
Хотел бы я знать, госпожа, какую каждая из них имеет силу?
\vs 1Er 16:13
Слушай,~--- отвечала она, силы их одинаковы:
они связаны между собою и следуют одна за другою,
как и рождены.
\vs 1Er 16:14
От Веры рождается Воздержание,
от Воздержания Простота,
от Простоты Невинность,
от Невинности Скромность,
от Скромности Знание,
от Знания Любовь.
\vs 1Er 16:15
Действия их чисты, целомудренны и святы,
и кто послужит им и будет в силе исполнять дела их, тот
будет иметь обитель в башне со святыми Божьими.
\vs 1Er 16:16
Я спросил её о времени, не конец ли уж теперь.
\vs 1Er 16:17
Но она громко воскликнула:
<<Неразумный человек!
Неужели не видишь ты, что башня всё ещё строится?
Когда башня будет построена, тогда и будет конец.
\vs 1Er 16:18
Не спрашивай у меня ничего более.
И этого напоминания и обновления душ ваших
достаточно для тебя и для всех святых.
\vs 1Er 16:19
Не для тебя одного это открыто, но чтобы ты возвестил всем.
\vs 1Er 16:20
Итак, по прошествии 3-х дней ты, Ерма,
должен уразуметь следующие слова, которые имею сказать тебе,
чтобы ты довел их до ушей святых, чтобы, слушая и исполняя их,
очистились от своих неправд~--- и ты вместе с ними.>>

\vs 1Er 17:1
Послушайте меня, дети.
Я воспитала вас в великой простоте,
невинности и целомудрии, по милосердию Господа,
\vs 1Er 17:2
Который излил на вас
правду, чтобы вы очистились от всякого беззакония и лжи, а вы не хотите
отступиться от неправд ваших. Итак, теперь послушайте меня.
\vs 1Er 17:3
Живите в мире, заботьтесь
друг о друге, поддерживайте себя взаимно и не пользуйтесь одни творениями
Божьими, но щедро раздавайте нуждающимся.
\vs 1Er 17:4
Некоторые от многих яств
наносят вред своей плоти и истощают её.
А у других, не имеющих пропитания,
также истощается плоть оттого,
что нет в достатке пищи и гибнут тела их.
\vs 1Er 17:5
Такое невоздержание пагубно для тех,
кто имеет и не делится с нуждающимися.
Подумайте о грядущем суде.
\vs 1Er 17:6
Вы, кто превосходит других, отыскивайте алчущих,
пока ещё не окончена башня.
\vs 1Er 17:7
Ибо после, когда завершится строительство,
пожелаете благотворить, но не будет вам места.
\vs 1Er 17:8
Итак, смотрите вы, гордящиеся своими богатствами,
чтобы не восстенали терпящие нужду,
\vs 1Er 17:9
стон их взойдет к Господу
и удалены вы будете со своими сокровищами за двери башни.
\vs 1Er 17:10
Тем теперь говорю, кто
начальствует в Церкви и главенствует: не будьте подобны злодеям.
\vs 1Er 17:11
Злодеи, по крайней мере, яд свой носят в сосудах,
а вы отраву свою и яд держите в сердце;
\vs 1Er 17:12
не хотите очистить сердец
ваших и чистым сердцем сойтись в единомыслие,
чтобы иметь милость от Великого Царя.
\vs 1Er 17:13
Смотрите, дети, чтобы такие разделения ваши не лишили вас жизни.
\vs 1Er 17:14
Как хотите вы воспитывать избранников Божьих,
когда сами не имеете научения?
\vs 1Er 17:15
Поэтому вразумляйте себя взаимно и будьте в мире между собою,
чтобы и я, радостно представ пред Отцом вашим,
могла дать отчёт за вас Господу.

\vs 1Er 18:1
Когда она перестала говорить со мною,
пришли те 6 юношей, которые строили,
и понесли её к башне,
а другие 4 взяли скамью и также отнесли её в башню.
\vs 1Er 18:2
Лица сих последних я не видел,
потому что они были обращены в другую сторону.
\vs 1Er 18:3
Когда она удалялась, я просил её объяснить различные облики,
в которых являлась она мне.
\vs 1Er 18:4
Но она сказала в ответ:
<<Это пусть другой объяснит тебе.>>
\vs 1Er 18:5
А явилась она мне, братья, в 1-м видении, в прошлом году,
очень старою, сидящею на кафедре.
\vs 1Er 18:6
Во 2-м видении она имела лицо юное,
но тело и волосы старческие, и беседовала со мною стоя;
впрочем, была веселее, нежели прежде.
\vs 1Er 18:7
В 3-м же видении она вся была гораздо моложе,
с прекрасным лицом, но со старческими волосами;
она была вполне весела и сидела на скамье.
\vs 1Er 18:8
И очень я печалился, что не понятны мне такие различия,
пока не увидел во сне ночном ту старицу,
\vs 1Er 18:9
и она сказала мне:
<<Всякая молитва нуждается в смирении,
поэтому постись и получишь от Господа, чего просишь.>>
\vs 1Er 18:10
Итак, я постился 1 день,
и в ту же ночь явился мне юноша и сказал:
<<Почему ты так часто в молитве просишь откровений?
\vs 1Er 18:11
Смотри, чтобы, прося многого, не повредить тебе своей плоти.
Достаточно для тебя и этих откровений.
\vs 1Er 18:12
Сможешь ли видеть откровения ещё больше тех, которые видел?>>
\vs 1Er 18:13
Господин, я об одном только прошу,
чтобы мне было дано полное объяснение насчёт 3-х обликов той старицы.
\vs 1Er 18:14
Доколе будете вы неразумны?~--- укорил он.~--- Сомнения ваши
делают вас неразумными, потому что не
имеете в сердцах ваших устремления к Господу.
\vs 1Er 18:15
Я отвечал ему:
<<От тебя мы узнаем об этом вернее.>>

\vs 1Er 19:1
Слушай,~--- сказал он,~--- об обликах, которые тебя интересуют.
Почему в 1-м видении явилась тебе старица, сидящая на кафедре?
\vs 1Er 19:2
Потому что дух ваш обветшал и ослабел
и не имеет силы от грехов ваших и сомнений сердца.
\vs 1Er 19:3
Ибо как старцы, не имеющие
надежды на обновление, ничего другого не желают,
кроме успокоения на ложе,
\vs 1Er 19:4
так и вы, обременённые житейскими делами,
впали в беспечность и не возложили попечений своих на
Господа; одряхлел ваш разум и состарились вы в печалях ваших.
\vs 1Er 19:5
Я желаю узнать, господин, почему она сидела на кафедре?
\vs 1Er 19:6
Потому,~--- отвечал он мне,
что всякий немощный сидит на седалище по причине своей слабости,
чтобы имело поддержку немощное тело его.
Вот тебе смысл 1-го явления.

\vs 1Er 20:1
Во 2-м видении ты видел её стоящей,
с помолодевшим лицом и более веселою, нежели прежде;
а тело и волосы были у неё старческие.
\vs 1Er 20:2
Выслушай и эту притчу.
Когда кто сильно состарится и отчается в самом себе
из-за своей слабости и бедности,
то ничего другого не ожидает,
только последнего дня своей жизни.
\vs 1Er 20:3
Но вдруг получает он наследство.
Узнав об этом, он вскакивает повеселевший, к нему возвращаются
силы, обновляется дух его, который одряхлел от прежних дел;
он уже не лежит, но, восставши, мужественно действует.
\vs 1Er 20:4
То же произошло и с вами, когда услышали вы об откровении,
которое Бог сообщил вам.
\vs 1Er 20:5
Господь сжалился над вами и обновил дух ваш~--- и вы отложили
свои немощи, пришло к вам мужество, вы укрепились в вере,
и Господь, видя вашу верность, возрадовался.
\vs 1Er 20:6
Поэтому показал он вам строение башни~--- и иное покажет,
если будет между вами чистосердечный мир.

\vs 1Er 21:1
В 3-м видении ты видел, что она ещё моложе, прекрасна,
весела и лицо её светло.
\vs 1Er 21:2
Сравнить это с тем можно, как если бы к печалящемуся
человеку пришёл добрый вестник
\vs 1Er 21:3
тотчас он забыл бы прежнюю скорбь,
ни о чём другом не думал, как об услышанной им вести;
ободряется и обновляется дух его от радости.
\vs 1Er 21:4
Так точно и вы получили обновление душ ваших,
узнав такие блага.
\vs 1Er 21:5
А что ты видел её сидящею на скамье~--- это означает
твёрдое положение, так как скамейка имеет 4 ножки и стоит твёрдо,
да и мир поддерживается 4-мя стихиями.
\vs 1Er 21:6
Поэтому и те, которые
всецело, от всего сердца покаются, помолодеют и окрепнут.
\vs 1Er 21:7
Теперь имеешь ты полное объяснение.
Не проси более никаких откровений.
Если же нужно будет, то откроется тебе.

\chhdr{Видение 4-е.}
\vs 1Er 22:1
Спустя 20 дней было мне, братья,
видение гонения, которое должно случиться.
\vs 1Er 22:2
Шел я по полю при дороге Шампанской,
от большой дороги до поля почти 10 стадиев:
через это место путь бывает редко.
\vs 1Er 22:3
Гуляя один, я молил
Господа, чтобы он подтвердил те откровения,
которые явил мне чрез святую свою Церковь,
укрепил меня и дал покаяние всем рабам своим,
которые соблазнились;
\vs 1Er 22:4
дабы прославлялось великое и досточтимое имя его за то,
что удостоил показать мне чудеса свои.
\vs 1Er 22:5
И в то время когда я прославлял и благодарил его, голос был мне:
<<Не сомневайся, Ерма!>>
\vs 1Er 22:6
Стал я думать:
<<Что мне сомневаться, когда я так укреплён Господом и видел дивные дела?>>
\vs 1Er 22:7
Пройдя немного, братья, вдруг увидел я пыль,
поднимающуюся до неба, и подумал, что это идёт скот,
пыль поднимая.
\vs 1Er 22:8
Расстояние между тучей пыли и мною было около стадия.
\vs 1Er 22:9
Между тем пыль поднималась гуще и гуще,
так что мне стало это казаться чем-то сверхъестественным.
\vs 1Er 22:10
Несколько проглянуло солнце, и увидел я огромного зверя
наподобие дракона, из уст которого выходила огненная саранча.
\vs 1Er 22:11
В длину это животное имело около 100 футов,
а голова была подобна глиняному сосуду.
\vs 1Er 22:12
И начал я плакать и молить Господа, чтобы спас меня от него.
\vs 1Er 22:13
Потом вспомнил я слова, которые слышал:
<<Не сомневайся, Ерма.>>
\vs 1Er 22:14
Итак, братья, облёкшись верою в Бога и вспомнив явленные
мне им великие дела, я смело пошёл к зверю.
\vs 1Er 22:15
Зверь же метался с такою яростью и был так свиреп и силён,
что, казалось, при нападении мог бы разрушить город.
\vs 1Er 22:16
Я приблизился к нему,
и это огромное устрашающее животное мирно
растянулось на земле, высунув язык.
\vs 1Er 22:17
Я прошёл мимо него, и оно не пошевелилось.
\vs 1Er 22:18
Голова этого зверя была 4-х цветов:
чёрного, потом красного, или кровавого,
далее золотистого и, наконец, белого.

\vs 1Er 23:1
После того как я прошёл мимо зверя и удалился почти на 30 футов,
встречается мне разукрашенная дева,
словно выходящая из брачного чертога,
\vs 1Er 23:2
в белых башмаках, покрытая белыми одеждами до самого чела;
митра была ее покрывалом, волосы у ней были белые.
\vs 1Er 23:3
По прежним видениям я догадался,
что это Церковь, и обрадовался.
\vs 1Er 23:4
Она приветствовала меня:
<<Здравствуй, человек.>>
И я ответил ей также приветствием.
\vs 1Er 23:5
Она спросила:
<<Ничто не встретилось тебе, человек?>>
\vs 1Er 23:6
Госпожа, мне встретилось такое животное,
которое могло бы истребить народы, но силою Бога
и по великому его милосердию я спасся от него.
\vs 1Er 23:7
Счастливо спасся ты, сказала она,~--- потому,
что заботу свою ты возложил на Господа и ему открыл своё сердце,
\vs 1Er 23:8
веруя, что никем другим не можешь быть спасён,
кроме его великого и преславного имени.
\vs 1Er 23:9
За это Господь послал ангела своего,
поставленного над зверями, которому имя Егрин,
и он заградил пасть его, чтобы не пожрал тебя.
\vs 1Er 23:10
Ты избежал великого бедствия по вере твоей,
так как ты не усомнился при виде такого зверя.
\vs 1Er 23:11
Итак, пойди и возвести избранникам Бога
о великих делах его и скажи им,
что зверь этот есть образ грядущей великой напасти.
\vs 1Er 23:12
Поэтому, если приготовите себя и от всего сердца
покаетесь перед Господом, то можете избежать её,
\vs 1Er 23:13
если сердце ваше будет чисто и непорочно
и в остальные дни жизни вашей
будете безукоризненно служить Богу.
\vs 1Er 23:14
Возложите на Господа печали ваши, и он сам уврачует их.
\vs 1Er 23:15
Вы, двоедушные, веруйте в Бога, что он всё может~--- и
отвратить от вас гнев свой, и послать бичи на двоедушных.
\vs 1Er 23:16
Горе тем, которые услышат эти слова и презрят их,
лучше было им не родиться.

\vs 1Er 24:1
Я спросил её о 4-х цветах, которые были у зверя на голове.
\vs 1Er 24:2
Она сказала на это:
<<Опять ты любопытствуешь, спрашивая о вещах такого рода.>>
\vs 1Er 24:3
Да, госпожа, объясни мне, что они означают.
\vs 1Er 24:4
Слушай же,~--- отвечала она.
Чёрный цвет означает мир, в котором вы живете.
\vs 1Er 24:5
Огненный и кровавый~--- то,
что этому миру должно погибнуть посредством крови и огня.
\vs 1Er 24:6
А золотистая часть~--- это все вы, избегающие этого мира.
\vs 1Er 24:7
Ибо как золото испытывается посредством огня
и становится годным, так испытываетесь и вы,
живущие среди них.
\vs 1Er 24:8
Те, которые сохранят твёрдость и будут искушены ими, очистятся.
\vs 1Er 24:9
И как золото оставляет в огне примеси свои,
так и вы оставите всякую скорбь и печаль, очиститесь и
будете годны для здания башни.
\vs 1Er 24:10
Белая же часть означает будущий век,
в котором станут жить избранники Божьи,
\vs 1Er 24:11
потому что непорочны и чисты будут те,
которые избраны Богом в жизнь вечную.
\vs 1Er 24:12
Итак, не переставай доносить это до слуха святых.
\vs 1Er 24:13
Имеете вы и образ грядущего великого бедствия.
Если захотите вы, оно будет не страшно для вас:
помните заповеданное вам.
\vs 1Er 24:14
Сказав это, она удалилась, и не видел я, куда она ушла.
\vs 1Er 24:15
Раздался шум, и я в страхе бросился назад,
думая, что приближается тот зверь \ldots

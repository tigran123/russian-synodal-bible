\bibbookdescr{2Ch}{
  inline={\LARGE Вторая книга\\\Huge Паралипоменон\fns{У Евреев: <<Летопись>>.}},
  toc={2-я Паралипоменон},
  bookmark={2-я Паралипоменон},
  header={2-я Паралипоменон},
  %headerleft={},
  %headerright={},
  abbr={2~Пар}
}
\vs 2Ch 1:1 И утвердился Соломон, сын Давидов, в царстве своем; и Господь Бог его \bibemph{был} с ним, и вознес его высоко.
\vs 2Ch 1:2 И приказал Соломон \bibemph{собраться} всему Израилю: тысяченачальникам и стоначальникам, и судьям, и всем начальствующим во всем Израиле~--- главам поколений.
\vs 2Ch 1:3 И пошли Соломон и все собрание с ним на высоту, что в Гаваоне, ибо там была Божия скиния собрания, которую устроил Моисей, раб Господень, в пустыне.
\vs 2Ch 1:4 Ковчег Божий принес Давид из Кириаф-Иарима на место, которое приготовил для него Давид, устроив для него скинию в Иерусалиме.
\vs 2Ch 1:5 А медный жертвенник, который сделал Веселеил, сын Урия, сына Орова, \bibemph{оставался} там, пред скиниею Господнею, и взыскал его Соломон с собранием.
\vs 2Ch 1:6 И там пред лицем Господа, на медном жертвеннике, который пред скиниею собрания, вознес Соломон тысячу всесожжений.
\rsbpar\vs 2Ch 1:7 В ту ночь явился Бог Соломону и сказал ему: проси, что Мне дать тебе.
\vs 2Ch 1:8 И сказал Соломон Богу: Ты сотворил Давиду, отцу моему, великую милость и поставил меня царем вместо него.
\vs 2Ch 1:9 Да исполнится же, Господи Боже, слово Твое к Давиду, отцу моему. Так как Ты воцарил меня над народом многочисленным, как прах земной,
\vs 2Ch 1:10 то ныне дай мне премудрость и знание, чтобы я \bibemph{умел} выходить пред народом сим и входить, ибо кто может управлять сим народом Твоим великим?
\vs 2Ch 1:11 И сказал Бог Соломону: за то, что это было на сердце твоем, и ты не просил богатства, имения и славы и души неприятелей твоих, и также не просил ты многих дней, а просил себе премудрости и знания, чтобы управлять народом Моим, над которым Я воцарил тебя,
\vs 2Ch 1:12 премудрость и знание дается тебе, а богатство и имение и славу Я дам тебе такие, подобных которым не бывало у царей прежде тебя и не будет после тебя.
\rsbpar\vs 2Ch 1:13 И пришел Соломон с высоты, что в Гаваоне, от скинии собрания, в Иерусалим и царствовал над Израилем.
\vs 2Ch 1:14 И набрал Соломон колесниц и всадников; и было у него тысяча четыреста колесниц и двенадцать тысяч всадников; и он разместил их в колесничных городах и при царе в Иерусалиме.
\vs 2Ch 1:15 И сделал царь серебро и золото в Иерусалиме равноценным \bibemph{простому} камню, а кедры, по множеству их, сделал равноценными сикоморам, которые на низких местах.
\vs 2Ch 1:16 Коней Соломону приводили из Египта и из Кувы; купцы царские из Кувы получали их за деньги.
\vs 2Ch 1:17 Колесница получаема и доставляема была из Египта за шестьсот \bibemph{сиклей} серебра, а конь за сто пятьдесят. Таким же образом они руками своими доставляли \bibemph{это} всем царям Хеттейским и царям Арамейским.
\vs 2Ch 2:1 И положил Соломон построить дом имени Господню и дом царский для себя.
\vs 2Ch 2:2 И отчислил Соломон семьдесят тысяч носильщиков и восемьдесят тысяч каменосеков в горах, и надзирателей над ними три тысячи шестьсот.
\vs 2Ch 2:3 И послал Соломон к Хираму, царю Тирскому, сказать: как поступал ты с Давидом, отцом моим, и присылал ему кедры на построение дома для его жительства, \bibemph{так поступи и со мною}.
\vs 2Ch 2:4 Вот я строю дом имени Господа Бога моего, для посвящения Ему, чтобы возжигать пред Ним благовонное курение, представлять постоянно хлебы предложения и \bibemph{возносить там} всесожжения утром и вечером в субботы, и в новомесячия, и в праздники Господа Бога нашего, что навсегда заповедано Израилю.
\vs 2Ch 2:5 И дом, который я строю, велик, потому что велик Бог наш, выше всех богов.
\vs 2Ch 2:6 И достанет ли у кого силы построить Ему дом, когда небо и небеса небес не вмещают Его? И кто я, чтобы мог построить Ему дом? Разве \bibemph{только} для курения пред лицем Его.
\vs 2Ch 2:7 Итак пришли мне человека, умеющего делать \bibemph{изделия} из золота, и из серебра, и из меди, и из железа, и из \bibemph{пряжи} пурпурового, багряного и яхонтового \bibemph{цвета}, и знающего вырезывать резную работу, вместе с художниками, какие есть у меня в Иудее и в Иерусалиме, которых приготовил Давид, отец мой.
\vs 2Ch 2:8 И пришли мне кедровых дерев, и кипарису и певгового дерева с Ливана, ибо я знаю, что рабы твои умеют рубить дерева Ливанские. И вот рабы мои пойдут с рабами твоими,
\vs 2Ch 2:9 чтобы мне приготовить множество дерев, потому что дом, который я строю, великий и чудный.
\vs 2Ch 2:10 И вот древосекам, рубящим дерева, рабам твоим, я даю в пищу: пшеницы двадцать тысяч к\acc{о}ров, и ячменю двадцать тысяч к\acc{о}ров, и вина двадцать тысяч батов, и оливкового масла двадцать тысяч батов.
\rsbpar\vs 2Ch 2:11 И отвечал Хирам, царь Тирский, письмом, которое прислал к Соломону: по любви к народу Своему, Господь поставил тебя царем над ним.
\vs 2Ch 2:12 И \bibemph{еще} сказал Хирам: благословен Господь Бог Израилев, создавший небо и землю, давший царю Давиду сына мудрого, имеющего смысл и разум, который намерен строить дом Господу и дом царский для себя.
\vs 2Ch 2:13 Итак я посылаю [тебе] человека умного, имеющего знания, Хирам-Авия,
\vs 2Ch 2:14 сына \bibemph{одной} женщины из дочерей Дановых,~--- а отец его Тирянин,~--- умеющего делать \bibemph{изделия} из золота и из серебра, из меди, из железа, из камней и из дерев, из \bibemph{пряжи} пурпурового, яхонтового \bibemph{цвета}, и из виссона, и из багряницы, и вырезывать всякую резьбу, и исполнять все, что будет поручено ему вместе с художниками твоими и с художниками господина моего Давида, отца твоего.
\vs 2Ch 2:15 А пшеницу и ячмень, оливковое масло и вино, о которых говорил ты, господин мой, пошли рабам твоим.
\vs 2Ch 2:16 Мы же нарубим дерев с Ливана, сколько нужно тебе, и пригоним их в плотах по морю в Яфу, а ты отвезешь их в Иерусалим.
\rsbpar\vs 2Ch 2:17 И исчислил Соломон всех пришельцев, бывших \bibemph{тогда} в земле Израилевой, после исчисления их, сделанного Давидом, отцом его,~--- и нашлось их сто пятьдесят три тысячи шестьсот.
\vs 2Ch 2:18 И сделал он из них семьдесят тысяч носильщиков и восемьдесят тысяч каменосеков на горах и три тысячи шестьсот надзирателей, чтобы они побуждали народ к работе.
\vs 2Ch 3:1 И начал Соломон строить дом Господень в Иерусалиме на горе Мориа, которая указана была Давиду, отцу его, на месте, которое приготовил Давид, на гумне Орны Иевусеянина.
\vs 2Ch 3:2 Начал же он строить во второй \bibemph{день} второго месяца, в четвертый год царствования своего.
\vs 2Ch 3:3 И вот основание, \bibemph{положенное} Соломоном при строении дома Божия: длина \bibemph{его} шестьдесят локтей, по прежней мере, а ширина двадцать локтей;
\vs 2Ch 3:4 и притвор, который пред домом, длиною по ширине дома в двадцать локтей, а вышиною во сто двадцать. И обложил его внутри чистым золотом.
\vs 2Ch 3:5 Дом же главный обшил деревом кипарисовым и обложил его лучшим золотом, и выделал на нем пальмы и цепочки.
\vs 2Ch 3:6 И обложил дом дорогими камнями для красоты; золото же \bibemph{было} золото Парваимское.
\vs 2Ch 3:7 И покрыл дом, бревна, пороги и стены его, [и окна] и двери его золотом, и вырезал на стенах херувимов.
\vs 2Ch 3:8 И сделал Святое Святых: длина его по широте дома в двадцать локтей, и ширина его в двадцать локтей; и покрыл его лучшим золотом на шестьсот талантов.
\vs 2Ch 3:9 В гвоздях весу до пятидесяти сиклей золота [в каждом гвозде]. Горницы также покрыл золотом.
\vs 2Ch 3:10 И сделал он во Святом Святых двух херувимов резной работы и покрыл их золотом.
\vs 2Ch 3:11 Крылья херувимов длиною \bibemph{были} в двадцать локтей. Одно крыло в пять локтей касалось стены дома, а другое крыло в пять же локтей сходилось с крылом другого херувима;
\vs 2Ch 3:12 \bibemph{равно} и крыло другого херувима в пять локтей касалось стены дома, а другое крыло в пять локтей сходилось с крылом другого херувима.
\vs 2Ch 3:13 Крылья сих херувимов \bibemph{были} распростерты на двадцать локтей; и они стояли на ногах своих, лицами своими к храму.
\vs 2Ch 3:14 И сделал завесу из яхонтовой, пурпуровой и багряной \bibemph{ткани} и из виссона и изобразил на ней херувимов.
\vs 2Ch 3:15 И сделал пред храмом два столба, длиною по тридцати пяти локтей, и капитель на верху каждого в пять локтей.
\vs 2Ch 3:16 И сделал цепочки, \bibemph{как} во святилище, и положил на верху столбов, и сделал сто гранатовых яблок и положил на цепочки.
\vs 2Ch 3:17 И поставил столбы пред храмом, один по правую сторону, другой по левую, и дал имя правому Иахин, а левому имя Воаз.
\vs 2Ch 4:1 И сделал медный жертвенник: двадцать локтей длина его и двадцать локтей ширина его и десять локтей вышина его.
\vs 2Ch 4:2 И сделал море литое,~--- от края его до края его десять локтей,~--- все круглое, вышиною в пять локтей; и снурок в тридцать локтей обнимал его кругом;
\vs 2Ch 4:3 и \bibemph{литые} подобия волов стояли под ним кругом со всех сторон; на десять локтей окружали море кругом два ряда волов, вылитых одним литьем с ним.
\vs 2Ch 4:4 Стояло оно на двенадцати волах: три глядели к северу и три глядели к западу, и три глядели к югу, и три глядели к востоку,~--- и море на них сверху; зады же их были обращены внутрь под него.
\vs 2Ch 4:5 Толщиною оно \bibemph{было} в ладонь; и края его, сделанные, как края чаши, \bibemph{походили} на распустившуюся лилию. Оно вмещало до трех тысяч батов.
\vs 2Ch 4:6 И сделал десять омывальниц, и поставил пять по правую сторону и пять по левую, чтоб омывать в них,~--- приготовляемое ко всесожжению омывали в них; море же~--- для священников, чтоб они омывались в нем.
\vs 2Ch 4:7 И сделал десять золотых светильников, как им быть надлежало, и поставил в храме, пять по правую сторону и пять по левую.
\vs 2Ch 4:8 И сделал десять столов и поставил в храме, пять по правую сторону и пять по левую, и сделал сто золотых чаш.
\vs 2Ch 4:9 И сделал священнический двор и большой двор и двери к двору, и вереи их обложил медью.
\vs 2Ch 4:10 Море поставил на правой стороне, к юго-востоку.
\vs 2Ch 4:11 И сделал Хирам тазы, и лопатки, и чаши [и кадильницы, и все жертвенные сосуды]. И кончил Хирам работу, которую производил для царя Соломона в доме Божием:
\vs 2Ch 4:12 два столба и две опояски венцов на верху столбов, и две сетки для покрытия двух опоясок венцов, которые на главе столбов,
\vs 2Ch 4:13 и четыреста гранатовых яблок на двух сетках, два ряда гранатовых яблок для каждой сетки, для покрытия двух опоясок венцов, которые на столбах.
\vs 2Ch 4:14 И подставы сделал он, и омывальницы сделал на подставах;
\vs 2Ch 4:15 одно море, и двенадцать волов под ним,
\vs 2Ch 4:16 и тазы, и лопатки, и вилки, и весь прибор их сделал Хирам-Авий царю Соломону для дома Господня из полированной меди.
\vs 2Ch 4:17 В окрестности Иордана выливал их царь, в глинистой земле, между Сокхофом и Цередою.
\vs 2Ch 4:18 И сделал Соломон все вещи сии в великом множестве, так что не знали веса меди.
\vs 2Ch 4:19 Также сделал Соломон все вещи для дома Божия и золотой жертвенник, и столы, на которых хлебы предложения,
\vs 2Ch 4:20 и светильники и лампады их, чтобы возжигать их по уставу пред давиром, из чистого золота;
\vs 2Ch 4:21 и цветы, и лампады, и щипцы из золота, из самого чистого золота,
\vs 2Ch 4:22 и ножи, и кропильницы, и чаши, и лотки из золота самого чистого, и двери храма,~--- двери его внутренние во Святое Святых, и двери храма во святилище,~--- из золота.
\vs 2Ch 5:1 И окончилась вся работа, которую производил Соломон для дома Господня. И принес Соломон посвященное Давидом, отцом его, и серебро и золото и все вещи отдал в сокровищницы дома Божия.
\rsbpar\vs 2Ch 5:2 Тогда собрал Соломон старейшин Израилевых и всех глав колен, начальников поколений сынов Израилевых, в Иерусалим, для перенесения ковчега завета Господня из города Давидова, то есть \bibemph{с} Сиона.
\vs 2Ch 5:3 И собрались к царю все Израильтяне на праздник, в седьмой месяц.
\vs 2Ch 5:4 И пришли все старейшины Израилевы. Левиты взяли ковчег
\vs 2Ch 5:5 и понесли ковчег и скинию собрания и все вещи священные, которые в скинии,~--- понесли их священники и левиты.
\vs 2Ch 5:6 Царь же Соломон и все общество Израилево, собравшееся к нему пред ковчегом, приносили жертвы из овец и волов, которых невозможно исчислить и определить, по причине множества.
\vs 2Ch 5:7 И принесли священники ковчег завета Господня на место его, в давир храма~--- во Святое Святых, под крылья херувимов.
\vs 2Ch 5:8 И херувимы распростирали крылья над местом ковчега, и покрывали херувимы ковчег и шесты его сверху.
\vs 2Ch 5:9 И выдвинулись шесты, так что головки шестов ковчега видны были пред давиром, но не выказывались наружу, и они там до сего дня.
\vs 2Ch 5:10 Не было в ковчеге ничего кроме двух скрижалей, которые положил Моисей на Хориве, когда Господь заключил \bibemph{завет} с сынами Израилевыми, по исходе их из Египта.
\vs 2Ch 5:11 Когда священники вышли из святилища, ибо все священники, находившиеся там, освятились без различия отделов;
\vs 2Ch 5:12 и левиты певцы,~--- все они, \bibemph{то есть} Асаф, Еман, Идифун и сыновья их, и братья их,~--- одетые в виссон, с кимвалами и с псалтирями и цитрами стояли на восточной стороне жертвенника, и с ними сто двадцать священников, трубивших трубами,
\vs 2Ch 5:13 и были, как один, трубящие и поющие, издавая один голос к восхвалению и славословию Господа; и когда загремел звук труб и кимвалов и музыкальных орудий, и восхваляли Господа, ибо Он благ, ибо вовек милость Его; тогда дом, дом Господень, наполнило облако,
\vs 2Ch 5:14 и не могли священники стоять на служении по причине облака, потому что слава Господня наполнила дом Божий.
\vs 2Ch 6:1 Тогда сказал Соломон: Господь сказал, что Он благоволит обитать во мгле,
\vs 2Ch 6:2 а я построил дом в жилище Тебе, [Святый,] место для вечного Твоего пребывания.
\vs 2Ch 6:3 И обратился царь лицем своим и благословил все собрание Израильтян,~--- все собрание Израильтян стояло,~---
\vs 2Ch 6:4 и сказал: благословен Господь Бог Израилев, Который, чт\acc{о} сказал устами Своими Давиду, отцу моему, исполнил \bibemph{ныне} рукою Своею! Он говорил:
\vs 2Ch 6:5 <<с того дня, как Я вывел народ Мой из земли Египетской, Я не избрал города ни в одном из колен Израилевых для построения дома, в котором пребывало бы имя Мое, и не избрал человека, который был бы правителем народа Моего Израиля,
\vs 2Ch 6:6 но избрал Иерусалим, чтобы там пребывало имя Мое, и избрал Давида, чтоб он был над народом Моим Израилем>>.
\vs 2Ch 6:7 И было на сердце у Давида, отца моего, построить дом имени Господа, Бога Израилева.
\vs 2Ch 6:8 Но Господь сказал Давиду, отцу моему: <<у тебя есть на сердце построить храм имени Моему; хорошо, что это на сердце у тебя.
\vs 2Ch 6:9 Однако не ты построишь храм, а сын твой, который произойдет из чресл твоих,~--- он построит храм имени Моему>>.
\vs 2Ch 6:10 И исполнил Господь слово Свое, которое изрек: я вступил на место Давида, отца моего, и воссел на престоле Израилевом, как сказал Господь, и построил дом имени Господа Бога Израилева.
\vs 2Ch 6:11 И я поставил там ковчег, в котором завет Господа, заключенный Им с сынами Израилевыми.
\rsbpar\vs 2Ch 6:12 И стал \bibemph{Соломон} у жертвенника Господня впереди всего собрания Израильтян, и воздвиг руки свои,~---
\vs 2Ch 6:13 ибо Соломон сделал медный амвон длиною в пять локтей и шириною в пять локтей, а вышиною в три локтя, и поставил его среди двора; и стал на нем, и преклонил колени впереди всего собрания Израильтян, и воздвиг руки свои к небу,~---
\vs 2Ch 6:14 и сказал: Господи Боже Израилев! Нет Бога, подобного Тебе, ни на небе, ни на земле. Ты хранишь завет и милость к рабам Твоим, ходящим пред Тобою всем сердцем своим:
\vs 2Ch 6:15 Ты исполнил рабу Твоему Давиду, отцу моему, что Ты говорил ему; что изрек Ты устами Твоими, то в день сей исполнил рукою Твоею.
\vs 2Ch 6:16 И ныне, Господи Боже Израилев! исполни рабу Твоему Давиду, отцу моему, то, что Ты сказал ему, говоря: не прекратится у тебя [муж,] сидящий пред лицем Моим на престоле Израилевом, если только сыновья твои будут наблюдать за путями своими, ходя по закону Моему так, как ты ходил предо Мною.
\vs 2Ch 6:17 И ныне, Господи Боже Израилев! да будет верно слово Твое, которое Ты изрек рабу Твоему Давиду.
\vs 2Ch 6:18 Поистине, Богу ли жить с человеками на земле? Если небо и небеса небес не вмещают Тебя, тем менее храм сей, который построил я.
\vs 2Ch 6:19 Но призри на молитву раба Твоего и на прошение его, Господи Боже мой! услышь воззвание и молитву, которою раб Твой молится пред Тобою.
\vs 2Ch 6:20 Да будут очи Твои отверсты на храм сей днем и ночью, на место, где Ты обещал положить имя Твое, чтобы слышать молитву, которою раб Твой будет молиться на месте сем.
\vs 2Ch 6:21 Услышь моления раба Твоего и народа Твоего Израиля, какими они будут молиться на месте сем; услышь с места обитания Твоего, с небес, услышь и помилуй!
\vs 2Ch 6:22 Когда кто согрешит против ближнего своего, и потребуют от него клятвы, чтоб он поклялся, и будет совершаться клятва пред жертвенником Твоим в храме сем,
\vs 2Ch 6:23 тогда Ты услышь с неба и соверши суд над рабами Твоими, воздай виновному, возложив поступок его на голову его, и оправдай правого, воздав ему по правде его.
\vs 2Ch 6:24 Когда поражен будет народ Твой Израиль неприятелем за то, что согрешил пред Тобою, и они обратятся \bibemph{к Тебе}, и исповедают имя Твое, и будут просить и молиться пред Тобою в храме сем,
\vs 2Ch 6:25 тогда Ты услышь с неба, и прости грех народа Твоего Израиля, и возврати их в землю, которую Ты дал им и отцам их.
\vs 2Ch 6:26 Когда заключится небо и не будет дождя за то, что они согрешили пред Тобою, и будут молиться на месте сем, и исповедают имя Твое, и обратятся от греха своего, потому что Ты смирил их,
\vs 2Ch 6:27 тогда Ты услышь с неба и прости грех рабов Твоих и народа Твоего Израиля, указав им добрый путь, по которому идти им, и пошли дождь на землю Твою, которую Ты дал народу Твоему в наследие.
\vs 2Ch 6:28 Голод ли будет на земле, будет ли язва моровая, будет ли ветер палящий или ржа, саранча или червь, будут ли теснить его неприятели его на земле владений его, \bibemph{будет ли} какое бедствие, какая болезнь,
\vs 2Ch 6:29 всякую молитву, всякое прошение, какое будет от какого-либо человека или от всего народа Твоего Израиля, когда они почувствуют каждый бедствие свое и горе свое и прострут руки свои к храму сему,
\vs 2Ch 6:30 Ты услышь с неба~--- места обитания Твоего, и прости, и воздай каждому по всем путям его, как Ты знаешь сердце его,~--- ибо Ты один знаешь сердце сынов человеческих,~---
\vs 2Ch 6:31 чтобы они боялись Тебя и ходили путями Твоими во все дни, доколе живут на земле, которую Ты дал отцам нашим.
\vs 2Ch 6:32 Даже и иноплеменник, который не от народа Твоего Израиля, когда он придет из земли далекой ради имени Твоего великого и руки Твоей могущественной и мышцы Твоей простертой, и придет и будет молиться у храма сего,
\vs 2Ch 6:33 Ты услышь с неба, с места обитания Твоего, и сделай все, о чем будет взывать к Тебе иноплеменник, чтобы все народы земли узнали имя Твое, и чтобы боялись Тебя, как народ Твой Израиль, и знали, что Твоим именем называется дом сей, который построил я.
\vs 2Ch 6:34 Когда выйдет народ Твой на войну против неприятелей своих путем, которым Ты пошлешь его, и будет молиться Тебе, обратившись к городу сему, который избрал Ты, и к храму, который я построил имени Твоему,
\vs 2Ch 6:35 тогда услышь с неба молитву их и прошение их и сделай, что потребно для них.
\vs 2Ch 6:36 Когда они согрешат пред Тобою,~--- ибо нет человека, который не согрешил бы,~--- и Ты прогневаешься на них, и предашь их врагу, и отведут их пленившие их в землю далекую или близкую,
\vs 2Ch 6:37 и когда они в земле, в которую будут пленены, войдут в себя и обратятся и будут молиться Тебе в земле пленения своего, говоря: мы согрешили, сделали беззаконие, мы виновны,
\vs 2Ch 6:38 и обратятся к Тебе всем сердцем своим и всею душею своею в земле пленения своего, куда отведут их в плен, и будут молиться, обратившись к земле своей, которую Ты дал отцам их, и к городу, который избрал Ты, и к храму, который я построил имени Твоему,~---
\vs 2Ch 6:39 тогда услышь с неба, с места обитания Твоего, молитву их и прошение их, и сделай, что потребно для них, и прости народу Твоему, в чем он согрешил пред Тобою.
\vs 2Ch 6:40 Боже мой! да будут очи Твои отверсты и уши Твои внимательны к молитве на месте сем.
\vs 2Ch 6:41 И ныне, Господи Боже, стань на \bibemph{место} покоя Твоего, Ты и ковчег могущества Твоего. Священники Твои, Господи Боже, да облекутся во спасение, и преподобные Твои да насладятся благами.
\vs 2Ch 6:42 Господи Боже! не отврати лица помазанника Твоего, помяни милости к Давиду, рабу Твоему.
\vs 2Ch 7:1 Когда окончил Соломон молитву, сошел огонь с неба и поглотил всесожжение и жертвы, и слава Господня наполнила дом.
\vs 2Ch 7:2 И не могли священники войти в дом Господень, потому что слава Господня наполнила дом Господень.
\vs 2Ch 7:3 И все сыны Израилевы, видя, как сошел огонь и слава Господня на дом, пали лицем на землю, на помост, и поклонились, и славословили Господа, ибо Он благ, ибо вовек милость Его.
\rsbpar\vs 2Ch 7:4 Царь же и весь народ стали приносить жертвы пред лицем Господа.
\vs 2Ch 7:5 И принес царь Соломон в жертву двадцать две тысячи волов и сто двадцать тысяч овец: так освятили дом Божий царь и весь народ.
\vs 2Ch 7:6 Священники стояли в служении своем, и левиты с музыкальными орудиями Господа, которые сделал царь Давид для прославления Господа, ибо вечна милость Его, так как Давид славословил чрез них; священники же трубили перед ним, и весь Израиль стоял.
\vs 2Ch 7:7 Освятил Соломон и внутреннюю часть двора, которая пред домом Господним: ибо принес там всесожжения и тук мирных жертв, так как жертвенник медный, сделанный Соломоном, не мог вмещать всесожжения и хлебного приношения, и туков.
\vs 2Ch 7:8 И сделал Соломон в то время семидневный праздник, и весь Израиль с ним~--- собрание весьма большое, \bibemph{сошедшееся} от входа в Емаф до реки Египетской;
\vs 2Ch 7:9 а в день восьмой сделали попразднство, ибо освящение жертвенника совершали семь дней и праздник семь дней.
\vs 2Ch 7:10 И в двадцать третий день седьмого месяца \bibemph{царь} отпустил народ в шатры их, радующийся и веселящийся в сердце о благе, какое сделал Господь Давиду и Соломону и Израилю, народу Своему.
\rsbpar\vs 2Ch 7:11 И окончил Соломон дом Господень и дом царский; и все, что предположил Соломон в сердце своем сделать в доме Господнем и в доме своем, совершил он успешно.
\vs 2Ch 7:12 И явился Господь Соломону ночью и сказал ему: Я услышал молитву твою и избрал Себе место сие в дом жертвоприношения.
\vs 2Ch 7:13 Если Я заключу небо и не будет дождя, и если повелю саранче поядать землю, или пошлю моровую язву на народ Мой,
\vs 2Ch 7:14 и смирится народ Мой, который именуется именем Моим, и будут молиться, и взыщут лица Моего, и обратятся от худых путей своих, то Я услышу с неба и прощу грехи их и исцелю землю их.
\vs 2Ch 7:15 Ныне очи Мои будут отверсты и уши Мои внимательны к молитве на месте сем.
\vs 2Ch 7:16 И ныне Я избрал и освятил дом сей, чтобы имя Мое было там во веки; и очи Мои и сердце Мое будут там во все дни.
\vs 2Ch 7:17 И если ты будешь ходить пред лицем Моим, как ходил Давид, отец твой, и будешь делать все, что Я повелел тебе, и будешь хранить уставы Мои и законы Мои,
\vs 2Ch 7:18 то утвержу престол царства твоего, как Я обещал Давиду, отцу твоему, говоря: не прекратится у тебя [муж,] владеющий Израилем.
\vs 2Ch 7:19 Если же вы отступите и оставите уставы Мои и заповеди Мои, которые Я дал вам, и пойдете и станете служить богам иным и поклоняться им,
\vs 2Ch 7:20 то Я истреблю \bibemph{Израиля} с лица земли Моей, которую Я дал им, и храм сей, который Я освятил имени Моему, отвергну от лица Моего и сделаю его притчею и посмешищем у всех народов.
\vs 2Ch 7:21 И о храме сем высоком всякий, проходящий мимо него, ужаснется и скажет: за что поступил так Господь с землею сею и с храмом сим?
\vs 2Ch 7:22 И скажут: за то, что они оставили Господа, Бога отцов своих, Который вывел их из земли Египетской, и прилепились к богам иным, и поклонялись им, и служили им,~--- за то Он навел на них все это бедствие.
\vs 2Ch 8:1 По окончании двадцати лет, в которые Соломон строил дом Господень и свой дом,
\vs 2Ch 8:2 Соломон обстроил и города, которые дал Соломону Хирам, и поселил в них сынов Израилевых.
\vs 2Ch 8:3 И пошел Соломон на Емаф-Сува и взял его.
\vs 2Ch 8:4 И построил он Фадмор в пустыне, и все города для запасов, какие основал в Емафе.
\vs 2Ch 8:5 Он обстроил Вефорон верхний и Вефорон нижний, города укрепленные, со стенами, воротами и запорами,
\vs 2Ch 8:6 и Ваалаф и все города для запасов, которые были у Соломона, и все города для колесниц, и города для конных, и все, что хотел Соломон построить в Иерусалиме и на Ливане и во всей земле владения своего.
\rsbpar\vs 2Ch 8:7 Весь народ, оставшийся от Хеттеев, и Аморреев, и Ферезеев, и Евеев и Иевусеев, которые были не из сынов Израилевых,~---
\vs 2Ch 8:8 детей их, оставшихся после них на земле, которых не истребили сыны Израилевы,~--- сделал Соломон оброчными до сего дня.
\vs 2Ch 8:9 Сынов же Израилевых не делал Соломон работниками по делам своим, но они были воинами, и начальниками телохранителей его, и вождями колесниц его и всадников его.
\vs 2Ch 8:10 И было главных приставников у царя Соломона, управлявших народом, двести пятьдесят.
\vs 2Ch 8:11 А дочь Фараонову перевел Соломон из города Давидова в дом, который построил для нее, потому что, говорил он, не должна жить женщина у меня в доме Давида, царя Израилева, ибо свят он, так как вошел в него ковчег Господень.
\rsbpar\vs 2Ch 8:12 Тогда стал возносить Соломон всесожжения Господу на жертвеннике Господнем, который он устроил пред притвором,
\vs 2Ch 8:13 чтобы по уставу каждого дня приносить всесожжения, по заповеди Моисеевой, в субботы, и в новомесячия, и в праздники три раза в год: в праздник опресноков, и в праздник седмиц, и в праздник кущей.
\vs 2Ch 8:14 И установил он, по распоряжению Давида, отца своего, череды священников по службе их и левитов по стражам их, чтобы они славословили и служили при священниках по уставу каждого дня, и привратников по чередам их, к каждым воротам, потому что таково было завещание Давида, человека Божия.
\vs 2Ch 8:15 И не отступали от повелений царя о священниках и левитах ни в чем, ни в отношении сокровищ.
\vs 2Ch 8:16 Так устроено было все дело Соломоново от дня основания дома Господня до совершенного окончания его~--- дома Господня.
\rsbpar\vs 2Ch 8:17 Тогда пошел Соломон в Ецион-Гавер и в Елаф, который на берегу моря, в земле Идумейской.
\vs 2Ch 8:18 И прислал ему Хирам чрез слуг своих корабли и рабов, знающих море, и отправились они с слугами Соломоновыми в Офир, и добыли оттуда четыреста пятьдесят талантов золота, и привезли царю Соломону.
\vs 2Ch 9:1 Царица Савская, услышав о славе Соломона, пришла испытать Соломона загадками в Иерусалим, с весьма большим богатством, и с верблюдами, навьюченными благовониями и множеством золота и драгоценных камней. И пришла к Соломону и беседовала с ним обо всем, что было на сердце у нее.
\vs 2Ch 9:2 И объяснил ей Соломон все слова ее, и не нашлось ничего незнакомого Соломону, чего он не объяснил бы ей.
\vs 2Ch 9:3 И увидела царица Савская мудрость Соломона и дом, который он построил,
\vs 2Ch 9:4 и пищу за столом его, и жилище рабов его, и чинность служащих ему и одежду их, и виночерпиев его и одежду их, и ход, которым он ходил в дом Господень,~--- и была она вне себя.
\vs 2Ch 9:5 И сказала царю: верно то, что я слышала в земле моей о делах твоих и о мудрости твоей,
\vs 2Ch 9:6 но я не верила словам о них, доколе не пришла и не увидела глазами своими. И вот, мне и вполовину не сказано о множестве мудрости твоей: ты превосходишь молву, какую я слышала.
\vs 2Ch 9:7 Блаженны люди твои, и блаженны сии слуги твои, всегда предстоящие пред тобою и слышащие мудрость твою!
\vs 2Ch 9:8 Да будет благословен Господь Бог твой, Который благоволил посадить тебя на престол Свой в царя у Господа Бога твоего. По любви Бога твоего к Израилю, чтоб утвердить его на веки, Он поставил тебя царем над ним~--- творить суд и правду.
\vs 2Ch 9:9 И подарила она царю сто двадцать талантов золота и великое множество благовоний и драгоценных камней; и не бывало таких благовоний, какие подарила царица Савская царю Соломону.
\vs 2Ch 9:10 И слуги Хирамовы и слуги Соломоновы, которые привезли золото из Офира, привезли и красного дерева и драгоценных камней.
\vs 2Ch 9:11 И сделал царь из этого красного дерева лестницы к дому Господню и к дому царскому, и цитры и псалтири для певцов. И не видано было подобного сему прежде в земле Иудейской.
\vs 2Ch 9:12 Царь же Соломон дал царице Савской все, чего она желала и чего она просила, кроме таких вещей, какие она привезла царю. И она отправилась обратно в землю свою, она и слуги ее.
\rsbpar\vs 2Ch 9:13 Весу в золоте, которое приходило к Соломону в один год, \bibemph{было} шестьсот шестьдесят шесть талантов золота.
\vs 2Ch 9:14 Сверх того, послы и купцы приносили, и все цари Аравийские и начальники областные приносили золото и серебро Соломону.
\vs 2Ch 9:15 И сделал царь Соломон двести больших щитов из кованого золота,~--- по шестисот \bibemph{сиклей} кованого золота пошло на каждый щит,~---
\vs 2Ch 9:16 и триста щитов меньших из кованого золота,~--- по триста \bibemph{сиклей} золота пошло на каждый щит; и поставил их царь в доме из Ливанского дерева.
\vs 2Ch 9:17 И сделал царь большой престол из слоновой кости и обложил его чистым золотом,
\vs 2Ch 9:18 и шесть ступеней к престолу и золотое подножие, к престолу приделанное, и локотники по обе стороны у места сидения, и двух львов, стоящих возле локотников,
\vs 2Ch 9:19 и \bibemph{еще} двенадцать львов, стоящих там на шести ступенях, по обе стороны. Не бывало такого [престола] ни в одном царстве.
\vs 2Ch 9:20 И все сосуды для питья у царя Соломона \bibemph{были} из золота, и все сосуды в доме из Ливанского дерева \bibemph{были} из золота отборного; серебро во дни Соломона вменялось ни во что,
\vs 2Ch 9:21 ибо корабли царя ходили в Фарсис с слугами Хирама, и в три года раз возвращались корабли из Фарсиса и привозили золото и серебро, слоновую кость и обезьян и павлинов.
\rsbpar\vs 2Ch 9:22 И превзошел царь Соломон всех царей земли богатством и мудростью.
\vs 2Ch 9:23 И все цари земли искали видеть Соломона, чтобы послушать мудрости его, которую вложил Бог в сердце его.
\vs 2Ch 9:24 И каждый из них подносил от себя в дар сосуды серебряные и сосуды золотые и одежды, оружие и благовония, коней и лошаков, из года в год.
\vs 2Ch 9:25 И было у Соломона четыре тысячи стойл для коней и колесниц и двенадцать тысяч всадников; и он разместил их в городах колесничных и при царе~--- в Иерусалиме;
\vs 2Ch 9:26 и господствовал он над всеми царями, от реки \bibemph{Евфрата} до земли Филистимской и до пределов Египта.
\vs 2Ch 9:27 И сделал царь [золото и] серебро в Иерусалиме равноценным \bibemph{простому} камню, а кедры, по их множеству, сделал равноценными сикоморам, которые на низких местах.
\vs 2Ch 9:28 Коней приводили Соломону из Египта и из всех земель.
\rsbpar\vs 2Ch 9:29 Прочие деяния Соломоновы, первые и последние, описаны в записях Нафана пророка и в пророчестве Ахии Силомлянина и в видениях прозорливца Иоиля о Иеровоаме, сыне Наватовом.
\vs 2Ch 9:30 Царствовал же Соломон в Иерусалиме над всем Израилем сорок лет.
\vs 2Ch 9:31 И почил Соломон с отцами своими, и похоронили его в городе Давида, отца его. И воцарился Ровоам, сын его, вместо него.
\vs 2Ch 10:1 И пошел Ровоам в Сихем, потому что в Сихем сошлись все Израильтяне, чтобы поставить его царем.
\vs 2Ch 10:2 Когда услышал \bibemph{о сем} Иеровоам, сын Наватов,~--- он находился в Египте, куда убежал от царя Соломона,~--- то возвратился Иеровоам из Египта.
\vs 2Ch 10:3 И послали и звали его; и пришел Иеровоам и весь Израиль, и говорили Ровоаму так:
\vs 2Ch 10:4 отец твой наложил на нас тяжкое иго; но ты облегчи жестокую работу отца твоего и тяжкое иго, которое он наложил на нас, и мы будем служить тебе.
\vs 2Ch 10:5 И сказал им \bibemph{Ровоам}: через три дня придите опять ко мне. И разошелся народ.
\vs 2Ch 10:6 И советовался царь Ровоам со старейшинами, которые предстояли пред лицем Соломона, отца его, при жизни его, и говорил: как вы посоветуете отвечать народу сему?
\vs 2Ch 10:7 Они сказали ему: если ты [ныне] будешь добр к народу сему и угодишь им и будешь говорить с ними ласково, то они будут тебе рабами на все дни.
\vs 2Ch 10:8 Но он оставил совет старейшин, который они давали ему, и стал советоваться с людьми молодыми, которые выросли вместе с ним, предстоящими пред лицем его;
\vs 2Ch 10:9 и сказал им: что вы посоветуете мне отвечать народу сему, говорившему мне так: облегчи иго, которое наложил на нас отец твой?
\vs 2Ch 10:10 И говорили ему молодые люди, выросшие вместе с ним, и сказали: так скажи народу, говорившему тебе: отец твой наложил на нас тяжкое иго, а ты облегчи нас,~--- так скажи им: мизинец мой толще чресл отца моего.
\vs 2Ch 10:11 Отец мой наложил на вас тяжкое иго, а я увеличу иго ваше; отец мой наказывал вас бичами, а я [буду бить вас] скорпионами.
\rsbpar\vs 2Ch 10:12 И пришел Иеровоам и весь народ к Ровоаму на третий день, как приказал царь, сказав: придите ко мне опять чрез три дня.
\vs 2Ch 10:13 Тогда царь отвечал им сурово, ибо оставил царь Ровоам совет старейшин, и говорил им по совету молодых людей так:
\vs 2Ch 10:14 отец мой наложил на вас тяжкое иго, а я увеличу его; отец мой наказывал вас бичами, а я [буду бить вас] скорпионами.
\vs 2Ch 10:15 И не послушал царь народа, потому что так устроено было от Бога, чтоб исполнить Господу слово Свое, которое изрек Он чрез Ахию Силомлянина Иеровоаму, сыну Наватову.
\rsbpar\vs 2Ch 10:16 Когда весь Израиль увидел, что не слушает его царь, то отвечал народ царю, говоря: какая нам часть в Давиде? Нет нам доли в сыне Иессеевом; по шатрам своим, Израиль! Теперь знай свой дом, Давид. И разошлись все Израильтяне по шатрам своим.
\vs 2Ch 10:17 Только над сынами Израилевыми, жившими в городах Иудиных, остался царем Ровоам.
\vs 2Ch 10:18 И послал царь Ровоам Адонирама, начальника над собиранием даней, и забросали его сыны Израилевы каменьями, и он умер. Царь же Ровоам поспешил сесть на колесницу, чтобы убежать в Иерусалим.
\vs 2Ch 10:19 Так отложились Израильтяне от дома Давидова до сего дня.
\vs 2Ch 11:1 И прибыл Ровоам в Иерусалим и созвал из дома Иудина и Вениаминова сто восемьдесят тысяч отборных воинов, чтобы воевать с Израилем и возвратить царство Ровоаму.
\vs 2Ch 11:2 И было слово Господне к Самею, человеку Божию, и сказано:
\vs 2Ch 11:3 скажи Ровоаму, сыну Соломонову, царю Иудейскому, и всему Израилю в \bibemph{колене} Иудином и Вениаминовом:
\vs 2Ch 11:4 так говорит Господь: не ходите и не начинайте войн\acc{ы} с братьями вашими; возвратитесь каждый в дом свой, ибо Мною сделано это. Они послушались слов Господних и возвратились из похода против Иеровоама.
\rsbpar\vs 2Ch 11:5 Ровоам жил в Иерусалиме; он обнес города в Иудее стенами.
\vs 2Ch 11:6 Он укрепил Вифлеем и Ефам, и Фекою,
\vs 2Ch 11:7 и Вефцур, и Сохо, и Одоллам,
\vs 2Ch 11:8 и Геф, и Марешу, и Зиф,
\vs 2Ch 11:9 и Адораим, и Лахис, и Азеку,
\vs 2Ch 11:10 и Цору, и Аиалон, и Хеврон, находившиеся в колене Иудином и Вениаминовом.
\vs 2Ch 11:11 И утвердил он крепости сии, и устроил в них начальников и хранилища для хлеба и деревянного масла и вина.
\vs 2Ch 11:12 И \bibemph{дал} в каждый город щиты и копья и утвердил их весьма сильно. И оставались за ним Иуда и Вениамин.
\rsbpar\vs 2Ch 11:13 И священники и левиты, какие \bibemph{были} по всей земле Израильской, собрались к нему из всех пределов,
\vs 2Ch 11:14 ибо оставили левиты свои городские предместья и свои владения и пришли в Иудею и в Иерусалим, так как отставил их Иеровоам и сыновья его от священства Господня
\vs 2Ch 11:15 и поставил у себя жрецов к высотам, и к козлам, и к тельцам, которых он сделал.
\vs 2Ch 11:16 А за ними и из всех колен Израилевых расположившие сердце свое, чтобы взыскать Господа Бога Израилева, приходили в Иерусалим, дабы приносить жертвы Господу Богу отцов своих.
\vs 2Ch 11:17 И укрепили они царство Иудино и поддерживали Ровоама, сына Соломонова, три года, потому что ходили путем Давида и Соломона в сии три года.
\rsbpar\vs 2Ch 11:18 И взял себе Ровоам в жену Махалафу, дочь Иеромофа, сына Давидова, и Авихаиль, дочь Елиава, сына Иессеева,
\vs 2Ch 11:19 и она родила ему сыновей: Иеуса и Шемарию и Загама.
\vs 2Ch 11:20 После нее он взял Мааху, дочь Авессалома, и она родила ему Авию и Аттая, и Зизу и Шеломифа.
\vs 2Ch 11:21 И любил Ровоам Мааху, дочь Авессалома, более всех жен и наложниц своих, ибо он имел восемнадцать жен и шестьдесят наложниц и родил двадцать восемь сыновей и шестьдесят дочерей.
\vs 2Ch 11:22 И поставил Ровоам Авию, сына Маахи, главою [и] князем над братьями его, потому что \bibemph{хотел} воцарить его.
\vs 2Ch 11:23 И действовал благоразумно, и разослал всех сыновей своих по всем землям Иуды и Вениамина во все укрепленные города, и дал им содержание большое и приискал много жен.
\vs 2Ch 12:1 Когда царство Ровоама утвердилось, и он сделался силен, тогда он оставил закон Господень, и весь Израиль с ним.
\rsbpar\vs 2Ch 12:2 На пятом году царствования Ровоама, Сусаким, царь Египетский, пошел на Иерусалим,~--- потому что они отступили от Господа,~---
\vs 2Ch 12:3 с тысячью и двумя стами колесниц и шестьюдесятью тысячами всадников; и не было числа народу, который пришел с ним из Египта, Ливиянам, Сукхитам и Ефиоплянам;
\vs 2Ch 12:4 и взял укрепленные города в Иудее и пришел к Иерусалиму.
\vs 2Ch 12:5 Тогда Самей пророк пришел к Ровоаму и князьям Иудеи, которые собрались в Иерусалим, \bibemph{спасаясь} от Сусакима, и сказал им: так говорит Господь: вы оставили Меня, за то и Я оставляю вас в руки Сусакиму.
\vs 2Ch 12:6 И смирились князья Израилевы и царь и сказали: праведен Господь!
\vs 2Ch 12:7 Когда увидел Господь, что они смирились, тогда было слово Господне к Самею, и сказано: они смирились; не истреблю их и вскоре дам им избавление, и не прольется гнев Мой на Иерусалим рукою Сусакима;
\vs 2Ch 12:8 однако же они будут слугами его, чтобы знали, каково служить Мне и служить царствам земным.
\vs 2Ch 12:9 И пришел Сусаким, царь Египетский, в Иерусалим и взял сокровища дома Господня и сокровища дома царского; всё взял он, взял и щиты золотые, которые сделал Соломон.
\vs 2Ch 12:10 И сделал царь Ровоам, вместо их, щиты медные, и отдал их на руки начальникам телохранителей, охранявших вход дома царского.
\vs 2Ch 12:11 Когда выходил царь в дом Господень, приходили телохранители и несли их, и потом опять относили их в палату телохранителей.
\vs 2Ch 12:12 И когда он смирился, тогда отвратился от него гнев Господа и не погубил его до конца; притом и в Иудее было нечто доброе.
\rsbpar\vs 2Ch 12:13 И утвердился царь Ровоам в Иерусалиме и царствовал. Сорок один год было Ровоаму, когда он воцарился, и семнадцать лет царствовал в Иерусалиме, в городе, который из всех колен Израилевых избрал Господь, чтобы там пребывало имя Его. Имя матери его Наама, Аммонитянка.
\vs 2Ch 12:14 И делал он зло, потому что не расположил сердца своего к тому, чтобы взыскать Господа.
\rsbpar\vs 2Ch 12:15 Деяния Ровоамовы, первые и последние, описаны в записях Самея пророка и Адды прозорливца при родословиях. И были войны у Ровоама с Иеровоамом во все дни.
\vs 2Ch 12:16 И почил Ровоам с отцами своими и погребен в городе Давидовом. И воцарился Авия, сын его, вместо него.
\vs 2Ch 13:1 В восемнадцатый год царствования Иеровоама воцарился Авия над Иудою.
\vs 2Ch 13:2 Три года он царствовал в Иерусалиме; имя матери его Михаия, дочь Уриилова, из Гивы. И была война у Авии с Иеровоамом.
\vs 2Ch 13:3 И вывел Авия на войну войско, состоявшее из людей храбрых, из четырехсот тысяч человек отборных; а Иеровоам выступил против него на войну с восемью стами тысяч человек, \bibemph{также} отборных, храбрых.
\vs 2Ch 13:4 И стал Авия на вершине горы Цемараимской, одной из гор Ефремовых, и говорил: послушайте меня, Иеровоам и все Израильтяне!
\vs 2Ch 13:5 Не знаете ли вы, что Господь Бог Израилев дал царство Давиду над Израилем навек, ему и сыновьям его, по завету соли [вечному]?
\vs 2Ch 13:6 Но восстал Иеровоам, сын Наватов, раб Соломона, сына Давидова, и возмутился против господина своего.
\vs 2Ch 13:7 И собрались вокруг него люди пустые, люди развращенные, и укрепились против Ровоама, сына Соломонова; Ровоам же был молод и слаб сердцем и не устоял против них.
\vs 2Ch 13:8 И ныне вы думаете устоять против царства Господня в руке сынов Давидовых, \bibemph{потому что} вас великое множество, и у вас золотые тельцы, которых Иеровоам сделал вам богами.
\vs 2Ch 13:9 Не вы ли изгнали священников Господних, сынов Аарона, и левитов, и поставили у себя священников, какие у народов \bibemph{других} земель? Всякий, кто приходит для посвящения своего с тельцом и с семью овнами, делается \bibemph{у вас} священником лжебогов.
\vs 2Ch 13:10 А у нас~--- Господь Бог наш; мы не оставляли Его, и Господу служат священники, сыны Аароновы, и левиты при \bibemph{своем} деле.
\vs 2Ch 13:11 И сожигают они Господу всесожжения каждое утро и каждый вечер, и благовонное курение, и полагают рядами хлебы на столе чистом, и \bibemph{зажигают} золотой светильник и лампады его, чтобы горели каждый вечер, потому что мы соблюдаем установление Господа Бога нашего, а вы оставили Его.
\vs 2Ch 13:12 И вот, у нас во главе Бог, и священники Его, и трубы громогласные, чтобы греметь против вас. Сыны Израилевы! не воюйте с Господом Богом отцов ваших, ибо не получите успеха.
\rsbpar\vs 2Ch 13:13 Между тем Иеровоам послал отряд в засаду с тыла им, так что \bibemph{сам он} был впереди Иудеев, а засада позади их.
\vs 2Ch 13:14 И оглянулись Иудеи, и вот, им битва спереди и сзади; и возопили они к Господу, а священники затрубили трубами.
\vs 2Ch 13:15 И воскликнули Иудеи. И когда воскликнули Иудеи, Бог поразил Иеровоама и всех Израильтян пред лицем Авии и Иуды.
\vs 2Ch 13:16 И побежали сыны Израилевы от Иудеев, и предал их Бог в руки им.
\vs 2Ch 13:17 И произвели у них Авия и народ его поражение сильное; и пало убитых у Израиля пятьсот тысяч человек отборных.
\vs 2Ch 13:18 И смирились тогда сыны Израилевы, и были сильны сыны Иудины, потому что уповали на Господа Бога отцов своих.
\vs 2Ch 13:19 И преследовал Авия Иеровоама и взял у него города: Вефиль и зависящие от него города, и Иешану и зависящие от нее города, и Ефрон и зависящие от него города.
\vs 2Ch 13:20 И не входил уже в силу Иеровоам во дни Авии. И поразил его Господь, и он умер.
\vs 2Ch 13:21 Авия же усилился; и взял себе четырнадцать жен и родил двадцать два сына и шестнадцать дочерей.
\rsbpar\vs 2Ch 13:22 Прочие деяния Авии и его поступки и слова описаны в сказании пророка Адды.
\vs 2Ch 14:1 И почил Авия с отцами своими, и похоронили его в городе Давидовом. И воцарился Аса, сын его, вместо него. Во дни его покоилась земля десять лет.
\vs 2Ch 14:2 И делал Аса доброе и угодное в очах Господа Бога своего:
\vs 2Ch 14:3 и отверг он жертвенники \bibemph{богов} чужих и высоты, и разбил статуи, и вырубил \bibemph{посвященные} дерева;
\vs 2Ch 14:4 и повелел Иудеям взыскать Господа Бога отцов своих, и исполнять закон [Его] и заповеди;
\vs 2Ch 14:5 и отменил он во всех городах Иудиных высоты и статуи солнца. И спокойно было при нем царство.
\vs 2Ch 14:6 И построил он укрепленные города в Иудее, ибо спокойна была земля, и не было у него войны в те годы, так как Господь дал покой ему.
\vs 2Ch 14:7 И сказал он Иудеям: построим города сии и обнесем их стенами с башнями, с воротами и запорами. Земля еще наша, потому что мы взыскали Господа Бога нашего: мы взыскали Его,~--- и Он дал нам покой со всех сторон. И стали строить, и имели успех.
\vs 2Ch 14:8 И было у Асы военной силы: вооруженных щитом и копьем из \bibemph{колена} Иудина триста тысяч, и из \bibemph{колена} Вениаминова вооруженных щитом и стрелявших из лука двести восемьдесят тысяч, людей храбрых.
\rsbpar\vs 2Ch 14:9 И вышел на них Зарай Ефиоплянин с войском в тысячу тысяч и с тремя стами колесниц и дошел до Мареши.
\vs 2Ch 14:10 И выступил Аса против него, и построились к сражению на долине Цефата у Мареши.
\vs 2Ch 14:11 И воззвал Аса к Господу Богу своему, и сказал: Господи! не в Твоей ли силе помочь сильному или бессильному? помоги же нам, Господи Боже наш: ибо мы на Тебя уповаем и во имя Твое вышли мы против множества сего. Господи! Ты Бог наш: да не превозможет Тебя человек.
\vs 2Ch 14:12 И поразил Господь Ефиоплян пред лицем Асы и пред лицем Иуды, и побежали Ефиопляне.
\vs 2Ch 14:13 И преследовал их Аса и народ, бывший с ним, до Герара, и пали Ефиопляне, так что у них никого \bibemph{не осталось} в живых, потому что они поражены были пред Господом и пред воинством Его. И набрали добычи великое множество.
\vs 2Ch 14:14 И разрушили все города вокруг Герара, потому что напал на них ужас от Господа; и разграбили все города и вынесли из них весьма много добычи.
\vs 2Ch 14:15 Также и пастушеские шалаши разорили и угнали множество стад мелкого скота и верблюдов и возвратились в Иерусалим.
\vs 2Ch 15:1 Тогда на Азарию, сына Одедова, сошел Дух Божий,
\vs 2Ch 15:2 и вышел он навстречу Асе и сказал ему: послушайте меня, Аса и весь Иуда и Вениамин: Господь с вами, когда вы с Ним; и если будете искать Его, Он будет найден вами; если же оставите Его, Он оставит вас.
\vs 2Ch 15:3 Многие дни Израиль \bibemph{будет} без Бога истинного, и без священника учащего, и без закона;
\rsbpar\vs 2Ch 15:4 но когда он обратится в тесноте своей к Господу Богу Израилеву и взыщет Его, Он даст им найти Себя.
\vs 2Ch 15:5 В те времена не будет мира ни выходящему, ни входящему, ибо великие волнения будут у всех жителей земель;
\vs 2Ch 15:6 народ будет сражаться с народом, и город с городом, потому что Бог приведет их в смятение всякими бедствиями.
\vs 2Ch 15:7 Но вы укрепитесь, и пусть не ослабевают руки ваши, потому что есть возмездие за дела ваши.
\rsbpar\vs 2Ch 15:8 Когда услышал Аса слова сии и пророчество [Азарии], \bibemph{сына} Одеда пророка, то ободрился и изверг мерзости \bibemph{языческие} из всей земли Иудиной и Вениаминовой и из городов, которые он взял на горе Ефремовой, и обновил жертвенник Господень, который пред притвором Господним.
\vs 2Ch 15:9 И собрал всего Иуду и Вениамина и живущих с ними переселенцев от Ефрема и Манассии и Симеона; ибо многие от Израиля перешли к нему, когда увидели, что Господь, Бог его, с ним.
\vs 2Ch 15:10 И собрались в Иерусалим в третий месяц, в пятнадцатый год царствования Асы;
\vs 2Ch 15:11 и принесли в день тот жертву Господу из добычи, которую привели, из крупного скота семьсот и из мелкого семь тысяч;
\vs 2Ch 15:12 и вступили в завет, чтобы взыскать Господа Бога отцов своих от всего сердца своего и от всей души своей;
\vs 2Ch 15:13 а всякий, кто не станет искать Господа Бога Израилева, должен умереть, малый ли он или большой, мужчина ли или женщина.
\vs 2Ch 15:14 И клялись Господу громогласно и с восклицанием и при \bibemph{звуке} труб и рогов.
\vs 2Ch 15:15 И радовались все Иудеи сей клятве, потому что от всего сердца своего клялись и со всем усердием взыскали Его, и Он дал им найти Себя. И дал им Господь покой со всех сторон.
\vs 2Ch 15:16 И Мааху, мать свою, царь Аса лишил царского достоинства за то, что она сделала истукан для дубравы. И ниспроверг Аса истукан ее, и изрубил в куски, и сжег на долине Кедрона.
\vs 2Ch 15:17 Хотя высоты не были отменены у Израиля, но сердце Асы было вполне предано \bibemph{Господу} во все дни его.
\vs 2Ch 15:18 И внес он посвященное отцом его и свое посвящение в дом Божий, серебро и золото и сосуды.
\vs 2Ch 15:19 И не было войны до тридцать пятого года царствования Асы.
\vs 2Ch 16:1 В тридцать шестой год царствования Асы, пошел Вааса, царь Израильский, на Иудею и начал строить Раму, чтобы не позволить \bibemph{никому} ни уходить от Асы, царя Иудейского, ни приходить \bibemph{к нему}.
\vs 2Ch 16:2 И вынес Аса серебро и золото из сокровищниц дома Господня и дома царского и послал к Венададу, царю Сирийскому, жившему в Дамаске, говоря:
\vs 2Ch 16:3 союз да будет между мною и тобою, как был между отцом моим и отцом твоим; вот, я посылаю тебе серебра и золота: пойди, расторгни союз твой с Ваасою, царем Израильским, чтоб он отступил от меня.
\vs 2Ch 16:4 И послушался Венадад царя Асы и послал военачальников, которые \bibemph{были} у него, против городов Израильских, и они опустошили Ийон и Дан и Авелмаим и все запасы в городах Неффалимовых.
\vs 2Ch 16:5 И когда услышал \bibemph{о сем} Вааса, то перестал строить Раму и прекратил работу свою.
\vs 2Ch 16:6 Аса же царь собрал всех Иудеев, и они вывезли \bibemph{из} Рамы камни и дерева, которые употреблял Вааса для строения,~--- и выстроил из них Геву и Мицфу.
\rsbpar\vs 2Ch 16:7 В то время пришел Ананий прозорливец к Асе, царю Иудейскому, и сказал ему: так как ты понадеялся на царя Сирийского и не уповал на Господа Бога твоего, потому и спаслось войско царя Сирийского от руки твоей.
\vs 2Ch 16:8 Не были ли Ефиопляне и Ливияне с силою большею и с колесницами и всадниками весьма многочисленными? Но как ты уповал на Господа, то Он предал их в руку твою,
\vs 2Ch 16:9 ибо очи Господа обозревают всю землю, чтобы поддерживать тех, \bibemph{чье} сердце вполне предано Ему. Безрассудно ты поступил теперь. За то отныне будут у тебя войны.
\vs 2Ch 16:10 И разгневался Аса на прозорливца, и заключил его в темницу, так как за это был в раздражении на него; притеснял Аса и \bibemph{некоторых} из народа в то время.
\rsbpar\vs 2Ch 16:11 И вот, деяния Асы, первые и последние, описаны в книге царей Иудейских и Израильских.
\vs 2Ch 16:12 И сделался Аса болен ногами на тридцать девятом году царствования своего, и болезнь его поднялась до верхних частей тела; но он в болезни своей взыскал не Господа, а врачей.
\vs 2Ch 16:13 И почил Аса с отцами своими, и умер на сорок первом году царствования своего.
\vs 2Ch 16:14 И похоронили его в гробнице, которую он устроил для себя в городе Давидовом; и положили его на одре, который наполнили благовониями и разными искусственными мастями, и сожгли их для него великое множество.
\vs 2Ch 17:1 И воцарился Иосафат, сын его, вместо него; и укрепился он против Израильтян.
\vs 2Ch 17:2 И поставил он войско во все укрепленные города Иудеи и расставил охранное войско по земле Иудейской и по городам Ефремовым, которыми овладел Аса, отец его.
\vs 2Ch 17:3 И был Господь с Иосафатом, потому что он ходил первыми путями Давида, отца своего, и не взыскал Ваалов,
\vs 2Ch 17:4 но взыскал он Бога отца своего и поступал по заповедям Его, а не по деяниям Израильтян.
\vs 2Ch 17:5 И утвердил Господь царство в руке его, и давали все Иудеи дары Иосафату, и было у него много богатства и славы.
\vs 2Ch 17:6 И возвысилось сердце его на путях Господних; притом и высоты отменил он и дубравы в Иудее.
\rsbpar\vs 2Ch 17:7 И в третий год царствования своего он послал князей своих Бенхаила и Овадию, и Захарию и Нафанаила и Михея, чтоб учили по городам Иудиным народ,
\vs 2Ch 17:8 и с ними левитов: Шемаию и Нефанию, и Зевадию и Азаила, и Шемирамофа и Ионафана, и Адонию и Товию и Тов-Адонию, и с ними Елишаму и Иорама, священников.
\vs 2Ch 17:9 И они учили в Иудее, имея с собою книгу закона Господня; и обходили все города Иудеи и учили народ.
\vs 2Ch 17:10 И был страх Господень на всех царствах земель, которые вокруг Иудеи, и не воевали с Иосафатом.
\vs 2Ch 17:11 А от Филистимлян приносили Иосафату дары и в дань серебро; также Аравитяне пригоняли к нему мелкий скот: овнов семь тысяч семьсот и козлов семь тысяч семьсот.
\rsbpar\vs 2Ch 17:12 И возвышался Иосафат все более и более и построил в Иудее крепости и города для запасов.
\vs 2Ch 17:13 Много было у него запасов в городах Иудейских, а в Иерусалиме людей военных, храбрых.
\vs 2Ch 17:14 И вот список их по поколениям их: у Иуды начальники тысяч: Адна начальник, и у него отличных воинов триста тысяч;
\vs 2Ch 17:15 за ним Иоханан начальник, и у него двести восемьдесят тысяч;
\vs 2Ch 17:16 за ним Амасия, сын Зихри, посвятивший себя Господу, и у него двести тысяч воинов отличных.
\vs 2Ch 17:17 У Вениамина: отличный воин Елиада, и у него вооруженных луком и щитом двести тысяч;
\vs 2Ch 17:18 за ним Иегозавад, и у него сто восемьдесят тысяч вооруженных воинов.
\vs 2Ch 17:19 Вот служившие царю, сверх тех, которых расставил царь в укрепленных городах по всей Иудее.
\vs 2Ch 18:1 И было у Иосафата много богатства и славы; и породнился он с Ахавом.
\vs 2Ch 18:2 И пошел чрез несколько лет к Ахаву в Самарию; и заколол для него Ахав множество скота мелкого и крупного, и для людей, бывших с ним, и склонял его идти на Рамоф Галаадский.
\vs 2Ch 18:3 И говорил Ахав, царь Израильский, Иосафату, царю Иудейскому: пойдешь ли со мною в Рамоф Галаадский? Тот сказал ему: как ты, так и я, как твой народ, так и мой народ: \bibemph{иду} с тобою на войну!
\vs 2Ch 18:4 И сказал Иосафат царю Израильскому: вопроси сегодня, что скажет Господь.
\vs 2Ch 18:5 И собрал царь Израильский пророков четыреста человек и сказал им: идти ли нам на Рамоф Галаадский войною, или удержаться? Они сказали: иди, и Бог предаст \bibemph{его} в руку царя.
\vs 2Ch 18:6 И сказал Иосафат: нет ли здесь еще пророка Господня? спросим и у него.
\vs 2Ch 18:7 И сказал царь Израильский Иосафату: есть еще один человек, чрез которого можно вопросить Господа; но я не люблю его, потому что он не пророчествует обо мне доброго, а постоянно пророчествует худое; это Михей, сын Иемвлая. И сказал Иосафат: не говори так, царь.
\vs 2Ch 18:8 И позвал царь Израильский одного евнуха, и сказал: сходи поскорее за Михеем, сыном Иемвлая.
\vs 2Ch 18:9 Царь же Израильский и Иосафат, царь Иудейский, сидели каждый на своем престоле, одетые в \bibemph{царские} одежды; сидели на площади у ворот Самарии, и все пророки пророчествовали пред ними.
\vs 2Ch 18:10 И сделал себе Седекия, сын Хенааны, железные рога и сказал: так говорит Господь: сими избодешь Сириян до истребления их.
\vs 2Ch 18:11 И все пророки пророчествовали то же, говоря: иди на Рамоф Галаадский; будет успех тебе, и предаст \bibemph{его} Господь в руку царя.
\rsbpar\vs 2Ch 18:12 Посланный, который пошел позвать Михея, говорил ему: вот, пророки единогласно предрекают доброе царю; пусть бы и твое слово было такое же, как каждого из них: изреки и ты доброе.
\vs 2Ch 18:13 И сказал Михей: жив Господь,~--- что скажет мне Бог мой, то изреку я.
\vs 2Ch 18:14 И пришел он к царю, и сказал ему царь: Михей, идти ли нам войной на Рамоф Галаадский, или удержаться? И сказал тот: идите, будет вам успех, и они преданы будут в руки ваши.
\vs 2Ch 18:15 И сказал ему царь: сколько раз мне заклинать тебя, чтобы ты не говорил мне ничего, кроме истины, во имя Господне?
\vs 2Ch 18:16 Тогда \bibemph{Михей} сказал: я видел всех сынов Израиля, рассеянных по горам, как овец, у которых нет пастыря,~--- и сказал Господь: нет у них начальника, пусть возвратятся каждый в дом свой с миром.
\vs 2Ch 18:17 И сказал царь Израильский Иосафату: не говорил ли я тебе, что он не пророчествует о мне доброго, а только худое?
\vs 2Ch 18:18 И сказал \bibemph{Михей}: так выслушайте слово Господне: я видел Господа, сидящего на престоле Своем, и все воинство небесное стояло по правую и по левую руку Его.
\vs 2Ch 18:19 И сказал Господь: кто увлек бы Ахава, царя Израильского, чтобы он пошел и пал в Рамофе Галаадском? И один говорил так, другой говорил иначе.
\vs 2Ch 18:20 И выступил один дух, и стал пред лицем Господа, и сказал: я увлеку его. И сказал ему Господь: чем?
\vs 2Ch 18:21 Тот сказал: я выйду, и буду духом лжи в устах всех пророков его. И сказал Он: ты увлечешь его, и успеешь; пойди и сделай так.
\vs 2Ch 18:22 И теперь, вот попустил Господь духу лжи \bibemph{войти} в уста сих пророков твоих, но Господь изрек о тебе недоброе.
\vs 2Ch 18:23 И подошел Седекия, сын Хенааны, и ударил Михея по щеке, и сказал: по какой это дороге отошел от меня Дух Господень, чтобы говорить в тебе?
\vs 2Ch 18:24 И сказал Михей: вот, ты увидишь \bibemph{это} в тот день, когда будешь бегать из комнаты в комнату, чтобы укрыться.
\vs 2Ch 18:25 И сказал царь Израильский: возьмите Михея и отведите его к Амону градоначальнику и к Иоасу, сыну царя,
\vs 2Ch 18:26 и скажите: так говорит царь: посад\acc{и}те этого в темницу и кормите его хлебом и водою скудно, доколе я не возвращусь в мире.
\vs 2Ch 18:27 И сказал Михей: если ты возвратишься в мире, то не Господь говорил чрез меня. И сказал: слушайте \bibemph{это}, все люди!
\rsbpar\vs 2Ch 18:28 И пошел царь Израильский и Иосафат, царь Иудейский, к Рамофу Галаадскому.
\vs 2Ch 18:29 И сказал царь Израильский Иосафату: я переоденусь и вступлю в сражение, а ты надень свои \bibemph{царские} одежды. И переоделся царь Израильский, и вступили в сражение.
\vs 2Ch 18:30 И царь Сирийский повелел начальникам колесниц, бывших у него, сказав: не сражайтесь ни с малым, ни с великим, а только с одним царем Израильским.
\vs 2Ch 18:31 И когда увидели Иосафата начальники колесниц, то подумали: это царь Израильский,~--- и окружили его, чтобы сразиться с ним. Но Иосафат закричал, и Господь помог ему, и отвел их Бог от него.
\vs 2Ch 18:32 И когда увидели начальники колесниц, что \bibemph{это} не был царь Израильский, то поворотили от него.
\vs 2Ch 18:33 Между тем один человек случайно натянул лук свой, и ранил царя Израильского сквозь швы лат. И сказал он вознице: повороти назад, и вези меня от войска, ибо я ранен.
\vs 2Ch 18:34 Но сражение в тот день усилилось; и царь Израильский стоял на колеснице напротив Сириян до вечера и умер на закате солнца.
\vs 2Ch 19:1 И возвращался Иосафат, царь Иудейский, в мире в дом свой в Иерусалим.
\vs 2Ch 19:2 И выступил навстречу ему Ииуй, сын Анании, прозорливец, и сказал царю Иосафату: \bibemph{следовало} ли тебе помогать нечестивцу и любить ненавидящих Господа? За это на тебя гнев от лица Господня.
\vs 2Ch 19:3 Впрочем и доброе найдено в тебе, потому что ты истребил кумиры в земле [Иудейской] и расположил сердце свое к тому, чтобы взыскать Бога.
\rsbpar\vs 2Ch 19:4 И жил Иосафат в Иерусалиме. И опять стал он обходить народ \bibemph{свой} от Вирсавии до горы Ефремовой, и обращал их к Господу, Богу отцов их.
\vs 2Ch 19:5 И поставил судей на земле по всем укрепленным городам Иудеи в каждом городе,
\vs 2Ch 19:6 и сказал судьям: смотрите, что вы делаете, вы творите не суд человеческий, но суд Господа; и \bibemph{Он} с вами в деле суда.
\vs 2Ch 19:7 Итак да будет страх Господень на вас: действуйте осмотрительно, ибо нет у Господа Бога нашего неправды, ни лицеприятия, ни мздоимства.
\vs 2Ch 19:8 И в Иерусалиме приставил Иосафат \bibemph{некоторых} из левитов и священников и глав поколений у Израиля~--- к суду Господню и к тяжбам. И возвратились в Иерусалим.
\vs 2Ch 19:9 И дал им повеление, говоря: так действуйте в страхе Господнем, с верностью и с чистым сердцем:
\vs 2Ch 19:10 во всяком деле спорном, какое поступит к вам от братьев ваших, живущих в городах своих, о кровопролитии ли, или о законе, заповеди, уставах и обрядах, наставляйте их, чтобы они не провинились пред Господом, и не было бы гнева \bibemph{Его} на вас и на братьев ваших; так действуйте,~--- и вы не погрешите.
\vs 2Ch 19:11 И вот Амария первосвященник, над вами во всяком деле Господнем, а Зевадия, сын Исмаилов, князь дома Иудина, во всяком деле царя, и надзиратели левиты пред вами. Будьте тверды и действуйте, и будет Господь с добрым.
\vs 2Ch 20:1 После сего Моавитяне и Аммонитяне, а с ними некоторые из страны Маонитской, пошли войною на Иосафата.
\vs 2Ch 20:2 И пришли, и донесли Иосафату, говоря: идет на тебя множество великое из-за моря, от Сирии, и вот они в Хацацон-Фамаре, то есть в Енгедди.
\vs 2Ch 20:3 И убоялся Иосафат, и обратил лице свое взыскать Господа, и объявил пост по всей Иудее.
\vs 2Ch 20:4 И собрались Иудеи просить \bibemph{помощи} у Господа; из всех городов Иудиных пришли они умолять Господа.
\rsbpar\vs 2Ch 20:5 И стал Иосафат в собрании Иудеев и Иерусалимлян в доме Господнем, пред новым двором,
\vs 2Ch 20:6 и сказал: Господи Боже отцов наших! Не Ты ли Бог на небе? И Ты владычествуешь над всеми царствами народов, и в Твоей руке сила и крепость, и никто не устоит против Тебя!
\vs 2Ch 20:7 Не Ты ли, Боже наш, изгнал жителей земли сей пред лицем народа Твоего Израиля и отдал ее семени Авраама, друга Твоего, навек?
\vs 2Ch 20:8 И они поселились на ней и построили Тебе на ней святилище во имя Твое, говоря:
\vs 2Ch 20:9 если придет на нас бедствие: меч наказующий, или язва, или голод, то мы станем пред домом сим и пред лицем Твоим, ибо имя Твое в доме сем; и воззовем к Тебе в тесноте нашей, и Ты услышишь и спасешь.
\vs 2Ch 20:10 И ныне вот Аммонитяне и Моавитяне и \bibemph{обитатели} горы Сеира, чрез земли которых Ты не позволил пройти Израильтянам, когда они шли из земли Египетской, а потому они миновали их и не истребили их,~---
\vs 2Ch 20:11 вот они платят нам \bibemph{тем}, что пришли выгнать нас из наследственного владения Твоего, которое Ты отдал нам.
\vs 2Ch 20:12 Боже наш! Ты суди их. Ибо нет в нас силы против множества сего великого, пришедшего на нас, и мы не знаем, чт\acc{о} делать, но к Тебе очи наши!
\vs 2Ch 20:13 И все Иудеи стояли пред лицем Господним, и малые дети их, жены их и сыновья их.
\rsbpar\vs 2Ch 20:14 Тогда на Иозиила, сына Захарии, сына Ванеи, сына Иеиела, сына Матфании, левита из сынов Асафовых, сошел Дух Господень среди собрания
\vs 2Ch 20:15 и сказал он: слушайте, все Иудеи и жители Иерусалима и царь Иосафат! Так говорит Господь к вам: не бойтесь и не ужасайтесь множества сего великого, ибо не ваша война, а Божия.
\vs 2Ch 20:16 Завтра выступите против них: вот они всходят на возвышенность Циц, и вы найдете их на конце долины, пред пустынею Иеруилом.
\vs 2Ch 20:17 Не вам сражаться на сей раз; вы станьте, стойте и смотрите на спасение Господне, \bibemph{посылаемое} вам. Иуда и Иерусалим! не бойтесь и не ужасайтесь. Завтра выступите навстречу им, и Господь будет с вами.
\vs 2Ch 20:18 И преклонился Иосафат лицем до земли, и все Иудеи и жители Иерусалима пали пред Господом, чтобы поклониться Господу.
\vs 2Ch 20:19 И встали левиты из сынов Каафовых и из сынов Кореевых~--- хвалить Господа Бога Израилева, голосом весьма громким.
\rsbpar\vs 2Ch 20:20 И встали они рано утром, и выступили к пустыне Фекойской; и когда они выступили, стал Иосафат и сказал: послушайте меня, Иудеи и жители Иерусалима! Верьте Господу Богу вашему, и будете тверды; верьте пророкам Его, и будет успех вам.
\vs 2Ch 20:21 И совещался он с народом, и поставил певцов Господу, чтобы они в благолепии святыни, выступая впереди вооруженных, славословили и говорили: славьте Господа, ибо вовек милость Его!
\vs 2Ch 20:22 И в то время, \bibemph{как} они стали восклицать и славословить, Господь возбудил несогласие между Аммонитянами, Моавитянами и \bibemph{обитателями} горы Сеира, пришедшими на Иудею, и были они поражены:
\vs 2Ch 20:23 ибо восстали Аммонитяне и Моавитяне на обитателей горы Сеира, побивая и истребляя \bibemph{их}, а когда покончили с жителями Сеира, тогда стали истреблять друг друга.
\vs 2Ch 20:24 И когда Иудеи пришли на возвышенность к пустыне и взглянули на то многолюдство, и вот~--- трупы, лежащие на земле, и нет уцелевшего.
\vs 2Ch 20:25 И пришел Иосафат и народ его забирать добычу, и нашли у них во множестве и имущество, и одежды, и драгоценные вещи, и набрали себе столько, что не \bibemph{могли} нести. И три дня они забирали добычу; так велика \bibemph{была} она!
\rsbpar\vs 2Ch 20:26 А в четвертый день собрались на долину благословения, так как там они благословили Господа. Посему и называют то место долиною благословения до сего дня.
\vs 2Ch 20:27 И пошли назад все Иудеи и Иерусалимляне и Иосафат во главе их, чтобы возвратиться в Иерусалим с веселием, потому что дал им Господь торжество над врагами их.
\vs 2Ch 20:28 И пришли в Иерусалим с псалтирями, и цитрами, и трубами, к дому Господню.
\vs 2Ch 20:29 И был страх Божий на всех царствах земных, когда они услышали, что \bibemph{Сам} Господь воевал против врагов Израиля.
\vs 2Ch 20:30 И спокойно стало царство Иосафатово, и дал ему Бог его покой со всех сторон.
\rsbpar\vs 2Ch 20:31 Так царствовал Иосафат над Иудеею: тридцати пяти лет он \bibemph{был}, когда воцарился, и двадцать пять лет царствовал в Иерусалиме. Имя матери его Азува, дочь Салаила.
\vs 2Ch 20:32 И ходил он путем отца своего Асы и не уклонился от него, делая угодное в очах Господних.
\vs 2Ch 20:33 Только высоты не были отменены, и народ еще не обратил твердо сердца своего к Богу отцов своих.
\rsbpar\vs 2Ch 20:34 Прочие деяния Иосафата, первые и последние, описаны в записях Ииуя, сына Ананиева, которые внесены в книгу царей Израилевых.
\rsbpar\vs 2Ch 20:35 Но после того вступил Иосафат, царь Иудейский в общение с Охозиею, царем Израильским, который поступал беззаконно,
\vs 2Ch 20:36 и соединился с ним, чтобы построить корабли для отправления в Фарсис; и построили они корабли в Ецион-Гавере.
\vs 2Ch 20:37 И изрек \bibemph{тогда} Елиезер, сын Додавы из Мареши, пророчество на Иосафата, говоря: так как ты вступил в общение с Охозиею, то разрушил Господь дело твое.~--- И разбились корабли, и не могли идти в Фарсис.
\vs 2Ch 21:1 И почил Иосафат с отцами своими, и похоронен с отцами своими в городе Давидовом. И воцарился Иорам, сын его, вместо него.
\vs 2Ch 21:2 И у него \bibemph{были} братья, сыновья Иосафата: Азария и Иехиил, и Захария и Азария, и Михаил и Сафатия: все сии сыновья Иосафата, царя Израилева.
\vs 2Ch 21:3 И дал им отец их большие подарки серебром и золотом и драгоценностями, вместе с укрепленными городами в Иудее; царство же отдал Иораму, потому что он первенец.
\vs 2Ch 21:4 И вступил Иорам на царство отца своего и утвердился, и умертвил всех братьев своих мечом и также \bibemph{некоторых} из князей Израилевых.
\rsbpar\vs 2Ch 21:5 Тридцати двух лет \bibemph{был} Иорам, когда воцарился, и восемь лет царствовал в Иерусалиме;
\vs 2Ch 21:6 и ходил он путем царей Израильских, как поступал дом Ахавов, потому что дочь Ахава была женою его,~--- и делал он неугодное в очах Господних.
\vs 2Ch 21:7 Однако же не хотел Господь погубить дома Давидова ради завета, который заключил с Давидом, и потому что обещал дать ему светильник и сыновьям его на все времена.
\rsbpar\vs 2Ch 21:8 Во дни его вышел Едом из-под власти Иуды, и поставили над собою царя.
\vs 2Ch 21:9 И пошел Иорам с военачальниками своими, и все колесницы с ним; и встав ночью, поразил Идумеян, которые окружили его, и начальствующих над колесницами [и побежал народ в жилища свои].
\vs 2Ch 21:10 Однако вышел Едом из-под власти Иуды до сего дня. В то же время вышла и Ливна из-под власти его, потому что он оставил Господа Бога отцов своих.
\vs 2Ch 21:11 Также высоты устроил он на горах Иудейских, и ввел в блужение жителей Иерусалима и соблазнил Иудею.
\rsbpar\vs 2Ch 21:12 И пришло к нему письмо от Илии пророка, в котором было сказано: так говорит Господь Бог Давида, отца твоего: за то, что ты не пошел путями Иосафата, отца твоего, и путями Асы, царя Иудейского,
\vs 2Ch 21:13 а пошел путем царей Израильских и ввел в блужение Иудею и жителей Иерусалима, как вводил в блужение дом Ахавов, еще же и братьев твоих, дом отца твоего, которые лучше тебя, ты умертвил,
\vs 2Ch 21:14 \bibemph{за то}, вот Господь поразит поражением великим народ твой и сыновей твоих, и жен твоих, и все имущество твое,
\vs 2Ch 21:15 тебя же \bibemph{самого}~--- болезнью сильною, болезнью внутренностей твоих до того, что будут выпадать внутренности твои от болезни со дня на день.
\rsbpar\vs 2Ch 21:16 И возбудил Господь против Иорама дух Филистимлян и Аравитян, сопредельных Ефиоплянам;
\vs 2Ch 21:17 и они пошли на Иудею и ворвались в нее, и захватили все имущество, находившееся в доме царя, также и сыновей его и жен его; и не осталось у него сына, кроме Охозии, меньшего из сыновей его.
\vs 2Ch 21:18 А после всего этого поразил Господь внутренности его болезнью неизлечимою.
\vs 2Ch 21:19 Так было со дня на день, а к концу второго года выпали внутренности его от болезни его, и он умер в жестоких страданиях; и не сожег для него народ его \bibemph{благовоний}, как делал то для отцов его.
\rsbpar\vs 2Ch 21:20 Тридцати двух \bibemph{лет} был он, когда воцарился, и восемь лет царствовал в Иерусалиме, и отошел неоплаканный, и похоронили его в городе Давидовом, но не в царских гробницах.
\vs 2Ch 22:1 И поставили царем жители Иерусалима Охозию, меньшего сына его, вместо него, так как всех старших избило полчище, приходившее с Аравитянами к стану,~--- и воцарился Охозия, сын Иорама, царя Иудейского.
\rsbpar\vs 2Ch 22:2 Двадцати двух лет \bibemph{был} Охозия, когда воцарился, и один год царствовал в Иерусалиме; имя матери его Гофолия, дочь Амврия.
\vs 2Ch 22:3 Он также ходил путями дома Ахавова, потому что мать его была советницею ему на беззаконные дела.
\vs 2Ch 22:4 И делал он неугодное в очах Господних, подобно дому Ахавову, потому что он был ему советником, по смерти отца его, на погибель ему.
\vs 2Ch 22:5 Также следуя их совету, он пошел с Иорамом, сыном Ахавовым, царем Израильским, на войну против Азаила, царя Сирийского, в Рамоф Галаадский. И ранили Сирияне Иорама,
\vs 2Ch 22:6 и возвратился он в Изреель лечиться от ран, которые причинили ему в Раме, когда он воевал с Азаилом, царем Сирийским. И Охозия, сын Иорама, царь Иудейский, пришел посетить Иорама, сына Ахавова, в Изреель, потому что тот был болен.
\vs 2Ch 22:7 И от Бога было это на погибель Охозии, что он пришел к Иораму: ибо, по приходе своем, он вышел с Иорамом против Ииуя, сына Намессиева, которого помазал Господь на истребление дома Ахавова.
\vs 2Ch 22:8 Когда совершал Ииуй суд над домом Ахава, тогда он нашел князей Иудейских и сыновей братьев Охозии, служивших Охозии, и умертвил их.
\vs 2Ch 22:9 И [велел] он искать Охозию, и взяли его, когда он скрывался в Самарии, и привели его к Ииую, и умертвили его, и похоронили его, ибо говорили: он сын Иосафата, который взыскал Господа от всего сердца своего. И не \bibemph{осталось} в доме Охозии, \bibemph{кто} мог бы царствовать.
\vs 2Ch 22:10 Ибо Гофолия, мать Охозии, увидев, что умер сын ее, встала и истребила все царское племя дома Иудина.
\vs 2Ch 22:11 Но Иосавеф, дочь царя, взяла Иоаса, сына Охозии, и похитила его из среды царских сыновей умерщвляемых, и поместила его и кормилицу его в спальной комнате; и таким образом Иосавеф, дочь царя Иорама, жена Иодая священника, сестра Охозии, скрыла Иоаса от Гофолии, и она не умертвила его.
\vs 2Ch 22:12 И был он у них в доме Божием скрываем шесть лет; Гофолия же царствовала над землею.
\vs 2Ch 23:1 Но в седьмой год ободрился Иодай и принял в союз с собою начальников сотен: Азарию, сына Иерохамова, и Исмаила, сына Иегохананова, и Азарию, сына Оведова, и Маасею, сына Адаии, и Елишафата, сына Зихри.
\vs 2Ch 23:2 И они прошли по Иудее и собрали левитов из всех городов Иудеи и глав поколений Израилевых, и пришли в Иерусалим.
\vs 2Ch 23:3 И заключило все собрание союз в доме Божием с царем. И сказал им \bibemph{Иодай}: вот сын царя должен быть царем, как изрек Господь о сыновьях Давидовых.
\vs 2Ch 23:4 Вот что вы сделайте: треть вас, приходящих в субботу, из священников и левитов, \bibemph{будет} привратниками у порогов,
\vs 2Ch 23:5 и треть при доме царском, и треть у ворот Иесод, а весь народ на дворах дома Господня.
\vs 2Ch 23:6 И \bibemph{никто} пусть не входит в дом Господень, кроме священников и служащих из левитов. Они могут войти, потому что освящены; весь же народ пусть стоит на страже Господней.
\vs 2Ch 23:7 И пусть левиты окружат царя со всех сторон, всякий с оружием своим в руке своей, и кто будет входить в храм, да будет умерщвлен. И будьте вы при царе, когда он будет входить и выходить.
\vs 2Ch 23:8 И сделали левиты и все Иудеи, что приказал Иодай священник; и взяли каждый людей своих, приходящих в субботу с отходящими в субботу, потому что не отпустил священник Иодай \bibemph{сменившихся} черед.
\vs 2Ch 23:9 И раздал Иодай священник начальникам сотен копья и малые и большие щиты царя Давида, которые \bibemph{были} в доме Божием;
\vs 2Ch 23:10 и поставил весь народ, каждого с оружием его в руке его, от правой стороны храма до левой стороны храма, у жертвенника и у дома, вокруг царя.
\vs 2Ch 23:11 И вывели сына царя, и возложили на него венец и украшения, и поставили его царем; и помазали его Иодай и сыновья его и сказали: да живет царь!
\vs 2Ch 23:12 И услышала Гофолия голос народа, бегущего и провозглашающего о царе, и вышла к народу в дом Господень,
\vs 2Ch 23:13 и увидела: и вот царь стоит на возвышении своем при входе, и князья и трубы подле царя, и весь народ земли веселится, и трубят трубами, и певцы с орудиями музыкальными и искусные в славословии. И разодрала Гофолия одежды свои и закричала: заговор! заговор!
\vs 2Ch 23:14 И вызвал Иодай священник начальников сотен, начальствующих над войском, и сказал им: выведите ее вон [из храма], и кто последует за нею, да будет умерщвлен мечом. Потому что священник сказал: не умертвите ее в доме Господнем.
\vs 2Ch 23:15 И дали ей место, и когда она пришла ко входу конских ворот царского дома, там умертвили ее.
\rsbpar\vs 2Ch 23:16 И заключил Иодай завет между собою и между всем народом и царем, чтобы быть \bibemph{им} народом Господним.
\vs 2Ch 23:17 И пошел весь народ в капище Ваала, и разрушили его, и жертвенники его и истуканов его сокрушили; и Матфана, жреца Ваалова, умертвили пред жертвенниками.
\vs 2Ch 23:18 И поручил Иодай дела дома Господня священникам и левитам, [и восстановил дневные череды священников и левитов,] как распределил Давид в доме Господнем, для возношения всесожжений Господу, как написано в законе Моисеевом, с радостью и пением, по уставу Давидову.
\vs 2Ch 23:19 И поставил он привратников у ворот дома Господня, чтобы не \bibemph{мог} входить нечистый почему-нибудь.
\vs 2Ch 23:20 И взял начальников сотен, и вельмож, и начальствующих в народе, и весь народ земли, и проводил царя из дома Господня, и прошли чрез верхние ворота в дом царский, и посадили царя на царский престол.
\vs 2Ch 23:21 И веселился весь народ земли, и город успокоился. А Гофолию умертвили мечом.
\vs 2Ch 24:1 Семи лет \bibemph{был} Иоас, когда воцарился, и сорок лет царствовал в Иерусалиме; имя матери его Цивья из Вирсавии.
\vs 2Ch 24:2 И делал Иоас угодное в очах Господних во все дни Иодая священника.
\vs 2Ch 24:3 И взял ему Иодай двух жен, и он имел \bibemph{от них} сыновей и дочерей.
\rsbpar\vs 2Ch 24:4 И после сего пришло на сердце Иоасу обновить дом Господень,
\vs 2Ch 24:5 и собрал он священников и левитов и сказал им: пойдите по городам Иудеи и собирайте со всех Израильтян серебро для поддержания дома Бога вашего из года в год, и поспешите в этом деле. Но не поспешили левиты.
\vs 2Ch 24:6 И призвал царь Иодая, главу \bibemph{их}, и сказал ему: почему ты не требуешь от левитов, чтобы они доставляли с Иудеи и Иерусалима дань, \bibemph{установленную} Моисеем, рабом Господним, и собранием Израильтян для скинии собрания?
\vs 2Ch 24:7 Ибо нечестивая Гофолия и сыновья ее разорили дом Божий и все посвященное для дома Господня употребили для Ваалов.
\vs 2Ch 24:8 И приказал царь, и сделали один ящик, и поставили его у входа в дом Господень извне.
\vs 2Ch 24:9 И провозгласили по Иудее и Иерусалиму, чтобы приносили Господу дань, \bibemph{наложенную} Моисеем, рабом Божиим, на Израильтян в пустыне.
\vs 2Ch 24:10 И обрадовались все начальствующие и весь народ, и приносили и клали в ящик дотоле, доколе он не наполнился.
\vs 2Ch 24:11 В то время, когда приносили ящик к царским чиновникам чрез левитов, и когда они видели, что серебра много, приходил писец царя и поверенный первосвященника, и высыпали из ящика, и относили его и ставили его на свое место. Так делали они изо дня в день, и собрали множество серебра.
\vs 2Ch 24:12 И отдавали его царь и Иодай производителям работ по дому Господню, и они нанимали каменотесов и плотников для подновления дома Господня, также кузнецов и медников для укрепления дома Господня.
\vs 2Ch 24:13 И работали производители работ, и совершилось исправление руками их, и привели дом Божий в надлежащее состояние его, и укрепили его.
\vs 2Ch 24:14 И кончив \bibemph{все}, они представили царю и Иодаю остаток серебра. И сделали из него сосуды для дома Господня, сосуды служебные и \bibemph{для} всесожжений, чаши и \bibemph{другие} сосуды золотые и серебряные. И приносили всесожжения в доме Господнем постоянно во все дни Иодая.
\rsbpar\vs 2Ch 24:15 И состарился Иодай и, насытившись днями \bibemph{жизни}, умер: сто тридцать лет \bibemph{было} ему, когда он умер.
\vs 2Ch 24:16 И похоронили его в городе Давидовом с царями, потому что он делал доброе в Израиле и для Бога, и для дома Его.
\rsbpar\vs 2Ch 24:17 Но по смерти Иодая пришли князья Иудейские и поклонились царю; тогда царь стал слушаться их.
\vs 2Ch 24:18 И оставили дом Господа Бога отцов своих и стали служить деревам \bibemph{посвященным} и идолам,~--- и был гнев \bibemph{Господень} на Иуду и Иерусалим за сию вину их.
\vs 2Ch 24:19 И он посылал к ним пророков для обращения их к Господу, и они увещевали их, но те не слушали.
\vs 2Ch 24:20 И Дух Божий облек Захарию, сына Иодая священника, и он стал на возвышении пред народом и сказал им: так говорит Господь: для чего вы преступаете повеления Господни? не будет успеха вам; и как вы оставили Господа, то и Он оставит вас.
\vs 2Ch 24:21 И сговорились против него, и побили его камнями, по приказанию царя [Иоаса], на дворе дома Господня.
\vs 2Ch 24:22 И не вспомнил царь Иоас благодеяния, какое сделал ему Иодай, отец его, и убил сына его. И он умирая говорил: да видит Господь и да взыщет!
\rsbpar\vs 2Ch 24:23 И по истечении года выступило против него войско Сирийское, и вошли в Иудею и в Иерусалим, и истребили из народа всех князей народа, и всю добычу, \bibemph{взятую} у них, отослали к царю в Дамаск.
\vs 2Ch 24:24 Хотя в небольшом числе людей приходило войско Сирийское, но Господь предал в руку их весьма многочисленную силу за то, что оставили Господа Бога отцов своих. И над Иоасом совершили они суд,
\vs 2Ch 24:25 и когда они ушли от него, оставив его в тяжкой болезни, то составили против него заговор рабы его, за кровь сына Иодая священника, и убили его на постели его, и он умер. И похоронили его в городе Давидовом, но не похоронили его в царских гробницах.
\vs 2Ch 24:26 Заговорщиками же против него были: Завад, сын Шимеафы Аммонитянки, и Иегозавад, сын Шимрифы Моавитянки.
\vs 2Ch 24:27 О сыновьях его и о множестве пророчеств против него и об устроении дома Божия написано в книге царей. И воцарился Амасия, сын его, вместо него.
\vs 2Ch 25:1 Двадцати пяти лет воцарился Амасия и двадцать девять лет царствовал в Иерусалиме; имя матери его Иегоаддань из Иерусалима.
\vs 2Ch 25:2 И делал он угодное в очах Господних, но не от полного сердца.
\vs 2Ch 25:3 Когда утвердилось за ним царство, тогда он умертвил рабов своих, убивших царя, отца его.
\vs 2Ch 25:4 Но детей их не умертвил, так как написано в законе, в книге Моисеевой, где заповедал Господь, говоря: не должны быть умерщвляемы отцы за детей, и дети не должны быть умерщвляемы за отцов, но каждый за свое преступление должен умереть.
\rsbpar\vs 2Ch 25:5 И собрал Амасия Иудеев и поставил их по поколениям под власть тысяченачальников и стоначальников, всех Иудеев и Вениаминян, и пересчитал их от двадцати лет и выше, и нашел их триста тысяч человек отборных, ходящих на войну, держащих копье и щит.
\vs 2Ch 25:6 И \bibemph{еще} нанял из Израильтян сто тысяч храбрых воинов за сто талантов серебра.
\vs 2Ch 25:7 Но человек Божий пришел к нему и сказал: царь! пусть не идет с тобою войско Израильское, потому что нет Господа с Израильтянами, со всеми сынами Ефрема.
\vs 2Ch 25:8 Но иди ты \bibemph{один}, делай дело, мужественно подвизайся на войне. \bibemph{Иначе} повергнет тебя Бог пред лицем врага, ибо есть сила у Бога поддержать и повергнуть.
\vs 2Ch 25:9 И сказал Амасия человеку Божию: что же делать со ста талантами, которые я отдал войску Израильскому? И сказал человек Божий: может Господь дать тебе более сего.
\vs 2Ch 25:10 И отделил их Амасия,~--- войско, пришедшее к нему из \bibemph{земли} Ефремовой,~--- чтоб они шли в свое место. И возгорелся сильно гнев их на Иудею, и они пошли назад в свое место, в пылу гнева.
\vs 2Ch 25:11 А Амасия отважился и повел народ свой, и пошел на долину Соляную и побил сынов Сеира десять тысяч;
\vs 2Ch 25:12 и десять тысяч живых взяли сыны Иудины в плен, и привели их на вершину скалы, и низринули их с вершины скалы, и все они разбились совершенно.
\vs 2Ch 25:13 Войско же, которое Амасия отослал обратно, чтоб оно не ходило с ним на войну, рассыпалось по городам Иудеи от Самарии до Вефорона и перебило в них три тысячи, и награбило множество добычи.
\rsbpar\vs 2Ch 25:14 Амасия, придя после поражения Идумеян, принес богов сынов Сеира и поставил их у себя богами, и пред ними кланялся и им кадил.
\vs 2Ch 25:15 И воспылал гнев Господа на Амасию, и послал Он к нему пророка, и тот сказал ему: зачем ты прибегаешь к богам народа сего, которые не избавили народа своего от руки твоей?
\vs 2Ch 25:16 Когда он говорил ему, \bibemph{царь} отвечал: разве советником царским поставили тебя? перестань, чтоб не убили тебя. И перестал пророк, сказав: знаю, что решил Бог погубить тебя, потому что ты сделал сие и не слушаешь совета моего.
\vs 2Ch 25:17 И посоветовался Амасия, царь Иудейский, и послал к Иоасу, сыну Иоахаза, сына Ииуева, царю Израильскому, сказать: выходи, повидаемся лично.
\vs 2Ch 25:18 И послал Иоас, царь Израильский, к Амасии, царю Иудейскому, сказать: терн, который на Ливане, послал к кедру, который на Ливане же, сказать: отдай дочь свою в жену сыну моему. Но прошли звери дикие, которые на Ливане, и истоптали этот терн.
\vs 2Ch 25:19 Ты говоришь: вот я побил Идумеян,~--- и вознеслось сердце твое до тщеславия. Сиди лучше у себя дома. К чему тебе затевать опасное дело? Падешь ты и Иудея с тобою.
\vs 2Ch 25:20 Но не послушался Амасия, так как от Бога \bibemph{было} это, дабы предать их в руку \bibemph{Иоаса} за то, что стали прибегать к богам Идумейским.
\vs 2Ch 25:21 И выступил Иоас, царь Израильский, и увиделись лично, он и Амасия, царь Иудейский, в Вефсамисе Иудейском.
\vs 2Ch 25:22 И были разбиты Иудеи Израильтянами, и разбежались каждый в шатер свой.
\vs 2Ch 25:23 И Амасию, царя Иудейского, сына Иоаса, сына Иоахазова, захватил Иоас, царь Израильский, в Вефсамисе и привел его в Иерусалим, и разрушил стену Иерусалимскую от ворот Ефремовых до ворот уг\acc{о}льных, на четыреста локтей;
\vs 2Ch 25:24 и \bibemph{взял} все золото и серебро, и все сосуды, находившиеся в доме Божием у Овед-Едома, и сокровища дома царского, и заложников, и возвратился в Самарию.
\rsbpar\vs 2Ch 25:25 И жил Амасия, сын Иоасов, царь Иудейский, по смерти Иоаса, сына Иоахазова, царя Израильского, пятнадцать лет.
\vs 2Ch 25:26 Прочие дела Амасии, первые и последние, описаны в книге царей Иудейских и Израильских.
\vs 2Ch 25:27 И после того времени, как Амасия отступил от Господа, составили против него заговор в Иерусалиме, и он убежал в Лахис. И послали за ним в Лахис, и умертвили его там.
\vs 2Ch 25:28 И привезли его на конях, и похоронили его с отцами его в городе Иудином.
\vs 2Ch 26:1 И взял весь народ Иудейский Озию, которому \bibemph{было} шестнадцать лет, и поставили его царем на место отца его Амасии.
\vs 2Ch 26:2 Он обстроил Елаф и возвратил его Иудее, после того как почил царь с отцами своими.
\rsbpar\vs 2Ch 26:3 Шестнадцати лет \bibemph{был} Озия, когда воцарился, и пятьдесят два года царствовал в Иерусалиме; имя матери его Иехолия из Иерусалима.
\vs 2Ch 26:4 И делал он угодное в очах Господних точно так, как делал Амасия, отец его;
\vs 2Ch 26:5 и прибегал он к Богу во дни Захарии, поучавшего страху Божию; и в те дни, когда он прибегал к Господу, споспешествовал ему Бог.
\vs 2Ch 26:6 И он вышел и сразился с Филистимлянами, и разрушил стены Гефа и стены Иавнеи и стены Азота; и построил города в \bibemph{области} Азотской и у Филистимлян.
\vs 2Ch 26:7 И помогал ему Бог против Филистимлян и против Аравитян, живущих в Гур-Ваале, и \bibemph{против} Меунитян;
\vs 2Ch 26:8 и давали Аммонитяне дань Озии, и дошло имя его до пределов Египта, потому что он был весьма силен.
\vs 2Ch 26:9 И построил Озия башни в Иерусалиме над воротами уг\acc{о}льными и над воротами долины и на углу, и укрепил их.
\vs 2Ch 26:10 И построил башни в пустыне, и иссек много водоемов, потому что имел много скота, и на низменности и на равнине, и земледельцев и садовников на горах и на Кармиле, ибо он любил земледелие.
\vs 2Ch 26:11 Было у Озии и войско, выходившее на войну отрядами, по счету в списке их, составленном рукою Иеиела писца и Маасеи надзирателя, под предводительством Ханании, \bibemph{одного} из главных сановников царских.
\vs 2Ch 26:12 Все число глав поколений, из храбрых воинов, \bibemph{было} две тысячи шестьсот,
\vs 2Ch 26:13 и под рукою их военной силы триста семь тысяч пятьсот, вступавших в сражение с воинским мужеством, на помощь царю против неприятеля.
\vs 2Ch 26:14 И заготовил для них Озия, для всего войска, щиты и копья, и шлемы и латы, и луки и пращные камни.
\vs 2Ch 26:15 И сделал он в Иерусалиме искусно придуманные машины, чтоб они находились на башнях и на углах для метания стрел и больших камней. И пронеслось имя его далеко, потому что он дивно оградил себя и сделался силен.
\rsbpar\vs 2Ch 26:16 Но когда он сделался силен, возгордилось сердце его на погибель \bibemph{его}, и он сделался преступником пред Господом Богом своим, ибо вошел в храм Господень, чтобы воскурить \bibemph{фимиам} на алтаре кадильном.
\vs 2Ch 26:17 И пошел за ним Азария священник, и с ним восемьдесят священников Господних, людей отличных,
\vs 2Ch 26:18 и воспротивились Озии царю и сказали ему: не тебе, Озия, кадить Господу; это \bibemph{дело} священников, сынов Аароновых, посвященных для каждения; выйди из святилища, ибо ты поступил беззаконно, и не [будет] тебе это в честь у Господа Бога.
\vs 2Ch 26:19 И разгневался Озия,~--- а в руке у него кадильница для каждения; и когда разгневался он на священников, проказа явилась на челе его, пред лицем священников, в доме Господнем, у алтаря кадильного.
\vs 2Ch 26:20 И взглянул на него Азария первосвященник и все священники; и вот у него проказа на челе его. И понуждали его выйти оттуда, да и сам он спешил удалиться, так как поразил его Господь.
\vs 2Ch 26:21 И был царь Озия прокаженным до дня смерти своей, и жил в отдельном доме и отлучен был от дома Господня. А Иоафам, сын его, начальствовал над домом царским и управлял народом земли.
\rsbpar\vs 2Ch 26:22 Прочие деяния Озии, первые и последние, описал Исаия, сын Амоса, пророк.
\vs 2Ch 26:23 И почил Озия с отцами своими, и похоронили его с отцами его на поле царских гробниц, ибо говорили: он прокаженный. И воцарился Иоафам, сын его, вместо него.
\vs 2Ch 27:1 Двадцати пяти лет \bibemph{был} Иоафам, когда воцарился, и шестнадцать лет царствовал в Иерусалиме; имя матери его Иеруша, дочь Садока.
\vs 2Ch 27:2 И делал он угодное в очах Господних точно так, как делал Озия, отец его, только он не входил в храм Господень, и народ продолжал еще грешить.
\vs 2Ch 27:3 Он построил верхние ворота дома Господня, и многое построил на стене Офела;
\vs 2Ch 27:4 и города построил на горе Иудейской, и в лесах построил дворцы и башни.
\vs 2Ch 27:5 Он воевал с царем Аммонитян и одолел их, и дали ему Аммонитяне в тот год сто талантов серебра и десять тысяч к\acc{о}ров пшеницы и ячменя десять тысяч. Это давали ему Аммонитяне и на другой год, и на третий.
\vs 2Ch 27:6 Так силен был Иоафам потому, что устроял пути свои пред лицем Господа Бога своего.
\rsbpar\vs 2Ch 27:7 Прочие деяния Иоафама и все войны его и поведение его описаны в книге царей Израильских и Иудейских:
\vs 2Ch 27:8 двадцати пяти лет был он, когда воцарился, и шестнадцать лет царствовал в Иерусалиме.
\vs 2Ch 27:9 И почил Иоафам с отцами своими, и похоронили его в городе Давидовом. И воцарился Ахаз, сын его, вместо него.
\vs 2Ch 28:1 Двадцати лет был Ахаз, когда воцарился, и шестнадцать лет царствовал в Иерусалиме; и он не делал угодного в очах Господних, как \bibemph{делал} Давид, отец его:
\vs 2Ch 28:2 он шел путями царей Израильских, и даже сделал литые статуи Ваалов;
\vs 2Ch 28:3 и он совершал курения на долине сынов Еннома, и проводил сыновей своих через огонь, подражая мерзостям народов, которых изгнал Господь пред лицем сынов Израилевых;
\vs 2Ch 28:4 и приносил жертвы и курения на высотах и на холмах и под всяким ветвистым деревом.
\vs 2Ch 28:5 И предал его Господь Бог его в руку царя Сириян, и они поразили его и взяли у него множество пленных и отвели в Дамаск. Также и в руку царя Израильского был предан он, и тот произвел у него великое поражение.
\vs 2Ch 28:6 И избил Факей, сын Ремалиин, [царь Израильский,] Иудеев сто двадцать тысяч в один день, людей воинственных, потому что они оставили Господа Бога отцов своих.
\vs 2Ch 28:7 Зихрий же, силач из Ефремлян, убил Маасею, сына царя, и Азрикама, начальствующего над дворцом, и Елкану, второго по царе.
\vs 2Ch 28:8 И взяли сыны Израилевы в плен у братьев своих, \bibemph{Иудеев}, двести тысяч жен, сыновей и дочерей; также и множество добычи награбили у них, и отправили добычу в Самарию.
\rsbpar\vs 2Ch 28:9 Там был пророк Господень, имя его Одед. Он вышел пред лице войска, шедшего в Самарию, и сказал им: вот Господь Бог отцов ваших, во гневе на Иудеев, предал их в руку вашу, и вы избили их с такою яростью, которая достигла до небес.
\vs 2Ch 28:10 И теперь вы думаете поработить сынов Иуды и Иерусалима в рабы и рабыни себе. А разве на самих вас нет вины пред Господом Богом вашим?
\vs 2Ch 28:11 Итак послушайте меня, и возвратите пленных, которых вы захватили из братьев ваших, ибо пламень гнева Господня на вас.
\vs 2Ch 28:12 И встали некоторые из начальников сынов Ефремовых: Азария, сын Иегоханана, Берехия, сын Мешиллемофа, и Езекия, сын Шаллума, и Амаса, сын Хадлая, против шедших с войны,
\vs 2Ch 28:13 и сказали им: не вводите сюда пленных, потому что грех был бы нам пред Господом. Неужели вы думаете прибавить к грехам нашим и к преступлениям нашим? велика вина наша, и пламень гнева [Господня] над Израилем.
\vs 2Ch 28:14 И оставили вооруженные пленных и добычу у военачальников и всего собрания.
\vs 2Ch 28:15 И встали мужи, упомянутые по именам, и взяли пленных, и всех нагих из них одели из добычи,~--- и одели их, и обули их, и накормили их, и напоили их, и помазали их елеем, и посадили на ослов всех слабых из них, и отправили их в Иерихон, город пальм, к братьям их, и возвратились в Самарию.
\rsbpar\vs 2Ch 28:16 В то время послал царь Ахаз к царям Ассирийским, чтоб они помогли ему,
\vs 2Ch 28:17 ибо Идумеяне и еще приходили, и \bibemph{многих} побили в Иудее, и взяли в плен;
\vs 2Ch 28:18 и Филистимляне рассыпались по городам низменного края и юга Иудеи и взяли Вефсамис и Аиалон, и Гедероф и Сохо и зависящие от него города, и Фимну и зависящие от нее города, и Гимзо и зависящие от него города, и поселились там.
\vs 2Ch 28:19 Так унизил Господь Иудею за Ахаза, царя Иудейского, потому что он развратил Иудею и тяжко грешил пред Господом.
\vs 2Ch 28:20 И пришел к нему Феглафелласар, царь Ассирийский, но был в тягость ему, вместо того, чтобы помочь ему,
\vs 2Ch 28:21 потому что Ахаз взял \bibemph{сокровища} из дома Господня и дома царского и у князей и отдал царю Ассирийскому, но не в помощь себе.
\rsbpar\vs 2Ch 28:22 И в тесное для себя время он продолжал беззаконно поступать пред Господом, он~--- царь Ахаз.
\vs 2Ch 28:23 И приносил он жертвы богам Дамасским, \bibemph{думая, что} они поражали его, и говорил: боги царей Сирийских помогают им; принесу я жертву им, и они помогут мне. Но они были на падение ему и всему Израилю.
\vs 2Ch 28:24 И собрал Ахаз сосуды дома Божия, и сокрушил сосуды дома Божия, и запер двери дома Господня, и устроил себе жертвенники по всем углам в Иерусалиме,
\vs 2Ch 28:25 и по всем городам Иудиным устроил высоты для каждения богам иным, и раздражал Господа Бога отцов своих.
\rsbpar\vs 2Ch 28:26 Прочие дела его и все поступки его, первые и последние, описаны в книге царей Иудейских и Израильских.
\vs 2Ch 28:27 И почил Ахаз с отцами своими, и похоронили его в городе, в Иерусалиме, но не внесли его в гробницы царей Израилевых. И воцарился Езекия, сын его, вместо него.
\vs 2Ch 29:1 Езекия воцарился двадцати пяти лет, и двадцать девять лет царствовал в Иерусалиме; имя матери его Авия, дочь Захарии.
\vs 2Ch 29:2 И делал он угодное в очах Господних точно так, как делал Давид, отец его.
\rsbpar\vs 2Ch 29:3 В первый же год царствования своего, в первый месяц, он отворил двери дома Господня и возобновил их,
\vs 2Ch 29:4 и велел прийти священникам и левитам, и собрал их на площади восточной,
\vs 2Ch 29:5 и сказал им: послушайте меня, левиты! Ныне освятитесь \bibemph{сами} и освятите дом Господа Бога отцов ваших, и выбросьте нечистоту из святилища.
\vs 2Ch 29:6 Ибо отцы наши поступали беззаконно, и делали неугодное в очах Господа Бога нашего, и оставили Его, и отвратили они лица свои от жилища Господня, и оборотились спиною,
\vs 2Ch 29:7 и заперли двери притвора, и погасили светильники, и не сожигали курения, и не возносили всесожжений во святилище Бога Израилева.
\vs 2Ch 29:8 И был гнев Господа на Иудею и на Иерусалим, и Он отдал их на позор, на опустошение и на посмеяние, как вы видите глазами вашими.
\vs 2Ch 29:9 И вот, пали отцы наши от меча, а сыновья наши и дочери наши и жены наши за это в плену [в земле не своей] доныне.
\vs 2Ch 29:10 Теперь у меня на сердце~--- заключить завет с Господом Богом Израилевым, да отвратит от нас пламень гнева Своего.
\vs 2Ch 29:11 Дети мои! не будьте небрежны, ибо вас избрал Господь предстоять лицу Его, служить Ему и быть у Него служителями и возжигателями курений.
\vs 2Ch 29:12 И встали левиты: Махаф, сын Амасая, и Иоель, сын Азарии, из сыновей Каафовых; и из сыновей Мерариных: Кис, сын Авдия, и Азария, сын Иегаллелела; и из племени Гирсонова: Иоах, сын Зиммы, и Еден, сын Иоаха;
\vs 2Ch 29:13 и из сыновей Елицафановых: Шимри и Иеиел; и из сыновей Асафовых: Захария и Матфания;
\vs 2Ch 29:14 и из сыновей Емановых: Иехиел и Шимей; и из сыновей Идифуновых: Шемаия и Уззиел.
\vs 2Ch 29:15 Они собрали братьев своих и освятились, и пошли по приказанию царя очищать дом Господень по словам Господа.
\vs 2Ch 29:16 И вошли священники внутрь дома Господня для очищения, и вынесли все нечистое, что нашли в храме Господнем, на двор дома Господня, а левиты взяли это, чтобы вынести вон к потоку Кедрону.
\vs 2Ch 29:17 И начали освящать в первый \bibemph{день} первого месяца, и в восьмой день \bibemph{того же} месяца вошли в притвор Господень; и освящали дом Господень восемь дней, и в шестнадцатый день первого месяца кончили.
\vs 2Ch 29:18 И пришли в дом к царю Езекии и сказали: мы очистили дом Господень, и жертвенник для всесожжения, и все сосуды его, и стол \bibemph{для хлебов} предложения, и все сосуды его;
\vs 2Ch 29:19 и все сосуды, которые забросил царь Ахаз во время царствования своего, в беззаконии своем, мы приготовили и освятили, и вот они пред жертвенником Господним.
\rsbpar\vs 2Ch 29:20 И встал царь Езекия рано утром и собрал начальников города, и пошел в дом Господень.
\vs 2Ch 29:21 И привели семь тельцов и семь овнов, и семь агнцев и семь козлов на жертву о грехе за царство и за святилище и за Иудею; и приказал он сынам Аароновым, священникам, вознести всесожжение на жертвенник Господень.
\vs 2Ch 29:22 И закололи тельцов, и взяли священники кровь, и окропили жертвенник, и закололи овнов, и окропили кровью жертвенник; и закололи агнцев, и окропили кровью жертвенник.
\vs 2Ch 29:23 И привели козлов за грех пред лице царя и собрания, и они возложили руки свои на них.
\vs 2Ch 29:24 И закололи их священники, и очистили кровью их жертвенник для заглаждения грехов всего Израиля, ибо за всего Израиля приказал царь \bibemph{принести} всесожжение и жертву о грехе.
\vs 2Ch 29:25 И поставил он левитов в доме Господнем с кимвалами, псалтирями и цитрами, по уставу Давида и Гада, прозорливца царева, и Нафана пророка, так как от Господа \bibemph{был} устав этот чрез пророков Его.
\vs 2Ch 29:26 И стали левиты с \bibemph{музыкальными} орудиями Давидовыми и священники с трубами.
\vs 2Ch 29:27 И приказал Езекия вознести всесожжение на жертвенник. И в то время, как началось всесожжение, началось пение Господу, при \bibemph{звуке} труб и орудий Давида, царя Израилева.
\vs 2Ch 29:28 И все собрание молилось, и певцы пели, и трубили трубы, доколе не окончилось всесожжение.
\vs 2Ch 29:29 По окончании же всесожжения царь и все находившиеся при нем преклонились и поклонились.
\vs 2Ch 29:30 И сказал царь Езекия и князья левитам, чтоб они славили Господа словами Давида и Асафа прозорливца, и они славили с радостью, и преклонялись и поклонялись.
\vs 2Ch 29:31 И продолжал Езекия и сказал: теперь вы посвятили себя Господу; приступайте и приносите жертвы и благодарственные приношения в дом Господень. И понесло \bibemph{все} собрание жертвы и благодарственные приношения, и всякий, кто расположен был сердцем,~--- всесожжения.
\vs 2Ch 29:32 И было число всесожжений, которые привели собравшиеся: семьдесят волов, сто овнов, двести агнцев~--- все это для всесожжения Господу.
\vs 2Ch 29:33 \bibemph{Других} священных жертв \bibemph{было}: шестьсот из крупного скота и три тысячи из мелкого скота.
\vs 2Ch 29:34 Но священников было мало, и они не могли сдирать кож со всех всесожжений, и помогали им братья их левиты, до окончания дела и доколе освятились \bibemph{прочие} священники, ибо левиты были более тщательны в освящении себя, нежели священники.
\vs 2Ch 29:35 Притом же всесожжений \bibemph{было} множество с туками мирных жертв и с возлияниями при всесожжении. Так восстановлено служение в доме Господнем.
\vs 2Ch 29:36 И радовался Езекия и весь народ о том, что Бог \bibemph{так} расположил народ, ибо это сделалось неожиданно.
\vs 2Ch 30:1 И послал Езекия по всей \bibemph{земле} Израильской и Иудее, и письма писал к Ефрему и Манассии, чтобы пришли в дом Господень, в Иерусалим, для совершения пасхи Господу Богу Израилеву.
\vs 2Ch 30:2 И положили на совете царь и князья его и все собрание в Иерусалиме~--- совершить пасху во второй месяц,
\vs 2Ch 30:3 ибо не могли совершить ее в свое время, потому что священники \bibemph{еще} не освятились в достаточном числе и народ не собрался в Иерусалим.
\vs 2Ch 30:4 И понравилось это царю и всему собранию.
\vs 2Ch 30:5 И определили объявить по всему Израилю, от Вирсавии до Дана, чтобы шли в Иерусалим для совершения пасхи Господу Богу Израилеву, потому что давно не совершали \bibemph{ее}, как предписано.
\vs 2Ch 30:6 И пошли гонцы с письмами от царя и от князей его по всей \bibemph{земле} Израильской и Иудее, и по повелению царя говорили: дети Израиля! обратитесь к Господу Богу Авраама, Исаака и Израиля, и Он обратится к остатку, уцелевшему у вас от руки царей Ассирийских.
\vs 2Ch 30:7 И не будьте таковы, как отцы ваши и братья ваши, которые беззаконно поступали пред Господом Богом отцов своих; и Он предал их на опустошение, как вы видите.
\vs 2Ch 30:8 Ныне не будьте жестоковыйны, как отцы ваши, покоритесь Господу и приходите во святилище Его, которое Он освятил навек; и служите Господу Богу вашему, и Он отвратит от вас пламень гнева Своего.
\vs 2Ch 30:9 Когда вы обратитесь к Господу, тогда братья ваши и дети ваши [будут] в милости у пленивших их и возвратятся в землю сию, ибо благ и милосерд Господь Бог ваш и не отвратит лица от вас, если вы обратитесь к Нему.
\vs 2Ch 30:10 И ходили гонцы из города в город по земле Ефремовой и Манассииной и до Завулоновой, но над ними смеялись и издевались.
\vs 2Ch 30:11 Однако некоторые из \bibemph{колена} Асирова, Манассиина и Завулонова смирились и пришли в Иерусалим.
\vs 2Ch 30:12 И над Иудеею была рука Божия, даровавшая им единое сердце, чтоб исполнить повеление царя и князей, по слову Господню.
\rsbpar\vs 2Ch 30:13 И собралось в Иерусалим множество народа для совершения праздника опресноков, во второй месяц,~--- собрание весьма многочисленное.
\vs 2Ch 30:14 И встали и ниспровергли жертвенники, которые были в Иерусалиме; и всё, на чем совершаемо было курение [идолам], разрушили и бросили в поток Кедрон;
\vs 2Ch 30:15 и закололи пасхального агнца в четырнадцатый \bibemph{день} второго месяца. Священники и левиты устыдившись освятились и принесли всесожжения в дом Господень,
\vs 2Ch 30:16 и стали на своем месте по уставу своему, по закону Моисея, человека Божия. Священники кропили кровью [принимая ее] из рук левитов.
\vs 2Ch 30:17 Так как много \bibemph{было} в собрании таких, которые не освятились, то вместо нечистых левиты закололи пасхального агнца, для посвящения Господу.
\vs 2Ch 30:18 Многие из народа, большею частью из колена Ефремова и Манассиина, Иссахарова и Завулонова, не очистились; однако же они ели пасху, не по уставу.
\vs 2Ch 30:19 Но Езекия помолился за них, говоря: Господь благий да простит каждого, кто расположил сердце свое к тому, чтобы взыскать Господа Бога, Бога отцов своих, хотя и без очищения священного!
\vs 2Ch 30:20 И услышал Господь Езекию и простил народ.
\rsbpar\vs 2Ch 30:21 И совершили сыны Израилевы, находившиеся в Иерусалиме, праздник опресноков в семь дней, с великим веселием; каждый день левиты и священники славили Господа на орудиях, \bibemph{устроенных} для славословия Господа.
\vs 2Ch 30:22 И говорил Езекия по сердцу всем левитам, имевшим доброе разумение \bibemph{в служении} Господу. И ели праздничное семь дней, принося жертвы мирные и славя Господа Бога отцов своих.
\vs 2Ch 30:23 И решило все собрание праздновать другие семь дней, и провели эти семь дней в веселии,
\vs 2Ch 30:24 потому что Езекия, царь Иудейский, выставил для собравшихся тысячу тельцов и десять тысяч мелкого скота, и вельможи выставили для собравшихся тысячу тельцов и десять тысяч мелкого скота; и священников освятилось \bibemph{уже} много.
\vs 2Ch 30:25 И веселились все собравшиеся из Иудеи, и священники и левиты, и все собрание, пришедшее от Израиля, и пришельцы, пришедшие из земли Израильской и обитавшие в Иудее.
\vs 2Ch 30:26 И было веселие великое в Иерусалиме, потому что со дней Соломона, сына Давидова, царя Израилева, \bibemph{не бывало} подобного сему в Иерусалиме.
\vs 2Ch 30:27 И встали священники и левиты, и благословили народ; и услышан был голос их, и взошла молитва их в святое жилище Его на небеса.
\vs 2Ch 31:1 И по окончании всего этого, пошли все Израильтяне, \bibemph{там} находившиеся, в города Иудейские и разбили статуи, срубили \bibemph{посвященные} дерева, и разрушили высоты и жертвенники во всей Иудее и в \bibemph{земле} Вениаминовой, Ефремовой и Манассииной, до конца. И \bibemph{потом} возвратились все сыны Израилевы, каждый во владение свое, в города свои.
\vs 2Ch 31:2 И поставил Езекия череды священников и левитов, по их распределению, каждого при деле своем, священническом или левитском, при всесожжении и при жертвах мирных, для службы, для хваления и славословия, у ворот дома Господня.
\vs 2Ch 31:3 И \bibemph{определил} царь часть из имущества своего на всесожжения: на всесожжения утренние и вечерние, и на всесожжения в субботы и в новомесячия, и в праздники, как написано в законе Господнем.
\vs 2Ch 31:4 И повелел он народу, живущему в Иерусалиме, давать определенное содержание священникам и левитам, чтоб они были ревностны в законе Господнем.
\rsbpar\vs 2Ch 31:5 Когда обнародовано было это повеление, тогда нанесли сыны Израилевы множество начатков хлеба, вина, и масла, и меду, и всяких произведений полевых; и десятин из всего нанесли множество.
\vs 2Ch 31:6 И Израильтяне и Иудеи, живущие по городам Иудейским, также представили десятины из крупного и мелкого скота и десятины из пожертвований, посвященных Господу Богу их; и наложили груды, груды.
\vs 2Ch 31:7 В третий месяц начали класть груды, и в седьмой месяц кончили.
\vs 2Ch 31:8 И пришли Езекия и вельможи, и увидели груды, и благодарили Господа и народ Его Израиля.
\vs 2Ch 31:9 И спросил Езекия священников и левитов об этих грудах.
\vs 2Ch 31:10 И отвечал ему Азария первосвященник из дома Садокова и сказал: с того времени, как начали носить приношения в дом Господень, мы ели досыта, и многое осталось, потому что Господь благословил народ Свой. Из оставшегося \bibemph{составилось} такое множество.
\vs 2Ch 31:11 И приказал Езекия приготовить комнаты при доме Господнем. И приготовили.
\vs 2Ch 31:12 И перенесли \bibemph{туда} приношения, и десятины, и пожертвования, со \bibemph{всею} точностью. И \bibemph{был} начальником при них Хонания левит, и Симей, брат его, вторым.
\vs 2Ch 31:13 А Иехиил и Азазия, и Нахаф и Асаил, и Иеримоф и Иозавад, и Елиел и Исмахия, и Махаф и Бенания \bibemph{были} смотрителями под рукою Хонании и Симея, брата его, по распоряжению царя Езекии и Азарии, начальника при доме Божием.
\vs 2Ch 31:14 Коре, сын Имны, левит, привратник на восточной стороне, \bibemph{был} при добровольных приношениях Богу, для выдачи принесенного Господу и важнейших из вещей посвященных.
\vs 2Ch 31:15 И под его \bibemph{ведением находились} Еден, и Миниамин, и Иешуа, и Шемаия, и Амария и Шехания в городах священнических, чтобы верно раздавать братьям своим части, как большому, так и малому,
\vs 2Ch 31:16 сверх списка их, \bibemph{всем} мужеского пола от трех лет и выше, всем ходящим в дом Господа для дел ежедневных, для служения их, по должностям их и по отделам их,
\vs 2Ch 31:17 и внесенным в список священникам, по поколениям их, и левитам от двадцати лет и выше, по должностям их, по отделам их,
\vs 2Ch 31:18 и внесенным в список, со всеми малолетними их, с женами их и с сыновьями их и с дочерями их,~--- всему обществу, ибо они со \bibemph{всею} верностью посвятили себя на священную службу.
\vs 2Ch 31:19 И для сынов Аароновых, священников в селениях вокруг городов их, при каждом городе \bibemph{поставлены были} мужи поименованные, чтобы раздавать участки всем мужеского пола у священников и всем внесенным в список у левитов.
\rsbpar\vs 2Ch 31:20 Вот что сделал Езекия во всей Иудее,~--- и он делал доброе, и справедливое, и истинное пред лицем Господа Бога своего.
\vs 2Ch 31:21 И во всем, что он предпринимал на служение дому Божию и для соблюдения закона и заповедей, помышляя о Боге своем, он действовал от всего сердца своего и имел успех.
\vs 2Ch 32:1 После таких дел и верности, пришел Сеннахирим, царь Ассирийский, и вступил в Иудею, и осадил укрепленные города, и думал отторгнуть их себе.
\vs 2Ch 32:2 Когда Езекия увидел, что пришел Сеннахирим с намерением воевать против Иерусалима,
\vs 2Ch 32:3 тогда решил с князьями своими и с военными людьми своими зас\acc{ы}пать источники воды, которые вне города, и те помогли ему.
\vs 2Ch 32:4 И собралось множество народа, и зас\acc{ы}пали все источники и поток, протекавший по стране, говоря: да не найдут цари Ассирийские, придя \bibemph{сюда}, много воды [и да не укрепятся].
\vs 2Ch 32:5 И ободрился он, и восстановил всю обрушившуюся стену, и поднял ее до башни, и извне \bibemph{построил} другую стену, и укрепил Милло в городе Давидовом, и наготовил множество оружия и щитов.
\vs 2Ch 32:6 И поставил военачальников над народом, и собрал их к себе на площадь у городских ворот, и говорил к сердцу их, и сказал:
\vs 2Ch 32:7 будьте тверды и мужественны, не бойтесь и не страшитесь царя Ассирийского и всего множества, которое с ним, потому что с нами более, нежели с ним;
\vs 2Ch 32:8 с ним мышца плотская, а с нами Господь Бог наш, чтобы помогать нам и сражаться на бранях наших. И подкрепился народ словами Езекии, царя Иудейского.
\rsbpar\vs 2Ch 32:9 После сего послал Сеннахирим, царь Ассирийский, рабов своих в Иерусалим,~--- сам он \bibemph{стоял} против Лахиса, и вся сила его с ним,~--- к Езекии, царю Иудейскому, и ко всем Иудеям, которые в Иерусалиме, сказать:
\vs 2Ch 32:10 так говорит Сеннахирим, царь Ассирийский: на что вы надеетесь и сидите в крепости в Иерусалиме?
\vs 2Ch 32:11 Не обольщает ли вас Езекия, чтобы предать вас смерти от голода и жажды, говоря: Господь Бог наш спасет нас от руки царя Ассирийского?
\vs 2Ch 32:12 Не этот ли Езекия разрушил высоты Его и жертвенники Его, и сказал Иудее и Иерусалиму: пред жертвенником единым поклоняйтесь и на нем совершайте курения?
\vs 2Ch 32:13 Разве вы не знаете, что сделал я и отцы мои со всеми народами земель? Могли ли боги народов земных спасти землю свою от руки моей?
\vs 2Ch 32:14 Кто из всех богов народов, истребленных отцами моими, мог спасти народ свой от руки моей? \bibemph{Как же} возможет ваш Бог спасти вас от руки моей?
\vs 2Ch 32:15 И ныне пусть не обольщает вас Езекия и не отклоняет вас таким образом; не верьте ему: если не в силах был ни один бог ни одного народа и царства спасти народ свой от руки моей и от руки отцов моих, то и ваш Бог не спасет вас от руки моей.
\vs 2Ch 32:16 И еще \bibemph{многое} говорили рабы его против Господа Бога и против Езекии, раба Его.
\vs 2Ch 32:17 И письма писал он, \bibemph{в которых} поносил Господа Бога Израилева и говорил против Него такие слова: как боги народов земных не спасли народов своих от руки моей, так Бог Езекии не спасет народа Своего от руки моей.
\vs 2Ch 32:18 И кричали громким голосом на Иудейском языке к народу Иерусалимскому, который \bibemph{был} на стене, чтоб устрашить его и напугать его, и взять город.
\vs 2Ch 32:19 И говорили о Боге Иерусалима, как о богах народов земли,~--- изделии рук человеческих.
\vs 2Ch 32:20 И помолился царь Езекия и Исаия, сын Амосов, пророк, и возопили к небу.
\vs 2Ch 32:21 И послал Господь Ангела, и он истребил всех храбрых и главноначальствующего и начальствующих в войске царя Ассирийского. И возвратился он со стыдом в землю свою; и когда пришел в дом бога своего,~--- исшедшие из чресл его поразили его там мечом.
\vs 2Ch 32:22 Так спас Господь Езекию и жителей Иерусалима от руки Сеннахирима, царя Ассирийского, и от руки всех и оберегал их отовсюду.
\vs 2Ch 32:23 Тогда многие приносили дары Господу в Иерусалим и дорогие вещи Езекии, царю Иудейскому. И он возвеличился после сего в глазах всех народов.
\rsbpar\vs 2Ch 32:24 В те дни заболел Езекия смертельно. И помолился Господу, и Он услышал его и дал ему знамение.
\vs 2Ch 32:25 Но не воздал Езекия за оказанные ему благодеяния, ибо возгордилось сердце его. И был на него гнев \bibemph{Божий} и на Иудею, и на Иерусалим.
\vs 2Ch 32:26 Но как смирился Езекия в гордости сердца своего,~--- сам и жители Иерусалима, то не пришел на них гнев Господень во дни Езекии.
\vs 2Ch 32:27 И было у Езекии богатства и славы весьма много, и хранилище он сделал у себя для серебра и золота, и камней драгоценных, и для ароматов и щитов, и для всяких драгоценных сосудов;
\vs 2Ch 32:28 и кладовые для произведений \bibemph{земли}, для хлеба, вина и масла, и стойла для всякого рода скота, и дворы для стад.
\vs 2Ch 32:29 И города построил себе. И стад мелкого и крупного скота \bibemph{было у него} множество, потому что дал ему Бог весьма большое имущество.
\vs 2Ch 32:30 Он же, Езекия, запер верхний проток вод Геона и провел их вниз к западной стороне города Давидова. И действовал успешно Езекия во всяком деле своем.
\vs 2Ch 32:31 Только при послах царей Вавилонских, которые присылали к нему спросить о знамении, бывшем на земле, оставил его Бог, чтоб испытать его и открыть все, что у него на сердце.
\rsbpar\vs 2Ch 32:32 Прочие деяния Езекии и добродетели его описаны в видении Исаии, сына Амосова, пророка, и в книге царей Иудейских и Израильских.
\vs 2Ch 32:33 И почил Езекия с отцами своими, и похоронили его над гробницами сыновей Давидовых, и почесть воздали ему по смерти его все Иудеи и жители Иерусалима. И воцарился Манассия, сын его, вместо него.
\vs 2Ch 33:1 Двенадцати лет \bibemph{был} Манассия, когда воцарился, и пятьдесят пять лет царствовал в Иерусалиме,
\vs 2Ch 33:2 и делал он неугодное в очах Господних, подражая мерзостям народов, которых прогнал Господь от лица сынов Израилевых,
\vs 2Ch 33:3 и снова построил высоты, которые разрушил Езекия, отец его, и поставил жертвенники Ваалам, и устроил дубравы, и поклонялся всему воинству небесному, и служил ему,
\vs 2Ch 33:4 и соорудил жертвенники в доме Господнем, о котором сказал Господь: в Иерусалиме будет имя Мое вечно;
\vs 2Ch 33:5 и соорудил жертвенники всему воинству небесному на обоих дворах дома Господня.
\vs 2Ch 33:6 Он же проводил сыновей своих чрез огонь в долине сына Енномова, и гадал, и ворожил, и чародействовал, и учредил вызывателей мертвецов и волшебников; много делал он неугодного в очах Господа, к прогневлению Его.
\vs 2Ch 33:7 И поставил резного идола, которого сделал, в доме Божием, о котором говорил Бог Давиду и Соломону, сыну его: в доме сем и в Иерусалиме, который Я избрал из всех колен Израилевых, Я положу имя Мое навек;
\vs 2Ch 33:8 и не дам впредь выступить ноге Израиля из земли сей, которую Я укрепил за отцами их, если только они будут стараться исполнять все, что Я заповедал им, по всему закону и уставам и повелениям, \bibemph{данным} рукою Моисея.
\vs 2Ch 33:9 Но Манассия довел Иудею и жителей Иерусалима до того, что они поступали хуже тех народов, которых истребил Господь от лица сынов Израилевых.
\rsbpar\vs 2Ch 33:10 И говорил Господь к Манассии и к народу его, но они не слушали.
\vs 2Ch 33:11 И привел Господь на них военачальников царя Ассирийского, и заковали они Манассию в кандалы и оковали его цепями, и отвели его в Вавилон.
\vs 2Ch 33:12 И в тесноте своей он стал умолять лице Господа Бога своего и глубоко смирился пред Богом отцов своих.
\vs 2Ch 33:13 И помолился Ему, и \bibemph{Бог} преклонился к нему и услышал моление его, и возвратил его в Иерусалим на царство его. И узнал Манассия, что Господь есть Бог.
\vs 2Ch 33:14 И после того построил внешнюю стену города Давидова, на западной стороне Геона, по лощине и до входа в Рыбные ворота, и провел ее вокруг Офела и высоко поднял ее. И поставил военачальников по всем укрепленным городам в Иудее,
\vs 2Ch 33:15 и низверг чужеземных богов и идола из дома Господня, и все капища, которые соорудил на горе дома Господня и в Иерусалиме, и выбросил их за город.
\vs 2Ch 33:16 И восстановил жертвенник Господень и принес на нем жертвы мирные и хвалебные, и сказал Иудеям, чтобы они служили Господу Богу Израилеву.
\vs 2Ch 33:17 Но народ еще приносил жертвы на высотах, хотя и Господу Богу своему.
\rsbpar\vs 2Ch 33:18 Прочие дела Манассии, и молитва его к Богу своему, и слова прозорливцев, говоривших к нему именем Господа Бога Израилева, находятся в записях царей Израилевых.
\vs 2Ch 33:19 И молитва его, и то, что \bibemph{Бог} преклонился к нему, и все грехи его и беззакония его, и места, на которых он построил высоты и поставил изображения Астарты и истуканов, прежде нежели смирился, описаны в записях Хозая.
\vs 2Ch 33:20 И почил Манассия с отцами своими, и похоронили его в доме его. И воцарился Амон, сын его, вместо него.
\rsbpar\vs 2Ch 33:21 Двадцати двух лет был Амон, когда воцарился, и два года царствовал в Иерусалиме.
\vs 2Ch 33:22 И делал неугодное в очах Господних так, как делал Манассия, отец его; и всем истуканам, которых сделал Манассия, отец его, приносил Амон жертвы и служил им.
\vs 2Ch 33:23 И не смирился пред лицем Господним, как смирился Манассия, отец его; напротив, Амон умножил \bibemph{свои} грехи.
\vs 2Ch 33:24 И составили против него заговор слуги его, и умертвили его в доме его.
\vs 2Ch 33:25 Но народ земли перебил всех, бывших в заговоре против царя Амона, и воцарил народ земли Иосию, сына его, вместо него.
\vs 2Ch 34:1 Восемь лет было Иосии, когда он воцарился, и тридцать один год царствовал в Иерусалиме,
\vs 2Ch 34:2 и делал он угодное в очах Господних, и ходил путями Давида, отца своего, и не уклонялся ни направо, ни налево.
\rsbpar\vs 2Ch 34:3 В восьмой год царствования своего, будучи еще отроком, он начал прибегать к Богу Давида, отца своего, а в двенадцатый год начал очищать Иудею и Иерусалим от высот и \bibemph{посвященных} дерев и от резных и литых кумиров.
\vs 2Ch 34:4 И разрушили пред лицем его жертвенники Ваалов и статуи, возвышавшиеся над ними; и \bibemph{посвященные} дерева он срубил, и резные и литые кумиры изломал и разбил в прах, и рассыпал на гробах тех, которые приносили им жертвы,
\vs 2Ch 34:5 и кости жрецов сжег на жертвенниках их, и очистил Иудею и Иерусалим,
\vs 2Ch 34:6 и в городах Манассии, и Ефрема, и Симеона, \bibemph{даже} до колена Неффалимова, и в опустошенных окрестностях их
\vs 2Ch 34:7 он разрушил жертвенники и \bibemph{посвященные} дерева, и кумиры разбил в прах, и все статуи сокрушил по всей земле Израильской, и возвратился в Иерусалим.
\rsbpar\vs 2Ch 34:8 В восемнадцатый год царствования своего, по очищении земли и дома \bibemph{Божия}, он послал Шафана, сына Ацалии, и Маасею градоначальника, и Иоаха, сына Иоахазова, дееписателя, возобновить дом Господа Бога своего.
\vs 2Ch 34:9 И пришли они к Хелкии первосвященнику, и отдали серебро, принесенное в дом Божий, которое левиты, стоящие на страже у порога, собрали из рук Манассии и Ефрема и всех прочих Израильтян, и от всего Иуды и Вениамина, и от жителей Иерусалима,
\vs 2Ch 34:10 и отдали в руки производителям работ, приставленным к дому Господню, чтоб они раздавали его работникам, которые работали в доме Господнем, при исправлении и возобновлении дома.
\vs 2Ch 34:11 И они раздавали плотникам и строителям на покупку тесаных камней и дерев для связей и для покрытия зданий, которые разорили цари Иудейские.
\vs 2Ch 34:12 Люди сии действовали честно при работе, и для надзора над ними поставлены были Иахаф и Овадия, левиты из сыновей Мерариных, и Захария и Мешуллам из сыновей Каафовых, и все левиты, умеющие играть на музыкальных орудиях.
\vs 2Ch 34:13 Они же \bibemph{были} приставниками над носильщиками и наблюдали над всеми работниками при каждой работе; из левитов же \bibemph{были и} писцы, и надзиратели, и привратники.
\rsbpar\vs 2Ch 34:14 Когда вынимали они серебро, принесенное в дом Господень, тогда Хелкия священник нашел книгу закона Господня, \bibemph{данную} рукою Моисея.
\vs 2Ch 34:15 И начал Хелкия, и сказал Шафану писцу: книгу закона нашел я в доме Господнем. И подал Хелкия ту книгу Шафану.
\vs 2Ch 34:16 И понес Шафан книгу к царю, и принес при этом царю известие: все, что поручено рабам твоим, они делают;
\vs 2Ch 34:17 и высыпали серебро, найденное в доме Господнем, и передали его в руки приставникам и в руки производителям работ.
\vs 2Ch 34:18 И \bibemph{также} донес Шафан писец царю, говоря: книгу дал мне Хелкия священник. И читал ее Шафан перед царем.
\vs 2Ch 34:19 Когда услышал царь слова закона, то разодрал одежды свои.
\vs 2Ch 34:20 И дал царь повеление Хелкии и Ахикаму, сыну Шафанову, и Авдону, сыну Михея, и Шафану писцу, и Асаии, слуге царскому, говоря:
\vs 2Ch 34:21 пойдите, вопросите Господа за меня и за оставшихся у Израиля и за Иуду о словах сей найденной книги, потому что велик гнев Господа, который воспылал на нас за то, что не соблюдали отцы наши слова Господня, чтобы поступать по всему написанному в книге сей.
\vs 2Ch 34:22 И пошел Хелкия и те, которые от царя, к Олдане пророчице, жене Шаллума, сына Тавкегафа, сына Хасры, хранителя одежд,~--- а жила она во второй части Иерусалима,~--- и говорили с нею об этом.
\vs 2Ch 34:23 И она сказала им: так говорит Господь Бог Израилев: скажите тому человеку, который послал вас ко мне:
\vs 2Ch 34:24 так говорит Господь: вот Я наведу бедствие на место сие и на жителей его все проклятия, написанные в книге, которую читали пред лицем царя Иудейского,
\vs 2Ch 34:25 за то, что они оставили Меня и кадили богам другим, чтобы прогневлять Меня всеми делами рук своих. И гнев Мой возгорится над местом сим и не угаснет.
\vs 2Ch 34:26 А царю Иудейскому, пославшему вас вопросить Господа, так скажите: так говорит Господь Бог Израилев о словах, которые ты слышал:
\vs 2Ch 34:27 так как смягчилось сердце твое, и ты смирился пред Богом, услышав слова Его о месте сем и о жителях его,~--- и ты смирился предо Мною, и разодрал одежды свои, и плакал предо Мною, то и Я услышал \bibemph{тебя}, говорит Господь.
\vs 2Ch 34:28 Вот Я приложу тебя к отцам твоим, и положен будешь в гробницу твою в мире, и не увидят глаза твои всего того бедствия, которое Я наведу на место сие и на жителей его. И принесли царю ответ.
\rsbpar\vs 2Ch 34:29 И послал царь, и собрал всех старейшин Иудеи и Иерусалима,
\vs 2Ch 34:30 и пошел царь в дом Господень, и \bibemph{с ним} все Иудеи и жители Иерусалима, и священники и левиты, и весь народ, от большого до малого; и он прочитал вслух их все слова книги завета, найденной в доме Господнем.
\vs 2Ch 34:31 И стал царь на месте своем, и заключил завет пред лицем Господа последовать Господу и соблюдать заповеди Его и откровения Его, и уставы Его, от всего сердца своего и от всей души своей, чтобы выполнить слова завета, написанные в книге сей.
\vs 2Ch 34:32 И велел царь подтвердить \bibemph{это} всем находившимся в Иерусалиме и в земле Вениаминовой; и стали поступать жители Иерусалима по завету Бога, Бога отцов своих.
\vs 2Ch 34:33 И изверг Иосия все мерзости из всех земель, которые у сынов Израилевых, и повелел всем, находившимся в \bibemph{земле} Израилевой служить Господу Богу своему. И во все дни \bibemph{жизни} его они не отступали от Господа Бога отцов своих.
\vs 2Ch 35:1 И совершил Иосия в Иерусалиме пасху Господу, и закололи пасхального агнца в четырнадцатый \bibemph{день} первого месяца.
\vs 2Ch 35:2 И поставил он священников на местах их, и ободрял их на служение в доме Господнем,
\vs 2Ch 35:3 и сказал левитам, наставникам всех Израильтян, посвященным Господу: поставьте ковчег святый в храме, который построил Соломон, сын Давидов, царь Израилев; нет вам нужды носить \bibemph{его} на раменах; служите теперь Господу Богу нашему и народу Его Израилю;
\vs 2Ch 35:4 станьте по поколениям вашим, по чередам вашим, как предписано Давидом, царем Израилевым, и как предписано Соломоном, сыном его,
\vs 2Ch 35:5 и стойте во святилище, по распределениям поколений у братьев ваших, сынов народа, и по разделению поколений у левитов,
\vs 2Ch 35:6 и заколите пасхального агнца, и освятитесь, и приготовьте его для братьев ваших, поступая согласно со словом Господним чрез Моисея.
\vs 2Ch 35:7 И дал Иосия в дар сынам народа, всем, находившимся там, из мелкого скота агнцев и козлов молодых, все для жертвы пасхальной, числом тридцать тысяч и три тысячи волов. Это из имущества царя.
\vs 2Ch 35:8 И князья его по усердию давали в дар народу, священникам и левитам: Хелкия и Захария и Иехиил, начальствующие в доме Божием, дали священникам для жертвы пасхальной две тысячи шестьсот [овец, агнцев и козлов] и триста волов;
\vs 2Ch 35:9 и Хонания, и Шемаия, и Нафанаил, братья его, и Хашавия, и Иеиел, и Иозавад, начальники левитов, подарили левитам для жертвы пасхальной [овец] пять тысяч и пятьсот волов.
\rsbpar\vs 2Ch 35:10 Так устроено было служение. И стали священники на место свое и левиты по чередам своим, по повелению царскому;
\vs 2Ch 35:11 и закололи пасхального агнца. И кропили священники \bibemph{кровью}, принимая ее из рук левитов, а левиты снимали кожу;
\vs 2Ch 35:12 и распределили \bibemph{назначенное} для всесожжения, чтобы раздать то по отделениям поколений у сынов народа, для принесения Господу, как написано в книге Моисеевой. То же \bibemph{сделали} и с волами.
\vs 2Ch 35:13 И испекли пасхального агнца на огне, по уставу; и священные жертвы сварили в котлах, горшках и кастрюлях, и поспешно раздали всему народу,
\vs 2Ch 35:14 а после приготовили для себя и для священников, ибо священники, сыны Аароновы, \bibemph{заняты были} приношением всесожжения и туков до ночи; потому-то и готовили левиты для себя и для священников, сынов Аароновых.
\vs 2Ch 35:15 И певцы, сыновья Асафовы, \bibemph{оставались} на местах своих, по установлению Давида и Асафа, и Емана и Идифуна, прозорливца царского, и привратники у каждых ворот: не для чего \bibemph{было} им отходить от служения своего, так как братья их левиты готовили для них.
\vs 2Ch 35:16 Так устроено было все служение Господу в тот день, чтобы совершить пасху и принести всесожжения на жертвеннике Господнем, по повелению царя Иосии.
\vs 2Ch 35:17 И совершали сыны Израилевы, находившиеся \bibemph{там}, пасху в то время и праздник опресноков в течение семи дней.
\vs 2Ch 35:18 И не была совершаема такая пасха у Израиля от дней Самуила пророка; и из всех царей Израилевых ни один не совершал такой пасхи, какую совершил Иосия, и священники, и левиты, и все Иудеи, и Израильтяне, \bibemph{там} находившиеся, и жители Иерусалима.
\vs 2Ch 35:19 В восемнадцатый год царствования Иосии совершена сия пасха.
\rsbpar\vs 2Ch 35:20 После всего того, что сделал Иосия в доме \bibemph{Божием} [и как сжег огнем царь Иосия и чревовещателей, и волхвов, и капища, и идолов, и дубравы, бывшие в Иерусалиме и Иудее, чтобы утвердить слова закона, написанные в книге, которую нашел Хелкия священник в доме Господнем, не было подобного ему прежде него, кто обратился бы к Господу всем сердцем своим, и всею душею своею, и всею крепостию своею, по всему закону Моисееву; не восстал и после него подобный ему. Однако же не отвратился Господь от великой ярости гнева Своего,~--- ярости, которою разгневался Господь на Иудею за все оскорбления, которыми прогневал Манассия. И сказал Господь: и Иуду отвергну от лица Моего, как отверг дом Израилев, и отвергну город Иерусалим, который избрал, и храм, о котором сказал: будет там имя Мое,] пошел Нехао, царь Египетский, на войну к Кархемису на Евфрате; и Иосия вышел навстречу ему.
\vs 2Ch 35:21 И послал к нему \bibemph{Нехао} послов сказать: что мне и тебе, царь Иудейский? Не против тебя теперь \bibemph{иду я}, но туда, где у меня война. И Бог повелел мне поспешать; не противься Богу, Который со мною, чтоб Он не погубил тебя.
\vs 2Ch 35:22 Но Иосия не отстранился от него, а приготовился, чтобы сразиться с ним, и не послушал слов Нехао от лица Божия и выступил на сражение на равнину Мегиддо.
\vs 2Ch 35:23 И выстрелили стрельцы в царя Иосию, и сказал царь слугам своим: уведите меня, потому что я тяжело ранен.
\vs 2Ch 35:24 И свели его слуги его с колесницы, и посадили его в другую повозку, которая \bibemph{была} у него, и отвезли его в Иерусалим. И умер он, и похоронен в гробницах отцов своих. И вся Иудея и Иерусалим оплакали Иосию.
\vs 2Ch 35:25 Оплакал Иосию и Иеремия в песне плачевной; и говорили все певцы и певицы об Иосии в плачевных песнях своих, \bibemph{известных} до сего дня, и передали их в употребление у Израиля; и вот они вписаны в \bibemph{книгу} плачевных песней.
\rsbpar\vs 2Ch 35:26 Прочие деяния Иосии и добродетели его, согласные с предписанным в законе Господнем,
\vs 2Ch 35:27 и деяния его, первые и последние, описаны в книге царей Израильских и Иудейских.
\vs 2Ch 36:1 И взял народ земли Иоахаза, сына Иосиина, [и помазали его] и воцарили его, вместо отца его, в Иерусалиме.
\vs 2Ch 36:2 Двадцати трех лет был Иоахаз, когда воцарился, и три месяца царствовал в Иерусалиме. [Имя матери его~--- Амитал, дочь Иеремии из Ловны. И сделал он лукавое пред Господом по всему, что сделали отцы его. И оковал его фараон Нехао в Девлафе, в земле Емафской, чтобы не царствовать ему в Иерусалиме.]
\vs 2Ch 36:3 И низложил его царь Египетский в Иерусалиме [и привел его царь в Египет], и наложил на землю пени сто талантов серебра и талант золота.
\vs 2Ch 36:4 И воцарил царь Египетский над Иудеею и Иерусалимом Елиакима, брата его, и переменил имя его на Иоакима, а Иоахаза, брата его, взял Нехао и отвел его в Египет [и он умер там. И серебро и золото давал фараону: тогда земля начала давать серебро по слову фараона, и каждый, по власти, требовал серебра и золота от народа земли для дани фараону Нехао].
\rsbpar\vs 2Ch 36:5 Двадцати пяти лет \bibemph{был} Иоаким, когда воцарился, и одиннадцать лет царствовал в Иерусалиме [имя матери его Зехора, дочь Нириева из Рамы]. И делал он неугодное в очах Господа Бога своего [по всему, что делали отцы его. Во дни его пришел Навуходоносор, царь Вавилонский, на землю, и он служил ему три года и отступил от него. И послал Господь на них Халдеев и разбойников Сирских, и разбойников Моавитских, и сынов Аммоновых и Самарийских, и отступили по слову сему,~--- по слову Господа устами рабов Его, пророков. Впрочем гнев Господа был на Иуде, чтоб отвергнуть его от лица Его, за все грехи Манассии, которые он сделал, и за кровь неповинную, которую пролил Иоаким и наполнил Иерусалим неповинною кровью. Но не восхотел Господь искоренить их].
\vs 2Ch 36:6 Против него вышел Навуходоносор, царь Вавилонский, и оковал его оковами, чтоб отвести его в Вавилон.
\vs 2Ch 36:7 И часть сосудов дома Господня перенес Навуходоносор в Вавилон и положил их в капище своем в Вавилоне.
\rsbpar\vs 2Ch 36:8 Прочие дела Иоакима и мерзости его, какие он делал и какие найдены в нем, описаны в книге царей Израильских и Иудейских. [И почил Иоаким с отцами своими, и погребен был в Ганозане с отцами своими.] И воцарился Иехония, сын его, вместо него.
\rsbpar\vs 2Ch 36:9 Восемнадцати лет \bibemph{был} Иехония, когда воцарился, и три месяца и десять дней царствовал в Иерусалиме, и делал он неугодное в очах Господних.
\vs 2Ch 36:10 По прошествии года послал царь Навуходоносор и велел взять его в Вавилон вместе с драгоценными сосудами дома Господня, и воцарил над Иудеею и Иерусалимом Седекию, брата его.
\rsbpar\vs 2Ch 36:11 Двадцати одного года \bibemph{был} Седекия, когда воцарился, и одиннадцать лет царствовал в Иерусалиме,
\vs 2Ch 36:12 и делал он неугодное в очах Господа Бога своего. Он не смирился пред Иеремиею пророком, \bibemph{пророчествовавшим} от уст Господних,
\vs 2Ch 36:13 и отложился от царя Навуходоносора, взявшего клятву с него \bibemph{именем} Бога,~--- и сделал упругою шею свою и ожесточил сердце свое до того, что не обратился к Господу Богу Израилеву.
\vs 2Ch 36:14 Да и все начальствующие над священниками и над народом много грешили, подражая всем мерзостям язычников, и сквернили дом Господа, который Он освятил в Иерусалиме.
\vs 2Ch 36:15 И посылал к ним Господь Бог отцов их, посланников Своих от раннего утра, потому что Он жалел Свой народ и Свое жилище.
\vs 2Ch 36:16 Но они издевались над посланными от Бога и пренебрегали словами Его, и ругались над пророками Его, доколе не сошел гнев Господа на народ Его, так что не было \bibemph{ему} спасения.
\vs 2Ch 36:17 И Он навел на них царя Халдейского,~--- и тот умертвил юношей их мечом в доме святыни их и не пощадил [ни Седекии,] ни юноши, ни девицы, ни старца, ни седовласого: все предал \bibemph{Бог} в руку его.
\vs 2Ch 36:18 И все сосуды дома Божия, большие и малые, и сокровища дома Господня, и сокровища царя и князей его, все принес он в Вавилон.
\vs 2Ch 36:19 И сожгли дом Божий, и разрушили стену Иерусалима, и все чертоги его сожгли огнем, и все драгоценности его истребили.
\vs 2Ch 36:20 И переселил он оставшихся от меча в Вавилон, и были они рабами его и сыновей его, до воцарения царя Персидского,
\vs 2Ch 36:21 доколе, во исполнение слова Господня, \bibemph{сказанного} устами Иеремии, земля не отпраздновала суббот своих. Во все дни запустения она субботствовала до исполнения семидесяти лет.
\rsbpar\vs 2Ch 36:22 А в первый год Кира, царя Персидского, во исполнение слова Господня, \bibemph{сказанного} устами Иеремии, возбудил Господь дух Кира, царя Персидского, и он велел объявить по всему царству своему, словесно и письменно, и сказать:
\vs 2Ch 36:23 так говорит Кир, царь Персидский: все царства земли дал мне Господь Бог небесный, и Он повелел мне построить Ему дом в Иерусалиме, что в Иудее. Кто есть из вас~--- из всего народа Его, [да будет] Господь Бог его с ним, и пусть он туда идет.
\chhdr{Молитва Манассии, царя Иудейского, когда он содержался в плену в Вавилоне.\fns{Переведена с греческого; в еврейском тексте ее нет.}}
\vs 2Ch 37:1 Господи Вседержителю, Боже отцев наших, Авраама и Исаака и Иакова, и семени их праведного,
\vs 2Ch 37:2 сотворивший небо и землю со всем благолепием их, связавший море словом повеления Твоего, заключивший бездну и запечатавший ее страшным и славным именем Твоим, которого все боятся, и трепещут от лица силы Твоея, потому что никто не может устоять пред великолепием славы Твоея, и нестерпим гнев
\vs 2Ch 37:3 прещения Твоего на грешников!
\vs 2Ch 37:4 Но безмерна и неисследима милость обетования Твоего,
\vs 2Ch 37:5 ибо Ты Господь вышний, благий, долготерпеливый и многомилостивый и кающийся о злобах человеческих. Ты, Господи, по множеству Твоей благости, обещал покаяние
\vs 2Ch 37:6 и отпущение согрешившим Тебе, и множеством щедрот Твоих определил покаяние грешникам во спасение. Итак Ты, Господи, Боже праведных, не положил покаяния праведным
\vs 2Ch 37:7 Аврааму и Исааку и Иакову, не согрешившим Тебе, но положил покаяние мне грешнику, потому что я согрешил паче числа песка морского.
\vs 2Ch 37:8 Многочисленны беззакония мои, Господи, многочисленны беззакония мои, и я недостоин взирать и смотреть на высоту небесную от множества неправд моих. Я согбен многими железными узами,
\vs 2Ch 37:9 так что не могу поднять головы моей, и нет мне отдохновения, потому что прогневал Тебя и сделал пред Тобою злое:
\vs 2Ch 37:10 не исполнил воли Твоей, не сохранил повелений Твоих, поставил мерзости и умножил соблазны. И ныне преклоняю колени сердца моего, умоляя Тебя о благости.
\vs 2Ch 37:11 Согрешил я, Господи, согрешил, и беззакония мои я знаю, но прошу, молясь Тебе: отпусти мне, Господи, отпусти мне, и не погуби меня с беззакониями моими и не осуди меня в преисподнюю. Ибо Ты Бог, Бог кающихся, и на мне яви всю благость Твою, спасши меня недостойного по великой милости Твоей, и буду прославлять Тебя во все дни жизни моей,
\vs 2Ch 37:12 потому что Тебя славят все силы небесные, и Твоя слава во веки веков. Аминь.

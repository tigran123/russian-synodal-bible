\bibbookdescr{8Sb}{
  inline={Восьмая книга Сивилл},
  toc={8-я Сивилл},
  bookmark={8-я Сивилл},
  header={8-я Сивилл},
  abbr={8~Сив}
}
\vs 8Sb 1:1 Гнев Господень ужасный грядет непокорному свету! 

\vs 8Sb 1:2 Все, чем Бог угрожает последнему веку, скажу я 

\vs 8Sb 1:3 Жителям каждого града, всем будет пророчество ясным. 

\vs 8Sb 1:4 С той поры, когда башня упала, а вслед человечий 

\vs 8Sb 1:5 Множеством говоров стал язык, внезапно распавшись, 

\vs 8Sb 1:6 Царство Египта вначале возникло, потом государства 

\vs 8Sb 1:7 Персов, Мидян, Эфиопов, в Ассирии вкруг Вавилона, 

\vs 8Sb 1:8 И в Македонии гордой  про всех уже сказано мною; 

\vs 8Sb 1:9 Ныне же я обращусь к пресловутой земле Италийской.

\vs 8Sb 1:10 Множество зол в конце времен причинит она смертным: 

\vs 8Sb 1:11 Всюду сведет на нет старания разных народов, 

\vs 8Sb 1:12 Многих отважных мужей она в плен угонит на Запад, 

\vs 8Sb 1:13 Все подчинит и народам свои предпишет законы. 

\vs 8Sb 1:14 Долго пусть мелют зерно жернова Господни, но мелко.

\vs 8Sb 1:15 Сгинет все от огня, и в тонкий пух превратятся 

\vs 8Sb 1:16 Гор высоких вершины, а всякая плоть станет пылью. 

\vs 8Sb 1:17 Алчность и безразсудство  всех бед и несчастий начало. 

\vs 8Sb 1:18 К золоту и серебру, к обманчивым, люди стремятся, 

\vs 8Sb 1:19 Лучшим, что есть на земле, металлы эти считая:

\vs 8Sb 1:20 Лучше, чем солнца сиянье, и лучше, чем небо иль море, 

\vs 8Sb 1:21 Или земля, что, простершись широко, все порождает, 

\vs 8Sb 1:22 Иль даже Бог, сотворивший весь мир и все подающий; 

\vs 8Sb 1:23 Веру и благочестье поставили ниже металлов. 

\vs 8Sb 1:24 Это безумье  источник неправды и смуты зачинщик,

\vs 8Sb 1:25 Мирной жизни оно враждебно, а войнам  причина, 

\vs 8Sb 1:26 Ведь от него и отцы с сыновьями своими враждуют, 

\vs 8Sb 1:27 Также и брак не в чести у тех, кто золото любит.

\vs 8Sb 1:28 Всюду на землях  границы, и всюду стражи на море, 

\vs 8Sb 1:29 Делится все хитроумно меж теми, кто златом владеет,

\vs 8Sb 1:30 Будто навечно хотят забрать плодоносную землю. 

\vs 8Sb 1:31 Грабят они бедняков, лишь бы только именье расширить, 

\vs 8Sb 1:32 Тех, кто им отдал свое, в рабов обращая хвастливо. 

\vs 8Sb 1:33 Если б земля не была далека от звездного неба, 

\vs 8Sb 1:34 Не был бы также и свет одинаково людям доступным,

\vs 8Sb 1:35 Но только тот, кто богат, покупать его мог бы за деньги, 

\vs 8Sb 1:36 Новый же мир сотворить для бедных Богу пришлось бы. 

\vs 8Sb 1:37 Рим надменный, тебе испытать когда-то придется 

\vs 8Sb 1:38 Неба удар справедливый, ты первый шею преклонишь, 

\vs 8Sb 1:39 Рухнешь наземь, огнем истребишься до основанья,

\vs 8Sb 1:40 Лежа на собственных землях, и все богатство погибнет, 

\vs 8Sb 1:41 А во дворцах будут жить лишь дикие волки и лисы; 

\vs 8Sb 1:42 Станешь пустынею, словно и не было города вовсе. 

\vs 8Sb 1:43 Где твой палладий? И где тот бог, что пришел бы на помощь, 

\vs 8Sb 1:44 Медный, иль золотой, иль каменный? Где же сената

\vs 8Sb 1:45 Постановления? Где потомки Кроноса, Реи

\vs 8Sb 1:46 Или же Зевса и всех, кто так были чтимы тобою? 

\vs 8Sb 1:47 То  божества без души, подобия трупов безсильных, 

\vs 8Sb 1:48 Скроет которых земля несчастного Крита, и станут 

\vs 8Sb 1:49 С гордостью там почитать мертвецов, что не чувствуют больше.

\vs 8Sb 1:50 После трижды пяти царей, о изнеженный город, 

\vs 8Sb 1:51 Что покорят весь мир от Восхода и вплоть до Заката, 

\vs 8Sb 1:52 Вождь воцарится седой, соименник ближнего моря. 

\vs 8Sb 1:53 Грозной ногою пройдет он по миру, дары добывая, 

\vs 8Sb 1:54 Многое множество злата, а с ним серебра еще больше

\vs 8Sb 1:55 Он у врагов заберет и, награбив, домой возвратится. 

\vs 8Sb 1:56 Царь тот в святилищах магов участником станет мистерий, 

\vs 8Sb 1:57 Мальчика сделает богом, но все богов почитанье 

\vs 8Sb 1:58 Сам низпровергнет, открыв всю лживость мистерий для смертных. 

\vs 8Sb 1:59 Будет ужасное время, когда сам Ужасный погибнет.

\vs 8Sb 1:60 Скажет однажды народ: О город, падет твоя сила!  

\vs 8Sb 1:61 Ибо почувствует вдруг дурного дня приближенье. 

\vs 8Sb 1:62 Горько заплачут тогда, предвидя удел твой несчастный, 

\vs 8Sb 1:63 Вместе родители все и все неразумные дети, 

\vs 8Sb 1:64 Скорбным рыданием их огласятся два берега Тибра.

\vs 8Sb 1:65 Трое за тем царем в последние дни будут править, 

\vs 8Sb 1:66 Имя собою исполнив Небесного Бога, чья сила 

\vs 8Sb 1:67 Вплоть до скончания всех времен пребудет, как ныне. 

\vs 8Sb 1:68 Скипетр удержит надолго один из них, муж престарелый, 

\vs 8Sb 1:69 Царь, сожаленья достойный, который сокровища мира

\vs 8Sb 1:70 Все в чертогах своих укроет, чтобы раздать их

\vs 8Sb 1:71 Людям, когда от границ земных беглец возвратится, 

\vs 8Sb 1:72 Мать погубивший; тогда богатой Азия станет. 

\vs 8Sb 1:73 Снимешь в те дни ты наряд, с широкою красной каймою 

\vs 8Sb 1:74 И облечешься, печалясь, в одежду скорби глубокой,

\vs 8Sb 1:75 О надменное царство, о дочь Латинского Рима! 

\vs 8Sb 1:76 Славною гордость твоя надменная быть перестанет, 

\vs 8Sb 1:77 Будешь лежать распростершись и больше тебе не подняться. 

\vs 8Sb 1:78 Честь легионов твоих падет со всеми орлами, 

\vs 8Sb 1:79 Где ж твоя сила? И кто союзником быть согласится,

\vs 8Sb 1:80 Пред неразумьем твоим безбожным главу преклоняя? 

\vs 8Sb 1:81 Тут среди смертных по всей земле начнется смятенье, 

\vs 8Sb 1:82 В день, как придет Вседержитель, возсядет на троне и будет 

\vs 8Sb 1:83 Души судить живых и мертвых  суд над вселенной. 

\vs 8Sb 1:84 Станут тогда немилы родители детям, а дети

\vs 8Sb 1:85 Тем, кто родил их, от горя нежданного и от нечестья. 

\vs 8Sb 1:86 Скрежет зубов тебя ждет, раскаянье и покоренье, 

\vs 8Sb 1:87 Грады рухнут когда и земля провалы разверзнет. 

\vs 8Sb 1:88 Тут пурпурный дракон по морским приплывет к тебе волнам, 

\vs 8Sb 1:89 Чревом наполненным он детей твоих вскармливать будет

\vs 8Sb 1:90 В дни, когда голод придет и войны гражданские грянут; 

\vs 8Sb 1:91 Все это значит, что день последний этого мира 

\vs 8Sb 1:92 Близок, и вскоре все люди на Суд будут призваны Богом. 

\vs 8Sb 1:93 Римлянам первым придется познать Его гнев безпощадный, 

\vs 8Sb 1:94 Ждет их кровавое время и в жизни сплошные несчастья.

\vs 8Sb 1:95 Горе тебе, о земля Италийская  варваров племя!  

\vs 8Sb 1:96 Ты позабыла о том, что, на свет появившись нагою, 

\vs 8Sb 1:97 И недостойной, назад уйдешь ты опять без одежды 

\vs 8Sb 1:98 В то же самое место, где суд над тобою свершится, 

\vs 8Sb 1:99 Ибо неправедно ты сама осуждала \ldots

\vs 8Sb 1:100 Руки твои словно руки Гигантов были, когда ты

\vs 8Sb 1:101 Шла в этот мир с высоты  теперь под землей твое место. 

\vs 8Sb 1:102 Нефтью, серой, асфальтом, великим огнем истребишься 

\vs 8Sb 1:103 И превратишься ты в груду горячего пепла навеки. 

\vs 8Sb 1:104 Каждый, кто ни посмотрит, услышит, из Тартара вопли,

\vs 8Sb 1:105 Громкие, полные скорби, и скрежет зубов, и удары 

\vs 8Sb 1:106 Жалких ладоней твоих, что в грудь безбожную бьются.

\vs 8Sb 1:107 Ночь одинаково всех ожидает  богатых и бедных, 

\vs 8Sb 1:108 Мы из земли нагими выходим, нагими же в землю 

\vs 8Sb 1:109 Снова ложимся, как только исполнится срок нашей жизни.

\vs 8Sb 1:110 Там уже нет никаких рабов, ни господ, ни тиранов, 

\vs 8Sb 1:111 Нет ни царей, ни вождей надменных и полных гордыни, 

\vs 8Sb 1:112 Ни хитроумных витий, ни начальников нет лихоимцев, 

\vs 8Sb 1:113 Больше уж на алтарях не прольется жертвенной крови, 

\vs 8Sb 1:114 Не зазвучат ни тимпан, ни кимвал \ldots

\vs 8Sb 1:115 Флейты многоотверстной безумных звуков не станет, 

\vs 8Sb 1:116 Нет и песни свирели, подобной извивам дракона, 

\vs 8Sb 1:117 Нет и варварских труб, что людям войну возвещают, 

\vs 8Sb 1:118 И не упьется вином уж никто на пирах нечестивых, ъ

\vs 8Sb 1:119 Нет ни плясок, ни пенья кифары; исчезнет коварство,

\vs 8Sb 1:120 Ссоры и гнев многовидный, и острых ножей не имеют 

\vs 8Sb 1:121 Те, кто жизнь завершил; единый лишь век остается. 

\vs 8Sb 1:122 Ключник великой ограды, привратник Божьего трона \ldots

\vs 8Sb 1:123 Идолы пусть вас украсят из золота, дерева, камня, 

\vs 8Sb 1:124 Пусть простоят до тех пор, когда день самый горький настанет,

\vs 8Sb 1:125 Чтоб твою кару узнать, о Рим, и вопли услышать. 

\vs 8Sb 1:126 Шею больше как раб под ярмо твое не преклонят 

\vs 8Sb 1:127 Ни Сириец, ни Эллин, ни варвар, ни племя иное. 

\vs 8Sb 1:128 Ждет разграбленье тебя, воздаcтся за то, что творил ты, 

\vs 8Sb 1:129 Будешь от страха стенать, пока за все не отплатишь.

\vs 8Sb 1:130 Целый мир над тобой, опозоренным, справит победу \ldots

\vs 8Sb 1:131 \ldots\ И в поколенье шестом царей Латинских угаснут 

\vs 8Sb 1:132 Жизни остатки, и руки удерживать скиптры не смогут. 

\vs 8Sb 1:133 Будет властвовать царь другой из того поколенья, 

\vs 8Sb 1:134 Скиптры все подчинит и землю всю покорит он, 

\vs 8Sb 1:135 Править будет один, но по воле Всевышнего Бога; 

\vs 8Sb 1:136 Дети и внуки его составят род нерушимый. 

\vs 8Sb 1:137 Так назначено Богом, когда круг времен совершится 

\vs 8Sb 1:138 И трижды пять царей над Египтом правленье закончат.

\vs 8Sb 1:139 Как подойдет к концу век птицы Феникса пятый \ldots

\vs 8Sb 1:140 \ldots\ Явится тут погубитель родов и племен без разбора, 

\vs 8Sb 1:141 И между ними  Евреев, Арес уничтожит Ареса, 

\vs 8Sb 1:142 Сам ведь на смерть обречет он угрозы надменные Рима. 

\vs 8Sb 1:143 Прежде цветущая, пала могучая Римлян держава, 

\vs 8Sb 1:144 Издавна всех городов, окрест лежащих, царица.

\vs 8Sb 1:145 Больше не будет победы для тучной Римской равнины, 

\vs 8Sb 1:146 После того как придет из Азии войск предводитель. 

\vs 8Sb 1:147 Все совершив, он захватит великий град; и в то время, 

\vs 8Sb 1:148 Как трижды триста исполнишь годов и еще сорок восемь, 

\vs 8Sb 1:149 Участь несчастная ждет тебя, от насилья погибнешь 

\vs 8Sb 1:150 И даже имя твое навеки будет забыто.

\vs 8Sb 1:151 Трижды несчастной, увы мне! когда же я день тот увижу, 

\vs 8Sb 1:152 Гибель несущий тебе, о Рим, и роду Латинян? 

\vs 8Sb 1:153 Радуйся, если желаешь, тому, кто, тайно рожденный 

\vs 8Sb 1:154 В Азии где-то, взошел на Троянскую колесницу, 

\vs 8Sb 1:155 Гневом кипя. Но когда перешеек Истмийский пробьет он, 

\vs 8Sb 1:156 Все озирая вокруг, враждебный ко всем, через море 

\vs 8Sb 1:157 Переплывет  тогда вслед за зверем великим польется 

\vs 8Sb 1:158 Черная кровь, но собака догонит губителя стада, 

\vs 8Sb 1:159 Льва; и, скиптра лишенный, пойдет он в царство Аида.

\vs 8Sb 1:160 Будет Родосцев несчастье последнее самым ужасным, 

\vs 8Sb 1:161 Фивы позорный захват ожидает затем, а Египет 

\vs 8Sb 1:162 От преступлений вождей своих негодных погибнет. 

\vs 8Sb 1:163 Тех же из смертных, кто смог избежать погибели страшной, 

\vs 8Sb 1:164 Трижды счастливыми нужно считать, и четырежды даже.

\vs 8Sb 1:165 Рим будет улочкой жалкой, а Делос невидимым станет, 

\vs 8Sb 1:166 Самос в песок превратится \ldots

\vs 8Sb 1:167 После придут, наконец, и к Персам великие беды 

\vs 8Sb 1:168 Карой за нрав их надменный  и вся гордыня исчезнет.

\vs 8Sb 1:169 Вождь святой подчинит себе все скиптры земные 

\vs 8Sb 1:170 С этой поры навсегда, и умерших от сна он пробудит.

\vs 8Sb 1:171 Волей Всевышнего жребий несчастный выпадет Риму:

\vs 8Sb 1:172 Всех, кто в городе этом живет, обрек Он на гибель.

\vs 8Sb 1:173 Но не хотят покориться, хоть лучше бы им это было.

\vs 8Sb 1:174 В пору, как день тот созреет, великой бедою чреватый  

\vs 8Sb 1:175 Голодом, мором и шумом войны, для людей нестерпимым,

\vs 8Sb 1:176 Снова тогда созовет несчастный, что раньше владыкой

\vs 8Sb 1:177 Был над Римом, совет, чтобы гибель тому уготовить \ldots

\vs 8Sb 1:178 \ldots\ Листья, едва распустившись, тотчас же станут сухими,

\vs 8Sb 1:179 Только на твердые скалы дожди с небосвода польются, 

\vs 8Sb 1:180 Почве же только ветра и огонь достанется жаркий, 

\vs 8Sb 1:181 Много семян оттого зазря пропадет в целом мире \ldots

\vs 8Sb 1:182 \ldots\ Зло продолжают творить, ибо всякий стыд потеряли 

\vs 8Sb 1:183 И не боятся уже ни людского, ни Божьего гнева, 

\vs 8Sb 1:184 Срам утратили все, променяли его на безстыдство; 

\vs 8Sb 1:185 Много насилий творят, закон попирают, тираны, 

\vs 8Sb 1:186 Лгут, обещаний не держат, ни слова правды не молвят, 

\vs 8Sb 1:187 Любят надутые речи, а веру поносят и гонят; 

\vs 8Sb 1:188 Нет для них насыщенья в богатстве, стремятся безстыдно 

\vs 8Sb 1:189 Больше и больше собрать  и сгинут под гнетом тиранов.

\vs 8Sb 1:190 Звезды все упадут прямо с неба в пучину морскую, 

\vs 8Sb 1:191 Но вместо них взойдут другие, одну из которых, 

\vs 8Sb 1:192 Что ярче всех заблестит, назовут кометою люди, 

\vs 8Sb 1:193 Знаменьем станет она войны и множества бедствий.

\vs 8Sb 1:194 Нет, не хотела б я жить во дни правленья нечистой, 

\vs 8Sb 1:195 Страстно желала б, напротив, когда небесная милость 

\vs 8Sb 1:196 В мире царицею станет, а всех коварных злодеев 

\vs 8Sb 1:197 Сын святой закует в оковы и в страшную бездну 

\vs 8Sb 1:198 Бросит  и смертных внезапно обнимет дом деревянный.

\vs 8Sb 1:199 После того как сойдет поколенье десятое в Тартар, 

\vs 8Sb 1:200 Женщина властью великой тогда завладеет, и много 

\vs 8Sb 1:201 Бед Господь низпошлет, когда она увенчает 

\vs 8Sb 1:202 Царским венцом главу  все в мире изменится сразу. 

\vs 8Sb 1:203 Жаркое солнце свой бег являть будет людям и ночью, 

\vs 8Sb 1:204 Звезды с неба исчезнут, и бешеный вихрь пронесется, 

\vs 8Sb 1:205 Опустошая весь мир, и мертвые всюду воскреснут, 

\vs 8Sb 1:206 Быстро хромые пойдут, и слышать смогут глухие, 

\vs 8Sb 1:207 Зренье получат слепцы, обретут дар речи немые. 

\vs 8Sb 1:208 Жизнь и богатство тогда для смертных общими будут, 

\vs 8Sb 1:209 Общей и вся земля; и, быть перестав разделенной 

\vs 8Sb 1:210 Стенами и рубежами, сама даст плод изобильный 

\vs 8Sb 1:211 И родники молока белоснежного, сладкого меда

\vs 8Sb 1:212 Даст и вино источит \ldots

\vs 8Sb 1:213 Суд бессмертного Бога \ldots

\vs 8Sb 1:214 Бог переменит все времена \ldots

\vs 8Sb 1:215 Зиму сделает летом; да сбудутся все предсказанья. 

\vs 8Sb 1:216 Но после гибели мира \ldots

\vs 8Sb 1:217 Из земли источит близость Судного дня капли пота,

\vs 8Sb 1:218 И снизойдет с небес тот Царь, что вечно пребудет.

\vs 8Sb 1:219 Суд учинит Он великий над миром и всякою плотью, 

\vs 8Sb 1:220 Узрят и верные Бога, и все неверные тоже,

\vs 8Sb 1:221 С высей как спустится Он в конце времен со святыми.

\vs 8Sb 1:222 Души плотских людей придут для суда к Его трону;

\vs 8Sb 1:223 Худо придется земле от засухи страшной и терний;

\vs 8Sb 1:224 Разных кумиров и все богатство смертные бросят, 

\vs 8Sb 1:225 И проникающий всюду огонь и сушу, и море

\vs 8Sb 1:226 Сгубит, и небосвод, и крепкие двери Аида.

\vs 8Sb 1:227 Тем, кто был жизни святой, свободы свет засияет;

\vs 8Sb 1:228 Огненной каре навек предаст нечестивцев Всевышний,

\vs 8Sb 1:229 Скажет всякий из них о зле, что тайно творил он, 

\vs 8Sb 1:230 Ибо сердечная тьма озарится светом Господним.

\vs 8Sb 1:231 Скрежет зубовный тогда и плач всеобщий раздастся,

\vs 8Sb 1:232 Солнца померкнет сиянье, и скроются звезд хороводы,

\vs 8Sb 1:233 Небо свернется как свиток, луны мерцанье угаснет,

\vs 8Sb 1:234 Высями станут долины, в низины холмы обратятся. 

\vs 8Sb 1:235 Больше высот на земле губительных вовсе не будет,

\vs 8Sb 1:236 Общее примут обличье равнины и горные кряжи;

\vs 8Sb 1:237 Глади морской не коснутся суда; земля запылает,

\vs 8Sb 1:238 А источники рек и бурные воды изсякнут.

\vs 8Sb 1:239 С неба труба пропоет печальным голосом песню, 

\vs 8Sb 1:240 Страшный позор несчастных оплачет и мира мученья,

\vs 8Sb 1:241 Пропасти, в почве разверзшись, покажут Тартара хаос,

\vs 8Sb 1:242 А пред небесным престолом Господним все люди сойдутся.

\vs 8Sb 1:243 С неба потоки огня и серы хлынут на землю.

\vs 8Sb 1:244 Будет для смертных тогда непреложным знаком, печатью, 

\vs 8Sb 1:245 Древом для верных тот рог, о котором так долго мечтали,

\vs 8Sb 1:246 Камень соблазна для мира, но жизнь для мужей справедливых,

\vs 8Sb 1:247 Радость света для званых двенадцатью водами давший,

\vs 8Sb 1:248 Есть у него и жезл железный, чтоб смертными править.

\vs 8Sb 1:249 Сам Господь наш Небесный записан тут акростихами  

\vs 8Sb 1:250 То Спаситель безсмертный и Царь, за нас пострадавший.

\vs 8Sb 1:251 Запечатлел же Его еще Моисей, распростерши 

\vs 8Sb 1:252 Руки святые, когда победил Амалика он верой; 

\vs 8Sb 1:253 Понял тогда народ, что избран Богом и славен 

\vs 8Sb 1:254 Жезл Давыда и Камень, что был заранее предсказан: 

\vs 8Sb 1:255 Тот, кто поверит в Него, сподобится жизни безсмертной.

\vs 8Sb 1:256 Он не во славе придет на суд, но как смертный несчастный, 

\vs 8Sb 1:257 Без красоты, без почета, чтоб дать несчастным надежду; 

\vs 8Sb 1:258 Тленному телу придаст Он форму, неверным дарует 

\vs 8Sb 1:259 Веру небесную Он, и вылепит вновь человека,

\vs 8Sb 1:260 Коего Сам Господь творил Своими руками. 

\vs 8Sb 1:261 Но обманул человека коварный змей и заставил 

\vs 8Sb 1:262 Смертную участь принять и познать благое и злое; 

\vs 8Sb 1:263 Так люди бросили Бога и тленному кланяться стали. 

\vs 8Sb 1:264 Сына в советники взяв изначально, рек Вседержитель:

\vs 8Sb 1:265 Сделаем вместе с Тобою, о Чадо, смертное племя, 

\vs 8Sb 1:266 Слепим его, отразив в нем Наше с Тобою обличье. 

\vs 8Sb 1:267 Я руками теперь, Ты позже словом послужишь 

\vs 8Sb 1:268 Делу Нашему, чтобы оно стало общим твореньем. 

\vs 8Sb 1:269 Помня об этом решенье, сойдет для суда Он на землю,

\vs 8Sb 1:270 В Деву святую вселившись подобным отображеньем, 

\vs 8Sb 1:271 Свет водой даровав через руки того, кто был старше, 

\vs 8Sb 1:272 Все Своим словом творя и любую болезнь исцеляя. 

\vs 8Sb 1:273 Словом же Он успокоит ветра и море разгладит, 

\vs 8Sb 1:274 Всюду покой принесет и веру, по свету скитаясь.

\vs 8Sb 1:275 Он хлебами пятью и рыбой из моря одною

\vs 8Sb 1:276 Целых пять тысяч мужей легко насытит в пустыне; 

\vs 8Sb 1:277 После, собрав те куски, что остались от пищи, наполнит 

\vs 8Sb 1:278 Ими двенадцать корзин, да придет надежда к народам. 

\vs 8Sb 1:279 Вызовет души блаженных и тех несчастных возлюбит,

\vs 8Sb 1:280 Кто, подвергаясь насмешкам, отплатит за зло только благом, 

\vs 8Sb 1:281 Кто нищету возлюбил, кто гоним и бичами терзаем. 

\vs 8Sb 1:282 Все доступно Его уму, Он все видит и слышит, 

\vs 8Sb 1:283 Узрит и то, что таится внутри, обнажит и очистит, 

\vs 8Sb 1:284 Ибо Он сам разуменье, и слух, и зренье всех сущих,

\vs 8Sb 1:285 Он же и Слово-Творец всех форм, и Ему все покорно. 

\vs 8Sb 1:286 Мертвых Он воскресит, исцелит любые болезни. 

\vs 8Sb 1:287 Он попадет, наконец, в безбожные руки неверных, 

\vs 8Sb 1:288 Станут грешной рукой наносить пощечины Богу, 

\vs 8Sb 1:289 Полную яда слюну из грязных уст извергая.

\vs 8Sb 1:290 Спину святую Свою ударам кнута Он подставит, 

\vs 8Sb 1:291 Ибо Он миру придет отдать непорочную Деву. 

\vs 8Sb 1:292 Будет молчанье хранить под ударами, чтоб не узнали, 

\vs 8Sb 1:293 Кто Он, чей и откуда; но к падшим речет Свое слово. 

\vs 8Sb 1:294 И увенчают Его венцом терновым, и станет

\vs 8Sb 1:295 Шип колючий наградой святым избранникам вечной. 

\vs 8Sb 1:296 Во исполненье закона пронзят тростником Его ребра, 

\vs 8Sb 1:297 Ведь подготовил тростник, не простым колеблемый ветром, 

\vs 8Sb 1:298 Душу Его к осужденью, обидам и наказанью. 

\vs 8Sb 1:299 Как совершится все то, что ныне предсказано мною,

\vs 8Sb 1:300 В Нем растворится всецело закон, что прежде народу 

\vs 8Sb 1:301 Дан непокорному был, в словах человечьих записан. 

\vs 8Sb 1:302 Руки раскинет и все, что есть в мире, Он ими обнимет; 

\vs 8Sb 1:303 Желчью кормили Его и уксусом горьким поили, 

\vs 8Sb 1:304 Кару получат они за враждебную трапезу эту.

\vs 8Sb 1:305 В храме порвется завеса, и день превратится внезапно 

\vs 8Sb 1:306 В ночь, что на три часа всю землю мраком покроет. 

\vs 8Sb 1:307 Скрытое бреднями мира, теперь станет ясно: не нужно 

\vs 8Sb 1:308 Больше храм почитать и закон, для людей непонятный  

\vs 8Sb 1:309 Сам ведь на землю сошел Безсмертный Владыка Небесный.

\vs 8Sb 1:310 Спустится после в Аид и для праведных вестником будет 

\vs 8Sb 1:311 Доброй надежды и дня, которым века прекратятся. 

\vs 8Sb 1:312 Смертный удел победит Он, на третий день пробудившись;

\vs 8Sb 1:313 Мертвых покинет тогда и вновь появится в мире, 

\vs 8Sb 1:314 Тем лишь, кто избран, сперва открыв воскресенья начало;

\vs 8Sb 1:315 Смоет водой родника, что дает безсмертия влагу, 

\vs 8Sb 1:316 Всякое прежнее зло, и, заново свыше родившись, 

\vs 8Sb 1:317 Быть перестанут рабами обычаев мира безбожных. 

\vs 8Sb 1:318 Явится прежде Господь своим, чтоб увидели ясно: 

\vs 8Sb 1:319 Вновь Он пришел во плоти, как прежде был, и покажет

\vs 8Sb 1:320 Им на руках и ногах четыре следа кровавых, 

\vs 8Sb 1:321 То будут Севера, Юга, Востока и Запада знаки, 

\vs 8Sb 1:322 Или число тех царств, что в мире свершат нечестиво 

\vs 8Sb 1:323 Дело позорное, кое нам всем в осужденье послужит.

\vs 8Sb 1:324 Дочь святая Сиона, ты много бед претерпела, 

\vs 8Sb 1:325 Радуйся ныне! Твой царь въезжает на ослике в Город, 

\vs 8Sb 1:326 Кротко, смиренно грядет, чтобы наше рабское иго, 

\vs 8Sb 1:327 Тяжко давившее спины, теперь упразднилось навеки, 

\vs 8Sb 1:328 Чтобы неправый закон исчез и гнетущие цепи.

\vs 8Sb 1:329 Так почти же Его как Бога и Божьего Сына, 

\vs 8Sb 1:330 В сердце свое прими и славь Его в радостных гимнах,

\vs 8Sb 1:331 Всею душой возлюби и себе возьми Его имя.

\vs 8Sb 1:332 Прежних всех удали и кровь, Им пролитую, смой ты;

\vs 8Sb 1:333 Жалких воплей твоих, и тленных жертв, и молений

\vs 8Sb 1:334 Вовсе не нужно Тому, Кто Сам безсмертен и вечен, 

\vs 8Sb 1:335 Но песнопений Он хочет из уст и сердца святого.

\vs 8Sb 1:336 Знай же, Кто Он таков  тогда и Отца Его узришь.

\vs 8Sb 1:337 Мира все элементы в то время придут в запустенье: 

\vs 8Sb 1:338 Воздух, море, земля и свет, от огня исходящий, 

\vs 8Sb 1:339 Ось небесная, ночь и все дни воедино сольются,

\vs 8Sb 1:340 В пламени формы свои они совершенно утратят. 

\vs 8Sb 1:341 Все светоносные звезды с небес упадут и исчезнут; 

\vs 8Sb 1:342 В воздухе больше не станут летать крылатые птицы, 

\vs 8Sb 1:343 Землю не тронет нога, ибо твари живые погибнут; 

\vs 8Sb 1:344 Смолкнут все голоса  людей, зверей и пернатых,

\vs 8Sb 1:345 Миру в его неустройстве не будет звука на пользу, 

\vs 8Sb 1:346 Но угрожающий шум издаст глубокое море, 

\vs 8Sb 1:347 Жители вод содрогнутся, и тут же конец им настанет; 

\vs 8Sb 1:348 Не поплывет по волнам и судно, везущее грузы. 

\vs 8Sb 1:349 Тяжко застонет земля, в сраженьях политая кровью;

\vs 8Sb 1:350 И заскрежещут зубами тогда все души людские,

\vs 8Sb 1:351 Грешные души погибель в стенаньях и ужасе встретят. 

\vs 8Sb 1:352 Жажда будет их жечь, убийства, болезни и голод, 

\vs 8Sb 1:353 И пожелают они умереть, но больше не смогут: 

\vs 8Sb 1:354 Не успокоит их смерть, и ночь не даст передышки.

\vs 8Sb 1:355 Долго Всевышнего Бога молить они будут напрасно  

\vs 8Sb 1:356 И отвратит Господь Свой лик, чтоб их больше не видеть: 

\vs 8Sb 1:357 Ибо ведь людям заблудшим Он семь веков предоставил 

\vs 8Sb 1:358 Для покаянья  за них просила Дева святая.

\vs 8Sb 1:359 Все эти речи Сам Бог вложил мне в сердце, и нужно, 

\vs 8Sb 1:360 Чтобы сбылось непременно все то, что уста мои молвят. 

\vs 8Sb 1:361 Мне известно число песчинок и вод в Океане, 

\vs 8Sb 1:362 Знаю земли тайники и Тартара мрачное царство, 

\vs 8Sb 1:363 Знаю все звезды на небе, и все деревья, и сколько 

\vs 8Sb 1:364 В мире четвероногих, плавучих и птиц оперенных, 

\vs 8Sb 1:365 Сколько людей живет, жило раньше и позже родится.

\vs 8Sb 1:366 В смертных запечатлел Я Сам и облик, и разум, 

\vs 8Sb 1:367 Дал им здравую мысль, научил их знаниям всяким,

\vs 8Sb 1:368 Создал Я ухо и глаз, и Сам все вижу и слышу, 

\vs 8Sb 1:369 Мысли чувствую все и всех событий Свидетель,

\vs 8Sb 1:370 Тот, который в начале молчит, а потом обличает, 

\vs 8Sb 1:371 Тяжко за тайное зло карая любого из смертных, 

\vs 8Sb 1:372 И у Господнего трона людей собеседником будет. 

\vs 8Sb 1:373 Я и глухого пойму и даже немого услышу. 

\vs 8Sb 1:374 Знаю и то, каково от земли до небес расстоянье,

\vs 8Sb 1:375 Знаю конец и начало, ведь Я создал небо и землю, 

\vs 8Sb 1:376 Создал Он все  от истока Он ведает все до предела. 

\vs 8Sb 1:377 Я  единственный Бог, иного не узрите Бога. 

\vs 8Sb 1:378 Люди подобье Мое деревянное обожествили, 

\vs 8Sb 1:379 Сделав своей же рукой изваянья безгласные, стали

\vs 8Sb 1:380 Кланяться им и молиться, служа нечестивые службы. 

\vs 8Sb 1:381 Чтили то, что несвято, Творца же совсем позабыли, 

\vs 8Sb 1:382 Все, рожденные Мной, приносят ненужные жертвы, 

\vs 8Sb 1:383 Точно во славу Мою, и достойным это считают, 

\vs 8Sb 1:384 Дым воскуряют, как будто заботясь о собственных мертвых.

\vs 8Sb 1:385 Мясо спаляют они и кости, полные мозга, 

\vs 8Sb 1:386 Жертвуя на алтарях и демонам кровь возливая. 

\vs 8Sb 1:387 Свечи Мне возжигают, хоть Я же свет даровал им, 

\vs 8Sb 1:388 Смертные Богу вино возливают, словно Он жаждет, 

\vs 8Sb 1:389 И напиваются сами у ног изваяний никчемных.

\vs 8Sb 1:390 Ваших не нужно Мне жертв, не нужно и возлияний; 

\vs 8Sb 1:391 Мне грязный дым неугоден и мерзостной крови потоки. 

\vs 8Sb 1:392 В память царей и тиранов свершают службы такие, 

\vs 8Sb 1:393 Кланяясь демонам мертвым, как будто в небе живущим,  

\vs 8Sb 1:394 Так почитаньем безбожным они себе гибель готовят.

\vs 8Sb 1:395 Бога лишенные люди зовут изваянья богами, 

\vs 8Sb 1:396 Но позабыли Творца и видят жизнь и надежду 

\vs 8Sb 1:397 В идолах, что не лишены и слуха, и голоса вовсе. 

\vs 8Sb 1:398 В злых лишь надежны делах, а добра не имеют и в мыслях.

\vs 8Sb 1:399 Мною даны два пути: есть к жизни дорога и к смерти,

\vs 8Sb 1:400 Я же благой дал совет  держаться праведной жизни, 

\vs 8Sb 1:401 Но предпочли эти люди погибель и вечное пламя. 

\vs 8Sb 1:402 Образ Мой  человек, наделенный здравою мыслью: 

\vs 8Sb 1:403 Вот ему и подай не кровавую  чистую пищу, 

\vs 8Sb 1:404 Сделай добро: удели тому, кто голоден, хлеба,

\vs 8Sb 1:405 Если жаждет  воды, если наг  одежды для тела;

\vs 8Sb 1:406 Милость твори от своих же трудов и ладонью святою. 

\vs 8Sb 1:407 Кто-то в унынье  утешь, устал кто  приди на подмогу: 

\vs 8Sb 1:408 Эту жертву живую воздай ты Богу Живому. 

\vs 8Sb 1:409 Сей зерно благочестья, тогда в награду получишь

\vs 8Sb 1:410 Плод безсмертный и свет негасимый, и жизни нетленной 

\vs 8Sb 1:411 Будешь достоин, когда всех людей огонь испытает. 

\vs 8Sb 1:412 Сплавлю Я все воедино и вновь разниму и очищу, 

\vs 8Sb 1:413 Небо сверну Я, как свиток, открою недра земные, 

\vs 8Sb 1:414 Мертвых в тот день воскрешу и судьбы людские разрушу,

\vs 8Sb 1:415 Жало смерти Я вырву; и суд последний устрою: 

\vs 8Sb 1:416 Стану жизни судить и праведных, и нечестивых, 

\vs 8Sb 1:417 Рядом встанет баран с бараном, и с пастырем пастырь, 

\vs 8Sb 1:418 Рядом с тельцом телец, чтоб друг друга они обличили. 

\vs 8Sb 1:419 Здесь обличатся все те, кто в жизни своей возвышался,

\vs 8Sb 1:420 Рты затыкая другим, и к зависти их побуждали, 

\vs 8Sb 1:421 Дабы и те справедливых людей в рабов обращали, 

\vs 8Sb 1:422 Всех заставляли молчать, влекомые только наживой. 

\vs 8Sb 1:423 Те же, кто праведно жил, все встанут рядом со Мною. 

\vs 8Sb 1:424 Больше в печали никто не скажет: То будет завтра,

\vs 8Sb 1:425 Или: То было вчера; и дней, заботами полных, 

\vs 8Sb 1:426 Также не станет, исчезнут четыре времени года, 

\vs 8Sb 1:427 С ними Восход и Закат, и все в долгий день обратится. 

\vs 8Sb 1:428 Свет пребудет вовеки, желанный и долгожданный  

\vs 8Sb 1:429 Навсегда Иисус Христос \ldots

\vs 8Sb 1:430 Боже, никем не рожденный, Чистейший, Вечный, Безсмертный!

\vs 8Sb 1:431 В небе живешь Ты и властью Своей усмиряешь дыханье 

\vs 8Sb 1:432 Пламени и поражаешь огнем могущество скиптров, 

\vs 8Sb 1:433 Гулких грома раскатов легко укрощаешь Ты силу, 

\vs 8Sb 1:434 Движешь Ты землю и держишь в узде волненье морское, 

\vs 8Sb 1:435 И ослабляешь бичи сверкающих огненных молний,

\vs 8Sb 1:436 Мощные ливней потоки, падение града и снега, 

\vs 8Sb 1:437 Что из тучи морозной летит, и бури порывы \ldots

\vs 8Sb 1:438 Ангелы, слуги Твои, заботливо распределяют

\vs 8Sb 1:439 Все, что Тобой решено и чему повелел Ты свершиться.

\vs 8Sb 1:440 Прежде творенья всего на совет призвав Свое Чадо, 

\vs 8Sb 1:441 Создал Ты смертных людей и жизни дал зародиться.

\vs 8Sb 1:442 Первым к Нему обратился Ты сладостной речью Твоею: 

\vs 8Sb 1:443 Сделаем ныне с Тобою подобного Нам человека 

\vs 8Sb 1:444 И дающее жизнь дыхание в грудь его вложим. 

\vs 8Sb 1:445 Смертен он будет, но все пускай на земле ему служит,

\vs 8Sb 1:446 Все ему подчиним, хоть его мы слепим из глины. 

\vs 8Sb 1:447 Это Ты Слову сказал, и стало так, как решил Ты. 

\vs 8Sb 1:448 Тотчас же все элементы велениям повиновались, 

\vs 8Sb 1:449 И со смертным твореньем все вечное соединилось: 

\vs 8Sb 1:450 Небо, воздух, огонь, земля и волны морские,

\vs 8Sb 1:451 Солнце, луна и созвездья \ldots

\vs 8Sb 1:452 Дни и ночи, и сон, пробуждение, дух и движенье, 

\vs 8Sb 1:453 И душа, и разсудок, искусство, сила и голос, 

\vs 8Sb 1:454 Стаи диких зверей, пернатые птицы и рыбы, 

\vs 8Sb 1:455 Те, кто ходит, и те, кто ползут, и амфибии также 

\vs 8Sb 1:456 Все человеку Он дал, Твоим решеньем наставлен.

\vs 8Sb 1:457 А у конца времен изменил Он землю: 

\vs 8Sb 1:458 Младенец Тело девы Марии покинул, и новый зажегся 

\vs 8Sb 1:459 Свет: пришел Он с небес и смертное принял обличье. 

\vs 8Sb 1:460 Мощный телом своим, сперва Гавриил появился,

\vs 8Sb 1:461 После же голос возвысил архангел и деве сказал он: 

\vs 8Sb 1:462 В чистое лоно свое прими ныне Бога, о дева! 

\vs 8Sb 1:463 Рек, и вдохнул благодать ей Господь; она же, услышав, 

\vs 8Sb 1:464 В страхе была и в смущенье, поскольку мужа не знала. 

\vs 8Sb 1:465 С места сойти не могла, в испуге дрожала, и сердце

\vs 8Sb 1:466 Быстро билось в груди от вести, еще непонятной. 

\vs 8Sb 1:467 Вскоре, однако, слова проникли в сердце, и тут же 

\vs 8Sb 1:468 Смех ее охватил, разрумянились юные щеки; 

\vs 8Sb 1:469 Перемешались в душе у девицы смущенье и радость, 

\vs 8Sb 1:470 И осмелела она. А слово, влетевшее в чрево,

\vs 8Sb 1:471 Плотью со временем стало и, в теле ожив материнском, 

\vs 8Sb 1:472 Облик людской обрело; и Мальчик на свет появился, 

\vs 8Sb 1:473 Девой рожден. Без сомненья  для смертных великое чудо, 

\vs 8Sb 1:474 Но не бывает великих чудес для Отца и для Сына. 

\vs 8Sb 1:475 Чада рожденье земле принесло великую радость,

\vs 8Sb 1:476 Возвеселился и в небе Престол, и мир  в ликованье. 

\vs 8Sb 1:477 Маги воздали честь звезде, невиданной прежде, 

\vs 8Sb 1:478 И, уверовав в Бога, лежащего в яслях узрели; 

\vs 8Sb 1:479 Пасшие коз и овец Дитя увидели также. 

\vs 8Sb 1:480 И наречен Вифлеем богоизбранный родиной Слова.

\vs 8Sb 1:481 \ldots\ Нужно в сердце иметь смиренье и зло ненавидеть, 

\vs 8Sb 1:482 К ближним питать любовь такую, как к собственной жизни, 

\vs 8Sb 1:483 Бога душою любить и Ему воздавать почитанье. 

\vs 8Sb 1:484 Ибо ведь мы от Него и святого Рожденья Христова 

\vs 8Sb 1:485 Небом произведены и единую кровь получили.

\vs 8Sb 1:486 Богу служа, мы всегда о блаженстве будущем помним, 

\vs 8Sb 1:487 Правды дорогой идя, прямым путем благочестья. 

\vs 8Sb 1:488 В храмы язычников мы никогда входить не дерзаем, 

\vs 8Sb 1:489 Нет для кумиров у нас ни молитвы, ни возлияний, 

\vs 8Sb 1:490 Ни аромата цветов, ни огня светильников ярких,

\vs 8Sb 1:491 Ни приношеньем даров богатых мы не почтим их 

\vs 8Sb 1:492 И благовонья на их алтарях никогда не воскурим; 

\vs 8Sb 1:493 В жертву быков и овец приносить им не станем, пытаясь 

\vs 8Sb 1:494 От наказания их таким путем откупиться; 

\vs 8Sb 1:495 Жирным дымом костра, который плоть пожирает,

\vs 8Sb 1:496 Мы осквернять не хотим сияния ясного неба.

\vs 8Sb 1:497 Но в помышлениях чистых, ликуя радостным сердцем, 

\vs 8Sb 1:498 Щедро дающей рукой и богатством любви безконечным, 

\vs 8Sb 1:499 Сладостной песней и в гимнах, достойных великого Бога, 

\vs 8Sb 1:500 Должно Тебя воспевать нам, неложно и непрестанно 

\vs 8Sb 1:501 Мудрый мира Создатель \ldots

\bibbookdescr{Gen}{
  inline={\LARGE Первая книга Моисеева\\\Huge Бытие},
  toc={Бытие},
  bookmark={Бытие},
  header={Бытие},
  %headerleft={},
  %headerright={},
  abbr={Быт}
}
\vs Gen 1:1 В начале сотворил Бог небо и землю.
\vs Gen 1:2 Земля же была безвидна и пуста, и тьма над бездною, и Дух Божий носился над водою.
\rsbpar\vs Gen 1:3 И сказал Бог: да будет свет. И стал свет.
\vs Gen 1:4 И увидел Бог свет, что он хорош, и отделил Бог свет от тьмы.
\vs Gen 1:5 И назвал Бог свет днем, а тьму ночью. И был вечер, и было утро: день один.
\rsbpar\vs Gen 1:6 И сказал Бог: да будет твердь посреди воды, и да отделяет она воду от воды. [И стало так.]
\vs Gen 1:7 И создал Бог твердь, и отделил воду, которая под твердью, от воды, которая над твердью. И стало так.
\vs Gen 1:8 И назвал Бог твердь небом. [И увидел Бог, что \bibemph{это} хорошо.] И был вечер, и было утро: день второй.
\rsbpar\vs Gen 1:9 И сказал Бог: да соберется вода, которая под небом, в одно место, и да явится суша. И стало так. [И собралась вода под небом в свои места, и явилась суша.]
\vs Gen 1:10 И назвал Бог сушу землею, а собрание вод назвал морями. И увидел Бог, что \bibemph{это} хорошо.
\vs Gen 1:11 И сказал Бог: да произрастит земля зелень, траву, сеющую семя [по роду и по подобию \bibemph{ее}, и] дерево плодовитое, приносящее по роду своему плод, в котором семя его на земле. И стало так.
\vs Gen 1:12 И произвела земля зелень, траву, сеющую семя по роду [и по подобию] ее, и дерево [плодовитое], приносящее плод, в котором семя его по роду его [на земле]. И увидел Бог, что \bibemph{это} хорошо.
\vs Gen 1:13 И был вечер, и было утро: день третий.
\rsbpar\vs Gen 1:14 И сказал Бог: да будут светила на тверди небесной [для освещения земли и] для отделения дня от ночи, и для знамений, и времен, и дней, и годов;
\vs Gen 1:15 и да будут они светильниками на тверди небесной, чтобы светить на землю. И стало так.
\vs Gen 1:16 И создал Бог два светила великие: светило большее, для управления днем, и светило меньшее, для управления ночью, и звезды;
\vs Gen 1:17 и поставил их Бог на тверди небесной, чтобы светить на землю,
\vs Gen 1:18 и управлять днем и ночью, и отделять свет от тьмы. И увидел Бог, что \bibemph{это} хорошо.
\vs Gen 1:19 И был вечер, и было утро: день четвёртый.
\rsbpar\vs Gen 1:20 И сказал Бог: да произведет вода пресмыкающихся, душу живую; и птицы да полетят над землею, по тверди небесной. [И стало так.]
\vs Gen 1:21 И сотворил Бог рыб больших и всякую душу животных пресмыкающихся, которых произвела вода, по роду их, и всякую птицу пернатую по роду ее. И увидел Бог, что \bibemph{это} хорошо.
\vs Gen 1:22 И благословил их Бог, говоря: плодитесь и размножайтесь, и наполняйте воды в морях, и птицы да размножаются на земле.
\vs Gen 1:23 И был вечер, и было утро: день пятый.
\rsbpar\vs Gen 1:24 И сказал Бог: да произведет земля душу живую по роду ее, скотов, и гадов, и зверей земных по роду их. И стало так.
\vs Gen 1:25 И создал Бог зверей земных по роду их, и скот по роду его, и всех гадов земных по роду их. И увидел Бог, что \bibemph{это} хорошо.
\rsbpar\vs Gen 1:26 И сказал Бог: сотворим человека по образу Нашему [и] по подобию Нашему, и да владычествуют они над рыбами морскими, и над птицами небесными, [и над зверями,] и над скотом, и над всею землею, и над всеми гадами, пресмыкающимися по земле.
\vs Gen 1:27 И сотворил Бог человека по образу Своему, по образу Божию сотворил его; мужчину и женщину сотворил их.
\vs Gen 1:28 И благословил их Бог, и сказал им Бог: плодитесь и размножайтесь, и наполняйте землю, и обладайте ею, и владычествуйте над рыбами морскими [и над зверями,] и над птицами небесными, [и над всяким скотом, и над всею землею,] и над всяким животным, пресмыкающимся по земле.
\vs Gen 1:29 И сказал Бог: вот, Я дал вам всякую траву, сеющую семя, какая есть на всей земле, и всякое дерево, у которого плод древесный, сеющий семя;~--- вам \bibemph{сие} будет в пищу;
\vs Gen 1:30 а всем зверям земным, и всем птицам небесным, и всякому [гаду,] пресмыкающемуся по земле, в котором душа живая, \bibemph{дал} Я всю зелень травную в пищу. И стало так.
\vs Gen 1:31 И увидел Бог все, что Он создал, и вот, хорошо весьма. И был вечер, и было утро: день шестой.
\vs Gen 2:1 Так совершены небо и земля и все воинство их.
\vs Gen 2:2 И совершил Бог к седьмому дню дела Свои, которые Он делал, и почил в день седьмый от всех дел Своих, которые делал.
\vs Gen 2:3 И благословил Бог седьмой день, и освятил его, ибо в оный почил от всех дел Своих, которые Бог творил и созидал.
\rsbpar\vs Gen 2:4 Вот происхождение неба и земли, при сотворении их, в то время, когда Господь Бог создал землю и небо,
\vs Gen 2:5 и всякий полевой кустарник, которого еще не было на земле, и всякую полевую траву, которая еще не росла, ибо Господь Бог не посылал дождя на землю, и не было человека для возделывания земли,
\vs Gen 2:6 но пар поднимался с земли и орошал все лице земли.
\vs Gen 2:7 И создал Господь Бог человека из праха земного, и вдунул в лице его дыхание жизни, и стал человек душею живою.
\vs Gen 2:8 И насадил Господь Бог рай в Едеме на востоке, и поместил там человека, которого создал.
\vs Gen 2:9 И произрастил Господь Бог из земли всякое дерево, приятное на вид и хорошее для пищи, и дерево жизни посреди рая, и дерево познания добра и зла.
\vs Gen 2:10 Из Едема выходила река для орошения рая; и потом разделялась на четыре реки.
\vs Gen 2:11 Имя одной Фисон: она обтекает всю землю Хавила, ту, где золото;
\vs Gen 2:12 и золото той земли хорошее; там бдолах и камень оникс.
\vs Gen 2:13 Имя второй реки Гихон [Геон]: она обтекает всю землю Куш.
\vs Gen 2:14 Имя третьей реки Хиддекель [Тигр]: она протекает пред Ассириею. Четвертая река Евфрат.
\vs Gen 2:15 И взял Господь Бог человека, [которого создал,] и поселил его в саду Едемском, чтобы возделывать его и хранить его.
\vs Gen 2:16 И заповедал Господь Бог человеку, говоря: от всякого дерева в саду ты будешь есть,
\vs Gen 2:17 а от дерева познания добра и зла не ешь от него, ибо в день, в который ты вкусишь от него, смертью умрешь.
\vs Gen 2:18 И сказал Господь Бог: не хорошо быть человеку одному; сотворим ему помощника, соответственного ему.
\vs Gen 2:19 Господь Бог образовал из земли всех животных полевых и всех птиц небесных, и привел [их] к человеку, чтобы видеть, как он назовет их, и чтобы, как наречет человек всякую душу живую, так и было имя ей.
\vs Gen 2:20 И нарек человек имена всем скотам и птицам небесным и всем зверям полевым; но для человека не нашлось помощника, подобного ему.
\vs Gen 2:21 И навел Господь Бог на человека крепкий сон; и, когда он уснул, взял одно из ребр его, и закрыл то место плотию.
\vs Gen 2:22 И создал Господь Бог из ребра, взятого у человека, жену, и привел ее к человеку.
\vs Gen 2:23 И сказал человек: вот, это кость от костей моих и плоть от плоти моей; она будет называться женою, ибо взята от мужа [своего].
\vs Gen 2:24 Потому оставит человек отца своего и мать свою и прилепится к жене своей; и будут [два] одна плоть.
\vs Gen 2:25 И были оба наги, Адам и жена его, и не стыдились.
\vs Gen 3:1 Змей был хитрее всех зверей полевых, которых создал Господь Бог. И сказал змей жене: подлинно ли сказал Бог: не ешьте ни от какого дерева в раю?
\vs Gen 3:2 И сказала жена змею: плоды с дерев мы можем есть,
\vs Gen 3:3 только плодов дерева, которое среди рая, сказал Бог, не ешьте их и не прикасайтесь к ним, чтобы вам не умереть.
\vs Gen 3:4 И сказал змей жене: нет, не умрете,
\vs Gen 3:5 но знает Бог, что в день, в который вы вкусите их, откроются глаза ваши, и вы будете, как боги, знающие добро и зло.
\vs Gen 3:6 И увидела жена, что дерево хорошо для пищи, и что оно приятно для глаз и вожделенно, потому что дает знание; и взяла плодов его и ела; и дала также мужу своему, и он ел.
\vs Gen 3:7 И открылись глаза у них обоих, и узнали они, что наги, и сшили смоковные листья, и сделали себе опоясания.
\rsbpar\vs Gen 3:8 И услышали голос Господа Бога, ходящего в раю во время прохлады дня; и скрылся Адам и жена его от лица Господа Бога между деревьями рая.
\vs Gen 3:9 И воззвал Господь Бог к Адаму и сказал ему: [Адам,] где ты?
\vs Gen 3:10 Он сказал: голос Твой я услышал в раю, и убоялся, потому что я наг, и скрылся.
\vs Gen 3:11 И сказал [Бог]: кто сказал тебе, что ты наг? не ел ли ты от дерева, с которого Я запретил тебе есть?
\vs Gen 3:12 Адам сказал: жена, которую Ты мне дал, она дала мне от дерева, и я ел.
\vs Gen 3:13 И сказал Господь Бог жене: что ты это сделала? Жена сказала: змей обольстил меня, и я ела.
\rsbpar\vs Gen 3:14 И сказал Господь Бог змею: за то, что ты сделал это, проклят ты пред всеми скотами и пред всеми зверями полевыми; ты будешь ходить на чреве твоем, и будешь есть прах во все дни жизни твоей;
\vs Gen 3:15 и вражду положу между тобою и между женою, и между семенем твоим и между семенем ее; оно будет поражать тебя в голову, а ты будешь жалить его в пяту\fns{По другому чтению: и между Семенем ее; Он будет поражать тебя в голову, а ты будешь жалить Его в пяту.}.
\rsbpar\vs Gen 3:16 Жене сказал: умножая умножу скорбь твою в беременности твоей; в болезни будешь рождать детей; и к мужу твоему влечение твое, и он будет господствовать над тобою.
\vs Gen 3:17 Адаму же сказал: за то, что ты послушал голоса жены твоей и ел от дерева, о котором Я заповедал тебе, сказав: не ешь от него, проклята земля за тебя; со скорбью будешь питаться от нее во все дни жизни твоей;
\vs Gen 3:18 терния и волчцы произрастит она тебе; и будешь питаться полевою травою;
\vs Gen 3:19 в поте лица твоего будешь есть хлеб, доколе не возвратишься в землю, из которой ты взят, ибо прах ты и в прах возвратишься.
\vs Gen 3:20 И нарек Адам имя жене своей: Ева\fns{Жизнь.}, ибо она стала матерью всех живущих.
\rsbpar\vs Gen 3:21 И сделал Господь Бог Адаму и жене его одежды кожаные и одел их.
\rsbpar\vs Gen 3:22 И сказал Господь Бог: вот, Адам стал как один из Нас, зная добро и зло; и теперь как бы не простер он руки своей, и не взял также от дерева жизни, и не вкусил, и не стал жить вечно.
\vs Gen 3:23 И выслал его Господь Бог из сада Едемского, чтобы возделывать землю, из которой он взят.
\vs Gen 3:24 И изгнал Адама, и поставил на востоке у сада Едемского Херувима и пламенный меч обращающийся, чтобы охранять путь к дереву жизни.
\vs Gen 4:1 Адам познал Еву, жену свою; и она зачала, и родила Каина, и сказала: приобрела я человека от Господа.
\vs Gen 4:2 И еще родила брата его, Авеля. И был Авель пастырь овец, а Каин был земледелец.
\rsbpar\vs Gen 4:3 Спустя несколько времени, Каин принес от плодов земли дар Господу,
\vs Gen 4:4 и Авель также принес от первородных стада своего и от тука их. И призрел Господь на Авеля и на дар его,
\vs Gen 4:5 а на Каина и на дар его не призрел. Каин сильно огорчился, и поникло лице его.
\vs Gen 4:6 И сказал Господь [Бог] Каину: почему ты огорчился? и отчего поникло лице твое?
\vs Gen 4:7 если делаешь доброе, то не поднимаешь ли лица? а если не делаешь доброго, то у дверей грех лежит; он влечет тебя к себе, но ты господствуй над ним.
\vs Gen 4:8 И сказал Каин Авелю, брату своему: [пойдем в поле]. И когда они были в поле, восстал Каин на Авеля, брата своего, и убил его.
\rsbpar\vs Gen 4:9 И сказал Господь [Бог] Каину: где Авель, брат твой? Он сказал: не знаю; разве я сторож брату моему?
\vs Gen 4:10 И сказал [Господь]: что ты сделал? голос крови брата твоего вопиет ко Мне от земли;
\vs Gen 4:11 и ныне проклят ты от земли, которая отверзла уста свои принять кровь брата твоего от руки твоей;
\vs Gen 4:12 когда ты будешь возделывать землю, она не станет более давать силы своей для тебя; ты будешь изгнанником и скитальцем на земле.
\vs Gen 4:13 И сказал Каин Господу [Богу]: наказание мое больше, нежели снести можно;
\vs Gen 4:14 вот, Ты теперь сгоняешь меня с лица земли, и от лица Твоего я скроюсь, и буду изгнанником и скитальцем на земле; и всякий, кто встретится со мною, убьет меня.
\vs Gen 4:15 И сказал ему Господь [Бог]: за то всякому, кто убьет Каина, отмстится всемеро. И сделал Господь [Бог] Каину знамение, чтобы никто, встретившись с ним, не убил его.
\vs Gen 4:16 И пошел Каин от лица Господня и поселился в земле Нод, на восток от Едема.
\vs Gen 4:17 И познал Каин жену свою; и она зачала и родила Еноха. И построил он город; и назвал город по имени сына своего: Енох.
\vs Gen 4:18 У Еноха родился Ирад [Гаидад]; Ирад родил Мехиаеля [Малелеила]; Мехиаель родил Мафусала; Мафусал родил Ламеха.
\vs Gen 4:19 И взял себе Ламех две жены: имя одной: Ада, и имя второй: Цилла [Селла].
\vs Gen 4:20 Ада родила Иавала: он был отец живущих в шатрах со стадами.
\vs Gen 4:21 Имя брату его Иувал: он был отец всех играющих на гуслях и свирели.
\vs Gen 4:22 Цилла также родила Тувалкаина [Фовела], который был ковачом всех орудий из меди и железа. И сестра Тувалкаина Ноема.
\vs Gen 4:23 И сказал Ламех женам своим: Ада и Цилла! послушайте голоса моего; жены Ламеховы! внимайте словам моим: я убил мужа в язву мне и отрока в рану мне;
\vs Gen 4:24 если за Каина отмстится всемеро, то за Ламеха в семьдесят раз всемеро.
\rsbpar\vs Gen 4:25 И познал Адам еще [Еву,] жену свою, и она родила сына, и нарекла ему имя: Сиф, потому что, [говорила она,] Бог положил мне другое семя, вместо Авеля, которого убил Каин.
\vs Gen 4:26 У Сифа также родился сын, и он нарек ему имя: Енос; тогда начали призывать имя Господа [Бога].
\vs Gen 5:1 Вот родословие Адама: когда Бог сотворил человека, по подобию Божию создал его,
\vs Gen 5:2 мужчину и женщину сотворил их, и благословил их, и нарек им имя: человек, в день сотворения их.
\vs Gen 5:3 Адам жил сто тридцать [230] лет и родил [сына] по подобию своему [и] по образу своему, и нарек ему имя: Сиф.
\vs Gen 5:4 Дней Адама по рождении им Сифа было восемьсот [700] лет, и родил он сынов и дочерей.
\vs Gen 5:5 Всех же дней жизни Адамовой было девятьсот тридцать лет; и он умер.
\rsbpar\vs Gen 5:6 Сиф жил сто пять [205] лет и родил Еноса.
\vs Gen 5:7 По рождении Еноса Сиф жил восемьсот семь [707] лет и родил сынов и дочерей.
\vs Gen 5:8 Всех же дней Сифовых было девятьсот двенадцать лет; и он умер.
\rsbpar\vs Gen 5:9 Енос жил девяносто [190] лет и родил Каинана.
\vs Gen 5:10 По рождении Каинана Енос жил восемьсот пятнадцать [715] лет и родил сынов и дочерей.
\vs Gen 5:11 Всех же дней Еноса было девятьсот пять лет; и он умер.
\rsbpar\vs Gen 5:12 Каинан жил семьдесят [170] лет и родил Малелеила.
\vs Gen 5:13 По рождении Малелеила Каинан жил восемьсот сорок [740] лет и родил сынов и дочерей.
\vs Gen 5:14 Всех же дней Каинана было девятьсот десять лет; и он умер.
\rsbpar\vs Gen 5:15 Малелеил жил шестьдесят пять [165] лет и родил Иареда.
\vs Gen 5:16 По рождении Иареда Малелеил жил восемьсот тридцать [730] лет и родил сынов и дочерей.
\vs Gen 5:17 Всех же дней Малелеила было восемьсот девяносто пять лет; и он умер.
\rsbpar\vs Gen 5:18 Иаред жил сто шестьдесят два года и родил Еноха.
\vs Gen 5:19 По рождении Еноха Иаред жил восемьсот лет и родил сынов и дочерей.
\vs Gen 5:20 Всех же дней Иареда было девятьсот шестьдесят два года; и он умер.
\rsbpar\vs Gen 5:21 Енох жил шестьдесят пять [165] лет и родил Мафусала.
\vs Gen 5:22 И ходил Енох пред Богом, по рождении Мафусала, триста [200] лет и родил сынов и дочерей.
\vs Gen 5:23 Всех же дней Еноха было триста шестьдесят пять лет.
\vs Gen 5:24 И ходил Енох пред Богом; и не стало его, потому что Бог взял его.
\rsbpar\vs Gen 5:25 Мафусал жил сто восемьдесят семь лет и родил Ламеха.
\vs Gen 5:26 По рождении Ламеха Мафусал жил семьсот восемьдесят два года и родил сынов и дочерей.
\vs Gen 5:27 Всех же дней Мафусала было девятьсот шестьдесят девять лет; и он умер.
\rsbpar\vs Gen 5:28 Ламех жил сто восемьдесят два [188] года и родил сына,
\vs Gen 5:29 и нарек ему имя: Ной, сказав: он утешит нас в работе нашей и в трудах рук наших при \bibemph{возделывании} земли, которую проклял Господь [Бог].
\vs Gen 5:30 И жил Ламех по рождении Ноя пятьсот девяносто пять [565] лет и родил сынов и дочерей.
\vs Gen 5:31 Всех же дней Ламеха было семьсот семьдесят семь [753] лет; и он умер.
\rsbpar\vs Gen 5:32 Ною было пятьсот лет и родил Ной [трех сынов]: Сима, Хама и Иафета.
\vs Gen 6:1 Когда люди начали умножаться на земле и родились у них дочери,
\vs Gen 6:2 тогда сыны Божии увидели дочерей человеческих, что они красивы, и брали \bibemph{их} себе в жены, какую кто избрал.
\vs Gen 6:3 И сказал Господь [Бог]: не вечно Духу Моему быть пренебрегаемым человеками [сими], потому что они плоть; пусть будут дни их сто двадцать лет.
\vs Gen 6:4 В то время были на земле исполины, особенно же с того времени, как сыны Божии стали входить к дочерям человеческим, и они стали рождать им: это сильные, издревле славные люди.
\rsbpar\vs Gen 6:5 И увидел Господь [Бог], что велико развращение человеков на земле, и что все мысли и помышления сердца их были зло во всякое время;
\vs Gen 6:6 и раскаялся Господь, что создал человека на земле, и восскорбел в сердце Своем.
\vs Gen 6:7 И сказал Господь: истреблю с лица земли человеков, которых Я сотворил, от человека до скотов, и гадов и птиц небесных истреблю, ибо Я раскаялся, что создал их.
\rsbpar\vs Gen 6:8 Ной же обрел благодать пред очами Господа [Бога].
\rsbpar\vs Gen 6:9 Вот житие Ноя: Ной был человек праведный и непорочный в роде своем; Ной ходил пред Богом.
\vs Gen 6:10 Ной родил трех сынов: Сима, Хама и Иафета.
\vs Gen 6:11 Но земля растлилась пред лицем Божиим, и наполнилась земля злодеяниями.
\vs Gen 6:12 И воззрел [Господь] Бог на землю, и вот, она растленна, ибо всякая плоть извратила путь свой на земле.
\vs Gen 6:13 И сказал [Господь] Бог Ною: конец всякой плоти пришел пред лице Мое, ибо земля наполнилась от них злодеяниями; и вот, Я истреблю их с земли.
\vs Gen 6:14 Сделай себе ковчег из дерева гофер; отделения сделай в ковчеге и осмоли его смолою внутри и снаружи.
\vs Gen 6:15 И сделай его так: длина ковчега триста локтей; ширина его пятьдесят локтей, а высота его тридцать локтей.
\vs Gen 6:16 И сделай отверстие в ковчеге, и в локоть сведи его вверху, и дверь в ковчег сделай с боку его; устрой в нем нижнее, второе и третье [жилье].
\vs Gen 6:17 И вот, Я наведу на землю потоп водный, чтоб истребить всякую плоть, в которой есть дух жизни, под небесами; все, что есть на земле, лишится жизни.
\vs Gen 6:18 Но с тобою Я поставлю завет Мой, и войдешь в ковчег ты, и сыновья твои, и жена твоя, и жены сынов твоих с тобою.
\vs Gen 6:19 Введи также в ковчег [из всякого скота, и из всех гадов, и] из всех животных, и от всякой плоти по паре, чтоб они остались с тобою в живых; мужеского пола и женского пусть они будут.
\vs Gen 6:20 Из [всех] птиц по роду их, и из [всех] скотов по роду их, и из всех пресмыкающихся по земле по роду их, из всех по паре войдут к тебе, чтобы остались в живых [с тобою, мужеского пола и женского].
\vs Gen 6:21 Ты же возьми себе всякой пищи, какою питаются, и собери к себе; и будет она для тебя и для них пищею.
\vs Gen 6:22 И сделал Ной всё: как повелел ему [Господь] Бог, так он и сделал.
\vs Gen 7:1 И сказал Господь [Бог] Ною: войди ты и все семейство твое в ковчег, ибо тебя увидел Я праведным предо Мною в роде сем;
\vs Gen 7:2 и всякого скота чистого возьми по семи, мужеского пола и женского, а из скота нечистого по два, мужеского пола и женского;
\vs Gen 7:3 также и из птиц небесных [чистых] по семи, мужеского пола и женского, [и из всех птиц нечистых по две, мужеского пола и женского,] чтобы сохранить племя для всей земли,
\vs Gen 7:4 ибо чрез семь дней Я буду изливать дождь на землю сорок дней и сорок ночей; и истреблю все существующее, что Я создал, с лица земли.
\vs Gen 7:5 Ной сделал все, что Господь [Бог] повелел ему.
\rsbpar\vs Gen 7:6 Ной же был шестисот лет, как потоп водный пришел на землю.
\vs Gen 7:7 И вошел Ной и сыновья его, и жена его, и жены сынов его с ним в ковчег от вод потопа.
\vs Gen 7:8 И [из птиц чистых и из птиц нечистых, и] из скотов чистых и из скотов нечистых, [и из зверей] и из всех пресмыкающихся по земле
\vs Gen 7:9 по паре, мужеского пола и женского, вошли к Ною в ковчег, как [Господь] Бог повелел Ною.
\rsbpar\vs Gen 7:10 Чрез семь дней воды потопа пришли на землю.
\vs Gen 7:11 В шестисотый год жизни Ноевой, во второй месяц, в семнадцатый [27] день месяца, в сей день разверзлись все источники великой бездны, и окна небесные отворились;
\vs Gen 7:12 и лился на землю дождь сорок дней и сорок ночей.
\vs Gen 7:13 В сей самый день вошел в ковчег Ной, и Сим, Хам и Иафет, сыновья Ноевы, и жена Ноева, и три жены сынов его с ними.
\vs Gen 7:14 Они, и все звери [земли] по роду их, и всякий скот по роду его, и все гады, пресмыкающиеся по земле, по роду их, и все летающие по роду их, все птицы, все крылатые,
\vs Gen 7:15 и вошли к Ною в ковчег по паре [мужеского пола и женского] от всякой плоти, в которой есть дух жизни;
\vs Gen 7:16 и вошедшие [к Ною в ковчег] мужеский и женский пол всякой плоти вошли, как повелел ему [Господь] Бог. И затворил Господь [Бог] за ним [ковчег].
\vs Gen 7:17 И продолжалось на земле наводнение сорок дней [и сорок ночей], и умножилась вода, и подняла ковчег, и он возвысился над землею;
\vs Gen 7:18 вода же усиливалась и весьма умножалась на земле, и ковчег плавал по поверхности вод.
\vs Gen 7:19 И усилилась вода на земле чрезвычайно, так что покрылись все высокие горы, какие есть под всем небом;
\vs Gen 7:20 на пятнадцать локтей поднялась над ними вода, и покрылись [все высокие] горы.
\vs Gen 7:21 И лишилась жизни всякая плоть, движущаяся по земле, и птицы, и скоты, и звери, и все гады, ползающие по земле, и все люди;
\vs Gen 7:22 все, что имело дыхание духа жизни в ноздрях своих на суше, умерло.
\vs Gen 7:23 Истребилось всякое существо, которое было на поверхности [всей] земли; от человека до скота, и гадов, и птиц небесных,~--- все истребилось с земли, остался только Ной и что \bibemph{было} с ним в ковчеге.
\vs Gen 7:24 Вода же усиливалась на земле сто пятьдесят дней.
\vs Gen 8:1 И вспомнил Бог о Ное, и о всех зверях, и о всех скотах, [и о всех птицах, и о всех гадах пресмыкающихся,] бывших с ним в ковчеге; и навел Бог ветер на землю, и воды остановились.
\vs Gen 8:2 И закрылись источники бездны и окна небесные, и перестал дождь с неба.
\vs Gen 8:3 Вода же постепенно возвращалась с земли, и стала убывать вода по окончании ста пятидесяти дней.
\vs Gen 8:4 И остановился ковчег в седьмом месяце, в семнадцатый день месяца, на горах Араратских.
\vs Gen 8:5 Вода постоянно убывала до десятого месяца; в первый день десятого месяца показались верхи гор.
\rsbpar\vs Gen 8:6 По прошествии сорока дней Ной открыл сделанное им окно ковчега
\vs Gen 8:7 и выпустил ворона, [чтобы видеть, убыла ли вода с земли,] который, вылетев, отлетал и прилетал, пока осушилась земля от воды.
\vs Gen 8:8 Потом выпустил от себя голубя, чтобы видеть, сошла ли вода с лица земли,
\vs Gen 8:9 но голубь не нашел места покоя для ног своих и возвратился к нему в ковчег, ибо вода была еще на поверхности всей земли; и он простер руку свою, и взял его, и принял к себе в ковчег.
\vs Gen 8:10 И помедлил еще семь дней других и опять выпустил голубя из ковчега.
\vs Gen 8:11 Голубь возвратился к нему в вечернее время, и вот, свежий масличный лист во рту у него, и Ной узнал, что вода сошла с земли.
\vs Gen 8:12 Он помедлил еще семь дней других и [опять] выпустил голубя; и он уже не возвратился к нему.
\rsbpar\vs Gen 8:13 Шестьсот первого года [жизни Ноевой] к первому [дню] первого месяца иссякла вода на земле; и открыл Ной кровлю ковчега и посмотрел, и вот, обсохла поверхность земли.
\vs Gen 8:14 И во втором месяце, к двадцать седьмому дню месяца, земля высохла.
\rsbpar\vs Gen 8:15 И сказал [Господь] Бог Ною:
\vs Gen 8:16 выйди из ковчега ты и жена твоя, и сыновья твои, и жены сынов твоих с тобою;
\vs Gen 8:17 выведи с собою всех животных, которые с тобою, от всякой плоти, из птиц, и скотов, и всех гадов, пресмыкающихся по земле: пусть разойдутся они по земле, и пусть плодятся и размножаются на земле.
\vs Gen 8:18 И вышел Ной и сыновья его, и жена его, и жены сынов его с ним;
\vs Gen 8:19 все звери, и [весь скот, и] все гады, и все птицы, все движущееся по земле, по родам своим, вышли из ковчега.
\rsbpar\vs Gen 8:20 И устроил Ной жертвенник Господу; и взял из всякого скота чистого и из всех птиц чистых и принес во всесожжение на жертвеннике.
\vs Gen 8:21 И обонял Господь приятное благоухание, и сказал Господь [Бог] в сердце Своем: не буду больше проклинать землю за человека, потому что помышление сердца человеческого~--- зло от юности его; и не буду больше поражать всего живущего, как Я сделал:
\vs Gen 8:22 впредь во все дни земли сеяние и жатва, холод и зной, лето и зима, день и ночь не прекратятся.
\vs Gen 9:1 И благословил Бог Ноя и сынов его и сказал им: плодитесь и размножайтесь, и наполняйте землю [и обладайте ею];
\vs Gen 9:2 да страшатся и да трепещут вас все звери земные, [и весь скот земной,] и все птицы небесные, все, что движется на земле, и все рыбы морские: в ваши руки отданы они;
\vs Gen 9:3 все движущееся, что живет, будет вам в пищу; как зелень травную даю вам все;
\vs Gen 9:4 только плоти с душею ее, с кровью ее, не ешьте;
\vs Gen 9:5 Я взыщу и вашу кровь, \bibemph{в которой} жизнь ваша, взыщу ее от всякого зверя, взыщу также душу человека от руки человека, от руки брата его;
\vs Gen 9:6 кто прольет кровь человеческую, того кровь прольется рукою человека: ибо человек создан по образу Божию;
\vs Gen 9:7 вы же плодитесь и размножайтесь, и распространяйтесь по земле, и умножайтесь на ней.
\rsbpar\vs Gen 9:8 И сказал Бог Ною и сынам его с ним:
\vs Gen 9:9 вот, Я поставляю завет Мой с вами и с потомством вашим после вас,
\vs Gen 9:10 и со всякою душею живою, которая с вами, с птицами и со скотами, и со всеми зверями земными, которые у вас, со всеми вышедшими из ковчега, со всеми животными земными;
\vs Gen 9:11 поставляю завет Мой с вами, что не будет более истреблена всякая плоть водами потопа, и не будет уже потопа на опустошение земли.
\vs Gen 9:12 И сказал [Господь] Бог: вот знамение завета, который Я поставляю между Мною и между вами и между всякою душею живою, которая с вами, в роды навсегда:
\vs Gen 9:13 Я полагаю радугу Мою в облаке, чтоб она была знамением [вечного] завета между Мною и между землею.
\vs Gen 9:14 И будет, когда Я наведу облако на землю, то явится радуга [Моя] в облаке;
\vs Gen 9:15 и Я вспомню завет Мой, который между Мною и между вами и между всякою душею живою во всякой плоти; и не будет более вода потопом на истребление всякой плоти.
\vs Gen 9:16 И будет радуга [Моя] в облаке, и Я увижу ее, и вспомню завет вечный между Богом [и между землею] и между всякою душею живою во всякой плоти, которая на земле.
\vs Gen 9:17 И сказал Бог Ною: вот знамение завета, который Я поставил между Мною и между всякою плотью, которая на земле.
\rsbpar\vs Gen 9:18 Сыновья Ноя, вышедшие из ковчега, были: Сим, Хам и Иафет. Хам же был отец Ханаана.
\vs Gen 9:19 Сии трое были сыновья Ноевы, и от них населилась вся земля.
\rsbpar\vs Gen 9:20 Ной начал возделывать землю и насадил виноградник;
\vs Gen 9:21 и выпил он вина, и опьянел, и \bibemph{лежал} обнаженным в шатре своем.
\vs Gen 9:22 И увидел Хам, отец Ханаана, наготу отца своего, и выйдя рассказал двум братьям своим.
\vs Gen 9:23 Сим же и Иафет взяли одежду и, положив ее на плечи свои, пошли задом и покрыли наготу отца своего; лица их были обращены назад, и они не видали наготы отца своего.
\vs Gen 9:24 Ной проспался от вина своего и узнал, что сделал над ним меньший сын его,
\vs Gen 9:25 и сказал: проклят Ханаан; раб рабов будет он у братьев своих.
\vs Gen 9:26 Потом сказал: благословен Господь Бог Симов; Ханаан же будет рабом ему;
\vs Gen 9:27 да распространит Бог Иафета, и да вселится он в шатрах Симовых; Ханаан же будет рабом ему.
\rsbpar\vs Gen 9:28 И жил Ной после потопа триста пятьдесят лет.
\vs Gen 9:29 Всех же дней Ноевых было девятьсот пятьдесят лет, и он умер.
\vs Gen 10:1 Вот родословие сынов Ноевых: Сима, Хама и Иафета. После потопа родились у них дети.
\rsbpar\vs Gen 10:2 Сыны Иафета: Гомер, Магог, Мадай, Иаван, [Елиса,] Фувал, Мешех и Фирас.
\vs Gen 10:3 Сыны Гомера: Аскеназ, Рифат и Фогарма.
\vs Gen 10:4 Сыны Иавана: Елиса, Фарсис, Киттим и Доданим.
\vs Gen 10:5 От сих населились острова народов в землях их, каждый по языку своему, по племенам своим, в народах своих.
\rsbpar\vs Gen 10:6 Сыны Хама: Хуш, Мицраим, Фут и Ханаан.
\vs Gen 10:7 Сыны Хуша: Сева, Хавила, Савта, Раама и Савтеха. Сыны Раамы: Шева и Дедан.
\vs Gen 10:8 Хуш родил также Нимрода; сей начал быть силен на земле;
\vs Gen 10:9 он был сильный зверолов пред Господом [Богом], потому и говорится: сильный зверолов, как Нимрод, пред Господом [Богом].
\vs Gen 10:10 Царство его вначале \bibemph{составляли}: Вавилон, Эрех, Аккад и Халне в земле Сеннаар.
\vs Gen 10:11 Из сей земли вышел Ассур и построил Ниневию, Реховоф-ир, Калах
\vs Gen 10:12 и Ресен между Ниневиею и между Калахом; это город великий.
\vs Gen 10:13 От Мицраима произошли Лудим, Анамим, Легавим, Нафтухим,
\vs Gen 10:14 Патрусим, Каслухим, откуда вышли Филистимляне, и Кафторим.
\vs Gen 10:15 От Ханаана родились: Сидон, первенец его, Хет,
\vs Gen 10:16 Иевусей, Аморрей, Гергесей,
\vs Gen 10:17 Евей, Аркей, Синей,
\vs Gen 10:18 Арвадей, Цемарей и Химафей. Впоследствии племена Ханаанские рассеялись,
\vs Gen 10:19 и были пределы Хананеев от Сидона к Герару до Газы, отсюда к Содому, Гоморре, Адме и Цевоиму до Лаши.
\vs Gen 10:20 Это сыны Хамовы, по племенам их, по языкам их, в землях их, в народах их.
\rsbpar\vs Gen 10:21 Были дети и у Сима, отца всех сынов Еверовых, старшего брата Иафетова.
\vs Gen 10:22 Сыны Сима: Елам, Ассур, Арфаксад, Луд, Арам [и Каинан].
\vs Gen 10:23 Сыны Арама: Уц, Хул, Гефер и Маш.
\vs Gen 10:24 Арфаксад родил [Каинана, Каинан родил] Салу, Сала родил Евера.
\vs Gen 10:25 У Евера родились два сына; имя одному: Фалек, потому что во дни его земля разделена; имя брату его: Иоктан.
\vs Gen 10:26 Иоктан родил Алмодада, Шалефа, Хацармавефа, Иераха,
\vs Gen 10:27 Гадорама, Узала, Диклу,
\vs Gen 10:28 Овала, Авимаила, Шеву,
\vs Gen 10:29 Офира, Хавилу и Иовава. Все эти сыновья Иоктана.
\vs Gen 10:30 Поселения их были от Меши до Сефара, горы восточной.
\vs Gen 10:31 Это сыновья Симовы по племенам их, по языкам их, в землях их, по народам их.
\vs Gen 10:32 Вот племена сынов Ноевых, по родословию их, в народах их. От них распространились народы на земле после потопа.
\vs Gen 11:1 На всей земле был один язык и одно наречие.
\vs Gen 11:2 Двинувшись с востока, они нашли в земле Сеннаар равнину и поселились там.
\vs Gen 11:3 И сказали друг другу: наделаем кирпичей и обожжем огнем. И стали у них кирпичи вместо камней, а земляная смола вместо извести.
\vs Gen 11:4 И сказали они: построим себе город и башню, высотою до небес, и сделаем себе имя, прежде нежели рассеемся по лицу всей земли.
\vs Gen 11:5 И сошел Господь посмотреть город и башню, которые строили сыны человеческие.
\vs Gen 11:6 И сказал Господь: вот, один народ, и один у всех язык; и вот что начали они делать, и не отстанут они от того, что задумали делать;
\vs Gen 11:7 сойдем же и смешаем там язык их, так чтобы один не понимал речи другого.
\vs Gen 11:8 И рассеял их Господь оттуда по всей земле; и они перестали строить город [и башню].
\vs Gen 11:9 Посему дано ему имя: Вавилон, ибо там смешал Господь язык всей земли, и оттуда рассеял их Господь по всей земле.
\rsbpar\vs Gen 11:10 Вот родословие Сима: Сим был ста лет и родил Арфаксада, чрез два года после потопа;
\vs Gen 11:11 по рождении Арфаксада Сим жил пятьсот лет и родил сынов и дочерей [и умер].
\vs Gen 11:12 Арфаксад жил тридцать пять [135] лет и родил [Каинана. По рождении Каинана Арфаксад жил триста тридцать лет и родил сынов и дочерей и умер. Каинан жил сто тридцать лет, и родил] Салу.
\vs Gen 11:13 По рождении Салы Арфаксад [Каинан] жил четыреста три [330] года и родил сынов и дочерей [и умер].
\vs Gen 11:14 Сала жил тридцать [130] лет и родил Евера.
\vs Gen 11:15 По рождении Евера Сала жил четыреста три [330] года и родил сынов и дочерей [и умер].
\vs Gen 11:16 Евер жил тридцать четыре [134] года и родил Фалека.
\vs Gen 11:17 По рождении Фалека Евер жил четыреста тридцать [370] лет и родил сынов и дочерей [и умер].
\vs Gen 11:18 Фалек жил тридцать [130] лет и родил Рагава.
\vs Gen 11:19 По рождении Рагава Фалек жил двести девять лет и родил сынов и дочерей [и умер].
\vs Gen 11:20 Рагав жил тридцать два [132] года и родил Серуха.
\vs Gen 11:21 По рождении Серуха Рагав жил двести семь лет и родил сынов и дочерей [и умер].
\vs Gen 11:22 Серух жил тридцать [130] лет и родил Нахора.
\vs Gen 11:23 По рождении Нахора Серух жил двести лет и родил сынов и дочерей [и умер].
\vs Gen 11:24 Нахор жил двадцать девять [79] лет и родил Фарру.
\vs Gen 11:25 По рождении Фарры Нахор жил сто девятнадцать [129] лет и родил сынов и дочерей [и умер].
\vs Gen 11:26 Фарра жил семьдесят лет и родил Аврама, Нахора и Арана.
\rsbpar\vs Gen 11:27 Вот родословие Фарры: Фарра родил Аврама, Нахора и Арана. Аран родил Лота.
\vs Gen 11:28 И умер Аран при Фарре, отце своем, в земле рождения своего, в Уре Халдейском.
\vs Gen 11:29 Аврам и Нахор взяли себе жен; имя жены Аврамовой: Сара; имя жены Нахоровой: Милка, дочь Арана, отца Милки и отца Иски.
\vs Gen 11:30 И Сара была неплодна и бездетна.
\vs Gen 11:31 И взял Фарра Аврама, сына своего, и Лота, сына Аранова, внука своего, и Сару, невестку свою, жену Аврама, сына своего, и вышел с ними из Ура Халдейского, чтобы идти в землю Ханаанскую; но, дойдя до Харрана, они остановились там.
\vs Gen 11:32 И было дней \bibemph{жизни} Фарры [в Харранской земле] двести пять лет, и умер Фарра в Харране.
\vs Gen 12:1 И сказал Господь Авраму: пойди из земли твоей, от родства твоего и из дома отца твоего [и иди] в землю, которую Я укажу тебе;
\vs Gen 12:2 и Я произведу от тебя великий народ, и благословлю тебя, и возвеличу имя твое, и будешь ты в благословение;
\vs Gen 12:3 Я благословлю благословляющих тебя, и злословящих тебя прокляну; и благословятся в тебе все племена земные.
\rsbpar\vs Gen 12:4 И пошел Аврам, как сказал ему Господь; и с ним пошел Лот. Аврам был семидесяти пяти лет, когда вышел из Харрана.
\vs Gen 12:5 И взял Аврам с собою Сару, жену свою, Лота, сына брата своего, и все имение, которое они приобрели, и всех людей, которых они имели в Харране; и вышли, чтобы идти в землю Ханаанскую; и пришли в землю Ханаанскую.
\vs Gen 12:6 И прошел Аврам по земле сей [по длине ее] до места Сихема, до дубравы Мор\acc{е}. В этой земле тогда [жили] Хананеи.
\vs Gen 12:7 И явился Господь Авраму и сказал [ему]: потомству твоему отдам Я землю сию. И создал там [Аврам] жертвенник Господу, Который явился ему.
\vs Gen 12:8 Оттуда двинулся он к горе, на восток от Вефиля; и поставил шатер свой \bibemph{так, что от него} Вефиль \bibemph{был} на запад, а Гай на восток; и создал там жертвенник Господу и призвал имя Господа [явившегося ему].
\vs Gen 12:9 И поднялся Аврам и продолжал идти к югу.
\rsbpar\vs Gen 12:10 И был голод в той земле. И сошел Аврам в Египет, пожить там, потому что усилился голод в земле той.
\vs Gen 12:11 Когда же он приближался к Египту, то сказал Саре, жене своей: вот, я знаю, что ты женщина, прекрасная видом;
\vs Gen 12:12 и когда Египтяне увидят тебя, то скажут: это жена его; и убьют меня, а тебя оставят в живых;
\vs Gen 12:13 скажи же, что ты мне сестра, дабы мне хорошо было ради тебя, и дабы жива была душа моя чрез тебя.
\vs Gen 12:14 И было, когда пришел Аврам в Египет, Египтяне увидели, что она женщина весьма красивая;
\vs Gen 12:15 увидели ее и вельможи фараоновы и похвалили ее фараону; и взята была она в дом фараонов.
\vs Gen 12:16 И Авраму хорошо было ради ее; и был у него мелкий и крупный скот и ослы, и рабы и рабыни, и лошаки и верблюды.
\vs Gen 12:17 Но Господь поразил тяжкими ударами фараона и дом его за Сару, жену Аврамову.
\vs Gen 12:18 И призвал фараон Аврама и сказал: что ты это сделал со мною? для чего не сказал мне, что она жена твоя?
\vs Gen 12:19 для чего ты сказал: она сестра моя? и я взял было ее себе в жену. И теперь вот жена твоя; возьми [ее] и пойди.
\vs Gen 12:20 И дал о нем фараон повеление людям, и проводили его, и жену его, и все, что у него было, [и Лота с ним].
\vs Gen 13:1 И поднялся Аврам из Египта, сам и жена его, и всё, что у него было, и Лот с ним, на юг.
\vs Gen 13:2 И был Аврам очень богат скотом, и серебром, и золотом.
\vs Gen 13:3 И продолжал он переходы свои от юга до Вефиля, до места, где прежде был шатер его между Вефилем и между Гаем,
\vs Gen 13:4 до места жертвенника, который он сделал там вначале; и там призвал Аврам имя Господа.
\rsbpar\vs Gen 13:5 И у Лота, который ходил с Аврамом, также был мелкий и крупный скот и шатры.
\vs Gen 13:6 И непоместительна была земля для них, чтобы жить вместе, ибо имущество их было так велико, что они не могли жить вместе.
\vs Gen 13:7 И был спор между пастухами скота Аврамова и между пастухами скота Лотова; и Хананеи и Ферезеи жили тогда в той земле.
\vs Gen 13:8 И сказал Аврам Лоту: да не будет раздора между мною и тобою, и между пастухами моими и пастухами твоими, ибо мы родственники;
\vs Gen 13:9 не вся ли земля пред тобою? отделись же от меня: если ты налево, то я направо; а если ты направо, то я налево.
\vs Gen 13:10 Лот возвел очи свои и увидел всю окрестность Иорданскую, что она, прежде нежели истребил Господь Содом и Гоморру, вся до Сигора орошалась водою, как сад Господень, как земля Египетская;
\vs Gen 13:11 и избрал себе Лот всю окрестность Иорданскую; и двинулся Лот к востоку. И отделились они друг от друга.
\vs Gen 13:12 Аврам стал жить на земле Ханаанской; а Лот стал жить в городах окрестности и раскинул шатры до Содома.
\vs Gen 13:13 Жители же Содомские были злы и весьма грешны пред Господом.
\rsbpar\vs Gen 13:14 И сказал Господь Авраму, после того как Лот отделился от него: возведи очи твои и с места, на котором ты теперь, посмотри к северу и к югу, и к востоку и к западу;
\vs Gen 13:15 ибо всю землю, которую ты видишь, тебе дам Я и потомству твоему навеки,
\vs Gen 13:16 и сделаю потомство твое, как песок земной; если кто может сосчитать песок земной, то и потомство твое сочтено будет;
\vs Gen 13:17 встань, пройди по земле сей в долготу и в широту ее, ибо Я тебе дам ее [и потомству твоему навсегда].
\vs Gen 13:18 И двинул Аврам шатер, и пошел, и поселился у дубравы Мамре, что в Хевроне; и создал там жертвенник Господу.
\vs Gen 14:1 И было во дни Амрафела, царя Сеннаарского, Ариоха, царя Елласарского, Кедорлаомера, царя Еламского, и Фидала, царя Гоимского,
\vs Gen 14:2 пошли они войною против Беры, царя Содомского, против Бирши, царя Гоморрского, Шинава, царя Адмы, Шемевера, царя Севоимского, и против царя Белы, которая есть Сигор.
\vs Gen 14:3 Все сии соединились в долине Сиддим, где \bibemph{ныне} море Соленое.
\vs Gen 14:4 Двенадцать лет были они в порабощении у Кедорлаомера, а в тринадцатом году возмутились.
\vs Gen 14:5 В четырнадцатом году пришел Кедорлаомер и цари, которые с ним, и поразили Рефаимов в Аштероф-Карнаиме, Зузимов в Гаме, Эмимов в Шаве-Кириафаиме,
\vs Gen 14:6 и Хорреев в горе их Сеире, до Эл-Фарана, что при пустыне.
\vs Gen 14:7 И возвратившись оттуда, они пришли к источнику Мишпат, который есть Кадес, и поразили всю страну Амаликитян, и также Аморреев, живущих в Хацацон-Фамаре.
\vs Gen 14:8 И вышли царь Содомский, царь Гоморрский, царь Адмы, царь Севоимский и царь Белы, которая есть Сигор; и вступили в сражение с ними в долине Сиддим,
\vs Gen 14:9 с Кедорлаомером, царем Еламским, Фидалом, царем Гоимским, Амрафелом, царем Сеннаарским, Ариохом, царем Елласарским,~--- четыре царя против пяти.
\vs Gen 14:10 В долине же Сиддим было много смоляных ям. И цари Содомский и Гоморрский, обратившись в бегство, упали в них, а остальные убежали в горы.
\vs Gen 14:11 \bibemph{Победители} взяли все имущество Содома и Гоморры и весь запас их и ушли.
\vs Gen 14:12 И взяли Лота, племянника Аврамова, жившего в Содоме, и имущество его и ушли.
\rsbpar\vs Gen 14:13 И пришел один из уцелевших и известил Аврама Еврея, жившего тогда у дубравы Мамре, Аморреянина, брата Эшколу и брата Анеру, которые были союзники Аврамовы.
\vs Gen 14:14 Аврам, услышав, что [Лот] сродник его взят в плен, вооружил рабов своих, рожденных в доме его, триста восемнадцать, и преследовал \bibemph{неприятелей} до Дана;
\vs Gen 14:15 и, разделившись, \bibemph{напал} на них ночью, сам и рабы его, и поразил их, и преследовал их до Ховы, что по левую сторону Дамаска;
\vs Gen 14:16 и возвратил все имущество и Лота, сродника своего, и имущество его возвратил, также и женщин и народ.
\rsbpar\vs Gen 14:17 Когда он возвращался после поражения Кедорлаомера и царей, бывших с ним, царь Содомский вышел ему навстречу в долину Шаве, что \bibemph{ныне} долина царская;
\vs Gen 14:18 и Мелхиседек, царь Салимский, вынес хлеб и вино,~--- он был священник Бога Всевышнего,~---
\vs Gen 14:19 и благословил его, и сказал: благословен Аврам от Бога Всевышнего, Владыки неба и земли;
\vs Gen 14:20 и благословен Бог Всевышний, Который предал врагов твоих в руки твои. [Аврам] дал ему десятую часть из всего.
\vs Gen 14:21 И сказал царь Содомский Авраму: отдай мне людей, а имение возьми себе.
\vs Gen 14:22 Но Аврам сказал царю Содомскому: поднимаю руку мою к Господу Богу Всевышнему, Владыке неба и земли,
\vs Gen 14:23 что даже нитки и ремня от обуви не возьму из всего твоего, чтобы ты не сказал: я обогатил Аврама;
\vs Gen 14:24 кроме того, что съели отроки, и кроме доли, принадлежащей людям, которые ходили со мною; Анер, Эшкол и Мамрий пусть возьмут свою долю.
\vs Gen 15:1 После сих происшествий было слово Господа к Авраму в видении [ночью], и сказано: не бойся, Аврам; Я твой щит; награда твоя [будет] весьма велика.
\vs Gen 15:2 Аврам сказал: Владыка Господи! что Ты дашь мне? я остаюсь бездетным; распорядитель в доме моем этот Елиезер из Дамаска.
\vs Gen 15:3 И сказал Аврам: вот, Ты не дал мне потомства, и вот, домочадец мой наследник мой.
\vs Gen 15:4 И было слово Господа к нему, и сказано: не будет он твоим наследником, но тот, кто произойдет из чресл твоих, будет твоим наследником.
\vs Gen 15:5 И вывел его вон и сказал [ему]: посмотри на небо и сосчитай звезды, если ты можешь счесть их. И сказал ему: столько будет у тебя потомков.
\vs Gen 15:6 Аврам поверил Господу, и Он вменил ему это в праведность.
\rsbpar\vs Gen 15:7 И сказал ему: Я Господь, Который вывел тебя из Ура Халдейского, чтобы дать тебе землю сию во владение.
\vs Gen 15:8 Он сказал: Владыка Господи! по чему мне узнать, что я буду владеть ею?
\vs Gen 15:9 \bibemph{Господь} сказал ему: возьми Мне трехлетнюю телицу, трехлетнюю козу, трехлетнего овна, горлицу и молодого голубя.
\vs Gen 15:10 Он взял всех их, рассек их пополам и положил одну часть против другой; только птиц не рассек.
\vs Gen 15:11 И налетели на трупы хищные птицы; но Аврам отгонял их.
\rsbpar\vs Gen 15:12 При захождении солнца крепкий сон напал на Аврама, и вот, напал на него ужас и мрак великий.
\vs Gen 15:13 И сказал \bibemph{Господь} Авраму: знай, что потомки твои будут пришельцами в земле не своей, и поработят их, и будут угнетать их четыреста лет,
\vs Gen 15:14 но Я произведу суд над народом, у которого они будут в порабощении; после сего они выйдут [сюда] с большим имуществом,
\vs Gen 15:15 а ты отойдешь к отцам твоим в мире \bibemph{и} будешь погребен в старости доброй;
\vs Gen 15:16 в четвертом роде возвратятся они сюда: ибо \bibemph{мера} беззаконий Аморреев доселе еще не наполнилась.
\vs Gen 15:17 Когда зашло солнце и наступила тьма, вот, дым \bibemph{как бы из} печи и пламя огня прошли между рассеченными \bibemph{животными}.
\vs Gen 15:18 В этот день заключил Господь завет с Аврамом, сказав: потомству твоему даю Я землю сию, от реки Египетской до великой реки, реки Евфрата:
\vs Gen 15:19 Кенеев, Кенезеев, Кедмонеев,
\vs Gen 15:20 Хеттеев, Ферезеев, Рефаимов,
\vs Gen 15:21 Аморреев, Хананеев, [Евеев,] Гергесеев и Иевусеев.
\vs Gen 16:1 Но Сара, жена Аврамова, не рождала ему. У ней была служанка Египтянка, именем Агарь.
\vs Gen 16:2 И сказала Сара Авраму: вот, Господь заключил чрево мое, чтобы мне не рождать; войди же к служанке моей: может быть, я буду иметь детей от нее. Аврам послушался слов Сары.
\vs Gen 16:3 И взяла Сара, жена Аврамова, служанку свою, Египтянку Агарь, по истечении десяти лет пребывания Аврамова в земле Ханаанской, и дала ее Авраму, мужу своему, в жену.
\vs Gen 16:4 Он вошел к Агари, и она зачала. Увидев же, что зачала, она стала презирать госпожу свою.
\vs Gen 16:5 И сказала Сара Авраму: в обиде моей ты виновен; я отдала служанку мою в недро твое; а она, увидев, что зачала, стала презирать меня; Господь пусть будет судьею между мною и между тобою.
\vs Gen 16:6 Аврам сказал Саре: вот, служанка твоя в твоих руках; делай с нею, что тебе угодно. И Сара стала притеснять ее, и она убежала от нее.
\vs Gen 16:7 И нашел ее Ангел Господень у источника воды в пустыне, у источника на дороге к Суру.
\vs Gen 16:8 И сказал [ей Ангел Господень]: Агарь, служанка Сарина! откуда ты пришла и куда идешь? Она сказала: я бегу от лица Сары, госпожи моей.
\vs Gen 16:9 Ангел Господень сказал ей: возвратись к госпоже своей и покорись ей.
\vs Gen 16:10 И сказал ей Ангел Господень: умножая умножу потомство твое, так что нельзя будет и счесть его от множества.
\vs Gen 16:11 И еще сказал ей Ангел Господень: вот, ты беременна, и родишь сына, и наречешь ему имя Измаил, ибо услышал Господь страдание твое;
\vs Gen 16:12 он будет \bibemph{между} людьми, \bibemph{как} дикий осел; руки его на всех, и руки всех на него; жить будет он пред лицем всех братьев своих.
\vs Gen 16:13 И нарекла [Агарь] Господа, Который говорил к ней, \bibemph{сим} именем: Ты Бог видящий меня. Ибо сказала она: точно я видела здесь в след видящего меня.
\vs Gen 16:14 Посему источник \bibemph{тот} называется: Беэр-лахай-рои\fns{Источник Живаго, видящего меня.}. Он находится между Кадесом и между Баредом.
\rsbpar\vs Gen 16:15 Агарь родила Авраму сына; и нарек [Аврам] имя сыну своему, рожденному от Агари: Измаил.
\vs Gen 16:16 Аврам был восьмидесяти шести лет, когда Агарь родила Авраму Измаила.
\vs Gen 17:1 Аврам был девяноста девяти лет, и Господь явился Авраму и сказал ему: Я Бог Всемогущий; ходи предо Мною и будь непорочен;
\vs Gen 17:2 и поставлю завет Мой между Мною и тобою, и весьма, весьма размножу тебя.
\vs Gen 17:3 И пал Аврам на лице свое. Бог продолжал говорить с ним и сказал:
\vs Gen 17:4 Я~--- вот завет Мой с тобою: ты будешь отцом множества народов,
\vs Gen 17:5 и не будешь ты больше называться Аврамом, но будет тебе имя: Авраам, ибо Я сделаю тебя отцом множества народов;
\vs Gen 17:6 и весьма, весьма распложу тебя, и произведу от тебя народы, и цари произойдут от тебя;
\vs Gen 17:7 и поставлю завет Мой между Мною и тобою и между потомками твоими после тебя в роды их, завет вечный в том, что Я буду Богом твоим и потомков твоих после тебя;
\vs Gen 17:8 и дам тебе и потомкам твоим после тебя землю, по которой ты странствуешь, всю землю Ханаанскую, во владение вечное; и буду им Богом.
\vs Gen 17:9 И сказал Бог Аврааму: ты же соблюди завет Мой, ты и потомки твои после тебя в роды их.
\vs Gen 17:10 Сей есть завет Мой, который вы \bibemph{должны} соблюдать между Мною и между вами и между потомками твоими после тебя [в роды их]: да будет у вас обрезан весь мужеский пол;
\vs Gen 17:11 обрезывайте крайнюю плоть вашу: и сие будет знамением завета между Мною и вами.
\vs Gen 17:12 Восьми дней от рождения да будет обрезан у вас в роды ваши всякий \bibemph{младенец} мужеского пола, рожденный в доме и купленный за серебро у какого-нибудь иноплеменника, который не от твоего семени.
\vs Gen 17:13 Непременно да будет обрезан рожденный в доме твоем и купленный за серебро твое, и будет завет Мой на теле вашем заветом вечным.
\vs Gen 17:14 Необрезанный же мужеского пола, который не обрежет крайней плоти своей [в восьмой день], истребится душа та из народа своего, \bibemph{ибо} он нарушил завет Мой.
\vs Gen 17:15 И сказал Бог Аврааму: Сару, жену твою, не называй Сарою, но да будет имя ей: Сарра;
\vs Gen 17:16 Я благословлю ее и дам тебе от нее сына; благословлю ее, и произойдут от нее народы, и цари народов произойдут от нее.
\vs Gen 17:17 И пал Авраам на лице свое, и рассмеялся, и сказал сам в себе: неужели от столетнего будет сын? и Сарра, девяностолетняя, неужели родит?
\vs Gen 17:18 И сказал Авраам Богу: о, хотя бы Измаил был жив пред лицем Твоим!
\vs Gen 17:19 Бог же сказал [Аврааму]: именно Сарра, жена твоя, родит тебе сына, и ты наречешь ему имя: Исаак; и поставлю завет Мой с ним заветом вечным [в том, что Я буду Богом ему и] потомству его после него.
\vs Gen 17:20 И о Измаиле Я услышал тебя: вот, Я благословлю его, и возращу его, и весьма, весьма размножу; двенадцать князей родятся от него; и Я произведу от него великий народ.
\vs Gen 17:21 Но завет Мой поставлю с Исааком, которого родит тебе Сарра в сие самое время на другой год.
\vs Gen 17:22 И Бог перестал говорить с Авраамом и восшел от него.
\rsbpar\vs Gen 17:23 И взял Авраам Измаила, сына своего, и всех рожденных в доме своем и всех купленных за серебро свое, весь мужеский пол людей дома Авраамова; и обрезал крайнюю плоть их в тот самый день, как сказал ему Бог.
\vs Gen 17:24 Авраам был девяноста девяти лет, когда была обрезана крайняя плоть его.
\vs Gen 17:25 А Измаил, сын его, был тринадцати лет, когда была обрезана крайняя плоть его.
\vs Gen 17:26 В тот же самый день обрезаны были Авраам и Измаил, сын его,
\vs Gen 17:27 и с ним обрезан был весь мужеский пол дома его, рожденные в доме и купленные за серебро у иноплеменников.
\vs Gen 18:1 И явился ему Господь у дубравы Мамре, когда он сидел при входе в шатер [свой], во время зноя дневного.
\vs Gen 18:2 Он возвел очи свои и взглянул, и вот, три мужа стоят против него. Увидев, он побежал навстречу им от входа в шатер [свой] и поклонился до земли,
\vs Gen 18:3 и сказал: Владыка! если я обрел благоволение пред очами Твоими, не пройди мимо раба Твоего;
\vs Gen 18:4 и принесут немного воды, и омоют ноги ваши; и отдохните под сим деревом,
\vs Gen 18:5 а я принесу хлеба, и вы подкрепите сердца ваши; потом пойдите [в путь свой]; так как вы идете мимо раба вашего. Они сказали: сделай так, как говоришь.
\vs Gen 18:6 И поспешил Авраам в шатер к Сарре и сказал [ей]: поскорее замеси три саты лучшей муки и сделай пресные хлебы.
\vs Gen 18:7 И побежал Авраам к стаду, и взял теленка нежного и хорошего, и дал отроку, и тот поспешил приготовить его.
\vs Gen 18:8 И взял масла и молока и теленка приготовленного, и поставил перед ними, а сам стоял подле них под деревом. И они ели.
\vs Gen 18:9 И сказали ему: где Сарра, жена твоя? Он отвечал: здесь, в шатре.
\vs Gen 18:10 И сказал \bibemph{один из них}: Я опять буду у тебя в это же время [в следующем году], и будет сын у Сарры, жены твоей. А Сарра слушала у входа в шатер, сзади его.
\vs Gen 18:11 Авраам же и Сарра были стары и в летах преклонных, и обыкновенное у женщин у Сарры прекратилось.
\vs Gen 18:12 Сарра внутренно рассмеялась, сказав: мне ли, когда я состарилась, иметь сие утешение? и господин мой стар.
\vs Gen 18:13 И сказал Господь Аврааму: отчего это [сама в себе] рассмеялась Сарра, сказав: <<неужели я действительно могу родить, когда я состарилась>>?
\vs Gen 18:14 Есть ли что трудное для Господа? В назначенный срок буду Я у тебя в следующем году, и [будет] у Сарры сын.
\vs Gen 18:15 Сарра же не призналась, а сказала: я не смеялась. Ибо она испугалась. Но Он сказал [ей]: нет, ты рассмеялась.
\vs Gen 18:16 И встали те мужи и оттуда отправились к Содому [и Гоморре]; Авраам же пошел с ними, проводить их.
\rsbpar\vs Gen 18:17 И сказал Господь: утаю ли Я от Авраама [раба Моего], что хочу делать!
\vs Gen 18:18 От Авраама точно произойдет народ великий и сильный, и благословятся в нем все народы земли,
\vs Gen 18:19 ибо Я избрал его для того, чтобы он заповедал сынам своим и дому своему после себя, ходить путем Господним, творя правду и суд; и исполнит Господь над Авраамом [все], что сказал о нем.
\vs Gen 18:20 И сказал Господь: вопль Содомский и Гоморрский, велик он, и грех их, тяжел он весьма;
\vs Gen 18:21 сойду и посмотрю, точно ли они поступают так, каков вопль на них, восходящий ко Мне, или нет; узнаю.
\vs Gen 18:22 И обратились мужи оттуда и пошли в Содом; Авраам же еще стоял пред лицем Господа.
\vs Gen 18:23 И подошел Авраам и сказал: неужели Ты погубишь праведного с нечестивым [и с праведником будет то же, что с нечестивым]?
\vs Gen 18:24 может быть, есть в этом городе пятьдесят праведников? неужели Ты погубишь, и не пощадишь [всего] места сего ради пятидесяти праведников, [если они находятся] в нем?
\vs Gen 18:25 не может быть, чтобы Ты поступил так, чтобы Ты погубил праведного с нечестивым, чтобы то же было с праведником, что с нечестивым; не может быть от Тебя! Судия всей земли поступит ли неправосудно?
\vs Gen 18:26 Господь сказал: если Я найду в городе Содоме пятьдесят праведников, то Я ради них пощажу [весь город и] все место сие.
\vs Gen 18:27 Авраам сказал в ответ: вот, я решился говорить Владыке, я, прах и пепел:
\vs Gen 18:28 может быть, до пятидесяти праведников недостанет пяти, неужели за \bibemph{недостатком} пяти Ты истребишь весь город? Он сказал: не истреблю, если найду там сорок пять.
\vs Gen 18:29 \bibemph{Авраам} продолжал говорить с Ним и сказал: может быть, найдется там сорок? Он сказал: не сделаю \bibemph{того} и ради сорока.
\vs Gen 18:30 И сказал \bibemph{Авраам}: да не прогневается Владыка, что я буду говорить: может быть, найдется там тридцать? Он сказал: не сделаю, если найдется там тридцать.
\vs Gen 18:31 \bibemph{Авраам} сказал: вот, я решился говорить Владыке: может быть, найдется там двадцать? Он сказал: не истреблю ради двадцати.
\vs Gen 18:32 \bibemph{Авраам} сказал: да не прогневается Владыка, что я скажу еще однажды: может быть, найдется там десять? Он сказал: не истреблю ради десяти.
\vs Gen 18:33 И пошел Господь, перестав говорить с Авраамом; Авраам же возвратился в свое место.
\vs Gen 19:1 И пришли те два Ангела в Содом вечером, когда Лот сидел у ворот Содома. Лот увидел, и встал, чтобы встретить их, и поклонился лицем до земли
\vs Gen 19:2 и сказал: государи мои! зайдите в дом раба вашего и ночуйте, и умойте ноги ваши, и встаньте поутру и пойдете в путь свой. Но они сказали: нет, мы ночуем на улице.
\vs Gen 19:3 Он же сильно упрашивал их; и они пошли к нему и пришли в дом его. Он сделал им угощение и испек пресные хлебы, и они ели.
\vs Gen 19:4 Еще не легли они спать, как городские жители, Содомляне, от молодого до старого, весь народ со \bibemph{всех} концов \bibemph{города}, окружили дом
\vs Gen 19:5 и вызвали Лота и говорили ему: где люди, пришедшие к тебе на ночь? выведи их к нам; мы позн\acc{а}ем их.
\vs Gen 19:6 Лот вышел к ним ко входу, и запер за собою дверь,
\vs Gen 19:7 и сказал [им]: братья мои, не делайте зла;
\vs Gen 19:8 вот у меня две дочери, которые не познали мужа; лучше я выведу их к вам, делайте с ними, что вам угодно, только людям сим не делайте ничего, так как они пришли под кров дома моего.
\vs Gen 19:9 Но они сказали [ему]: пойди сюда. И сказали: вот пришлец, и хочет судить? теперь мы хуже поступим с тобою, нежели с ними. И очень приступали к человеку сему, к Лоту, и подошли, чтобы выломать дверь.
\vs Gen 19:10 Тогда мужи те простерли руки свои и ввели Лота к себе в дом, и дверь [дома] заперли;
\vs Gen 19:11 а людей, бывших при входе в дом, поразили слепотою, от малого до большого, так что они измучились, искав входа.
\vs Gen 19:12 Сказали мужи те Лоту: кто у тебя есть еще здесь? зять ли, сыновья ли твои, дочери ли твои, и кто бы ни был у тебя в городе, всех выведи из сего места,
\vs Gen 19:13 ибо мы истребим сие место, потому что велик вопль на жителей его к Господу, и Господь послал нас истребить его.
\vs Gen 19:14 И вышел Лот, и говорил с зятьями своими, которые брали за себя дочерей его, и сказал: встаньте, выйдите из сего места, ибо Господь истребит сей город. Но зятьям его показалось, что он шутит.
\rsbpar\vs Gen 19:15 Когда взошла заря, Ангелы начали торопить Лота, говоря: встань, возьми жену твою и двух дочерей твоих, которые у тебя, чтобы не погибнуть тебе за беззакония города.
\vs Gen 19:16 И как он медлил, то мужи те [Ангелы], по милости к нему Господней, взяли за руку его и жену его, и двух дочерей его, и вывели его и поставили его вне города.
\vs Gen 19:17 Когда же вывели их вон, \bibemph{то один из них} сказал: спасай душу свою; не оглядывайся назад и нигде не останавливайся в окрестности сей; спасайся на гору, чтобы тебе не погибнуть.
\vs Gen 19:18 Но Лот сказал им: нет, Владыка!
\vs Gen 19:19 вот, раб Твой обрел благоволение пред очами Твоими, и велика милость Твоя, которую Ты сделал со мною, что спас жизнь мою; но я не могу спасаться на гору, чтоб не застигла меня беда и мне не умереть;
\vs Gen 19:20 вот, ближе бежать в сей город, он же мал; побегу я туда,~--- он же мал; и сохранится жизнь моя [ради Тебя].
\vs Gen 19:21 И сказал ему: вот, в угодность тебе Я сделаю и это: не ниспровергну города, о котором ты говоришь;
\vs Gen 19:22 поспешай, спасайся туда, ибо Я не могу сделать дела, доколе ты не придешь туда. Потому и назван город сей: Сигор.
\vs Gen 19:23 Солнце взошло над землею, и Лот пришел в Сигор.
\rsbpar\vs Gen 19:24 И пролил Господь на Содом и Гоморру дождем серу и огонь от Господа с неба,
\vs Gen 19:25 и ниспроверг города сии, и всю окрестность сию, и всех жителей городов сих, и [все] произрастания земли.
\vs Gen 19:26 Жена же \bibemph{Лотова} оглянулась позади его, и стала соляным столпом.
\rsbpar\vs Gen 19:27 И встал Авраам рано утром [и пошел] на место, где стоял пред лицем Господа,
\vs Gen 19:28 и посмотрел к Содому и Гоморре и на все пространство окрестности и увидел: вот, дым поднимается с земли, как дым из печи.
\vs Gen 19:29 И было, когда Бог истреблял [все] города окрестности сей, вспомнил Бог об Аврааме и выслал Лота из среды истребления, когда ниспровергал города, в которых жил Лот.
\rsbpar\vs Gen 19:30 И вышел Лот из Сигора и стал жить в гор\acc{е}, и с ним две дочери его, ибо он боялся жить в Сигоре. И жил в пещере, и с ним две дочери его.
\vs Gen 19:31 И сказала старшая младшей: отец наш стар, и нет человека на земле, который вошел бы к нам по обычаю всей земли;
\vs Gen 19:32 итак напоим отца нашего вином, и переспим с ним, и восставим от отца нашего племя.
\vs Gen 19:33 И напоили отца своего вином в ту ночь; и вошла старшая и спала с отцом своим [в ту ночь]; а он не знал, когда она легла и когда встала.
\vs Gen 19:34 На другой день старшая сказала младшей: вот, я спала вчера с отцом моим; напоим его вином и в эту ночь; и ты войди, спи с ним, и восставим от отца нашего племя.
\vs Gen 19:35 И напоили отца своего вином и в эту ночь; и вошла младшая и спала с ним; и он не знал, когда она легла и когда встала.
\vs Gen 19:36 И сделались обе дочери Лотовы беременными от отца своего,
\vs Gen 19:37 и родила старшая сына, и нарекла ему имя: Моав [говоря: \bibemph{он} от отца моего]. Он отец Моавитян доныне.
\vs Gen 19:38 И младшая также родила сына, и нарекла ему имя: Бен-Амми [говоря: \bibemph{он} сын рода моего]. Он отец Аммонитян доныне.
\vs Gen 20:1 Авраам поднялся оттуда к югу и поселился между Кадесом и между Суром; и был на время в Гераре.
\vs Gen 20:2 И сказал Авраам о Сарре, жене своей: она сестра моя. [Ибо он боялся сказать, что это жена его, чтобы жители города того не убили его за нее.] И послал Авимелех, царь Герарский, и взял Сарру.
\vs Gen 20:3 И пришел Бог к Авимелеху ночью во сне и сказал ему: вот, ты умрешь за женщину, которую ты взял, ибо она имеет мужа.
\vs Gen 20:4 Авимелех же не прикасался к ней и сказал: Владыка! неужели Ты погубишь [не знавший \bibemph{сего}] и невинный народ?
\vs Gen 20:5 Не сам ли он сказал мне: она сестра моя? И она сама сказала: он брат мой. Я сделал это в простоте сердца моего и в чистоте рук моих.
\vs Gen 20:6 И сказал ему Бог во сне: и Я знаю, что ты сделал сие в простоте сердца твоего, и удержал тебя от греха предо Мною, потому и не допустил тебя прикоснуться к ней;
\vs Gen 20:7 теперь же возврати жену мужу, ибо он пророк и помолится о тебе, и ты будешь жив; а если не возвратишь, то знай, что непременно умрешь ты и все твои.
\vs Gen 20:8 И встал Авимелех утром рано, и призвал всех рабов своих, и пересказал все слова сии в уши их; и люди сии [все] весьма испугались.
\vs Gen 20:9 И призвал Авимелех Авраама и сказал ему: что ты с нами сделал? чем согрешил я против тебя, что ты навел было на меня и на царство мое великий грех? Ты сделал со мною дела, каких не делают.
\vs Gen 20:10 И сказал Авимелех Аврааму: что ты имел в виду, когда делал это дело?
\vs Gen 20:11 Авраам сказал: я подумал, что нет на месте сем страха Божия, и убьют меня за жену мою;
\vs Gen 20:12 да она и подлинно сестра мне: она дочь отца моего, только не дочь матери моей; и сделалась моею женою;
\vs Gen 20:13 когда Бог повел меня странствовать из дома отца моего, то я сказал ей: сделай со мною сию милость, в какое ни придем мы место, везде говори обо мне: это брат мой.
\vs Gen 20:14 И взял Авимелех [серебра тысячу сиклей и] мелкого и крупного скота, и рабов и рабынь, и дал Аврааму; и возвратил ему Сарру, жену его.
\vs Gen 20:15 И сказал Авимелех [Аврааму]: вот, земля моя пред тобою; живи, где тебе угодно.
\vs Gen 20:16 И Сарре сказал: вот, я дал брату твоему тысячу \bibemph{сиклей} серебра; вот, это тебе покрывало для очей пред всеми, которые с тобою, и пред всеми ты оправдана.
\vs Gen 20:17 И помолился Авраам Богу, и исцелил Бог Авимелеха, и жену его, и рабынь его, и они стали рождать;
\vs Gen 20:18 ибо заключил Господь всякое чрево в доме Авимелеха за Сарру, жену Авраамову.
\vs Gen 21:1 И призрел Господь на Сарру, как сказал; и сделал Господь Сарре, как говорил.
\vs Gen 21:2 Сарра зачала и родила Аврааму сына в старости его во время, о котором говорил ему Бог;
\vs Gen 21:3 и нарек Авраам имя сыну своему, родившемуся у него, которого родила ему Сарра, Исаак;
\vs Gen 21:4 и обрезал Авраам Исаака, сына своего, в восьмой день, как заповедал ему Бог.
\vs Gen 21:5 Авраам был ста лет, когда родился у него Исаак, сын его.
\vs Gen 21:6 И сказала Сарра: смех сделал мне Бог; кто ни услышит обо мне, рассмеется.
\vs Gen 21:7 И сказала: кто сказал бы Аврааму: Сарра будет кормить детей грудью? ибо в старости его я родила сына.
\vs Gen 21:8 Дитя выросло и отнято от груди; и Авраам сделал большой пир в тот день, когда Исаак [сын его] отнят был от груди.
\rsbpar\vs Gen 21:9 И увидела Сарра, что сын Агари Египтянки, которого она родила Аврааму, насмехается [над ее сыном, Исааком],
\vs Gen 21:10 и сказала Аврааму: выгони эту рабыню и сына ее, ибо не наследует сын рабыни сей с сыном моим Исааком.
\vs Gen 21:11 И показалось это Аврааму весьма неприятным ради сына его [Измаила].
\vs Gen 21:12 Но Бог сказал Аврааму: не огорчайся ради отрока и рабыни твоей; во всем, что скажет тебе Сарра, слушайся голоса ее, ибо в Исааке наречется тебе семя;
\vs Gen 21:13 и от сына рабыни Я произведу [великий] народ, потому что он семя твое.
\vs Gen 21:14 Авраам встал рано утром, и взял хлеба и мех воды, и дал Агари, положив ей на плечи, и отрока, и отпустил ее. Она пошла, и заблудилась в пустыне Вирсавии;
\vs Gen 21:15 и не стало воды в мехе, и она оставила отрока под одним кустом
\vs Gen 21:16 и пошла, села вдали, в расстоянии на \bibemph{один} выстрел из лука. Ибо она сказала: не \bibemph{хочу} видеть смерти отрока. И она села [поодаль] против [него], и подняла вопль, и плакала;
\vs Gen 21:17 и услышал Бог голос отрока [оттуда, где он был]; и Ангел Божий с неба воззвал к Агари и сказал ей: что с тобою, Агарь? не бойся; Бог услышал голос отрока оттуда, где он находится;
\vs Gen 21:18 встань, подними отрока и возьми его за руку, ибо Я произведу от него великий народ.
\vs Gen 21:19 И Бог открыл глаза ее, и она увидела колодезь с водою [живою], и пошла, наполнила мех водою и напоила отрока.
\vs Gen 21:20 И Бог был с отроком; и он вырос, и стал жить в пустыне, и сделался стрелком из лука.
\vs Gen 21:21 Он жил в пустыне Фаран; и мать его взяла ему жену из земли Египетской.
\rsbpar\vs Gen 21:22 И было в то время, Авимелех с [Ахузафом невестоводителем и] Фихолом, военачальником своим, сказал Аврааму: с тобою Бог во всем, что ты ни делаешь;
\vs Gen 21:23 и теперь поклянись мне здесь Богом, что ты не обидишь ни меня, ни сына моего, ни внука моего; и как я хорошо поступал с тобою, так и ты будешь поступать со мною и землею, в которой ты гостишь.
\vs Gen 21:24 И сказал Авраам: я клянусь.
\vs Gen 21:25 И Авраам упрекал Авимелеха за колодезь с водою, который отняли рабы Авимелеховы.
\vs Gen 21:26 Авимелех же сказал [ему]: не знаю, кто это сделал, и ты не сказал мне; я даже и не слыхал \bibemph{о том} доныне.
\vs Gen 21:27 И взял Авраам мелкого и крупного скота и дал Авимелеху, и они оба заключили союз.
\vs Gen 21:28 И поставил Авраам семь агниц из \bibemph{стада} мелкого скота особо.
\vs Gen 21:29 Авимелех же сказал Аврааму: на что здесь сии семь агниц [\bibemph{из стада} овец], которых ты поставил особо?
\vs Gen 21:30 [Авраам] сказал: семь агниц сих возьми от руки моей, чтобы они были мне свидетельством, что я выкопал этот колодезь.
\vs Gen 21:31 Потому и назвал он сие место: Вирсавия, ибо тут оба они клялись
\vs Gen 21:32 и заключили союз в Вирсавии. И встал Авимелех, и [Ахузаф, невестоводитель его, и] Фихол, военачальник его, и возвратились в землю Филистимскую.
\vs Gen 21:33 И насадил [Авраам] при Вирсавии рощу и призвал там имя Господа, Бога вечного.
\vs Gen 21:34 И жил Авраам в земле Филистимской, как странник, дни многие.
\vs Gen 22:1 И было, после сих происшествий Бог искушал Авраама и сказал ему: Авраам! Он сказал: вот я.
\vs Gen 22:2 \bibemph{Бог} сказал: возьми сына твоего, единственного твоего, которого ты любишь, Исаака; и пойди в землю Мориа и там принеси его во всесожжение на одной из гор, о которой Я скажу тебе.
\vs Gen 22:3 Авраам встал рано утром, оседлал осла своего, взял с собою двоих из отроков своих и Исаака, сына своего; наколол дров для всесожжения, и встав пошел на место, о котором сказал ему Бог.
\vs Gen 22:4 На третий день Авраам возвел очи свои, и увидел то место издалека.
\vs Gen 22:5 И сказал Авраам отрокам своим: останьтесь вы здесь с ослом, а я и сын пойдем туда и поклонимся, и возвратимся к вам.
\vs Gen 22:6 И взял Авраам дрова для всесожжения, и возложил на Исаака, сына своего; взял в руки огонь и нож, и пошли оба вместе.
\vs Gen 22:7 И начал Исаак говорить Аврааму, отцу своему, и сказал: отец мой! Он отвечал: вот я, сын мой. Он сказал: вот огонь и дрова, где же агнец для всесожжения?
\vs Gen 22:8 Авраам сказал: Бог усмотрит Себе агнца для всесожжения, сын мой. И шли \bibemph{далее} оба вместе.
\rsbpar\vs Gen 22:9 И пришли на место, о котором сказал ему Бог; и устроил там Авраам жертвенник, разложил дрова и, связав сына своего Исаака, положил его на жертвенник поверх дров.
\vs Gen 22:10 И простер Авраам руку свою и взял нож, чтобы заколоть сына своего.
\vs Gen 22:11 Но Ангел Господень воззвал к нему с неба и сказал: Авраам! Авраам! Он сказал: вот я.
\vs Gen 22:12 \bibemph{Ангел} сказал: не поднимай руки твоей на отрока и не делай над ним ничего, ибо теперь Я знаю, что боишься ты Бога и не пожалел сына твоего, единственного твоего, для Меня.
\vs Gen 22:13 И возвел Авраам очи свои и увидел: и вот, позади овен, запутавшийся в чаще рогами своими. Авраам пошел, взял овна и принес его во всесожжение вместо [Исаака], сына своего.
\vs Gen 22:14 И нарек Авраам имя месту тому: Иегова-ире\fns{Господь усмотрит.}. Посему \bibemph{и} ныне говорится: на горе Иеговы усмотрится.
\vs Gen 22:15 И вторично воззвал к Аврааму Ангел Господень с неба
\vs Gen 22:16 и сказал: Мною клянусь, говорит Господь, что, так как ты сделал сие дело, и не пожалел сына твоего, единственного твоего, [для Меня,]
\vs Gen 22:17 то Я благословляя благословлю тебя и умножая умножу семя твое, как звезды небесные и как песок на берегу моря; и овладеет семя твое городами врагов своих;
\vs Gen 22:18 и благословятся в семени твоем все народы земли за то, что ты послушался гласа Моего.
\vs Gen 22:19 И возвратился Авраам к отрокам своим, и встали и пошли вместе в Вирсавию; и жил Авраам в Вирсавии.
\rsbpar\vs Gen 22:20 После сих происшествий Аврааму возвестили, сказав: вот, и Милка родила Нахору, брату твоему, сынов:
\vs Gen 22:21 Уца, первенца его, Вуза, брата сему, Кемуила, отца Арамова,
\vs Gen 22:22 Кеседа, Хазо, Пилдаша, Идлафа и Вафуила;
\vs Gen 22:23 от Вафуила родилась Ревекка. Восьмерых сих [сынов] родила Милка Нахору, брату Авраамову;
\vs Gen 22:24 и наложница его, именем Реума, также родила Теваха, Гахама, Тахаша и Мааху.
\vs Gen 23:1 Жизни Сарриной было сто двадцать семь лет: \bibemph{вот} лета жизни Сарриной;
\vs Gen 23:2 и умерла Сарра в Кириаф-Арбе, [который на долине,] что \bibemph{ныне} Хеврон, в земле Ханаанской. И пришел Авраам рыдать по Сарре и оплакивать ее.
\vs Gen 23:3 И отошел Авраам от умершей своей, и говорил сынам Хетовым, и сказал:
\vs Gen 23:4 я у вас пришлец и поселенец; дайте мне в собственность \bibemph{место для} гроба между вами, чтобы мне умершую мою схоронить от глаз моих.
\vs Gen 23:5 Сыны Хета отвечали Аврааму и сказали ему:
\vs Gen 23:6 послушай нас, господин наш; ты князь Божий посреди нас; в лучшем из наших погребальных мест похорони умершую твою; никто из нас не откажет тебе в погребальном месте, для погребения [на нем] умершей твоей.
\vs Gen 23:7 Авраам встал и поклонился народу земли той, сынам Хетовым;
\vs Gen 23:8 и говорил им [Авраам] и сказал: если вы согласны, чтобы я похоронил умершую мою, то послушайте меня, попросите за меня Ефрона, сына Цохарова,
\vs Gen 23:9 чтобы он отдал мне пещеру Махпелу, которая у него на конце поля его, чтобы за довольную цену отдал ее мне посреди вас, в собственность для погребения.
\vs Gen 23:10 Ефрон же сидел посреди сынов Хетовых; и отвечал Ефрон Хеттеянин Аврааму вслух сынов Хета, всех входящих во врата города его, и сказал:
\vs Gen 23:11 нет, господин мой, послушай меня: я даю тебе поле и пещеру, которая на нем, даю тебе, пред очами сынов народа моего дарю тебе ее, похорони умершую твою.
\vs Gen 23:12 Авраам поклонился пред народом земли той
\vs Gen 23:13 и говорил Ефрону вслух [всего] народа земли той и сказал: если послушаешь, я даю тебе за поле серебро; возьми у меня, и я похороню там умершую мою.
\vs Gen 23:14 Ефрон отвечал Аврааму и сказал ему:
\vs Gen 23:15 господин мой! послушай меня: земля \bibemph{стоит} четыреста сиклей серебра; для меня и для тебя что это? похорони умершую твою.
\vs Gen 23:16 Авраам выслушал Ефрона; и отвесил Авраам Ефрону серебра, сколько он объявил вслух сынов Хетовых, четыреста сиклей серебра, какое ходит у купцов.
\vs Gen 23:17 И стало поле Ефроново, которое при Махпеле, против Мамре, поле и пещера, которая на нем, и все деревья, которые на поле, во всех пределах его вокруг,
\vs Gen 23:18 владением Авраамовым пред очами сынов Хета, всех входящих во врата города его.
\rsbpar\vs Gen 23:19 После сего Авраам похоронил Сарру, жену свою, в пещере поля в Махпеле, против Мамре, что \bibemph{ныне} Хеврон, в земле Ханаанской.
\vs Gen 23:20 Так достались Аврааму от сынов Хетовых поле и пещера, которая на нем, в собственность для погребения.
\vs Gen 24:1 Авраам был уже стар и в летах преклонных. Господь благословил Авраама всем.
\vs Gen 24:2 И сказал Авраам рабу своему, старшему в доме его, управлявшему всем, что у него было: положи руку твою под стегно мое
\vs Gen 24:3 и клянись мне Господом, Богом неба и Богом земли, что ты не возьмешь сыну моему [Исааку] жены из дочерей Хананеев, среди которых я живу,
\vs Gen 24:4 но пойдешь в землю мою, на родину мою [и к племени моему], и возьмешь [оттуда] жену сыну моему Исааку.
\vs Gen 24:5 Раб сказал ему: может быть, не захочет женщина идти со мною в эту землю, должен ли я возвратить сына твоего в землю, из которой ты вышел?
\vs Gen 24:6 Авраам сказал ему: берегись, не возвращай сына моего туда;
\vs Gen 24:7 Господь, Бог неба [и Бог земли], Который взял меня из дома отца моего и из земли рождения моего, Который говорил мне и Который клялся мне, говоря: [тебе и] потомству твоему дам сию землю,~--- Он пошлет Ангела Своего пред тобою, и ты возьмешь жену сыну моему [Исааку] оттуда;
\vs Gen 24:8 если же не захочет женщина идти с тобою [в землю сию], ты будешь свободен от сей клятвы моей; только сына моего не возвращай туда.
\vs Gen 24:9 И положил раб руку свою под стегно Авраама, господина своего, и клялся ему в сем.
\vs Gen 24:10 И взял раб из верблюдов господина своего десять верблюдов и пошел. В руках у него были также всякие сокровища господина его. Он встал и пошел в Месопотамию, в город Нахора,
\vs Gen 24:11 и остановил верблюдов вне города, у колодезя воды, под вечер, в то время, когда выходят женщины черпать [воду],
\vs Gen 24:12 и сказал: Господи, Боже господина моего Авраама! пошли \bibemph{ее} сегодня навстречу мне и сотвори милость с господином моим Авраамом;
\vs Gen 24:13 вот, я стою у источника воды, и дочери жителей города выходят черпать воду;
\vs Gen 24:14 и девица, которой я скажу: наклони кувшин твой, я напьюсь, и которая скажет [мне]: пей, я и верблюдам твоим дам пить, [пока не напьются,]~--- вот та, которую Ты назначил рабу Твоему Исааку; и по сему узн\acc{а}ю я, что Ты творишь милость с господином моим [Авраамом].
\vs Gen 24:15 Еще не перестал он говорить [в уме своем], и вот, вышла Ревекка, которая родилась от Вафуила, сына Милки, жены Нахора, брата Авраамова, и кувшин ее на плече ее;
\vs Gen 24:16 девица \bibemph{была} прекрасна видом, дева, которой не познал муж. Она сошла к источнику, наполнила кувшин свой и пошла вверх.
\vs Gen 24:17 И побежал раб навстречу ей и сказал: дай мне испить немного воды из кувшина твоего.
\vs Gen 24:18 Она сказала: пей, господин мой. И тотчас спустила кувшин свой на руку свою и напоила его.
\vs Gen 24:19 И, когда напоила его, сказала: я стану черпать и для верблюдов твоих, пока не напьются [все].
\vs Gen 24:20 И тотчас вылила воду из кувшина своего в поило и побежала опять к колодезю почерпнуть [воды], и начерпала для всех верблюдов его.
\vs Gen 24:21 Человек тот смотрел на нее с изумлением в молчании, желая уразуметь, благословил ли Господь путь его, или нет.
\vs Gen 24:22 Когда верблюды перестали пить, тогда человек тот взял золотую серьгу, весом полсикля, и два запястья на руки ей, весом в десять \bibemph{сиклей} золота;
\vs Gen 24:23 [и спросил ее] и сказал: чья ты дочь? скажи мне, есть ли в доме отца твоего место нам ночевать?
\vs Gen 24:24 Она сказала ему: я дочь Вафуила, сына Милки, которого она родила Нахору.
\vs Gen 24:25 И еще сказала ему: у нас много соломы и корму, и \bibemph{есть} место для ночлега.
\vs Gen 24:26 И преклонился человек тот и поклонился Господу,
\vs Gen 24:27 и сказал: благословен Господь Бог господина моего Авраама, Который не оставил господина моего милостью Своею и истиною Своею! Господь прямым путем привел меня к дому брата господина моего.
\vs Gen 24:28 Девица побежала и рассказала об этом в доме матери своей.
\vs Gen 24:29 У Ревекки был брат, именем Лаван. Лаван выбежал к тому человеку, к источнику.
\vs Gen 24:30 И когда он увидел серьгу и запястья на руках у сестры своей и услышал слова Ревекки, сестры своей, которая говорила: так говорил со мною этот человек,~--- то пришел к человеку, и вот, он стоит при верблюдах у источника;
\vs Gen 24:31 и сказал [ему]: войди, благословенный Господом; зачем ты стоишь вне? я приготовил дом и место для верблюдов.
\vs Gen 24:32 И вошел человек. \bibemph{Лаван} расседлал верблюдов и дал соломы и корму верблюдам, и воды умыть ноги ему и людям, которые были с ним;
\vs Gen 24:33 и предложена была ему пища; но он сказал: не стану есть, доколе не скажу дела своего. И сказали: говори.
\vs Gen 24:34 Он сказал: я раб Авраамов;
\vs Gen 24:35 Господь весьма благословил господина моего, и он сделался великим: Он дал ему овец и волов, серебро и золото, рабов и рабынь, верблюдов и ослов;
\vs Gen 24:36 Сарра, жена господина моего, уже состарившись, родила господину моему [одного] сына, которому он отдал все, что у него;
\vs Gen 24:37 и взял с меня клятву господин мой, сказав: не бери жены сыну моему из дочерей Хананеев, в земле которых я живу,
\vs Gen 24:38 а пойди в дом отца моего и к родственникам моим, и возьмешь [оттуда] жену сыну моему.
\vs Gen 24:39 Я сказал господину моему: может быть, не пойдет женщина со мною.
\vs Gen 24:40 Он сказал мне: Господь [Бог], пред лицем Которого я хожу, пошлет с тобою Ангела Своего и благоустроит путь твой, и возьмешь жену сыну моему из родных моих и из дома отца моего;
\vs Gen 24:41 тогда будешь ты свободен от клятвы моей, когда сходишь к родственникам моим; и если они не дадут тебе, то будешь свободен от клятвы моей.
\vs Gen 24:42 И пришел я ныне к источнику, и сказал: Господи, Боже господина моего Авраама! Если Ты благоустроишь путь, который я совершаю,
\vs Gen 24:43 то вот, я стою у источника воды, [и дочери жителей города выходят черпать воду,] и девица, которая выйдет почерпать, и которой я скажу: дай мне испить немного из кувшина твоего,
\vs Gen 24:44 и которая скажет мне: и ты пей, и верблюдам твоим я начерпаю,~--- вот жена, которую Господь назначил сыну господина моего [рабу Своему Исааку; и по сему узнаю я, что Ты творишь милость с господином моим Авраамом].
\vs Gen 24:45 Еще не перестал я говорить в уме моем, и вот вышла Ревекка, и кувшин ее на плече ее, и сошла к источнику и почерпнула [воды]; и я сказал ей: напой меня.
\vs Gen 24:46 Она тотчас спустила с себя кувшин свой [на руку свою] и сказала: пей, и верблюдов твоих я напою. И я пил, и верблюдов [моих] она напоила.
\vs Gen 24:47 Я спросил ее и сказал: чья ты дочь? [скажи мне]. Она сказала: дочь Вафуила, сына Нахорова, которого родила ему Милка. И дал я серьги ей и запястья на руки ее.
\vs Gen 24:48 И преклонился я и поклонился Господу, и благословил Господа, Бога господина моего Авраама, Который прямым путем привел меня, чтобы взять дочь брата господина моего за сына его.
\vs Gen 24:49 И ныне скажите мне: намерены ли вы оказать милость и правду господину моему или нет? скажите мне, и я обращусь направо, или налево.
\vs Gen 24:50 И отвечали Лаван и Вафуил и сказали: от Господа пришло это дело; мы не можем сказать тебе вопреки ни худого, ни доброго;
\vs Gen 24:51 вот Ревекка пред тобою; возьми [ее] и пойди; пусть будет она женою сыну господина твоего, как сказал Господь.
\vs Gen 24:52 Когда раб Авраамов услышал слова их, то поклонился Господу до земли.
\vs Gen 24:53 И вынул раб серебряные вещи и золотые вещи и одежды и дал Ревекке; также и брату ее и матери ее дал богатые подарки.
\vs Gen 24:54 И ели и пили он и люди, бывшие с ним, и переночевали. Когда же встали поутру, то он сказал: отпустите меня [и я пойду] к господину моему.
\vs Gen 24:55 Но брат ее и мать ее сказали: пусть побудет с нами девица дней хотя десять, потом пойдешь.
\vs Gen 24:56 Он сказал им: не удерживайте меня, ибо Господь благоустроил путь мой; отпустите меня, и я пойду к господину моему.
\vs Gen 24:57 Они сказали: призовем девицу и спросим, что она скажет.
\vs Gen 24:58 И призвали Ревекку и сказали ей: пойдешь ли с этим человеком? Она сказала: пойду.
\vs Gen 24:59 И отпустили Ревекку, сестру свою, и кормилицу ее, и раба Авраамова, и людей его.
\vs Gen 24:60 И благословили Ревекку и сказали ей: сестра наша! да родятся от тебя тысячи тысяч, и да владеет потомство твое жилищами врагов твоих!
\vs Gen 24:61 И встала Ревекка и служанки ее, и сели на верблюдов, и поехали за тем человеком. И раб взял Ревекку и пошел.
\vs Gen 24:62 А Исаак пришел из Беэр-лахай-рои, ибо жил он в земле полуденной.
\vs Gen 24:63 При наступлении вечера Исаак вышел в поле поразмыслить, и возвел очи свои, и увидел: вот, идут верблюды.
\vs Gen 24:64 Ревекка взглянула, и увидела Исаака, и спустилась с верблюда.
\vs Gen 24:65 И сказала рабу: кто этот человек, который идет по полю навстречу нам? Раб сказал: это господин мой. И она взяла покрывало и покрылась.
\vs Gen 24:66 Раб же сказал Исааку все, что сделал.
\vs Gen 24:67 И ввел ее Исаак в шатер Сарры, матери своей, и взял Ревекку, и она сделалась ему женою, и он возлюбил ее; и утешился Исаак в \bibemph{печали} по [Сарре,] матери своей.
\vs Gen 25:1 И взял Авраам еще жену, именем Хеттуру.
\vs Gen 25:2 Она родила ему Зимрана, Иокшана, Медана, Мадиана, Ишбака и Шуаха.
\vs Gen 25:3 Иокшан родил Шеву, [Фемана] и Дедана. Сыны Дедана были: [Рагуил, Навдеил,] Ашурим, Летушим и Леюмим.
\vs Gen 25:4 Сыны Мадиана: Ефа, Ефер, Ханох, Авида и Елдага. Все сии сыны Хеттуры.
\vs Gen 25:5 И отдал Авраам все, что было у него, Исааку [сыну своему],
\vs Gen 25:6 а сынам наложниц, которые были у Авраама, дал Авраам подарки и отослал их от Исаака, сына своего, еще при жизни своей, на восток, в землю восточную.
\vs Gen 25:7 Дней жизни Авраамовой, которые он прожил, было сто семьдесят пять лет;
\vs Gen 25:8 и скончался Авраам, и умер в старости доброй, престарелый и насыщенный [жизнью], и приложился к народу своему.
\vs Gen 25:9 И погребли его Исаак и Измаил, сыновья его, в пещере Махпеле, на поле Ефрона, сына Цохара, Хеттеянина, которое против Мамре,
\vs Gen 25:10 на поле [и в пещере], которые Авраам приобрел от сынов Хетовых. Там погребены Авраам и Сарра, жена его.
\vs Gen 25:11 По смерти Авраама Бог благословил Исаака, сына его. Исаак жил при Беэр-лахай-рои.
\vs Gen 25:12 Вот родословие Измаила, сына Авраамова, которого родила Аврааму Агарь Египтянка, служанка Саррина;
\vs Gen 25:13 и вот имена сынов Измаиловых, имена их по родословию их: первенец Измаилов Наваиоф, \bibemph{за ним} Кедар, Адбеел, Мивсам,
\vs Gen 25:14 Мишма, Дума, Масса,
\vs Gen 25:15 Хадад, Фема, Иетур, Нафиш и Кедма.
\vs Gen 25:16 Сии суть сыны Измаиловы, и сии имена их, в селениях их, в кочевьях их. \bibemph{Это} двенадцать князей племен их.
\vs Gen 25:17 Лет же жизни Измаиловой было сто тридцать семь лет; и скончался он, и умер, и приложился к народу своему.
\vs Gen 25:18 Они жили от Хавилы до Сура, что пред Египтом, как идешь к Ассирии. Они поселились пред лицем всех братьев своих.
\rsbpar\vs Gen 25:19 Вот родословие Исаака, сына Авраамова. Авраам родил Исаака.
\vs Gen 25:20 Исаак был сорока лет, когда он взял себе в жену Ревекку, дочь Вафуила Арамеянина из Месопотамии, сестру Лавана Арамеянина.
\vs Gen 25:21 И молился Исаак Господу о [Ревекке] жене своей, потому что она была неплодна; и Господь услышал его, и зачала Ревекка, жена его.
\vs Gen 25:22 Сыновья в утробе ее стали биться, и она сказала: если так будет, то для чего мне это? И пошла вопросить Господа.
\vs Gen 25:23 Господь сказал ей: два племени во чреве твоем, и два различных народа произойдут из утробы твоей; один народ сделается сильнее другого, и больший будет служить меньшему.
\vs Gen 25:24 И настало время родить ей: и вот близнецы в утробе ее.
\vs Gen 25:25 Первый вышел красный, весь, как кожа, косматый; и нарекли ему имя Исав.
\vs Gen 25:26 Потом вышел брат его, держась рукою своею за пяту Исава; и наречено ему имя Иаков. Исаак же был шестидесяти лет, когда они родились [от Ревекки].
\rsbpar\vs Gen 25:27 Дети выросли, и стал Исав человеком искусным в звероловстве, человеком полей; а Иаков человеком кротким, живущим в шатрах.
\vs Gen 25:28 Исаак любил Исава, потому что дичь его была по вкусу его, а Ревекка любила Иакова.
\vs Gen 25:29 И сварил Иаков кушанье; а Исав пришел с поля усталый.
\vs Gen 25:30 И сказал Исав Иакову: дай мне поесть красного, красного этого, ибо я устал. От сего дано ему прозвание: Едом.
\vs Gen 25:31 Но Иаков сказал [Исаву]: продай мне теперь же свое первородство.
\vs Gen 25:32 Исав сказал: вот, я умираю, что мне в этом первородстве?
\vs Gen 25:33 Иаков сказал [ему]: поклянись мне теперь же. Он поклялся ему, и продал [Исав] первородство свое Иакову.
\vs Gen 25:34 И дал Иаков Исаву хлеба и кушанья из чечевицы; и он ел и пил, и встал и пошел; и пренебрег Исав первородство.
\vs Gen 26:1 Был голод в земле, сверх прежнего голода, который был во дни Авраама; и пошел Исаак к Авимелеху, царю Филистимскому, в Герар.
\vs Gen 26:2 Господь явился ему и сказал: не ходи в Египет; живи в земле, о которой Я скажу тебе,
\vs Gen 26:3 странствуй по сей земле, и Я буду с тобою и благословлю тебя, ибо тебе и потомству твоему дам все земли сии и исполню клятву [Мою], которою Я клялся Аврааму, отцу твоему;
\vs Gen 26:4 умножу потомство твое, как звезды небесные, и дам потомству твоему все земли сии; благословятся в семени твоем все народы земные,
\vs Gen 26:5 за то, что Авраам [отец твой] послушался гласа Моего и соблюдал, что Мною \bibemph{заповедано} было соблюдать: повеления Мои, уставы Мои и законы Мои.
\vs Gen 26:6 Исаак поселился в Гераре.
\vs Gen 26:7 Жители места того спросили о [Ревекке] жене его, и он сказал: это сестра моя; потому что боялся сказать: жена моя, чтобы не убили меня, \bibemph{думал он}, жители места сего за Ревекку, потому что она прекрасна видом.
\vs Gen 26:8 Но когда уже много времени он там прожил, Авимелех, царь Филистимский, посмотрев в окно, увидел, что Исаак играет с Ревеккою, женою своею.
\vs Gen 26:9 И призвал Авимелех Исаака и сказал: вот, это жена твоя; как же ты сказал: она сестра моя? Исаак сказал ему: потому что я думал, не умереть бы мне ради ее.
\vs Gen 26:10 Но Авимелех сказал [ему]: что это ты сделал с нами? едва один из народа [моего] не совокупился с женою твоею, и ты ввел бы нас в грех.
\vs Gen 26:11 И дал Авимелех повеление всему народу, сказав: кто прикоснется к сему человеку и к жене его, тот предан будет смерти.
\vs Gen 26:12 И сеял Исаак в земле той и получил в тот год ячменя во сто крат: так благословил его Господь.
\vs Gen 26:13 И стал великим человек сей и возвеличивался больше и больше до того, что стал весьма великим.
\vs Gen 26:14 У него были стада мелкого и стада крупного скота и множество пахотных полей, и Филистимляне стали завидовать ему.
\vs Gen 26:15 И все колодези, которые выкопали рабы отца его при жизни отца его Авраама, Филистимляне завалили и засыпали землею.
\vs Gen 26:16 И Авимелех сказал Исааку: удались от нас, ибо ты сделался гораздо сильнее нас.
\vs Gen 26:17 И Исаак удалился оттуда, и расположился шатрами в долине Герарской, и поселился там.
\vs Gen 26:18 И вновь выкопал Исаак колодези воды, которые выкопаны были во дни Авраама, отца его, и которые завалили Филистимляне по смерти Авраама [отца его]; и назвал их теми же именами, которыми назвал их [Авраам,] отец его.
\vs Gen 26:19 И копали рабы Исааковы в долине [Герарской] и нашли там колодезь воды живой.
\vs Gen 26:20 И спорили пастухи Герарские с пастухами Исаака, говоря: наша вода. И он нарек колодезю имя: Есек, потому что спорили с ним.
\vs Gen 26:21 [Когда двинулся оттуда Исаак,] выкопали другой колодезь; спорили также и о нем; и он нарек ему имя: Ситна.
\vs Gen 26:22 И он двинулся отсюда и выкопал иной колодезь, о котором уже не спорили, и нарек ему имя: Реховоф, ибо, сказал он, теперь Господь дал нам пространное место, и мы размножимся на земле.
\rsbpar\vs Gen 26:23 Оттуда перешел он в Вирсавию.
\vs Gen 26:24 И в ту ночь явился ему Господь и сказал: Я Бог Авраама, отца твоего; не бойся, ибо Я с тобою; и благословлю тебя и умножу потомство твое, ради [отца твоего] Авраама, раба Моего.
\vs Gen 26:25 И он устроил там жертвенник и призвал имя Господа. И раскинул там шатер свой, и выкопали там рабы Исааковы колодезь, [в долине Герарской].
\vs Gen 26:26 Пришел к нему из Герара Авимелех и Ахузаф, друг его, и Фихол, военачальник его.
\vs Gen 26:27 Исаак сказал им: для чего вы пришли ко мне, когда вы возненавидели меня и выслали меня от себя?
\vs Gen 26:28 Они сказали: мы ясно увидели, что Господь с тобою, и потому мы сказали: поставим между нами и тобою клятву и заключим с тобою союз,
\vs Gen 26:29 чтобы ты не делал нам зла, как и мы не коснулись до тебя, а делали тебе одно доброе и отпустили тебя с миром; теперь ты благословен Господом.
\vs Gen 26:30 Он сделал им пиршество, и они ели и пили.
\vs Gen 26:31 И встав рано утром, поклялись друг другу; и отпустил их Исаак, и они пошли от него с миром.
\vs Gen 26:32 В тот же день пришли рабы Исааковы и известили его о колодезе, который копали они, и сказали ему: мы нашли воду.
\vs Gen 26:33 И он назвал его: Шива. Посему имя городу тому Беэршива [Вирсавия] до сего дня.
\vs Gen 26:34 И был Исав сорока лет, и взял себе в жены Иегудифу, дочь Беэра Хеттеянина, и Васемафу, дочь Елона Хеттеянина;
\vs Gen 26:35 и они были в тягость Исааку и Ревекке.
\vs Gen 27:1 Когда Исаак состарился и притупилось зрение глаз его, он призвал старшего сына своего Исава и сказал ему: сын мой! Тот сказал ему: вот я.
\vs Gen 27:2 [Исаак] сказал: вот, я состарился; не знаю дня смерти моей;
\vs Gen 27:3 возьми теперь орудия твои, колчан твой и лук твой, пойди в поле, и налови мне дичи,
\vs Gen 27:4 и приготовь мне кушанье, какое я люблю, и принеси мне есть, чтобы благословила тебя душа моя, прежде нежели я умру.
\vs Gen 27:5 Ревекка слышала, когда Исаак говорил сыну своему Исаву. И пошел Исав в поле достать и принести дичи;
\vs Gen 27:6 а Ревекка сказала [меньшему] сыну своему Иакову: вот, я слышала, как отец твой говорил брату твоему Исаву:
\vs Gen 27:7 принеси мне дичи и приготовь мне кушанье; я поем и благословлю тебя пред лицем Господним, пред смертью моею.
\vs Gen 27:8 Теперь, сын мой, послушайся слов моих в том, что я прикажу тебе:
\vs Gen 27:9 пойди в \bibemph{стадо} и возьми мне оттуда два козленка [молодых] хороших, и я приготовлю из них отцу твоему кушанье, какое он любит,
\vs Gen 27:10 а ты принесешь отцу твоему, и он поест, чтобы благословить тебя пред смертью своею.
\vs Gen 27:11 Иаков сказал Ревекке, матери своей: Исав, брат мой, человек косматый, а я человек гладкий;
\vs Gen 27:12 может статься, ощупает меня отец мой, и я буду в глазах его обманщиком и наведу на себя проклятие, а не благословение.
\vs Gen 27:13 Мать его сказала ему: на мне пусть будет проклятие твое, сын мой, только послушайся слов моих и пойди, принеси мне.
\vs Gen 27:14 Он пошел, и взял, и принес матери своей; и мать его сделала кушанье, какое любил отец его.
\vs Gen 27:15 И взяла Ревекка богатую одежду старшего сына своего Исава, бывшую у ней в доме, и одела [в нее] младшего сына своего Иакова;
\vs Gen 27:16 а руки его и гладкую шею его обложила кожею козлят;
\vs Gen 27:17 и дала кушанье и хлеб, которые она приготовила, в руки Иакову, сыну своему.
\vs Gen 27:18 Он вошел к отцу своему и сказал: отец мой! Тот сказал: вот я; кто ты, сын мой?
\vs Gen 27:19 Иаков сказал отцу своему: я Исав, первенец твой; я сделал, как ты сказал мне; встань, сядь и поешь дичи моей, чтобы благословила меня душа твоя.
\vs Gen 27:20 И сказал Исаак сыну своему: что так скоро нашел ты, сын мой? Он сказал: потому что Господь Бог твой послал мне навстречу.
\vs Gen 27:21 И сказал Исаак Иакову: подойди [ко мне], я ощупаю тебя, сын мой, ты ли сын мой Исав, или нет?
\vs Gen 27:22 Иаков подошел к Исааку, отцу своему, и он ощупал его и сказал: голос, голос Иакова; а руки, руки Исавовы.
\vs Gen 27:23 И не узнал его, потому что руки его были, как руки Исава, брата его, косматые; и благословил его
\vs Gen 27:24 и сказал: ты ли сын мой Исав? Он отвечал: я.
\vs Gen 27:25 \bibemph{Исаак} сказал: подай мне, я поем дичи сына моего, чтобы благословила тебя душа моя. \bibemph{Иаков} подал ему, и он ел; принес ему и вина, и он пил.
\vs Gen 27:26 Исаак, отец его, сказал ему: подойди [ко мне], поцелуй меня, сын мой.
\vs Gen 27:27 Он подошел и поцеловал его. И ощутил \bibemph{Исаак} запах от одежды его и благословил его и сказал: вот, запах от сына моего, как запах от поля [полного], которое благословил Господь;
\vs Gen 27:28 да даст тебе Бог от росы небесной и от тука земли, и множество хлеба и вина;
\vs Gen 27:29 да послужат тебе народы, и да поклонятся тебе племена; будь господином над братьями твоими, и да поклонятся тебе сыны матери твоей; проклинающие тебя~--- прокляты; благословляющие тебя~--- благословенны!
\rsbpar\vs Gen 27:30 Как скоро совершил Исаак благословение над Иаковом [сыном своим], и как только вышел Иаков от лица Исаака, отца своего, Исав, брат его, пришел с ловли своей.
\vs Gen 27:31 Приготовил и он кушанье, и принес отцу своему, и сказал отцу своему: встань, отец мой, и поешь дичи сына твоего, чтобы благословила меня душа твоя.
\vs Gen 27:32 Исаак же, отец его, сказал ему: кто ты? Он сказал: я сын твой, первенец твой, Исав.
\vs Gen 27:33 И вострепетал Исаак весьма великим трепетом, и сказал: кто ж это, который достал [мне] дичи и принес мне, и я ел от всего, прежде нежели ты пришел, и я благословил его? он и будет благословен.
\vs Gen 27:34 Исав, выслушав слова отца своего [Исаака], поднял громкий и весьма горький вопль и сказал отцу своему: отец мой! благослови и меня.
\vs Gen 27:35 Но он сказал [ему]: брат твой пришел с хитростью и взял благословение твое.
\vs Gen 27:36 И сказал [Исав]: не потому ли дано ему имя: Иаков, что он запнул меня уже два раза? Он взял первородство мое, и вот, теперь взял благословение мое. И \bibemph{еще} сказал [Исав отцу своему]: неужели ты не оставил [и] мне благословения?
\vs Gen 27:37 Исаак отвечал Исаву: вот, я поставил его господином над тобою и всех братьев его отдал ему в рабы; одарил его хлебом и вином; что же я сделаю для тебя, сын мой?
\vs Gen 27:38 Но Исав сказал отцу своему: неужели, отец мой, одно у тебя благословение? благослови и меня, отец мой! И [как Исаак молчал,] возвысил Исав голос свой и заплакал.
\vs Gen 27:39 И отвечал Исаак, отец его, и сказал ему: вот, от тука земли будет обитание твое и от росы небесной свыше;
\vs Gen 27:40 и ты будешь жить мечом твоим и будешь служить брату твоему; будет же \bibemph{время}, когда воспротивишься и свергнешь иго его с выи твоей.
\vs Gen 27:41 И возненавидел Исав Иакова за благословение, которым благословил его отец его; и сказал Исав в сердце своем: приближаются дни плача по отце моем, и я убью Иакова, брата моего.
\vs Gen 27:42 И пересказаны были Ревекке слова Исава, старшего сына ее; и она послала, и призвала младшего сына своего Иакова, и сказала ему: вот, Исав, брат твой, грозит убить тебя;
\vs Gen 27:43 и теперь, сын мой, послушайся слов моих, встань, беги [в Месопотамию] к Лавану, брату моему, в Харран,
\vs Gen 27:44 и поживи у него несколько времени, пока утолится ярость брата твоего,
\vs Gen 27:45 пока утолится гнев брата твоего на тебя, и он позабудет, что ты сделал ему: тогда я пошлю и возьму тебя оттуда; для чего мне в один день лишиться обоих вас?
\vs Gen 27:46 И сказала Ревекка Исааку: я жизни не рада от дочерей Хеттейских; если Иаков возьмет жену из дочерей Хеттейских, каковы эти, из дочерей этой земли, то к чему мне и жизнь?
\vs Gen 28:1 И призвал Исаак Иакова и благословил его, и заповедал ему и сказал: не бери себе жены из дочерей Ханаанских;
\vs Gen 28:2 встань, пойди в Месопотамию, в дом Вафуила, отца матери твоей, и возьми себе жену оттуда, из дочерей Лавана, брата матери твоей;
\vs Gen 28:3 Бог же Всемогущий да благословит тебя, да расплодит тебя и да размножит тебя, и да будет от тебя множество народов,
\vs Gen 28:4 и да даст тебе благословение Авраама [отца моего], тебе и потомству твоему с тобою, чтобы тебе наследовать землю странствования твоего, которую Бог дал Аврааму!
\vs Gen 28:5 И отпустил Исаак Иакова, и он пошел в Месопотамию к Лавану, сыну Вафуила Арамеянина, к брату Ревекки, матери Иакова и Исава.
\rsbpar\vs Gen 28:6 Исав увидел, что Исаак благословил Иакова и благословляя послал его в Месопотамию, взять себе жену оттуда, и заповедал ему, сказав: не бери жены из дочерей Ханаанских;
\vs Gen 28:7 и что Иаков послушался отца своего и матери своей и пошел в Месопотамию.
\vs Gen 28:8 И увидел Исав, что дочери Ханаанские не угодны Исааку, отцу его;
\vs Gen 28:9 и пошел Исав к Измаилу и взял себе жену Махалафу, дочь Измаила, сына Авраамова, сестру Наваиофову, сверх \bibemph{других} жен своих.
\rsbpar\vs Gen 28:10 Иаков же вышел из Вирсавии и пошел в Харран,
\vs Gen 28:11 и пришел на \bibemph{одно} место, и \bibemph{остался} там ночевать, потому что зашло солнце. И взял \bibemph{один} из камней того места, и положил себе изголовьем, и лег на том месте.
\vs Gen 28:12 И увидел во сне: вот, лестница стоит на земле, а верх ее касается неба; и вот, Ангелы Божии восходят и нисходят по ней.
\vs Gen 28:13 И вот, Господь стоит на ней и говорит: Я Господь, Бог Авраама, отца твоего, и Бог Исаака; [не бойся]. Землю, на которой ты лежишь, Я дам тебе и потомству твоему;
\vs Gen 28:14 и будет потомство твое, как песок земной; и распространишься к морю и к востоку, и к северу и к полудню; и благословятся в тебе и в семени твоем все племена земные;
\vs Gen 28:15 и вот Я с тобою, и сохраню тебя везде, куда ты ни пойдешь; и возвращу тебя в сию землю, ибо Я не оставлю тебя, доколе не исполню того, что Я сказал тебе.
\vs Gen 28:16 Иаков пробудился от сна своего и сказал: истинно Господь присутствует на месте сем; а я не знал!
\vs Gen 28:17 И убоялся и сказал: как страшно сие место! это не иное что, как дом Божий, это врата небесные.
\vs Gen 28:18 И встал Иаков рано утром, и взял камень, который он положил себе изголовьем, и поставил его памятником, и возлил елей на верх его.
\vs Gen 28:19 И нарек [Иаков] имя месту тому: Вефиль\fns{Дом Божий.}, а прежнее имя того города было: Луз.
\vs Gen 28:20 И положил Иаков обет, сказав: если [Господь] Бог будет со мною и сохранит меня в пути сем, в который я иду, и даст мне хлеб есть и одежду одеться,
\vs Gen 28:21 и я в мире возвращусь в дом отца моего, и будет Господь моим Богом,~---
\vs Gen 28:22 то этот камень, который я поставил памятником, будет [у меня] домом Божиим; и из всего, что Ты, \bibemph{Боже}, даруешь мне, я дам Тебе десятую часть.
\vs Gen 29:1 И встал Иаков и пошел в землю сынов востока [к Лавану, сыну Вафуила Арамеянина, к брату Ревекки, матери Иакова и Исава].
\vs Gen 29:2 И увидел: вот, на поле колодезь, и там три стада мелкого скота, лежавшие около него, потому что из того колодезя поили стада. Над устьем колодезя был большой камень.
\vs Gen 29:3 Когда собирались туда все стада, отваливали камень от устья колодезя и поили овец; потом опять клали камень на свое место, на устье колодезя.
\vs Gen 29:4 Иаков сказал им [пастухам]: братья мои! откуда вы? Они сказали: мы из Харрана.
\vs Gen 29:5 Он сказал им: знаете ли вы Лавана, сына Нахорова? Они сказали: знаем.
\vs Gen 29:6 Он еще сказал им: здравствует ли он? Они сказали: здравствует; и вот, Рахиль, дочь его, идет с овцами.
\vs Gen 29:7 И сказал [Иаков]: вот, дня еще много; не время собирать скот; напойте овец и пойдите, пасите.
\vs Gen 29:8 Они сказали: не можем, пока не соберутся все стада, и не отвалят камня от устья колодезя; тогда будем мы поить овец.
\vs Gen 29:9 Еще он говорил с ними, как пришла Рахиль [дочь Лавана] с мелким скотом отца своего, потому что она пасла [мелкий скот отца своего].
\vs Gen 29:10 Когда Иаков увидел Рахиль, дочь Лавана, брата матери своей, и овец Лавана, брата матери своей, то подошел Иаков, отвалил камень от устья колодезя и напоил овец Лавана, брата матери своей.
\vs Gen 29:11 И поцеловал Иаков Рахиль и возвысил голос свой и заплакал.
\vs Gen 29:12 И сказал Иаков Рахили, что он родственник отцу ее и что он сын Ревеккин. А она побежала и сказала отцу своему [всё сие].
\vs Gen 29:13 Лаван, услышав о Иакове, сыне сестры своей, выбежал ему навстречу, обнял его и поцеловал его, и ввел его в дом свой; и он рассказал Лавану всё сие.
\vs Gen 29:14 Лаван же сказал ему: подлинно ты кость моя и плоть моя. И жил у него \bibemph{Иаков} целый месяц.
\rsbpar\vs Gen 29:15 И Лаван сказал Иакову: неужели ты даром будешь служить мне, потому что ты родственник? скажи мне, что заплатить тебе?
\vs Gen 29:16 У Лавана же было две дочери; имя старшей: Лия; имя младшей: Рахиль.
\vs Gen 29:17 Лия была слаба глазами, а Рахиль была красива станом и красива лицем.
\vs Gen 29:18 Иаков полюбил Рахиль и сказал: я буду служить тебе семь лет за Рахиль, младшую дочь твою.
\vs Gen 29:19 Лаван сказал [ему]: лучше отдать мне ее за тебя, нежели отдать ее за другого кого; живи у меня.
\vs Gen 29:20 И служил Иаков за Рахиль семь лет; и они показались ему за несколько дней, потому что он любил ее.
\vs Gen 29:21 И сказал Иаков Лавану: дай жену мою, потому что мне уже исполнилось время, чтобы войти к ней.
\vs Gen 29:22 Лаван созвал всех людей того места и сделал пир.
\vs Gen 29:23 Вечером же взял [Лаван] дочь свою Лию и ввел ее к нему; и вошел к ней [Иаков].
\vs Gen 29:24 И дал Лаван служанку свою Зелфу в служанки дочери своей Лии.
\vs Gen 29:25 Утром же оказалось, что это Лия. И [Иаков] сказал Лавану: что это сделал ты со мною? не за Рахиль ли я служил у тебя? зачем ты обманул меня?
\vs Gen 29:26 Лаван сказал: в нашем месте так не делают, чтобы младшую выдать прежде старшей;
\vs Gen 29:27 окончи неделю этой, потом дадим тебе и ту за службу, которую ты будешь служить у меня еще семь лет других.
\vs Gen 29:28 Иаков так и сделал и окончил неделю этой. И [Лаван] дал Рахиль, дочь свою, ему в жену.
\vs Gen 29:29 И дал Лаван служанку свою Валлу в служанки дочери своей Рахили.
\vs Gen 29:30 [Иаков] вошел и к Рахили, и любил Рахиль больше, нежели Лию; и служил у него еще семь лет других.
\rsbpar\vs Gen 29:31 Господь [Бог] узрел, что Лия была нелюбима, и отверз утробу ее, а Рахиль была неплодна.
\vs Gen 29:32 Лия зачала и родила [Иакову] сына, и нарекла ему имя: Рувим, потому что сказала она: Господь призрел на мое бедствие [и дал мне сына], ибо теперь будет любить меня муж мой.
\vs Gen 29:33 И зачала [Лия] опять и родила [Иакову второго] сына, и сказала: Господь услышал, что я нелюбима, и дал мне и сего. И нарекла ему имя: Симеон.
\vs Gen 29:34 И зачала еще и родила сына, и сказала: теперь-то прилепится ко мне муж мой, ибо я родила ему трех сынов. От сего наречено ему имя: Левий.
\vs Gen 29:35 И еще зачала и родила сына, и сказала: теперь-то я восхвалю Господа. Посему нарекла ему имя Иуда. И перестала рождать.
\vs Gen 30:1 И увидела Рахиль, что она не рождает детей Иакову, и позавидовала Рахиль сестре своей, и сказала Иакову: дай мне детей, а если не так, я умираю.
\vs Gen 30:2 Иаков разгневался на Рахиль и сказал [ей]: разве я Бог, Который не дал тебе плода чрева?
\vs Gen 30:3 Она сказала: вот служанка моя Валла; войди к ней; пусть она родит на колени мои, чтобы и я имела детей от нее.
\vs Gen 30:4 И дала она Валлу, служанку свою, в жену ему; и вошел к ней Иаков.
\vs Gen 30:5 Валла [служанка Рахилина] зачала и родила Иакову сына.
\vs Gen 30:6 И сказала Рахиль: судил мне Бог, и услышал голос мой, и дал мне сына. Посему нарекла ему имя: Дан.
\vs Gen 30:7 И еще зачала и родила Валла, служанка Рахилина, другого сына Иакову.
\vs Gen 30:8 И сказала Рахиль: борьбою сильною боролась я с сестрою моею и превозмогла. И нарекла ему имя: Неффалим.
\vs Gen 30:9 Лия увидела, что перестала рождать, и взяла служанку свою Зелфу, и дала ее Иакову в жену, [и он вошел к ней].
\vs Gen 30:10 И Зелфа, служанка Лиина, [зачала и] родила Иакову сына.
\vs Gen 30:11 И сказала Лия: прибавилось. И нарекла ему имя: Гад.
\vs Gen 30:12 И [еще зачала] Зелфа, служанка Лии, [и] родила другого сына Иакову.
\vs Gen 30:13 И сказала Лия: к благу моему, ибо блаженною будут называть меня женщины. И нарекла ему имя: Асир.
\vs Gen 30:14 Рувим пошел во время жатвы пшеницы, и нашел мандрагоровые яблоки в поле, и принес их Лии, матери своей. И Рахиль сказала Лии [сестре своей]: дай мне мандрагоров сына твоего.
\vs Gen 30:15 Но [Лия] сказала ей: неужели мало тебе завладеть мужем моим, что ты домогаешься и мандрагоров сына моего? Рахиль сказала: так пусть он ляжет с тобою эту ночь, за мандрагоры сына твоего.
\vs Gen 30:16 Иаков пришел с поля вечером, и Лия вышла ему навстречу и сказала: войди ко мне [сегодня], ибо я купила тебя за мандрагоры сына моего. И лег он с нею в ту ночь.
\vs Gen 30:17 И услышал Бог Лию, и она зачала и родила Иакову пятого сына.
\vs Gen 30:18 И сказала Лия: Бог дал возмездие мне за то, что я отдала служанку мою мужу моему. И нарекла ему имя: Иссахар [что значит возмездие].
\vs Gen 30:19 И еще зачала Лия и родила Иакову шестого сына.
\vs Gen 30:20 И сказала Лия: Бог дал мне прекрасный дар; теперь будет жить у меня муж мой, ибо я родила ему шесть сынов. И нарекла ему имя: Завулон.
\vs Gen 30:21 Потом родила дочь и нарекла ей имя: Дина.
\vs Gen 30:22 И вспомнил Бог о Рахили, и услышал ее Бог, и отверз утробу ее.
\vs Gen 30:23 Она зачала и родила [Иакову] сына, и сказала [Рахиль]: снял Бог позор мой.
\vs Gen 30:24 И нарекла ему имя: Иосиф, сказав: Господь даст мне и другого сына.
\rsbpar\vs Gen 30:25 После того, как Рахиль родила Иосифа, Иаков сказал Лавану: отпусти меня, и пойду я в свое место, и в свою землю;
\vs Gen 30:26 отдай [мне] жен моих и детей моих, за которых я служил тебе, и я пойду, ибо ты знаешь службу мою, какую я служил тебе.
\vs Gen 30:27 И сказал ему Лаван: о, если бы я нашел благоволение пред очами твоими! я примечаю, что за тебя Господь благословил меня.
\vs Gen 30:28 И сказал: назначь себе награду от меня, и я дам [тебе].
\vs Gen 30:29 И сказал ему [Иаков]: ты знаешь, как я служил тебе, и каков стал скот твой при мне;
\vs Gen 30:30 ибо мало было у тебя до меня, а стало много; Господь благословил тебя с приходом моим; когда же я буду работать для своего дома?
\vs Gen 30:31 И сказал [ему Лаван]: что дать тебе? Иаков сказал [ему]: не давай мне ничего. Если только сделаешь мне, чт\acc{о} я скажу, то я опять буду пасти и стеречь овец твоих.
\vs Gen 30:32 Я пройду сегодня по всему \bibemph{стаду} овец твоих; отдели из него всякий скот с крапинами и с пятнами, всякую скотину черную из овец, также с пятнами и с крапинами из коз. \bibemph{Такой скот} будет наградою мне [и будет мой].
\vs Gen 30:33 И будет говорить за меня пред тобою справедливость моя в следующее время, когда придешь посмотреть награду мою. Всякая из коз не с крапинами и не с пятнами, и из овец не черная, краденое это у меня.
\vs Gen 30:34 Лаван сказал [ему]: хорошо, пусть будет по твоему слову.
\vs Gen 30:35 И отделил в тот день козлов пестрых и с пятнами, и всех коз с крапинами и с пятнами, всех, на которых было \bibemph{несколько} белого, и всех черных овец, и отдал на руки сыновьям своим;
\vs Gen 30:36 и назначил расстояние между собою и между Иаковом на три дня пути. Иаков же пас остальной мелкий скот Лаванов.
\vs Gen 30:37 И взял Иаков свежих прутьев тополевых, миндальных и яворовых, и вырезал на них [Иаков] белые полосы, сняв кору до белизны, которая на прутьях,
\vs Gen 30:38 и положил прутья с нарезкою перед скотом в водопойных корытах, куда скот приходил пить, и где, приходя пить, зачинал пред прутьями.
\vs Gen 30:39 И зачинал скот пред прутьями, и рождался скот пестрый, и с крапинами, и с пятнами.
\vs Gen 30:40 И отделял Иаков ягнят и ставил скот лицем к пестрому и всему черному скоту Лаванову; и держал свои стада особо и не ставил их вместе со скотом Лавана.
\vs Gen 30:41 Каждый раз, когда зачинал скот крепкий, Иаков клал прутья в корытах пред глазами скота, чтобы он зачинал пред прутьями.
\vs Gen 30:42 А когда зачинал скот слабый, тогда он не клал. И доставался слабый \bibemph{скот} Лавану, а крепкий Иакову.
\vs Gen 30:43 И сделался этот человек весьма, весьма богатым, и было у него множество мелкого скота [и крупного скота], и рабынь, и рабов, и верблюдов, и ослов.
\vs Gen 31:1 И услышал [Иаков] слова сынов Лавановых, которые говорили: Иаков завладел всем, что было у отца нашего, и из имения отца нашего составил все богатство сие.
\vs Gen 31:2 И увидел Иаков лице Лавана, и вот, оно не таково к нему, как было вчера и третьего дня.
\vs Gen 31:3 И сказал Господь Иакову: возвратись в землю отцов твоих и на родину твою; и Я буду с тобою.
\vs Gen 31:4 И послал Иаков, и призвал Рахиль и Лию в поле, к \bibemph{стаду} мелкого скота своего,
\vs Gen 31:5 и сказал им: я вижу лице отца вашего, что оно ко мне не таково, как было вчера и третьего дня; но Бог отца моего был со мною;
\vs Gen 31:6 вы сами знаете, что я всеми силами служил отцу вашему,
\vs Gen 31:7 а отец ваш обманывал меня и раз десять переменял награду мою; но Бог не попустил ему сделать мне зло.
\vs Gen 31:8 Когда сказал он, что \bibemph{скот} с крапинами будет тебе в награду, то скот весь родил с крапинами. А когда он сказал: пестрые будут тебе в награду, то скот весь и родил пестрых.
\vs Gen 31:9 И отнял Бог [весь] скот у отца вашего и дал [его] мне.
\vs Gen 31:10 Однажды в такое время, когда скот зачинает, я взглянул и увидел во сне, и вот козлы [и овны], поднявшиеся на скот [на коз и овец] пестрые, с крапинами и пятнами.
\vs Gen 31:11 Ангел Божий сказал мне во сне: Иаков! Я сказал: вот я.
\vs Gen 31:12 Он сказал: возведи очи твои и посмотри: все козлы [и овны], поднявшиеся на скот [на коз и овец], пестрые, с крапинами и с пятнами, ибо Я вижу все, что Лаван делает с тобою;
\vs Gen 31:13 Я Бог [явившийся тебе] в Вефиле, где ты возлил елей на памятник и где ты дал Мне обет; теперь встань, выйди из земли сей и возвратись в землю родины твоей [и Я буду с тобою].
\vs Gen 31:14 Рахиль и Лия сказали ему в ответ: есть ли еще нам доля и наследство в доме отца нашего?
\vs Gen 31:15 не за чужих ли он нас почитает? ибо он продал нас и съел даже серебро наше;
\vs Gen 31:16 посему все [имение и] богатство, которое Бог отнял у отца нашего, есть наше и детей наших; итак делай все, что Бог сказал тебе.
\rsbpar\vs Gen 31:17 И встал Иаков, и посадил детей своих и жен своих на верблюдов,
\vs Gen 31:18 и взял с собою весь скот свой и все богатство свое, которое приобрел, скот собственный его, который он приобрел в Месопотамии, [и все свое,] чтобы идти к Исааку, отцу своему, в землю Ханаанскую.
\vs Gen 31:19 И как Лаван пошел стричь скот свой, то Рахиль похитила идолов, которые были у отца ее.
\vs Gen 31:20 Иаков же похитил сердце у Лавана Арамеянина, потому что не известил его, что удаляется.
\vs Gen 31:21 И ушел со всем, что у него было; и, встав, перешел реку и направился к горе Галаад.
\vs Gen 31:22 На третий день сказали Лавану [Арамеянину], что Иаков ушел.
\vs Gen 31:23 Тогда он взял с собою [сынов и] родственников своих, и гнался за ним семь дней, и догнал его на горе Галаад.
\vs Gen 31:24 И пришел Бог к Лавану Арамеянину ночью во сне и сказал ему: берегись, не говори Иакову ни доброго, ни худого.
\vs Gen 31:25 И догнал Лаван Иакова; Иаков же поставил шатер свой на горе, и Лаван со сродниками своими поставил на горе Галаад.
\vs Gen 31:26 И сказал Лаван Иакову: что ты сделал? для чего ты обманул меня, и увел дочерей моих, как плененных оружием?
\vs Gen 31:27 зачем ты убежал тайно, и укрылся от меня, и не сказал мне? я отпустил бы тебя с веселием и с песнями, с тимпаном и с гуслями;
\vs Gen 31:28 ты не позволил мне даже поцеловать внуков моих и дочерей моих; безрассудно ты сделал.
\vs Gen 31:29 Есть в руке моей сила сделать вам зло; но Бог отца вашего вчера говорил ко мне и сказал: берегись, не говори Иакову ни хорошего, ни худого.
\vs Gen 31:30 Но пусть бы ты ушел, потому что ты нетерпеливо захотел быть в доме отца твоего,~--- зачем ты украл богов моих?
\vs Gen 31:31 Иаков отвечал Лавану и сказал: \bibemph{я} боялся, ибо я думал, не отнял бы ты у меня дочерей своих [и всего моего].
\vs Gen 31:32 [И сказал Иаков:] у кого найдешь богов твоих, тот не будет жив; при родственниках наших узнавай, что [есть твоего] у меня, и возьми себе. [Но он ничего у него не узнал.] Иаков не знал, что Рахиль [жена его] украла их.
\vs Gen 31:33 И ходил Лаван в шатер Иакова, и в шатер Лии, и в шатер двух рабынь, [и обыскивал,] но не нашел. И, выйдя из шатра Лии, вошел в шатер Рахили.
\vs Gen 31:34 Рахиль же взяла идолов, и положила их под верблюжье седло и села на них. И обыскал Лаван весь шатер; но не нашел.
\vs Gen 31:35 Она же сказала отцу своему: да не прогневается господин мой, что я не могу встать пред тобою, ибо у меня обыкновенное женское. И [Лаван] искал [во всем шатре], но не нашел идолов.
\vs Gen 31:36 Иаков рассердился и вступил в спор с Лаваном. И начал Иаков говорить и сказал Лавану: какая вина моя, какой грех мой, что ты преследуешь меня?
\vs Gen 31:37 ты осмотрел у меня все вещи [в доме моем], что нашел ты из всех вещей твоего дома? покажи здесь пред родственниками моими и пред родственниками твоими; пусть они рассудят между нами обоими.
\vs Gen 31:38 Вот, двадцать лет я \bibemph{был} у тебя; овцы твои и козы твои не выкидывали; овнов стада твоего я не ел;
\vs Gen 31:39 растерзанного зверем я не приносил к тебе, это был мой убыток; ты с меня взыскивал, днем ли что пропадало, ночью ли пропадало;
\vs Gen 31:40 я томился днем от жара, а ночью от стужи, и сон мой убегал от глаз моих.
\vs Gen 31:41 Таковы мои двадцать лет в доме твоем. Я служил тебе четырнадцать лет за двух дочерей твоих и шесть лет за скот твой, а ты десять раз переменял награду мою.
\vs Gen 31:42 Если бы не был со мною Бог отца моего, Бог Авраама и страх Исаака, ты бы теперь отпустил меня ни с чем. Бог увидел бедствие мое и труд рук моих и вступился \bibemph{за меня} вчера.
\vs Gen 31:43 И отвечал Лаван и сказал Иакову: дочери~--- мои дочери; дети~--- мои дети; скот~--- мой скот, и все, что ты видишь, это мое: могу ли я что сделать теперь с дочерями моими и с детьми их, которые рождены ими?
\vs Gen 31:44 Теперь заключим союз я и ты, и это будет свидетельством между мною и тобою. [При сем Иаков сказал ему: вот, с нами нет никого; смотри, Бог свидетель между мною и тобою.]
\rsbpar\vs Gen 31:45 И взял Иаков камень и поставил его памятником.
\vs Gen 31:46 И сказал Иаков родственникам своим: наберите камней. Они взяли камни, и сделали холм, и ели [и пили] там на холме. [И сказал ему Лаван: холм сей свидетель сегодня между мною и тобою.]
\vs Gen 31:47 И назвал его Лаван: Иегар-Сагадуфа; а Иаков назвал его Галаадом.
\vs Gen 31:48 И сказал Лаван [Иакову]: сегодня этот холм [и памятник, который я поставил,] между мною и тобою свидетель. Посему и наречено ему имя: Галаад,
\vs Gen 31:49 \bibemph{также}: Мицпа, оттого, что Лаван сказал: да надзирает Господь надо мною и над тобою, когда мы скроемся друг от друга;
\vs Gen 31:50 если ты будешь худо поступать с дочерями моими, или если возьмешь жен сверх дочерей моих, то, хотя нет человека между нами, [который бы видел,] но смотри, Бог свидетель между мною и между тобою.
\vs Gen 31:51 И сказал Лаван Иакову: вот холм сей и вот памятник, который я поставил между мною и тобою;
\vs Gen 31:52 этот холм свидетель, и этот памятник свидетель, что ни я не перейду к тебе за этот холм, ни ты не перейдешь ко мне за этот холм и за этот памятник, для зла;
\vs Gen 31:53 Бог Авраамов и Бог Нахоров да судит между нами, Бог отца их. Иаков поклялся страхом отца своего Исаака.
\vs Gen 31:54 И заколол Иаков жертву на горе и позвал родственников своих есть хлеб; и они ели хлеб [и пили] и ночевали на горе.
\vs Gen 31:55 И встал Лаван рано утром и поцеловал внуков своих и дочерей своих, и благословил их. И пошел и возвратился Лаван в свое место.
\vs Gen 32:1 А Иаков пошел путем своим. [И, взглянув, увидел ополчение Божие ополчившееся.] И встретили его Ангелы Божии.
\vs Gen 32:2 Иаков, увидев их, сказал: это ополчение Божие. И нарек имя месту тому: Маханаим.
\vs Gen 32:3 И послал Иаков пред собою вестников к брату своему Исаву в землю Сеир, в область Едом,
\vs Gen 32:4 и приказал им, сказав: так скажите господину моему Исаву: вот что говорит раб твой Иаков: я жил у Лавана и прожил доныне;
\vs Gen 32:5 и есть у меня волы и ослы и мелкий скот, и рабы и рабыни; и я послал известить \bibemph{о себе} господина моего [Исава], дабы приобрести [рабу твоему] благоволение пред очами твоими.
\vs Gen 32:6 И возвратились вестники к Иакову и сказали: мы ходили к брату твоему Исаву; он идет навстречу тебе, и с ним четыреста человек.
\vs Gen 32:7 Иаков очень испугался и смутился; и разделил людей, бывших с ним, и скот мелкий и крупный и верблюдов на два стана.
\vs Gen 32:8 И сказал [Иаков]: если Исав нападет на один стан и побьет его, то остальной стан может спастись.
\vs Gen 32:9 И сказал Иаков: Боже отца моего Авраама и Боже отца моего Исаака, Господи [Боже], сказавший мне: возвратись в землю твою, на родину твою, и Я буду благотворить тебе!
\vs Gen 32:10 Недостоин я всех милостей и всех благодеяний, которые Ты сотворил рабу Твоему, ибо я с посохом моим перешел этот Иордан, а теперь у меня два стана.
\vs Gen 32:11 Избавь меня от руки брата моего, от руки Исава, ибо я боюсь его, чтобы он, придя, не убил меня [и] матери с детьми.
\vs Gen 32:12 Ты сказал: Я буду благотворить тебе и сделаю потомство твое, как песок морской, которого не исчислить от множества.
\vs Gen 32:13 И ночевал там \bibemph{Иаков} в ту ночь. И взял из того, что у него было, [и послал] в подарок Исаву, брату своему:
\vs Gen 32:14 двести коз, двадцать козлов, двести овец, двадцать овнов,
\vs Gen 32:15 тридцать верблюдиц дойных с жеребятами их, сорок коров, десять волов, двадцать ослиц, десять ослов.
\vs Gen 32:16 И дал в руки рабам своим каждое стадо особо и сказал рабам своим: пойдите предо мною и оставляйте расстояние от стада до стада.
\vs Gen 32:17 И приказал первому, сказав: когда брат мой Исав встретится тебе и спросит тебя, говоря: чей ты? и куда идешь? и чье это \bibemph{стадо} [идет] пред тобою?
\vs Gen 32:18 то скажи: раба твоего Иакова; это подарок, посланный господину моему Исаву; вот, и сам он за нами [идет].
\vs Gen 32:19 То же [что первому] приказал он и второму, и третьему, и всем, которые шли за стадами, говоря: так скажите Исаву, когда встретите его;
\vs Gen 32:20 и скажите: вот, и раб твой Иаков [идет] за нами. Ибо он сказал \bibemph{сам в себе}: умилостивлю его дарами, которые идут предо мною, и потом увижу лице его; может быть, и примет меня.
\vs Gen 32:21 И пошли дары пред ним, а он ту ночь ночевал в стане.
\vs Gen 32:22 И встал в ту ночь, и, взяв двух жен своих и двух рабынь своих, и одиннадцать сынов своих, перешел через Иавок вброд;
\vs Gen 32:23 и, взяв их, перевел через поток, и перевел все, что у него \bibemph{было}.
\rsbpar\vs Gen 32:24 И остался Иаков один. И боролся Некто с ним до появления зари;
\vs Gen 32:25 и, увидев, что не одолевает его, коснулся состава бедра его и повредил состав бедра у Иакова, когда он боролся с Ним.
\vs Gen 32:26 И сказал [ему]: отпусти Меня, ибо взошла заря. Иаков сказал: не отпущу Тебя, пока не благословишь меня.
\vs Gen 32:27 И сказал: как имя твое? Он сказал: Иаков.
\vs Gen 32:28 И сказал [ему]: отныне имя тебе будет не Иаков, а Израиль, ибо ты боролся с Богом, и человеков одолевать будешь.
\vs Gen 32:29 Спросил и Иаков, говоря: скажи [мне] имя Твое. И Он сказал: на что ты спрашиваешь о имени Моем? [оно чудно.] И благословил его там.
\vs Gen 32:30 И нарек Иаков имя месту тому: Пенуэл; ибо, \bibemph{говорил он}, я видел Бога лицем к лицу, и сохранилась душа моя.
\vs Gen 32:31 И взошло солнце, когда он проходил Пенуэл; и хромал он на бедро свое.
\vs Gen 32:32 Поэтому и доныне сыны Израилевы не едят жилы, которая на составе бедра, потому что \bibemph{Боровшийся} коснулся жилы на составе бедра Иакова.
\vs Gen 33:1 Взглянул Иаков и увидел, и вот, идет Исав, [брат его,] и с ним четыреста человек. И разделил [Иаков] детей Лии, Рахили и двух служанок.
\vs Gen 33:2 И поставил [двух] служанок и детей их впереди, Лию и детей ее за ними, а Рахиль и Иосифа позади.
\vs Gen 33:3 А сам пошел пред ними и поклонился до земли семь раз, подходя к брату своему.
\vs Gen 33:4 И побежал Исав к нему навстречу и обнял его, и пал на шею его и целовал его, и плакали [оба].
\vs Gen 33:5 И взглянул [Исав] и увидел жен и детей и сказал: кто это у тебя? \bibemph{Иаков} сказал: дети, которых Бог даровал рабу твоему.
\vs Gen 33:6 И подошли служанки и дети их и поклонились;
\vs Gen 33:7 подошла и Лия и дети ее и поклонились; наконец подошли Иосиф и Рахиль и поклонились.
\vs Gen 33:8 И сказал Исав: для чего у тебя это множество, которое я встретил? И сказал Иаков: дабы [рабу твоему] приобрести благоволение в очах господина моего.
\vs Gen 33:9 Исав сказал: у меня много, брат мой; пусть будет твое у тебя.
\vs Gen 33:10 Иаков сказал: нет, если я приобрел благоволение в очах твоих, прими дар мой от руки моей, ибо я увидел лице твое, как бы кто увидел лице Божие, и ты был благосклонен ко мне;
\vs Gen 33:11 прими благословение мое, которое я принес тебе, потому что Бог даровал мне, и есть у меня всё. И упросил его, и тот взял
\vs Gen 33:12 и сказал: поднимемся и пойдем; и я пойду пред тобою.
\vs Gen 33:13 Иаков сказал ему: господин мой знает, что дети нежны, а мелкий и крупный скот у меня дойный: если погнать его один день, то помрет весь скот;
\vs Gen 33:14 пусть господин мой пойдет впереди раба своего, а я пойду медленно, как пойдет скот, который предо мною, и как пойдут дети, и приду к господину моему в Сеир.
\vs Gen 33:15 Исав сказал: оставлю я с тобою \bibemph{несколько} из людей, которые при мне. Иаков сказал: к чему это? только бы мне приобрести благоволение в очах господина моего!
\vs Gen 33:16 И возвратился Исав в тот же день путем своим в Сеир.
\vs Gen 33:17 А Иаков двинулся в Сокхоф, и построил себе дом, и для скота своего сделал шалаши. От сего он нарек имя месту: Сокхоф.
\vs Gen 33:18 Иаков, возвратившись из Месопотамии, благополучно пришел в город Сихем, который в земле Ханаанской, и расположился пред городом.
\vs Gen 33:19 И купил часть поля, на котором раскинул шатер свой, у сынов Еммора, отца Сихемова, за сто монет.
\vs Gen 33:20 И поставил там жертвенник, и призвал имя Господа Бога Израилева.
\vs Gen 34:1 Дина, дочь Лии, которую она родила Иакову, вышла посмотреть на дочерей земли той.
\vs Gen 34:2 И увидел ее Сихем, сын Еммора Евеянина, князя земли той, и взял ее, и спал с нею, и сделал ей насилие.
\vs Gen 34:3 И прилепилась душа его к Дине, дочери Иакова, и он полюбил девицу и говорил по сердцу девицы.
\vs Gen 34:4 И сказал Сихем Еммору, отцу своему, говоря: возьми мне эту девицу в жену.
\vs Gen 34:5 Иаков слышал, что [сын Емморов] обесчестил Дину, дочь его, но как сыновья его были со скотом его в поле, то Иаков молчал, пока не пришли они.
\vs Gen 34:6 И вышел Еммор, отец Сихемов, к Иакову, поговорить с ним.
\vs Gen 34:7 Сыновья же Иакова пришли с поля, и когда услышали, то огорчились мужи те и воспылали гневом, потому что бесчестие сделал он Израилю, переспав с дочерью Иакова, а так не надлежало делать.
\vs Gen 34:8 Еммор стал говорить им, и сказал: Сихем, сын мой, прилепился душею к дочери вашей; дайте же ее в жену ему;
\vs Gen 34:9 породнитесь с нами; отдавайте за нас дочерей ваших, а наших дочерей берите себе [за сыновей ваших];
\vs Gen 34:10 и живите с нами; земля сия [пространна] пред вами, живите и промышляйте на ней и приобретайте ее во владение.
\vs Gen 34:11 Сихем же сказал отцу ее и братьям ее: только бы мне найти благоволение в очах ваших, я дам, что ни скажете мне;
\vs Gen 34:12 назначьте самое большое вено и дары; я дам, что ни скажете мне, только отдайте мне девицу в жену.
\vs Gen 34:13 И отвечали сыновья Иакова Сихему и Еммору, отцу его, с лукавством; а говорили так потому, что он обесчестил Дину, сестру их;
\vs Gen 34:14 и сказали им [Симеон и Левий, братья Дины, сыновья Лиины]: не можем этого сделать, выдать сестру нашу за человека, который необрезан, ибо это бесчестно для нас;
\vs Gen 34:15 только на том условии мы согласимся с вами [и поселимся у вас], если вы будете как мы, чтобы и у вас весь мужеский пол был обрезан;
\vs Gen 34:16 и будем отдавать за вас дочерей наших и брать за себя ваших дочерей, и будем жить с вами, и составим один народ;
\vs Gen 34:17 а если не послушаетесь нас в том, чтобы обрезаться, то мы возьмем дочь нашу и удалимся.
\vs Gen 34:18 И понравились слова сии Еммору и Сихему, сыну Емморову.
\vs Gen 34:19 Юноша не умедлил исполнить это, потому что любил дочь Иакова. А он более всех уважаем был из дома отца своего.
\vs Gen 34:20 И пришел Еммор и Сихем, сын его, к воротам города своего, и стали говорить жителям города своего и сказали:
\vs Gen 34:21 сии люди мирны с нами; пусть они селятся на земле и промышляют на ней; земля же вот пространна пред ними. Станем брать дочерей их себе в жены и наших дочерей выдавать за них.
\vs Gen 34:22 Только на том условии сии люди соглашаются жить с нами и быть одним народом, чтобы и у нас обрезан был весь мужеский пол, как они обрезаны.
\vs Gen 34:23 Не для нас ли стада их, и имение их, и весь скот их? Только [в том] согласимся с ними, и будут жить с нами.
\vs Gen 34:24 И послушались Еммора и Сихема, сына его, все выходящие из ворот города его: и обрезан был весь мужеский пол,~--- все выходящие из ворот города его.
\vs Gen 34:25 На третий день, когда они были в болезни, два сына Иакова, Симеон и Левий, братья Динины, взяли каждый свой меч, и смело напали на город, и умертвили весь мужеский пол;
\vs Gen 34:26 и самого Еммора и Сихема, сына его, убили мечом; и взяли Дину из дома Сихемова и вышли.
\vs Gen 34:27 Сыновья Иакова пришли к убитым и разграбили город за то, что обесчестили [Дину] сестру их.
\vs Gen 34:28 Они взяли мелкий и крупный скот их, и ослов их, и что ни было в городе, и что ни было в поле;
\vs Gen 34:29 и все богатство их, и всех детей их, и жен их взяли в плен, и разграбили всё, что было в [городе, и всё, что было в] домах.
\vs Gen 34:30 И сказал Иаков Симеону и Левию: вы возмутили меня, сделав меня ненавистным для [всех] жителей сей земли, для Хананеев и Ферезеев. У меня людей мало; соберутся против меня, поразят меня, и истреблен буду я и дом мой.
\vs Gen 34:31 Они же сказали: а разве можно поступать с сестрою нашею, как с блудницею!
\vs Gen 35:1 Бог сказал Иакову: встань, пойди в Вефиль и живи там, и устрой там жертвенник Богу, явившемуся тебе, когда ты бежал от лица Исава, брата твоего.
\vs Gen 35:2 И сказал Иаков дому своему и всем бывшим с ним: бросьте богов чужих, находящихся у вас, и очиститесь, и перемените одежды ваши;
\vs Gen 35:3 встанем и пойдем в Вефиль; там устрою я жертвенник Богу, Который услышал меня в день бедствия моего и был со мною [и хранил меня] в пути, которым я ходил.
\vs Gen 35:4 И отдали Иакову всех богов чужих, бывших в руках их, и серьги, бывшие в ушах у них, и закопал их Иаков под дубом, который близ Сихема. [И оставил их безвестными даже до нынешнего дня.]
\vs Gen 35:5 И отправились они [от Сихема]. И был ужас Божий на окрестных городах, и не преследовали сынов Иаковлевых.
\vs Gen 35:6 И пришел Иаков в Луз, что в земле Ханаанской, то есть в Вефиль, сам и все люди, бывшие с ним,
\vs Gen 35:7 и устроил там жертвенник, и назвал сие место: Эл-Вефиль, ибо тут явился ему Бог, когда он бежал от лица [Исава] брата своего.
\vs Gen 35:8 И умерла Девора, кормилица Ревеккина, и погребена ниже Вефиля под дубом, который и назвал \bibemph{Иаков} дубом плача.
\rsbpar\vs Gen 35:9 И явился Бог Иакову [в Лузе] по возвращении его из Месопотамии, и благословил его,
\vs Gen 35:10 и сказал ему Бог: имя твое Иаков; отныне ты не будешь называться Иаковом, но будет имя тебе: Израиль. И нарек ему имя: Израиль.
\vs Gen 35:11 И сказал ему Бог: Я Бог Всемогущий; плодись и умножайся; народ и множество народов будет от тебя, и цари произойдут из чресл твоих;
\vs Gen 35:12 землю, которую Я дал Аврааму и Исааку, Я дам тебе, и потомству твоему по тебе дам землю сию.
\vs Gen 35:13 И восшел от него Бог с места, на котором говорил ему.
\vs Gen 35:14 И поставил Иаков памятник на месте, на котором говорил ему [Бог], памятник каменный, и возлил на него возлияние, и возлил на него елей;
\vs Gen 35:15 и нарек Иаков имя месту, на котором Бог говорил ему: Вефиль.
\rsbpar\vs Gen 35:16 И отправились из Вефиля. [И раскинул он шатер свой за башнею Гадер.] И когда еще оставалось некоторое расстояние земли до Ефрафы, Рахиль родила, и роды ее были трудны.
\vs Gen 35:17 Когда же она страдала в родах, повивальная бабка сказала ей: не бойся, ибо и это тебе сын.
\vs Gen 35:18 И когда выходила из нее душа, ибо она умирала, то нарекла ему имя: Бенони. Но отец его назвал его Вениамином.
\vs Gen 35:19 И умерла Рахиль, и погребена на дороге в Ефрафу, то есть Вифлеем.
\vs Gen 35:20 Иаков поставил над гробом ее памятник. Это надгробный памятник Рахили до сего дня.
\vs Gen 35:21 И отправился [оттуда] Израиль и раскинул шатер свой за башнею Гадер.
\vs Gen 35:22 Во время пребывания Израиля в той стране, Рувим пошел и переспал с Валлою, наложницею отца своего [Иакова]. И услышал Израиль [и принял то с огорчением].\rsbpar Сынов же у Иакова было двенадцать.
\vs Gen 35:23 Сыновья Лии: первенец Иакова Рувим, \bibemph{по нем} Симеон, Левий, Иуда, Иссахар и Завулон.
\vs Gen 35:24 Сыновья Рахили: Иосиф и Вениамин.
\vs Gen 35:25 Сыновья Валлы, служанки Рахилиной: Дан и Неффалим.
\vs Gen 35:26 Сыновья Зелфы, служанки Лииной: Гад и Асир. Сии сыновья Иакова, родившиеся ему в Месопотамии.
\rsbpar\vs Gen 35:27 И пришел Иаков к Исааку, отцу своему, [ибо он был еще жив,] в Мамре, в Кириаф-Арбу, то есть Хеврон [в земле Ханаанской,] где странствовал Авраам и Исаак.
\vs Gen 35:28 И было дней [жизни] Исааковой сто восемьдесят лет.
\vs Gen 35:29 И испустил Исаак дух и умер, и приложился к народу своему, будучи стар и насыщен жизнью; и погребли его Исав и Иаков, сыновья его.
\vs Gen 36:1 Вот родословие Исава, он же Едом.
\vs Gen 36:2 Исав взял себе жен из дочерей Ханаанских: Аду, дочь Елона Хеттеянина, и Оливему, дочь Аны, сына Цивеона Евеянина,
\vs Gen 36:3 и Васемафу, дочь Измаила, сестру Наваиофа.
\vs Gen 36:4 Ада родила Исаву Елифаза, Васемафа родила Рагуила,
\vs Gen 36:5 Оливема родила Иеуса, Иеглома и Корея. Это сыновья Исава, родившиеся ему в земле Ханаанской.
\vs Gen 36:6 И взял Исав жен своих и сыновей своих, и дочерей своих, и всех людей дома своего, и [все] стада свои, и весь скот свой, и всё имение свое, которое он приобрел в земле Ханаанской, и пошел [Исав] в \bibemph{другую} землю от лица Иакова, брата своего,
\vs Gen 36:7 ибо имение их было так велико, что они не могли жить вместе, и земля странствования их не вмещала их, по множеству стад их.
\vs Gen 36:8 И поселился Исав на горе Сеир, Исав, он же Едом.
\rsbpar\vs Gen 36:9 И вот родословие Исава, отца Идумеев, на горе Сеир.
\vs Gen 36:10 Вот имена сынов Исава: Елифаз, сын Ады, жены Исавовой, и Рагуил, сын Васемафы, жены Исавовой.
\vs Gen 36:11 У Елифаза были сыновья: Феман, Омар, Цефо, Гафам и Кеназ.
\vs Gen 36:12 Фамна же была наложница Елифаза, сына Исавова, и родила Елифазу Амалика. Вот сыновья Ады, жены Исавовой.
\vs Gen 36:13 И вот сыновья Рагуила: Нахаф и Зерах, Шамма и Миза. Это сыновья Васемафы, жены Исавовой.
\vs Gen 36:14 И сии были сыновья Оливемы, дочери Аны, сына Цивеонова, жены Исавовой: она родила Исаву Иеуса, Иеглома и Корея.
\vs Gen 36:15 Вот старейшины сынов Исавовых. Сыновья Елифаза, первенца Исавова: старейшина Феман, старейшина Омар, старейшина Цефо, старейшина Кеназ,
\vs Gen 36:16 старейшина Корей, старейшина Гафам, старейшина Амалик. Сии старейшины Елифазовы в земле Едома; сии сыновья Ады.
\vs Gen 36:17 Сии сыновья Рагуила, сына Исавова: старейшина Нахаф, старейшина Зерах, старейшина Шамма, старейшина Миза. Сии старейшины Рагуиловы в земле Едома; сии сыновья Васемафы, жены Исавовой.
\vs Gen 36:18 Сии сыновья Оливемы, жены Исавовой: старейшина Иеус, старейшина Иеглом, старейшина Корей. Сии старейшины Оливемы, дочери Аны, жены Исавовой.
\vs Gen 36:19 Вот сыновья Исава, и вот старейшины их. Это Едом.
\rsbpar\vs Gen 36:20 Сии сыновья Сеира Хорреянина, жившие в земле той: Лотан, Шовал, Цивеон, Ана,
\vs Gen 36:21 Дишон, Эцер и Дишан. Сии старейшины Хорреев, сынов Сеира, в земле Едома.
\vs Gen 36:22 Сыновья Лотана были: Хори и Геман; а сестра у Лотана: Фамна.
\vs Gen 36:23 Сии сыновья Шовала: Алван, Манахаф, Эвал, Шефо и Онам.
\vs Gen 36:24 Сии сыновья Цивеона: Аиа и Ана. Это тот Ана, который нашел теплые воды в пустыне, когда пас ослов Цивеона, отца своего.
\vs Gen 36:25 Сии дети Аны: Дишон и Оливема, дочь Аны.
\vs Gen 36:26 Сии сыновья Дишона: Хемдан, Эшбан, Ифран и Херан.
\vs Gen 36:27 Сии сыновья Эцера: Билган, Зааван, [Укам] и Акан.
\vs Gen 36:28 Сии сыновья Дишана: Уц и Аран.
\vs Gen 36:29 Сии старейшины Хорреев: старейшина Лотан, старейшина Шовал, старейшина Цивеон, старейшина Ана,
\vs Gen 36:30 старейшина Дишон, старейшина Эцер, старейшина Дишан. Вот старейшины Хорреев, по старшинствам их в земле Сеир.
\rsbpar\vs Gen 36:31 Вот цари, царствовавшие в земле Едома, прежде царствования царей у сынов Израилевых:
\vs Gen 36:32 царствовал в Едоме Бела, сын Веоров, а имя городу его Дингава.
\vs Gen 36:33 И умер Бела, и воцарился по нем Иовав, сын Зераха, из Восоры.
\vs Gen 36:34 Умер Иовав, и воцарился по нем Хушам, из земли Феманитян.
\vs Gen 36:35 И умер Хушам, и воцарился по нем Гадад, сын Бедадов, который поразил Мадианитян на поле Моава; имя городу его Авиф.
\vs Gen 36:36 И умер Гадад, и воцарился по нем Самла из Масреки.
\vs Gen 36:37 И умер Самла, и воцарился по нем Саул из Реховофа, что при реке.
\vs Gen 36:38 И умер Саул, и воцарился по нем Баал-Ханан, сын Ахбора.
\vs Gen 36:39 И умер Баал-Ханан, сын Ахбора, и воцарился по нем Гадар [сын Варадов]; имя городу его Пау; имя жене его Мегетавеель, дочь Матреды, сына Мезагава.
\rsbpar\vs Gen 36:40 Сии имена старейшин Исавовых, по племенам их, по местам их, по именам их, [по народам их]: старейшина Фимна, старейшина Алва, старейшина Иетеф,
\vs Gen 36:41 старейшина Оливема, старейшина Эла, старейшина Пинон,
\vs Gen 36:42 старейшина Кеназ, старейшина Феман, старейшина Мивцар,
\vs Gen 36:43 старейшина Магдиил, старейшина Ирам. Вот старейшины Идумейские, по их селениям, в земле обладания их. Вот Исав, отец Идумеев.
\vs Gen 37:1 Иаков жил в земле странствования отца своего [Исаака], в земле Ханаанской.
\vs Gen 37:2 Вот житие Иакова. Иосиф, семнадцати лет, пас скот [отца своего] вместе с братьями своими, будучи отроком, с сыновьями Валлы и с сыновьями Зелфы, жен отца своего. И доводил Иосиф худые о них слухи до [Израиля] отца их.
\vs Gen 37:3 Израиль любил Иосифа более всех сыновей своих, потому что он был сын старости его,~--- и сделал ему разноцветную одежду.
\vs Gen 37:4 И увидели братья его, что отец их любит его более всех братьев его; и возненавидели его и не могли говорить с ним дружелюбно.
\vs Gen 37:5 И видел Иосиф сон, и рассказал [его] братьям своим: и они возненавидели его еще более.
\vs Gen 37:6 Он сказал им: выслушайте сон, который я видел:
\vs Gen 37:7 вот, мы вяжем снопы посреди поля; и вот, мой сноп встал и стал прямо; и вот, ваши снопы стали кругом и поклонились моему снопу.
\vs Gen 37:8 И сказали ему братья его: неужели ты будешь царствовать над нами? неужели будешь владеть нами? И возненавидели его еще более за сны его и за слова его.
\vs Gen 37:9 И видел он еще другой сон и рассказал его [отцу своему и] братьям своим, говоря: вот, я видел еще сон: вот, солнце и луна и одиннадцать звезд поклоняются мне.
\vs Gen 37:10 И он рассказал отцу своему и братьям своим; и побранил его отец его и сказал ему: что это за сон, который ты видел? неужели я и твоя мать, и твои братья придем поклониться тебе до земли?
\vs Gen 37:11 Братья его досадовали на него, а отец его заметил это слово.
\rsbpar\vs Gen 37:12 Братья его пошли пасти скот отца своего в Сихем.
\vs Gen 37:13 И сказал Израиль Иосифу: братья твои не пасут ли в Сихеме? пойди, я пошлю тебя к ним. Он отвечал ему: вот я.
\vs Gen 37:14 [Израиль] сказал ему: пойди, посмотри, здоровы ли братья твои и цел ли скот, и принеси мне ответ. И послал его из долины Хевронской; и он пришел в Сихем.
\vs Gen 37:15 И нашел его некто блуждающим в поле, и спросил его тот человек, говоря: чего ты ищешь?
\vs Gen 37:16 Он сказал: я ищу братьев моих; скажи мне, где они пасут?
\vs Gen 37:17 И сказал тот человек: они ушли отсюда, ибо я слышал, как они говорили: пойдем в Дофан. И пошел Иосиф за братьями своими и нашел их в Дофане.
\vs Gen 37:18 И увидели они его издали, и прежде нежели он приблизился к ним, стали умышлять против него, чтобы убить его.
\vs Gen 37:19 И сказали друг другу: вот, идет сновидец;
\vs Gen 37:20 пойдем теперь, и убьем его, и бросим его в какой-нибудь ров, и скажем, что хищный зверь съел его; и увидим, что будет из его снов.
\vs Gen 37:21 И услышал \bibemph{сие} Рувим и избавил его от рук их, сказав: не убьем его.
\vs Gen 37:22 И сказал им Рувим: не проливайте крови; бросьте его в ров, который в пустыне, а руки не налагайте на него. \bibemph{Сие говорил он} [с тем намерением], чтобы избавить его от рук их и возвратить его к отцу его.
\vs Gen 37:23 Когда Иосиф пришел к братьям своим, они сняли с Иосифа одежду его, одежду разноцветную, которая была на нем,
\vs Gen 37:24 и взяли его и бросили его в ров; ров же тот был пуст; воды в нем не было.
\vs Gen 37:25 И сели они есть хлеб, и, взглянув, увидели, вот, идет из Галаада караван Измаильтян, и верблюды их несут стираксу, бальзам и ладан: идут они отвезти это в Египет.
\vs Gen 37:26 И сказал Иуда братьям своим: что пользы, если мы убьем брата нашего и скроем кровь его?
\vs Gen 37:27 Пойдем, продадим его Измаильтянам, а руки наши да не будут на нем, ибо он брат наш, плоть наша. Братья его послушались
\vs Gen 37:28 и, когда проходили купцы Мадиамские, вытащили Иосифа изо рва и продали Иосифа Измаильтянам за двадцать сребреников; а они отвели Иосифа в Египет.
\vs Gen 37:29 Рувим же пришел опять ко рву; и вот, нет Иосифа во рве. И разодрал он одежды свои,
\vs Gen 37:30 и возвратился к братьям своим, и сказал: отрока нет, а я, куда я денусь?
\vs Gen 37:31 И взяли одежду Иосифа, и закололи козла, и вымарали одежду кровью;
\vs Gen 37:32 и послали разноцветную одежду, и доставили к отцу своему, и сказали: мы это нашли; посмотри, сына ли твоего эта одежда, или нет.
\vs Gen 37:33 Он узнал ее и сказал: \bibemph{это} одежда сына моего; хищный зверь съел его; верно, растерзан Иосиф.
\vs Gen 37:34 И разодрал Иаков одежды свои, и возложил вретище на чресла свои, и оплакивал сына своего многие дни.
\vs Gen 37:35 И собрались все сыновья его и все дочери его, чтобы утешить его; но он не хотел утешиться и сказал: с печалью сойду к сыну моему в преисподнюю. Так оплакивал его отец его.
\vs Gen 37:36 Мадианитяне же продали его в Египте Потифару, царедворцу фараонову, начальнику телохранителей.
\vs Gen 38:1 В то время Иуда отошел от братьев своих и поселился близ одного Одолламитянина, которому имя: Хира.
\vs Gen 38:2 И увидел там Иуда дочь одного Хананеянина, которому имя: Шуа; и взял ее и вошел к ней.
\vs Gen 38:3 Она зачала и родила сына; и он нарек ему имя: Ир.
\vs Gen 38:4 И зачала опять, и родила сына, и нарекла ему имя: Онан.
\vs Gen 38:5 И еще родила сына [третьего] и нарекла ему имя: Шела. Иуда был в Хезиве, когда она родила его.
\vs Gen 38:6 И взял Иуда жену Иру, первенцу своему; имя ей Фамарь.
\vs Gen 38:7 Ир, первенец Иудин, был неугоден пред очами Господа, и умертвил его Господь.
\vs Gen 38:8 И сказал Иуда Онану: войди к жене брата твоего, женись на ней, как деверь, и восстанови семя брату твоему.
\vs Gen 38:9 Онан знал, что семя будет не ему, и потому, когда входил к жене брата своего, изливал [семя] на землю, чтобы не дать семени брату своему.
\vs Gen 38:10 Зло было пред очами Господа то, что он делал; и Он умертвил и его.
\vs Gen 38:11 И сказал Иуда Фамари, невестке своей [по смерти двух сыновей своих]: живи вдовою в доме отца твоего, пока подрастет Шела, сын мой. Ибо он сказал [в уме своем]: не умер бы и он подобно братьям его. Фамарь пошла и стала жить в доме отца своего.
\rsbpar\vs Gen 38:12 Прошло много времени, и умерла дочь Шуи, жена Иудина. Иуда, утешившись, пошел в Фамну к стригущим скот его, сам и Хира, друг его, Одолламитянин.
\vs Gen 38:13 И уведомили Фамарь, говоря: вот, свекор твой идет в Фамну стричь скот свой.
\vs Gen 38:14 И сняла она с себя одежду вдовства своего, покрыла себя покрывалом и, закрывшись, села у ворот Енаима, что на дороге в Фамну. Ибо видела, что Шела вырос, и она не дана ему в жену.
\vs Gen 38:15 И увидел ее Иуда и почел ее за блудницу, потому что она закрыла лице свое. [И не узнал ее.]
\vs Gen 38:16 Он поворотил к ней и сказал: войду я к тебе. Ибо не знал, что это невестка его. Она сказала: что ты дашь мне, если войдешь ко мне?
\vs Gen 38:17 Он сказал: я пришлю тебе козленка из стада [моего]. Она сказала: дашь ли ты мне залог, пока пришлешь?
\vs Gen 38:18 Он сказал: какой дать тебе залог? Она сказала: печать твою, и перевязь твою, и трость твою, которая в руке твоей. И дал он ей и вошел к ней; и она зачала от него.
\vs Gen 38:19 И, встав, пошла, сняла с себя покрывало свое и оделась в одежду вдовства своего.
\vs Gen 38:20 Иуда же послал козленка чрез друга своего Одолламитянина, чтобы взять залог из руки женщины, но он не нашел ее.
\vs Gen 38:21 И спросил жителей того места, говоря: где блудница, \bibemph{которая была} в Енаиме при дороге? Но они сказали: здесь не было блудницы.
\vs Gen 38:22 И возвратился он к Иуде и сказал: я не нашел ее; да и жители места того сказали: здесь не было блудницы.
\vs Gen 38:23 Иуда сказал: пусть она возьмет себе, чтобы только не стали над нами смеяться; вот, я посылал этого козленка, но ты не нашел ее.
\vs Gen 38:24 Прошло около трех месяцев, и сказали Иуде, говоря: Фамарь, невестка твоя, впала в блуд, и вот, она беременна от блуда. Иуда сказал: выведите ее, и пусть она будет сожжена.
\vs Gen 38:25 Но когда повели ее, она послала сказать свекру своему: я беременна от того, чьи эти вещи. И сказала: узнавай, чья эта печать и перевязь и трость.
\vs Gen 38:26 Иуда узнал и сказал: она правее меня, потому что я не дал ее Шеле, сыну моему. И не познавал ее более.
\vs Gen 38:27 Во время родов ее оказалось, что близнецы в утробе ее.
\vs Gen 38:28 И во время родов ее показалась рука [одного]; и взяла повивальная бабка и навязала ему на руку красную нить, сказав: этот вышел первый.
\vs Gen 38:29 Но он возвратил руку свою; и вот, вышел брат его. И она сказала: как ты расторг себе преграду? И наречено ему имя: Фарес.
\vs Gen 38:30 Потом вышел брат его с красной нитью на руке. И наречено ему имя: Зара.
\vs Gen 39:1 Иосиф же отведен был в Египет, и купил его из рук Измаильтян, приведших его туда, Египтянин Потифар, царедворец фараонов, начальник телохранителей.
\vs Gen 39:2 И был Господь с Иосифом: он был успешен в делах и жил в доме господина своего, Египтянина.
\vs Gen 39:3 И увидел господин его, что Господь с ним и что всему, что он делает, Господь в руках его дает успех.
\vs Gen 39:4 И снискал Иосиф благоволение в очах его и служил ему. И он поставил его над домом своим, и все, что имел, отдал на руки его.
\vs Gen 39:5 И с того времени, как он поставил его над домом своим и над всем, что имел, Господь благословил дом Египтянина ради Иосифа, и было благословение Господне на всем, что имел он в доме и в поле [его].
\vs Gen 39:6 И оставил он все, что имел, в руках Иосифа и не знал при нем ничего, кроме хлеба, который он ел.\rsbpar Иосиф же был красив станом и красив лицем.
\vs Gen 39:7 И обратила взоры на Иосифа жена господина его и сказала: спи со мною.
\vs Gen 39:8 Но он отказался и сказал жене господина своего: вот, господин мой не знает при мне ничего в доме, и все, что имеет, отдал в мои руки;
\vs Gen 39:9 нет больше меня в доме сем; и он не запретил мне ничего, кроме тебя, потому что ты жена ему; как же сделаю я сие великое зло и согрешу пред Богом?
\vs Gen 39:10 Когда так она ежедневно говорила Иосифу, а он не слушался ее, чтобы спать с нею и быть с нею,
\vs Gen 39:11 случилось в один день, что он вошел в дом делать дело свое, а никого из домашних тут в доме не было;
\vs Gen 39:12 она схватила его за одежду его и сказала: ложись со мной. Но он, оставив одежду свою в руках ее, побежал и выбежал вон.
\vs Gen 39:13 Она же, увидев, что он оставил одежду свою в руках ее и побежал вон,
\vs Gen 39:14 кликнула домашних своих и сказала им так: посмотрите, он привел к нам Еврея ругаться над нами. Он пришел ко мне, чтобы лечь со мною, но я закричала громким голосом,
\vs Gen 39:15 и он, услышав, что я подняла вопль и закричала, оставил у меня одежду свою, и побежал, и выбежал вон.
\vs Gen 39:16 И оставила одежду его у себя до прихода господина его в дом свой.
\vs Gen 39:17 И пересказала ему те же слова, говоря: раб Еврей, которого ты привел к нам, приходил ко мне ругаться надо мною [и говорил мне: лягу я с тобою],
\vs Gen 39:18 но, когда [услышал, что] я подняла вопль и закричала, он оставил у меня одежду свою и убежал вон.
\vs Gen 39:19 Когда господин его услышал слова жены своей, которые она сказала ему, говоря: так поступил со мною раб твой, то воспылал гневом;
\vs Gen 39:20 и взял Иосифа господин его и отдал его в темницу, где заключены узники царя. И был он там в темнице.
\rsbpar\vs Gen 39:21 И Господь был с Иосифом, и простер к нему милость, и даровал ему благоволение в очах начальника темницы.
\vs Gen 39:22 И отдал начальник темницы в руки Иосифу всех узников, находившихся в темнице, и во всем, что они там ни делали, он был распорядителем.
\vs Gen 39:23 Начальник темницы и не смотрел ни за чем, что было у него в руках, потому что Господь был с \bibemph{Иосифом}, и во всем, что он делал, Господь давал успех.
\vs Gen 40:1 После сего виночерпий царя Египетского и хлебодар провинились пред господином своим, царем Египетским.
\vs Gen 40:2 И прогневался фараон на двух царедворцев своих, на главного виночерпия и на главного хлебодара,
\vs Gen 40:3 и отдал их под стражу в дом начальника телохранителей, в темницу, в место, где заключен был Иосиф.
\vs Gen 40:4 Начальник телохранителей приставил к ним Иосифа, и он служил им. И пробыли они под стражею несколько времени.
\rsbpar\vs Gen 40:5 Однажды виночерпию и хлебодару царя Египетского, заключенным в темнице, виделись сны, каждому свой сон, обоим в одну ночь, каждому сон особенного значения.
\vs Gen 40:6 И пришел к ним Иосиф поутру, увидел их, и вот, они в смущении.
\vs Gen 40:7 И спросил он царедворцев фараоновых, находившихся с ним в доме господина его под стражею, говоря: отчего у вас сегодня печальные лица?
\vs Gen 40:8 Они сказали ему: нам виделись сны; а истолковать их некому. Иосиф сказал им: не от Бога ли истолкования? расскажите мне.
\vs Gen 40:9 И рассказал главный виночерпий Иосифу сон свой и сказал ему: мне снилось, вот виноградная лоза предо мною;
\vs Gen 40:10 на лозе три ветви; она развилась, показался на ней цвет, выросли и созрели на ней ягоды;
\vs Gen 40:11 и чаша фараонова в руке у меня; я взял ягод, выжал их в чашу фараонову и подал чашу в руку фараону.
\vs Gen 40:12 И сказал ему Иосиф: вот истолкование его: три ветви~--- это три дня;
\vs Gen 40:13 через три дня фараон вознесет главу твою и возвратит тебя на место твое, и ты подашь чашу фараонову в руку его, по прежнему обыкновению, когда ты был у него виночерпием;
\vs Gen 40:14 вспомни же меня, когда хорошо тебе будет, и сделай мне благодеяние, и упомяни обо мне фараону, и выведи меня из этого дома,
\vs Gen 40:15 ибо я украден из земли Евреев; а также и здесь ничего не сделал, за что бы бросить меня в темницу.
\vs Gen 40:16 Главный хлебодар увидел, что истолковал он хорошо, и сказал Иосифу: мне также снилось: вот на голове у меня три корзины решетчатых;
\vs Gen 40:17 в верхней корзине всякая пища фараонова, изделие пекаря, и птицы [небесные] клевали ее из корзины на голове моей.
\vs Gen 40:18 И отвечал Иосиф и сказал [ему]: вот истолкование его: три корзины~--- это три дня;
\vs Gen 40:19 через три дня фараон снимет с тебя голову твою и повесит тебя на дереве, и птицы [небесные] будут клевать плоть твою с тебя.
\vs Gen 40:20 На третий день, день рождения фараонова, сделал он пир для всех слуг своих и вспомнил о главном виночерпии и главном хлебодаре среди слуг своих;
\vs Gen 40:21 и возвратил главного виночерпия на прежнее место, и он подал чашу в руку фараону,
\vs Gen 40:22 а главного хлебодара повесил [на дереве], как истолковал им Иосиф.
\vs Gen 40:23 И не вспомнил главный виночерпий об Иосифе, но забыл его.
\vs Gen 41:1 По прошествии двух лет фараону снилось: вот, он стоит у реки;
\vs Gen 41:2 и вот, вышли из реки семь коров, хороших видом и тучных плотью, и паслись в тростнике;
\vs Gen 41:3 но вот, после них вышли из реки семь коров других, худых видом и тощих плотью, и стали подле тех коров, на берегу реки;
\vs Gen 41:4 и съели коровы худые видом и тощие плотью семь коров хороших видом и тучных. И проснулся фараон,
\vs Gen 41:5 и заснул опять, и снилось ему в другой раз: вот, на одном стебле поднялось семь колосьев тучных и хороших;
\vs Gen 41:6 но вот, после них выросло семь колосьев тощих и иссушенных восточным ветром;
\vs Gen 41:7 и пожрали тощие колосья семь колосьев тучных и полных. И проснулся фараон и \bibemph{понял, что} это сон.
\vs Gen 41:8 Утром смутился дух его, и послал он, и призвал всех волхвов Египта и всех мудрецов его, и рассказал им фараон сон свой; но не было никого, кто бы истолковал его фараону.
\vs Gen 41:9 И стал говорить главный виночерпий фараону и сказал: грехи мои вспоминаю я ныне;
\vs Gen 41:10 фараон прогневался на рабов своих и отдал меня и главного хлебодара под стражу в дом начальника телохранителей;
\vs Gen 41:11 и снился нам сон в одну ночь, мне и ему, каждому снился сон особенного значения;
\vs Gen 41:12 там же был с нами молодой Еврей, раб начальника телохранителей; мы рассказали ему сны наши, и он истолковал нам каждому соответственно с его сновидением;
\vs Gen 41:13 и как он истолковал нам, так и сбылось: я возвращен на место мое, а тот повешен.
\vs Gen 41:14 И послал фараон и позвал Иосифа. И поспешно вывели его из темницы. Он остригся и переменил одежду свою и пришел к фараону.
\vs Gen 41:15 Фараон сказал Иосифу: мне снился сон, и нет никого, кто бы истолковал его, а о тебе я слышал, что ты умеешь толковать сны.
\vs Gen 41:16 И отвечал Иосиф фараону, говоря: это не мое; Бог даст ответ во благо фараону.
\vs Gen 41:17 И сказал фараон Иосифу: мне снилось: вот, стою я на берегу реки;
\vs Gen 41:18 и вот, вышли из реки семь коров тучных плотью и хороших видом и паслись в тростнике;
\vs Gen 41:19 но вот, после них вышли семь коров других, худых, очень дурных видом и тощих плотью: я не видывал во всей земле Египетской таких худых, как они;
\vs Gen 41:20 и съели тощие и худые коровы прежних семь коров тучных;
\vs Gen 41:21 и вошли \bibemph{тучные} в утробу их, но не приметно было, что они вошли в утробу их: они были так же худы видом, как и сначала. И я проснулся.
\vs Gen 41:22 \bibemph{Потом} снилось мне: вот, на одном стебле поднялись семь колосьев полных и хороших;
\vs Gen 41:23 но вот, после них выросло семь колосьев тонких, тощих и иссушенных восточным ветром;
\vs Gen 41:24 и пожрали тощие колосья семь колосьев хороших. Я рассказал это волхвам, но никто не изъяснил мне.
\vs Gen 41:25 И сказал Иосиф фараону: сон фараонов один: чт\acc{о} Бог сделает, т\acc{о} Он возвестил фараону.
\vs Gen 41:26 Семь коров хороших, это семь лет; и семь колосьев хороших, это семь лет: сон один;
\vs Gen 41:27 и семь коров тощих и худых, вышедших после тех, это семь лет, также и семь колосьев тощих и иссушенных восточным ветром, это семь лет голода.
\vs Gen 41:28 Вот почему сказал я фараону: чт\acc{о} Бог сделает, т\acc{о} Он показал фараону.
\vs Gen 41:29 Вот, наступает семь лет великого изобилия во всей земле Египетской;
\vs Gen 41:30 после них настанут семь лет голода, и забудется все то изобилие в земле Египетской, и истощит голод землю,
\vs Gen 41:31 и неприметно будет прежнее изобилие на земле, по причине голода, который последует, ибо он будет очень тяжел.
\vs Gen 41:32 А что сон повторился фараону дважды, \bibemph{это значит}, что сие истинно слово Божие, и что вскоре Бог исполнит сие.
\vs Gen 41:33 И ныне да усмотрит фараон мужа разумного и мудрого и да поставит его над землею Египетскою.
\vs Gen 41:34 Да повелит фараон поставить над землею надзирателей и собирать в семь лет изобилия пятую часть [всех произведений] земли Египетской;
\vs Gen 41:35 пусть они берут всякий хлеб этих наступающих хороших годов и соберут в городах хлеб под ведение фараона в пищу, и пусть берегут;
\vs Gen 41:36 и будет сия пища в запас для земли на семь лет голода, которые будут в земле Египетской, дабы земля не погибла от голода.
\rsbpar\vs Gen 41:37 Сие понравилось фараону и всем слугам его.
\vs Gen 41:38 И сказал фараон слугам своим: найдем ли мы такого, как он, человека, в котором был бы Дух Божий?
\vs Gen 41:39 И сказал фараон Иосифу: так как Бог открыл тебе все сие, то нет столь разумного и мудрого, как ты;
\vs Gen 41:40 ты будешь над домом моим, и твоего слова держаться будет весь народ мой; только престолом я буду больше тебя.
\vs Gen 41:41 И сказал фараон Иосифу: вот, я поставляю тебя над всею землею Египетскою.
\vs Gen 41:42 И снял фараон перстень свой с руки своей и надел его на руку Иосифа; одел его в виссонные одежды, возложил золотую цепь на шею ему;
\vs Gen 41:43 велел везти его на второй из своих колесниц и провозглашать пред ним: преклоняйтесь! И поставил его над всею землею Египетскою.
\vs Gen 41:44 И сказал фараон Иосифу: я фараон; без тебя никто не двинет ни руки своей, ни ноги своей во всей земле Египетской.
\vs Gen 41:45 И нарек фараон Иосифу имя: Цафнаф-панеах, и дал ему в жену Асенефу, дочь Потифера, жреца Илиопольского. И пошел Иосиф по земле Египетской.
\vs Gen 41:46 Иосифу было тридцать лет от рождения, когда он предстал пред лице фараона, царя Египетского. И вышел Иосиф от лица фараонова и прошел по всей земле Египетской.
\vs Gen 41:47 Земля же в семь лет изобилия приносила \bibemph{из зерна} по горсти.
\vs Gen 41:48 И собрал он всякий хлеб семи лет, которые были [плодородны] в земле Египетской, и положил хлеб в городах; в \bibemph{каждом} городе положил хлеб полей, окружающих его.
\vs Gen 41:49 И скопил Иосиф хлеба весьма много, как песку морского, так что перестал и считать, ибо не стало счета.
\rsbpar\vs Gen 41:50 До наступления годов голода, у Иосифа родились два сына, которых родила ему Асенефа, дочь Потифера, жреца Илиопольского.
\vs Gen 41:51 И нарек Иосиф имя первенцу: Манассия, потому что [говорил он] Бог дал мне забыть все несчастья мои и весь дом отца моего.
\vs Gen 41:52 А другому нарек имя: Ефрем, потому что [говорил он] Бог сделал меня плодовитым в земле страдания моего.
\rsbpar\vs Gen 41:53 И прошли семь лет изобилия, которое было в земле Египетской,
\vs Gen 41:54 и наступили семь лет голода, как сказал Иосиф. И был голод во всех землях, а во всей земле Египетской был хлеб.
\vs Gen 41:55 Но когда и вся земля Египетская начала терпеть голод, то народ начал вопиять к фараону о хлебе. И сказал фараон всем Египтянам: пойдите к Иосифу и делайте, что он вам скажет.
\vs Gen 41:56 И был голод по всей земле; и отворил Иосиф все житницы, и стал продавать хлеб Египтянам. Голод же усиливался в земле Египетской.
\vs Gen 41:57 И из всех стран приходили в Египет покупать хлеб у Иосифа, ибо голод усилился по всей земле.
\vs Gen 42:1 И узнал Иаков, что в Египте есть хлеб, и сказал Иаков сыновьям своим: что вы смотрите?
\vs Gen 42:2 И сказал: вот, я слышал, что есть хлеб в Египте; пойдите туда и купите нам оттуда хлеба, чтобы нам жить и не умереть.
\vs Gen 42:3 Десять братьев Иосифовых пошли купить хлеба в Египте,
\vs Gen 42:4 а Вениамина, брата Иосифова, не послал Иаков с братьями его, ибо сказал: не случилось бы с ним беды.
\vs Gen 42:5 И пришли сыны Израилевы покупать хлеб, вместе с другими пришедшими, ибо в земле Ханаанской был голод.
\vs Gen 42:6 Иосиф же был начальником в земле той; он и продавал хлеб всему народу земли. Братья Иосифа пришли и поклонились ему лицем до земли.
\vs Gen 42:7 И увидел Иосиф братьев своих и узнал их; но показал, будто не знает их, и говорил с ними сурово и сказал им: откуда вы пришли? Они сказали: из земли Ханаанской, купить пищи.
\vs Gen 42:8 Иосиф узнал братьев своих, но они не узнали его.
\vs Gen 42:9 И вспомнил Иосиф сны, которые снились ему о них; и сказал им: вы соглядатаи, вы пришли высмотреть наготу\fns{Слабые места.} земли сей.
\vs Gen 42:10 Они сказали ему: нет, господин наш; рабы твои пришли купить пищи;
\vs Gen 42:11 мы все дети одного человека; мы люди честные; рабы твои не бывали соглядатаями.
\vs Gen 42:12 Он сказал им: нет, вы пришли высмотреть наготу земли сей.
\vs Gen 42:13 Они сказали: нас, рабов твоих, двенадцать братьев; мы сыновья одного человека в земле Ханаанской, и вот, меньший теперь с отцом нашим, а одного не стало.
\vs Gen 42:14 И сказал им Иосиф: это самое я и говорил вам, сказав: вы соглядатаи;
\vs Gen 42:15 вот как вы будете испытаны: \bibemph{клянусь} жизнью фараона, вы не выйдете отсюда, если не придет сюда меньший брат ваш;
\vs Gen 42:16 пошлите одного из вас, и пусть он приведет брата вашего, а вы будете задержаны; и откроется, правда ли у вас; и если нет, \bibemph{то клянусь} жизнью фараона, что вы соглядатаи.
\vs Gen 42:17 И отдал их под стражу на три дня.
\vs Gen 42:18 И сказал им Иосиф в третий день: вот что сделайте, и останетесь живы, ибо я боюсь Бога:
\vs Gen 42:19 если вы люди честные, то один брат из вас пусть содержится в доме, где вы заключены; а вы пойдите, отвезите хлеб, ради голода семейств ваших;
\vs Gen 42:20 брата же вашего меньшого приведите ко мне, чтобы оправдались слова ваши и чтобы не умереть вам. Так они и сделали.
\vs Gen 42:21 И говорили они друг другу: точно мы наказываемся за грех против брата нашего; мы видели страдание души его, когда он умолял нас, но не послушали [его]; за то и постигло нас горе сие.
\vs Gen 42:22 Рувим отвечал им и сказал: не говорил ли я вам: не грешите против отрока? но вы не послушались; вот, кровь его взыскивается.
\vs Gen 42:23 А того не знали они, что Иосиф понимает; ибо между ними был переводчик.
\vs Gen 42:24 И отошел от них [Иосиф] и заплакал. И возвратился к ним, и говорил с ними, и, взяв из них Симеона, связал его пред глазами их.
\vs Gen 42:25 И приказал Иосиф наполнить мешки их хлебом, а серебро их возвратить каждому в мешок его, и дать им запасов на дорогу. Так и сделано с ними.
\vs Gen 42:26 Они положили хлеб свой на ослов своих, и пошли оттуда.
\vs Gen 42:27 И открыл один \bibemph{из них} мешок свой, чтобы дать корму ослу своему на ночлеге, и увидел серебро свое в отверстии мешка его,
\vs Gen 42:28 и сказал своим братьям: серебро мое возвращено; вот оно в мешке у меня. И смутилось сердце их, и они с трепетом друг другу говорили: что это Бог сделал с нами?
\vs Gen 42:29 И пришли к Иакову, отцу своему, в землю Ханаанскую и рассказали ему всё случившееся с ними, говоря:
\vs Gen 42:30 начальствующий над тою землею говорил с нами сурово и принял нас за соглядатаев земли той.
\vs Gen 42:31 И сказали мы ему: мы люди честные; мы не бывали соглядатаями;
\vs Gen 42:32 нас двенадцать братьев, сыновей у отца нашего; одного не стало, а меньший теперь с отцом нашим в земле Ханаанской.
\vs Gen 42:33 И сказал нам начальствующий над тою землею: вот как узнаю я, честные ли вы люди: оставьте у меня одного брата из вас, а вы возьмите хлеб ради голода семейств ваших и пойдите,
\vs Gen 42:34 и приведите ко мне меньшого брата вашего; и узнаю я, что вы не соглядатаи, но люди честные; отдам вам брата вашего, и вы можете промышлять в этой земле.
\vs Gen 42:35 Когда же они опорожняли мешки свои, вот, у каждого узел серебра его в мешке его. И увидели они узлы серебра своего, они и отец их, и испугались.
\vs Gen 42:36 И сказал им Иаков, отец их: вы лишили меня детей: Иосифа нет, и Симеона нет, и Вениамина взять хотите,~--- все это на меня!
\vs Gen 42:37 И сказал Рувим отцу своему, говоря: убей двух моих сыновей, если я не приведу его к тебе; отдай его на мои руки; я возвращу его тебе.
\vs Gen 42:38 Он сказал: не пойдет сын мой с вами; потому что брат его умер, и он один остался; если случится с ним несчастье на пути, в который вы пойдете, то сведете вы седину мою с печалью во гроб.
\vs Gen 43:1 Голод усилился на земле.
\vs Gen 43:2 И когда они съели хлеб, который привезли из Египта, тогда отец их сказал им: пойдите опять, купите нам немного пищи.
\vs Gen 43:3 И сказал ему Иуда, говоря: тот человек решительно объявил нам, сказав: не являйтесь ко мне на лице, если брата вашего не будет с вами.
\vs Gen 43:4 Если пошлешь с нами брата нашего, то пойдем и купим тебе пищи,
\vs Gen 43:5 а если не пошлешь, то не пойдем, ибо тот человек сказал нам: не являйтесь ко мне на лице, если брата вашего не будет с вами.
\vs Gen 43:6 Израиль сказал: для чего вы сделали мне такое зло, сказав тому человеку, что у вас есть еще брат?
\vs Gen 43:7 Они сказали: расспрашивал тот человек о нас и о родстве нашем, говоря: жив ли еще отец ваш? есть ли у вас брат? Мы и рассказали ему по этим расспросам. Могли ли мы знать, что он скажет: приведите брата вашего?
\vs Gen 43:8 Иуда же сказал Израилю, отцу своему: отпусти отрока со мною, и мы встанем и пойдем, и живы будем и не умрем и мы, и ты, и дети наши;
\vs Gen 43:9 я отвечаю за него, из моих рук потребуешь его; если я не приведу его к тебе и не поставлю его пред лицем твоим, то останусь я виновным пред тобою во все дни жизни;
\vs Gen 43:10 если бы мы не медлили, то уже сходили бы два раза.
\vs Gen 43:11 Израиль, отец их, сказал им: если так, то вот что сделайте: возьмите с собою плодов земли сей и отнесите в дар тому человеку несколько бальзама и несколько меду, стираксы и ладану, фисташков и миндальных орехов;
\vs Gen 43:12 возьмите и другое серебро в руки ваши; а серебро, обратно положенное в отверстие мешков ваших, возвратите руками вашими: может быть, это недосмотр;
\vs Gen 43:13 и брата вашего возьмите и, встав, пойдите опять к человеку тому;
\vs Gen 43:14 Бог же Всемогущий да даст вам найти милость у человека того, чтобы он отпустил вам и другого брата вашего и Вениамина, а мне если уже быть бездетным, то пусть буду бездетным.
\vs Gen 43:15 И взяли те люди дары эти, и серебра вдвое взяли в руки свои, и Вениамина, и встали, пошли в Египет и предстали пред лице Иосифа.
\vs Gen 43:16 Иосиф, увидев между ними Вениамина [брата своего, сына матери своей], сказал начальнику дома своего: введи сих людей в дом и заколи что-нибудь из скота, и приготовь, потому что со мною будут есть эти люди в полдень.
\vs Gen 43:17 И сделал человек тот, как сказал Иосиф, и ввел человек тот людей сих в дом Иосифов.
\vs Gen 43:18 И испугались люди эти, что ввели их в дом Иосифов, и сказали: это за серебро, возвращенное прежде в мешки наши, ввели нас, чтобы придраться к нам и напасть на нас, и взять нас в рабство, и ослов наших.
\vs Gen 43:19 И подошли они к начальнику дома Иосифова, и стали говорить ему у дверей дома,
\vs Gen 43:20 и сказали: послушай, господин наш, мы приходили уже прежде покупать пищи,
\vs Gen 43:21 и случилось, что, когда пришли мы на ночлег и открыли мешки наши,~--- вот серебро каждого в отверстии мешка его, серебро наше по весу его, и мы возвращаем его своими руками;
\vs Gen 43:22 а для покупки пищи мы принесли другое серебро в руках наших, мы не знаем, кто положил серебро наше в мешки наши.
\vs Gen 43:23 Он сказал: будьте спокойны, не бойтесь; Бог ваш и Бог отца вашего дал вам клад в мешках ваших; серебро ваше дошло до меня. И привел к ним Симеона.
\vs Gen 43:24 И ввел тот человек людей сих в дом Иосифов и дал воды, и они омыли ноги свои; и дал корму ослам их.
\vs Gen 43:25 И они приготовили дары к приходу Иосифа в полдень, ибо слышали, что там будут есть хлеб.
\vs Gen 43:26 И пришел Иосиф домой; и они принесли ему в дом дары, которые были на руках их, и поклонились ему до земли.
\vs Gen 43:27 Он спросил их о здоровье и сказал: здоров ли отец ваш старец, о котором вы говорили? жив ли еще он?
\vs Gen 43:28 Они сказали: здоров раб твой, отец наш; еще жив. [Он сказал: благословен человек сей от Бога.] И преклонились они и поклонились.
\vs Gen 43:29 И поднял глаза свои [Иосиф], и увидел Вениамина, брата своего, сына матери своей, и сказал: это брат ваш меньший, о котором вы сказывали мне? И сказал: да будет милость Божия с тобою, сын мой!
\vs Gen 43:30 И поспешно удалился Иосиф, потому что воскипела любовь к брату его, и он готов был заплакать, и вошел он во внутреннюю комнату и плакал там.
\vs Gen 43:31 И умыв лице свое, вышел, и скрепился и сказал: подавайте кушанье.
\vs Gen 43:32 И подали ему особо, и им особо, и Египтянам, обедавшим с ним, особо, ибо Египтяне не могут есть с Евреями, потому что это мерзость для Египтян.
\vs Gen 43:33 И сели они пред ним, первородный по первородству его, и младший по молодости его, и дивились эти люди друг пред другом.
\vs Gen 43:34 И посылались им кушанья от него, и доля Вениамина была впятеро больше долей каждого из них. И пили, и довольно пили они с ним.
\vs Gen 44:1 И приказал [Иосиф] начальнику дома своего, говоря: наполни мешки этих людей пищею, сколько они могут нести, и серебро каждого положи в отверстие мешка его,
\vs Gen 44:2 а чашу мою, чашу серебряную, положи в отверстие мешка к младшему вместе с серебром за купленный им хлеб. И сделал тот по слову Иосифа, которое сказал он.
\vs Gen 44:3 Утром, когда рассвело, эти люди были отпущены, они и ослы их.
\vs Gen 44:4 Еще не далеко отошли они от города, как Иосиф сказал начальнику дома своего: ступай, догоняй этих людей и, когда догонишь, скажи им: для чего вы заплатили злом за добро? [для чего украли у меня серебряную чашу?]
\vs Gen 44:5 Не та ли это, из которой пьет господин мой и он гадает на ней? Худо это вы сделали.
\vs Gen 44:6 Он догнал их и сказал им эти слова.
\vs Gen 44:7 Они сказали ему: для чего господин наш говорит такие слова? Нет, рабы твои не сделают такого дела.
\vs Gen 44:8 Вот, серебро, найденное нами в отверстии мешков наших, мы обратно принесли тебе из земли Ханаанской: как же нам украсть из дома господина твоего серебро или золото?
\vs Gen 44:9 У кого из рабов твоих найдется [чаша], тому смерть, и мы будем рабами господину нашему.
\vs Gen 44:10 Он сказал: хорошо; как вы сказали, так пусть и будет: у кого найдется [чаша], тот будет мне рабом, а вы будете не виноваты.
\vs Gen 44:11 Они поспешно спустили каждый свой мешок на землю и открыли каждый свой мешок.
\vs Gen 44:12 Он обыскал, начал со старшего и окончил младшим; и нашлась чаша в мешке Вениаминовом.
\vs Gen 44:13 И разодрали они одежды свои, и, возложив каждый на осла своего ношу, возвратились в город.
\vs Gen 44:14 И пришли Иуда и братья его в дом Иосифа, который был еще дома, и пали пред ним на землю.
\vs Gen 44:15 Иосиф сказал им: что это вы сделали? разве вы не знали, что такой человек, как я, конечно угадает?
\vs Gen 44:16 Иуда сказал: что нам сказать господину нашему? что говорить? чем оправдываться? Бог нашел неправду рабов твоих; вот, мы рабы господину нашему, и мы, и тот, в чьих руках нашлась чаша.
\vs Gen 44:17 Но [Иосиф] сказал: нет, я этого не сделаю; тот, в чьих руках нашлась чаша, будет мне рабом, а вы пойдите с миром к отцу вашему.
\vs Gen 44:18 И подошел Иуда к нему и сказал: господин мой, позволь рабу твоему сказать слово в уши господина моего, и не прогневайся на раба твоего, ибо ты то же, что фараон.
\vs Gen 44:19 Господин мой спрашивал рабов своих, говоря: есть ли у вас отец или брат?
\vs Gen 44:20 Мы сказали господину нашему, что у нас есть отец престарелый, и [у него] младший сын, сын старости, которого брат умер, а он остался один \bibemph{от} матери своей, и отец любит его.
\vs Gen 44:21 Ты же сказал рабам твоим: приведите его ко мне, чтобы мне взглянуть на него.
\vs Gen 44:22 Мы сказали господину нашему: отрок не может оставить отца своего, и если он оставит отца своего, то сей умрет.
\vs Gen 44:23 Но ты сказал рабам твоим: если не придет с вами меньший брат ваш, то вы более не являйтесь ко мне на лице.
\vs Gen 44:24 Когда мы пришли к рабу твоему, отцу нашему, то пересказали ему слова господина моего.
\vs Gen 44:25 И сказал отец наш: пойдите опять, купите нам немного пищи.
\vs Gen 44:26 Мы сказали: нельзя нам идти; а если будет с нами меньший брат наш, то пойдем; потому что нельзя нам видеть лица того человека, если не будет с нами меньшого брата нашего.
\vs Gen 44:27 И сказал нам раб твой, отец наш: вы знаете, что жена моя родила мне двух \bibemph{сынов};
\vs Gen 44:28 один пошел от меня, и я сказал: верно он растерзан; и я не видал его доныне;
\vs Gen 44:29 если и сего возьмете от глаз моих, и случится с ним несчастье, то сведете вы седину мою с горестью во гроб.
\vs Gen 44:30 Теперь если я приду к рабу твоему, отцу нашему, и не будет с нами отрока, с душею которого связана душа его,
\vs Gen 44:31 то он, увидев, что нет отрока, умрет; и сведут рабы твои седину раба твоего, отца нашего, с печалью во гроб.
\vs Gen 44:32 Притом я, раб твой, взялся отвечать за отрока отцу моему, сказав: если не приведу его к тебе [и не поставлю его пред тобою], то останусь я виновным пред отцом моим во все дни жизни.
\vs Gen 44:33 Итак пусть я, раб твой, вместо отрока останусь рабом у господина моего, а отрок пусть идет с братьями своими:
\vs Gen 44:34 ибо как пойду я к отцу моему, когда отрока не будет со мною? я увидел бы бедствие, которое постигло бы отца моего.
\vs Gen 45:1 Иосиф не мог более удерживаться при всех стоявших около него и закричал: удалите от меня всех. И не оставалось при Иосифе никого, когда он открылся братьям своим.
\vs Gen 45:2 И громко зарыдал он, и услышали Египтяне, и услышал дом фараонов.
\vs Gen 45:3 И сказал Иосиф братьям своим: я~--- Иосиф, жив ли еще отец мой? Но братья его не могли отвечать ему, потому что они смутились пред ним.
\vs Gen 45:4 И сказал Иосиф братьям своим: подойдите ко мне. Они подошли. Он сказал: я~--- Иосиф, брат ваш, которого вы продали в Египет;
\vs Gen 45:5 но теперь не печальтесь и не жалейте о том, что вы продали меня сюда, потому что Бог послал меня перед вами для сохранения вашей жизни;
\vs Gen 45:6 ибо теперь два года голода на земле: [остается] еще пять лет, в которые ни орать, ни жать не будут;
\vs Gen 45:7 Бог послал меня перед вами, чтобы оставить вас на земле и сохранить вашу жизнь великим избавлением.
\vs Gen 45:8 Итак не вы послали меня сюда, но Бог, Который и поставил меня отцом фараону и господином во всем доме его и владыкою во всей земле Египетской.
\vs Gen 45:9 Идите скорее к отцу моему и скажите ему: так говорит сын твой Иосиф: Бог поставил меня господином над всем Египтом; приди ко мне, не медли;
\vs Gen 45:10 ты будешь жить в земле Гесем; и будешь близ меня, ты, и сыны твои, и сыны сынов твоих, и мелкий и крупный скот твой, и все твое;
\vs Gen 45:11 и прокормлю тебя там, ибо голод будет еще пять лет, чтобы не обнищал ты и дом твой и все твое.
\vs Gen 45:12 И вот, очи ваши и очи брата моего Вениамина видят, что это мои уста говорят с вами;
\vs Gen 45:13 скажите же отцу моему о всей славе моей в Египте и о всем, что вы видели, и приведите скорее отца моего сюда.
\vs Gen 45:14 И пал он на шею Вениамину, брату своему, и плакал; и Вениамин плакал на шее его.
\vs Gen 45:15 И целовал всех братьев своих и плакал, обнимая их. Потом говорили с ним братья его.
\rsbpar\vs Gen 45:16 Дошел в дом фараона слух, что пришли братья Иосифа; и приятно было фараону и рабам его.
\vs Gen 45:17 И сказал фараон Иосифу: скажи братьям твоим: вот что сделайте: навьючьте скот ваш [хлебом] и ступайте в землю Ханаанскую;
\vs Gen 45:18 и возьмите отца вашего и семейства ваши и придите ко мне; я дам вам лучшее [место] в земле Египетской, и вы будете есть тук земли.
\vs Gen 45:19 Тебе же повелеваю сказать им: сделайте сие: возьмите себе из земли Египетской колесниц для детей ваших и для жен ваших, и привезите отца вашего и придите;
\vs Gen 45:20 и не жалейте вещей ваших, ибо лучшее из всей земли Египетской \bibemph{дам} вам.
\rsbpar\vs Gen 45:21 Так и сделали сыны Израилевы. И дал им Иосиф колесницы по приказанию фараона, и дал им путевой запас,
\vs Gen 45:22 каждому из них он дал перемену одежд, а Вениамину дал триста сребреников и пять перемен одежд;
\vs Gen 45:23 также и отцу своему послал десять ослов, навьюченных лучшими \bibemph{произведениями} Египетскими, и десять ослиц, навьюченных зерном, хлебом и припасами отцу своему на путь.
\vs Gen 45:24 И отпустил братьев своих, и они пошли. И сказал им: не ссорьтесь на дороге.
\vs Gen 45:25 И пошли они из Египта, и пришли в землю Ханаанскую к Иакову, отцу своему,
\vs Gen 45:26 и известили его, сказав: Иосиф [сын твой] жив и теперь владычествует над всею землею Египетскою. Но сердце его смутилось, ибо он не верил им.
\vs Gen 45:27 Когда же они пересказали ему все слова Иосифа, которые он говорил им, и когда увидел колесницы, которые прислал Иосиф, чтобы везти его, тогда ожил дух Иакова, отца их,
\vs Gen 45:28 и сказал Израиль: довольно [сего для меня], еще жив сын мой Иосиф; пойду и увижу его, пока не умру.
\vs Gen 46:1 И отправился Израиль со всем, что у него было, и пришел в Вирсавию, и принес жертвы Богу отца своего Исаака.
\vs Gen 46:2 И сказал Бог Израилю в видении ночном: Иаков! Иаков! Он сказал: вот я.
\vs Gen 46:3 \bibemph{Бог} сказал: Я Бог, Бог отца твоего; не бойся идти в Египет, ибо там произведу от тебя народ великий;
\vs Gen 46:4 Я пойду с тобою в Египет, Я и выведу тебя обратно. Иосиф своею рукою закроет глаза твои.
\rsbpar\vs Gen 46:5 Иаков отправился из Вирсавии; и повезли сыны Израилевы Иакова отца своего, и детей своих, и жен своих на колесницах, которые послал фараон, чтобы привезти его.
\vs Gen 46:6 И взяли они скот свой и имущество свое, которое приобрели в земле Ханаанской, и пришли в Египет,~--- Иаков и весь род его с ним.
\vs Gen 46:7 Сынов своих и внуков своих с собою, дочерей своих и внучек своих и весь род свой привел он с собою в Египет.
\rsbpar\vs Gen 46:8 Вот имена сынов Израилевых, пришедших в Египет: Иаков и сыновья его. Первенец Иакова Рувим.
\vs Gen 46:9 Сыны Рувима: Ханох и Фаллу, Хецрон и Харми.
\vs Gen 46:10 Сыны Симеона: Иемуил и Иамин, и Огад, и Иахин, и Цохар, и Саул, сын Хананеянки.
\vs Gen 46:11 Сыны Левия: Гирсон, Кааф и Мерари.
\vs Gen 46:12 Сыны Иуды: Ир и Онан, и Шела, и Фарес, и Зара; но Ир и Онан умерли в земле Ханаанской. Сыны Фареса были: Есром и Хамул.
\vs Gen 46:13 Сыны Иссахара: Фола и Фува, Иов и Шимрон.
\vs Gen 46:14 Сыны Завулона: Серед и Елон, и Иахлеил.
\vs Gen 46:15 Это сыны Лии, которых она родила Иакову в Месопотамии, и Дину, дочь его. Всех душ сынов его и дочерей его~--- тридцать три.
\vs Gen 46:16 Сыны Гада: Цифион и Хагги, Шуни и Эцбон, Ери и Ароди и Арели.
\vs Gen 46:17 Сыны Асира: Имна и Ишва, и Ишви, и Бриа, и Серах, сестра их. Сыны Брии: Хевер и Малхиил.
\vs Gen 46:18 Это сыны Зелфы, которую Лаван дал Лии, дочери своей; она родила их Иакову шестнадцать душ.
\vs Gen 46:19 Сыны Рахили, жены Иакова: Иосиф и Вениамин.
\vs Gen 46:20 И родились у Иосифа в земле Египетской Манассия и Ефрем, которых родила ему Асенефа, дочь Потифера, жреца Илиопольского.
\vs Gen 46:21 Сыны Вениамина: Бела и Бехер и Ашбел; [сыны Белы были:] Гера и Нааман, Эхи и Рош, Муппим и Хуппим и Ард.
\vs Gen 46:22 Это сыны Рахили, которые родились у Иакова, всего четырнадцать душ.
\vs Gen 46:23 Сын Дана: Хушим.
\vs Gen 46:24 Сыны Неффалима: Иахцеил и Гуни, и Иецер, и Шиллем.
\vs Gen 46:25 Это сыны Валлы, которую дал Лаван дочери своей Рахили; она родила их Иакову всего семь душ.
\vs Gen 46:26 Всех душ, пришедших с Иаковом в Египет, которые произошли из чресл его, кроме жен сынов Иаковлевых, всего шестьдесят шесть душ.
\vs Gen 46:27 Сынов Иосифа, которые родились у него в Египте, две души. Всех душ дома Иаковлева, перешедших [с Иаковом] в Египет, семьдесят [пять].
\vs Gen 46:28 Иуду послал он пред собою к Иосифу, чтобы он указал \bibemph{путь} в Гесем. И пришли в землю Гесем.
\vs Gen 46:29 Иосиф запряг колесницу свою и выехал навстречу Израилю, отцу своему, в Гесем, и, увидев его, пал на шею его, и долго плакал на шее его.
\vs Gen 46:30 И сказал Израиль Иосифу: умру я теперь, увидев лице твое, ибо ты еще жив.
\vs Gen 46:31 И сказал Иосиф братьям своим и дому отца своего: я пойду, извещу фараона и скажу ему: братья мои и дом отца моего, которые были в земле Ханаанской, пришли ко мне;
\vs Gen 46:32 эти люди пастухи овец, ибо скотоводы они; и мелкий и крупный скот свой, и все, что у них, привели они.
\vs Gen 46:33 Если фараон призовет вас и скажет: какое занятие ваше?
\vs Gen 46:34 то вы скажите: \bibemph{мы}, рабы твои, скотоводами были от юности нашей доныне, и мы и отцы наши, чтобы вас поселили в земле Гесем. Ибо мерзость для Египтян всякий пастух овец.
\vs Gen 47:1 И пришел Иосиф и известил фараона и сказал: отец мой и братья мои, с мелким и крупным скотом своим и со всем, что у них, пришли из земли Ханаанской; и вот, они в земле Гесем.
\vs Gen 47:2 И из братьев своих он взял пять человек и представил их фараону.
\vs Gen 47:3 И сказал фараон братьям его: какое ваше занятие? Они сказали фараону: пастухи овец рабы твои, и мы и отцы наши.
\vs Gen 47:4 И сказали они фараону: мы пришли пожить в этой земле, потому что нет пажити для скота рабов твоих, ибо в земле Ханаанской сильный голод; итак позволь поселиться рабам твоим в земле Гесем.
\vs Gen 47:5 И сказал фараон Иосифу: отец твой и братья твои пришли к тебе;
\vs Gen 47:6 земля Египетская пред тобою; на лучшем месте земли посели отца твоего и братьев твоих; пусть живут они в земле Гесем; и если знаешь, что между ними есть способные люди, поставь их смотрителями над моим скотом.
\vs Gen 47:7 И привел Иосиф Иакова, отца своего, и представил его фараону; и благословил Иаков фараона.
\vs Gen 47:8 Фараон сказал Иакову: сколько лет жизни твоей?
\vs Gen 47:9 Иаков сказал фараону: дней странствования моего сто тридцать лет; малы и несчастны дни жизни моей и не достигли до лет жизни отцов моих во днях странствования их.
\vs Gen 47:10 И благословил фараона Иаков и вышел от фараона.
\vs Gen 47:11 И поселил Иосиф отца своего и братьев своих, и дал им владение в земле Египетской, в лучшей части земли, в земле Раамсес, как повелел фараон.
\vs Gen 47:12 И снабжал Иосиф отца своего и братьев своих и весь дом отца своего хлебом, по потребностям каждого семейства.
\rsbpar\vs Gen 47:13 И не было хлеба по всей земле, потому что голод весьма усилился, и изнурены были от голода земля Египетская и земля Ханаанская.
\vs Gen 47:14 Иосиф собрал все серебро, какое было в земле Египетской и в земле Ханаанской, за хлеб, который покупали, и внес Иосиф серебро в дом фараонов.
\vs Gen 47:15 И серебро истощилось в земле Египетской и в земле Ханаанской. Все Египтяне пришли к Иосифу и говорили: дай нам хлеба; зачем нам умирать пред тобою, потому что серебро вышло у нас?
\vs Gen 47:16 Иосиф сказал: пригоняйте скот ваш, и я буду давать вам [хлеб] за скот ваш, если серебро вышло у вас.
\vs Gen 47:17 И пригоняли они к Иосифу скот свой; и давал им Иосиф хлеб за лошадей, и за стада мелкого скота, и за стада крупного скота, и за ослов; и снабжал их хлебом в тот год за весь скот их.
\vs Gen 47:18 И прошел этот год; и пришли к нему на другой год и сказали ему: не скроем от господина нашего, что серебро истощилось и стада скота нашего у господина нашего; ничего не осталось у нас пред господином нашим, кроме тел наших и земель наших;
\vs Gen 47:19 для чего нам погибать в глазах твоих, и нам и землям нашим? купи нас и земли наши за хлеб, и мы с землями нашими будем рабами фараону, а ты дай нам семян, чтобы нам быть живыми и не умереть, и чтобы не опустела земля.
\vs Gen 47:20 И купил Иосиф всю землю Египетскую для фараона, потому что продали Египтяне каждый свое поле, ибо голод одолевал их. И досталась земля фараону.
\vs Gen 47:21 И народ сделал он рабами от одного конца Египта до другого.
\vs Gen 47:22 Только земли жрецов не купил [Иосиф], ибо жрецам от фараона положен был участок, и они питались своим участком, который дал им фараон; посему и не продали земли своей.
\vs Gen 47:23 И сказал Иосиф народу: вот, я купил теперь для фараона вас и землю вашу; вот вам семена, и засевайте землю;
\vs Gen 47:24 когда будет жатва, давайте пятую часть фараону, а четыре части останутся вам на засеяние полей, на пропитание вам и тем, кто в домах ваших, и на пропитание детям вашим.
\vs Gen 47:25 Они сказали: ты спас нам жизнь; да обретем милость в очах господина нашего и да будем рабами фараону.
\vs Gen 47:26 И поставил Иосиф в закон земле Египетской, даже до сего дня: пятую часть давать фараону, исключая только землю жрецов, которая не принадлежала фараону.
\rsbpar\vs Gen 47:27 И жил Израиль в земле Египетской, в земле Гесем, и владели они ею, и плодились, и весьма умножились.
\vs Gen 47:28 И жил Иаков в земле Египетской семнадцать лет; и было дней Иакова, годов жизни его, сто сорок семь лет.
\vs Gen 47:29 И пришло время Израилю умереть, и призвал он сына своего Иосифа и сказал ему: если я нашел благоволение в очах твоих, положи руку твою под стегно мое и \bibemph{клянись}, что ты окажешь мне милость и правду, не похоронишь меня в Египте,
\vs Gen 47:30 дабы мне лечь с отцами моими; вынесешь меня из Египта и похоронишь меня в их гробнице. \bibemph{Иосиф} сказал: сделаю по слову твоему.
\vs Gen 47:31 И сказал: клянись мне. И клялся ему. И поклонился Израиль на возглавие постели\fns{По переводу 70-ти: на верх жезла его.}.
\vs Gen 48:1 После того Иосифу сказали: вот, отец твой болен. И он взял с собою двух сынов своих, Манассию и Ефрема [и пошел к Иакову].
\vs Gen 48:2 Иакова известили и сказали: вот, сын твой Иосиф идет к тебе. Израиль собрал силы свои и сел на постели.
\vs Gen 48:3 И сказал Иаков Иосифу: Бог Всемогущий явился мне в Лузе, в земле Ханаанской, и благословил меня,
\vs Gen 48:4 и сказал мне: вот, Я распложу тебя, и размножу тебя, и произведу от тебя множество народов, и дам землю сию потомству твоему после тебя, в вечное владение.
\vs Gen 48:5 И ныне два сына твои, родившиеся тебе в земле Египетской, до моего прибытия к тебе в Египет, мои они; Ефрем и Манассия, как Рувим и Симеон, будут мои;
\vs Gen 48:6 дети же твои, которые родятся от тебя после них, будут твои; они под именем братьев своих будут именоваться в их уделе.
\vs Gen 48:7 Когда я шел из Месопотамии, умерла у меня Рахиль [мать твоя] в земле Ханаанской, по дороге, не доходя несколько до Ефрафы, и я похоронил ее там на дороге к Ефрафе, что \bibemph{ныне} Вифлеем.
\vs Gen 48:8 И увидел Израиль сыновей Иосифа и сказал: кто это?
\vs Gen 48:9 И сказал Иосиф отцу своему: это сыновья мои, которых Бог дал мне здесь. [Иаков] сказал: подведи их ко мне, и я благословлю их.
\vs Gen 48:10 Глаза же Израилевы притупились от старости; не мог он видеть \bibemph{ясно. Иосиф} подвел их к нему, и он поцеловал их и обнял их.
\vs Gen 48:11 И сказал Израиль Иосифу: не надеялся я видеть твое лице; но вот, Бог показал мне и детей твоих.
\vs Gen 48:12 И отвел их Иосиф от колен его и поклонился ему лицем своим до земли.
\vs Gen 48:13 И взял Иосиф обоих [сыновей своих], Ефрема в правую свою руку против левой Израиля, а Манассию в левую против правой Израиля, и подвел к нему.
\vs Gen 48:14 Но Израиль простер правую руку свою и положил на голову Ефрему, хотя сей был меньший, а левую на голову Манассии. С намерением положил он так руки свои, хотя Манассия был первенец.
\vs Gen 48:15 И благословил Иосифа и сказал: Бог, пред Которым ходили отцы мои Авраам и Исаак, Бог, пасущий меня с тех пор, как я существую, до сего дня,
\vs Gen 48:16 Ангел, избавляющий меня от всякого зла, да благословит отроков сих; да будет на них наречено имя мое и имя отцов моих Авраама и Исаака, и да возрастут они во множество посреди земли.
\vs Gen 48:17 И увидел Иосиф, что отец его положил правую руку свою на голову Ефрема; и прискорбно было ему это. И взял он руку отца своего, чтобы переложить ее с головы Ефрема на голову Манассии,
\vs Gen 48:18 и сказал Иосиф отцу своему: не так, отец мой, ибо это~--- первенец; положи на его голову правую руку твою.
\vs Gen 48:19 Но отец его не согласился и сказал: знаю, сын мой, знаю; и от него произойдет народ, и он будет велик; но меньший его брат будет больше его, и от семени его произойдет многочисленный народ.
\vs Gen 48:20 И благословил их в тот день, говоря: тобою будет благословлять Израиль, говоря: Бог да сотворит тебе, как Ефрему и Манассии. И поставил Ефрема выше Манассии.
\vs Gen 48:21 И сказал Израиль Иосифу: вот, я умираю; и Бог будет с вами и возвратит вас в землю отцов ваших;
\vs Gen 48:22 я даю тебе, преимущественно пред братьями твоими, один участок, который я взял из рук Аморреев мечом моим и луком моим.
\vs Gen 49:1 И призвал Иаков сыновей своих и сказал: соберитесь, и я возвещу вам, чт\acc{о} будет с вами в грядущие дни;
\vs Gen 49:2 сойдитесь и послушайте, сыны Иакова, послушайте Израиля, отца вашего.
\rsbpar\vs Gen 49:3 Рувим, первенец мой! ты~--- крепость моя и начаток силы моей, верх достоинства и верх могущества;
\vs Gen 49:4 но ты бушевал, как вода,~--- не будешь преимуществовать, ибо ты взошел на ложе отца твоего, ты осквернил постель мою, [на которую] взошел.
\rsbpar\vs Gen 49:5 Симеон и Левий братья, орудия жестокости мечи их;
\vs Gen 49:6 в совет их да не внидет душа моя, и к собранию их да не приобщится слава моя, ибо они во гневе своем убили мужа и по прихоти своей перерезали жилы тельца;
\vs Gen 49:7 проклят гнев их, ибо жесток, и ярость их, ибо свирепа; разделю их в Иакове и рассею их в Израиле.
\rsbpar\vs Gen 49:8 Иуда! тебя восхвалят братья твои. Рука твоя на хребте врагов твоих; поклонятся тебе сыны отца твоего.
\vs Gen 49:9 Молодой лев Иуда, с добычи, сын мой, поднимается. Преклонился он, лег, как лев и как львица: кто поднимет его?
\vs Gen 49:10 Не отойдет скипетр от Иуды и законодатель от чресл его, доколе не приидет Примиритель, и Ему покорность народов.
\vs Gen 49:11 Он привязывает к виноградной лозе осленка своего и к лозе лучшего винограда сына ослицы своей; моет в вине одежду свою и в крови гроздов одеяние свое;
\vs Gen 49:12 блестящи очи [его] от вина, и белы зубы [его] от молока.
\rsbpar\vs Gen 49:13 Завулон при береге морском будет жить и у пристани корабельной, и предел его до Сидона.
\rsbpar\vs Gen 49:14 Иссахар осел крепкий, лежащий между протоками вод;
\vs Gen 49:15 и увидел он, что покой хорош, и что земля приятна: и преклонил плечи свои для ношения бремени и стал работать в уплату дани.
\rsbpar\vs Gen 49:16 Дан будет судить народ свой, как одно из колен Израиля;
\vs Gen 49:17 Дан будет змеем на дороге, аспидом на пути, уязвляющим ногу коня, так что всадник его упадет назад.
\vs Gen 49:18 На помощь твою надеюсь, Господи!
\rsbpar\vs Gen 49:19 Гад,~--- толпа будет теснить его, но он оттеснит ее по пятам.
\rsbpar\vs Gen 49:20 Для Асира~--- слишком тучен хлеб его, и он будет доставлять царские яства.
\rsbpar\vs Gen 49:21 Неффалим~--- теревинф рослый, распускающий прекрасные ветви\fns{По другому чтению: Неффалим~--- серна стройная; он говорит прекрасные изречения.}.
\rsbpar\vs Gen 49:22 Иосиф~--- отрасль плодоносного \bibemph{дерева}, отрасль плодоносного \bibemph{дерева} над источником; ветви его простираются над стеною;
\vs Gen 49:23 огорчали его, и стреляли и враждовали на него стрельцы,
\vs Gen 49:24 но тверд остался лук его, и крепки мышцы рук его, от рук мощного \bibemph{Бога} Иаковлева. Оттуда Пастырь и твердыня Израилева,
\vs Gen 49:25 от Бога отца твоего, \bibemph{Который} и да поможет тебе, и от Всемогущего, Который и да благословит тебя благословениями небесными свыше, благословениями бездны, лежащей долу, благословениями сосцов и утробы,
\vs Gen 49:26 благословениями отца твоего, которые превышают благословения гор древних и приятности холмов вечных; да будут они на голове Иосифа и на темени избранного между братьями своими.
\rsbpar\vs Gen 49:27 Вениамин, хищный волк, утром будет есть ловитву и вечером будет делить добычу.
\rsbpar\vs Gen 49:28 Вот все двенадцать колен Израилевых; и вот что сказал им отец их; и благословил их, и дал им благословение, каждому свое.
\vs Gen 49:29 И заповедал он им и сказал им: я прилагаюсь к народу моему; похороните меня с отцами моими в пещере, которая на поле Ефрона Хеттеянина,
\vs Gen 49:30 в пещере, которая на поле Махпела, что пред Мамре, в земле Ханаанской, которую [пещеру] купил Авраам с полем у Ефрона Хеттеянина в собственность для погребения;
\vs Gen 49:31 там похоронили Авраама и Сарру, жену его; там похоронили Исаака и Ревекку, жену его; и там похоронил я Лию;
\vs Gen 49:32 это поле и пещера, которая на нем, куплена у сынов Хеттеевых.
\vs Gen 49:33 И окончил Иаков завещание сыновьям своим, и положил ноги свои на постель, и скончался, и приложился к народу своему.
\vs Gen 50:1 Иосиф пал на лице отца своего, и плакал над ним, и целовал его.
\vs Gen 50:2 И повелел Иосиф слугам своим~--- врачам, бальзамировать отца его; и врачи набальзамировали Израиля.
\vs Gen 50:3 И исполнилось ему сорок дней, ибо столько дней употребляется на бальзамирование, и оплакивали его Египтяне семьдесят дней.
\vs Gen 50:4 Когда же прошли дни плача по нем, Иосиф сказал придворным фараона, говоря: если я обрел благоволение в очах ваших, то скажите фараону так:
\vs Gen 50:5 отец мой заклял меня, сказав: вот, я умираю; во гробе моем, который я выкопал себе в земле Ханаанской, там похорони меня. И теперь хотел бы я пойти и похоронить отца моего и возвратиться. [Слова Иосифа пересказали фараону.]
\vs Gen 50:6 И сказал фараон: пойди и похорони отца твоего, как он заклял тебя.
\vs Gen 50:7 И пошел Иосиф хоронить отца своего. И пошли с ним все слуги фараона, старейшины дома его и все старейшины земли Египетской,
\vs Gen 50:8 и весь дом Иосифа, и братья его, и дом отца его. Только детей своих и мелкий и крупный скот свой оставили в земле Гесем.
\vs Gen 50:9 С ним отправились также колесницы и всадники, так что сонм был весьма велик.
\vs Gen 50:10 И дошли они до Горен-гаатада при Иордане и плакали там плачем великим и весьма сильным; и сделал \bibemph{Иосиф} плач по отце своем семь дней.
\vs Gen 50:11 И видели жители земли той, Хананеи, плач в Горен-гаатаде, и сказали: велик плач этот у Египтян! Посему наречено имя [месту] тому: плач Египтян, что при Иордане.
\vs Gen 50:12 И сделали сыновья \bibemph{Иакова} с ним, как он заповедал им;
\vs Gen 50:13 и отнесли его сыновья его в землю Ханаанскую и похоронили его в пещере на поле Махпела, которую купил Авраам с полем в собственность для погребения у Ефрона Хеттеянина, пред Мамре.
\rsbpar\vs Gen 50:14 И возвратился Иосиф в Египет, сам и братья его и все ходившие с ним хоронить отца его, после погребения им отца своего.
\vs Gen 50:15 И увидели братья Иосифовы, что умер отец их, и сказали: что, если Иосиф возненавидит нас и захочет отмстить нам за всё зло, которое мы ему сделали?
\vs Gen 50:16 И послали они сказать Иосифу: отец твой пред смертью своею завещал, говоря:
\vs Gen 50:17 так скажите Иосифу: прости братьям твоим вину и грех их, так как они сделали тебе зло. И ныне прости вины рабов Бога отца твоего. Иосиф плакал, когда ему говорили это.
\vs Gen 50:18 Пришли и сами братья его, и пали пред лицем его, и сказали: вот, мы рабы тебе.
\vs Gen 50:19 И сказал Иосиф: не бойтесь, ибо я боюсь Бога;
\vs Gen 50:20 вот, вы умышляли против меня зло; но Бог обратил это в добро, чтобы сделать то, что теперь есть: сохранить жизнь великому числу людей;
\vs Gen 50:21 итак не бойтесь: я буду питать вас и детей ваших. И успокоил их и говорил по сердцу их.
\rsbpar\vs Gen 50:22 И жил Иосиф в Египте сам и дом отца его; жил же Иосиф всего сто десять лет.
\vs Gen 50:23 И видел Иосиф детей у Ефрема до третьего рода, также и сыновья Махира, сына Манассиина, родились на колени Иосифа.
\vs Gen 50:24 И сказал Иосиф братьям своим: я умираю, но Бог посетит вас и выведет вас из земли сей в землю, о которой клялся Аврааму, Исааку и Иакову.
\vs Gen 50:25 И заклял Иосиф сынов Израилевых, говоря: Бог посетит вас, и вынесите кости мои отсюда.
\vs Gen 50:26 И умер Иосиф ста десяти лет. И набальзамировали его и положили в ковчег в Египте.

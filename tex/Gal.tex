\bibbookdescr{Gal}{
  inline={Послание к Галатам\\\LARGE Святого Апостола Павла},
  toc={к Галатам},
  bookmark={к Галатам},
  header={к Галатам},
  %headerleft={},
  %headerright={},
  abbr={Гал}
}
\vs Gal 1:1 Павел Апостол, \bibemph{избранный} не человеками и не через человека, но Иисусом Христом и Богом Отцем, воскресившим Его из мертвых,
\vs Gal 1:2 и все находящиеся со мною братия~--- церквам Галатийским:
\vs Gal 1:3 благодать вам и мир от Бога Отца и Господа нашего Иисуса Христа,
\vs Gal 1:4 Который отдал Себя Самого за грехи наши, чтобы избавить нас от настоящего лукавого века, по воле Бога и Отца нашего;
\vs Gal 1:5 Ему слава во веки веков. Аминь.
\rsbpar\vs Gal 1:6 Удивляюсь, что вы от призвавшего вас благодатью Христовою так скоро переходите к иному благовествованию,
\vs Gal 1:7 которое \bibemph{впрочем} не иное, а только есть люди, смущающие вас и желающие превратить благовествование Христово.
\vs Gal 1:8 Но если бы даже мы или Ангел с неба стал благовествовать вам не то, чт\acc{о} мы благовествовали вам, да будет анафема.
\vs Gal 1:9 Как прежде мы сказали, \bibemph{так} и теперь еще говорю: кто благовествует вам не то, чт\acc{о} вы приняли, да будет анафема.
\vs Gal 1:10 У людей ли я ныне ищу благоволения, или у Бога? людям ли угождать стараюсь? Если бы я и поныне угождал людям, то не был бы рабом Христовым.
\rsbpar\vs Gal 1:11 Возвещаю вам, братия, что Евангелие, которое я благовествовал, не есть человеческое,
\vs Gal 1:12 ибо и я принял его и научился не от человека, но через откровение Иисуса Христа.
\vs Gal 1:13 Вы слышали о моем прежнем образе жизни в Иудействе, что я жестоко гнал Церковь Божию, и опустошал ее,
\vs Gal 1:14 и преуспевал в Иудействе более многих сверстников в роде моем, будучи неумеренным ревнителем отеческих моих преданий.
\vs Gal 1:15 Когда же Бог, избравший меня от утробы матери моей и призвавший благодатью Своею, благоволил
\vs Gal 1:16 открыть во мне Сына Своего, чтобы я благовествовал Его язычникам,~--- я не стал тогда же советоваться с плотью и кровью,
\vs Gal 1:17 и не пошел в Иерусалим к предшествовавшим мне Апостолам, а пошел в Аравию, и опять возвратился в Дамаск.
\vs Gal 1:18 Потом, спустя три года, ходил я в Иерусалим видеться с Петром и пробыл у него дней пятнадцать.
\vs Gal 1:19 Другого же из Апостолов я не видел \bibemph{никого}, кроме Иакова, брата Господня.
\vs Gal 1:20 А в том, чт\acc{о} пишу вам, пред Богом, не лгу.
\vs Gal 1:21 После сего отошел я в страны Сирии и Киликии.
\vs Gal 1:22 Церквам Христовым в Иудее лично я не был известен,
\vs Gal 1:23 а только слышали они, что гнавший их некогда ныне благовествует веру, которую прежде истреблял,~---
\vs Gal 1:24 и прославляли за меня Бога.
\vs Gal 2:1 Потом, через четырнадцать лет, опять ходил я в Иерусалим с Варнавою, взяв с собою и Тита.
\vs Gal 2:2 Ходил же по откровению, и предложил там, и особо знаменитейшим, благовествование, проповедуемое мною язычникам, не напрасно ли я подвизаюсь или подвизался.
\vs Gal 2:3 Но они и Тита, бывшего со мною, хотя и Еллина, не принуждали обрезаться,
\vs Gal 2:4 а вкравшимся лжебратиям, скрытно приходившим подсмотреть за нашею свободою, которую мы имеем во Христе Иисусе, чтобы поработить нас,
\vs Gal 2:5 мы ни на час не уступили и не покорились, дабы истина благовествования сохранилась у вас.
\vs Gal 2:6 И в знаменитых чем-либо, какими бы ни были они когда-либо, для меня нет ничего особенного: Бог не взирает на лице человека. И знаменитые не возложили на меня ничего более.
\vs Gal 2:7 Напротив того, увидев, что мне вверено благовестие для необрезанных, как Петру для обрезанных
\vs Gal 2:8 (ибо Содействовавший Петру в апостольстве у обрезанных содействовал и мне у язычников),
\vs Gal 2:9 и, узнав о благодати, данной мне, Иаков и Кифа и Иоанн, почитаемые столпами, подали мне и Варнаве руку общения, чтобы нам \bibemph{идти} к язычникам, а им к обрезанным,
\vs Gal 2:10 только чтобы мы помнили нищих, что и старался я исполнять в точности.
\rsbpar\vs Gal 2:11 Когда же Петр пришел в Антиохию, то я лично противостал ему, потому что он подвергался нареканию.
\vs Gal 2:12 Ибо, до прибытия некоторых от Иакова, ел вместе с язычниками; а когда те пришли, стал таиться и устраняться, опасаясь обрезанных.
\vs Gal 2:13 Вместе с ним лицемерили и прочие Иудеи, так что даже Варнава был увлечен их лицемерием.
\vs Gal 2:14 Но когда я увидел, что они не прямо поступают по истине Евангельской, то сказал Петру при всех: если ты, будучи Иудеем, живешь по-язычески, а не по-иудейски, то для чего язычников принуждаешь жить по-иудейски?
\vs Gal 2:15 Мы по природе Иудеи, а не из язычников грешники;
\vs Gal 2:16 однако же, узнав, что человек оправдывается не делами закона, а только верою в Иисуса Христа, и мы уверовали во Христа Иисуса, чтобы оправдаться верою во Христа, а не делами закона; ибо делами закона не оправдается никакая плоть.
\vs Gal 2:17 Если же, ища оправдания во Христе, мы и сами оказались грешниками, то неужели Христос есть служитель греха? Никак.
\vs Gal 2:18 Ибо если я снова созидаю, что разрушил, то сам себя делаю преступником.
\vs Gal 2:19 Законом я умер для закона, чтобы жить для Бога. Я сораспялся Христу,
\vs Gal 2:20 и уже не я живу, но живет во мне Христос. А что ныне живу во плоти, то живу верою в Сына Божия, возлюбившего меня и предавшего Себя за меня.
\vs Gal 2:21 Не отвергаю благодати Божией; а если законом оправдание, то Христос напрасно умер.
\vs Gal 3:1 О, несмысленные Галаты! кто прельстил вас не покоряться истине, \bibemph{вас}, у которых перед глазами предначертан был Иисус Христос, \bibemph{как бы} у вас распятый?
\vs Gal 3:2 Сие только хочу знать от вас: через дела ли закона вы получили Духа, или через наставление в вере?
\vs Gal 3:3 Так ли вы несмысленны, что, начав духом, теперь оканчиваете плотью?
\vs Gal 3:4 Столь многое потерпели вы неужели без пользы? О, если бы только без пользы!
\vs Gal 3:5 Подающий вам Духа и совершающий между вами чудеса через дела ли закона \bibemph{сие производит}, или через наставление в вере?
\vs Gal 3:6 Так Авраам поверил Богу, и это вменилось ему в праведность.
\vs Gal 3:7 Познайте же, что верующие суть сыны Авраама.
\vs Gal 3:8 И Писание, провидя, что Бог верою оправдает язычников, предвозвестило Аврааму: в тебе благословятся все народы.
\vs Gal 3:9 Итак верующие благословляются с верным Авраамом,
\vs Gal 3:10 а все, утверждающиеся на делах закона, находятся под клятвою. Ибо написано: проклят всяк, кто не исполняет постоянно всего, что написано в книге закона.
\vs Gal 3:11 А что законом никто не оправдывается пред Богом, это ясно, потому что праведный верою жив будет.
\vs Gal 3:12 А закон не по вере; но кто исполняет его, тот жив будет им.
\vs Gal 3:13 Христос искупил нас от клятвы закона, сделавшись за нас клятвою (ибо написано: проклят всяк, висящий на древе),
\vs Gal 3:14 дабы благословение Авраамово через Христа Иисуса распространилось на язычников, чтобы нам получить обещанного Духа верою.
\rsbpar\vs Gal 3:15 Братия! говорю по \bibemph{рассуждению} человеческому: даже человеком утвержденного завещания никто не отменяет и не прибавляет \bibemph{к нему}.
\vs Gal 3:16 Но Аврааму даны были обетования и семени его. Не сказано: и потомкам, как бы о многих, но как об одном: и семени твоему, которое есть Христос.
\vs Gal 3:17 Я говорю то, что завета о Христе, прежде Богом утвержденного, закон, явившийся спустя четыреста тридцать лет, не отменяет т\acc{а}к, чтобы обетование потеряло силу.
\vs Gal 3:18 Ибо если по закону наследство, то уже не по обетованию; но Аврааму Бог даровал \bibemph{оное} по обетованию.
\rsbpar\vs Gal 3:19 Для чего же закон? Он дан после по причине преступлений, до времени пришествия семени, к которому \bibemph{относится} обетование, и преподан через Ангелов, рукою посредника.
\vs Gal 3:20 Но посредник при одном не бывает, а Бог один.
\vs Gal 3:21 Итак закон противен обетованиям Божиим? Никак! Ибо если бы дан был закон, могущий животворить, то подлинно праведность была бы от закона;
\vs Gal 3:22 но Писание всех заключило под грехом, дабы обетование верующим дано было по вере в Иисуса Христа.
\vs Gal 3:23 А до пришествия веры мы заключены были под стражею закона, до того \bibemph{времени}, как надлежало открыться вере.
\vs Gal 3:24 Итак закон был для нас детоводителем ко Христу, дабы нам оправдаться верою;
\vs Gal 3:25 по пришествии же веры, мы уже не под \bibemph{руководством} детоводителя.
\vs Gal 3:26 Ибо все вы сыны Божии по вере во Христа Иисуса;
\vs Gal 3:27 все вы, во Христа крестившиеся, во Христа облеклись.
\vs Gal 3:28 Нет уже Иудея, ни язычника; нет раба, ни свободного; нет мужеского пола, ни женского: ибо все вы одно во Христе Иисусе.
\vs Gal 3:29 Если же вы Христовы, то вы семя Авраамово и по обетованию наследники.
\vs Gal 4:1 Еще скажу: наследник, доколе в детстве, ничем не отличается от раба, хотя и господин всего:
\vs Gal 4:2 он подчинен попечителям и домоправителям до срока, отцом \bibemph{назначенного}.
\vs Gal 4:3 Так и мы, доколе были в детстве, были порабощены вещественным началам мира;
\vs Gal 4:4 но когда пришла полнота времени, Бог послал Сына Своего (Единородного), Который родился от жены, подчинился закону,
\vs Gal 4:5 чтобы искупить подзаконных, дабы нам получить усыновление.
\vs Gal 4:6 А как вы~--- сыны, то Бог послал в сердца ваши Духа Сына Своего, вопиющего: <<Авва, Отче!>>
\vs Gal 4:7 Посему ты уже не раб, но сын; а если сын, то и наследник Божий через Иисуса Христа.
\vs Gal 4:8 Но тогда, не знав Бога, вы служили \bibemph{богам}, которые в существе не боги.
\vs Gal 4:9 Ныне же, познав Бога, или, лучше, получив познание от Бога, для чего возвращаетесь опять к немощным и бедным вещественным началам и хотите еще снова поработить себя им?
\vs Gal 4:10 Наблюдаете дни, месяцы, времена и годы.
\vs Gal 4:11 Боюсь за вас, не напрасно ли я трудился у вас.
\rsbpar\vs Gal 4:12 Прошу вас, братия, будьте, как я, потому что и я, как вы. Вы ничем не обидели меня:
\vs Gal 4:13 знаете, что, \bibemph{хотя} я в немощи плоти благовествовал вам в первый раз,
\vs Gal 4:14 но вы не презрели искушения моего во плоти моей и не возгнушались \bibemph{им}, а приняли меня, как Ангела Божия, как Христа Иисуса.
\vs Gal 4:15 Как вы были блаженны! Свидетельствую о вас, что, если бы возможно было, вы исторгли бы очи свои и отдали мне.
\vs Gal 4:16 Итак, неужели я сделался врагом вашим, говоря вам истину?
\vs Gal 4:17 Ревнуют по вас нечисто, а хотят вас отлучить, чтобы вы ревновали по них.
\vs Gal 4:18 Хорошо ревновать в добром всегда, а не в моем только присутствии у вас.
\vs Gal 4:19 Дети мои, для которых я снова в м\acc{у}ках рождения, доколе не изобразится в вас Христос!
\vs Gal 4:20 Хотел бы я теперь быть у вас и изменить голос мой, потому что я в недоумении о вас.
\rsbpar\vs Gal 4:21 Скажите мне вы, желающие быть под законом: разве вы не слушаете закона?
\vs Gal 4:22 Ибо написано: Авраам имел двух сынов, одного от рабы, а другого от свободной.
\vs Gal 4:23 Но который от рабы, тот рожден по плоти; а который от свободной, тот по обетованию.
\vs Gal 4:24 В этом есть иносказание. Это два завета: один от горы Синайской, рождающий в рабство, который есть Агарь,
\vs Gal 4:25 ибо Агарь означает гору Синай в Аравии и соответствует нынешнему Иерусалиму, потому что он с детьми своими в рабстве;
\vs Gal 4:26 а вышний Иерусалим свободен: он~--- матерь всем нам.
\vs Gal 4:27 Ибо написано: возвеселись, неплодная, нерождающая; воскликни и возгласи, не мучившаяся родами; потому что у оставленной гораздо более детей, нежели у имеющей мужа.
\vs Gal 4:28 Мы, братия, дети обетования по Исааку.
\vs Gal 4:29 Но, как тогда рожденный по плоти гнал \bibemph{рожденного} по духу, так и ныне.
\vs Gal 4:30 Что же говорит Писание? Изгони рабу и сына ее, ибо сын рабы не будет наследником вместе с сыном свободной.
\vs Gal 4:31 Итак, братия, мы дети не рабы, но свободной.
\vs Gal 5:1 Итак стойте в свободе, которую даровал нам Христос, и не подвергайтесь опять игу рабства.
\vs Gal 5:2 Вот, я, Павел, говорю вам: если вы обрезываетесь, не будет вам никакой пользы от Христа.
\vs Gal 5:3 Еще свидетельствую всякому человеку обрезывающемуся, что он должен исполнить весь закон.
\vs Gal 5:4 Вы, оправдывающие себя законом, остались без Христа, отпали от благодати,
\vs Gal 5:5 а мы духом ожидаем и надеемся праведности от веры.
\vs Gal 5:6 Ибо во Христе Иисусе не имеет силы ни обрезание, ни необрезание, но вера, действующая любовью.
\vs Gal 5:7 Вы шли хорошо: кто остановил вас, чтобы вы не покорялись истине?
\vs Gal 5:8 Такое убеждение не от Призывающего вас.
\vs Gal 5:9 Малая закваска заквашивает все тесто.
\vs Gal 5:10 Я уверен о вас в Господе, что вы не будете мыслить иначе; а смущающий вас, кто бы он ни был, понесет на себе осуждение.
\vs Gal 5:11 За что же гонят меня, братия, если я и теперь проповедую обрезание? Тогда соблазн креста прекратился бы.
\vs Gal 5:12 О, если бы удалены были возмущающие вас!
\rsbpar\vs Gal 5:13 К свободе призваны вы, братия, только бы свобода ваша не была поводом к \bibemph{угождению} плоти, но любовью служ\acc{и}те друг другу.
\vs Gal 5:14 Ибо весь закон в одном слове заключается: люби ближнего твоего, как самого себя.
\vs Gal 5:15 Если же друг друга угрызаете и съедаете, берегитесь, чтобы вы не были истреблены друг другом.
\rsbpar\vs Gal 5:16 Я говорю: поступайте по духу, и вы не будете исполнять вожделений плоти,
\vs Gal 5:17 ибо плоть желает противного духу, а дух~--- противного плоти: они друг другу противятся, так что вы не то делаете, что хотели бы.
\vs Gal 5:18 Если же вы духом водитесь, то вы не под законом.
\vs Gal 5:19 Дела плоти известны; они суть: прелюбодеяние, блуд, нечистота, непотребство,
\vs Gal 5:20 идолослужение, волшебство, вражда, ссоры, зависть, гнев, распри, разногласия, (соблазны,) ереси,
\vs Gal 5:21 ненависть, убийства, пьянство, бесчинство и тому подобное. Предваряю вас, как и прежде предварял, что поступающие так Царствия Божия не наследуют.
\vs Gal 5:22 Плод же духа: любовь, радость, мир, долготерпение, благость, милосердие, вера,
\vs Gal 5:23 кротость, воздержание. На таковых нет закона.
\vs Gal 5:24 Но те, которые Христовы, распяли плоть со страстями и похотями.
\vs Gal 5:25 Если мы живем духом, то по духу и поступать должны.
\vs Gal 5:26 Не будем тщеславиться, друг друга раздражать, друг другу завидовать.
\vs Gal 6:1 Братия! если и впадет человек в какое согрешение, вы, духовные, исправляйте такового в духе кротости, наблюдая каждый за собою, чтобы не быть искушенным.
\vs Gal 6:2 Нос\acc{и}те бремена друг друга, и таким образом исполните закон Христов.
\vs Gal 6:3 Ибо кто почитает себя чем-нибудь, будучи ничто, тот обольщает сам себя.
\vs Gal 6:4 Каждый да испытывает свое дело, и тогда будет иметь похвалу только в себе, а не в другом,
\vs Gal 6:5 ибо каждый понесет свое бремя.
\rsbpar\vs Gal 6:6 Наставляемый словом, делись всяким добром с наставляющим.
\vs Gal 6:7 Не обманывайтесь: Бог поругаем не бывает. Что посеет человек, то и пожнет:
\vs Gal 6:8 сеющий в плоть свою от плоти пожнет тление, а сеющий в дух от духа пожнет жизнь вечную.
\vs Gal 6:9 Делая добро, да не унываем, ибо в свое время пожнем, если не ослабеем.
\vs Gal 6:10 Итак, доколе есть время, будем делать добро всем, а наипаче своим по вере.
\rsbpar\vs Gal 6:11 Видите, как много написал я вам своею рукою.
\vs Gal 6:12 Желающие хвалиться по плоти принуждают вас обрезываться только для того, чтобы не быть гонимыми за крест Христов,
\vs Gal 6:13 ибо и сами обрезывающиеся не соблюдают закона, но хотят, чтобы вы обрезывались, дабы похвалиться в вашей плоти.
\vs Gal 6:14 А я не желаю хвалиться, разве только крестом Господа нашего Иисуса Христа, которым для меня мир распят, и я для мира.
\vs Gal 6:15 Ибо во Христе Иисусе ничего не значит ни обрезание, ни необрезание, а новая тварь.
\vs Gal 6:16 Тем, которые поступают по сему правилу, мир им и милость, и Израилю Божию.
\vs Gal 6:17 Впрочем никто не отягощай меня, ибо я ношу язвы Господа Иисуса на теле моем.
\rsbpar\vs Gal 6:18 Благодать Господа нашего Иисуса Христа со духом вашим, братия. Аминь.

\bibbookdescr{7Sb}{
  inline={Седьмая книга Сивилл},
  toc={7-я Сивилл},
  bookmark={7-я Сивилл},
  header={7-я Сивилл},
  abbr={7~Сив}
}
\vs 7Sb 1:1 Родос злосчастный, тебя, тебя я первым оплачу! 

\vs 7Sb 1:2 Первым среди городов ты будешь и первым погибнешь, 

\vs 7Sb 1:3 Жизни лишен, без людей, одинокий и очень несчастный.

\vs 7Sb 1:4 Делос, ты поплывешь и на волнах будешь качаться. 

\vs 7Sb 1:5 Кипр, однажды тебя затопят воды морские.

\vs 7Sb 1:6 Остров Сицилия, ты погибнешь, охвачен пожаром.

\vs 7Sb 1:7 То, о чем говорю: ужасный, невиданный прежде 

\vs 7Sb 1:8 Хлынет на землю потоп, Самим низпосланный Богом.

\vs 7Sb 1:9 Ной лишь один уцелел, от всех людей убежавший.

\vs 7Sb 1:10 Все поплывет  и земля, и горы, и небо над ними; 

\vs 7Sb 1:11 Мир весь станет водой и водами будет погублен. 

\vs 7Sb 1:12 Ветры дуть прекратят, наступит другая эпоха.

\vs 7Sb 1:13 Фригия! Первой на свет суждено тебе снова подняться, 

\vs 7Sb 1:14 Первой, в нечестие впав, сама отречешься от Бога 

\vs 7Sb 1:15 И, предпочтенье отдав немым изваяньям, за это, 

\vs 7Sb 1:16 Жалкая, годы спустя ужасною смертью погибнешь.

\vs 7Sb 1:17 Много бед претерпев, Эфиопы несчастные, в страхе 

\vs 7Sb 1:18 Телом дрожа, под мечи себя безрассудно поставят.

\vs 7Sb 1:19 Трудолюбивый Египет, от века растящий колосья, 

\vs 7Sb 1:20 Тот, которого Нил питает семью рукавами, 

\vs 7Sb 1:21 Междуусобная рознь погубит. Тогда же, нежданно, 

\vs 7Sb 1:22 Аписа люди изгонят за то, что вовсе не бог он.

\vs 7Sb 1:23 Лаодикия, увы! Ты Бога впредь не увидишь,

\vs 7Sb 1:24 Но, погрязнув во лжи, будешь смыта Ликской волною.

\vs 7Sb 1:25 Сам рожденный Господь, великий, Который без счета 

\vs 7Sb 1:26 Звезд сотворит и ось проденет сквозь неба средину, 

\vs 7Sb 1:27 Людям на страх, в вышине, чтобы видели все, установит 

\vs 7Sb 1:28 Столп, измерив его огнем великим, чьи капли 

\vs 7Sb 1:29 Жизнь отнимут у тех, кто себя запятнал преступленьем. 

\vs 7Sb 1:30 Время такое наступит однажды, и смертные люди 

\vs 7Sb 1:31 Бога тут станут молить, но не будет предела страданьям 

\vs 7Sb 1:32 Их безконечным. Тогда чрез дом все свершится Давидов, 

\vs 7Sb 1:33 Ибо сам Бог удостоил его небесного трона. 

\vs 7Sb 1:34 Ангелы лягут у ног, которые именем Божьим 

\vs 7Sb 1:35 Свет огням алтарей дают и воды  потокам: 

\vs 7Sb 1:36 Те хранят города, другие  ветра посылают.

\vs 7Sb 1:37 Многих людей ожидают невзгоды, что, в души несчастных 

\vs 7Sb 1:38 Путь пролагая, сердца их всех измениться заставят.

\vs 7Sb 1:39 В пору, как юный росток, на корне возросший, прозреет, 

\vs 7Sb 1:40 Власть он, что некогда всех в избытке пищей снабжала.

\vs 7Sb 1:41 Это случиться должно с исполнением срока. Но стоит 

\vs 7Sb 1:42 Править воинственным Персам начать, как тогда же покои 

\vs 7Sb 1:43 Брачные чистых невест омрачатся всеобщим нечестьем. 

\vs 7Sb 1:44 Сына мать своего как супруга на ложе допустит, 

\vs 7Sb 1:45 Мать соблазнит ее сын. Уснет, к отцу прижимаясь,

\vs 7Sb 1:46 Дочь, исполняя обычай их варварский. Позже над ними 

\vs 7Sb 1:47 Римский Арей заблестит оружьем несметного войска. 

\vs 7Sb 1:48 Много смешают земли тут с кровью убитых в сраженье, 

\vs 7Sb 1:49 Но от твердости копий бежит Италийский воитель. 

\vs 7Sb 1:50 Бросят они в той стране из золота сделанный символ 

\vs 7Sb 1:51 Тот, что вперед выходя, всегда означал неизбежность.

\vs 7Sb 1:52 Время настанет, и весь погрязший в пороке, несчастный 

\vs 7Sb 1:53 Смерть обретет Илион вместо свадеб, когда зарыдают 

\vs 7Sb 1:54 Горько юные жены о том, что не ведали Бога, 

\vs 7Sb 1:55 Но ударяли в тимпаны и били ногами о землю.

\vs 7Sb 1:56 Бога спроси, Колофон: тебя страшный пожар ожидает.

\vs 7Sb 1:57 Ты, несчастная в браке, Фессалия! Снова увидеть 

\vs 7Sb 1:58 Лик твой земле не дано, как и пепел. Отсюда по морю, 

\vs 7Sb 1:59 Храбрая, вдаль отплывешь и войны испражнением станешь, 

\vs 7Sb 1:60 Пав под ударом мечей и сгинув в стремительных реках. 

\vs 7Sb 1:61 Стойкий Коринф! Ты у стен Арея грозного примешь: 

\vs 7Sb 1:62 Горе тебе, ибо вы падете сраженные оба.

\vs 7Sb 1:63 Тир, тебе одному пережить уготовано столько: 

\vs 7Sb 1:64 Набожных граждан твоих безсилье тебя же разрушит.

\vs 7Sb 1:65 Горная Сирия, ты поднялась над землей Финикийской, 

\vs 7Sb 1:66 Где к берегам приливают валы Беритского моря. 

\vs 7Sb 1:67 Бога узнать своего не смогла, несчастная  влагой 

\vs 7Sb 1:68 Кто Иорданской омыт, на Кого Божий Дух опустился. 

\vs 7Sb 1:69 Кто, до того, как земля и звездное небо возникли, 

\vs 7Sb 1:70 Словом Отца был рожден, Властелин, и, плотью облекшись 

\vs 7Sb 1:71 Через Духа Святого, к Отцу вскоре в домы вознесся. 

\vs 7Sb 1:72 Три Ему башни Уран великий поставил, в которых 

\vs 7Sb 1:73 Матери Бога теперь живут благородные. Имя 

\vs 7Sb 1:74 Первой  Надежда, второй  Благочестье и Набожность  третьей.

\vs 7Sb 1:75 Ни серебра не хотят, ни золота  радость приносят 

\vs 7Sb 1:76 Им поклоненье людей, их жертвы и чистые мысли.

\vs 7Sb 1:77 Жертвовать вечному Богу, великому, славному станешь, 

\vs 7Sb 1:78 Не растопив на огне крупицу ладана, нож свой 

\vs 7Sb 1:79 Не занеся над бараном с волнистым руном, но со всеми, 

\vs 7Sb 1:80 В ком течет твоя кровь, взяв птицу дикую в руки, 

\vs 7Sb 1:81 Вверх направишь ее, с молитвою глядя на небо.

\vs 7Sb 1:82 Воду на чистый огонь прольешь и скажешь при этом: 

\vs 7Sb 1:83 Твой Отец Тебя создал как Слово, Отче. Я птицу 

\vs 7Sb 1:84 Быструю выпустил с вестью  о Слове Слово, крещенье 

\vs 7Sb 1:85 Влагой Твое окропив  огонь, из какого Ты вышел.

\vs 7Sb 1:86 Ты не закроешь дверей, когда чужестранец безвестный 

\vs 7Sb 1:87 К дому придет твоему, нуждаясь в пище и крове. 

\vs 7Sb 1:88 Но, его голову взяв в ладони, обрызгав водою, 

\vs 7Sb 1:89 Трижды мольбу вознеси, обратись к своему Господину: 

\vs 7Sb 1:90 Я не жажду богатства, простой  простого я принял.

\vs 7Sb 1:91 Вдвое подай нам, Отец, склони Свой слух, Покровитель! 

\vs 7Sb 1:92 Даст Он, мольбе твоей вняв, когда же уйдет чужеземец: 

\vs 7Sb 1:93 Мукам меня не предай, о праведной веры Святыня, 

\vs 7Sb 1:94 Чистый, свободный, прошедший сквозь пламя \ldots

\vs 7Sb 1:95 Слабый мой дух укрепи. Отец. На Тебя я взираю, 

\vs 7Sb 1:96 Кто всякой скверны далек, руками не создан людскими \ldots

\vs 7Sb 1:97 Жребий, Сардиния, твой несчастен  в золу превратишься, 

\vs 7Sb 1:98 Сменится десять эпох  и островом быть перестанешь. 

\vs 7Sb 1:99 Тщетно тебя среди волн искать мореходам придется, 

\vs 7Sb 1:100 Птицы свой жалобный плач по тебе над морем поднимут.

\vs 7Sb 1:101 Камнем покрыта сплошным, Мигдония, крепость на море, 

\vs 7Sb 1:102 Славиться будешь века, чтобы после навеки погибнуть

\vs 7Sb 1:103 Всей под горячим дыханьем, от боли придя в изступленье.

\vs 7Sb 1:104 Кельтов земля! По горам, у подножия Альп недоступных 

\vs 7Sb 1:105 Скроет глубокий песок тебя; не выплатишь дани, 

\vs 7Sb 1:106 Колос не дашь и траву  безлюдною ляжешь пустыней, 

\vs 7Sb 1:107 Вечно покрытая льдом, под слоем холодных кристаллов, 

\vs 7Sb 1:108 Будешь страдать за вину, которой преступно не помнишь.

\vs 7Sb 1:109 Рим, чей дух непреклонен! Вослед Македонскому царству 

\vs 7Sb 1:110 Дротик метнешь ты в Олимп  за это немым и печальным 

\vs 7Sb 1:111 Бог тебя сделает, пусть казаться ты будешь в то время 

\vs 7Sb 1:112 Сильным как никогда  тут я обращусь к тебе с речью. 

\vs 7Sb 1:113 Чувствуя гибель, оплачешь ты блеск и славу былую  

\vs 7Sb 1:114 Я во второй раз, о Рим, возьмусь тебе это напомнить.

\vs 7Sb 1:115 Ныне же я по тебе, несчастная Сирия, плачу.

\vs 7Sb 1:116 Разум оставил вас, Фивы; нависли ужасные звуки 

\vs 7Sb 1:117 Громко взывающих флейт, труба им грозная вторит  

\vs 7Sb 1:118 Так что увидите вы поверженной в прах вашу землю.

\vs 7Sb 1:119 Горе, о горе тебе, несчастное злобное море?

\vs 7Sb 1:120 Все тебя пламя пожрет, людей ты погубишь волнами 

\vs 7Sb 1:121 Ибо такой на земле пожар забушует, что воды 

\vs 7Sb 1:122 Станут огнем, потекут и землю безкрайнюю сгубят, 

\vs 7Sb 1:123 Горы заставят они пылать, ключи, и потоки. 

\vs 7Sb 1:124 Мир же со смертью людей прекрасный свой облик утратит, 

\vs 7Sb 1:125 В муках сгорая, тогда не увидят несчастные неба,

\vs 7Sb 1:126 Полного звезд, но огнем оно все выжжено будет. 

\vs 7Sb 1:127 Быстро они не умрут: под гибнущей в пламени плотью 

\vs 7Sb 1:128 Души их будут пылать в продолжение многих столетий. 

\vs 7Sb 1:129 Так, злые муки терпя, Закон познают Господень  

\vs 7Sb 1:130 Тот, что всегда справедлив. Земля же под гнетом несчастья

\vs 7Sb 1:131 Всяких богов приняла на своих алтарях без разбора 

\vs 7Sb 1:132 И обманулась, понять не сумев зловещего дыма. 

\vs 7Sb 1:133 Тем страдать суждено сверх меры, кто ради корысти 

\vs 7Sb 1:134 Станет предсказывать зло, продляя тяжелое время. 

\vs 7Sb 1:135 Эти, надев на себя овец густорунные шкуры,

\vs 7Sb 1:136 Будут себя выдавать за Евреев, хоть рода иного, 

\vs 7Sb 1:137 Хитрые речи плести, наживаясь на общем несчастье. 

\vs 7Sb 1:138 Жизнь поменяют свою, но праведных не убедить им  

\vs 7Sb 1:139 Тех, что, от чистого сердца уверовав, молятся Богу.

\vs 7Sb 1:140 В третьем жребии лет, что пройдут, друг друга сменяя,

\vs 7Sb 1:141 В первой восьмерке опять грядет обновление мира. 

\vs 7Sb 1:142 Долгая ночь на земле тогда неподвижная ляжет, 

\vs 7Sb 1:143 Запах серы зловещий начнет повсюду носиться  

\vs 7Sb 1:144 Вестник насильственной смерти, другие же люди в то время 

\vs 7Sb 1:145 Будут под пологом ночи от голода гибнуть. Тут явит

\vs 7Sb 1:146 Бог чистый ум средь людей и род возстановит, который 

\vs 7Sb 1:147 Некогда жил на земле. Никто больше пашню не взрежет 

\vs 7Sb 1:148 Выгнутым плугом, быки железо вглубь не опустят, 

\vs 7Sb 1:149 Больше ростки не взойдут, не будет колосьев. Все вместе 

\vs 7Sb 1:150 Манну росистую есть белоснежными станут зубами.

\vs 7Sb 1:151 С ними пребудет тогда Господь, и Он их научит 

\vs 7Sb 1:152 Так же, как и меня, несчастную  столько свершила 

\vs 7Sb 1:153 Прежде недобрых я дел и знала об этом. Другое 

\vs 7Sb 1:154 Было содеяно мною невольно. Со многими ложе 

\vs 7Sb 1:155 Я разделила без мысли о браке; ужасную клятву

\vs 7Sb 1:156 Всем вероломно дала. На порог не пустила я бедных. 

\vs 7Sb 1:157 И, среди первых спускаясь в долину смерти, Господних

\vs 7Sb 1:158 Слов не сумела понять  за то и пожрет меня пламень. 

\vs 7Sb 1:159 Вечно мне жить не дано  погубит жестокое время, 

\vs 7Sb 1:160 Люди меня погребут, проплывая по морю мимо. 

\vs 7Sb 1:161 Буду побита камнями вещав, что велел мне родитель, 

\vs 7Sb 1:162 Сына я предала! Все киньте по камню, кидайте  

\vs 7Sb 1:163 Так сохраню себе жизнь и к небу очи воздену.

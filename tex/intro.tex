\pagestyle{fancy}
\thispagestyle{empty}
\bibmark{book}{ВВЕДЕНИЕ}
\bibpdfbookmark{Введение}{intro}
\begin{center}
\Large\bfseries ВВЕДЕНИЕ\\
\end{center}
\fontsize{12}{15}\selectfont

%\begin{multicols}{2}
В 1804 году было основано \bibemph{Британское и Иностранное Библейское Общество} (BFBS),
полу-автономным филиалом которого стало \bibemph{Российское Библейское Общество} (РБО),
основанное 6 декабря 1812 года.
В своей работе РБО опиралось на поддержку царя Александра~I, а председателем Общества
был избран князь Александр Голицын (1773--1844),
который тогда был обер-прокурором Святейшего Правительствующего Синода
Русской Православной Церкви, а позже Министром Религии и Народного Образования ---
так называемого <<сугубого министерства>>.
Общество было открыто под именем Санкт-Петербургского, а в сентябре 1814 года
переименовано на Российское.

О русском библейском переводе впервые открыто заговорили в 1816 году.
Князь Голицын, как председатель РБО, получил Высочайшее изустное повеление,
<<дабы предложить Святейшему Синоду искреннее и точное желание Его Величества
доставить и россиянам способ читать Слово Божие на природном своем российском
языке, яко вразумительнейшем для них славянского наречия, на коем книги
Священного Писания у нас издаются>>.
Предполагалось при этом, что новый перевод будет издаваться со славянским
текстом совокупно, как еще раньше уже было выпущено послание к Римлянам,
с дозволения Синода (имелась в виду книга архиепископа Мефодия Смирнова,
перевод и толкование; первое издание в 1794 г., третье в 1815 г.).

Голицын в оправдание предложенного перевода на современный русский язык
ссылался на то, что греческой патриаршей грамотой одобрено народу
чтение священного писания Нового Завета на новейшем греческом наречии
вместо древнего (сама грамота патр. Кирилла была припечатана в отчете
РБО за 1814 год).

Синод не принял на себя руководства библейским переводом и не взял за
него ответственности на себя.
Перевод был отдан в ведение Комиссии духовных училищ, которой надлежало
избрать надежных переводчиков в местной Духовной Академии. 

Перевод был поставлен под охрану Высочайшего имени.
Замысел сей принадлежал самому Государю, или был ему приписан:
<<Не токмо одобряет все споспешествующее сему спасительному делу, но и
одушевляет деятельность Общества внушениями собственного сердца.
Он сам снимает печать невразумительного наречия, заграждавшую доныне от
многих из Россиян евангелие Иисусово, и открывает сию книгу для самых
младенцев народа, от которых не ея назначение, но единственно мрак
времен закрыл оную.>>
Невразумительное наречие закрывало Библию не столько
от народа, сколько именно от высшего круга, от самого императора, прежде
всего, он сам привык читать Новый Завет по-французски (в известном
переводе Де-Саси), и не изменил этой привычке и с изданием российского
перевода.

Ведение перевода от Комиссии духовных училищ было поручено Филарету,
тогда архимандриту и ректору Санкт-Пе\-тер\-бургс\-кой Академии, и он имел
избрать переводчиков по своему усмотрению.
Считалось, что перевод производится при Академии.
Филарет сам взял на себя Евангелие от Иоанна.
От Матфея переводил Павский, от Марка архим. Поликарп (Гайтанников),
тогда ректор Санкт-Петербургской семинарии, а вскоре и Московской
Академии, и от Луки архим. Моисей (Антипов-Платонов), ректор Киевской
семинарии, а потом и Академии, бывший перед тем бакалавром в
Санкт-Петербурге, впоследствии Экзарх Грузии.
Работа отдельных сотрудников пересматривалась и сверялась в особом
комитете из членов Библейского общества.
В нем участвовали: митр. Михаил (Десницкий), впоследствии митрополит
Санкт-Петербургский;
Серафим (Глаголевский), тоже будущий митрополит;
Филарет;
Лабзин;
В.~М.~Попов, директор департамента в двойном министерстве и секретарь
Библейского общества --- человек крайних мистических взглядов, переводчик
Линдля и Госнера, член кружка Татариновой, окончивший жизнь свою в
Зилантовом монастыре в Казани, как заточенный, кроткий изувер, как
его остроумно называет Вигель.

Правила для перевода были составлены Филаретом, это сразу чувствуется
уже в их стиле.
Переводить надлежало с греческого, как первоначального, преимущественно
перед славянским, с тем, чтобы в переводе удерживать или употреблять
слова славянские, если они ближе русских подходят к греческим, не
производя в речи темноты или нестройности, или если соответственные
русские не принадлежат к чистому книжному языку.
В переводе всего важнее точность, затем ясность, наконец, чистота.
Очень характерны некоторые стилистические директивы.
Величие Священного Писания состоит в силе, а не в блеске слов; из сего
следует, что не должно слишком привязываться к славянским словам и выражениям,
ради мнимой их важности.
Еще важнее другое замечание.
Тщательно наблюдать должно дух речи, дабы разговор перелагать слогом
разговорным, повествование повествовательным, и так далее.

Эти положения литературным архаистам показались дурной стилистической
ересью, и это был один из решающих моментов взволнованного восстания
или интриги против русской Библии в 20-х годах.

В этот период были переведены на русский язык и в сотрудничестве с BFBS
опубликованы: Евангелие (1819), Новый Завет (1820) и Псалтирь (1822).
В то же время началась работа над Пятикнижием.
Филарет в своих Записках на книгу Бытия (первое издание уже в 1816 г.)
всюду дает библейский текст в русском переводе, с еврейского.
К переводческим работам были привлечены и вновь открытые Академии:
Московская и Киевская, также и некоторые семинарии.
Сразу же встал трудный и сложный вопрос о соотношении еврейского и
греческого текстов, о достоинстве и достоинствах перевода Семидесяти, о
значении Массоретских чтений, и эти вопросы обострялись тем, что
всякое отступление от Семидесяти означало практически и расхождение
со славянской Библией, остававшейся в богослужебном употреблении,
а потому нуждалось в нарочитых оправданиях и оговорках.
Для начала вопрос был решен просто.
В основу был положен еврейский (Массоретский) текст, как подлинный,
а в объяснение расхождений со славянской Библией было составлено
особое предисловие, убедительное и для незнающих древних языков.
Составил его Филарет, и подписано оно было митр. Михаилом, митр.
Серафимом, тогда еще Московским, и самим Филаретом, тогда
архиепископом Ярославским.

Окончательная корректура Пятикнижия была поручена Герасиму Павскому.
Печатание было закончено в 1825 году, но по изменившимся
обстоятельствам  издание не только не было выпущено в свет, но было
арестовано и вскоре сожжено.
Само библейское дело было остановлено и Библейское общество закрыто
и запрещено 12 апреля 1826 года, в основном благодаря интриграм архимандрита
Фотия, адмирала Шишкова и Аракчеева.

В 1840-х годах профессор Павский впервые перевел на русский язык весь
Ветхий Завет непосредственно с еврейских оригиналов, за что был отдан
под суд и результаты его стараний были уничтожены.

С приходом к власти царя Александра~II работа РБО была возобновлена
под руководством Митрополита Московкого Филарета.
В декабре 1857 года библейское дело получило официальное движение.
Синодальное определение состоялось 20 марта 1858 года, а Высочайшее
повеление о возобновлении русского перевода было опубликовано в мае.

Перевод был возобновлен с Нового Завета, к участию в работах снова были
привлечены все академии, а редактирование поручено петербургскому
профессору Е.~И.~Ловягину.
Высшее наблюдение и последний просмотр были доверены Филарету.
Несмотря на свой преклонный возраст, он очень деятельно участвовал в
работе, со вниманием перечитывая и проверяя весь материал.

В 1860 году было издано русское Четвероевангелие, а в 1862 и полный
Новый Завет.

Перевод Ветхого Завета потребовал больше времени. Уже с самого начала
60-х годов в различных духовных журналах стали появляться частные опыты
перевода отдельных книг.
И, прежде всего, были опубликованы эти так незадолго перед тем запретные
переводы Павского (в журнале Дух Христианина за 1862 и 1863 годы) и
арх. Макария (в Православном Обозрении с 1860-го по 1867-ой, особым
приложением).
Это был очень живой и яркий симптом сдвига и поворота.
Было признано полезным и нужным предать гласности эти опыты, чтобы через
свободное обсуждение в печати подготовить окончательное издание.
С этой целью было предложено и профессорам академии заняться переводами
отдельных книг, с тем чтобы эти новые опыты были в свое время использованы
Синодальной комиссией.
Нечто подобное предлагал в свое время о. Макарий Глухарев,
издавать при Петербургской Академии особый журнал: Опыты в переводе с
еврейского и греческого, и рассылать по академиям и семинариям, с
примечаниями и сносками, потом этот материал пригодится.

В академических изданиях, в Христианском Чтении и в Трудах Киевской
духовной академии в эти годы появляется перевод многих книг.
В Киеве особенно потрудился проф. М.~С.~Гуляев, а в Петербурге проф.
М.~А.~Голубев в сотрудничестве с П.~И.~Савваитовым, Д.~А.~Хвольсоном и др.
Появились и отдельные издания.
Издавал свои библейские переводы с греческого Порфирий Успенский, тогда епископ
Чигиринский.
Это был полный разрыв с режимом предыдущего царствования.

Но встречались и трудности.
Не сразу удалось решить вопрос о принципах перевода.
Было заявлено мнение, что и Ветхий Завет переводить нужно с греческого,
к этому мнению удалось склонить и митр. Григория.
Филарет Московский настоял, чтобы перевод делался по сличению обоих
текстов, и расхождение в важнейших местах было отмечаемо под чертой.
Сперва предложено было начать с Псалмов; над исправлением перевода
Псалмов Филарет работал в свои последние годы.
Но затем он сам предложил издавать в порядке обычного текста,
т.~к. Пятикнижие легче Псалмов по языку.
Синодальный перевод начал выходить с 1868 года отдельными томами, а
всё издание закончилось в 1875 со включением и книг неканонических.

Особенно резким противником еврейского текста был епископ Феофан
Говоров, тогда уже Вышенский затворник.
Новый русский перевод Ветхого Завета он называл Синодальным сочинением,
совсем как Афанасий, и мечтал, что эту Библию новомодную доведет до
сожжения на Исаакиевской площади.
Употребление еврейского текста, никогда не бывшего в церковном
употреблении, означало в его понимании прямое отступничество.
Еврейская библия к нам нейдет, потому что никогда не было ее в Церкви и
в церковном употреблении.
Поэтому принимать ее значит отступать от того, что всегда было в Церкви,
т.~е. сдвигаться с коренного основания православия.
Феофан вполне признавал нужду в русском переводе, он возражал только
против еврейского образца.
И синодальный перевод считал поэтому соблазнительным и вредным.
Церковь Божия не знала другого Слова Божия, кроме 70-ти толковников, и
когда говорила, что Писание богодухновенно, разумела Писание именно в
этом переводе.
Об этом он очень резко писал в Душеполезном Чтении (1875 и 1876),
ему отвечал в Православном Обозрении проф. П. И. Горский-Платонов с
неменьшей резкостью.
Но Феофан не ограничивался критикой.
Он предлагал заняться изданием общедоступных толкований Библии по
славянскому тексту (и особенно книг учительных и пророческих),
чтобы приучить именно к этому тексту, т.~е. к Семидесяти.
Выйдет, что, несмотря на существование Библии в переводе с
еврейского, знать ее и понимать и читать все будут по
Семидесяти, по причине сего толкования.
Проект этот не был осуществлен, сам Феофан издал только
толкование на Псалом Сто Осмьнадцатый (сто восемнадцатый).
Возникла у него и мысль сесть за перевод всей Библии с греческого,
с замечаниями в оправдание греческого текста и в осуждение
еврейского.
Это намерение осталось тоже без исполнения.
Уже только много позже некоторые книги Ветхого Завета были переведены с
греческого казанским профессором П.~А. Юнгеровым (пророки, Псалтирь,
Притчи, Бытие, книги неканонические).

В процессе работы над переводом Ветхого Завета снова и снова
открывалось, что соотношение Массоретской редакции и Семидесяти слишком
сложно, чтобы можно было ставить вопрос о выборе между ними в общем
виде.
Можно спрашивать только о предпочтительном или надежном чтении
отдельных отрывков или стихов, и приходится выбирать иногда еврейскую
истину, иногда же греческое чтение.
Филологически лучшим будет именно сводный текст.
Богословскому заключению о догматическом достоинстве определенного
текста, во всяком случае, должно предшествовать подробное исследование
отдельных книг.
Примером такой работы в те годы была диссертация И.~С.~Якимова о книге
пророка Иеремии (1874).
Следует упомянуть и работы Д.~А.~Хвольсона и И.~А.~Олесницкого.  

Обнаруживалась и другая трудность.
Оказывалось, что и Славянскую Библию не приходится в целом приравнивать
к Семидесяти, что и сам славянский текст есть уже сводный, в известном
смысле и пределах.
В этом и была принципиальная важность описания библейских рукописей Горским
и Невоструевым в Московской Синодальной библиотеке.
Начинается историческое изучение Славянской Библии.
И уже нельзя так упрощенно спрашивать о выборе между славянским и русским.  

Оживает интерес и к вопросам библейской критики.
Большинство русских исследователей придерживались умеренных
взглядов, но и у них влияние западной критической
литературы сказывалось очень заметно.
Достаточно назвать работы архим. Филарета Филаретова (ректора
Киевской академии, впоследствии епископа Рижского, 1824-1882).
В его диссертации о Происхождении книги Иова (1872) он не только
принимал позднюю послепленную датировку книги, но и разбирал ее скорее,
как памятник литературы, нежели как книгу священного канона.
К тому же всё исследование было проведено по еврейскому тексту,
безо всякого внимания к славянским чтениям.
Митр. Арсений Киевский нашел сам тон диссертации
несоответствующим богодухновенному характеру библейской книги, и
публичная защита диссертации была запрещена Святейшим Синодом.
А в следующем (1873) году в Трудах Киевской Академии были
напечатаны устаревшие лекции по введению в священные книги
Ветхого Завета, читанные самим митр. Арсением в Петербургской
академии еще в 1823--1825 годах.
Впрочем, в кратком предисловии от редакции было оговорено, что
читатель сам сможет судить, насколько вперед подвинулись у нас
библиологическая наука с того времени до настоящего.

Переводы, выполненные в 1810--25~гг. и отредактированные
в 1860--70~гг., составили книгу именуемую \bibemph{Русской Синодальной Библией}.
Однако не все отнеслись благосклонно к появлению Библии на русском языке,
предпочитая старо-славянский перевод используемый и по сей день в церковном служении.
Даже Святейший Синод благословил Библию 1876 года \bibemph{исключительно}
для приватного употребления, для чтения дома, но не для церковного служения.

Впоследствии, текст Русской Синодальной Библии был существенно изменен
с целью распространения <<протестантизма>>.
А именно, слова и целые фразы соответствующие текстам греческой
Септуагинты и латинской Вульгаты были удалены, хотя и не полностью и
с многочисленными ошибками, чтобы поддержать \bibemph{миф} о том, что,
якобы, Бог чудодейственным образом <<сохранил>> Свое
Слово в одном единственном варианте и естественно выбор такого
<<идеального>> варианта пал на Массоретский Текст.
В данном издании мы не делаем идола из Слова Божиего и, посему,
приводим текст Синодального издания в его изначальной форме (за исключением
использования архаичного правописания и букв), включая неканонические
книги в том порядке и виде, в каком они были приведены в Библии 1876 года.

Тексты Книг Священного Писания Ветхого и Нового Завета и приложения,
использованные в данном издании, взяты с сайта Издательства Московской
Патриархии,
и соответствуют Си\-но\-даль\-но\-му переводу издания Московской Патриархии
кроме 70-и стихов находящихся между 35 и 36 стихами 7-й главы
3-й Книги Ездры, взятых нами из ``Толковой Библии'' А.~П.~Лопухина (Петербург, 1904)
и имеющихся также в Брюссельской Библии (Брюссель, 1973).
Электронные тексты были переведены в формат типографской системы,
используемой для всех Библий, издаваемых Bibles.org.uk, основанной
на \XeLaTeX\ в системе Linux.
Выражаем благодарность Самуэлю Ким за найденные опечатки.
Мы будем очень признательны, если найденные Вами в этом издании
опечатки, будут отправлены по электронной почте по адресу
{\makeatletter aivazian.tigran@gmail.com\makeatother}.

Оформление текста Библии в данном издании имеет следующие особенности:
\begin{itemize}
\item Для облегчения ссылок и чтения нумерация стихов выведена на поля.
\item Слова, напечатанные \bibemph{курсивом}, приведены для ясности
      и отсутствуют в оригиналах.
\item В тексте Ветхого Завета в квадратные скобки заключены слова,
      заимствованные из греческого перевода 70-ти толковников (III в.~до Р.~Х.)
      --- Септуагинты.
\end{itemize}

Несмотря на <<до-Пушкинский>> язык Русской Синодальной Библии, она
продолжает успешно служить миллионам людей на планете как самый
достоверный и читаемый перевод Священного Писания на русский язык.

Да благословит Господь Бог ваше изучение Его Слова, дабы подчинить
Сыну своему Иисусу Христу Господу нашему всякую мысль вашего сердца
и всякое слово, исходящее из уст ваших. Аминь.
%\end{multicols}

\begingroup
\vfill
\noindent
\itshape
\parbox{4cm}{
Владимир Волович,\\
Воронеж, Россия.
}
\hfill
\parbox{4cm}{
Тигран Айвазян,\\
Лондон, Англия.
}
\vfill
\endgroup

\bibbookdescr{1En}{
  inline={Первая книга Еноха},
  toc={1-я Еноха},
  bookmark={1-я Еноха},
  header={1-я Еноха},
  abbr={1~Ено}
}
\vs 1En 1:1
Слова благословения Еноха, которыми он благословил избранных и
праведных, которые будут жить в день скорби, когда все злые и нечестивые
будут отвержены.
И отвечал и сказал Енох,~--- праведный муж, которому были открыты
Богом очи,~--- что он видел на небесах святое видение: Его показали мне
ангелы, и от них я слышал всё, и уразумел, что видел, но не для этого рода,
а для родов отдалённых, которые явятся.
Об избранных говорил я и о них беседовал со Святым и Великим, с
Богом мира, Который выйдет из Своего жилища.
И оттуда Он придёт на гору Синай, и явится со Своими воинствами, и в
силе Своего могущества явится с неба.
И всё устрашится, и стражи содрогнутся, и великий страх и трепет
обоймёт их до пределов земли.
Поколеблются возвышенные горы, и высокие холмы опустятся, и растают,
как сотовый мёд от пламени.
Земля погрузится, и всё, что на земле, погибнет, и совершится суд
над всем и над всеми праведными.
Но праведным Он уготовит мир и будет охранять избранных, и милость
будет господствовать над ними; они все будут Божии, и хорошо им будет, и они
будут благословлены, и свет Божий будет светить им.
И вот Он идёт с мириадами святых,  чтобы совершить суд над ними,  и
Он уничтожит нечестивых, и будет судиться со всякою плотью относительно
всего, что грешники и нечестивые сделали и совершили против Него.
Я наблюдал всё, что происходит на небе: как светила, которые на
небе, не изменяют своих путей, как все они восходят и заходят по порядку,
каждое в своё время, и не преступают своих законов.
Взгляните на землю и обратите внимание на вещи, которые на ней, от
первой до последней, как каждое произведение Божие правильно обнаруживает
себя!
Взгляните на лето и зиму, как тогда (зимою) вся земля изобилует
водою, и тучи, и роса, и дождь стелются над нею!
Я наблюдал и видел, как зимою все деревья кажутся, будто они
высохли, и все листья их опали, кроме четырнадцати деревьев, которые не
обнажаются, но ожидают, оставаясь со старой листвой, появления новой в
течение двух--трёх лет.
И опять я наблюдал дни летние, как тогда солнце стоит над нею
(землёю), прямо против неё, а вы ищете прохладных мест и тени от солнечной
жары, и как тогда даже земля горит от зноя, а вы не можете ступить ни на
землю, ни на скалу (камень) вследствие их жара.
Я наблюдал, как деревья покрываются зеленью листьев и приносят
плоды; и вы обратите внимание на всё и узнайте, что всё это для вас сотворил
Тот, Который живёт вечно; посмотрите, как Его произведения существуют пред
Ним в каждом новом году и все Его произведения служат Ему и не изнемогают,
но как установил Бог, так всё и происходит!
И посмотрите, как моря и реки все вместе выполняют своё дело!
А вы не претерпели до конца и не выполнили закона Господня; но
преступили его и надменными, хульными словами поносили Его величие из своих
нечестивых уст; вы, жестокосердые, не обретёте никакого мира!
И посему вы проклянёте ваши дни, и годы вашей жизни прекратятся;
велико будет вечное осуждение, и вы не обретёте никакой милости.
В те дни вы лишитесь мира, чтобы быть вечным проклятием для всех
праведных, и они будут всегда проклинать вас как грешников,~--- вас вместе со
всеми грешниками.
Для избранных же настанет свет, и радость, и мир, и они наследуют
землю; а для вас, нечестивые, наступит проклятие.
Тогда избранным будет дана мудрость и они все будут жить и не
согрешат опять ни по небрежности, ни по надменности, но будут смирёнными, не
согрешая опять, так как имеют мудрость.
И они будут наказаны в продолжение своей жизни, и не умрут в муках
и в гневном осуждении, но окончат число дней своей жизни, а состареются в
мире, и годы их счастья будут многими: они будут пребывать в вечном
наслаждении и в мире в продолжение всей своей жизни.
\vs 1En 2:1
И случилось,~--- после того как сыны человеческие умножились в те
дни, у них родились красивые и прелестные дочери.
И ангелы, сыны неба, увидели их, и возжелали их, и сказали друг
другу: "давайте выберем себе жён в среде сынов человеческих и родим себе
детей"!
И Семъйяза,  начальник их,  сказал им: "Я боюсь, что вы не захотите
привести в исполнение это дело и тогда я  один  должен  буду  искупать
этот великий грех".
Тогда все они ответили ему и сказали: "Мы все поклянёмся клятвою и
обяжемся друг другу заклятиями не оставлять этого намерения, но привести его в
исполнение".
Тогда  поклялись  все  они вместе и обязались в этом все друг другу
заклятиями: было же их всего двести.
И они спустились на Ардис,  который есть вершина горы Ермон;  и они
назвали  её  горою  Ермон,  потому что поклялись на ней и изрекли друг другу
заклятия.
И вот имена их начальников:  Семъйяза, их начальник, Уракибарамеел,
Акибеел,  Тамиел,  Рамуел,  Данел, Езекеел, Саракуйял, Азаел, Батраал, Анани,
Цакебе, Самсавеел, Сартаел, Турел, Иомъйяел, Аразъйял. Это управители двухсот
ангелов, и другие все были с ними.
И они взяли себе жён, и каждый выбрал для себя одну; и они начали
входить к ним и смешиваться с ними, и научили их волшебству и заклятиям, и
открыли им срезывания корней и деревьев.
Они зачали и родили великих исполинов, рост которых был в три тысячи
локтей.
Они поели всё приобретение людей, так что люди уже не могли
прокармливать их.
Тогда исполины обратились против самих людей, чтобы пожирать их.
И они стали согрешать по отношению к птицам и зверям, и тому, что
движется, и рыбам, и стали пожирать друг с другом их мясо и пить из него кровь.
Тогда сетовала земля на нечестивых.
И Азазел научил людей делать мечи, и ножи, и щиты, и панцири, и
научил их видеть, что было позади них, и научил их искусствам: запястьям, и
предметам украшения, и употреблению белил и румян, и украшению бровей, и
украшению драгоценнейших и превосходнейших камней, и всяких цветных материй и
металлов земли.
И явилось великое нечестие и много непотребств, и люди согрешали, и
все пути их развратились.
Амезарак научил всяким заклинаниям и срезыванию корней, Армарос~---
расторжению заклятий, Баракал~--- наблюдению над звёздами, Кокабел~--- знамениям;
и Темел научил наблюдению над звёздами, и Астрадел научил движению Луны.
И когда люди погибли, они возопили и голос их проник к небу.
Тогда взглянули Михаил, Гавриил, Суръйян и Уръйян с неба и
увидели много крови, которая текла на земле, и всю неправду, которая
совершалась на земле.
И они сказали друг другу: "Голос вопля их (людей) достиг от
опустошённой земли до врат неба.
И ныне к вам, о святые неба, обращаются с мольбою души людей, говоря:
испросите нам правду у Всевышнего".
И они сказали своему Господу Царю: "Господь господей, Бог богов, Царь
царей!
Престол Твоей славы существует во все роды мира: Ты прославлен и
восхвалён!
Ты всё сотворил, и владычество над всем Тебе принадлежит: всё пред
Тобою обнаружено и открыто, и Ты видишь всё, и ничто не могло сокрыться пред
Тобою.
Так посмотри же, что сделал Азазел, как он научил на земле всякому
нечестию и открыл небесные тайны мира.
И заклинания открыл Семъйяза, которому ты дал власть быть вождём его
сообщников.
И пришли они (стражи) друг с другом к дочерям человеческими переспали
с ними, с этими жёнами, и осквернились, и открыли им эти грехи.
Жёны же родили исполинов, и чрез это вся земля наполнилась кровью и
нечестием.
И вот теперь разлученные души сетуют и вопиют к вратам неба и их
воздыхание возносится: они не могут убежать от нечестия, которое совершается
на земле.
И Ты знаешь всё, прежде чем это случилось, и Ты знаешь это и их дела,
и, однако же, ничего не говоришь нам.
Что мы теперь должны сделать с ними за это?
Тогда стал говорить Всевышний, Великий и Святый, и послал
Арсъйялалйюра к сыну Лемеха (Ною) и сказал ему: "Скажи ему Моим именем:
"скройся"!
и объяви ему предстоящий конец!
Ибо вся земля погибнет, и вода потопа готовится прийти на всю землю,
и то, что есть на ней, погибнет.
И теперь научи его, чтобы он спасся и его семя сохранилось для всей
земли"!
И сказал опять Господь Рафуилу: "Свяжи Азазела по рукам и ногам и
положи его во мрак; сделай отверстие в пустыне, которая находится в Дудаеле, и
опусти его туда.
И положи на него грубый и острый камень, и покрой его мраком, чтобы
он оставался там навсегда, и закрой ему лицо, чтобы он не смотрел на свет!
И в великий день суда он будет брошен в жар (в геенну).
И исцели землю, которую развратили ангелы, и возвести земле
исцеление, что Я исцелю её и что не все сыны человеческие погибнут чрез тайну
всего того, что сказали стражи и чему научили сыновей своих; и вся земля
развратилась чрез научения делам Азазела: ему припиши все грехи"!
И Гавриилу Бог сказал: "Иди к незаконным детям, и любодейцам, и к
детям любодеяния и уничтожь детей любодеяния и детей стражей из среды людей;
выведи их и выпусти, чтобы они сами погубили себя чрез избиения друг друга:
ибо они не должны иметь долгой жизни.
И  все они будут просить тебя, но отцы их (исполинов) ничего не
добьются для них (в пользу их), хотя они и надеются на вечную жизнь и на то,
что каждый из них проживёт пятьсот лет".
И Михаилу Бог сказал: "Извести Семъйязу и его соучастников, которые
соединились с жёнами, чтобы развратиться с ними во всей их нечистоте.
Когда все сыны их взаимно будут избивать друг друга и они увидят
погибель своих любимцев, то крепко свяжи их под холмами земли на семьдесят
родов до дня суда над ними и до окончания родов, пока не совершится последний
суд над всею вечностью.
В те дни их бросят в огненную бездну; на муку и в узы они будут
заключены на всю вечность.
И немедленно Семъйяза сгорит и отныне погибнет с ними; они будут
связаны друг с другом до окончания всех родов.
И уничтожь все сладострастные души и детей стражей, ибо они дурно
поступили с людьми.
Уничтожь всякое насилие с лица земли, и всякое злое деяние должно
прекратиться; и явится растение справедливости и правды, и всякое дело будет
сопровождаться благословением; справедливость и правда будут насаждать полную
радость в века.
И теперь в смирении будут поклоняться все праведные и будут пребывать
в жизни, пока не родят тысячу детей, и все дни своей юности и свои субботы они
окончат в мире.
В те дни вся земля будет обработана в справедливости, и будет вся
обсажена деревьями, и исполнятся благословения.
Всякие деревья веселия насадятся на ней, и виноградники насадят на
ней; виноградник, который будет насажен на ней, принесёт плод в изобилии, и от
всякого семени, которое будет на ней посеяно, одна мера принесёт десять тысяч,
и мера маслин даст десять пресов елея.
И ты очисть землю от всякого насилия, и от всякой неправды, и от
всякого греха, и от всякой нечистоты, какая совершается на земле, уничтожь их
с земли.
И все сыны человеческие должны сделаться праведными, и все народы
будут оказывать Мне почесть и прославлять Меня, и все будут поклоняться Мне.
И земля будет очищена от всякого развращения, и от всякого греха, и
от всякого наказания, и от всякого мучения; И Я никогда не пошлю опять на неё
потопа, от рода до рода вовек.
В те дни Я открою сокровищницы благословения, которые на небе,
чтобы низвести их на землю, на произведение и на труд сынов человеческих.
Мир и правда соединятся тогда на все дни мира и на все роды земли.
\vs 1En 3:1
И прежде чем всё это случилось, Енох был сокрыт, и никто из
людей не знал, где он сокрыт, и где он пребывает, и что с ним стало.
И вся его деятельность в течение земной жизни была со святыми и со
стражами.
--- И едва я, Енох, прославил великого Господа и Царя мира, как меня
призвали стражи,~--- меня, Еноха, писца,~--- и сказали мне: "Енох, писец правды!
Иди, возвести стражам неба, которые оставили вышнее небо и святые
вечные места, и развратились с жёнами, и поступили так, как делают сыны
человеческие, и взяли себе жён,  и погрузились на земле в великое
развращение: они не будут иметь на земле ни мира, ни прощение грехов: ибо они
не могут радоваться своим детям.
Избиение своих любимцев увидят они, и о погибели своих детей будут
воздыхать; и будут умолять, но милосердия и мира не будет для них".
И Енох пошёл и сказал Азазелу: "Ты не будешь иметь мира; тяжкий
суд учинён над тобою, чтобы взять тебя, связать тебя, и облегчение, ходатайство
и милосердие не будут долею для тебя за то насилие, которому ты научил, и за
все дела хулы, насилия и греха, которые ты показал сынам человеческим".
Тогда я пошёл далее и сказал всем им вместе; и они устрашились все,
страх и трепет объял их.
И они просили меня написать за них просьбу, чтобы чрез это они обрели
прощение, и вознести их просьбу на небо к Богу.
Ибо сами они не могли отныне ни говорить с Ним, ни поднять очей своих
к небу от стыда за свою греховную вину, за которую они были наказаны.
Тогда я составил им письменную просьбу и мольбу относительно
состояния их духа и их отдельных поступков и относительно того, о чём они
просили, чтобы чрез это получили они прощение и долготерпение.
И я пошёл, и сел при водах Дана в области Дан (т.е. к югу) от
западной стороны Ермона, и читал их просьбу, пока не заснул.
И вот нашёл на меня сон, и напало на меня видение; и я видел видение
суда, которое я должен был возвестить сынам неба и сделать им порицание.
И как только я пробудился от сна, то пришёл к ним; и все они сидели
печальные с закрытыми лицами, собравшись в Ублес-йяеле, который лежит между
Ливаном и Сенезером.
И я рассказал им все видения, которые видел во время своего сна, и
начал говорить те слова правды и порицать стражей неба.
То, что здесь далее написано, есть слово правды и наставления,
данное мне вечными стражами, как повелел им Святый и Великий в том видении.
Я видел во время видения моего сна то, что я буду теперь рассказывать
моим плотским языком и моим дыханием, которое Великий вложил в уста людям,
чтобы они говорили им и понимали это сердцем (мыслию).
Как сотворил Он всех людей и даровал им понимание слова благоразумия,
так Он сотворил и меня и дал мне право порицать стражей~--- сынов неба.
"Я написал вашу просьбу, и мне было открыто в видении, что именно
ваша просьба не будет для вас исполнена до всей вечности, дабы совершился над
вами суд, и ничто не будет для вас исполнено.
И отныне вы не взойдёте уже на небо до всей вечности и на земле вас
должны связать на все дни мира: такой произнесён приговор.
Но прежде этого вы увидите уничтожение ваших возлюбленных сынов, и вы
будете обладать ими, но они падут пред вами от меча.
Ваша просьба за них не будет исполнена для вас, как и та (моя)
просьба за вас; вы не можете даже в плаче и воздыхании произносить устами ни
одного слова из писания, которое я написал".
И видение мне явилось таким образом: вот тучи звали меня в видении и
облако звало меня; движение звёзд и молний гнало и влекло меня; и ветры в
видении дали мне крылья и гнали меня.
Они вознесли меня на небо, и я приблизился к одной стене, которая
была устроена из кристалловых камней и окружена огненным пламенем; и она стала
устрашать меня.
И я вошёл в огненное пламя, и приблизился к великому дому, который
был устроен из кристалловых камней; стены этого дома были подобны наборному
полу (паркет или мозаика) из кристалловых камней, и почвою его был кристалл.
Его крыша была подобна пути звёзд и молний с огненными херувимами
между нею (крышею) и водным небом.
Пылающий огонь окружал стены дома, и дверь его горела огнём.
И я вступил в тот дом, который был горяч как огонь и холоден как лёд;
не было в нём ни веселия, ни жизни~--- страх покрыл меня и трепет объял меня.
И так как я был потрясён и трепетал, то упал на своё лицо; и я видел
в видении.
И вот там был другой дом, больший, нежели тот; все врата его стояли
предо мной отворёнными, и он был выстроен из огненного пламени.
И во всём было так преизобильно: во славе, в великолепии и величии,
что я не могу дать описания вам его величия и его славы.
Почвою же дома был огонь, а поверх его была молния и путь звёзд, и
даже его крышею был пылающий огонь.
И я взглянул и увидел в нём возвышенный престол; его вид был как
иней, и вокруг него было как бы блистающее солнце и херувимские голоса.
И из-под великого престола выходили реки пылающего огня, так что
нельзя было смотреть на него.
И Тот, Кто велик во славе, сидел на нём; одежда Его была блестящее,
чем само солнце, и белее чистого снега.
Ни ангел не мог вступить сюда, ни смертный созерцать вид лица самого
Славного и Величественного.
Пламень пылающего огня был вокруг Него, и великий огонь находился
пред Ним, и никто не мог к Нему приблизиться из тех, которые находились около
Него: тьмы тем были пред Ним, но Он не нуждался в святом совете.
И святые, которые были вблизи Его, не удалялись ни днём, ни ночью и
никогда не отходили от Него.
И  я с тех пор имел покрывало на своём челе, потому что трепетал;
тогда позвал меня Господь собственными устами и сказал мне: "Пойди, Енох, сюда
и к Моему святому слову"!
И Он повелел подняться мне и подойти к вратам~--- я же опустил своё
лицо.
И Он отвечал и сказал мне Своим словом: Слушай!
Не страшись, Енох, ты праведный муж и писец правды; подойди сюда и
выслушай Моё слово!
И ступай, скажи стражам неба, которые послали тебя, чтобы ты просил
за них: вы должны попросить за людей, а не люди за вас.
Зачем вы оставили вышнее, святое, вечное небо, и преспали с жёнами,
и осквернились с дочерьми человеческими, и взяли себе жён, и поступали как сыны
земли, и родили сынов-исполинов?
Будучи духовными, святыми, в наслаждении вечной жизни, вы
осквернились с жёнами, кровию плотской родили детей, возжелали крови людей и
произвели плоть и кровь, как производят те, которые смертны и тленны.
Ради того-то Я им и дал жён, чтобы они оплодотворяли их, и чрез них
рождали бы детей, как это обыкновенно происходит на земле.
Но вы были прежде духовны, призваны к наслаждению вечной, бессмертной
жизни на все роды мира.
Посему Я не сотворил для вас жён, ибо духовные имеют своё жилище на
небе.
И теперь исполины, которые родились от тела и плоти, будут называться
на земле злыми духами и на земле будет их жилище.
Злые существа выходят из тела их; так как они сотворены свыше и их
начало и первое происхождение было от святых стражей, то они будут на земле
злыми духами, и будут называться злыми духами.
А духи неба имеют своё жилище на небе, а духи земли, родившиеся на
земле, имеют своё жилище на земле.
И духи исполинов, которые устремляются на облака, погибнут, и будут
низринуты, и станут совершать насилие, и производить разрушения на земле, и
причинять бедствия; они не будут принимать пищи, и не будут жаждать, и будут
невидимы.
И те существа не восстанут против сынов человеческих и против жён,
так как они произошли от них.
В дни избиения и погибели и смерти исполинов, лишь только души
выйдут из тел, их тело должно предаться тлению без суда; так будут погибать
они до того дня, когда великий суд совершится над великим миром,~--- над стражами
и нечестивыми людьми.
И теперь скажи стражам, которые послали тебя, чтобы ты просил за них,
и которые жили прежде на небе, теперь скажи им: "Вы были на небе, и хотя
сокровенные вещи не были ещё открыты вам, однако вы узнали незначительную тайну
и рассказали её в своём жестокосердии жёнам, и чрез эту тайну жёны и мужья
причиняют земле много зла".
Скажи им: "Для вас нет мира".
\vs 1En 4:1
И они (ангелы) унесли меня в одно место, где были фигуры, как
пылающий огонь, и когда они хотели, то казались людьми.
И они привели меня к месту бури и на одну гору, конец вершины которой
доходил до неба.
И я увидел ярко блестящие места и гром на краях их; в глубине этого
огненный лук стрелы и колчан для них, и огненный меч, и все молнии.
И они донесли меня до так называемой воды и до огня запада, который
принимает в себя каждый вечер заходящее солнце.
И я пришёл к огненной реке, огонь которой жидкий, как вода, и которая
впадает в великое море к западу.
И я видел все великие реки, и дошёл до великого мрака, и пришёл туда,
где шествуют все смертные.
И  я  видел горы мрачных туч зимнего времени и место, куда впадает
вода целой бездны.
И я видел устье всех рек земли и устье бездны.
И я видел хранилища всех ветров, и видел, как Он изукрасил этим
всё творение, и видел основание земли.
И я видел краеугольный камень земли, и видел четыре ветра, которые
носят землю и основание неба.
И я видел, как ветры растягивают высоты неба, и они носятся между
небом и землёю~--- это столпы неба.
И я видел ветры, которые кружат небо, которые несут солнечный круг и
все звёзды к заходу.
И я видел ветры на земле, которые носят тучи; и видел пути ангелов,
и видел в конце земли вверху основание неба.
И я пошёл далее к югу, который горит день и ночь,~--- туда, где
находятся семь гор из драгоценных камней,~--- три к востоку и три к югу: и
именно те, которые к востоку, одна из цветных камней, и одна из перловых
камней, и одна из сурьмы; а те, которые к югу, из красных камней.
Средняя же, достигавшая до неба, как престол Божий, была из
алебастра, и вершина престола из сапфира.
И я видел пылающий огонь, который был во всех горах.
И я видел там одно место по ту сторону великой земли: там собирались
воды.
И я видел глубокую расселину в земле со столбами небесного огня; и я
видел между ними ниспадающие столбы небесного огня, которые нельзя было
сосчитать ни в направлении к верху, ни к низу.
И над тою расселиной я видел одно место, которое не имело ни небесной
тверди над собою, ни земного основания под собою; на нём не было ни воды, ни
птиц, но это было пустое место.
И было ужасно то, что я видел там: семь звёзд, как великие горящие
горы и как духи, которые просили меня.
Ангел сказал мне: "Это то место, где оканчивается небо и земля; оно
служит темницей для звёзд небесных и для воинства небесного.
И эти звёзды, которые катятся над огнём, суть те самые, которые
преступили повеление Божие пред своим восходом, так как они пришли не в своё
определённое время.
И Он разгневался на них и связал их до времени, когда окончится их
вина,~--- в год тайны".
И Уриил сказал мне: "Здесь будут находиться духи ангелов,
которые соединились с жёнами и, принявши различные виды, осквернили людей и
соблазнили их, чтобы они приносили жертвы демонам, как богам,~--- будут
находиться именно в тот день, когда над ними будет произведён великий суд, пока
не постигнет их конечная участь.
Так же и с жёнами их, которые соблазнили ангелов неба, будет
поступлено точно так же, как и с друзьями их.
И только я, Енох, созерцал пределы всего, и ни один человек не видел
их так, как видел их я.
И вот имена святых ангелов, которые стерегут: Уриил, один из
святых ангелов, ангел грома и колебания; Руфаил, один из святых ангелов, ангел
духов людей; Рагуил, один из святых ангелов, который карает мир и светила;
Михаил, один из святых ангелов, поставленный над лучшею частью людей,~--- над
избранным народом; Саракаел, один из святых ангелов, который поставлен над
душами сынов человеческих, склонивших духов к греху; Гавриил, один из святых
ангелов, который поставлен над змеями, и над раем, и над херувимами.
И я обошёл кругом до одного места, где не было никакой вещи.
И я видел там нечто страшное, ни небо возвышенное и ни землю
утверждённую, но одно пустое (пустынное) место, величественное и страшное.
И здесь я видел семь звёзд небесных, вместе связанных в этом месте,
подобным великим горам и пылающим как бы огнём.
На этот раз я сказал: "За какой грех они связаны и за что они сюда
изгнаны"?
Тогда мне сказал Уриил, один из святых ангелов, который был при мне
как мой путеводитель: "Енох, для чего ты разведываешь, и для чего разузнаёшь,
и спрашиваешь, и любопытствуешь?
Это те звёзды, которые преступили повеление Всевышнего Бога, и они
связаны здесь до тех пор, пока не окончится тьма миров,~--- число дней их вины".
И отсюда я пошёл в другое место, которое было ещё страшнее, чем это,
и увидел нечто страшное: там был великий огонь, который пылал и горел, и он
имел разделения; он был ограничен (окружён) совершенною пропастью; великие
огненные столбы низвергались туда; но его (огня) протяжения и величины я не мог
рассмотреть, и не в состоянии был даже взглянуть, откуда он происходит.
Тогда я сказал: "Как страшно это место и как мучительно осматривать
его!"
Тогда отвечал мне Уриил, один из святых ангелов, который был при мне;
он отвечал мне и сказал: "Енох, к чему такой страх и трепет в тебе на этом
ужасном месте и при виде этого мучения?"
И он сказал мне: "Это место~--- темница ангелов, и здесь они будут
содержаться заключёнными до вечности".
\vs 1En 5:1
Отсюда я пошёл в другое место, и он (Руфаил) показал мне на
западе большой высокий горный хребет, твёрдые скалы и четыре прекрасных места.
И между ними (последними) были глубокие, и обширные, и совершенно
выглаженные настолько гладко, как нечто, что катится, и глубокое, и мрачное
на вид.
На этот раз ответил мне Руфаил, один из святых ангелов, который был
со мною, и сказал мне: "Эти прекрасные места назначены для того, чтобы на них
собирались духи,~--- души умерших; для них они созданы, чтобы все души сынов
человеческих собирались здесь.
Места эти созданы для них местами жилища до дня их суда и до
определённого для них срока, и срок этот велик: он продолжится дотоле, пока не
совершится над ними великий суд".
И я видел духов сынов человеческих, которые умерли, и их голос
проникал до неба и сетовал.
На этот раз я спросил ангела Руфаила, который был со мною, и сказал
ему: "Чей это там дух, голос которого так проникает вверх и сетует?"
И он отвечал мне и сказал мне так: "Это дух, который вышел из Авеля,
убитого своим братом Каином; и он жалуется на него, пока семя его (Каина) не
будет изглажено с лица земли и из семени людей не будет уничтожено его семя".
И поэтому я спросил тогда о нём (об Авеле) и о суде над всеми и
сказал: "Почему одно место отделено от другого?"
И он отвечал мне и сказал мне: "Эти три остальные отделения сделаны
для того, чтобы разделять души умерших.
И души праведных отделены таким образом: там есть источник воды, над
которым свет.
Точно также сделано такое отделение и для грешников, когда они
умирают и погребаются на земле без того, что суд над ними ещё не произведён при
их жизни.
Здесь отделены их души, в этом великом мучении, пока не наступит
великий день суда и наказания, и мучения для хулителей до вечности, и мщения
для их душ; и он (ангел наказания) связал их здесь до вечности.
И если это было пред вечностью, тогда это (последнее) отделение
сделано для душ тех, которые сетуют и возвещают о своей погибели, так как они
были умерщвлены в дни грешников.
Таким образом, это отделение сделано для душ людей, которые были не
праведными, а грешниками, скончавшись в вине; они будут находиться возле
виновных и подобны им, но их души не умрут до дня суда и не выйдут отсюда.
Тогда я прославил Господа славы и сказал: "Будь прославлен, Господь
мой, Господь славы и справедливости, всё направляющий в вечность!"
Оттуда я пошёл в другое место к западу, к пределам земли.
И я видел здесь горящий огонь, который тёк беспрерывно, и ни днём, ни
ночью не прекращал своего течения, но равномерно тёк.
И я спросил Рагуила, говоря: "Что это такое там, что не имеет покоя?"
На этот раз отвечал мне Рагуил, один из святых ангелов, который был
со мною, и сказал мне: "Этот горящий огонь на западе, течение которого ты
видел, есть огонь всех светил небесных".
Оттуда я пошёл в другое место земли, и он (Михаил) показал мне
там горный хребет огненный, который горел день и ночь.
И я взошёл на него и увидел семь великолепных гор, из которых каждая
отделена от другой, и великолепные (драгоценные), прекрасные камни; всё было
великолепно и славного вида и прекрасной видимости; три горы расположены к
востоку, одна над другой укреплена, и три к югу, одна над другой укреплена;
здесь были и глубокие вьющиеся долины, из которых ни одна не примыкала к
другой.
И седьмая гора была между ними; в своей же вышине они все были
подобны тронному седалищу, которое было окружено благовонными деревьями.
И  между ними было одно дерево с благоуханием, которого я ещё никогда
не обонял ни от тех, ни от других деревьев; и никакой другой запах не был похож
на его запах; его листья, цветы, ствол не гниют вечно, и плод его прекрасен; а
его плод подобен грозду пальмы.
На этот раз я сказал: "Посмотри на это прекрасное дерево: прекрасны
на вид и приятны его листья (ветви), и его плод очень приятен для ока".
Тогда отвечал мне Михаил, один из святых и почитаемых ангелов, бывший
со мною, который был поставлен над этим.
И он сказал мне: "Енох, что ты спрашиваешь меня о запахе этого
дерева и стремишься узнать?"
Тогда я, Енох, отвечал ему, говоря: "Обо всём желал бы я узнать, но
особенно об этом дереве".
И он отвечал мне, говоря: "Эта высокая гора, которую ты видел, и
вершина, которая подобна престолу Господа, есть Его престол, где воссядет
Святый и Великий, Господь славы, вечный Царь, когда Он сойдёт, чтобы посетить
землю с милостью.
И к этому дереву с драгоценным запахом не позволено прикасаться ни
одному из смертных до времени великого суда; когда всё будет искуплено и
окончено для вечности, оно будет отдано праведным и смиренным.
От его плода будет дана жизнь избранным; оно будет пересажено на
север к святому месту,~--- к храму Господа, великого Царя.
Тогда они будут радоваться полною радостью и ликовать в Святом; они
будут воспринимать запах его в свои кости, и продолжительную жизнь они будут
жить на земле, как жили их отцы; и в дни их жизни не коснётся их ни печаль, ни
горе, ни труд, ни мучение".
Тогда я прославил Господа славы, вечного Царя, за то, что Он уготовал
это для праведных людей, и создал такое, и обещал дать им.
И оттуда я пошёл в средину земли, и видел благословенное и
плодородное место, где были ветви, которые укоренялись и вырастали из
срубленного дерева.
И там я видел святую гору, и под горой~--- к востоку от неё~--- воду,
которая текла к югу.
И я видел к востоку другую гору такой же вышины, и между ними обоими
глубокую долину, но неширокую; в ней также текла вода возле горы.
И на западе от неё была другая гора, ниже той и невысокая, и внизу
её, между ними (горами) обоими, была долина; и другие долины глубокие и сухие
были в конце всех трёх.
И все долины были глубокие, но не широкие, из твёрдого скалистого
камня; и деревья были насажены на них.
И я удивился скалам, и удивился долине, и удивился чрезвычайно.
Тогда я сказал: "Для чего эта благословенная страна, которая
вся наполнена деревьями, и в промежутке (между горами) эта проклятая долина?"
Тогда отвечал мне Уриил, один из святых ангелов, который был со мною,
и сказал мне: "Эта проклятая долина для тех, которые прокляты до вечности;
здесь должны собраться все те, которые говорят своими устами непристойные речи
против Бога, и дерзко говорят о Его славе; здесь соберут их, и здесь место их
наказания.
И в последнее время будет зрелище праведного суда над ними пред лицом
праведных навсегда в вечности; за это те, которые обрели милосердие, будут
прославлять Господа славы, вечного Царя.
И в дни суда над ними (грешниками) они (праведные) прославят Его за
милосердие, по которому он назначил им такой жребий".
Тогда и я прославил Господа славы, и говорил к Нему, и вспоминал Его
величие, как подобает.
Оттуда я пошёл к востоку, в самую средину горного хребта,
(находящегося в) пустыне, и здесь я не видел ничего, кроме одной равнины.
Но она была наполнена деревьями тех же семян, и вода струилась на неё
сверху.
Можно было видеть, насколько орошение, которое она поглощала, было
обильное, можно было видеть и то, что как на севере, так и на западе и как
повсюду, так и здесь поднималась вода и роса.
И я пошёл в другое место, прочь от пустыни, приближаясь к
горному хребту на востоке.
И там я видел деревья суда, особенно же такие, которые издают запах
ладана и мирры и которые были не похожи на обыкновенные деревья.
И над этим, высоко над ними (деревьями), над восточною горою и
недалеко от неё, видел я другое место, именно~--- долины с водой, которые не
иссякают.
И я видел прекрасное дерево, запах которого, как запах мастикса.
И по сторонам тех долин я видел благовонную корицу.
И я поднялся вверх над ними (долинами или деревьями), направляясь
ближе к востоку.
И я видел другую гору с деревьями, из которой текла вода и из
которой выходило нечто подобное нектару, что называют сарира и гальбан.
И  над той горой я видел другую гору, на которой были алойные
деревья; и те деревья изобиловали миндалеподобным твёрдым веществом.
И если взять тот плод, то он был лучше, чем всякие благовония.
И после этих благовоний, как только я взглянул к северу выше тех
гор, я увидел там ещё семь гор, изобиловавших драгоценными нардами и
благовонными деревьями, корицей и перцем.
Оттуда я пошёл на вершину тех гор далеко к востоку, и подвинулся
далее, пройдя над Эритрейским морем, и ушёл далеко от него, и прошёл над
ангелом Цутелем.
И я пришёл в сад правды и увидел разнообразное множество тех
деревьев; там росло много больших деревьев,~--- благовонных, великих, очень
прекрасных и великолепных,~--- и дерево мудрости, доставляющее великую мудрость
тем, которые вкушают от него.
И оно похоже на кератонию; его плод, подобный виноградной кисти,
очень прекрасен; запах дерева распространяется и проникает далеко.
И я сказал: "Как прекрасно это дерево и как прекрасен и прелестен его
вид!"
И святой ангел Руфаил, который был со мною, отвечал мне и сказал:
"Это то самое дерево мудрости, от которого твои предки, твой старый отец и
старая мать вкусили и обрели познание мудрости, и у них открылись очи, и они
узнали, что были наги и были изгнаны из сада".
Оттуда я пошёл к пределам земли и увидел там великих зверей, из
которых каждый был отличен от другого, а также птиц, разнообразных по наружной
красоте и по голосу, из которых каждая была отлична от другой.
И на востоке от тех зверей я видел пределы земли, на которых покоится
небо, и открытые врата неба.
И я видел, как выходят звёзды небесные, и сосчитал врата, из которых
они выходят, и записал все выходы их,~--- о каждой из них особо, по числу их, их
именам, их связи, их положению, их времени и их месяцам,~--- так, как показал мне
это ангел Уриил, который был со мною.
Всё показал он мне и записал мне; их имена он также записал для меня,
и их законы и их отправления.
Оттуда я пошёл к северу к пределам земли, и там я видел великое
и славное чудо на пределах всей земли.
Здесь я видел трое открытых небесных врат на небе; из них выходят
северные ветры; если там (из них) дует, то бывает холод, град, иней, снег, роса
и дождь.
Из одних врат (средних) дует ко благу; но если они (ветры) дуют чрез
двое других врат, то бывает бурно и на землю приносится бедствие, и они дуют
тогда бурно.
Оттуда я пошёл к западу к пределам земли и увидел тогда трое
открытых врат подобно тому, как я видел их на востоке,~--- одинаковые врата и
одинаковые выходы.
Оттуда я пошёл на юг к пределам земли и видел там двое открытых
врат неба; из них выходит южный ветер, а с ним~--- роса, дождь и ветер.
Оттуда я пошёл к востоку к пределам неба и видел здесь трое восточных
небесных врат открытых и над ними маленькие врата.
Чрез каждые маленькие врата проходят звёзды небесные и бегут к вечеру
(к западу) на колеснице, которая им назначена.
И как только я увидел это, то прославил Господа, и таким образом я
всякий раз прославлял Господа славы, который сотворил великие и славные чудеса,
чтобы показать величие Своего творения ангелам и душам людей, дабы они
восхваляли Его творение и дабы все Его твари видели дело Его могущества,
восхваляли великое дело Его рук и славили Его довеку.
\vs 1En 6:1
Второе видение мудрости, которое видел Енох, сын Иареда, сына
Малелеила, сына Каинана, сына Еноса, сына Сифа, сына Адама.
И вот начало речи мудрости,  которую я начал говорить и высказывать
живущим на земле; слушайте вы, древнейшие, и обратите внимание, потомки, на
святые слова, которые я буду говорить пред Господом духов.
Справедливо назвать тех (древних) прежде всего, но и потомков мы не
будем удерживать от начала премудрости.
И до сегодня никогда не была дарована от Господа духов кому-либо та
мудрость, которую я получил по моему разумению, по благоволению
Господа духов, от которого мне назначен жребий вечной жизни.
Три притчи были долею для меня, и я начал их рассказывать тем, которые
населяют твердь.
\vs 1En 7:1
Первая притча.
Когда откроется общество праведных, и грешники будут судимы за свои
грехи, и будут изгнаны с лица земли, и когда Праведный явится пред очами
избранных праведников, дела которых взвешены Господом духов, и свет откроется
праведным и избранным, живущим на земле,~--- то где тогда будет жилище грешников
и убежище тех, которые отвергли Господа духов?
Было бы лучше для них, если б они никогда не рождались.
И когда тайны праведных будут открыты, тогда грешники будут судимы и
нечестивые будут отвергнуты от праведных и избранных.
И отныне не будут более сильными и вознесёнными те, которые владеют
землёю, и не будут в состоянии видеть лицо святых, ибо свет Господа духов будет
сиять на лица святых, и праведных, и избранных.
И сильные цари погибнут в то время и будут преданы в руки праведных и
святых.
И с тех пор никто не будет (иметь возможности) молить Господа духов о
милости, ибо жизнь их (людей) окончится.
И это случится в те дни, когда избранные и святые дети сойдут с
высоких небес и их семя соединится с сынами человеческими.
В те дни Енох получил книги гнева и ярости и книги беспокойства и
смятения, и в это самое время меня унесла прочь от земли туча и буря, и
принесла меня к пределам неба.
И здесь я видел другое видение, именно~--- жилища праведных и ложа
святых.
Здесь мои очи видели жилища возле ангелов и их ложа возле святых,
видел, как они молились, и просили, и умоляли за сынов человеческих, и правда
текла пред ними, как вода, и милосердие, как роса на земле: так бывает между
ними от века до века.
И в те дни мои очи видели место избранных правды и веры, и как правда
господствует в те дни, и как неисчислимо велико множество праведных и избранных
пред Ним от века до века.
И я видел жилища их под крыльями Господа духов, и видел, как все
праведные и избранные украшены пред Ним как бы огненным сиянием, и их уста
полны славословия, и их губы хвалят имя Господа духов, и правда не преходит
пред Ним.
Здесь желал я жить, и моя душа стремилась к тому жилищу; здесь уже
прежде была уготована мне участь, ибо так постановлено относительно меня у
Господа духов.
В те дни я хвалил и превозносил имя Господа духов благословениями и
славословиями, ибо Он определил мне благословение и славу.
Долго рассматривали мои очи то место, и я прославил Его (Господа),
говоря: Хвала Ему и да прославится Он от начала до вечности!
Пред Ним нет прехождения; Он знает, прежде чем создан мир, что он
такое и что будет от рода до рода.
Тебя славят те, которые не спят они стоят пред Тобою славою, и
прославляют, хвалят и превозносят Тебя, говоря: "свят, свят, свят
Господь духов, Он наполняет землю духами!"
И здесь мои очи видели всех тех, которые не спят, как они стоят пред
Ним, и прославляют, и говорят: "Будь прославлен Ты и да будет прославлено имя
Господа от века до века!"
И моё лицо изменилось, так что я не мог больше видеть.
И после этого я видел тысячу тысяч и тьму тем, несметно и
неисчислимо многих, стоящих пред славою Господа духов.
Я видел, и на четырёх сторонах престола Господа духов я заметил
четыре лица, отличные от тех, которые стояли там (1 ст.), и я узнал имена их,
так как ангел, пришедший со мною (или ко мне), открыл мне имена их и показал
мне все сокровенные вещи.
И я слышал глас тех четырёх лиц, как они пели хвалу пред Господом
славы.
Первый голос прославляет Господа духов от века и до века.
И другой голос слышал я, прославляет Избранного и избранных, которые
взвешены Господом духов.
И третий голос слышал я, просит, и молится за живущих на земле, и
умоляет во имя Господа духов.
И слышал я четвёртый голос, как он отражал врагов и не дозволял им
приступить к Господу духов, чтобы клеветать или жаловаться на живущих на земле.
После этого я спросил ангела мира, шедшего со мною, который показал
мне всё, что сокрыто, и сказал ему: "Кто эти четыре лица, которых я видел и
глас, которых я слышал и записал?"
И он сказал мне: "Этот первый~--- есть милосердный и долготерпеливый
святой Михаил; и другой, поставленный над всеми болезнями и над всеми ранами
сынов человеческих, есть Руфаил; и третий, поставленный над всеми силами, есть
святой Гавриил; и четвёртый, поставленный над покаянием и надеждою тех,
которые получат в наследие вечную жизнь, есть Фануил".
И вот четыре ангела всевышнего Бога, и четыре голоса их я слышал в те
дни.
И после этого я видел все тайны неба, и как разделено царство, и
как дела людей взвешены на весах.
Там видел я жилища избранных и жилища святых; и мои очи видели там,
как изгоняются оттуда все грешники, которые отвергли имя Господа духов, как
отражают их, и для них там нет места вследствие наказания, которое исходит от
Господа духов.
И там мои очи видели тайны молний и грома, и тайны ветров, как они
распределены,  чтобы дуть на землю, и тайны туч и росы; и там видел я,
откуда они выходят в том самом месте и как отсюда насыщается пыль
земная.
И там видел я замкнутые хранилища, из которых распределяются ветра,
и хранилища града, и хранилища тумана и туч, и Его туча, которая носится над
землёю до вечности.
И я видел хранилища Солнца и Луны, откуда они выходят и куда
возвращаются, и их славное возвращение; и я видел, как одно (т.е. Солнце) имеет
преимущество перед другой, видел и их определённое движение, как они не
преступают пути, ничего не прибавляя к своему пути и ничего не убавляя от него,
и соблюдают верность между собой, сохраняя клятву.
Прежде всего, выходит Солнце и совершает свой путь по Вселенной
Господа духов, и могущественно имя Его от века до века; за ним следует видимый
и невидимый путь Луны; и я видел, как она оканчивает движение по своему пути в
том месте днём и ночью, одно (светило, т.е. Луна), противостоя другому
(Солнцу), пред Господом духов; и они благодарят, и прославляют, и не
успокаиваются, так как их благодарение служит для них покоем.
Ибо сияющее Солнце совершает много обращений для благословения и для
проклятия; и движение Луны по её пути есть свет для праведных и для грешников
во имя Господа, Который положил разделение между светом и тьмою, и разделил
души людей, и утвердил души праведных во имя Своей правды.
Ибо ни ангел не нарушает этого, и никакая сила не может нарушить
этого (установленного Богом), но Судья видит их все (души людей) и судит их все
пред Собою.
Мудрость не нашла на земле места, где бы ей жить, и потому
жилище её стало на небесах.
Пришла мудрость, чтобы жить между сынами человеческими, не нашла себе
места; тогда мудрость возвратилась назад в своё место, и заняла своё положение
между ангелами.
И неправда вышла из своих хранилищ: не искавшая его (приёма), она
нашла его и жила между людьми, как дождь в пустыне и как роса в земле жаждущей.
И видел я опять молнии и звёзды небесные, как Он призывал их
всех отдельно по именам и они внимали Ему.
И я видел, как они взвешены правильными весами по мере их света, по
обширности их мест и времени их появления и обращения (видел, как одна молния
рождает другую), и их обращение по числу ангелов, и как они сохраняют между
собой верность.
И я спросил ангела, который шёл со мною и показал мне, что сокрыто:
"Кто это"?
И он сказал мне: "Образ их показал тебе Господь духов: это имена
праведных, которые живут на земле и веруют во имя Господа духов во всю
вечность".
И иное также видел я относительно молний, как они возникают из
звёзд, и становятся молниями, и ничего не могут удержать при себе.
\vs 1En 8:1
И  вот вторая притча относительно тех, которые отвергают имя
жилища святых и имя Господа духов.
Они не взойдут на небо, и на землю не придут они: таков будет жребий
грешников, которые отвергают имя Господа духов и которые сохраняются, таким
образом, на день страданий и скорби.
В тот день Избранный сядет на престоле славы и произведёт выбор между
делами их (людей) и местами без числа, и дух их сделается сильным в их
внутренности, ибо они увидят моего Избранного и тех, которые умаляли Моё святое
и славное имя.
И в тот день Я пошлю Моего Избранного жить между ними, и преобразую
небо, и приготовлю его для вечного благословения и света.
И я изменю землю, и приготовлю её для благословения,  и поселю  на
ней Моих избранных; грех же и преступления исчезнут на ней,~--- они не появятся.
Ибо Я увидел и насытил миром Моих праведных, и поставил их пред Собою;
для грешников же у Меня предстоит суд, дабы уничтожить их с лица земли.
И там я видел Единого, имевшего главу дней (престарелую  главу), и
Его глава была бела, как руно; и при Нём был другой, лице которого было подобно
виду человека, и Его лице полно было прелести и подобно одному из святых
ангелов.
И я спросил одного из ангелов, который шёл со мною и показывал мне все
сокровенные вещи, о том Сыне человеческом, кто Он, и откуда Он, и почему Он
идёт с Главою дней?
И он отвечал мне и сказал: "Это Сын человеческий, Который имеет правду,
при Котором живёт правда, и Который открывает все сокровища того, что сокрыто,
ибо Господь духов избрал Его, и жребий Его пред Господом духов превзошёл всё,
благодаря праведности.
И этот Сын человеческий, Которого ты видел, поднимет царей и
могущественных с их лож и сильных с их престолов, и развяжет узы сильных, и
зубы грешников сокрушит.
И Он изгонит царей с их престолов и из их царств, ибо они не
превозносят Его, и не прославляют Его, и не признают с благодарностью, откуда
досталось им царство.
И лицо сильных Он отвергнет, и краска стыда покроет их; мрак будет их
жилищем, и слёзы их ложем, и они не будут иметь надежды встать со своих лож,
так как они не превозносят имя Господа духов.
И это те, которые осуждают звёзды небесные и возвышают свои руки
против Всевышнего, и попирают землю и на ней живут; все дела их неправда, и они
открывают неправду; сила их основывается на богатстве, и вера их относится к
богам, сделанным их же руками; и они отвергли Господа духов.
И они изгоняются из домов их общественного собрания и из домов
верующих, которые взвешены во имя Господа духов.
И в те дни восходит молитва праведных и кровь праведного от земли
к Господу духов.
В те дни святые ангелы, живущие вверху на небесах, соединившись
вместе,
будут единым гласом просить, и молить, и прославлять, и благодарить, и
восхвалять имя Господа духов ради крови праведных, которая пролита, и ради
молитв праведных, что она не может быть тщетной пред Господом духов и что
совершён суд для них, и им не нужно терпеть (или дожидаться суда) вечного.
И в те дни я видел Главу дней как Он восседал на престоле своей славы
и книги живых были раскрыты пред Ним, и видели всё Его воинство, которое
находится вверху и на небесах и окружает Его, предстоя пред Ним.
И сердца святых были полны радостью, ибо исполнилось число правды, и
молитвы праведных услышана, и кровь праведного искуплена (или отомщена) пред
Господом духов.
И в том месте я видел источник правды, который был неисчерпаем;
его окружали вокруг многие источники мудрости, и все жаждущие пили из них и
исполнялись мудростью, и имели свои жилища около праведных, и святых, и
избранных.
И в тот час был назван тот Сын человеческий возле Господа духов и Его
имя пред Главою дней.
И прежде чем Солнце и знамения были сотворены, прежде чем звёзды
небесные были созданы, Его имя было названо пред Господом духов.
Он будет жезлом для праведных и святых, чтобы они опёрлись на Него и
не падали; и Он будет светом народов и чаянием тех, которые опечалены в своём
сердце.
Пред Ним упадут и поклонятся все живущие на земле, и будут хвалить и
прославлять, и петь хвалу имени Господа духов.
И посему Он был избран и сокрыт пред Ним, прежде даже чем создан мир;
и Он будет пред Ним до вечности.
И премудрость Господа духов открыла Его святым и избранным, ибо Он
охраняет жребий праведных, так как они возненавидели и презрели этот мир
неправды, и все его произведения и пути возненавидели во имя Господа духов; ибо
во имя Его они спасаются, и Он становится мстителем за их жизнь.
И в те дни потупили взор цари земли и сильные, владеющие твердью,
страшась за дела своих рук, ибо в день своей печали и бедствия они не спасут
своих душ.
И Я предал их в руки Моих избранных: как солома в огне и как свинец в
воде, они сгорят пред лицом праведных и потонут пред лицом святых, и никакого
следа более не останется от них.
И в день их бедствия водворится покой на земле; они падут пред Ним и
не восстанут опять; не будет никого, кто бы взял их в свои руки и поднял: ибо
они отвергли Господа духов и Его Помазанника.
Имя Господа духов будет прославлено.
Ибо мудрость излилась на Сына человеческого, как вода, и слава не
прекращается пред Ним от века до века.
Ибо Он силён во всех тайнах правды, и неправда прейдёт пред Ним, как
тень, и не будет иметь постоянства, так как Избранный восстал пред Господом
духов; и Его слава от века до века, и Его могущество от рода до рода.
В Нём живёт дух мудрости и дух Того, Кто даёт проницательность, и дух
учения и силы, и дух тех, которые почили в правде.
И Он будет судить сокровенные вещи, и никто не осмелится вести пред
Ним пустую речь, ибо Он избран пред Господом духов, и Его благоволению.
И в те дни совершится перемена со святыми и избранными: свет дней
будет обитать пред ними, и слава, и честь будут дарованы святым.
И в день бедствия соберётся нечестие на грешников, праведные же
победят во имя Господа духов; и Он покажет это другим, чтобы они принесли
покаяние и оставили дела своих рук.
Они не будут иметь чести пред Господом духов, но будут спасены во имя
Его; И Господь духов умилосердится над ними, ибо Его милосердие велико.
И праведен Он в Своём суде и пред Его славою, и на Его суде не устоит
неправда: кто не принесёт покаяния пред Ним, тот погибнет.
Но отныне Я не буду более милосердным к ним, говорит Господь духов.
И в те дни земля возвратит вверенное ей и царство мёртвых
возвратит вверенное ему, что оно получило, и преисподняя отдаст назад то, что
обязана отдать.
И Он изберёт между ними праведных и святых, ибо пришёл день, чтобы
спастись им.
И Избранный в те дни сядет на престоле Своём, и все тайны мудрости
будут истекать из мыслей Его уст, ибо Господь духов даровал Ему это и прославил
Его.
И в те дни горы будут скакать, как овны, и холмы будут прыгать, как
агнцы, насытившиеся молоком; и все они сделаются ангелами на небе.
Их лицо будет сиять от радости, так как в те дни восстанет Избранный;
и земля возрадуется, и на ней будут жить праведные, и избранные будут ходить и
шествовать по ней.
И после тех дней, в том месте, где я видел все видения
относительно того, что сокрыто,~--- я был восхищён в вихре ветра и приведён к
западу,~--- там очи мои видели сокровенные предметы неба, всё, что произойдёт на
земле: одну гору из железа, одну из меди, одну из серебра, одну из золота,
одну из жидкого металла и одну из свинца.
И я спросил ангела, который шёл со мною, говоря: "Что это за предметы,
которые я видел в сокровенном месте?"
И он сказал мне: "Все эти предметы, которые ты видел, служат
владычеству Его Помазанника, дабы Он был сильным и могущественным на земле".
И отвечал мне тот ангел мира, говоря: "Подожди немного, тогда ты
увидишь и тебе будет открыто всё, что сокровенно и что насадил Господь духов.
И те горы, которые ты видел: гора из железа, гора из меди, гора из
серебра, гора из золота, гора из жидкого металла и гора из свинца~--- все они
будут пред Избранным, как сотовый мёд пред огнём и как та вода, которая стекает
сверху на эти горы, и они окажутся слабыми под Его ногами.
И случится в те дни, что нельзя будет спастись ни золотом, ни
серебром: нельзя будет тогда ни спастись, ни убежать.
И не будет дано тогда для битвы ни железа, ни панцирной одежды; руда
не будет пригодна ни на что, и олово не будет пригодным ни на что и не пойдёт в
прок, и свинец не будет добываться.
Все эти вещи исчезнут и уничтожатся с поверхности земли, когда
появится Избранный пред лицом Господа духов.
И там мои очи видели глубокую долину, устье которой было открыто;
и все живущие на тверди, и в море, и на островах принесут Ему дары, и подарки,
и знаки верности, но та глубокая долина не наполнится.
И они совершают преступление своими руками, и всё, что они, грешники,
добывают, то преступным образом пожирают сами, так они, грешники, погибнут пред
лицом Господа, и будут изгнаны с лица Его земли без прекращения на всю
вечность.
Ибо я видел ангелов наказания, как они шли и готовили сатане все
орудия.
И я спросил ангела мира, шедшего со мною: "Те орудия,~--- для кого они
их готовят?"
И он сказал мне: "Они готовят их для царей и для сильных земли сей,
чтобы уничтожить их чрез это.
И после этого Праведный и Избранный откроет дом Своего общественного
собрания, которое отныне не должно быть стесняемо, во имя Господа духов.
И эти горы будут пред Его лицом, как земля и холмы будут, как водный
источник; и праведники будут иметь покой при унижении грешников".
И я взглянул и обратился к другой стороне земли, и увидел там
глубокую долину с пылающим огнём.
И они (ангелы наказания) принесли царей и сильных и положили их в
глубокую долину.
И там мои очи видели, как сделали для них орудия,~--- железные цепи
безмерного веса.
И я спросил ангела мира, говоря: "Эти цепи-орудия,~--- для кого они
приготовлены?"
И он сказал мне: "Они приготовлены для отрядов Азазела, чтобы взять их
и бросить в преисподний ад: и челюсти их будут покрыты грубыми камнями, как
повелел Господь духов.
Михаил и Гавриил, Руфаил и Фануил схватят их в тот великий день суда и
бросят в этот день в печь с пылающим огнём, дабы Господь духов отмстил им за их
неправду,~--- за то, что они покорились сатане и прельстили живущих на земле.
И в те дни наступит осуждение Господа духов, и откроется хранилище
вод, которые сверху на небесах, и, кроме них, те источники, которые под
небесами и внизу в земле.
И все воды на земле соединятся с водами, которые в верху на небесах;
вода же, которая вверху на небе, есть мужская, и вода, которая внизу на земле,
есть женская.
И тогда будут уничтожены все, которые живут на земле и которые живут
между пределами неба.
И чрез это они узнают всю неправду, которую они совершили на земле и
за которую погибают.
И после этого раскаялся Глава дней и сказал: "Напрасно Я погубил
всех живущих на земле".
И Он поклялся Своим великим именем: "Отныне Я не буду более поступать
так с живущими на земле; и Я положу знамение на небе: оно будет залогом между
Мною и ими до вечности, пока существует небо над землёю.
И тогда произойдёт по Моему повелению: когда Я в Моём гневе и в Моём
осуждении решу схватить их рукою ангелов в день скорби и печали, то Мой гнев и
Моё осуждение будут оставаться над ними навсегда,~--- говорит Бог, Господь духов.
Вы, могущественные цари, которые будете жить на земле, вы должны
увидеть Моего Избранного, как Он сидит на престоле Моей славы и судит Азазела,
и всё его сообщество, и все его отряды, во имя Господа духов".
И я видел там воинство идущих ангелов наказания, которые держали
верёвки из железа и руды.
И я спросил ангела мира, шедшего со мною, говоря: "К кому идут те,
которые держат верёвки?"
И он сказал мне: "Каждый к своим избранным и возлюбленным, чтобы
бросить их в глубокую пропасть долины.
И тот час та долина наполниться избранными и возлюбленными, и день их
жизни окончится, и день их обольщения не будет с тех пор более считаться".
И в те дни соберутся ангелы, и их начальники направятся к востоку к
Парфеянам и Мидянам~--- приготовить там возмущение между царями, чтобы нашёл на
них дух возмущения; и они поднимутся со своих престолов, чтобы выступить в
середину их стада, как львы из своих логовищ и как голодные волки.
И они поднимутся и обступят землю их избранных, и земля Его избранных
будет пред ними гумном и тропою.
Но город Моих праведных будет преградой для их коней; и они начнут
борьбу друг с другом, и их правая рука будет сильна против них самих, и никто
не будет знать своего ближнего и брата, ни сын своего отца и своей матери, пока
не будет достаточно трупов вследствие из смерти, и осуждение над ними не будет
тщетным.
И в те дни царство мёртвых откроет свою пасть, и они будут опущены в
него; и вот их погибель: царство мёртвых поглотит грешников пред лицом
избранных.
И случилось после этого: там опять я увидел отряд колесниц, на
которых ехали люди, и они шли на крыльях ветра от восхода и захода к полудню.
И был слышен шум их колесниц; и как только это смятение произошло,
святые ангелы заметили это с неба; и столпы земли подвинулись со своих мест, и
это было слышно от пределов земли до пределов неба, в один день.
И они все упадут и поклонятся Господу духов.
И это конец второй притчи.
\vs 1En 9:1
И я начал говорить третью притчу о праведных и избранных.
Будьте блаженными вы, праведные и избранные, ибо жребий ваш будет
славен!
И праведные будут жить в свете солнца и избранные в свете вечной жизни;
дни вечной жизни их не кончаются, и дни святых бесчисленны.
И они будут искать света и обретут правду у Господа духов: мир будут
иметь праведные у Господа мира.
И после этого будет сказано святым, чтобы они искали на небе тайны
справедливости и наследие веры, ибо оно стало ясно, как сияние солнца на земле,
и мрак исчез.
И не прекращаемый свет будет существовать, и дни, в которые они будут
жить, бесчисленны, ибо мрак заранее будет уничтожен, и силен будет свет пред
Господом духов, и свет праведности будет силен во век пред Господом духов.
И в те дни очи мои видели тайны молний и массы света, и их правду;
и они блестят для благословения и для проклятия, как желает этого Господь
духов.
И там я видел тайны грома, и слышал, как раздаётся глас его, когда он
гремит вверху на небе, и они (ангелы проводники) показали мне места жилищ на
земле и глас грома, как он служит для благополучия и благословения или для
проклятия, по слову Господа духов.
И после этого мне были показаны все тайны масс света и молний, как они
блестят для благословения и для насыщения.
\vs 1En 10:1
В пятисотый год, в седьмой месяц, в четырнадцатый день месяца
жизни Еноха.
В той притче я видел, как небо небес поколебалось от сильного трепета,
и воинство Всевышнего, и тысячи тысяч и тьмы тем ангелов были потрясены
вследствие сильного волнения.
И тот час я увидел Главу дней, сидящего на престоле Своей славы, и
ангелов и праведных, стоящих вокруг Него.
И меня объял сильный трепет, и страх охватил меня; моё бедро согнулось
и ослабело, всё моё существо сплавилось, и я упал на своё лицо.
Тогда святой Михаил послал другого святого ангела~--- одного из святых
ангелов~--- и он поднял меня; и как только он меня поднял, мой дух обратился
назад, ибо я не мог вынести вида этого воинства, и колебания и трепета неба.
И сказал мне святой Михаил: "что за вид так взволновал тебя?
До сего дня был день Его милосердия, ибо Он был милосерден и
долготерпелив к населяющим почву земную.
Но вот придет день, и власть, и наказание, и суд, что приготовил
Господь духов для тех, которые преклоняются пред праведным судом, и для тех,
которые отвергают праведный суд, и для тех, которые напрасно употребляют Его
имя; и тот день будет для избранных защитою, а для грешников расследованием.
И в тот день будут распределены два чудовища: женское чудовище,
называемое Левияфа, чтобы оно жило в бездне моря над источниками вод, мужеское
же называется Бегемотом, который своею грудью занимает необитаемую пустыню,
называемую Дендаин, находящуюся на востоке сада, где живут избранные и
праведные и куда взят мой дед, седьмой от Адама~--- первого человека, которого
сотворил Господь духов.
И я молил того другого ангела, чтобы он показал мне власть тех
чудовищ, как они разделены в один день, и одно было поставлено в глубину моря,
а другое на твердую почву пустыни.
И он сказал мне: "ты, сын человеческий,~--- ты добиваешься здесь узнать,
что сокрыто".
И сказал мне другой ангел, который шел со мною и показал мне, что
находится в сокровенных местах, первое и последнее, что на небе в высоте и на
земле в глубине, и что на пределах неба, и в хранилищах при основании неба, и в
хранилищах ветров; и он показал, как распределены духи, и как возвышаются
(явления в природе), и как исчислены источники и ветры по силе духа, и какова
сила лунного света, и как все это есть сила правды, и (показал) отделения звезд
по их именам, и как все отделения разделены; и он показал громы по местам их
падения, и все отделения; которые сделаны между молниям, чтобы они сверкали и
их отряды тотчас бы повиновались (следовали за ними); ибо гром имеет места
отдыха и ему определено выжидать свой удар; и они оба~--- гром и молния~---
неотделимы; и хотя они не одно, однако оба чрез посредство духа идут вместе и
не разделяются.
Ибо, когда сверкает молния, то и гром дает свой глас, и дух
задерживает во время удара и одинаково делает разделение между ними; ибо запас
их ударов, как песок, и каждый в отдельности из них удерживается при своем
ударе уздою, и силою духа они возвращаются назад, и таким образом посылаются
далее соразмерно с множеством стран земли.
И дух моря есть мужественный и сильный; и соразмерно с крепостью своей
силы он притягивает его (море) назад уздою; и таким образом оно перегоняется
вперед и разливается во все горы земли.
И дух инея есть его (собственный, особенный) ангел, и дух града есть
добрый ангел.
И духа снега Он назначил ради его силы, и он (снег) имеет особенного
духа; и то, что поднимается из него, есть как бы дым и его имя мороз.
Но дух облака не соединён с ними (духами инея, града и снега) в их
хранилищах, а имеет особое хранилище; ибо его движение бывает при ясности и
свете и при мраке, и зимой и летом, и его хранилище есть свет; и он (дух
облака) есть его ангел.
И дух росы имеет свое жилище на пределах неба, и оно связано с
хранилищем дождя, и его движение бывает зимою и летом; и его тучи и тучи
дождевого облака находятся в связи и сообщаются друг с другом.
И когда дух дождя выходит из своего хранилища, приходят ангелы, и
открывают хранилище, и выпускают его, и тогда он рассевается по всей суше и
таким образом соединяется с водою на земле.
Ибо воды существуют для живущих на земле, так как они составляют пищу
для земли от Всевышнего, Который существует на небе; посему дождь имеет меру,
и ангелы владеют им.
Я видел все эти вещи вплоть до сада праведных.
И ангел мира, который был со мною, сказал мне: "эти два чудовища
приготовлены сообразно с величием Божиим для того, чтобы быть накормленными,
дабы осуждение Божие не было тщетным; и будут умерщвлены сыны со своими
матерями и дети со своими отцами.
Когда осуждение Господа духов будет пребывать над ними, то будет
пребывать для того, чтобы осуждение Господа духов не сделалось тщетным по
отношению к ним; после этого будет суд по Его милосердию и терпению.
И я видел в те самые дни, как даны были тем ангелам длинные
веревки, и они подняли крылья и полетели, и достигли севера.
И я спросил ангела, говоря: "для чего они держали те длинные веревки и
удалились?"
И он сказал мне: "они ушли, чтобы измерять".
И ангел, шедший со мною, сказал мне: "они несут меры праведных и
канаты праведных, чтобы они опирались на имя Господа духов навсегда и навеки.
И начнут и будут жить избранные с избранными, и эти меры будут даны
вере и будут укреплять слова правды.
И эти меры откроют всё сокровенное в глубине земли, и погибших по
пустыням, и пожранных рыбами морскими и зверями, чтобы они возвратились и
оперлись на день Избранного; ибо никто не погибнет пред Господом духов, и никто
не может погибнуть.
И сохранили повеления все те, которые вверху на небе, и одна сила,
один голос и один свет, подобный огню, был дан им.
И Того, прежде всего, прославили, и возвеличили, и восхвалили они с
мудростью, и показали себя мудрыми в слове и духе жизни.
И Господь духов посадил Избранного на престол Своей славы, и Он будет
судить все деяния святых ангелов на небе и взвесит их поступки на весах.
И когда Он поднимает Свое лицо, чтобы судить их сокрытые пути по слову
имени Господа духов и их стезю по пути праведного суда всевышнего Бога, тогда
все они возглаголят одним гласом, и прославят, и восхвалят, и вознесут, и будут
хвалить имя Господа духов.
И будет взывать все воинство небесное и все святые, которые вверху, и
воинство Божие,~--- херувимы и серафимы, и офанимы, и все ангелы власти, и все
ангелы господства, и Избранный, и другие силы, которые на тверди и над водою,~---
все они будут взывать в тот день и будут возносить одним гласом, и прославлять,
и восхвалят, и хвалить, и превозносить в духе веры, и в духе мудрости и
терпения, и в духе милосердия, и в духе правды и мира, и в духе благости; и
будут все говорить одним гласом: "славь Его, и да будет прославлено имя Господа
духов во веки и до века!"
Его будут хвалить все, которые не спят вверху на небе; Его будут
прославлять все Его святые, которые на небе, и все избранные, живущие в саду
жизни, и каждый дух света, способный прославлять и восхвалять, и превозносить,
и святить Твое имя, и всякая плоть, которая будет чрезмерно прославлять и
восхвалять Твое имя вовеки.
Ибо велико милосердие Господа духов, и Он долготерпелив, и все Свои
творения и всю Свою силу,~--- так много Он сотворил,~--- Он открыл праведным и
избранным, во имя Господа духов.
И Господь духов так повелел царям, и сильным, и вознесенным, и
населяющим землю, и сказал: "откройте свои глаза и вознесите ваши роги, ибо вы
можете узнать Избранного!"
И Господь духов сел на престол славы, и дух правды изливался на Него,
и слово уст Его умертвило всех грешников и всех неправедных, и они погибли
перед лицом Его.
И будут стоять в тот день все цари, и сильные, и вознесенные, и
владеющие твердью, и увидят Его и узнают, как Он сидит на престоле Своей славы,
и пред Ним судятся праведные в правде и никакая пустая речь не говорится пред
Ним.
Тогда постигнет их боль, как жену, которая в родильных потугах и
которой трудно бывает родить, когда ее сын входит в проход утробы, и которая
имеет боли при родах.
И одна часть из них будет смотреть на другую, и они устрашатся и
потупят свой взор, и боль обоймёт их, когда они увидят того Сына жены, сидящим
на престоле Своей славы.
И цари, и сильные, и все владеющие землею будут восхвалять, и
прославлять, и превозносить Владычествующего над всем, Который был сокрыт.
Ибо прежде Сын человеческий был сокрыт, и Всевышний сохранял Его пред
Своим могуществом, и открыл Его избранным; и будет посеяно общество святых и
избранных, и будут стоять пред Ним в тот день все избранные.
И все могущественные цари, и вознесенные, и господствующие над
твердью, упадут пред Ним на свое лицо, и поклонятся, и возложат на того Сына
человеческого свою надежду, и будут умолять Его и просить у Него милосердия.
И Господь духов будет теперь теснить их, чтобы они немедленно
удалились прочь от Его лица; и их лица исполнятся стыдом, и мрак соберется на
них.
И ангелы наказания возьмут их, чтобы совершить над ними возмездие за
то, что они притеснили Его детей и избранных.
И они сделаются зрелищем для праведных и избранных Его: они
(праведные) будут радоваться, взирая на них, ибо гнев Господа духов будет
пребывать на них, и меч Господа духов упьется ими.
И праведные и избранные будут спасены в тот день, и не будут более
видеть отныне лица грешников и неправедных.
И Господь духов будет обитать над ними, и они будут жить вместе с тем
Сыном человеческим, и есть, и ложиться, и вставать, от века до века.
И праведные и избранные будут вознесены от земли, и перестанут
опускать свой взор, и будут облечены в одежду жизни.
И это будет одежда жизни у Господа духов.
В те дни могущественные цари, владеющие твердью, будут вымаливать
у Его ангелов наказания, которым они преданы,~--- даровать им немного успокоения,
и просить, чтобы им можно было пасть ниц перед Господом духов и поклониться, и
сознаться перед Ним в своих грехах.
И они будут прославлять и восхвалять Господа духов, и говорить: "да
будет прославлен Он, Господь духов и Господь царей, Господь сильных и Господь
властителей, Господь славы и Господь мудрости, пред которым всякая тайна ясна.
И Твое могущество от рода до рода, и Твоя слава от века до века;
глубоки все Твои тайны и бесчисленны, и слава Твоя неисчислима.
Теперь узнали мы, что нам нужно восхвалять и прославлять Господа царей
и Того, Кто царь над всеми царями".
И они скажут: "о, если бы нам дали успокоение, чтобы мы восхвалили
Его, и возблагодарили Его, и прославили Его, и уверовали пред Его славой!
И теперь мы домогаемся небольшого успокоения, но не находим его: мы
прогнаны, и не получили его, свет исчез пред нами, и мрак служит нашим жилищем
навсегда и навеки.
Ибо мы не уверовали в Него, и не восхвалили имя Господа царей за
всякое Его дело, и наша надежда была бы на скипетр нашего владычества и на наше
величие.
И в тот день нашего страдания и нашей печали Он не спасет нас, и мы не
найдем успокоения, дабы уверовать, что Господь наш истинен во всяком Своем
деле, и во всех Своих судах, и в Своей правде, и суды Его не лицеприятны.
И мы погибнем пред Его лицом за свои дела, и все грехи наши исчислены
по справедливости".
Теперь они скажут себе: "душа наша насытилась неправедным стяжанием,
но оно не отвратит того, что мы будем низвергнуты в пламя адского мучения".
И после этого их лицо исполнится мраком перед тем Сыном человеческим и
они будут отвергнуты от Его лица, и меч будет жить между ними перед Его лицом.
И Господь духов так сказал: "вот повеление и суд над сильными, и
царями, и вознесёнными, и владеющими твердью, пред Господом духов".
Также и другие виды я видел в том сокровенном месте.
Я слышал глас ангела, как он сказал: "это ангелы, которые сошли с неба
на землю и открыли сынам человеческим то, что было сокрыто, и соблазнили сынов
человеческих совершать грехи".
\vs 1En 11:1
И в те дни Ной увидел землю, как она согнулась, и ее гибель была
близка.
И он направил оттуда свои стоны и пришел к пределам земли, и вскликнул
к своему деду Еноху; и Ной трижды сказал опечаленным голосом: "послушай меня,
послушай меня, послушай меня!"
И он (Ной) сказал ему: "скажи мне, что это такое происходит на земле
что земля так ослабела и поколебалась?
как бы я не погиб вместе с нею!"
И после этого мгновения было великое колебание на земле, и голос был
слышен с неба, и я упал на свое лицо.
И пришел мой дед Енох, и встал около меня, и сказал мне: "Почему ты
восклицал ко мне опечаленным криком и плачем?
От лица Господа вышло повеление относительно живущих на тверди, что
должен наступить их конец, так как они знают все тайны ангелов, и всю власть
дьяволов, и всю их сокровенную силу, и всю силу тех, которые совершают
волшебства, и силу заклинаний, и силу тех, которые льют для всей земли
изображения идолов; и хорошо также знают, как серебро производится из праха
земли, и как жидкий металл образуется на земле.
Ибо свинец и олово не так производятся из земли, как первое (серебро):
существует особый источник, производящий их, и ангел, стоящий в нем, и он
преимущественно тот ангел".
И после этого дед Енох обнял меня своей рукою, поднял меня и сказал
мне: "иди, ибо я спрашивал Господа духов об этом колебании на земле.
И он сказал мне: за их нечестие над ними совершен суд, и он уже не
вычисляется предо Мною ради месяцев, которые они расследовали и через это
узнали, что земля и живущие на ней погибнут.
И для них (ангелов) не будет убежища вовеки, так как они показали им
(людям) то, что сокрыто, и они осуждены; но не так ты, мой сын: Господь духов
знает, что ты чист и свободен от этой укоризны за тайны.
И Он утвердил твое имя между святыми, и сохранит тебя между живущими
на тверди; и Он определил в правде твое семя для царей и для великой славы, и
из твоего семени произойдет источник праведных и святых без числа во веки".
И после этого он показал мне ангелов наказания, готовых идти и
выпустить все силы воды, которая внизу на земле, чтобы принести суд и погибель
всем, покоящимся и живущим на тверди.
И Господь духов дал повеление ангелам, вышедшим теперь, чтобы они не
простирали рук, а дожидались: ибо те ангелы были поставлены над силами вод.
И я удалился от лица Еноха.
И в те дни было слово Господа ко мне, и Он сказал мне: "Ной!
вот твой жребий предстал предо Мною, жребий без порока, жребий любви
и милосердия.
И теперь ангелы делают деревянное здание; и так как они вышли на это
дело, то и Я приложу к нему Свою руку и буду охранять его (ковчег); и выйдет из
него семя жизни, и земля должна подвергнуться превращению, чтобы ей не остаться
пустою.
И Я укреплю твое семя предо Мною на всю вечность, и живущие с тобою
распространятся по поверхности земли, и оно (семя) будет благословенно и
умножится на земле во имя Господа".
И они заключат тех ангелов, показавших неправду, в ту пылающую долину
на западе, которую показал мне прежде дед Енох, возле гор золота, и серебра, и
железа, и жидкого металла, и свинца.
И я видел ту долину, в которой было великое колебание и волнение вод.
И когда всё это случилось, то из той огненной металлической лавы и от
колебания, которое их (воды) колебало, в том месте (в долине) явился серный
запах, и он соединился с теми водами; и та долина ангелов, которые прельстили
людей, разгоралась все далее под тою землею.
И через долины этой самой земли проходят реки огня,~--- именно там, где
осуждены пребывать те ангелы, которые соблазнили живущих на тверди.
Но те воды будут служить в те дни для царей, и сильных, и вознесённых,
и для живущих на тверди к исцелению души и тела и к наказанию духа, так как
дух их исполнен сладострастием, чтобы они были наказаны со своим телом, ибо
они отвергли Господа духов; и они изо дня в день видят свое будущее наказание и
однако не веруют в Его имя.
И в той самой мере, насколько становятся сильными жар их тела, будет
происходить изменение и в их духе (от века до века), ибо не может быть сказано
пред Господом духов пустое слово.
Ибо придет суд на них, так как они веруют в сладострастии своего тела
и отвергают дух Господа.
И те воды сами в те дни претерпят изменение: ибо, когда те ангелы
будут наказаны в те дни, будет изменяться жар тех водных источников, и когда
ангелы будут подниматься, та вода источников будет изменяться и охлаждаться.
И я слышал святого Михаила, когда он отвечал и говорил: "этот суд,
которым осуждены ангелы, есть свидетельство для царей, и сильных, и владеющих
твердью.
Ибо эти воды суда служат к исцелению ангелов и для смерти их тела; но
они (владыки) не увидят того и не уверуют, что те воды изменятся и превратятся
в огонь, который горит вовек".
И после этого мой дед Енох дал мне в книге знамения всех тайн и
притч, которые ему были даны, и собрал их для меня в словах книги притчей.
И в тот день отвечал святой Михаил Руфаилу, говоря: "сила духа увлекает
меня и возбуждает меня, и строгость суда тайн, суда над ангелами, поражает
меня; кто может вынести строгость суда, который совершен и до сих пор пребывает
и от которого они распаляются?"
И опять отвечал и сказал святой Михаил Руфаилу: "есть ли кто такой,
который не размягчился бы сердцем и почки которого не содрогнулись бы от этого
слова?
суд вышел относительно них, относительно тех, которых выгнали они
таким образом".
И случилось, когда святой Михаил стоял пред Господом духов, то он
сказал Руфаилу так: "и я не буду представительствовать за них пред очами
Господа, ибо Господь духов разгневался на них, потому что они действуют так,
как если бы были равны Богу.
Посему на них грядет суд, который сокрыт, от века до века; ибо ни
ангел, ни человек не получат своей доли, но только они получат свой суд, от
века до века".
И после этого суда они навлекут на них гнев и ярость, так как они
показали это живущим на тверди.
И вот имена тех ангелов, и эти имена их: первый из них Семъяйза,
второй Арестикифа, третий Армен, четвертый Кокабаел, пятый Тураел, шестой
Румъйял, седьмой Данел, восьмой Нукаел, девятый Баракел, десятый Азазел:
одиннадцатый Армерс, двенадцатый Батаръйял, тринадцатый Базазаел, четырнадцатый
Ананел, пятнадцатый Турхйял, шестнадцатый Симанизиел, семнадцатый Иетарел,
восемнадцатый Тумаел, девятнадцатый Тарел, двадцатый Румаел, двадцать первый
Изезеел.
И это главы их ангелов и имена их предводителей над сотнею,
пятьюдесятью и десятью.
Имя первому Иекун; это тот, который соблазнил всех детей святых
ангелов, и свел их на землю, и соблазнил их чрез дочерей человеческих.
И имя другому Асбеел: этот внушил детям святых ангелов злой совет, и
соблазнил их, чтобы они осквернили свои тела с дочерьми человеческими.
И имя третьему Гадрел: это тот, который показал сынам человеческим все
смертоносные удары, и он соблазнил Еву, и показал сынам человеческим орудия
смерти, и панцырь, и щит, и меч для битвы, и показал сынам человеческим все
орудия смерти.
И из его руки они перешли к живущим на тверди, от того часа до века.
И имя четвертому Пенемуэ: этот показал сынам человеческим горькое и
сладкое, и показал им все тайны их мудрости.
Он научил людей письму чернилами и употреблению бумаги, и чрез это
многие согрешили от века до века и до сего дня.
Ибо люди сотворены не для того, чтобы они, таким образом, тростью и
чернилами закрепляли свою верность (свое слово).
Ибо люди сотворены не иначе, чем ангелы, чтобы им пребывать праведными
и чистыми, и смерть, которая губит всех, не касалась бы их, но они погибают
чрез это свое знание, и чрез эту силу она пожирает меня.
И имя пятому Касдейя: этот показал людям все злые удары духов и
демонов, и улары рождения в утробе матери, дабы устранить его, и удары души,
укушения змей, и удары, случающиеся в полдень,~--- сына змеи, именуемого Табает.
И это число Кесбеела, который показал святым главу клятвы, когда он
жил высоко вверху во славе, и имя ее (клятвы) Бека.
И этот ангел сказал святому Михаилу, чтобы он показал им сокровенное
имя Божие, дабы они видели то сокровенное имя и упоминали его при клятве, чтобы
содрогались пред тем именем и клятвой те, которые показали сынам человеческим
все, что было сокрыто.
И такова сила той клятвы, ибо она сильна и могущественна, и Он положил
эту клятву Акаэ в руку святого Михаила.
И таковы тайны клятвы, и они (тайны мира) утверждены чрез его клятву,
и силою его небо повешено, прежде чем был создан мир, и до века.
И чрез нее была основана земля на воде, и силою ее
выходит из сокровищ гор прекрасная вода для живущих от сотворения мира
до века.
И чрез ту клятву было сотворено море, и, как его основание, Он положил
ему на время ярости песок, и оно не должно преступать его от сотворения мира до
века.
И чрез ту клятву основания земли утверждены, и стоят и не движутся со
своего места от века до века.
И чрез ту клятву совершают свое движение солнце и луна, и не отступают
от предписанного им от века до века.
И чрез клятву звезды совершают свое движение, и Он зовет их по именам
и они отвечают Ему от века до века; и точно также духи воды, ветров и всего
воздуха, и их пути по всем соединениям духов.
И в ней (силою клятв) сберегаются хранилища гласов грома и света
молний; и в ней сберегаются хранилища града и инея, и хранилища дождя и росы.
И они все веруют и воссылают благодарение Господу духов, и восхваляют
всею своею силою, и их пища состоит в громких благодарениях; они благодарят, и
прославляют, и превозносят имя Господа духов от века до века.
И могущественна над ними эта клятва, и они сохраняются чрез неё, и их
пути сохраняются, и их движения не нарушаются.
И было для них (для праведников) великою радостью, и прославляли, и
восхваляли за то, что им было открыто имя того Сына человеческого.
И Он сел на престол Своей славы и весь суд был предан Ему, Сыну
человеческому, и Он допустил прийти и погибнуть с лица земли грешникам и тем,
которые соблазнили мир.
Они связаны ценою и заключены в своих сборных местах разврата, и все
дела их исчезают с земли.
И отныне не будет более там ничего тленного, ибо Он, Сын мужа, явился
и сел на престоле Своей славы: и всякое зло исчезнет и прейдет пред Его лицом;
слово же того Сына мужа будет иметь силу пред Господом духов.
Это третья притча Еноха.
\vs 1En 12:1
И случилось после этого: вот его Еноха имя было вознесено при
жизни к тому Сыну человеческому, к Господу духов, от живущих на тверди.
И оно было вознесено на колесницах духа, и имя его вышло среди людей.
И с того дня я не входил в их среду; и Он посадил меня между двумя
ветрами, между севером и западом,~--- там, где ангелы взяли веревки, чтобы
измерить около меня место для избранных и праведных.
И там я видел первых отцов праведных, от древнейшего времени живущих в
том месте.
И после того случилось, что мой дух был сокрыт (восхищен) и
вознесен на небеса; там я видел сынов ангелов, как они ходят по огненному
пламени; и их одежды и их одеяния белы, и свет лица их как кристалл.
И я видел две реки из огня, и свет того огня блистал, как гиацинт: и я
пал на свое лицо пред Господом духов.
И ангел Михаил, один из архангелов, взял меня за правую руку и поднял
меня, и привел меня ко всем тайнам милосердия и правды.
И он показал мне все тайны пределов неба и все хранилища всех звезд и
светил, откуда они выходят пред святых.
И дух восхитил Еноха на небо небес, и я видел там, в средине того
света, нечто такое, что было устроено из кристалловых камней, и между теми
камнями было пламя живого огня.
И мой дух видел, как вокруг того дома обходил огонь, на четырех же
сторонах его реки, наполненные живым огнем, и видел, как они окружают тот дом.
И вокруг были серафимы, херувимы и офанимы: это те которые не спят и
охраняют престол его славы.
И я видел ангелов, которые не могут быть исчислены, тысячу тысяч и
тьму тем, окружающих тот дом: и Михаил и Руфаил, Гавриил и Фануил, и святые
ангелы, которые вверху на небесах, выходят и входят в тот дом.
И вышли из того дома Михаил и Гавриил, Руфаил и Фануил, и многие
святые ангелы без числа, и с ними Глава дней; Его глава была чиста как волна
(руно) И Его одежда неописуема.
И я упал на свое лицо, и все мое тело сплавилось, и мой дух изменился:
и я воскликнул громким голосом, духом силы, и прославил и восхвалил и
превознес.
И эти прославления, которые вышли из моих уст, были приятны для того
Главы дней.
И сам Глава дней шел с Михаилом и Гавриилом, Руфаилом и Фануилом, и с
тысячами и со тьмами тысяч, с ангелами без числа.
И тот ангел пришел ко мне, и приветствовал меня своим гласом, и
сказал: "ты~--- сын человеческий, рожденный для правды", и правда обитает над
тобою, и правда Главы дней не оставляет тебя".
И он сказал мне: "Он призывает тебе мира, во имя будущего мира, ибо
оттуда исходит мир со времени сотворения вселенной, и таким образом ты будешь
иметь его во веки и от века до века.
И все, которые в будущем пойдут по твоему пути,~--- ты, которого правда
не оставляет вовек, жилища тех будут возле тебя и наследие их около тебя, и они
не будут отделены от тебя во век и от века до века.
И таким образом возле того Сына человеческого будет долгая жизнь, и
мир наступит для праведных, во имя Господа духов от века до века.
\vs 1En 13:1
Книга об обращении светил небесных, как это обращение происходит
с каждым из них, по их классам, по их господству и их времени, по их именам и
местам происхождения, и по их месяцам, которые показал мне их путеводитель,
святой ангел Уриил, бывший при мне; и он показал мне все их описание, что с
ними происходит со всеми годами мира и до века, пока не создано новое творение,
которое продолжится во век.
И вот первый закон светил: светило солнце имеет свой восход в восточных
вратах неба и свой заход в западных вратах неба.
И я видел шесть врат, в которых солнце заходит; луна также восходит и
заходит чрез те же врата, путеводители звёзд вместе со своими путеводными
восходят и заходят там же: шесть врат на востоке и шесть на западе, следующих
друг за другом в строго соответствующем порядке, а также много окон направо и
налево от тех врат.
Прежде всего, выходит великое светило, называемое солнцем, его
окружность как окружность неба, и оно совершенно наполнено блистающим и
согревающим огнем.
Колесницы, в которых оно поднимается, гонит ветер, и солнце, заходя,
исчезает с неба и возвращается назад через север, чтобы достигнуть востока; и
оно направляется таким образом, что приходит к соответствующим вратам и светит
на небе.
Таким образом, оно восходит в первый месяц в великих вратах и
именно оно восходит через четвёртые из тех шести восточных врат.
И при тех четвёртых вратах, через которые солнце восходит в первый
месяц, находятся двенадцать оконных отверстий, из которых выходит пламя, когда
они в свое время открываются.
Когда солнце поднимается на небо, то оно выходит чрез те четвёртые
врата в продолжение тридцати утров, и заходит прямо, напротив, в четвёртых
вратах на западе неба.
И в этот период день становится день за днем длиннее, и ночь становится
ночь за ночью короче до тридцатого утра.
И в тот день, день бывает длиннее на две части, чем ночь, и день
включает ровно десять частей и ночь восемь частей.
И солнце восходит из тех четвёртых врат и заходит в четвёртых, и
возвращается к пятым вратам востока в продолжение тридцати утров, и восходит из
них, и заходит в пятых вратах.
Тогда день становится длиннее на две части и заключает одиннадцать
частей, и ночь становится короче и заключает семь частей.
И солнце возвращается к востоку, и вступает в шестые врата, и восходит
и заходит в шестых вратах в продолжение тридцати одного утра ради их знака.
И в тот день, день становится длиннее ночи настолько, что заключает
двойное число частей ночи,~--- именно двенадцать частей, и ночь делается короче и
заключает шесть частей.
И поднимается солнце, чтобы день стал короче и ночь длиннее, и солнце
возвращается к востоку и вступает в шестые врата, и восходит из них и заходит в
продолжение тридцати утров.
И когда пройдет тридцать утров, день уменьшается ровно на одну часть,
и заключает одиннадцать частей и ночь семь частей.
И солнце выступает на западе их тех шести врат и идёт к востоку, и
восходит в пятых вратах в продолжение тридцати утров, и опять заходит на западе
в пятых западных вратах.
В тот день, день уменьшится на две части и заключает десять частей и
ночь восемь частей.
И солнце выходит из тех пятых врат запада, и поднимается в четвёртых
вратах ради их знака тридцать одно утро, и заходит на западе.
В тот день сравнивается день с ночью, и они становятся одинаково
длинными, и ночь заключает девять частей и день девять частей.
И солнце восходит из тех врат и заходит на западе, возвращается к
востоку и восходит в третьих вратах тридцать утров, и заходит на западе в
третьих вратах.
И в тот день ночь становится длиннее дня до тридцатого утра, и день
становится ежедневно короче до тридцатого дня, и ночь заключает ровно десять
частей и день восемь частей.
И солнце восходит из тех третьих врат, и заходит в третьих вратах на
западе, возвращается к востоку и восходит во вторых вратах востока в
продолжение тридцати утров, и точно также заходит во вторых вратах на западе
неба.
И в тот день ночь заключает одиннадцать частей и день из тех вторых
врат и заходит на западе во вторых вратах, и возвращается к востоку в первые
врата в продолжение тридцати одного утра и заходит на западе в первых вратах.
И в тот день ночь становится настолько длинною, что включает двойное
число частей дня; ночь заключает ровно двенадцать частей и день шесть частей.
Этим солнце закончило свои путевые становища, и оно опять поворачивает
на эти же становища, и вступает в те первые врата в продолжение тридцати утров,
и заходит также на западе напротив них.
И в тот день ночь уменьшается в продолжительности на одну часть, и она
заключает одиннадцать частей в день семь частей.
И солнце возвращается и вступает во вторые врата востока, и
возвращается на те свои путевые становища в продолжение тридцати утров, восходя
и заходя.
И в тот день ночь уменьшается в продолжительности, и ночь заключает
десять частей и день восемь частей.
И в тот день солнце восходит из тех вторых врат и заходит на западе,
потом возвращается к востоку поднимается в третьих вратах в продолжение
тридцати одного утра, и заходит на западе неба.
В тот день ночь уменьшается и заключает десять частей и день девять
частей, и ночь сравнивается с днем, и год заключает ровно триста шестьдесят
четыре дня, и продолжительность дня и ночи, и краткость дня и ночи в следствие
движения солнца становятся различными.
По причине этого дневное движение ежедневно становится длиннее, и его
ночное движение становится каждоночно короче.
И таков закон и движение солнца и его возвращение, насколько оно часто
возвращается: шестьдесят раз возвращается и восходит оно, именно то великое
вечное светило, которое навеки именуется солнцем.
И то, что таким образом восходит, есть великое светило, как оно
называется по своему появлению в силу повеления Господа.
И таким образом оно восходит и заходит, и не уменьшается и не
покоится, но движется день и ночь в колеснице, и его свет в семь раз светлее
лунного, но по величине они оба одинаковы.
\vs 1En 14:1
И после этого закона я видел другой закон, касающийся малого
светила, который называется луною.
Ее окружность подобна окружности неба, и ее колесница, в которой она
идет, гонится ветром; и ей дается свет по определённой мере.
В каждый месяц изменяется ее восход и заход; её дни как дни солнца; и
если ее свет равномерен (полон), то она содержит седьмую часть солнечного
света.
И она восходит таким образом: и ее начало на востоке выступает в
тридцатое утро; в тот день она становится видимою, и тогда бывает для всех
начало луны, в тридцатое утро, одинаково с солнцем в тех же вратах, где
восходит солнце.
И одна половина ее выступает на одну седьмую часть, и весь ее круг
бывает пуст, без света, кроме одной седьмой части из ее четырнадцати частей
света.
И когда она получает одну седьмую часть с половиной от своего света, то
ее свет заключает одну седьмую и седьмую часть с половиной.
Она заходит в новолуние вместе с солнцем, и когда солнце восходит,
восходит и луна вместе с ним, и получает половину одной седьмой части света, и
в ту ночь, в начале ее утра, луна заходит в первый день месяца вместе с
солнцем, и бывает невидима в ту ночь семью и семью частями с половиной.
И она в тот день становится видимою ровно одной седьмой частью, и
восходит и отклоняется от солнца, и дает света в остальные дни семь и семь (14)
частей.
И я видел другой закон и движение её, как она по тому закону
совершает своё месячное обращение.
И всё показал мне святой ангел Уриил, который служит вождём всех их
(светил); и я описал все её (луны) положения, и показал их мне, и описал ее
месяцы, как они бывают, и появление её света до истечения пятнадцати дней.
В каждых семи частях весь ее свет делается полным на востоке, и в
каждых седьми частях весь ее мрак делается полным на западе.
И в определенные месяцы она изменяет свой заход, и в определенные
месяцы она идет своим особенным (от солнца) движением.
И в двоих вратах луна заходит вместе с солнцем,~--- в тех двоих вратах,
в третьих и четвертых вратах.
Именно,~--- она выходит в продолжение семи дней и поворачивает, и
возвращается опять через врата, где восходит солнце; и в них ее свет делается
полным; и она отклоняется от солнца, и вступает в течение 8 дней в шестые
врата, из которых выходит солнце.
И когда солнце выходит из четвертых врат, она выходит семь дней, так
что она выходит из пятых, и возвращается опять в течение семи дней в четвёртые
врата, и весь ее свет делается полным, и она отклоняется и вступает в первые
врата в течение восьми дней.
И опять она возвращается в течение семи дней в четвертые врата, из
которых выходит солнце.
Так видел я их положения, как солнце восходит и заходит по порядку
своих месяцев.
И между теми днями, если взять вместе пять лет, солнце имеет излишку
тридцать дней; которые приходятся на один из тех пяти лет, если они полны,
составляют триста шестьдесят четыре дня.
И излишек солнца и звезд простирается до шести дней; а в пять лет, в
каждый по шести, до тридцати дней, и луна отстает от солнца и звезд на тридцать
дней.
И луна точно ведет все года, так что их положение вовек ни поспешает,
ни запаздывает ни на один день, но действительно правильно совершает годовую
смену в триста шестьдесят четыре дня.
Три года имеют тысячу девяносто два дня и пять лет тысячу восемьсот
двадцать дней, так что на восемь лет приходится две тысячи девятьсот двенадцать
дней.
На луну же приходится в три года тысяча шестьдесят два дня, и в пять
лет она отстаёт на пятьдесят дней; именно с суммою этого нужно прибавить к
шестидесяти двум дням.
И на пять лет приходится тысяча семьсот семьдесят пять дней, так что
лунные дни в восемь лет составляют две тысячи восемьсот тридцать два дня.
Именно ее отставание образует в восемь лет восемьдесят дней, и всех
дней, на которые она отстает в восемь лет, восемьдесят.
И правильный год достигает конца сообразно с положением их (фаз луны?)
и с положением солнца, так как оно восходит из врат, из которых оно восходит и
заходит тридцать дней.
И путеводители глав тысячей, которые поставлены над всем
творением и над всеми звёздами, существует с четырьмя добавочными днями,
которые не могут быть отделены от своего места сообразно со всем исчислением
года; и эти путеводители служат для четырёх дней, которые не считаются при
исчисление года.
И из-за них люди ошибаются в том (в исчислении), ибо те светила
действительно служат для положения мира, одно в первых, одно в третьих, одно в
четвертых и одно в шестых вратах; и точность движения мира оканчивается всегда
чрез триста шестьдесят четыре положения его.
Ибо знаки, времена, и годы и дни показал мне ангел Уриил, которого
вечный Господь славы поставил над всеми небесными светилами на небе и в мире,
чтобы они управляли на поверхности неба, и явились над землею, и были
путеводителями для дня и ночи, именно солнце, луна и звезды, и все служебные
творения, которые совершают свое обращение во всех колесницах неба.
Точно так же Уриил дал мне увидеть двенадцать дверных отверстие в
кругу солнечных колесниц на небе, из которых пробиваются лучи солнца; и от них
исходит теплота на землю, когда они открываются в определённые времена.
Такие же отверстия есть также для ветров для духа росы, когда они по
временам открываются, стоя открытыми в небесах на пределах.
И я видел двенадцать врат на небе на приделах земли, из которых
солнце, луна и звёзды, и все произведения неба выходят на востоке и на западе.
И много оконных отверстий находится направо и налево от них, и каждое
окно выбрасывает в свое время тепло, соответствуя тем вратам, из которых
выходят звезды по повелению, которое Он дал им, и в которые они заходят,
соответствуя их числу.
И я видел на небе колесницы, как они неслись в мире,~--- вверху и внизу
от тех врат,~--- в которых обращаются никогда не заходящие звезды.
И одна их них больше всех их, и она проходит чрез весь мир.
\vs 1En 15:1
И на пределах земли я видел открытыми для всех ветров двенадцать
врат, из которых выходят ветры и дуют на землю.
Трое из них открыты на лице неба (на востоке), и трое на заходе, и трое
на правой стороне неба и трое на левой.
И трое первых лежат к востоку, и трое к северу, и трое, противостоящих
им налево, к югу, и трое на западе.
Чрез четверо из них выходят ветры благословения и благополучия, а из
тех (из остальных) восьми выходят ветры бедствия; когда они посылаются, то
производят разрушение на всей земле, и в воде, существующей на ней, и во всех
тварях, живущих на ней, и во всем, что находится в воде и на суше.
И первый ветер, дующий из тех врат и называющийся восточным,
выходит в первых восточных вратах, склоняющихся к югу; из них выходит
разрушение, сухость, зной и гибель.
И чрез вторые врата, что лежат в средине, выходит правильное смещение,
и именно~--- из них выходит дождь и плодородие, и благополучие, и роса; и чрез
третьи врата, которые лежат к северу, выходит холод и сухость.
И после этих, выходят южные ветры чрез трое врат: во-первых, через
первые из них, которые склоняются к востоку, выходит жгучий ветер.
И через прилежащие к ним средние врата выходят благовония, и роса, и
дождь, и благополучие, и здоровье.
И чрез третьи врата, лежащие к западу, выходит роса, и дождь, и саранча,
и разрушение.
И после этих северные ветры: из седьмых врат, которые на восточной
стороне склоняются к югу выходит роса и дождь, саранча и разрушение.
И из средних врат на прямом направлении выходит дождь, и роса, и
здоровье, и благополучие; и через третьи врата на северо-западной стороне
выходит туман, и иней, и снег, и дождь, и роса, и саранча.
И после этих западные ветры: чрез первые врата, склоняющиеся к северу,
выходит роса, и дождь, и иней, и холод, и снег, и мороз.
И из средних врат выходит роса и дождь, благополучие и благословение;
и чрез последние врата, лежащие к югу, выходит сухость и разрушение, жар и
гибель.
Этим оканчиваются двенадцать врат четырех небесных стран; и все их
законы, и все их бедствия, и все их благодеяния я показал тебе, мой сын
Мафусаил.
Первый ветер называют восточным, так как он передний (первый); и
второй ветер называется южным ветром, так как там нисходит Всевышний; и там
предпочтительнее всего сходит Тот, Который да будет прославлен вовеки.
И западный ветер называется ветром уменьшения, так как там небесные
светила уменьшаются и опускаются.
И четвертый ветер называется северным: он разделяется на три части:
первая из них назначена для жилища людей, вторая для водных морей и с долинами,
и лесами, и реками, и мраком, и туманом; и третья часть с садом правды.
Я видел семь высоких гор, выше всех гор, находящихся на земле; оттуда
выходит иней; и приходят и исчезают дни, времена и годы.
Семь рек видел я на земле, больше всех других; одна из них, текущая с
запада, изливает свою воду в великое море.
И две из них текут с севера к морю, и изливают свою воду в эритрейское
море на востоке.
И четыре остальные вытекают на северной стране к своему морю, две к
эритрейскому морю, и две имеют устье в великом море, по другим~--- в пустыне.
Семь великих островов я видел на море и на суше: два на суше и пять на
великом море.
Имена солнца следующие: первое Оререс, второе Томас.
И луна имеет четыре имени: первое Азонъйя, второе Эбла, третье Беназэ,
и четвертое Эраэ.
Это оба великие светила: их окружность, как окружность неба, и по
величине они оба равны.
В кругу солнца находится одна седьмая часть света, в которою
прибавляется свет луне, и именно~--- в определенной мере прибавляется он, пока не
истощится седьмая часть солнца.
И они заходят и входят в западные врата, и совершают обращение через
север, и через восточные врата они выходят на поверхность неба.
И когда луна поднимается, то она появляется на небе имея в себе света
половину одной седьмой части; и в течение четырнадцати дней весь ее свет
делается полным.
В нее прибавляется также трижды пять (15) частей света, так что к
пятнадцатому дню свет ее становится полным по знаку года, и составляется трижды
пять частей, и луна рождается чрез половину одной седьмой части.
И при своем ущербе она уменьшается в первый день до четырнадцати своих
частей света, во второй до тринадцати, в третий до двенадцати, в четвертый до
одиннадцати, в пятый до десяти, в шестой до девяти, в седьмой до восьми, в
восьмой до семи, в девятый до шести, в десятый до пяти, в одиннадцатый до
четырех, в двенадцатый до трех, в тринадцатый до двух, в четырнадцатый до
половины одной седьмой части: и ее свет, который оставался от целого,
совершенно исчезает в пятнадцатый день.
И в определенные месяцы месяц имеет по двадцати девяти дней и один раз
двадцать восемь.
Также и другое установление показал мне Уриил относительно того, когда
прибавляется луне свет и на которой стороне он прибавляется ей от солнца.
Во все время, когда луна усиливается в своем свете, она лежит по
отношению к солнцу напротив; к четырнадцатому дню ее свет становится полным на
небе; и когда она вся освещена, ее свет бывает полным на небе.
И в первый день она называется новолунием, ибо в тот день начинается в
ней свет.
И она становится полною ровно в тот день (в 15-ый), когда солнце
заходит на западе, а она ночью восходит с востока и светит целую ночь, пока
солнце не взойдет напротив нее, и она бывает видима напротив солнца.
На той стороне, где бывает свет луны, она также опять уменьшается,
пока не исчезнет весь ее свет, и дни месяца оканчиваются, и ее круг остается
пустым без света.
И в продолжение трех месяцев она делает тридцать дней в свое время, и
в продолжение трех месяцев она делает по двадцати девяти дней, в которых
происходит ее ущерб в первое время и в первых вратах в течение ста семидесяти
семи дней.
И во время своего восхода она показывается в продолжение трех месяцев
по тридцать дней, и в продолжение трех месяцев по двадцати девяти дней.
Ночью она показывается приблизительно в течение двадцати дней как муж,
и днем как небо, ибо нет ничего другого в ней, кроме ее света.
И теперь, мой сын Мафусаил, я показал тебе все, и весь закон
звезд (светил) небесных окончен.
И он (Уриил) показал мне весь закон их для каждого дня, для каждого
времени (года), для каждого господства, и для каждого года, и его выход по Его
предписанию для каждого месяца и каждой недели; и он показал ущерб луны,
который происходит в шестых вратах; именно~--- в этих шестых вратах оканчивается
весь ее свет, и после этого там бывает начало месяца; и он показал ущерб,
который происходит в первых вратах в свое время, пока не пройдет сто семьдесят
семь дней, а по исчислению по неделям~--- двадцать недель и два дня; и он
показал, как она отстает от солнца и от порядка звезд ровно на пять дней в одно
время, и когда это место, которое ты видишь, оканчивается.
Таков образ, и описание каждого светила, как их показал мне вождь их~---
великий ангел Уриил.
И в те дни отвечал мне Уриил и сказал мне: "вот я показал тебе
все, о Енох, и открыл тебе все, чтобы ты увидел это, это солнце, и эту луну, и
путеводителей звезд небесных, и всех тех, которые вращают их, их соотношения, и
времена, и выходы.
И в дни грешников годы будут укорочены, и их посев будет запаздывать в
их странах и на их пастбищах (полях), и все вещи на земле изменятся и не будут
являться в свое время; дождь будет задержан, и небо удержит его.
И в те времена плоды земли будут запаздывать и не будут вырастать в
свое время; и плоды деревьев будут задержаны от созревания в свое время.
И луна изменит свой порядок и не будет являться в свое время.
И в те дни будет видимо на небе, как приходит великое неплодородие, на
самой крайней колеснице на западе; и оно (небо или солнце) будет светить ярче,
чем по обыкновенному порядку света.
И многие главы начальственных звезд будут ошибаться и они нарушат свои
пути и отправления, и подчинённые им не будут появляться в свои времена.
И весь порядок звезд сокрыт для грешников, и мысли тех, которые живут
на земле, будут ошибаться из-за них, и они уклонятся от всех своих путей, и
будут грешить и станут считать их (звезды) за богов.
И много зол придет на них, и осуждение придет на них, чтобы уничтожить
их всех".
И он сказал мне: "о Энох, рассмотри писание небесных скрижалей и
прочитай, что на них написано, и заметь для себя все в отдельности".
И я рассмотрел все на небесных скрижалях, и прочитал все, что на них,
и заметил для себя все, и прочитал книгу и все, что было на ней, все дела людей
и всех телесно-рожденных, которые будут на земле до самых отдаленных родов.
И после этого я тотчас прославил Господа, вечного Царя славы, за то,
что Он сотворил все произведения мира и восхвалил Господа за Его терпение, и
благословил Его за детей мира.
И в тот час я сказал: "блажен муж, который умирает как праведный и
благой, о котором не написано никакое писание неправды и против которого не
найдено вины"!
И те трое святых ангелов принесли меня и поставили меня на землю пред
дверями моего дома, и сказали мне: "возвести все своему сыне Мафусаилу и открой
всем своим детям, что ни один из смертных не праведен пред Господом, ибо Он
Творец их.
На один год мы оставим тебя при твоих детях,~--- пока ты не укрепишься
снова,~--- чтобы ты научил своих детей, и записал им это, и засвидетельствовал
им, всем твоим детям, и на другой год ты будешь взят из среды их.
Ибо добрые будут возвещать правду; праведный будет радоваться с
праведными, и они будут благожелать друг другу.
Грешник же умрет с грешником и отпадший потонет с отпадшим.
И те, которые сохранят справедливость, умрут ради дел людей и будут
соединены ради деяния нечестивых".
И в те дни они перестали говорить со мною.
И я пришел к своим домочадцам, прославляя Господа мира.
И теперь, сын мой Мафусаил, я рассказываю тебе все эти вещи и
записываю тебе; и я открыл тебе все и дал тебе писание обо всех них (светилах);
итак, сохрани же, мой сын Мафусаил, писания ради твоего отца, и передай их
грядущим родам.
Мудрость я дал тебе и твоим детям, и тем твоим детям, которые еще
придут, чтобы они передали ее своим детям и грядущим родам до вечности,~---
именно эту мудрость, превышающую их мысли.
И разумеющие ее не будут спать, и будут прислушиваться своим ухом,
чтобы научиться этой мудрости, ибо она понравится тем, которые кушают от неё,
лучше приятной пищи.
Блаженны все праведные, блаженны все, ходящие по пути правды, и не
погрешающие, подобно грешникам, в исчислении всех своих дней, в течение которых
солнце ходит на небе, входя и выходя через врата по тридцати дней вместе с
главами над тысячью этого порядка звезд, именно~--- вместе с четырьмя, которые
прибавляются и разделяют четыре части года, которые их направляют, и с ними
входят четыре дня.
И из-за них люди будут ошибаться, и не будут считать их при исчислении
целого движения мира; напротив люди будут ошибаться в них и не узнают их в
точности.
Ибо они (добавочные дни) относятся к исчислению года и действительно
отмечены навсегда~--- один в первых вратах, и один в третьих, и один в четвертых,
и один в шестых; и год завершается в 364 дня.
И рассказ об этом праведен, и точно указано исчисление этого
(т.е.года) ибо светила, и месяцы, и праздники, и годы, и дни мне показал и
внушил Уриил, которому Господь всего мироздания дал повеление ради меня
относительно воинства небесного; и он имеет власть над ночью и днем на небе,
чтобы заставлять свет светить над людьми,~--- солнце, луну и звёзды, и все силы
небесные, которые вращаются в своих кругах.
И таковы порядки звезд, которые заходят в своих местах и в свое время,
и праздники и месяцы.
И таковы имена тех, которые путеводят их (звезды) и которые
бодрствуют, чтобы он вступил в определенные им времена, в своих порядках, в
свои сроки, и месяцы, и времена господства, и по своим местам.
Четыре их путеводителя, которые разделяют четыре части года, вступают
прежде всех и после них двенадцать путеводителей порядков, которые разделяют
месяцы и год на 364 дня, рядом с главами над тысячью (хилиархами), которые
делают дни; и для четырех добавочных дней существуют те же путеводители,
которые разделяют четыре части года.
И из тех начальников над тысячью один расположен между путеводителем и
путеводимым позади мест, но только путеводители их делают разделение.
И вот имена путеводителей, разделяющих четыре установленные части
года: Мелкеел, и Гелеммелех, и Мелейял, и Нарел, И имена тех, которых они
ведут: Аднарел, и Ийязузаел, и Ийелумиел.
Эти трое следуют за путеводителями порядков, и один следует за троими
путеводителями порядков, следующими за теми место начальниками (топархами),
которые разделяют четыре части года.
В начале года первым восходит и управляет Мелкейял, который называется
Таммани и солнцем; и всего времени его господства, в продолжение которого он
управляет, девяносто один день.
И вот признаки дней, которые должны появляться на земле во время его
господства: пот, и жар, и тоска; все деревья тогда производят плоды, и листва
появляется на всех деревьях, и бывает жатва пшеницы и расцвет роз, и все цветы
тогда цветут на поле, но зимние деревья становятся сухими.
И вот имена подчиненных им (топархам) путеводителей: Беркеел,
Цалбезаел и еще другой, который присоединяется,~--- глава над тысячью, называемый
Голойязеф, и дни господства этого заканчиваются.
Другой путеводитель (топарх), который следует за ними, есть
Гелеммелек, которого называют также светящим солнцем; и все время его света
девяносто один день, и вот признаки дней на земле в то время: жар и сухость, и
плоды деревьев становятся зрелыми и спелыми, и плоды их сохнут; и овцы тогда
спариваются и становятся суягными; и тогда собираются все плоды земли и все,
что есть на полях, и бывает выжимание винограда: все это происходит во дни его
господства.
И вот имена, и порядки, и подчиненные им путеводители тех глав над
тысячью: Гедаел, и Кеел, и Геел, и имя начальника над тысячью, который
присоединяется к ним, Асфаел; и оканчиваются дни его господства.
\vs 1En 16:1
И теперь, мой сын Мафусаил, я хочу открыть тебе все видения,
которые я видел, рассказавши тебе их.
Два видения видел я, прежде чем взял жену, и они не похожи одно на
другое; в первый раз, когда я изучал писание, и во второй раз, прежде чем взять
твою мать, я видел страшные видения: и из-за них я молил Господа.
Я лег в доме моего деда Малелеила, и тогда я увидел в видении, как небо
опустилось и уменьшилось, и упало к земле.
И когда оно упало к земле.
И когда оно упало на землю, я увидел землю, как она была поглощена
великою бездною, и горы опустились на горы, и холмы погрузились на холмы, и
высокие деревья оторвались от своих стволов (корней), и низверглись и потонули
в бездне.
И от этого в моих устах обрелась речь, и я начал восклицать и сказал: "
погибла земля"!
И мой дед Малелеил разбудил меня, ибо я лежал около него, и сказал мне:
"отчего ты восклицаешь так, мой сын, и от чего ты так сетуешь"?
Тогда я рассказал ему видение, которые видел, и он сказал мне: "ужасно
то, что ты видел, мой сын!
И твое сновидение обнимает тайну всех грехов земли: она должна
погрузиться в бездну и потерпеть насильственную гибель.
И теперь, сын мой, встань и молись Господу славы,~--- ибо ты верующий,~---
чтобы остаток сохранился на земле целым и чтобы Он истребил не всю землю.
Сын мой!
С неба все это придет на землю, и на земле совершится насильственная
гибель".
После этого я встал, и просил, и умолял, и записал свою молитву для
грядущих родов, и я все покажу тебе, сын мой Мафусаил.
И когда я вышел вниз (т.е.из дому), и увидел небо и солнце, восходящее
на востоке, и луну, опускающуюся на западе, и все, как Он узнал это в начале,
то я прославил Господа суда, и превознес Его, ибо Он повелел солнцу выходить из
окон востока, чтобы оно поднималось, и восходило на плоскости неба, и
возносилось, и проходило теперь путь, который ему указан.
И я воздвиг руки свои в правде, и прославил Святого и Великого, и
говорил дыханием моих уст и телесным языком, который сотворил Бог для сынов
человеческих, чтобы они говорили им, и дал им дыхание, и язык, и уста, чтобы
они говорили благодаря этому.
"Будь прославлен Ты, о Господи, Царь Великий и Могущественный в Своем
величии, Господь всего небесного творения, Царь царей, и Бог всего мира!
И Твое божество, и царство, и величие пребывает во век и от века до
века, и Твое господство~--- чрез все роды, и все небеса служат Тебе престолом
вовек, и вся земля~--- подножием Твоих ног вовек и от века до века.
Ибо Ты сотворил и господствуешь над всем, и для Тебя совершенно ничего
нет трудного, и никакая мудрость не ускользнет от Тебя; она не отвращается от
своего престола,~--- Твоего престола,~--- ни от Твоего лица; и Ты знаешь, и видишь,
и слышишь все, и нет ничего, чтобы было сокровенно для Тебя, ибо Ты видишь все.
И теперь ангелы Твоего неба беззаконнуют, и гнев Твой пребывает на
плоти людей до дня великого суда.
И теперь, о Боже и Господи, и великий Царь, я молю и прошу, чтобы Ты
исполнил для меня мою просьбу, прошу оставить мне на земле потомство целым, и
не истреблять всю плоть человеческую, и не делать землю безлюдною, чтобы была
вечная гибель, и теперь, Господь мой, истреби от земли плоть, которая
разгневала Тебя, но плоть правды и праведности утверди как растение семени
навсегда, и не отвращай Твоего лица от молитвы раба Твоего, о Господи"!
\vs 1En 17:1
И после этого я видел другой сон, и я вполне открою его тебе, мой
сын.
И Енох начал и сказал своему сыну Мафусаилу: "тебе я буду говорить,
мой сын; слушай речь мою и приклони ухо свое к сновидению твоего отца!
Прежде чем я взял твою мать Едну, я видел в видении на своем ложе, и
вот телец вышел из земли, и тот телец был белый; и за ним вышло женское рогатое
животное, и вместе с ним вышли другие рогатые животные: одно из них было
черное, другое красное.
И то черное рогатое животное бодало красное и преследовало его на
земле; и скоро я не мог более видеть того красного рогатого животного.
Но то черное рогатое животное выросло и к нему пришло женское рогатое
животное, и я видел, как многие тельцы, которые были похожи на него и следовали
за ним, вышли от него.
И та корова,~--- та первая,~--- вышла от лица того первого тельца, чтобы
искать то красное животное, но не нашла его, и тотчас подняла великий жалобный
вопль, и искала его.
И я видел, как пришел к ней тот первый телец и успокоил ее, и с того
часа она более не ревела.
После этого она родила другого белого тельца, а после него родила
многих других тельцов и черных коров.
И я видел в моем сновидении, как тот белый вол также вырос и сделался
большим белым волом, и от него произошло много белых тельцов, которые были
похожи на него, И они стали производить белых тельцов, которые были похожи на
них, следуя один за другим.
И я опять видел своими очами, в то время как спал, и увидел
вверху небо, и вот одна звезда упала с неба, и она поднялась, и ела, и паслась
между теми тельцами.
И после этого я видел больших и черных тельцов, и вот они все
переменили свои загороди, и пастбища, и своих рогатых животных, и начали
сетовать друг с другом.
И я опять видел в видении, и посмотрел на небо, и вот я увидел много
звезд, как они упали и были низвергнуты с неба к той первой звезде и в среду
тех рогатых животных и тельцов; и вот они были с теми и паслись в среде их.
И я посмотрел на них и увидел, и вот все они обнаружили свои срамные
члены, как кони, и начали подниматься на тельцовых коров; и все они стали
стельными, и родили слонов, верблюдов и ослов.
И все тельцы устрашились и испугались их; и они начали кусаться своими
зубами и пожирать, и бодать своими рогами.
И они начали теперь поедать тех тельцов; и вот все дети земли начали
трепетать пред ними, и дрожать, и спасаться бегством.
И я опять видел их, как они начали бодаться сами между собою и
пожирать друг друга, и земля стала взывать.
И я опять поднял свои очи к небу и увидел в видении: и вот там вышли
из неба имевшие вид белых людей; из того места вышел один и вместе с ним трое.
И те трое, которые вышли после, взяли меня за руку и подняли меня
прочь от рода земли, и вознесли меня на высокое место, и показали мне башню,
высоко стоящую над землей, и все холмы были ниже ее.
И они сказали мне: "оставайся здесь, чтобы видеть всё, что произойдет
со всеми теми слонами, и верблюдами, и ослами, со звездами, и со всем
тельцами"!
И я видел одного из тех четверых, которые вышли прежде, как он
схватил звезду, прежде всех ниспадшую с неба, связал ей руки и ноги, и положил
ее в пропасть; пропасть же та была тесна и глубока, ужасна и мрачна.
И один из них обнажил свой меч и отдал его тем слонам, и верблюдам, и
ослам; тогда они начали поражать друг друга, так что вся земля дрожала
вследствие этого.
И когда я видел в видении,~--- вот там бросился теперь с неба вниз один
из тех четверых, которые спустились, и собрал и взял великие звезды, срамные
члены которых были как срамные члены коней, и связал их всех по рукам и ногам,
и положил их в ущелье земли.
И один из тех четверых пришел к тем белым тельцам, и научал его
(одного из них) тайне, в то время как он трепетал: он был рожден подобно тельцу
и сделался человеком, и выстроил себе большое судно и поселился в нем; вместе с
ним расположились также в том судне трое тельцов; и оно было закрыто над ними.
И я опять поднял свои очи к небу и увидел высокую крышу с семью
шлюзами на ней, и те шлюзы изливали много воды во двор.
И я видел опять, и вот, тогда открылись источники на почве в том
великом дворе, и эта самая вода начала волноваться и подниматься выше почвы, и
сделала тот двор невидимым, так что вся почва его закрылась водою.
И выростала на нем (дворе) вода, мрак и облако; и тогда я посмотрел на
высоту той воды, как она поднялась выше того двора, и текла поверх него, и
остановилась на земле.
И все тельцы того двора столпились вместе, так что я тотчас увидел,
как они потонули, и были поглощены и погибли в той воде.
Само же судно плавало по воде, между тем как все тельцы, и слоны, и
верблюды, и ослы на земле погрузились вместе со всем скотом, так что я не мог
более видеть их, и они не могли выйти, но потонули и погрузились в бездне.
И я опять видел в видении, как те шлюзы отложились от той высокой
крыши, и источники земли иссякли, и другие бездны открылись.
Тогда вода начала стекать в них, пока земля не сделалась видимою; а то
судно твердо встало на земле, и отступил мрак, и просиял свет.
А тот белый телец, который стал мужем, вышел из того судна и три
тельца с ним; и один из трех был белый, подобно тому тельцу, и один из них был
красный, как кровь, и один черный; и этот самый,~--- тот белый телец, отошел от
них.
И они начали рождать диких зверей и птиц, так что от всех их вместе
произошло разнообразное множество видов,~--- львы, тигры, псы, волки, шакалы,
дикие свиньи, соколы, коршуны, ястребы, орлы и вороны; и в среде их родился
белый телец.
И они начали грызться друг с другом: но тот белый телец, родившийся в
среде их, произвел дикого осла и вместе с ним белого вола; и дикий осел
умножился.
А тот телец, родившийся от него, произвел черную дикую свинью и белую
овцу; и та дикая свинья произвела многих свиней, та овца произвела двенадцать
овец.
И когда те двенадцать овец выросли, они передали одну из своей среды
ослам, и эти ослы опять передали ту овцу волкам, и та овца росла между волками.
И Господь привел одиннадцать овец~--- жить вместе с нею и пастись при
ней среди волков, и они размножились и выросли во многие овечьи стада.
И волки начали бояться их, и притесняли их, так что, наконец, стали
лишать жизни их агнцев; и они бросали их агнцев в многоводную реку; а те овцы
начали кричать о своих агнцах и жаловаться своему Господу.
И одна овца, которая была спасена от волков, убежала и ушла к диким
ослам; и я видел овец, как они сетовали, и кричали, и просили своего Господа
изо всех сил, пока тот Господь не сошел из высокого покоя на зов овец, и не
пошёл к ним и не посетил их.
И Он позвал ту овцу, удалившуюся от волков, и говорил с нею
относительно волков, чтобы она уговорила их не трогать овец.
И овца пошла к волкам по слову Господа, и другая овца сошлась с той
овцой и пошла с нею, и они обе вместе одна с другой пришли на сборище тех
волков, и говорили с ними, и увещевали их отныне не трогать впредь более овец.
При этом я видел волков, и как они стали еще более смирять овец всею
своею силою; и овцы кричали.
И Господь их пришел к овцам и начал бить тех волков; тогда волки
начали сетовать, овцы же сделались спокойными и тотчас не стали более кричать.
И я видел овец, как они ушли от волков; у волков же глаза были
ослеплены, и те волки вышли для преследования овец со всею своею силою.
И Господь овец шел с ними, предводительствуя ими, и все Его овцы
следовали за Ним; лицо же Его было блестящее, и вид Его страшен и величествен.
А волки стали преследовать тех овец, пока не настигли их при водном
озере.
И это самое водное озеро разделилось, и вода остановилась пред ними по
обеим сторонам; и их Господь, Который вел их, встал между ними и волками.
И так как те волки не стали уже видеть овец, то они вошли в средину
того водного озера и преследовали овец, и те волки погнались за ними в водном
озере.
И когда они увидели Господа овец, то воротились, чтобы убежать от
Него, но то водное озеро соединилось, внезапно приняло свою природу, и вода
поднялась и возвысилась, так что покрыла тех волков.
И я видел, как все волки, преследовавшие тех овец, погибли и потонули.
Но овцы вышли из той воды и перешли пустыню, где не было воды и травы;
и они начали открывать свои глаза и видеть; и я видел Господа овец, как Он пас
их и дал им воды и травы, и как та овца шла и вела их.
И та овца поднялась на вершину высокой скалы; и Господь овец послал ее
к ним.
И после этого я видел Господа овец, стоящего пред ними; и Его вид был
величествен и чрезмерно велик, и все те овцы видели Его и устрашились пред Его
лицом.
И все они устрашились и трепетали пред Ним, и кричали после ухода той
овцы, которая была при Нем, к другой овце, находившейся между ними: "мы не
можем вынести этого пред нашим Господом и взирать на Него".
И та овца, которая вела их, возвратилась и поднялась на вершину той
скалы; но овцы начали слепнуть и уклоняться от пути, который она показала им;
между тем та овца ничего не знала об этом.
И Господь овец сильно разгневался на них, и та овца узнала это и
спустилась с вершины скалы, и пришла к овцам, и нашла самую большую часть из
них ослепленною и уклонившеюся от своего пути.
И как только они увидели ее, устрашились и затрепетали пред ее лицом,
и пожелали возвратиться в свои загороди.
И та овца взяла с собою других овец и пришла к тем уклонившимся овцам,
и при этом начала умерщвлять их, и овцы устрашились пред ее лицом; и таким
образом та овца направила уклонившихся овец, и они возвратились в свои
загороди.
И я видел там видение, как та овца сделалась мужем, и выстроила
Господу овец дом, и повелела всем овцам стоять в том доме.
И я видел, как овца, сошедшаяся с той овцою, которая вела их, заснула;
и я видел, как все большие овцы погибли, и малые направились к своему месту, и
они пошли на пастбище и приблизились к водной реке.
Тогда отделилась от них та овца, которая вела их и которая сделалась
мужем, и заснула; и все овцы искали ее и подняли по ней великий вопль.
И я видел, как они прекратили вопль по той овце и переправились через
ту водную реку; и стояли всегда овцы, ведшие их, на месте тех, которые заснули
и которые вели их.
И я видел, как овцы пришли в хорошее место и в вожделенную и
великолепную страну, и видел, как те овцы насытились; а тот дом стоял между
ними в вожделенной стране.
И глаза их то открывались, то ослеплялись, пока не восстала другая
овца, и не повела их, и не направила их всех, и глаза их открылись.
И псы, и лисицы, и дикие свиньи начали пожирать тех овец, пока не
восстала другая овца,~--- баран из их среды, который вел их.
И тот баран начал бодать на обе стороны тех псов, лисиц и диких
свиней, пока не уничтожил их всех.
И у той овцы раскрылись глаза, и она увидела того барана, бывшего
между овцами, как он отрекся от своего достоинства и начал бодать тех овец, и
попирал их, и действовал непристойно.
И Господь овец послал овцу к другой овце, и возвысил ее (последнюю),
чтобы она была бараном и вела овец вместо той овцы, которая оказалась неверной
в своем достоинстве.
И она пошла к ней и говорила только с ней, и поставила ее бараном, и
сделала ее царем и вождем овец; а между всем этим псы притесняли овец.
И первый баран преследовал того второго барана, и тот второй баран
встал и убежал от него; и я увидел, как те псы низвергли того первого барана.
И тот второй баран возвысился и вел малых овец; и тот баран родил
многих овец и заснул; и малая овца сделалась бараном вместо него, и стала
вождём и царем тех овец.
И выросли и размножились те овцы, и все псы, и лисицы, и дикие свиньи
устрашились и разбежались от него; и тот баран бодал и убивал диких зверей; и
те дикие звери не могли уже осилить овец, и никогда уже не похищали у них
ничего.
И тот дом стал великим и широким, и тем овцам была выстроена высокая
башня над тем домом для Господа овец; и тот дом был низок, а башня была
возвышена и высока; и Господь овец стоял на той башне, и пред Ним поставили
полный стол.
И я видел опять тех овец, как они опять заблудились и пошли
многоразличными путями, и оставили тот свой дом; и Господь овец призвал
некоторых из овец и послал их к овцам, но овцы начали умерщвлять их.
И одна из них спаслась и не была умерщвлена, и она убежала и кричала
об овцах, и он хотел ее умертвить; но Господь овец спас ее из рук их и возвел
ее ко мне, и позволил ей жить там.
И многих других овец Он посылал к тем овцам, чтобы свидетельствовать
(или увещевать) и сетовать о них.
И после этого я видел: вот они оставили дом Господа овец и его башню;
они уклонились совершенно и их глаза ослепли; и я видел Господа овец, как он
допустил много поражений над ними в их отдельных стадах, так что те овцы начали
жаловаться на такие поражения и переменили место.
И Он предал их в руки львов и тигров, и волков, и шакалов, и в руки
лисиц и всех диких зверей; и дикие звери стали разрывать тех овец.
И я видел, что Он оставил тот дом их и их башню, и предал их всех в
руки львов, в руки всех диких зверей, чтобы они разрывали их и пожирали.
И я начал кричать изо всех сил, и призывать Господа овец, и
представлять Ему относительно овец, что они пожираются всеми дикими зверями.
Но Он оставался спокойным, когда видел это, и радовался, что они
пожираются, и истребляются и расхищаются; и Он оставил их в руках всех диких
зверей на съедение.
И Он призвал семьдесят пастырей,~--- и отверг тех овец,~--- чтобы они
пасли их, и сказал пастырям и их товарищам: "каждый из вас должен отныне пасти
овец, и все, что Я вам прикажу, то делайте!
И Я передаю их вам по числу, и буду вам объявлять: кто из их должен
погибнуть, тех истребляйте"!
И Он предал им тех овец.
И Он призвал другого и сказал ему: "замечай и смотри за всем, что
будут делать пастыри с этими овцами: ибо они будут губить их более, чем Я им
повелел.
И всякий излишек и уничтожение, которое будет совершаемо пастухами,
запиши,~--- именно сколько губят они по Моему повелению и сколько по своей
собственной воле; и запиши о каждом пастыре в отдельности все, что они губят.
И прочитай это предо Мною по числу (с указанием числа), сколько они
погубили по собственной воле и сколько предано им на погибель, чтобы это было
для Меня свидетельством против них, дабы я знал всякое действие пастырей, чтобы
передать их суду; и смотри, что они делают,~--- пребывают ли в Моем повелении,
которое Я им дал, или нет.
Но они не должны открывать им этого и наставлять их на путь, но запиши
только все, что они погубят, всякий раз о каждом в отдельности, и представь все
Мне"!
И я видел, как те пастыри пасли в определенное им время, и они начали
умерщвлять и погубят более чем им было повелено, и предали тех овец в руки
львов.
И львы и тигры пожирали и истребляли большую часть тех овец, и дикие
свиньи пожирали вместе с ними; и они сожгли ту башню и разрушили тот дом.
И я сильно опечалился из-за башни, так как самый дом овец был
разрушен; и после этого я не мог уже видеть тех овец, входили ли они в тот дом.
И пастыри и их товарищи предали тех овец всем диким зверям, чтобы они
пожирали их; и каждый в отдельности из них получил в своё время определенное
число, и о каждом в отдельности записал другой в книгу, сколько он погубил.
И каждый из них умертвил и погубил гораздо более чем ему было
позволено; и я начал плакать и сильно сетовать о тех овцах.
И я видел в видении того писца, как он записал каждую овцу, погибшую
от тех пастырей, день за днем, и всю книгу вознес к Господу овец, и представил
и показал, что они сделали, и всех, которых каждый из них уничтожил, и всех,
которых они предали погибели.
И книга была прочитана пред Господом овец, и он взял книгу в свои
руки, и прочитал ее, и сложил ее.
И тотчас я увидел, как пастыри пасли в продолжение двенадцати часов; и
вот три из тех овец возвратились, и пришли, и приступили, и начали строить все,
что было разрушено в том доме; но дикие свиньи помешали им, так что они не
могли продолжать этого.
И они начали опять строить, как прежде, и возвели ту башню, и она была
названа высокой башней; и они начали опять ставить стол пред башнею, но весь
хлеб на нем был скверен и нечист.
И по отношению ко всему глаза у тех овец были ослеплены, так что они
не видели, а также и у пастырей их, весьма многие из них были преданы пастырям
на погибель, и они попирали овец своими ногами и пожирали их.
И Господь овец оставался спокойным, пока все овцы не рассеялись по
полю и не перемешались с ними (диким зверями), и они (пастыри) не спасли их от
рук зверей.
И тот, который писал книгу, вознес ее к обителям Господа овец, и
показал ее, и умолял Его за их, и просил Его, показав Ему всю деятельность
пастырей их, и представил Ему свидетельство против всех пастырей.
И он взял книгу, сложил ее у Него и вышел.
И я смотрел до тех пор, пока таким образом не приняли паству
тридцать семь пастырей, и они все окончили каждый свое время, как первые; и
другие приняли их (овец) в свою власть, чтобы каждый пас их по определенному им
времени,~--- каждый пастырь в свое время.
И после этого я видел в видении, как пришли птицы небесные,~--- орлы,
коршуны, ястреба, вороны; орлы же предводительствовали всеми птицами; и они
начали пожирать тех овец, и выклевывать им глаза, и пожирать их мясо.
И овцы кричали, так как их мясо было пожираемо птицами, и я восклицал
и жаловался во время моего сна на того пастыря, который пас овец.
И я видел, как те овцы были пожраны псами, и орлами, и ястребами, и
они не оставили им ни мяса, ни кожи, ни сухожилий, так что от них остался
только остов, но и остов их упал на землю, и овец стало мало.
И я смотрел до тех пор, пока не приняли паству двадцать три пастыря,
и окончили, управляя каждый по определенному им времени, пятьдесят восемь
времен.
Но от тех белых овец родились малые агнцы, и они стали открывать свои
глаза, и видеть, и кричать овцам.
И овцы не кричали им и не слышали, что и сказали им, но были
чрезвычайно глухи, и их глаза были слишком и чрезмерно ослеплены.
И я видел в видении, как вороны налетели на тех агнцев и взяли одного
из тех агнцев, овец же разорвали и пожрали.
И я видел, как у тех агнцев выросли рога, и вороны низвергли их рога;
и я видел, как выскочил один великий рог,~--- одна из тех овец; и их глаза
открылись.
И я смотрел за ним, и глаза их раскрылись; и она кричала к овцам, и
юнцы увидели ее и все побежали к ней.
И между всем тем те орлы, и коршуны, и вороны, и ястреба все еще
разрывали овец беспрестанно, и слетались, на них и пожирали их; но овцы
оставались покойными, и юнцы сетовали и кричали.
И те вороны сражались и боролись с ними, и хотели сломить его рог, но
ничего не могли сделать с ним.
И я видел их, пока не пришли пастыри, и орлы, и те коршуны и ястреба,
и они кричали воронам, чтобы они сломили рог того юнца; и они боролись и
сражались с ними, и он боролся с ним, и кричал, чтобы пришла к нему помощь.
И я видел, как пришел тот муж, который записывал имена пастырей и
представлял Господу овец, и он помог тому юнцу, и показал ему все, чтобы пришла
его помощь.
И я видел, как тот Господь овец пришел к ним во гневе, и все видевшие
Его убежали, и упали все в Его тени пред лицом Его.
Все орлы, и коршуны, и вороны, и ястребы собрались и привели с собою
всех полевых овец, и все они сошлись и помогали друг другу, сломить тот рог
юнца.
И я видел того мужа, который писал книгу по повелению Господа, как он
развернул ту книгу умертвления, которое совершили те двенадцать последних
пастырей, и он показал пред Господом овец, что они умертвили гораздо более, чем
предшествовавшие.
И я видел, как пришел к ним (к хищным птицам и зверям) Господь овец,
и взял в Свою руку посох гнева, и ударил в землю, так что она расторгалась, и
все звери и птицы небесные упали с овец, и погрузились в землю, и она
замкнулась над ними.
И я видел, как овцам дан был великий меч: тогда овцы выступили против
тех полевых зверей, чтобы умертвить их, и все звери и птицы небесные
разбежались от их лица.
И я видел, как был воздвигнут престол в любимой земле, и Господь овец
воссел на нем; и он взял все запечатанные книги и раскрыл их пред Господом
овец.
И Господь призвал тех шесть (или семь) первых белых, чтобы они
принесли к Нему, начиная от первой звезды, пришедшей вперёд, все звезды, у
которых срамные члены были как срамные члены коней, и первую звезду, которая
ниспала прежде всех; и они принесли их все к Нему.
И Он сказал тому мужу, который писал пред Ним и который был одним из
тех шести (или семи) белых, и сказал ему: "возьми тех семьдесят пастырей,
которым Я предал овец, и которые взяли их и умертвили из них более, чем Я им
повелел, самовластно"!
И вот я видел их всех связанными, и они все стояли пред Ним.
И суд совершился, прежде всего, над звездами, и они были судимы и
оказались виновными, и пришли к месту осуждения, и их бросили в глубокое место,
наполненное огнем, пылающее и наполненное огненными столбами.
И те семьдесят пастырей были судимы и оказались виновными, и точно
также были брошены в ту огненную пропасть.
И я видел тогда, как была открыта подобная пропасть в средине земли,
наполненная огнем, и как принесли тех ослепленных овец, и они все были судимы и
найдены виновными, и брошены в ту огненную пропасть, и они сгорели: а пропасть
эта была направо от того дома.
И я видел, как сгорели те овцы, и кости их сгорели.
И я встал, чтобы видеть, как Он украшал тот древний дом: и выломали в
нем все столбы, и все балки и украшения этого дома были завернуты вместе с
ними; и выломали их совсем, и положили их в одно место на юге страны.
И я видел Господа овец, как он принес новый дом больше и выше того
первого, и поставил его на месте первого, который был завернут; все его столбы
были новы и больше, чем украшения первого древнего, который Он выломал; и все
овцы были в нем.
И я видел всех овец, которые остались целыми, и всех зверей на земле
и всех птиц небесных, как они пали ниц и воздавали честь тем овцам, и умоляли
их, и слушались их в каждом слове.
И после этого меня взяли те трое, одетые в белом, которые подняли
меня прежде, за мою руку, и в то время, как рука того юнца взяла меня, они
подняли меня и посадили меня посреди тех овец, прежде чем совершился суд.
А те овцы были все белы и их шерсть была большая и чистая.
И все и все погибшие и рассеянные овцы, и все звери полевые, и все
птицы небесные собрались в том доме, и у Господа овец была великая радость, так
как все они были добры и возвратились к Его дому.
И я видел, как они сложили тот меч, который был дан овцам, и принесли
назад в Его дом, и он был запечатан пред лицом Господа; и все овцы были
заключены в тот дом, и он не вмещал их.
И у них у всех были открыты глаза, так что они видели доброе, и не
было между ними ни одной, которая бы не сделалась видящею.
И я видел, что тот дом был велик и широк, и весьма наполнен.
И я видел, что родился белый телец с большими рогами, и все звери
полевые и все птицы небесные устрашились его и умоляли его все время.
И я видел, как весь род их изменился, и все они стали белыми
тельцами; и первый между ними (был Слово, и это Слово сделалось) сделался
великим зверем, и имел большие черные рога на своей голове; и Господь овец
радовался, взирая на них и на всех тельцов.
И я спал в среде их, затем пробудился и видел все.
Таково видение, которое я видел в то время, как спал, и я пробудился
и прославил Господа правды, и воздал Ему хвалу.
И после этого я поднял великий вопль, и мои слезы не останавливались,
так как я не мог более удержаться; когда я смотрел, то у меня лились слезы по
поводу того, что я видел, ибо все придет и исполнится; и всякое деяние людей
мне было показано по порядку.
И в ту ночь я вспомнил о моем первом сне; также и из-за этого я
плакал и трепетал, ибо я видел то видение.
\vs 1En 18:1
И теперь, мой сын Мафусаил, призови ко мне всех своих братьев, и
собери ко мне всех сыновей твоей матери; ибо слово побуждает меня и дух излился
на меня, чтобы я открыл вам все, что придет на вас до вечности.
После этого Мафусаил пошел и призвал всех своих братьев к себе, и
собрал своих родственников.
И он (Енох) говорил со всеми своими детьми о правде, и сказал:
"вслушайтесь, сыны мои, каждую речь вашего отца и должным образом внемли гласу
моих уст, ибо я увещеваю вам, возлюбленные мои: любите праведность и ходите в
ней.
И не приближайтесь к праведности с двояким сердцем, и не
присоединяйтесь к тем, у которых двоякое сердце, но ходите в правде, сыны мои;
и она приведет вас на добрые пути, и правда будет вашей помощницей.
Ибо я знаю, что дела насилия возьмут верх на земле, и великое осуждение
совершится на земле; и всякая неправда прекратится и будет отделена от своих
корней, и все здание ее исчезнет.
И неправда опять повторится, и все дела неправды и все дела насилия и
беззакония вторично совершатся на земле.
И так как тогда усилится неправда, и грех, и хула, и насилие, и другого
рода действия, и увеличится отпадение, и беззаконие, и нечистота, то придет
великое осуждение с неба на всех них, и святой Господь выйдет с гневом и
наказанием, чтобы совершить суд на земле.
В те дни насилие будет отделено от своих корней, и корни неправды
погибнут вместе с ложью, и они исчезнут из-под неба.
И все идолы язычников будут преданы погибели; башни будут сожжены
огнем, и их уберут со всей земли; и они будут брошены по осуждению в огнь, и
погибнут в гневе и жестоком осуждении, которое продолжится вовек.
И восстанет тогда праведный от сна, и мудрость восстанет и будет дана
им.
И после того корни неправды будет отделены, и грешники погибнут от
меча, у клеветников будут отделены корни во всяком месте, и те, которые
замышляют насилие и произносят хулу, погибнут от острия меча.
И после этого будет другая седмина~--- восьмая, седмина правды; и будет
дан ей меч, чтобы судить и справедливость исполнить над теми, которые поступают
насильственно, и грешники будут преданы в руки праведных.
И в конце ее они приобретут домы своею спаведливостю, и создастся дом
великому Царю в прославление навсегда и навечно.
И после этого в девятую седмину откроется всему миру праведный суд, и
все деяния нечестивых исчезнут со всей земли; и мир будет присужден к погибели,
и все люди будут взирать на путь праведности.
И после этого в десятую седьмину, в седьмую ее часть, будет суд на
вечность, который совершится над стражами, и явится великое небо,
произрастающее из среды ангелов.
И прежнее небо уменьшится и исчезнет, и явится новое небо, и все силы
небесные седмерицею будут светить вовек.
И после этого будет много седьмин без числа в вечность во благо и в
правду, и с тех пор грех не будет более именоваться до вечности.
И теперь я говорю вам, мои сыны, и указываю вам пути правды и пути
насилия, и я укажу вам их опять, чтобы вы знали, что придет.
И теперь послушайте, мои сыны, и ходите в путях правды, и не ходите по
путям насилия, ибо навеки погибнут все, ходящие путями неправды.
\vs 1En 19:1
Написанное Енохом писцом пространное учение мудрости,~---
которое заслуживает прославления от всех людей и есть судья всей земли,~---
для всех моих детей, которые будут жить на земле, и для будущих родов,
которые будут ходить в праведности и мире.
Да не смущается дух ваш из-за времен, ибо Святой и Великий всему
положил дни.
И праведный восстанет от сна, восстанет и пойдет по пути правды, и весь
его путь и стезя будут в вечном благе и милости для праведного, и даст
господство, и он будет жить во благе и правде, будет ходить в вечном свете.
И погибнет грех во мраке навсегда и навечно, и более уже не появится от
того дня до вечности.
И после этого Енох начал возвещать из книг.
И сказал Енох: "о детях правды, и об избранных мира, и о растении
справедливости и праведности, говорю я это вам, мои сыны,~--- я Енох,~--- согласно с
тем, что мне открыто в небесном видении, и что я знаю чрез слово святых ангелов
и что узнал из скрижалей небесных".
И Енох начал теперь повествовать из книг и сказал: "я родился седьмым
в первую седьмину, когда суд и правда еще медлили.
И после меня во вторую седьмину восстанет великая злоба и произрастет
обман, и во время нее будет первый конец, и во время ее спасется один муж; и
после того, как он (конец) совершится, возрастет неправда, и Он даст закон
грешникам.
И после этого в третью седьмину, в конце ее, будет избран в растение
праведного суда один муж, и после него явится растение правды навсегда и
навечно.
И после этого в четвертую седьмину, в конце ее, будут видимы видения
святых и праведных, и закон для всех будущих родов и двор будет сделан (дан)
им, И после этого в пятую седьмину, в конце ее, будет устроен дом славы и
господства навсегда и навечно.
И после этого в шестую седмину все, которые будут жить во время ее,
будут ослеплены, и все они погрузятся своею мыслью в неразумие, забыв мудрость;
и во время нее будет взят вверх один муж; и в конце его господства будет сожжен
огнем, и весь род избранного корня будет рассеян.
И после этого в седьмую седьмину восстанет отпадший (или развращенный)
род, и много будет деяний его, и все его деяния будут отпадением.
И в конце ее будут награждены избранные и праведные от вечного
растения правды, между тем как им будет дано седьмикратное наставление обо всем
Его творении.
Ибо есть ли где-нибудь сын человеческий, который услышал бы голос
Святого и не был бы потрясен?
И есть ли где-нибудь такой, кто мог бы мыслить его мысли?
И где есть такой, кто мог бы видеть все произведения неба?
И как мог бы существовать тот, кто узнал бы произведения неба, и
увидел бы Его дыхание, и Его дух, и подсказал бы о том, или вошел бы наверх и
увидел все концы (буквально~--- крылья) их (небес), и мог бы придумать их, или
сделать что подобное им?
И есть ли где-нибудь такой муж, который мог бы знать, какова широта и
длина земли, и кому открыта мера всего этого?
И найдется ли кто-нибудь, который мог бы знать длину неба, и какова
его высота, и на чем оно утверждено, и как велико число звезд, и где покоятся
все светила?
И теперь я говорю вам, мои сыны, любите правду и ходите в ней,
ибо пути правды достойны, чтобы принять их; а пути неправды исчезают внезапно и
погибают.
И некоторым людям из грядущих родов будут открыты пути насилия и
смерти, и они будут держать себя далеко от них, и не будут им следовать.
И теперь я говорю вам~--- праведным: ходите не по злому пути и не в
насилии, и не по путям смерти, и не приближайтесь к ним, чтобы вам не
погибнуть.
Но ищите и изберите себе правду и приятную для Бога жизнь, и ходите по
путям мира, чтобы выжили и имели радость.
И держите в мыслях вашего сердца и не допускайте, чтобы речь моя
искоренилась из вашего сердца, ибо я знаю, что грешники соблазнят людей~---
унижать мудрость, и она не приобретет нигде места, и искушения всякого рода не
уменьшатся.
Горе тем, которые созидают неправду и насилие, и полагают основание
обману; ибо они внезапно будут искоренены и не будут иметь мира.
Горе тем, которые строят свои дома грехом, ибо они будут искоренены до
основания и падут от меча; и приобретающие золото и серебро внезапно погибнут
на суде.
Горе вам, богатые, ибо вы положитесь на ваше богатство, и вы лишитесь
своего богатства, так как вы не думали о Всевышнем в дни своего богатства.
Вы творили хулу и неправду, и приготовили себя ко дню кровопролития, и
ко дню мрака, и ко дню великого суда.
Это я говорю вам, что вас истребит до основания Тот, Кто сотворил вас:
и не будет никакого сострадания к вашему падению; и ваш Творец будет радоваться
вашей погибели.
И ваши праведники в те дни будут служить поношением для грешников и
нечестивых.
О, если бы мои очи были водной тучей, чтобы плакать о вас, и
излить мои слезы как водную тучу, дабы я получил успокоение для своего сердца
от печали!
Кто позволил вам совершать ненависть и злобу?
Так пусть же постигнет вас, грешники, суд!
Не страшитесь грешников, вы~--- праведные, ибо Господь опять предаст их
в ваши руки, чтобы вы совершили над ними суд, как желаете.
Горе вам, изрекающим проклятие, чтобы проклинать неразрешимо; и ваше
исцеление должно быть далеко от вас вследствие ваших грехов!
Горе вам, воздающим своему ближнему злом, ибо вам будет уготовано по
вашим делам!
Горе вам лжесвидетелям и тем, которые показывают неправду, ибо вы
внезапно погибнете!
Горе вам, грешникам, так как вы преследуете праведных; ибо вы будете
преданы и преследуемы, вы~--- люди неправды, и тяжело будет на вас их (праведных)
ярмо.
Вы, праведные, надейтесь, ибо грешники внезапно погибнут пред
вами, и вы будете господствовать над ними, как желаете!
И в день страдания грешников ваши юнцы вознесутся и взлетят, как орлы,
и выше, чем у коршуна, будет ваше гнездо, вознесетесь; и как кролики вы
проникнете в ущелье земли и в расселины скал навсегда пред праведными; а они
будут воздыхать из-за вас и плакать, как лесные духи.
Но и вы не бойтесь,~--- вы страдающие, ибо для вас будет исцеление, и
будет светить вам блестящий свет, и призыв к покою вы услышите с неба.
Горе вам, вы~--- грешники, ибо ваше богатство позволяет вам казаться
праведным, но ваше сердце изобличает вас, что вы грешники, и эта речь будет
свидетельствовать против вас для напоминания о ваших злодеяниях.
Горе вам, которые едите тук пшеницы и пьете силу корня источника, и
попираете своею силою приниженных!
Горе вам, которые всегда пьете воду, ибо вам внезапно будет воздано, и
вы завянете и иссохнете, так как вы оставили источник жизни!
Горе вам, совершающим неправду, и обман, и хулу: это будет памятью
против вас к вашему злу!
Горе вам, сильные, поражающие своею силою праведного, ибо придет день
вашей погибели, в то время много хороших дней придет для праведных день~---
вашего суда.
Веруйте вы, праведные, ибо грешники будут позором для вас и
погибнут в день неправды.
Да будет вам (грешникам) известно, что Всевышний думает о вашей
погибели, и ангелы радуются вашей погибели.
Что будете вы делать, грешники, и куда убежите в тот день суда, когда
услышите голос молитвы праведных?
И для вас не будет того, что для них,~--- для вас, против которых будет
свидетельством это слово: "вы сделались союзниками грешников".
И в те дни молитва праведных проникнет к Господу, и для вас наступят
дни вашего суда.
И все ваши неправедные речи будут прочитаны пред Великим и Святым, и
ваше лицо пристыдится, и всякое дело, основанное на неправде, будет отринуто.
Горе вам, грешникам, в средине моря и на суше, воспоминание которых о вас
недоброе!
Горе вам, приобретающим себе серебро и золото не по правде и
говорящим: "мы сделались богатыми и имеем сокровища, и владеем всем, чего
хотим; и теперь мы исполним все то, что нам думается, ибо мы собрали серебра и
наполнили наши кладовые, и как воды много у нас оберегающих наши дома".
как вода, разольется ваша ложь, ибо богатство не сохранится у вас, но
внезапно будет у вас отнято, так как вы все приобрели неправдою, и вы сами
подвергнетесь великому осуждению.
И теперь я клянусь вам, мудрым и безумным; ибо вы много
переживете (или увидите) на земле.
Ибо вы, мужи, будете возлагать на себя украшений более, нежели жены, и
разноцветного более, чем дева, в царском достоинстве и величии и власти, и в
серебре, и в золоте, и в пурпуре, и в почести, и в пище они разольются, как
вода.
Посему им не достает учения и мудрости, и чрез то они погибнут вместе
со своими сокровищами, и со всею своею силою и почестью; и в позоре, и в
умертвлении, и в великой бедности их дух будет брошен в огненную печь.
Я клянусь вам, грешники: как гора не была и не будет рабой, ни
возвышенность служанкой жены, так точно и грех не был послан на землю, но люди
произвели его из своей головы; и великому осуждению подпадут те, которые
совершают его.
И неплодие не дано было жене, но ради дела своих рук она умирает без
детей, Я клянусь вам, грешники, Святым и Великим, что всякое злое дело ваше
открыто на небесах, и ни одно из ваших деяний насилия не утаено или прикрыто.
И не думайте в своем духе и не говорите в своем сердце,~--- вы не знаете
и не видите, что каждый грех записывается ежедневно на небе пред Всевышним.
Отныне вы знайте, что все ваше насилие, которое вы совершаете,
записывается каждый день до дня вашего суда, Горе вам, безумные, ибо вы
погибнете чрез ваше безумие; и так как вы не слушаетесь мудрых, то ничто доброе
не будет вашим уделом.
И теперь знайте, что вы приготовлены на день погибели, и не надейтесь,
что вы будете жить,~--- вы грешники,~--- но вы погибнете и умрете, так как вы не
знаете никакого выкупа: ибо вы приготовлены на день великого суда, и на день
страдания и великого позора для вашего духа.
Горе вам,~--- вы ожесточенные, которые делаете зло и едите кровь!
Откуда у вас хорошая пища, и питье, и насыщение?
От всякого блага, которое наш Господь, Всевышний в изобилии послал на
землю: посему вы не должны иметь мира.
Горе вам, любящим свои деяния неправды!
Почему вы чаете блага для себя?
Знайте, что вы будете преданы в руки праведных; они перережут ваши шеи
и умертвят вас, и не будут иметь сострадания к вам.
Горе вам, радующимся страданию праведных, ибо для вас не будет вырыта
могила!
Горе вам, для которых слова праведных только пустые речи, ибо для вас
не будет надежды на жизнь!
Горе вам, записывающим лживые речи и беззаконные слова; ибо они
записывают свою ложь, чтобы их слушали и не забывали их безумия; так не будет
же для них мира, но они умрут внезапной смертью!
Горе тем, которые совершают нечестие, и похваляют и сохраняют в
уважении лживые речи: вы погибнете чрез это и для вас нет хорошей жизни!
Горе вам, искажающим слова праведности!
И они отпадут от вечного закона и сами себя делают тем, чем не были,
именно~--- грешниками; они будут попираемы на земле.
В те дни вы, праведные, приготовьтесь вознести свои мысленные молитвы,
вы представите их как свидетельство ангелам, чтобы они представили беззакония
грешников Всевышнему в напоминание.
В те дни народы придут в смятение, и поколения народов восстанут ко
дню погибели.
И в те дни выйдет плод материного чрева, и они (матери) растерзают
своих собственных детей; они оттолкнут от себя своих детей, и у них выпадет
недоношенный плод; грудных детей они оттолкнут от себя, и не возвратятся опять
к ним, и не сжалятся над своими любимцами.
Опять клянусь вам, грешники, что грех уготован на день беспрерывного
кровопролития.
И они будут поклоняться камням, и другие будут делать изображения из
золота и серебра, и из дерева и глины; и другие будут поклоняться нечистым
духам, и демонам, и разного рода идолам в идольских капищах: между тем у них
(идолов) нельзя найти никакой помощи.
И они погрузятся в неведение вследствие безумия своего сердца, и их
очи будут ослеплены страхом их сердца и сновидениями.
Чрез них они впадут в неведенье и страх, ибо они все свои дела
совершают во лжи, и поклоняются камням; и они погибнут все разом.
Но в те дни блаженны все те, которые принимают слова мудрости, и знают
ее, и исполняют пути Всевышнего, и ходят по пути правды, и с безбожными: ибо
они будут спасены.
Горе вам, распространяющим зло между своими ближними, ибо вы будете
умерщвленны в геенне.
Горе вам, полагающим основание греху и лжи, и вызывающим ожесточение
на земле: ибо за это их постигнет конец.
Горе вам, которые строите свои дома потом других и у которых
строительный материал есть не что иное, как черепица и камень греха; я говорю
вам, что для вас нет мира.
Горе тем, которые отвергают меру и наследие своих отцов, пребывающее
вечно, и прилепляют свои души к идолам: ибо для них не будет покоя.
Горе тем, которые делают неправду, и помогают насилию, и умерщвляют
своих ближних, в день великого суда: ибо Он низринет вашу славу, и положит вам
злобу на сердце, и возбудит дух Своего гнева, чтобы погубить вас всех мечом; и
все праведные и святые припомнят ваши грехи.
И в те дни будут умерщвлены в одном месте отцы вместе со своими
сынами, и братья друг с другом упадут от смерти, пока их кровь не потечет
подобно потоку.
Ибо муж не будет из сострадания удерживать свою руку от своих сынов и
от своих внуков, убивая их; и грешник не будет сдерживать своей руки от своего
почетнейшего брата; от утренней зари до солнечного захода они будут умерщвлять
друг друга.
И конь будет по самую грудь ходить в крови грешников, и колесница
потонет до своего верха.
И в те дни ангелы сойдут в убежища грешников и соберут в одно место
всех тех, которые помогали греху; и Всевышний восстанет в тот день, чтобы
произвести великий суд над всеми грешниками.
Но над всеми праведными и святыми Он поставит стражами святых ангелов,
чтобы они берегли их, как зеницу ока, пока не наступит конец всякой злобе и
всякого греха; и если даже праведные спят продолжительным сном, то и тогда они
не должны ничего бояться.
И кто мудр между людьми, тот увидит истину, и дети земли поймут все
слова этой книги, и узнают, что их богатство не может спасти их при погибели их
греха.
Горе вам, грешники, если вы мучите праведных,~--- в день жестокого
страдания,~--- и сжигаете их огнем: вам будет воздано по вашим делам.
Горе вам, развращенные сердцем, заботящиеся о том, чтобы измышлять
злое; на вас нападет страх, и никто не поможет вам.
Горе вам, грешники, ибо вы будете гореть в озере огненного пламени за
слова своих уст и за дела своих рук, которыми вы действуете нечестиво.
И теперь знайте, что ангелы на небе будут выведывать о ваших делах у
солнца, и луны, и звезд,~--- выведывать о ваших греховных делах, ибо вы
совершаете на земле суд над праведными.
И Он сделает свидетелями против вас каждую тучу, и облако, и росу, и
дождь, ибо все они задерживаются вами, чтобы не ходить на вас; и не должны ли
они думать о ваших грехах?
И теперь дайте дары дождю, чтобы он не был задержан от снисхождения
на вас, а также не была бы задержана роса, если она получила от вас золото и
серебро.
Когда будут падать на вас иней и снег вместе с их холодом, и все
снежные ветры со всеми своими бедствиями, то вы не устоите в те дни против них.
Рассмотрите небо, все дети земли, и каждое произведение
Всевышнего, и устрашайтесь пред Ним, и не делайте пред Ним ничего злого!
Если бы Он закрыл окна небесные и задержал из-за вас дождь и росу,
чтобы они не падали на землю, то что вы стали бы тогда делать?
И если Он посылает Свой гнев на вас и на все ваши произведения, то
можете ли вы не поклоняться Ему, так как вы высказываете надменные и бесстыдные
речи против Его правды, и для вас не будет мира.
И не видите ли вы управителей кораблей, как их корабли бросаются
волнами, качаются ветрами, и подвергаются опасности; и они вследствие этого
впадают в страх, так как они взяли с собою в море самое лучшее из своего
имения, и они беспокоятся в своем сердце, как бы море не поглотило их и как бы
они не погибли в нем?
Все море, и все его воды, и все его движение~--- не есть ли творение
Всевышнего, и не запечатал ли Он все Свое дело и не заключил ли его совсем в
песок?
Оно засыхает от Его угроз и устрашается, и все его рабы и все, что
есть в нем, умирают: и вы, грешники, живущие на земле, не боитесь Его.
Не сотворил ли Он небо, и землю, и все, что есть на них?
И кто дал учение и мудрость всем, которые движутся на земле и которые
живут в море?
Не боятся ли моря все цари кораблей?
А грешники не боятся Всевышнего.
В те дни, когда Он пошлет на вас мучительный огонь, куда вы
убежите, и где спасетесь?
И когда Он пошлет на вас Свое слово, не будете ли вы поражены тогда и
не устрашитесь ли?
Все светила потрясутся тогда от великого страха, и вся земля будет
поражена, и она задрожит и устрашится.
И все ангелы выполнят данные им повеления и будут стараться укрыться
пред Тем, Кто велик во славе, и дети земли задрожат и затрепещут; и вы, о
грешники, будете прокляты навеки, и пусть не будет для вас мира!
--- Не страшитесь вы, души праведных, и уповайте на день своей смерти в
правде!
И не печальтесь, что ваша душа нисходит в царство мертвых в великой
скорби, в горе, и воздыхании, и печали, и что ваше тело не обрело в вашей жизни
того, чего заслужила ваша благость, скорее теперь в день, когда вы стали
одинаковыми с грешниками, и в день проклятия и осуждения.
И когда вы умираете, грешники говорят над вами: "праведники умирают,
как и мы, и какая для них польза от их дел?
Вот они, как и мы, умерли в печали и мраке, и какое преимущество они
имеют пред нами?
отныне мы одинаковы.
И чего они достигнут этим, и что они увидят в вечности?
Ибо вот они также умерли и отныне не увидят света до века".
Я говорю вам, грешники: для вас достаточно есть, и пить, и обнажать
человека, и расхищать, и согрешать, и приобретать силу, и видеть хорошие дни.
Видели ли вы праведных, как конец их был мирен, ибо никакого рода
насилия не было в них по день их смерти, "И они погибли, как бы и не
существовали, и их души в печали сошли в царство мертвых".
И теперь я клянусь вам праведным Его великою славою и честью, и
Его достохвальным царством, и Его владычеством я клянусь вам: я знаю эту тайну
и прочитал ее на небесных скрижалях, и видел книгу святых, и нашел написанное и
отмеченное в ней относительно них, что для них уготовано всякое благо, и
радость и почесть; и я нашел записанное относительно духов тех, которые умерли
в правде; и узнали, что вам будет воздано многими благами за ваши труды, и ваша
участь лучше, чем участь живущих.
И будут жить ваши духи,~--- вы, умершие в правде; и будут радоваться и
ликовать их духи, и память о них будет пред лицом Великого на все роды мира:
так не страшитесь же их поношения!
Горе вам, грешники, когда вы умираете в своих грехах и подобные вам
говорят о вас: "блаженны грешники, они видели все свои дни; и теперь они умерли
в счастье и в богатстве, и не видели в своей жизни ни горести, ни убийства; в
славе они умерли, и во время их жизни суд не совершился над ними".
Но знаете ли вы, что души их должны сойти в царство мертвых, и они
найдут его невыносимым, и велика будет печаль их?
И во время великого суда ваш дух сойдет во мраке, и в сети, и в
плавающее пламя, и великий суд будет для всех родов до века: горе вам, ибо для
вас нет мира!
Не говорите праведным и добрым, которые еще живут: "в дни нашего
бедствия мы трудились, и побеждали всякую нужду, и встречались со всякими
бедствиями; мы не могли ничего сделать против врагов ни словом, ни делом, и
совершенно ничего не достигли; мы мучились и погибали, и не могли надеяться
видеть жизнь день за днем.
Мы надеялись быть главою, а сделались хвостом; мы измучились в
работах и не получили плодов своего труда, мы сделались пищею для грешников,
неправедные сделали для нас тяжким свое ярмо.
Владыками над нами были те, которые ненавидели нас и били нас: и мы
должны были склонять свои головы пред ненавидящими нас, и они не имели
сострадания к нам.
Мы старались ускользнуть от них, чтобы убежать и получить успокоение,
но мы не находили, куда бежать нам и спастись от них.
Мы жаловались на них в своей горести властителям, и сетовали на тех,
которые поедали нас; но они не взирали на наш вопль и не хотели слышать нашего
голоса.
И они помогали тем, которые обкрадывали нас и поедали, и тем, которые
принижали нас; и они утаивали их притеснения, так что не снимали с нас их ярма,
но поедали нас, и прогоняли, и убивали: и они утаивали умерщвление нас, и не
думали о том, что они подняли свои руки против нас".
Я клянусь вам, праведные, что ангелы на небе напоминают о вас
пред славою Великого к вашему благу, и ваши имена записаны пред славою
Великого.
Надейтесь вы, праведные, ибо прежде вы были в позоре, и несчастии, и
бедствии, а теперь вы будете светить, как светила небесные, и будете видимы, и
врата небесные отверзнутся для вас.
И ваш вопль о суде продолжается: он откроется для вас, ибо
властителям отомстится за ваше страдание, и всем помощникам тех, которые
обкрадывали вас.
Надейтесь и не покидайте свои надежды: ибо вы будете иметь великую
радость, как ангелы небесные, Так как вам предстоит таковое, то вы не будете
скрываться в день великого суда, и не будете найдены подобными грешникам, и от
вас далеко будет вечное осуждение, на все роды мира.
И теперь вы не бойтесь, праведные, когда видите грешников
усиливающимися и услаждающимися в своем веселие, и не имейте никакого общения с
ними, но держитесь в отдалении, ибо вы должны быть союзниками небесных воинств.
Вы грешники, хотя и говорите: "вы не можете разузнать этого и наши
грехи не записаны все", однако же они (ангелы) каждый день записывают ваши
грехи.
И теперь я открываю вам, что свет и мрак, день и ночь видят все ваши
грехи.
Не будьте нечестивыми в своем сердце, и не лгите, не изменяйте слов
праведности (или истины), и не выдавайте за ложь слов Святого и Великого, и не
прославляйте своих идолов; ибо вся ваша ложь и ваше нечестие служит не к
правде, а к великому греху.
И теперь я знаю эту тайну, что многие грешники изменят слова
праведности (или истины) и отпадут от них, и будут говорить двойные речи, и
говорить ложь, и творить великие (греховные) дела, и писать книги о своих
речах.
Но когда они все мои слова пишут правильно на своих языках, и ничего
не изменяют и не пропускают из моих слов, но все пишут правильно,~--- все, что я
прежде утверждал относительно них; то я знаю другую тайну, что именно только
праведным и мудрым даны книги к радости, и к праведности, и к великой мудрости,
и им даны книги, и они уверуют в них и возрадуются о них; и получат награду все
праведные, научившиеся из них знать все пути праведности.
"И в те дни, говорит Господь, они (праведные) должны воззвать к
сынам земли и представить свидетельство относительно мудрости их (книг);
покажите им их,~--- ибо вы их вожди,~--- и награды для всей земли.
Ибо Я и Мой Сын соединимся с ними навсегда и навечно на путях
праведности в их жизни.
И мир будет с вами: радуйтесь, вы~--- дети праведности, воистину"!
\vs 1En 20:1
И после некоторого времени мой сын Мафусаил взял своему сыну
Лемеху жену, и она зачала от него и родила сына.
Тело его было бело, как снег, и красно, как роза, и его волосы головные
и темянные были, как волна (руно), и его глаза были прекрасны; и когда он
открыл свои глаза, то они осветили весь дом подобно солнцу, так что весь дом
сделался необычайно светлым.
И как только он был взят из руки повивальной бабки, то открыл свои уста
и начал говорить к Господу правды.
И его отец Ламех устрашился этого, и удалился, и пришел к своему отцу
Мафусаилу.
И он сказал ему: "я родил необыкновенного сына; он не как человек, а
похож на детей небесных ангелов, ибо он родился иначе, нежели мы: его глаза
подобны лучам солнца и его лицо блестящее.
И мне кажется, что он происходит не от меня, а от ангелов; и я боюсь,
как бы в его дни не произошло на земле чудо.
И теперь, мой отец, я здесь с неотступною просьбою к тебе о том, чтобы
ты отправился к нашему отцу Еноху и выведал от него истину, ибо он имеет свое
жилище возле ангелов".
И когда Мафусаил слушал речь своего сына, то пришел ко мне к пределам
земли,~--- ибо он слышал, что я там,~--- и воскликнул; и я услышал его голос, и
пришел к нему, и сказал ему: "вот я здесь, мой сын, ибо ты пришел ко мне".
И он отвечал мне и сказал: "ради важного дела я пришел к тебе, и из-за
тревожного случая я приблизился сюда.
И теперь, отец мой, выслушай меня: у моего сына Ламеха родился сын,
образ и вид которого не как вид человека; его цвет белее, нежели снег, и
краснее розы, и его головные волосы белее, чем белое руно, и его глаза, как
лучи солнца; и он открыл свои глаза, и вот они осветили весь дом.
И взятый из руки повивальной бабки он открыл свои уста и прославил
Господа неба.
Тогда устрашился отец его Ламех и прибежал ко мне; и он не верит, что
он произошел от него, но что будто он подобие ангелов небесных; и вот я пришел
к тебе, чтобы ты открыл мне истину".
И я, Енох, отвечал и сказал ему: "Господь совершит на земле новое, и
это я знаю, и я видел в видении, и открыл тебе, что в век моего отца Иареда
некоторые ангелы, сошедшие с высоты неба преступили слово Господне.
И вот они совершили грех, и преступили закон, и соединились с женами,
и совершили с ними грех, и взяли жен из них, и родили с ними детей.
И великая погибель придет на всю землю, придет потоп, и будет великая
погибель в продолжение года.
Этот сын, родившийся у вас, останется на земле, и три его сына
спасутся вместе с ним; когда все люди живущие на земле, умрут, он и его сыновья
спасутся.
[Они рождают на земле исполинов не по духу, а по плоти, и за это
придет великое наказание на землю, земля будет вполне омыта от всей нечистоты].
И теперь извести сына своего Ламеха, что родившийся есть действительно
его сын, и нареки ему имя Ной, ибо он будет для вас остатком; и он и его
сыновья спасутся от уничтожения, которое придет на землю за все грехи и за
всякую неправду, которые совершаются на земле в его дни.
И после того неправда будет еще гораздо больше, чем та, которая
совершалась на земле прежде, ибо я знаю тайны святых,
так как Он~--- Господь~--- дозволил мне видеть их и открыл их мне, и я почитал их на скрижалях небесных.
И я видел написанное на них, что род за родом будет
беззаконовать, пока не восстанет род правды, и беззаконие будет обречено на
погибель, и грех исчезнет с земли, и все доброе появится на ней.
И теперь, мой сын, иди и возвести своему сыну Ламеху, что этот
родившийся сын есть действительно его сын, и это не ложь".
И когда Мафусаил выслушал речь своего отца Еноха,~--- ибо все тайные
вещи он открыл ему,~--- то возвратился, увидевшись с ним (Енохом), назад, и нарёк
тому сыну имя Ной, ибо он утешит землю в вознаграждение за всю погибель.
Другое писание, которое Енох написал для своего сына Мафусаила и
для всех, которые придут после него и будут сохранять закон в последние дни.
Вы, исполнившие его и теперь ожидающие, как в те дни совершится конец
над теми, которые делают злое, и сила беззаконников окончится,~--- вы ожидайте
только, когда минует грех, ибо имя их (грешников) будет изглажено из книг
святых и семя их погибнет навсегда и навечно, и их духи будут умерщвлены, и они
будут восклицать и взывать в пустом необитаемом месте и гореть в огне, где нет
земли.
И я видел там нечто похожее на облако, чего нельзя было узнать, ибо
вследствие глубины его (этого места) я не мог взглянуть на него; и я увидел там
ярко~--- пылающее пламя огня, и там кружились предметы, как блестящие горы, и
двигались туда и сюда.
И я спросил одного из святых ангелов, бывших при мне, и сказал ему:
"что это такое блестящее?
ибо это не небо, а только пламя пылающего огня и звуки вопля, и плача,
и сетования, и жестокого страдания".
28 И он сказал мне: "в это место, которое ты видишь,~--- сюда приносятся
духи грешников, и хулителей, и тех, которые делают злое и изменяют всё, что Бог
сказал устами пророков о будущем.
Ибо об этом есть писания и начертания вверху на небе, чтобы ангелы
читали их и знали, что случится с грешниками и духами покорных и тех, которые
умерщвляли свою плоть и за это получили от Бога награду, и тех, которые были
обесчещены злыми людьми; которые любили Бога, не любили ни золота, ни серебра,
ни всех благ мира, но предавали свое тело мучению; и которые, со времени своего
бытия, домогались не земных явлений, а считали самих себя за преходящее дыхание
и сообразно с этим жили, и были многократно испытываемы Господом, но их души
были обретены в чистоте, чтобы прославить Его имя.
Все благословения, которые они получают, я представил в книгах; и Он
назначал им за это награду, ибо они обрелись возлюбившими более вечное небо,
чем свою жизнь, и в то время, как были попираемы злыми людьми, и должны были
выслушывать от них оскорбления и хуления, и были обесчещиваемы, они прославили
Меня".
И теперь Я призову духов добрых людей из поколения света, и произведу
перемену с теми, которые родились во тьме и которые в своей плоти не были
награждены почестью, как надлежало за их верность.
И Я введу в блистающий свет любивших Мое святое имя, и посажу каждого
из них отдельно на престоле почести,~--- его почести.
И они будут блистать в продолжение бесчисленных времен, ибо
справедливость есть суд Божий и верным Он даст верность в жилище праведных
путей.
И они (праведные) увидят, как родившиеся во тьме будут брошены во
тьму, между тем как праведные будут блистать.
И грешники воскликнут и увидят, как они блистают: и они также пойдут
туда, где им написаны дни и времена.

\bibbookdescr{Job}{
  inline={\LARGE Книга\\\Huge Иова},
  toc={Иов},
  bookmark={Иов},
  header={Иов},
  %headerleft={},
  %headerright={},
  abbr={Иов}
}
\vs Job 1:1 Был человек в земле Уц, имя его Иов; и был человек этот непорочен, справедлив и богобоязнен и удалялся от зла.
\vs Job 1:2 И родились у него семь сыновей и три дочери.
\vs Job 1:3 Имения у него было: семь тысяч мелкого скота, три тысячи верблюдов, пятьсот пар волов и пятьсот ослиц и весьма много прислуги; и был человек этот знаменитее всех сынов Востока.
\vs Job 1:4 Сыновья его сходились, делая пиры каждый в своем доме в свой день, и посылали и приглашали трех сестер своих есть и пить с ними.
\vs Job 1:5 Когда круг пиршественных дней совершался, Иов посылал \bibemph{за ними} и освящал их и, вставая рано утром, возносил всесожжения по числу всех их [и одного тельца за грех о душах их]. Ибо говорил Иов: может быть, сыновья мои согрешили и похулили Бога в сердце своем. Так делал Иов во все \bibemph{такие} дни.
\rsbpar\vs Job 1:6 И был день, когда пришли сыны Божии предстать пред Господа; между ними пришел и сатана.
\vs Job 1:7 И сказал Господь сатане: откуда ты пришел? И отвечал сатана Господу и сказал: я ходил по земле и обошел ее.
\vs Job 1:8 И сказал Господь сатане: обратил ли ты внимание твое на раба Моего Иова? ибо нет такого, как он, на земле: человек непорочный, справедливый, богобоязненный и удаляющийся от зла.
\vs Job 1:9 И отвечал сатана Господу и сказал: разве даром богобоязнен Иов?
\vs Job 1:10 Не Ты ли кругом оградил его и дом его и все, что у него? Дело рук его Ты благословил, и стада его распространяются по земле;
\vs Job 1:11 но простри руку Твою и коснись всего, что у него,~--- благословит ли он Тебя?
\vs Job 1:12 И сказал Господь сатане: вот, все, что у него, в руке твоей; только на него не простирай руки твоей. И отошел сатана от лица Господня.
\rsbpar\vs Job 1:13 И был день, когда сыновья его и дочери его ели и вино пили в доме первородного брата своего.
\vs Job 1:14 И \bibemph{вот}, приходит вестник к Иову и говорит:
\vs Job 1:15 волы орали, и ослицы паслись подле них, как напали Савеяне и взяли их, а отроков поразили острием меча; и спасся только я один, чтобы возвестить тебе.
\vs Job 1:16 Еще он говорил, как приходит другой и сказывает: огонь Божий упал с неба и опалил овец и отроков и пожрал их; и спасся только я один, чтобы возвестить тебе.
\vs Job 1:17 Еще он говорил, как приходит другой и сказывает: Халдеи расположились тремя отрядами и бросились на верблюдов и взяли их, а отроков поразили острием меча; и спасся только я один, чтобы возвестить тебе.
\vs Job 1:18 Еще этот говорил, приходит другой и сказывает: сыновья твои и дочери твои ели и вино пили в доме первородного брата своего;
\vs Job 1:19 и вот, большой ветер пришел от пустыни и охватил четыре угла дома, и дом упал на отроков, и они умерли; и спасся только я один, чтобы возвестить тебе.
\rsbpar\vs Job 1:20 Тогда Иов встал и разодрал верхнюю одежду свою, остриг голову свою и пал на землю и поклонился
\vs Job 1:21 и сказал: наг я вышел из чрева матери моей, наг и возвращусь. Господь дал, Господь и взял; [как угодно было Господу, так и сделалось;] да будет имя Господне благословенно!
\vs Job 1:22 Во всем этом не согрешил Иов и не произнес ничего неразумного о Боге.
\vs Job 2:1 Был день, когда пришли сыны Божии предстать пред Господа; между ними пришел и сатана предстать пред Господа.
\vs Job 2:2 И сказал Господь сатане: откуда ты пришел? И отвечал сатана Господу и сказал: я ходил по земле и обошел ее.
\vs Job 2:3 И сказал Господь сатане: обратил ли ты внимание твое на раба Моего Иова? ибо нет такого, как он, на земле: человек непорочный, справедливый, богобоязненный и удаляющийся от зла, и доселе тверд в своей непорочности; а ты возбуждал Меня против него, чтобы погубить его безвинно.
\vs Job 2:4 И отвечал сатана Господу и сказал: кожу за кожу, а за жизнь свою отдаст человек все, что есть у него;
\vs Job 2:5 но простри руку Твою и коснись кости его и плоти его,~--- благословит ли он Тебя?
\vs Job 2:6 И сказал Господь сатане: вот, он в руке твоей, только душу его сбереги.
\rsbpar\vs Job 2:7 И отошел сатана от лица Господня и поразил Иова проказою лютою от подошвы ноги его по самое темя его.
\vs Job 2:8 И взял он себе черепицу, чтобы скоблить себя ею, и сел в пепел [вне селения].
\vs Job 2:9 И сказала ему жена его: ты все еще тверд в непорочности твоей! похули Бога и умри.\fns{Этот стих по переводу 70-ти: По многом времени сказала ему жена его: доколе ты будешь терпеть? Вот, подожду еще немного в надежде спасения моего. Ибо погибли с земли память твоя, сыновья и дочери, болезни чрева моего и труды, которыми напрасно трудилась. Сам ты сидишь в смраде червей, проводя ночь без покрова, а я скитаюсь и служу, перехожу с места на место, из дома в дом, ожидая, когда зайдет солнце, чтобы успокоиться от трудов моих и болезней, которые ныне удручают меня. Но скажи некое слово к Богу и умри.}
\vs Job 2:10 Но он сказал ей: ты говоришь как одна из безумных: неужели доброе мы будем принимать от Бога, а злого не будем принимать? Во всем этом не согрешил Иов устами своими.
\vs Job 2:11 И услышали трое друзей Иова о всех этих несчастьях, постигших его, и пошли каждый из своего места: Елифаз Феманитянин, Вилдад Савхеянин и Софар Наамитянин, и сошлись, чтобы идти вместе сетовать с ним и утешать его.
\vs Job 2:12 И подняв глаза свои издали, они не узнали его; и возвысили голос свой и зарыдали; и разодрал каждый верхнюю одежду свою, и бросали пыль над головами своими к небу.
\vs Job 2:13 И сидели с ним на земле семь дней и семь ночей; и никто не говорил ему ни слова, ибо видели, что страдание его весьма велико.
\vs Job 3:1 После того открыл Иов уста свои и проклял день свой.
\vs Job 3:2 И начал Иов и сказал:
\vs Job 3:3 погибни день, в который я родился, и ночь, в которую сказано: зачался человек!
\vs Job 3:4 День тот да будет тьмою; да не взыщет его Бог свыше, и да не воссияет над ним свет!
\vs Job 3:5 Да омрачит его тьма и тень смертная, да обложит его туча, да страшатся его, как палящего зноя!
\vs Job 3:6 Ночь та,~--- да обладает ею мрак, да не сочтется она в днях года, да не войдет в число месяцев!
\vs Job 3:7 О! ночь та~--- да будет она безлюдна; да не войдет в нее веселье!
\vs Job 3:8 Да проклянут ее проклинающие день, способные разбудить левиафана!
\vs Job 3:9 Да померкнут звезды рассвета ее: пусть ждет она света, и он не приходит, и да не увидит она ресниц денницы
\vs Job 3:10 за то, что не затворила дверей чрева \bibemph{матери} моей и не сокрыла горести от очей моих!
\vs Job 3:11 Для чего не умер я, выходя из утробы, и не скончался, когда вышел из чрева?
\vs Job 3:12 Зачем приняли меня колени? зачем было мне сосать сосцы?
\vs Job 3:13 Теперь бы лежал я и почивал; спал бы, и мне было бы покойно
\vs Job 3:14 с царями и советниками земли, которые застраивали для себя пустыни,
\vs Job 3:15 или с князьями, у которых было золото, и которые наполняли домы свои серебром;
\vs Job 3:16 или, как выкидыш сокрытый, я не существовал бы, как младенцы, не увидевшие света.
\vs Job 3:17 Там беззаконные перестают наводить страх, и там отдыхают истощившиеся в силах.
\vs Job 3:18 Там узники вместе наслаждаются покоем и не слышат криков приставника.
\vs Job 3:19 Малый и великий там равны, и раб свободен от господина своего.
\vs Job 3:20 На что дан страдальцу свет, и жизнь огорченным душею,
\vs Job 3:21 которые ждут смерти, и нет ее, которые вырыли бы ее охотнее, нежели клад,
\vs Job 3:22 обрадовались бы до восторга, восхитились бы, что нашли гроб?
\vs Job 3:23 \bibemph{На что дан свет} человеку, которого путь закрыт, и которого Бог окружил мраком?
\vs Job 3:24 Вздохи мои предупреждают хлеб мой, и стоны мои льются, как вода,
\vs Job 3:25 ибо ужасное, чего я ужасался, то и постигло меня; и чего я боялся, то и пришло ко мне.
\vs Job 3:26 Нет мне мира, нет покоя, нет отрады: постигло несчастье.
\vs Job 4:1 И отвечал Елифаз Феманитянин и сказал:
\vs Job 4:2 \bibemph{если} попытаемся мы \bibemph{сказать} к тебе слово,~--- не тяжело ли будет тебе? Впрочем кто может возбранить слову!
\vs Job 4:3 Вот, ты наставлял многих и опустившиеся руки поддерживал,
\vs Job 4:4 падающего восставляли слова твои, и гнущиеся колени ты укреплял.
\vs Job 4:5 А теперь дошло до тебя, и ты изнемог; коснулось тебя, и ты упал духом.
\vs Job 4:6 Богобоязненность твоя не должна ли быть твоею надеждою, и непорочность путей твоих~--- упованием твоим?
\vs Job 4:7 Вспомни же, погибал ли кто невинный, и где праведные бывали искореняемы?
\vs Job 4:8 Как я видал, то оравшие нечестие и сеявшие зло пожинают его;
\vs Job 4:9 от дуновения Божия погибают и от духа гнева Его исчезают.
\vs Job 4:10 Рев льва и голос рыкающего \bibemph{умолкает}, и зубы скимнов сокрушаются;
\vs Job 4:11 могучий лев погибает без добычи, и дети львицы рассеиваются.
\vs Job 4:12 И вот, ко мне тайно принеслось слово, и ухо мое приняло нечто от него.
\vs Job 4:13 Среди размышлений о ночных видениях, когда сон находит на людей,
\vs Job 4:14 объял меня ужас и трепет и потряс все кости мои.
\vs Job 4:15 И дух прошел надо мною; дыбом стали волосы на мне.
\vs Job 4:16 Он стал,~--- но я не распознал вида его,~--- только облик был пред глазами моими; тихое веяние,~--- и я слышу голос:
\vs Job 4:17 человек праведнее ли Бога? и муж чище ли Творца своего?
\vs Job 4:18 Вот, Он и слугам Своим не доверяет и в Ангелах Своих усматривает недостатки:
\vs Job 4:19 тем более~--- в обитающих в храминах из брения, которых основание прах, которые истребляются скорее моли.
\vs Job 4:20 Между утром и вечером они распадаются; не увидишь, как они вовсе исчезнут.
\vs Job 4:21 Не погибают ли с ними и достоинства их? Они умирают, не достигнув мудрости.
\vs Job 5:1 Взывай, если есть отвечающий тебе. И к кому из святых обратишься ты?
\vs Job 5:2 Так, глупца убивает гневливость, и несмысленного губит раздражительность.
\vs Job 5:3 Видел я, как глупец укореняется, и тотчас проклял дом его.
\vs Job 5:4 Дети его далеки от счастья, их будут бить у ворот, и не будет заступника.
\vs Job 5:5 Жатву его съест голодный и из-за терна возьмет ее, и жаждущие поглотят имущество его.
\vs Job 5:6 Так, не из праха выходит горе, и не из земли вырастает беда;
\vs Job 5:7 но человек рождается на страдание, \bibemph{как} искры, чтобы устремляться вверх.
\vs Job 5:8 Но я к Богу обратился бы, предал бы дело мое Богу,
\vs Job 5:9 Который творит дела великие и неисследимые, чудные без числа,
\vs Job 5:10 дает дождь на лице земли и посылает воды на лице полей;
\vs Job 5:11 униженных поставляет на высоту, и сетующие возносятся во спасение.
\vs Job 5:12 Он разрушает замыслы коварных, и руки их не довершают предприятия.
\vs Job 5:13 Он уловляет мудрецов их же лукавством, и совет хитрых становится тщетным:
\vs Job 5:14 днем они встречают тьму и в полдень ходят ощупью, как ночью.
\vs Job 5:15 Он спасает бедного от меча, от уст их и от руки сильного.
\vs Job 5:16 И есть несчастному надежда, и неправда затворяет уста свои.
\vs Job 5:17 Блажен человек, которого вразумляет Бог, и потому наказания Вседержителева не отвергай,
\vs Job 5:18 ибо Он причиняет раны и Сам обвязывает их; Он поражает, и Его же руки врачуют.
\vs Job 5:19 В шести бедах спасет тебя, и в седьмой не коснется тебя зло.
\vs Job 5:20 Во время голода избавит тебя от смерти, и на войне~--- от руки меча.
\vs Job 5:21 От бича языка укроешь себя и не убоишься опустошения, когда оно придет.
\vs Job 5:22 Опустошению и голоду посмеешься и зверей земли не убоишься,
\vs Job 5:23 ибо с камнями полевыми у тебя союз, и звери полевые в мире с тобою.
\vs Job 5:24 И узн\acc{а}ешь, что шатер твой в безопасности, и будешь смотреть за домом твоим, и не согрешишь.
\vs Job 5:25 И увидишь, что семя твое многочисленно, и отрасли твои, как трава на земле.
\vs Job 5:26 Войдешь во гроб в зрелости, как укладываются снопы пшеницы в свое время.
\vs Job 5:27 Вот, что мы дознали; так оно и есть; выслушай это и заметь для себя.
\vs Job 6:1 И отвечал Иов и сказал:
\vs Job 6:2 о, если бы верно взвешены были вопли мои, и вместе с ними положили на весы страдание мое!
\vs Job 6:3 Оно верно перетянуло бы песок морей! Оттого слова мои неистовы.
\vs Job 6:4 Ибо стрелы Вседержителя во мне; яд их пьет дух мой; ужасы Божии ополчились против меня.
\vs Job 6:5 Ревет ли дикий осел на траве? мычит ли бык у месива своего?
\vs Job 6:6 Едят ли безвкусное без соли, и есть ли вкус в яичном белке?
\vs Job 6:7 До чего не хотела коснуться душа моя, то составляет отвратительную пищу мою.
\vs Job 6:8 О, когда бы сбылось желание мое и чаяние мое исполнил Бог!
\vs Job 6:9 О, если бы благоволил Бог сокрушить меня, простер руку Свою и сразил меня!
\vs Job 6:10 Это было бы еще отрадою мне, и я крепился бы в моей беспощадной болезни, ибо я не отвергся изречений Святаго.
\vs Job 6:11 Что за сила у меня, чтобы надеяться мне? и какой конец, чтобы длить мне жизнь мою?
\vs Job 6:12 Твердость ли камней твердость моя? и медь ли плоть моя?
\vs Job 6:13 Есть ли во мне помощь для меня, и есть ли для меня какая опора?
\vs Job 6:14 К страждущему должно быть сожаление от друга его, если только он не оставил страха к Вседержителю.
\vs Job 6:15 Но братья мои неверны, как поток, как быстро текущие ручьи,
\vs Job 6:16 которые черны от льда и в которых скрывается снег.
\vs Job 6:17 Когда становится тепло, они умаляются, а во время жары исчезают с мест своих.
\vs Job 6:18 Уклоняют они направление путей своих, заходят в пустыню и теряются;
\vs Job 6:19 смотрят на них дороги Фемайские, надеются на них пути Савейские,
\vs Job 6:20 но остаются пристыженными в своей надежде; приходят туда и от стыда краснеют.
\vs Job 6:21 Так и вы теперь ничто: увидели страшное и испугались.
\vs Job 6:22 Говорил ли я: дайте мне, или от достатка вашего заплатите за меня;
\vs Job 6:23 и избавьте меня от руки врага, и от руки мучителей выкупите меня?
\vs Job 6:24 Науч\acc{и}те меня, и я замолчу; укажите, в чем я погрешил.
\vs Job 6:25 Как сильны слова правды! Но что доказывают обличения ваши?
\vs Job 6:26 Вы придумываете речи для обличения? На ветер пускаете слова ваши.
\vs Job 6:27 Вы нападаете на сироту и роете яму другу вашему.
\vs Job 6:28 Но прошу вас, взгляните на меня; буду ли я говорить ложь пред лицем вашим?
\vs Job 6:29 Пересмотрите, есть ли неправда? пересмотрите,~--- правда моя.
\vs Job 6:30 Есть ли на языке моем неправда? Неужели гортань моя не может различить горечи?
\vs Job 7:1 Не определено ли человеку время на земле, и дни его не то же ли, что дни наемника?
\vs Job 7:2 Как раб жаждет тени, и как наемник ждет окончания работы своей,
\vs Job 7:3 так я получил в удел месяцы суетные, и ночи горестные отчислены мне.
\vs Job 7:4 Когда ложусь, то говорю: <<когда-то встану?>>, а вечер длится, и я ворочаюсь досыта до самого рассвета.
\vs Job 7:5 Тело мое одето червями и пыльными струпами; кожа моя лопается и гноится.
\vs Job 7:6 Дни мои бегут скорее челнока и кончаются без надежды.
\vs Job 7:7 Вспомни, что жизнь моя дуновение, что око мое не возвратится видеть доброе.
\vs Job 7:8 Не увидит меня око видевшего меня; очи Твои на меня,~--- и нет меня.
\vs Job 7:9 Редеет облако и уходит; так нисшедший в преисподнюю не выйдет,
\vs Job 7:10 не возвратится более в дом свой, и место его не будет уже знать его.
\vs Job 7:11 Не буду же я удерживать уст моих; буду говорить в стеснении духа моего; буду жаловаться в горести души моей.
\vs Job 7:12 Разве я море или морское чудовище, что Ты поставил надо мною стражу?
\vs Job 7:13 Когда подумаю: утешит меня постель моя, унесет горесть мою ложе мое,
\vs Job 7:14 Ты страшишь меня снами и видениями пугаешь меня;
\vs Job 7:15 и душа моя желает лучше прекращения дыхания, лучше смерти, нежели \bibemph{сбережения} костей моих.
\vs Job 7:16 Опротивела мне жизнь. Не вечно жить мне. Отступи от меня, ибо дни мои суета.
\vs Job 7:17 Что такое человек, что Ты столько ценишь его и обращаешь на него внимание Твое,
\vs Job 7:18 посещаешь его каждое утро, каждое мгновение испытываешь его?
\vs Job 7:19 Доколе же Ты не оставишь, доколе не отойдешь от меня, доколе не дашь мне проглотить слюну мою?
\vs Job 7:20 Если я согрешил, то что я сделаю Тебе, страж человеков! Зачем Ты поставил меня противником Себе, так что я стал самому себе в тягость?
\vs Job 7:21 И зачем бы не простить мне греха и не снять с меня беззакония моего? ибо, вот, я лягу в прахе; завтра поищешь меня, и меня нет.
\vs Job 8:1 И отвечал Вилдад Савхеянин и сказал:
\vs Job 8:2 долго ли ты будешь говорить так?~--- слов\acc{а} уст твоих бурный ветер!
\vs Job 8:3 Неужели Бог извращает суд, и Вседержитель превращает правду?
\vs Job 8:4 Если сыновья твои согрешили пред Ним, то Он и предал их в руку беззакония их.
\vs Job 8:5 Если же ты взыщешь Бога и помолишься Вседержителю,
\vs Job 8:6 и если ты чист и прав, то Он ныне же встанет над тобою и умиротворит жилище правды твоей.
\vs Job 8:7 И если вначале у тебя было мало, то впоследствии будет весьма много.
\vs Job 8:8 Ибо спроси у прежних родов и вникни в наблюдения отцов их;
\vs Job 8:9 а мы~--- вчерашние и ничего не знаем, потому что наши дни на земле тень.
\vs Job 8:10 Вот, они научат тебя, скажут тебе и от сердца своего произнесут слова:
\vs Job 8:11 поднимается ли тростник без влаги? растет ли камыш без воды?
\vs Job 8:12 Еще он в свежести своей и не срезан, а прежде всякой травы засыхает.
\vs Job 8:13 Таковы пути всех забывающих Бога, и надежда лицемера погибнет;
\vs Job 8:14 упование его подсечено, и уверенность его~--- дом паука.
\vs Job 8:15 Обопрется о дом свой и не устоит; ухватится за него и не удержится.
\vs Job 8:16 Зеленеет он пред солнцем, за сад простираются ветви его;
\vs Job 8:17 в кучу \bibemph{камней} вплетаются корни его, между камнями врезываются.
\vs Job 8:18 Но когда вырвут его с места его, оно откажется от него: <<я не видало тебя!>>
\vs Job 8:19 Вот радость пути его! а из земли вырастают другие.
\vs Job 8:20 Видишь, Бог не отвергает непорочного и не поддерживает рук\acc{и} злодеев.
\vs Job 8:21 Он еще наполнит смехом уста твои и губы твои радостным восклицанием.
\vs Job 8:22 Ненавидящие тебя облекутся в стыд, и шатра нечестивых не станет.
\vs Job 9:1 И отвечал Иов и сказал:
\vs Job 9:2 правда! знаю, что так; но как оправдается человек пред Богом?
\vs Job 9:3 Если захочет вступить в прение с Ним, то не ответит Ему ни на одно из тысячи.
\vs Job 9:4 Премудр сердцем и могущ силою; кто восставал против Него и оставался в покое?
\vs Job 9:5 Он передвигает горы, и не узна\acc{ю}т их: Он превращает их в гневе Своем;
\vs Job 9:6 сдвигает землю с места ее, и столбы ее дрожат;
\vs Job 9:7 скажет солнцу,~--- и не взойдет, и на звезды налагает печать.
\vs Job 9:8 Он один распростирает небеса и ходит по высотам моря;
\vs Job 9:9 сотворил Ас, Кесиль и Хима\fns{Созвездия, соответствующие нынешним названиям: Медведицы, Ориона и Плеяд.} и тайники юга;
\vs Job 9:10 делает великое, неисследимое и чудное без числа!
\vs Job 9:11 Вот, Он пройдет предо мною, и не увижу Его; пронесется, и не замечу Его.
\vs Job 9:12 Возьмет, и кто возбранит Ему? кто скажет Ему: что Ты делаешь?
\vs Job 9:13 Бог не отвратит гнева Своего; пред Ним падут поборники гордыни.
\vs Job 9:14 Тем более могу ли я отвечать Ему и приискивать себе слова пред Ним?
\vs Job 9:15 Хотя бы я и прав был, но не буду отвечать, а буду умолять Судию моего.
\vs Job 9:16 Если бы я воззвал, и Он ответил мне,~--- я не поверил бы, что голос мой услышал Тот,
\vs Job 9:17 Кто в вихре разит меня и умножает безвинно мои раны,
\vs Job 9:18 не дает мне перевести духа, но пресыщает меня горестями.
\vs Job 9:19 Если \bibemph{действовать} силою, то Он могуществен; если судом, кто сведет меня с Ним?
\vs Job 9:20 Если я буду оправдываться, то мои же уста обвинят меня; \bibemph{если} я невинен, то Он призн\acc{а}ет меня виновным.
\vs Job 9:21 Невинен я; не хочу знать души моей, презираю жизнь мою.
\vs Job 9:22 Все одно; поэтому я сказал, что Он губит и непорочного и виновного.
\vs Job 9:23 Если этого поражает Он бичом вдруг, то пытке невинных посмевается.
\vs Job 9:24 Земля отдана в руки нечестивых; лица судей ее Он закрывает. Если не Он, то кто же?
\vs Job 9:25 Дни мои быстрее гонца,~--- бегут, не видят добра,
\vs Job 9:26 несутся, как легкие ладьи, как орел стремится на добычу.
\vs Job 9:27 Если сказать мне: забуду я жалобы мои, отложу мрачный вид свой и ободрюсь;
\vs Job 9:28 то трепещу всех страданий моих, зная, что Ты не объявишь меня невинным.
\vs Job 9:29 Если же я виновен, то для чего напрасно томлюсь?
\vs Job 9:30 Хотя бы я омылся и снежною водою и совершенно очистил руки мои,
\vs Job 9:31 то и тогда Ты погрузишь меня в грязь, и возгнушаются мною одежды мои.
\vs Job 9:32 Ибо Он не человек, как я, чтоб я мог отвечать Ему и идти вместе с Ним на суд!
\vs Job 9:33 Нет между нами посредника, который положил бы руку свою на обоих нас.
\vs Job 9:34 Да отстранит Он от меня жезл Свой, и страх Его да не ужасает меня,~---
\vs Job 9:35 и тогда я буду говорить и не убоюсь Его, ибо я не таков сам в себе.
\vs Job 10:1 Опротивела душе моей жизнь моя; предамся печали моей; буду говорить в горести души моей.
\vs Job 10:2 Скажу Богу: не обвиняй меня; объяви мне, за что Ты со мною борешься?
\vs Job 10:3 Хорошо ли для Тебя, что Ты угнетаешь, что презираешь дело рук Твоих, а на совет нечестивых посылаешь свет?
\vs Job 10:4 Разве у Тебя плотские очи, и Ты смотришь, как смотрит человек?
\vs Job 10:5 Разве дни Твои, как дни человека, или лета Твои, как дни мужа,
\vs Job 10:6 что Ты ищешь порока во мне и допытываешься греха во мне,
\vs Job 10:7 хотя знаешь, что я не беззаконник, и что некому избавить меня от руки Твоей?
\vs Job 10:8 Твои руки трудились надо мною и образовали всего меня кругом,~--- и Ты губишь меня?
\vs Job 10:9 Вспомни, что Ты, как глину, обделал меня, и в прах обращаешь меня?
\vs Job 10:10 Не Ты ли вылил меня, как молоко, и, как творог, сгустил меня,
\vs Job 10:11 кожею и плотью одел меня, костями и жилами скрепил меня,
\vs Job 10:12 жизнь и милость даровал мне, и попечение Твое хранило дух мой?
\vs Job 10:13 Но и то скрывал Ты в сердце Своем,~--- знаю, что это было у Тебя,~---
\vs Job 10:14 что если я согрешу, Ты заметишь и не оставишь греха моего без наказания.
\vs Job 10:15 Если я виновен, горе мне! если и прав, то не осмелюсь поднять головы моей. Я пресыщен унижением; взгляни на бедствие мое:
\vs Job 10:16 оно увеличивается. Ты гонишься за мною, как лев, и снова нападаешь на меня и чудным являешься во мне.
\vs Job 10:17 Выводишь новых свидетелей Твоих против меня; усиливаешь гнев Твой на меня; и беды, одни за другими, ополчаются против меня.
\vs Job 10:18 И зачем Ты вывел меня из чрева? пусть бы я умер, когда еще ничей глаз не видел меня;
\vs Job 10:19 пусть бы я, как небывший, из чрева перенесен был во гроб!
\vs Job 10:20 Не малы ли дни мои? Оставь, отступи от меня, чтобы я немного ободрился,
\vs Job 10:21 прежде нежели отойду,~--- и уже не возвращусь,~--- в страну тьмы и сени смертной,
\vs Job 10:22 в страну мрака, каков есть мрак тени смертной, где нет устройства, \bibemph{где} темно, как самая тьма.
\vs Job 11:1 И отвечал Софар Наамитянин и сказал:
\vs Job 11:2 разве на множество слов нельзя дать ответа, и разве человек многоречивый прав?
\vs Job 11:3 Пустословие твое заставит ли молчать мужей, чтобы ты глумился, и некому было постыдить тебя?
\vs Job 11:4 Ты сказал: суждение мое верно, и чист я в очах Твоих.
\vs Job 11:5 Но если бы Бог возглаголал и отверз уста Свои к тебе
\vs Job 11:6 и открыл тебе тайны премудрости, что тебе вдвое больше следовало бы понести! Итак знай, что Бог для тебя некоторые из беззаконий твоих предал забвению.
\vs Job 11:7 Можешь ли ты исследованием найти Бога? Можешь ли совершенно постигнуть Вседержителя?
\vs Job 11:8 Он превыше небес,~--- что можешь сделать? глубже преисподней,~--- что можешь узнать?
\vs Job 11:9 Длиннее земли мера Его и шире моря.
\vs Job 11:10 Если Он пройдет и заключит кого в оковы и представит на суд, то кто отклонит Его?
\vs Job 11:11 Ибо Он знает людей лживых и видит беззаконие, и оставит ли его без внимания?
\vs Job 11:12 Но пустой человек мудрствует, хотя человек рождается подобно дикому осленку.
\vs Job 11:13 Если ты управишь сердце твое и прострешь к Нему руки твои,
\vs Job 11:14 и если есть порок в руке твоей, а ты удалишь его и не дашь беззаконию обитать в шатрах твоих,
\vs Job 11:15 то поднимешь незапятнанное лице твое и будешь тверд и не будешь бояться.
\vs Job 11:16 Тогда забудешь горе: как о воде протекшей, будешь вспоминать о нем.
\vs Job 11:17 И яснее полдня пойдет жизнь твоя; просветлеешь, как утро.
\vs Job 11:18 И будешь спокоен, ибо есть надежда; ты огражден, и можешь спать безопасно.
\vs Job 11:19 Будешь лежать, и не будет устрашающего, и многие будут заискивать у тебя.
\vs Job 11:20 А глаза беззаконных истают, и убежище пропадет у них, и надежда их исчезнет.
\vs Job 12:1 И отвечал Иов и сказал:
\vs Job 12:2 подлинно, \bibemph{только} вы люди, и с вами умрет мудрость!
\vs Job 12:3 И у меня \bibemph{есть} сердце, как у вас; не ниже я вас; и кто не знает того же?
\vs Job 12:4 Посмешищем стал я для друга своего, я, который взывал к Богу, и которому Он отвечал, посмешищем~--- \bibemph{человек} праведный, непорочный.
\vs Job 12:5 Так презрен по мыслям сидящего в покое факел, приготовленный для спотыкающихся ногами.
\vs Job 12:6 Покойны шатры у грабителей и безопасны у раздражающих Бога, которые как бы Бога носят в руках своих.
\vs Job 12:7 И подлинно: спроси у скота, и научит тебя, у птицы небесной, и возвестит тебе;
\vs Job 12:8 или побеседуй с землею, и наставит тебя, и скажут тебе рыбы морские.
\vs Job 12:9 Кто во всем этом не узнает, что рука Господа сотворила сие?
\vs Job 12:10 В Его руке душа всего живущего и дух всякой человеческой плоти.
\vs Job 12:11 Не ухо ли разбирает слова, и не язык ли распознает вкус пищи?
\vs Job 12:12 В старцах~--- мудрость, и в долголетних~--- разум.
\vs Job 12:13 У Него премудрость и сила; Его совет и разум.
\vs Job 12:14 Что Он разрушит, то не построится; кого Он заключит, тот не высвободится.
\vs Job 12:15 Остановит воды, и все высохнет; пустит их, и превратят землю.
\vs Job 12:16 У Него могущество и премудрость, пред Ним заблуждающийся и вводящий в заблуждение.
\vs Job 12:17 Он приводит советников в необдуманность и судей делает глупыми.
\vs Job 12:18 Он лишает перевязей царей и поясом обвязывает чресла их;
\vs Job 12:19 князей лишает достоинства и низвергает храбрых;
\vs Job 12:20 отнимает язык у велеречивых и старцев лишает смысла;
\vs Job 12:21 покрывает стыдом знаменитых и силу могучих ослабляет;
\vs Job 12:22 открывает глубокое из среды тьмы и выводит на свет тень смертную;
\vs Job 12:23 умножает народы и истребляет их; рассевает народы и собирает их;
\vs Job 12:24 отнимает ум у глав народа земли и оставляет их блуждать в пустыне, где нет пути:
\vs Job 12:25 ощупью ходят они во тьме без света и шатаются, как пьяные.
\vs Job 13:1 Вот, все \bibemph{это} видело око мое, слышало ухо мое и заметило для себя.
\vs Job 13:2 Сколько знаете вы, знаю и я: не ниже я вас.
\vs Job 13:3 Но я к Вседержителю хотел бы говорить и желал бы состязаться с Богом.
\vs Job 13:4 А вы сплетчики лжи; все вы бесполезные врачи.
\vs Job 13:5 О, если бы вы только молчали! это было бы \bibemph{вменено} вам в мудрость.
\vs Job 13:6 Выслушайте же рассуждения мои и вникните в возражение уст моих.
\vs Job 13:7 Надлежало ли вам ради Бога говорить неправду и для Него говорить ложь?
\vs Job 13:8 Надлежало ли вам быть лицеприятными к Нему и за Бога так препираться?
\vs Job 13:9 Хорошо ли будет, когда Он испытает вас? Обманете ли Его, как обманывают человека?
\vs Job 13:10 Строго накажет Он вас, хотя вы и скрытно лицемерите.
\vs Job 13:11 Неужели величие Его не устрашает вас, и страх Его не нападает на вас?
\vs Job 13:12 Напоминания ваши подобны пеплу; оплоты ваши~--- оплоты глиняные.
\vs Job 13:13 Замолчите предо мною, и я буду говорить, что бы ни постигло меня.
\vs Job 13:14 Для чего мне терзать тело мое зубами моими и душу мою полагать в руку мою?
\vs Job 13:15 Вот, Он убивает меня, но я буду надеяться; я желал бы только отстоять пути мои пред лицем Его!
\vs Job 13:16 И это уже в оправдание мне, потому что лицемер не пойдет пред лице Его!
\vs Job 13:17 Выслушайте внимательно слово мое и объяснение мое ушами вашими.
\vs Job 13:18 Вот, я завел судебное дело: знаю, что буду прав.
\vs Job 13:19 Кто в состоянии оспорить меня? Ибо я скоро умолкну и испущу дух.
\vs Job 13:20 Двух только \bibemph{вещей} не делай со мною, и тогда я не буду укрываться от лица Твоего:
\vs Job 13:21 удали от меня руку Твою, и ужас Твой да не потрясает меня.
\vs Job 13:22 Тогда зови, и я буду отвечать, или буду говорить я, а Ты отвечай мне.
\vs Job 13:23 Сколько у меня пороков и грехов? покажи мне беззаконие мое и грех мой.
\vs Job 13:24 Для чего скрываешь лице Твое и считаешь меня врагом Тебе?
\vs Job 13:25 Не сорванный ли листок Ты сокрушаешь и не сухую ли соломинку преследуешь?
\vs Job 13:26 Ибо Ты пишешь на меня горькое и вменяешь мне грехи юности моей,
\vs Job 13:27 и ставишь в колоду ноги мои и подстерегаешь все стези мои,~--- гонишься по следам ног моих.
\vs Job 13:28 А он, как гниль, распадается, как одежда, изъеденная молью.
\vs Job 14:1 Человек, рожденный женою, краткодневен и пресыщен печалями:
\vs Job 14:2 как цветок, он выходит и опадает; убегает, как тень, и не останавливается.
\vs Job 14:3 И на него-то Ты отверзаешь очи Твои, и меня ведешь на суд с Тобою?
\vs Job 14:4 Кто родится чистым от нечистого? Ни один.
\vs Job 14:5 Если дни ему определены, и число месяцев его у Тебя, если Ты положил ему предел, которого он не перейдет,
\vs Job 14:6 то уклонись от него: пусть он отдохнет, доколе не окончит, как наемник, дня своего.
\vs Job 14:7 Для дерева есть надежда, что оно, если и будет срублено, снова оживет, и отрасли от него \bibemph{выходить} не перестанут:
\vs Job 14:8 если и устарел в земле корень его, и пень его замер в пыли,
\vs Job 14:9 но, лишь почуяло воду, оно дает отпрыски и пускает ветви, как бы вновь посаженное.
\vs Job 14:10 А человек умирает и распадается; отошел, и где он?
\vs Job 14:11 Уходят воды из озера, и река иссякает и высыхает:
\vs Job 14:12 так человек ляжет и не встанет; до скончания неба он не пробудится и не воспрянет от сна своего.
\vs Job 14:13 О, если бы Ты в преисподней сокрыл меня и укрывал меня, пока пройдет гнев Твой, положил мне срок и потом вспомнил обо мне!
\vs Job 14:14 Когда умрет человек, то будет ли он опять жить? Во все дни определенного мне времени я ожидал бы, пока придет мне смена.
\vs Job 14:15 Воззвал бы Ты, и я дал бы Тебе ответ, и Ты явил бы благоволение творению рук Твоих;
\vs Job 14:16 ибо тогда Ты исчислял бы шаги мои и не подстерегал бы греха моего;
\vs Job 14:17 в свитке было бы запечатано беззаконие мое, и Ты закрыл бы вину мою.
\vs Job 14:18 Но гора падая разрушается, и скала сходит с места своего;
\vs Job 14:19 вода стирает камни; разлив ее смывает земную пыль: так и надежду человека Ты уничтожаешь.
\vs Job 14:20 Теснишь его до конца, и он уходит; изменяешь ему лице и отсылаешь его.
\vs Job 14:21 В чести ли дети его~--- он не знает, унижены ли~--- он не замечает;
\vs Job 14:22 но плоть его на нем болит, и душа его в нем страдает.
\vs Job 15:1 И отвечал Елифаз Феманитянин и сказал:
\vs Job 15:2 станет ли мудрый отвечать знанием пустым и наполнять чрево свое ветром палящим,
\vs Job 15:3 оправдываться словами бесполезными и речью, не имеющею никакой силы?
\vs Job 15:4 Да ты отложил и страх и за малость считаешь речь к Богу.
\vs Job 15:5 Нечестие твое настроило так уста твои, и ты избрал язык лукавых.
\vs Job 15:6 Тебя обвиняют уста твои, а не я, и твой язык говорит против тебя.
\vs Job 15:7 Разве ты первым человеком родился и прежде холмов создан?
\vs Job 15:8 Разве совет Божий ты слышал и привлек к себе премудрость?
\vs Job 15:9 Что знаешь ты, чего бы не знали мы? что разумеешь ты, чего не было бы и у нас?
\vs Job 15:10 И седовласый и старец есть между нами, днями превышающий отца твоего.
\vs Job 15:11 Разве малость для тебя утешения Божии? И это неизвестно тебе?
\vs Job 15:12 К чему порывает тебя сердце твое, и к чему так гордо смотришь?
\vs Job 15:13 Что устремляешь против Бога дух твой и устами твоими произносишь такие речи?
\vs Job 15:14 Что такое человек, чтоб быть ему чистым, и чтобы рожденному женщиною быть праведным?
\vs Job 15:15 Вот, Он и святым Своим не доверяет, и небеса нечисты в очах Его:
\vs Job 15:16 тем больше нечист и растлен человек, пьющий беззаконие, как воду.
\vs Job 15:17 Я буду говорить тебе, слушай меня; я расскажу тебе, что видел,
\vs Job 15:18 что слышали мудрые и не скрыли слышанного от отцов своих,
\vs Job 15:19 которым одним отдана была земля, и среди которых чужой не ходил.
\vs Job 15:20 Нечестивый мучит себя во все дни свои, и число лет закрыто от притеснителя;
\vs Job 15:21 звук ужасов в ушах его; среди мира идет на него губитель.
\vs Job 15:22 Он не надеется спастись от тьмы; видит пред собою меч.
\vs Job 15:23 Он скитается за куском хлеба повсюду; знает, что уже готов, в руках у него день тьмы.
\vs Job 15:24 Устрашает его нужда и теснота; одолевает его, как царь, приготовившийся к битве,
\vs Job 15:25 за то, что он простирал против Бога руку свою и противился Вседержителю,
\vs Job 15:26 устремлялся против Него с \bibemph{гордою} выею, под толстыми щитами своими;
\vs Job 15:27 потому что он покрыл лице свое жиром своим и обложил туком лядвеи свои.
\vs Job 15:28 И он селится в городах разоренных, в домах, в которых не живут, которые обречены на развалины.
\vs Job 15:29 Не пребудет он богатым, и не уцелеет имущество его, и не распрострется по земле приобретение его.
\vs Job 15:30 Не уйдет от тьмы; отрасли его иссушит пламя и дуновением уст своих увлечет его.
\vs Job 15:31 Пусть не доверяет суете заблудший, ибо суета будет и воздаянием ему.
\vs Job 15:32 Не в свой день он скончается, и ветви его не будут зеленеть.
\vs Job 15:33 Сбросит он, как виноградная лоза, недозрелую ягоду свою и, как маслина, стряхнет цвет свой.
\vs Job 15:34 Так опустеет дом нечестивого, и огонь пожрет шатры мздоимства.
\vs Job 15:35 Он зачал зло и родил ложь, и утроба его приготовляет обман.
\vs Job 16:1 И отвечал Иов и сказал:
\vs Job 16:2 слышал я много такого; жалкие утешители все вы!
\vs Job 16:3 Будет ли конец ветреным словам? и что побудило тебя так отвечать?
\vs Job 16:4 И я мог бы так же говорить, как вы, если бы душа ваша была на месте души моей; ополчался бы на вас словами и кивал бы на вас головою моею;
\vs Job 16:5 подкреплял бы вас языком моим и движением губ утешал бы.
\vs Job 16:6 Говорю ли я, не утоляется скорбь моя; перестаю ли, что отходит от меня?
\vs Job 16:7 Но ныне Он изнурил меня. Ты разрушил всю семью мою.
\vs Job 16:8 Ты покрыл меня морщинами во свидетельство против меня; восстает на меня изможденность моя, в лицо укоряет меня.
\vs Job 16:9 Гнев Его терзает и враждует против меня, скрежещет на меня зубами своими; неприятель мой острит на меня глаза свои.
\vs Job 16:10 Разинули на меня пасть свою; ругаясь бьют меня по щекам: все сговорились против меня.
\vs Job 16:11 Предал меня Бог беззаконнику и в руки нечестивым бросил меня.
\vs Job 16:12 Я был спокоен, но Он потряс меня; взял меня за шею и избил меня и поставил меня целью для Себя.
\vs Job 16:13 Окружили меня стрельцы Его; Он рассекает внутренности мои и не щадит, пролил на землю желчь мою,
\vs Job 16:14 пробивает во мне пролом за проломом, бежит на меня, как ратоборец.
\vs Job 16:15 Вретище сшил я на кожу мою и в прах положил голову мою.
\vs Job 16:16 Лицо мое побагровело от плача, и на веждах моих тень смерти,
\vs Job 16:17 при всем том, что нет хищения в руках моих, и молитва моя чиста.
\vs Job 16:18 Земля! не закрой моей крови, и да не будет места воплю моему.
\vs Job 16:19 И ныне вот на небесах Свидетель мой, и Заступник мой в вышних!
\vs Job 16:20 Многоречивые друзья мои! К Богу слезит око мое.
\vs Job 16:21 О, если бы человек мог иметь состязание с Богом, как сын человеческий с ближним своим!
\vs Job 16:22 Ибо летам моим приходит конец, и я отхожу в путь невозвратный.
\vs Job 17:1 Дыхание мое ослабело; дни мои угасают; гробы предо мною.
\vs Job 17:2 Если бы не насмешки их, то и среди споров их око мое пребывало бы спокойно.
\vs Job 17:3 Заступись, поручись \bibemph{Сам} за меня пред Собою! иначе кто поручится за меня?
\vs Job 17:4 Ибо Ты закрыл сердце их от разумения, и потому не дашь восторжествовать \bibemph{им}.
\vs Job 17:5 Кто обрекает друзей своих в добычу, у детей того глаза истают.
\vs Job 17:6 Он поставил меня притчею для народа и посмешищем для него.
\vs Job 17:7 Помутилось от горести око мое, и все члены мои, как тень.
\vs Job 17:8 Изумятся о сем праведные, и невинный вознегодует на лицемера.
\vs Job 17:9 Но праведник будет крепко держаться пути своего, и чистый руками будет больше и больше утверждаться.
\vs Job 17:10 Выступайте, все вы, и подойдите; не найду я мудрого между вами.
\vs Job 17:11 Дни мои прошли; думы мои~--- достояние сердца моего~--- разбиты.
\vs Job 17:12 А они ночь \bibemph{хотят} превратить в день, свет приблизить к лицу тьмы.
\vs Job 17:13 Если бы я и ожидать стал, то преисподняя~--- дом мой; во тьме постелю я постель мою;
\vs Job 17:14 гробу скажу: ты отец мой, червю: ты мать моя и сестра моя.
\vs Job 17:15 Где же после этого надежда моя? и ожидаемое мною кто увидит?
\vs Job 17:16 В преисподнюю сойдет она и будет покоиться со мною в прахе.
\vs Job 18:1 И отвечал Вилдад Савхеянин и сказал:
\vs Job 18:2 когда же положите вы конец таким речам? обдумайте, и потом будем говорить.
\vs Job 18:3 Зачем считаться нам за животных и быть униженными в собственных глазах ваших?
\vs Job 18:4 \bibemph{О ты}, раздирающий душу твою в гневе твоем! Неужели для тебя опустеть земле, и скале сдвинуться с места своего?
\vs Job 18:5 Да, свет у беззаконного потухнет, и не останется искры от огня его.
\vs Job 18:6 Померкнет свет в шатре его, и светильник его угаснет над ним.
\vs Job 18:7 Сократятся шаги могущества его, и низложит его собственный замысл его,
\vs Job 18:8 ибо он попадет в сеть своими ногами и по тенетам ходить будет.
\vs Job 18:9 Петля зацепит за ногу его, и грабитель уловит его.
\vs Job 18:10 Скрытно разложены по земле силки для него и западни на дороге.
\vs Job 18:11 Со всех сторон будут страшить его ужасы и заставят его бросаться туда и сюда.
\vs Job 18:12 Истощится от голода сила его, и гибель готова, сбоку у него.
\vs Job 18:13 Съест члены тела его, съест члены его первенец смерти.
\vs Job 18:14 Изгнана будет из шатра его надежда его, и это низведет его к царю ужасов.
\vs Job 18:15 Поселятся в шатре его, потому что он уже не его; жилище его посыпано будет серою.
\vs Job 18:16 Снизу подсохнут корни его, и сверху увянут ветви его.
\vs Job 18:17 Память о нем исчезнет с земли, и имени его не будет на площади.
\vs Job 18:18 Изгонят его из света во тьму и сотрут его с лица земли.
\vs Job 18:19 Ни сына его, ни внука не будет в народе его, и никого не останется в жилищах его.
\vs Job 18:20 О дне его ужаснутся потомки, и современники будут объяты трепетом.
\vs Job 18:21 Таковы жилища беззаконного, и таково место того, кто не знает Бога.
\vs Job 19:1 И отвечал Иов и сказал:
\vs Job 19:2 доколе будете мучить душу мою и терзать меня речами?
\vs Job 19:3 Вот, уже раз десять вы срамили меня и не стыдитесь теснить меня.
\vs Job 19:4 Если я и действительно погрешил, то погрешность моя при мне остается.
\vs Job 19:5 Если же вы хотите повеличаться надо мною и упрекнуть меня позором моим,
\vs Job 19:6 то знайте, что Бог ниспроверг меня и обложил меня Своею сетью.
\vs Job 19:7 Вот, я кричу: обида! и никто не слушает; вопию, и нет суда.
\vs Job 19:8 Он преградил мне дорогу, и не могу пройти, и на стези мои положил тьму.
\vs Job 19:9 Совлек с меня славу мою и снял венец с головы моей.
\vs Job 19:10 Кругом разорил меня, и я отхожу; и, как дерево, Он исторг надежду мою.
\vs Job 19:11 Воспылал на меня гневом Своим и считает меня между врагами Своими.
\vs Job 19:12 Полки Его пришли вместе и направили путь свой ко мне и расположились вокруг шатра моего.
\vs Job 19:13 Братьев моих Он удалил от меня, и знающие меня чуждаются меня.
\vs Job 19:14 Покинули меня близкие мои, и знакомые мои забыли меня.
\vs Job 19:15 Пришлые в доме моем и служанки мои чужим считают меня; посторонним стал я в глазах их.
\vs Job 19:16 Зову слугу моего, и он не откликается; устами моими я должен умолять его.
\vs Job 19:17 Дыхание мое опротивело жене моей, и я должен умолять ее ради детей чрева моего.
\vs Job 19:18 Даже малые дети презирают меня: поднимаюсь, и они издеваются надо мною.
\vs Job 19:19 Гнушаются мною все наперсники мои, и те, которых я любил, обратились против меня.
\vs Job 19:20 Кости мои прилипли к коже моей и плоти моей, и я остался только с кожею около зубов моих.
\vs Job 19:21 Помилуйте меня, помилуйте меня вы, друзья мои, ибо рука Божия коснулась меня.
\vs Job 19:22 Зачем и вы преследуете меня, как Бог, и плотью моею не можете насытиться?
\vs Job 19:23 О, если бы записаны были слова мои! Если бы начертаны были они в книге
\vs Job 19:24 резцом железным с оловом,~--- на вечное время на камне вырезаны были!
\vs Job 19:25 А я знаю, Искупитель мой жив, и Он в последний день восставит из праха распадающуюся кожу мою сию,
\vs Job 19:26 и я во плоти моей узрю Бога.
\vs Job 19:27 Я узрю Его сам; мои глаза, не глаза другого, увидят Его. Истаевает сердце мое в груди моей!
\vs Job 19:28 Вам надлежало бы сказать: зачем мы преследуем его? Как будто корень зла найден во мне.
\vs Job 19:29 Убойтесь меча, ибо меч есть отмститель неправды, и знайте, что есть суд.
\vs Job 20:1 И отвечал Софар Наамитянин и сказал:
\vs Job 20:2 размышления мои побуждают меня отвечать, и я поспешаю выразить их.
\vs Job 20:3 Упрек, позорный для меня, выслушал я, и дух разумения моего ответит за меня.
\vs Job 20:4 Разве не знаешь ты, что от века,~--- с того времени, как поставлен человек на земле,~---
\vs Job 20:5 веселье беззаконных кратковременно, и радость лицемера мгновенна?
\vs Job 20:6 Хотя бы возросло до небес величие его, и голова его касалась облаков,~---
\vs Job 20:7 как помет его, на веки пропадает он; видевшие его скажут: где он?
\vs Job 20:8 Как сон, улетит, и не найдут его; и, как ночное видение, исчезнет.
\vs Job 20:9 Глаз, видевший его, больше не увидит его, и уже не усмотрит его место его.
\vs Job 20:10 Сыновья его будут заискивать у нищих, и руки его возвратят похищенное им.
\vs Job 20:11 Кости его наполнены грехами юности его, и с ним лягут они в прах.
\vs Job 20:12 Если сладко во рту его зло, и он таит его под языком своим,
\vs Job 20:13 бережет и не бросает его, а держит его в устах своих,
\vs Job 20:14 то эта пища его в утробе его превратится в желчь аспидов внутри его.
\vs Job 20:15 Имение, которое он глотал, изблюет: Бог исторгнет его из чрева его.
\vs Job 20:16 Змеиный яд он сосет; умертвит его язык ехидны.
\vs Job 20:17 Не видать ему ручьев, рек, текущих медом и молоком!
\vs Job 20:18 Нажитое трудом возвратит, не проглотит; по мере имения его будет и расплата его, а он не порадуется.
\vs Job 20:19 Ибо он угнетал, отсылал бедных; захватывал домы, которых не строил;
\vs Job 20:20 не знал сытости во чреве своем и в жадности своей не щадил ничего.
\vs Job 20:21 Ничего не спаслось от обжорства его, зато не устоит счастье его.
\vs Job 20:22 В полноте изобилия будет тесно ему; всякая рука обиженного поднимется на него.
\vs Job 20:23 Когда будет чем наполнить утробу его, Он пошлет на него ярость гнева Своего и одождит на него болезни в плоти его.
\vs Job 20:24 Убежит ли он от оружия железного,~--- пронзит его лук медный;
\vs Job 20:25 станет вынимать \bibemph{стрелу},~--- и она выйдет из тела, выйдет, сверкая сквозь желчь его; ужасы смерти найдут на него!
\vs Job 20:26 Все мрачное сокрыто внутри его; будет пожирать его огонь, никем не раздуваемый; зло постигнет и оставшееся в шатре его.
\vs Job 20:27 Небо откроет беззаконие его, и земля восстанет против него.
\vs Job 20:28 Исчезнет стяжание дома его; все расплывется в день гнева Его.
\vs Job 20:29 Вот удел человеку беззаконному от Бога и наследие, определенное ему Вседержителем!
\vs Job 21:1 И отвечал Иов и сказал:
\vs Job 21:2 выслушайте внимательно речь мою, и это будет мне утешением от вас.
\vs Job 21:3 Потерпите меня, и я буду говорить; а после того, как поговорю, насмехайся.
\vs Job 21:4 Разве к человеку речь моя? как же мне и не малодушествовать?
\vs Job 21:5 Посмотрите на меня и ужаснитесь, и положите перст на уста.
\vs Job 21:6 Лишь только я вспомню,~--- содрогаюсь, и трепет объемлет тело мое.
\vs Job 21:7 Почему беззаконные живут, достигают старости, да и силами крепки?
\vs Job 21:8 Дети их с ними перед лицем их, и внуки их перед глазами их.
\vs Job 21:9 Домы их безопасны от страха, и нет жезла Божия на них.
\vs Job 21:10 Вол их оплодотворяет и не извергает, корова их зачинает и не выкидывает.
\vs Job 21:11 Как стадо, выпускают они малюток своих, и дети их прыгают.
\vs Job 21:12 Восклицают под \bibemph{голос} тимпана и цитры и веселятся при \bibemph{звуках} свирели;
\vs Job 21:13 проводят дни свои в счастьи и мгновенно нисходят в преисподнюю.
\vs Job 21:14 А между тем они говорят Богу: отойди от нас, не хотим мы знать путей Твоих!
\vs Job 21:15 Что Вседержитель, чтобы нам служить Ему? и что пользы прибегать к Нему?
\vs Job 21:16 Видишь, счастье их не от их рук.~--- Совет нечестивых будь далек от меня!
\vs Job 21:17 Часто ли угасает светильник у беззаконных, и находит на них беда, и Он дает им в удел страдания во гневе Своем?
\vs Job 21:18 Они должны быть, как соломинка пред ветром и как плева, уносимая вихрем.
\vs Job 21:19 \bibemph{Скажешь}: Бог бережет для детей его несчастье его.~--- Пусть воздаст Он ему самому, чтобы он это знал.
\vs Job 21:20 Пусть его глаза увидят несчастье его, и пусть он сам пьет от гнева Вседержителева.
\vs Job 21:21 Ибо какая ему забота до дома своего после него, когда число месяцев его кончится?
\vs Job 21:22 Но Бога ли учить мудрости, когда Он судит и горних?
\vs Job 21:23 Один умирает в самой полноте сил своих, совершенно спокойный и мирный;
\vs Job 21:24 внутренности его полны жира, и кости его напоены мозгом.
\vs Job 21:25 А другой умирает с душею огорченною, не вкусив добра.
\vs Job 21:26 И они вместе будут лежать во прахе, и червь покроет их.
\vs Job 21:27 Знаю я ваши мысли и ухищрения, какие вы против меня сплетаете.
\vs Job 21:28 Вы скажете: где дом князя, и где шатер, в котором жили беззаконные?
\vs Job 21:29 Разве вы не спрашивали у путешественников и незнакомы с их наблюдениями,
\vs Job 21:30 что в день погибели пощажен бывает злодей, в день гнева отводится в сторону?
\vs Job 21:31 Кто представит ему пред лице путь его, и кто воздаст ему за то, что он делал?
\vs Job 21:32 Его провожают ко гробам и на его могиле ставят стражу.
\vs Job 21:33 Сладки для него глыбы долины, и за ним идет толпа людей, а идущим перед ним нет числа.
\vs Job 21:34 Как же вы хотите утешать меня пустым? В ваших ответах остается \bibemph{одна} ложь.
\vs Job 22:1 И отвечал Елифаз Феманитянин и сказал:
\vs Job 22:2 разве может человек доставлять пользу Богу? Разумный доставляет пользу себе самому.
\vs Job 22:3 Что за удовольствие Вседержителю, что ты праведен? И будет ли Ему выгода от того, что ты содержишь пути твои в непорочности?
\vs Job 22:4 Неужели Он, боясь тебя, вступит с тобою в состязание, пойдет судиться с тобою?
\vs Job 22:5 Верно, злоба твоя велика, и беззакониям твоим нет конца.
\vs Job 22:6 Верно, ты брал залоги от братьев твоих ни за что и с полунагих снимал одежду.
\vs Job 22:7 Утомленному жаждою не подавал воды напиться и голодному отказывал в хлебе;
\vs Job 22:8 а человеку сильному ты \bibemph{давал} землю, и сановитый селился на ней.
\vs Job 22:9 Вдов ты отсылал ни с чем и сирот оставлял с пустыми руками.
\vs Job 22:10 За то вокруг тебя петли, и возмутил тебя неожиданный ужас,
\vs Job 22:11 или тьма, в которой ты ничего не видишь, и множество вод покрыло тебя.
\vs Job 22:12 Не превыше ли небес Бог? посмотри вверх на звезды, как они высоко!
\vs Job 22:13 И ты говоришь: что знает Бог? может ли Он судить сквозь мрак?
\vs Job 22:14 Облака~--- завеса Его, так что Он не видит, а ходит \bibemph{только} по небесному кругу.
\vs Job 22:15 Неужели ты держишься пути древних, по которому шли люди беззаконные,
\vs Job 22:16 которые преждевременно были истреблены, когда вода разлилась под основание их?
\vs Job 22:17 Они говорили Богу: отойди от нас! и что сделает им Вседержитель?
\vs Job 22:18 А Он наполнял домы их добром. Но совет нечестивых будь далек от меня!
\vs Job 22:19 Видели праведники и радовались, и непорочный смеялся им:
\vs Job 22:20 враг наш истреблен, а оставшееся после них пожрал огонь.
\vs Job 22:21 Сблизься же с Ним~--- и будешь спокоен; чрез это придет к тебе добро.
\vs Job 22:22 Прими из уст Его закон и положи слова Его в сердце твое.
\vs Job 22:23 Если ты обратишься к Вседержителю, то вновь устроишься, удалишь беззаконие от шатра твоего
\vs Job 22:24 и будешь вменять в прах блестящий металл, и в камни потоков~--- \bibemph{золото} Офирское.
\vs Job 22:25 И будет Вседержитель твоим золотом и блестящим серебром у тебя,
\vs Job 22:26 ибо тогда будешь радоваться о Вседержителе и поднимешь к Богу лице твое.
\vs Job 22:27 Помолишься Ему, и Он услышит тебя, и ты исполнишь обеты твои.
\vs Job 22:28 Положишь намерение, и оно состоится у тебя, и над путями твоими будет сиять свет.
\vs Job 22:29 Когда кто уничижен будет, ты скажешь: возвышение! и Он спасет поникшего лицем,
\vs Job 22:30 избавит и небезвинного, и он спасется чистотою рук твоих.
\vs Job 23:1 И отвечал Иов и сказал:
\vs Job 23:2 еще и ныне горька речь моя: страдания мои тяжелее стонов моих.
\vs Job 23:3 О, если бы я знал, где найти Его, и мог подойти к престолу Его!
\vs Job 23:4 Я изложил бы пред Ним дело мое и уста мои наполнил бы оправданиями;
\vs Job 23:5 узнал бы слова, какими Он ответит мне, и понял бы, что Он скажет мне.
\vs Job 23:6 Неужели Он в полном могуществе стал бы состязаться со мною? О, нет! Пусть Он только обратил бы внимание на меня.
\vs Job 23:7 Тогда праведник мог бы состязаться с Ним,~--- и я навсегда получил бы свободу от Судии моего.
\vs Job 23:8 Но вот, я иду вперед~--- и нет Его, назад~--- и не нахожу Его;
\vs Job 23:9 делает ли Он что на левой стороне, я не вижу; скрывается ли на правой, не усматриваю.
\vs Job 23:10 Но Он знает путь мой; пусть испытает меня,~--- выйду, как золото.
\vs Job 23:11 Нога моя твердо держится стези Его; пути Его я хранил и не уклонялся.
\vs Job 23:12 От заповеди уст Его не отступал; глаголы уст Его хранил больше, нежели мои правила.
\vs Job 23:13 Но Он тверд; и кто отклонит Его? Он делает, чего хочет душа Его.
\vs Job 23:14 Так, Он выполнит положенное мне, и подобного этому много у Него.
\vs Job 23:15 Поэтому я трепещу пред лицем Его; размышляю~--- и страшусь Его.
\vs Job 23:16 Бог расслабил сердце мое, и Вседержитель устрашил меня.
\vs Job 23:17 Зачем я не уничтожен прежде этой тьмы, и Он не сокрыл мрака от лица моего!
\vs Job 24:1 Почему не сокрыты от Вседержителя времена, и знающие Его не видят дней Его?
\vs Job 24:2 Межи передвигают, угоняют стада и пасут \bibemph{у себя}.
\vs Job 24:3 У сирот уводят осла, у вдовы берут в залог вола;
\vs Job 24:4 бедных сталкивают с дороги, все уничиженные земли принуждены скрываться.
\vs Job 24:5 Вот они, \bibemph{как} дикие ослы в пустыне, выходят на дело свое, вставая рано на добычу; степь \bibemph{дает} хлеб для них и для детей их;
\vs Job 24:6 жнут они на поле не своем и собирают виноград у нечестивца;
\vs Job 24:7 нагие ночуют без покрова и без одеяния на стуже;
\vs Job 24:8 мокнут от горных дождей и, не имея убежища, жмутся к скале;
\vs Job 24:9 отторгают от сосцов сироту и с нищего берут залог;
\vs Job 24:10 заставляют ходить нагими, без одеяния, и голодных кормят колосьями;
\vs Job 24:11 между стенами выжимают масло оливковое, топчут в точилах и жаждут.
\vs Job 24:12 В городе люди стонут, и душа убиваемых вопит, и Бог не воспрещает того.
\vs Job 24:13 Есть из них враги света, не знают путей его и не ходят по стезям его.
\vs Job 24:14 С рассветом встает убийца, умерщвляет бедного и нищего, а ночью бывает вором.
\vs Job 24:15 И око прелюбодея ждет сумерков, говоря: ничей глаз не увидит меня,~--- и закрывает лице.
\vs Job 24:16 В темноте подкапываются под домы, которые днем они заметили для себя; не знают света.
\vs Job 24:17 Ибо для них утро~--- смертная тень, так как они знакомы с ужасами смертной тени.
\vs Job 24:18 Легок такой на поверхности воды, проклята часть его на земле, и не смотрит он на дорогу садов виноградных.
\vs Job 24:19 Засуха и жара поглощают снежную воду: так преисподняя~--- грешников.
\vs Job 24:20 Пусть забудет его утроба \bibemph{матери}; пусть лакомится им червь; пусть не остается о нем память; как дерево, пусть сломится беззаконник,
\vs Job 24:21 который угнетает бездетную, не рождавшую, и вдове не делает добра.
\vs Job 24:22 Он и сильных увлекает своею силою; он встает, и никто не уверен за жизнь свою.
\vs Job 24:23 А Он дает ему \bibemph{все} для безопасности, и он \bibemph{на то} опирается, и очи Его видят пути их.
\vs Job 24:24 Поднялись высоко,~--- и вот, нет их; падают и умирают, как и все, и, как верхушки колосьев, срезываются.
\vs Job 24:25 Если это не так,~--- кто обличит меня во лжи и в ничто обратит речь мою?
\vs Job 25:1 И отвечал Вилдад Савхеянин и сказал:
\vs Job 25:2 держава и страх у Него; Он творит мир на высотах Своих!
\vs Job 25:3 Есть ли счет воинствам Его? и над кем не восходит свет Его?
\vs Job 25:4 И как человеку быть правым пред Богом, и как быть чистым рожденному женщиною?
\vs Job 25:5 Вот даже луна, и та несветла, и звезды нечисты пред очами Его.
\vs Job 25:6 Тем менее человек, \bibemph{который} есть червь, и сын человеческий, \bibemph{который} есть моль.
\vs Job 26:1 И отвечал Иов и сказал:
\vs Job 26:2 как ты помог бессильному, поддержал мышцу немощного!
\vs Job 26:3 Какой совет подал ты немудрому и как во всей полноте объяснил дело!
\vs Job 26:4 Кому ты говорил эти слова, и чей дух исходил из тебя?
\vs Job 26:5 Рефаимы трепещут под водами, и живущие в них.
\vs Job 26:6 Преисподняя обнажена пред Ним, и нет покрывала Аваддону.
\vs Job 26:7 Он распростер север над пустотою, повесил землю ни на чем.
\vs Job 26:8 Он заключает воды в облаках Своих, и облако не расседается под ними.
\vs Job 26:9 Он поставил престол Свой, распростер над ним облако Свое.
\vs Job 26:10 Черту провел над поверхностью воды, до границ света со тьмою.
\vs Job 26:11 Столпы небес дрожат и ужасаются от грозы Его.
\vs Job 26:12 Силою Своею волнует море и разумом Своим сражает его дерзость.
\vs Job 26:13 От духа Его~--- великолепие неба; рука Его образовала быстрого скорпиона.
\vs Job 26:14 Вот, это части путей Его; и как мало мы слышали о Нем! А гром могущества Его кто может уразуметь?
\vs Job 27:1 И продолжал Иов возвышенную речь свою и сказал:
\vs Job 27:2 жив Бог, лишивший \bibemph{меня} суда, и Вседержитель, огорчивший душу мою,
\vs Job 27:3 что, доколе еще дыхание мое во мне и дух Божий в ноздрях моих,
\vs Job 27:4 не скажут уста мои неправды, и язык мой не произнесет лжи!
\vs Job 27:5 Далек я от того, чтобы признать вас справедливыми; доколе не умру, не уступлю непорочности моей.
\vs Job 27:6 Крепко держал я правду мою и не опущу ее; не укорит меня сердце мое во все дни мои.
\vs Job 27:7 Враг мой будет, как нечестивец, и восстающий на меня, как беззаконник.
\vs Job 27:8 Ибо какая надежда лицемеру, когда возьмет, когда исторгнет Бог душу его?
\vs Job 27:9 Услышит ли Бог вопль его, когда придет на него беда?
\vs Job 27:10 Будет ли он утешаться Вседержителем и призывать Бога во всякое время?
\vs Job 27:11 Возвещу вам, чт\acc{о} в руке Божией; чт\acc{о} у Вседержителя, не скрою.
\vs Job 27:12 Вот, все вы и сами видели; и для чего вы столько пустословите?
\vs Job 27:13 Вот доля человеку беззаконному от Бога, и наследие, какое получают от Вседержителя притеснители.
\vs Job 27:14 Если умножаются сыновья его, то под меч; и потомки его не насытятся хлебом.
\vs Job 27:15 Оставшихся по нем смерть низведет во гроб, и вдовы их не будут плакать.
\vs Job 27:16 Если он наберет кучи серебра, как праха, и наготовит одежд, как брение,
\vs Job 27:17 то он наготовит, а одеваться будет праведник, и серебро получит себе на долю беспорочный.
\vs Job 27:18 Он строит, как моль, дом свой и, как сторож, делает себе шалаш;
\vs Job 27:19 ложится спать богачом и таким не встанет; открывает глаза свои, и он уже не тот.
\vs Job 27:20 Как в\acc{о}ды, постигнут его ужасы; в ночи похитит его буря.
\vs Job 27:21 Поднимет его восточный ветер и понесет, и он быстро побежит от него.
\vs Job 27:22 Устремится на него и не пощадит, как бы он ни силился убежать от руки его.
\vs Job 27:23 Всплеснут о нем руками и посвищут над ним с места его!
\vs Job 28:1 Так! у серебра есть источная жила, и у золота место, \bibemph{где его} плавят.
\vs Job 28:2 Железо получается из земли; из камня выплавляется медь.
\vs Job 28:3 \bibemph{Человек} полагает предел тьме и тщательно разыскивает камень во мраке и тени смертной.
\vs Job 28:4 Вырывают рудокопный колодезь в местах, забытых ногою, спускаются вглубь, висят \bibemph{и} зыблются вдали от людей.
\vs Job 28:5 Земля, на которой вырастает хлеб, внутри изрыта как бы огнем.
\vs Job 28:6 Камни ее~--- место сапфира, и в ней песчинки золота.
\vs Job 28:7 Стези \bibemph{туда} не знает хищная птица, и не видал ее глаз коршуна;
\vs Job 28:8 не попирали ее скимны, и не ходил по ней шакал.
\vs Job 28:9 На гранит налагает он руку свою, с корнем опрокидывает горы;
\vs Job 28:10 в скалах просекает каналы, и все драгоценное видит глаз его;
\vs Job 28:11 останавливает течение потоков и сокровенное выносит на свет.
\vs Job 28:12 Но где премудрость обретается? и где место разума?
\vs Job 28:13 Не знает человек цены ее, и она не обретается на земле живых.
\vs Job 28:14 Бездна говорит: не во мне она; и море говорит: не у меня.
\vs Job 28:15 Не дается она за золото и не приобретается она за вес серебра;
\vs Job 28:16 не оценивается она золотом Офирским, ни драгоценным ониксом, ни сапфиром;
\vs Job 28:17 не равняется с нею золото и кристалл, и не выменяешь ее на сосуды из чистого золота.
\vs Job 28:18 А о кораллах и жемчуге и упоминать нечего, и приобретение премудрости выше рубинов.
\vs Job 28:19 Не равняется с нею топаз Ефиопский; чистым золотом не оценивается она.
\vs Job 28:20 Откуда же исходит премудрость? и где место разума?
\vs Job 28:21 Сокрыта она от очей всего живущего и от птиц небесных утаена.
\vs Job 28:22 Аваддон и смерть говорят: ушами нашими слышали мы слух о ней.
\vs Job 28:23 Бог знает путь ее, и Он ведает место ее.
\vs Job 28:24 Ибо Он прозирает до концов земли и видит под всем небом.
\vs Job 28:25 Когда Он ветру полагал вес и располагал воду по мере,
\vs Job 28:26 когда назначал устав дождю и путь для молнии громоносной,
\vs Job 28:27 тогда Он видел ее и явил ее, приготовил ее и еще испытал ее
\vs Job 28:28 и сказал человеку: вот, страх Господень есть истинная премудрость, и удаление от зла~--- разум.
\vs Job 29:1 И продолжал Иов возвышенную речь свою и сказал:
\vs Job 29:2 о, если бы я был, как в прежние месяцы, как в те дни, когда Бог хранил меня,
\vs Job 29:3 когда светильник Его светил над головою моею, и я при свете Его ходил среди тьмы;
\vs Job 29:4 как был я во дни молодости моей, когда милость Божия \bibemph{была} над шатром моим,
\vs Job 29:5 когда еще Вседержитель \bibemph{был} со мною, и дети мои вокруг меня,
\vs Job 29:6 когда пути мои обливались молоком, и скала источала для меня ручьи елея!
\vs Job 29:7 когда я выходил к воротам города и на площади ставил седалище свое,~---
\vs Job 29:8 юноши, увидев меня, прятались, а старцы вставали и стояли;
\vs Job 29:9 князья удерживались от речи и персты полагали на уста свои;
\vs Job 29:10 голос знатных умолкал, и язык их прилипал к гортани их.
\vs Job 29:11 Ухо, слышавшее меня, ублажало меня; око видевшее восхваляло меня,
\vs Job 29:12 потому что я спасал страдальца вопиющего и сироту беспомощного.
\vs Job 29:13 Благословение погибавшего приходило на меня, и сердцу вдовы доставлял я радость.
\vs Job 29:14 Я облекался в правду, и суд мой одевал меня, как мантия и увясло.
\vs Job 29:15 Я был глазами слепому и ногами хромому;
\vs Job 29:16 отцом был я для нищих и тяжбу, которой я не знал, разбирал внимательно.
\vs Job 29:17 Сокрушал я беззаконному челюсти и из зубов его исторгал похищенное.
\vs Job 29:18 И говорил я: в гнезде моем скончаюсь, и дни \bibemph{мои} будут многи, как песок;
\vs Job 29:19 корень мой открыт для воды, и роса ночует на ветвях моих;
\vs Job 29:20 слава моя не стареет, лук мой крепок в руке моей.
\vs Job 29:21 Внимали мне и ожидали, и безмолвствовали при совете моем.
\vs Job 29:22 После слов моих уже не рассуждали; речь моя капала на них.
\vs Job 29:23 Ждали меня, как дождя, и, \bibemph{как} дождю позднему, открывали уста свои.
\vs Job 29:24 Бывало, улыбнусь им~--- они не верят; и света лица моего они не помрачали.
\vs Job 29:25 Я назначал пути им и сидел во главе и жил как царь в кругу воинов, как утешитель плачущих.
\vs Job 30:1 А ныне смеются надо мною младшие меня летами, те, которых отцов я не согласился бы поместить с псами стад моих.
\vs Job 30:2 И сила рук их к чему мне? Над ними уже прошло время.
\vs Job 30:3 Бедностью и голодом истощенные, они убегают в степь безводную, мрачную и опустевшую;
\vs Job 30:4 щиплют зелень подле кустов, и ягоды можжевельника~--- хлеб их.
\vs Job 30:5 Из общества изгоняют их, кричат на них, как на воров,
\vs Job 30:6 чтобы жили они в рытвинах потоков, в ущельях земли и утесов.
\vs Job 30:7 Ревут между кустами, жмутся под терном.
\vs Job 30:8 Люди отверженные, люди без имени, отребье земли!
\vs Job 30:9 Их-то сделался я ныне песнью и пищею разговора их.
\vs Job 30:10 Они гнушаются мною, удаляются от меня и не удерживаются плевать пред лицем моим.
\vs Job 30:11 Так как Он развязал повод мой и поразил меня, то они сбросили с себя узду пред лицем моим.
\vs Job 30:12 С правого боку встает это исчадие, сбивает меня с ног, направляет гибельные свои пути ко мне.
\vs Job 30:13 А мою стезю испортили: всё успели сделать к моей погибели, не имея помощника.
\vs Job 30:14 Они пришли ко мне, как сквозь широкий пролом; с шумом бросились на меня.
\vs Job 30:15 Ужасы устремились на меня; как ветер, развеялось величие мое, и счастье мое унеслось, как облако.
\vs Job 30:16 И ныне изливается душа моя во мне: дни скорби объяли меня.
\vs Job 30:17 Ночью ноют во мне кости мои, и жилы мои не имеют покоя.
\vs Job 30:18 С великим трудом снимается с меня одежда моя; края хитона моего жмут меня.
\vs Job 30:19 Он бросил меня в грязь, и я стал, как прах и пепел.
\vs Job 30:20 Я взываю к Тебе, и Ты не внимаешь мне,~--- стою, а Ты \bibemph{только} смотришь на меня.
\vs Job 30:21 Ты сделался жестоким ко мне, крепкою рукою враждуешь против меня.
\vs Job 30:22 Ты поднял меня и заставил меня носиться по ветру и сокрушаешь меня.
\vs Job 30:23 Так, я знаю, что Ты приведешь меня к смерти и в дом собрания всех живущих.
\vs Job 30:24 Верно, Он не прострет руки Своей на дом костей: будут ли они кричать при своем разрушении?
\vs Job 30:25 Не плакал ли я о том, кто был в горе? не скорбела ли душа моя о бедных?
\vs Job 30:26 Когда я чаял добра, пришло зло; когда ожидал света, пришла тьма.
\vs Job 30:27 Мои внутренности кипят и не перестают; встретили меня дни печали.
\vs Job 30:28 Я хожу почернелый, но не от солнца; встаю в собрании и кричу.
\vs Job 30:29 Я стал братом шакалам и другом страусам.
\vs Job 30:30 Моя кожа почернела на мне, и кости мои обгорели от жара.
\vs Job 30:31 И цитра моя сделалась унылою, и свирель моя~--- голосом плачевным.
\vs Job 31:1 Завет положил я с глазами моими, чтобы не помышлять мне о девице.
\vs Job 31:2 Какая же участь \bibemph{мне} от Бога свыше? И какое наследие от Вседержителя с небес?
\vs Job 31:3 Не для нечестивого ли гибель, и не для делающего ли зло напасть?
\vs Job 31:4 Не видел ли Он путей моих, и не считал ли всех моих шагов?
\vs Job 31:5 Если я ходил в суете, и если нога моя спешила на лукавство,~---
\vs Job 31:6 пусть взвесят меня на весах правды, и Бог узнает мою непорочность.
\vs Job 31:7 Если стопы мои уклонялись от пути и сердце мое следовало за глазами моими, и если что-либо \bibemph{нечистое} пристало к рукам моим,
\vs Job 31:8 то пусть я сею, а другой ест, и пусть отрасли мои искоренены будут.
\vs Job 31:9 Если сердце мое прельщалось женщиною и я строил ковы у дверей моего ближнего,~---
\vs Job 31:10 пусть моя жена мелет на другого, и пусть другие издеваются над нею,
\vs Job 31:11 потому что это~--- преступление, это~--- беззаконие, подлежащее суду;
\vs Job 31:12 это~--- огонь, поядающий до истребления, который искоренил бы все добро мое.
\vs Job 31:13 Если я пренебрегал правами слуги и служанки моей, когда они имели спор со мною,
\vs Job 31:14 то что стал бы я делать, когда бы Бог восстал? И когда бы Он взглянул на меня, что мог бы я отвечать Ему?
\vs Job 31:15 Не Он ли, Который создал меня во чреве, создал и его и равно образовал нас в утробе?
\vs Job 31:16 Отказывал ли я нуждающимся в их просьбе и томил ли глаза вдовы?
\vs Job 31:17 Один ли я съедал кусок мой, и не ел ли от него и сирота?
\vs Job 31:18 Ибо с детства он рос со мною, как с отцом, и от чрева матери моей я руководил \bibemph{вдову}.
\vs Job 31:19 Если я видел кого погибавшим без одежды и бедного без покрова,~---
\vs Job 31:20 не благословляли ли меня чресла его, и не был ли он согрет шерстью овец моих?
\vs Job 31:21 Если я поднимал руку мою на сироту, когда видел помощь себе у ворот,
\vs Job 31:22 то пусть плечо мое отпадет от спины, и рука моя пусть отломится от локтя,
\vs Job 31:23 ибо страшно для меня наказание от Бога: пред величием Его не устоял бы я.
\vs Job 31:24 Полагал ли я в золоте опору мою и говорил ли сокровищу: ты~--- надежда моя?
\vs Job 31:25 Радовался ли я, что богатство мое было велико, и что рука моя приобрела много?
\vs Job 31:26 Смотря на солнце, как оно сияет, и на луну, как она величественно шествует,
\vs Job 31:27 прельстился ли я в тайне сердца моего, и целовали ли уста мои руку мою?
\vs Job 31:28 Это также было бы преступление, подлежащее суду, потому что я отрекся бы \bibemph{тогда} от Бога Всевышнего.
\vs Job 31:29 Радовался ли я погибели врага моего и торжествовал ли, когда несчастье постигало его?
\vs Job 31:30 Не позволял я устам моим грешить проклятием души его.
\vs Job 31:31 Не говорили ли люди шатра моего: о, если бы мы от мяс его не насытились?
\vs Job 31:32 Странник не ночевал на улице; двери мои я отворял прохожему.
\vs Job 31:33 Если бы я скрывал проступки мои, как человек, утаивая в груди моей пороки мои,
\vs Job 31:34 то я боялся бы большого общества, и презрение одноплеменников страшило бы меня, и я молчал бы и не выходил бы за двери.
\vs Job 31:35 О, если бы кто выслушал меня! Вот мое желание, чтобы Вседержитель отвечал мне, и чтобы защитник мой составил запись.
\vs Job 31:36 Я носил бы ее на плечах моих и возлагал бы ее, как венец;
\vs Job 31:37 объявил бы ему число шагов моих, сблизился бы с ним, как с князем.
\vs Job 31:38 Если вопияла на меня земля моя и жаловались на меня борозды ее;
\vs Job 31:39 если я ел плоды ее без платы и отягощал жизнь земледельцев,
\vs Job 31:40 то пусть вместо пшеницы вырастает волчец и вместо ячменя куколь. Слова Иова кончились.
\vs Job 32:1 Когда те три мужа перестали отвечать Иову, потому что он был прав в глазах своих,
\vs Job 32:2 тогда воспылал гнев Елиуя, сына Варахиилова, Вузитянина из племени Рамова: воспылал гнев его на Иова за то, что он оправдывал себя больше, нежели Бога,
\vs Job 32:3 а на трех друзей его воспылал гнев его за то, что они не нашли, что отвечать, а между тем обвиняли Иова.
\vs Job 32:4 Елиуй ждал, пока Иов говорил, потому что они летами были старше его.
\vs Job 32:5 Когда же Елиуй увидел, что нет ответа в устах тех трех мужей, тогда воспылал гнев его.
\vs Job 32:6 И отвечал Елиуй, сын Варахиилов, Вузитянин, и сказал: я молод летами, а вы~--- старцы; поэтому я робел и боялся объявлять вам мое мнение.
\vs Job 32:7 Я говорил сам себе: пусть говорят дни, и многолетие поучает мудрости.
\vs Job 32:8 Но дух в человеке и дыхание Вседержителя дает ему разумение.
\vs Job 32:9 Не многолетние \bibemph{только} мудры, и не старики разумеют правду.
\vs Job 32:10 Поэтому я говорю: выслушайте меня, объявлю вам мое мнение и я.
\vs Job 32:11 Вот, я ожидал слов ваших,~--- вслушивался в суждения ваши, доколе вы придумывали, чт\acc{о} сказать.
\vs Job 32:12 Я пристально смотрел на вас, и вот никто из вас не обличает Иова и не отвечает на слова его.
\vs Job 32:13 Не скажите: мы нашли мудрость: Бог опровергнет его, а не человек.
\vs Job 32:14 Если бы он обращал слова свои ко мне, то я не вашими речами отвечал бы ему.
\vs Job 32:15 Испугались, не отвечают более; перестали говорить.
\vs Job 32:16 И как я ждал, а они не говорят, остановились и не отвечают более,
\vs Job 32:17 то и я отвечу с моей стороны, объявлю мое мнение и я,
\vs Job 32:18 ибо я полон речами, и дух во мне теснит меня.
\vs Job 32:19 Вот, утроба моя, как вино неоткрытое: она готова прорваться, подобно новым мехам.
\vs Job 32:20 Поговорю, и будет легче мне; открою уста мои и отвечу.
\vs Job 32:21 На лице человека смотреть не буду и никакому человеку льстить не стану,
\vs Job 32:22 потому что я не умею льстить: сейчас убей меня, Творец мой.
\vs Job 33:1 Итак слушай, Иов, речи мои и внимай всем словам моим.
\vs Job 33:2 Вот, я открываю уста мои, язык мой говорит в гортани моей.
\vs Job 33:3 Слова мои от искренности моего сердца, и уста мои произнесут знание чистое.
\vs Job 33:4 Дух Божий создал меня, и дыхание Вседержителя дало мне жизнь.
\vs Job 33:5 Если можешь, отвечай мне и стань передо мною.
\vs Job 33:6 Вот я, по желанию твоему, вместо Бога. Я образован также из брения;
\vs Job 33:7 поэтому страх передо мною не может смутить тебя, и рука моя не будет тяжела для тебя.
\vs Job 33:8 Ты говорил в уши мои, и я слышал звук слов:
\vs Job 33:9 чист я, без порока, невинен я, и нет во мне неправды;
\vs Job 33:10 а Он нашел обвинение против меня и считает меня Своим противником;
\vs Job 33:11 поставил ноги мои в колоду, наблюдает за всеми путями моими.
\vs Job 33:12 Вот в этом ты неправ, отвечаю тебе, потому что Бог выше человека.
\vs Job 33:13 Для чего тебе состязаться с Ним? Он не дает отчета ни в каких делах Своих.
\vs Job 33:14 Бог говорит однажды и, если того не заметят, в другой раз:
\vs Job 33:15 во сне, в ночном видении, когда сон находит на людей, во время дремоты на ложе.
\vs Job 33:16 Тогда Он открывает у человека ухо и запечатлевает Свое наставление,
\vs Job 33:17 чтобы отвести человека от какого-либо предприятия и удалить от него гордость,
\vs Job 33:18 чтобы отвести душу его от пропасти и жизнь его от поражения мечом.
\vs Job 33:19 Или он вразумляется болезнью на ложе своем и жестокою болью во всех костях своих,~---
\vs Job 33:20 и жизнь его отвращается от хлеба и душа его от любимой пищи.
\vs Job 33:21 Плоть на нем пропадает, так что ее не видно, и показываются кости его, которых не было видно.
\vs Job 33:22 И душа его приближается к могиле и жизнь его~--- к смерти.
\vs Job 33:23 Если есть у него Ангел-наставник, один из тысячи, чтобы показать человеку прямой \bibemph{путь} его,~---
\vs Job 33:24 \bibemph{Бог} умилосердится над ним и скажет: освободи его от могилы; Я нашел умилостивление.
\vs Job 33:25 Тогда тело его сделается свеж\acc{е}е, нежели в молодости; он возвратится к дням юности своей.
\vs Job 33:26 Будет молиться Богу, и Он~--- милостив к нему; с радостью взирает на лице его и возвращает человеку праведность его.
\vs Job 33:27 Он будет смотреть на людей и говорить: грешил я и превращал правду, и не воздано мне;
\vs Job 33:28 Он освободил душу мою от могилы, и жизнь моя видит свет.
\vs Job 33:29 Вот, все это делает Бог два-три раза с человеком,
\vs Job 33:30 чтобы отвести душу его от могилы и просветить его светом живых.
\vs Job 33:31 Внимай, Иов, слушай меня, молчи, и я буду говорить.
\vs Job 33:32 Если имеешь, что сказать, отвечай; говори, потому что я желал бы твоего оправдания;
\vs Job 33:33 если же нет, то слушай меня: молчи, и я научу тебя мудрости.
\vs Job 34:1 И продолжал Елиуй и сказал:
\vs Job 34:2 выслушайте, мудрые, речь мою, и приклоните ко мне ухо, рассудительные!
\vs Job 34:3 Ибо ухо разбирает слова, как гортань различает вкус в пище.
\vs Job 34:4 Установим между собою рассуждение и распознаем, что хорошо.
\vs Job 34:5 Вот, Иов сказал: я прав, но Бог лишил меня суда.
\vs Job 34:6 Должен ли я лгать на правду мою? Моя рана неисцелима без вины.
\vs Job 34:7 Есть ли такой человек, как Иов, который пьет глумление, как воду,
\vs Job 34:8 вступает в сообщество с делающими беззаконие и ходит с людьми нечестивыми?
\vs Job 34:9 Потому что он сказал: нет пользы для человека в благоугождении Богу.
\vs Job 34:10 Итак послушайте меня, мужи мудрые! Не может быть у Бога неправда или у Вседержителя неправосудие,
\vs Job 34:11 ибо Он по делам человека поступает с ним и по путям мужа воздает ему.
\vs Job 34:12 Истинно, Бог не делает неправды и Вседержитель не извращает суда.
\vs Job 34:13 Кто кроме Его промышляет о земле? И кто управляет всею вселенною?
\vs Job 34:14 Если бы Он обратил сердце Свое к Себе и взял к Себе дух ее и дыхание ее,~---
\vs Job 34:15 вдруг погибла бы всякая плоть, и человек возвратился бы в прах.
\vs Job 34:16 Итак, если ты имеешь разум, то слушай это и внимай словам моим.
\vs Job 34:17 Ненавидящий правду может ли владычествовать? И можешь ли ты обвинить Всеправедного?
\vs Job 34:18 Можно ли сказать царю: ты~--- нечестивец, и князьям: вы~--- беззаконники?
\vs Job 34:19 Но Он не смотрит и на лица князей и не предпочитает богатого бедному, потому что все они дело рук Его.
\vs Job 34:20 Внезапно они умирают; среди ночи народ возмутится, и они исчезают; и сильных изгоняют не силою.
\vs Job 34:21 Ибо очи Его над путями человека, и Он видит все шаги его.
\vs Job 34:22 Нет тьмы, ни тени смертной, где могли бы укрыться делающие беззаконие.
\vs Job 34:23 Потому Он уже не требует от человека, чтобы шел на суд с Богом.
\vs Job 34:24 Он сокрушает сильных без исследования и поставляет других на их места;
\vs Job 34:25 потому что Он делает известными дела их и низлагает их ночью, и они истребляются.
\vs Job 34:26 Он поражает их, как беззаконных людей, пред глазами других,
\vs Job 34:27 за то, что они отвратились от Него и не уразумели всех путей Его,
\vs Job 34:28 так что дошел до Него вопль бедных, и Он услышал стенание угнетенных.
\vs Job 34:29 Дарует ли Он тишину, кто может возмутить? скрывает ли Он лице Свое, кто может увидеть Его? Будет ли это для народа, или для одного человека,
\vs Job 34:30 чтобы не царствовал лицемер к соблазну народа.
\vs Job 34:31 К Богу должно говорить: я потерпел, больше не буду грешить.
\vs Job 34:32 А чего я не знаю, Ты научи меня; и если я сделал беззаконие, больше не буду.
\vs Job 34:33 По твоему ли \bibemph{рассуждению} Он должен воздавать? И как ты отвергаешь, то тебе следует избирать, а не мне; говори, что знаешь.
\vs Job 34:34 Люди разумные скажут мне, и муж мудрый, слушающий меня:
\vs Job 34:35 Иов не умно говорит, и слова его не со смыслом.
\vs Job 34:36 Я желал бы, чтобы Иов вполне был испытан, по ответам его, свойственным людям нечестивым.
\vs Job 34:37 Иначе он ко греху своему прибавит отступление, будет рукоплескать между нами и еще больше наговорит против Бога.
\vs Job 35:1 И продолжал Елиуй и сказал:
\vs Job 35:2 считаешь ли ты справедливым, что сказал: я правее Бога?
\vs Job 35:3 Ты сказал: что пользы мне? и какую прибыль я имел бы пред тем, как если бы я и грешил?
\vs Job 35:4 Я отвечу тебе и твоим друзьям с тобою:
\vs Job 35:5 взгляни на небо и смотри; воззри на облака, они выше тебя.
\vs Job 35:6 Если ты грешишь, что делаешь ты Ему? и если преступления твои умножаются, что причиняешь ты Ему?
\vs Job 35:7 Если ты праведен, что даешь Ему? или что получает Он от руки твоей?
\vs Job 35:8 Нечестие твое относится к человеку, как ты, и праведность твоя к сыну человеческому.
\vs Job 35:9 От множества притеснителей стонут притесняемые, и от руки сильных вопиют.
\vs Job 35:10 Но никто не говорит: где Бог, Творец мой, Который дает песни в ночи,
\vs Job 35:11 Который научает нас более, нежели скотов земных, и вразумляет нас более, нежели птиц небесных?
\vs Job 35:12 Там они вопиют, и Он не отвечает им, по причине гордости злых людей.
\vs Job 35:13 Но неправда, что Бог не слышит и Вседержитель не взирает на это.
\vs Job 35:14 Хотя ты сказал, что ты не видишь Его, но суд пред Ним, и~--- жди его.
\vs Job 35:15 Но ныне, потому что гнев Его не посетил его и он не познал его во всей строгости,
\vs Job 35:16 Иов и открыл легкомысленно уста свои и безрассудно расточает слова.
\vs Job 36:1 И продолжал Елиуй и сказал:
\vs Job 36:2 подожди меня немного, и я покажу тебе, что я имею еще что сказать за Бога.
\vs Job 36:3 Начну мои рассуждения издалека и воздам Создателю моему справедливость,
\vs Job 36:4 потому что слова мои точно не ложь: пред тобою~--- совершенный в познаниях.
\vs Job 36:5 Вот, Бог могуществен и не презирает сильного крепостью сердца;
\vs Job 36:6 Он не поддерживает нечестивых и воздает должное угнетенным;
\vs Job 36:7 Он не отвращает очей Своих от праведников, но с царями навсегда посаждает их на престоле, и они возвышаются.
\vs Job 36:8 Если же они окованы цепями и содержатся в узах бедствия,
\vs Job 36:9 то Он указывает им на дела их и на беззакония их, потому что умножились,
\vs Job 36:10 и открывает их ухо для вразумления и говорит им, чтоб они отстали от нечестия.
\vs Job 36:11 Если послушают и будут служить Ему, то проведут дни свои в благополучии и лета свои в радости;
\vs Job 36:12 если же не послушают, то погибнут от стрелы и умрут в неразумии.
\vs Job 36:13 Но лицемеры питают в сердце гнев и не взывают к Нему, когда Он заключает их в узы;
\vs Job 36:14 поэтому душа их умирает в молодости и жизнь их с блудниками.
\vs Job 36:15 Он спасает бедного от беды его и в угнетении открывает ухо его.
\vs Job 36:16 И тебя вывел бы Он из тесноты на простор, где нет стеснения, и поставляемое на стол твой было бы наполнено туком;
\vs Job 36:17 но ты преисполнен суждениями нечестивых: суждение и осуждение~--- близки.
\vs Job 36:18 Да не поразит тебя гнев \bibemph{Божий} наказанием! Большой выкуп не спасет тебя.
\vs Job 36:19 Даст ли Он какую цену твоему богатству? Нет,~--- ни золоту и никакому сокровищу.
\vs Job 36:20 Не желай той ночи, когда народы истребляются на своем месте.
\vs Job 36:21 Берегись, не склоняйся к нечестию, которое ты предпочел страданию.
\vs Job 36:22 Бог высок могуществом Своим, и кто такой, как Он, наставник?
\vs Job 36:23 Кто укажет Ему путь Его; кто может сказать: Ты поступаешь несправедливо?
\vs Job 36:24 Помни о том, чтобы превозносить дела его, которые люди видят.
\vs Job 36:25 Все люди могут видеть их; человек может усматривать их издали.
\vs Job 36:26 Вот, Бог велик, и мы не можем познать Его; число лет Его неисследимо.
\vs Job 36:27 Он собирает капли воды; они во множестве изливаются дождем:
\vs Job 36:28 из облаков каплют и изливаются обильно на людей.
\vs Job 36:29 Кто может также постигнуть протяжение облаков, треск шатра Его?
\vs Job 36:30 Вот, Он распространяет над ним свет Свой и покрывает дно моря.
\vs Job 36:31 Оттуда Он судит народы, дает пищу в изобилии.
\vs Job 36:32 Он сокрывает в дланях Своих молнию и повелевает ей, кого разить.
\vs Job 36:33 Треск ее дает знать о ней; скот также чувствует происходящее.
\vs Job 37:1 И от сего трепещет сердце мое и подвиглось с места своего.
\vs Job 37:2 Слушайте, слушайте голос Его и гром, исходящий из уст Его.
\vs Job 37:3 Под всем небом раскат его, и блистание его~--- до краев земли.
\vs Job 37:4 За ним гремит глас; гремит Он гласом величества Своего и не останавливает его, когда голос Его услышан.
\vs Job 37:5 Дивно гремит Бог гласом Своим, делает дела великие, для нас непостижимые.
\vs Job 37:6 Ибо снегу Он говорит: будь на земле; равно мелкий дождь и большой дождь в Его власти.
\vs Job 37:7 Он полагает печать на руку каждого человека, чтобы все люди знали дело Его.
\vs Job 37:8 Тогда зверь уходит в убежище и остается в своих логовищах.
\vs Job 37:9 От юга приходит буря, от севера~--- стужа.
\vs Job 37:10 От дуновения Божия происходит лед, и поверхность воды сжимается.
\vs Job 37:11 Также влагою Он наполняет тучи, и облака сыплют свет Его,
\vs Job 37:12 и они направляются по намерениям Его, чтоб исполнить то, что Он повелит им на лице обитаемой земли.
\vs Job 37:13 Он повелевает им идти или для наказания, или в благоволение, или для помилования.
\vs Job 37:14 Внимай сему, Иов; стой и разумевай чудные дела Божии.
\vs Job 37:15 Знаешь ли, как Бог располагает ими и повелевает свету блистать из облака Своего?
\vs Job 37:16 Разумеешь ли равновесие облаков, чудное дело Совершеннейшего в знании?
\vs Job 37:17 Как нагревается твоя одежда, когда Он успокаивает землю от юга?
\vs Job 37:18 Ты ли с Ним распростер небеса, твердые, как литое зеркало?
\vs Job 37:19 Научи нас, что сказать Ему? Мы в этой тьме ничего не можем сообразить.
\vs Job 37:20 Будет ли возвещено Ему, что я говорю? Сказал ли кто, что сказанное доносится Ему?
\vs Job 37:21 Теперь не видно яркого света в облаках, но пронесется ветер и расчистит их.
\vs Job 37:22 Светлая погода приходит от севера, и окрест Бога страшное великолепие.
\vs Job 37:23 Вседержитель! мы не постигаем Его. Он велик силою, судом и полнотою правосудия. Он \bibemph{никого} не угнетает.
\vs Job 37:24 Посему да благоговеют пред Ним люди, и да трепещут пред Ним все мудрые сердцем!
\vs Job 38:1 [Когда Елиуй перестал говорить,] Господь отвечал Иову из бури и сказал:
\vs Job 38:2 кто сей, омрачающий Провидение словами без смысла?
\vs Job 38:3 Препояшь ныне чресла твои, как муж: Я буду спрашивать тебя, и ты объясняй Мне:
\vs Job 38:4 где был ты, когда Я полагал основания земли? Скажи, если знаешь.
\vs Job 38:5 Кто положил меру ей, если знаешь? или кто протягивал по ней вервь?
\vs Job 38:6 На чем утверждены основания ее, или кто положил краеугольный камень ее,
\vs Job 38:7 при общем ликовании утренних звезд, когда все сыны Божии восклицали от радости?
\vs Job 38:8 Кто затворил море воротами, когда оно исторглось, вышло как бы из чрева,
\vs Job 38:9 когда Я облака сделал одеждою его и мглу пеленами его,
\vs Job 38:10 и утвердил ему Мое определение, и поставил запоры и ворота,
\vs Job 38:11 и сказал: доселе дойдешь и не перейдешь, и здесь предел надменным волнам твоим?
\vs Job 38:12 Давал ли ты когда в жизни своей приказания утру и указывал ли заре место ее,
\vs Job 38:13 чтобы она охватила края земли и стряхнула с нее нечестивых,
\vs Job 38:14 чтобы \bibemph{земля} изменилась, как глина под печатью, и стала, как разноцветная одежда,
\vs Job 38:15 и чтобы отнялся у нечестивых свет их и дерзкая рука их сокрушилась?
\vs Job 38:16 Нисходил ли ты во глубину моря и входил ли в исследование бездны?
\vs Job 38:17 Отворялись ли для тебя врата смерти, и видел ли ты врата тени смертной?
\vs Job 38:18 Обозрел ли ты широту земли? Объясни, если знаешь все это.
\vs Job 38:19 Где путь к жилищу света, и где место тьмы?
\vs Job 38:20 Ты, конечно, доходил до границ ее и знаешь стези к дому ее.
\vs Job 38:21 Ты знаешь это, потому что ты был уже тогда рожден, и число дней твоих очень велико.
\vs Job 38:22 Входил ли ты в хранилища снега и видел ли сокровищницы града,
\vs Job 38:23 которые берегу Я на время смутное, на день битвы и войны?
\vs Job 38:24 По какому пути разливается свет и разносится восточный ветер по земле?
\vs Job 38:25 Кто проводит протоки для излияния воды и путь для громоносной молнии,
\vs Job 38:26 чтобы шел дождь на землю безлюдную, на пустыню, где нет человека,
\vs Job 38:27 чтобы насыщать пустыню и степь и возбуждать травные зародыши к возрастанию?
\vs Job 38:28 Есть ли у дождя отец? или кто рождает капли росы?
\vs Job 38:29 Из чьего чрева выходит лед, и иней небесный,~--- кто рождает его?
\vs Job 38:30 Воды, как камень, крепнут, и поверхность бездны замерзает.
\vs Job 38:31 Можешь ли ты связать узел Хима и разрешить узы Кесиль?
\vs Job 38:32 Можешь ли выводить созвездия в свое время и вести Ас с ее детьми?
\vs Job 38:33 Знаешь ли ты уставы неба, можешь ли установить господство его на земле?
\vs Job 38:34 Можешь ли возвысить голос твой к облакам, чтобы вода в обилии покрыла тебя?
\vs Job 38:35 Можешь ли посылать молнии, и пойдут ли они и скажут ли тебе: вот мы?
\vs Job 38:36 Кто вложил мудрость в сердце, или кто дал смысл разуму?
\vs Job 38:37 Кто может расчислить облака своею мудростью и удержать сосуды неба,
\vs Job 38:38 когда пыль обращается в грязь и глыбы слипаются?
\vs Job 38:39 Ты ли ловишь добычу львице и насыщаешь молодых львов,
\vs Job 38:40 когда они лежат в берлогах или покоятся под тенью в засаде?
\vs Job 38:41 Кто приготовляет в\acc{о}рону корм его, когда птенцы его кричат к Богу, бродя без пищи?
\vs Job 39:1 Знаешь ли ты время, когда рождаются дикие козы на скалах, и замечал ли роды ланей?
\vs Job 39:2 можешь ли расчислить месяцы беременности их? и знаешь ли время родов их?
\vs Job 39:3 Они изгибаются, рождая детей своих, выбрасывая свои ноши;
\vs Job 39:4 дети их приходят в силу, растут на поле, уходят и не возвращаются к ним.
\vs Job 39:5 Кто пустил дикого осла на свободу, и кто разрешил узы онагру,
\vs Job 39:6 которому степь Я назначил домом и солончаки~--- жилищем?
\vs Job 39:7 Он посмевается городскому многолюдству и не слышит криков погонщика,
\vs Job 39:8 по горам ищет себе пищи и гоняется за всякою зеленью.
\vs Job 39:9 Захочет ли единорог служить тебе и переночует ли у яслей твоих?
\vs Job 39:10 Можешь ли веревкою привязать единорога к борозде, и станет ли он боронить за тобою поле?
\vs Job 39:11 Понадеешься ли на него, потому что у него сила велика, и предоставишь ли ему работу твою?
\vs Job 39:12 Поверишь ли ему, что он семена твои возвратит и сложит на гумно твое?
\vs Job 39:13 Ты ли дал красивые крылья павлину и перья и пух страусу?
\vs Job 39:14 Он оставляет яйца свои на земле, и на песке согревает их,
\vs Job 39:15 и забывает, что нога может раздавить их и полевой зверь может растоптать их;
\vs Job 39:16 он жесток к детям своим, как бы не своим, и не опасается, что труд его будет напрасен;
\vs Job 39:17 потому что Бог не дал ему мудрости и не уделил ему смысла;
\vs Job 39:18 а когда поднимется на высоту, посмевается коню и всаднику его.
\vs Job 39:19 Ты ли дал коню силу и облек шею его гривою?
\vs Job 39:20 Можешь ли ты испугать его, как саранчу? Храпение ноздрей его~--- ужас;
\vs Job 39:21 роет ногою землю и восхищается силою; идет навстречу оружию;
\vs Job 39:22 он смеется над опасностью и не робеет и не отворачивается от меча;
\vs Job 39:23 колчан звучит над ним, сверкает копье и дротик;
\vs Job 39:24 в порыве и ярости он глотает землю и не может стоять при звуке трубы;
\vs Job 39:25 при трубном звуке он издает голос: гу! гу! и издалека чует битву, громкие голоса вождей и крик.
\vs Job 39:26 Твоею ли мудростью летает ястреб и направляет крылья свои на полдень?
\vs Job 39:27 По твоему ли слову возносится орел и устрояет на высоте гнездо свое?
\vs Job 39:28 Он живет на скале и ночует на зубце утесов и на местах неприступных;
\vs Job 39:29 оттуда высматривает себе пищу: глаза его смотрят далеко;
\vs Job 39:30 птенцы его пьют кровь, и где труп, там и он.
\vs Job 39:31 И продолжал Господь и сказал Иову:
\vs Job 39:32 будет ли состязающийся со Вседержителем еще учить? Обличающий Бога пусть отвечает Ему.
\vs Job 39:33 И отвечал Иов Господу и сказал:
\vs Job 39:34 вот, я ничтожен; что буду я отвечать Тебе? Руку мою полагаю на уста мои.
\vs Job 39:35 Однажды я говорил,~--- теперь отвечать не буду, даже дважды, но более не буду.
\vs Job 40:1 И отвечал Господь Иову из бури и сказал:
\vs Job 40:2 препояшь, как муж, чресла твои: Я буду спрашивать тебя, а ты объясняй Мне.
\vs Job 40:3 Ты хочешь ниспровергнуть суд Мой, обвинить Меня, чтобы оправдать себя?
\vs Job 40:4 Такая ли у тебя мышца, как у Бога? И можешь ли возгреметь голосом, как Он?
\vs Job 40:5 Укрась же себя величием и славою, облекись в блеск и великолепие;
\vs Job 40:6 излей ярость гнева твоего, посмотри на все гордое и смири его;
\vs Job 40:7 взгляни на всех высокомерных и унизь их, и сокруши нечестивых на местах их;
\vs Job 40:8 зарой всех их в землю и лица их покрой тьмою.
\vs Job 40:9 Тогда и Я признаю, что десница твоя может спасать тебя.
\vs Job 40:10 Вот бегемот, которого Я создал, как и тебя; он ест траву, как вол;
\vs Job 40:11 вот, его сила в чреслах его и крепость его в мускулах чрева его;
\vs Job 40:12 поворачивает хвостом своим, как кедром; жилы же на бедрах его переплетены;
\vs Job 40:13 ноги у него, как медные трубы; кости у него, как железные прутья;
\vs Job 40:14 это~--- верх путей Божиих; только Сотворивший его может приблизить к нему меч Свой;
\vs Job 40:15 горы приносят ему пищу, и там все звери полевые играют;
\vs Job 40:16 он ложится под тенистыми деревьями, под кровом тростника и в болотах;
\vs Job 40:17 тенистые дерева покрывают его своею тенью; ивы при ручьях окружают его;
\vs Job 40:18 вот, он пьет из реки и не торопится; остается спокоен, хотя бы Иордан устремился ко рту его.
\vs Job 40:19 Возьмет ли кто его в глазах его и проколет ли ему нос багром?
\vs Job 40:20 Можешь ли ты удою вытащить левиафана и веревкою схватить за язык его?
\vs Job 40:21 вденешь ли кольцо в ноздри его? проколешь ли иглою челюсть его?
\vs Job 40:22 будет ли он много умолять тебя и будет ли говорить с тобою кротко?
\vs Job 40:23 сделает ли он договор с тобою, и возьмешь ли его навсегда себе в рабы?
\vs Job 40:24 станешь ли забавляться им, как птичкою, и свяжешь ли его для девочек твоих?
\vs Job 40:25 будут ли продавать его товарищи ловли, разделят ли его между Хананейскими купцами?
\vs Job 40:26 можешь ли пронзить кожу его копьем и голову его рыбачьею острогою?
\vs Job 40:27 Клади на него руку твою, и помни о борьбе: вперед не будешь.
\vs Job 41:1 Надежда тщетна: не упадешь ли от одного взгляда его?
\vs Job 41:2 Нет столь отважного, который осмелился бы потревожить его; кто же может устоять перед Моим лицем?
\vs Job 41:3 Кто предварил Меня, чтобы Мне воздавать ему? под всем небом все Мое.
\vs Job 41:4 Не умолчу о членах его, о силе и красивой соразмерности их.
\vs Job 41:5 Кто может открыть верх одежды его, кто подойдет к двойным челюстям его?
\vs Job 41:6 Кто может отворить двери лица его? круг зубов его~--- ужас;
\vs Job 41:7 крепкие щиты его~--- великолепие; они скреплены как бы твердою печатью;
\vs Job 41:8 один к другому прикасается близко, так что и воздух не проходит между ними;
\vs Job 41:9 один с другим лежат плотно, сцепились и не раздвигаются.
\vs Job 41:10 От его чихания показывается свет; глаза у него как ресницы зари;
\vs Job 41:11 из пасти его выходят пламенники, выскакивают огненные искры;
\vs Job 41:12 из ноздрей его выходит дым, как из кипящего горшка или котла.
\vs Job 41:13 Дыхание его раскаляет угли, и из пасти его выходит пламя.
\vs Job 41:14 На шее его обитает сила, и перед ним бежит ужас.
\vs Job 41:15 Мясистые части тела его сплочены между собою твердо, не дрогнут.
\vs Job 41:16 Сердце его твердо, как камень, и жестко, как нижний жернов.
\vs Job 41:17 Когда он поднимается, силачи в страхе, совсем теряются от ужаса.
\vs Job 41:18 Меч, коснувшийся его, не устоит, ни копье, ни дротик, ни латы.
\vs Job 41:19 Железо он считает за солому, медь~--- за гнилое дерево.
\vs Job 41:20 Дочь лука не обратит его в бегство; пращные камни обращаются для него в плеву.
\vs Job 41:21 Булава считается у него за соломину; свисту дротика он смеется.
\vs Job 41:22 Под ним острые камни, и он на острых камнях лежит в грязи.
\vs Job 41:23 Он кипятит пучину, как котел, и море претворяет в кипящую мазь;
\vs Job 41:24 оставляет за собою светящуюся стезю; бездна кажется сединою.
\vs Job 41:25 Нет на земле подобного ему; он сотворен бесстрашным;
\vs Job 41:26 на все высокое смотрит смело; он царь над всеми сынами гордости.
\vs Job 42:1 И отвечал Иов Господу и сказал:
\vs Job 42:2 знаю, что Ты все можешь, и что намерение Твое не может быть остановлено.
\vs Job 42:3 Кто сей, омрачающий Провидение, ничего не разумея?~--- Так, я говорил о том, чего не разумел, о делах чудных для меня, которых я не знал.
\vs Job 42:4 Выслушай, \bibemph{взывал я}, и я буду говорить, и что буду спрашивать у Тебя, объясни мне.
\vs Job 42:5 Я слышал о Тебе слухом уха; теперь же мои глаза видят Тебя;
\vs Job 42:6 поэтому я отрекаюсь и раскаиваюсь в прахе и пепле.
\rsbpar\vs Job 42:7 И было после того, как Господь сказал слова те Иову, сказал Господь Елифазу Феманитянину: горит гнев Мой на тебя и на двух друзей твоих за то, что вы говорили о Мне не так верно, как раб Мой Иов.
\vs Job 42:8 Итак возьмите себе семь тельцов и семь овнов и пойдите к рабу Моему Иову и принесите за себя жертву; и раб Мой Иов помолится за вас, ибо только лице его Я приму, дабы не отвергнуть вас за то, что вы говорили о Мне не так верно, как раб Мой Иов.
\vs Job 42:9 И пошли Елифаз Феманитянин и Вилдад Савхеянин и Софар Наамитянин, и сделали так, как Господь повелел им,~--- и Господь принял лице Иова.
\rsbpar\vs Job 42:10 И возвратил Господь потерю Иова, когда он помолился за друзей своих; и дал Господь Иову вдвое больше того, что он имел прежде.
\vs Job 42:11 Тогда пришли к нему все братья его и все сестры его и все прежние знакомые его, и ели с ним хлеб в доме его, и тужили с ним, и утешали его за все зло, которое Господь навел на него, и дали ему каждый по кесите и по золотому кольцу.
\rsbpar\vs Job 42:12 И благословил Бог последние дни Иова более, нежели прежние: у него было четырнадцать тысяч мелкого скота, шесть тысяч верблюдов, тысяча пар волов и тысяча ослиц.
\vs Job 42:13 И было у него семь сыновей и три дочери.
\vs Job 42:14 И нарек он имя первой Емима, имя второй~--- Кассия, а имя третьей~--- Керенгаппух.
\vs Job 42:15 И не было на всей земле таких прекрасных женщин, как дочери Иова, и дал им отец их наследство между братьями их.
\vs Job 42:16 После того Иов жил сто сорок лет, и видел сыновей своих и сыновей сыновних до четвертого рода;
\vs Job 42:17 и умер Иов в старости, насыщенный днями.\fns{В Славянской Библии к книге Иова имеется следующее добавление: <<Написано, что он опять восстанет с теми, коих воскресит Господь. О нем толкуется в Сирской книге, что жил он в земле Авситидийской на пределах Идумеи и Аравии: прежде же было имя ему Иовав. Взяв жену Аравитянку, родил сына, которому имя Еннон. Происходил он от отца Зарефа, сынов Исавовых сын, матери же Воссоры, так что был он пятым от Авраама. И сии цари, царствовавшие в Едоме, какою страною и он обладал: первый Валак, сын Веора, и имя городу его Деннава; после же Валака Иовав, называемый Иовом; после сего Ассом, игемон из Феманитской страны; после него Адад, сын Варада, поразивший Мадиама на поле Моава,~--- и имя городу его Гефем. Пришедшие же к нему друзья, Елифаз (сын Софана) от сынов Исавовых, царь Феманский, Валдад (сын Амнона Ховарского) савхейский властитель, Софар Минейский царь. (Феман сын Елифаза, игемон Идумеи. О нем говорится в книге Сирской, что жил в земле Авситидийской, около берегов Евфрата; прежде имя его было Иовав, отец же его был Зареф, от востока солнца.)>>.}

\bibbookdescr{1Cl}{
  inline={Послание Климента к Коринфянам},
  toc={1-е Климента},
  bookmark={1-е Климента},
  header={1-е Климента},
  abbr={1~Кли}
}
\vs 1Cl 1:1
Церковь Божья,
находящаяся в Риме, Церкви Божьей, находящейся в Коринфе, званным, освященным
по воле Божьей через Господа нашего Иисуса Христа.
\vs 1Cl 1:2
Благодать вам и мир от
Всемогущего Бога чрез Иисуса Христа да умножится.
\vs 1Cl 1:3
Внезапные и одно за другим случившиеся с нами
несчастия и бедствия были причиною того, братья, что поздно, как думается нам,
обратили мы внимание на спорные у вас дела, возлюбленные, и на неприличный и
чуждый избранникам Божьим, преступный и нечестивый мятеж,
\vs 1Cl 1:4
который немногие дерзкие и высокомерные люди
разожгли до такого безумия, что почтенное, славное и для всех достолюбезное
имя ваше подверглось великому поруганию.
\vs 1Cl 1:5
Ибо кто, побывавший у вас, не хвалил вашей,
всеми добродетелями исполненной и твердой, веры, не удивлялся вашему
трезвенному и кроткому во Христе благочестию, не превозносил вашей великой
щедрости в гостеприимстве, не прославлял вашего совершенного и верного знания?
\vs 1Cl 1:6
Во всем вы поступали нелицеприятно, ходили в
заповедях Божьих, повинуясь предводителям вашим и воздавая должную честь
старшим между вами.
\vs 1Cl 1:7
Юношам внушили скромность и благопристойность;
жен наставляли, чтобы они все делали с неукоризненной, честной и чистой
совестью, любя, как должно, своих мужей,
\vs 1Cl 1:8
и учили их, чтобы они, не выступая из правила
повиновения, пристойно распоряжались домашними делами, и вели себя вполне
целомудренно.

\vs 1Cl 2:1
Все вы были смиренны и
чужды тщеславия, любили более подчиняться, нежели повелевать, и давать, нежели
принимать.
\vs 1Cl 2:2
Довольствуясь тем, что Бог
дал вам на путь, и тщательно внимая словам Его, вы хранили их в глубине
сердца, и страдания Его были пред очами вашими.
\vs 1Cl 2:3
Таким образом всем был
дарован глубокий и прекрасный мир и ненасытное стремление делать добро: и на
всех было полное излияние Святого Духа.
\vs 1Cl 2:4
Полные святых желаний, с
искренним усердием и благочестивым упованием, вы простирали руки свои ко
Всемогущему Богу и умоляли Его быть милосердым, если вы в чем невольно
погрешили.
\vs 1Cl 2:5
День и ночь подвигом вашим
было попечение о всем братстве, чтобы число избранных Его спасалось в
добродушии и единомыслии.
\vs 1Cl 2:6
Вы были искренни,
чистосердечны, и не помнили зла друг на друге.
\vs 1Cl 2:7
Всякий мятеж и всякое
разделение было вам противно.
\vs 1Cl 2:8
Вы плакали о проступках
ближних; их недостатки считали собственными.
\vs 1Cl 2:9
Не скучали делать добро,
готовые на всякое дело доброе.
\vs 1Cl 2:10
Будучи украшены такою
добродетельною и почтенною жизнью, вы все совершали в страхе Господа: Его
повеления и заповеди были написаны на скрижалях сердца вашего.

\vs 1Cl 3:1
Вся слава и широта дана
была вам, и исполнилось, что написано: он ел и пил, разжирел и растолстел, и
сделался непокорен возлюбленный.
\vs 1Cl 3:2
А отсюда ревность и
зависть, вражда и раздор, гонение и возмущение, война и плен.
\vs 1Cl 3:3
Таким образом, люди
бесчестные восстали против почтенных, бесславные против славных, глупые против
разумных, молодые против старших.
\vs 1Cl 3:4
Поэтому удалились правда и
мир,~--- так как всякий оставил страх Божий, сделался туп в вере Его, не ходит
по правилам заповедей Его, и не ведет жизни, достойной Христа,
\vs 1Cl 3:5
но каждый последовал злым
своим пожеланиям, допустив снова беззаконную и нечестивую зависть, чрез
которую и смерть вошла в мир.

\vs 1Cl 4:1
Ибо так написано: и было
спустя несколько дней, приносил Каин от плодов земли жертву Богу: и также
Авель приносил от первородных овец и от туков их;
\vs 1Cl 4:2
и призрел Бог на Авеля и
на дары Его; на Каина же и на жертвы его не посмотрел.
\vs 1Cl 4:3
И весьма опечалился Каин,
и поникло лицо его.
\vs 1Cl 4:4
И сказал Бог Каину: что ты
стал печален, и от чего поникло лицо твое? Не согрешил ли ты, если ты
правильно принес, но неправильно разделил?
\vs 1Cl 4:5
Успокойся. К тебе
обращение его и ты будешь обладать тем.
\vs 1Cl 4:6
И сказал Каин Авелю, брату
своему: пойдем в поле;
\vs 1Cl 4:7
и было в то время, как они
находились в поле, возстал Каин на Авеля, брата своего, и убил его.
\vs 1Cl 4:8
Видите, братья, ревность и
зависть произвели братоубийство.
\vs 1Cl 4:9
По причине зависти отец
наш Иаков убежал от лица Исава, брата своего.
\vs 1Cl 4:10
Зависть была причиною,
что Иосиф гоним был на смерть и подвергся рабству.
\vs 1Cl 4:11
Зависть принудила Моисея
бежать от лица фараона, царя Египетского, когда услышал он от единоплеменника
своего:
\vs 1Cl 4:12
кто поставил тебя
решителем или судьею над нами? Не хочешь ли убить меня, как убил вчера
египтянина?
\vs 1Cl 4:13
За зависть Аарон и
Мариамь жили вне стана.
\vs 1Cl 4:14
Зависть Дафана и Авирона
живых низвела в ад за то, что они возмутились против Моисея, служителя
Божьего.
\vs 1Cl 4:15
По причине зависти Давид
не только подвергся ненависти иноплеменных, но был гоним и от Саула, царя
Израильского.

\vs 1Cl 5:1
Но оставив древние
примеры, перейдем к ближайшим подвижникам: возьмем достойные примеры нашего
поколения.
\vs 1Cl 5:2
По ревности и зависти
величайшие и праведные столпы подверглись гонению и смерти. Представим пред
глазами нашими блаженных апостолов.
\vs 1Cl 5:3
Петр от беззаконной
зависти понес не одно, не два, но многие страдания, и таким образом
претерпевши мученичество, отошел в подобающее место славы.
\vs 1Cl 5:4
Павел, по причине зависти,
получил награду за терпение: он был в узах семь раз, был изгоняем, побиваем
камнями.
\vs 1Cl 5:5
Будучи проповедником на
Востоке и Западе, он приобрел благородную славу за свою веру, так как научил
весь мир правде,
\vs 1Cl 5:6
и доходил до границы
Запада, и мученически засвидетельствовал истину перед правителями.
\vs 1Cl 5:7
Так он переселился из
мира, и перешел в место святое, сделавшись величайшим образцом терпения.

\vs 1Cl 6:1
К этим мужам, свято
провождавшим жизнь, присовокупилось великое множество избранных,
\vs 1Cl 6:2
которые по причине зависти
претерпели многие поругания и мучения, и оставили среди нас прекрасный пример.
\vs 1Cl 6:3
Завистью гонимы были
женщины Данаида и Дирка;
\vs 1Cl 6:4
претерпевши тяжкие и
ужасные мучения, они прошли твердым путем веры, и, немощные телом, получили
славную награду.
\vs 1Cl 6:5
Зависть отлучала жен от
мужей и извращала слова праотца нашего Адама: вот ныне кость от костей моих,
и плоть от плоти моей.
\vs 1Cl 6:6
Зависть и раздор
ниспровергли великие города и совершенно истребили великие народы.

\vs 1Cl 7:1
Это, возлюбленные, пишем
мы не только для вашего наставления, но и для собственного напоминания;
\vs 1Cl 7:2
потому что мы находимся на
том же поприще, и тот же подвиг предлежит нам.
\vs 1Cl 7:3
Итак, оставим пустые и
суетные заботы, и обратимся к славному и досточтимому правилу святого звания
нашего.
\vs 1Cl 7:4
Будем смотреть на то, что
добро, что угодно и приятно Создателю нашему.
\vs 1Cl 7:5
Обратим внимание на кровь
Христа,~--- и увидим, как драгоценна пред Богом кровь Его, которая была пролита
для нашего спасения, и всему миру принесла благодать покаяния.
\vs 1Cl 7:6
Пройдем все поколения и
узнаем, что Господь в каждом поколении милостиво принимал покаяние желавших
обратиться к Нему.
\vs 1Cl 7:7
Ной проповедовал покаяние,
и послушавшиеся его спаслись.
\vs 1Cl 7:8
Иона возвестил ниневитянам
погибель, но они, раскаявшись в своих грехах, умилостивили Бога своими
молитвами и получили спасение, хотя были далеки от Бога.

\vs 1Cl 8:1
Служители благодати
Божьей по вдохновению Духа Святого говорили о покаянии;
\vs 1Cl 8:2
и Сам Владыка всего
говорил о покаянии с клятвою: жив Я, говорит ЯХВЕ, не хочу смерти грешника,
но покаяния;
\vs 1Cl 8:3
и присовокупил еще
следующую прекрасную мысль: дом Израилев, обратитесь от нечестия вашего.
\vs 1Cl 8:4
Скажи сынам народа Моего:
хотя грехи ваши будут простираться от земли до неба, и хотя будут краснее
червленицы и чернее власяницы,
\vs 1Cl 8:5
но если вы обратитесь ко
Мне от всего сердца, и скажете: Отец! то Я услышу вас как народ святой.
\vs 1Cl 8:6
И в другом месте так
говорит: омойтесь, и очиститесь, удалите лукавство из душ ваших пред очами
Моими, отстаньте от злодейств ваших;
\vs 1Cl 8:7
научитесь делать добро,
ищите правды, избавьте обиженного, рассудите о сироте, оправдайте вдовицу, и
придите и будем судиться, говорит ЯХВЕ:
\vs 1Cl 8:8
и если будут грехи ваши,
как пурпур, то убелю их как снег; и если будут как червленица, то убелю их как
волну; и если хотите и послушаете Меня, то будете наслаждаться благами земли;
\vs 1Cl 8:9
если же не хотите и не
послушаете Меня, то меч истребит вас: ибо уста ЯХВЕ сказали это.
\vs 1Cl 8:10
Итак, Он всех Своих
возлюбленных хочет сделать участниками покаяния, и утвердил это всемогущею
Своею волею!

\vs 1Cl 9:1
Поэтому покоримся
величественной и славной воле Его, и, оставив суетные дела, раздор и зависть,
ведущую к смерти, припадем и обратимся к Его милосердию, умоляя Его милость и
благость.
\vs 1Cl 9:2
Будем постоянно взирать на
тех, которые совершенно послужили величественной Его славе.
\vs 1Cl 9:3
Возьмем Еноха, который по
своему послушанию был найден праведным, и преставился, и не видели его смерти.
\vs 1Cl 9:4
Ной был найден верным, и
по своему служению проповедал миру обновление, и через него спас Господь
животных, согласно вошедших в ковчег.

\vs 1Cl 10:1
Авраам, названный другом,
найден верным по своему послушанию словам Божьим.
\vs 1Cl 10:2
Он из послушания вышел из
земли своей, и от родства своего, и из дома отца своего, чтобы оставить землю
малую, родство малосильное и небольшой дом, наследовал обетованию Божьему.
\vs 1Cl 10:3
Ибо так сказал ему:
удались из земли твоей, и от родства твоего, и из дома отца твоего в землю,
которую покажут тебе.
\vs 1Cl 10:4
И сделаю тебя народом
великим; и благословлю тебя, и возвеличу имя твое, и будешь благословен.
\vs 1Cl 10:5
И благословлю
благословляющих тебя, и проклинающих тебя прокляну, и благословятся в тебе все
племена земные.
\vs 1Cl 10:6
И опять по разделении его
с Лотом сказал ему Бог: подними глаза твои, и взгляни с места, где ты теперь,
к северу и югу, и к востоку и к морю: ибо всю землю, которую ты видишь, отдам
тебе и семени твоему на век.
\vs 1Cl 10:7
И сделаю семя твое как
песок земной: если кто может сосчитать песок земной, то и семя твое сочтется.
\vs 1Cl 10:8
И еще сказано: вывел Бог
Авраама и сказал ему: взгляни на небо и сосчитай звезды, если можешь счесть
их: так будет семя твое.
\vs 1Cl 10:9
И поверил Авраам Богу, и
это вменилось ему в праведность.
\vs 1Cl 10:10
За веру и гостеприимство
был дан ему в старости сын, но он из послушания принес его в жертву Богу на
одной из показанных от Него гор.

\vs 1Cl 11:1
За гостеприимство и
благочестие Лот вышел невредимым из Содома, тогда как вся окрестная страна
была наказана огнем и серою:
\vs 1Cl 11:2
и тем ясно показал
Господь, что Он не оставляет уповающих на Него; а уклоняющихся от Него
подвергает мучениям и казни.
\vs 1Cl 11:3
Ибо вышедшая с ним жена
его, так как была других мыслей и не согласна с ним, поставлена в знамение:
\vs 1Cl 11:4
она сделалась соляным
столбом, и даже до сего дня, чтобы все знали, что двоедушные и сомневающиеся о
могуществе Божьем служат примером суда и знамением для всех родов.

\vs 1Cl 12:1
За веру и гостеприимство
была спасена Раав блудница.
\vs 1Cl 12:2
Когда Иисусом Нуном были
посланы соглядатаи в Иерихон, и царь земли той узнал, что они пришли
осматривать его землю, то послал людей схватить их, чтобы схватив, предать их
смерти.
\vs 1Cl 12:3
Но гостеприимная Раав,
приняв их к себе, скрыла на верху своего дома в снопах льна.
\vs 1Cl 12:4
И когда от царя явились к
ней и говорили: люди пришли к тебе, соглядатаи земли нашей, выведи их, так
повелевает царь,
\vs 1Cl 12:5
то она отвечала: приходили
ко мне два человека, которых вы ищете, но они скоро ушли, и теперь в пути;
таким образом, она не показала их посланным.
\vs 1Cl 12:6
А мужам тем сказала: знаю
верно, что ЯХВЕ, Бог ваш, предаст вам этот город; потому что страх и трепет от
вас напал на живущих в нем. Итак, когда удастся вам взять его, сохраните меня
и дом отца моего.
\vs 1Cl 12:7
А они сказали ей: будет
так, как ты сказала нам. Как скоро узнаешь о приближении нашем, собери всех
своих под кровлю твою и будут целы, а кто будет найден вне дома, погибнет.
\vs 1Cl 12:8
Притом дали ей знак, чтобы
она свесила из дома своего красную вервь,~--- и тем показали, что всем верующим
и уповающим на Бога будет искупление кровью Господа.
\vs 1Cl 12:9
Видите, возлюбленные, в
этой жене была не только вера, но и пророчество.

\vs 1Cl 13:1
Итак, будем смиренны,
братья, отложив всякое надмение, гордость, неразумие и гнев, и будем
поступать, как написано.
\vs 1Cl 13:2
Ибо говорит Дух Святой:
да не похвалится мудрый мудростью своей, ни сильный силой своей, ни богатый
богатством своим,
\vs 1Cl 13:3
но хвалящийся пусть
хвалится ЯХВЕ, ища Его, и творя суд и правду.
\vs 1Cl 13:4
Особенно будем помнить
слова Господа Иисуса, которые изрек Он, научая кротости и великодушию.
\vs 1Cl 13:5
Он так сказал: милуйте,
чтобы быть помилованными, отпускайте, дабы вам было отпущено;
\vs 1Cl 13:6
как вы делаете, так вам
будут делать; как даете, так вам дано будет;
\vs 1Cl 13:7
как судите, так сами
судимы будете; как будете снисходить, так к вам будут снисходить; какою мерою
мерите, такою отмерится вам.
\vs 1Cl 13:8
Этой заповедью и этими
внушениями утвердим себя, чтобы ходить со смирением, повинуясь святым
повелениям Его.
\vs 1Cl 13:9
Ибо святое слово говорит:
на кого воззрю,~--- только на кроткого и тихого, и трепещущего слов Моих.

\vs 1Cl 14:1
Итак, праведное и святое
дело, братья, более повиноваться Богу, нежели последовать тем, которые в
надменности и кичливости стали предводителями презренной зависти.
\vs 1Cl 14:2
Ибо не малому вреду, а
напротив, подвергнемся великой опасности, если опрометчиво отдадим себя на
волю тех людей, которые подстрекают нас к раздору и мятежам, чтобы отвести нас
от добродетели.
\vs 1Cl 14:3
Будем снисходительны друг
к другу, как милосерд и благ Сотворивший нас;
\vs 1Cl 14:4
ибо написано: добрые
будут обитателями земли, и невинные останутся на ней; а беззаконные истребятся
с нее.
\vs 1Cl 14:5
И опять говорит Писание:
я видел нечестивого превозносящегося и возвышающегося, как кедры ливанские; и
прошел я мимо, и вот его уже не стало, и искал я места его, и не нашел.
\vs 1Cl 14:6
Храни невинность и
соблюдай правоту, потому что мирного человека ожидают добрые последствия.

\vs 1Cl 15:1
Итак, присоединимся к
тем, которые с благочестием хранят мир, а не к тем, которые с лицемерием
желают мира;
\vs 1Cl 15:2
ибо сказано где-то: эти
люди почитают Меня устами, сердце же их далеко отстоит от Меня.
\vs 1Cl 15:3
И в другом месте: устами
своими они благословляли, а сердцем своим проклинали.
\vs 1Cl 15:4
И еще сказано: возлюбили
Его устами своими, и языком своим солгали Ему; сердце же их не было право с
Ним; и они не были верны в завете Его.
\vs 1Cl 15:5
Да будут немы уста
льстивые, и да истребит Господь уста льстивых и язык велеречивый,~--- тех,
которые говорят: язык наш возвеличим, уста наши при нас: кто нам Господь?
\vs 1Cl 15:6
Ради бедствий нищих, и
воздыхания убогих, ныне Я восстану, говорит ЯХВЕ: послужу им спасением, и буду
поступать с ними честно.

\vs 1Cl 16:1
Ибо Христос принадлежит
смиренным, а не тем, которые возносятся над стадом Его.
\vs 1Cl 16:2
Жезл величия Божьего,
Господь наш Иисус Христос, не пришел в блеске великолепия и надменности, хотя
и мог бы, но смиренно, как сказал о Нем Дух Святой.
\vs 1Cl 16:3
Ибо говорит Он: ЯХВЕ, кто
верил слуху нашему? и рука ЯХВЕ кому открылась?
\vs 1Cl 16:4
Мы возвестили пред Ним; Он
как малый отрок, как корень в земле жаждущей,~--- не имеет ни вида, ни славы.
\vs 1Cl 16:5
И мы видели Его, и не имел
Он ни вида, ни красоты; но вид Его бесчестен, унижен более вида человеков: Он
человек в язве и страдании, умеющий переносить болезнь;
\vs 1Cl 16:6
потому что отвратилось
лицо Его,~--- было поругано и презрено. Он грехи наши носит и за нас страдает;
\vs 1Cl 16:7
а мы думали, что Он
праведно подвержен страданию, и язве, и мучению;
\vs 1Cl 16:8
но Он уязвлен был за грехи
наши и мучен был за беззакония наши; наказание мира нашего на Нем, чрез рану
Его мы исцелились.
\vs 1Cl 16:9
Все мы, как овцы,
заблудились; человек блуждал на пути своем, и ЯХВЕ предал Его за грехи наши.
\vs 1Cl 16:10
И Он, будучи мучим, не
отверзает уст: как овца был веден на заклание, и как агнец безгласный пред
стригущим его, так Он не отверзает уст Своих. За смирение Его с Него снят был
суд.
\vs 1Cl 16:11
Кто расскажет Его род,
когда жизнь Его берется от земли? За беззакония людей Моих Он идет на смерть.
\vs 1Cl 16:12
И потому помилую злых за
гроб Его, и богатых за смерть Его; ибо Он не сделал беззакония и обмана не
нашлось в устах Его.
\vs 1Cl 16:13
И ЯХВЕ угодно очистить
Его от язвы; если дадите жертву о грехе, то душа ваша узрит семя долговечное.
\vs 1Cl 16:14
И Господь хочет спасти
Его от страдания души Его, показать Ему свет и образовать разумом, и оправдать
праведного, который благодетельно послужил многим; и грехи их Он понесет.
\vs 1Cl 16:15
Поэтому Он будет обладать
многими и разделит добычи сильных,~--- за то, что предана была на смерть душа
Его и был причтен к злодеям; и Он уничтожил грехи многих и за беззакония их
был предан.
\vs 1Cl 16:16
И опять Он же говорит: Я
червь, а не человек, поношение человеков и уничижение людей.
\vs 1Cl 16:17
Все видящие Меня
издевались надо Мною, говорили устами и кивали головою, говоря: Он уповал на
Господа, пусть избавит Его и сохранит Его, так как благоволит к Нему.
\vs 1Cl 16:18
Видите возлюбленные,
какой дан нам образец: ибо если Господь так смирил Себя, то что должны делать
мы, которые чрез Него пришли под иго благодати Его?

\vs 1Cl 17:1
Будем подражать и тем,
которые скитались в козьих и овечьих кожах, проповедуя о пришествии Христовом:
\vs 1Cl 17:2
разумеем пророков Илию,
Елисея и Иезекииля, также и тех, которые получили прекрасное свидетельство.
\vs 1Cl 17:3
Авраам получил великое
свидетельство, и назван другом Божьим: но, взирая на славу Божью, со смирением
говорит: я земля и пепел.
\vs 1Cl 17:4
Далее и об Иове так
написано: Иов был праведен и непорочен, истинен и благочестив, и удалялся от
всякого зла.
\vs 1Cl 17:5
Но он, сам себя осуждая,
сказал: никто не чист от скверны, хотя бы и один день была жизнь Его.
\vs 1Cl 17:6
Моисей назван верным во
всем доме его, и Бог через его служение совершил суд над Египтом посредством
мучений и казней:
\vs 1Cl 17:7
но и он, столько
прославленный, не величался, но, когда из купины было к нему Божественное
слово, сказал: кто я, что Ты меня посылаешь? Я заика и косноязычен. И опять
говорит: я пар из котла.

\vs 1Cl 18:1
Что же скажем о
прославленном Давиде, о котором сказал Бог: Я нашел человека по сердцу Моему,
Давида сына Иессеева, милостью вечной Я помазал его?
\vs 1Cl 18:2
Но и он говорит Богу:
помилуй меня, Боже, по великой милости Твоей, и по множеству щедрот Твоих
очисти беззаконие мое.
\vs 1Cl 18:3
Еще более~--- омой меня от
неправды моей, и очисти меня от греха моего, ибо я знаю неправду свою и грех
мой всегда предо мною.
\vs 1Cl 18:4
Тебе одному согрешил я и
пред Тобою сделал зло, чтобы Ты оправдался в словах Твоих и победил, когда
станут судить Тебя.
\vs 1Cl 18:5
Ибо в беззакониях зачат я
и в грехах родила меня мать моя.
\vs 1Cl 18:6
Ты возлюбил истину:
сокровенные тайны премудрости Твоей Ты открыл мне.
\vs 1Cl 18:7
Окропи меня иссопом, и
буду чист, омой меня, и буду белее снега.
\vs 1Cl 18:8
Слуху моему дай радость и
веселье: и сокрушенные кости мои возрадуются.
\vs 1Cl 18:9
Отврати лицо от грехов
моих, и изгладь все беззакония мои.
\vs 1Cl 18:10
Создай во мне сердце
чистое, Боже, и дух правый обнови в утробе моей.
\vs 1Cl 18:11
Не отвергни меня от лица
Твоего, и Духа Твоего Святого не отними от меня.
\vs 1Cl 18:12
Воздай мне радость
спасения Твоего, и укрепи меня Духом Адонаи.
\vs 1Cl 18:13
Научу грешников путям
Твоим и нечестивые обратятся к Тебе.
\vs 1Cl 18:14
Избавь меня от пролития
крови, Боже, Бог спасения моего. Язык мой воспоет правду Твою.
\vs 1Cl 18:15
Адонай, открой уста мои,
и уста мои возвестят хвалу Твою.
\vs 1Cl 18:16
Если бы Ты восхотел иной
жертвы, я принес бы; но всесожжения Тебе неугодны.
\vs 1Cl 18:17
Жертва Богу~--- дух
сокрушенный; сердце сокрушенное и смиренное Бог не презрит.

\vs 1Cl 19:1
Смирение и послушливая
покорность этих мужей, получивших столь славное свидетельство от Самого Бога,
сделали лучшими не только нас, но и прежде бывшие поколения,
\vs 1Cl 19:2
именно тех, которые со
страхом и искренностью принимали глаголы Его.
\vs 1Cl 19:3
Итак, имея пред собою
столь многие великие и славные деяния, обратимся к цели мира, указанной нам
изначала,
\vs 1Cl 19:4
и взирая к Отцу и
Создателю всего мира, вникнем в Его величественные и превосходные дары мира и
в Его благодеяния.
\vs 1Cl 19:5
Воззрим на Него умом и
душевными очами, посмотрим на долготерпение Его воли, и помыслим, как Он
кроток ко всему творению Своему.

\vs 1Cl 20:1
Небеса, по Его
распоряжению движущиеся, в мире повинуются Ему: и день и ночь совершают
определенное им течение, не препятствуя друг другу.
\vs 1Cl 20:2
Солнце и лики звезд, по
Его велению, согласно, без малейшего уклонения проникают на назначенные им
пути.
\vs 1Cl 20:3
Плодоносящая земля, по Его
воле, в определенные времена производит изобильную пищу человекам, зверям и
всем находящимся на ней животным, не замедляя и не изменяя ничего из
предписанного им.
\vs 1Cl 20:4
Неисследуемые и
непостижимые области бездны и преисподней держатся теми же велениями.
\vs 1Cl 20:5
Беспредельное море, по Его
устроению совокупленное в большие водные массы, не выступает за положенные ему
преграды, но делает так, как Он повелел.
\vs 1Cl 20:6
Ибо Он сказал: доселе
дойдешь, и волны твои в тебе сокрушатся.
\vs 1Cl 20:7
Непроходимый для людей
океан, и миры за ним находящиеся, управляются теми же повелениями Господа.
\vs 1Cl 20:8
Времена года~--- весна,
лето, осень и зима мирно сменяются одни другими.
\vs 1Cl 20:9
Определенные ветры, каждый
в свое время, беспрепятственно совершают свое служение.
\vs 1Cl 20:10
Неиссякающие источники,
созданные для наслаждения и здравия, непрестанно доставляют людям свою влагу,
необходимую для их жизни.
\vs 1Cl 20:11
Наконец, малейшие
животные мирно и согласно составляют сожительства между собою.
\vs 1Cl 20:12
Всему этому повелел быть
в согласии и мире великий Создатель и Владыка всего,
\vs 1Cl 20:13
Который благотворит всем,
а преимущественно нам, которые прибегли к милосердию Его чрез Господа нашего
Иисуса Христа, Которому слава и величие во веки веков. Аминь.

\vs 1Cl 21:1
Смотрите, возлюбленные,
чтобы столь многие благодеяния Его не обратились всем нам в осуждение, если
мы, живя достойно Его, не будем единодушно совершать благое и угодное Ему.
\vs 1Cl 21:2
Ибо сказано где-то: Дух
Господа есть светильник, испытующий тайны утробы.
\vs 1Cl 21:3
Помыслим, как Он близок к
нам, и что ни одна из наших мыслей или совещаний, какие мы делаем, не закрыты
от Него.
\vs 1Cl 21:4
Итак, надлежит нам не
отступать от воли Его: лучше воспротивимся глупым и несмысленным,
превозносящимся и хвалящимся пышностью слова своего людям, нежели Богу.
\vs 1Cl 21:5
Будем благоговеть перед
Господом Иисусом Христом, кровь Которого предана за нас, будем почитать
предстоятелей наших, уважать пресвитеров, юношей воспитывать в страхе Божьем,
\vs 1Cl 21:6
жен своих направлять к
добру, чтобы они отличались достолюбезным нравом целомудрия, показывали чистое
свое расположение к кротости, скромность языка своего обнаруживали молчанием,
любовь свою оказывали не по склонностям, но равную ко всем, свято боящимся
Бога.
\vs 1Cl 21:7
Дети ваши пусть получают
воспитание христианина; пусть научаются, как сильно пред Богом смирение, что
значит пред Богом чистая любовь, как прекрасен и велик страх Божий и
спасителен для всех, свято ходящих в нем с чистым умом.
\vs 1Cl 21:8
Ибо Он есть испытатель
мыслей и желаний наших: Его дыхание в нас, и когда захочет, возьмет его.

\vs 1Cl 22:1
Все сие подтверждает вера
христианская. Ибо Сам Христос чрез Духа Святого так взывает к нам:
\vs 1Cl 22:2
приходите, дети,
послушайте Меня; страху ЯХВЕ научу вас.
\vs 1Cl 22:3
Кто есть человек, хотящий
жизни, любящий видеть дни благие?
\vs 1Cl 22:4
Удержи язык твой от зла, и
уста твои, чтобы не говорить коварства.
\vs 1Cl 22:5
Уклонись от зла и сотвори
доброе; ищи мира, и гонись за ним.
\vs 1Cl 22:6
Очи ЯХВЕ~--- на праведных, и
уши Его~--- на молитву их:
\vs 1Cl 22:7
а на делающих злое лице
ЯХВЕ для того, чтобы истребить с земли память их.
\vs 1Cl 22:8
Воззвал праведник, и ЯХВЕ
услышал его, и избавил его от всех скорбей его.
\vs 1Cl 22:9
Много бичей грешному:
уповающих же на ЯХВЕ будет окружать милость.

\vs 1Cl 23:1
Милосердый во всем и
благодетельный Отец милостив к боящимся Его, и дары Свои охотно и ласково
раздает приступающим к Нему с чистым расположением.
\vs 1Cl 23:2
Посему не будем
сомневаться, и душа наша да не отчаивается о превосходных и славных дарах Его:
\vs 1Cl 23:3
да будет далеко от нас
сказанное в Писании, где оно говорит: несчастны двоедушные, колеблющиеся
душою и говорящие:
\vs 1Cl 23:4
это мы слышали и во время
отцов наших, и вот мы состарились, но ничего такого с нами не случилось.
\vs 1Cl 23:5
Неразумные! Сравните себя
с деревом, возьмите виноградную лозу:
\vs 1Cl 23:6
сперва она теряет лист,
потом образуется отпрыск, потом лист, потом цвет, и после этого незрелый,
наконец, спелый виноград.
\vs 1Cl 23:7
Видите, как в короткое
время древесный плод достигает зрелости.
\vs 1Cl 23:8
Скоро поистине и внезапно
совершится воля Господа по свидетельству самого Писания: скоро придет, и не
замедлит, и внезапно придет в храм Свой ЯХВЕ и Святой, Которого вы ожидаете.

\vs 1Cl 24:1
Рассмотрим, возлюбленные,
как Господь постоянно показывает нам будущее воскресение, которого начатком
сделал Господа Иисуса Христа, воскресив Его из мертвых.
\vs 1Cl 24:2
Посмотрим, возлюбленные,
на воскресение, совершающееся во всякое время.
\vs 1Cl 24:3
День и ночь представляют
нам воскресение: ночь отходит ко сну,~--- встает день; проходит день,~--- настает
ночь.
\vs 1Cl 24:4
Посмотрим на плоды, каким
образом происходит сеяние зерен.
\vs 1Cl 24:5
Вышел сеятель, бросил их в
землю, и брошенные семена, которые упали на землю сухие и голые, сгнивают;
\vs 1Cl 24:6
но после этого разрушения
великая сила Промысла Господня воскрешает их, и из одного возвращает многие и
производит плод.

\vs 1Cl 25:1
Взглянем на необычайное
знамение, бывающее в восточных странах, то есть около Аравии.
\vs 1Cl 25:2
Есть там птица, которая
называется Феникс. Она рождается только одна и живет по пяти сот лет.
\vs 1Cl 25:3
Приближаясь к своему
разрушению смертному, она из ливана, смирны и других ароматов делает себе
гнездо, в которое, по исполнении своего времени, входит и умирает.
\vs 1Cl 25:4
Из гниющего же тела
рождается червь, который, питаясь влагою умершего животного, оперяется;
\vs 1Cl 25:5
потом, пришедши в
крепость, берет то гнездо, в котором лежат кости его предка, и с этою ношею
совершает путь из Аравии в Египет, в город, называемый Илиополь,
\vs 1Cl 25:6
и прилетая днем, в виду
всех кладет это на жертвенник солнца, и таким образом назад удаляется.
\vs 1Cl 25:7
Жрецы рассматривают
летописи, и находят, что она являлась по исполнении пятисот лет.

\vs 1Cl 26:1
Итак, почтем ли мы
великим и удивительным, если Творец всего воскресит тех, которые в уповании
благой веры свято служили Ему,
\vs 1Cl 26:2
когда Он и посредством
птицы открывает нам Свое великое обещания Своего?
\vs 1Cl 26:3
Ибо говорится где-то: и
Ты воскресишь меня и восхвалю Тебя.
\vs 1Cl 26:4
И еще: я уснул, и спал,
восстал, потому что Ты со мной.
\vs 1Cl 26:5
Так же Иов говорит: и Ты
воскресишь эту плоть мою, которая терпит все это.

\vs 1Cl 27:1
В этой надежде да
прилепятся души наши к Тому, Кто верен в обещаниях и праведен в судах.
\vs 1Cl 27:2
Заповедавший не лгать, тем
более Сам не солжет; ибо для Бога ничего нет невозможного: невозможно только
солгать.
\vs 1Cl 27:3
Итак, да воспламенится в
нас вера Его, и будем помышлять, что все близко к Нему.
\vs 1Cl 27:4
Словом величества Своего
Он все создал, словом же может и разрушить это.
\vs 1Cl 27:5
Кто скажет Ему: зачем
сделал? или кто воспротивится могуществу силы Его.
\vs 1Cl 27:6
Когда Ему угодно, Он все
сделает, и ничего из определенного Им не останется без исполнения.
\vs 1Cl 27:7
Все пред Ним, и ничто не
скрыто от совета Его.
\vs 1Cl 27:8
Если небеса поведают
славу Божью, то твердь возвещает о творении рук Его;
\vs 1Cl 27:9
день дню отрыгает слово, и
ночь ночи возвращает ведение.
\vs 1Cl 27:10
И нет слов, ни речей,
звуки которых не были бы слышимы.

\vs 1Cl 28:1
Итак, если Бог все видит
и слышит, то убоимся Его, и оставим нечистые стремления к худым делам, чтобы
милосердием Его покрыться от будущих судов.
\vs 1Cl 28:2
Ибо куда может кто-либо из
нас убежать от крепкой руки Его? Какой мир примет убежавшего от Него?
\vs 1Cl 28:3
Ибо говорит негде Писание:
куда пойду и где скроюсь от лица Твоего?
\vs 1Cl 28:4
Если взойду на небо, Ты
там; если пойду на конец земли, и там десница Твоя; если расположусь в
безднах, и там Дух Твой.
\vs 1Cl 28:5
Итак, куда мог бы кто
удалиться, или куда убежать от Того, Кто все объемлет?

\vs 1Cl 29:1
Итак, приступим к Нему в
святости души, поднимая к Нему чистые и нескверные руки,
\vs 1Cl 29:2
и любя кроткого
милосердого Отца нашего, Который избрал нас в достояние Себе;
\vs 1Cl 29:3
ибо так написано: когда
Вышний разделял народы, когда расселял сынов Адамовых, то Он поставил пределы
народов по числу ангелов Божьих:
\vs 1Cl 29:4
и уделом ЯХВЕ стал народ
Его Иаков, межею наследия Его~--- Израиль.
\vs 1Cl 29:5
И в другом месте говорится
вот ЯХВЕ избирает Себе народ из среды народов, как человек берет начатки с
гумна своего и произойдет из того народа святое святых.

\vs 1Cl 30:1
Итак, будучи уделом
Святого, будем делать все относящееся к святости,
\vs 1Cl 30:2
убегая злословия, нечистых
и порочных связей, пьянства, страсти к нововведениям,
\vs 1Cl 30:3
низких пожеланий,
скверного расового смешения и гнусной гордости.
\vs 1Cl 30:4
Ибо говорится: Бог гордым
противится, смиренным же дает благодать.
\vs 1Cl 30:5
Итак, присоединимся к тем,
которым дана от Бога благодать.
\vs 1Cl 30:6
Облечемся в единомыслие,
будем смиренны, воздержны, далеки от всякой клеветы и злоречия, оправдывая
себя делами, а не словами.
\vs 1Cl 30:7
Ибо сказано: кто говорит
много, тот должен и слушать в свою очередь; или многоречивый будет праведен?
Благословен рожденный от жены, малолетний. Не будь многоречив.
\vs 1Cl 30:8
Хвала наша да будет у
Бога, а не от нас самих; Бог ненавидит тех, которые сами хвалят себя.
\vs 1Cl 30:9
Пусть свидетельство о
добром поведении нашем будет дано от других, так как дано было оно отцам нашим
праведным.
\vs 1Cl 30:10
Наглость, надменность и
дерзость свойственны проклятым от Бога; умеренность, смиренномудрие и кротость
у благословенных от Бога.

\vs 1Cl 31:1
Итак, взыщем
благословения Его, и посмотрим, какие пути приводят к благословению. Вспомним,
что было от начала.
\vs 1Cl 31:2
За что был благословен
отец наш Авраам? Не за то ли, что по вере своей творил правду и истину?
\vs 1Cl 31:3
Исаак, с уверенностью зная
будущее, охотно стал жертвою.
\vs 1Cl 31:4
Иаков со смирением оставил
из-за брата землю свою, пошел к Лавану и служил; и даны ему двенадцать колен
Израилевых.

\vs 1Cl 32:1
Если кто рассмотрит все в
подробности, то познает величие даров, данных от Бога.
\vs 1Cl 32:2
От Иакова все священники и
левиты, служащие при жертвеннике Божьем.
\vs 1Cl 32:3
От него Господь Иисус по
плоти: от него цари, начальники, вожди чрез Иуду;
\vs 1Cl 32:4
и прочие его колена в
немалой славе, так как обещал Бог: будет семя твое, как звезды небесные.
\vs 1Cl 32:5
И все они прославились и
возвеличились не сами собой, и не делами своими, и не правотой действий,
совершенных ими, но волей Божьей.
\vs 1Cl 32:6
Так и мы, будучи призваны
по воле Его во Христе Иисусе, оправдываемся не сами собою, и не своею
мудростью, или разумом, или благочестием, или делами, в святости сердца нами
совершаемыми,
\vs 1Cl 32:7
но посредством веры,
которую Вседержитель Бог от века всех оправдывал. Ему да будет слава во веки
веков. Аминь.

\vs 1Cl 33:1
Итак, что нам делать,
братья? Отстать ли от добродетели и любви?~--- Отнюдь нет, не дай Господь, чтоб
это сталось с нами;
\vs 1Cl 33:2
напротив, со всем усилием
и готовностью поспешим совершать доброе дело.
\vs 1Cl 33:3
Ибо Сам Творец и Владыка
всего веселится о делах Своих.
\vs 1Cl 33:4
Он высочайшею Своею силою
утвердил небеса и непостижимою Своею мудростью украсил их;
\vs 1Cl 33:5
Он отделил землю от
окружающей ее воды, и утвердил на прочном основании Своего хотения,
\vs 1Cl 33:6
и Своею властью повелел
быть ходящим на ней животным.
\vs 1Cl 33:7
Он также сотворил море и в
нем животных, и оградил Своим могуществом.
\vs 1Cl 33:8
Сверх всего этого Он
святыми и чистыми руками образовал отличнейшее и по разуму превосходнейшее
существо, человека, начертание Своего образа.
\vs 1Cl 33:9
Ибо так говорит Бог:
сотворим человека по образу и подобию Нашему. И сотворил Бог человека, мужа и
жену сотворил их.
\vs 1Cl 33:10
Совершив все это, Он
одобрил и благословил и сказал: раститесь и умножайтесь.
\vs 1Cl 33:11
Познаем также, что все
праведные украсились добрыми делами; и Сам Господь радовался, украсив Себя
делами.
\vs 1Cl 33:12
Имея такой пример,
неленостно последуем воле Его, и всею силою будем творить дело правды.

\vs 1Cl 34:1
Добрый работник смело
получает хлеб за труд свой; ленивый же и беспечный не смеет и взглянуть на
того, кто дал ему работу.
\vs 1Cl 34:2
И нам надлежит быть
ревностными в делании добра, ибо все от Него.
\vs 1Cl 34:3
Ибо предсказывает нам:
вот ЯХВЕ, и награда Его перед лицом Его, чтобы воздать каждому по делу его.
\vs 1Cl 34:4
Так увещевает Он нас всем
сердцем обратиться к Нему, и ни в каком добром деле не быть беспечными и
нерадивыми;
\vs 1Cl 34:5
в Нем да будет похвала и
надежда наша; покоримся воле Его.
\vs 1Cl 34:6
Помыслим о всем множестве
ангелов Его, как они, предстоя, исполняют волю Его.
\vs 1Cl 34:7
Ибо говорит Писание: тьмы
тем предстояли пред Ним и тысячи тысяч служили Ему, и взывали: свят, свят,
свят ЯХВЕ Цебаот; полно все творение славы Его.
\vs 1Cl 34:8
Так и мы, в единомысленном
собрании, единым духом, как бы из одних уст, будем взывать к Нему непрестанно,
чтобы сделаться нам участниками великих и славных обетований Его.
\vs 1Cl 34:9
Ибо говорит: око не
видело, и ухо не слышало, и на сердце человеку не приходило то, что Он
уготовал уповающим на Него.

\vs 1Cl 35:1
Как блаженны и чудны дары
Божьи, возлюбленные~--- жизнь в бессмертии, сияние в правде, истина в свободе,
вера в уповании, воздержание в святости: все это доступно нашему разумению.
\vs 1Cl 35:2
Какие же еще уготовляются
ждущим? Творец и Отец веков, Всесвятой, Он Сам знает их величие и красоту.
\vs 1Cl 35:3
Итак, употребим все усилия
быть в числе уповающих на Него, чтобы участвовать в обетованных дарах.
\vs 1Cl 35:4
Каким же образом это
будет, возлюбленные? Если ум наш будет утвержден в вере в Бога; если будем
искать того, что Ему угодно и приятно;
\vs 1Cl 35:5
если будем исполнять то,
что согласно с Его святою волею, и ходить путем истины, отвергнув от себя
всякую неправду и беззаконие,
\vs 1Cl 35:6
любостяжание, распри,
злонравие и коварство, клеветы и злословие, нечестие, гордость и величавость,
тщеславие и негостеприимность.
\vs 1Cl 35:7
Ибо делающие это
ненавистны Богу, и не только делающие, но и одобряющие это.
\vs 1Cl 35:8
Писание говорит: грешнику
сказал Бог: зачем ты познаешь заповеди Мои и принимаешь завет Мой устами
твоими, а возненавидел вразумление и отверг слова Мои?
\vs 1Cl 35:9
Если ты видел вора, то
бежал с ним, и с прелюбодеем принимал участие.
\vs 1Cl 35:10
Уста твои были исполнены
злобы, и язык твой сплетал обманы.
\vs 1Cl 35:11
Сидя на суде, ты клеветал
на брата твоего и сыну матери твоей строил ковы.
\vs 1Cl 35:12
Ты это делал, и Я молчал;
ты, беззаконный, подумал, что буду тебе подобен.
\vs 1Cl 35:13
Но обличу тебя и
представлю тебя перед лицом твоим.
\vs 1Cl 35:14
Разумейте же это вы,
забывающие Бога, чтобы вам не быть похищенными как бы львом, и некому будет
избавить вас.
\vs 1Cl 35:15
Жертва хвалы прославит
Меня, и там путь, на котором явлю ему спасение Божье.

\vs 1Cl 36:1
Таков путь, возлюбленные,
которым мы обретаем наше спасение, Иисуса Христа, Первосвященника наших
приношений, заступника и помощника в немощи нашей.
\vs 1Cl 36:2
Посредством Него взираем
мы на высоту небес; чрез Него, как бы в зеркале видим чистое и пресветлое лицо
Бога;
\vs 1Cl 36:3
чрез Него отверзлись очи
сердца нашего; чрез Него несмысленный и омраченный ум наш возникает в чудный
Его свет;
\vs 1Cl 36:4
чрез Него восхотел
Господь, чтобы мы вкусили бессмертного знания.
\vs 1Cl 36:5
Он, будучи сиянием величия
Его, столько превосходнее ангелов, сколько славнейшее пред ними наследовал
имя.
\vs 1Cl 36:6
Ибо так написано: Он
творит ангелов Своих духами и служителей Своих пламенем огненным.
\vs 1Cl 36:7
О Сыне же Своем так сказал
Господь: Сын Мой Ты, Я ныне родил Тебя, проси от Меня и дам Тебе народы в
достояние Твое, и пределы земли~--- в обладание Твое.
\vs 1Cl 36:8
И еще говорит к Нему:
сиди одесную Меня, доколе положу врагов Твоих в подножие ног Твоих.
\vs 1Cl 36:9
Кто же враги?~--- Порочные,
противящиеся воле Божьей.

\vs 1Cl 37:1
Итак, братья! будем всеми
силами воинствовать под святыми Его повелениями.
\vs 1Cl 37:2
Представим себе
воинствующих под начальством вождей наших; как стройно, как усердно, как
покорно исполняют они приказания.
\vs 1Cl 37:3
Не все епархи, не все
тысяченачальники, или стоначальники или пятидесятиначальники и так далее, но
каждый в своем чине исполняет приказания царя и полководцев.
\vs 1Cl 37:4
Ни великие без малых, ни
малые без великих не могут существовать.
\vs 1Cl 37:5
Все они как бы связаны
вместе, и это доставляет пользу.
\vs 1Cl 37:6
Возьмем тело наше: голова
без ног ничего не значит, равно и ноги без головы, и малейшие члены в теле
нашем нужны и полезны для целого тела;
\vs 1Cl 37:7
все они согласны и
стройным подчинением служат для целого тела.

\vs 1Cl 38:1
Так пусть будет здраво и
все тело наше в Иисусе Христе, и каждый повинуется ближнему своему сообразно
со степенью, на которой он поставлен дарованием Его.
\vs 1Cl 38:2
Сильный не пренебрегай
слабого, и слабый почитай сильного; богатый подавай бедному, и бедный
благодари Бога, что Он даровал ему, чрез кого может быть восполнена его
скудость.
\vs 1Cl 38:3
Мудрый показывай мудрость
свою не в словах, но в добрых делах.
\vs 1Cl 38:4
Смиренный не сам о себе
свидетельствуй, но предоставляй другому дать о тебе свидетельство.
\vs 1Cl 38:5
Чистый по плоти молчи и не
превозносись, зная, что есть другой, дарующий ему воздержание.
\vs 1Cl 38:6
Помыслим, братья, из
какого вещества мы произошли, и какими вошли в мир, как бы из гроба и мрака.
\vs 1Cl 38:7
Творец и Создатель наш
ввел нас в мир Свой, наперед приготовил нам Свои благодеяния прежде рождения
нашего.
\vs 1Cl 38:8
Итак, все имея от Него, мы
должны за все благодарить Его. Ему слава во веки веков. Аминь.

\vs 1Cl 39:1
Безумные, несмысленные,
глупые и невежды смеются и ругаются над нами, желая самих себя возвысить в
собственных мыслях своих.
\vs 1Cl 39:2
Но что может смертный, или
какая крепость в земнородном?
\vs 1Cl 39:3
Ибо написано: не было
образа пред глазами моими; но я слышал тихое веяние и голос:
\vs 1Cl 39:4
что же? будет ли человек
чист пред ЯХВЕ, или в делах своих непорочен, если Он на служителей Своих не
полагается и в ангелах Своих усматривает недостатки?
\vs 1Cl 39:5
Небо не чисто пред Ним;
тем менее живущие в бренных храминах, из числа которых и мы сами из того же
брения.
\vs 1Cl 39:6
Как бы моль поела их, и от
утра до вечера их уже нет: от того, что не могут помочь самим себе, они
погибли.
\vs 1Cl 39:7
Дунул на них и погибли,
потому что не имеют мудрости.
\vs 1Cl 39:8
Призови же, услышит ли
тебя кто-нибудь, или увидишь ли кого из святых ангелов?
\vs 1Cl 39:9
Безумного губит гнев и
глупого умерщвляет рвение.
\vs 1Cl 39:10
Я видел безумных
укореняющихся, но тотчас истреблено было их жилище.
\vs 1Cl 39:11
Да будут сыны их далеко
от спасения и да будут презрены при дверях меньших, и некому будет спасти их.
\vs 1Cl 39:12
Ибо что они собрали,
поедят праведные, сами же от зол не будут изъяты.

\vs 1Cl 40:1
Будучи убеждены в этом и
проникая в глубины божественного ведения, мы должны в порядке совершать все,
что Господь повелел совершать в определенные времена.
\vs 1Cl 40:2
Он повелел, чтобы жертвы и
священные действия совершались не случайно и не без порядка, но в определенные
времена и часы.
\vs 1Cl 40:3
Также где и через кого
должно быть это совершаемо, Сам Он определил высочайшим Своим изволением,
чтобы все совершалось свято и богоугодно, и было приятно воле Его.
\vs 1Cl 40:4
Итак, приятны Ему и
блаженны те, которые в установленные времена приносят жертвы свои;
\vs 1Cl 40:5
ибо, следуя заповедям
Господним, они не погрешают.
\vs 1Cl 40:6
Первосвященнику дано свое
служение, священникам назначено свое дело, и на левитов возложены свои
должности;
\vs 1Cl 40:7
из народа человек связан
постановлениями для народа.

\vs 1Cl 41:1
Каждый из вас, братья,
благодари Бога за свое собственное положение, храня добрую совесть и с
благоговением не преступая определенного правила служения своего.
\vs 1Cl 41:2
Не повсюду, братья,
приносятся жертвы непрерывные, или обетные или жертвы за грех, и жертвы
повинности, но только в Иерусалиме,
\vs 1Cl 41:3
и там не на всяком месте
совершается приношение, а пред храмом на жертвеннике, после того как жертва
будет осмотрена первосвященником и вышеназванными служителями.
\vs 1Cl 41:4
Те же, которые делают
что-либо вопреки Его воле, наказываются смертью.
\vs 1Cl 41:5
Видите, братья, чем
большего сподобились мы видения, тем большей подлежим опасности.

\vs 1Cl 42:1
Апостолы были посланы
проповедовать благовестие нам от Господа Иисуса Христа, Иисус Христос~--- от
Бога.
\vs 1Cl 42:2
Христос был послан от
Бога, а апостолы~--- от Христа; то и другое было в порядке по воле Божьей.
\vs 1Cl 42:3
Итак принявши повеление,
совершенно убежденные чрез воскресение Господа нашего Иисуса Христа и
утвержденные в вере словом Божьим, с полнотой Духа Святого пошли
благовествовать наступающее царство Божье.
\vs 1Cl 42:4
Проповедуя по странам и
городам, они первенцев, по духовном испытании поставляли в епископы и диаконы
для будущих верующих.
\vs 1Cl 42:5
И это не новое
установление; ибо много веков прежде было писано о епископах и диаконах.
\vs 1Cl 42:6
Так говорит Писание:
поставлю епископов их в правде и диаконов в вере.

\vs 1Cl 43:1
И чему дивиться, если те,
которым во Христе вверено было бы от Бога это дело, поставляли вышеупомянутых?
\vs 1Cl 43:2
Блаженный Моисей,
вверенный служитель во всем доме Божьем, все заповеданное Ему изобразил в
священных книгах;
\vs 1Cl 43:3
ему последовали и прочие
пророки, утверждая своим свидетельством его узаконения.
\vs 1Cl 43:4
Когда возникла распря о
священстве, и колена разногласили о том, какое из них должно быть украшено
этим славным именем,
\vs 1Cl 43:5
то повелел двенадцати
начальникам колен принести к нему жезлы, на которых было написано имя каждого
колена;
\vs 1Cl 43:6
и взявши их, связал,
запечатал перстнями начальников колен, положил их в скинии свидетельства на
трапезе Господней.
\vs 1Cl 43:7
И, заключив скинию,
запечатал замки также, как и жезлы, и сказал им: братья, которого колена жезл
расцветет, то избрал Бог к священству и служению Себе.
\vs 1Cl 43:8
На другой день утром
созвал он всего Израиля, шесть сот тысяч человек, и показал начальникам колен
печати их, и отворил скинию свидетельства и вынес жезлы:
\vs 1Cl 43:9
и оказалось, что жезл
Аарона не только расцвел но даже имел на себе плод.
\vs 1Cl 43:10
Как вы думаете,
возлюбленные, не знал ли Моисей прежде, что это будет?
\vs 1Cl 43:11
Конечно знал, но так
поступил он для того, чтобы не было возмущения в Израиле, для прославления
имени истинного и единого Бога. Ему слава во веки веков. Аминь.

\vs 1Cl 44:1
И апостолы наши знали
чрез Господа нашего Иисуса Христа, что будет раздор о епископском достоинстве.
\vs 1Cl 44:2
По этой самой причине они,
получивши совершенное предведение, поставили вышеозначенных, и потом
присовокупили закон, чтобы когда они почиют, другие испытанные мужи принимали
на себя их служение.
\vs 1Cl 44:3
Итак, почитаем
несправедливым лишить служения тех, которые поставлены самими апостолами или
после них другими достоуважаемыми мужами, с согласия всей Церкви, и служили
стаду Христову неукоризненно, со смирением, кротко и беспорочно, и притом в
течение долгого времени от всех получили одобрение.
\vs 1Cl 44:4
И не малый будет на нас
грех, если неукоризненно и свято приносящих дары будем лишать епископства.
\vs 1Cl 44:5
Блаженны предшествовавшие
нам пресвитеры, которые разрешились от тела после многоплодной и совершенной
жизни:
\vs 1Cl 44:6
им нечего опасаться, чтобы
кто мог свергнуть их с занимаемого ими места.
\vs 1Cl 44:7
Ибо мы видим, что вы
некоторых, похвально провождающих жизнь, лишили служения безукоризненно ими
проходимого.

\vs 1Cl 45:1
Вы, братья, спорливы и
ревностны в том, что ни мало не относится к спасению.
\vs 1Cl 45:2
Загляните в Писания, эти
истинные глаголы Духа Святого. Заметьте, что в них ничего несправедливого и
превратного не написано.
\vs 1Cl 45:3
Вы не найдете чтобы люди
праведные были низвергаемы людьми святыми.
\vs 1Cl 45:4
Были гонимы праведные, но
от беззаконных; были заключаемы в темницу, но от нечестивых; были побиваемы
камнями от злодеев; были убиваемы от порочных, увлекавшихся преступною
завистью. Все эти страдания они перенесли со славою.
\vs 1Cl 45:5
Ибо что скажем, братья?
Даниил от богобоязненных ли людей был брошен в ров львиный? Анания, Азария и
Мисаил от чтителей ли благолепного и славного служения Всевышнему были
ввержены в пещь огненную?~--- Отнюдь нет.
\vs 1Cl 45:6
Кто же сделал это?~--- Люди
порочные, полные всякого зла, дошли до такого неистовства, что святою и
непорочною волею служащих Богу подвергли мучениям:
\vs 1Cl 45:7
они не знали того, что
Вышний есть заступник и защитник тех, которые с чистой совестью чтут
всесовершенное имя Его. Ему слава во веки веков. Аминь.
\vs 1Cl 45:8
А они, терпя в уповании, и
были превознесены Богом, и сделались достолюбезными в памяти их во веки веков.
Аминь.

\vs 1Cl 46:1
Таким примерам и мы
должны подражать, братья.
\vs 1Cl 46:2
Ибо написано: прилепитесь
к святым; ибо прилепляющиеся к ним освятятся.
\vs 1Cl 46:3
И опять в другом месте
сказано: с мужем невинным будешь невинен, и с избранным будешь избран, а с
развращенным развратишься.
\vs 1Cl 46:4
Итак, присоединимся к
невинным и праведным, они-то суть избранные Божьи.
\vs 1Cl 46:5
К чему у вас распри; гнев
несогласия, разделения, война?
\vs 1Cl 46:6
Не одного ли Бога и одного
Христа имеем мы? Не один ли Дух благодати излит на нас, не одно ли призвание
во Христе?
\vs 1Cl 46:7
Для чего раздираем и
расторгаем члены Христовы, восстаем против собственного тела, и до такого
доходим безумия, что забываем, что мы друг другу члены?
\vs 1Cl 46:8
Вспомните слова Иисуса,
Господа нашего. Он сказал: горе тому человеку; хорошо было бы ему не
родиться, нежели соблазнить одного из избранных Моих;
\vs 1Cl 46:9
было бы лучше для него,
если бы он повесил камень жерновный и ввергнулся в море, нежели соблазнить
одного из малых Моих.
\vs 1Cl 46:10
Ваше разделение многих
развратило, многих повергло в уныние, многих в сомнение, и всех нас в печаль,
а смятение ваше все еще продолжается.

\vs 1Cl 47:1
Возьмите послание
блаженного апостола Павла. О чем он прежде всего писал вам в начале ангельской
проповеди?
\vs 1Cl 47:2
Истинно он по вдохновению
написал вам как о себе самом, так и о Кифе и Аполлосе, потому что и тогда
произошло у вас разделение на различные стороны.
\vs 1Cl 47:3
Но тогдашнее разделение
подвергло вас меньшему греху; ибо вы преклонялись на стороны прославленных
апостолов, и на сторону мужа, им одобренного.
\vs 1Cl 47:4
А теперь подумайте, какие
люди развратили вас и уменьшили красоту знаменитой братской любви вашей.
\vs 1Cl 47:5
Постыдное, возлюбленные, и
чрезвычайно постыдное и христианской жизни недостойное слышится дело:
твердейшая и древняя церковь Коринфская из-за одного или двух человек
возмутилась против пресвитеров.
\vs 1Cl 47:6
И этот слух дошел не
только до нас, но и до самых врагов наших, так что чрез ваше безумие имя
Господне подвергается поруганию, и вам самим готовится опасность.

\vs 1Cl 48:1
Итак, прекратим это как
можно скорее, и припадем к Господу, и слезно будем умолять Его, чтобы Он,
умилосердившись, примирился с нами, и восстановил в нас прежнюю прекрасную и
чистую жизнь братской любви.
\vs 1Cl 48:2
Это~--- врата правды,
отверстые к жизни как написано: откройте Мне врата правды; Я войду с вами и
восхвалю ЯХВЕ. Это~--- врата ЯХВЕ, праведные войдут ими.
\vs 1Cl 48:3
Из многих открытых врат
врата правды суть врата Христовы, и блаженны те, которые входят ими и
направляют шествие свое в святости и правде, все совершая без возмущения.
\vs 1Cl 48:4
Если кто тверд в вере, или
способен предлагать ведение, или мудр в обсуждении речей, или чист по своим
делам; тем более он должен смиряться, чем более кажется великим, и должен
искать общей пользы, а не своей.

\vs 1Cl 49:1
Кто имеет любовь во
Христе, тот должен соблюдать заповеди Христовы.
\vs 1Cl 49:2
Кто может изъяснить союз
любви Божьей? Кто способен, как должно, высказать величие благости Его?
\vs 1Cl 49:3
Несказанна высота, на
которую возводит любовь. Любовь соединяет нас с Богом; любовь покрывает
множество грехов, любовь все принимает, все терпит великодушно.
\vs 1Cl 49:4
В любви нет ничего
низкого, ничего надменного, любовь не допускает разделения, любовь не заводит
возмущения,
\vs 1Cl 49:5
любовь все делает в
согласии, любовью все избранные Божьи достигли совершенства, без любви нет
ничего благоугодного Богу.
\vs 1Cl 49:6
По любви воспринял нас
Господь; по любви, которую имел к нам Иисус Христос, Господь наш, по воле
Божьей дал кровь за нас, и плоть за плоть нашу, и душу за души наши.

\vs 1Cl 50:1
Видите ли, возлюбленные,
как велика и дивна любовь, и невыразимо ее совершенство.
\vs 1Cl 50:2
Кто может иметь ее, если
кого Сам Бог не удостоит?
\vs 1Cl 50:3
Итак будем просить и
умолять Его милосердие, чтобы жить нам в любви непорочно, без человеческого
разделения.
\vs 1Cl 50:4
Все роды от Адама до сего
дня миновали; но усовершившиеся в любви по благодати Божьей находятся на месте
благочестивых: они откроются с пришествием царства Христова.
\vs 1Cl 50:5
Ибо написано: войди на
некоторое время в храмины, пока пройдет гнев и негодование Мое, и вспомню о
дне добром, и воскрешу вас от гробов ваших.
\vs 1Cl 50:6
Блаженны мы, возлюбленные,
если исполняем заповеди Божьи в единомыслии любви, дабы чрез любовь были
прощены нам грехи наши.
\vs 1Cl 50:7
Ибо написано: блаженны
те, которых беззакония отпущены и которых покрылись грехи. Блажен человек,
которому ЯХВЕ не вменит греха, и в устах его нет обмана.
\vs 1Cl 50:8
Это обещание блаженства
относится к тем, которые избраны Богом чрез Иисуса Христа, Господа нашего. Ему
слава во веки веков. Аминь.

\vs 1Cl 51:1
Итак, в чем мы согрешили
по каким-либо наветам врага, должны мы просить прощения.
\vs 1Cl 51:2
И те, которые были
предводителями возмущения и раздора, должны иметь в виду общую надежду.
\vs 1Cl 51:3
Ибо провождающие жизнь со
страхом и любовью лучше хотят сами подвергнуться неприятностям, нежели ближних
своих,
\vs 1Cl 51:4
и охотнее на себя примут
осуждение, нежели на преданное нам доброе и святое согласие.
\vs 1Cl 51:5
И лучше человеку
признаться в своих грехах, нежели ожесточать сердце свое,
\vs 1Cl 51:6
как ожесточилось сердце
возмутившихся против раба Божьего Моисея:
\vs 1Cl 51:7
суд над ними совершился
явно, ибо они живые снизошли во ад и поглотила их смерть.
\vs 1Cl 51:8
Фараон, войско его, все
начальники Египетские, и колесницы и всадники их не по другой какой причине
потонули в море Суф и погибли, но потому, что ожесточились их несмысленные
сердца, после стольких знамений и чудес, совершенных в земле Египетской через
раба Божьего Моисея.

\vs 1Cl 52:1
Братья! Господь ни в чем
не имел нужды, и ничего ни от кого не требует, кроме исповедания Ему.
\vs 1Cl 52:2
Ибо говорит избранный
Давид: исповедуюсь ЯХВЕ, и это будет Ему приятнее, нежели молодой телец, у
которого растут рога и копыта. Пусть видят это бедные и возрадуются.
\vs 1Cl 52:3
И опять говорит: принеси
Богу жертву хвалы и воздай Вышнему молитвы твои.
\vs 1Cl 52:4
И призови Меня в день
скорби твоей, и избавлю тебя, и ты прославишь Меня.
\vs 1Cl 52:5
Ибо жертва Богу~--- дух
сокрушенный.

\vs 1Cl 53:1
Вы знаете, возлюбленные,
и хорошо знаете священные Писания, и разумеете слова Божьи.
\vs 1Cl 53:2
Итак, приведите себе на
память: когда Моисей взошел на гору и провел сорок дней и сорок ночей в посте
и смирении,
\vs 1Cl 53:3
тогда сказал ему ЯХВЕ:
Моисей, Моисей, сойди поскорей отсюда, потому что совершил преступление народ
твой, который ты вывел из земли Египетской;
\vs 1Cl 53:4
скоро они совратились с
пути, который ты заповедал им,~--- сделали себе слияния.
\vs 1Cl 53:5
И сказал ему ЯХВЕ: говорил
Я тебе раз и два, говоря: видел Я народ этот, и вот он~--- народ жестоковыйный.
\vs 1Cl 53:6
Дай Мне истребить его, и
погублю имя его под небом, а тебя сделаю народом великим и дивным и
многочисленнее этого.
\vs 1Cl 53:7
Моисей же сказал: нет,
ЯХВЕ, прости грех народу этому, или и меня истреби из книги живых.
\vs 1Cl 53:8
О, великая любовь! О,
несравненное совершенство! Раб смело говорит Господу, просит прощения народу;
в противном случае хочет и сам быть истребленным вместе с ними.

\vs 1Cl 54:1
Итак, кто из вас
благороден, кто добродушен, кто исполнен любви, тот пусть скажет:
\vs 1Cl 54:2
если из-за меня мятеж
раздор и разделение, я отхожу, иду, куда вам угодно, и исполню все, что велит
народ, только бы стадо Христово было в мире с поставленными пресвитерами.
\vs 1Cl 54:3
Кто поступит таким
образом, тот приобретет себе великую славу в Господе, и всякое место примет
его:
\vs 1Cl 54:4
ибо ЯХВЕ земля и
исполнение ее.
\vs 1Cl 54:5
Так поступали и будут
поступать все, провождающие похвальную божественную жизнь.

\vs 1Cl 55:1
Но представим примеры
народов. Многие цари и вожди во время моровой язвы, по внушениям прорицалища,
предавали себя на смерть, чтобы своею кровью спасти граждан.
\vs 1Cl 55:2
Многие удалялись из своих
городов, чтобы прекратилось возмущение в них.
\vs 1Cl 55:3
И из своих мы знаем
многих, которые предали себя в узы, дабы других освободить.
\vs 1Cl 55:4
Многие предали себя в
рабство, и, взявши за себя цену, питали других.
\vs 1Cl 55:5
Многие женщины,
укрепленные благодатью Божьей, совершили много дел мужественных.
\vs 1Cl 55:6
Блаженная Иудифь во время
осады города испросила позволения у старейшин пойти в стан иноплеменников.
\vs 1Cl 55:7
И пошла она, подвергая
себя опасности из любви к своему отечеству и народу осажденному, и Господь
предал Олоферна в руки женщины.
\vs 1Cl 55:8
Не меньшей опасности
подвергла себя совершенная по вере Есфирь, дабы избавить от предстоявшей
погибели двенадцать колен Израилевых.
\vs 1Cl 55:9
В посте и смирении она
умоляла всевидящего ЯХВЕ, Бога веков, Который, видя смирение души ее, избавил
народ, для блага которого она подвергла себя опасности.

\vs 1Cl 56:1
Будем и мы молиться о
тех, которые находятся во грехе, чтобы дарована им была кротость и смирение,
чтобы они послушались не нас, но воли Божьей.
\vs 1Cl 56:2
Ибо таким образом будет
для них плодотворно и совершенно милосердное воспоминание их пред Богом и
святыми.
\vs 1Cl 56:3
Примем наказание, на
которое никто не должен досадовать, возлюбленные!
\vs 1Cl 56:4
Взаимно делаемое нами друг
другу вразумление хорошо и весьма полезно, ибо оно не прилепляет нас к воле
Божьей.
\vs 1Cl 56:5
Ибо так говорит Святое
Слово: тяжко наказал меня ЯХВЕ, но смерти не предал меня;
\vs 1Cl 56:6
ибо кого любит ЯХВЕ, того
наказывает, и бьет всякого сына, которого принимает.
\vs 1Cl 56:7
Праведник накажет меня
милостиво и обличит меня; елей же грешного да не намастит головы моей.
\vs 1Cl 56:8
И еще говорит: блажен
человек, которого обличил ЯХВЕ; и вразумления Вседержителя не отвращайся,
\vs 1Cl 56:9
ибо Он производит скорбь и
опять восстановляет, поражает и руки Его исцеляют.
\vs 1Cl 56:10
Шесть раз избавит тебя от
бед, в седьмой же не коснется тебя зло.
\vs 1Cl 56:11
Во время голода избавит
тебя от смерти, во время войны спасет тебя от руки железа;
\vs 1Cl 56:12
от бича языка защитит
тебя, и не убоишься пред наступающими бедствиями.
\vs 1Cl 56:13
Ты посмеешься над
неправедными и беззаконными, и диких зверей не устрашишься; ибо звери дикие
будут мирны с тобою.
\vs 1Cl 56:14
Потом ты узнаешь, что дом
твой будет наслаждаться миром и не будет недостатка в помещении твоего шатра.
\vs 1Cl 56:15
Узнаешь также, что велико
семя твое и дети твои будут, как различные злаки полевые.
\vs 1Cl 56:16
Во гроб же сойдешь, как
пшеница созрелая, вовремя пожатая, или как стог гумна, вовремя свезенный.
\vs 1Cl 56:17
Видите, возлюбленные, что
наказуемые Господом~--- под Его защитою,
\vs 1Cl 56:18
ибо, как благой, Бог
наказывает для того, чтобы мы вразумились святым Его наказанием.

\vs 1Cl 57:1
Итак, вы, положившие
начало возмущению, покоритесь пресвитерам, и примите вразумление к покаянию,
преклонив колена сердца своего.
\vs 1Cl 57:2
Научитесь покорности,
отложивши тщеславную и надменную дерзость языка.
\vs 1Cl 57:3
Ибо лучше вам быть в стаде
Христа малыми и уважаемыми, нежели казаться чрезмерно высокими и лишиться
упования Его.
\vs 1Cl 57:4
Ибо так говорит
всесовершенная Премудрость: вот предложу вам слово Моего дыхания и научу вас
Моему разуму.
\vs 1Cl 57:5
Поскольку Я звала, и вы не
послушали, Я простирала слова, и вы не внимали, но отвергали Мои советы, и не
покорялись Моим обличениям:
\vs 1Cl 57:6
то Я посмеюсь вашей
погибели, и порадуюсь, когда придет вам пагуба, и когда внезапно настигнет вас
смятение, явится переворот подобно буре, или когда придет вам скорбь и
бедствие.
\vs 1Cl 57:7
Будет тогда, что призовете
Меня, а Я не послушаю вас; будут искать Меня злые и не найдут.
\vs 1Cl 57:8
Ибо они возненавидели
премудрость, страха ЯХВЕ не приняли, и не хотели внимать Моим советам, но
смеялись Моим обличениям.
\vs 1Cl 57:9
И потому они вкусят плоды
своих путей и насытятся своего нечестия.
\vs 1Cl 57:10
Ибо за то, что обидели
младенцев, они убиты будут, и суд нечестивых погубит.
\vs 1Cl 57:11
Меня же слушающий будет
обитать уверенно в надежде и упокоится без страха от всякого зла.

\vs 1Cl 58:1
Итак, будем повиноваться
всесвятому и славному имени Его, избегая прореченных Премудростью угроз
непокорным, дабы обитать уверенно в пресвятом имени величия Его.
\vs 1Cl 58:2
Примите совет наш, и не
раскаетесь. Ибо жив Бог и жив Господь Иисус Христос и Дух Святой, вера и
надежда избранных,
\vs 1Cl 58:3
так что выполнивший в
смиренномудрии, с непрестанной кротостью, Богом данные заповеди и повеления,
не раскаиваясь,~---
\vs 1Cl 58:4
сей поставится и изберется
в число спасающихся чрез Иисуса Христа, чрез Которого Ему слава во веки веков.
Аминь.

\vs 1Cl 59:1
Если же некоторые не
покорятся сказанному Им через нас,~--- пусть знают, что свяжут себя падением и
немалою опасностью.
\vs 1Cl 59:2
Мы же неповинны будем во
грехе сем и будем непрестанно молиться, прося и умоляя:
\vs 1Cl 59:3
да сохранит Творец всех
нерушимо исчисленное число избранных Своих во всем мире чрез возлюбленного
Отрока Своего, Иисуса Христа,~---
\vs 1Cl 59:4
чрез Которого Ты призвал
нас из тьмы в свет, из неведения~--- в познание славы Имени Его,
\vs 1Cl 59:5
надеяться на прежде всего
творения Имя Твое~--- ЯХВЕ.
\vs 1Cl 59:6
Отверзший очи сердца
нашего, чтобы познать Тебя, Единого Вышнего в вышних,
\vs 1Cl 59:7
Святого во святых
почивающего, смиряющего надмение гордых, разрушающего замыслы народов,
\vs 1Cl 59:8
смиренных возносящего и
смиряющего вознесенных,
\vs 1Cl 59:9
обогащающего и
разоряющего, убивающего и животворящего, Единого Благодетеля духов и Бога
всякой плоти,
\vs 1Cl 59:10
видящего бездны, Всевидца
человеческих дел, в опасности пребывающих Помощника,
\vs 1Cl 59:11
отчаявшихся Спасителя,
всякого духа Творца и Надзирателя, умножающего на земле народы и из всех
избравшего любящих Тебя чрез возлюбленного Отрока Твоего Иисуса Христа, чрез
Которого Ты нас научил, освятил, почтил.
\vs 1Cl 59:12
Просим, Владыка,
Помощником и Заступником нашим быть, пребывающих из нас в скорби спаси,
смиренных помилуй,
\vs 1Cl 59:13
падших воздвигни,
просящим явись, немощных исцели,
\vs 1Cl 59:14
заблуждающихся от народа
Твоего обрати, напитай алчущих, плененных из нас освободи, восставь немощных,
утешь малодушных:
\vs 1Cl 59:15
да познают Тебя все
народы, ибо Ты~--- Един Бог, и Иисус Христос~--- Отрок Твой, и мы люди Твои и
овцы пажити Твоей.

\vs 1Cl 60:1
Ибо Ты через совершаемое
Тобой сделал зримым вечный состав мира;
\vs 1Cl 60:2
Ты, ЯХВЕ, сотворил
вселенную, верный во всех родах и праведный в судах, чудный в силе и
великолепии,
\vs 1Cl 60:3
мудрый в творении и
разумный в основании сотворенного, благой в видимом и верный в надеющихся на
Тебя, милостивый и щедрый, оставь нам беззакония наши и неправды и грехи и
прегрешения.
\vs 1Cl 60:4
Не вмени всякого греха
рабов Твоих и рабынь, но очисти нас очищением истины Твоей и исправь стопы
наши, чтобы ходить в святости, правде и простоте сердца и творить благое и
угодное пред Тобою и пред князьями нашими.
\vs 1Cl 60:5
О, ЯХВЕ, яви лице Твое нам
во благо в мире, чтобы осениться нам рукою Твоею сильною и избавиться от
всякого зла Твоею мышцею высокою, и избавь нас от ненавидящих нас неправедно.
\vs 1Cl 60:6
Подай единомыслие и мир
нам и всем населяющим землю также, как Ты дал отцам нашим, призывающим им Тебя
свято в вере и истине,
\vs 1Cl 60:7
чтобы покорными быть
всемогущему и всесовершенному Имени Твоему, и князьям и вождям нашим на земле.

\vs 1Cl 61:1
Ты, Владыка, дал власть
царства им ради великолепия и неизреченной Твоей державы, чтобы познать нам
данную Тобою им славу и честь покоряться им, ни в чем не противиться воле
Твоей;
\vs 1Cl 61:2
подай им, ЯХВЕ, здравие,
мир, единомыслие, благостояние, дабы исполнять им Тобою данное им водительство
без соблазна.
\vs 1Cl 61:3
Ибо Ты, Владыка
пренебесный, Царь веков, дающий сынам человеческим славу и честь и власть над
сущими на земле;
\vs 1Cl 61:4
Ты, ЯХВЕ, исправь совет их
ко благу и угодному пред Тобою, да совершая в мире и кротости благочестно
данную им Тобою власть, обретут Тебя милостива.
\vs 1Cl 61:5
Единый могущий творить сие
и великое благо с нами, Тебе исповедуемся чрез Первосвященника и Ходатая душ
наших Иисуса Христа, чрез Которого Тебе слава и величие и ныне и в род родов
и во веки веков. Аминь.

\vs 1Cl 62:1
Итак, о делах приличных
богопочтению нашему и полезнейших для жизни добродетельной, желающим вести ее
благочестно и праведно, мы достаточно написали вам, мужи братия.
\vs 1Cl 62:2
Ибо мы всюду касались
того, что относится к вере, покаянию, искренней любви, воздержанию,
целомудрию и терпению,
\vs 1Cl 62:3
напоминая, что должно вам
в справедливости, истине и великодушии свято благоугождать Вседержителю Богу,
в единомыслии, незлопамятно, в любви и мире, с непрестанною кротостью,
\vs 1Cl 62:4
как и названные выше отцы
наши благоугождали, смиренномудрствуя по отношению к Отцу, Богу и Творцу, и ко
всем людям.
\vs 1Cl 62:5
И тем приятнее нам было
напомнить об этом, что мы пишем~--- как мы ясно знаем~--- мужам верным и славным,
вникающим в изречения учения Божьего.
1
Итак, справедливо,~--- следуя столь великим и многим примерам,~--- склонить выю и
занять место послушания,
\vs 1Cl 63:2
дабы успокоившись от
суетного волнения, достигли мы в истине предлежащей нам цели без всякого
позора.
\vs 1Cl 63:3
Ибо вы доставите нам
радость и веселье, если послушаетесь написанного нами чрез Святого Духа и
пресечете несправедливый гнев ревности вашей, сообразно увещанию к миру и
согласию, нами обращенному к вам в этом послании.
\vs 1Cl 63:4
Послали же мы мужей верных
и мудрых, от юности до старости обращавшихся непорочно среди нас, которые и
будут свидетелями между нами и вами.
\vs 1Cl 63:5
А поступили мы так, дабы
знали вы, что вся забота наша и была и есть~--- чтобы в скорости достигли вы
мира.

\vs 1Cl 64:1
Всевидящий Бог и Владыка
духов и Господь всякой плоти, избравший Господа Иисуса Христа и чрез Него~---
нас в народ избранный,
\vs 1Cl 64:2
да даст всякой душе,
призывающей великое и святое имя Его, веру, страх, мир, терпение, великодушие,
воздержание, чистоту и целомудрие
\vs 1Cl 64:3
в благоугождение имени его чрез первосвященника и ходатая
нашего Иисуса Христа, чрез которого ему слава,
величие, держава и честь ныне и во веки веков.
Аминь.

\vs 1Cl 65:1
Посланных от нас, Клавдия Эфеба и Валерия Витона с Фортунатом,
немедленно отпустите к нам в мире с радостью,
\vs 1Cl 65:2
чтобы они скорее известили нас о желаемом
и вожделенном для нас мире и согласии вашем,
\vs 1Cl 65:3
дабы и мы скорее могли порадоваться о вашем благоустройстве.
\vs 1Cl 65:4
Благодать Господа нашего Иисуса Христа да будет
с вами и со всеми, которые повсюду призваны Богом и чрез него:
\vs 1Cl 65:5
чрез которого ему слава,
честь, держава и величие, престол вечный,
от веков во веки веков. Аминь.

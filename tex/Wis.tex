\bibbookdescr{Wis}{
  inline={\LARGE Книга\\\Huge Премудрости Соломона\fns{Переведена с греческого.}},
  toc={Премудрость Соломона*},
  bookmark={Премудрость Соломона},
  header={Премудрость Соломона},
  %headerleft={},
  %headerright={},
  abbr={Прем}
}
\vs Wis 1:1 Любите справедливость, судьи земли, право мыслите о Господе, и в простоте сердца ищите Его,
\vs Wis 1:2 ибо Он обретается неискушающими Его и является не неверующим Ему.
\vs Wis 1:3 Ибо неправые умствования отдаляют от Бога, и испытание силы Его обличит безумных.
\vs Wis 1:4 В лукавую душу не войдет премудрость и не будет обитать в теле, порабощенном греху,
\vs Wis 1:5 ибо святый Дух премудрости удалится от лукавства и уклонится от неразумных умствований, и устыдится приближающейся неправды.
\vs Wis 1:6 Человеколюбивый дух~--- премудрость, но не оставит безнаказанным богохульствующего устами, потому что Бог есть свидетель внутренних чувств его и истинный зритель сердца его, и слышатель языка его.
\vs Wis 1:7 Дух Господа наполняет вселенную и, как все объемлющий, знает \bibemph{всякое} слово.
\vs Wis 1:8 Посему никто, говорящий неправду, не утаится, и не минет его обличающий суд.
\vs Wis 1:9 Ибо будет испытание помыслов нечестивого, и слов\acc{а} его взойдут к Господу в обличение беззаконий его;
\vs Wis 1:10 потому что ухо ревности слышит все, и ропот не скроется.
\vs Wis 1:11 Итак, хранитесь от бесполезного ропота и берегитесь от злоречия языка, ибо и тайное слово не пройдет даром, а клевещущие уста убивают душу.
\vs Wis 1:12 Не ускоряйте смерти заблуждениями вашей жизни и не привлекайте к себе погибели делами рук ваших.
\vs Wis 1:13 Бог не сотворил смерти и не радуется погибели живущих,
\vs Wis 1:14 ибо Он создал все для бытия, и все в мире спасительно, и нет пагубного яда, нет и царства ада на земле.
\vs Wis 1:15 Праведность бессмертна, а неправда причиняет смерть:
\vs Wis 1:16 нечестивые привлекли ее и руками и словами, сочли ее другом и исчахли, и заключили союз с нею, ибо они достойны быть ее жребием.
\vs Wis 2:1 Неправо умствующие говорили сами в себе: <<коротка и прискорбна наша жизнь, и нет человеку спасения от смерти, и не знают, чтобы кто освободил из ада.
\vs Wis 2:2 Случайно мы рождены и после будем как небывшие: дыхание в ноздрях наших~--- дым, и слово~--- искра в движении нашего сердца.
\vs Wis 2:3 Когда она угаснет, тело обратится в прах, и дух рассеется, как жидкий воздух;
\vs Wis 2:4 и имя наше забудется со временем, и никто не вспомнит о делах наших; и жизнь наша пройдет, как след облака, и рассеется, как туман, разогнанный лучами солнца и отягченный теплотою его.
\vs Wis 2:5 Ибо жизнь наша~--- прохождение тени, и нет нам возврата от смерти: ибо положена печать, и никто не возвращается.
\vs Wis 2:6 Будем же наслаждаться настоящими благами и спешить пользоваться миром, как юностью;
\vs Wis 2:7 преисполнимся дорогим вином и благовониями, и да не пройдет мимо нас весенний цвет жизни;
\vs Wis 2:8 увенчаемся цветами роз прежде, нежели они увяли;
\vs Wis 2:9 никто из нас не лишай себя участия в нашем наслаждении; везде оставим следы веселья, ибо это наша доля и наш жребий.
\vs Wis 2:10 Будем притеснять бедняка праведника, не пощадим вдовы и не постыдимся многолетних седин старца.
\vs Wis 2:11 Сила наша да будет законом правды, ибо бессилие оказывается бесполезным.
\vs Wis 2:12 Устроим ковы праведнику, ибо он в тягость нам и противится делам нашим, укоряет нас в грехах против закона и поносит нас за грехи нашего воспитания;
\vs Wis 2:13 объявляет себя имеющим познание о Боге и называет себя сыном Господа;
\vs Wis 2:14 он пред нами~--- обличение помыслов наших.
\vs Wis 2:15 Тяжело нам и смотреть на него, ибо жизнь его не похожа на жизнь других, и отличны пути его:
\vs Wis 2:16 он считает нас мерзостью и удаляется от путей наших, как от нечистот, ублажает кончину праведных и тщеславно называет отцом своим Бога.
\vs Wis 2:17 Увидим, истинны ли слова его, и испытаем, какой будет исход его;
\vs Wis 2:18 ибо если этот праведник есть сын Божий, то \bibemph{Бог} защитит его и избавит его от руки врагов.
\vs Wis 2:19 Испытаем его оскорблением и мучением, дабы узнать смирение его и видеть незлобие его;
\vs Wis 2:20 осудим его на бесчестную смерть, ибо, по словам его, о нем попечение будет>>.
\vs Wis 2:21 Так они умствовали, и ошиблись; ибо злоба их ослепила их,
\vs Wis 2:22 и они не познали тайн Божиих, не ожидали воздаяния за святость и не считали достойными награды душ непорочных.
\vs Wis 2:23 Бог создал человека для нетления и соделал его образом вечного бытия Своего;
\vs Wis 2:24 но завистью диавола вошла в мир смерть, и испытывают ее принадлежащие к уделу его.
\vs Wis 3:1 А души праведных в руке Божией, и мучение не коснется их.
\vs Wis 3:2 В глазах неразумных они казались умершими, и исход их считался погибелью,
\vs Wis 3:3 и отшествие от нас~--- уничтожением; но они пребывают в мире.
\vs Wis 3:4 Ибо, хотя они в глазах людей и наказываются, но надежда их полна бессмертия.
\vs Wis 3:5 И немного наказанные, они будут много облагодетельствованы, потому что Бог испытал их и нашел их достойными Его.
\vs Wis 3:6 Он испытал их как золото в горниле и принял их как жертву всесовершенную.
\vs Wis 3:7 Во время воздаяния им они воссияют как искры, бегущие по стеблю.
\vs Wis 3:8 Будут судить племена и владычествовать над народами, а над ними будет Господь царствовать во веки.
\vs Wis 3:9 Надеющиеся на Него познают истину, и верные в любви пребудут у Него; ибо благодать и милость со святыми Его и промышление об избранных Его.
\vs Wis 3:10 Нечестивые же, как умствовали, так и понесут наказание за то, что презрели праведного и отступили от Господа.
\vs Wis 3:11 Ибо презирающий мудрость и наставление несчастен, и надежда их суетна, и труды бесплодны, и дела их непотребны.
\vs Wis 3:12 Жены их несмысленны, и дети их злы, проклят род их.
\vs Wis 3:13 Блаженна неплодная неосквернившаяся, которая не познала беззаконного ложа; она получит плод при воздаянии святых душ.
\vs Wis 3:14 \bibemph{Блажен} и евнух, не сделавший беззакония рукою и не помысливший лукавого против Господа, ибо дастся ему особенная благодать веры и приятнейший жребий в храме Господнем.
\vs Wis 3:15 Плод добрых трудов славен, и корень мудрости неподвижен.
\vs Wis 3:16 Дети прелюбодеев будут несовершенны, и семя беззаконного ложа исчезнет.
\vs Wis 3:17 Если и будут они долгожизненны, но будут почитаться за ничто, и поздняя старость их будет без почета.
\vs Wis 3:18 А если скоро умрут, не будут иметь надежды и утешения в день суда;
\vs Wis 3:19 ибо ужасен конец неправедного рода.
\vs Wis 4:1 Лучше бездетность с добродетелью, ибо память о ней бессмертна: она признается и у Бога и у людей.
\vs Wis 4:2 Когда она присуща, ей подражают, а когда отойдет, стремятся к ней: и в вечности увенчанная она торжествует, как одержавшая победу непорочными подвигами.
\vs Wis 4:3 А плодородное множество нечестивых не принесет пользы, и прелюбодейные отрасли не дадут корней в глубину и не достигнут незыблемого основания;
\vs Wis 4:4 и хотя на время позеленеют в ветвях, но, не имея твердости, поколеблются от ветра и порывом ветров искоренятся;
\vs Wis 4:5 некрепкие ветви переломятся, и плод их \bibemph{будет} бесполезен, незрел для пищи и ни к чему не годен;
\vs Wis 4:6 ибо дети, рождаемые от беззаконных сожитий, суть свидетели разврата против родителей при допросе их.
\vs Wis 4:7 А праведник, если и рановременно умрет, будет в покое,
\vs Wis 4:8 ибо не в долговечности честная старость и не числом лет измеряется:
\vs Wis 4:9 мудрость есть седина для людей, и беспорочная жизнь~--- возраст старости.
\vs Wis 4:10 Как благоугодивший Богу, он возлюблен, и, как живший посреди грешников, преставлен,
\vs Wis 4:11 восхищен, чтобы злоба не изменила разума его, или коварство не прельстило души его.
\vs Wis 4:12 Ибо упражнение в нечестии помрачает доброе, и волнение похоти развращает ум незлобивый.
\vs Wis 4:13 Достигнув совершенства в короткое время, он исполнил долгие лета;
\vs Wis 4:14 ибо душа его была угодна Господу, потому и ускорил он из среды нечестия. А люди видели это и не поняли, даже и не подумали о том,
\vs Wis 4:15 что благодать и милость со святыми Его и промышление об избранных Его.
\vs Wis 4:16 Праведник, умирая, осудит живых нечестивых, и скоро достигшая совершенства юность~--- долголетнюю старость неправедного;
\vs Wis 4:17 ибо они увидят кончину мудрого и не поймут, что Господь определил о нем и для чего поставил его в безопасность;
\vs Wis 4:18 они увидят и уничтожат его, но Господь посмеется им;
\vs Wis 4:19 и после сего будут они бесчестным трупом и позором между умершими навек, ибо Он повергнет их ниц безгласными и сдвинет их с оснований, и они вконец запустеют и будут в скорби, и память их погибнет;
\vs Wis 4:20 в сознании грехов своих они предстанут со страхом, и беззакония их осудят их в лице их.
\vs Wis 5:1 Тогда праведник с великим дерзновением станет пред лицем тех, которые оскорбляли его и презирали подвиги его;
\vs Wis 5:2 они же, увидев, смутятся великим страхом и изумятся неожиданности спасения его
\vs Wis 5:3 и, раскаиваясь и воздыхая от стеснения духа, будут говорить сами в себе: <<это тот самый, который был у нас некогда в посмеянии и притчею поругания.
\vs Wis 5:4 Безумные, мы почитали жизнь его сумасшествием и кончину его бесчестною!
\vs Wis 5:5 Как же он причислен к сынам Божиим, и жребий его~--- со святыми?
\vs Wis 5:6 Итак, мы заблудились от пути истины, и свет правды не светил нам, и солнце не озаряло нас.
\vs Wis 5:7 Мы преисполнились делами беззакония и погибели и ходили по непроходимым пустыням, а пути Господня не познали.
\vs Wis 5:8 Какую пользу принесло нам высокомерие, и что доставило нам богатство с тщеславием?
\vs Wis 5:9 Все это прошло как тень и как молва быстротечная.
\vs Wis 5:10 Как после прохождения корабля, идущего по волнующейся воде, невозможно найти следа, ни стези дна его в волнах;
\vs Wis 5:11 или как от птицы, пролетающей по воздуху, никакого не остается знака ее пути, но легкий воздух, ударяемый крыльями и рассекаемый быстротою движения, пройден движущимися крыльями, и после того не осталось никакого знака прохождения по нему;
\vs Wis 5:12 или как от стрелы, пущенной в цель, разделенный воздух тотчас опять сходится, так что нельзя узнать, где прошла она;
\vs Wis 5:13 так и мы родились и умерли, и не могли показать никакого знака добродетели, но истощились в беззаконии нашем>>.
\vs Wis 5:14 Ибо надежда нечестивого исчезает, как прах, уносимый ветром, и как тонкий иней, разносимый бурею, и как дым, рассеиваемый ветром, и проходит, как память об однодневном госте.
\vs Wis 5:15 А праведники живут во веки; награда их~--- в Господе, и попечение о них~--- у Вышнего.
\vs Wis 5:16 Посему они получат царство славы и венец красоты от руки Господа, ибо Он покроет их десницею и защитит их мышцею.
\vs Wis 5:17 Он возьмет всеоружие~--- ревность Свою, и тварь вооружит к отмщению врагам;
\vs Wis 5:18 облечется в броню~--- в правду, и возложит на Себя шлем~--- нелицеприятный суд;
\vs Wis 5:19 возьмет непобедимый щит~--- святость;
\vs Wis 5:20 строгий гнев Он изострит, как меч, и мир ополчится с Ним против безумцев.
\vs Wis 5:21 Понесутся меткие стрелы молний и из облаков, как из туго натянутого лука, полетят в цель.
\vs Wis 5:22 И, как из каменометного орудия, с яростью посыплется град; вознегодует на них вода морская и реки свирепо потопят их;
\vs Wis 5:23 восстанет против них дух силы и, как вихрь, развеет их.
\vs Wis 5:24 Так беззаконие опустошит всю землю, и злодеяние ниспровергнет престолы сильных.
\vs Wis 6:1 Итак, слушайте, цари, и разумейте, научитесь, судьи концов земли!
\vs Wis 6:2 Внимайте, обладатели множества и гордящиеся пред народами!
\vs Wis 6:3 От Господа дана вам держава, и сила~--- от Вышнего, Который исследует ваши дела и испытает намерения.
\vs Wis 6:4 Ибо вы, будучи служителями Его царства, не судили справедливо, не соблюдали закона и не поступали по воле Божией.
\vs Wis 6:5 Страшно и скоро Он явится вам,~--- и строг суд над начальствующими,
\vs Wis 6:6 ибо меньший заслуживает помилование, а сильные сильно будут истязаны.
\vs Wis 6:7 Господь всех не убоится лица и не устрашится величия, ибо Он сотворил и малого и великого и одинаково промышляет о всех;
\vs Wis 6:8 но начальствующим предстоит строгое испытание.
\vs Wis 6:9 Итак, к вам, цари, слова мои, чтобы вы научились премудрости и не падали.
\vs Wis 6:10 Ибо свято хранящие святое освятятся, и научившиеся тому найдут оправдание.
\vs Wis 6:11 Итак, возжелайте слов моих, полюбите и научитесь.
\vs Wis 6:12 Премудрость светла и неувядающа, и легко созерцается любящими ее, и обретается ищущими ее;
\vs Wis 6:13 она \bibemph{даже} упреждает желающих познать ее.
\vs Wis 6:14 С раннего утра ищущий ее не утомится, ибо найдет ее сидящею у дверей своих.
\vs Wis 6:15 Помышлять о ней есть уже совершенство разума, и бодрствующий ради нее скоро освободится от забот,
\vs Wis 6:16 ибо она сама обходит и ищет достойных ее, и благосклонно является им на путях, и при всякой мысли встречается с ними.
\vs Wis 6:17 Начало ее есть искреннейшее желание учения,
\vs Wis 6:18 а забота об учении~--- любовь, любовь же~--- хранение законов ее, а наблюдение законов~--- залог бессмертия,
\vs Wis 6:19 а бессмертие приближает к Богу;
\vs Wis 6:20 поэтому желание премудрости возводит к царству.
\vs Wis 6:21 Итак, властители народов, если вы услаждаетесь престолами и скипетрами, то почтите премудрость, чтобы вам царствовать во веки.
\vs Wis 6:22 Что же есть премудрость, и как она произошла, я возвещу,
\vs Wis 6:23 и не скрою от вас тайн, но исследую от начала рождения,
\vs Wis 6:24 и открою познание ее, и не миную истины;
\vs Wis 6:25 и не пойду вместе с истаевающим от зависти, ибо таковой не будет причастником премудрости.
\vs Wis 6:26 Множество мудрых~--- спасение миру, и царь разумный~--- благосостояние народа.
\vs Wis 6:27 Итак учитесь от слов моих, и получите пользу.
\vs Wis 7:1 И я человек смертный, подобный всем, потомок первозданного земнородного.
\vs Wis 7:2 И я в утробе матерней образовался в плоть в десятимесячное время, сгустившись в крови от семени мужа и услаждения, соединенного со сном,
\vs Wis 7:3 и я, родившись, начал дышать общим воздухом и ниспал на ту же землю, первый голос обнаружил плачем одинаково со всеми,
\vs Wis 7:4 вскормлен в пеленах и заботах;
\vs Wis 7:5 ибо ни один царь не имел иного начала рождения:
\vs Wis 7:6 один для всех вход в жизнь и одинаковый исход.
\vs Wis 7:7 Посему я молился, и дарован мне разум; я взывал, и сошел на меня дух премудрости.
\vs Wis 7:8 Я предпочел ее скипетрам и престолам и богатство почитал за ничто в сравнении с нею;
\vs Wis 7:9 драгоценного камня я не сравнил с нею, потому что перед нею все золото~--- ничтожный песок, а серебро~--- грязь в сравнении с нею.
\vs Wis 7:10 Я полюбил ее более здоровья и красоты и избрал ее предпочтительно перед светом, ибо свет ее неугасим.
\vs Wis 7:11 А вместе с нею пришли ко мне все блага и несметное богатство через руки ее;
\vs Wis 7:12 я радовался всему, потому что премудрость руководствовала ими, но я не знал, что она~--- виновница их.
\vs Wis 7:13 Без хитрости я научился, и без зависти преподаю, не скрываю богатства ее,
\vs Wis 7:14 ибо она есть неистощимое сокровище для людей; пользуясь ею, они входят в содружество с Богом, посредством даров учения.
\vs Wis 7:15 Только дал бы мне Бог говорить по разумению и достойно мыслить о дарованном, ибо Он есть руководитель к мудрости и исправитель мудрых.
\vs Wis 7:16 Ибо в руке Его и мы и слова наши, и всякое разумение и искусство делания.
\vs Wis 7:17 Сам Он даровал мне неложное познание существующего, чтобы познать устройство мира и действие стихий,
\vs Wis 7:18 начало, конец и средину времен, смены поворотов и перемены времен,
\vs Wis 7:19 круги годов и положение звезд,
\vs Wis 7:20 природу животных и свойства зверей, стремления ветров и мысли людей, различия растений и силы корней.
\vs Wis 7:21 Познал я все, и сокровенное и явное, ибо научила меня Премудрость, художница всего.
\vs Wis 7:22 Она есть дух разумный, святый, единородный, многочастный, тонкий, удобоподвижный, светлый, чистый, ясный, невредительный, благолюбивый, скорый, неудержимый,
\vs Wis 7:23 благодетельный, человеколюбивый, твердый, непоколебимый, спокойный, беспечальный, всевидящий и проникающий все умные, чистые, тончайшие духи.
\vs Wis 7:24 Ибо премудрость подвижнее всякого движения, и по чистоте своей сквозь все проходит и проникает.
\vs Wis 7:25 Она есть дыхание силы Божией и чистое излияние славы Вседержителя: посему ничто оскверненное не войдет в нее.
\vs Wis 7:26 Она есть отблеск вечного света и чистое зеркало действия Божия и образ благости Его.
\vs Wis 7:27 Она~--- одна, но может все, и, пребывая в самой себе, все обновляет, и, переходя из рода в род в святые души, приготовляет друзей Божиих и пророков;
\vs Wis 7:28 ибо Бог никого не любит, кроме живущего с премудростью.
\vs Wis 7:29 Она прекраснее солнца и превосходнее сонма звезд; в сравнении со светом она выше;
\vs Wis 7:30 ибо свет сменяется ночью, а премудрости не превозмогает злоба.
\vs Wis 8:1 Она быстро распростирается от одного конца до другого и все устрояет на пользу.
\vs Wis 8:2 Я полюбил ее и взыскал от юности моей, и пожелал взять ее в невесту себе, и стал любителем красоты ее.
\vs Wis 8:3 Она возвышает \bibemph{свое} благородство тем, что имеет сожитие с Богом, и Владыка всех возлюбил ее:
\vs Wis 8:4 она таинница ума Божия и избирательница дел Его.
\vs Wis 8:5 Если богатство есть вожделенное приобретение в жизни, то что богаче премудрости, которая все делает?
\vs Wis 8:6 Если же благоразумие делает \bibemph{многое}, то какой художник лучше ее?
\vs Wis 8:7 Если кто любит праведность,~--- плоды ее суть добродетели: она научает целомудрию и рассудительности, справедливости и мужеству, полезнее которых ничего нет для людей в жизни.
\vs Wis 8:8 Если кто желает большой опытности, мудрость знает давнопрошедшее и угадывает будущее, знает тонкости слов и разрешение загадок, предузнает знамения и чудеса и последствия лет и времен.
\vs Wis 8:9 Посему я рассудил принять ее в сожитие с собою, зная, что она будет мне советницею на доброе и утешеньем в заботах и печали.
\vs Wis 8:10 Через нее я буду иметь славу в народе и честь перед старейшими, будучи юношею;
\vs Wis 8:11 окажусь проницательным в суде, и в глазах сильных заслужу удивление.
\vs Wis 8:12 Когда я буду молчать, они будут ожидать, и когда начну говорить, будут внимать, и когда продлю беседу, положат руку на уста свои.
\vs Wis 8:13 Чрез нее я достигну бессмертия и оставлю вечную память будущим после меня.
\vs Wis 8:14 Я буду управлять народами, и племена покорятся мне;
\vs Wis 8:15 убоятся меня, когда услышат обо мне страшные тираны; в народе явлюсь добрым и на войне мужественным.
\vs Wis 8:16 Войдя в дом свой, я успокоюсь ею, ибо в обращении ее нет суровости, ни в сожитии с нею скорби, но веселие и радость.
\vs Wis 8:17 Размышляя о сем сам в себе и обдумывая в сердце своем, что в родстве с премудростью~--- бессмертие,
\vs Wis 8:18 и в дружестве с нею~--- благое наслаждение, и в трудах рук ее~--- богатство неоскудевающее, и в собеседовании с нею~--- разум, и в общении слов ее~--- добрая слава,~--- я ходил и искал, как бы мне взять ее себе.
\vs Wis 8:19 Я был отрок даровитый и душу получил добрую;
\vs Wis 8:20 притом, будучи добрым, я вошел и в тело чистое.
\vs Wis 8:21 Познав же, что иначе не могу овладеть ею, как если дарует Бог,~--- и что уже было делом разума, чтобы познать, чей этот дар,~--- я обратился к Господу и молился Ему, и говорил от всего сердца моего:
\vs Wis 9:1 Боже отцов и Господи милости, сотворивший все словом Твоим
\vs Wis 9:2 и премудростию Твоею устроивший человека, чтобы он владычествовал над созданными Тобою тварями
\vs Wis 9:3 и управлял миром свято и справедливо, и в правоте души производил суд!
\vs Wis 9:4 Даруй мне приседящую престолу Твоему премудрость и не отринь меня от отроков Твоих,
\vs Wis 9:5 ибо я раб Твой и сын рабы Твоей, человек немощный и кратковременный и слабый в разумении суда и законов.
\vs Wis 9:6 Да хотя бы кто и совершен был между сынами человеческими, без Твоей премудрости он будет признан за ничто.
\vs Wis 9:7 Ты избрал меня царем народа Твоего и судьею сынов Твоих и дщерей;
\vs Wis 9:8 Ты сказал, чтобы я построил храм на святой горе Твоей и алтарь в городе обитания Твоего, по подобию святой скинии, которую Ты предуготовил от начала.
\vs Wis 9:9 С Тобою премудрость, которая знает дела Твои и присуща была, когда Ты творил мир, и ведает, что угодно пред очами Твоими и что право по заповедям Твоим:
\vs Wis 9:10 ниспошли ее от святых небес и от престола славы Твоей ниспошли ее, чтобы она споспешествовала мне в трудах моих, и чтобы я знал, что благоугодно пред Тобою;
\vs Wis 9:11 ибо она все знает и разумеет, и мудро будет руководить меня в делах моих, и сохранит меня в своей славе;
\vs Wis 9:12 и дела мои будут благоприятны, и буду судить народ Твой справедливо, и буду достойным престола отца моего.
\vs Wis 9:13 Ибо какой человек в состоянии познать совет Божий? или кто может уразуметь, что угодно Господу?
\vs Wis 9:14 Помышления смертных нетверды, и мысли наши ошибочны,
\vs Wis 9:15 ибо тленное тело отягощает душу, и эта земная храмина подавляет многозаботливый ум.
\vs Wis 9:16 Мы едва можем постигать и то, что на земле, и с трудом понимаем то, что под руками, а что на небесах~--- кто исследовал?
\vs Wis 9:17 Волю же Твою кто познал бы, если бы Ты не даровал премудрости и не ниспослал свыше святаго Твоего Духа?
\vs Wis 9:18 И так исправились пути живущих на земле, и люди научились тому, что угодно Тебе,
\vs Wis 9:19 и спаслись премудростью.
\vs Wis 10:1 Она сохраняла первозданного отца мира, который сотворен был один, и спасала его от собственного его падения:
\vs Wis 10:2 она дала ему силу владычествовать над всем.
\vs Wis 10:3 А отступивший от нее неправедный во гневе своем погиб от братоубийственной ярости.
\vs Wis 10:4 Ради него потопляемую землю опять премудрость спасла, сохранив праведника посредством малого дерева.
\vs Wis 10:5 Она же между народами, смешанными в единомыслии зла, нашла праведника и соблюла его неукоризненным пред Богом, и сохранила мужественным в жалости к сыну.
\vs Wis 10:6 Она во время погибели нечестивых спасла праведного, который избежал огня, нисшедшего на пять городов,
\vs Wis 10:7 от которых во свидетельство нечестия осталась дымящаяся пустая земля и растения, не в свое время приносящие плоды, и памятником неверной души~--- стоящий соляной столб.
\vs Wis 10:8 Ибо они, презрев премудрость, не только повредили себе тем, что не познали добра, но и оставили живущим память о своем безумии, дабы не могли скрыть того, в чем заблудились.
\vs Wis 10:9 Премудрость же спасла от бед служащих ей.
\vs Wis 10:10 Праведного, бежавшего от братнего гнева, она наставляла на правые пути, показала ему царство Божие и даровала ему познание святых, помогала ему в огорчениях и обильно вознаградила труды его.
\vs Wis 10:11 Когда из корыстолюбия обижали его, она предстала и обогатила его,
\vs Wis 10:12 сохранила его от врагов, и обезопасила от коварствовавших против него, и в крепкой борьбе доставила ему победу, дабы он знал, что благочестие всего сильнее.
\vs Wis 10:13 Она не оставила проданного праведника, но спасла его от греха:
\vs Wis 10:14 она нисходила с ним в ров и не оставляла его в узах, и потом принесла ему скипетр царства и власть над угнетавшими его, показала лжецами обвинявших его и даровала ему вечную славу.
\vs Wis 10:15 Она освободила святой народ и непорочное семя от народа угнетавших \bibemph{его},
\vs Wis 10:16 вошла в душу служителя Господня и противостала страшным царям чудесами и знамениями.
\vs Wis 10:17 Она воздала святым награду за труды их, вела их путем дивным; и днем была им покровом, а ночью~--- звездным светом.
\vs Wis 10:18 Она перевела их чрез Чермное море и провела их сквозь большую воду,
\vs Wis 10:19 а врагов их потопила и извергла их из глубины бездны.
\vs Wis 10:20 Итак, праведные завладели доспехами нечестивых и воспели святое имя Твое, Господи, и единодушно прославили поборающую руку Твою;
\vs Wis 10:21 ибо премудрость отверзла уста немых и сделала внятными языки младенцев.
\vs Wis 11:1 Она благоустроила дела их рукою святого пророка:
\vs Wis 11:2 они прошли по необитаемой пустыне, и на непроходных \bibemph{местах} поставили шатры;
\vs Wis 11:3 противостали неприятелям и отмстили врагам;
\vs Wis 11:4 томились жаждою и воззвали к Тебе, и дана им была вода из утесистой скалы и утоление жажды~--- из твердого камня.
\vs Wis 11:5 Ибо, чем наказаны были враги их,
\vs Wis 11:6 тем они, находясь в затруднении, были облагодетельствованы:
\vs Wis 11:7 вместо источника постоянно текущей реки, смрадною кровью возмущенной,
\vs Wis 11:8 в обличение их детоубийственного повеления, Ты неожиданно дал им обильную воду,
\vs Wis 11:9 показав тогда чрез жажду, как Ты наказал их противников.
\vs Wis 11:10 Ибо, когда они были испытываемы, подвергаясь, впрочем, милостивому вразумлению, тогда познали, как мучились во гневе судимые нечестивые;
\vs Wis 11:11 потому что их Ты испытывал, как отец, поучая, а тех, как гневный царь, осуждая, истязал.
\vs Wis 11:12 И отсутствовавшие и присутствовавшие одинаково пострадали:
\vs Wis 11:13 их постигла сугубая скорбь и стенание от воспоминания о прошедшем.
\vs Wis 11:14 Они, когда услышали, что чрез их наказания те были облагодетельствованы, познали Господа.
\vs Wis 11:15 Кого они прежде, как отверженного, отреклись с ругательством, Тому в последствие событий удивлялись, потерпев неодинаковую с праведными жажду.
\vs Wis 11:16 А за неразумные помышления их неправды, по которым они в заблуждении служили бессловесным пресмыкающимся и презренным чудовищам, Ты в наказание наслал на них множество бессловесных животных,
\vs Wis 11:17 чтобы они познали, что, чем кто согрешает, тем и наказывается.
\vs Wis 11:18 Не невозможно было бы для всемогущей руки Твоей, создавшей мир из необразного вещества, наслать на них множество медведей или свирепых львов,
\vs Wis 11:19 или неизвестных новосозданных лютых зверей, или дышащих огненным дыханием, или извергающих клубы дыма, или бросающих из глаз ужасные искры,
\vs Wis 11:20 которые не только повреждением могли истребить их, но и ужасающим видом погубить.
\vs Wis 11:21 Да и без этого они могли погибнуть от одного дуновения, преследуемые правосудием и рассеваемые духом силы Твоей; но Ты все расположил мерою, числом и весом.
\vs Wis 11:22 Ибо великая сила всегда присуща Тебе, и кто противостанет силе мышцы Твоей?
\vs Wis 11:23 Весь мир пред Тобою, как колебание чашки весов, или как капля утренней росы, сходящей на землю.
\vs Wis 11:24 Ты всех милуешь, потому что все можешь, и покрываешь грехи людей ради покаяния.
\vs Wis 11:25 Ты любишь все существующее, и ничем не гнушаешься, что сотворил, ибо не создал бы, если бы что ненавидел.
\vs Wis 11:26 И как могло бы пребывать что-либо, если бы Ты не восхотел? Или как сохранилось бы то, что не было призвано Тобою?
\vs Wis 11:27 Но Ты все щадишь, потому что все Твое, душелюбивый Господи.
\vs Wis 12:1 Нетленный Твой дух пребывает во всем.
\vs Wis 12:2 Посему заблуждающихся Ты мало-помалу обличаешь и, напоминая \bibemph{им}, в чем они согрешают, вразумляешь, чтобы они, отступив от зла, уверовали в Тебя, Господи.
\vs Wis 12:3 Так, возгнушавшись древними обитателями святой земли Твоей,
\vs Wis 12:4 совершавшими ненавистные дела волхвований и нечестивые жертвоприношения,
\vs Wis 12:5 и безжалостными убийцами детей, и на жертвенных пирах пожиравшими внутренности человеческой плоти и крови в тайных собраниях,
\vs Wis 12:6 и родителями, убивавшими беспомощные души,~--- Ты восхотел погубить \bibemph{их} руками отцов наших,
\vs Wis 12:7 дабы земля, драгоценнейшая всех у Тебя, приняла достойное население чад Божиих.
\vs Wis 12:8 Но и их, как людей, Ты щадил, послав предтечами воинства Твоего шершней, дабы они мало-помалу истребляли их.
\vs Wis 12:9 Хотя не невозможно было Тебе войною покорить нечестивых праведным, или истребить их страшными зверями, или грозным словом в один раз;
\vs Wis 12:10 но Ты, мало-помалу наказывая \bibemph{их}, давал место покаянию, зная, однако, что племя их негодное и зло их врожденное, и помышление их не изменится во веки.
\vs Wis 12:11 Ибо семя их было проклятое от начала, и не из опасения перед кем-либо Ты допускал безнаказанность грехов их.
\vs Wis 12:12 Ибо кто скажет: <<что Ты сделал?>> или кто противостанет суду Твоему? и кто обвинит Тебя в погублении народов, которых Ты сотворил? Или какой защитник придет к Тебе с ходатайством за неправедных людей?
\vs Wis 12:13 Ибо кроме Тебя нет Бога, который имеет попечение о всех, чтобы доказывать Тебе, что Ты несправедливо судил.
\vs Wis 12:14 Ни царь, ни властелин не в состоянии явиться к Тебе на глаза за тех, которых Ты погубил.
\vs Wis 12:15 Будучи праведен, Ты всем управляешь праведно, почитая не свойственным Твоей силе осудить того, кто не заслуживает наказания.
\vs Wis 12:16 Ибо сила Твоя есть начало правды, и то самое, что Ты господствуешь над всеми, располагает Тебя щадить всех.
\vs Wis 12:17 Силу Твою Ты показываешь не верующим всемогуществу Твоему и в не признающих Тебя обличаешь дерзость;
\vs Wis 12:18 но, обладая силою, Ты судишь снисходительно и управляешь нами с великою милостью, ибо могущество Твое всегда в Твоей воле.
\vs Wis 12:19 Но такими делами Ты поучал народ Твой, что праведному должно быть человеколюбивым, и внушал сынам Твоим благую надежду, что Ты даешь время покаянию во грехах.
\vs Wis 12:20 Ибо, если врагов сынам Твоим и повинных смерти Ты наказывал с таким снисхождением и пощадою, давая \bibemph{им} время и побуждение освободиться от зла,
\vs Wis 12:21 то с каким вниманием Ты судил сынов Твоих, которых отцам Ты дал клятвы и заветы благих обетований!
\vs Wis 12:22 Итак, вразумляя нас, Ты наказываешь врагов наших тысячекратно, дабы мы, когда судим, помышляли о Твоей благости и, когда бываем судимы, ожидали помилования.
\vs Wis 12:23 Посему-то и тех нечестивых, которые проводили жизнь в неразумии, Ты истязал собственными их мерзостями,
\vs Wis 12:24 ибо они очень далеко уклонились на путях заблуждения, обманываясь подобно неразумным детям и почитая за богов тех из животных, которые и у врагов были презренными.
\vs Wis 12:25 Посему, как неразумным детям, в посмеяние послал Ты им и наказание.
\vs Wis 12:26 Но, не вразумившись обличительным посмеянием, они испытывали заслуженный суд Божий.
\vs Wis 12:27 Ибо, что они сами терпели с досадою, то же увидев на тех, которых считали богами и чрез которых были наказываемы, они познали Бога истинного, Которого прежде отрекались знать;
\vs Wis 12:28 посему и пришло на них окончательное осуждение.
\vs Wis 13:1 Подлинно суетны по природе все люди, у которых не было ведения о Боге, которые из видимых совершенств не могли познать Сущего и, взирая на дела, не познали Виновника,
\vs Wis 13:2 а почитали за богов, правящих миром, или огонь, или ветер, или движущийся воздух, или звездный круг, или бурную воду, или небесные светила.
\vs Wis 13:3 Если, пленяясь их красотою, они почитали их за богов, то должны были бы познать, сколько лучше их Господь, ибо Он, Виновник красоты, создал их.
\vs Wis 13:4 А если удивлялись силе и действию их, то должны были бы узнать из них, сколько могущественнее Тот, Кто сотворил их;
\vs Wis 13:5 ибо от величия красоты созданий сравнительно познается Виновник бытия их.
\vs Wis 13:6 Впрочем, они меньше заслуживают порицания, ибо заблуждаются, может быть, ища Бога и желая найти Его:
\vs Wis 13:7 потому что, обращаясь к делам Его, они исследуют и убеждаются зрением, что все видимое прекрасно.
\vs Wis 13:8 Но и они неизвинительны:
\vs Wis 13:9 если они столько могли разуметь, что в состоянии были исследовать временный мир, то почему они тотчас не обрели Господа его?
\vs Wis 13:10 Но более жалки те, и надежды их~--- на бездушных, которые называют богами дела рук человеческих, золото и серебро, изделия художества, изображения животных, или негодный камень, дело давней руки.
\vs Wis 13:11 Или какой-либо древодел, вырубив годное дерево, искусно снял с него всю кору и, обделав красиво, устроил из него сосуд, полезный к употреблению в жизни,
\vs Wis 13:12 а обрезки от работы употребил на приготовление пищи и насытился;
\vs Wis 13:13 один же из обрезков, ни к чему не годный, дерево кривое и сучковатое, взяв, старательно округлил на досуге и, с опытностью знатока обделав его, уподобил его образу человека,
\vs Wis 13:14 или сделал подобным какому-нибудь низкому животному, намазал суриком и покрыл краскою поверхность его, и закрасил в нем всякий недостаток,
\vs Wis 13:15 и, устроив для него достойное его место, повесил его на стене, укрепив железом.
\vs Wis 13:16 Итак, чтобы \bibemph{произведение} его не упало, он наперед озаботился, зная, что оно само себе помочь не может, ибо это кумир и имеет нужду в помощи.
\vs Wis 13:17 Молясь же \bibemph{пред ним} о своих стяжаниях, о браке и о детях, он не стыдится говорить бездушному,
\vs Wis 13:18 и о здоровье взывает к немощному, о жизни просит мертвое, о помощи умоляет совершенно неспособное, о путешествии~--- не могущее ступить,
\vs Wis 13:19 о прибытке, о ремесле и об успехе рук~--- совсем не могущее делать руками, о силе просит самое бессильное.
\vs Wis 14:1 Еще: иной, собираясь плыть и переплывать свирепые волны, призывает на помощь дерево, слабейшее носящего его корабля;
\vs Wis 14:2 ибо стремление к приобретениям выдумало оный, а художник искусно устроил,
\vs Wis 14:3 но промысл Твой, Отец, управляет кораблем, ибо Ты дал и путь \bibemph{в море} и безопасную стезю в волнах,
\vs Wis 14:4 показывая, что Ты можешь от всего спасать, хотя бы кто отправлялся \bibemph{в море} и без искусства.
\vs Wis 14:5 Ты хочешь, чтобы не тщетны были дела Твоей премудрости; поэтому люди вверяют свою жизнь малейшему дереву и спасаются, проходя по волнам на ладье.
\vs Wis 14:6 Ибо и вначале, когда погубляемы были гордые исполины, надежда мира, управленная Твоею рукою, прибегнув к кораблю, оставила миру семя рода.
\vs Wis 14:7 Благословенно дерево, чрез которое бывает правда!
\vs Wis 14:8 А это рукотворенное проклято и само, и сделавший его~--- за то, что сделал; а это тленное названо богом.
\vs Wis 14:9 Ибо равно ненавистны Богу и нечестивец и нечестие его;
\vs Wis 14:10 и сделанное вместе со сделавшим будет наказано.
\vs Wis 14:11 Посему и на идолов языческих будет суд, так как они среди создания Божия сделались мерзостью, соблазном душ человеческих и сетью ногам неразумных.
\vs Wis 14:12 Ибо вымысл идолов~--- начало блуда, и изобретение их~--- растление жизни.
\vs Wis 14:13 Не было их вначале, и не во веки они будут.
\vs Wis 14:14 Они вошли в мир по человеческому тщеславию, и потому близкий сужден им конец.
\vs Wis 14:15 Отец, терзающийся горькою скорбью о рано умершем сыне, сделав изображение его, как уже мертвого человека, затем стал почитать его, как бога, и передал подвластным тайны и жертвоприношения.
\vs Wis 14:16 Потом утвердившийся временем этот нечестивый обычай соблюдаем был, как закон, и по повелениям властителей изваяние почитаемо было, как божество.
\vs Wis 14:17 Кого в лицо люди не могли почитать по отдаленности жительства, того отдаленное лицо они изображали: делали видимый образ почитаемого царя, дабы этим усердием польстить отсутствующему, как бы присутствующему.
\vs Wis 14:18 К усилению же почитания и от незнающих поощряло тщание художника,
\vs Wis 14:19 ибо он, желая, может быть, угодить властителю, постарался искусством сделать подобие покрасивее;
\vs Wis 14:20 а народ, увлеченный красотою отделки, незадолго пред тем почитаемого, как человека, признал теперь божеством.
\vs Wis 14:21 И это было соблазном для людей, потому что они, покоряясь или несчастью, или тиранству, несообщимое Имя прилагали к камням и деревам.
\vs Wis 14:22 Потом не довольно было для них заблуждаться в познании о Боге, но они, живя в великой борьбе невежества, такое великое зло называют миром.
\vs Wis 14:23 Совершая или детоубийственные жертвы, или скрытные тайны, или \bibemph{заимствованные} от чужих обычаев неистовые пиршества,
\vs Wis 14:24 они не берегут ни жизни, ни чистых браков, но один другого или коварством убивает, или прелюбодейством обижает.
\vs Wis 14:25 Всеми же без различия обладают кровь и убийство, хищение и коварство, растление, вероломство, мятеж, клятвопреступление, расхищение имуществ,
\vs Wis 14:26 забвение благодарности, осквернение душ, превращение полов, бесчиние браков, прелюбодеяние и распутство.
\vs Wis 14:27 Служение идолам, недостойным именования, есть начало и причина, и конец всякого зла,
\vs Wis 14:28 ибо они или веселясь неистовствуют, или прорицают ложь, или живут беззаконно, или скоро нарушают клятву.
\vs Wis 14:29 Надеясь на бездушных идолов, они не думают быть наказанными за то, что несправедливо клянутся.
\vs Wis 14:30 Но за то и другое придет на них осуждение, \bibemph{и за то}, что нечестиво мыслили о Боге, обращаясь к идолам, и \bibemph{за то}, что ложно клялись, коварно презирая святое.
\vs Wis 14:31 Ибо не сила тех, которыми они клянутся, но суд над согрешающими следует всегда за преступлением неправедных.
\vs Wis 15:1 Но Ты, Бог наш, благ и истинен, долготерпелив и управляешь всем милостиво.
\vs Wis 15:2 Если мы и согрешаем, мы~--- Твои, признающие власть Твою; но мы не будем грешить, зная, что мы признаны Твоими.
\vs Wis 15:3 Знать Тебя есть полная праведность, и признавать власть Твою~--- корень бессмертия.
\vs Wis 15:4 Не обольщает нас лукавое человеческое изобретение, ни бесплодный труд художников~--- изображения, испещренные различными красками,
\vs Wis 15:5 взгляд на которые возбуждает в безумных похотение и вожделение к бездушному виду мертвого образа.
\vs Wis 15:6 И делающие, и похотствующие, и чествующие суть любители зла, достойные таких надежд.
\vs Wis 15:7 Горшечник мнет мягкую землю, заботливо лепит всякий \bibemph{сосуд} на службу нашу; из одной и той же глины выделывает сосуды, потребные и для чистых дел и для нечистых~--- все одинаково; но какое каждого из них употребление, судья~--- тот же горшечник.
\vs Wis 15:8 И суетный труженик из той же глины лепит суетного бога, тогда как сам недавно родился из земли и вскоре пойдет туда же, откуда он взят, и взыщется с него долг души его.
\vs Wis 15:9 Но у него забота не о том, что он должен много трудиться, и не о том, что жизнь его кратка; но он соревнует художникам золотых и серебряных изделий, и подражает медникам, и вменяет себе в славу, что делает мерзости.
\vs Wis 15:10 Сердце его~--- пепел, и надежда его ничтожнее земли, и жизнь его презреннее грязи;
\vs Wis 15:11 ибо он не познал Сотворившего его и вдунувшего в него деятельную душу и вдохнувшего в него дух жизни.
\vs Wis 15:12 Они считают жизнь нашу забавою и житие прибыльною торговлею, ибо говорят, что должно же откуда-либо извлекать прибыль, хотя бы и из зла.
\vs Wis 15:13 Впрочем такой более всех знает, что он грешит, делая из земляного вещества бренные сосуды и изваяния.
\vs Wis 15:14 Самые же неразумные из всех и беднее умом самых младенцев~--- враги народа Твоего, угнетающие его,
\vs Wis 15:15 потому что они почитают богами всех идолов языческих, у которых нет употребления ни глаз для зрения, ни ноздрей для привлечения воздуха, ни ушей для слышания, ни перстов рук для осязания и которых ноги негодны для хождения.
\vs Wis 15:16 Хотя человек сделал их, и заимствовавший дух образовал их, но никакой человек не может образовать бога, как он сам.
\vs Wis 15:17 Будучи смертным, он делает нечестивыми руками мертвое, поэтому он превосходнее божеств своих, ибо он жил, а те~--- никогда.
\vs Wis 15:18 Притом они почитают животных самых отвратительных, которые по бессмыслию сравнительно хуже всех.
\vs Wis 15:19 Они даже некрасивы по виду, как \bibemph{другие} животные, чтобы могли привлекать к себе, но лишены и одобрения Божия и благословения Его.
\vs Wis 16:1 Посему они достойно были наказаны чрез подобных \bibemph{животных} и терзаемы множеством чудовищ.
\vs Wis 16:2 Вместо такого наказания Ты благодетельствовал народу Твоему: в удовлетворение прихоти их Ты приготовил им в насыщение необычайную пищу~--- перепелов,
\vs Wis 16:3 дабы те, мучимые голодом, по отвратительному виду насланных \bibemph{гадов}, отказывали и необходимому позыву на пищу, а эти, кратковременно потерпев недостаток, вкусили необычайной пищи.
\vs Wis 16:4 Ибо тех притеснителей должен был постигнуть неотвратимый недостаток, а этим только нужно было показать, как мучились враги их.
\vs Wis 16:5 И тогда, как постигла их ужасная ярость зверей и они были истребляемы угрызениями коварных змиев, гнев Твой не продолжился до конца.
\vs Wis 16:6 Но они были смущены на краткое время для вразумления, получив знамение спасения на воспоминание о заповеди закона Твоего,
\vs Wis 16:7 ибо обращавшийся исцелялся не тем, на что взирал, но Тобою, Спасителем всех.
\vs Wis 16:8 И этим Ты показал врагам нашим, что Ты~--- избавляющий от всякого зла:
\vs Wis 16:9 ибо их убивали уязвления саранчи и мух, и не нашлось врачевства для души их, потому что они достойны были мучения от сих.
\vs Wis 16:10 А сынов Твоих не одолели и зубы ядовитых змиев, ибо милость Твоя пришла на помощь и исцелила их.
\vs Wis 16:11 Хотя они и были уязвляемы в напоминание им слов Твоих, но скоро были и исцеляемы, дабы, впав в глубокое забвение \bibemph{оных}, не лишились Твоего благодеяния.
\vs Wis 16:12 Не трава и не пластырь врачевали их, но Твое, Господи, всеисцеляющее слово.
\vs Wis 16:13 Ты имеешь власть жизни и смерти и низводишь до врат ада и возводишь.
\vs Wis 16:14 Человек по злобе своей убивает, но не может возвратить исшедшего духа и не может призвать взятой души.
\vs Wis 16:15 А Твоей руки невозможно избежать,
\vs Wis 16:16 ибо нечестивые, отрекшиеся познать Тебя, наказаны силою мышцы Твоей, быв преследуемы необыкновенными дождями, градами и неотвратимыми бурями и истребляемы огнем.
\vs Wis 16:17 Но самое чудное было то, что огонь сильнее оказывал действие в воде, все погашающей, ибо самый мир есть поборник за праведных.
\vs Wis 16:18 Иногда пламя укрощалось, чтобы не сжечь животных, посланных на нечестивых, и чтобы они, видя это, познали, что преследуются судом Божиим.
\vs Wis 16:19 А иногда и среди воды жгло сильнее огня, дабы истребить произведения земли неправедной.
\vs Wis 16:20 Вместо того народ Твой Ты питал пищею ангельскою и послал им, нетрудящимся, с неба готовый хлеб, имевший всякую приятность по вкусу каждого.
\vs Wis 16:21 Ибо свойство пищи Твоей показывало Твою любовь к детям и в удовлетворение желания вкушающего изменялось по вкусу каждого.
\vs Wis 16:22 А снег и лед выдерживали огонь и не таяли, дабы они знали, что огонь, горящий в граде и блистающий в дождях, истреблял плоды врагов.
\vs Wis 16:23 Но тот же огонь, дабы напитались праведные, терял свою силу.
\vs Wis 16:24 Ибо тварь, служа Тебе, Творцу, устремляется к наказанию нечестивых и утихает для благодеяния верующим в Тебя.
\vs Wis 16:25 Посему и тогда она, изменяясь во всё, повиновалась Твоей благодати, питающей всех, по желанию нуждающихся,
\vs Wis 16:26 дабы сыны Твои, которых Ты, Господи, возлюбил, познали, что не роды плодов питают человека, но слово Твое сохраняет верующих в Тебя.
\vs Wis 16:27 Ибо неповреждаемое огнем, будучи согреваемо слабым солнечным лучом, тотчас растаявало,
\vs Wis 16:28 дабы известно было, что должно предупреждать солнце благодарением Тебе и обращаться к Тебе на восток света.
\vs Wis 16:29 Ибо надежда неблагодарного растает, как зимний иней, и выльется, как негодная вода.
\vs Wis 17:1 Велики и непостижимы суды Твои, посему ненаученные души впали в заблуждение.
\vs Wis 17:2 Ибо беззаконные, которые задумали угнетать святой народ, узники тьмы и пленники долгой ночи, затворившись в домах, скрывались от вечного Промысла.
\vs Wis 17:3 Думая укрыться в тайных грехах, они, под темным покровом забвения, рассеялись, сильно устрашаемые и смущаемые призраками,
\vs Wis 17:4 ибо и самое потаенное место, заключавшее их, не спасало их от страха, но страшные звуки вокруг них приводили их в смущение, и являлись свирепые чудовища со страшными лицами.
\vs Wis 17:5 И никакая сила огня не могла озарить, ни яркий блеск звезд не в состоянии был осветить этой мрачной ночи.
\vs Wis 17:6 Являлись им только сами собою горящие костры, полные ужаса, и они, страшась невидимого~--- призрака, представляли себе видимое еще худшим.
\vs Wis 17:7 Пали обольщения волшебного искусства, и хвастовство мудростью подверглось посмеянию,
\vs Wis 17:8 ибо обещавшиеся отогнать от страдавшей души ужасы и страхи, сами страдали позорною боязливостью.
\vs Wis 17:9 И хотя никакие устрашения не тревожили их, но, преследуемые брожениями ядовитых зверей и свистами пресмыкающихся, они исчезали от страха, боясь взглянуть даже на воздух, от которого никуда нельзя убежать,
\vs Wis 17:10 ибо осуждаемое собственным свидетельством нечестие боязливо и, преследуемое совестью, всегда придумывает ужасы.
\vs Wis 17:11 Страх есть не что иное, как лишение помощи от рассудка.
\vs Wis 17:12 Чем меньше надежды внутри, тем больше представляется неизвестность причины, производящей мучение.
\vs Wis 17:13 И они в эту истинно невыносимую и из глубин нестерпимого ада исшедшую ночь, располагаясь заснуть обыкновенным сном,
\vs Wis 17:14 то были тревожимы страшными призраками, то расслабляемы душевным унынием, ибо находил на них внезапный и неожиданный страх.
\vs Wis 17:15 Итак, где кто тогда был застигнут, делался пленником и заключаем был в эту темницу без оков.
\vs Wis 17:16 Был ли то земледелец или пастух, или занимающийся работами в пустыне, всякий, быв застигнут, подвергался этой неизбежной судьбе,
\vs Wis 17:17 ибо все были связаны одними неразрешимыми узами тьмы. Свищущий ли ветер, или среди густых ветвей сладкозвучный голос птиц, или сила быстро текущей воды, или сильный треск низвергающихся камней,
\vs Wis 17:18 или незримое бегание скачущих животных, или голос ревущих свирепейших зверей, или отдающееся из горных углублений эхо, \bibemph{все это}, ужасая их, повергало в расслабление.
\vs Wis 17:19 Ибо весь мир был освещаем ясным светом и занимался беспрепятственно делами;
\vs Wis 17:20 а над ними одними была распростерта тяжелая ночь, образ тьмы, имевшей некогда объять их; но сами для себя они были тягостнее тьмы.
\vs Wis 18:1 А для святых Твоих был величайший свет. И те, слыша голос их, а образа не видя, называли их блаженными, потому что они не страдали.
\vs Wis 18:2 А за то, что, быв прежде обижаемы ими, не мстили им, благодарили и просили прощения в том, что заставляли переносить их.
\vs Wis 18:3 Вместо того, Ты дал им указателем на незнакомом пути огнесветлый столп, а для благополучного странствования~--- безвредное солнце.
\vs Wis 18:4 Ибо те достойны были лишения света и заключения во тьме, потому что держали в заключении сынов Твоих, чрез которых имел быть дан миру нетленный свет закона.
\vs Wis 18:5 Когда определили они избить детей святых, хотя одного сына покинутого и спасли, в наказание за то Ты отнял множество их детей и самих всех погубил в сильной воде.
\vs Wis 18:6 Та ночь была предвозвещена отцам нашим, дабы они, твердо зная обетования, каким верили, были благодушны.
\vs Wis 18:7 И народ Твой ожидал как спасения праведных, так и погибели врагов,
\vs Wis 18:8 ибо, чем Ты наказывал врагов, тем самым возвеличил нас, которых Ты призвал.
\vs Wis 18:9 Святые дети добрых тайно совершали жертвоприношение и единомысленно постановили божественным законом, чтобы святые равно участвовали в одних и тех же благах и опасностях, когда отцы уже воспевали хвалы.
\vs Wis 18:10 С противной же стороны отдавался нестройный крик врагов, и разносился жалобный вопль над оплакиваемыми детьми.
\vs Wis 18:11 Одинаковым судом был наказан раб с господином, и простолюдин терпел одно и то же с царем:
\vs Wis 18:12 все вообще имели бесчисленных мертвецов, \bibemph{умерших} одинаковою смертью; и живых недоставало для погребения, так как в одно мгновение погублено было \bibemph{все} драгоценнейшее их поколение.
\vs Wis 18:13 И не верившие ничему ради чародейства, при погублении первенцев, признали, что \bibemph{этот} народ есть сын Божий,
\vs Wis 18:14 ибо, когда все окружало тихое безмолвие и ночь в своем течении достигла средины,
\vs Wis 18:15 сошло с небес от царственных престолов на средину погибельной земли всемогущее слово Твое, как грозный воин.
\vs Wis 18:16 Оно несло острый меч~--- неизменное Твое повеление и, став, наполнило все смертью: оно касалось неба и ходило по земле.
\vs Wis 18:17 Тогда вдруг сильно встревожили их мечты сновидений, и наступили неожиданные ужасы;
\vs Wis 18:18 и, будучи поражаем~--- один там, другой тут, полумертвый объявлял причину, по которой он умирал:
\vs Wis 18:19 ибо встревожившие их сновидения предварительно показали \bibemph{им} это, чтобы они не погибли, не зная того, за что терпят зло.
\vs Wis 18:20 Хотя искушение смерти коснулось и праведных, и много их погибло в пустыне, но недолго продолжался этот гнев,
\vs Wis 18:21 ибо непорочный муж поспешил защитить их; принеся оружие своего служения, молитву и умилостивление кадильное, он противостал гневу и положил конец бедствию, показав тем, что он слуга Твой.
\vs Wis 18:22 Он победил истребителя не силою телесною и не действием оружия, но словом покорил наказывавшего, воспомянув клятвы и заветы отцов.
\vs Wis 18:23 Ибо, когда уже грудами лежали мертвые одни на других, он, став в средине, остановил гнев и пресек \bibemph{ему} путь к живым.
\vs Wis 18:24 На подире его был целый мир, и славные \bibemph{имена} отцов были вырезаны на камнях в четыре ряда, и величие Твое~--- на диадиме головы его.
\vs Wis 18:25 Этому уступил истребитель, и этого убоялся: ибо довольно было одного этого испытания гневного.
\vs Wis 19:1 А над нечестивыми до конца тяготел немилостивый гнев, ибо Он предвидел и будущие их \bibemph{дела},
\vs Wis 19:2 что они, позволив им отправиться и с поспешностью выслав их, раскаются и погонятся за ними,
\vs Wis 19:3 ибо, еще имея в руках печали и рыдая над гробами мертвых, они возымели другой безумный помысл, и тех, кого с мольбою высылали, преследовали, как беглецов.
\vs Wis 19:4 Влекла же их к тому концу судьба, которой они были достойны, и она навела забвение о случившемся, дабы они восполнили наказание, недостававшее к их мучениям,
\vs Wis 19:5 и дабы народ Твой совершил славное путешествие, а они нашли себе необычайную смерть.
\vs Wis 19:6 Ибо вся тварь снова свыше преобразовалась в своей природе, повинуясь особым повелениям, дабы сыны Твои сохранились невредимыми.
\vs Wis 19:7 Явилось облако, осеняющее стан, а где стояла прежде вода, показалась сухая земля, из Чермного моря~--- беспрепятственный путь, и из бурной пучины~--- зеленая долина.
\vs Wis 19:8 Покрываемые Твоею рукою, они прошли по ней всем народом, видя дивные чудеса.
\vs Wis 19:9 Они паслись как кони и играли как агнцы, славя Тебя, Господи, Избавителя их,
\vs Wis 19:10 ибо они еще помнили о том, что случилось во время пребывания их там, как земля вместо рождения \bibemph{других} животных произвела скнипов и река вместо рыб извергла множество жаб.
\vs Wis 19:11 А после они увидели и новый род птиц, когда, увлекшись пожеланием, просили приятной пищи,
\vs Wis 19:12 ибо в утешение им налетели с моря перепелы, а грешных постигли наказания не без знамений, бывших силою молний. Они справедливо страдали за свою злобу,
\vs Wis 19:13 ибо они более сильную питали ненависть к чужеземцам: иные не принимали незнаемых странников, а эти порабощали благодетельных пришельцев.
\vs Wis 19:14 И мало этого, но еще будет суд на них за то, что те враждебно принимали чужих,
\vs Wis 19:15 а эти, с радостью приняв, потом уже пользовавшихся одинаковыми правами стали угнетать ужасными работами.
\vs Wis 19:16 Посему они поражены были слепотою, как те \bibemph{некогда} при дверях праведника, когда, будучи объяты густою тьмою, искали каждый входа в его двери.
\vs Wis 19:17 Самые стихии изменились, как в арфе звуки изменяют свой характер, всегда оставаясь теми же звуками; это можно усмотреть чрез тщательное наблюдение бывшего.
\vs Wis 19:18 Ибо земные \bibemph{животные} переменялись в водяные, а плавающие в водах выходили на землю.
\vs Wis 19:19 Огонь в воде удерживал свою силу, а вода теряла угашающее свое свойство;
\vs Wis 19:20 пламя, наоборот, не вредило телам бродящих удоборазрушимых животных, и не таял легко растаявающий снеговидный род небесной пищи.
\vs Wis 19:21 Так, Господи, Ты во всем возвеличил и прославил народ Твой, и не оставлял его, но во всякое время и на всяком месте пребывал с ним.

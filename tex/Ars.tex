\bibbookdescr{Ars}{
  inline={Письмо Аристея},
  toc={Письмо Аристея},
  bookmark={Письмо Аристея},
  header={Письмо Аристея},
  abbr={Арист}
}
\chhdr{Аристей Филократу.}
\vs Ars 1:1
Так как у нас имеется заслуживающее внимания повествование о посольстве к иудейскому первосвященнику Елеазару, а ты, Филократ, при всяком случае напоминал, что считаешь важным знать, для чего и почему мы были посланы, то я, зная твою любознательность, попытался изобразить тебе.
\vs Ars 1:2
Самое важное для человека это всегда учиться и приобретать что-либо новое, путем ли исторических повествований, или путем собственного опыта. Ибо чистое настроение души приобретается в том случаe, если она, усвоив прекраснейшее и одобрив то, что важнее всего, устраивает благочестие при помощи твердого правила.
\vs Ars 1:3
Имея склонность к тщательному размышлению о божественном, мы посвятили себя посольству к выше упомянутому мужу, который своим благородством и славой снискал особую честь как у сограждан, так и иноземцев, и принес величайшую пользу иудеям Палестины и других мест переводом божественного Закона, потому что написан у них на пергаменте еврейскими буквами.
\vs Ars 1:4
Это-то мы и выполнили со всяким тщанием. Следует сообщить тебе и о том, что мы говорили царю получив удобный случай о переселенных в Египет из Иудеи отцом царя, прежнем владельце столицы и владыке Египта.
\vs Ars 1:5
Я убежден, что ты, имея значительное расположениe к нравственной чистоте и душевной настроенности мужей, живущих согласно священному законодательству, охотно услышишь о том, что мы желаем сообщить, так как ты недавно приходил к нам с острова и выражал желание узнать о том, что способствует исправлению души.
\vs Ars 1:6
И ранее я отправил тебе описание замечательного, по моему мнению, об иудейском народе, полученное нами от ученейших жрецов в Египте.
\vs Ars 1:7
А так как ты любознателен в том, что может принести пользу душе, то необходимо передать по преимуществу всем единомышленникам, а тем более тебе, ибо у тебя подлинное расположение; ты брат не только по родству, но и по настроенности, влечение к прекрасному у нас одно и то же.
\vs Ars 1:8
Ведь удовольствие от золота, или какое-либо иное имущество, почитаемое пустыми, не приносит столько пользы, как образование и забота о нем. Дабы не впасть в мнoгocлoвиe, удлиняя предисловие, мы вернемся к дальнейшему ходу повествования.
\vs Ars 1:9
Димитрий Фалирей, заведующий царской библиотекой, получил крупные суммы на то, чтобы собрать, по возможности, все книги мира. Скупая и снимая копии, он, по мере сил, довел до конца желание царя.
\vs Ars 1:10
Однажды в нашем присутствии он был спрошен, сколько у него тысяч книг, и ответил: свыше двухсот тысяч, царь, а в непродолжительном времени я позабочусь об остальных, чтобы довести до пятисот тысяч. Но мне сообщают, что и законы иудеев заслуживают того, чтобы их переписать и иметь в твоей библиотеке.
\vs Ars 1:11
Что же препятствует тeбе, спросил, сделать это? Ведь в твоем распоряжении есть всё, касающееся этого дела!. Димитрий ответил: необходим еще перевод, так как среди иудеев пользуются особым письмом, подобно тому как египтяне своим расположением букв, почему имеют и особый язык. Предполагают, что говорят на сирийском, но их не этот, а иного типа. Узнав обо всем, царь повелел написать иудейскому первосвященнику, чтобы привел в исполнение этот план.
\vs Ars 1:12
А я, считая настоящий момент удобным, просил начальников телохранителей Сосивия тарентинца и Андрея о том же, о чем часто об освобождении переселенных из иудеи отцом царя действительно, он столь же удачно, как и храбро напал на всю территорию нижней Сирии и Финикии и одних переселил, а других взял в плен, всё подчинив, благодаря страху. В это время он и переселил около ста тысяч из Иудеи в Египет.
\vs Ars 1:13
Около тридцати тысяч из них, лучших воинов, он, вооружив, поселил в крепостях своей страны (хотя много и раньше прибыло с персидским царем, а до этого и иные были отправлены на помощь Псаммитиху, чтобы сражаться против эфиопского царя, но их прибыло не так много, как переселил Птолемей, сын Лага).
\vs Ars 1:14
Выбрав, как мы сказали, цветущих возрастом и отличающихся силою, он вооружил, а остальную массу старцев, юношей, а также женщин, он обратил в рабство, не столько по собственному желанию, сколько по требованию воинов, за услуги, которые они оказали на войнe. А так как мы, о чем ранее сказано, получили известный предлог к освобождению их, то обратились к царю с такими словами:
\vs Ars 1:15
Царь, не будь настолько безразсуден, чтобы тебя обличали сами факты. Ведь законодательство, которое мы намереваемся не только переписать, но и перевести, имеет силу для всех иудеев; какое же основание у нас будет для отправления, если в твоем царстве огромная масса находится в рабстве? Освободи же, по совершенству и богатству души, угнетаемых бедствиями, ибо, как я тщательно изследовал, Бог, управляющий твоим царством, даровал закон и им.
\vs Ars 1:16
Они, царь, чтут Зрителя всяческих и создателя Бога, Которого почитают и все, а мы иначе называем Его Зевсом и Дием. Древние дали это удачное наименование Тому, Кем оживотворяется и создано всё; Он же управляет и владычествует над всем. А так как величием души ты превосходишь всех людей, то освободи находящихся в рабстве.
\vs Ars 1:17
Подождав немного, когда мы в душе молились Богу, чтобы Он внушил ему мысль об освобождении всех, ибо человеческий род, творение Бога, Он переменяет и снова изменяет. Поэтому я часто и многообразно призывал Владыку сердец, чтобы Он побудил исполнить то, чего я просил.
\vs Ars 1:18
A выступая с речью о спасении людей, я твердо надеялся, что Бог исполнит просимое; ибо, если люди делают по благочестию то, что, по их мнению относится к справедливости и попечению о прекрасном, то владычествующий над всем Бог руководит их действиями и намерениями он, подняв голову и милостиво взглянув, спросил:
\vs Ars 1:19
сколько будет тысяч, по твоему мнению? Присутствовавший Андрей ответил: немногим более ста тысяч. Царь сказал: немногого же просит у нас Аристей. А Сосивий и некоторые из присутствующих ответили это: действительно, величия твоей души достойно принести величайшему Богу в качестве благодарственной жертвы освобождение их. Так как Владыкой всяческих ты удостоен высочайшей чести и прославлен более твоих предков, то тебе следует принести в благодарность и величайшую жертву.
\vs Ars 1:20
Сильно обрадованный, приказал добавить к жалованью и за каждого человека получать по двадцать драхм; издать об этом указ, а списки изготовить немедленно. Он обнаружил величайшее расположение, ибо Бог исполнил все наши желания и побудил его освободить не только тех, которые пришли с войском его отца, но и тех, которые жили ранее, или впоследствие были приведены в государство, хотя ему и заявляли, что дар обойдется более четырехсот талантов.
\vs Ars 1:21
A копия указа была сделана, по моему мнению, не напрасно, ведь великодушие царя будет гораздо яснее и очевиднее, ибо Бог дал ему возможность послужить спасению множества. Содержание же указа таково:
\vs Ars 1:22
По приказанию царя, те из соратников нашего отца в Сирии и Финикии, которые при нападении на Иудею захватили пленников иудеев и переселили их в столицу и страну, или продали иным, точно также, если некоторые жили ранее, или впоследствие были приведены оттуда, владеющие должны немедленно отпустить на свободу, получив тотчас же по двадцать драхм за человека: воины при выдаче жалованья, а остальные из царской казны.
\vs Ars 1:23
Ибо по нашему мнению они были взяты в плен и вопреки воле отца нашего и вопреки благородству, а страна их была опустошена и иудеи были переселены в Египет вследствие запальчивости воинов. Ведь добыча, захваченная воинами на поле брани, была достаточно велика, почему и порабощение этих людей совершенно несправедливо.
\vs Ars 1:24
Итак, воздавая, по общему мнению, справедливое всем людям, а особенно угнетаемым неразумно, и во всём стремясь к полному coглacию со справедливостью и благочестием в отношении всех, мы определяем всех иудеев нашего государства, каким бы то ни было образом в рабстве, отпустить на свободу, уплатив владельцам назначенную сумму. Никто не должен медлить исполнением этого; а списки доставить назначенным для этого в течение трех дней со времени издания настоящего указа, предъявляя вместе с тем и самих людей.
\vs Ars 1:25
Ибо мы решили, что осуществление этого полезно и нам и государству. А о неповинующихся должен доносить всякий желающий, с условием, что он станет господином того, кто окажется виновным, а имущество таковых будет взято в царскую казну.
\vs Ars 1:26
Когда этот указ, в котором находилось всё, кроме: и если некоторые жили ранее, или впоследствие были приведены оттуда, был подан царю для просмотра, то он, по своему благородству и великодушию, это добавил сам и приказал дать назначение казначеям легионов и царским менялам на всю сумму издержек.
\vs Ars 1:27
В таком виде это постановление было утверждено в течение семи дней, а выкупная сумма достигла более шестисот талантов, ибо много и грудных детей было освобождено вместе с их матерями. Когда же к царю обратились с запросом, выдавать ли и за них по двадцать драхм, то царь приказал делать и это, выполняя во всём его волю полностью.
\vs Ars 1:28
Когда это было исполнено, приказал Димитрию сделать доклад относительно копии иудейских книг. (Ибо эти цари управляли всем при посредстве указов и с великой осмотрительностью. Вот почему я помещаю копии доклада и писем, также количество отправленного и устройство их, так как все они отличались роскошью и искусством.) Копия доклада такова:
\vs Ars 1:29
Великому царю от Димитрия.
Так как ты, царь, для пополнения отсутствующих в твоей библиотеке книг приказал собрать, а распавшиеся надлежащим образом исправить, то я, тщательно потрудившись над этим, доношу тебе следующее:
\vs Ars 1:30
отсутствуют, наряду с немногими другими, книги иудейского закона. Они написаны еврейскими буквами и языком, но, как сообщают знающие, слишком небрежно и не так, как должно, ибо не привлекали царского внимания.
\vs Ars 1:31
Teбе необходимо иметь у себя и эти, но тщательно исправив, ибо это законодательство, как божественное, чисто и исполнено мудрости? Поэтому, прозаики, поэты и многие историки были далеки от упоминания о названных книгах и о мужах, которые управлялись на основании их, так как, по словам Экатея Авдиритского, учение в них чисто и священно.
\vs Ars 1:32
Итак, если, царь, угодно, пусть напишут первосвященнику в Иерусалиме, чтобы он прислал старцев особенно добродетельной жизни, сведущих в своём Законе, по шести от каждого колена, чтобы, достигнув coгласия по большинству и получив точный перевод, мы положили на видном месте, достойно и самого дела и твоего намерения. Будь счастлив всегда.
\vs Ars 1:33
После этого доклада царь приказал написать об этом Елеазару, сообщив и об освобождении пленников. А для изготовления сосудов, бокалов, трапезы и чаш для возлияния он дал золота весом пятьдесят талантов, серебра семьдесят талантов и достаточное количество драгоценных камней (ибо он приказал хранителям сокровищ, чтобы они предоставили мастерам выбирать, что те пожелают) и денег для жертв и на остальное около ста талантов.
\vs Ars 1:34
Об изготовлении мы скажем после того, как передадим копии писем. Письмо царя было такого содержания:
\vs Ars 1:35
Царь Птолемей первосвященнику Елеазару радоваться и здравствовать!
Так как в нашу страну было переселено много иудеев, силою уведенных из Иеросалима персами во время их господства, а кроме того пленников прибыло в Египет и вместе с отцом нашим
\vs Ars 1:36
(большинство их он зачислил в войско на большое жалованье, равным образом и тем, которые жили paнee, он, по доверию к ним, поручил охрану построенных им крепостей, чтобы, таким образом, египтяне были в безопасности; а мы, получив царскую власть, проявили большее человеколюбие по отношению ко всем, а особенно твоим согражданам),
\vs Ars 1:37
то мы освободили более ста тысяч пленников, уплатив их господам следуемую денежную плату и исправляя вместе с тем зло, причиненное им яростью черни. Мы решили, что этим поступаем благочестиво и приносим благодарственную жертву величайшему Богу, Который сохраняет наше царство в мире и величайшей славе во всей вселенной. Зрелых возрастом мы зачислили в войско, а пригодных для нашей службы и заслуживающих доверия при дворе мы определили на должности.
\vs Ars 1:38
Желая сделать угодное и им, и иудеям всего миpa и последующим, мы предрешили перевести ваш Закон греческими буквами с букв, называемых у вас еврейскими, чтобы в нашей библиотеке, наряду с другими царскими книгами, находились и эти.
\vs Ars 1:39
Поэтому ты поступишь прекрасно и согласно нашему желанию, если выберешь старцев добродетельной жизни, сведущих в Законе и сильных в переводе, по шести от каждого колена, чтобы достигнуть согласия по большинству, ибо изследование касается очень важных предметов. Мы полагаем, что, исполнив это, ты приобретешь себе великую славу.
\vs Ars 1:40
Для этого мы посылаем Андрея, начальника телохранителей, и Аристея, которые пользуются у нас почетом; они будут вести с тобою переговоры и доставят начатки приношений в храм, а для жертв и на остальное сто талантов серебра. А сообщив нам о желаниях, ты приобретешь благосклонность и поступишь согласно дружбе, так как мы возможно скорее исполним то, что тебе угодно. Будь здоров.
\vs Ars 1:41
На это письмо Елеазар тотчас же ответил следующее:
Первосвященник Елеазар царю Птолемею, истинному другу радоваться!
Нам приятно было бы, если бы ты, царица Арсиноя, твоя сестра и дети были здоровы; этого мы и желаем, а мы здоровы.
\vs Ars 1:42
Получив от тебя письмо, мы весьма возрадовались твоему намерению и прекрасному желанию; собрав весь народ, мы прочли ему, чтобы он знал о твоем благоговении к нашему Богу. Мы показали и присланные тобою бокалы, двадцать золотых и тридцать серебряных, пять сосудов, трапезу для возношения и сто талантов серебра для принесения жертв и необходимых исправлений в храме.
\vs Ars 1:43
Это доставили пользующиеся у тебя почетом Андрей и Аристей, мужи добрые, прекрасные, отличающиеся образованием и во всём достойные твоего настроения и справедливости. Они-то и передали нам твоё, на что и с нашей стороны услышали соответствующее твоему письму.
\vs Ars 1:44
Мы исполним всё, что полезно для тебя, даже если бы это было противно природе (ведь это свидетельствует о дружбе и любви), ибо и ты оказал нашим согражданам великие, разнообразные и никогда не забываемые благодеяния.
\vs Ars 1:45
Поэтому мы тотчас же принесли жертвы за тебя, твою сестру, детей и любезных), и весь народ молился, чтобы исполнилось всё, что тебе угодно, чтобы владычествующий над всем Бог сохранил твоё царство в мире и славе и чтобы перевод святого Закона был сделан с пользою для тебя и тщательно.
\vs Ars 1:46
В присутствии всех мы избрали старцев, мужей добрых и благородных, из каждого колена по шести; их мы отправили вместе с Законом. А ты, праведный царь, прекрасно поступишь, приказав, чтобы эти мужи, по окончании перевода книг, снова безпрепятственно вернулись к нам. Будь здоров.
\vs Ars 1:47
Из первого колена: Иосиф, Езекия, Захария, Иоанн, Езекия, Елисей.
Из второго: Иуда, Симон, Самуил, Адей, Матафия, Есхлемия.
Из третьего: Неемия, Иосиф, Феодосий, Васея, Орния, Дакис.
\vs Ars 1:48
Из четвертого: Ионафан, Аврей, Елисей, Анания, Хаврий, 3ахария.
Из пятого: Исаак, Иаков, Иесуа, Савватий, Симон, Левий.
Из шестого: Иуда, Иосиф, Симон, Захария, Самуил, Шелемия.
\vs Ars 1:49
Из седьмого: Савватий, Седекия, Иаков, Исайя, Иесия, Натфей.
Из восьмого: Феодосий, Иасон, Иесуа, Феодот, Иоанн, Ионафан.
Из десятого: Феофил, Авраам, Арсам, Иасон, Эндемия, Даниил.
\vs Ars 1:50
Из десятого: Иеремия, Елеазар, Захария, Ванея, Елисей, Дафей.
Из одиннадцатого: Самуил, Иосиф, Иуда, Ионафан, Хавев, Досифей.
Из двенадцатого: Исаил, Иоанн, Феодосий, Арсам, Авиит, Иезекииль.
Всего семьдесят два.
\vs Ars 1:51
Таков был ответ Елеазара и его приближенных на письмо царя.
Согласно обещанию, я опишу тебе также приготовленное. Они были сделаны с необыкновенным искусством, так как царь отпустил большие средства и всегда наблюдал за мастерами. Поэтому они ничего не могли упустить из виду и сделать небрежно.
\vs Ars 1:52
Прежде всего я опишу тебе устройство трапезы. Царь желал сделать это сооружение огромных размеров; но он приказал справиться у местных, какова величина уже существующей и стоящей в Иеросалимском храме.
\vs Ars 1:53
Когда же сообщили её размеры, он снова спросил, можно ли делать большую? Некоторые из священников и другие говорили, что нет препятствий, но он сказал, что желает сделать в пять раз большую, однако опасается, что она окажется непригодной для богослужения.
\vs Ars 1:54
А он, конечно, не хотел, чтобы приготовленная им только стояла на месте; ему будет гораздо приятнее, если соответствующие службы будут совершаться, как и должно, назначенными для этого на приготовленной им.
\vs Ars 1:55
Для прежней трапезы были указаны меньшие размеры не по недостатку золота, но, сказал, она была сделана таких размеров по известным, как кажется, основаниям. А если бы оказалось необходимым увеличить её, то ни в чем не было бы недостатка. Поэтому не следует ни уменьшать, ни увеличивать удачно избранные.
\vs Ars 1:56
Итак приказал широко пользоваться различными искусствами, ибо он всё замышлял в величественных чертах и обладал природной способностью представлять предметы в их готовом виде. Что не было записано, он приказал делать согласно с красотой, а что было указано, в этом следовать размерам.
\vs Ars 1:57
Из чистого золота было сделано массивное сооружение длиною в два локтя, шириною в один локоть и высотою в полтора локтя; говорю же не о накладном золоте, но о том, что была положена массивная доска.
\vs Ars 1:58
Вокруг был сделан ободок, шириною в ладонь, с витыми бортами, украшенными рельефной плетеной резьбой, удивительно искуссно выгравированной с трех сторон, ибо был треугольным.
\vs Ars 1:59
На каждой стороне работа была выполнена одинаково, так что, в какую бы сторону ни поворачивать, вид был один и тот же. Но в то время, как художественная работа под ободком была обращена к трапезе, наружная поверхность была видима приходящему.
\vs Ars 1:60
Поэтому верхний край с обоих сторон был острым, ибо, как сказано ранее, был сделан треугольным (в какую бы сторону его ни поворачивать). Посредине плетения в него были вставлены различные драгоценные камни, прикрепленные один к другому с неподражаемым искусством.
\vs Ars 1:61
Все они для безопасности были укреплены в отверстиях золотыми гвоздями, а на углах для прочности оправы связывались вместе.
\vs Ars 1:62
По бокам у ободка в верхней части кругом было сделано всё усеянное драгоценными камнями подобие яиц), изображенное выступами при помощи сплошного барельефа в виде полос, плотно прилегающих одна к другой вокруг всей трапезы.
\vs Ars 1:63
A под изображенным из драгоценных камней подобием яиц художники превосходно и очень отчетливо сделали венок, изобилующий всякими плодами: виноградными кистями, колосьями, финиками, масличными ягодами, гранатовыми яблоками и т. п. Для изображения этих плодов они употребили камни, соответствующие цвету каждого плода и прикрепили их к золотому кольцу вокруг всей трапезы, сбоку её.
\vs Ars 1:64
Украсив ободок, они внизу под изображением подобия яиц устроили таким же образом и остальные части рельефных украшений и резьбы, так что трапеза была сделана для пользования с обоих сторон, с какой угодно. Были сделаны и борты и ободок в нижней части у ножек.
\vs Ars 1:65
По всей ширине трапезы они сделали массивную доску в четыре пальца толщиною, в неё были вставлены ножки и под ободком укреплены шипами, находящимися в углублениях, чтобы можно было пользоваться с какой угодно стороны. Это можно было видеть на верхней доске, так как это произведение было устроено для употребления с обоих сторон.
\vs Ars 1:66
На самой же трапезе превосходно и рельефно изобразили мэандр), посредине которого находилось множество драгоценных камней разных пород: рубины, смарагды, ониксы и другие породы камней превосходного качества.
\vs Ars 1:67
Под изображением мэандра находилась сделанная удивительно искусно сетка, посредине имеющая узор в форме ромба. В него были вставлены горный хрусталь и так называемый янтарь, производя необычайное впечатление на зрителей.
\vs Ars 1:68
Ножки были сделаны в форме головок лилий, под трапезой лилии загибались, а с лицевой стороны имели прямые листья.
\vs Ars 1:69
Основание ножек на полу из рубина и всюду имело четыре пальца, с лицевой стороны имея форму башмака в восемь пальцев ширины. На нем и удерживалась вся тяжесть ножек.
\vs Ars 1:70
Вырезали из камня плющ, обвитый тернием и виноградной лозой, которая вместе с виноградными кистями, вытесанными из камня, окружала ножки до верху. Таково было устройство четырех ножек. Всё было сделано и выступало ясно; опытность и искусство неизменно превосходили действительность, так что при дуновении воздуха листья начинали двигаться, ибо всё было сделано так, чтобы изображать действительность.
\vs Ars 1:71
Переднюю сторону трапезы сделали из трех частей, как бы в форме триптиха, и по толщине сооружения скрепили одну с другой шипами в форме гусиной лапки, так что соединение скреп не было видно и нельзя было найти. А толщина всей трапезы была не меньше полулоктя, так что на всё сооружение пошло много талантов.
\vs Ars 1:72
А так как царь не запрещал увеличивать размеров, то, если нужно было издержать больше приготовленного, царь отпускал на это и больше. Согласно его желанию всё было исполнено удивительно и достойным образом, безподобно со стороны искусства и безукоризненно в отношении красоты.
\vs Ars 1:73
Два сосуда были сделаны из золота; от основания и до середины они были покрыты чешуйчатой резьбой, а между чешуей были весьма искуссно устроены скрепы из драгоценных камней.
\vs Ars 1:74
Далее лежал мэандр высотою в локоть; он был рельефно изображен при помощи драгоценных камней различного цвета, свидетельствуя как о зрелости искусства, так и о тщательности. За ним рельефное украшение в виде ромба, которое до отверстия имело форму сети.
\vs Ars 1:75
Впечатление красоты дополняли небольшие щиты величиною не меньше четырех пальцев из различных драгоценных камней и расположенные посредине один подле другого. А по краю отверстия, кругом, были изображены лилии с цветками и виноградные лозы, переплетающиеся с виноградными кистями.
\vs Ars 1:76
Таково было устройство золотых сосудов, которые вмещали более двух метритов). Что касается серебряных, то они были сделаны гладкими, вроде зеркала, и уже это было удивительно, так как в них гораздо яснее, чем в зеркале, отражалось всё, что подносили.
\vs Ars 1:77
По сравнению с отражаемой ими действительностью их действие описать невозможно. Когда всё было окончено и предметы были поставлены один подле другого, то есть сначала серебряный сосуд, затем золотой, снова серебряный и золотой, то действие их вида было совершенно неописуемо, так что приходившие посмотреть на них не могли уйти вследствие их необычайного блеска и прелести для взоров.
\vs Ars 1:78
Впечатление от их наружного вида было разнообразно. Когда смотрели на работу из золота, являлась радость с удивлением, так как внимание непрерывно устремлялось на каждое из этих художественных произведений. А если кто, напротив, хотел взглянуть на серебряные сосуды, то они всюду и кругом начинали блестеть, где бы кто ни стоял, и доставляли зрителю еще большее удовольствие. Таким образом, изящество их совершенно нельзя описать.
\vs Ars 1:79
На золотых бокалах посредине были выгравированы венки виноградной лозы, а по краям вырезали венок, сплетенный из плюща, мирты и маслины, вставив различные драгоценные камни. Остальные части граверной работы они сделали из различных узоров, усердно стремясь всё сделать для большей славы царя.
\vs Ars 1:80
Вообще, таких роскошных и художественных предметов нет не только в царских сокровищницах, но и в ничьих других. Ибо славолюбивый царь не мало подумал над тем, чтобы всё было исполнено прекрасно.
\vs Ars 1:81
Часто он оставлял публичные аудиенции и внимательно следил за художниками, чтобы они выполняли свою работу достойно того места, куда отправлялись их произведения. Поэтому всё делалось великолепно и достойно, как царя, отправляющего, так и первосвященника, управляющего этим местом.
\vs Ars 1:82
На работу пошло великое множество драгоценных камней, притом большой величины, не менее пяти тысяч и всё отличалось художественностью исполнения, так что количество драгоценных камней и работа ювелиров стоили в пять раз дороже золота.
\vs Ars 1:83
Я сообщил тебe описание их, предполагая, что это необходимо. Дальнейшее содержит наше путешествие к Елеазару. Сначала я опишу устройство всей страны. Когда мы прибыли на место, то увидели город, лежащий посредине всех иудеев, на очень высокой гopе.
\vs Ars 1:84
На краю был построен храм превосходного вида, три стены, высотой более семидесяти локтей, а их ширина и длина соответствовали устройству храма, так как всё было построено с необыкновенными во всех отношениях великолепием и роскошью.
\vs Ars 1:85
Очевидно было, что на двери, на прикрепление их к косякам и укрепление притолоков затрачены были также огромные суммы.
\vs Ars 1:86
Устройство завесы во всем было совершено подобно дверям. Особенно приятный вид, от которого с трудом можно было оторваться, она получала при дуновении ветра, когда ткань приходила в непрерывное движение, так как дуновение от основания передавалось по складкам до верхнего края.
\vs Ars 1:87
Устройство жертвенника отвечало месту и сожигаемым на огне жертвам, точно также и подъем к нему; место это, вследствие необходимого благоприличия, имело наклон, так как священники совершали служение одетыми до пят в льняные хитоны.
\vs Ars 1:88
Храм лицом был обращен к востоку, а задней стороной на запад. Весь пол был вымощен камнем, а для стока воды от замывания жертвенной крови имел в соответствующих местах наклон; ибо в праздничные дни приводились для жертв тысячи скота.
\vs Ars 1:89
Скопление же воды было неисчерпаемо, так как внутри протекал обильный естественный источник, а под землею, кроме того, находились удивительные и неописуемые водоемы. И показывали на пять стадий вокруг основания храма безчисленные галереи каждого из них, так как потоки в каждой части соединялись друг с другом.
\vs Ars 1:90
Всё это на полу и стенах было обложено свинцом, а поверх этого покрыто толстым слоем штукатурки, так что всё было сделано прочно. Частые отверстия в полу не были известны никому, кроме служащих, как будто всё множество жертвенной крови очищалось одним движением и мановением.
\vs Ars 1:91
Объясню, насколько я, по моему убеждению, сам удостоверился, и устройство водоемов. Меня вывели за город дальше, чем на четыре стадии, и приказали, наклонившись в известном месте, прислушаться к шуму от встречи вод. Вследствие этого мне, как сказано, ясной стала величина водоемов.
\vs Ars 1:92
Служение священников по силе, а также настроению благоприличия и тишины несравненно. Все усердно трудятся по доброй воле и с великим напряжением; каждый же заботится о порученном. Они непрерывно работают: одни доставляют дрова, другие масло, иные крупинчатую муку; иные ароматы; другие сожигают части жертвенного мяса, обнаруживая необыкновенную силу.
\vs Ars 1:93
Взяв обоими руками ноги телят, каждая из которых весит почти более двух талантов, они удивительно ловко и без промаха бросают их обоими руками на значительную высоту, точно также и овец и коз, отличающихся значительным весом и тучностью. Назначенные для этого выбирают безпорочных и особенно тучных и совершается вышеуказанное.
\vs Ars 1:94
Для отдыха им назначено место, где сидят отдыхающие. В это время пробуждаются те из отдыхавших, которые имеют желание, хотя никто не приказывает им служить.
\vs Ars 1:95
А тишина такова, что можно подумать, будто здесь нет никого, хотя служащих находится около семи тысяч (количество приносящих жертвы также очень велико), но всё совершается в страхе и достойно великого Божества.
\vs Ars 1:96
Нас охватило великое изумление, когда мы увидели Елеазара в служении, его облачение и славу, которая обнаруживалась в носимом им хитоне и камнях на нем. Вокруг его подира были золотые позвонки, которые издавали своеобразные гармонические звуки, а около каждого из них разноцветные гранатовые яблочки поразительной окраски.
\vs Ars 1:97
Он быль опоясан превосходным и великолепным поясом, вытканным из красивейших цветов. На груди он носит так называемый наперсник судный, в который были вставлены оправленные в золото двенадцать камней различной породы, с расположенными согласно первоначальному порядку именами начальников колен. Каждый сверкал своим характерным и неописуемым природным блеском.
\vs Ars 1:98
На голове имеет так называемый кидар, а на нем безподобная митра, то есть святая диадема, на которой над бровями священным шрифтом на золотом листке было вырезано имя Божие, полное славы. В таком виде выходит тот, кто был признан быть достойным этого, на служение.
\vs Ars 1:99
Всё это вместе вызывало страх и трепет, так что казалось, будто приходишь в иное место, вне этого миpa. И я утверждаю, что каждый человек, приходя посмотреть на это, повергался в изумление и невыразимое удивление, так как мысль его изменялась вследствие святого устройства во всём.
\vs Ars 1:100
Чтобы узнать всё, мы производили осмотр, поднявшись на лежащую около города крепость. Она расположена на самом высоком месте и укреплена множеством башен, так как они доверху выстроены из больших каменных плит, для охраны, как мы понимаем, мест около храма
\vs Ars 1:101
(чтобы, в случаe какого-либо нападения, бунта, или вторжения неприятелей, никто не мог проникнуть за стены, окружающие храм, так как на башнях крепости есть метательные машины и разные снаряды, а место это лежит выше упомянутых ранее стен),
\vs Ars 1:102
так как эти башни охраняются наиболее надежными мужами, давшими отечеству великие доказательства. Им разрешается выходить из крепости только по праздникам, притом по частям. Точно также никого не разрешается и впускать.
\vs Ars 1:103
Большую осторожность соблюдают они, если начальник дал разрешение впустить кого-либо для осмотра. Это случилось и с нами. С трудом безоружных нас двух впустили, чтобы посмотреть на принесение жертв.
\vs Ars 1:104
Говорили, что они обязались в этом клятвою. Bcе они, числом пятьсот, поклялись (конечно, при клятве дело по необходимости выполняется по-божески) не впускать в крепость болee пяти человек одновременно. Ведь крепость является единственной защитой храма и строитель укрепил её так для охраны указанного ранее.
\vs Ars 1:105
Город средней величины, так как стена, насколько можно догадываться, имеет около сорока стадий. Башни расположены в нем в форме театра; в нижних входы не видны, а в верхних заметны; в них и выходы. Местность имеет подъем, так как город построен на горе.
\vs Ars 1:106
Ко входам лестницы; одни вверху, а другие внизу, и очень удалены от дороги, чтобы те, которые живут в чистоте, не соприкасались с недозволенным.
\vs Ars 1:107
И начальники города неслучайно построили его симметрично, но по мудром размышлении. Так как эта страна обширна и прекрасна, и одни части её ровны, как например по направлению к Самарии и граничащие с Идумеей, а другие, как например посредине страны, гористы, то необходимо постоянно возделывать и обрабатывать землю, чтобы таким путем и эти стали плодородными. Если делать это, то во всей указанной стране всё приносит обильные плоды.
\vs Ars 1:108
В городах, отличающихся своей величиной и соответствующим благоденствием, население многочисленно, а страна оставляется в пренебрежении, так как все склоняются к жизненным радостям, ибо все люди склонны к yдoвольcтвиям.
\vs Ars 1:109
Это имеет место и в Александрии, превосходящей все города своей величиной и благоденствием. Именно, те из поселян, которые, прибыв в неё погостить, остаются надолго, отвыкают от земледельческого труда.
\vs Ars 1:110
Поэтому, чтобы они не задерживались, царь разрешил оставаться не более двадцати дней. Соответственно этому он сделал письменное распоряжение чиновникам: если необходимо вызвать кого-либо, то разбираться в течение пяти дней.
\vs Ars 1:111
В виду важности дела он назначил для каждого округа судей и их помощников, чтобы земледельцы и поверенные, получая доходы, не уменьшали городских кладовых; говорю же я о земледельческих налогах.
\vs Ars 1:112
Мы уклонились в сторону, потому что Елеазар прекрасно на примерах разъяснил нам вышеизложенное. Действительно, труд при обработке земли велик. И страна их изобилует масличными деревьями, хлебными плодами, овощами, а кроме того виноградом и массою меда (других плодовых деревьев и финиковых пальм у них нет), множеством различного скота и обилием пастбищ для него.
\vs Ars 1:113
Поэтому они прекрасно обратили внимание на то, что эта страна требует многочисленного населения, и установили надлежащее отношение между городом и деревнями.
\vs Ars 1:114
Кроме того, сюда доставляется арабами масса благовоний, различных драгоценных камней и золота. Страна эта, удобная для земледелия, пригодна и для торговли, а город для: занятия различными ремеслами. Она не имеет недостатка ни в чем, что доставляется морем.
\vs Ars 1:115
Есть в ней и удобные гавани доставляющие: у Аскалона, Яфы, Газы, а также у основанной царем Птолемаиды, которая находится посредине первых, на небольшом от них разстоянии. Страна эта имеет всё в изобилии, так как всюду хорошо орошается и прочно защищена.
\vs Ars 1:116
Её окружает река, называемая Иорданом, которая никогда не пересыхает (первоначально страна эта была не менее шестидесяти миллионов арур), но впоследствие, когда соседи были вытеснены из неё, шестьсот тысяч мужей получили в удел сто арур). Разливаясь, подобно Нилу, она около времени жатвы увлажняет большую часть страны.
\vs Ars 1:117
Он впадает в другую реку, в стране Птолемеев, а эта выходит в море. Текут и иные горные потоки, охватывающие окрестности Газы и местность Азота.
\vs Ars 1:118
Охраняется самою природой, будучи недоступна для вторжения и непроходима для больших масс, так как дороги узки, её окружают утесы и глубокие ущелья, a кромe того все горы вокруг этой страны скалисты.
\vs Ars 1:119
Говорили также, что ранее в соседних горах Аравии существовали медные и железные рудники, но во время господства персов они были оставлены, так как начальники того времени пустили ложные слухи, будто разработка их безполезна и дорого обходится,
\vs Ars 1:120
чтобы вследствие добывания оных не погибла страна и, при их тирании, не отпала, тогда как путем этой клеветы они получили предлог к доступу в эту местность. Итак, брат Филократ, я указал тебе главное и сколько нужно было об этом; далее же мы изложим то, что касается перевода.
\vs Ars 1:121
Елеазар выбрал лучших мужей, отличающихся образованием и знатностью рода, которые приобрели навык не только в иудейской литературе, но тщательно позаботились и об изучении греческой.
\vs Ars 1:122
Поэтому они были пригодны для посольства и в необходимых случаях исполняли его; они обладали большими дарованиями к беседам и изследованию в области Закона, стремясь к среднему положению (ибо оно прекраснее всего); они оставили грубость и неотделанность мысли, а также пренебрегли самомнением и своим превосходством над другими; в беседах они были примером для других, как своим умением слушать, так и отвечать каждому должное; все они соблюдают это, желая в этом более всего превосходить друг друга, все быть достойными своего начальника и его добродетели.
\vs Ars 1:123
А что они любили Елеазара, видно было, как они с неохотой покидали его. И сам он не только царю написал о возвращении их, но настойчиво просил Андрея и нас содействовать, насколько можем.
\vs Ars 1:124
И хотя мы обещали внимательно заботиться о них, он говорил, что сильно безпокоится. Действительно, он знал, что любящий доброе и добрых царь выше всего ставит приглашеше таких людей, которые где-либо признаются, как отличающиеся от других своим образованием и разумом.
\vs Ars 1:125
Царь, я полагаю, прекрасно говорит, что, имея около себя мужей праведных и мудрых, он приобретет лучшую охрану для своего царства, так как друзья с полной откровенностью советуют ему полезное. А этим именно и обладали посланные Елеазара.
\vs Ars 1:126
И он клятвенно уверял, что он не отпустил бы этих людей, если бы того требовало какое-либо иное личное его дело, но он отправляет их для общего исправления всех граждан.
\vs Ars 1:127
Ибо добродетельная жизнь заключается в соблюдении законов, а это гораздо лучше достигается путем слушания, чем путем чтения. Итак, предлагая их и подобное им, Елеазар ясно показывал свое расположение к ним.
\vs Ars 1:128
Следует вкратце упомянуть и о том, что Елеазар ответил нам на наши вопросы (ибо я полагаю, что многиe серьезно интересуются некоторыми из законов о пище и питье, а также о животных, признаваемых нечистыми).
\vs Ars 1:129
Итак, когда мы спросили, почему, несмотря на одинаковое происхождение, одни считаются нечистыми для еды, а другие и для прикосновения (ибо большинство из Закона отличается суеверием, а в этих частях полным), то он начал на это следующее.
\vs Ars 1:130
Ты видишь, сказал он, как влияют образ жизни и знакомства; поддерживая знакомство с порочными, люди совращаются и становятся несчастными на всю жизнь; а если они живут в обществе мудрых и разумных, то из неведения вступают в жизнь лучшую.
\vs Ars 1:131
Поэтому наш законодатель, определив прежде всего то, что относится к благочестию и справедливости, научив всему этому не только в форме запрещений, но и путем разъяснений, показав вредные последствия, а также наказания, посылаемые Богом виновникам этого
\vs Ars 1:132
(ибо прежде всего он и указал, что Бог един и сила Его очевидна всюду, так как всякое место полно Его господства и от Него не скроется ни одно из тайных дел людей на земле, но Ему видно всё, что делает и намеревается делать человек),
\vs Ars 1:133
тщательно выполнив это и сделав вполне очевидным, он показал, что, если бы кто и замыслил сделать дурное, то не не скрыл, но даже и не сделал бы, на протяжении всего законодательства указывая могущество Божие.
\vs Ars 1:134
Итак, положив такое начало и показав, что все остальные люди, кроме нас, почитают многих богов, хотя сами гораздо сильнее тех, кого безразсудно чтут
\vs Ars 1:135
(действительно, сделав из дерева и камней статуи, они говорят, что это образы тех, которые изобрели нечто полезное для их жизни; им они покланяются, сразу обнаруживая глупость.
\vs Ars 1:136
Разве они не [обнаружили бы] полное неразумие, если бы на этом основании, вследствие изобретения, кто-либо был обоготворен? Действительно, взяв одну из тварей, они лишь заметили и указали пользу, но не создали её устройства.
\vs Ars 1:137
Поэтому тщетно и безразсудно обоготворять подобных. И теперь ведь еще есть много людей более ученых и изобретательных, чем прежде, и не доходят до того, чтобы поклоняться им. Создавшие эти образы и составители мифов считают себя самыми мудрыми из греков.
\vs Ars 1:138
Что же говорить о других, более безразсудных, о египтянах и подобных им? Они останавливаются на зверях, на большинстве гадов и животных, покланяются им и приносят жертвы, как живым, так и павшим),
\vs Ars 1:139
и вот, имея в виду всё это, мудрый законодатель, которого Бог наделил способностями к познанию всего, огородил нас частоколом, которого нельзя прорубить, и железными стенами, чтобы мы ни в чем не смешивались с другими народами, пребывая чистыми по телу и душе, свободными от пустых учений, выше всех тварей почитая единого и могущественного Бoгa.
\vs Ars 1:140
На этом основании начальники египтян, их жрецы, постигшие многое и знакомые с письменами, называют нас людьми Божиими. А это неприложимо к остальным, если они не почитают истинного Бога, но являются людьми пищи, питья и одежды.
\vs Ars 1:141
Действительно, к этому направлено всё настроение их души, а у наших это вменяется ни во что; мы в течение всей жизни занимаемся изследованием божественного правления.
\vs Ars 1:142
Поэтому, чтобы мы ни с кем не смешивались и, имея общение с порочными, не испортились, всюду оградил нас законами о чистоте, в пище, питье, прикосновениях, в том, что мы слышим и видим.
\vs Ars 1:143
Вообще, всё подобное согласно с естественным разумом, так как установлено одной Силой и каждое в отдельности о том, почему мы воздерживаемся и пользуемся, имеет глубокое основание.
\vs Ars 1:144
Для примера я кратко объясню тебе одно или два, чтобы ты не впал в опровергнутое мнение, будто Моисей заповедует это, заботясь о мышах, куницах и тому подобных. Напротив, все важные определения сделаны ради справедливости, с целью чистых размышлений и образования нравов.
\vs Ars 1:145
Все птицы, которыми мы питаемся, ручные, чистые, питающиеся овощами и пшеницей, как например голуби, горлицы, куры, куропатки, гуси и прочие, подобные им.
\vs Ars 1:146
А в ряду запрещенных птиц ты найдешь диких, плотоядных, порабощающих благодаря своей силе остальных и несправедливо пожирающих названных выше ручных. Да и не только этих; они похищают даже ягнят, козлят и причиняют вред людям, как мертвым, так и живым.
\vs Ars 1:147
Поэтому, назвав еще нечистыми, обозначил этим, что те, для кого назначено законодательство, должны быть справедливыми по душе, никого не угнетать, полагаясь на свою силу, и ничего не похищать, но управлять своею жизнью согласно справедливости, подобно тому, как названные выше ручные птицы питаются овощами, растущими на земле, и не пользуются своей силой для угнетения более слабых и родственных.
\vs Ars 1:148
И вот, посредством такого законодатель дал знамение разумным, чтобы они были справедливыми, не насильничали и, полагаясь на свою силу, не угнетали других.
\vs Ars 1:149
А если этих, вследствие их природных особенностей, не должно даже и касаться, то как же не охранять себя всячески от того, чтобы наши нравы не извратились в этом.
\vs Ars 1:150
Итак, всё о разрешении этих и животных изложены нам в форме символов. Так, раздвоенность копыт и разделение когтей является символом того, чтобы разграничивать каждое из дел, стремясь к прекрасному.
\vs Ars 1:151
Ибо сила всего тела и деятельность его опору имеет в плечах и бедрах. Поэтому, этим символом принуждает всё направлять к справедливости с разделением, а также, что мы отличаемся от всех людей.
\vs Ars 1:152
Действительно, большинство остальных людей оскверняют себя совокуплениями, совершая тяжкую несправедливость, и этим хвалятся целые страны и города; они не только вступают в сношения с мужчинами, но оскверняют матерей и дочерей. Мы же воздерживаемся от этого.
\vs Ars 1:153
Но кто владеет указанным выше способом различения, тот, как он указал, владеет им и в отношении памяти. Ибо все, раздвояющие копыта, отрыгают и жвачку, ясно показывая размышляющим свойства памяти;
\vs Ars 1:154
ведь жвачность есть ничто иное, как воспоминание о жизни и устройстве, и полагает, что жизнь поддерживается благодаря питанию.
\vs Ars 1:155
Поэтому и чрез писание он предписывает следующее: помни ЯХВЕ, Бога твоего, сотворившего среди тебя великое и чудное. Действительно, если подумать, то великим и славным окажется прежде всего скрепление тела, потребление пищи и разделение каждого члена.
\vs Ars 1:156
Еще более безконечной мудрости заключает устройство чувств, деятельность мысли и невидимое движение, а также быстрота действия в каждом и изобретение искусств.
\vs Ars 1:157
Поэтому предписывает помнить, что указанное выше вместе с устройством хранится божественной силой. Всякое время и место он определил для постоянного памятования о Боге, владыке и хранителе.
\vs Ars 1:158
Поэтому он повелевает, чтобы при пище и питье сначала принести начатки Богу и затем пользоваться. Далее он дал нам знак воспоминания и на покровах); точно также, для памяти о том, что Бог есть, он приказал нам поместить изречения на дверях и воротах).
\vs Ars 1:159
И на руках он ясно приказал привесить знак), ясно показывая, что всякое действие должно совершать справедливо, памятуя о своем устройстве, а более всего питая страх к Богу.
\vs Ars 1:160
Призывает также, отходя ко сну, вставая и путешествуя, изучать творения Божии и не только на словах, но и в мыслях рассматривать свои движения и представления, когда мы отходим ко сну и когда пробуждаемся, так как смена этого божественна и непостижима.
\vs Ars 1:161
Итак, тебе показано превосходное учение в отношении к различию и памяти, как мы истолковали раздвоенность копыт и жвачность. Это заповедуется душе не безцельно и случайно, но ради истины и руководства к здравому учению.
\vs Ars 1:162
Дав предписания относительно пищи, питья, а также прикосновений, приказывает ничего не делать и слушать необдуманно и, пользуясь силою разума, не обращать его на неправду.
\vs Ars 1:163
То же можно найти и в отношении животных. Куницы, мыши и подобные им, сколько указано, вредны.
\vs Ars 1:164
Мыши всё оскверняют и портят, не только для собственного питания, но и делают совершенно безполезным для человека всё, чему бы они ни начали вредить.
\vs Ars 1:165
А куницы оригинальны. Кроме указанного выше они имеют постыдное устройство: они зачинают ушами, а детей рождают через рот.
\vs Ars 1:166
Поэтому такой характер нечист для людей. То, что они воспринимают слухом, это воплощают в слове и погружают других во зло, производя не случайную нечистоту, но пятная себя всюду осквернением нечестия. Ваш царь прекрасно делает, убивая, как мы слышим, таковых.
\vs Ars 1:167
А я сказал: я полагаю, ты говоришь о доносчиках, так как их он всегда подвергает побоям и мучительной смерти.
Он: и я говорю о них. Действительно, подкарауливать погибель людей нечестиво.
\vs Ars 1:168
А наш закон предписывает никому не вредить ни словом, ни делом.
Итак, тебе показано, насколько это можно было сделать вкратце, что все определения имеют в виду справедливость и писание не предписывает ничего безцельного или баснословного, но чтобы в течение всей жизни в своих действиях мы упражнялись в справедливости по отношению ко всем людям, помня о господствующем Боге.
\vs Ars 1:169
Поэтому всё разсуждение о пище и о нечистых гадах и животных относится к справедливости и справедливым отношениям к людям.
\vs Ars 1:170
По моему мнению он прекрасно защищал каждое. А относительно приносимых в жертву телят, баранов и козлов он говорил, что, взяв из стада быков и овец ручных, их должно приносить в жертву, но не диких, чтобы приносящие жертву, воспользовавшись указаниями законодателя, ничем не гордились и знали свою природу, ибо приносящий жертву приносит в жертву всё настроение своей души.
\vs Ars 1:171
Итак, я полагаю, что и в этом отношении его беседы заслуживают внимания. Поэтому, вследствие твоей любознательности, у меня, Филократ, были побуждения объяснить тебе святость Закона и его согласие с природой.
\vs Ars 1:172
И Елеазар, совершив жертвоприношение и избрав посланцев и приготовив много даров для
\vs Ars 1:173
царя, отправил нас в путь в великой безопасности. И когда мы достигли Александрии, царю тотчас же доложили о нашем прибытии. Будучи приняты во дворце, Андрей и я тепло приветствовали
\vs Ars 1:174
царя и передали ему письмо, написанное Елеазаром. Царь весьма безпокоился о том, чтобы принять посланных, и повелел, чтобы все прочие чиновники вышли, а посланные
\vs Ars 1:175
тотчас же были приведены к нему. Это вызвало всеобщее изумление, ибо обычно те, кто добивается быть допущенным пред царем по важным делам, ждут пять дней, а послы царя или больших городов с трудом добиваются придворного приема на третий день; но этих людей он счел достойными больших почестей, поскольку он имел столь великое почтение к их наставнику, и так он отослал тех, чье присутствие он счел излишним, и прогуливался, пока они не вошли и он смог приветствовать их.
\vs Ars 1:176
Когда они вошли с дарами, которые были посланы с ними, и драгоценными пергаментами, на которых был золотыми еврейскими письменами записан Закон (ибо пергамент был чудно изготовлен и соединение страниц было сделано так, что было невидимо), царь, как только
\vs Ars 1:177
увидел их, стал спрашивать их о книгах. И когда они вынули свитки из коробов и развернули их, царь долго простоял в безмолвии и поклонившись семь раз, он сказал: Благодарю вас, друзья, и еще более благодарю того, кто послал вас,
\vs Ars 1:178
превыше же всего Бога, вещавшего это. И когда все, посланные и прочие, бывшие там, вместе воскликнули в один голос: Бог да хранит царя!, он пролил слёзы радости. Ибо в душе его восторг и переполняющее ощущение оказанной ему чести
\vs Ars 1:179
побудили его плакать от счастья. Он повелел свернуть свитки обратно и затем, поклонившись этим мужам, сказал: Достойно было, люди Божии, мне уделить прежде всего почтение книгам, ради которых я призвал вас сюда, и теперь, после того, как я это сделал, протянуть вам десницу моей дружбы. Ради этого я
\vs Ars 1:180
сделал это прежде всего. Я дал указ о том, чтобы этот день, в который вы прибыли, стал считаться великим днем и стал ежегодно торжественно справляться в течение всей моей жизни. Вышло так, что это еще и годовщина
\vs Ars 1:181
моей победы на море над Антигоном. Поэтому я буду рад пировать с вами сегодня. Все, что вам может потребоваться, сказал он, будет приготовлено как подобает и вместе с вами и для меня тоже. И они выразили своё восхищение, и он повелел отвести их в лучший квартал, прилежащий к цитадели) и готовить пир.
\vs Ars 1:182
И Никанор вызвал главного дворцового распорядителя Дорофея, чиновника, особо назначенного смотреть за евреями, и приказал ему приготовить всё необходимое для каждого из них. Ибо так было установлено царем, и это установление вы увидите соблюденным сегодня. Ибо поскольку многие города имеют свои обычаи в том, что касается еды, питья и возлежания, есть особые чиновники, назначение которых узнавать, что им требуется. И всякий раз, когда те приходят к царю, для них готовят, соблюдая их собственные обычаи, так чтобы они не испытывали безпокойства, наслаждаясь посещением. Та же предусмотрительность была соблюдена и для еврейских посланцев. Дорофей же, назначенный старшим приставником при еврейских гостях, был
\vs Ars 1:183
человеком весьма тщательным. Ради такого пира он открыл все хранилища, бывшие под его надзором и державшиеся особо для подобных гостей. Он расположил сидения в два ряда согласно царскому повелению. Ибо он повелел ему усадить половину мужей справа от себя, а остальных позади, так чтобы он не лишил их величайшей из возможных чести. Когда они заняли сидения, он повелел Дорофею всё делать,
\vs Ars 1:184
сообразуясь с обычаями, принятыми среди еврейских гостей. Поэтому он прибег к услугам священных глашатаев и священников, приносящих жертвы, и прочих, кто привык возносить молитвы, и призвал одного из нашего числа, по имени Елеазар, старейшего из еврейских священников, вознести молитву вместо [себя]. И тот поднялся и сотворил превосходную молитву: Да обогатит
\vs Ars 1:185
тебя Всемогущий Бог, о царь, всяким благом, созданным Им, и да наградит Он тебя и твою жену, и твоих детей и твоих товарищей) непрерывным владычеством их во всю вашу жизнь. При этих словах поднялось громкое и радостное одобрение, длившееся весьма долго, и потом
\vs Ars 1:186
они обратились к наслаждению приготовленным пиром. Все распоряжения за столом совершались согласно внушениям Дорофея. Среди прислуживающих были юноши из свиты царя и иные, занимавшие почетные должности при царском дворе.
\vs Ars 1:187
Воспользовавшись моментом, когда пир приостановился, царь спросил посланца, занимавшего почетное место (ибо их расположили по старшинству), как ему сохранить царство
\vs Ars 1:188
неослабным до конца. Поразмыслив немного, тот отвечал: Лучше всего ты утвердишь его безопасность, если ты будешь подражать безконечной Божьей благости. Ибо если ты будешь являть милость и налагать кроткие наказания на тех, кто их заслуживает сообразно сделанному ими, ты
\vs Ars 1:189
обратишь их от зла и приведешь их к покаянию.
Царь похвалил ответ и затем спросил у еще одного мужа, как может совершать самое лучшее во всём. И тот отвечал: Если муж держится правильного отношения ко всему, он всегда будет поступать правильно во всяком случае, помня, что всякая мысль известна Богу. Если страх Божий станет для тебя исходною чертою, ты никогда не пройдешь мимо цели.
\vs Ars 1:190
Царь похвалил и этого мужа и спросил у иного, как приобрести друзей, мыслящих одинаково с собою. Тот отвечал: Если они будут видеть, что ты ревнуешь о нуждах множества, которым ты правишь, ты сам увидишь, как Бог одаривает Своими благодеяниями
\vs Ars 1:191
человеческий род, подавая им здоровье и пищу и всё остальное в должное время.
Выразив согласие с ответом, царь спросил следующего, как, принимая просящих и вынося суд, он может стяжать похвалу даже от тех, кто не добился исполнения иска. И тот сказал: Если твоя речь будет прилична для всех равно и ты никогда не будешь поступать надменно или как тиранн с
\vs Ars 1:192
преступниками. И ты будешь делать это, если рассмотришь образ деяний Божьих. Прошения достойных всегда исполняются, тогда как те, кто не получает ответа на свои молитвы, уведоляются снами или событиями о том, что в их просьбах было вредное, и что Бог не поражает их по их грехам или по величию Своей силы, но долготерпит им.
\vs Ars 1:193
Царь горячо похвалил мужа за его ответ и спросил следующего за ним, как он может стать непобедимым в делах войны. И тот ответил, что если он не будет полагаться только на множество своих сил и их воинственность, но будет непрестанно взывать к Богу привести начатое к счастливому исполнению, а сам
\vs Ars 1:194
же будет исполнять все свои обязанности в духе праведности.
Поблагодарив за ответ, он спросил другого, как он может сделаться грозен для своих врагов. И тот ответил, что если сохраняя мощный запас оружия и войска, он будет помнить, что этим нельзя добиться постоянного и окончательного итога. Ибо даже Бог внушает страх в умы людей, откладывая исполнение и лишь являя величие Своего могущества.
\vs Ars 1:195
Этого мужа царь похвалил и спросил следующего, что есть высшее благо в жизни. И тот ответил: Познать, что Бог есть Господь Вселенной, и что в конечном исполнении всех наших дел не мы добиваемся успеха, но Бог, Своею властью приводящий всё к завершению и ведущий нас к цели.
\vs Ars 1:196
Царь воскликнул, что этот муж ответил хорошо и затем спросил следующего, как он может сохранить всё, чем владеет в целости и в конце передать своим наследникам таким же. И тот ответил: Постоянно моля Бога вдохновить тебя высокою целью во всех твоих начинаниях и предупреждая твоих наследников не ослепляться молвою или богатством, ибо все эти дары подает Бог, а люди никогда сами по себе не достигают превосходства.
\vs Ars 1:197
Царь выразил своё согласие с ответом и осведомился у следующего гостя, как ему суметь переносить с душевным спокойствием всё, что выпадет ему. И тот сказал: Если ты будешь иметь твердое понимание того, что всем людям надлежит от Бога иметь часть как в величайшем зле, так и в величайшем благе, поскольку невозможно для человека уйти от этого. Но Бог, Которому мы все обязаны молитвою, вселяет в нас мужество претерпевать.
\vs Ars 1:198
Восхищенный ответами мужей, царь сказал, что все их ответы были хороши. Я задам еще один вопрос одному мужу, прибавил он, и тогда я прервусь на время, чтобы мы могли обратить наше внимание
\vs Ars 1:199
к наслаждению пиром и провести время с удовольствием. И тогда он спросил мужа, какова истинная цель мужества. И тот ответил: Если правильный замысел исполняется в час опасности в согласии с первоначальным намерением. Ибо всё совершается Богом к твоему преимуществу, о царь, когда твой умысел благ.
\vs Ars 1:200
Когда все своим одобрением выразили согласие с ответом, царь сказал философам (ибо их там было немало): По моему мнению, эти мужи сияют добродетелями и владеют необычайным знанием, ибо в мгновение они дали правильные ответы на те вопросы, что я им задавал, и все положили Бога источником своих слов.
\vs Ars 1:201
И Менедем, философ из Эритреи, сказал: Истинно, о царь, ибо вселенная управляется провидением, и поскольку мы верно постигаем то, что человек есть создание Бога, отсюда следует,
\vs Ars 1:202
что всякая сила и красота слова исходит от Бога. Когда царь показал, что он согласен с этим чувством, разговор прекратился, и они предались удовольствию. Когда наступил вечер, пир закончился.
\vs Ars 1:203
На следующий день они вновь сели за стол и продолжили пир в прежнем распорядке. Когда царь счел, что настал подходящий момент, чтобы предложить изследование, он стал задавать вопросы тем мужам, которые
\vs Ars 1:204
сидели вслед за отвечавшими накануне. Он приступил к началу беседы с одиннадцатым мужем, ибо десять уже отвечали на прежние вопросы. Когда установилось
\vs Ars 1:205
молчание, он спросил как он сможет и далее оставаться богатым. После краткого раздумья, муж, которому был задан вопрос, отвечал, что если он никогда не делал ничего недостойного своего звания, никогда не вел себя распутно, никогда не расточительствовал ради пустого и тщетного, но своею благотворительностью располагал подданных к себе. Ибо Бог Творец всяческого блага и
\vs Ars 1:206
Ему человек обязан послушанием.
Царь воздал этому хвалу и затем спросил у другого, как ему соблюсти истину. В ответ на вопрос тот сказал: Признав, что ложь наводит великий позор на всех людей, и еще более на царей. Ибо поскольку они имеют власть делать, что хотят, зачем им прибегать ко лжи? В прибавление к этому ты должен помнить, о царь, что Бог любит истину.
\vs Ars 1:207
Царь принял ответ с великим удовольствием и, глядя на другого, сказал: Что есть научение истине. И тот отвечал: Если ты не хочешь, чтобы зло случилось с тобою, но хочешь быть причастником всего благого, тогда ты должен соблюдать одно и то же в отношении к твоим подданным и к преступникам, и ты должен кротко увещевать благородного и доброго. Ибо Бог привлекает к Себе всех людей Своей благостью.
\vs Ars 1:208
Царь похвалил его и спросил у следующего по порядку, как ему быть другом людей. И тот ответил: Наблюдая то, что человеческий род возрастает и рождается во многом смятении и великом страдании. Посему ты не должен легкомысленно наказывать или подвергать их пыткам, поскольку ты знаешь, что человеческая жизнь состоит из боли и наказаний. Ибо если ты поймешь всё, ты преисполнишься жалости, ибо Бог также преисполнен жалости.
\vs Ars 1:209
Царь принял ответ с одобрением и спросил у следующего: Что есть основное отличие правления? Блюсти себя, отвечал тот, свободным от пьянства и соблюдать трезвость в течение большей части жизни, почитать праведность превыше всего и делать своими друзьями людей подобного рода. Ибо Бог любит также праведность.
\vs Ars 1:210
Выказав своё одобрение, царь спросил у другого: Что есть истинный признак благочестия? И тот ответил: Постигать то, что Бог непрестанно творит во Вселенной и знает всё, и ни один человек, делающий несправедливое и творящий развращенное, не может избежать Его взора. Поскольку Бог благотворит всему миру, также и ты должен подражать Ему и быть свободным от преступлений.
\vs Ars 1:211
Царь выказал своё согласие и сказал другому: Что есть сущность царствования? И тот ответил: Хорошо управлять собою и не уклоняться ради славы или богатства к неумеренным или непристойным желаниям, вот истинный путь правления, если ты хорошо уразумеваешь дело. Ибо всё, что тебе нужно, у тебя есть, и Бог свободен от нужд и добросердечен. Пусть твои мысли будут такими, какие достойны мужа, и желай немногого, но только того, что необходимо для правления.
\vs Ars 1:212
Царь похвалил его и спросил у другого мужа, как его разсуждения могут привести к наилучшему. И тот ответил, что если он будет постоянно полагать пред собою праведность во всём и думать, что неправедность равносильна лишению жизни. Ибо Бог всяческих обещает высочайшее благословение праведному.
\vs Ars 1:213
Похвалив его, царь спросил следующего, как он может быть свободен от безпокойных мыслей во время сна. И тот отвечал: Ты задал мне вопрос, на который очень трудно ответить, ибо мы не можем руководить собою в часы сна, но твердо удерживаемся в них
\vs Ars 1:214
воображением, которое не может управляться разумом. Ибо наши души обладают чувствами, которые на самом деле видят то, что входит в наше сознание во время сна. Но мы ошибаемся, если мы полагаем, что мы на самом деле плывем на корабле по морю или летаем по воздуху или путешествуем по другим странам или что-нибудь иное в этом роде. И всё-таки мы на самом деле воображаем, что
\vs Ars 1:215
эти вещи происходят. Насколько я могу решить, я достиг следующего решения. Ты должен, о царь, управлять своими словами и делами в законе благочестия всяким возможным образом, так чтобы ты мог сознавать, что ты соблюдаешь добродетель и никогда не решал вознаграждать себя за расточение разума и никогда не превышать своей власти до того, чтобы
\vs Ars 1:216
презирать праведность. Ибо ум по большей части занимается во сне тем же, чем он занят, бодрствуя. И тот, кто направил все свои мысли и дела к самым благородным целям, утверждает себя в праведности и во время бодрствования и во время сна. Благодаря этому ты можешь пребывать неуклонно в постоянном самоблюдении.
\vs Ars 1:217
Царь воздал хвалу мужу и сказал другому: Поскольку ты отвечаешь десятым, то когда ты выскажешься, мы предадимся пиру. И затем он задал вопрос:
\vs Ars 1:218
Как я могу избегать недостойных дел сам по себе. И тот ответил: Всегда обращай внимание на твою славу и твоё верховное звание, так чтобы ты мог говорить и думать только то,
\vs Ars 1:219
что совместимо с ними, зная что все твои подданные думают и говорят о тебе. Ибо ты не должен казаться хуже актеров, которые тщательно изучают свои роли, которые им надо сыграть, и сообразовывают с ними все свои дела. Ты не играешь роль, но ты истинный царь, поскольку Бог наделил тебя царскою властью, чтобы ты соблюдал её вместе с твоей славой.
\vs Ars 1:220
Когда царь похвалил громко и долго и весьма любезно, гостей стали побуждать отдохнуть. И так, когда прекратился разговор, они предались течению пира.
\vs Ars 1:221
На следующий день всё было устроено по-прежнему, и когда царь уловил подходящий момент, чтобы задавать вопросы мужам, он спросил первого из тех, кто оставался
\vs Ars 1:222
неспрошенным: Что есть высший образ правления? И тот ответил: Управлять собою и не поддаваться порывам. Ибо каждй человек от природы обладает особым умственным увлечением.
\vs Ars 1:223
Возможно, большинство людей склонны к еде, питью и наслаждениям, а царь увлекается приобретением земель и великою славой. Но хорошо, когда во всём этом соблюдается умеренность. Что Бог дает, то мы должны принимать и хранить, но никогда не стремиться приобретать то, чего мы не в силах достичь.
\vs Ars 1:224
Довольный этими словами царь спросил у следующего, как ему быть свободным от зависти. И после краткого молчания тот ответил: Если ты прежде всего будешь смотреть на то, что славою и великим богатством всех царей наделяет Бог, а не сам царь своею властью. Все люди хотят иметь такую славу, но не могут, потому что это дар Божий.
\vs Ars 1:225
Царь похвалил мужа в длинной речи и затем спросил у другого, как ему научиться презирать врагов. И тот ответил: Если ты будешь выказывать добродушие ко всем и добиваться их дружбы, тебе не надо будеть никого бояться. Пользоваться всеобщею любовью это наилучший из даров, которые можно получить от Бога.
\vs Ars 1:226
Похвалив этот ответ, царь велел следующему мужу ответить на вопрос, как он может сохранить свою высокую славу. И тот отвечал так: Если ты будешь щедр и открыт сердцем, оделяя других добродушием и благодеяниями, ты никогда не утратишь твою славу, но если ты хочешь, чтобы и впредь милость пребывала с тобою, ты должен постоянно призывать Бога.
\vs Ars 1:227
Царь изъявил своё согласие и спросил у следующего, кому надлежит мужу являть щедрость. И тот ответил: Всеми признано, что нам надлежит являть щедрость по отношению к тем, кто благорасположен к нам, но я думаю, что нам должно являть то же самое щедрое расположение духа и к тем, кто враждебен нам, чтобы мы могли привлечь их к правде и выгоде для них самих. Но мы должны молить Бога о том, чтобы это совершилось, ибо Он управляет умами всех.
\vs Ars 1:228
Выразив своё согласие с ответом, царь просил шестого мужа ответить на вопрос, кому мы должны выказывать благодарность. И тот ответил: Нашим родителям постоянно, ибо Бог дал нам важнейшую из заповедей относительно должного почитания родителей. Затем Он поместил отношение к друзьям, ибо Он говорит: друг словно твоя душа. Хорошо тебе постараться сделать всех твоими друзьями.
\vs Ars 1:229
Царь сказал ему ласковое слово и затем спросил следующего, что по цене подобно красоте. И тот сказал: Благочестие, ибо оно есть преимущественный образ красоты, и его сила в любви, она же Божий дар. Ты уже приобрел это, и вместе все благословения жизни.
\vs Ars 1:230
Царь с великою любезностью одобрил ответ и спросил у другого, как ему, совершив ошибку, вновь вернуть своё имя на прежнюю степень. И тот сказал: Невозможно тебе совершить ошибку, ибо во всех людях ты взыскал семена благодарности, приносящие урожай благожелательности,
\vs Ars 1:231
который могущественней самого сильного оружия и обезпечивает величайшую безопасность. Но если кто-либо совершает ошибку, он никогда не должен повторять того, что привело к ней, но ему надлежит сообразовываться с дружбою и творить справедливость. Ибо дар от Бога быть способным творить дела добра, а не противные ему.
\vs Ars 1:232
Восхищенный этими словами царь спросил у другого, как ему быть свободным от печали. И тот ответил: Если ты никогда никому не причинял вреда, но творил добро и следовал стезею
\vs Ars 1:233
праведности, ибо её плоды дают свободу от печали. Но нам надлежит молить Бога о том, чтобы нежданное зло вроде смерти, болезни, скорби или чего-либо в этом роде не нашло на нас и не принесло вреда. Но поскольку ты предался благочестию, никакое из этих несчастий никогда не постигнет тебя.
\vs Ars 1:234
Царь одарил его великою похвалой и спросил десятого, что есть высший образ славы. И тот ответил: Почитать Бога, и делать это не [только] дарами и жертвоприношениями, но в чистоте души и святом убеждении, поскольку всё образовано и управляется Богом по Его воле. Этой цели ты добиваешься неотменно, как это видно для всех в твоих свершениях в прошлом и настоящем.
\vs Ars 1:235
Громким голосом царь поблагодарил их всех и говорил к ним милостиво и выразил своё одобрение всем, кто был тут, особенно же философам. Ибо те превосходили их и поведением и доводами, поскольку они положили Бога источником себе. Затем царь, чтобы показать свои добрые чувства, начал пить за здоровье гостей.
\vs Ars 1:236
На следующий день пир был приготовлен так же, и царь, как только представился момент, стал задавать вопросы мужам, которые сидели вслед за теми, кто уже отвечал; и у первого он спросил: Можно ли научить мудрости? И тот сказал: Душа устроена так, что может божественною силою принять всё доброе и отвергнуть противное.
\vs Ars 1:237
Царь выразил одобрение и спросил следующего мужа: Что самое полезное для здоровья? И тот сказал: Умеренность, а её невозможно обрести, доколе Бог не внушит расположение к ней.
\vs Ars 1:238
Царь сказал ему ласковое слово и спросил другого: Как человек может достойно заплатить долг благодарности родителям? И тот сказал: Никогда не причиняя им скорби, а это невозможно, доколе Бог не расположит ум к поиску самых благородных целей.
\vs Ars 1:239
Царь выразил согласие и спросил следующего, как ему сделаться жаждущим слушателем. И тот сказал: Помня, что всякое знание полезно, потому что с Божьей помощью оно делает тебя способным во время опасности выбрать что-либо из того, что ты изучил и применить против нашедшей на тебя напасти. И усилия человеческие в этом исполняются через присутствие Божие.
\vs Ars 1:240
Царь похвалил его и спросил другого, как ему избежать делать что-либо противное закону. И тот сказал: Если ты признаешь, что в сердца законодателей помышления охранять человеческие жизни вложил Бог, ты последуешь им.
\vs Ars 1:241
Царь подтвердил ответ этого мужа и сказал другому: В чем преимущество родства? И тот ответил: Если мы обратим внимание на то, что мы сами поражаемся невзгодами, выпадающими нашим ближним, и если их страдания делаются нашими, тогда сразу
\vs Ars 1:242
же становится явною сила родства, ибо лишь явив такие чувства, мы приобретем честь и достоинство в их глазах. Ибо помощь, когда она связана с добротою, есть сама в себе связь, разорвать которую никак нельзя. И в день их процветания мы не должны вожделеть того, чем они владеют, но молить Бога даровать им блага всякого рода.
\vs Ars 1:243
И вознаградив его тою же похвалою, что и прочих, царь спросил другого, как ему достичь свободы от страха. И тот отвечал: Когда ум сознает, что он не творил зла, и когда Бог направляет его на всякий благородный замысел.
\vs Ars 1:244
Царь выразил одобрение и спросил другого, как ему всегда приходить к правильному суждению. И тот отвечал, что если он будет постоянно держать перед глазами выпадающие людям невзгоды и признавать, что Бог иных лишает благоденствия, а других приводит к великим почестям и славе.
\vs Ars 1:245
Царь оказал мужу ласковый прием и попросил другого ответить на вопрос: Как ему избегать жизни в праздности и наслаждениях? И тот отвечал, что если он будет постоянно помнить о том, что он правитель великого царства и господин над великим множеством, и что его ум не должен заниматься иными вещами, но должен всегда изследовать, как ему наилучшим образом обезпечить их благоденствие. Он также должен молиться Богу о том, чтобы ничто из должного не было в пренебрежении.
\vs Ars 1:246
Воздав ему хвалу, царь спросил десятого, как ему распознать тех, кто поступает с ним лукаво. И тот ответил на вопрос: Если он будет наблюдать, насколько естественно их отношение к нему, и придерживаются ли они правил старшинства во время приемов и совета, а в ежедневном общении выходят ли когда-нибудь за пределы
\vs Ars 1:247
приличий в поздравлениях и в прочих манерах. Но Бог направит твой ум, о царь, ко всему благородному. Когда царь выразил своё громкое одобрение и похвалил всех по-одному (и вместе их похвалили все, кто был там), они обратились к радостям пира.
\vs Ars 1:248
И на следующий день, когда подошло время, царь спросил следующего мужа: Что есть грубейший вид пренебрежения? И тот ответил: Если человек не заботится о своих детях и не прилагает всяческих усилий к их воспитанию. Ибо мы всегда молим Бога не столько о самих себе, сколько о наших детях, чтобы им стяжать всякое благословение. Наши пожелания о том, чтобы наши дети сумели владеть собою, может быть исполнено только Божьей властью.
\vs Ars 1:249
Царь сказал, что он говорил хорошо, и потом спросил другого, как ему любить отечество. Сохраняя в уме, отвечал тот, мысль о том, что хорошо жить и умереть для своей страны. Пребывание на чужбине наводит презрение на бедняка и позор на богача, как если бы они были изгнаны за преступление. Если ты даруешь благодеяния всем, как ты это делаешь постоянно, Бог даст тебе милость во всём, и ты будешь признан любящим отечество.
\vs Ars 1:250
Выслушав этого мужа, царь спросил у следующего по порядку, как ему жить в дружбе с женою. И тот ответил: Признавая, что женщины по природе упорны и деятельны, когда добиваются исполнения своих желаний, и склонны резко менять свои суждения от ложных разсуждений, и их природа слаба по сути своей. Необходимо быть мудрым в обращении с ними
\vs Ars 1:251
и не порождать споров. Для того, чтобы пройти жизнь успешно, кормчий должен знать цель, к которой ему надлежит направиться. Лишь призыванием Божьей помощи люди будут соблюдать истинное направление жизни во всякое время.
\vs Ars 1:252
Царь выразил своё согласие и спросил следующего, как ему быть свободным от заблуждений. И тот отвечал: Если ты всегда будешь действовать с разсуждением и никогда не давать веры клевете, но сам испытывать то, что тебе говорят, и решать своим собственным судом просьбы, которые тебе приносят, и приводить всё на свет твоего суждения, ты будешь свободен от заблуждения, о царь! Но познание и исполнение этого есть дело Божественной силы.
\vs Ars 1:253
Восхищенный этими словами царь спросил другого, как ему быть свободным от гнева. И тот сказал в ответ на вопрос, что если он будет признавать, что он имеет власть надо всеми вплоть до предания их смерти, если он даст место гневу и будет безполезно и жалко, если он, потому что он повелитель,
\vs Ars 1:254
лишит жизни многих. Что за нужда гневаться, когда все покорны и никто не противится ему? Надлежит признавать, что Бог правит всем миром в духе добросердечия и без гнева во всём, и ты, о царь, сказал он, необходимо должен следовать Его примеру.
\vs Ars 1:255
Царь сказал, что он отвечал хорошо и потом спросил у следующего мужа: Что есть добрый совет? Делать добро во всякое время и с должным разсуждением, объяснил тот, сравнивая то, что выгодно для твоих дел с уроном, который может быть следствием принятия противного решения, и таким образом взвешивая каждый шаг, мы можем умудряться, и наши намерения могут быть исполнены. И самый важный из всех твой замысел Божьей силою найдет своё завершение потому, что ты соблюдаешь благочестие.
\vs Ars 1:256
Царь сказал, что этот муж отвечал хорошо, и спросил другого: Что такое философия? И тот объяснил: Правильное обсуждение всякого возникшего вопроса с тем, чтобы никогда не увлекаться порывами, но взвешивать всякий ущерб, причиняемый страстями, и действовать в правильной сообразности с тем, как того требуют обстоятельства, соблюдая умеренность. Но мы должны молиться Богу вселить в наш ум почтительное отношение к этому.
\vs Ars 1:257
Царь выразил своё согласие и спросил другого, как ему встречаться с признательностью, путешествуя в чужих странах. Будучи любезным ко всем, отвечал тот, казаться ниже, нежели выше тех, с кем ты путешествуешь. Ибо признано правило, что Бог по Своей истинной природе приемлет смиренного. А человеческий род любит тех, кто охотно подчиняется им.
\vs Ars 1:258
Выразив своё одобрение с этим ответом, царь спросил другого, как ему строить так, чтобы построенное сохранилось после него. И тот ответил на вопрос, что если сделанное им будет принадлежать к числу прекрасного и благородного, так что обладатели сохранят это ради его красоты, и он никогда не лишит себя тех, кто делает подобные вещи, и никогда не будет вынуждать других служить его
\vs Ars 1:259
нуждам без вознаграждения. Ибо наблюдая, как Бог печется о человеческом роде, наделяя его здоровьем и умственными способностями и иными дарами, он сам должен следовать Его примеру, воздавая людям вознаграждение за их тяжкий труд. Ибо дела, творимые в праведности, пребывают всегда.
\vs Ars 1:260
Царь сказал, что этот муж также отвечал правильно и спросил у десятого: Что есть плод мудрости? И тот ответил: То, что муж должен сознавать в себе, что он не делал зла
\vs Ars 1:261
и что ему надлежит прожить жизнь в истине, ибо от этого, о могучий царь, величайшая радость и стойкость души и крепкая вера в Бога умножатся в тебе, если ты будешь править твоим царством в благочестии. И когда они выслушали ответ, они все приветствовали его громкими восклицаниями, и после этого царь, преисполнившись радости, стал пить за их здоровье.
\vs Ars 1:262
И на следующий день пир продолжился также, как и в предыдущие, и когда подошел момент, царь стал задавать вопросы оставшимся гостям, и
\vs Ars 1:263
сказал первому: Как человеку уберечься от гордыни? И тот ответил: Если он остается уравновешенным и во всех обстоятельствах помнит, что он муж, правящий людьми; и: Бог низводит гордых в ничто, и возносит слабых и смиренных.
\vs Ars 1:264
Царь сказал ему милостивое слово и спросил следующего: Кого мужу следует избирать себе в советники? И тот ответил: Тех, кто был испытан во многих делах и сохранил непревратной благожелательность по отношению к нему и разделял с ним его намерения. И Бог Сам является тем, кто достоин исполнения этих целей.
\vs Ars 1:265
Царь похвалил его и спросил другого: Чем прежде всего надлежит владеть царю? Дружбою и любовью своих подданных, отвечал тот, ибо через них узы доброжелательности делаются нерасторжимыми. И Бог делает прочным это настолько, чтобы оно происходило согласно твоему желанию.
\vs Ars 1:266
Царь похвалил его и спросил у другого: Что есть цель слова? И тот ответил: Убедить твоего противника, показав ему его ошибки хорошо и правильно выстроенными доводами. Ибо так ты приобретешь слушателя, не споря с ним, но хваля его с тем, чтобы убедить его. А убеждение совершается силою Божьей.
\vs Ars 1:267
Царь сказал, что он дал хороший ответ, и спросил другого, как ему жить в дружбе со множеством различных племен, составляющих население его царства. Делая то, что надлежит в отношении каждого из них, отвечал тот, и беря себе праведность вождем, как ты это делаешь ныне с помощью проницательности, которою тебя наделил Бог.
\vs Ars 1:268
Царь был восхищен этим ответом и спросил у другого: При каких обстоятельствах человеку надлежит выносить скорбь? В невзгодах, которые выпадают нашим друзьям, ответил тот, когда мы видим, что они длительны и неизгонимы. Разум не дает нам печалиться о тех, кто умер и освободился от зла, но все люди печалятся о них, потому что думают, лишь о себе и своей собственной выгоде. Одною лишь силою Божьей мы можем избежать всякого зла.
\vs Ars 1:269
Царь сказал, что тот дал приличный ответ, и спросил другого: Как утрачивают доброе имя? И тот отвечал: Когда гордыня и необузданная самоуверенность держат власть, рождаются позор и потеря доброго имени. Ибо Бог есть Господь доброй славы и наделяет ею, когда хочет.
\vs Ars 1:270
Царь подтвердил ответ и спросил следующего мужа, кому люди должны доверяться. Тем, отвечал тот, кто служит тебе по доброй воле, а не из страха или своего интереса, помышляя о собственной наживе. Ибо знак любви это одно, а другое признак дурной воли и временного услужения. Ибо человек, который всегда ищет собственной наживы, в сердце своём предатель. Но ты владеешь привязанностью всех твоих подданных с помощью разсудительности, которою тебя одарил Бог.
\vs Ars 1:271
Царь сказал, что тот отвечал мудро, и спросил у другого, чем сохраняется царство. И тот ответил на этот вопрос: Заботой и предусмотрительностью о том, чтобы никакое зло не было сделано теми, кто стоит у власти над народом, и тем, что ты всегда делаешь с Божьей помощью, внушающей тебе важные суждения.
\vs Ars 1:272
Царь сказал ему одобряющее слово и спросил следующего: Чем соблюдаются благодарность и честь? И тот ответил: Добродетелью, ибо она есть творец добрых дел, и она уничтожается злом, даже если ты выказываешь благородство характера ко всем благодаря Божьему дару, которым Он наделил тебя.
\vs Ars 1:273
Царь милостиво принял ответ и спросил одиннадцатого (поскольку их было семьдесят и еще двое), как во время войны ему сохранять душевное спокойствие. И тот отвечал: Помня, что он не сделал зла никому из подданных, и что все они в ответ будут сражаться за него ради благодеяний, которые они получили, зная, что даже если они потеряют жизнь, ты позаботишься о тех,
\vs Ars 1:274
кто находится на их иждивении. Ибо ты никогда не забудешь о таком воздаянии, таково добросердечие, которым тебя одарил Бог.
Царь громко похвалил их всех и говорил с ними милостиво и потом много пил за здоровье каждого, сам предаваясь радости и осыпая своих гостей самыми щедрыми и радостными выражениями дружбы.
\vs Ars 1:275
На седьмой день были сделаны самые обильные приготовления и собраны и другие мужи от различных городов (и среди них большое число посланников). Когда подошел момент, царь спросил у первого из тех, кто еще не был спрошен, как ему избегать
\vs Ars 1:276
обмана неверным разсуждением. И тот ответил: Обращая тщательное внимание на говорящего, на то, что говорится и на обсуждаемый предмет, и задавая те же самые вопросы через некоторое время по-другому. Но обладать зорким умом и уметь выносить здравое суждение в каждом случае есть один из благих даров от Бога, и ты имеешь его, о царь!
\vs Ars 1:277
Царь громко одобрил ответ и спросил у другого: Почему большинство людей никогда не бывает добродетельно? Потому что, отвечал тот, все люди неумеренны по природе и склонны
\vs Ars 1:278
к наслаждениям. Отсюда возникает неправедность и потоком любостяжание. Обычай добродетели препятствие для тех, кто отдает жизнь на наслаждения, потому что он обязывается их предпочитать умеренности и праведности. Ибо хозяин сего Бог.
\vs Ars 1:279
Царь сказал, что он отвечал хорошо, и спросил [следующего], чему должны повиноваться цари. И тот сказал: Законам, так чтобы праведными поступками они могли возвращать людям жизнь. Когда ты делашь это в повиновении Божественным заповедям, ты откладываешь для себя запас в хранилищах вечной памяти.
\vs Ars 1:280
Царь сказал, что этот муж также говорил хорошо, и спросил у следующего: Кого мы должны назначать наместниками? И тот ответил: Всех тех, кто ненавидит порочность и, подражая твоему поведению, творит праведность с тем, чтобы ему поддерживать постоянно своё доброе имя. Ибо это то, что делаешь ты, о могучий царь, сказал он, и Бог возложил на тебя венец праведности.
\vs Ars 1:281
Царь громко приветствовал этот ответ и затем, глядя на следующего мужа, сказал: Кого мы должны назначать начальниками над войсками? И тот объяснил: Тех, кто блистает храбростью и праведностью и тех, кто заботится более о том, чтобы беречь своих людей, чем о том, чтобы добиться победы, опрометчиво подвергая опасности их жизнь. Ибо так же, как Бог действует ко благу всех людей, так и ты, последуя Ему, творишь благо для всех твоих подданных.
\vs Ars 1:282
Царь сказал, что он дал хороший ответ и спросил у другого: Какой человек достоин восхищения? И тот ответил: Человек, наделенный доброю славой и богатством и силой, и душа которого относится одинаково ко всему. Ты сам своими деяниями являешь себя наидостойнейшим восхищения благодаря Божьей помощи, направляющей твоё внимание к этому.
\vs Ars 1:283
Царь выразил своё одобрение и сказал следующему: Чему царь должен уделять наибольшее время? И тот отвечал: Чтению и изучению записей путешествий твоих чиновников, описывающих различные царства для того, чтобы изменять и охранять твоих подданных. И благодаря таким деяниям ты достиг славы, которой не приближался никто другой с Божьей помощью, исполняющей все твои желания.
\vs Ars 1:284
Царь сказал воодушевленное слово этому мужу и спросил у следующего, чем заниматься человеку в часы отдыха и развлечения. И тот отвечал: Смотреть те игры), в которых можно соблюсти уместность и представлять перед глазами сцены, взятые из жизни и показанные
\vs Ars 1:285
достойно и прилично, и полезно и благопристойно. Ибо даже в таких развлечениях можно найти некое наставление, потому что часто какой-нибудь нужный урок извлекается из самого незначительного житейского обстоятельства. Но соблюдая наивысшую пристойность во всех твоих деяниях, ты явил себя философом и был почтен у Бога ради твоей добродетели.
\vs Ars 1:286
Царь, обрадованный весьма хорошо сказанными словами, сказал девятому мужу: Как надлежит вести себя человеку на пирах? И тот ответил: Тебе надлежит собирать у себя ученых мужей и тех, кто способен указать тебе нечто полезное касательно дел твоего царства и жизни твоих подданных (ибо тебе не найти предмета, более подходящего или более
\vs Ars 1:287
наставительного, чем это), поскольку эти люди угодны Богу, потому что они обучили свой ум созерцать самые благородные предметы, как и ты сам, конечно, делаешь это, поскольку все твои дела направляются Богом.
\vs Ars 1:288
Восхищенный ответом, царь спросил у следующего мужа: Что для народа лучше всего? То ли, что царем над ним может стать частный гражданин или же член царской семьи? И тот
\vs Ars 1:289
ответил: Лучший по природе. Ибо цари, происходящие от царской крови, часто грубы и суровы к своим подданным. И всё же часто бывает с некоторыми из тех, кто вознесся из рядов частных граждан, что испытав злое и пожив некое время
\vs Ars 1:290
в бедности, они, управляя многими, становятся свирепее безбожных тираннов. Но, как я сказал, добрая природа, получив пристойное обучение, способна к управлению, и ты великий царь, не столько потому, что блистаешь славою твоего правления и твоими богатствами, сколько оттого, что ты превзошел всех людей в милости и человеколюбии, благодаря Богу, одарившему тебя этими достоинствами.
\vs Ars 1:291
Царь некоторое время восхвалял этого мужа, а после спросил самого последнего: Что есть величайшее свершение в правлении царством? И тот отвечал: То, когда все подданные могут непрерывно жить в мире, и правосудие быстро разрешает споры.
\vs Ars 1:292
Это может быть достигнуто благодаря силе правителя, когда это муж, ненавидящий зло и любящий благо и отдающий свои устремления спасению человеческой жизни, точно также, как ты считаешь несправедливость худшею разновидностью зла и своим управлением создал себе безсмертное доброе имя, поскольку Бог даровал тебе ум чистый и незапятнанный злом.
\vs Ars 1:293
И когда он закончил, раздались длительные, громкие и радостные голоса одобрения. Когда они замолкли, царь взял чашу и сказал слово в честь всех своих гостей и тех слов, что они произнесли. И в конце он сказал: Я извлек величайшую выгоду из вашего прихода,
\vs Ars 1:294
я многое получил от мудрого учения, преподанного вами мне об искусстве правления. Потом он повелел дать каждому по три таланта серебра и указал одному из своих рабов раздать деньги. Тут же все восклицанием выразили своё согласие, и пир превратился в зрелище радости, в то время как царь предался непрерывной череде застолья.
\vs Ars 1:295
Я писал долго и должен просить тебя о прощении, Филострат. Я был безмерно изумлен этими мужами и тем, как они мгновенно давали ответы,
\vs Ars 1:296
требующие поистине длительного размышления. Ибо несмотря на то, что спрашивающий предлагал каждому сложную мысль в каждом вопросе, отвечавшие один за другим давали свои ответы уже готовыми тотчас же и так, что мне и всем, кто был там, а особенно философам, они казались достойными восхищения. И я полагаю, что это покажется невероятным тем, кто
\vs Ars 1:297
прочтет мой рассказ в будущем. Но не подобает искажать то, что записано в государственных архивах. И неправедным с моей стороны было бы извращать подобный предмет. Я рассказываю эту историю так, как она происходила, тщательно избегая любой ошибки. Сила их высказываний запечатлелась во мне настолько, что я особенно переговорил с теми, чья обязанность состоит в том, чтобы
\vs Ars 1:298
записывать всё, что происходит на царских приемах и пирах. Ибо существует, как ты знаешь, обычай с того момента, как царь затевает какое-либо дело и вплоть до того, как он удаляется на покой, вести запись всего, что он говорит или делает, устроение превосходное и полезное.
\vs Ars 1:299
Ибо на следующий день всё по времени, что говорилось и делалось накануне, читается прежде начала дела, и если находится в этом какая-нибудь неправильность, её тотчас же исправляют.
\vs Ars 1:300
И благодаря этому, как уже сказано, я добился точных сведений из государственных архивов и расположил события в правильном порядке, поскольку я знаю, сколь жадно ты стремишься получать полезные сведения.
\vs Ars 1:301
Через три дня Димитрий собрал этих мужей, проведя их вдоль дамбы за семь стадий к острову, перешел мост и направился к северным кварталам Фароса. Там он собрал их в доме, построенном на дамбе, весьма красивом и уединенном, и пригласил их приступить к переводу, поскольку всё, что было им нужно для этого
\vs Ars 1:302
находилось в их распоряжении. И так они принялись за работу, сравнивая сделанное каждым и согласовываясь между собою, и всё согласованное надлежащим образом копировалось под руководством Димитрия.
\vs Ars 1:303
И работа длилась до девятого часа, после чего они могли свободно исполнять свои
\vs Ars 1:304
телесные нужды. Всё, в чем они нуждались, им доставлялось весьма щедро. В дополнение к этому Дорофей приготовлял для них ежедневно то же, что и для самого царя, ибо таково было повеление царя. Каждое утро они рано появлялись во дворце, и,
\vs Ars 1:305
поклонившись царю, возвращались на своё место. И по еврейскому обычаю они мыли руки в море и молились Богу и затем предавались чтению и
\vs Ars 1:306
переводу того или иного начатого места; и я спросил у них, почему они моют руки перед тем, как молятся. И они объяснили, что это делается в знак того, что они не делают зла (ибо всё, что ни делается, делается руками), поскольку благородно и свято видят во всем символ праведности и истины.
\vs Ars 1:307
Как я уже сказал, они собирались ежедневно на месте, восхитительном из-за его спокойствия и веселости, и принимались за работу. И так случилось, что весь перевод был исполнен за семьдесят два дня, будто бы преднамерено заранее.
\vs Ars 1:308
И когда работа была закончена, Димитрий собрал всё еврейское население туда, где работали переводчики, и прочитал им перевод вслух в присутствии переводчиков, которых также, как и народ, принял с почестями из-за великих благодеяний, которые они привлекли
\vs Ars 1:309
на него. Они горячо похвалили Димитрия и побуждали его переписать весь Закон целиком и передать список их начальникам.
\vs Ars 1:310
После того, как книга была прочитана, священники и старейшие из переводчиков, и еврейская община, и начальники народа поднялись и сказали, что теперь, когда совершен столь превосходный и столь тщательный перевод, будет только справедливо, если он останется таким, каков он есть и никакое
\vs Ars 1:311
изменение не будет внесено в него. И когда все выразили согласие с этим, они просили произнести проклятие согласно их обычаю на тех, кто внесет туда какое-нибудь изменение, или прибавив что-нибудь или изменив как-нибудь хоть одно слово из написанного, или опустив что-нибудь. Это было весьма мудрою предосторожностью для того, чтобы наверняка сохранить эту книгу неизменной на всё время в будущем.
\vs Ars 1:312
Когда об этом доложили царю, он весьма обрадовался, ибо он чувствовал, что его желание было исполнено в целости. Вся книга была прочитана ему вслух, и он весьма изумился духу законодателя. И он сказал Димитрию: Как это никто из историков или поэтов не подумал когда-либо посвятить хотя бы слово столь чудесному
\vs Ars 1:313
деянию? И Димитрий ответил: Потому что Закон свят и исходит от Бога. И некоторые из имевших намерение прикоснуться к нему были поражены Богом и отступили от
\vs Ars 1:314
своих замыслов. Он сказал, что слышал от Феопомпа, как тот на тридцать дней потерял разсудок, потому что попытался вставить в свою историю какие-то события из раннего и малодостоверного перевода Закона. Когда он немного пришел
\vs Ars 1:315
в себя, он просил Бога открыть ему, за что ему выпала эта напасть. И ему было открыто во сне, что из пустого любопытства он хотел передать священную истину обычным людям, и что если он отступится, он обретет вновь здоровье. Я также слышал из уст
\vs Ars 1:316
Феодекта, одного из тех поэтов, что пишут трагедии, что когда он попытался переделать кое-что из происшествий, записанных в этой книге для своей трагедии, оба его глаза покрылись бельмами. Когда он понял причину своего несчастья, он много дней молил Бога и затем обрел здоровье.
\vs Ars 1:317
И после того, как царь, как я уже сказал, получил от Димитрия объяснение этому вопросу, он поклонился и повелел, чтобы с книгами обращались с великою тщательностью, и чтобы их
\vs Ars 1:318
хранили как святыню. И он горячо приглашал переводчиков часто навещать его после того, как они возвратятся в Иудею, ибо, говорил он, лишь так будет справедливо, если теперь он отпустит их домой. Но когда они придут снова, он
\vs Ars 1:319
примет их как своих друзей, как подобает, и они получат богатые подарки от него. Он повелел сделать приготовления к их возвращению домой и явил к ним великую щедрость. Он подарил каждому три превосходнейших одежды, два золотых таланта, сундук весом в талант, всё, что необходимо для трех лож.)
\vs Ars 1:320
А в сопровождение он послал Елеазару десять лож на серебряных ножках и всё необходимое к ним, сундук ценою в тридцать талантов, десять одежд, багряницу и великолепный венец, и сто штук тончайшей шерсти, а также кубки и блюда, и две золотых чаши для посвящения Богу.
\vs Ars 1:321
Он просил его также в письме о том, что если кто из этих мужей захочет вернуться к нему, не препятствовать ему. Ибо он считал за честь наслаждаться обществом столь ученых мужей, предпочитая расточать свои богатства на них, нежели на суетное.
\vs Ars 1:322
И теперь, Филострат, ты получил полный рассказ, согласно моему обещанию. Я полагаю, ты испытаешь большее удовольствие от этого, чем от сочинений баснописцев. Ибо ты предан тому, что может принести пользу душе, и уделяешь этому много времени. Я постараюсь рассказать о каких-нибудь еще событиях, стоящих записывания, так что внимательно читая их, ты сможешь удостоверить высочайшую награду за твоё усердие.

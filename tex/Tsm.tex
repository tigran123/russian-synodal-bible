\bibbookdescr{Tsm}{
  inline={Завещание Симеона,\\второго сына Иакова и Лии\fns{В греч. тексте $+$ ``о зависти''.}},
  toc={Завещание Симеона},
  bookmark={Завещание Симеона},
  header={Завещание Симеона},
  abbr={Сим}
}
\vs Tsm 1:1
Список слов Симеона, речённых им к сыновьям его перед тем,
как умер он в 120-ый год жизни своей,
в тот же год, что и брат его Иосиф.
\vs Tsm 1:2
Когда занемог Симеон, пришли проведать его дети его, и, сделав
усилие, сел он, поцеловал их и сказал:
\vs Tsm 2:1
послушайте, дети мои, Симеона, отца вашего; возвещу вам то,
что имею я в сердце моём.

\vs Tsm 2:2
Родился я от Иакова и был вторым сыном отца моего, и Лия, мать моя,
нарекла меня Симеоном, ибо услышал Господь мольбу ее.
\vs Tsm 2:3
Сделался я весьма сильным, не боялся труда и не страшился никакого дела.
\vs Tsm 2:4
Ибо сердце моё было сухим, печень моя недвижимой,
а внутренности мои нечувствительными.
\vs Tsm 2:5
Ведь и мужество даётся от Всевышнего людям в душах и телах.
\vs Tsm 2:6
Во время юности моей завидовал я сильно Иосифу,
ибо возлюбил его отец мой более всех.
\vs Tsm 2:7
И утвердился я против него в сердце моём, возжелав убить его,
так как Князь обмана и дух зависти ослепили мне ум, и забыл я,
что это брат мой, и не пощадил отца моего Иакова.
\vs Tsm 2:8
Но Бог его и Бог отцов наших послал ангела своего и избавил
Иосифа от рук моих.

\vs Tsm 2:9
Ибо, когда я отправился в Сиким, чтобы принести притирание для стада,
а Рувим  в Дофаим, где было необходимое нам и все хранилища наши,
Иуда, брат мой, продал Иосифа Измаильтянам.
\vs Tsm 2:10
Рувим, услышав об этом, опечалился, ибо он хотел отвести его к отцу.
\vs Tsm 2:11
Я же, услышав это, сильно разгневался на Иуду,
ибо он отпустил Иосифа живым,
и 5 месяцев пребывал я в гневе на него.
\vs Tsm 2:12
И сковал меня Господь и удалил от меня дело рук моих,
ибо правая рука моя стала наполовину сухой на 7 дней.
\vs Tsm 2:13
И познал я, дети, что из-за Иосифа случилось это со мною.
И, раскаявшись, заплакал я и молил Господа Бога,
чтобы восстановилась рука моя и удержался я от всякой скверны
и зависти и ото всякого безрассудства.
\vs Tsm 2:14
Ибо понял я, что злое дело замыслил перед лицом Господа и Иакова,
отца моего, против Иосифа, брата моего, позавидовав ему.

\vs Tsm 3:1
Ныне, дети мои, послушайте меня и остерегитесь духа обмана и зависти.
\vs Tsm 3:2
Ведь зависть властвует надо всем помыслом человека
и не дает ему ни есть, ни пить, ни делать ничего доброго.
\vs Tsm 3:3
Но всечасно подстрекает она убить того, кому человек завидует,
но тот всечасно процветает, а завистник чахнет.
\vs Tsm 3:4
И вот, 2 года сокрушал я в страхе Господнем душу мою постом.
И узнал я, что избавление от зависти происходит через страх Божий.
\vs Tsm 3:5
Если кто прибегает к Господу, оставляет его злой дух
и становится разум лёгким.
\vs Tsm 3:6
И наконец, начинает он сочувствовать тому, кому завидовал, и
примиряется с любящими его, и так избавляется от зависти.

\vs Tsm 4:1
Спросил отец мой, что со мною, ибо заметил меня скорбящим, и
сказал я ему, что переполняется печень моя.
\vs Tsm 4:2
Ибо печалился я чрезвычайно, что виновен в продаже Иосифа.
\vs Tsm 4:3
И когда пошли мы в Египет и связали меня как соглядатая,
познал я, что справедливо страдаю и не опечалился.
\vs Tsm 4:4
Иосиф же был добрый муж, дух Божий в себе имевший,
милостивый и сострадательный; не вспомнил мне зла, но
возлюбил меня с братьями моими.

\vs Tsm 4:5
Так остерегайтесь же, дети мои, всякой ревности и зависти и живите в
простоте сердечной, чтобы дал и вам Бог милость и славу и благословение на
головы ваши, как вы видите то на Иосифе.
\vs Tsm 4:6
Ни в какой день не стыдил он нас за дело это,
но возлюбил нас как душу свою, и более сыновей своих почтил нас,
и богатство, и скот, и плоды даровал нам.

\vs Tsm 4:7
И вы, дети мои, возлюбите каждый брата своего в доброте сердечной,
и отойдёт от вас дух зависти.
\vs Tsm 4:8
Ибо озлобляет он душу и губит тело, гнев и вражду вводит в
помышление и побуждает к крови и вводит разум в экстаз,
и смятение создает в душе и дрожь в теле.
\vs Tsm 4:9
Даже во сне злая зависть, соблазняя человека,
пожирает его и духами злыми возмущает душу его,
и заставляет тело его содрогаться,
и смятением лишает сна ум его,
и как дух злой и губительный является людям.
\vs Tsm 5:1
Оттого Иосиф был прекрасен лицом и приятен видом своим,
что не поселялось в нем ничто злое;
ибо смущение духа проступает явно на лице человека.

\vs Tsm 5:2
Ныне, дети мои, смягчите сердце ваше пред Господом
и выпрямите пути ваши пред людьми,
и стяжаете благодать пред лицом Господа и людей.
\vs Tsm 5:3
И остерегайтесь блуда, ибо блуд порождает всякое зло,
отдаляя от Бога и приближая к Велиару.
\vs Tsm 5:4
Видел я в книге Еноха, что сыновья ваши совратятся
от блуда и обиду нанесут мечом своим сыновьям Левия.
\vs Tsm 5:5
Но не смогут они противостоять Левию,
ибо поведёт он брань Господню и одолеет всякое войско ваше.
\vs Tsm 5:6
И будут они малочисленны, разделенные в Левин и в Иуде, и
не будет из вас никого, кто властвовал бы,
как и пророчествовал отец наш в благословениях своих.

\vs Tsm 6:1
И вот, сказал я вам всё, дабы оправдать себя от греха вашего.
\vs Tsm 6:2
И если удалите от себя зависть и всякое жестокосердие,
словно роза расцветут кости мои в Израиле,
и словно лилия плоть моя в Иакове,
и будет благоухание моё словно аромат Ливана,
и умножатся святые от меня во веки веков,
и взрастут отрасли их.
\vs Tsm 6:3
Тогда погибнет семя Ханаана,
и не будет остатка у Амалика,
и сгинут все Каппадокийцы,
и все Хетты истребятся.
\vs Tsm 6:4
Тогда угаснет земля Хама, и погибнет весь народ.
Тогда почиет вся земля от смуты, и всё, что под небесами, от войны.
\vs Tsm 6:5
Тогда прославится Сим,
ибо Господь Бог Израиля придет на землю [как человек] и тем
спасёт Адама.
\vs Tsm 6:6
Тогда предан будет всякий дух соблазна на поругание,
и люди обретут власть над злыми духами.
\vs Tsm 6:7
Тогда воскресну и я в радости и благословлю Всевышнего ради чудес его,
[ибо Господь, приняв тело и вкусив пищу с людьми, спас людей.]

\vs Tsm 7:1
Ныне, дети мои, слушайте Левия и Иуду,
и не восставайте на два эти колена,
ибо от них исполнится нам спасение Божие.
\vs Tsm 7:2
Ибо восстанет Господь из Левия как Первосвященник,
а из Иуды как Царь [Бог и человек].
Он спасёт [все народы и] род Израиля.
\vs Tsm 7:3
Для того внушаю вам это, дабы и вы внушили детям вашим,
да сохранят всё в поколениях своих.

\vs Tsm 8:1
Завершил Симеон наставление сыновей своих и почил с отцами
своими, будучи 120-и лет.
\vs Tsm 8:2
И положили его во гроб деревянный,
чтобы отнести кости его в Хеврон.
И отнесли их втайне, пока Египтяне вели войну.
\vs Tsm 8:3
Ибо кости Иосифа сохранили Египтяне в гробнице царей.
\vs Tsm 8:4
Сказали им прорицатели, что, если вынесут кости Иосифа,
тьма и мрак будут по всей земле и несчастье великое Египтянам,
так что и со светильником не узнает никто брата своего.

\vs Tsm 9:1
И оплакали сыновья Симеона, отца своего.
И пребывали в Египте вплоть до дней, когда Моисей вывел их рукою своею.

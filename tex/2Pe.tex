\bibbookdescr{2Pe}{
  inline={Второе Соборное Послание\\\LARGE Святого Апостола Петра},
  toc={2-е Петра},
  bookmark={2-е Петра},
  header={2-е Петра},
  %headerleft={},
  %headerright={},
  abbr={2~Пет}
}
\vs 2Pe 1:1 Симон Петр, раб и Апостол Иисуса Христа, принявшим с нами равно драгоценную веру по правде Бога нашего и Спасителя Иисуса Христа:
\vs 2Pe 1:2 благодать и мир вам да умножится в познании Бога и Христа Иисуса, Господа нашего.
\rsbpar\vs 2Pe 1:3 Как от Божественной силы Его даровано нам все потребное для жизни и благочестия, через познание Призвавшего нас славою и благостию,
\vs 2Pe 1:4 которыми дарованы нам великие и драгоценные обетования, дабы вы через них соделались причастниками Божеского естества, удалившись от господствующего в мире растления похотью:
\vs 2Pe 1:5 то вы, прилагая к сему все старание, покажите в вере вашей добродетель, в добродетели рассудительность,
\vs 2Pe 1:6 в рассудительности воздержание, в воздержании терпение, в терпении благочестие,
\vs 2Pe 1:7 в благочестии братолюбие, в братолюбии любовь.
\vs 2Pe 1:8 Если это в вас есть и умножается, то вы не останетесь без успеха и плода в познании Господа нашего Иисуса Христа.
\vs 2Pe 1:9 А в ком нет сего, тот слеп, закрыл глаза, забыл об очищении прежних грехов своих.
\vs 2Pe 1:10 Посему, братия, более и более старайтесь делать твердым ваше звание и избрание; так поступая, никогда не преткнетесь,
\vs 2Pe 1:11 ибо так откроется вам свободный вход в вечное Царство Господа нашего и Спасителя Иисуса Христа.
\rsbpar\vs 2Pe 1:12 Для того я никогда не перестану напоминать вам о сем, хотя вы то и знаете, и утверждены в настоящей истине.
\vs 2Pe 1:13 Справедливым же почитаю, доколе нахожусь в этой \bibemph{телесной} храмине, возбуждать вас напоминанием,
\vs 2Pe 1:14 зная, что скоро должен оставить храмину мою, как и Господь наш Иисус Христос открыл мне.
\vs 2Pe 1:15 Буду же стараться, чтобы вы и после моего отшествия всегда приводили это на память.
\vs 2Pe 1:16 Ибо мы возвестили вам силу и пришествие Господа нашего Иисуса Христа, не хитросплетенным басням последуя, но быв очевидцами Его величия.
\vs 2Pe 1:17 Ибо Он принял от Бога Отца честь и славу, когда от велелепной славы принесся к Нему такой глас: Сей есть Сын Мой возлюбленный, в Котором Мое благоволение.
\vs 2Pe 1:18 И этот глас, принесшийся с небес, мы слышали, будучи с Ним на святой горе.
\vs 2Pe 1:19 И притом мы имеем вернейшее пророческое слово; и вы хорошо делаете, что обращаетесь к нему, как к светильнику, сияющему в темном месте, доколе не начнет рассветать день и не взойдет утренняя звезда в сердцах ваших,
\vs 2Pe 1:20 зная прежде всего то, что никакого пророчества в Писании нельзя разрешить самому собою.
\vs 2Pe 1:21 Ибо никогда пророчество не было произносимо по воле человеческой, но изрекали его святые Божии человеки, будучи движимы Духом Святым.
\vs 2Pe 2:1 Были и лжепророки в народе, как и у вас будут лжеучители, которые введут пагубные ереси и, отвергаясь искупившего их Господа, навлекут сами на себя скорую погибель.
\vs 2Pe 2:2 И многие последуют их разврату, и через них путь истины будет в поношении.
\vs 2Pe 2:3 И из любостяжания будут уловлять вас льстивыми словами; суд им давно готов, и погибель их не дремлет.
\vs 2Pe 2:4 Ибо, если Бог ангелов согрешивших не пощадил, но, связав узами адского мрака, предал блюсти на суд для наказания;
\vs 2Pe 2:5 и если не пощадил первого мира, но в восьми душах сохранил семейство Ноя, проповедника правды, когда навел потоп на мир нечестивых;
\vs 2Pe 2:6 и если города Содомские и Гоморрские, осудив на истребление, превратил в пепел, показав пример будущим нечестивцам,
\vs 2Pe 2:7 а праведного Лота, утомленного обращением между людьми неистово развратными, избавил
\vs 2Pe 2:8 (ибо сей праведник, живя между ними, ежедневно мучился в праведной душе, видя и слыша дела беззаконные)~---
\vs 2Pe 2:9 то, конечно, знает Господь, как избавлять благочестивых от искушения, а беззаконников соблюдать ко дню суда, для наказания,
\vs 2Pe 2:10 а наипаче тех, которые идут вслед скверных похотей плоти, презирают начальства, дерзки, своевольны и не страшатся злословить высших,
\vs 2Pe 2:11 тогда как и Ангелы, превосходя их крепостью и силою, не произносят на них пред Господом укоризненного суда.
\vs 2Pe 2:12 Они, как бессловесные животные, водимые природою, рожденные на уловление и истребление, злословя то, чего не понимают, в растлении своем истребятся.
\vs 2Pe 2:13 Они получат возмездие за беззаконие, ибо они полагают удовольствие во вседневной роскоши; срамники и осквернители, они наслаждаются обманами своими, пиршествуя с вами.
\vs 2Pe 2:14 Глаза у них исполнены любострастия и непрестанного греха; они прельщают неутвержденные души; сердце их приучено к любостяжанию: это сыны проклятия.
\vs 2Pe 2:15 Оставив прямой путь, они заблудились, идя по следам Валаама, сына Восорова, который возлюбил мзду неправедную,
\vs 2Pe 2:16 но был обличен в своем беззаконии: бессловесная ослица, проговорив человеческим голосом, остановила безумие пророка.
\vs 2Pe 2:17 Это безводные источники, облака и мглы, гонимые бурею: им приготовлен мрак вечной тьмы.
\vs 2Pe 2:18 Ибо, произнося надутое пустословие, они уловляют в плотские похоти и разврат тех, которые едва отстали от находящихся в заблуждении.
\vs 2Pe 2:19 Обещают им свободу, будучи сами рабы тления; ибо, кто кем побежден, тот тому и раб.
\vs 2Pe 2:20 Ибо если, избегнув скверн мира чрез познание Господа и Спасителя нашего Иисуса Христа, опять запутываются в них и побеждаются ими, то последнее бывает для таковых хуже первого.
\vs 2Pe 2:21 Лучше бы им не познать пути правды, нежели, познав, возвратиться назад от преданной им святой заповеди.
\vs 2Pe 2:22 Но с ними случается по верной пословице: пес возвращается на свою блевотину, и: вымытая свинья \bibemph{идет} валяться в грязи.
\vs 2Pe 3:1 Это уже второе послание пишу к вам, возлюбленные; в них напоминанием возбуждаю ваш чистый смысл,
\vs 2Pe 3:2 чтобы вы помнили слова, прежде реченные святыми пророками, и заповедь Господа и Спасителя, преданную Апостолами вашими.
\vs 2Pe 3:3 Прежде всего знайте, что в последние дни явятся наглые ругатели, поступающие по собственным своим похотям
\vs 2Pe 3:4 и говорящие: где обетование пришествия Его? Ибо с тех пор, как стали умирать отцы, от начала творения, всё остается так же.
\vs 2Pe 3:5 Думающие так не знают, что вначале словом Божиим небеса и земля составлены из воды и водою:
\vs 2Pe 3:6 потому тогдашний мир погиб, быв потоплен водою.
\vs 2Pe 3:7 А нынешние небеса и земля, содержимые тем же Словом, сберегаются огню на день суда и погибели нечестивых человеков.
\rsbpar\vs 2Pe 3:8 Одно т\acc{о} не должно быть сокрыто от вас, возлюбленные, что у Господа один день, как тысяча лет, и тысяча лет, как один день.
\vs 2Pe 3:9 Не медлит Господь \bibemph{исполнением} обетования, как некоторые почитают то медлением; но долготерпит нас, не желая, чтобы кто погиб, но чтобы все пришли к покаянию.
\vs 2Pe 3:10 Придет же день Господень, как тать ночью, и тогда небеса с шумом прейдут, стихии же, разгоревшись, разрушатся, земля и все дела на ней сгорят.
\vs 2Pe 3:11 Если так всё это разрушится, то какими должно быть в святой жизни и благочестии вам,
\vs 2Pe 3:12 ожидающим и желающим пришествия дня Божия, в который воспламененные небеса разрушатся и разгоревшиеся стихии растают?
\vs 2Pe 3:13 Впрочем мы, по обетованию Его, ожидаем нового неба и новой земли, на которых обитает правда.
\rsbpar\vs 2Pe 3:14 Итак, возлюбленные, ожидая сего, потщитесь явиться пред Ним неоскверненными и непорочными в мире;
\vs 2Pe 3:15 и долготерпение Господа нашего почитайте спасением, как и возлюбленный брат наш Павел, по данной ему премудрости, написал вам,
\vs 2Pe 3:16 как он говорит об этом и во всех посланиях, в которых есть нечто неудобовразумительное, что невежды и неутвержденные, к собственной своей погибели, превращают, как и прочие Писания.
\vs 2Pe 3:17 Итак вы, возлюбленные, будучи предварены о сем, берегитесь, чтобы вам не увлечься заблуждением беззаконников и не отпасть от своего утверждения,
\vs 2Pe 3:18 но возрастайте в благодати и познании Господа нашего и Спасителя Иисуса Христа. Ему слава и ныне и в день вечный. Аминь.

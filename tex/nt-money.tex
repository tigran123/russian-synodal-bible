%\bibpdfbookmark{Денежные Единицы}{ntmoney}
\bibmark{book}{ДЕНЕЖНЫЕ ЕДИНИЦЫ}
\thispagestyle{empty}
\pagestyle{fancy}

\begin{center}
\normalsize\bfseries
ДЕНЕЖНЫЕ ЕДИНИЦЫ В НОВОМ ЗАВЕТЕ
\end{center}

%\begin{multicols}{2}
В новозаветное время в Палестине в основном имели хождение монеты
греческие и римские. В книгах Нового Завета упоминается десять видов
монет: одна монета иудейской чеканки, пять --- греческой и четыре ---
римской.

Ходячей \bibemph{иудейской монетой} был \textbf{сребреник}, остаток
маккавейской чеканки. Он равнялся сиклю и считался национальной
монетой, употреблявшейся предпочтительно пред всеми другими при
храме. За тридцать таких сребреников Иуда предал Христа (\bibref{Mat
26:15}; \bibref[27:3--6,9]{Mat 27:3}). По тогдашним ценам это была
достаточная сумма, чтобы купить небольшой участок земли даже в
окрестностях Иерусалима.

\bibemph{Греческие монеты}. Основная денежная единица \textbf{драхма}
(\bibref[\bk{Luk}~15:8,~9]{Luk 15:8}) --- серебряная монета, равная
римскому динарию. Одна драхма составляла 6000-ю часть аттического таланта,
100-ю часть мины и разделялась на 6 оволов (оболов). В зависимости от
места чеканки драхма имела разный вес: аттическая драхма ---
4,37~\bibemph{г} серебра (т.~е.\ в два раза меньше нашего серебряного
полтинника чеканки 1922 и 1924 года), эгинская --- 6,3~\bibemph{г}. В
разное время вес монеты и ее цена тоже колебались, поэтому вообще
сравнивать покупную способность древних денег с современными можно
только приблизительно. Серебряная монета достоинством в две драхмы
называлась \textbf{дидрахма} (\bibref{Mat 17:24}); внешне дидрахма
могла походить на серебряный полтинник. Дидрахма приравнивалась к
полусиклю, так что принималась вместо последнего в уплату храмовой
подати. Четыре драхмы составляли \textbf{статир} (\bibref{Mat 17:27}) ---
серебряную монету, называвшуюся также \textbf{тетрадрахмой} (он мог
быть вроде серебряного рубля чеканки 1924 года). Статир приравнивался
к полному священному сиклю или сребренику. Такой статир был найден
ап. Петром в пойманной им рыбе и отдан в уплату храмовой подати за
Иисуса Христа и за себя. Сто драхм или 25 статиров составляли
\textbf{мину} (\bibref{Luk 19:13}). Высшей денежной единицей был
\textbf{талант}, золотой или серебряный (\bibref{Mat 18:24};
\bibref[25:15]{Mat 25:15}; \bibref{Rev 16:21}). Золотой талант был
равен десяти серебряным. Аттический талант равнялся 60 минам или 6000
драхм, а коринфский талант --- 100 минам. Последний более подходит к
ценности собственно еврейского (ветхозаветного) серебряного таланта,
но к I~веку по Р.~Х.\ вес и стоимость таланта понизились.


\bibemph{Римские монеты}. \textbf{Динарий} (denarius) --- серебряная
монета, часто упоминаемая в евангелиях (\bibref{Mat 18:28};
\bibref[20:2]{Mat 20:2}; \bibref{Mar 6:37}; \bibref[12:15]{Mar 12:15};
\bibref{Luk 7:41}; \bibref[20:24]{Luk 20:24}; \bibref{Joh 6:7};
\bibref[12:5]{Joh 12:5}, также \bibref{Rev 6:6}). По весу и ценности
динарий приравнивался к греческой драхме или 1/4 сикля, но во время
земной жизни Спасителя он имел меньшую ценность. На лицевой стороне
монеты изображался царствующий император
(\bibref[\bk{Mat}~22:19--21]{Mat 22:19}).  Динарий составлял
ежедневную плату римскому воину, как драхма --- ежедневную плату
афинским воинам. Он же составлял обычную поденную плату рабочим
(\bibref{Mat 20:2}). Динарию же равнялась поголовная подать, которую
иудеи обязаны были платить римлянам (\bibref{Mat 22:19}). Динарий
разделялся на десять, а позднее --- на шестнадцать \textbf{ассариев}
или \textbf{асов} (\bibref{Mat 10:29}; \bibref{Luk 12:6}). Это была
медная монета. Четвертую часть ассария составлял \textbf{кодрант}
(quadrans) (\bibref{Mat 5:26}; \bibref{Mar 12:42}). На этих монетах
тоже изображался император. Половину кодранта составляла минута
(minutum) или \textbf{лепта} --- в русском переводе <<полушка>>
(\bibref{Luk 12:59}; \bibref{Mar 12:42}) --- самая мелкая медная
монета. Две такие монеты и положила в сокровищницу храма бедная
вдовица (\bibref[\bk{Mar}~12:41--44]{Mar 12:41}).
%\end{multicols}

\bibbookdescr{Mar}{
  inline={От Марка\\\LARGE святое благовествование},
  toc={От Марка},
  bookmark={От Марка},
  header={От Марка},
  %headerleft={},
  %headerright={},
  abbr={Мк}
}
\vs Mar 1:1 Начало Евангелия Иисуса Христа, Сына Божия,
\vs Mar 1:2 как написано у пророков: вот, Я посылаю Ангела Моего пред лицем Твоим, который приготовит путь Твой пред Тобою.
\vs Mar 1:3 Глас вопиющего в пустыне: приготовьте путь Господу, прямыми сделайте стези Ему.
\rsbpar\vs Mar 1:4 Явился Иоанн, крестя в пустыне и проповедуя крещение покаяния для прощения грехов.
\vs Mar 1:5 И выходили к нему вся страна Иудейская и Иерусалимляне, и крестились от него все в реке Иордане, исповедуя грехи свои.
\vs Mar 1:6 Иоанн же носил одежду из верблюжьего волоса и пояс кожаный на чреслах своих, и ел акриды и дикий мед.
\vs Mar 1:7 И проповедовал, говоря: идет за мною Сильнейший меня, у Которого я недостоин, наклонившись, развязать ремень обуви Его;
\vs Mar 1:8 я крестил вас водою, а Он будет крестить вас Духом Святым.
\rsbpar\vs Mar 1:9 И было в те дни, пришел Иисус из Назарета Галилейского и крестился от Иоанна в Иордане.
\vs Mar 1:10 И когда выходил из воды, тотчас увидел \bibemph{Иоанн} разверзающиеся небеса и Духа, как голубя, сходящего на Него.
\vs Mar 1:11 И глас был с небес: Ты Сын Мой возлюбленный, в Котором Мое благоволение.
\rsbpar\vs Mar 1:12 Немедленно после того Дух ведет Его в пустыню.
\vs Mar 1:13 И был Он там в пустыне сорок дней, искушаемый сатаною, и был со зверями; и Ангелы служили Ему.
\rsbpar\vs Mar 1:14 После же того, как предан был Иоанн, пришел Иисус в Галилею, проповедуя Евангелие Царствия Божия
\vs Mar 1:15 и говоря, что исполнилось время и приблизилось Царствие Божие: покайтесь и веруйте в Евангелие.
\rsbpar\vs Mar 1:16 Проходя же близ моря Галилейского, увидел Симона и Андрея, брата его, закидывающих сети в море, ибо они были рыболовы.
\vs Mar 1:17 И сказал им Иисус: идите за Мною, и Я сделаю, что вы будете ловцами человеков.
\vs Mar 1:18 И они тотчас, оставив свои сети, последовали за Ним.
\vs Mar 1:19 И, пройдя оттуда немного, Он увидел Иакова Зеведеева и Иоанна, брата его, также в лодке починивающих сети;
\vs Mar 1:20 и тотчас призвал их. И они, оставив отца своего Зеведея в лодке с работниками, последовали за Ним.
\rsbpar\vs Mar 1:21 И приходят в Капернаум; и вскоре в субботу вошел Он в синагогу и учил.
\vs Mar 1:22 И дивились Его учению, ибо Он учил их, как власть имеющий, а не как книжники.
\vs Mar 1:23 В синагоге их был человек, \bibemph{одержимый} духом нечистым, и вскричал:
\vs Mar 1:24 оставь! что Тебе до нас, Иисус Назарянин? Ты пришел погубить нас! знаю Тебя, кто Ты, Святый Божий.
\vs Mar 1:25 Но Иисус запретил ему, говоря: замолчи и выйди из него.
\vs Mar 1:26 Тогда дух нечистый, сотрясши его и вскричав громким голосом, вышел из него.
\vs Mar 1:27 И все ужаснулись, так что друг друга спрашивали: что это? что это за новое учение, что Он и духам нечистым повелевает со властью, и они повинуются Ему?
\vs Mar 1:28 И скоро разошлась о Нем молва по всей окрестности в Галилее.
\rsbpar\vs Mar 1:29 Выйдя вскоре из синагоги, пришли в дом Симона и Андрея, с Иаковом и Иоанном.
\vs Mar 1:30 Теща же Симонова лежала в горячке; и тотчас говорят Ему о ней.
\vs Mar 1:31 Подойдя, Он поднял ее, взяв ее за руку; и горячка тотчас оставила ее, и она стала служить им.
\vs Mar 1:32 При наступлении же вечера, когда заходило солнце, приносили к Нему всех больных и бесноватых.
\vs Mar 1:33 И весь город собрался к дверям.
\vs Mar 1:34 И Он исцелил многих, страдавших различными болезнями; изгнал многих бесов, и не позволял бесам говорить, что они знают, что Он Христос.
\rsbpar\vs Mar 1:35 А утром, встав весьма рано, вышел и удалился в пустынное место, и там молился.
\vs Mar 1:36 Симон и бывшие с ним пошли за Ним
\vs Mar 1:37 и, найдя Его, говорят Ему: все ищут Тебя.
\vs Mar 1:38 Он говорит им: пойдем в ближние селения и города, чтобы Мне и там проповедовать, ибо Я для того пришел.
\vs Mar 1:39 И Он проповедовал в синагогах их по всей Галилее и изгонял бесов.
\rsbpar\vs Mar 1:40 Приходит к Нему прокаженный и, умоляя Его и падая пред Ним на колени, говорит Ему: если хочешь, можешь меня очистить.
\vs Mar 1:41 Иисус, умилосердившись над ним, простер руку, коснулся его и сказал ему: хочу, очистись.
\vs Mar 1:42 После сего слова проказа тотчас сошла с него, и он стал чист.
\vs Mar 1:43 И, посмотрев на него строго, тотчас отослал его
\vs Mar 1:44 и сказал ему: смотри, никому ничего не говори, но пойди, покажись священнику и принеси за очищение твое, что повелел Моисей, во свидетельство им.
\vs Mar 1:45 А он, выйдя, начал провозглашать и рассказывать о происшедшем, так что \bibemph{Иисус} не мог уже явно войти в город, но находился вне, в местах пустынных. И приходили к Нему отовсюду.
\vs Mar 2:1 Через \bibemph{несколько} дней опять пришел Он в Капернаум; и слышно стало, что Он в доме.
\vs Mar 2:2 Тотчас собрались многие, так что уже и у дверей не было места; и Он говорил им слово.
\vs Mar 2:3 И пришли к Нему с расслабленным, которого несли четверо;
\vs Mar 2:4 и, не имея возможности приблизиться к Нему за многолюдством, раскрыли кровлю \bibemph{дома}, где Он находился, и, прокопав ее, спустили постель, на которой лежал расслабленный.
\vs Mar 2:5 Иисус, видя веру их, говорит расслабленному: чадо! прощаются тебе грехи твои.
\vs Mar 2:6 Тут сидели некоторые из книжников и помышляли в сердцах своих:
\vs Mar 2:7 что Он так богохульствует? кто может прощать грехи, кроме одного Бога?
\vs Mar 2:8 Иисус, тотчас узнав духом Своим, что они так помышляют в себе, сказал им: для чего так помышляете в сердцах ваших?
\vs Mar 2:9 Что легче? сказать ли расслабленному: прощаются тебе грехи? или сказать: встань, возьми свою постель и ходи?
\vs Mar 2:10 Но чтобы вы знали, что Сын Человеческий имеет власть на земле прощать грехи,~--- говорит расслабленному:
\vs Mar 2:11 тебе говорю: встань, возьми постель твою и иди в дом твой.
\vs Mar 2:12 Он тотчас встал и, взяв постель, вышел перед всеми, так что все изумлялись и прославляли Бога, говоря: никогда ничего такого мы не видали.
\rsbpar\vs Mar 2:13 И вышел \bibemph{Иисус} опять к морю; и весь народ пошел к Нему, и Он учил их.
\vs Mar 2:14 Проходя, увидел Он Левия Алфеева, сидящего у сбора пошлин, и говорит ему: следуй за Мною. И \bibemph{он}, встав, последовал за Ним.
\vs Mar 2:15 И когда Иисус возлежал в доме его, возлежали с Ним и ученики Его и многие мытари и грешники: ибо много их было, и они следовали за Ним.
\vs Mar 2:16 Книжники и фарисеи, увидев, что Он ест с мытарями и грешниками, говорили ученикам Его: как это Он ест и пьет с мытарями и грешниками?
\vs Mar 2:17 Услышав \bibemph{сие}, Иисус говорит им: не здоровые имеют нужду во враче, но больные; Я пришел призвать не праведников, но грешников к покаянию.
\rsbpar\vs Mar 2:18 Ученики Иоанновы и фарисейские постились. Приходят к Нему и говорят: почему ученики Иоанновы и фарисейские постятся, а Твои ученики не постятся?
\vs Mar 2:19 И сказал им Иисус: могут ли поститься сыны чертога брачного, когда с ними жених? Доколе с ними жених, не могут поститься,
\vs Mar 2:20 но придут дни, когда отнимется у них жених, и тогда будут поститься в те дни.
\vs Mar 2:21 Никто к ветхой одежде не приставляет заплаты из небеленой ткани: иначе вновь пришитое отдерет от старого, и дыра будет еще хуже.
\vs Mar 2:22 Никто не вливает вина молодого в мехи ветхие: иначе молодое вино прорвет мехи, и вино вытечет, и мехи пропадут; но вино молодое надобно вливать в мехи новые.
\rsbpar\vs Mar 2:23 И случилось Ему в субботу проходить засеянными \bibemph{полями}, и ученики Его дорогою начали срывать колосья.
\vs Mar 2:24 И фарисеи сказали Ему: смотри, чт\acc{о} они делают в субботу, чего не должно \bibemph{делать}?
\vs Mar 2:25 Он сказал им: неужели вы не читали никогда, чт\acc{о} сделал Давид, когда имел нужду и взалкал сам и бывшие с ним?
\vs Mar 2:26 как вошел он в дом Божий при первосвященнике Авиафаре и ел хлебы предложения, которых не должно было есть никому, кроме священников, и дал и бывшим с ним?
\vs Mar 2:27 И сказал им: суббота для человека, а не человек для субботы;
\vs Mar 2:28 посему Сын Человеческий есть господин и субботы.
\vs Mar 3:1 И пришел опять в синагогу; там был человек, имевший иссохшую руку.
\vs Mar 3:2 И наблюдали за Ним, не исцелит ли его в субботу, чтобы обвинить Его.
\vs Mar 3:3 Он же говорит человеку, имевшему иссохшую руку: стань на средину.
\vs Mar 3:4 А им говорит: должно ли в субботу добро делать, или зло делать? душу спасти, или погубить? Но они молчали.
\vs Mar 3:5 И, воззрев на них с гневом, скорбя об ожесточении сердец их, говорит тому человеку: протяни руку твою. Он протянул, и стала рука его здорова, как другая.
\rsbpar\vs Mar 3:6 Фарисеи, выйдя, немедленно составили с иродианами совещание против Него, как бы погубить Его.
\vs Mar 3:7 Но Иисус с учениками Своими удалился к морю; и за Ним последовало множество народа из Галилеи, Иудеи,
\vs Mar 3:8 Иерусалима, Идумеи и из-за Иордана. И \bibemph{живущие} в окрестностях Тира и Сидона, услышав, что Он делал, шли к Нему в великом множестве.
\vs Mar 3:9 И сказал ученикам Своим, чтобы готова была для Него лодка по причине многолюдства, дабы не теснили Его.
\vs Mar 3:10 Ибо многих Он исцелил, так что имевшие язвы бросались к Нему, чтобы коснуться Его.
\vs Mar 3:11 И духи нечистые, когда видели Его, падали пред Ним и кричали: Ты Сын Божий.
\vs Mar 3:12 Но Он строго запрещал им, чтобы не делали Его известным.
\rsbpar\vs Mar 3:13 Потом взошел на гору и позвал к Себе, кого Сам хотел; и пришли к Нему.
\vs Mar 3:14 И поставил \bibemph{из них} двенадцать, чтобы с Ним были и чтобы посылать их на проповедь,
\vs Mar 3:15 и чтобы они имели власть исцелять от болезней и изгонять бесов;
\vs Mar 3:16 \bibemph{поставил} Симона, нарекши ему имя Петр,
\vs Mar 3:17 Иакова Зеведеева и Иоанна, брата Иакова, нарекши им имена Воанергес, то есть <<сыны громовы>>,
\vs Mar 3:18 Андрея, Филиппа, Варфоломея, Матфея, Фому, Иакова Алфеева, Фаддея, Симона Кананита
\vs Mar 3:19 и Иуду Искариотского, который и предал Его.
\rsbpar\vs Mar 3:20 Приходят в дом; и опять сходится народ, так что им невозможно было и хлеба есть.
\vs Mar 3:21 И, услышав, ближние Его пошли взять Его, ибо говорили, что Он вышел из себя.
\vs Mar 3:22 А книжники, пришедшие из Иерусалима, говорили, что Он имеет \bibemph{в Себе} веельзевула и что изгоняет бесов силою бесовского князя.
\vs Mar 3:23 И, призвав их, говорил им притчами: как может сатана изгонять сатану?
\vs Mar 3:24 Если царство разделится само в себе, не может устоять царство т\acc{о};
\vs Mar 3:25 и если дом разделится сам в себе, не может устоять дом тот;
\vs Mar 3:26 и если сатана восстал на самого себя и разделился, не может устоять, но пришел конец его.
\vs Mar 3:27 Никто, войдя в дом сильного, не может расхитить вещей его, если прежде не свяжет сильного, и тогда расхитит дом его.
\vs Mar 3:28 Истинно говорю вам: будут прощены сынам человеческим все грехи и хуления, какими бы ни хулили;
\vs Mar 3:29 но кто будет хулить Духа Святаго, тому не будет прощения вовек, но подлежит он вечному осуждению.
\vs Mar 3:30 \bibemph{Сие сказал Он}, потому что говорили: в Нем нечистый дух.
\rsbpar\vs Mar 3:31 И пришли Матерь и братья Его и, стоя вне \bibemph{дома}, послали к Нему звать Его.
\vs Mar 3:32 Около Него сидел народ. И сказали Ему: вот, Матерь Твоя и братья Твои и сестры Твои, вне \bibemph{дома}, спрашивают Тебя.
\vs Mar 3:33 И отвечал им: кто матерь Моя и братья Мои?
\vs Mar 3:34 И обозрев сидящих вокруг Себя, говорит: вот матерь Моя и братья Мои;
\vs Mar 3:35 ибо кто будет исполнять волю Божию, тот Мне брат, и сестра, и матерь.
\vs Mar 4:1 И опять начал учить при море; и собралось к Нему множество народа, так что Он вошел в лодку и сидел на море, а весь народ был на земле, у моря.
\vs Mar 4:2 И учил их притчами много, и в учении Своем говорил им:
\vs Mar 4:3 слушайте: вот, вышел сеятель сеять;
\vs Mar 4:4 и, когда сеял, случилось, что иное упало при дороге, и налетели птицы и поклевали т\acc{о}.
\vs Mar 4:5 Иное упало на каменистое \bibemph{место}, где немного было земли, и скоро взошло, потому что земля была неглубока;
\vs Mar 4:6 когда же взошло солнце, увяло и, как не имело корня, засохло.
\vs Mar 4:7 Иное упало в терние, и терние выросло, и заглушило \bibemph{семя}, и оно не дало плода.
\vs Mar 4:8 И иное упало на добрую землю и дало плод, который взошел и вырос, и принесло иное тридцать, иное шестьдесят, и иное сто.
\vs Mar 4:9 И сказал им: кто имеет уши слышать, да слышит!
\vs Mar 4:10 Когда же остался без народа, окружающие Его, вместе с двенадцатью, спросили Его о притче.
\vs Mar 4:11 И сказал им: вам дано знать тайны Царствия Божия, а тем внешним все бывает в притчах;
\vs Mar 4:12 так что они своими глазами смотрят, и не видят; своими ушами слышат, и не разумеют, да не обратятся, и прощены будут им грехи.
\vs Mar 4:13 И говорит им: не понимаете этой притчи? Как же вам уразуметь все притчи?
\vs Mar 4:14 Сеятель слово сеет.
\vs Mar 4:15 \bibemph{Посеянное} при дороге означает тех, в которых сеется слово, но \bibemph{к которым}, когда услышат, тотчас приходит сатана и похищает слово, посеянное в сердцах их.
\vs Mar 4:16 Подобным образом и посеянное на каменистом \bibemph{месте} означает тех, которые, когда услышат слово, тотчас с радостью принимают его,
\vs Mar 4:17 но не имеют в себе корня и непостоянны; потом, когда настанет скорбь или гонение за слово, тотчас соблазняются.
\vs Mar 4:18 Посеянное в тернии означает слышащих слово,
\vs Mar 4:19 но в которых заботы века сего, обольщение богатством и другие пожелания, входя в них, заглушают слово, и оно бывает без плода.
\vs Mar 4:20 А посеянное на доброй земле означает тех, которые слушают слово и принимают, и приносят плод, один в тридцать, другой в шестьдесят, иной во сто крат.
\rsbpar\vs Mar 4:21 И сказал им: для того ли приносится свеча, чтобы поставить ее под сосуд или под кровать? не для того ли, чтобы поставить ее на подсвечнике?
\vs Mar 4:22 Нет ничего тайного, что не сделалось бы явным, и ничего не бывает потаенного, что не вышло бы наружу.
\vs Mar 4:23 Если кто имеет уши слышать, да слышит!
\vs Mar 4:24 И сказал им: замечайте, что слышите: какою мерою мерите, такою отмерено будет вам и прибавлено будет вам, слушающим.
\vs Mar 4:25 Ибо кто имеет, тому дано будет, а кто не имеет, у того отнимется и то, что имеет.
\rsbpar\vs Mar 4:26 И сказал: Царствие Божие подобно тому, как если человек бросит семя в землю,
\vs Mar 4:27 и спит, и встает ночью и днем; и к\acc{а}к семя всходит и растет, не знает он,
\vs Mar 4:28 ибо земля сама собою производит сперва зелень, потом колос, потом полное зерно в колосе.
\vs Mar 4:29 Когда же созреет плод, немедленно посылает серп, потому что настала жатва.
\rsbpar\vs Mar 4:30 И сказал: чему уподобим Царствие Божие? или какою притчею изобразим его?
\vs Mar 4:31 Оно~--- как зерно горчичное, которое, когда сеется в землю, есть меньше всех семян на земле;
\vs Mar 4:32 а когда посеяно, всходит и становится больше всех злаков, и пускает большие ветви, так что под тенью его могут укрываться птицы небесные.
\vs Mar 4:33 И таковыми многими притчами проповедовал им слово, сколько они могли слышать.
\vs Mar 4:34 Без притчи же не говорил им, а ученикам наедине изъяснял все.
\rsbpar\vs Mar 4:35 Вечером того дня сказал им: переправимся на ту сторону.
\vs Mar 4:36 И они, отпустив народ, взяли Его с собою, как Он был в лодке; с Ним были и другие лодки.
\vs Mar 4:37 И поднялась великая буря; волны били в лодку, так что она уже наполнялась \bibemph{водою}.
\vs Mar 4:38 А Он спал на корме на возглавии. Его будят и говорят Ему: Учитель! неужели Тебе нужды нет, что мы погибаем?
\vs Mar 4:39 И, встав, Он запретил ветру и сказал морю: умолкни, перестань. И ветер утих, и сделалась великая тишина.
\vs Mar 4:40 И сказал им: что вы так боязливы? как у вас нет веры?
\vs Mar 4:41 И убоялись страхом великим и говорили между собою: кто же Сей, что и ветер и море повинуются Ему?
\vs Mar 5:1 И пришли на другой берег моря, в страну Гадаринскую.
\vs Mar 5:2 И когда вышел Он из лодки, тотчас встретил Его вышедший из гробов человек, \bibemph{одержимый} нечистым духом,
\vs Mar 5:3 он имел жилище в гробах, и никто не мог его связать даже цепями,
\vs Mar 5:4 потому что многократно был он скован оковами и цепями, но разрывал цепи и разбивал оковы, и никто не в силах был укротить его;
\vs Mar 5:5 всегда, ночью и днем, в горах и гробах, кричал он и бился о камни;
\vs Mar 5:6 увидев же Иисуса издалека, прибежал и поклонился Ему,
\vs Mar 5:7 и, вскричав громким голосом, сказал: что Тебе до меня, Иисус, Сын Бога Всевышнего? заклинаю Тебя Богом, не мучь меня!
\vs Mar 5:8 Ибо \bibemph{Иисус} сказал ему: выйди, дух нечистый, из сего человека.
\vs Mar 5:9 И спросил его: как тебе имя? И он сказал в ответ: легион имя мне, потому что нас много.
\vs Mar 5:10 И много просили Его, чтобы не высылал их вон из страны той.
\vs Mar 5:11 Паслось же там при горе большое стадо свиней.
\vs Mar 5:12 И просили Его все бесы, говоря: пошли нас в свиней, чтобы нам войти в них.
\vs Mar 5:13 Иисус тотчас позволил им. И нечистые духи, выйдя, вошли в свиней; и устремилось стадо с крутизны в море, а их было около двух тысяч; и потонули в море.
\vs Mar 5:14 Пасущие же свиней побежали и рассказали в городе и в деревнях. И \bibemph{жители} вышли посмотреть, что случилось.
\vs Mar 5:15 Приходят к Иисусу и видят, что бесновавшийся, в котором был легион, сидит и одет, и в здравом уме; и устрашились.
\vs Mar 5:16 Видевшие рассказали им о том, как это произошло с бесноватым, и о свиньях.
\vs Mar 5:17 И начали просить Его, чтобы отошел от пределов их.
\vs Mar 5:18 И когда Он вошел в лодку, бесновавшийся просил Его, чтобы быть с Ним.
\vs Mar 5:19 Но Иисус не дозволил ему, а сказал: иди домой к своим и расскажи им, что сотворил с тобою Господь и \bibemph{как} помиловал тебя.
\vs Mar 5:20 И пошел и начал проповедовать в Десятиградии, что сотворил с ним Иисус; и все дивились.
\rsbpar\vs Mar 5:21 Когда Иисус опять переправился в лодке на другой берег, собралось к Нему множество народа. Он был у моря.
\vs Mar 5:22 И вот, приходит один из начальников синагоги, по имени Иаир, и, увидев Его, падает к ногам Его
\vs Mar 5:23 и усильно просит Его, говоря: дочь моя при смерти; приди и возложи на нее руки, чтобы она выздоровела и осталась жива.
\vs Mar 5:24 \bibemph{Иисус} пошел с ним. За Ним следовало множество народа, и теснили Его.
\rsbpar\vs Mar 5:25 Одна женщина, которая страдала кровотечением двенадцать лет,
\vs Mar 5:26 много потерпела от многих врачей, истощила всё, что было у ней, и не получила никакой пользы, но пришла еще в худшее состояние,~---
\vs Mar 5:27 услышав об Иисусе, подошла сзади в народе и прикоснулась к одежде Его,
\vs Mar 5:28 ибо говорила: если хотя к одежде Его прикоснусь, то выздоровею.
\vs Mar 5:29 И тотчас иссяк у ней источник крови, и она ощутила в теле, что исцелена от болезни.
\vs Mar 5:30 В то же время Иисус, почувствовав Сам в Себе, что вышла из Него сила, обратился в народе и сказал: кто прикоснулся к Моей одежде?
\vs Mar 5:31 Ученики сказали Ему: Ты видишь, что народ теснит Тебя, и говоришь: кто прикоснулся ко Мне?
\vs Mar 5:32 Но Он смотрел вокруг, чтобы видеть ту, которая сделала это.
\vs Mar 5:33 Женщина в страхе и трепете, зная, что с нею произошло, подошла, пала пред Ним и сказала Ему всю истину.
\vs Mar 5:34 Он же сказал ей: дщерь! вера твоя спасла тебя; иди в мире и будь здорова от болезни твоей.
\rsbpar\vs Mar 5:35 Когда Он еще говорил сие, приходят от начальника синагоги и говорят: дочь твоя умерла; что еще утруждаешь Учителя?
\vs Mar 5:36 Но Иисус, услышав сии слова, тотчас говорит начальнику синагоги: не бойся, только веруй.
\vs Mar 5:37 И не позволил никому следовать за Собою, кроме Петра, Иакова и Иоанна, брата Иакова.
\vs Mar 5:38 Приходит в дом начальника синагоги и видит смятение и плачущих и вопиющих громко.
\vs Mar 5:39 И, войдя, говорит им: что смущаетесь и плачете? девица не умерла, но спит.
\vs Mar 5:40 И смеялись над Ним. Но Он, выслав всех, берет с Собою отца и мать девицы и бывших с Ним и входит туда, где девица лежала.
\vs Mar 5:41 И, взяв девицу за руку, говорит ей: <<талиф\acc{а} кум\acc{и}>>, что значит: девица, тебе говорю, встань.
\vs Mar 5:42 И девица тотчас встала и начала ходить, ибо была лет двенадцати. \bibemph{Видевшие} пришли в великое изумление.
\vs Mar 5:43 И Он строго приказал им, чтобы никто об этом не знал, и сказал, чтобы дали ей есть.
\vs Mar 6:1 Оттуда вышел Он и пришел в Свое отечество; за Ним следовали ученики Его.
\vs Mar 6:2 Когда наступила суббота, Он начал учить в синагоге; и многие слышавшие с изумлением говорили: откуда у Него это? что за премудрость дана Ему, и как такие чудеса совершаются руками Его?
\vs Mar 6:3 Не плотник ли Он, сын Марии, брат Иакова, Иосии, Иуды и Симона? Не здесь ли, между нами, Его сестры? И соблазнялись о Нем.
\vs Mar 6:4 Иисус же сказал им: не бывает пророк без чести, разве только в отечестве своем и у сродников и в доме своем.
\vs Mar 6:5 И не мог совершить там никакого чуда, только на немногих больных возложив руки, исцелил \bibemph{их}.
\vs Mar 6:6 И дивился неверию их; потом ходил по окрестным селениям и учил.
\rsbpar\vs Mar 6:7 И, призвав двенадцать, начал посылать их по два, и дал им власть над нечистыми духами.
\vs Mar 6:8 И заповедал им ничего не брать в дорогу, кроме одного посоха: ни сумы, ни хлеба, ни меди в поясе,
\vs Mar 6:9 но обуваться в простую обувь и не носить двух одежд.
\vs Mar 6:10 И сказал им: если где войдете в дом, оставайтесь в нем, доколе не выйдете из того места.
\vs Mar 6:11 И если кто не примет вас и не будет слушать вас, то, выходя оттуда, отрясите прах от ног ваших, во свидетельство на них. Истинно говорю вам: отраднее будет Содому и Гоморре в день суда, нежели тому городу.
\vs Mar 6:12 Они пошли и проповедовали покаяние;
\vs Mar 6:13 изгоняли многих бесов и многих больных мазали маслом и исцеляли.
\rsbpar\vs Mar 6:14 Царь Ирод, услышав \bibemph{об Иисусе} (ибо имя Его стало гласно), говорил: это Иоанн Креститель воскрес из мертвых, и потому чудеса делаются им.
\vs Mar 6:15 Другие говорили: это Илия, а иные говорили: это пророк, или как один из пророков.
\vs Mar 6:16 Ирод же, услышав, сказал: это Иоанн, которого я обезглавил; он воскрес из мертвых.
\vs Mar 6:17 Ибо сей Ирод, послав, взял Иоанна и заключил его в темницу за Иродиаду, жену Филиппа, брата своего, потому что женился на ней.
\vs Mar 6:18 Ибо Иоанн говорил Ироду: не должно тебе иметь жену брата твоего.
\vs Mar 6:19 Иродиада же, злобясь на него, желала убить его; но не могла.
\vs Mar 6:20 Ибо Ирод боялся Иоанна, зная, что он муж праведный и святой, и берёг его; многое делал, слушаясь его, и с удовольствием слушал его.
\vs Mar 6:21 Настал удобный день, когда Ирод, по случаю \bibemph{дня} рождения своего, делал пир вельможам своим, тысяченачальникам и старейшинам Галилейским,~---
\vs Mar 6:22 дочь Иродиады вошла, плясала и угодила Ироду и возлежавшим с ним; царь сказал девице: проси у меня, чего хочешь, и дам тебе;
\vs Mar 6:23 и клялся ей: чего ни попросишь у меня, дам тебе, даже до половины моего царства.
\vs Mar 6:24 Она вышла и спросила у матери своей: чего просить? Та отвечала: головы Иоанна Крестителя.
\vs Mar 6:25 И она тотчас пошла с поспешностью к царю и просила, говоря: хочу, чтобы ты дал мне теперь же на блюде голову Иоанна Крестителя.
\vs Mar 6:26 Царь опечалился, но ради клятвы и возлежавших с ним не захотел отказать ей.
\vs Mar 6:27 И тотчас, послав оруженосца, царь повелел принести голову его.
\vs Mar 6:28 Он пошел, отсек ему голову в темнице, и принес голову его на блюде, и отдал ее девице, а девица отдала ее матери своей.
\vs Mar 6:29 Ученики его, услышав, пришли и взяли тело его, и положили его во гробе.
\rsbpar\vs Mar 6:30 И собрались Апостолы к Иисусу и рассказали Ему всё, и что сделали, и чему научили.
\vs Mar 6:31 Он сказал им: пойдите вы одни в пустынное место и отдохните немного,~--- ибо много было приходящих и отходящих, так что и есть им было некогда.
\vs Mar 6:32 И отправились в пустынное место в лодке одни.
\vs Mar 6:33 Народ увидел, \bibemph{как} они отправлялись, и многие узнали их; и бежали туда пешие из всех городов, и предупредили их, и собрались к Нему.
\vs Mar 6:34 Иисус, выйдя, увидел множество народа и сжалился над ними, потому что они были, как овцы, не имеющие пастыря; и начал учить их много.
\vs Mar 6:35 И как времени прошло много, ученики Его, приступив к Нему, говорят: место \bibemph{здесь} пустынное, а времени уже много,~---
\vs Mar 6:36 отпусти их, чтобы они пошли в окрестные деревни и селения и купили себе хлеба, ибо им нечего есть.
\vs Mar 6:37 Он сказал им в ответ: вы дайте им есть. И сказали Ему: разве нам пойти купить хлеба динариев на двести и дать им есть?
\vs Mar 6:38 Но Он спросил их: сколько у вас хлебов? пойдите, посмотрите. Они, узнав, сказали: пять хлебов и две рыбы.
\vs Mar 6:39 Тогда повелел им рассадить всех отделениями на зеленой траве.
\vs Mar 6:40 И сели рядами, по сто и по пятидесяти.
\vs Mar 6:41 Он взял пять хлебов и две рыбы, воззрев на небо, благословил и преломил хлебы и дал ученикам Своим, чтобы они раздали им; и две рыбы разделил на всех.
\vs Mar 6:42 И ели все, и насытились.
\vs Mar 6:43 И набрали кусков хлеба и \bibemph{остатков} от рыб двенадцать полных коробов.
\vs Mar 6:44 Было же евших хлебы около пяти тысяч мужей.
\rsbpar\vs Mar 6:45 И тотчас понудил учеников Своих войти в лодку и отправиться вперед на другую сторону к Вифсаиде, пока Он отпустит народ.
\vs Mar 6:46 И, отпустив их, пошел на гору помолиться.
\vs Mar 6:47 Вечером лодка была посреди моря, а Он один на земле.
\vs Mar 6:48 И увидел их бедствующих в плавании, потому что ветер им был противный; около же четвертой стражи ночи подошел к ним, идя по морю, и хотел миновать их.
\vs Mar 6:49 Они, увидев Его идущего по морю, подумали, что это призрак, и вскричали.
\vs Mar 6:50 Ибо все видели Его и испугались. И тотчас заговорил с ними и сказал им: ободритесь; это Я, не бойтесь.
\vs Mar 6:51 И вошел к ним в лодку, и ветер утих. И они чрезвычайно изумлялись в себе и дивились,
\vs Mar 6:52 ибо не вразумились \bibemph{чудом} над хлебами, потому что сердце их было окаменено.
\vs Mar 6:53 И, переправившись, прибыли в землю Геннисаретскую и пристали \bibemph{к берегу}.
\vs Mar 6:54 Когда вышли они из лодки, тотчас \bibemph{жители}, узнав Его,
\vs Mar 6:55 обежали всю окрестность ту и начали на постелях приносить больных туда, где Он, как слышно было, находился.
\vs Mar 6:56 И куда ни приходил Он, в селения ли, в города ли, в деревни ли, клали больных на открытых местах и просили Его, чтобы им прикоснуться хотя к краю одежды Его; и которые прикасались к Нему, исцелялись.
\vs Mar 7:1 Собрались к Нему фарисеи и некоторые из книжников, пришедшие из Иерусалима,
\vs Mar 7:2 и, увидев некоторых из учеников Его, евших хлеб нечистыми, то есть неумытыми, руками, укоряли.
\vs Mar 7:3 Ибо фарисеи и все Иудеи, держась предания старцев, не едят, не умыв тщательно рук;
\vs Mar 7:4 и, \bibemph{придя} с торга, не едят не омывшись. Есть и многое другое, чего они приняли держаться: наблюдать омовение чаш, кружек, котлов и скамей.
\vs Mar 7:5 Потом спрашивают Его фарисеи и книжники: зачем ученики Твои не поступают по преданию старцев, но неумытыми руками едят хлеб?
\vs Mar 7:6 Он сказал им в ответ: хорошо пророчествовал о вас, лицемерах, Исаия, как написано: люди сии чтут Меня устами, сердце же их далеко отстоит от Меня,
\vs Mar 7:7 но тщетно чтут Меня, уча учениям, заповедям человеческим.
\vs Mar 7:8 Ибо вы, оставив заповедь Божию, держитесь предания человеческого, омовения кружек и чаш, и делаете многое другое, сему подобное.
\vs Mar 7:9 И сказал им: хорошо ли, \bibemph{что} вы отменяете заповедь Божию, чтобы соблюсти свое предание?
\vs Mar 7:10 Ибо Моисей сказал: почитай отца своего и мать свою; и: злословящий отца или мать смертью да умрет.
\vs Mar 7:11 А вы говорите: кто скажет отцу или матери: корван, то есть дар \bibemph{Богу} т\acc{о}, чем бы ты от меня пользовался,
\vs Mar 7:12 тому вы уже попускаете ничего не делать для отца своего или матери своей,
\vs Mar 7:13 устраняя слово Божие преданием вашим, которое вы установили; и делаете многое сему подобное.
\vs Mar 7:14 И, призвав весь народ, говорил им: слушайте Меня все и разумейте:
\vs Mar 7:15 ничто, входящее в человека извне, не может осквернить его; но что исходит из него, то оскверняет человека.
\vs Mar 7:16 Если кто имеет уши слышать, да слышит!
\vs Mar 7:17 И когда Он от народа вошел в дом, ученики Его спросили Его о притче.
\vs Mar 7:18 Он сказал им: неужели и вы так непонятливы? Неужели не разумеете, что ничто, извне входящее в человека, не может осквернить его?
\vs Mar 7:19 Потому что не в сердце его входит, а в чрево, и выходит вон, \bibemph{чем} очищается всякая пища.
\vs Mar 7:20 Далее сказал: исходящее из человека оскверняет человека.
\vs Mar 7:21 Ибо извнутрь, из сердца человеческого, исходят злые помыслы, прелюбодеяния, любодеяния, убийства,
\vs Mar 7:22 кражи, лихоимство, злоба, коварство, непотребство, завистливое око, богохульство, гордость, безумство,~---
\vs Mar 7:23 всё это зло извнутрь исходит и оскверняет человека.
\rsbpar\vs Mar 7:24 И, отправившись оттуда, пришел в пределы Тирские и Сидонские; и, войдя в дом, не хотел, чтобы кто узнал; но не мог утаиться.
\vs Mar 7:25 Ибо услышала о Нем женщина, у которой дочь одержима была нечистым духом, и, придя, припала к ногам Его;
\vs Mar 7:26 а женщина та была язычница, родом сирофиникиянка; и просила Его, чтобы изгнал беса из ее дочери.
\vs Mar 7:27 Но Иисус сказал ей: дай прежде насытиться детям, ибо нехорошо взять хлеб у детей и бросить псам.
\vs Mar 7:28 Она же сказала Ему в ответ: так, Господи; но и псы под столом едят крохи у детей.
\vs Mar 7:29 И сказал ей: за это слово, пойди; бес вышел из твоей дочери.
\vs Mar 7:30 И, придя в свой дом, она нашла, что бес вышел и дочь лежит на постели.
\rsbpar\vs Mar 7:31 Выйдя из пределов Тирских и Сидонских, \bibemph{Иисус} опять пошел к морю Галилейскому через пределы Десятиградия.
\vs Mar 7:32 Привели к Нему глухого косноязычного и просили Его возложить на него руку.
\vs Mar 7:33 \bibemph{Иисус}, отведя его в сторону от народа, вложил персты Свои в уши ему и, плюнув, коснулся языка его;
\vs Mar 7:34 и, воззрев на небо, вздохнул и сказал ему: <<еффаф\acc{а}>>, то есть: отверзись.
\vs Mar 7:35 И тотчас отверзся у него слух и разрешились узы его языка, и стал говорить чисто.
\vs Mar 7:36 И повелел им не сказывать никому. Но сколько Он ни запрещал им, они еще более разглашали.
\vs Mar 7:37 И чрезвычайно дивились, и говорили: всё хорошо делает,~--- и глухих делает слышащими, и немых~--- говорящими.
\vs Mar 8:1 В те дни, когда собралось весьма много народа и нечего было им есть, Иисус, призвав учеников Своих, сказал им:
\vs Mar 8:2 жаль Мне народа, что уже три дня находятся при Мне, и нечего им есть.
\vs Mar 8:3 Если неевшими отпущу их в домы их, ослабеют в дороге, ибо некоторые из них пришли издалека.
\vs Mar 8:4 Ученики Его отвечали Ему: откуда мог бы кто \bibemph{взять} здесь в пустыне хлебов, чтобы накормить их?
\vs Mar 8:5 И спросил их: сколько у вас хлебов? Они сказали: семь.
\vs Mar 8:6 Тогда велел народу возлечь на землю; и, взяв семь хлебов и воздав благодарение, преломил и дал ученикам Своим, чтобы они раздали; и они раздали народу.
\vs Mar 8:7 Было у них и немного рыбок: благословив, Он велел раздать и их.
\vs Mar 8:8 И ели, и насытились; и набрали оставшихся кусков семь корзин.
\vs Mar 8:9 Евших же было около четырех тысяч. И отпустил их.
\rsbpar\vs Mar 8:10 И тотчас войдя в лодку с учениками Своими, прибыл в пределы Далмануфские.
\vs Mar 8:11 Вышли фарисеи, начали с Ним спорить и требовали от Него знамения с неба, искушая Его.
\vs Mar 8:12 И Он, глубоко вздохнув, сказал: для чего род сей требует знамения? Истинно говорю вам, не дастся роду сему знамение.
\vs Mar 8:13 И, оставив их, опять вошел в лодку и отправился на ту сторону.
\vs Mar 8:14 При сем ученики Его забыли взять хлебов и кроме одного хлеба не имели с собою в лодке.
\vs Mar 8:15 А Он заповедал им, говоря: смотрите, берегитесь закваски фарисейской и закваски Иродовой.
\vs Mar 8:16 И, рассуждая между собою, говорили: \bibemph{это значит}, что хлебов нет у нас.
\vs Mar 8:17 Иисус, уразумев, говорит им: что рассуждаете о том, что нет у вас хлебов? Еще ли не понимаете и не разумеете? Еще ли окаменено у вас сердце?
\vs Mar 8:18 Имея очи, не видите? имея уши, не слышите? и не помните?
\vs Mar 8:19 Когда Я пять хлебов преломил для пяти тысяч \bibemph{человек}, сколько полных коробов набрали вы кусков? Говорят Ему: двенадцать.
\vs Mar 8:20 А когда семь для четырех тысяч, сколько корзин набрали вы оставшихся кусков? Сказали: семь.
\vs Mar 8:21 И сказал им: как же не разумеете?
\rsbpar\vs Mar 8:22 Приходит в Вифсаиду; и приводят к Нему слепого и просят, чтобы прикоснулся к нему.
\vs Mar 8:23 Он, взяв слепого за руку, вывел его вон из селения и, плюнув ему на глаза, возложил на него руки и спросил его: видит ли что?
\vs Mar 8:24 Он, взглянув, сказал: вижу проходящих людей, как деревья.
\vs Mar 8:25 Потом опять возложил руки на глаза ему и велел ему взглянуть. И он исцелел и стал видеть все ясно.
\vs Mar 8:26 И послал его домой, сказав: не заходи в селение и не рассказывай никому в селении.
\rsbpar\vs Mar 8:27 И пошел Иисус с учениками Своими в селения Кесарии Филипповой. Дорогою Он спрашивал учеников Своих: за кого почитают Меня люди?
\vs Mar 8:28 Они отвечали: за Иоанна Крестителя; другие же~--- за Илию; а иные~--- за одного из пророков.
\vs Mar 8:29 Он говорит им: а вы за кого почитаете Меня? Петр сказал Ему в ответ: Ты Христос.
\vs Mar 8:30 И запретил им, чтобы никому не говорили о Нем.
\vs Mar 8:31 И начал учить их, что Сыну Человеческому много должно пострадать, быть отвержену старейшинами, первосвященниками и книжниками, и быть убиту, и в третий день воскреснуть.
\vs Mar 8:32 И говорил о сем открыто. Но Петр, отозвав Его, начал прекословить Ему.
\vs Mar 8:33 Он же, обратившись и взглянув на учеников Своих, воспретил Петру, сказав: отойди от Меня, сатана, потому что ты думаешь не о том, что Божие, но что человеческое.
\vs Mar 8:34 И, подозвав народ с учениками Своими, сказал им: кто хочет идти за Мною, отвергнись себя, и возьми крест свой, и следуй за Мною.
\vs Mar 8:35 Ибо кто хочет душу свою сберечь, тот потеряет ее, а кто потеряет душу свою ради Меня и Евангелия, тот сбережет ее.
\vs Mar 8:36 Ибо какая польза человеку, если он приобретет весь мир, а душе своей повредит?
\vs Mar 8:37 Или какой выкуп даст человек за душу свою?
\vs Mar 8:38 Ибо кто постыдится Меня и Моих слов в роде сем прелюбодейном и грешном, того постыдится и Сын Человеческий, когда приидет в славе Отца Своего со святыми Ангелами.
\vs Mar 9:1 И сказал им: истинно говорю вам: есть некоторые из стоящих здесь, которые не вкусят смерти, как уже увидят Царствие Божие, пришедшее в силе.
\vs Mar 9:2 И, по прошествии дней шести, взял Иисус Петра, Иакова и Иоанна, и возвел на гору высокую особо их одних, и преобразился перед ними.
\vs Mar 9:3 Одежды Его сделались блистающими, весьма белыми, как снег, как на земле белильщик не может выбелить.
\vs Mar 9:4 И явился им Илия с Моисеем; и беседовали с Иисусом.
\vs Mar 9:5 При сем Петр сказал Иисусу: Равв\acc{и}! хорошо нам здесь быть; сделаем три кущи: Тебе одну, Моисею одну, и одну Илии.
\vs Mar 9:6 Ибо не знал, что сказать; потому что они были в страхе.
\vs Mar 9:7 И явилось облако, осеняющее их, и из облака исшел глас, глаголющий: Сей есть Сын Мой возлюбленный; Его слушайте.
\vs Mar 9:8 И, внезапно посмотрев вокруг, никого более с собою не видели, кроме одного Иисуса.
\vs Mar 9:9 Когда же сходили они с горы, Он не велел никому рассказывать о том, что видели, доколе Сын Человеческий не воскреснет из мертвых.
\vs Mar 9:10 И они удержали это слово, спрашивая друг друга, что значит: воскреснуть из мертвых.
\vs Mar 9:11 И спросили Его: как же книжники говорят, что Илии надлежит прийти прежде?
\vs Mar 9:12 Он сказал им в ответ: правда, Илия должен прийти прежде и устроить всё; и Сыну Человеческому, как написано о Нем, \bibemph{надлежит} много пострадать и быть уничижену.
\vs Mar 9:13 Но говорю вам, что и Илия пришел, и поступили с ним, как хотели, как написано о нем.
\rsbpar\vs Mar 9:14 Придя к ученикам, увидел много народа около них и книжников, спорящих с ними.
\vs Mar 9:15 Тотчас, увидев Его, весь народ изумился, и, подбегая, приветствовали Его.
\vs Mar 9:16 Он спросил книжников: о чем спорите с ними?
\vs Mar 9:17 Один из народа сказал в ответ: Учитель! я привел к Тебе сына моего, одержимого духом немым:
\vs Mar 9:18 где ни схватывает его, повергает его на землю, и он испускает пену, и скрежещет зубами своими, и цепенеет. Говорил я ученикам Твоим, чтобы изгнали его, и они не могли.
\vs Mar 9:19 Отвечая ему, Иисус сказал: о, род неверный! доколе буду с вами? доколе буду терпеть вас? Приведите его ко Мне.
\vs Mar 9:20 И привели его к Нему. Как скоро \bibemph{бесноватый} увидел Его, дух сотряс его; он упал на землю и валялся, испуская пену.
\vs Mar 9:21 И спросил \bibemph{Иисус} отца его: как давно это сделалось с ним? Он сказал: с детства;
\vs Mar 9:22 и многократно \bibemph{дух} бросал его и в огонь и в воду, чтобы погубить его; но, если что можешь, сжалься над нами и помоги нам.
\vs Mar 9:23 Иисус сказал ему: если сколько-нибудь можешь веровать, всё возможно верующему.
\vs Mar 9:24 И тотчас отец отрока воскликнул со слезами: верую, Господи! помоги моему неверию.
\vs Mar 9:25 Иисус, видя, что сбегается народ, запретил духу нечистому, сказав ему: дух немой и глухой! Я повелеваю тебе, выйди из него и впредь не входи в него.
\vs Mar 9:26 И, вскрикнув и сильно сотрясши его, вышел; и он сделался, как мертвый, так что многие говорили, что он умер.
\vs Mar 9:27 Но Иисус, взяв его за руку, поднял его; и он встал.
\vs Mar 9:28 И как вошел \bibemph{Иисус} в дом, ученики Его спрашивали Его наедине: почему мы не могли изгнать его?
\vs Mar 9:29 И сказал им: сей род не может выйти иначе, как от молитвы и поста.
\rsbpar\vs Mar 9:30 Выйдя оттуда, проходили через Галилею; и Он не хотел, чтобы кто узнал.
\vs Mar 9:31 Ибо учил Своих учеников и говорил им, что Сын Человеческий предан будет в руки человеческие и убьют Его, и, по убиении, в третий день воскреснет.
\vs Mar 9:32 Но они не разумели сих слов, а спросить Его боялись.
\rsbpar\vs Mar 9:33 Пришел в Капернаум; и когда был в доме, спросил их: о чем дорогою вы рассуждали между собою?
\vs Mar 9:34 Они молчали; потому что дорогою рассуждали между собою, кто больше.
\vs Mar 9:35 И, сев, призвал двенадцать и сказал им: кто хочет быть первым, будь из всех последним и всем слугою.
\vs Mar 9:36 И, взяв дитя, поставил его посреди них и, обняв его, сказал им:
\vs Mar 9:37 кто примет одно из таких детей во имя Мое, тот принимает Меня; а кто Меня примет, тот не Меня принимает, но Пославшего Меня.
\vs Mar 9:38 При сем Иоанн сказал: Учитель! мы видели человека, который именем Твоим изгоняет бесов, а не ходит за нами; и запретили ему, потому что не ходит за нами.
\vs Mar 9:39 Иисус сказал: не запрещайте ему, ибо никто, сотворивший чудо именем Моим, не может вскоре злословить Меня.
\vs Mar 9:40 Ибо кто не против вас, тот за вас.
\vs Mar 9:41 И кто напоит вас чашею воды во имя Мое, потому что вы Христовы, истинно говорю вам, не потеряет награды своей.
\vs Mar 9:42 А кто соблазнит одного из малых сих, верующих в Меня, тому лучше было бы, если бы повесили ему жерновный камень на шею и бросили его в море.
\vs Mar 9:43 И если соблазняет тебя рука твоя, отсеки ее: лучше тебе увечному войти в жизнь, нежели с двумя руками идти в геенну, в огонь неугасимый,
\vs Mar 9:44 где червь их не умирает и огонь не угасает.
\vs Mar 9:45 И если нога твоя соблазняет тебя, отсеки ее: лучше тебе войти в жизнь хромому, нежели с двумя ногами быть ввержену в геенну, в огонь неугасимый,
\vs Mar 9:46 где червь их не умирает и огонь не угасает.
\vs Mar 9:47 И если глаз твой соблазняет тебя, вырви его: лучше тебе с одним глазом войти в Царствие Божие, нежели с двумя глазами быть ввержену в геенну огненную,
\vs Mar 9:48 где червь их не умирает и огонь не угасает.
\vs Mar 9:49 Ибо всякий огнем осолится, и всякая жертва солью осолится.
\vs Mar 9:50 Соль~--- добрая \bibemph{вещь}; но ежели соль не солона будет, чем вы ее поправите? Имейте в себе соль, и мир имейте между собою.
\vs Mar 10:1 Отправившись оттуда, приходит в пределы Иудейские за Иорданскою стороною. Опять собирается к Нему народ, и, по обычаю Своему, Он опять учил их.
\vs Mar 10:2 Подошли фарисеи и спросили, искушая Его: позволительно ли разводиться мужу с женою?
\vs Mar 10:3 Он сказал им в ответ: что заповедал вам Моисей?
\vs Mar 10:4 Они сказали: Моисей позволил писать разводное письмо и разводиться.
\vs Mar 10:5 Иисус сказал им в ответ: по жестокосердию вашему он написал вам сию заповедь.
\vs Mar 10:6 В начале же создания, Бог мужчину и женщину сотворил их.
\vs Mar 10:7 Посему оставит человек отца своего и мать
\vs Mar 10:8 и прилепится к жене своей, и будут два одною плотью; так что они уже не двое, но одна плоть.
\vs Mar 10:9 Итак, что Бог сочетал, того человек да не разлучает.
\vs Mar 10:10 В доме ученики Его опять спросили Его о том же.
\vs Mar 10:11 Он сказал им: кто разведется с женою своею и женится на другой, тот прелюбодействует от нее;
\vs Mar 10:12 и если жена разведется с мужем своим и выйдет за другого, прелюбодействует.
\rsbpar\vs Mar 10:13 Приносили к Нему детей, чтобы Он прикоснулся к ним; ученики же не допускали приносящих.
\vs Mar 10:14 Увидев \bibemph{то}, Иисус вознегодовал и сказал им: пустите детей приходить ко Мне и не препятствуйте им, ибо таковых есть Царствие Божие.
\vs Mar 10:15 Истинно говорю вам: кто не примет Царствия Божия, как дитя, тот не войдет в него.
\vs Mar 10:16 И, обняв их, возложил руки на них и благословил их.
\rsbpar\vs Mar 10:17 Когда выходил Он в путь, подбежал некто, пал пред Ним на колени и спросил Его: Учитель благий! что мне делать, чтобы наследовать жизнь вечную?
\vs Mar 10:18 Иисус сказал ему: что ты называешь Меня благим? Никто не благ, как только один Бог.
\vs Mar 10:19 Знаешь заповеди: не прелюбодействуй, не убивай, не кради, не лжесвидетельствуй, не обижай, почитай отца твоего и мать.
\vs Mar 10:20 Он же сказал Ему в ответ: Учитель! всё это сохранил я от юности моей.
\vs Mar 10:21 Иисус, взглянув на него, полюбил его и сказал ему: одного тебе недостает: пойди, всё, что имеешь, продай и раздай нищим, и будешь иметь сокровище на небесах; и приходи, последуй за Мною, взяв крест.
\vs Mar 10:22 Он же, смутившись от сего слова, отошел с печалью, потому что у него было большое имение.
\vs Mar 10:23 И, посмотрев вокруг, Иисус говорит ученикам Своим: как трудно имеющим богатство войти в Царствие Божие!
\vs Mar 10:24 Ученики ужаснулись от слов Его. Но Иисус опять говорит им в ответ: дети! как трудно надеющимся на богатство войти в Царствие Божие!
\vs Mar 10:25 Удобнее верблюду пройти сквозь игольные уши, нежели богатому войти в Царствие Божие.
\vs Mar 10:26 Они же чрезвычайно изумлялись и говорили между собою: кто же может спастись?
\vs Mar 10:27 Иисус, воззрев на них, говорит: человекам это невозможно, но не Богу, ибо всё возможно Богу.
\rsbpar\vs Mar 10:28 И начал Петр говорить Ему: вот, мы оставили всё и последовали за Тобою.
\vs Mar 10:29 Иисус сказал в ответ: истинно говорю вам: нет никого, кто оставил бы дом, или братьев, или сестер, или отца, или мать, или жену, или детей, или з\acc{е}мли, ради Меня и Евангелия,
\vs Mar 10:30 и не получил бы ныне, во время сие, среди гонений, во сто крат более домов, и братьев, и сестер, и отцов, и матерей, и детей, и земель, а в веке грядущем жизни вечной.
\vs Mar 10:31 Многие же будут первые последними, и последние первыми.
\rsbpar\vs Mar 10:32 Когда были они на пути, восходя в Иерусалим, Иисус шел впереди их, а они ужасались и, следуя за Ним, были в страхе. Подозвав двенадцать, Он опять начал им говорить о том, чт\acc{о} будет с Ним:
\vs Mar 10:33 вот, мы восходим в Иерусалим, и Сын Человеческий предан будет первосвященникам и книжникам, и осудят Его на смерть, и предадут Его язычникам,
\vs Mar 10:34 и поругаются над Ним, и будут бить Его, и оплюют Его, и убьют Его; и в третий день воскреснет.
\vs Mar 10:35 \bibemph{Тогда} подошли к Нему сыновья Зеведеевы Иаков и Иоанн и сказали: Учитель! мы желаем, чтобы Ты сделал нам, о чем попросим.
\vs Mar 10:36 Он сказал им: что хотите, чтобы Я сделал вам?
\vs Mar 10:37 Они сказали Ему: дай нам сесть у Тебя, одному по правую сторону, а другому по левую в славе Твоей.
\vs Mar 10:38 Но Иисус сказал им: не знаете, чего просите. Можете ли пить чашу, которую Я пью, и креститься крещением, которым Я крещусь?
\vs Mar 10:39 Они отвечали: можем. Иисус же сказал им: чашу, которую Я пью, будете пить, и крещением, которым Я крещусь, будете креститься;
\vs Mar 10:40 а дать сесть у Меня по правую сторону и по левую~--- не от Меня \bibemph{зависит}, но кому уготовано.
\vs Mar 10:41 И, услышав, десять начали негодовать на Иакова и Иоанна.
\vs Mar 10:42 Иисус же, подозвав их, сказал им: вы знаете, что почитающиеся князьями народов господствуют над ними, и вельможи их властвуют ими.
\vs Mar 10:43 Но между вами да не будет так: а кто хочет быть б\acc{о}льшим между вами, да будем вам слугою;
\vs Mar 10:44 и кто хочет быть первым между вами, да будет всем рабом.
\vs Mar 10:45 Ибо и Сын Человеческий не для того пришел, чтобы Ему служили, но чтобы послужить и отдать душу Свою для искупления многих.
\rsbpar\vs Mar 10:46 Приходят в Иерихон. И когда выходил Он из Иерихона с учениками Своими и множеством народа, Вартимей, сын Тимеев, слепой сидел у дороги, прося \bibemph{милостыни}.
\vs Mar 10:47 Услышав, что это Иисус Назорей, он начал кричать и говорить: Иисус, Сын Давидов! помилуй меня.
\vs Mar 10:48 Многие заставляли его молчать; но он еще более стал кричать: Сын Давидов! помилуй меня.
\vs Mar 10:49 Иисус остановился и велел его позвать. Зовут слепого и говорят ему: не бойся, вставай, зовет тебя.
\vs Mar 10:50 Он сбросил с себя верхнюю одежду, встал и пришел к Иисусу.
\vs Mar 10:51 Отвечая ему, Иисус спросил: чего ты хочешь от Меня? Слепой сказал Ему: Учитель! чтобы мне прозреть.
\vs Mar 10:52 Иисус сказал ему: иди, вера твоя спасла тебя. И он тотчас прозрел и пошел за Иисусом по дороге.
\vs Mar 11:1 Когда приблизились к Иерусалиму, к Виффагии и Вифании, к горе Елеонской, \bibemph{Иисус} посылает двух из учеников Своих
\vs Mar 11:2 и говорит им: пойдите в селение, которое прямо перед вами; входя в него, тотчас найдете привязанного молодого осла, на которого никто из людей не садился; отвязав его, приведите.
\vs Mar 11:3 И если кто скажет вам: что вы это делаете?~--- отвечайте, что он надобен Господу; и тотчас пошлет его сюда.
\vs Mar 11:4 Они пошли, и нашли молодого осла, привязанного у ворот на улице, и отвязали его.
\vs Mar 11:5 И некоторые из стоявших там говорили им: что делаете? \bibemph{зачем} отвязываете осленка?
\vs Mar 11:6 Они отвечали им, к\acc{а}к повелел Иисус; и те отпустили их.
\vs Mar 11:7 И привели осленка к Иисусу, и возложили на него одежды свои; \bibemph{Иисус} сел на него.
\vs Mar 11:8 Многие же постилали одежды свои по дороге; а другие резали ветви с дерев и постилали по дороге.
\vs Mar 11:9 И предшествовавшие и сопровождавшие восклицали: осанна! благословен Грядущий во имя Господне!
\vs Mar 11:10 благословенно грядущее во имя Господа царство отца нашего Давида! осанна в вышних!
\rsbpar\vs Mar 11:11 И вошел Иисус в Иерусалим и в храм; и, осмотрев всё, как время уже было позднее, вышел в Вифанию с двенадцатью.
\rsbpar\vs Mar 11:12 На другой день, когда они вышли из Вифании, Он взалкал;
\vs Mar 11:13 и, увидев издалека смоковницу, покрытую листьями, пошел, не найдет ли чего на ней; но, придя к ней, ничего не нашел, кроме листьев, ибо еще не время было \bibemph{собирания} смокв.
\vs Mar 11:14 И сказал ей Иисус: отныне да не вкушает никто от тебя плода вовек! И слышали т\acc{о} ученики Его.
\vs Mar 11:15 Пришли в Иерусалим. Иисус, войдя в храм, начал выгонять продающих и покупающих в храме; и столы меновщиков и скамьи продающих голубей опрокинул;
\vs Mar 11:16 и не позволял, чтобы кто пронес через храм какую-либо вещь.
\vs Mar 11:17 И учил их, говоря: не написано ли: дом Мой домом молитвы наречется для всех народов? а вы сделали его вертепом разбойников.
\vs Mar 11:18 Услышали \bibemph{это} книжники и первосвященники, и искали, как бы погубить Его, ибо боялись Его, потому что весь народ удивлялся учению Его.
\vs Mar 11:19 Когда же стало поздно, Он вышел вон из города.
\rsbpar\vs Mar 11:20 Поутру, проходя мимо, увидели, что смоковница засохла до корня.
\vs Mar 11:21 И, вспомнив, Петр говорит Ему: Равв\acc{и}! посмотри, смоковница, которую Ты проклял, засохла.
\vs Mar 11:22 Иисус, отвечая, говорит им:
\vs Mar 11:23 имейте веру Божию, ибо истинно говорю вам, если кто скажет горе сей: поднимись и ввергнись в море, и не усомнится в сердце своем, но поверит, что сбудется по словам его,~--- будет ему, что ни скажет.
\vs Mar 11:24 Потому говорю вам: всё, чего ни будете просить в молитве, верьте, что получите,~--- и будет вам.
\vs Mar 11:25 И когда стоите на молитве, прощайте, если чт\acc{о} имеете на кого, дабы и Отец ваш Небесный простил вам согрешения ваши.
\vs Mar 11:26 Если же не прощаете, то и Отец ваш Небесный не простит вам согрешений ваших.
\rsbpar\vs Mar 11:27 Пришли опять в Иерусалим. И когда Он ходил в храме, подошли к Нему первосвященники и книжники, и старейшины
\vs Mar 11:28 и говорили Ему: какою властью Ты это делаешь? и кто Тебе дал власть делать это?
\vs Mar 11:29 Иисус сказал им в ответ: спрошу и Я вас об одном, отвечайте Мне; \bibemph{тогда} и Я скажу вам, какою властью это делаю.
\vs Mar 11:30 Крещение Иоанново с небес было, или от человеков? отвечайте Мне.
\vs Mar 11:31 Они рассуждали между собою: если скажем: с небес,~--- то Он скажет: почему же вы не поверили ему?
\vs Mar 11:32 а сказать: от человеков~--- боялись народа, потому что все полагали, что Иоанн точно был пророк.
\vs Mar 11:33 И сказали в ответ Иисусу: не знаем. Тогда Иисус сказал им в ответ: и Я не скажу вам, какою властью это делаю.
\vs Mar 12:1 И начал говорить им притчами: некоторый человек насадил виноградник и обнес оградою, и выкопал точило, и построил башню, и, отдав его виноградарям, отлучился.
\vs Mar 12:2 И послал в свое время к виноградарям слугу~--- принять от виноградарей плодов из виноградника.
\vs Mar 12:3 Они же, схватив его, били, и отослали ни с чем.
\vs Mar 12:4 Опять послал к ним другого слугу; и тому камнями разбили голову и отпустили его с бесчестьем.
\vs Mar 12:5 И опять иного послал: и того убили; и многих других то били, то убивали.
\vs Mar 12:6 Имея же еще одного сына, любезного ему, напоследок послал и его к ним, говоря: постыдятся сына моего.
\vs Mar 12:7 Но виноградари сказали друг другу: это наследник; пойдем, убьем его, и наследство будет наше.
\vs Mar 12:8 И, схватив его, убили и выбросили вон из виноградника.
\vs Mar 12:9 Что же сделает хозяин виноградника?~--- Придет и предаст смерти виноградарей, и отдаст виноградник другим.
\vs Mar 12:10 Неужели вы не читали сего в Писании: камень, который отвергли строители, тот самый сделался главою угла;
\vs Mar 12:11 это от Господа, и есть дивно в очах наших.
\vs Mar 12:12 И старались схватить Его, но побоялись народа, ибо поняли, что о них сказал притчу; и, оставив Его, отошли.
\rsbpar\vs Mar 12:13 И посылают к Нему некоторых из фарисеев и иродиан, чтобы уловить Его в слове.
\vs Mar 12:14 Они же, придя, говорят Ему: Учитель! мы знаем, что Ты справедлив и не заботишься об угождении кому-либо, ибо не смотришь ни на какое лице, но истинно пути Божию учишь. Позволительно ли давать п\acc{о}дать кесарю или нет? давать ли нам или не давать?
\vs Mar 12:15 Но Он, зная их лицемерие, сказал им: что искушаете Меня? принесите Мне динарий, чтобы Мне видеть его.
\vs Mar 12:16 Они принесли. Тогда говорит им: чье это изображение и надпись? Они сказали Ему: кесаревы.
\vs Mar 12:17 Иисус сказал им в ответ: отдавайте кесарево кесарю, а Божие Богу. И дивились Ему.
\rsbpar\vs Mar 12:18 Потом пришли к Нему саддукеи, которые говорят, что нет воскресения, и спросили Его, говоря:
\vs Mar 12:19 Учитель! Моисей написал нам: если у кого умрет брат и оставит жену, а детей не оставит, то брат его пусть возьмет жену его и восстановит семя брату своему.
\vs Mar 12:20 Было семь братьев: первый взял жену и, умирая, не оставил детей.
\vs Mar 12:21 Взял ее второй и умер, и он не оставил детей; также и третий.
\vs Mar 12:22 Брали ее \bibemph{за себя} семеро и не оставили детей. После всех умерла и жена.
\vs Mar 12:23 Итак, в воскресении, когда воскреснут, которого из них будет она женою? Ибо семеро имели ее женою.
\vs Mar 12:24 Иисус сказал им в ответ: этим ли приводитесь вы в заблуждение, не зная Писаний, ни силы Божией?
\vs Mar 12:25 Ибо, когда из мертвых воскреснут, \bibemph{тогда} не будут ни жениться, ни замуж выходить, но будут, как Ангелы на небесах.
\vs Mar 12:26 А о мертвых, что они воскреснут, разве не читали вы в книге Моисея, как Бог при купине сказал ему: Я Бог Авраама, и Бог Исаака, и Бог Иакова?
\vs Mar 12:27 \bibemph{Бог} не есть Бог мертвых, но Бог живых. Итак, вы весьма заблуждаетесь.
\rsbpar\vs Mar 12:28 Один из книжников, слыша их прения и видя, что \bibemph{Иисус} хорошо им отвечал, подошел и спросил Его: какая первая из всех заповедей?
\vs Mar 12:29 Иисус отвечал ему: первая из всех заповедей: слушай, Израиль! Господь Бог наш есть Господь единый;
\vs Mar 12:30 и возлюби Господа Бога твоего всем сердцем твоим, и всею душею твоею, и всем разумением твоим, и всею крепостию твоею,~--- вот первая заповедь!
\vs Mar 12:31 Вторая подобная ей: возлюби ближнего твоего, как самого себя. Иной большей сих заповеди нет.
\vs Mar 12:32 Книжник сказал Ему: хорошо, Учитель! истину сказал Ты, что один есть Бог и нет иного, кроме Его;
\vs Mar 12:33 и любить Его всем сердцем и всем умом, и всею душею, и всею крепостью, и любить ближнего, как самого себя, есть больше всех всесожжений и жертв.
\vs Mar 12:34 Иисус, видя, что он разумно отвечал, сказал ему: недалеко ты от Царствия Божия. После того никто уже не смел спрашивать Его.
\rsbpar\vs Mar 12:35 Продолжая учить в храме, Иисус говорил: как говорят книжники, что Христос есть Сын Давидов?
\vs Mar 12:36 Ибо сам Давид сказал Духом Святым: сказал Господь Господу моему: седи одесную Меня, доколе положу врагов Твоих в подножие ног Твоих.
\vs Mar 12:37 Итак, сам Давид называет Его Господом: как же Он Сын ему? И множество народа слушало Его с услаждением.
\vs Mar 12:38 И говорил им в учении Своем: остерегайтесь книжников, любящих ходить в длинных одеждах и \bibemph{принимать} приветствия в народных собраниях,
\vs Mar 12:39 сидеть впереди в синагогах и возлежать на первом \bibemph{месте} на пиршествах,~---
\vs Mar 12:40 сии, поядающие домы вдов и напоказ долго молящиеся, примут тягчайшее осуждение.
\rsbpar\vs Mar 12:41 И сел Иисус против сокровищницы и смотрел, как народ кладет деньги в сокровищницу. Многие богатые клали много.
\vs Mar 12:42 Придя же, одна бедная вдова положила две лепты, что составляет кодрант.
\vs Mar 12:43 Подозвав учеников Своих, \bibemph{Иисус} сказал им: истинно говорю вам, что эта бедная вдова положила больше всех, клавших в сокровищницу,
\vs Mar 12:44 ибо все клали от избытка своего, а она от скудости своей положила всё, что имела, всё пропитание свое.
\vs Mar 13:1 И когда выходил Он из храма, говорит Ему один из учеников Его: Учитель! посмотри, какие камни и какие здания!
\vs Mar 13:2 Иисус сказал ему в ответ: видишь сии великие здания? всё это будет разрушено, так что не останется здесь камня на камне.
\vs Mar 13:3 И когда Он сидел на горе Елеонской против храма, спрашивали Его наедине Петр, и Иаков, и Иоанн, и Андрей:
\vs Mar 13:4 скажи нам, когда это будет, и какой признак, когда всё сие должно совершиться?
\vs Mar 13:5 Отвечая им, Иисус начал говорить: берегитесь, чтобы кто не прельстил вас,
\vs Mar 13:6 ибо многие придут под именем Моим и будут говорить, что это Я; и многих прельстят.
\vs Mar 13:7 Когда же услышите о войнах и о военных слухах, не ужасайтесь: ибо надлежит \bibemph{сему} быть,~--- но \bibemph{это} еще не конец.
\vs Mar 13:8 Ибо восстанет народ на народ и царство на царство; и будут землетрясения по местам, и будут глады и смятения. Это~--- начало болезней.
\vs Mar 13:9 Но вы смотр\acc{и}те за собою, ибо вас будут предавать в судилища и бить в синагогах, и перед правителями и царями поставят вас за Меня, для свидетельства перед ними.
\vs Mar 13:10 И во всех народах прежде должно быть проповедано Евангелие.
\vs Mar 13:11 Когда же поведут предавать вас, не заботьтесь наперед, чт\acc{о} вам говорить, и не обдумывайте; но чт\acc{о} дано будет вам в тот час, т\acc{о} и говорите, ибо не вы будете говорить, но Дух Святый.
\vs Mar 13:12 Предаст же брат брата на смерть, и отец~--- детей; и восстанут дети на родителей и умертвят их.
\vs Mar 13:13 И будете ненавидимы всеми за имя Мое; претерпевший же до конца спасется.
\vs Mar 13:14 Когда же увидите мерзость запустения, реченную пророком Даниилом, стоящую, где не должно,~--- читающий да разумеет,~--- тогда находящиеся в Иудее да бегут в горы;
\vs Mar 13:15 а кто на кровле, тот не сходи в дом и не входи взять что-нибудь из дома своего;
\vs Mar 13:16 и кто на поле, не обращайся назад взять одежду свою.
\vs Mar 13:17 Горе беременным и питающим сосцами в те дни.
\vs Mar 13:18 Мол\acc{и}тесь, чтобы не случилось бегство ваше зимою.
\vs Mar 13:19 Ибо в те дни будет такая скорбь, какой не было от начала творения, которое сотворил Бог, даже доныне, и не будет.
\vs Mar 13:20 И если бы Господь не сократил тех дней, то не спаслась бы никакая плоть; но ради избранных, которых Он избрал, сократил те дни.
\vs Mar 13:21 Тогда, если кто вам скажет: вот, здесь Христос, или: вот, там,~--- не верьте.
\vs Mar 13:22 Ибо восстанут лжехристы и лжепророки и дадут знамения и чудеса, чтобы прельстить, если возможно, и избранных.
\vs Mar 13:23 Вы же берегитесь. Вот, Я наперед сказал вам всё.
\vs Mar 13:24 Но в те дни, после скорби той, солнце померкнет, и луна не даст света своего,
\vs Mar 13:25 и звезды спадут с неба, и силы небесные поколеблются.
\vs Mar 13:26 Тогда увидят Сына Человеческого, грядущего на облаках с силою многою и славою.
\vs Mar 13:27 И тогда Он пошлет Ангелов Своих и соберет избранных Своих от четырех ветров, от края земли до края неба.
\vs Mar 13:28 От смоковницы возьмите подобие: когда ветви ее становятся уже мягки и пускают листья, то знаете, что близко лето.
\vs Mar 13:29 Так и когда вы увидите т\acc{о} сбывающимся, знайте, что близко, при дверях.
\vs Mar 13:30 Истинно говорю вам: не прейдет род сей, как всё это будет.
\vs Mar 13:31 Небо и земля прейдут, но слова Мои не прейдут.
\vs Mar 13:32 О дне же том, или часе, никто не знает, ни Ангелы небесные, ни Сын, но только Отец.
\vs Mar 13:33 Смотрите, бодрствуйте, молитесь, ибо не знаете, когда наступит это время.
\vs Mar 13:34 Подобно как бы кто, отходя в путь и оставляя дом свой, дал слугам своим власть и каждому свое дело, и приказал привратнику бодрствовать.
\vs Mar 13:35 Итак бодрствуйте, ибо не знаете, когда придет хозяин дома: вечером, или в полночь, или в пение петухов, или поутру;
\vs Mar 13:36 чтобы, придя внезапно, не нашел вас спящими.
\vs Mar 13:37 А чт\acc{о} вам говорю, говорю всем: бодрствуйте.
\vs Mar 14:1 Через два дня \bibemph{надлежало} быть \bibemph{празднику} Пасхи и опресноков. И искали первосвященники и книжники, как бы взять Его хитростью и убить;
\vs Mar 14:2 но говорили: \bibemph{только} не в праздник, чтобы не произошло возмущения в народе.
\rsbpar\vs Mar 14:3 И когда был Он в Вифании, в доме Симона прокаженного, и возлежал,~--- пришла женщина с алавастровым сосудом мира из нарда чистого, драгоценного и, разбив сосуд, возлила Ему на голову.
\vs Mar 14:4 Некоторые же вознегодовали и говорили между собою: к чему сия трата мира?
\vs Mar 14:5 Ибо можно было бы продать его более нежели за триста динариев и раздать нищим. И роптали на нее.
\vs Mar 14:6 Но Иисус сказал: оставьте ее; чт\acc{о} ее смущаете? Она доброе дело сделала для Меня.
\vs Mar 14:7 Ибо нищих всегда имеете с собою и, когда захотите, можете им благотворить; а Меня не всегда имеете.
\vs Mar 14:8 Она сделала, чт\acc{о} могла: предварила помазать тело Мое к погребению.
\vs Mar 14:9 Истинно говорю вам: где ни будет проповедано Евангелие сие в целом мире, сказано будет, в память ее, и о том, чт\acc{о} она сделала.
\rsbpar\vs Mar 14:10 И пошел Иуда Искариот, один из двенадцати, к первосвященникам, чтобы предать Его им.
\vs Mar 14:11 Они же, услышав, обрадовались, и обещали дать ему сребреники. И он искал, как бы в удобное время предать Его.
\rsbpar\vs Mar 14:12 В первый день опресноков, когда заколали пасхального \bibemph{агнца}, говорят Ему ученики Его: где хочешь есть пасху? мы пойдем и приготовим.
\vs Mar 14:13 И посылает двух из учеников Своих и говорит им: пойдите в город; и встретится вам человек, несущий кувшин воды; последуйте за ним
\vs Mar 14:14 и куда он войдет, скажите хозяину дома того: Учитель говорит: где комната, в которой бы Мне есть пасху с учениками Моими?
\vs Mar 14:15 И он покажет вам горницу большую, устланную, готовую: там приготовьте нам.
\vs Mar 14:16 И пошли ученики Его, и пришли в город, и нашли, как сказал им; и приготовили пасху.
\vs Mar 14:17 Когда настал вечер, Он приходит с двенадцатью.
\vs Mar 14:18 И, когда они возлежали и ели, Иисус сказал: истинно говорю вам, один из вас, ядущий со Мною, предаст Меня.
\vs Mar 14:19 Они опечалились и стали говорить Ему, один за другим: не я ли? и другой: не я ли?
\vs Mar 14:20 Он же сказал им в ответ: один из двенадцати, обмакивающий со Мною в блюдо.
\vs Mar 14:21 Впрочем Сын Человеческий идет, как писано о Нем; но горе тому человеку, которым Сын Человеческий предается: лучше было бы тому человеку не родиться.
\rsbpar\vs Mar 14:22 И когда они ели, Иисус, взяв хлеб, благословил, преломил, дал им и сказал: приимите, ядите; сие есть Тело Мое.
\vs Mar 14:23 И, взяв чашу, благодарив, подал им: и пили из нее все.
\vs Mar 14:24 И сказал им: сие есть Кровь Моя Нового Завета, за многих изливаемая.
\vs Mar 14:25 Истинно говорю вам: Я уже не буду пить от плода виноградного до того дня, когда буду пить новое вино в Царствии Божием.
\rsbpar\vs Mar 14:26 И, воспев, пошли на гору Елеонскую.
\vs Mar 14:27 И говорит им Иисус: все вы соблазнитесь о Мне в эту ночь; ибо написано: поражу пастыря, и рассеются овцы.
\vs Mar 14:28 По воскресении же Моем, Я предварю вас в Галилее.
\vs Mar 14:29 Петр сказал Ему: если и все соблазнятся, но не я.
\vs Mar 14:30 И говорит ему Иисус: истинно говорю тебе, что ты ныне, в эту ночь, прежде нежели дважды пропоет петух, трижды отречешься от Меня.
\vs Mar 14:31 Но он еще с б\acc{о}льшим усилием говорил: хотя бы мне надлежало и умереть с Тобою, не отрекусь от Тебя. Т\acc{о} же и все говорили.
\rsbpar\vs Mar 14:32 Пришли в селение, называемое Гефсимания; и Он сказал ученикам Своим: посидите здесь, пока Я помолюсь.
\vs Mar 14:33 И взял с Собою Петра, Иакова и Иоанна; и начал ужасаться и тосковать.
\vs Mar 14:34 И сказал им: душа Моя скорбит смертельно; побудьте здесь и бодрствуйте.
\vs Mar 14:35 И, отойдя немного, пал на землю и молился, чтобы, если возможно, миновал Его час сей;
\vs Mar 14:36 и говорил: Авва Отче! всё возможно Тебе; пронеси чашу сию мимо Меня; но не чего Я хочу, а чего Ты.
\vs Mar 14:37 Возвращается и находит их спящими, и говорит Петру: Симон! ты спишь? не мог ты бодрствовать один час?
\vs Mar 14:38 Бодрствуйте и молитесь, чтобы не впасть в искушение: дух бодр, плоть же немощна.
\vs Mar 14:39 И, опять отойдя, молился, сказав то же слово.
\vs Mar 14:40 И, возвратившись, опять нашел их спящими, ибо глаза у них отяжелели, и они не знали, чт\acc{о} Ему отвечать.
\vs Mar 14:41 И приходит в третий раз и говорит им: вы всё еще спите и почиваете? Кончено, пришел час: вот, предается Сын Человеческий в руки грешников.
\vs Mar 14:42 Встаньте, пойдем; вот, приблизился предающий Меня.
\rsbpar\vs Mar 14:43 И тотчас, как Он еще говорил, приходит Иуда, один из двенадцати, и с ним множество народа с мечами и кольями, от первосвященников и книжников и старейшин.
\vs Mar 14:44 Предающий же Его дал им знак, сказав: Кого я поцелую, Тот и есть, возьмите Его и ведите осторожно.
\vs Mar 14:45 И, придя, тотчас подошел к Нему и говорит: Равв\acc{и}! Равв\acc{и}! и поцеловал Его.
\vs Mar 14:46 А они возложили на Него руки свои и взяли Его.
\vs Mar 14:47 Один же из стоявших тут извлек меч, ударил раба первосвященникова и отсек ему ухо.
\vs Mar 14:48 Тогда Иисус сказал им: как будто на разбойника вышли вы с мечами и кольями, чтобы взять Меня.
\vs Mar 14:49 Каждый день бывал Я с вами в храме и учил, и вы не брали Меня. Но да сбудутся Писания.
\vs Mar 14:50 Тогда, оставив Его, все бежали.
\vs Mar 14:51 Один юноша, завернувшись по нагому телу в покрывало, следовал за Ним; и воины схватили его.
\vs Mar 14:52 Но он, оставив покрывало, нагой убежал от них.
\rsbpar\vs Mar 14:53 И привели Иисуса к первосвященнику; и собрались к нему все первосвященники и старейшины и книжники.
\vs Mar 14:54 Петр издали следовал за Ним, даже внутрь двора первосвященникова; и сидел со служителями, и грелся у огня.
\vs Mar 14:55 Первосвященники же и весь синедрион искали свидетельства на Иисуса, чтобы предать Его смерти; и не находили.
\vs Mar 14:56 Ибо многие лжесвидетельствовали на Него, но свидетельства сии не были достаточны.
\vs Mar 14:57 И некоторые, встав, лжесвидетельствовали против Него и говорили:
\vs Mar 14:58 мы слышали, как Он говорил: Я разрушу храм сей рукотворенный, и через три дня воздвигну другой, нерукотворенный.
\vs Mar 14:59 Но и такое свидетельство их не было достаточно.
\vs Mar 14:60 Тогда первосвященник стал посреди и спросил Иисуса: чт\acc{о} Ты ничего не отвечаешь? чт\acc{о} они против Тебя свидетельствуют?
\vs Mar 14:61 Но Он молчал и не отвечал ничего. Опять первосвященник спросил Его и сказал Ему: Ты ли Христос, Сын Благословенного?
\vs Mar 14:62 Иисус сказал: Я; и вы \acc{у}зрите Сына Человеческого, сидящего одесную силы и грядущего на облаках небесных.
\vs Mar 14:63 Тогда первосвященник, разодрав одежды свои, сказал: на что еще нам свидетелей?
\vs Mar 14:64 Вы слышали богохульство; как вам кажется? Они же все признали Его повинным смерти.
\vs Mar 14:65 И некоторые начали плевать на Него и, закрывая Ему лице, ударять Его и говорить Ему: прореки. И слуги били Его по ланитам.
\rsbpar\vs Mar 14:66 Когда Петр был на дворе внизу, пришла одна из служанок первосвященника
\vs Mar 14:67 и, увидев Петра греющегося и всмотревшись в него, сказала: и ты был с Иисусом Назарянином.
\vs Mar 14:68 Но он отрекся, сказав: не знаю и не понимаю, что ты говоришь. И вышел вон на передний двор; и запел петух.
\vs Mar 14:69 Служанка, увидев его опять, начала говорить стоявшим тут: этот из них.
\vs Mar 14:70 Он опять отрекся. Спустя немного, стоявшие тут опять стали говорить Петру: точно ты из них; ибо ты Галилеянин, и наречие твое сходно.
\vs Mar 14:71 Он же начал клясться и божиться: не знаю Человека Сего, о Котором говорите.
\vs Mar 14:72 Тогда петух запел во второй раз. И вспомнил Петр слово, сказанное ему Иисусом: прежде нежели петух пропоет дважды, трижды отречешься от Меня; и начал плакать.
\vs Mar 15:1 Немедленно поутру первосвященники со старейшинами и книжниками и весь синедрион составили совещание и, связав Иисуса, отвели и предали Пилату.
\vs Mar 15:2 Пилат спросил Его: Ты Царь Иудейский? Он же сказал ему в ответ: ты говоришь.
\vs Mar 15:3 И первосвященники обвиняли Его во многом.
\vs Mar 15:4 Пилат же опять спросил Его: Ты ничего не отвечаешь? видишь, как много против Тебя обвинений.
\vs Mar 15:5 Но Иисус и на это ничего не отвечал, так что Пилат дивился.
\vs Mar 15:6 На всякий же праздник отпускал он им одного узника, о котором просили.
\vs Mar 15:7 Тогда был в узах \bibemph{некто}, по имени Варавва, со своими сообщниками, которые во время мятежа сделали убийство.
\vs Mar 15:8 И народ начал кричать и просить \bibemph{Пилата} о том, чт\acc{о} он всегда делал для них.
\vs Mar 15:9 Он сказал им в ответ: хотите ли, отпущу вам Царя Иудейского?
\vs Mar 15:10 Ибо знал, что первосвященники предали Его из зависти.
\vs Mar 15:11 Но первосвященники возбудили народ \bibemph{просить}, чтобы отпустил им лучше Варавву.
\vs Mar 15:12 Пилат, отвечая, опять сказал им: что же хотите, чтобы я сделал с Тем, Которого вы называете Царем Иудейским?
\vs Mar 15:13 Они опять закричали: распни Его.
\vs Mar 15:14 Пилат сказал им: какое же зло сделал Он? Но они еще сильнее закричали: распни Его.
\vs Mar 15:15 Тогда Пилат, желая сделать угодное народу, отпустил им Варавву, а Иисуса, бив, предал на распятие.
\rsbpar\vs Mar 15:16 А воины отвели Его внутрь двора, то есть в преторию, и собрали весь полк,
\vs Mar 15:17 и одели Его в багряницу, и, сплетши терновый венец, возложили на Него;
\vs Mar 15:18 и начали приветствовать Его: радуйся, Царь Иудейский!
\vs Mar 15:19 И били Его по голове тростью, и плевали на Него, и, становясь на колени, кланялись Ему.
\rsbpar\vs Mar 15:20 Когда же насмеялись над Ним, сняли с Него багряницу, одели Его в собственные одежды Его и повели Его, чтобы распять Его.
\vs Mar 15:21 И заставили проходящего некоего Киринеянина Симона, отца Александрова и Руфова, идущего с поля, нести крест Его.
\vs Mar 15:22 И привели Его на место Голгофу, чт\acc{о} значит: Лобное место.
\vs Mar 15:23 И давали Ему пить вино со смирною; но Он не принял.
\vs Mar 15:24 Распявшие Его делили одежды Его, бросая жребий, кому чт\acc{о} взять.
\vs Mar 15:25 Был час третий, и распяли Его.
\vs Mar 15:26 И была надпись вины Его: Царь Иудейский.
\vs Mar 15:27 С Ним распяли двух разбойников, одного по правую, а другого по левую \bibemph{сторону} Его.
\vs Mar 15:28 И сбылось слово Писания: и к злодеям причтен.
\vs Mar 15:29 Проходящие злословили Его, кивая головами своими и говоря: э! разрушающий храм, и в три дня созидающий!
\vs Mar 15:30 спаси Себя Самого и сойди со креста.
\vs Mar 15:31 Подобно и первосвященники с книжниками, насмехаясь, говорили друг другу: других спасал, а Себя не может спасти.
\vs Mar 15:32 Христос, Царь Израилев, пусть сойдет теперь с креста, чтобы мы видели, и уверуем. И распятые с Ним поносили Его.
\rsbpar\vs Mar 15:33 В шестом же часу настала тьма по всей земле и \bibemph{продолжалась} до часа девятого.
\vs Mar 15:34 В девятом часу возопил Иисус громким голосом: Эло\acc{и}! Эло\acc{и}! ламм\acc{а} савахфан\acc{и}?~--- что значит: Боже Мой! Боже Мой! для чего Ты Меня оставил?
\vs Mar 15:35 Некоторые из стоявших тут, услышав, говорили: вот, Илию зовет.
\vs Mar 15:36 А один побежал, наполнил губку уксусом и, наложив на трость, давал Ему пить, говоря: постойте, посмотрим, придет ли Илия снять Его.
\vs Mar 15:37 Иисус же, возгласив громко, испустил дух.
\vs Mar 15:38 И завеса в храме раздралась надвое, сверху донизу.
\vs Mar 15:39 Сотник, стоявший напротив Его, увидев, что Он, т\acc{а}к возгласив, испустил дух, сказал: истинно Человек Сей был Сын Божий.
\vs Mar 15:40 Были \bibemph{тут} и женщины, которые смотрели издали: между ними была и Мария Магдалина, и Мария, мать Иакова меньшего и Иосии, и Саломия,
\vs Mar 15:41 которые и тогда, как Он был в Галилее, следовали за Ним и служили Ему, и другие многие, вместе с Ним пришедшие в Иерусалим.
\rsbpar\vs Mar 15:42 И как уже настал вечер,~--- потому что была пятница, то есть \bibemph{день} перед субботою,~---
\vs Mar 15:43 пришел Иосиф из Аримафеи, знаменитый член совета, который и сам ожидал Царствия Божия, осмелился войти к Пилату, и просил тела Иисусова.
\vs Mar 15:44 Пилат удивился, что Он уже умер, и, призвав сотника, спросил его, давно ли умер?
\vs Mar 15:45 И, узнав от сотника, отдал тело Иосифу.
\vs Mar 15:46 Он, купив плащаницу и сняв Его, обвил плащаницею, и положил Его во гробе, который был высечен в скале, и привалил камень к двери гроба.
\vs Mar 15:47 Мария же Магдалина и Мария Иосиева смотрели, где Его полагали.
\vs Mar 16:1 По прошествии субботы Мария Магдалина и Мария Иаковлева и Саломия купили ароматы, чтобы идти помазать Его.
\vs Mar 16:2 И весьма рано, в первый \bibemph{день} недели, приходят ко гробу, при восходе солнца,
\vs Mar 16:3 и говорят между собою: кто отвалит нам камень от двери гроба?
\vs Mar 16:4 И, взглянув, видят, что камень отвален; а он был весьма велик.
\vs Mar 16:5 И, войдя во гроб, увидели юношу, сидящего на правой стороне, облеченного в белую одежду; и ужаснулись.
\vs Mar 16:6 Он же говорит им: не ужасайтесь. Иисуса ищете Назарянина, распятого; Он воскрес, Его нет здесь. Вот место, где Он был положен.
\vs Mar 16:7 Но идите, скажите ученикам Его и Петру, что Он предваряет вас в Галилее; там Его увидите, как Он сказал вам.
\vs Mar 16:8 И, выйдя, побежали от гроба; их объял трепет и ужас, и никому ничего не сказали, потому что боялись.
\rsbpar\vs Mar 16:9 Воскреснув рано в первый \bibemph{день} недели, \bibemph{Иисус} явился сперва Марии Магдалине, из которой изгнал семь бесов.
\vs Mar 16:10 Она пошла и возвестила бывшим с Ним, плачущим и рыдающим;
\vs Mar 16:11 но они, услышав, что Он жив и она видела Его,~--- не поверили.
\rsbpar\vs Mar 16:12 После сего явился в ином образе двум из них на дороге, когда они шли в селение.
\vs Mar 16:13 И те, возвратившись, возвестили прочим; но и им не поверили.
\rsbpar\vs Mar 16:14 Наконец, явился самим одиннадцати, возлежавшим \bibemph{на вечери}, и упрекал их за неверие и жестокосердие, что видевшим Его воскресшего не поверили.
\vs Mar 16:15 И сказал им: идите по всему миру и проповедуйте Евангелие всей твари.
\vs Mar 16:16 Кто будет веровать и креститься, спасен будет; а кто не будет веровать, осужден будет.
\vs Mar 16:17 Уверовавших же будут сопровождать сии знамения: именем Моим будут изгонять бесов; будут говорить новыми языками;
\vs Mar 16:18 будут брать змей; и если чт\acc{о} смертоносное выпьют, не повредит им; возложат руки на больных, и они будут здоровы.
\rsbpar\vs Mar 16:19 И так Господь, после беседования с ними, вознесся на небо и воссел одесную Бога.
\vs Mar 16:20 А они пошли и проповедовали везде, при Господнем содействии и подкреплении слова последующими знамениями. Аминь.

\bibbookdescr{2Ti}{
  inline={Второе Послание к Тимофею\\\LARGE Святого Апостола Павла},
  toc={2-е Тимофею},
  bookmark={2-е Тимофею},
  header={2-е Тимофею},
  %headerleft={},
  %headerright={},
  abbr={2~Тим}
}
\vs 2Ti 1:1 Павел, волею Божиею Апостол Иисуса Христа, по обетованию жизни во Христе Иисусе,
\vs 2Ti 1:2 Тимофею, возлюбленному сыну: благодать, милость, мир от Бога Отца и Христа Иисуса, Господа нашего.
\rsbpar\vs 2Ti 1:3 Благодарю Бога, Которому служу от прародителей с чистою совестью, что непрестанно вспоминаю о тебе в молитвах моих днем и ночью,
\vs 2Ti 1:4 и желаю видеть тебя, вспоминая о слезах твоих, дабы мне исполниться радости,
\vs 2Ti 1:5 приводя на память нелицемерную веру твою, которая прежде обитала в бабке твоей Лоиде и матери твоей Евнике; уверен, что она и в тебе.
\vs 2Ti 1:6 По сей причине напоминаю тебе возгревать дар Божий, который в тебе через мое рукоположение;
\vs 2Ti 1:7 ибо дал нам Бог духа не боязни, но силы и любви и целомудрия.
\vs 2Ti 1:8 Итак, не стыдись свидетельства Господа нашего Иисуса Христа, ни меня, узника Его; но страдай с благовестием Христовым силою Бога,
\vs 2Ti 1:9 спасшего нас и призвавшего званием святым, не по делам нашим, но по Своему изволению и благодати, данной нам во Христе Иисусе прежде вековых времен,
\vs 2Ti 1:10 открывшейся же ныне явлением Спасителя нашего Иисуса Христа, разрушившего смерть и явившего жизнь и нетление через благовестие,
\vs 2Ti 1:11 для которого я поставлен проповедником и Апостолом и учителем язычников.
\vs 2Ti 1:12 По сей причине я и страдаю так; но не стыжусь. Ибо я знаю, в Кого уверовал, и уверен, что Он силен сохранить залог мой на оный день.
\vs 2Ti 1:13 Держись образца здравого учения, которое ты слышал от меня, с верою и любовью во Христе Иисусе.
\vs 2Ti 1:14 Храни добрый залог Духом Святым, живущим в нас.
\rsbpar\vs 2Ti 1:15 Ты знаешь, что все Асийские оставили меня; в числе их Фигелл и Ермоген.
\vs 2Ti 1:16 Да даст Господь милость дому Онисифора за то, что он многократно покоил меня и не стыдился уз моих,
\vs 2Ti 1:17 но, быв в Риме, с великим тщанием искал меня и нашел.
\vs 2Ti 1:18 Да даст ему Господь обрести милость у Господа в оный день; а сколько он служил мне в Ефесе, ты лучше знаешь.
\vs 2Ti 2:1 Итак укрепляйся, сын мой, в благодати Христом Иисусом,
\vs 2Ti 2:2 и что слышал от меня при многих свидетелях, то передай верным людям, которые были бы способны и других научить.
\vs 2Ti 2:3 Итак переноси страдания, как добрый воин Иисуса Христа.
\vs 2Ti 2:4 Никакой воин не связывает себя делами житейскими, чтобы угодить военачальнику.
\vs 2Ti 2:5 Если же кто и подвизается, не увенчивается, если незаконно будет подвизаться.
\vs 2Ti 2:6 Трудящемуся земледельцу первому должно вкусить от плодов.
\vs 2Ti 2:7 Разумей, что я говорю. Да даст тебе Господь разумение во всем.
\rsbpar\vs 2Ti 2:8 Помни Господа Иисуса Христа от семени Давидова, воскресшего из мертвых, по благовествованию моему,
\vs 2Ti 2:9 за которое я страдаю даже до уз, как злодей; но для слова Божия нет уз.
\vs 2Ti 2:10 Посему я все терплю ради избранных, дабы и они получили спасение во Христе Иисусе с вечною славою.
\vs 2Ti 2:11 Верно слово: если мы с Ним умерли, то с Ним и оживем;
\vs 2Ti 2:12 если терпим, то с Ним и царствовать будем; если отречемся, и Он отречется от нас;
\vs 2Ti 2:13 если мы неверны, Он пребывает верен, ибо Себя отречься не может.
\rsbpar\vs 2Ti 2:14 Сие напоминай, заклиная пред Господом не вступать в словопрения, что нимало не служит к пользе, а к расстройству слушающих.
\vs 2Ti 2:15 Старайся представить себя Богу достойным, делателем неукоризненным, верно преподающим слово истины.
\vs 2Ti 2:16 А непотребного пустословия удаляйся; ибо они еще более будут преуспевать в нечестии,
\vs 2Ti 2:17 и слово их, как рак, будет распространяться. Таковы Именей и Филит,
\vs 2Ti 2:18 которые отступили от истины, говоря, что воскресение уже было, и разрушают в некоторых веру.
\vs 2Ti 2:19 Но твердое основание Божие сто\acc{и}т, имея печать сию: <<познал Господь Своих>>; и: <<да отступит от неправды всякий, исповедующий имя Господа>>.
\vs 2Ti 2:20 А в большом доме есть сосуды не только золотые и серебряные, но и деревянные и глиняные; и одни в почетном, а другие в низком употреблении.
\vs 2Ti 2:21 Итак, кто будет чист от сего, тот будет сосудом в чести, освященным и благопотребным Владыке, годным на всякое доброе дело.
\vs 2Ti 2:22 Юношеских похотей убегай, а держись правды, веры, любви, мира со всеми призывающими Господа от чистого сердца.
\vs 2Ti 2:23 От глупых и невежественных состязаний уклоняйся, зная, что они рождают ссоры;
\vs 2Ti 2:24 рабу же Господа не должно ссориться, но быть приветливым ко всем, учительным, незлобивым,
\vs 2Ti 2:25 с кротостью наставлять противников, не даст ли им Бог покаяния к познанию истины,
\vs 2Ti 2:26 чтобы они освободились от сети диавола, который уловил их в свою волю.
\vs 2Ti 3:1 Знай же, что в последние дни наступят времена тяжкие.
\vs 2Ti 3:2 Ибо люди будут самолюбивы, сребролюбивы, горды, надменны, злоречивы, родителям непокорны, неблагодарны, нечестивы, недружелюбны,
\vs 2Ti 3:3 непримирительны, клеветники, невоздержны, жестоки, не любящие добра,
\vs 2Ti 3:4 предатели, наглы, напыщенны, более сластолюбивы, нежели боголюбивы,
\vs 2Ti 3:5 имеющие вид благочестия, силы же его отрекшиеся. Таковых удаляйся.
\vs 2Ti 3:6 К сим принадлежат те, которые вкрадываются в домы и обольщают женщин, утопающих во грехах, водимых различными похотями,
\vs 2Ti 3:7 всегда учащихся и никогда не могущих дойти до познания истины.
\vs 2Ti 3:8 Как Ианний и Иамврий противились Моисею, так и сии противятся истине, люди, развращенные умом, невежды в вере.
\vs 2Ti 3:9 Но они не много успеют; ибо их безумие обнаружится перед всеми, как и с теми случилось.
\vs 2Ti 3:10 А ты последовал мне в учении, житии, расположении, вере, великодушии, любви, терпении,
\vs 2Ti 3:11 в гонениях, страданиях, постигших меня в Антиохии, Иконии, Листрах; каковые гонения я перенес, и от всех избавил меня Господь.
\vs 2Ti 3:12 Да и все, желающие жить благочестиво во Христе Иисусе, будут гонимы.
\vs 2Ti 3:13 Злые же люди и обманщики будут преуспевать во зле, вводя в заблуждение и заблуждаясь.
\vs 2Ti 3:14 А ты пребывай в том, чему научен и что тебе вверено, зная, кем ты научен.
\vs 2Ti 3:15 Притом же ты из детства знаешь священные писания, которые могут умудрить тебя во спасение верою во Христа Иисуса.
\vs 2Ti 3:16 Все Писание богодухновенно и полезно для научения, для обличения, для исправления, для наставления в праведности,
\vs 2Ti 3:17 да будет совершен Божий человек, ко всякому доброму делу приготовлен.
\vs 2Ti 4:1 Итак заклинаю тебя пред Богом и Господом нашим Иисусом Христом, Который будет судить живых и мертвых в явление Его и Царствие Его:
\vs 2Ti 4:2 проповедуй слово, настой во время и не во время, обличай, запрещай, увещевай со всяким долготерпением и назиданием.
\vs 2Ti 4:3 Ибо будет время, когда здравого учения принимать не будут, но по своим прихотям будут избирать себе учителей, которые льстили бы слуху;
\vs 2Ti 4:4 и от истины отвратят слух и обратятся к басням.
\vs 2Ti 4:5 Но ты будь бдителен во всем, переноси скорби, совершай дело благовестника, исполняй служение твое.
\rsbpar\vs 2Ti 4:6 Ибо я уже становлюсь жертвою, и время моего отшествия настало.
\vs 2Ti 4:7 Подвигом добрым я подвизался, течение совершил, веру сохранил;
\vs 2Ti 4:8 а теперь готовится мне венец правды, который даст мне Господь, праведный Судия, в день оный; и не только мне, но и всем, возлюбившим явление Его.
\rsbpar\vs 2Ti 4:9 Постарайся прийти ко мне скоро.
\vs 2Ti 4:10 Ибо Димас оставил меня, возлюбив нынешний век, и пошел в Фессалонику, Крискент в Галатию, Тит в Далматию; один Лука со мною.
\vs 2Ti 4:11 Марка возьми и приведи с собою, ибо он мне нужен для служения.
\vs 2Ti 4:12 Тихика я послал в Ефес.
\vs 2Ti 4:13 Когда пойдешь, принеси фелонь, который я оставил в Троаде у Карпа, и книги, особенно кожаные.
\vs 2Ti 4:14 Александр медник много сделал мне зла. Да воздаст ему Господь по делам его!
\vs 2Ti 4:15 Берегись его и ты, ибо он сильно противился нашим словам.
\rsbpar\vs 2Ti 4:16 При первом моем ответе никого не было со мною, но все меня оставили. Да не вменится им!
\vs 2Ti 4:17 Господь же предстал мне и укрепил меня, дабы через меня утвердилось благовестие и услышали все язычники; и я избавился из львиных челюстей.
\vs 2Ti 4:18 И избавит меня Господь от всякого злого дела и сохранит для Своего Небесного Царства, Ему слава во веки веков. Аминь.
\rsbpar\vs 2Ti 4:19 Приветствуй Прискиллу и Акилу и дом Онисифоров.
\vs 2Ti 4:20 Ераст остался в Коринфе; Трофима же я оставил больного в Милите.
\vs 2Ti 4:21 Постарайся прийти до зимы. Приветствуют тебя Еввул, и Пуд, и Лин, и Клавдия, и все братия.
\rsbpar\vs 2Ti 4:22 Господь Иисус Христос со духом твоим. Благодать с вами. Аминь.

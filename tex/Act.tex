\bibbookdescr{Act}{
  inline={Деяния\\\LARGE святых Апостолов},
  toc={Деяния},
  bookmark={Деяния},
  header={Деяния},
  %headerleft={},
  %headerright={},
  abbr={Деян}
}
\vs Act 1:1 Первую книгу написал я \bibemph{к тебе}, Феофил, о всем, что Иисус делал и чему учил от начала
\vs Act 1:2 до того дня, в который Он вознесся, дав Святым Духом повеления Апостолам, которых Он избрал,
\vs Act 1:3 которым и явил Себя живым, по страдании Своем, со многими верными доказательствами, в продолжение сорока дней являясь им и говоря о Царствии Божием.
\rsbpar\vs Act 1:4 И, собрав их, Он повелел им: не отлучайтесь из Иерусалима, но ждите обещанного от Отца, о чем вы слышали от Меня,
\vs Act 1:5 ибо Иоанн крестил водою, а вы, через несколько дней после сего, будете крещены Духом Святым.
\vs Act 1:6 Посему они, сойдясь, спрашивали Его, говоря: не в сие ли время, Господи, восстановляешь Ты царство Израилю?
\vs Act 1:7 Он же сказал им: не ваше дело знать времена или сроки, которые Отец положил в Своей власти,
\vs Act 1:8 но вы примете силу, когда сойдет на вас Дух Святый; и будете Мне свидетелями в Иерусалиме и во всей Иудее и Самарии и даже до края земли.
\vs Act 1:9 Сказав сие, Он поднялся в глазах их, и облако взяло Его из вида их.
\vs Act 1:10 И когда они смотрели на небо, во время восхождения Его, вдруг предстали им два мужа в белой одежде
\vs Act 1:11 и сказали: мужи Галилейские! что вы стоите и смотрите на небо? Сей Иисус, вознесшийся от вас на небо, придет таким же образом, как вы видели Его восходящим на небо.
\rsbpar\vs Act 1:12 Тогда они возвратились в Иерусалим с горы, называемой Елеон, которая находится близ Иерусалима, в расстоянии субботнего пути.
\vs Act 1:13 И, придя, взошли в горницу, где и пребывали, Петр и Иаков, Иоанн и Андрей, Филипп и Фома, Варфоломей и Матфей, Иаков Алфеев и Симон Зилот, и Иуда, \bibemph{брат} Иакова.
\vs Act 1:14 Все они единодушно пребывали в молитве и молении, с \bibemph{некоторыми} женами и Мариею, Материю Иисуса, и с братьями Его.
\rsbpar\vs Act 1:15 И в те дни Петр, став посреди учеников, сказал
\vs Act 1:16 (было же собрание человек около ста двадцати): мужи братия! Надлежало исполниться тому, что в Писании предрек Дух Святый устами Давида об Иуде, бывшем вожде тех, которые взяли Иисуса;
\vs Act 1:17 он был сопричислен к нам и получил жребий служения сего;
\vs Act 1:18 но приобрел землю неправедною мздою, и когда низринулся, расселось чрево его, и выпали все внутренности его;
\vs Act 1:19 и это сделалось известно всем жителям Иерусалима, так что земля та на отечественном их наречии названа Акелдам\acc{а}, то есть земля крови.
\vs Act 1:20 В книге же Псалмов написано: да будет двор его пуст, и да не будет живущего в нем; и: достоинство его да приимет другой.
\vs Act 1:21 Итак надобно, чтобы один из тех, которые находились с нами во всё время, когда пребывал и обращался с нами Господь Иисус,
\vs Act 1:22 начиная от крещения Иоаннова до того дня, в который Он вознесся от нас, был вместе с нами свидетелем воскресения Его.
\vs Act 1:23 И поставили двоих: Иосифа, называемого Варсавою, который прозван Иустом, и Матфия;
\vs Act 1:24 и помолились и сказали: Ты, Господи, Сердцеведец всех, покажи из сих двоих одного, которого Ты избрал
\vs Act 1:25 принять жребий сего служения и Апостольства, от которого отпал Иуда, чтобы идти в свое место.
\vs Act 1:26 И бросили о них жребий, и выпал жребий Матфию, и он сопричислен к одиннадцати Апостолам.
\vs Act 2:1 При наступлении дня Пятидесятницы все они были единодушно вместе.
\vs Act 2:2 И внезапно сделался шум с неба, как бы от несущегося сильного ветра, и наполнил весь дом, где они находились.
\vs Act 2:3 И явились им разделяющиеся языки, как бы огненные, и почили по одному на каждом из них.
\vs Act 2:4 И исполнились все Духа Святаго, и начали говорить на иных языках, как Дух давал им провещевать.
\rsbpar\vs Act 2:5 В Иерусалиме же находились Иудеи, люди набожные, из всякого народа под небом.
\vs Act 2:6 Когда сделался этот шум, собрался народ, и пришел в смятение, ибо каждый слышал их говорящих его наречием.
\vs Act 2:7 И все изумлялись и дивились, говоря между собою: сии говорящие не все ли Галилеяне?
\vs Act 2:8 Как же мы слышим каждый собственное наречие, в котором родились.
\vs Act 2:9 Парфяне, и Мидяне, и Еламиты, и жители Месопотамии, Иудеи и Каппадокии, Понта и Асии,
\vs Act 2:10 Фригии и Памфилии, Египта и частей Ливии, прилежащих к Киринее, и пришедшие из Рима, Иудеи и прозелиты\fns{Обращенные из язычников.},
\vs Act 2:11 критяне и аравитяне, слышим их нашими языками говорящих о великих \bibemph{делах} Божиих?
\vs Act 2:12 И изумлялись все и, недоумевая, говорили друг другу: что это значит?
\vs Act 2:13 А иные, насмехаясь, говорили: они напились сладкого вина.
\rsbpar\vs Act 2:14 Петр же, став с одиннадцатью, возвысил голос свой и возгласил им: мужи Иудейские, и все живущие в Иерусалиме! сие да будет вам известно, и внимайте словам моим:
\vs Act 2:15 они не пьяны, как вы думаете, ибо теперь третий час дня;
\vs Act 2:16 но это есть предреченное пророком Иоилем:
\vs Act 2:17 И будет в последние дни, говорит Бог, излию от Духа Моего на всякую плоть, и будут пророчествовать сыны ваши и дочери ваши; и юноши ваши будут видеть видения, и старцы ваши сновидениями вразумляемы будут.
\vs Act 2:18 И на рабов Моих и на рабынь Моих в те дни излию от Духа Моего, и будут пророчествовать.
\vs Act 2:19 И покажу чудеса на небе вверху и знамения на земле внизу, кровь и огонь и курение дыма.
\vs Act 2:20 Солнце превратится во тьму, и луна~--- в кровь, прежде нежели наступит день Господень, великий и славный.
\vs Act 2:21 И будет: всякий, кто призовет имя Господне, спасется.
\vs Act 2:22 Мужи Израильские! выслушайте слова сии: Иисуса Назорея, Мужа, засвидетельствованного вам от Бога силами и чудесами и знамениями, которые Бог сотворил через Него среди вас, как и сами знаете,
\vs Act 2:23 Сего, по определенному совету и предведению Божию преданного, вы взяли и, пригвоздив руками беззаконных, убили;
\vs Act 2:24 но Бог воскресил Его, расторгнув узы смерти, потому что ей невозможно было удержать Его.
\vs Act 2:25 Ибо Давид говорит о Нем: видел я пред собою Господа всегда, ибо Он одесную меня, дабы я не поколебался.
\vs Act 2:26 Оттого возрадовалось сердце мое и возвеселился язык мой; даже и плоть моя упокоится в уповании,
\vs Act 2:27 ибо Ты не оставишь души моей в аде и не дашь святому Твоему увидеть тления.
\vs Act 2:28 Ты дал мне познать путь жизни, Ты исполнишь меня радостью пред лицем Твоим.
\vs Act 2:29 Мужи братия! да будет позволено с дерзновением сказать вам о праотце Давиде, что он и умер и погребен, и гроб его у нас до сего дня.
\vs Act 2:30 Будучи же пророком и зная, что Бог с клятвою обещал ему от плода чресл его воздвигнуть Христа во плоти и посадить на престоле его,
\vs Act 2:31 он прежде сказал о воскресении Христа, что не оставлена душа Его в аде, и плоть Его не видела тления.
\vs Act 2:32 Сего Иисуса Бог воскресил, чему все мы свидетели.
\vs Act 2:33 Итак Он, быв вознесен десницею Божиею и приняв от Отца обетование Святаго Духа, излил то, что вы ныне видите и слышите.
\vs Act 2:34 Ибо Давид не восшел на небеса; но сам говорит: сказал Господь Господу моему: седи одесную Меня,
\vs Act 2:35 доколе положу врагов Твоих в подножие ног Твоих.
\vs Act 2:36 Итак твердо знай, весь дом Израилев, что Бог соделал Господом и Христом Сего Иисуса, Которого вы распяли.
\rsbpar\vs Act 2:37 Услышав это, они умилились сердцем и сказали Петру и прочим Апостолам: что нам делать, мужи братия?
\vs Act 2:38 Петр же сказал им: покайтесь, и да крестится каждый из вас во имя Иисуса Христа для прощения грехов; и пол\acc{у}чите дар Святаго Духа.
\vs Act 2:39 Ибо вам принадлежит обетование и детям вашим и всем дальним, кого ни призовет Господь Бог наш.
\vs Act 2:40 И другими многими словами он свидетельствовал и увещевал, говоря: спасайтесь от рода сего развращенного.
\vs Act 2:41 Итак охотно принявшие слово его крестились, и присоединилось в тот день душ около трех тысяч.
\vs Act 2:42 И они постоянно пребывали в учении Апостолов, в общении и преломлении хлеба и в молитвах.
\vs Act 2:43 Был же страх на всякой душе; и много чудес и знамений совершилось через Апостолов в Иерусалиме.
\vs Act 2:44 Все же верующие были вместе и имели всё общее.
\vs Act 2:45 И продавали имения и всякую собственность, и разделяли всем, смотря по нужде каждого.
\vs Act 2:46 И каждый день единодушно пребывали в храме и, преломляя по домам хлеб, принимали пищу в веселии и простоте сердца,
\vs Act 2:47 хваля Бога и находясь в любви у всего народа. Господь же ежедневно прилагал спасаемых к Церкви.
\vs Act 3:1 Петр и Иоанн шли вместе в храм в час молитвы девятый.
\vs Act 3:2 И был человек, хромой от чрева матери его, которого носили и сажали каждый день при дверях храма, называемых Красными, просить милостыни у входящих в храм.
\vs Act 3:3 Он, увидев Петра и Иоанна перед входом в храм, просил у них милостыни.
\vs Act 3:4 Петр с Иоанном, всмотревшись в него, сказали: взгляни на нас.
\vs Act 3:5 И он пристально смотрел на них, надеясь получить от них что-нибудь.
\vs Act 3:6 Но Петр сказал: серебра и золота нет у меня; а что имею, то даю тебе: во имя Иисуса Христа Назорея встань и ходи.
\vs Act 3:7 И, взяв его за правую руку, поднял; и вдруг укрепились его ступни и колени,
\vs Act 3:8 и вскочив, стал, и начал ходить, и вошел с ними в храм, ходя и скача, и хваля Бога.
\vs Act 3:9 И весь народ видел его ходящим и хвалящим Бога;
\vs Act 3:10 и узнали его, что это был тот, который сидел у Красных дверей храма для милостыни; и исполнились ужаса и изумления от случившегося с ним.
\rsbpar\vs Act 3:11 И как исцеленный хромой не отходил от Петра и Иоанна, то весь народ в изумлении сбежался к ним в притвор, называемый Соломонов.
\vs Act 3:12 Увидев это, Петр сказал народу: мужи Израильские! что дивитесь сему, или что смотрите на нас, как будто бы мы своею силою или благочестием сделали то, что он ходит?
\vs Act 3:13 Бог Авраама и Исаака и Иакова, Бог отцов наших, прославил Сына Своего Иисуса, Которого вы предали и от Которого отреклись перед лицом Пилата, когда он полагал освободить Его.
\vs Act 3:14 Но вы от Святого и Праведного отреклись, и просили даровать вам человека убийцу,
\vs Act 3:15 а Начальника жизни убили. Сего Бог воскресил из мертвых, чему мы свидетели.
\vs Act 3:16 И ради веры во имя Его, имя Его укрепило сего, которого вы видите и знаете, и вера, которая от Него, даровала ему исцеление сие перед всеми вами.
\vs Act 3:17 Впрочем я знаю, братия, что вы, как и начальники ваши, сделали это по неведению;
\vs Act 3:18 Бог же, как предвозвестил устами всех Своих пророков пострадать Христу, так и исполнил.
\vs Act 3:19 Итак покайтесь и обратитесь, чтобы загладились грехи ваши,
\vs Act 3:20 да придут времена отрады от лица Господа, и да пошлет Он предназначенного вам Иисуса Христа,
\vs Act 3:21 Которого небо должно было принять до времен совершения всего, что говорил Бог устами всех святых Своих пророков от века.
\vs Act 3:22 Моисей сказал отцам: Господь Бог ваш воздвигнет вам из братьев ваших Пророка, как меня, слушайтесь Его во всем, что Он ни будет говорить вам;
\vs Act 3:23 и будет, что всякая душа, которая не послушает Пророка того, истребится из народа.
\vs Act 3:24 И все пророки, от Самуила и после него, сколько их ни говорили, также предвозвестили дни сии.
\vs Act 3:25 Вы сыны пророков и завета, который завещевал Бог отцам вашим, говоря Аврааму: и в семени твоем благословятся все племена земные.
\vs Act 3:26 Бог, воскресив Сына Своего Иисуса, к вам первым послал Его благословить вас, отвращая каждого от злых дел ваших.
\vs Act 4:1 Когда они говорили к народу, к ним приступили священники и начальники стражи при храме и саддукеи,
\vs Act 4:2 досадуя на то, что они учат народ и проповедуют в Иисусе воскресение из мертвых;
\vs Act 4:3 и наложили на них руки и отдали \bibemph{их} под стражу до утра; ибо уже был вечер.
\vs Act 4:4 Многие же из слушавших слово уверовали; и было число таковых людей около пяти тысяч.
\rsbpar\vs Act 4:5 На другой день собрались в Иерусалим начальники их и старейшины, и книжники,
\vs Act 4:6 и Анна первосвященник, и Каиафа, и Иоанн, и Александр, и прочие из рода первосвященнического;
\vs Act 4:7 и, поставив их посреди, спрашивали: какою силою или каким именем вы сделали это?
\vs Act 4:8 Тогда Петр, исполнившись Духа Святаго, сказал им: начальники народа и старейшины Израильские!
\vs Act 4:9 Если от нас сегодня требуют ответа в благодеянии человеку немощному, как он исцелен,
\vs Act 4:10 то да будет известно всем вам и всему народу Израильскому, что именем Иисуса Христа Назорея, Которого вы распяли, Которого Бог воскресил из мертвых, Им поставлен он перед вами здрав.
\vs Act 4:11 Он есть камень, пренебреженный вами зиждущими, но сделавшийся главою угла, и нет ни в ком ином спасения,
\vs Act 4:12 ибо нет другого имени под небом, данного человекам, которым надлежало бы нам спастись.
\rsbpar\vs Act 4:13 Видя смелость Петра и Иоанна и приметив, что они люди некнижные и простые, они удивлялись, между тем узнавали их, что они были с Иисусом;
\vs Act 4:14 видя же исцеленного человека, стоящего с ними, ничего не могли сказать вопреки.
\vs Act 4:15 И, приказав им выйти вон из синедриона, рассуждали между собою,
\vs Act 4:16 говоря: чт\acc{о} нам делать с этими людьми? Ибо всем, живущим в Иерусалиме, известно, что ими сделано явное чудо, и мы не можем отвергнуть \bibemph{сего};
\vs Act 4:17 но, чтобы более не разгласилось это в народе, с угрозою запретим им, чтобы не говорили об имени сем никому из людей.
\vs Act 4:18 И, призвав их, приказали им отнюдь не говорить и не учить о имени Иисуса.
\vs Act 4:19 Но Петр и Иоанн сказали им в ответ: суд\acc{и}те, справедливо ли пред Богом слушать вас более, нежели Бога?
\vs Act 4:20 Мы не можем не говорить того, что видели и слышали.
\vs Act 4:21 Они же, пригрозив, отпустили их, не находя возможности наказать их, по причине народа; потому что все прославляли Бога за происшедшее.
\vs Act 4:22 Ибо лет более сорока было тому человеку, над которым сделалось сие чудо исцеления.
\rsbpar\vs Act 4:23 Быв отпущены, они пришли к своим и пересказали, что говорили им первосвященники и старейшины.
\vs Act 4:24 Они же, выслушав, единодушно возвысили голос к Богу и сказали: Владыко Боже, сотворивший небо и землю и море и всё, что в них!
\vs Act 4:25 Ты устами отца нашего Давида, раба Твоего, сказал Духом Святым: что мятутся язычники, и народы замышляют тщетное?
\vs Act 4:26 Восстали цари земные, и князи собрались вместе на Господа и на Христа Его.
\vs Act 4:27 Ибо поистине собрались в городе сем на Святаго Сына Твоего Иисуса, помазанного Тобою, Ирод и Понтий Пилат с язычниками и народом Израильским,
\vs Act 4:28 чтобы сделать то, чему быть предопределила рука Твоя и совет Твой.
\vs Act 4:29 И ныне, Господи, воззри на угрозы их, и дай рабам Твоим со всею смелостью говорить слово Твое,
\vs Act 4:30 тогда как Ты простираешь руку Твою на исцеления и на соделание знамений и чудес именем Святаго Сына Твоего Иисуса.
\vs Act 4:31 И, по молитве их, поколебалось место, где они были собраны, и исполнились все Духа Святаго, и говорили слово Божие с дерзновением.
\rsbpar\vs Act 4:32 У множества же уверовавших было одно сердце и одна душа; и никто ничего из имения своего не называл своим, но всё у них было общее.
\vs Act 4:33 Апостолы же с великою силою свидетельствовали о воскресении Господа Иисуса Христа; и великая благодать была на всех их.
\vs Act 4:34 Не было между ними никого нуждающегося; ибо все, которые владели землями или домами, продавая их, приносили цену проданного
\vs Act 4:35 и полагали к ногам Апостолов; и каждому давалось, в чем кто имел нужду.
\vs Act 4:36 Так Иосия, прозванный от Апостолов Варнавою, что значит~--- сын утешения, левит, родом Кипрянин,
\vs Act 4:37 у которого была своя земля, продав ее, принес деньги и положил к ногам Апостолов.
\vs Act 5:1 Некоторый же муж, именем Анания, с женою своею Сапфирою, продав имение,
\vs Act 5:2 утаил из цены, с ведома и жены своей, а некоторую часть принес и положил к ногам Апостолов.
\vs Act 5:3 Но Петр сказал: Анания! Для чего \bibemph{ты допустил} сатане вложить в сердце твое \bibemph{мысль} солгать Духу Святому и утаить из цены земли?
\vs Act 5:4 Чем ты владел, не твое ли было, и приобретенное продажею не в твоей ли власти находилось? Для чего ты положил это в сердце твоем? Ты солгал не человекам, а Богу.
\vs Act 5:5 Услышав сии слова, Анания пал бездыханен; и великий страх объял всех, слышавших это.
\vs Act 5:6 И встав, юноши приготовили его к погребению и, вынеся, похоронили.
\vs Act 5:7 Часа через три после сего пришла и жена его, не зная о случившемся.
\vs Act 5:8 Петр же спросил ее: скажи мне, за столько ли продали вы землю? Она сказала: да, за столько.
\vs Act 5:9 Но Петр сказал ей: что это согласились вы искусить Духа Господня? вот, входят в двери погребавшие мужа твоего; и тебя вынесут.
\vs Act 5:10 Вдруг она упала у ног его и испустила дух. И юноши, войдя, нашли ее мертвою и, вынеся, похоронили подле мужа ее.
\vs Act 5:11 И великий страх объял всю церковь и всех слышавших это.
\rsbpar\vs Act 5:12 Руками же Апостолов совершались в народе многие знамения и чудеса; и все единодушно пребывали в притворе Соломоновом.
\vs Act 5:13 Из посторонних же никто не смел пристать к ним, а народ прославлял их.
\vs Act 5:14 Верующих же более и более присоединялось к Господу, множество мужчин и женщин,
\vs Act 5:15 так что выносили больных на улицы и полагали на постелях и кроватях, дабы хотя тень проходящего Петра осенила кого из них.
\vs Act 5:16 Сходились также в Иерусалим многие из окрестных городов, неся больных и нечистыми духами одержимых, которые и исцелялись все.
\rsbpar\vs Act 5:17 Первосвященник же и с ним все, принадлежавшие к ереси саддукейской, исполнились зависти,
\vs Act 5:18 и наложили руки свои на Апостолов, и заключили их в народную темницу.
\vs Act 5:19 Но Ангел Господень ночью отворил двери темницы и, выведя их, сказал:
\vs Act 5:20 идите и, став в храме, говорите народу все сии слова жизни.
\vs Act 5:21 Они, выслушав, вошли утром в храм и учили. Между тем первосвященник и которые с ним, придя, созвали синедрион и всех старейшин из сынов Израилевых и послали в темницу привести \bibemph{Апостолов}.
\vs Act 5:22 Но служители, придя, не нашли их в темнице и, возвратившись, донесли,
\vs Act 5:23 говоря: темницу мы нашли запертою со всею предосторожностью и стражей стоящими перед дверями; но, отворив, не нашли в ней никого.
\vs Act 5:24 Когда услышали эти слова первосвященник, начальник стражи и \bibemph{прочие} первосвященники, недоумевали, что бы это значило.
\vs Act 5:25 Пришел же некто и донес им, говоря: вот, мужи, которых вы заключили в темницу, стоят в храме и учат народ.
\vs Act 5:26 Тогда начальник стражи пошел со служителями и привел их без принуждения, потому что боялись народа, чтобы не побили их камнями.
\vs Act 5:27 Приведя же их, поставили в синедрионе; и спросил их первосвященник, говоря:
\vs Act 5:28 не запретили ли мы вам накрепко учить о имени сем? и вот, вы наполнили Иерусалим учением вашим и хотите навести на нас кровь Того Человека.
\vs Act 5:29 Петр же и Апостолы в ответ сказали: должно повиноваться больше Богу, нежели человекам.
\vs Act 5:30 Бог отцов наших воскресил Иисуса, Которого вы умертвили, повесив на древе.
\vs Act 5:31 Его возвысил Бог десницею Своею в Начальника и Спасителя, дабы дать Израилю покаяние и прощение грехов.
\vs Act 5:32 Свидетели Ему в сем мы и Дух Святый, Которого Бог дал повинующимся Ему.
\rsbpar\vs Act 5:33 Слышав это, они разрывались от гнева и умышляли умертвить их.
\vs Act 5:34 Встав же в синедрионе, некто фарисей, именем Гамалиил, законоучитель, уважаемый всем народом, приказал вывести Апостолов на короткое время,
\vs Act 5:35 а им сказал: мужи Израильские! подумайте сами с собою о людях сих, чт\acc{о} вам с ними делать.
\vs Act 5:36 Ибо незадолго перед сим явился Февда, выдавая себя за кого-то великого, и к нему пристало около четырехсот человек; но он был убит, и все, которые слушались его, рассеялись и исчезли.
\vs Act 5:37 После него во время переписи явился Иуда Галилеянин и увлек за собою довольно народа; но он погиб, и все, которые слушались его, рассыпались.
\vs Act 5:38 И ныне, говорю вам, отстаньте от людей сих и оставьте их; ибо если это предприятие и это дело~--- от человеков, то оно разрушится,
\vs Act 5:39 а если от Бога, то вы не можете разрушить его; \bibemph{берегитесь}, чтобы вам не оказаться и богопротивниками.
\vs Act 5:40 Они послушались его; и, призвав Апостолов, били \bibemph{их} и, запретив им говорить о имени Иисуса, отпустили их.
\vs Act 5:41 Они же пошли из синедриона, радуясь, что за имя Господа Иисуса удостоились принять бесчестие.
\vs Act 5:42 И всякий день в храме и по домам не переставали учить и благовествовать об Иисусе Христе.
\vs Act 6:1 В эти дни, когда умножились ученики, произошел у Еллинистов\fns{Евреи из стран языческих.} ропот на Евреев за то, что вдовицы их пренебрегаемы были в ежедневном раздаянии потребностей.
\vs Act 6:2 Тогда двенадцать \bibemph{Апостолов}, созвав множество учеников, сказали: нехорошо нам, оставив слово Божие, пещись о столах.
\vs Act 6:3 Итак, братия, выберите из среды себя семь человек изведанных, исполненных Святаго Духа и мудрости; их поставим на эту службу,
\vs Act 6:4 а мы постоянно пребудем в молитве и служении слова.
\vs Act 6:5 И угодно было это предложение всему собранию; и избрали Стефана, мужа, исполненного веры и Духа Святаго, и Филиппа, и Прохора, и Никанора, и Тимона, и Пармена, и Николая Антиохийца, обращенного из язычников;
\vs Act 6:6 их поставили перед Апостолами, и \bibemph{сии}, помолившись, возложили на них руки.
\rsbpar\vs Act 6:7 И слово Божие росло, и число учеников весьма умножалось в Иерусалиме; и из священников очень многие покорились вере.
\rsbpar\vs Act 6:8 А Стефан, исполненный веры и силы, совершал великие чудеса и знамения в народе.
\vs Act 6:9 Некоторые из так называемой синагоги Либертинцев и Киринейцев и Александрийцев и некоторые из Киликии и Асии вступили в спор со Стефаном;
\vs Act 6:10 но не могли противостоять мудрости и Духу, Которым он говорил.
\vs Act 6:11 Тогда научили они некоторых сказать: мы слышали, как он говорил хульные слова на Моисея и на Бога.
\vs Act 6:12 И возбудили народ и старейшин и книжников и, напав, схватили его и повели в синедрион.
\vs Act 6:13 И представили ложных свидетелей, которые говорили: этот человек не перестает говорить хульные слова на святое место сие и на закон.
\vs Act 6:14 Ибо мы слышали, как он говорил, что Иисус Назорей разрушит место сие и переменит обычаи, которые передал нам Моисей.
\vs Act 6:15 И все, сидящие в синедрионе, смотря на него, видели лице его, как лице Ангела.
\vs Act 7:1 Тогда сказал первосвященник: так ли это?
\vs Act 7:2 Но он сказал: мужи братия и отцы! послушайте. Бог славы явился отцу нашему Аврааму в Месопотамии, прежде переселения его в Харран,
\vs Act 7:3 и сказал ему: выйди из земли твоей и из родства твоего и из дома отца твоего, и пойди в землю, которую покажу тебе.
\vs Act 7:4 Тогда он вышел из земли Халдейской и поселился в Харране; а оттуда, по смерти отца его, переселил его \bibemph{Бог} в сию землю, в которой вы ныне живете.
\vs Act 7:5 И не дал ему на ней наследства ни на стопу ноги, а обещал дать ее во владение ему и потомству его по нем, когда еще был он бездетен.
\vs Act 7:6 И сказал ему Бог, что потомки его будут переселенцами в чужой земле и будут в порабощении и притеснении лет четыреста.
\vs Act 7:7 Но Я, сказал Бог, произведу суд над тем народом, у которого они будут в порабощении; и после того они выйдут и будут служить Мне на сем месте.
\vs Act 7:8 И дал ему завет обрезания. По сем родил он Исаака и обрезал его в восьмой день; а Исаак \bibemph{родил} Иакова, Иаков же двенадцать патриархов.
\vs Act 7:9 Патриархи, по зависти, продали Иосифа в Египет; но Бог был с ним,
\vs Act 7:10 и избавил его от всех скорбей его, и даровал мудрость ему и благоволение царя Египетского фараона, \bibemph{который} и поставил его начальником над Египтом и над всем домом своим.
\vs Act 7:11 И пришел голод и великая скорбь на всю землю Египетскую и Ханаанскую, и отцы наши не находили пропитания.
\vs Act 7:12 Иаков же, услышав, что есть хлеб в Египте, послал \bibemph{туда} отцов наших в первый раз.
\vs Act 7:13 А когда \bibemph{они пришли} во второй раз, Иосиф открылся братьям своим, и известен стал фараону род Иосифов.
\vs Act 7:14 Иосиф, послав, призвал отца своего Иакова и все родство свое, душ семьдесят пять.
\vs Act 7:15 Иаков перешел в Египет, и скончался сам и отцы наши;
\vs Act 7:16 и перенесены были в Сихем и положены во гробе, который купил Авраам ценою серебра у сынов Еммора Сихемова.
\vs Act 7:17 А по мере, как приближалось время \bibemph{исполниться} обетованию, о котором клялся Бог Аврааму, народ возрастал и умножался в Египте,
\vs Act 7:18 до тех пор, как восстал иной царь, который не знал Иосифа.
\vs Act 7:19 Сей, ухищряясь против рода нашего, притеснял отцов наших, принуждая их бросать детей своих, чтобы не оставались в живых.
\vs Act 7:20 В это время родился Моисей, и был прекрасен пред Богом. Три месяца он был питаем в доме отца своего.
\vs Act 7:21 А когда был брошен, взяла его дочь фараонова и воспитала его у себя, как сына.
\vs Act 7:22 И научен был Моисей всей мудрости Египетской, и был силен в словах и делах.
\vs Act 7:23 Когда же исполнилось ему сорок лет, пришло ему на сердце посетить братьев своих, сынов Израилевых.
\vs Act 7:24 И, увидев одного из них обижаемого, вступился и отмстил за оскорбленного, поразив Египтянина.
\vs Act 7:25 Он думал, поймут братья его, что Бог рукою его дает им спасение; но они не поняли.
\vs Act 7:26 На следующий день, когда некоторые из них дрались, он явился и склонял их к миру, говоря: вы братья; зачем обижаете друг друга?
\vs Act 7:27 Но обижающий ближнего оттолкнул его, сказав: кто тебя поставил начальником и судьею над нами?
\vs Act 7:28 Не хочешь ли ты убить и меня, как вчера убил Египтянина?
\vs Act 7:29 От сих слов Моисей убежал и сделался пришельцем в земле Мадиамской, где родились от него два сына.
\vs Act 7:30 По исполнении сорока лет явился ему в пустыне горы Синая Ангел Господень в пламени горящего тернового куста.
\vs Act 7:31 Моисей, увидев, дивился видению; а когда подходил рассмотреть, был к нему глас Господень:
\vs Act 7:32 Я Бог отцов твоих, Бог Авраама и Бог Исаака и Бог Иакова. Моисей, объятый трепетом, не смел смотреть.
\vs Act 7:33 И сказал ему Господь: сними обувь с ног твоих, ибо место, на котором ты стоишь, есть земля святая.
\vs Act 7:34 Я вижу притеснение народа Моего в Египте, и слышу стенание его, и нисшел избавить его: итак пойди, Я пошлю тебя в Египет.
\vs Act 7:35 Сего Моисея, которого они отвергли, сказав: кто тебя поставил начальником и судьею? сего Бог чрез Ангела, явившегося ему в терновом кусте, послал начальником и избавителем.
\vs Act 7:36 Сей вывел их, сотворив чудеса и знамения в земле Египетской, и в Чермном море, и в пустыне в продолжение сорока лет.
\vs Act 7:37 Это тот Моисей, который сказал сынам Израилевым: Пророка воздвигнет вам Господь Бог ваш из братьев ваших, как меня; Его слушайте.
\vs Act 7:38 Это тот, который был в собрании в пустыне с Ангелом, говорившим ему на горе Синае, и с отцами нашими, и который принял живые слова, чтобы передать нам,
\vs Act 7:39 которому отцы наши не хотели быть послушными, но отринули его и обратились сердцами своими к Египту,
\vs Act 7:40 сказав Аарону: сделай нам богов, которые предшествовали бы нам; ибо с Моисеем, который вывел нас из земли Египетской, не знаем, что случилось.
\vs Act 7:41 И сделали в те дни тельца, и принесли жертву идолу, и веселились перед делом рук своих.
\vs Act 7:42 Бог же отвратился и оставил их служить воинству небесному, как написано в книге пророков: дом Израилев! приносили ли вы Мне заколения и жертвы в продолжение сорока лет в пустыне?
\vs Act 7:43 Вы приняли скинию Молохову и звезду бога вашего Ремфана, изображения, которые вы сделали, чтобы поклоняться им: и Я переселю вас далее Вавилона.
\vs Act 7:44 Скиния свидетельства была у отцов наших в пустыне, как повелел Говоривший Моисею сделать ее по образцу, им виденному.
\vs Act 7:45 Отцы наши с Иисусом, взяв ее, внесли во владения народов, изгнанных Богом от лица отцов наших. \bibemph{Так было} до дней Давида.
\vs Act 7:46 Сей обрел благодать пред Богом и молил, \bibemph{чтобы} найти жилище Богу Иакова.
\vs Act 7:47 Соломон же построил Ему дом.
\vs Act 7:48 Но Всевышний не в рукотворенных храмах живет, как говорит пророк:
\vs Act 7:49 Небо~--- престол Мой, и земля~--- подножие ног Моих. Какой дом созиждете Мне, говорит Господь, или какое место для покоя Моего?
\vs Act 7:50 Не Моя ли рука сотворила всё сие?
\vs Act 7:51 Жестоковыйные! люди с необрезанным сердцем и ушами! вы всегда противитесь Духу Святому, как отцы ваши, так и вы.
\vs Act 7:52 Кого из пророков не гнали отцы ваши? Они убили предвозвестивших пришествие Праведника, Которого предателями и убийцами сделались ныне вы,~---
\vs Act 7:53 вы, которые приняли закон при служении Ангелов и не сохранили.
\rsbpar\vs Act 7:54 Слушая сие, они рвались сердцами своими и скрежетали на него зубами.
\vs Act 7:55 Стефан же, будучи исполнен Духа Святаго, воззрев на небо, увидел славу Божию и Иисуса, стоящего одесную Бога,
\vs Act 7:56 и сказал: вот, я вижу небеса отверстые и Сына Человеческого, стоящего одесную Бога.
\vs Act 7:57 Но они, закричав громким голосом, затыкали уши свои, и единодушно устремились на него,
\vs Act 7:58 и, выведя за город, стали побивать его камнями. Свидетели же положили свои одежды у ног юноши, именем Савла,
\vs Act 7:59 и побивали камнями Стефана, который молился и говорил: Господи Иисусе! приими дух мой.
\vs Act 7:60 И, преклонив колени, воскликнул громким голосом: Господи! не вмени им греха сего. И, сказав сие, почил.
\vs Act 8:1 Савл же одобрял убиение его. В те дни произошло великое гонение на церковь в Иерусалиме; и все, кроме Апостолов, рассеялись по разным местам Иудеи и Самарии.
\vs Act 8:2 Стефана же погребли мужи благоговейные, и сделали великий плач по нем.
\vs Act 8:3 А Савл терзал церковь, входя в домы и влача мужчин и женщин, отдавал в темницу.
\rsbpar\vs Act 8:4 Между тем рассеявшиеся ходили и благовествовали слово.
\vs Act 8:5 Так Филипп пришел в город Самарийский и проповедовал им Христа.
\vs Act 8:6 Народ единодушно внимал тому, что говорил Филипп, слыша и видя, какие он творил чудеса.
\vs Act 8:7 Ибо нечистые духи из многих, одержимых ими, выходили с великим воплем, а многие расслабленные и хромые исцелялись.
\vs Act 8:8 И была радость великая в том городе.
\rsbpar\vs Act 8:9 Находился же в городе некоторый муж, именем Симон, который перед тем волхвовал и изумлял народ Самарийский, выдавая себя за кого-то великого.
\vs Act 8:10 Ему внимали все, от малого до большого, говоря: сей есть великая сила Божия.
\vs Act 8:11 А внимали ему потому, что он немалое время изумлял их волхвованиями.
\vs Act 8:12 Но, когда поверили Филиппу, благовествующему о Царствии Божием и о имени Иисуса Христа, то крестились и мужчины и женщины.
\vs Act 8:13 Уверовал и сам Симон и, крестившись, не отходил от Филиппа; и, видя совершающиеся великие силы и знамения, изумлялся.
\rsbpar\vs Act 8:14 Находившиеся в Иерусалиме Апостолы, услышав, что Самаряне приняли слово Божие, послали к ним Петра и Иоанна,
\vs Act 8:15 которые, придя, помолились о них, чтобы они приняли Духа Святаго.
\vs Act 8:16 Ибо Он не сходил еще ни на одного из них, а только были они крещены во имя Господа Иисуса.
\vs Act 8:17 Тогда возложили руки на них, и они приняли Духа Святаго.
\vs Act 8:18 Симон же, увидев, что через возложение рук Апостольских подается Дух Святый, принес им деньги,
\vs Act 8:19 говоря: дайте и мне власть сию, чтобы тот, на кого я возложу руки, получал Духа Святаго.
\vs Act 8:20 Но Петр сказал ему: серебро твое да будет в погибель с тобою, потому что ты помыслил дар Божий получить за деньги.
\vs Act 8:21 Нет тебе в сем части и жребия, ибо сердце твое неправо пред Богом.
\vs Act 8:22 Итак покайся в сем грехе твоем, и молись Богу: может быть, отпустится тебе помысел сердца твоего;
\vs Act 8:23 ибо вижу тебя исполненного горькой желчи и в узах неправды.
\vs Act 8:24 Симон же сказал в ответ: помолитесь вы за меня Господу, дабы не постигло меня ничто из сказанного вами.
\vs Act 8:25 Они же, засвидетельствовав и проповедав слово Господне, обратно пошли в Иерусалим и во многих селениях Самарийских проповедали Евангелие.
\rsbpar\vs Act 8:26 А Филиппу Ангел Господень сказал: встань и иди на полдень, на дорогу, идущую из Иерусалима в Газу, на ту, которая пуста.
\vs Act 8:27 Он встал и пошел. И вот, муж Ефиоплянин, евнух, вельможа Кандакии, царицы Ефиопской, хранитель всех сокровищ ее, приезжавший в Иерусалим для поклонения,
\vs Act 8:28 возвращался и, сидя на колеснице своей, читал пророка Исаию.
\vs Act 8:29 Дух сказал Филиппу: подойди и пристань к сей колеснице.
\vs Act 8:30 Филипп подошел и, услышав, что он читает пророка Исаию, сказал: разумеешь ли, что читаешь?
\vs Act 8:31 Он сказал: как могу разуметь, если кто не наставит меня? и попросил Филиппа взойти и сесть с ним.
\vs Act 8:32 А место из Писания, которое он читал, было сие: как овца, веден был Он на заклание, и, как агнец пред стригущим его безгласен, так Он не отверзает уст Своих.
\vs Act 8:33 В уничижении Его суд Его совершился. Но род Его кто разъяснит? ибо вземлется от земли жизнь Его.
\vs Act 8:34 Евнух же сказал Филиппу: прошу тебя \bibemph{сказать}: о ком пророк говорит это? о себе ли, или о ком другом?
\vs Act 8:35 Филипп отверз уста свои и, начав от сего Писания, благовествовал ему об Иисусе.
\vs Act 8:36 Между тем, продолжая путь, они приехали к воде; и евнух сказал: вот вода; что препятствует мне креститься?
\vs Act 8:37 Филипп же сказал ему: если веруешь от всего сердца, можно. Он сказал в ответ: верую, что Иисус Христос есть Сын Божий.
\vs Act 8:38 И приказал остановить колесницу, и сошли оба в воду, Филипп и евнух; и крестил его.
\vs Act 8:39 Когда же они вышли из воды, Дух Святый сошел на евнуха, а Филиппа восхитил Ангел Господень, и евнух уже не видел его, и продолжал путь, радуясь.
\vs Act 8:40 А Филипп оказался в Азоте и, проходя, благовествовал всем городам, пока пришел в Кесарию.
\vs Act 9:1 Савл же, еще дыша угрозами и убийством на учеников Господа, пришел к первосвященнику
\vs Act 9:2 и выпросил у него письма в Дамаск к синагогам, чтобы, кого найдет последующих сему учению, и мужчин и женщин, связав, приводить в Иерусалим.
\rsbpar\vs Act 9:3 Когда же он шел и приближался к Дамаску, внезапно осиял его свет с неба.
\vs Act 9:4 Он упал на землю и услышал голос, говорящий ему: Савл, Савл! что ты гонишь Меня?
\vs Act 9:5 Он сказал: кто Ты, Господи? Господь же сказал: Я Иисус, Которого ты гонишь. Трудно тебе идти против рожна.
\vs Act 9:6 Он в трепете и ужасе сказал: Господи! что повелишь мне делать? и Господь \bibemph{сказал} ему: встань и иди в город; и сказано будет тебе, что тебе надобно делать.
\vs Act 9:7 Люди же, шедшие с ним, стояли в оцепенении, слыша голос, а никого не видя.
\vs Act 9:8 Савл встал с земли, и с открытыми глазами никого не видел. И повели его за руки, и привели в Дамаск.
\vs Act 9:9 И три дня он не видел, и не ел, и не пил.
\rsbpar\vs Act 9:10 В Дамаске был один ученик, именем Анания; и Господь в видении сказал ему: Анания! Он сказал: я, Господи.
\vs Act 9:11 Господь же \bibemph{сказал} ему: встань и пойди на улицу, так называемую Прямую, и спроси в Иудином доме Тарсянина, по имени Савла; он теперь молится,
\vs Act 9:12 и видел в видении мужа, именем Ананию, пришедшего к нему и возложившего на него руку, чтобы он прозрел.
\vs Act 9:13 Анания отвечал: Господи! я слышал от многих о сем человеке, сколько зла сделал он святым Твоим в Иерусалиме;
\vs Act 9:14 и здесь имеет от первосвященников власть вязать всех, призывающих имя Твое.
\vs Act 9:15 Но Господь сказал ему: иди, ибо он есть Мой избранный сосуд, чтобы возвещать имя Мое перед народами и царями и сынами Израилевыми.
\vs Act 9:16 И Я покажу ему, сколько он должен пострадать за имя Мое.
\vs Act 9:17 Анания пошел и вошел в дом и, возложив на него руки, сказал: брат Савл! Господь Иисус, явившийся тебе на пути, которым ты шел, послал меня, чтобы ты прозрел и исполнился Святаго Духа.
\vs Act 9:18 И тотчас как бы чешуя отпала от глаз его, и вдруг он прозрел; и, встав, крестился,
\vs Act 9:19 и, приняв пищи, укрепился.\rsbpar И был Савл несколько дней с учениками в Дамаске.
\vs Act 9:20 И тотчас стал проповедовать в синагогах об Иисусе, что Он есть Сын Божий.
\vs Act 9:21 И все слышавшие дивились и говорили: не тот ли это самый, который гнал в Иерусалиме призывающих имя сие? да и сюда за тем пришел, чтобы вязать их и вести к первосвященникам.
\vs Act 9:22 А Савл более и более укреплялся и приводил в замешательство Иудеев, живущих в Дамаске, доказывая, что Сей есть Христос.
\rsbpar\vs Act 9:23 Когда же прошло довольно времени, Иудеи согласились убить его.
\vs Act 9:24 Но Савл узнал об этом умысле их. А они день и ночь стерегли у ворот, чтобы убить его.
\vs Act 9:25 Ученики же ночью, взяв его, спустили по стене в корзине.
\vs Act 9:26 Савл прибыл в Иерусалим и старался пристать к ученикам; но все боялись его, не веря, что он ученик.
\vs Act 9:27 Варнава же, взяв его, пришел к Апостолам и рассказал им, как на пути он видел Господа, и что говорил ему Господь, и как он в Дамаске смело проповедовал во имя Иисуса.
\vs Act 9:28 И пребывал он с ними, входя и исходя, в Иерусалиме, и смело проповедовал во имя Господа Иисуса.
\vs Act 9:29 Говорил также и состязался с Еллинистами; а они покушались убить его.
\vs Act 9:30 Братия, узнав \bibemph{о сем}, отправили его в Кесарию и препроводили в Тарс.
\rsbpar\vs Act 9:31 Церкви же по всей Иудее, Галилее и Самарии были в покое, назидаясь и ходя в страхе Господнем; и, при утешении от Святаго Духа, умножались.
\rsbpar\vs Act 9:32 Случилось, что Петр, обходя всех, пришел и к святым, живущим в Лидде.
\vs Act 9:33 Там нашел он одного человека, именем Энея, который восемь уже лет лежал в постели в расслаблении.
\vs Act 9:34 Петр сказал ему: Эней! исцеляет тебя Иисус Христос; встань с постели твоей. И он тотчас встал.
\vs Act 9:35 И видели его все, живущие в Лидде и в Сароне, которые и обратились к Господу.
\rsbpar\vs Act 9:36 В Иоппии находилась одна ученица, именем Тавифа, что значит: <<серна>>; она была исполнена добрых дел и творила много милостынь.
\vs Act 9:37 Случилось в те дни, что она занемогла и умерла. Ее омыли и положили в горнице.
\vs Act 9:38 А как Лидда была близ Иоппии, то ученики, услышав, что Петр находится там, послали к нему двух человек просить, чтобы он не замедлил прийти к ним.
\vs Act 9:39 Петр, встав, пошел с ними; и когда он прибыл, ввели его в горницу, и все вдовицы со слезами предстали перед ним, показывая рубашки и платья, какие делала Серна, живя с ними.
\vs Act 9:40 Петр выслал всех вон и, преклонив колени, помолился, и, обратившись к телу, сказал: Тавифа! встань. И она открыла глаза свои и, увидев Петра, села.
\vs Act 9:41 Он, подав ей руку, поднял ее, и, призвав святых и вдовиц, поставил ее перед ними живою.
\vs Act 9:42 Это сделалось известным по всей Иоппии, и многие уверовали в Господа.
\vs Act 9:43 И довольно дней пробыл он в Иоппии у некоторого Симона кожевника.
\vs Act 10:1 В Кесарии был некоторый муж, именем Корнилий, сотник из полка, называемого Италийским,
\vs Act 10:2 благочестивый и боящийся Бога со всем домом своим, творивший много милостыни народу и всегда молившийся Богу.
\vs Act 10:3 Он в видении ясно видел около девятого часа дня Ангела Божия, который вошел к нему и сказал ему: Корнилий!
\vs Act 10:4 Он же, взглянув на него и испугавшись, сказал: чт\acc{о}, Господи? \bibemph{Ангел} отвечал ему: молитвы твои и милостыни твои пришли на память пред Богом.
\vs Act 10:5 Итак пошли людей в Иоппию и призови Симона, называемого Петром.
\vs Act 10:6 Он гостит у некоего Симона кожевника, которого дом находится при море; он скажет тебе слова, которыми спасешься ты и весь дом твой.
\vs Act 10:7 Когда Ангел, говоривший с Корнилием, отошел, то он, призвав двоих из своих слуг и благочестивого воина из находившихся при нем
\vs Act 10:8 и, рассказав им все, послал их в Иоппию.
\rsbpar\vs Act 10:9 На другой день, когда они шли и приближались к городу, Петр около шестого часа взошел на верх дома помолиться.
\vs Act 10:10 И почувствовал он голод, и хотел есть. Между тем, как приготовляли, он пришел в исступление
\vs Act 10:11 и видит отверстое небо и сходящий к нему некоторый сосуд, как бы большое полотно, привязанное за четыре угла и опускаемое на землю;
\vs Act 10:12 в нем находились всякие четвероногие земные, звери, пресмыкающиеся и птицы небесные.
\vs Act 10:13 И был глас к нему: встань, Петр, заколи и ешь.
\vs Act 10:14 Но Петр сказал: нет, Господи, я никогда не ел ничего скверного или нечистого.
\vs Act 10:15 Тогда в другой раз \bibemph{был} глас к нему: что Бог очистил, того ты не почитай нечистым.
\vs Act 10:16 Это было трижды; и сосуд опять поднялся на небо.
\vs Act 10:17 Когда же Петр недоумевал в себе, что бы значило видение, которое он видел,~--- вот, мужи, посланные Корнилием, расспросив о доме Симона, остановились у ворот,
\vs Act 10:18 и, крикнув, спросили: здесь ли Симон, называемый Петром?
\vs Act 10:19 Между тем, как Петр размышлял о видении, Дух сказал ему: вот, три человека ищут тебя;
\vs Act 10:20 встань, сойди и иди с ними, нимало не сомневаясь; ибо Я послал их.
\vs Act 10:21 Петр, сойдя к людям, присланным к нему от Корнилия, сказал: я тот, которого вы ищете; за каким делом пришли вы?
\vs Act 10:22 Они же сказали: Корнилий сотник, муж добродетельный и боящийся Бога, одобряемый всем народом Иудейским, получил от святаго Ангела повеление призвать тебя в дом свой и послушать речей твоих.
\vs Act 10:23 Тогда Петр, пригласив их, угостил. А на другой день, встав, пошел с ними, и некоторые из братий Иоппийских пошли с ним.
\rsbpar\vs Act 10:24 В следующий день пришли они в Кесарию. Корнилий же ожидал их, созвав родственников своих и близких друзей.
\vs Act 10:25 Когда Петр входил, Корнилий встретил его и поклонился, пав к ногам его.
\vs Act 10:26 Петр же поднял его, говоря: встань; я тоже человек.
\vs Act 10:27 И, беседуя с ним, вошел \bibemph{в дом}, и нашел многих собравшихся.
\vs Act 10:28 И сказал им: вы знаете, что Иудею возбранено сообщаться или сближаться с иноплеменником; но мне Бог открыл, чтобы я не почитал ни одного человека скверным или нечистым.
\vs Act 10:29 Посему я, будучи позван, и пришел беспрекословно. Итак спрашиваю: для какого дела вы призвали меня?
\vs Act 10:30 Корнилий сказал: четвертого дня я постился до теперешнего часа, и в девятом часу молился в своем доме, и вот, стал предо мною муж в светлой одежде,
\vs Act 10:31 и говорит: Корнилий! услышана молитва твоя, и милостыни твои воспомянулись пред Богом.
\vs Act 10:32 Итак пошли в Иоппию и призови Симона, называемого Петром; он гостит в доме кожевника Симона при море; он придет и скажет тебе.
\vs Act 10:33 Тотчас послал я к тебе, и ты хорошо сделал, что пришел. Теперь все мы предстоим пред Богом, чтобы выслушать все, что повелено тебе от Бога.
\rsbpar\vs Act 10:34 Петр отверз уста и сказал: истинно позна\acc{ю}, что Бог нелицеприятен,
\vs Act 10:35 но во всяком народе боящийся Его и поступающий по правде приятен Ему.
\vs Act 10:36 Он послал сынам Израилевым слово, благовествуя мир чрез Иисуса Христа; Сей есть Господь всех.
\vs Act 10:37 Вы знаете происходившее по всей Иудее, начиная от Галилеи, после крещения, проповеданного Иоанном:
\vs Act 10:38 как Бог Духом Святым и силою помазал Иисуса из Назарета, и Он ходил, благотворя и исцеляя всех, обладаемых диаволом, потому что Бог был с Ним.
\vs Act 10:39 И мы свидетели всего, что сделал Он в стране Иудейской и в Иерусалиме, и что наконец Его убили, повесив на древе.
\vs Act 10:40 Сего Бог воскресил в третий день, и дал Ему являться
\vs Act 10:41 не всему народу, но свидетелям, предъизбранным от Бога, нам, которые с Ним ели и пили, по воскресении Его из мертвых.
\vs Act 10:42 И Он повелел нам проповедовать людям и свидетельствовать, что Он есть определенный от Бога Судия живых и мертвых.
\vs Act 10:43 О Нем все пророки свидетельствуют, что всякий верующий в Него получит прощение грехов именем Его.
\rsbpar\vs Act 10:44 Когда Петр еще продолжал эту речь, Дух Святый сошел на всех, слушавших слово.
\vs Act 10:45 И верующие из обрезанных, пришедшие с Петром, изумились, что дар Святаго Духа излился и на язычников,
\vs Act 10:46 ибо слышали их говорящих языками и величающих Бога. Тогда Петр сказал:
\vs Act 10:47 кто может запретить креститься водою тем, которые, как и мы, получили Святаго Духа?
\vs Act 10:48 И велел им креститься во имя Иисуса Христа. Потом они просили его пробыть у них несколько дней.
\vs Act 11:1 Услышали Апостолы и братия, бывшие в Иудее, что и язычники приняли слово Божие.
\vs Act 11:2 И когда Петр пришел в Иерусалим, обрезанные упрекали его,
\vs Act 11:3 говоря: ты ходил к людям необрезанным и ел с ними.
\vs Act 11:4 Петр же начал пересказывать им по порядку, говоря:
\vs Act 11:5 в городе Иоппии я молился, и в исступлении видел видение: сходил некоторый сосуд, как бы большое полотно, за четыре угла спускаемое с неба, и спустилось ко мне.
\vs Act 11:6 Я посмотрел в него и, рассматривая, увидел четвероногих земных, зверей, пресмыкающихся и птиц небесных.
\vs Act 11:7 И услышал я голос, говорящий мне: встань, Петр, заколи и ешь.
\vs Act 11:8 Я же сказал: нет, Господи, ничего скверного или нечистого никогда не входило в уста мои.
\vs Act 11:9 И отвечал мне голос вторично с неба: что Бог очистил, того ты не почитай нечистым.
\vs Act 11:10 Это было трижды, и опять поднялось всё на небо.
\vs Act 11:11 И вот, в тот самый час три человека стали перед домом, в котором я был, посланные из Кесарии ко мне.
\vs Act 11:12 Дух сказал мне, чтобы я шел с ними, нимало не сомневаясь. Пошли со мною и сии шесть братьев, и мы пришли в дом \bibemph{того} человека.
\vs Act 11:13 Он рассказал нам, как он видел в доме своем Ангела (святого), который стал и сказал ему: пошли в Иоппию людей и призови Симона, называемого Петром;
\vs Act 11:14 он скажет тебе слова, которыми спасешься ты и весь дом твой.
\vs Act 11:15 Когда же начал я говорить, сошел на них Дух Святый, как и на нас вначале.
\vs Act 11:16 Тогда вспомнил я слово Господа, как Он говорил: <<Иоанн крестил водою, а вы будете крещены Духом Святым>>.
\vs Act 11:17 Итак, если Бог дал им такой же дар, как и нам, уверовавшим в Господа Иисуса Христа, то кто же я, чтобы мог воспрепятствовать Богу?
\vs Act 11:18 Выслушав это, они успокоились и прославили Бога, говоря: видно, и язычникам дал Бог покаяние в жизнь.
\rsbpar\vs Act 11:19 Между тем рассеявшиеся от гонения, бывшего после Стефана, прошли до Финикии и Кипра и Антиохии, никому не проповедуя слово, кроме Иудеев.
\vs Act 11:20 Были же некоторые из них Кипряне и Киринейцы, которые, придя в Антиохию, говорили Еллинам, благовествуя Господа Иисуса.
\vs Act 11:21 И была рука Господня с ними, и великое число, уверовав, обратилось к Господу.
\vs Act 11:22 Дошел слух о сем до церкви Иерусалимской, и поручили Варнаве идти в Антиохию.
\vs Act 11:23 Он, прибыв и увидев благодать Божию, возрадовался и убеждал всех держаться Господа искренним сердцем;
\vs Act 11:24 ибо он был муж добрый и исполненный Духа Святаго и веры. И приложилось довольно народа к Господу.
\vs Act 11:25 Потом Варнава пошел в Тарс искать Савла и, найдя его, привел в Антиохию.
\vs Act 11:26 Целый год собирались они в церкви и учили немалое число людей, и ученики в Антиохии в первый раз стали называться Христианами.
\rsbpar\vs Act 11:27 В те дни пришли из Иерусалима в Антиохию пророки.
\vs Act 11:28 И один из них, по имени Агав, встав, предвозвестил Духом, что по всей вселенной будет великий голод, который и был при кесаре Клавдии.
\vs Act 11:29 Тогда ученики положили, каждый по достатку своему, послать пособие братьям, живущим в Иудее,
\vs Act 11:30 что и сделали, послав \bibemph{собранное} к пресвитерам через Варнаву и Савла.
\vs Act 12:1 В то время царь Ирод поднял руки на некоторых из принадлежащих к церкви, чтобы сделать им зло,
\vs Act 12:2 и убил Иакова, брата Иоаннова, мечом.
\vs Act 12:3 Видя же, что это приятно Иудеям, вслед за тем взял и Петра,~--- тогда были дни опресноков,~---
\vs Act 12:4 и, задержав его, посадил в темницу, и приказал четырем четверицам воинов стеречь его, намереваясь после Пасхи вывести его к народу.
\vs Act 12:5 Итак Петра стерегли в темнице, между тем церковь прилежно молилась о нем Богу.
\rsbpar\vs Act 12:6 Когда же Ирод хотел вывести его, в ту ночь Петр спал между двумя воинами, скованный двумя цепями, и стражи у дверей стерегли темницу.
\vs Act 12:7 И вот, Ангел Господень предстал, и свет осиял темницу. \bibemph{Ангел}, толкнув Петра в бок, пробудил его и сказал: встань скорее. И цепи упали с рук его.
\vs Act 12:8 И сказал ему Ангел: опояшься и обуйся. Он сделал так. Потом говорит ему: надень одежду твою и иди за мною.
\vs Act 12:9 \bibemph{Петр} вышел и следовал за ним, не зная, что делаемое Ангелом было действительно, а думая, что видит видение.
\vs Act 12:10 Пройдя первую и вторую стражу, они пришли к железным воротам, ведущим в город, которые сами собою отворились им: они вышли, и прошли одну улицу, и вдруг Ангела не стало с ним.
\vs Act 12:11 Тогда Петр, придя в себя, сказал: теперь я вижу воистину, что Господь послал Ангела Своего и избавил меня из руки Ирода и от всего, чего ждал народ Иудейский.
\vs Act 12:12 И, осмотревшись, пришел к дому Марии, матери Иоанна, называемого Марком, где многие собрались и молились.
\vs Act 12:13 Когда же Петр постучался у ворот, то вышла послушать служанка, именем Рода,
\vs Act 12:14 и, узнав голос Петра, от радости не отворила ворот, но, вбежав, объявила, что Петр стоит у ворот.
\vs Act 12:15 А те сказали ей: в своем ли ты уме? Но она утверждала свое. Они же говорили: это Ангел его.
\vs Act 12:16 Между тем Петр продолжал стучать. Когда же отворили, то увидели его и изумились.
\vs Act 12:17 Он же, дав знак рукою, чтобы молчали, рассказал им, как Господь вывел его из темницы, и сказал: уведомьте о сем Иакова и братьев. Потом, выйдя, пошел в другое место.
\rsbpar\vs Act 12:18 По наступлении дня между воинами сделалась большая тревога о том, что сделалось с Петром.
\vs Act 12:19 Ирод же, поискав его и не найдя, судил стражей и велел казнить их. Потом он отправился из Иудеи в Кесарию и \bibemph{там} оставался.
\rsbpar\vs Act 12:20 Ирод был раздражен на Тирян и Сидонян; они же, согласившись, пришли к нему и, склонив на свою сторону Власта, постельника царского, просили мира, потому что область их питалась от \bibemph{области} царской.
\vs Act 12:21 В назначенный день Ирод, одевшись в царскую одежду, сел на возвышенном месте и говорил к ним;
\vs Act 12:22 а народ восклицал: \bibemph{это} голос Бога, а не человека.
\vs Act 12:23 Но вдруг Ангел Господень поразил его за то, что он не воздал славы Богу; и он, быв изъеден червями, умер.
\rsbpar\vs Act 12:24 Слово же Божие росло и распространялось.
\vs Act 12:25 А Варнава и Савл, по исполнении поручения, возвратились из Иерусалима (в Антиохию), взяв с собою и Иоанна, прозванного Марком.
\vs Act 13:1 В Антиохии, в тамошней церкви были некоторые пророки и учители: Варнава, и Симеон, называемый Нигер, и Луций Киринеянин, и Манаил, совоспитанник Ирода четвертовластника, и Савл.
\vs Act 13:2 Когда они служили Господу и постились, Дух Святый сказал: отделите Мне Варнаву и Савла на дело, к которому Я призвал их.
\vs Act 13:3 Тогда они, совершив пост и молитву и возложив на них руки, отпустили их.
\rsbpar\vs Act 13:4 Сии, быв посланы Духом Святым, пришли в Селевкию, а оттуда отплыли в Кипр;
\vs Act 13:5 и, быв в Саламине, проповедовали слово Божие в синагогах Иудейских; имели же при себе и Иоанна для служения.
\vs Act 13:6 Пройдя весь остров до Пафа, нашли они некоторого волхва, лжепророка, Иудеянина, именем Вариисуса,
\vs Act 13:7 который находился с проконсулом Сергием Павлом, мужем разумным. Сей, призвав Варнаву и Савла, пожелал услышать слово Божие.
\vs Act 13:8 А Елима волхв (ибо т\acc{о} значит имя его) противился им, стараясь отвратить проконсула от веры.
\vs Act 13:9 Но Савл, он же и Павел, исполнившись Духа Святаго и устремив на него взор,
\vs Act 13:10 сказал: о, исполненный всякого коварства и всякого злодейства, сын диавола, враг всякой правды! перестанешь ли ты совращать с прямых путей Господних?
\vs Act 13:11 И ныне вот, рука Господня на тебя: ты будешь слеп и не увидишь солнца до времени. И вдруг напал на него мрак и тьма, и он, обращаясь туда и сюда, искал вожатого.
\vs Act 13:12 Тогда проконсул, увидев происшедшее, уверовал, дивясь учению Господню.
\rsbpar\vs Act 13:13 Отплыв из Пафа, Павел и бывшие при нем прибыли в Пергию, в Памфилии. Но Иоанн, отделившись от них, возвратился в Иерусалим.
\vs Act 13:14 Они же, проходя от Пергии, прибыли в Антиохию Писидийскую и, войдя в синагогу в день субботний, сели.
\vs Act 13:15 После чтения закона и пророков, начальники синагоги послали сказать им: мужи братия! если у вас есть слово наставления к народу, говорите.
\rsbpar\vs Act 13:16 Павел, встав и дав знак рукою, сказал: мужи Израильтяне и боящиеся Бога! послушайте.
\vs Act 13:17 Бог народа сего избрал отцов наших и возвысил сей народ во время пребывания в земле Египетской, и мышцею вознесенною вывел их из нее,
\vs Act 13:18 и около сорока лет времени питал их в пустыне.
\vs Act 13:19 И, истребив семь народов в земле Ханаанской, разделил им в наследие землю их.
\vs Act 13:20 И после сего, около четырехсот пятидесяти лет, давал им судей до пророка Самуила.
\vs Act 13:21 Потом просили они царя, и Бог дал им Саула, сына Кисова, мужа из колена Вениаминова. \bibemph{Так прошло} лет сорок.
\vs Act 13:22 Отринув его, поставил им царем Давида, о котором и сказал, свидетельствуя: нашел Я мужа по сердцу Моему, Давида, сына Иессеева, который исполнит все хотения Мои.
\vs Act 13:23 Из его-то потомства Бог по обетованию воздвиг Израилю Спасителя Иисуса.
\vs Act 13:24 Перед самым явлением Его Иоанн проповедовал крещение покаяния всему народу Израильскому.
\vs Act 13:25 При окончании же поприща своего, Иоанн говорил: за кого почитаете вы меня? я не тот; но вот, идет за мною, у Которого я недостоин развязать обувь на ногах.
\vs Act 13:26 Мужи братия, дети рода Авраамова, и боящиеся Бога между вами! вам послано слово спасения сего.
\vs Act 13:27 Ибо жители Иерусалима и начальники их, не узнав Его и осудив, исполнили слова пророческие, читаемые каждую субботу,
\vs Act 13:28 и, не найдя в Нем никакой вины, достойной смерти, просили Пилата убить Его.
\vs Act 13:29 Когда же исполнили всё написанное о Нем, то, сняв с древа, положили Его во гроб.
\vs Act 13:30 Но Бог воскресил Его из мертвых.
\vs Act 13:31 Он в продолжение многих дней являлся тем, которые вышли с Ним из Галилеи в Иерусалим и которые ныне суть свидетели Его перед народом.
\vs Act 13:32 И мы благовествуем вам, что обетование, данное отцам, Бог исполнил нам, детям их, воскресив Иисуса,
\vs Act 13:33 как и во втором псалме написано: Ты Сын Мой: Я ныне родил Тебя.
\vs Act 13:34 А что воскресил Его из мертвых, так что Он уже не обратится в тление, \bibemph{о сем} сказал так: Я дам вам милости, \bibemph{обещанные} Давиду, верно.
\vs Act 13:35 Посему и в другом \bibemph{месте} говорит: не дашь Святому Твоему увидеть тление.
\vs Act 13:36 Давид, в свое время послужив изволению Божию, почил и приложился к отцам своим, и увидел тление;
\vs Act 13:37 а Тот, Которого Бог воскресил, не увидел тления.
\vs Act 13:38 Итак, да будет известно вам, мужи братия, что ради Него возвещается вам прощение грехов;
\vs Act 13:39 и во всем, в чем вы не могли оправдаться законом Моисеевым, оправдывается Им всякий верующий.
\vs Act 13:40 Берегитесь же, чтобы не пришло на вас сказанное у пророков:
\vs Act 13:41 смотрите, презрители, подивитесь и исчезните; ибо Я делаю дело во дни ваши, дело, которому не поверили бы вы, если бы кто рассказывал вам.
\rsbpar\vs Act 13:42 При выходе их из Иудейской синагоги язычники просили их говорить о том же в следующую субботу.
\vs Act 13:43 Когда же собрание было распущено, то многие Иудеи и чтители \bibemph{Бога}, обращенные из язычников, последовали за Павлом и Варнавою, которые, беседуя с ними, убеждали их пребывать в благодати Божией.
\rsbpar\vs Act 13:44 В следующую субботу почти весь город собрался слушать слово Божие.
\vs Act 13:45 Но Иудеи, увидев народ, исполнились зависти и, противореча и злословя, сопротивлялись тому, что говорил Павел.
\vs Act 13:46 Тогда Павел и Варнава с дерзновением сказали: вам первым надлежало быть проповедану слову Божию, но как вы отвергаете его и сами себя делаете недостойными вечной жизни, то вот, мы обращаемся к язычникам.
\vs Act 13:47 Ибо так заповедал нам Господь: Я положил Тебя во свет язычникам, чтобы Ты был во спасение до края земли.
\vs Act 13:48 Язычники, слыша это, радовались и прославляли слово Господне, и уверовали все, которые были предуставлены к вечной жизни.
\vs Act 13:49 И слово Господне распространялось по всей стране.
\vs Act 13:50 Но Иудеи, подстрекнув набожных и почетных женщин и первых в городе \bibemph{людей}, воздвигли гонение на Павла и Варнаву и изгнали их из своих пределов.
\vs Act 13:51 Они же, отрясши на них прах от ног своих, пошли в Иконию.
\vs Act 13:52 А ученики исполнялись радости и Духа Святаго.
\vs Act 14:1 В Иконии они вошли вместе в Иудейскую синагогу и говорили так, что уверовало великое множество Иудеев и Еллинов.
\vs Act 14:2 А неверующие Иудеи возбудили и раздражили против братьев сердца язычников.
\vs Act 14:3 Впрочем они пробыли \bibemph{здесь} довольно времени, смело действуя о Господе, Который, во свидетельство слову благодати Своей, творил руками их знамения и чудеса.
\vs Act 14:4 Между тем народ в городе разделился: и одни были на стороне Иудеев, а другие на стороне Апостолов.
\vs Act 14:5 Когда же язычники и Иудеи со своими начальниками устремились на них, чтобы посрамить и побить их камнями,
\vs Act 14:6 они, узнав \bibemph{о сем}, удалились в Ликаонские города Листру и Дервию и в окрестности их,
\vs Act 14:7 и там благовествовали.
\rsbpar\vs Act 14:8 В Листре некоторый муж, не владевший ногами, сидел, будучи хром от чрева матери своей, и никогда не ходил.
\vs Act 14:9 Он слушал говорившего Павла, который, взглянув на него и увидев, что он имеет веру для получения исцеления,
\vs Act 14:10 сказал громким голосом: тебе говорю во имя Господа Иисуса Христа: стань на ноги твои прямо. И он тотчас вскочил и стал ходить.
\vs Act 14:11 Народ же, увидев, что сделал Павел, возвысил свой голос, говоря по-ликаонски: боги в образе человеческом сошли к нам.
\vs Act 14:12 И называли Варнаву Зевсом, а Павла Ермием, потому что он начальствовал в слове.
\vs Act 14:13 Жрец же \bibemph{идола} Зевса, находившегося перед их городом, приведя к воротам волов и \bibemph{принеся} венки, хотел вместе с народом совершить жертвоприношение.
\vs Act 14:14 Но Апостолы Варнава и Павел, услышав \bibemph{о сем}, разодрали свои одежды и, бросившись в народ, громогласно говорили:
\vs Act 14:15 мужи! чт\acc{о} вы это делаете? И мы~--- подобные вам человеки, и благовествуем вам, чтобы вы обратились от сих ложных к Богу живому, Который сотворил небо и землю, и море, и все, что в них,
\vs Act 14:16 Который в прошедших родах попустил всем народам ходить своими путями,
\vs Act 14:17 хотя и не переставал свидетельствовать о Себе благодеяниями, подавая нам с неба дожди и времена плодоносные и исполняя пищею и веселием сердца наши.
\vs Act 14:18 И, говоря сие, они едва убедили народ не приносить им жертвы и идти каждому домой. Между тем, как они, оставаясь там, учили,
\vs Act 14:19 из Антиохии и Иконии пришли некоторые Иудеи и, когда \bibemph{Апостолы} смело проповедовали, убедили народ отстать от них, говоря: они не говорят ничего истинного, а все лгут. И, возбудив народ, побили Павла камнями и вытащили за город, почитая его умершим.
\vs Act 14:20 Когда же ученики собрались около него, он встал и пошел в город, а на другой день удалился с Варнавою в Дервию.
\rsbpar\vs Act 14:21 Проповедав Евангелие сему городу и приобретя довольно учеников, они обратно проходили Листру, Иконию и Антиохию,
\vs Act 14:22 утверждая души учеников, увещевая пребывать в вере и \bibemph{поучая}, что многими скорбями надлежит нам войти в Царствие Божие.
\vs Act 14:23 Рукоположив же им пресвитеров к каждой церкви, они помолились с постом и предали их Господу, в Которого уверовали.
\vs Act 14:24 Потом, пройдя через Писидию, пришли в Памфилию,
\vs Act 14:25 и, проповедав слово Господне в Пергии, сошли в Атталию;
\vs Act 14:26 а оттуда отплыли в Антиохию, откуда были преданы благодати Божией на дело, которое и исполнили.
\rsbpar\vs Act 14:27 Прибыв туда и собрав церковь, они рассказали всё, что сотворил Бог с ними и как Он отверз дверь веры язычникам.
\vs Act 14:28 И пребывали там немалое время с учениками.
\vs Act 15:1 Некоторые, пришедшие из Иудеи, учили братьев: если не обрежетесь по обряду Моисееву, не можете спастись.
\vs Act 15:2 Когда же произошло разногласие и немалое состязание у Павла и Варнавы с ними, то положили Павлу и Варнаве и некоторым другим из них отправиться по сему делу к Апостолам и пресвитерам в Иерусалим.
\vs Act 15:3 Итак, быв провожены церковью, они проходили Финикию и Самарию, рассказывая об обращении язычников, и производили радость великую во всех братиях.
\vs Act 15:4 По прибытии же в Иерусалим они были приняты церковью, Апостолами и пресвитерами, и возвестили всё, что Бог сотворил с ними и как отверз дверь веры язычникам.
\vs Act 15:5 Тогда восстали некоторые из фарисейской ереси уверовавшие и говорили, что должно обрезывать \bibemph{язычников} и заповедовать соблюдать закон Моисеев.
\rsbpar\vs Act 15:6 Апостолы и пресвитеры собрались для рассмотрения сего дела.
\vs Act 15:7 По долгом рассуждении Петр, встав, сказал им: мужи братия! вы знаете, что Бог от дней первых избрал из нас \bibemph{меня}, чтобы из уст моих язычники услышали слово Евангелия и уверовали;
\vs Act 15:8 и Сердцеведец Бог дал им свидетельство, даровав им Духа Святаго, как и нам;
\vs Act 15:9 и не положил никакого различия между нами и ими, верою очистив сердца их.
\vs Act 15:10 Что же вы ныне искушаете Бога, \bibemph{желая} возложить на выи учеников иго, которого не могли понести ни отцы наши, ни мы?
\vs Act 15:11 Но мы веруем, что благодатию Господа Иисуса Христа спасемся, как и они.
\vs Act 15:12 Тогда умолкло все собрание и слушало Варнаву и Павла, рассказывавших, какие знамения и чудеса сотворил Бог через них среди язычников.
\vs Act 15:13 После же того, как они умолкли, начал речь Иаков и сказал: мужи братия! послушайте меня.
\vs Act 15:14 Симон изъяснил, как Бог первоначально призрел на язычников, чтобы составить из них народ во имя Свое.
\vs Act 15:15 И с сим согласны слова пророков, как написано:
\vs Act 15:16 Потом обращусь и воссоздам скинию Давидову падшую, и то, что в ней разрушено, воссоздам, и исправлю ее,
\vs Act 15:17 чтобы взыскали Господа прочие человеки и все народы, между которыми возвестится имя Мое, говорит Господь, творящий все сие.
\vs Act 15:18 Ведомы Богу от вечности все дела Его.
\vs Act 15:19 Посему я полагаю не затруднять обращающихся к Богу из язычников,
\vs Act 15:20 а написать им, чтобы они воздерживались от оскверненного идолами, от блуда, удавленины и крови, и чтобы не делали другим того, чего не хотят себе.
\vs Act 15:21 Ибо \bibemph{закон} Моисеев от древних родов по всем городам имеет проповедующих его и читается в синагогах каждую субботу.
\rsbpar\vs Act 15:22 Тогда Апостолы и пресвитеры со всею церковью рассудили, избрав из среды себя мужей, послать их в Антиохию с Павлом и Варнавою, \bibemph{именно}: Иуду, прозываемого Варсавою, и Силу, мужей, начальствующих между братиями,
\vs Act 15:23 написав и вручив им следующее: <<Апостолы и пресвитеры и братия~--- находящимся в Антиохии, Сирии и Киликии братиям из язычников: радоваться.
\vs Act 15:24 Поелику мы услышали, что некоторые, вышедшие от нас, смутили вас \bibemph{своими} речами и поколебали ваши души, говоря, что должно обрезываться и соблюдать закон, чего мы им не поручали,
\vs Act 15:25 то мы, собравшись, единодушно рассудили, избрав мужей, послать их к вам с возлюбленными нашими Варнавою и Павлом,
\vs Act 15:26 человеками, предавшими души свои за имя Господа нашего Иисуса Христа.
\vs Act 15:27 Итак мы послали Иуду и Силу, которые изъяснят вам то же и словесно.
\vs Act 15:28 Ибо угодно Святому Духу и нам не возлагать на вас никакого бремени более, кроме сего необходимого:
\vs Act 15:29 воздерживаться от идоложертвенного и крови, и удавленины, и блуда, и не делать другим того, чего себе не хотите. Соблюдая сие, хорошо сделаете. Будьте здравы>>.
\rsbpar\vs Act 15:30 Итак, отправленные пришли в Антиохию и, собрав людей, вручили письмо.
\vs Act 15:31 Они же, прочитав, возрадовались о сем наставлении.
\vs Act 15:32 Иуда и Сила, будучи также пророками, обильным словом преподали наставление братиям и утвердили их.
\vs Act 15:33 Пробыв там \bibemph{некоторое} время, они с миром отпущены были братиями к Апостолам.
\vs Act 15:34 Но Силе рассудилось остаться там. (А Иуда возвратился в Иерусалим.)
\vs Act 15:35 Павел же и Варнава жили в Антиохии, уча и благовествуя, вместе с другими многими, слово Господне.
\rsbpar\vs Act 15:36 По некотором времени Павел сказал Варнаве: пойдем опять, посетим братьев наших по всем городам, в которых мы проповедали слово Господне, как они живут.
\vs Act 15:37 Варнава хотел взять с собою Иоанна, называемого Марком.
\vs Act 15:38 Но Павел полагал не брать отставшего от них в Памфилии и не шедшего с ними на дело, на которое они были посланы.
\vs Act 15:39 Отсюда произошло огорчение, так что они разлучились друг с другом; и Варнава, взяв Марка, отплыл в Кипр;
\vs Act 15:40 а Павел, избрав себе Силу, отправился, быв поручен братиями благодати Божией,
\vs Act 15:41 и проходил Сирию и Киликию, утверждая церкви.
\vs Act 16:1 Дошел он до Дервии и Листры. И вот, там был некоторый ученик, именем Тимофей, которого мать была Иудеянка уверовавшая, а отец Еллин,
\vs Act 16:2 и о котором свидетельствовали братия, находившиеся в Листре и Иконии.
\vs Act 16:3 Его пожелал Павел взять с собою; и, взяв, обрезал его ради Иудеев, находившихся в тех местах; ибо все знали об отце его, что он был Еллин.
\vs Act 16:4 Проходя же по городам, они предавали \bibemph{верным} соблюдать определения, постановленные Апостолами и пресвитерами в Иерусалиме.
\vs Act 16:5 И церкви утверждались верою и ежедневно увеличивались числом.
\rsbpar\vs Act 16:6 Пройдя через Фригию и Галатийскую страну, они не были допущены Духом Святым проповедовать слово в Асии.
\vs Act 16:7 Дойдя до Мисии, предпринимали идти в Вифинию; но Дух не допустил их.
\vs Act 16:8 Миновав же Мисию, сошли они в Троаду.
\vs Act 16:9 И было ночью видение Павлу: предстал некий муж, Македонянин, прося его и говоря: приди в Македонию и помоги нам.
\vs Act 16:10 После сего видения, тотчас мы положили отправиться в Македонию, заключая, что призывал нас Господь благовествовать там.
\rsbpar\vs Act 16:11 Итак, отправившись из Троады, мы прямо прибыли в Самофракию, а на другой день в Неаполь,
\vs Act 16:12 оттуда же в Филиппы: это первый город в той части Македонии, колония. В этом городе мы пробыли несколько дней.
\vs Act 16:13 В день же субботний мы вышли за город к реке, где, по обыкновению, был молитвенный дом, и, сев, разговаривали с собравшимися \bibemph{там} женщинами.
\vs Act 16:14 И одна женщина из города Фиатир, именем Лидия, торговавшая багряницею, чтущая Бога, слушала; и Господь отверз сердце ее внимать тому, что говорил Павел.
\vs Act 16:15 Когда же крестилась она и домашние ее, то просила нас, говоря: если вы признали меня верною Господу, то войдите в дом мой и живите \bibemph{у меня}. И убедила нас.
\rsbpar\vs Act 16:16 Случилось, что, когда мы шли в молитвенный дом, встретилась нам одна служанка, одержимая духом прорицательным, которая через прорицание доставляла большой доход господам своим.
\vs Act 16:17 Идя за Павлом и за нами, она кричала, говоря: сии человеки~--- рабы Бога Всевышнего, которые возвещают нам путь спасения.
\vs Act 16:18 Это она делала много дней. Павел, вознегодовав, обратился и сказал духу: именем Иисуса Христа повелеваю тебе выйти из нее. И \bibemph{дух} вышел в тот же час.
\vs Act 16:19 Тогда господа ее, видя, что исчезла надежда дохода их, схватили Павла и Силу и повлекли на площадь к начальникам.
\vs Act 16:20 И, приведя их к воеводам, сказали: сии люди, будучи Иудеями, возмущают наш город
\vs Act 16:21 и проповедуют обычаи, которых нам, Римлянам, не следует ни принимать, ни исполнять.
\vs Act 16:22 Народ также восстал на них, а воеводы, сорвав с них одежды, велели бить их палками
\vs Act 16:23 и, дав им много ударов, ввергли в темницу, приказав темничному стражу крепко стеречь их.
\vs Act 16:24 Получив такое приказание, он ввергнул их во внутреннюю темницу и ноги их забил в колоду.
\rsbpar\vs Act 16:25 Около полуночи Павел и Сила, молясь, воспевали Бога; узники же слушали их.
\vs Act 16:26 Вдруг сделалось великое землетрясение, так что поколебалось основание темницы; тотчас отворились все двери, и у всех узы ослабели.
\vs Act 16:27 Темничный же страж, пробудившись и увидев, что двери темницы отворены, извлек меч и хотел умертвить себя, думая, что узники убежали.
\vs Act 16:28 Но Павел возгласил громким голосом, говоря: не делай себе никакого зла, ибо все мы здесь.
\vs Act 16:29 Он потребовал огня, вбежал \bibemph{в темницу} и в трепете припал к Павлу и Силе,
\vs Act 16:30 и, выведя их вон, сказал: государи \bibemph{мои}! что мне делать, чтобы спастись?
\vs Act 16:31 Они же сказали: веруй в Господа Иисуса Христа, и спасешься ты и весь дом твой.
\vs Act 16:32 И проповедали слово Господне ему и всем, бывшим в доме его.
\vs Act 16:33 И, взяв их в тот час ночи, он омыл раны их и немедленно крестился сам и все \bibemph{домашние} его.
\vs Act 16:34 И, приведя их в дом свой, предложил трапезу и возрадовался со всем домом своим, что уверовал в Бога.
\rsbpar\vs Act 16:35 Когда же настал день, воеводы послали городских служителей сказать: отпусти тех людей.
\vs Act 16:36 Темничный страж объявил о сем Павлу: воеводы прислали отпустить вас; итак выйдите теперь и идите с миром.
\vs Act 16:37 Но Павел сказал к ним: нас, Римских граждан, без суда всенародно били и бросили в темницу, а теперь тайно выпускают? нет, пусть придут и сами выведут нас.
\vs Act 16:38 Городские служители пересказали эти слова воеводам, и те испугались, услышав, что это Римские граждане.
\vs Act 16:39 И, придя, извинились перед ними и, выведя, просили удалиться из города.
\vs Act 16:40 Они же, выйдя из темницы, пришли к Лидии и, увидев братьев, поучали их, и отправились.
\vs Act 17:1 Пройдя через Амфиполь и Аполлонию, они пришли в Фессалонику, где была Иудейская синагога.
\vs Act 17:2 Павел, по своему обыкновению, вошел к ним и три субботы говорил с ними из Писаний,
\vs Act 17:3 открывая и доказывая им, что Христу надлежало пострадать и воскреснуть из мертвых и что Сей Христос есть Иисус, Которого я проповедую вам.
\vs Act 17:4 И некоторые из них уверовали и присоединились к Павлу и Силе, как из Еллинов, чтущих \bibemph{Бога}, великое множество, так и из знатных женщин немало.
\vs Act 17:5 Но неуверовавшие Иудеи, возревновав и взяв с площади некоторых негодных людей, собрались толпою и возмущали город и, приступив к дому Иасона, домогались вывести их к народу.
\vs Act 17:6 Не найдя же их, повлекли Иасона и некоторых братьев к городским начальникам, крича, что эти всесветные возмутители пришли и сюда,
\vs Act 17:7 а Иасон принял их, и все они поступают против повелений кесаря, почитая другого царем, Иисуса.
\vs Act 17:8 И встревожили народ и городских начальников, слушавших это.
\vs Act 17:9 Но \bibemph{сии}, получив удостоверение от Иасона и прочих, отпустили их.
\rsbpar\vs Act 17:10 Братия же немедленно ночью отправили Павла и Силу в Верию, куда они прибыв, пошли в синагогу Иудейскую.
\vs Act 17:11 Здешние были благомысленнее Фессалоникских: они приняли слово со всем усердием, ежедневно разбирая Писания, точно ли это так.
\vs Act 17:12 И многие из них уверовали, и из Еллинских почетных женщин и из мужчин немало.
\vs Act 17:13 Но когда Фессалоникские Иудеи узнали, что и в Верии проповедано Павлом слово Божие, то пришли и туда, возбуждая и возмущая народ.
\vs Act 17:14 Тогда братия тотчас отпустили Павла, как будто идущего к морю; а Сила и Тимофей остались там.
\vs Act 17:15 Сопровождавшие Павла проводили его до Афин и, получив приказание к Силе и Тимофею, чтобы они скорее пришли к нему, отправились.
\rsbpar\vs Act 17:16 В ожидании их в Афинах Павел возмутился духом при виде этого города, полного идолов.
\vs Act 17:17 Итак он рассуждал в синагоге с Иудеями и с чтущими \bibemph{Бога}, и ежедневно на площади со встречающимися.
\vs Act 17:18 Некоторые из эпикурейских и стоических философов стали спорить с ним; и одни говорили: <<чт\acc{о} хочет сказать этот суеслов?>>, а другие: <<кажется, он проповедует о чужих божествах>>, потому что он благовествовал им Иисуса и воскресение.
\vs Act 17:19 И, взяв его, привели в ареопаг и говорили: можем ли мы знать, что это за новое учение, проповедуемое тобою?
\vs Act 17:20 Ибо что-то странное ты влагаешь в уши наши. Посему хотим знать, чт\acc{о} это такое?
\vs Act 17:21 Афиняне же все и живущие \bibemph{у них} иностранцы ни в чем охотнее не проводили время, как в том, чтобы говорить или слушать что-нибудь новое.
\rsbpar\vs Act 17:22 И, став Павел среди ареопага, сказал: Афиняне! по всему вижу я, что вы как бы особенно набожны.
\vs Act 17:23 Ибо, проходя и осматривая ваши святыни, я нашел и жертвенник, на котором написано <<неведомому Богу>>. Сего-то, Которого вы, не зная, чтите, я проповедую вам.
\vs Act 17:24 Бог, сотворивший мир и всё, что в нем, Он, будучи Господом неба и земли, не в рукотворенных храмах живет
\vs Act 17:25 и не требует служения рук человеческих, \bibemph{как бы} имеющий в чем-либо нужду, Сам дая всему жизнь и дыхание и всё.
\vs Act 17:26 От одной крови Он произвел весь род человеческий для обитания по всему лицу земли, назначив предопределенные времена и пределы их обитанию,
\vs Act 17:27 дабы они искали Бога, не ощутят ли Его и не найдут ли, хотя Он и недалеко от каждого из нас:
\vs Act 17:28 ибо мы Им живем и движемся и существуем, как и некоторые из ваших стихотворцев говорили: <<мы Его и род>>.
\vs Act 17:29 Итак мы, будучи родом Божиим, не должны думать, что Божество подобно золоту, или серебру, или камню, получившему образ от искусства и вымысла человеческого.
\vs Act 17:30 Итак, оставляя времена неведения, Бог ныне повелевает людям всем повсюду покаяться,
\vs Act 17:31 ибо Он назначил день, в который будет праведно судить вселенную, посредством предопределенного Им Мужа, подав удостоверение всем, воскресив Его из мертвых.
\vs Act 17:32 Услышав о воскресении мертвых, одни насмехались, а другие говорили: об этом послушаем тебя в другое время.
\vs Act 17:33 Итак Павел вышел из среды их.
\vs Act 17:34 Некоторые же мужи, пристав к нему, уверовали; между ними был Дионисий Ареопагит и женщина, именем Дамарь, и другие с ними.
\vs Act 18:1 После сего Павел, оставив Афины, пришел в Коринф;
\vs Act 18:2 и, найдя некоторого Иудея, именем Акилу, родом Понтянина, недавно пришедшего из Италии, и Прискиллу, жену его,~--- потому что Клавдий повелел всем Иудеям удалиться из Рима,~--- пришел к ним;
\vs Act 18:3 и, по одинаковости ремесла, остался у них и работал; ибо ремеслом их было делание палаток.
\vs Act 18:4 Во всякую же субботу он говорил в синагоге и убеждал Иудеев и Еллинов.
\rsbpar\vs Act 18:5 А когда пришли из Македонии Сила и Тимофей, то Павел понуждаем был духом свидетельствовать Иудеям, что Иисус есть Христос.
\vs Act 18:6 Но как они противились и злословили, то он, отрясши одежды свои, сказал к ним: кровь ваша на главах ваших; я чист; отныне иду к язычникам.
\vs Act 18:7 И пошел оттуда, и пришел к некоторому чтущему Бога, именем Иусту, которого дом был подле синагоги.
\vs Act 18:8 Крисп же, начальник синагоги, уверовал в Господа со всем домом своим, и многие из Коринфян, слушая, уверовали и крестились.
\vs Act 18:9 Господь же в видении ночью сказал Павлу: не бойся, но говори и не умолкай,
\vs Act 18:10 ибо Я с тобою, и никто не сделает тебе зла, потому что у Меня много людей в этом городе.
\vs Act 18:11 И он оставался там год и шесть месяцев, поучая их слову Божию.
\rsbpar\vs Act 18:12 Между тем, во время проконсульства Галлиона в Ахаии, напали Иудеи единодушно на Павла и привели его пред судилище,
\vs Act 18:13 говоря, что он учит людей чтить Бога не по закону.
\vs Act 18:14 Когда же Павел хотел открыть уста, Галлион сказал Иудеям: Иудеи! если бы какая-нибудь была обида или злой умысел, то я имел бы причину выслушать вас,
\vs Act 18:15 но когда идет спор об учении и об именах и о законе вашем, то разбирайте сами; я не хочу быть судьею в этом.
\vs Act 18:16 И прогнал их от судилища.
\vs Act 18:17 А все Еллины, схватив Сосфена, начальника синагоги, били его перед судилищем; и Галлион нимало не беспокоился о том.
\rsbpar\vs Act 18:18 Павел, пробыв еще довольно дней, простился с братиями и отплыл в Сирию,~--- и с ним Акила и Прискилла,~--- остригши голову в Кенхреях, по обету.
\vs Act 18:19 Достигнув Ефеса, оставил их там, а сам вошел в синагогу и рассуждал с Иудеями.
\vs Act 18:20 Когда же они просили его побыть у них долее, он не согласился,
\vs Act 18:21 а простился с ними, сказав: мне нужно непременно провести приближающийся праздник в Иерусалиме; к вам же возвращусь опять, если будет угодно Богу. И отправился из Ефеса. (Акила же и Прискилла остались в Ефесе.)
\vs Act 18:22 Побывав в Кесарии, он приходил \bibemph{в Иерусалим}, приветствовал церковь и отошел в Антиохию.
\vs Act 18:23 И, проведя \bibemph{там} несколько времени, вышел, и проходил по порядку страну Галатийскую и Фригию, утверждая всех учеников.
\rsbpar\vs Act 18:24 Некто Иудей, именем Аполлос, родом из Александрии, муж красноречивый и сведущий в Писаниях, пришел в Ефес.
\vs Act 18:25 Он был наставлен в начатках пути Господня и, горя духом, говорил и учил о Господе правильно, зная только крещение Иоанново.
\vs Act 18:26 Он начал смело говорить в синагоге. Услышав его, Акила и Прискилла приняли его и точнее объяснили ему путь Господень.
\vs Act 18:27 А когда он вознамерился идти в Ахаию, то братия послали к \bibemph{тамошним} ученикам, располагая их принять его; и он, прибыв туда, много содействовал уверовавшим благодатью,
\vs Act 18:28 ибо он сильно опровергал Иудеев всенародно, доказывая Писаниями, что Иисус есть Христос.
\vs Act 19:1 Во время пребывания Аполлоса в Коринфе Павел, пройдя верхние страны, прибыл в Ефес и, найдя \bibemph{там} некоторых учеников,
\vs Act 19:2 сказал им: приняли ли вы Святаго Духа, уверовав? Они же сказали ему: мы даже и не слыхали, есть ли Дух Святый.
\vs Act 19:3 Он сказал им: во что же вы крестились? Они отвечали: во Иоанново крещение.
\vs Act 19:4 Павел сказал: Иоанн крестил крещением покаяния, говоря людям, чтобы веровали в Грядущего по нем, то есть во Христа Иисуса.
\vs Act 19:5 Услышав это, они крестились во имя Господа Иисуса,
\vs Act 19:6 и, когда Павел возложил на них руки, нисшел на них Дух Святый, и они стали говорить \bibemph{иными} языками и пророчествовать.
\vs Act 19:7 Всех их было человек около двенадцати.
\rsbpar\vs Act 19:8 Придя в синагогу, он небоязненно проповедовал три месяца, беседуя и удостоверяя о Царствии Божием.
\vs Act 19:9 Но как некоторые ожесточились и не верили, злословя путь Господень перед народом, то он, оставив их, отделил учеников, и ежедневно проповедовал в училище некоего Тиранна.
\vs Act 19:10 Это продолжалось до двух лет, так что все жители Асии слышали проповедь о Господе Иисусе, как Иудеи, так и Еллины.
\rsbpar\vs Act 19:11 Бог же творил немало чудес руками Павла,
\vs Act 19:12 так что на больных возлагали платки и опоясания с тела его, и у них прекращались болезни, и злые духи выходили из них.
\vs Act 19:13 Даже некоторые из скитающихся Иудейских заклинателей стали употреблять над имеющими злых духов имя Господа Иисуса, говоря: заклинаем вас Иисусом, Которого Павел проповедует.
\vs Act 19:14 Это делали какие-то семь сынов Иудейского первосвященника Скевы.
\vs Act 19:15 Но злой дух сказал в ответ: Иисуса знаю, и Павел мне известен, а вы кто?
\vs Act 19:16 И бросился на них человек, в котором был злой дух, и, одолев их, взял над ними такую силу, что они, нагие и избитые, выбежали из того дома.
\vs Act 19:17 Это сделалось известно всем живущим в Ефесе Иудеям и Еллинам, и напал страх на всех их, и величаемо было имя Господа Иисуса.
\vs Act 19:18 Многие же из уверовавших приходили, исповедуя и открывая дела свои.
\vs Act 19:19 А из занимавшихся чародейством довольно многие, собрав книги свои, сожгли перед всеми, и сложили цены их, и оказалось их на пятьдесят тысяч \bibemph{драхм}.
\vs Act 19:20 С такою силою возрастало и возмогало слово Господне.
\rsbpar\vs Act 19:21 Когда же это совершилось, Павел положил в духе, пройдя Македонию и Ахаию, идти в Иерусалим, сказав: побывав там, я должен видеть и Рим.
\vs Act 19:22 И, послав в Македонию двоих из служивших ему, Тимофея и Ераста, сам остался на время в Асии.
\rsbpar\vs Act 19:23 В то время произошел немалый мятеж против пути Господня,
\vs Act 19:24 ибо некто серебряник, именем Димитрий, делавший серебряные храмы Артемиды и доставлявший художникам немалую прибыль,
\vs Act 19:25 собрав их и других подобных ремесленников, сказал: друзья! вы знаете, что от этого ремесла зависит благосостояние наше;
\vs Act 19:26 между тем вы видите и слышите, что не только в Ефесе, но почти во всей Асии этот Павел своими убеждениями совратил немалое число людей, говоря, что делаемые руками человеческими не суть боги.
\vs Act 19:27 А это нам угрожает тем, что не только ремесло наше придет в презрение, но и храм великой богини Артемиды ничего не будет значить, и испровергнется величие той, которую почитает вся Асия и вселенная.
\vs Act 19:28 Выслушав это, они исполнились ярости и стали кричать, говоря: велика Артемида Ефесская!
\vs Act 19:29 И весь город наполнился смятением. Схватив Македонян Гаия и Аристарха, спутников Павловых, они единодушно устремились на зрелище.
\vs Act 19:30 Когда же Павел хотел войти в народ, ученики не допустили его.
\vs Act 19:31 Также и некоторые из Асийских начальников, будучи друзьями его, послав к нему, просили не показываться на зрелище.
\vs Act 19:32 Между тем одни кричали одно, а другие другое, ибо собрание было беспорядочное, и большая часть \bibemph{собравшихся} не знали, зачем собрались.
\vs Act 19:33 По предложению Иудеев, из народа вызван был Александр. Дав знак рукою, Александр хотел говорить к народу.
\vs Act 19:34 Когда же узнали, что он Иудей, то закричали все в один голос, и около двух часов кричали: велика Артемида Ефесская!
\vs Act 19:35 Блюститель же порядка, утишив народ, сказал: мужи Ефесские! какой человек не знает, что город Ефес есть служитель великой богини Артемиды и Диопета?
\vs Act 19:36 Если же в этом нет спора, то надобно вам быть спокойными и не поступать опрометчиво.
\vs Act 19:37 А вы привели этих мужей, которые ни храма Артемидина не обокрали, ни богини вашей не хулили.
\vs Act 19:38 Если же Димитрий и другие с ним художники имеют жалобу на кого-нибудь, то есть судебные собрания и есть проконсулы: пусть жалуются друг на друга.
\vs Act 19:39 А если вы ищете чего-нибудь другого, то это будет решено в законном собрании.
\vs Act 19:40 Ибо мы находимся в опасности~--- за происшедшее ныне быть обвиненными в возмущении, так как нет никакой причины, которою мы могли бы оправдать такое сборище. Сказав это, он распустил собрание.
\vs Act 20:1 По прекращении мятежа Павел, призвав учеников и дав им наставления и простившись с ними, вышел и пошел в Македонию.
\vs Act 20:2 Пройдя же те места и преподав \bibemph{верующим} обильные наставления, пришел в Елладу.
\vs Act 20:3 \bibemph{Там} пробыл он три месяца. Когда же, по случаю возмущения, сделанного против него Иудеями, он хотел отправиться в Сирию, то пришло ему на мысль возвратиться через Македонию.
\vs Act 20:4 Его сопровождали до Асии Сосипатр Пирров, Вериянин, и из Фессалоникийцев Аристарх и Секунд, и Гаий Дервянин и Тимофей, и Асийцы Тихик и Трофим.
\vs Act 20:5 Они, пойдя вперед, ожидали нас в Троаде.
\vs Act 20:6 А мы, после дней опресночных, отплыли из Филипп и дней в пять прибыли к ним в Троаду, где пробыли семь дней.
\rsbpar\vs Act 20:7 В первый же день недели, когда ученики собрались для преломления хлеба, Павел, намереваясь отправиться в следующий день, беседовал с ними и продолжил слово до полуночи.
\vs Act 20:8 В горнице, где мы собрались, было довольно светильников.
\vs Act 20:9 Во время продолжительной беседы Павловой один юноша, именем Евтих, сидевший на окне, погрузился в глубокий сон и, пошатнувшись, сонный упал вниз с третьего жилья, и поднят мертвым.
\vs Act 20:10 Павел, сойдя, пал на него и, обняв его, сказал: не тревожьтесь, ибо душа его в нем.
\vs Act 20:11 Взойдя же и преломив хлеб и вкусив, беседовал довольно, даже до рассвета, и потом вышел.
\vs Act 20:12 Между тем отрока привели живого, и немало утешились.
\rsbpar\vs Act 20:13 Мы пошли вперед на корабль и поплыли в Асс, чтобы взять оттуда Павла; ибо он так приказал нам, намереваясь сам идти пешком.
\vs Act 20:14 Когда же он сошелся с нами в Ассе, то, взяв его, мы прибыли в Митилину.
\vs Act 20:15 И, отплыв оттуда, в следующий день мы остановились против Хиоса, а на другой пристали к Самосу и, побывав в Трогиллии, в следующий \bibemph{день} прибыли в Милит,
\vs Act 20:16 ибо Павлу рассудилось миновать Ефес, чтобы не замедлить ему в Асии; потому что он поспешал, если можно, в день Пятидесятницы быть в Иерусалиме.
\rsbpar\vs Act 20:17 Из Милита же послав в Ефес, он призвал пресвитеров церкви,
\vs Act 20:18 и, когда они пришли к нему, он сказал им: вы знаете, как я с первого дня, в который пришел в Асию, все время был с вами,
\vs Act 20:19 работая Господу со всяким смиренномудрием и многими слезами, среди искушений, приключавшихся мне по злоумышлениям Иудеев;
\vs Act 20:20 как я не пропустил ничего полезного, о чем вам не проповедовал бы и чему не учил бы вас всенародно и по домам,
\vs Act 20:21 возвещая Иудеям и Еллинам покаяние пред Богом и веру в Господа нашего Иисуса Христа.
\vs Act 20:22 И вот, ныне я, по влечению Духа, иду в Иерусалим, не зная, чт\acc{о} там встретится со мною;
\vs Act 20:23 только Дух Святый по всем городам свидетельствует, говоря, что узы и скорби ждут меня.
\vs Act 20:24 Но я ни на что не взираю и не дорожу своею жизнью, только бы с радостью совершить поприще мое и служение, которое я принял от Господа Иисуса, проповедать Евангелие благодати Божией.
\vs Act 20:25 И ныне, вот, я знаю, что уже не увидите лица моего все вы, между которыми ходил я, проповедуя Царствие Божие.
\vs Act 20:26 Посему свидетельствую вам в нынешний день, что чист я от крови всех,
\vs Act 20:27 ибо я не упускал возвещать вам всю волю Божию.
\vs Act 20:28 Итак внимайте себе и всему стаду, в котором Дух Святый поставил вас блюстителями, пасти Церковь Господа и Бога, которую Он приобрел Себе Кровию Своею.
\vs Act 20:29 Ибо я знаю, что, по отшествии моем, войдут к вам лютые волки, не щадящие стада;
\vs Act 20:30 и из вас самих восстанут люди, которые будут говорить превратно, дабы увлечь учеников за собою.
\vs Act 20:31 Посему бодрствуйте, памятуя, что я три года день и ночь непрестанно со слезами учил каждого из вас.
\vs Act 20:32 И ныне предаю вас, братия, Богу и слову благодати Его, могущему назидать \bibemph{вас} более и дать вам наследие со всеми освященными.
\vs Act 20:33 Ни серебра, ни золота, ни одежды я ни от кого не пожелал:
\vs Act 20:34 сами знаете, что нуждам моим и \bibemph{нуждам} бывших при мне послужили руки мои сии.
\vs Act 20:35 Во всем показал я вам, что, так трудясь, надобно поддерживать слабых и памятовать слова Господа Иисуса, ибо Он Сам сказал: <<блаженнее давать, нежели принимать>>.
\vs Act 20:36 Сказав это, он преклонил колени свои и со всеми ими помолился.
\vs Act 20:37 Тогда немалый плач был у всех, и, падая на выю Павла, целовали его,
\vs Act 20:38 скорбя особенно от сказанного им слова, что они уже не увидят лица его. И провожали его до корабля.
\vs Act 21:1 Когда же мы, расставшись с ними, отплыли, то прямо пришли в Кос, на другой день в Родос и оттуда в Патару,
\vs Act 21:2 и, найдя корабль, идущий в Финикию, взошли на него и отплыли.
\vs Act 21:3 Быв в виду Кипра и оставив его слева, мы плыли в Сирию, и пристали в Тире, ибо тут надлежало сложить груз с корабля.
\vs Act 21:4 И, найдя учеников, пробыли там семь дней. Они, по \bibemph{внушению} Духа, говорили Павлу, чтобы он не ходил в Иерусалим.
\vs Act 21:5 Проведя эти дни, мы вышли и пошли, и нас провожали все с женами и детьми даже за город; а на берегу, преклонив колени, помолились.
\vs Act 21:6 И, простившись друг с другом, мы вошли в корабль, а они возвратились домой.
\rsbpar\vs Act 21:7 Мы же, совершив плавание, прибыли из Тира в Птолемаиду, где, приветствовав братьев, пробыли у них один день.
\vs Act 21:8 А на другой день Павел и мы, бывшие с ним, выйдя, пришли в Кесарию и, войдя в дом Филиппа благовестника, одного из семи \bibemph{диаконов}, остались у него.
\vs Act 21:9 У него были четыре дочери девицы, пророчествующие.
\vs Act 21:10 Между тем как мы пребывали у них многие дни, пришел из Иудеи некто пророк, именем Агав,
\vs Act 21:11 и, войдя к нам, взял пояс Павлов и, связав себе руки и ноги, сказал: так говорит Дух Святый: мужа, чей этот пояс, так свяжут в Иерусалиме Иудеи и предадут в руки язычников.
\vs Act 21:12 Когда же мы услышали это, то и мы и тамошние просили, чтобы он не ходил в Иерусалим.
\vs Act 21:13 Но Павел в ответ сказал: что вы делаете? что плачете и сокрушаете сердце мое? я не только хочу быть узником, но готов умереть в Иерусалиме за имя Господа Иисуса.
\vs Act 21:14 Когда же мы не могли уговорить его, то успокоились, сказав: да будет воля Господня!
\rsbpar\vs Act 21:15 После сих дней, приготовившись, пошли мы в Иерусалим.
\vs Act 21:16 С нами шли и некоторые ученики из Кесарии, провожая \bibemph{нас} к некоему давнему ученику, Мнасону Кипрянину, у которого можно было бы нам жить.
\rsbpar\vs Act 21:17 По прибытии нашем в Иерусалим братия радушно приняли нас.
\vs Act 21:18 На другой день Павел пришел с нами к Иакову; пришли и все пресвитеры.
\vs Act 21:19 Приветствовав их, \bibemph{Павел} рассказывал подробно, что сотворил Бог у язычников служением его.
\vs Act 21:20 Они же, выслушав, прославили Бога и сказали ему: видишь, брат, сколько тысяч уверовавших Иудеев, и все они ревнители закона.
\vs Act 21:21 А о тебе наслышались они, что ты всех Иудеев, живущих между язычниками, учишь отступлению от Моисея, говоря, чтобы они не обрезывали детей своих и не поступали по обычаям.
\vs Act 21:22 Итак что же? Верно соберется народ; ибо услышат, что ты пришел.
\vs Act 21:23 Сделай же, что мы скажем тебе: есть у нас четыре человека, имеющие на себе обет.
\vs Act 21:24 Взяв их, очистись с ними, и возьми на себя издержки на \bibemph{жертву} за них, чтобы остригли себе голову, и узнают все, что слышанное ими о тебе несправедливо, но что и сам ты продолжаешь соблюдать закон.
\vs Act 21:25 А об уверовавших язычниках мы писали, положив, чтобы они ничего такого не наблюдали, а только хранили себя от идоложертвенного, от крови, от удавленины и от блуда.
\vs Act 21:26 Тогда Павел, взяв тех мужей и очистившись с ними, в следующий день вошел в храм и объявил окончание дней очищения, когда должно быть принесено за каждого из них приношение.
\rsbpar\vs Act 21:27 Когда же семь дней оканчивались, тогда Асийские Иудеи, увидев его в храме, возмутили весь народ и наложили на него руки,
\vs Act 21:28 крича: мужи Израильские, помогите! этот человек всех повсюду учит против народа и закона и места сего; притом и Еллинов ввел в храм и осквернил святое место сие.
\vs Act 21:29 Ибо перед тем они видели с ним в городе Трофима Ефесянина и думали, что Павел его ввел в храм.
\vs Act 21:30 Весь город пришел в движение, и сделалось стечение народа; и, схватив Павла, повлекли его вон из храма, и тотчас заперты были двери.
\vs Act 21:31 Когда же они хотели убить его, до тысяченачальника полка дошла весть, что весь Иерусалим возмутился.
\vs Act 21:32 Он, тотчас взяв воинов и сотников, устремился на них; они же, увидев тысяченачальника и воинов, перестали бить Павла.
\vs Act 21:33 Тогда тысяченачальник, приблизившись, взял его и велел сковать двумя цепями, и спрашивал: кто он, и что сделал.
\vs Act 21:34 В народе одни кричали одно, а другие другое. Он же, не могши по причине смятения узнать ничего верного, повелел вести его в крепость.
\vs Act 21:35 Когда же он был на лестнице, то воинам пришлось нести его по причине стеснения от народа,
\vs Act 21:36 ибо множество народа следовало и кричало: смерть ему!
\rsbpar\vs Act 21:37 При входе в крепость Павел сказал тысяченачальнику: можно ли мне сказать тебе нечто? А тот сказал: ты знаешь по-гречески?
\vs Act 21:38 Так не ты ли тот Египтянин, который перед сими днями произвел возмущение и вывел в пустыню четыре тысячи человек разбойников?
\vs Act 21:39 Павел же сказал: я Иудеянин, Тарсянин, гражданин небезызвестного Киликийского города; прошу тебя, позволь мне говорить к народу.
\vs Act 21:40 Когда же тот позволил, Павел, стоя на лестнице, дал знак рукою народу; и, когда сделалось глубокое молчание, начал говорить на еврейском языке так:
\vs Act 22:1 Мужи братия и отцы! выслушайте теперь мое оправдание перед вами.
\vs Act 22:2 Услышав же, что он заговорил с ними на еврейском языке, они еще более утихли. Он сказал:
\vs Act 22:3 я Иудеянин, родившийся в Тарсе Киликийском, воспитанный в сем городе при ногах Гамалиила, тщательно наставленный в отеческом законе, ревнитель по Боге, как и все вы ныне.
\vs Act 22:4 Я даже до смерти гнал \bibemph{последователей} сего учения, связывая и предавая в темницу и мужчин и женщин,
\vs Act 22:5 как засвидетельствует о мне первосвященник и все старейшины, от которых и письма взяв к братиям, живущим в Дамаске, я шел, чтобы тамошних привести в оковах в Иерусалим на истязание.
\vs Act 22:6 Когда же я был в пути и приближался к Дамаску, около полудня вдруг осиял меня великий свет с неба.
\vs Act 22:7 Я упал на землю и услышал голос, говоривший мне: Савл, Савл! что ты гонишь Меня?
\vs Act 22:8 Я отвечал: кто Ты, Господи? Он сказал мне: Я Иисус Назорей, Которого ты гонишь.
\vs Act 22:9 Бывшие же со мною свет видели, и пришли в страх; но голоса Говорившего мне не слыхали.
\vs Act 22:10 Тогда я сказал: Господи! что мне делать? Господь же сказал мне: встань и иди в Дамаск, и там тебе сказано будет всё, что назначено тебе делать.
\vs Act 22:11 А как я от славы света того лишился зрения, то бывшие со мною за руку привели меня в Дамаск.
\vs Act 22:12 Некто Анания, муж благочестивый по закону, одобряемый всеми Иудеями, живущими в Дамаске,
\vs Act 22:13 пришел ко мне и, подойдя, сказал мне: брат Савл! прозри. И я тотчас увидел его.
\vs Act 22:14 Он же сказал мне: Бог отцов наших предъизбрал тебя, чтобы ты познал волю Его, увидел Праведника и услышал глас из уст Его,
\vs Act 22:15 потому что ты будешь Ему свидетелем пред всеми людьми о том, что ты видел и слышал.
\vs Act 22:16 Итак, что ты медлишь? Встань, крестись и омой грехи твои, призвав имя Господа Иисуса.
\vs Act 22:17 Когда же я возвратился в Иерусалим и молился в храме, пришел я в исступление,
\vs Act 22:18 и увидел Его, и Он сказал мне: поспеши и выйди скорее из Иерусалима, потому что \bibemph{здесь} не примут твоего свидетельства о Мне.
\vs Act 22:19 Я сказал: Господи! им известно, что я верующих в Тебя заключал в темницы и бил в синагогах,
\vs Act 22:20 и когда проливалась кровь Стефана, свидетеля Твоего, я там стоял, одобрял убиение его и стерег одежды побивавших его.
\vs Act 22:21 И Он сказал мне: иди; Я пошлю тебя далеко к язычникам.
\rsbpar\vs Act 22:22 До этого слова слушали его; а за сим подняли крик, говоря: истреби от земли такого! ибо ему не должно жить.
\vs Act 22:23 Между тем как они кричали, метали одежды и бросали пыль на воздух,
\vs Act 22:24 тысяченачальник повелел ввести его в крепость, приказав бичевать его, чтобы узнать, по какой причине так кричали против него.
\vs Act 22:25 Но когда растянули его ремнями, Павел сказал стоявшему сотнику: разве вам позволено бичевать Римского гражданина, да и без суда?
\vs Act 22:26 Услышав это, сотник подошел и донес тысяченачальнику, говоря: смотри, что ты хочешь делать? этот человек~--- Римский гражданин.
\vs Act 22:27 Тогда тысяченачальник, подойдя к нему, сказал: скажи мне, ты Римский гражданин? Он сказал: да.
\vs Act 22:28 Тысяченачальник отвечал: я за большие деньги приобрел это гражданство. Павел же сказал: а я и родился в нем.
\vs Act 22:29 Тогда тотчас отступили от него хотевшие пытать его. А тысяченачальник, узнав, что он Римский гражданин, испугался, что связал его.
\rsbpar\vs Act 22:30 На другой день, желая достоверно узнать, в чем обвиняют его Иудеи, освободил его от оков и повелел собраться первосвященникам и всему синедриону и, выведя Павла, поставил его перед ними.
\vs Act 23:1 Павел, устремив взор на синедрион, сказал: мужи братия! я всею доброю совестью жил пред Богом до сего дня.
\vs Act 23:2 Первосвященник же Анания стоявшим перед ним приказал бить его по устам.
\vs Act 23:3 Тогда Павел сказал ему: Бог будет бить тебя, стена подбеленная! ты сидишь, чтобы судить по закону, и, вопреки закону, велишь бить меня.
\vs Act 23:4 Предстоящие же сказали: первосвященника Божия поносишь?
\vs Act 23:5 Павел сказал: я не знал, братия, что он первосвященник; ибо написано: начальствующего в народе твоем не злословь.
\vs Act 23:6 Узнав же Павел, что \bibemph{тут} одна часть саддукеев, а другая фарисеев, возгласил в синедрионе: мужи братия! я фарисей, сын фарисея; за чаяние воскресения мертвых меня судят.
\vs Act 23:7 Когда же он сказал это, произошла распря между фарисеями и саддукеями, и собрание разделилось.
\vs Act 23:8 Ибо саддукеи говорят, что нет воскресения, ни Ангела, ни духа; а фарисеи признают и то и другое.
\vs Act 23:9 Сделался большой крик; и, встав, книжники фарисейской стороны спорили, говоря: ничего худого мы не находим в этом человеке; если же дух или Ангел говорил ему, не будем противиться Богу.
\vs Act 23:10 Но как раздор увеличился, то тысяченачальник, опасаясь, чтобы они не растерзали Павла, повелел воинам сойти взять его из среды их и отвести в крепость.
\rsbpar\vs Act 23:11 В следующую ночь Господь, явившись ему, сказал: дерзай, Павел; ибо, как ты свидетельствовал о Мне в Иерусалиме, так надлежит тебе свидетельствовать и в Риме.
\vs Act 23:12 С наступлением дня некоторые Иудеи сделали умысел, и заклялись не есть и не пить, доколе не убьют Павла.
\vs Act 23:13 Было же более сорока сделавших такое заклятие.
\vs Act 23:14 Они, придя к первосвященникам и старейшинам, сказали: мы клятвою заклялись не есть ничего, пока не убьем Павла.
\vs Act 23:15 Итак ныне же вы с синедрионом дайте знать тысяченачальнику, чтобы он завтра вывел его к вам, как будто вы хотите точнее рассмотреть дело о нем; мы же, прежде нежели он приблизится, готовы убить его.
\vs Act 23:16 Услышав о сем умысле, сын сестры Павловой пришел и, войдя в крепость, уведомил Павла.
\vs Act 23:17 Павел же, призвав одного из сотников, сказал: отведи этого юношу к тысяченачальнику, ибо он имеет нечто сказать ему.
\vs Act 23:18 Тот, взяв его, привел к тысяченачальнику и сказал: узник Павел, призвав меня, просил отвести к тебе этого юношу, который имеет нечто сказать тебе.
\vs Act 23:19 Тысяченачальник, взяв его за руку и отойдя с ним в сторону, спрашивал: что такое имеешь ты сказать мне?
\vs Act 23:20 Он отвечал, что Иудеи согласились просить тебя, чтобы ты завтра вывел Павла пред синедрион, как будто они хотят точнее исследовать дело о нем.
\vs Act 23:21 Но ты не слушай их; ибо его подстерегают более сорока человек из них, которые заклялись не есть и не пить, доколе не убьют его; и они теперь готовы, ожидая твоего распоряжения.
\vs Act 23:22 Тогда тысяченачальник отпустил юношу, сказав: никому не говори, что ты объявил мне это.
\rsbpar\vs Act 23:23 И, призвав двух сотников, сказал: приготовьте мне воинов \bibemph{пеших} двести, конных семьдесят и стрелков двести, чтобы с третьего часа ночи шли в Кесарию.
\vs Act 23:24 Приготовьте также ослов, чтобы, посадив Павла, препроводить его к правителю Феликсу.
\vs Act 23:25 Написал и письмо следующего содержания:
\vs Act 23:26 <<Клавдий Лисий достопочтенному правителю Феликсу~--- радоваться.
\vs Act 23:27 Сего человека Иудеи схватили и готовы были убить; я, придя с воинами, отнял его, узнав, что он Римский гражданин.
\vs Act 23:28 Потом, желая узнать, в чем обвиняли его, привел его в синедрион их
\vs Act 23:29 и нашел, что его обвиняют в спорных мнениях, касающихся закона их, но что нет в нем никакой вины, достойной смерти или оков.
\vs Act 23:30 А как до меня дошло, что Иудеи злоумышляют на этого человека, то я немедленно послал его к тебе, приказав и обвинителям говорить на него перед тобою. Будь здоров>>.
\rsbpar\vs Act 23:31 Итак воины, по \bibemph{данному} им приказанию, взяв Павла, повели ночью в Антипатриду.
\vs Act 23:32 А на другой день, предоставив конным идти с ним, возвратились в крепость.
\vs Act 23:33 А те, придя в Кесарию и отдав письмо правителю, представили ему и Павла.
\vs Act 23:34 Правитель, прочитав письмо, спросил, из какой он области, и, узнав, что из Киликии, сказал:
\vs Act 23:35 я выслушаю тебя, когда явятся твои обвинители. И повелел ему быть под стражею в Иродовой претории.
\vs Act 24:1 Через пять дней пришел первосвященник Анания со старейшинами и с некоторым ритором Тертуллом, которые жаловались правителю на Павла.
\vs Act 24:2 Когда же он был призван, то Тертулл начал обвинять его, говоря:
\vs Act 24:3 всегда и везде со всякою благодарностью признаём мы, что тебе, достопочтенный Феликс, обязаны мы многим миром, и твоему попечению благоустроением сего народа.
\vs Act 24:4 Но, чтобы много не утруждать тебя, прошу тебя выслушать нас кратко, со свойственным тебе снисхождением.
\vs Act 24:5 Найдя сего человека язвою \bibemph{общества}, возбудителем мятежа между Иудеями, живущими по вселенной, и представителем Назорейской ереси,
\vs Act 24:6 который отважился даже осквернить храм, мы взяли его и хотели судить его по нашему закону.
\vs Act 24:7 Но тысяченачальник Лисий, придя, с великим насилием взял его из рук наших и послал к тебе,
\vs Act 24:8 повелев и нам, обвинителям его, идти к тебе. Ты можешь сам, разобрав, узнать от него о всем том, в чем мы обвиняем его.
\vs Act 24:9 И Иудеи подтвердили, сказав, что это так.
\rsbpar\vs Act 24:10 Павел же, когда правитель дал ему знак говорить, отвечал: зная, что ты многие годы справедливо судишь народ сей, я тем свободнее буду защищать мое дело.
\vs Act 24:11 Ты можешь узнать, что не более двенадцати дней тому, как я пришел в Иерусалим для поклонения.
\vs Act 24:12 И ни в святилище, ни в синагогах, ни по городу они не находили меня с кем-либо спорящим или производящим народное возмущение,
\vs Act 24:13 и не могут доказать того, в чем теперь обвиняют меня.
\vs Act 24:14 Но в том признаюсь тебе, что по учению, которое они называют ересью, я действительно служу Богу отцов \bibemph{моих}, веруя всему, написанному в законе и пророках,
\vs Act 24:15 имея надежду на Бога, что будет воскресение мертвых, праведных и неправедных, чего и сами они ожидают.
\vs Act 24:16 Посему и сам подвизаюсь всегда иметь непорочную совесть пред Богом и людьми.
\vs Act 24:17 После многих лет я пришел, чтобы доставить милостыню народу моему и приношения.
\vs Act 24:18 При сем нашли меня, очистившегося в храме не с народом и не с шумом.
\vs Act 24:19 \bibemph{Это были} некоторые Асийские Иудеи, которым надлежало бы предстать пред тебя и обвинять меня, если что имеют против меня.
\vs Act 24:20 Или пусть сии самые скажут, какую нашли они во мне неправду, когда я стоял перед синедрионом,
\vs Act 24:21 разве только т\acc{о} одно слово, которое громко произнес я, стоя между ними, что за \bibemph{учение о} воскресении мертвых я ныне судим вами.
\rsbpar\vs Act 24:22 Выслушав это, Феликс отсрочил \bibemph{дело} их, сказав: рассмотрю ваше дело, когда придет тысяченачальник Лисий, и я обстоятельно узнаю об этом учении.
\vs Act 24:23 А Павла приказал сотнику стеречь, но не стеснять его и не запрещать никому из его близких служить ему или приходить к нему.
\rsbpar\vs Act 24:24 Через несколько дней Феликс, придя с Друзиллою, женою своею, Иудеянкою, призвал Павла, и слушал его о вере во Христа Иисуса.
\vs Act 24:25 И как он говорил о правде, о воздержании и о будущем суде, то Феликс пришел в страх и отвечал: теперь пойди, а когда найду время, позову тебя.
\vs Act 24:26 Притом же надеялся он, что Павел даст ему денег, чтобы отпустил его: посему часто призывал его и беседовал с ним.
\vs Act 24:27 Но по прошествии двух лет на место Феликса поступил Порций Фест. Желая доставить удовольствие Иудеям, Феликс оставил Павла в узах.
\vs Act 25:1 Фест, прибыв в область, через три дня отправился из Кесарии в Иерусалим.
\vs Act 25:2 Тогда первосвященник и знатнейшие из Иудеев явились к нему \bibemph{с жалобою} на Павла и убеждали его,
\vs Act 25:3 прося, чтобы он сделал милость, вызвал его в Иерусалим; и злоумышляли убить его на дороге.
\vs Act 25:4 Но Фест отвечал, что Павел содержится в Кесарии под стражею и что он сам скоро отправится туда.
\vs Act 25:5 Итак, сказал он, которые из вас могут, пусть пойдут со мною, и если есть что-нибудь за этим человеком, пусть обвиняют его.
\rsbpar\vs Act 25:6 Пробыв же у них не больше восьми или десяти дней, возвратился в Кесарию, и на другой день, сев на судейское место, повелел привести Павла.
\vs Act 25:7 Когда он явился, стали кругом пришедшие из Иерусалима Иудеи, принося на Павла многие и тяжкие обвинения, которых не могли доказать.
\vs Act 25:8 Он же в оправдание свое сказал: я не сделал никакого преступления ни против закона Иудейского, ни против храма, ни против кесаря.
\vs Act 25:9 Фест, желая сделать угождение Иудеям, сказал в ответ Павлу: хочешь ли идти в Иерусалим, чтобы я там судил тебя в этом?
\vs Act 25:10 Павел сказал: я сто\acc{ю} перед судом кесаревым, где мне и следует быть судиму. Иудеев я ничем не обидел, как и ты хорошо знаешь.
\vs Act 25:11 Ибо, если я неправ и сделал что-нибудь, достойное смерти, то не отрекаюсь умереть; а если ничего того нет, в чем сии обвиняют меня, то никто не может выдать меня им. Требую суда кесарева.
\vs Act 25:12 Тогда Фест, поговорив с советом, отвечал: ты потребовал суда кесарева, к кесарю и отправишься.
\rsbpar\vs Act 25:13 Через несколько дней царь Агриппа и Вереника прибыли в Кесарию поздравить Феста.
\vs Act 25:14 И как они провели там много дней, то Фест предложил царю дело Павлово, говоря: \bibemph{здесь} есть человек, оставленный Феликсом в узах,
\vs Act 25:15 на которого, в бытность мою в Иерусалиме, \bibemph{с жалобою} явились первосвященники и старейшины Иудейские, требуя осуждения его.
\vs Act 25:16 Я отвечал им, что у Римлян нет обыкновения выдавать какого-нибудь человека на смерть, прежде нежели обвиняемый будет иметь обвинителей налицо и получит свободу защищаться против обвинения.
\vs Act 25:17 Когда же они пришли сюда, то, без всякого отлагательства, на другой же день сел я на судейское место и повелел привести того человека.
\vs Act 25:18 Обступив его, обвинители не представили ни одного из обвинений, какие я предполагал;
\vs Act 25:19 но они имели некоторые споры с ним об их Богопочитании и о каком-то Иисусе умершем, о Котором Павел утверждал, что Он жив.
\vs Act 25:20 Затрудняясь в решении этого вопроса, я сказал: хочет ли он идти в Иерусалим и там быть судимым в этом?
\vs Act 25:21 Но как Павел потребовал, чтобы он оставлен был на рассмотрение Августово, то я велел содержать его под стражею до тех пор, как пошлю его к кесарю.
\vs Act 25:22 Агриппа же сказал Фесту: хотел бы и я послушать этого человека. Завтра же, отвечал тот, услышишь его.
\rsbpar\vs Act 25:23 На другой день, когда Агриппа и Вереника пришли с великою пышностью и вошли в судебную палату с тысяченачальниками и знатнейшими гражданами, по приказанию Феста приведен был Павел.
\vs Act 25:24 И сказал Фест: царь Агриппа и все присутствующие с нами мужи! вы видите того, против которого всё множество Иудеев приступали ко мне в Иерусалиме и здесь и кричали, что ему не должно более жить.
\vs Act 25:25 Но я нашел, что он не сделал ничего, достойного смерти; и как он сам потребовал суда у Августа, то я решился послать его \bibemph{к нему}.
\vs Act 25:26 Я не имею ничего верного написать о нем государю; посему привел его пред вас, и особенно пред тебя, царь Агриппа, дабы, по рассмотрении, было мне что написать.
\vs Act 25:27 Ибо, мне кажется, нерассудительно послать узника и не показать обвинений на него.
\vs Act 26:1 Агриппа сказал Павлу: позволяется тебе говорить за себя. Тогда Павел, простерши руку, стал говорить в свою защиту:
\vs Act 26:2 царь Агриппа! почитаю себя счастливым, что сегодня могу защищаться перед тобою во всем, в чем обвиняют меня Иудеи,
\vs Act 26:3 тем более, что ты знаешь все обычаи и спорные мнения Иудеев. Посему прошу тебя выслушать меня великодушно.
\vs Act 26:4 Жизнь мою от юности \bibemph{моей}, которую сначала проводил я среди народа моего в Иерусалиме, знают все Иудеи;
\vs Act 26:5 они издавна знают обо мне, если захотят свидетельствовать, что я жил фарисеем по строжайшему в нашем вероисповедании учению.
\vs Act 26:6 И ныне я сто\acc{ю} перед судом за надежду на обетование, данное от Бога нашим отцам,
\vs Act 26:7 которого исполнение надеются увидеть наши двенадцать колен, усердно служа \bibemph{Богу} день и ночь. За сию-то надежду, царь Агриппа, обвиняют меня Иудеи.
\vs Act 26:8 Что же? Неужели вы невероятным почитаете, что Бог воскрешает мертвых?
\vs Act 26:9 Правда, и я думал, что мне должно много действовать против имени Иисуса Назорея.
\vs Act 26:10 Это я и делал в Иерусалиме: получив власть от первосвященников, я многих святых заключал в темницы, и, когда убивали их, я подавал на то голос;
\vs Act 26:11 и по всем синагогам я многократно мучил их и принуждал хулить \bibemph{Иисуса} и, в чрезмерной против них ярости, преследовал даже и в чужих городах.
\vs Act 26:12 Для сего, идя в Дамаск со властью и поручением от первосвященников,
\vs Act 26:13 среди дня на дороге я увидел, государь, с неба свет, превосходящий солнечное сияние, осиявший меня и шедших со мною.
\vs Act 26:14 Все мы упали на землю, и я услышал голос, говоривший мне на еврейском языке: Савл, Савл! что ты гонишь Меня? Трудно тебе идти против рожна.
\vs Act 26:15 Я сказал: кто Ты, Господи? Он сказал: <<Я Иисус, Которого ты гонишь.
\vs Act 26:16 Но встань и стань на ноги твои; ибо Я для того и явился тебе, чтобы поставить тебя служителем и свидетелем того, что ты видел и что Я открою тебе,
\vs Act 26:17 избавляя тебя от народа Иудейского и от язычников, к которым Я теперь посылаю тебя
\vs Act 26:18 открыть глаза им, чтобы они обратились от тьмы к свету и от власти сатаны к Богу, и верою в Меня получили прощение грехов и жребий с освященными>>.
\vs Act 26:19 Поэтому, царь Агриппа, я не воспротивился небесному видению,
\vs Act 26:20 но сперва жителям Дамаска и Иерусалима, потом всей земле Иудейской и язычникам проповедовал, чтобы они покаялись и обратились к Богу, делая дела, достойные покаяния.
\vs Act 26:21 За это схватили меня Иудеи в храме и покушались растерзать.
\vs Act 26:22 Но, получив помощь от Бога, я до сего дня стою, свидетельствуя малому и великому, ничего не говоря, кроме того, о чем пророки и Моисей говорили, что это будет,
\vs Act 26:23 \bibemph{то есть} что Христос имел пострадать и, восстав первый из мертвых, возвестить свет народу (Иудейскому) и язычникам.
\vs Act 26:24 Когда он так защищался, Фест громким голосом сказал: безумствуешь ты, Павел! большая ученость доводит тебя до сумасшествия.
\vs Act 26:25 Нет, достопочтенный Фест, сказал он, я не безумствую, но говорю слова истины и здравого смысла.
\vs Act 26:26 Ибо знает об этом царь, перед которым и говорю смело. Я отнюдь не верю, чтобы от него было что-нибудь из сего скрыто; ибо это не в углу происходило.
\vs Act 26:27 Веришь ли, царь Агриппа, пророкам? Знаю, что веришь.
\vs Act 26:28 Агриппа сказал Павлу: ты немного не убеждаешь меня сделаться Христианином.
\vs Act 26:29 Павел сказал: молил бы я Бога, чтобы мало ли, много ли, не только ты, но и все, слушающие меня сегодня, сделались такими, как я, кроме этих уз.
\vs Act 26:30 Когда он сказал это, царь и правитель, Вереника и сидевшие с ними встали;
\vs Act 26:31 и, отойдя в сторону, говорили между собою, что этот человек ничего, достойного смерти или уз, не делает.
\vs Act 26:32 И сказал Агриппа Фесту: можно было бы освободить этого человека, если бы он не потребовал суда у кесаря. Посему и решился правитель послать его к кесарю.
\vs Act 27:1 Когда решено было плыть нам в Италию, то отдали Павла и некоторых других узников сотнику Августова полка, именем Юлию.
\vs Act 27:2 Мы взошли на Адрамитский корабль и отправились, намереваясь плыть около Асийских мест. С нами был Аристарх, Македонянин из Фессалоники.
\vs Act 27:3 На другой \bibemph{день} пристали к Сидону. Юлий, поступая с Павлом человеколюбиво, позволил ему сходить к друзьям и воспользоваться их усердием.
\vs Act 27:4 Отправившись оттуда, мы приплыли в Кипр, по причине противных ветров,
\vs Act 27:5 и, переплыв море против Киликии и Памфилии, прибыли в Миры Ликийские.
\vs Act 27:6 Там сотник нашел Александрийский корабль, плывущий в Италию, и посадил нас на него.
\vs Act 27:7 Медленно плавая многие дни и едва поровнявшись с Книдом, по причине неблагоприятного нам ветра, мы подплыли к Криту при Салмоне.
\vs Act 27:8 Пробравшись же с трудом мимо него, прибыли к одному месту, называемому Хорошие Пристани, близ которого был город Ласея.
\vs Act 27:9 Но как прошло довольно времени, и плавание было уже опасно, потому что и пост уже прошел, то Павел советовал,
\vs Act 27:10 говоря им: мужи! я вижу, что плавание будет с затруднениями и с большим вредом не только для груза и корабля, но и для нашей жизни.
\vs Act 27:11 Но сотник более доверял кормчему и начальнику корабля, нежели словам Павла.
\vs Act 27:12 А как пристань не была приспособлена к зимовке, то многие давали совет отправиться оттуда, чтобы, если можно, дойти до Финика, пристани Критской, лежащей против юго-западного и северо-западного ветра, и \bibemph{там} перезимовать.
\vs Act 27:13 Подул южный ветер, и они, подумав, что уже получили желаемое, отправились, и поплыли поблизости Крита.
\vs Act 27:14 Но скоро поднялся против него ветер бурный, называемый эвроклидон.
\vs Act 27:15 Корабль схватило так, что он не мог противиться ветру, и мы носились, отдавшись волнам.
\vs Act 27:16 И, набежав на один островок, называемый Кл\acc{а}вдой, мы едва могли удержать лодку.
\vs Act 27:17 Подняв ее, стали употреблять пособия и обвязывать корабль; боясь же, чтобы не сесть на мель, спустили парус и таким образом носились.
\vs Act 27:18 На другой день, по причине сильного обуревания, начали выбрасывать \bibemph{груз},
\vs Act 27:19 а на третий мы своими руками побросали с корабля вещи.
\vs Act 27:20 Но как многие дни не видно было ни солнца, ни звезд и продолжалась немалая буря, то наконец исчезала всякая надежда к нашему спасению.
\vs Act 27:21 И как долго не ели, то Павел, став посреди них, сказал: мужи! надлежало послушаться меня и не отходить от Крита, чем и избежали бы сих затруднений и вреда.
\vs Act 27:22 Теперь же убеждаю вас ободриться, потому что ни одна душа из вас не погибнет, а только корабль.
\vs Act 27:23 Ибо Ангел Бога, Которому принадлежу я и Которому служу, явился мне в эту ночь
\vs Act 27:24 и сказал: <<не бойся, Павел! тебе должно предстать пред кесаря, и вот, Бог даровал тебе всех плывущих с тобою>>.
\vs Act 27:25 Посему ободритесь, мужи, ибо я верю Богу, что будет так, как мне сказано.
\vs Act 27:26 Нам должно быть выброшенными на какой-нибудь остров.
\rsbpar\vs Act 27:27 В четырнадцатую ночь, как мы носимы были в Адриатическом море, около полуночи корабельщики стали догадываться, что приближаются к какой-то земле,
\vs Act 27:28 и, вымерив глубину, нашли двадцать сажен; потом на небольшом расстоянии, вымерив опять, нашли пятнадцать сажен.
\vs Act 27:29 Опасаясь, чтобы не попасть на каменистые места, бросили с кормы четыре якоря, и ожидали дня.
\vs Act 27:30 Когда же корабельщики хотели бежать с корабля и спускали на море лодку, делая вид, будто хотят бросить якоря с носа,
\vs Act 27:31 Павел сказал сотнику и воинам: если они не останутся на корабле, то вы не можете спастись.
\vs Act 27:32 Тогда воины отсекли веревки у лодки, и она упала.
\vs Act 27:33 Перед наступлением дня Павел уговаривал всех принять пищу, говоря: сегодня четырнадцатый день, как вы, в ожидании, остаетесь без пищи, не вкушая ничего.
\vs Act 27:34 Потому прошу вас принять пищу: это послужит к сохранению вашей жизни; ибо ни у кого из вас не пропадет волос с головы.
\vs Act 27:35 Сказав это и взяв хлеб, он возблагодарил Бога перед всеми и, разломив, начал есть.
\vs Act 27:36 Тогда все ободрились и также приняли пищу.
\vs Act 27:37 Было же всех нас на корабле двести семьдесят шесть душ.
\vs Act 27:38 Насытившись же пищею, стали облегчать корабль, выкидывая пшеницу в море.
\vs Act 27:39 Когда настал день, земли не узнавали, а усмотрели только некоторый залив, имеющий \bibemph{отлогий} берег, к которому и решились, если можно, пристать с кораблем.
\vs Act 27:40 И, подняв якоря, пошли по морю и, развязав рули и подняв малый парус по ветру, держали к берегу.
\vs Act 27:41 Попали на косу, и корабль сел на мель. Нос увяз и остался недвижим, а корма разбивалась силою волн.
\vs Act 27:42 Воины согласились было умертвить узников, чтобы кто-нибудь, выплыв, не убежал.
\vs Act 27:43 Но сотник, желая спасти Павла, удержал их от сего намерения, и велел умеющим плавать первым броситься и выйти на землю,
\vs Act 27:44 прочим же \bibemph{спасаться} кому на досках, а кому на чем-нибудь от корабля; и таким образом все спаслись на землю.
\vs Act 28:1 Спасшись же, бывшие с Павлом узнали, что остров называется Мелит.
\vs Act 28:2 Иноплеменники оказали нам немалое человеколюбие, ибо они, по причине бывшего дождя и холода, разложили огонь и приняли всех нас.
\vs Act 28:3 Когда же Павел набрал множество хвороста и клал на огонь, тогда ехидна, выйдя от жара, повисла на руке его.
\vs Act 28:4 Иноплеменники, когда увидели висящую на руке его змею, говорили друг другу: верно этот человек~--- убийца, когда его, спасшегося от моря, суд \bibemph{Божий} не оставляет жить.
\vs Act 28:5 Но он, стряхнув змею в огонь, не потерпел никакого вреда.
\vs Act 28:6 Они ожидали было, что у него будет воспаление, или он внезапно упадет мертвым; но, ожидая долго и видя, что не случилось с ним никакой беды, переменили мысли и говорили, что он Бог.
\rsbpar\vs Act 28:7 Около того места были поместья начальника острова, именем Публия; он принял нас и три дня дружелюбно угощал.
\vs Act 28:8 Отец Публия лежал, страдая горячкою и болью в животе; Павел вошел к нему, помолился и, возложив на него руки свои, исцелил его.
\vs Act 28:9 После сего события и прочие на острове, имевшие болезни, приходили и были исцеляемы,
\vs Act 28:10 и оказывали нам много почести и при отъезде снабдили нужным.
\rsbpar\vs Act 28:11 Через три месяца мы отплыли на Александрийском корабле, называемом Диоскуры, зимовавшем на том острове,
\vs Act 28:12 и, приплыв в Сиракузы, пробыли там три дня.
\vs Act 28:13 Оттуда отплыв, прибыли в Ригию; и как через день подул южный ветер, прибыли на второй день в Путеол,
\vs Act 28:14 где нашли братьев, и были упрошены пробыть у них семь дней, а потом пошли в Рим.
\vs Act 28:15 Тамошние братья, услышав о нас, вышли нам навстречу до Аппиевой площади и трех гостиниц. Увидев их, Павел возблагодарил Бога и ободрился.
\vs Act 28:16 Когда же пришли мы в Рим, то сотник передал узников военачальнику, а Павлу позволено жить особо с воином, стерегущим его.
\rsbpar\vs Act 28:17 Через три дня Павел созвал знатнейших из Иудеев и, когда они сошлись, говорил им: мужи братия! не сделав ничего против народа или отеческих обычаев, я в узах из Иерусалима предан в руки Римлян.
\vs Act 28:18 Они, судив меня, хотели освободить, потому что нет во мне никакой вины, достойной смерти;
\vs Act 28:19 но так как Иудеи противоречили, то я принужден был потребовать суда у кесаря, впрочем не с тем, чтобы обвинить в чем-либо мой народ.
\vs Act 28:20 По этой причине я и призвал вас, чтобы увидеться и поговорить с вами, ибо за надежду Израилеву обложен я этими узами.
\vs Act 28:21 Они же сказали ему: мы ни писем не получали о тебе из Иудеи, ни из приходящих братьев никто не известил о тебе и не сказал чего-либо худого.
\vs Act 28:22 Впрочем желательно нам слышать от тебя, как ты мыслишь; ибо известно нам, что об этом учении везде спорят.
\vs Act 28:23 И, назначив ему день, очень многие пришли к нему в гостиницу; и он от утра до вечера излагал им \bibemph{учение} о Царствии Божием, приводя свидетельства и удостоверяя их о Иисусе из закона Моисеева и пророков.
\vs Act 28:24 Одни убеждались словами его, а другие не верили.
\vs Act 28:25 Будучи же не согласны между собою, они уходили, когда Павел сказал следующие слова: хорошо Дух Святый сказал отцам нашим через пророка Исаию:
\vs Act 28:26 пойди к народу сему и скажи: слухом услышите, и не уразумеете, и очами смотреть будете, и не увидите.
\vs Act 28:27 Ибо огрубело сердце людей сих, и ушами с трудом слышат, и очи свои сомкнули, да не узрят очами, и не услышат ушами, и не уразумеют сердцем, и не обратятся, чтобы Я исцелил их.
\vs Act 28:28 Итак да будет вам известно, что спасение Божие послано язычникам: они и услышат.
\vs Act 28:29 Когда он сказал это, Иудеи ушли, много споря между собою.
\rsbpar\vs Act 28:30 И жил Павел целых два года на своем иждивении и принимал всех, приходивших к нему,
\vs Act 28:31 проповедуя Царствие Божие и уча о Господе Иисусе Христе со всяким дерзновением невозбранно.
\vs Act 29:1 И Павел, исполненный благословений Христа и преизобилующий в духе, удалился из Рима, решив идти в Испанию; ибо он давно имел намерение на путешествие туда, и задумал также идти оттуда в Британию.\fns{Гл.~29: Оригинальная греческая рукопись была найдена во второй половине XVIII столетия в архивах Константинополя и подарена султаном Абдул Ахметом французскому учёному К.С.~Соннини, который перевёл её на английский язык и издал в 1801 году.}
\vs Act 29:2
Ибо он слышал в Финикии, что некоторые дети Израилевы,
со времени Ассирийского плена, бежали морем
на острова отдалённые, как сказано пророком,
и называвшиеся римлянами Британией.
\vs Act 29:3
И Господь повелел, чтобы евангелие было проповедано
далеко отсюда народам и заблудшим овцам дома Израилева.
\vs Act 29:4
И никто не препятствовал Павлу;
ибо он смело свидетельствовал об Исусе
пред трибунами и среди людей;
и он взял с собою некоторых из братьев,
которые находились с ним в Риме, и они,
погрузившись в Остриуме и имея попутные ветры,
благополучно прибыли в пристань Испании.
\vs Act 29:5
И множество людей собралось вместе из городов
и селений и горной местности;
ибо они услышали об обращении к апостолу и многих чудесах,
которые он совершал.
\vs Act 29:6
И Павел дерзновенно проповедывал в Испании,
и великое множество уверовало и было обращено;
ибо они уразумели, что он~--- апостол, посланный от Бога.
\vs Act 29:7
И они удалились из Испании, и Павел и его спутники,
найдя судно, отплывающее в Арморику, к Британии,
куда они желали, и продвигаясь вдоль южного берега,
они достигли порта, называемого Рафин. 
\vs Act 29:8
Ныне, когда об этом было возвещено повсюду,
что апостол высадится на их берег,
великое множество обитателей встретило его,
и они обходились с Павлом учтиво и он вошёл
в восточные ворота их города и поселился
в доме еврея и одного из его племени.

\vs Act 29:9
И назавтра он пришёл и стал на горе Луд;
и народ толпился у прохода, и собрались на дороге,
и он проповедовал им Христа,
и они поверили слову и свидетельству об Исусе.
\vs Act 29:10
И к вечеру Святой Дух сошёл на Павла,
и он пророчествовал, говоря:
<<Вот, в последние дни Бог мира пребудет
в городах и поэтому жители будут исчислены:
и в 7-м исчислении людей их глаза откроются,
и слава их наследия впредь возсияет перед ними.
Народы пойдут поклоняться на гору,
свидетельствовавшую о страдании
и долготерпении раба Господня. 
\vs Act 29:11
И в последние дни новая весть
о евангелии произойдет из Иерусалима,
и сердца людей возрадуются,
и вот, источники разверзнутся,
и больше не будет безпокойства.
\vs Act 29:12
В те дни будут в\acc{о}йны и слухи войн;
и царь возстанет, и его меч будет для исцеления народов,
и его миротворчество будет неизменно,
и слава его царства~--- удивление среди князей.>>
\vs Act 29:13
И вот пришли передать, что некоторые из друидов
пришли к Павлу лично; и показали свои обряды и церемонии,
унаследованные ими от иудеев, которые бежали из рабства
в земле Египетской; и апостол поверил их словам,
и он дал им целование мира.
\vs Act 29:14
И Павел пребывал в своём жилище 3 месяца,
утверждая в вере и непрестанно проповедуя Христа.
\vs Act 29:15
И после этих деяний Павел и его братья удалились из Рафина,
и поплыли на Атиум в Галлию.
\vs Act 29:16
И Павел проповедовал в Римском гарнизоне и среди народа,
увещевая всех мужчин раскаяться и исповедовать их грехи.

\vs Act 29:17
И вот пришли к нему некоторые из бельгов разузнать
о его новом учении и человеке Иисусе;
и Павел открыл им своё сердце, и поведал им всё то,
что произошло с ним, в том числе, как Иисус Христос
пришёл в мир, чтобы спасти грешников;
и они ушли, размышляя между собою о том,
что они услышали.

\vs Act 29:18
И после длительной проповеди и труда Павел
и его соработники перешли в Гельветию,
и пришли к горе Понтия Пилата, где тот,
кто осудил Господа Иисуса,
бросился вниз головой, и так жалко погиб.
\vs Act 29:19
И тотчас поток хлынул из скалы и смыл его тело,
разбившееся на части, в озеро.
\vs Act 29:20
И Павел простёр свои руки на воду, 
и молился Господу, говоря:
<<О Господи Боже, дай знамение всем народам,
что здесь Понтий Пилат, который осудил твоего единородного сына,
низвергся вниз головой в бездну.>>
\vs Act 29:21
И пока Павел ещё говорил, вот,
там случилось сильное землетрясение,
и лицо вод изменилось, и вид озера стал подобен
сыну человеческому, висящему в мучении на кресте.
\vs Act 29:22
И голос изшёл с небес, говоря:
<<Даже Пилат избегает грядущего гнева,
ибо он умыл свои руки перед толпой
при пролитии крови Господа Иисуса.>>
\vs Act 29:23
Поэтому когда Павел и те, что были с ним,
увидeли землетрясение и услышали голос ангела,
они прославили Бога и сильно укрепились в духе.

\vs Act 29:24
И они путешествовали и пришли к горе Юлия,
где стояли 2 столпа, один справа, а другой слева,
воздвигнутые кесарем Августом.
\vs Act 29:25
И Павел, исполнившись Святым Духом,
стал между 2-я столпами, говоря:
<<Мужи и братья, эти камни, которые вы видите сегодня,
будут свидетельствовать о моём путешествии отсюда;
и поистине я скажу: они останутся до излияния Духа на все народы,
никакой путь не будет препятствовать этому во всех поколениях.>>

\vs Act 29:26
И они отправились дальше и пришли в Иллирик,
собираясь идти через Македонию в Асию,
и благодать обреталась во всех церквях,
и они преуспевали и имели мир. Аминь!

\bibbookdescr{Tnf}{
  inline={Завещание Неффалима,\\восьмого сына Иакова и Баллы},
  toc={Завещание Неффалима},
  bookmark={Завещание Неффалима},
  header={Завещание Неффалима},
  abbr={Неф}
}
\vs Tnf 1:1
Список завещания Неффалима,
данного им в час кончины его в 130-ый год жизни его.
\vs Tnf 1:2
Когда собрались сыновья его в 7-ой месяц 1-го числа, устроил им пир.
\vs Tnf 1:3
И пробудившись наутро, сказал он им:
я умираю.
И они не поверили ему.
\vs Tnf 1:4
И восславив Господа, собрал он силы и сказал:
после пира, бывшего вчера, умерла плоть моя.

\vs Tnf 1:5
И начал говорить:
слушайте, дети мои, сыновья Неффалима, слушайте слова отца вашего.
\vs Tnf 1:6
Я родился от Баллы, ибо хитрость сотворила Рахиль,
и вместо себя дала Баллу Иакову,
и та зачала и родила меня на колени Рахили,
и оттого наречено мне было имя Неффалим.
\vs Tnf 1:7
Премного возлюбила меня Рахиль, ибо на колени её родился я,
и когда был я ещё мал, целовала меня, говоря:
да будет мне дан брат твой от чрева моего, такой, как ты.
\vs Tnf 1:8
Оттого сходен был со мною во всем Иосиф по мольбам Рахили.
\vs Tnf 1:9
Мать же моя Балла была дочерью Руфея, брата Деворы,
кормилицы Ревекки, и родилась в один день с Рахилью.
\vs Tnf 1:10
Руфей же был из рода Авраама, Халдей,
чтущий Бога, свободный и знатный.
\vs Tnf 1:11
И попав в плен, был он куплен Лаваном,
и тот дал ему в жёны Енан, служанку свою,
которая родила дочь и нарекла ей имя Зелфа по имени того города,
где Руфей был взят в плен.
\vs Tnf 1:12
После же родила она Баллу и сказала:
к новому торопится дочь моя,
ибо родилась быстро и, взяв грудь,
сразу принялась сосать.

\vs Tnf 2:1
Был я лёгок ногами, словно серна, и поручал мне всякую весть
отец мой Иаков, и как серну благословил меня.
\vs Tnf 2:2
Как знает гончар сосуд, сколько вмещает он,
и сообразно с этим берёт глину для него,
так же и Господь в согласии с духом творит тело,
а по силе телесной влагает дух.
\vs Tnf 2:3
И нет расхождения ни на треть волоса,
ибо всё творение весами, и мерою, и правилом совершается.
\vs Tnf 2:4
И как знает гончар пользу всякого сосуда, для чего он пригоден,
так же и Господь знает тело,
до какого предела пребывает оно в добре,
а когда ко злу переходит.
\vs Tnf 2:5
Ибо нет творения и никакого помысла нет,
которых не ведал бы Господь.
Ибо всякого человека сотворил он по образу своему.
\vs Tnf 2:6
И какова сила его, таково и дело его;
каково око его, таков и сон его;
какова душа его, таково и слово его~--- либо по закону Господа,
либо по закону Велиара.
\vs Tnf 2:7
И как различают между светом и тьмою,
между зрением и слухом, так и между мужем и мужем различают,
и между женщиной и женщиной, и нельзя сказать,
что один подобен другому лицом или помыслом.
\vs Tnf 2:8
Ибо всё сделал Бог прекрасно в порядке своём:
5 чувств поместил в голове,
и горло приладил к голове,
и волосы на ней взрастил для красоты и славы;
после сердце сотворил для рассуждения,
желудок~--- для пищеварения,
чрево~--- для очищения тела,
горло~--- для дыхания,
печень~--- для гнева,
желчь~--- для огорчения,
селезёнку~--- для веселья,
почки~--- для разумения,
бока~--- для сна,
бёдра~--- для мощи,
и так далее.
\vs Tnf 2:9
Так да будут, дети мои, все дела ваши в порядке своём,
и в добром помышлении,
и в страхе Божием,
и ничего безрассудного не делайте в небрежении,
или же не в свой час.
\vs Tnf 2:10
Ибо если скажешь оку: слушай,~--- не сможет оно.
Так и вы, пребывая во тьме, не сможете творить дела света.

\vs Tnf 3:1
Так не стремитесь в любостяжании погубить дела ваши,
и словами пустыми не обманывайте душ ваших,
ибо молча, в чистоте сердца узрите,
как поддержать волю Божию, а волю Велиара отвергнуть.
\vs Tnf 3:2
Солнце, луна и звёзды не меняют порядка своего;
так и вы не меняйте закона Божьего,
лишая порядка дела ваши.
\vs Tnf 3:3
Язычники, заблуждаясь и отвергая Господа,
изменили порядок свой,
стали слушаться они дерева и камня,
духов соблазна.
\vs Tnf 3:4
Вы же не делайте так, дети мои;
узнавайте Господа на небосводе,
на земле, на море и во всех творениях его,
создавшего всё,
дабы не уподобиться вам Содому,
изменившему строй естества своего.
\vs Tnf 3:5
Так же и Стражи изменили строй естества своего,
и низверг их Господь потопом,
из-за них сделав землю лишённой поселений и плодов.

\vs Tnf 4:1
То говорю я вам, дети мои, что узнал я из писаний Еноха,
что и вы отступитесь от Господа,
и станете жить во всём беззаконии языческом,
и сотворите всё зло Содомское.
\vs Tnf 4:2
И наведёт на вас Господь пленение,
и рабами будете врагам вашим,
и всякой беде и горю подвергнетесь,
пока не избавит Господь всех вас.
\vs Tnf 4:3
Когда уменьшитесь вы и умалитесь,
обратитесь вы и познаете Бога вашего,
и он возвратит вас в землю вашу по великому милосердию своему.
\vs Tnf 4:4
И будет: пришедшие в землю отцов своих вновь забудут Господа
и совершат нечестия.
\vs Tnf 4:5
И рассеет их Господь по лицу всей земли,
пока не придет милосердие Господне,~--- Человек,
справедливость творящий и милость всем дальним и ближним.
\vs Tnf 5:1
Ибо в 40-ой год жизни моей узрел я видение
на горе Елеонской к востоку от Иерусалима,
что солнце и луна остановились.
\vs Tnf 5:2
И вот, Исаак, отец отца моего, сказал нам:
бегите и возьмите каждый по силе своей,
и получит овладевший солнце и луну.
\vs Tnf 5:3
И побежали все разом, и Левий овладел солнцем,
а Иуда успел взять луну,
и возвысились оба с тем, что взяли они.
\vs Tnf 5:4
И когда Левий стал как солнце, вот,
некий юноша дал ему 12 ветвей пальмовых,
а Иуда стал светел как луна,
и было под ногами их 12 лучей.
\vs Tnf 5:5
[И побежали оба, Левий и Иуда, и взяли их себе.]
\vs Tnf 5:6
И вот, явился на земле бык, имеющий 2 рога великих
и крылья орла на спине своей;
и когда хотели мы взять его, не смогли.
\vs Tnf 5:7
Иосиф же, придя, схватил его и взошел с ним на высоту.
\vs Tnf 5:8
И увидел я, что был я там,
и вот, святое писание увидели мы,
говорящее:
Ассирийцы, Мидяне, Персы, Халдеи, Сирияне
унаследуют пленение 12-ти скипетров Израиля.

\vs Tnf 6:1
И опять, через 5 дней, узрел я,
что отец мой Иаков стоит на море Ямнийском, и мы с ним.
\vs Tnf 6:2
И вот, корабль подошёл, плывущий без моряков и рулевых,
и написано было на нём, что это корабль Иакова.
\vs Tnf 6:3
И сказал нам отец наш: взойдем на корабль наш.
\vs Tnf 6:4
Когда же взошли мы, сделалась сильная буря и вихрь великий,
и ушёл от нас отец наш, державший руль.
\vs Tnf 6:5
А мы, гонимые бурей, носились по морю,
и наполнился корабль водою, и заливали его валы огромные,
и от них рассыпался он.
\vs Tnf 6:6
И поплыл Иосиф в лодке, а мы разделились на 10 досок.
Левий же и Иуда были на одной доске.
\vs Tnf 6:7
И разметало нас всех по разным концам земли.
\vs Tnf 6:8
И Левий, облачившись во вретище, молился Господу.
\vs Tnf 6:9
Когда же утихла буря, прибыло судно к земле в мире.
\vs Tnf 6:10
И вот, пришёл отец наш, и все мы вместе возвеселились.

\vs Tnf 7:1
2 этих сна поведал я отцу моему, и сказал он мне:
должно тому исполниться в свои времена,
когда многое вынесет Израиль.
\vs Tnf 7:2
Потом сказал отец мой:
верю я Богу, что жив Иосиф, ибо всечасно вижу я,
что числит его с живыми Господь.
\vs Tnf 7:3
И сказал плача: увы, дитя мое Иосиф, ты жив,
a я не вижу тебя, и ты не видишь Иакова,
породившего тебя.
\vs Tnf 7:4
От этих слов его заплакал и я;
и воспылал я сердцем моим возвестить,
что продан был Иосиф, но побоялся я братьев моих.

\vs Tnf 8:1
И вот, дети мои, показал я вам последние времена,
когда всё совершится в Израиле.
\vs Tnf 8:2
А вы поведайте о том детям вашим,
дабы они едины были с Левием и с Иудою.
Ибо через них придёт спасение Израилю,
и в них благословен будет Иаков.
\vs Tnf 8:3
От скипетра их явится Бог [живущий в людях] на земле,
дабы спасти род Израиля и привести к нему праведных из язычников.
\vs Tnf 8:4
И если вы также будете делать добро,
благословят вас люди и ангелы,
и через вас прославлен будет Бог среди народов,
а дьявол бежит от вас, и звери убоятся вас,
и Господь возлюбит вас [и ангелы обнимут вас].
\vs Tnf 8:5
Вырастивший доброго сына стяжает память добрую,
так же и память добрая о благом деле у Бога пребывает.
\vs Tnf 8:6
Того же, кто сотворит недоброе, проклянут его и ангелы,
и люди, а Бога хулить станут язычники из-за него,
а дьявол поселится в нём, как в орудии своём,
а всякий зверь власть будет иметь над ним,
и возненавидит его Господь.
\vs Tnf 8:7
Заповеди же закона двояки, и нужно искусство для их исполнения.
\vs Tnf 8:8
Ибо время сообщению с женой, и время воздержанию для молитвы.
\vs Tnf 8:9
И обе заповеди~--- от Бога,
и если бы не исполнялись они в порядке своём,
грех великий учинялся бы людьми.
Так же и с остальными заповедями.
\vs Tnf 8:10
Будьте же мудры в Боге, дети мои, и благоразумны,
видя порядок заповедей его и законы всех дел,
дабы возлюбил вас Господь.

\vs Tnf 9:1
И много подобного завещав им, просил,
чтобы отнесли кости его в Хеврон и погребли там с отцами его.
\vs Tnf 9:2
И вкушал он, и пил в веселии души, после же закрыл лицо своё и умер.
\vs Tnf 9:3
И сделали сыновья его всё так, как завещал им Неффалим, отец их.

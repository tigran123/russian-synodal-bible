\bibbookdescr{3Er}{
  inline={Пастырь Ермы. Книга 3. Подобия},
  toc={3-я Ермы},
  bookmark={3-я Ермы},
  header={3-я Ермы},
  abbr={3~Ермы}
}
\chhdr{Подобие 1-е.}
\vs 3Er 1:1
Мы, не имея в этом мире постоянного города, должны искать будущего.
\vs 3Er 1:2
Пастырь сказал мне: знаете
ли, что вы, рабы Божьи, находитесь в странствии? Ваш город далеко отсюда. Если
знаете ваше отечество, в котором надлежит вам жить, то зачем здесь покупаете
поместья, строите великолепные здания и ненужные жилища?
\vs 3Er 1:3
Ибо кто занимается подобными приготовлениями в этом городе, тот не помышляет о
возвращении в свое отечество. Несмысленный, двоедушный и жалкий человек, разве
не понимаешь, что всё это чужое и под властью другого?
\vs 3Er 1:4
Ибо господин этого города
говорит: или следуй моим законам, или убирайся вон из моих пределов. Что же
поэтому сделаешь ты, имея собственный закон в твоем отечестве? Ужели ради
полей или других стяжаний своих откажешься от отечественного закона?
\vs 3Er 1:5
Если же ты откажешься, а
потом пожелаешь возвратиться в свое отечество, то не будешь принят, но изгнан
оттуда.
\vs 3Er 1:6
Итак, смотри, подобно
страннику на чужой стороне, не приготовляй себе ничего более того, сколько
тебе необходимо для жизни;
\vs 3Er 1:7
и будь готов к тому,
чтобы, когда господин этого города захочет изгнать тебя за то, что не
повинуешься закону его,~--- идти тебе в своё отечество и жить по своему закону
беспечально и радостно.
\vs 3Er 1:8
Итак, вы, служащие Богу и
имеющие Его в сердцах своих, смотрите: делайте дела Божьи, помня о заповедях
Его и обетованиях, Им данных, и веруйте Ему, что Он исполнит их, если будут
соблюдены Его заповеди.
\vs 3Er 1:9
Вместо полей искупайте
души от нужд, сколько кто может, помогайте вдовам и сиротам; богатство и все
стяжания ваши употребляйте на такого рода дела, ради которых вы и получили их
от Бога.
\vs 3Er 1:10
Ибо Господь обогатил вас
для того, чтобы вы исполняли такое служение Ему.
\vs 3Er 1:11
Гораздо лучше делать это,
нежели покупать дома и поместья, ибо имущество тленно, тогда как то, что
сделаешь во имя Божье, обретешь в своём городе и будешь иметь радость без
печали и страха.
\vs 3Er 1:12
Итак, не желайте богатств
народов, ибо несвойственны они рабам Божьим; избытком же своим распоряжайтесь
так, чтобы могли вы получить радость.
\vs 3Er 1:13
И не делайте фальшивой
монеты, не касайтесь и не желайте чужого. Делай своё дело~--- и спасешься.

\chhdr{Подобие 2-е.}
\vs 3Er 2:1
Однажды, когда я, прогуливаясь по полю, увидел вяз и
виноградное дерево и размышлял о плодах их~--- пастырь явился мне и спросил: что
ты думаешь об этом виноградном дереве и вязе?
\vs 3Er 2:2
Думаю, что они пригодны
друг для друга.
\vs 3Er 2:3
И сказал он мне: эти два
дерева являют рабам Божьим глубокий смысл.
\vs 3Er 2:4
Желал бы я познать,
господин, этот смысл.
\vs 3Er 2:5
Смотрите же,~--- сказал он.
Это виноградное дерево имеет плод, а вяз~--- дерево бесплодное; но виноградное
дерево не может приносить обильных плодов, если не будет опираться на вяз.
\vs 3Er 2:6
Ибо, лёжа на земле, оно
дает гнилой плод; но если виноградная лоза будет висеть на вязе, то дает плод
и за себя, и за вяз.
\vs 3Er 2:7
Итак, видишь, что вяз дает
плод не меньший, а гораздо больший, нежели виноградная лоза, потому что
виноградная лоза, поддерживаемая вязом, дает плод и обильный и хороший, но,
лёжа на земле, дает плод плохой и малый.
\vs 3Er 2:8
Это служит притчею для
рабов Божьих, для бедного и богатого.
\vs 3Er 2:9
Каким образом, объясни
мне?
\vs 3Er 2:10
Слушай,~--- говорит он.
Богатый имеет много сокровищ, но беден перед Господом. Занятый своими
богатствами, он очень мало молится Господу и если имеет какую молитву, то
скудную и не имеющую силы.
\vs 3Er 2:11
Но когда богатый подает
бедному то, в чем он нуждается, тогда бедный молит Господа за богатого, и Бог
подает богатому все блага, потому что бедный богат в молитве и молитва его
имеет великую силу пред Господом.
\vs 3Er 2:12
Богатый подает бедному,
веруя, что ему внимает Господь, и охотно и без сомнения подает ему всё,
заботясь, чтобы у него не было в чем-нибудь недостатка.
\vs 3Er 2:13
Бедный благодарит Бога за
богатого, дающего ему.
\vs 3Er 2:14
Так люди, думая, что вяз
не дает плода, не понимают того, что во время засухи вяз, имея в себе влагу
питает виноградную лозу, и виноградная лоза благодаря этому дает двойной плод
и за себя, и за вяз.
\vs 3Er 2:15
Так и бедные, моля
Господа за богатых, бывают услышаны и умножают богатства их, а богатые,
помогая бедным, ободряют их души. Те и другие участвуют в добром деле.
\vs 3Er 2:16
Итак, кто поступает таким
образом, не будет оставлен Господом, но будет вписан в книгу жизни.
\vs 3Er 2:17
Блаженны те, которые,
имея богатство, сознают, что они обогащаются от Господа, ибо кто почувствует
это, тот может совершать добро.

\chhdr{Подобие 3-е.}
\vs 3Er 3:1
Пастырь показал мне много деревьев без листьев, казавшихся
иссохшими: все они были похожи.
\vs 3Er 3:2
Видишь эти деревья?
\vs 3Er 3:3
Вижу,~--- говорю я.~--- Они
похожи друг на друга и сухи.
\vs 3Er 3:4
Эти деревья служат образом
людей, живущих в этом мире.
\vs 3Er 3:5
Почему же, господин,~--- спросил я,~--- они как бы засохли и похожи друг на друга?
\vs 3Er 3:6
Потому,~--- отвечал он,~--- что в этом веке не различимы
ни праведные, ни нечестивые люди: одни походят на других.
\vs 3Er 3:7
Ибо настоящий век есть
зима для праведных, которые, живя с грешниками, по виду не отличаются от них.
\vs 3Er 3:8
Как во время зимы все
деревья с облетевшими листьями сходны между собою, и не видно, которые из них
действительно засохли, а которые живы, так точно в настоящем веке нельзя
распознать праведников и грешников, но все похожи одни на других.

\chhdr{Подобие 4-е.}
\vs 3Er 4:1
Снова показал мне пастырь многие деревья, из которых одни
расцвели, а другие были иссохшие.
\vs 3Er 4:2
Видишь ли эти деревья?
\vs 3Er 4:3
Вижу, господин,~--- отвечал
я,~--- одни засохли, а другие покрыты листьями.
\vs 3Er 4:4
Эти зеленеющие деревья,~--- сказал он,~--- означают праведных, которые будут жить в грядущем веке.
\vs 3Er 4:5
Ибо будущий век есть лето
для праведных и зима для грешников.
\vs 3Er 4:6
Итак, когда воссияет
благость Господа, тогда явятся служащие Богу и все будут видимы.
\vs 3Er 4:7
Ибо как летом созревает
плод всякого дерева, и становится понятно, каково оно, так точно обнаружится и
будет видим и плод праведных, и все они явятся радостными в том веке.
\vs 3Er 4:8
Народы же и грешники суть
сухие деревья, которые ты видел. Они обретутся в будущем веке сухими и
бесплодными, и будут преданы огню, как дрова, и обнаружится, что во время их
жизни дела их были злы.
\vs 3Er 4:9
Грешники будут преданы
огню, потому что согрешили и не раскаялись в грехах своих, народы же потому,
что не познали Бога~--- Творца своего.
\vs 3Er 4:10
Посему ты приноси плод
добрый, чтобы он явился во время того лета. Воздерживайся от многих попечений
и никогда не согрешишь.
\vs 3Er 4:11
Ибо имеющие многие заботы
согрешают во многом, потому что озабочены своими делами и не служат Богу.
\vs 3Er 4:12
Каким же образом человек,
не служащий Богу, может просить и получить что-либо от Бога?
\vs 3Er 4:13
Те, которые служат Богу,
просят и получат свои прошения, а не служащие Богу~--- не получат.
\vs 3Er 4:14
Кто занимается одним
делом, тот может и служить Богу; потому что дух его не отчуждается от Господа,
но чистою мыслию служит Богу.
\vs 3Er 4:15
Итак, если исполнишь это,
будешь иметь плод в грядущем веке; равно как и все, которые исполнят это,
будут иметь плод.

\chhdr{Подобие 5-е.}
\vs 3Er 5:1
Однажды во время поста сидел я на горе, благодарил Господа за
то, что сделал Он со мною, и увидел вдруг пастыря рядом с собою.
\vs 3Er 5:2
И спрашивает он у меня: что так рано пришел ты сюда?
\vs 3Er 5:3
Потому, господин, что нахожусь на стоянии.
\vs 3Er 5:4
А что такое стояние?
\vs 3Er 5:5
То есть пощусь, господин,~--- объяснил я.
\vs 3Er 5:6
Каким же образом,~--- спросил он,~--- постишься ты?
\vs 3Er 5:7
Как постился по обыкновению, так и пощусь.
\vs 3Er 5:8
Не умеете вы,~--- сказал он,
поститься Богу; и пост, который совершаете, бесполезен.
\vs 3Er 5:9
Почему, господин, говоришь так?
\vs 3Er 5:10
То, как вы думаете
поститься, не есть пост, но я научу тебя, какой пост есть совершенный и
угодный Богу.
\vs 3Er 5:11
Слушай: Бог не хочет
такого суетного поста, ибо, постясь таким образом, ты не совершаешь правды.
\vs 3Er 5:12
Постись же Богу следующим
постом: не лукавствуй в жизни, но служи Богу чистым сердцем; соблюдай Его
заповеди, ходи в Его повелениях и не допускай никакой злой похоти в сердце
своем.
\vs 3Er 5:13
Веруй в Бога, и если
исполнишь это и будешь иметь страх Божий и удержишься от всякого злого дела,
то будешь жить с Богом.
\vs 3Er 5:14
И таким образом ты
совершишь великий и угодный Богу пост.

\vs 3Er 6:1
Послушай притчу
относительно поста, которую я намерен поведать тебе.
\vs 3Er 6:2
Некто имел поместье и
много рабов. В одной части земли своей он насадил виноградник, и потом,
отправляясь в дальнее путешествие, избрал раба, самого верного и честного, и
поручил ему виноградник с тем, чтобы он к виноградным лозам приставил
подпорки, обещая за исполнение этого приказания дать ему свободу.
\vs 3Er 6:3
Только это хозяин приказал
рабу сделать в винограднике и с тем отправился.
\vs 3Er 6:4
Раб тщательно сделал, что
господин повелел: он расставил подпорки в винограднике, но, приметив в нём
много сорных трав, стал рассуждать сам с собою: я исполнил приказание
господина, вскопаю теперь виноградник, и он будет красивее; а если выполоть
сорную траву, он, не заглушаемый сорняками, даст больше плода.
\vs 3Er 6:5
И принялся за работу;
вскопал виноградник и выполол в нём все сорняки, и стал виноградник красивым и
цветущим, не засоренным травами.
\vs 3Er 6:6
Через некоторое время
возвратился господин его и пришел в виноградник. Когда он увидел, что
виноградник хорошо обставлен и сверх того вскопан, прополот, и лозы обильны
плодами, то был весьма доволен поступком раба своего.
\vs 3Er 6:7
Итак, пригласил он
любимого сына, своего наследника, и друзей, своих советников, и рассказал им,
что приказал он сделать рабу своему и что тот сверх этого сделал.
\vs 3Er 6:8
Они тотчас приветствовали
раба с тем, что он получил столь высокую похвалу от своего господина.
\vs 3Er 6:9
Господин же говорит им:
<<Я обещал свободу этому рабу, если он исполнит данное приказание, он исполнил его
и сверх того приложил к винограднику добрый труд, который мне весьма
понравился.
\vs 3Er 6:10
Поэтому за его усердие я
хочу сделать его сонаследником моего сына, потому что, помысливши доброе, он
не оставил его, но исполнил.>>
\vs 3Er 6:11
Это намерение господина,
то есть чтобы раб был сонаследником сыну, одобрили и сын, и друзья его.
\vs 3Er 6:12
Потом, спустя несколько
дней, когда созваны были гости, господин со своего пира посылал тому рабу
много яств.
\vs 3Er 6:13
Получая их, раб брал из
них то, что было для него достаточно, остальное же делил между товарищами
своими.
\vs 3Er 6:14
Они, обрадованные, начали
желать ему, чтобы он еще большую любовь нашел у хозяина за свою доброту и
щедрость.
\vs 3Er 6:15
Когда обо всем этом узнал
господин его, он опять весьма обрадовался и снова рассказал друзьям и сыну о
поступке своего раба, и они еще более одобрили мысль господина, чтобы раб этот
был сонаследником сына.

\vs 3Er 7:1
Я сказал: господин, не
знаю этих притчей и не смогу понять, если ты не объяснишь мне их.
\vs 3Er 7:2
Всё,~--- обещал он,~--- объясню, что только скажу и покажу тебе. Соблюдай заповеди Господа, и будешь
угоден Богу и включен в число тех, которые соблюли Его заповеди.
\vs 3Er 7:3
Если же сделаешь что-либо
доброе сверх заповеданного Господом, то приобретешь себе еще большее
достоинство и будешь пред Господом славнее, нежели мог быть прежде.
\vs 3Er 7:4
Итак, если соблюдешь
заповеди Господа и к ним присоединишь эти стояния, то получишь великую
радость, особенно если будешь исполнять их согласно с моим внушением.
\vs 3Er 7:5
Господин,~--- говорю,~--- я
исполню все, что ни повелишь мне, ибо я знаю, что ты будешь со мною.
\vs 3Er 7:6
Буду,~--- сказал он,~--- с
тобою, потому что имеешь такое доброе намерение; буду также и со всеми
имеющими такое намерение.
\vs 3Er 7:7
Этот пост,~--- продолжал он,
при исполнении заповедей Господа очень хорош, и соблюдай его таким образом:
прежде всего воздерживайся от всякого дурного слова и злой похоти и очисти
сердце своё от всех сует века сего.
\vs 3Er 7:8
Если соблюдать это, пост у
тебя будет праведный.
\vs 3Er 7:9
Поступай же так: исполнив
вышесказанное, в тот день, в который постишься, ничего не вкушай, кроме хлеба
и воды; а то из пищи, что ты в этот день сбережешь таким образом, отложи и
отдай вдове, сироте или бедному;
\vs 3Er 7:10
таким образом ты смиришь
свою душу; а получивший от тебя насытит свою душу и будет за тебя молиться
Господу.
\vs 3Er 7:11
Если будешь совершать
пост так, как я повелел тебе, то жертва твоя будет приятна Господу, и этот
пост будет написан, и дело, таким образом совершаемое, прекрасно, радостно и
угодно Господу.
\vs 3Er 7:12
Если ты соблюдешь это с
детьми своими и со всеми домашними твоими, то будешь блажен; и все, кто только
соблюдут это, будут блаженны и что ни попросят у Господа, всё получат.

\vs 3Er 8:1
И упрашивал я его, чтобы
объяснил мне эту притчу о поместье и господине, о винограднике и рабе,
поставившем подпорки в нем, о травах, выполотых в винограднике, о сыне и
друзьях, призванных для совета: ибо я понял, что все это~--- притча.
\vs 3Er 8:2
Он сказал мне: очень смел
ты на вопросы. Ты ни о чем не должен спрашивать; что должно быть объяснено, то
объяснится тебе.
\vs 3Er 8:3
Господин, я напрасно буду
видеть то, что ты покажешь мне, не истолковав, что это значит; напрасно буду
слушать и притчи, если ты будешь предлагать их мне без объяснения.
\vs 3Er 8:4
Он сказал мне снова: кто
раб Божий и в сердце своем имеет Господа, тот просит у Него разума и получает,
и постигает всякую притчу, и понимает слова Господа, сказанные приточно.
\vs 3Er 8:5
А беспечные и ленивые к
молитве колеблются просить Господа, тогда как Господь многомилостив и
непрестанно дает всем просящим у Него.
\vs 3Er 8:6
Ты же утвержден тем
достопоклоняемым ангелом и получил от Него столь могущественную молитву.
Почему, если не ленив ты, не просишь разума и не получаешь от Господа?
\vs 3Er 8:7
Если ты при мне,~--- сказал
я ему~--- надлежит мне тебя обо всем просить и спрашивать, ибо ты всё мне
показываешь и говоришь со мною. Если бы без тебя я видел это или слышал, тогда
бы Господа просил, чтобы было мне объяснено.

\vs 3Er 9:1
И он отвечал: я и прежде
говорил тебе, что ты искусен и смел на то, чтобы спрашивать смысл притчей.
\vs 3Er 9:2
Так как ты настойчив, то
объясню тебе притчу о поместье и о прочем, чтобы ты рассказал всем.
\vs 3Er 9:3
Слушай же и разумей.
Поместье, о котором говорится в притче, означает мир. Владелец поместья есть
Творец, который всё создал и утвердил. Сын есть Дух Святой. Раб~--- Сын Божий.
\vs 3Er 9:4
Виноградник означает
народ, который насадил Господь. Подпорки суть ангелы, приставленные Господом
для сохранения Его народа.
\vs 3Er 9:5
Травы, уничтоженные в
винограднике, суть преступления рабов Божьих. Яства, которые с пира посылал
господин рабу, суть заповеди, которые через Сына своего дал Господь своему
народу.
\vs 3Er 9:6
Друзья, призванные на
совет, суть святые ангелы первозданные.
\vs 3Er 9:7
Отсутствие же господина
означает время, остающееся до Его Пришествия.

\vs 3Er 10:1
Я сказал тогда: господин,
величественно, дивно и славно всё, что ты поведал, но мог ли я, господин,
понять это?
\vs 3Er 10:2
Да ни один человек, хотя
бы и очень разумный, не может постичь этого. Теперь же спрошу тебя вот о чем.
\vs 3Er 10:3
Спрашивай, что хочешь.
\vs 3Er 10:4
Почему Сын Божий в этой
притче представляется рабом?
\vs 3Er 10:5
Слушай,~--- сказал он. Сын
Божий предстает в рабском положении, но имеет великое могущество и власть.
\vs 3Er 10:6
Каким образом, господин,
не понимаю?
\vs 3Er 10:7
Бог насадил виноградник,
то есть создал народ и поручил Сыну своему; Сын же приставил ангелов для
сохранения каждого из людей и сам усердно трудился и изрядно пострадал, чтобы
искупить грехи их.
\vs 3Er 10:8
Ибо никакой виноградник не
может быть очищен без труда и подвига.
\vs 3Er 10:9
Итак, очистив грехи народа
Своего, Он показал им путь жизни и дал им закон, принятый Им от Отца.
\vs 3Er 10:10
Видишь, что Он есть
Господь народа со всею властью, полученною от Отца.
\vs 3Er 10:11
А вот почему Господь
держал совет о наследстве с Сыном Своим и славными ангелами. Дух Святой,
прежде Сущий, создавший всю тварь, Бог поселил в плоть, какую Он пожелал.
\vs 3Er 10:12
И эта плоть, в которую
вселился Дух Святой, хорошо послужила Духу, ходя в чистоте и святости и ничем
не осквернив Духа.
\vs 3Er 10:13
И так как жила она
непорочно, и подвизалась вместе с Духом, и мужественно содействовала Ему во
всяком деле, то Бог принял её в общение, ибо Ему угодно было житие плоти,
которая не осквернилась на земле, имея в себе Дух Святой.
\vs 3Er 10:14
И призвал Он в совет Сына
и добрых ангелов, чтобы и эта плоть, непорочно послужившая Духу, обрела место
успокоения, чтобы не оказалась без награды непорочная и чистая, в которой
поселился Святой Дух. Вот тебе объяснение этой притчи.

\vs 3Er 11:1
Возрадовался я, господин,
сказал я, услышав такое объяснение.
\vs 3Er 11:2
Слушай далее. Эту плоть
храни неоскверненною и чистою, чтобы дух, живущий в ней, был доволен ею и
спаслась твоя плоть.
\vs 3Er 11:3
Смотри также, никогда не
допускай мысли, что эта плоть погибнет, и не злоупотребляй ею в какой-либо
похоти.
\vs 3Er 11:4
Ибо если осквернишь плоть
свою, то осквернишь и Духа Святого, если же осквернишь Духа Святого, не будешь
жить.
\vs 3Er 11:5
И спросил я: что же, если
кто по неведению, до того, как услышать эти слова, осквернил свою плоть, каким
образом получит он спасение?
\vs 3Er 11:6
Прежние грехи неведения,~--- сказал он,~--- исцелить может один Бог, ибо Ему принадлежит всякая власть.
\vs 3Er 11:7
Но теперь храни себя; и
Господь Всемогущий и милостивый даст искупление для прежних грехов, если
впредь не осквернишь плоти своей и духа. Ибо они взаимопричастны, и одна без
другого не оскверняется.
\vs 3Er 11:8
Итак, и то и другое
сохраняй чистым и будешь жить с Богом.

\chhdr{Подобие 6-е.}
\vs 3Er 12:1
Когда я, сидя дома, прославлял Господа за всё то, что видел,
и размышлял о заповедях, как они прекрасны, тверды, почтенны и сладостны и
могут спасти душу человека,
\vs 3Er 12:2
то я говорил сам себе:
<<Блажен буду, если стану поступать по этим заповедям; и всякий поступающий по
ним, будет блажен!>>
\vs 3Er 12:3
Когда рассуждал таким
образом, вдруг пастырь появился возле меня
\vs 3Er 12:4
и сказал: что раздумываешь
о заповедях моих, которые я тебе преподал? Они прекрасны, нисколько не
сомневайся; но облекись верою в Господа и будешь исполнять их, ибо наделю тебя
для этого силой.
\vs 3Er 12:5
Заповеди эти полезны для
тех, которые хотят покаяться; если не будут исполнять их, то тщетным будет их
покаяние.
\vs 3Er 12:6
Итак, вы, кающиеся,
отриньте от себя лукавства этого века, губящие вас. Облекитесь же во всякую
добродетель, чтобы вы могли соблюсти эти заповеди, и ничего не прибавляйте к
грехам вашим.
\vs 3Er 12:7
Ибо если снова не будете
грешить, то загладите прежние грехи. Поступайте по заповедям моим и будете
жить с Богом. Все это мною наказано вам.
\vs 3Er 12:8
После этих слов он
продолжал: пойдем в поле, и я покажу тебе пастухов овец.
\vs 3Er 12:9
Пойдем, господин,~--- согласился я.
\vs 3Er 12:10
Пошли мы и в поле увидели
молодого пастуха, одетого в богатые одежды багряного цвета; стадо его было
многочисленно, и ухоженные овцы весело резвились в травах. И сам пастух
радовался на свое стадо и с довольным лицом ходил около овец.

\vs 3Er 13:1
Ангел указал мне на
пастуха и сказал: это~--- ангел наслаждения и лжи, он изводит души рабов Божьих,
отвращая их от истины, обольщая злыми пожеланиями;
\vs 3Er 13:2
и они забывают заповеди
живого Бога и живут в роскоши и суетных удовольствиях, и этот злой ангел губит
их~--- некоторых до смерти, а некоторых до растления.
\vs 3Er 13:3
Господин,~--- спросил я,~--- как понять <<до смерти>>
и что значит <<до растления>>?
\vs 3Er 13:4
Слушай. Овцы, которых ты
видел резвящимися, это те, которые навсегда отреклись от Бога и предались
удовольствиям этого века.
\vs 3Er 13:5
Поэтому им нет возврата к
жизни через покаяние, ибо они к другим своим преступлениям прибавили еще
больше~--- нечестиво хулили имя Господа. Жизнь таких людей подобна смерти.
\vs 3Er 13:6
А овцы, которые не скакали
по полю, а скучно паслись, означают тех, которые хоть и предавались
наслаждениям и удовольствиям, но не возводили хулы против Господа: они не
отошли от истины, и для них есть еще покаяние, посредством которого они спасут
жизнь.
\vs 3Er 13:7
В растлении есть некоторая
надежда на восстановление; а смерть имеет окончательную погибель.
\vs 3Er 13:8
Еще прошли мы немного, и
он показал мне большого пастуха, дикого на вид, одетого в белую козью шкуру, с
сумой на плечах, сучковатой и крепкой палкой и большим бичом в руках; лицо его
было суровое и грозное, так что становилось страшно.
\vs 3Er 13:9
Он принимал от юного
пастуха овец, которые жили в неге и наслаждении, но не скакали; он отгонял их
в местность скалистую и тернистую, и овцы, запутавшись в колючках, сильно
страдали, а пастух осыпал их ударами, гонял туда и сюда, не давая им покоя и
не позволяя где-либо остановиться.

\vs 3Er 14:1
Видя, что овцы
подвергаются побоям, терпят такие мучения и не находят покоя, я пожалел их и
спросил пастыря, кто этот безжалостный и жестокий пастух, не имеющий ни
малейшего сострадания к овцам.
\vs 3Er 14:2
Это,~--- ответил пастырь,~--- ангел наказания; он из праведных ангелов, но приставлен для наказания. Ему
вверяются те, которые уклонились от Бога и предались похотям и удовольствиям
этого века; и он наказывает их, как они того заслуживают, различными жестокими
мучениями.
\vs 3Er 14:3
Расскажи мне, господин,~---
попросил я,~--- что это за мучения, какого рода они?
\vs 3Er 14:4
Слушай: эти различные
наказания и мучения~--- те, которые люди терпят в своей ежедневной жизни. Одни
терпят убытки, другие~--- бедность, иные~--- различные болезни,
некоторые~--- непостоянство в жизни, другие подвергаются обидам от людей недостойных и
многим иным неприятностям.
\vs 3Er 14:5
Очень многие с
непостоянными намерениями принимаются за различные дела, но ничто им не
удается, и жалуются они, что не имеют успеха в своих начинаниях; не приходит
им мысль, что они творят худые дела, но жалуются на Господа.
\vs 3Er 14:6
После того, как натерпятся
они всякой скорби, они предаются мне для доброго увещевания, укрепляются в
вере в Господа и в остальные дни жизни своей служат Господу чистым сердцем.
\vs 3Er 14:7
И когда начнут они каяться
в преступлениях, тогда на сердце их приходят беззаконные дела их и они воздают
славу Господу, говоря, что Он~--- Судия праведный и что они всё претерпели
достойно по делам своим.
\vs 3Er 14:8
И в остальное время служат
Богу чистым сердцем и имеют успех во всех делах своих, получая от Бога всё,
чего ни попросят; и тогда благодарят Бога, что вручены мне, и уже не
подвергаются более никакой жестокости.

\vs 3Er 15:1
И захотел я узнать,
столько ли времени мучаются оставившие страх Божий, сколько наслаждались
удовольствиями, и спросил пастыря об этом.
\vs 3Er 15:2
Столько же времени и
мучаются,~--- ответил он.
\vs 3Er 15:3
Мало они мучаются, надобно
бы предавшимся удовольствиям и забывшим Бога терпеть наказания в семь раз
более.
\vs 3Er 15:4
Неразумен ты,~--- сказал он,
и не понимаешь силы наказания.
\vs 3Er 15:5
Господин, если бы я
понимал, то и не просил бы тебя объяснить мне.
\vs 3Er 15:6
Слушай,~--- сказал он,~--- какова сила того и другого~--- наслаждения и наказания. Один час наслаждения
ограничивается своим протяжением, а один час наказания имеет силу тридцати
дней.
\vs 3Er 15:7
Кто один день предавался
наслаждению и удовольствию, тот будет мучиться один день, но день мучения
будет стоить целого года.
\vs 3Er 15:8
Следовательно, сколько
дней кто наслаждается, столько лет мучится. Видишь,~--- заключил он,~--- что время
мирского наслаждения и обольщения очень кратко, а время наказания и мучения
велико.

\vs 3Er 16:1
Я сказал ему: не совсем
понимаю относительно времени наслаждения и наказания, объясни мне лучше.
\vs 3Er 16:2
Он ответил: неразумие твоё
упорно остается с тобою, и ты не хочешь очистить сердце своё и служить Богу.
Смотри, чтобы не оказаться тебе неразумным, когда исполнится время.
\vs 3Er 16:3
А теперь слушай, если
желаешь понять. Кто один день предавался удовольствиям и делал, что было
угодно душе его, тот исполняется великим неразумием и наутро не понимает своих
действий и не помнит, что делал накануне, ибо наслаждение и обольщение не
имеют никакой памяти по причине неразумия, которым человек исполняется.
\vs 3Er 16:4
Но когда на один день
придет человеку наказание и мучение, то он страдает целый год, потому что
наказание и мучение имеют великую память.
\vs 3Er 16:5
Страдающий в течение
целого года вспоминает и о суетном наслаждении и сознаёт, что за него он
терпит зло.
\vs 3Er 16:6
Таким-то образом
наказываются те, которые предались наслаждению и обольщению; потому что,
наделенные жизнью, сами себя предали смерти.
\vs 3Er 16:7
Я спросил: господин, какие
удовольствия вредны?
\vs 3Er 16:8
Любое дело,~--- ответил он,
доставляет удовольствие человеку, если он выполняет его с приятностью.
\vs 3Er 16:9
Ибо и гневливый, исполняя
свое дело, получает удовольствие, и расово смешивающийся, и пьяница, и
клеветник, и лжец, и любостяжательный человек, и хищник, и всякий совершающий
что-либо подобное удовлетворяет свою страсть и наслаждается своим делом. Все
эти наслаждения вредны рабам Божьим, и за них-то они страдают и терпят
наказания.
\vs 3Er 16:11
Но есть также
удовольствия, спасительные для людей: многие, совершая добрые дела, получают
удовольствие, находя в них для себя сладость. Это удовольствие полезно рабам
Божьим и приготовляет жизнь таким людям.
\vs 3Er 16:12
А те, о которых сказано
прежде, заслуживают наказания и мучения, и те, которые будут нести их и не
покаются в своих преступлениях, обрекут себя на смерть.

\chhdr{Подобие 7-е.}
\vs 3Er 17:1
Спустя несколько дней я встретил пастыря на том поле, на
котором прежде видел пастухов,
\vs 3Er 17:2
и спросил он меня: чего ты
ищешь?
\vs 3Er 17:3
Я пришел, господин,
просить тебя, чтобы ты приказал удалиться из моего дома пастырю,
приставленному для наказания, потому что он сильно поражает меня.
\vs 3Er 17:4
Он сказал мне в ответ:
необходимо пережить тебе бедствия и скорби, потому что так заповедал тебе тот
славный ангел, который хочет испытать тебя.
\vs 3Er 17:5
Какое же зло, господин,
сделал я, что предан этому ангелу?
\vs 3Er 17:6
Слушай,~--- сказал он. Ты
имеешь очень много грехов, но не столь много, чтобы следовало тебя предать
этому ангелу;
\vs 3Er 17:7
но домочадцы твои
совершили великие грехи и преступления, и тот славный ангел прогневался на их
дела и повелел понести тебе наказание некоторое время, чтобы и они покаялись в
своих прегрешениях и очистились от всякой скверны этого века. И когда они
покаются и очистятся, тогда удалится от тебя ангел наказания.
\vs 3Er 17:8
Я сказал ему: господин,
если они так вели себя, что рассердили славного ангела, в чем же моя вина?
\vs 3Er 17:9
Он отвечал: они не могут
быть наказаны, если ты, глава всего дома, не подвергнешься наказанию. Ибо всё,
что претерпишь ты, неизбежно претерпят и они, а при твоем благополучии они не
могут испытать никакого мучения.
\vs 3Er 17:10
Но теперь, господин,~--- сказал я,~--- они уже покаялись от всего сердца своего.
\vs 3Er 17:11
Знаю, что они покаялись
от всего сердца. Но не думаешь ли ты, что тотчас отпускаются грехи кающихся?
Нет, кающийся должен помучить свою душу, смириться во всяком деле своем и
перенести многие и различные скорби.
\vs 3Er 17:12
И когда перенесет всё,
что ему назначено, тогда, конечно, Тот, Который всё сотворил и утвердил,
подвигнется к нему Своею милостью и даст ему спасительное врачевание, и лишь
тогда, когда увидит, что сердце кающегося чисто от всякого злого дела.
\vs 3Er 17:13
А тебе и семейству твоему
пострадать теперь полезно. Нужно пострадать так, как повелел тот ангел
Господа, который мне предал тебя.
\vs 3Er 17:14
А ты лучше благодари
Господа, что Он удостоил предварительно открыть тебе наказание, чтобы, наперед
зная о нём, ты стойко перенёс его.
\vs 3Er 17:15
И я просил его: господин,
будь со мною, и я легко перенесу всякое бедствие.
\vs 3Er 17:16
Я буду с тобою и даже
попрошу ангела наказания, чтобы он легче поражал тебя; впрочем, не долго ты
потерпишь бедствие и снова возвратишься в свое благосостояние, только пребывай
в смиренномудрии и повинуйся Господу от чистого сердца.
\vs 3Er 17:17
Пусть и дети твои, и весь
дом твой живут по заповедям, которые я тебе преподал,~--- и покаяние ваше может
сделаться твердым и чистым.
\vs 3Er 17:18
И если ты с семьей своей
соблюдешь мои заповеди, то удалится от тебя всякое бедствие; и от всех тех,
которые будут придерживаться этих заповедей, удалится всякое бедствие.

\chhdr{Подобие 8-е.}
\vs 3Er 18:1
Пастырь показал мне заросли ивы, покрывшие поля и горы, в
тень которых пришли все призванные в имени Господа.
\vs 3Er 18:2
И подле этой ивы стоял славный, весьма высокий ангел, он большим серпом срезал
с ивы ветки и раздавал их народу.
\vs 3Er 18:3
После того, как все получили ветки, ангел положил серп, но дерево осталось
таким же целым, каким я видел его прежде. Очень я удивился этому,
\vs 3Er 18:4
а пастырь сказал: не удивляйся, что дерево осталось невредимо после того, как
срезано было с него столько веток. Подожди, что будет дальше, и станет
понятным тебе, что всё это означает.
\vs 3Er 18:5
Ангел, раздававший ветки,
потребовал их назад. В том же порядке, в каком получали, он подзывал людей:
они подходили и возвращали ветки.
\vs 3Er 18:6
Ангел Господень принимал
их и рассматривал. От некоторых он получал сухие, как бы изъеденные молью
ветки, и тем он повелел стать отдельно; те, которые вернули ветки сухие, но не
тронутые молью, тоже стали отдельно.
\vs 3Er 18:7
Особо стали и те, кто
принес ветки полусухие и с трещинами, и те, чьи ветки были наполовину сухие,
наполовину зеленые.
\vs 3Er 18:8
Некоторые возвращали ветки
на две трети сухими, а на треть~--- зелеными; а некоторые~--- наоборот: на две
трети зелеными и на треть~--- сухими. Ангел их также поставил отдельно.
\vs 3Er 18:9
Иные подавали ветки
полностью зеленые, и только малая часть их, самая верхушка, была сухая, и они
были потрескавшиеся.
\vs 3Er 18:10
А в других ветках было
совсем мало зеленого.
\vs 3Er 18:11
А у большинства людей
были такие же зеленые ветки, какими они их и получили; ангел весьма радовался
им.
\vs 3Er 18:12
Иные отдавали ветки
зелеными и с молодыми побегами, ангел принимал их также с большим
удовольствием.
\vs 3Er 18:13
У некоторых зеленые ветки
были и с новыми отростками, и с плодами на них. Мужи, возвращающие такие
ветки, приходили с очень довольным видом, и сам ангел был весьма весел, и
пастырь тоже радовался.

\vs 3Er 19:1
Потом ангел Господа велел
принести венцы.
\vs 3Er 19:2
Принесены были венцы,
словно сплетенные из пальмовых листьев, и ангел надел их на тех мужей, ветки
которых были с отростками и плодами, и велел им идти в башню;
\vs 3Er 19:3
и других мужей, ветки
которых были зелены и с побегами, но без плодов, послал туда же, дав им
печать.
\vs 3Er 19:4
На всех входивших в башню
была одежда, белая как снег.
\vs 3Er 19:5
В ту же башню послал он и
тех, которые возвратили свои ветки такими же зелеными, как приняли, дав им
печать и белую одежду.
\vs 3Er 19:6
По окончании этого он
обратился к пастырю: я пойду, а ты впусти их внутрь стен, на то место, какое
каждый заслужил, но прежде рассмотри внимательно их ветки; следи, чтобы
кто-нибудь не миновал тебя; если же кто пройдет мимо, я обличу их перед
алтарем.
\vs 3Er 19:7
Он удалился, после чего
пастырь сказал мне: возьмем у них ветки и посадим их в землю, может быть,
некоторые из них зазеленеют снова?
\vs 3Er 19:8
Я удивился: господин,
каким образом могут снова зазеленеть ветки, которые уже засохли?
\vs 3Er 19:9
Он ответил мне: это дерево
ива, и оно всегда любит жизнь: поэтому, если эти ветки будут посажены и
получат чуть-чуть влаги, очень многие из них опять зазеленеют.
\vs 3Er 19:10
Попробую полью их водой,
и если какая из них сможет ожить, порадуюсь за неё; если же нет, по крайней
мере, видно будет, что я не был небрежен.
\vs 3Er 19:11
Потом пастырь приказал
мне позвать их в том порядке, в каком они стояли; подошли они и передали свои
ветки. Получив их, пастырь каждую посадил по порядку.
\vs 3Er 19:12
И, рассадив, так обильно
поливал их водою, что вода полностью покрыла их.
\vs 3Er 19:13
Полив, он сказал: пойдем,
а через несколько дней возвратимся и осмотрим все ветки. Ибо Сотворивший это
дерево хочет, чтобы были живы все происшедшие от него ветки.
\vs 3Er 19:14
А я надеюсь, что после
того, как эти ветки политы водою, очень многие из них оживут, напоенные
влагою.

\vs 3Er 20:1
Я попросил: господин,
объясни мне, что означает это дерево; я недоумеваю, почему оно остается целым:
ведь срезано с него столько веток, но не видно, чтобы от него что-нибудь
убавилось?
\vs 3Er 20:2
Слушай,~--- сказал он. Это
большое дерево, покрывающее поля и горы и всю землю, означает Закон Божий,
данный всему миру.
\vs 3Er 20:3
Закон этот есть Сын Божий,
проповеданный во всех концах земли. Люди, стоящие под сенью его, означают тех,
которые услышали проповедь и уверовали в Него.
\vs 3Er 20:4
Величественный и сильный
ангел есть Михаил, который имеет власть над этим народом и управляет им: он
насаждает Закон в сердцах верующих и наблюдает за теми, которым дал Закон,
соблюдают ли они его.
\vs 3Er 20:5
У каждого есть ветки:
ветки означают также Закон Господа.
\vs 3Er 20:6
Видишь, многие из них
сделались негодными, и ты узнаешь всех тех, которые не соблюли Закона, и
увидишь место каждого из них.
\vs 3Er 20:7
Почему же, господин, одних
Он отослал в башню, а других здесь оставил, при тебе?
\vs 3Er 20:8
Те, которые преступили
Закон, от Него принятый, оставлены в моей власти, чтобы покаялись в своих
преступлениях; а которые удовлетворили Закону и его соблюли, находятся под
собственною Его властью.
\vs 3Er 20:9
Кто же, господин, те,
которые увенчаны и вошли в башню?
\vs 3Er 20:10
Он ответил: это те,
которые вели борьбу с дьяволом и победили его; те, которые, соблюдая Закон,
пострадали за него;
\vs 3Er 20:11
другие, которые
возвратили ветки зелеными и с отростками, но без плодов,~--- это те, которые,
хотя и потерпели мучение за тот Закон, но не вкусили смерти и не отреклись от
своего Закона;
\vs 3Er 20:12
те же, которые возвратили
зелеными, какими и взяли, суть кроткие и праведные, которые жили с чистым
сердцем и соблюли заповеди Божии.
\vs 3Er 20:13
Остальное ты узнаешь
тогда, когда пересмотрю ветки, которые я посадил в землю и полил.

\vs 3Er 21:1
Через несколько дней мы
возвратились туда, и пастырь сел на месте того ангела, а я стал подле него, и
он велел мне подпоясаться полотенцем и помогать ему.
\vs 3Er 21:2
Я подпоясался чистым
платом, сделанным из мешка. Видя, что я готов служить ему,
\vs 3Er 21:3
он сказал: зови тех мужей,
ветки которых посажены в землю, в том порядке, в каком каждый их подавал.
\vs 3Er 21:4
И отправился я в поле,
созвал всех, и они стали на свои места. Пусть каждый вынет свою ветку и подаст
мне,~--- указал он.
\vs 3Er 21:5
Прежде всего подали те, у
которых были ветки сухие и гнилые. И так как они опять оказались загнившими и
сухими, то он повелел им стать отдельно.
\vs 3Er 21:6
После подали те, у которых
ранее они были сухие, но не гнилые. Одни из них подали ветки зеленые, а другие
сухие и загнившие, как бы тронутые молью.
\vs 3Er 21:7
Тем, которые подали
зеленые, велел он стать отдельно; а тем, которые подали сухие и загнившие,
велел стать вместе с первыми.
\vs 3Er 21:8
Потом подали те, чьи были
полузасохшие и с трещинами; многие из них принесли ветки зеленые и без трещин;
а некоторые~--- зеленые, имеющие побеги и даже плоды~--- как те, которые
увенчанные вошли в башню;
\vs 3Er 21:9
другие подали сухие и
поврежденные, иные сухие, но не гнилые, а некоторые полусухие и с трещинами,
какими и прежде были.
\vs 3Er 21:10
И всех их пастырь
разделил на группы, повелел каждой стать отдельно.

\vs 3Er 22:1
Потом принесли ветки те, у
которых они были хотя зеленые, но с трещинами: все они подали их теперь
зелеными и стали на своем месте, и пастырь радовался за них, что все они
оправились и заживили свои трещины.
\vs 3Er 22:2
Подали и те, которые
прежде имели ветки наполовину сухие; ветки некоторых из них оказались все
зелеными, других~--- полусухими, иных~--- сухими и поврежденными,
а иных~--- зелёными и с отростками.
\vs 3Er 22:3
Потом подали те, у которых
ветки на две трети были зеленые и на треть сухие; многие из них подали ветки
зеленые, многие полусухие, прочие же сухие и гнилые.
\vs 3Er 22:4
Далее подали те, у которых
до того ветки на две трети были сухие, а на треть зеленые; из них многие
подали полусухие, некоторые сухие и гнилые, другие полусухие и с трещинами, а
иные зеленые.
\vs 3Er 22:5
Потом подали те, у которых
ветки были зелены, но немного и сухи и с трещинами; из них некоторые
возвратили ветки зеленые, другие же зеленые и с побегами; и они отошли на свое
место.
\vs 3Er 22:6
Наконец, у тех, у которых
в ветках было немного зелени, а остальное засохло, ветки большею частью
оказались зелеными, с отростками и даже с плодом на них, а остальные были
зеленые. Этими ветками пастырь весьма был доволен.
\vs 3Er 22:7
И каждого он отправлял на
своё место.

\vs 3Er 23:1
Пересмотрев все ветки,
сказал мне пастырь: я говорил тебе, что дерево это любит жизнь. Видишь, многие
покаялись и получили спасение.
\vs 3Er 23:2
Вижу, господин.
\vs 3Er 23:3
Знай же,~--- продолжал он,~--- велики и славны благость и милость Господа, Который дал дух, способный
покаяться.
\vs 3Er 23:4
Почему же, господин,~--- спросил я,~--- не все покаялись?
\vs 3Er 23:5
Он ответил: Господь дал
покаяние тем, чьи сердца, он видел, будут чисты и кто будет служить Ему
усердно и праведно.
\vs 3Er 23:6
А тем, у которых
чувствовал лукавство, и неправду, и притворное к Нему обращение, не дал
покаяния, чтобы они снова не осквернили имени Его.
\vs 3Er 23:7
Теперь, господин, объясни
мне, что означает каждый из тех, кто возвратил ветки, и где его место, чтобы
узнали об этом уверовавшие, которые получили печать, но сокрушили её и не
сохранили в целости
\vs 3Er 23:8
и, дабы, познав дела свои,
покаялись и, приняв от тебя печать, воздали славу Господу, что подвигся Он к
ним Своею милостью, и послал тебя для обновления душ их.
\vs 3Er 23:9
Слушай,~--- сказал он. У
кого ветки найдены сухими и гнилыми, как бы поврежденными тлёю,~--- это суть
отступники и предатели Церкви, которые во грехах своих хулили Господа и
постыдились имени Его, на них призванного: все они умерли для Бога.
\vs 3Er 23:10
И ты видишь, что никто из
них не покаялся, и они презрели слова Божьи, которые я заповедал тебе; от этих
людей отступила жизнь.
\vs 3Er 23:11
Равным образом недалеко
от них те, которые возвратили ветки сухими, хотя не гнилыми, ибо они были
лицемеры, вводили чуждые учения и совращали рабов Божьих, особенно тех,
которые согрешили, не дозволяя им возвращаться к покаянию, но внушая им
вредные мысли.
\vs 3Er 23:12
Они имеют надежду
покаяния; и ты видишь, что многие из них уже покаялись после того, как я
возвестил им мои заповеди, и еще покаются.
\vs 3Er 23:13
Те, которые не покаются,
потеряли жизнь свою; те же, которые покаялись, сделались добрыми и
местопребыванием их стали первые стены, а некоторые вошли даже внутрь башни.
\vs 3Er 23:14
Итак, видишь, покаяние
грешников несет в себе жизнь, а нераскаянность~--- смерть.

\vs 3Er 24:1
Послушай и о тех, которые
вернули ветки полусухие и с трещинами,~--- говорил пастырь далее.
\vs 3Er 24:2
Те, у которых ветки были
только полусухие,~--- это сомневающиеся: они ни живы, ни мертвы; а те, которые
подали ветки полусухие и с трещинами,~--- это сомневающиеся и вместе с тем
злоязычные, которые поносят отсутствующих, никогда не живут в мире, но
постоянно находятся в раздоре.
\vs 3Er 24:3
Впрочем, и им есть
покаяние. Видишь, и из них некоторые покаялись.
\vs 3Er 24:4
Из них немедленно
покаявшиеся найдут себе место в башне, а те, которые позднее покаялись, будут
обитать на стенах.
\vs 3Er 24:5
Те же, которые не
покаялись, но остались при своих делах, обретут погибель.
\vs 3Er 24:6
Те, которые подали ветки
зеленые, но с трещинами, всегда были верными и добрыми, хотя имеют между собою
зависть и соперничество о первенстве и достоинстве: только глупы люди,
спорящие между собою о первенстве.
\vs 3Er 24:7
Впрочем, они были добры,
послушались моих заповедей, исправились и скоро покаялись, потому и место их в
башне.
\vs 3Er 24:8
Если же кто-нибудь из них
возвратится к раздору, будет изгнан из башни и погубит жизнь свою.
\vs 3Er 24:9
Ибо жизнь званных Богом
состоит в соблюдении заповедей Господа: в этом жизнь, а не в первенстве или
каком-либо достоинстве.
\vs 3Er 24:10
Чрез терпение и смирение
духа люди получат жизнь от Господа, а пренебрегающие Законом приобретут себе
смерть.

\vs 3Er 25:1
Те, у которых ветки
наполовину сухи, наполовину зелены,~--- это привязанные к мирским занятиям и
отчуждавшиеся от общения со святыми, и потому половина их жива, половина
мертва.
\vs 3Er 25:2
И из них многие,
послушавшись заповедей моих, покаялись и получили место в башне; некоторые же
вовсе отпали.
\vs 3Er 25:3
Для них нет покаяния,
потому что они хулили Господа и наконец отвергли Его, и за это нечестие они
потеряли жизнь свою.
\vs 3Er 25:4
Но многие из них
двоедушествовали: этим еще есть покаяние, и если вскоре покаются, будут иметь
жилище в башне; если позднее~--- будут обитать на стенах; если же совсем не
покаются~--- потеряют жизнь свою.
\vs 3Er 25:5
Те, у которых ветки на две
трети были зеленые, а на треть сухие, означают тех, которые, будучи различным
образом совращены, отреклись от Господа:
\vs 3Er 25:6
из них многие покаялись и
уже получили место в башне; а иные навсегда отпали от Бога и совсем потеряли
жизнь.
\vs 3Er 25:7
А некоторые из них
двоедушествовали и возбуждали раздоры: им еще есть покаяние, если вскоре
покаются и откажутся от своих удовольствий; если же останутся при своих делах,
то приготовят себе смерть.

\vs 3Er 26:1
Подавшие свои ветки на две
трети сухими, а на треть зелеными суть верные, но, обогатившись и обретя славу
среди народов, они впали в большую гордость, стали высокомерными, оставили
истину и не имели общения с праведными, но жили вместе с народами, и эта жизнь
казалась им приятнее; от Бога, впрочем, они не отпали и сохраняли веру; только
не творили дела веры.
\vs 3Er 26:2
Многие из них уже
покаялись и стали обитать в башне.
\vs 3Er 26:3
Другие, живя с народами и
набравшись надменного тщеславия у них, совершенно отошли от Бога, предавшись
делам народов: такие люди причислились к народам.
\vs 3Er 26:4
Некоторые же из них начали
колебаться, не надеясь спастись по делам, ими совершаемым; другие пришли в
сомнение и стали возбуждать несогласия.
\vs 3Er 26:5
И тем и другим еще есть
покаяние, но покаяние их должно быть немедленным, чтобы осталось для них место
в башне.
\vs 3Er 26:6
А тем, которые не
раскаются, пребывая в своих удовольствиях, скоро предстоит смерть.

\vs 3Er 27:1
Те, которые подали ветки
зеленые, за исключением их сухих верхушек, и с трещинами, те всегда были
добрыми, верными и славными у Бога, но согрешили несколько раз по причине
небольших удовольствий и мелких несогласий, которые имели между собою.
\vs 3Er 27:2
Услышав слова мои, очень
многие тотчас покаялись, и место их стало в башне.
\vs 3Er 27:3
Некоторые из них пришли в
сомнение, а некоторые, сверх того, произвели большой раздор. Для таких есть
надежда покаяния, потому что всегда были добрыми и едва ли кто из них умрет.
\vs 3Er 27:4
Те же, которые подали
сухие ветки с зелеными верхушками, они только уверовали в Бога, но творили
беззаконие; впрочем, они никогда не отступали от Бога, но всегда охотно носили
Его имя и с любовью принимали рабов Божьих в дома свои.
\vs 3Er 27:5
Услышав о покаянии, они
немедленно покаялись и делают всякую добродетель и правду.
\vs 3Er 27:6
Некоторые из них
претерпели смерть, а другие охотно перенесли несчастия,
помня о делах своих,~--- всем таковым место будет в башне.

\vs 3Er 28:1
Окончив объяснение всех
веток, он повелел мне: пойди и скажи всем, чтобы покаялись и жили для Бога,
потому что Господь по Своему милосердию послал меня дать всем покаяние, даже и
тем, которые по делам своим не заслуживают спасения. Но терпелив Господь и
хочет, чтобы спаслись призванные Его Сыном.
\vs 3Er 28:2
Я надеюсь, господин,~--- ответил я,~--- что все услышавшие это покаются. Ибо я убежден, что всякий
обратится к покаянию, познав дела свои и убоявшись Бога.
\vs 3Er 28:3
Все те, которые от всего
сердца покаются и очистятся от всех неправедных дел, о которых говорилось
прежде, и не приумножат еще чем-либо свои преступления, получат от Господа
прощение прежних грехов своих, если не усомнятся в этих заповедях моих и будут
жить с Богом.
\vs 3Er 28:4
И ты ходи в этих заповедях
и будешь жить с Богом; и все, кто только будет верно исполнять их, будут жить
с Богом.
\vs 3Er 28:5
Показав мне всё это, он
пообещал: остальное я покажу тебе спустя несколько дней.

\chhdr{Подобие 9-е.}
\vs 3Er 29:1
После того как я написал заповеди и притчи пастыря, ангела
покаяния, он пришел ко мне и сказал: я хочу показать тебе всё, что показал
тебе Дух Святой, Который беседовал с тобою в образе Церкви: Дух тот есть Сын
Божий.
\vs 3Er 29:2
И так как ты был слаб телом, то не было открываемо тебе через ангела, доколе
ты не утвердился духом и не укрепился силами, чтобы мог видеть ангела.
\vs 3Er 29:3
Тогда Церковью показано было тебе строение башни хорошо и величественно; но ты
видел, как было показано тебе всё девою.
\vs 3Er 29:4
А теперь ты получишь откровение через ангела, но от того же Духа. Ты должен
тщательно всё узнать от меня; ибо для того и послан я тем досточтимым ангелом
обитать в доме твоём, чтобы ты рассмотрел всё хорошо, ничего не страшась, как
прежде.
\vs 3Er 29:5
И повел он меня в Аркадию,
на гору, имеющую форму груди, и сели мы на её вершине. И показал он мне
большое поле, которое окружали двенадцать гор, не похожих одна на другую.
\vs 3Er 29:6
Первая из них была черная
как сажа. Вторая была голая, без растений. Третья заросла сорняками и
терниями. На четвертой были растения полузасохшие, с зеленой верхушкой и
мёртвым стеблем, а некоторые растения совсем засохли от солнечного жара.
\vs 3Er 29:7
Пятая гора была скалистая,
но на ней зеленели растения. Шестая гора была с расселинами, в иных местах
малыми, в других большими; в этих расселинах были растения, но не цветущие, а
слегка увядшие.
\vs 3Er 29:8
На седьмой горе цвели
растения, и была она плодородна: всякий скот и птицы небесные собирали там
корм, и чем более питались они на ней, тем обильнее росли растения.
\vs 3Er 29:9
Восьмую гору сплошь
покрывали источники, и из этих источников утоляли жажду твари Божьи. Девятая
гора вовсе не имела никакой воды и вся была обнажена: на ней обитали ядовитые
змеи, гибельные для людей.
\vs 3Er 29:10
Десятая гора вся была
затенена огромными деревьями, на ней растущими, и в тени лежал скот, отдыхая и
пережевывая жвачку.
\vs 3Er 29:11
На одиннадцатой горе тоже
во множестве росли деревья, и они изобиловали разными плодами, и видевший их
желал вкусить этих плодов. Двенадцатая гора, вся белая, имела вид самый
приятный, всё было на ней прекрасно.

\vs 3Er 30:1
В середине поля он показал
мне огромный белый камень; камень этот, квадратный по форме, был выше тех гор,
так что мог бы держать всю землю.
\vs 3Er 30:2
Он был древний, но имел
высеченную дверь, которая казалась недавно сделанною. Дверь эта сияла ярче
солнца, так что я поразился ее блеску
\vs 3Er 30:3
Двенадцать дев стояли
возле двери, по четырем сторонам её, в середине попарно.
\vs 3Er 30:4
Четверо из них, стоявшие
по углам двери, показались мне самыми великолепными, но и остальные были
прекрасны.
\vs 3Er 30:5
Веселые и радостные, эти
девы одеты были в полотняные туники, красиво подпоясанные; их правые плечи
были обнажены, словно девы намеревались нести какую-то ношу.
\vs 3Er 30:6
Я залюбовался этим
величественным и дивным зрелищем, но в то же время недоумевая, что девы,
будучи столь нежны, стояли мужественно, будто готовясь понести на себе целое
небо.
\vs 3Er 30:7
И когда размышлял я так,
пастырь сказал мне: что размышляешь ты и недоумеваешь и сам на себя навлекаешь
заботу? Чего не можешь понять, за то не берись, но проси Господа, чтобы
вразумил понять это.
\vs 3Er 30:8
Что за тобою, того не
можешь видеть; а видишь, что перед тобою. Чего не можешь видеть, то оставь и
не мучь себя.
\vs 3Er 30:9
Владей тем, что видишь, о
прочем же не беспокойся. Я объясню тебе всё, что покажу; а теперь смотри, что
будет дальше.

\vs 3Er 31:1
И вот увидел я, что пришли
шесть высоких и почтенных мужей, и все были похожи один на другого; они
призвали множество других мужей, которые также были высоки, красивы и сильны.
\vs 3Er 31:2
И те шесть мужей приказали
строить башню над дверью.
\vs 3Er 31:3
Тогда мужи, которые пришли
для строительства башни, подняли великий шум и беготню около двери.
\vs 3Er 31:4
Девы, стоявшие при двери,
сказали им поспешить со строительством и сами протянули свои руки, как бы
готовясь что-нибудь брать у них.
\vs 3Er 31:5
Те шестеро приказали
доставать камни со дна и подносить их к башне. И подняты были десять камней
белых, квадратных, обтесанных.
\vs 3Er 31:6
Те шесть мужей подозвали
дев и приказали им носить все камни, которые должны были идти на
строительство, проходить через дверь и передавать камни строителям башни.
\vs 3Er 31:7
И тотчас же девы начали
возлагать друг на друга первые камни, извлеченные со дна, и носить их вместе
по одному камню.

\vs 3Er 32:1
Как стояли девы около
двери, так они и носили: те, которые казались сильнее, брались за углы камня,
а другие держали по бокам.
\vs 3Er 32:2
И таким образом носили они
все камни, проходили через дверь, как было велено, и передавали строителям
башни; а те, принимая их, строили.
\vs 3Er 32:3
Башня строилась на большом
камне, над дверью. Те десять камней были положены в основание башни: камень же
и дверь держали на себе всю башню.
\vs 3Er 32:4
После извлекли со дна
другие двадцать пять камней, и они были принесены девами и использованы для
строительства башни.
\vs 3Er 32:5
После них подняли другие
тридцать пять, которые подобным же образом уложили в башню.
\vs 3Er 32:6
Затем подняли еще сорок
камней, и они все пошли на строительство этой башни.
\vs 3Er 32:7
Таким образом в основание
башни легло четыре ряда камней.
\vs 3Er 32:8
Когда закончились все
камни, которые брали со дна, немного отдохнули строители.
\vs 3Er 32:9
Потом те шесть мужей
приказали народу приносить для башни камни с двенадцати гор.
\vs 3Er 32:10
И стали мужи приносить со
всех гор камни обсеченные, различных цветов, и подавали их девам, а те
проносили их через дверь и подавали строителям.
\vs 3Er 32:11
И когда эти разнообразные
камни были положены в здание, то изменили свои прежние цвета и сделались
белыми и одинаковыми.
\vs 3Er 32:12
Но некоторые камни не
были передаваемы девами и не проносились через дверь, а подавались самими
мужами прямо в строение и не делались светлыми, а оставались такими, какими
клались.
\vs 3Er 32:13
Эти камни безобразно
смотрелись в здании башни. Увидев их, те шесть мужей приказали вынуть и
положить на то место, откуда их взяли.
\vs 3Er 32:14
И сказали они тем,
которые приносили эти камни: вы совсем не подавайте камней для строения, но
кладите их возле башни, чтобы девы проносили через дверь и подавали их, иначе
камни не смогут изменить цветов своих, так что не трудитесь понапрасну.

\vs 3Er 33:1
И кончились в тот день
работы, но башня не была завершена; строительство её должно было опять
возобновиться, и только на время сделана некоторая остановка.
\vs 3Er 33:2
Те шесть мужей приказали
строившим удалиться и отдохнуть немного; девам же повелели не отходить от
башни, чтобы охранять её.
\vs 3Er 33:3
После того как ушли все, я
спросил пастыря, почему не окончено здание башни.
\vs 3Er 33:4
Не может оно быть
завершено прежде, нежели придет господин башни и испытает это строение, чтобы,
если окажутся некоторые камни негодными, заменить их, ибо по его воле строится
эта башня,~--- отвечал он.
\vs 3Er 33:5
Господин,~--- попросил я,~---
я желал бы знать, что означает строение башни, а также узнать и об этом камне,
и о двери, и о горах, и о девах, и о камнях, извлеченных со дна и не
отёсанных, но сразу положенных в здание;
\vs 3Er 33:6
и почему сперва положены в
основание десять камней, потом двадцать пять, затем тридцать пять и, наконец,
сорок;
\vs 3Er 33:7
равно и о тех камнях,
которые положены были в строение, но потом вынуты и отнесены на свое место;
всё это, господин, объясни и успокой душу мою.
\vs 3Er 33:8
И сказал он мне: если не
будешь попусту любопытен, то всё узнаешь и увидишь, что дальше будет с этой
башней, и все притчи обстоятельно узнаешь.
\vs 3Er 33:9
Через несколько дней
пришли мы на то же самое место, где сидели прежде, и позвал он меня: пойдем к
башне, ибо господин её придет, чтобы испытать её.
\vs 3Er 33:10
И пришли мы к башне и
никого другого не нашли, кроме дев.
\vs 3Er 33:11
Пастырь спросил их, не
прибыл ли господин башни. И они ответили, что он скоро придет осмотреть это
здание.

\vs 3Er 34:1
И вот, спустя немного
времени, увидел я, что идет великое множество мужей, и в середине муж такого
величайшего роста, что он превышал саму башню;
\vs 3Er 34:2
окружали его шесть мужей,
которые распоряжались строительством, и все те, которые строили эту башню, и
сверх того еще очень многие славные мужи.
\vs 3Er 34:3
Девы, охранявшие башню,
поспешили к нему навстречу; облобызали его, и стали они вместе ходить вокруг
башни.
\vs 3Er 34:4
И он так внимательно
осматривал строение, что испытал каждый камень: по каждому камню он ударил
трижды тростью, которую держал в руке.
\vs 3Er 34:5
Некоторые камни после его
ударов сделались черны как сажа, некоторые шероховаты, другие потрескались,
иные стали коротки, некоторые ни черны, ни белы, другие неровны и не подходили
к прочим камням, иные покрылись множеством пятен. Так разнообразны были камни,
найденные негодными для здания.
\vs 3Er 34:6
Господин повелел убрать
все их из башни и оставить подле неё, а на место их принести другие камни.
\vs 3Er 34:7
И спросили его строившие:
с какой горы прикажешь принести камни и положить на место выброшенных?
\vs 3Er 34:8
Он запретил приносить с
гор, но велел носить с ближайшего поля.
\vs 3Er 34:9
Взрыли поле и нашли камни
блестящие, квадратные, а некоторые и круглые.
\vs 3Er 34:10
И все камни, сколько их
было на этом поле, были принесены и девами пронесены через дверь;
\vs 3Er 34:11
из них квадратные были
обтёсаны и положены на место выброшенных, а круглые не употреблены в здание,
ибо трудно и долго было их обсекать.
\vs 3Er 34:12
Их оставили около башни,
чтобы после обсечь и употребить в здание, потому как они были очень блестящи.

\vs 3Er 35:1
Окончив это,
величественный муж, господин этой башни, призвал пастыря и поручил ему камни,
не одобренные для здания и положенные около башни.
\vs 3Er 35:2
Тщательно очисти эти
камни,~--- велел он,~--- и положи в здание башни те, которые могут приладиться к
прочим, а неподходящие отбрасывай далеко в сторону.
\vs 3Er 35:3
Приказав это, он удалился
со всеми, с кем пришел к башне. Девы же остались около башни охранять её.
\vs 3Er 35:4
И спросил я пастыря: каким
образом эти камни могут снова пойти в здание башни, когда они уже найдены
негодными?
\vs 3Er 35:5
Он отвечал: я из этих
камней большую часть обсеку и использую для строения, и они придутся к прочим.
\vs 3Er 35:6
Господин,~--- сказал я,~--- каким образом, обсечённые, они могут занять то же самое место?
\vs 3Er 35:7
Те, которые кажутся
малыми, пойдут в середину здания; а большие лягут снаружи и будут их
удерживать.
\vs 3Er 35:8
Потом он сказал: пойдем и
через два дня возвратимся и, очистив эти камни, положим в здание. Ибо всё, что
находится около башни, должно быть очищено, а то вдруг случайно явится
господин, увидит, что нечисто около башни, и прогневается; тогда эти камни не
пойдут на строительство башни, и сочтет он меня нерадивым.
\vs 3Er 35:9
Спустя два дня, когда
пришли мы к башне, он сказал мне: рассмотрим все эти камни и узнаем, которые
из них могут идти в здание.
\vs 3Er 35:10
Рассмотрим, господин,~--- ответил я.

\vs 3Er 36:1
Сначала мы рассмотрели
черные камни. Они оказались такими же, какими были отложены от здания.
\vs 3Er 36:2
Он приказал отнести их от
башни и положить отдельно.
\vs 3Er 36:3
Потом он рассмотрел камни
шероховатые и многие из них велел обсечь и девам взять их и положить в здание;
\vs 3Er 36:4
и они, взяв их, положили в
середину башни.
\vs 3Er 36:5
Остальные же он велел
положить с черными камнями, потому что и они оказались черными.
\vs 3Er 36:6
Затем он рассмотрел камни
с трещинами и из них многие обсек и велел чрез дев отнести в здание: они были
положены снаружи, как более крепкие;
\vs 3Er 36:7
остальные же, по множеству
трещин, не могли быть обработанными и потому были удалены от здания башни.
\vs 3Er 36:8
Далее он рассмотрел камни,
которые были коротки: многие из них оказались черными, а некоторые с большими
трещинами, и он велел положить их с теми, которые были отброшены;
\vs 3Er 36:9
остальные же, очищенные и
обработанные, он велел использовать, и девы, взяв их, положили в середину
здания башни, потому что они были не так крепки.
\vs 3Er 36:10
Потом он рассмотрел камни
наполовину белые и наполовину черные: многие из них оказались черными, и он
велел их перенести к отброшенным.
\vs 3Er 36:11
Остальные же все были
найдены белыми и взяты девами и положены снаружи, будучи крепкими, так что
могли удерживать камни, помещенные в середине, ибо в них ничего не было
отсечено.
\vs 3Er 36:12
Затем он рассмотрел камни
неровные и крепкие: некоторые из них отбросил, потому что по причине твердости
нельзя было обработать их;
\vs 3Er 36:13
остальные же были
обсечены и положены девами в середину здания башни, как более слабые.
\vs 3Er 36:14
Далее он рассмотрел камни
с пятнами, и из них немногие оказались черными и были отброшены к прочим;
остальные же оказались белыми~--- они в целости были использованы девами для
строительства и уложены снаружи по причине их твердости.

\vs 3Er 37:1
Потом стал он
рассматривать камни белые и круглые и спросил меня, что делать с ними.
\vs 3Er 37:2
Не знаю, господин,~--- ответил я.
\vs 3Er 37:3
Значит, ты ничего не
можешь придумать насчет них?
\vs 3Er 37:4
Господин,~--- сказал я,~--- не
владею этим искусством, я не каменщик и ничего не могу придумать.
\vs 3Er 37:5
И сказал он: разве не
видишь, что они круглы? Если я захочу сделать их квадратными, то нужно очень
много от них отсекать, но необходимо, чтобы некоторые из них вошли в здание
башни.
\vs 3Er 37:6
Если необходимо,~--- сказал
я,~--- что же ты затрудняешься, не выбираешь, что хочешь, и не подгоняешь в это
здание?
\vs 3Er 37:7
И он выбрал камни большие
и блестящие и обсек их; а девы, взяв их, положили во внешних частях здания.
\vs 3Er 37:8
Остальные же были отнесены
на то же поле, откуда взяты, но не отброшены. Потому что,~--- объяснил пастырь,
несколько еще недостает башне для окончания; господину угодно, чтобы эти
камни пошли в здание башни, так как они очень белы.
\vs 3Er 37:9
Потом призваны были
двенадцать очень красивых женщин, одетых в черное, с обнаженными плечами и
распущенными волосами. Эти женщины казались деревенскими.
\vs 3Er 37:10
Пастырь приказал им взять
отброшенные от здания камни и отнести их на горы, откуда они были принесены.
\vs 3Er 37:11
И они с радостью подняли,
отнесли все камни и положили туда, откуда они взяты.
\vs 3Er 37:12
Когда же не осталось
возле башни ни одного камня, он сказал: обойдем башню и посмотрим, нет ли в
ней какого изъяна.
\vs 3Er 37:13
Обойдя башню, пастырь
увидел, что она прекрасна и построена безукоризненно, и очень развеселился.
\vs 3Er 37:14
И всякий залюбовался бы
постройкою, потому что не было видно ни одного соединения и башня казалась
высеченною из единого камня.

\vs 3Er 38:1
И я, ходя вместе с
пастырем, весьма был доволен таким прекрасным зрелищем.
\vs 3Er 38:2
И повелел он мне: принеси
известь и мелкие черепицы, чтобы мне исправить вид тех камней, которые опять
вынули из здания, ибо всё вокруг башни должно быть ровно и гладко.
\vs 3Er 38:3
И я всё принес, как
приказал он мне, и он добавил: послужи мне: это дело скоро окончится.
\vs 3Er 38:4
Он исправил вид тех камней
и приказал навести порядок около башни.
\vs 3Er 38:5
Тогда девы, взяв веники,
убрали всю грязь и полили водою~--- и место около башни стало красивым и
веселым.
\vs 3Er 38:6
Пастырь сказал мне: всё
очищено; если Господь придет посмотреть эту башню, не найдет ничего, за что бы
укорить нас,
\vs 3Er 38:7
и он хотел удалиться, но я
схватил его за суму и начал умолять его Господом, чтобы объяснил мне
показанное.
\vs 3Er 38:8
Мне нужно отдохнуть
немного, потом я всё объясню тебе,~--- пообещал он.~--- Дожидайся меня здесь.
\vs 3Er 38:9
Господин, что я здесь буду
один делать?
\vs 3Er 38:10
Ты не один,~--- отвечал он,
все девы с тобою.
\vs 3Er 38:11
Господин,~--- попросил я,~--- передай им меня. И он позвал их и сказал: поручаю вам его, пока не вернусь.
\vs 3Er 38:12
И так я остался один с
теми девами. И они были веселы и ласковы со мною, особенно же четыре из них,
превосходнейшие.

\vs 3Er 39:1
Девы сказали: сегодня
пастырь сюда не придет.
\vs 3Er 39:2
Что же я буду делать?
\vs 3Er 39:3
Подожди до вечера, может
быть, придет и будет говорить с тобою, если же не придет, пробудешь с нами,
доколе придет.
\vs 3Er 39:4
Буду дожидаться его до
вечера,~--- решил я,~--- если же не придет, пойду домой и возвращусь поутру.
\vs 3Er 39:5
Но они воспротивились: ты
нам перепоручен и не можешь уйти от нас.
\vs 3Er 39:6
Я спросил тогда: где я
останусь?
\vs 3Er 39:7
С нами,~--- ответили они,~--- ты уснешь, как брат, а не как муж, ибо ты~--- брат наш и после мы будем обитать
с тобою, потому что очень тебя полюбили.
\vs 3Er 39:8
Мне же стыдно было
оставаться с ними. Но та, которая из них казалась главною, обняла меня и
начала лобызать. И прочие, увидев это, тоже начали лобызать меня, как брата,
водить около башни и играть со мною.
\vs 3Er 39:9
Некоторые из них пели
псалмы, а иные водили хороводы. А я в молчании ходил с ними около башни, и
казалось мне, что я помолодел.
\vs 3Er 39:10
С наступлением вечера я
хотел уйти домой, но они удержали меня и не позволили уйти.
\vs 3Er 39:11
И так я провел с ними эту
ночь около башни. Они постлали на землю свои полотняные туники и уложили меня
на них, сами же ничего другого не делали, только молились.
\vs 3Er 39:12
И я с ними молился
непрерывно и столь же усердно, и девы радовались моему усердию. Так оставался
я с девами до следующего дня.
\vs 3Er 39:13
Потом пришел пастырь и
спросил их: вы не причинили ему никакой обиды?
\vs 3Er 39:14
И отвечали они: спроси
его самого.
\vs 3Er 39:15
Господин,~--- сказал я,~--- я
получил великое удовольствие оттого, что остался с ними.
\vs 3Er 39:16
Что ты ужинал?~--- спросил
он.
\vs 3Er 39:17
Я ответил: всю ночь,
господин, я питался словами Господа.
\vs 3Er 39:18
Хорошо ли они тебя
приняли?
\vs 3Er 39:19
Хорошо, господин.
\vs 3Er 39:20
Теперь что прежде всего
желаешь услышать?
\vs 3Er 39:21
Чтобы ты, господин,
объяснил мне, всё, что до этого показал.
\vs 3Er 39:22
Как желаешь,~--- сказал он,
так и буду объяснять тебе и ничего от тебя не скрою.

\vs 3Er 40:1
Прежде всего, господин,~--- попросил я,~--- объясни мне, что означают камень и дверь.
\vs 3Er 40:2
Камень и дверь,~--- сказал
он,~--- это Сын Божий.
\vs 3Er 40:3
Как же так, господин,~--- удивился я,~--- ведь камень древний, а дверь новая?
\vs 3Er 40:4
Слушай, неразумный, и
понимай. Сын Божий древнее всякой твари, так что присутствовал на совете Отца
Своего о создании твари.
\vs 3Er 40:5
А дверь новая потому, что
Он явился в последние дни, сделался новою дверью для того, чтобы желающие
спастись через неё вошли в царство Божье.
\vs 3Er 40:6
Ты видел, что камни через
дверь были пронесены в здание башни, а те, которые не пронесены через неё,
были возвращены на своё место.
\vs 3Er 40:7
Так,~--- продолжал он,~--- никто не войдет в царство Божье, если не примет имени Сына Божьего.
\vs 3Er 40:8
Ибо если бы ты захотел
войти в какой-либо город, окруженный стеною с одними только воротами, не мог
бы ты проникнуть в этот город иначе как только через эти ворота.
\vs 3Er 40:9
По-другому и быть не
может, господин,~--- согласился я.

\vs 3Er 41:1
Итак, как в этот город
можно войти только через ворота его, так и в царство Божье не попадет человек
иначе как только через имя Сына Божьего возлюбленного.
\vs 3Er 41:2
Видел ли ты множество
строящих этими духовными силами? Будут один дух и одно тело, и будет один цвет
одежд их; тот именно заслужит место в башне, кто будет носить имена этих дев.
\vs 3Er 41:3
Почему же, господин,~--- спросил я,~--- отброшены и забракованы были некоторые камни, тогда как и их
пронесли через дверь и передали через руки дев в здание башни?
\vs 3Er 41:4
Так как у тебя есть
обыкновение всё тщательно исследовать, то слушай и об отброшенных камнях.
\vs 3Er 41:5
Все они приняли имя Сына
Божьего и силу этих дев. Приняв эти дары Духа, они укрепились и были в числе
рабов Божьих, и стали у них один дух, одно тело и одна одежда, потому что они
были единомысленны и делали правду.
\vs 3Er 41:6
Но спустя некоторое время
они увлеклись теми красивыми женщинами, которых ты видел одетыми в черную
одежду с обнаженными плечами и распущенными волосами;
\vs 3Er 41:7
увидев их, они возжелали
их и облеклись их силою, а силу дев свергли с себя.
\vs 3Er 41:8
Поэтому они изгнаны из
дома Божьего и преданы тем женщинам. А не соблазнившиеся красотою их остались
в доме Божьем.
\vs 3Er 41:9
Вот тебе,~--- заключил он,~--- значение камней отброшенных.

\vs 3Er 42:1
Что если, господин,~--- продолжал я расспросы,~--- такие люди покаются, отринут пожелания тех женщин и,
вновь обратившись к девам, облекутся их силою,~--- то войдут ли они в дом Божий?
\vs 3Er 42:2
Войдут, если отвергнут
дела тех женщин и снова приобретут силу дев и будут ходить в делах их.
\vs 3Er 42:3
Для того и остановлено
строительство, чтобы они покаялись и вошли в здание башни; если же не
покаются, то другие займут их место, а они будут отвержены навсегда.
\vs 3Er 42:4
За всё это я возблагодарил
Господа, что Он, подвигнутый милостью ко всем призывающим Его имя, послал
ангела покаяния к ним, согрешившим против Него, и обновил души наши, уже
ослабевшие и не имеющие надежды на спасение, восстановив нас к жизни.
\vs 3Er 42:5
Теперь, господин,~--- сказал
я,~--- объясни мне, почему башня строится не на земле, но на камне и двери?
\vs 3Er 42:6
Ты спрашиваешь, потому что
неразумен.
\vs 3Er 42:7
Господин, я вынужден обо
всем тебя спрашивать, потому что совершенно не могу ничего понять, ведь всё
это так величественно и дивно, что людям трудно постичь.
\vs 3Er 42:8
Слушай,~--- сказал мне
пастырь.~--- Имя Сына Божьего велико и неизмеримо, и оно держит весь мир.
\vs 3Er 42:9
Если всё творение держится
Сыном Божьим,~--- спросил я,~--- то как думаешь, поддерживает ли Он тех, которые
призваны Им, носят имя Его и живут по Его заповедям?
\vs 3Er 42:10
Видишь, Он поддерживает
тех, которые от всего сердца носят Его имя. Он Сам служит для них основанием и
с любовью держит их, потому что они не стыдятся носить Его имя.

\vs 3Er 43:1
Открой мне, господин,~--- попросил я,~--- имена дев и тех женщин, облеченных в черную одежду
\vs 3Er 43:2
Слушай. Из тех, которые
могущественнее и стоят по углам двери, первая зовется Верою, вторая~--- Воздержанием, третья~--- Мощью, четвертая~--- Терпением.
\vs 3Er 43:3
Прочие же, которые в
середине, имеют следующие имена: Простота, Невинность, Целомудрие, Радость,
Правдивость, Разумение, Согласие и Любовь.
\vs 3Er 43:4
Носящие эти имена и имя
Сына Божьего могут войти в царство Божье.
\vs 3Er 43:5
Слушай теперь имена
женщин, одетых в черную одежду. Четыре самые могущественные: первую зовут
Вероломством, вторую~--- Неумеренностью, третью~--- Неверием,
четвертую~--- Сластолюбием.
\vs 3Er 43:6
Имена следующих за ними:
Печаль, Лукавство, Похоть, Гнев, Ложь, Неразумие, Злословие, Ненависть.
\vs 3Er 43:7
Раб Божий, носящий такие
имена, хоть и увидит царство Божье, но не войдет в него!
\vs 3Er 43:8
Тогда решил узнать я у
пастыря, что означают камни, которые со дна подняты для здания.
\vs 3Er 43:9
Первые 10,~--- ответил он,~--- положенные в основание,
означают первый век,
следующие 25~--- второй век мужей праведных;
\vs 3Er 43:35
означают пророков и служителей
Господа; 40 же означают апостолов и учителей Евангелия Сына Божьего.
\vs 3Er 43:10
Почему же, господин, девы
подавали и эти камни в здание башни, пронеся их через дверь?
\vs 3Er 43:11
Потому, что они первые
имели силы этих дев, и те и другие не отступали~--- ни духовные силы от людей,
ни люди от сил; но эти силы пребывали с ними до дня упокоения;
\vs 3Er 43:12
если бы они не имели этих
сил духовных, то не годились бы для здания башни.

\vs 3Er 44:1
И снова я попросил: еще,
господин, объясни мне, почему эти камни были извлечены со дна и положены в
здание башни, тогда как они уже имели этих духов?
\vs 3Er 44:2
Им было необходимо пройти
через воду, чтобы оживотвориться; не могли они иначе войти в царство Божие,
как отринув мертвость прежней жизни.
\vs 3Er 44:3
Посему эти почившие
получили печать Сына Божьего и вошли в царство Божье.
\vs 3Er 44:4
Ибо человек до принятия
имени Сына Божьего мертв; но как скоро примет эту печать, он отлагает
мертвость и воспринимает жизнь.
\vs 3Er 44:5
Печать же эта есть вода, в
неё сходят люди мертвыми, а восходят из неё живыми; посему и им проповедана
была эта печать, и они воспользовались ею, чтобы войти в царство Божье.
\vs 3Er 44:6
Почему же,~--- спросил я,~--- вместе с ними взяты со дна и те сорок камней, уже имеющие эту печать?
\vs 3Er 44:7
Потому, что эти апостолы и
учители, проповедовавшие имя Сына Божьего, скончавшись с верою в Него и с
силою, проповедовали Его и прежде почившим, и сами дали им эту печать; они
вместе с ними нисходили в воду и с ними опять восходили.
\vs 3Er 44:8
Но они нисходили живыми, а
те, которые почили прежде, нисходили мертвыми, а вышли живыми; через апостолов
они восприняли жизнь и познали имя Сына Божьего и потому взяты вместе с ними и
положены в здание башни;
\vs 3Er 44:9
они употреблены в строение
не обсеченные, потому что они скончались в праведности и чистоте, только не
имели этой печати. Вот тебе объяснение этих камней.

\vs 3Er 45:1
Теперь, господин,~--- сказал
я,~--- объясни мне значение тех гор: почему они такие разные?
\vs 3Er 45:2
Слушай. Эти двенадцать
гор, которые ты видишь, означают двенадцать племен, населяющих весь мир; среди
них был проповедан Сын Божий через апостолов.
\vs 3Er 45:3
Почему же они различны и
вид имеют неодинаковый?
\vs 3Er 45:4
Эти двенадцать племен,
населяющие весь мир, суть двенадцать народов; и как различны, ты видел, горы,
так различны мысль и внутреннее настроение этих народов. Я поясню тебе смысл
каждого из них.
\vs 3Er 45:5
Прежде всего, господин,
скажи мне вот что: если эти горы так различны, то каким образом камни с них,
будучи положены в здание башни, сделались одноцветными и блестящими, как и
камни, поднятые со дна?
\vs 3Er 45:6
Потому, что все народы под
небом, услышав проповедь, уверовали и нареклись одним именем Сына Божьего и,
приняв печать Его, все получили один дух и один разум, и стала у них одна вера
и одна любовь, и вместе с именем Его они облеклись духовными силами дев.
\vs 3Er 45:7
Потому-то здание башни
сделалось одноцветным и сияющим, подобно солнцу.
\vs 3Er 45:8
Но после того как они
сошлись воедино и стали одним телом, некоторые из них осквернили себя и были
извергнуты из рода праведных; опять возвратились к прежнему состоянию и даже
сделались хуже.

\vs 3Er 46:1
Каким образом, господин,~--- говорю я,~--- они, познав Господа, сделались худшими?
\vs 3Er 46:2
Если не познавший Господа,
сказал он,~--- сделает зло, он подлежит наказанию за свою неправду. Но кто
познал Господа, тот уже должен удерживаться от зла и делать добро.
\vs 3Er 46:3
И если тот, который должен
совершать добро, вместо этого причиняет зло, то не более ли он преступен,
нежели не ведающий Бога?
\vs 3Er 46:4
Посему хотя и не познавшие
Бога и делающие зло обречены на смерть; но те, которые познали Господа и
видели дивные дела Его, делая зло, будут вдвойне наказаны и умрут навеки.
\vs 3Er 46:5
Так очистится Церковь
Божья.
\vs 3Er 46:6
Ты видел: забракованные
камни были выброшены из башни и преданы злым духам, и башня так очистилась,
что казалась вся высеченною из одного камня;
\vs 3Er 46:7
такою будет и Церковь
Божья, когда она очистится и будут изринуты из неё злые, лицемеры,
богохульники, двоедушные и все виновные в различной неправде;
\vs 3Er 46:8
она будет единое тело,
единый дух, единый разум, единая вера и единая любовь, и тогда Сын Божий будет
торжествовать между ними и радоваться, приняв Свой народ чистым.
\vs 3Er 46:9
Господин,~--- сказал я,~--- всё это величественно и славно. Теперь объясни мне значение каждой из гор,
чтобы всякая душа, уповающая на Господа, услышав это, прославляла великое,
дивное и славное имя Его.
\vs 3Er 46:10
Слушай и об этих
различных горах, то есть двенадцати народах.

\vs 3Er 47:1
Первая гора черная
означает верующих отступников, хулителей Господа и предателей рабов Божиих: им
назначена смерть и нет покаяния, и они черны потому, что род их беззаконен.
\vs 3Er 47:2
Вторая гора голая~--- это
верующие лицемеры и учители неправды; они весьма близки к первым и не имеют
плода правды.
\vs 3Er 47:3
Ибо как гора их пуста и
бесплодна, так и эти люди, хотя и имеют имя, но не имеют веры и нет в них
никакого плода истины.
\vs 3Er 47:4
Им, впрочем, есть
покаяние, если только немедленно покаются; а если замедлят, то и им будет
смерть вместе с первыми.
\vs 3Er 47:5
Почему же, господин,
последним есть доступ к покаянию, а первым нет? Ведь дела их почти те же
самые.
\vs 3Er 47:6
Потому для них есть
покаяние, что они не хулили Господа своего и не были предателями рабов Божьих;
но, стремясь к корысти, они обольщали людей, и каждый потворствовал похотям
грешных;
\vs 3Er 47:7
за это дело они понесут
наказание, но так как не были хулителями и предателями Господа, есть у них
возможность покаяния.

\vs 3Er 48:1
Третья гора,~--- продолжал
пастырь,~--- покрытая терниями и сорняками, знаменует верующих, из которых одни
богаты, а другие занялись множеством дел,
\vs 3Er 48:2
ибо сорняки означают
богатых, а тернии~--- тех, которые предались многим попечениям.
\vs 3Er 48:3
Таковые не имеют общения с
рабами Божьими, но удаляются от них, увлекаемые делами своими.
\vs 3Er 48:4
А богатые с трудом
вступают в общение с рабами Божьими, опасаясь, чтобы у них не попросили
чего-либо.
\vs 3Er 48:5
И как разутыми ногами
трудно ходить по колючкам, так и людям такого рода трудно попасть в царство
Божье.
\vs 3Er 48:6
Но и им есть покаяние,
только они должны немедленно обратиться к нему, чтобы упущенное ими прежде
вознаградить в остающиеся дни и делать добро.
\vs 3Er 48:7
Покаявшись и творя добрые
дела, они будут жить с Богом; если же пребудут в своих делах, то будут преданы
тем женщинам, которые лишат их жизни.

\vs 3Er 49:1
И далее повествовал
пастырь: четвертая гора, на которой очень много растений, в верхней части
зеленых, а к корням сухих и даже увядших от солнечного зноя, означает
верующих, которые колеблются или же имеют Господа только на устах, но не в
сердце.
\vs 3Er 49:2
Потому они в основании
сухи и лишены силы, и только слова их живы, а дела мертвы~--- и сами они ни
мертвы, ни живы.
\vs 3Er 49:3
Подобным образом и
колеблющиеся~--- ни зелены, ни сухи, то есть ни живы и ни мертвы.
\vs 3Er 49:4
Как те растения засохли,
едва лишь показалось солнце, так точно и двоедушные, услышав о гонении, по
малодушию поклоняются идолам и стыдятся имени своего Господа;
\vs 3Er 49:5
такие люди ни живы и ни
мертвы; но и они могут жить, если скоро покаются; если же не покаются, то
будут преданы тем женщинам, которые лишат их жизни.

\vs 3Er 50:1
Пятая гора скалистая, но
поросшая зелеными травами, означает верующих таких, которые хоть и веруют, но
мало учатся, дерзки и самодовольны, желают казаться всезнающими, но ничего не
знают.
\vs 3Er 50:2
За эту дерзость разум
отступил от них и вошло в них тщеславное безрассудство.
\vs 3Er 50:3
Они выдают себя за умных
и, будучи глупы, желают быть учителями.
\vs 3Er 50:4
За это высокомудрие многие
из них уничижены, ибо великое беснование~--- дерзость и суетная самонадеянность.
\vs 3Er 50:5
Из них многие отвержены,
другие же, осознав свое заблуждение, покаялись и покорились имеющим разум.
\vs 3Er 50:6
Но и прочим, подобным им,
есть покаяние, потому что они не столько были злы, сколько неразумны и глупы.
\vs 3Er 50:7
Посему, если покаются, они
будут жить с Богом; если же не покаются, будут обитать с женщинами,
коварствующими над ними.

\vs 3Er 51:1
Шестая гора, с большими и
малыми расселинами и с сухими растениями в них, означает верующих.
\vs 3Er 51:2
Малые расселины~--- тех,
которые имели между собою распри и от взаимных пререканий притупилась их вера;
\vs 3Er 51:3
многие из них покаялись,
то же сделают и прочие, услышав мои заповеди, потому что незначительны их
распри и легко они обратятся к покаянию.
\vs 3Er 51:4
Большие~--- это те, что
упорствуют в распрях, злопамятны и гневливы;
\vs 3Er 51:5
они отброшены от башни и
не годятся для здания, трудно им жить с Богом.
\vs 3Er 51:6
Если Бог и Господь наш,
владычествующий над всем Своим творением, не помнит зла на исповедующих грехи
свои, но умилостивляется, то пристало ли человеку, смертному и исполненному
грехов, упорно гневаться на другого, словно он может спасти или погубить его?
\vs 3Er 51:7
Я, ангел покаяния, убеждаю
вас, склонных к этому: одумайтесь и обратитесь к покаянию~--- и Господь уврачует
прежние ваши прегрешения, если очиститесь от этого бесовского зла, если же нет
будете преданы смерти.

\vs 3Er 52:1
Седьмая гора,~--- продолжал
пастырь свои объяснения,~--- на которой растительность зеленая, цветущая и
обильная, так что всякий скот и птицы небесные питаются ею, и она, будучи
срываема, растет еще лучше,
\vs 3Er 52:2
знаменует верующих,
которые просты и добры, не враждуют между собою, но всегда радуются за рабов
Божьих, исполнены духом дев, милосердны к любому человеку и плодами от трудов
своих делятся со всяким немедленно и без колебания.
\vs 3Er 52:3
Посему Господь, видя
простоту и доброту их, благопоспешал трудам рук их и даровал успех во всяком
деле.
\vs 3Er 52:4
Я, ангел покаяния,
убеждаю вас пребывать в таком расположении, и семя ваше не искоренится вовек.
\vs 3Er 52:5
Господь одобрил вас и
вписал в наше число, и всё семя ваше будет обитать с Сыном Божьим, потому что
вы~--- от Духа Его.

\vs 3Er 53:1
Восьмая гора, со многими
источниками, которые поили всякую тварь Божью, означает апостолов и учителей,
\vs 3Er 53:2
которые проповедовали по
всему миру, и свято и чисто учили слову Господню, и не склонялись к дурным
желаниям, но постоянно пребывали в правде и истине, приняв Святого Духа.
\vs 3Er 53:3
Посему они обитают с
ангелами.

\vs 3Er 54:1
Камни с пятнами на девятой
горе, пустынной и населенной вредоносными змеями, означают дьяконов,
\vs 3Er 54:2
которые плохо проходили
служение, расхищая блага вдов и сирот и обогащаясь от своего служения.
\vs 3Er 54:3
Если останутся в своем
пороке, то они мертвы и нет в них никакой надежды жизни; если же обратятся и
будут непорочно исполнять свое служение, то смогут жить.
\vs 3Er 54:4
А камни шероховатые
означают тех, которые отреклись и не обратились к Господу, одичали и
уподобились пустыне, не общаются с рабами Божьими, но, живя одиноко, губят
свои души.
\vs 3Er 54:5
Как виноградная лоза,
оставленная без всякого ухода, пропадает, заглушается травами, со временем
делается дикой и бесполезной для хозяина, так и эти люди, отчаявшись в себе
самих, одичали и стали бесполезны для своего Господа.
\vs 3Er 54:6
Для них возможно покаяние,
если отреклись они не от сердца; если же кто сделал это от сердца, не знаю,
сможет ли он возродиться.
\vs 3Er 54:7
Я не о настоящих днях
говорю, чтобы отрекшийся мог покаяться;
\vs 3Er 54:8
невозможно обрести
спасение тем, кто намерен отречься от своего Господа; но покаяние дается
тем, кто отрекся в прошлом.
\vs 3Er 54:9
Итак, кто намерен
покаяться, пусть сделает это немедленно, прежде чем закончится строительство
башни.
\vs 3Er 54:10
Если же кто не поспешит,
то будет предан смерти теми женщинами.
\vs 3Er 54:11
Камни короткие означают
людей коварных и клеветников: они подобны змеям, которых ты видел на девятой
горе.
\vs 3Er 54:12
Ибо как яд змеи
смертоносен для человека, так и слова таких людей губительны для других.
Несовершенны они в своей вере по причине их образа действий.
\vs 3Er 54:13
Впрочем, некоторые из
них покаялись и спаслись.
\vs 3Er 54:14
Равным образом прочие
таковые получат спасение, если покаются; если же не покаются, то погибнут от
тех женщин, силою и властью которых они обладают.

\vs 3Er 55:1
Деревья на десятой горе,
которые служат кровом для скота, означают епископов и верующих страннолюбцев,
\vs 3Er 55:2
которые всегда непритворно
и радушно принимали в домах своих рабов Божьих;
\vs 3Er 55:3
епископов, которые
беспрестанно покровительствовали бедным и вдовствующим и жили всегда
непорочно.
\vs 3Er 55:4
Таким людям
покровительствует сам Господь: они почтенны у Бога, и им место среди ангелов,
если пребудут до конца в служении Господу.

\vs 3Er 56:1
Одиннадцатая гора, деревья
на которой обильны разными плодами, означает верующих, пострадавших за имя
Сына Божьего, пострадавших с любовью и от всего сердца своего.
\vs 3Er 56:2
Я спросил: почему же,
господин, все деревья имеют плоды, но на некоторых плоды менее приятны?
\vs 3Er 56:3
И это объясню тебе.
Пострадавшие за имя Господне почтенны у Бога, и всем им отпущены грехи, потому
что пострадали за имя Сына Божьего.
\vs 3Er 56:4
Но некоторые, будучи
допущены ко властям и спрошены, не отреклись от Господа, но с готовностью
пострадали,~--- они почтенны у Бога, и плод их превосходнее.
\vs 3Er 56:5
А некоторые, охваченные
страхами и смущением, колебание имели в своем сердце, проповедать ли Бога или
отречься, и пострадали~--- их плоды хуже, потому что в сердце их был лукавый
помысел раба отречься от своего господина.
\vs 3Er 56:6
Смотрите вы, помышляющие
так, чтобы эта мысль не утвердилась в ваших сердцах и чтобы не умереть вам для
Бога.
\vs 3Er 56:7
А вы, страдающие за имя
Божье, должны прославлять Господа, что удостоил вас носить Его имя, ибо
исцелятся все грехи ваши.
\vs 3Er 56:8
Ужели вы не почитаете себя
более других блаженными? Вы думаете, что совершили великое дело, если кто из
вас пострадал?
\vs 3Er 56:9
Но Господь дарует вам
жизнь, и вы об этом не помышляете. Вас отягощали грехи ваши, и если бы не
пострадали вы за имя Господне, то вы умерли бы для Бога за грехи свои.
\vs 3Er 56:10
Это я говорю вам,
сомневающимся, исповедать ли Бога или отречься.
\vs 3Er 56:11
Исповедуйте, что вы
имеете Господа, и, не отрекаясь, отдавайте себя в оковы.
\vs 3Er 56:12
Если все народы
наказывают рабов за отречение от своего хозяина, то что, думаете вы, сделает с
вами Господь, имеющий власть над всеми?
\vs 3Er 56:13
Итак, удалите из сердец
своих такие помыслы, чтобы вовеки жить вам с Богом.

\vs 3Er 57:1
Двенадцатая гора, белая,
означает верующих, подобных младенцам, коим не всходила на сердце никакая
злоба, которые не знают, что такое лукавство, но всегда пребывают в простоте.
\vs 3Er 57:2
Такие люди, без сомнения,
будут обитать в царстве Божьем, потому что они ни в одном деле не преступили
заповедей Божьих, но с простотою пребывали в том же расположении все дни своей
жизни.
\vs 3Er 57:3
Те, которые останутся как
младенцы, не имеющие злобы, будут почетнее всех, о которых сказано выше: все
младенцы славны у Господа и почитаются у Него первыми.
\vs 3Er 57:4
Итак, блаженны вы, которые
удалили от себя лукавство и облеклись в невинность, потому что вы первые
будете жить с Богом.
\vs 3Er 57:5
После того как пастырь
истолковал мне все горы, я сказал ему:
\vs 3Er 57:6
<<Господин, теперь поведай о тех камнях, которые принесены с поля и заложены в
башню вместо вынутых, а также о тех круглых камнях, которые вошли в здание
башни, и о тех, которые доселе остаются круглыми.>>

\vs 3Er 58:1
Слушай и об этом. Камни,
которые были принесены с поля и заложены в здание башни
вместо отвергнутых,~--- это суть отроги белой горы.
\vs 3Er 58:2
Поскольку верующие с этой
горы оказались невинными, то господин башни поместил их в здание башни, ибо
знал, что, войдя в здание, они останутся белыми и ни один из них не почернеет.
\vs 3Er 58:3
А если бы он приказал
положить в здание башни камни и с прочих гор, то нужно было бы ему снова
осматривать эту башню и очищать.
\vs 3Er 58:4
Эти белые камни суть
новообращенные, которые уверовали и уверуют, ибо они веруют от сердца. Блажен
этот род, потому что невинен.
\vs 3Er 58:5
Слушай теперь и о круглых
блестящих камнях. И они все от белой горы.
\vs 3Er 58:6
Круглыми же они оказались
потому, что богатство немного омрачило их, но они не отступили от Бога и ни
единое слово хулы не сошло с языка их~--- только правда, добродетель и истина.
\vs 3Er 58:7
Посему Господь, зная душу
их и то, что они родились и остаются добрыми, повелел отсечь их богатства, но
не совсем отнять их, чтобы из оставшегося они могли делать добро и жить с
Богом, ибо и они из доброго рода.
\vs 3Er 58:8
Посему их несколько
отесали и положили в здание башни.

\vs 3Er 59:1
А прочие камни, которые
остались круглыми и были негодны для здания, еще не получили печати и
возвращены на свое место, ибо оказались слишком круглыми.
\vs 3Er 59:2
Должно лишить их благ
настоящего века и суетного богатства~--- и тогда они будут годны в царстве
Божьем.
\vs 3Er 59:3
Они должны войти в царство
Божье, ибо Господь благословил этот род, и из него никто не погибнет;
\vs 3Er 59:4
может быть, кто из них,
искушенный злым дьяволом, и согрешит в чем-либо, но скоро вновь обратится к
Господу своему.
\vs 3Er 59:5
Я, ангел покаяния, почитаю
счастливыми вас, которые невинны, как дети, потому что ваша участь благая и
почтенная перед Богом.
\vs 3Er 59:6
И всем, которые приняли
печать Сына Божьего, говорю: имейте простоту, не помните обид, не пребывайте в
злобе, да не будет в душе кого-либо из вас горечи злопамятства;
\vs 3Er 59:7
врачуйте и удаляйте от
себя злые раздоры, чтобы господин стада пришел и возрадовался, найдя целыми
овец своих.
\vs 3Er 59:8
Если же какая овца будет
потеряна пастырями или самих пастырей господин найдет дурными, что ответят
ему? Ужели скажут, что они измучены стадом?
\vs 3Er 59:9
Не поверят им, ибо не
может пастырь потерпеть что от овец и еще более будет наказан за ложь свою.
\vs 3Er 59:10
И я~--- пастырь и должен
дать Всевышнему отчет за вас.

\vs 3Er 60:1
Итак, позаботьтесь о себе,
пока еще строится башня.
\vs 3Er 60:2
Господь обитает в людях,
любящих мир, ибо Он Сам любит мир и далек от сварливых и развращенных злобою.
\vs 3Er 60:3
Возвратите Ему дух целым,
какой приняли от Него.
\vs 3Er 60:4
Ибо если ты отдашь
валяльщику одежду целую, то желаешь и получить ее обратно целою, а если
валяльщик возвратит тебе её изодранною, возьмешь ли ты ее?
\vs 3Er 60:5
Не прогневаешься ли и не
будешь ли бранить его, говоря: я дал тебе одежду целою, а ты изодрал её, и
теперь она из-за дыр, которые ты на ней сделал, стала непригодна. Разве не так
будешь пенять ты валяльщику и скорбеть о своей одежде?
\vs 3Er 60:6
Так что же, думаешь,
сделает тебе Господь, который вручил тебе дух чистый, а ты повредил его и
привел в негодность, так что он никак не может служить Господу?
\vs 3Er 60:7
И за это Господь предаст
тебя смерти.
\vs 3Er 60:8
Так накажет Он всех тех,
которых найдет упорно помнящими обиды.
\vs 3Er 60:9
Не пренебрегайте Его
милосердием, но лучше прославляйте Его за то, что Он, не в пример вам, столь
терпим к вашим преступлениям.
\vs 3Er 60:10
Покайтесь, ибо это
полезно для вас.

\vs 3Er 61:1
Всё, что описано выше,
показал я, пастырь, ангел покаяния, ради покаяния.
\vs 3Er 61:2
Я всегда говорил и теперь
говорю рабам Божьим: если поверите и послушаетесь слов моих, будете поступать
по ним и исправите пути ваши, то сможете спастись.
\vs 3Er 61:3
Если же будете
упорствовать в лукавстве и злопамятстве, ни один из таких грешников не будет
жить с Богом: ибо всё это мною наперед сказано вам.
\vs 3Er 61:4
И после этих слов пастырь
спросил меня: всё ли ты проведал у меня?
\vs 3Er 61:5
Я ответил, что всё.
\vs 3Er 61:6
Почему же ты не спросил
меня,~--- сказал тогда он,~--- о камнях, положенных в здание, вид которых мы
исправили?
\vs 3Er 61:7
Забыл, господин.
\vs 3Er 61:8
Выслушай и о них. Это те,
до которых дошли теперь мои заповеди, и они от всего сердца покаялись, и
Господь, видя, что покаяние их доброе и чистое и что пребудут они в нем,
повелел загладить прежние грехи их.
\vs 3Er 61:9
Так грехи их изглажены,
чтобы после они не были видны.

\chhdr{Подобие 10-е.}
\vs 3Er 62:1
После того как я написал эту книгу, тот ангел, который
вручил меня пастырю, пришел в дом мой и сел на ложе, а справа от него стал
пастырь.
\vs 3Er 62:2
Позвал ангел меня и сказал: я поручил тебя и дом твой этому пастырю под его
покровительство.
\vs 3Er 62:3
Так, господин,~--- подтвердил я.
\vs 3Er 62:4
Итак, если хочешь быть
защищен от всякого бедствия и злополучия, иметь успех во всяком благом деле и
слове и во всякой истинной добродетели, то поступай по тем заповедям, которые
он дал тебе, и будешь господствовать над всякою неправдою.
\vs 3Er 62:5
Ибо, если будешь соблюдать
эти заповеди, покорятся тебе всякое пожелание и сладость этого века и будет
сопровождать тебя удача во всяком добром деле.
\vs 3Er 62:6
Почитай его достоинство и
святость и скажи всем, что он в великой чести и славе у Бога и имеет великую
власть и силу.
\vs 3Er 62:7
Ему одному во всей
вселенной вручена власть покаяния. Разве он не кажется тебе могущественным?
\vs 3Er 62:8
Но вы пренебрегаете его
достоинством и властью, которую он имеет над вами.

\vs 3Er 63:1
Я сказал: спроси,
господин, самого его, сделал ли я что дурное или оскорбил его чем-нибудь за то
время, что он находится в доме моем.
\vs 3Er 63:2
И я знаю, что ты не сделал
и не сделаешь ничего дурного, потому я и говорю это тебе, чтобы ты всегда был
таков. Ибо он предо мною хорошо засвидетельствовал о тебе.
\vs 3Er 63:3
Скажи это и прочим, чтобы
и они, если покаялись или намерены покаяться, чувствовали то же, что и ты,~--- и
он засвидетельствует доброе о них предо мною, а я пред Господом.
\vs 3Er 63:4
Господин,~--- ответил я,~--- я
всякому человеку возвещу великие дела Божьи и надеюсь, что все прежде
согрешившие, услышав это, покаются, чтобы получить жизнь.
\vs 3Er 63:5
Итак, совершай неуклонно
это служение и впредь.
\vs 3Er 63:6
Кто исполнит заповеди Его,
будет иметь жизнь и великую честь у Господа.
\vs 3Er 63:7
А кто не соблюдет Его
заповедей, бежит от своей жизни; кто не чтит Его, теряет свою честь у Господа.
\vs 3Er 63:8
Презирающие Его и не
соблюдающие Его заповедей обрекают себя на смерть, и любой из них виновен в
крови своей.
\vs 3Er 63:9
Тебе же наказываю
соблюдать эти заповеди~--- и получишь искупление всех грехов своих.

\vs 3Er 64:1
Я послал к тебе также и
этих дев, чтобы они жили с тобою, ибо я видел, что они очень ласковы к тебе.
\vs 3Er 64:2
Они станут тебе
помощниками, чтобы усерднее ты мог соблюдать заповеди, ибо без этих дев
невозможно соблюсти заповеди.
\vs 3Er 64:3
Я вижу, что им приятно
быть с тобою, и я прикажу, чтобы они вовсе не выходили из твоего дома.
\vs 3Er 64:4
Ты только очисти дом свой:
в чистом доме они живут охотно.
\vs 3Er 64:5
Они сами чисты, непорочны
и рачительны и весьма угодны Господу.
\vs 3Er 64:6
Итак, если будет чист дом
твой, они останутся с тобою. Если же чем осквернится дом твой, они совсем
удалятся из него, ибо не любят никакой нечистоты.
\vs 3Er 64:7
Я надеюсь угодить им, так
что они охотно и безотлучно будут жить в доме моем. И как тот, которому ты
передал меня, ни в чем на меня не жалуется, так и они не будут жаловаться.
\vs 3Er 64:8
Ангел сказал пастырю: я
вижу, что раб Божий хочет соблюдать эти заповеди и поместить дев в чистом
жилище.
\vs 3Er 64:9
Произнеся это, он опять
поручил меня пастырю и обратился к девам: так как я вижу, что вам приятно жить
в этом доме, то вручаю вам Ерму и семью его с тем, чтобы вы не покидали этого
дома.
\vs 3Er 64:10
И они с удовольствием
вняли этим словам.

\vs 3Er 65:1
Потом он сказал мне:
мужественно проходи это служение и поведай всякому человеку величие Божье~--- и
будешь иметь благодать в своем служении.
\vs 3Er 65:2
Всякий, кто исполнит эти
заповеди, будет жить и будет блажен; а кто пренебрежет ими, не будет жить и
будет несчастлив в своей жизни.
\vs 3Er 65:3
Скажи всем, чтобы не
переставали, кто может, благотворить, ибо благотворение полезно им.
\vs 3Er 65:4
Говорю о том, что должно
всякого человека вызволять из бедствия. Неимущий в ежедневной жизни терпит
великое мучение и скорбь.
\vs 3Er 65:5
Кто вырвет из нужды душу
такого человека, тот обретет великую радость, ибо терпящий подобное бедствие
испытывает страдания сродни заключенному в узах.
\vs 3Er 65:6
Многие, не вынеся
бедственного положения, причиняют себе смерть. Посему кто знает о бедствии
такого человека и не избавляет его, тот совершает великий грех и принимает
вину за кровь его.
\vs 3Er 65:7
Итак, благотворите,
сколько кто получил от Господа. Не медлите, пока не окончилось строительство
башни, ибо ради вас приостановлено оно.
\vs 3Er 65:8
Если не поспешите
исправиться, будет достроена башня и вы не попадете в неё.
\vs 3Er 65:9
После этих слов он встал с
ложа и, взяв пастыря и дев, удалился, но обещал мне, что пастыря и дев
отпустит обратно в дом мой.

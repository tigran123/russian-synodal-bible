\bibbookdescr{Sol}{
  inline={\LARGE Книга\\\Huge Песни Песней Соломона},
  toc={Песнь Песней},
  bookmark={Песнь Песней},
  header={Песнь Песней},
  %headerleft={},
  %headerright={},
  abbr={Песн}
}
\vs Sol 1:1 Да лобзает он меня лобзанием уст своих! Ибо ласки твои лучше вина.
\vs Sol 1:2 От благовония мастей твоих имя твое~--- как разлитое миро; поэтому девицы любят тебя.
\vs Sol 1:3 Влеки меня, мы побежим за тобою;~--- царь ввел меня в чертоги свои,~--- будем восхищаться и радоваться тобою, превозносить ласки твои больше, нежели вино; достойно любят тебя!
\rsbpar\vs Sol 1:4 Дщери Иерусалимские! черна я, но красива, как шатры Кидарские, как завесы Соломоновы.
\vs Sol 1:5 Не смотрите на меня, что я смугла, ибо солнце опалило меня: сыновья матери моей разгневались на меня, поставили меня стеречь виноградники,~--- моего собственного виноградника я не стерегла.
\rsbpar\vs Sol 1:6 Скажи мне, ты, которого любит душа моя: где пасешь ты? где отдыхаешь в полдень? к чему мне быть скиталицею возле стад товарищей твоих?
\vs Sol 1:7 Если ты не знаешь этого, прекраснейшая из женщин, то иди себе по следам овец и паси козлят твоих подле шатров пастушеских.
\vs Sol 1:8 Кобылице моей в колеснице фараоновой я уподобил тебя, возлюбленная моя.
\vs Sol 1:9 Прекрасны ланиты твои под подвесками, шея твоя в ожерельях;
\vs Sol 1:10 золотые подвески мы сделаем тебе с серебряными блестками.
\vs Sol 1:11 Доколе царь был за столом своим, нард мой издавал благовоние свое.
\vs Sol 1:12 Мирровый пучок~--- возлюбленный мой у меня, у грудей моих пребывает.
\vs Sol 1:13 Как кисть кипера, возлюбленный мой у меня в виноградниках Енгедских.
\vs Sol 1:14 О, ты прекрасна, возлюбленная моя, ты прекрасна! глаза твои голубиные.
\vs Sol 1:15 О, ты прекрасен, возлюбленный мой, и любезен! и ложе у нас~--- зелень;
\vs Sol 1:16 кровли домов наших~--- кедры, потолки наши~--- кипарисы.
\vs Sol 2:1 Я нарцисс Саронский, лилия долин!
\vs Sol 2:2 Что лилия между тернами, то возлюбленная моя между девицами.
\vs Sol 2:3 Что яблоня между лесными деревьями, то возлюбленный мой между юношами. В тени ее люблю я сидеть, и плоды ее сладки для гортани моей.
\rsbpar\vs Sol 2:4 Он ввел меня в дом пира, и знамя его надо мною~--- любовь.
\vs Sol 2:5 Подкрепите меня вином, освежите меня яблоками, ибо я изнемогаю от любви.
\vs Sol 2:6 Левая рука его у меня под головою, а правая обнимает меня.
\vs Sol 2:7 Заклинаю вас, дщери Иерусалимские, сернами или полевыми ланями: не будите и не тревожьте возлюбленной, доколе ей угодно.
\rsbpar\vs Sol 2:8 Голос возлюбленного моего! вот, он идет, скачет по горам, прыгает по холмам.
\vs Sol 2:9 Друг мой похож на серну или на молодого оленя. Вот, он стоит у нас за стеною, заглядывает в окно, мелькает сквозь решетку.
\vs Sol 2:10 Возлюбленный мой начал говорить мне: встань, возлюбленная моя, прекрасная моя, выйди!
\vs Sol 2:11 Вот, зима уже прошла; дождь миновал, перестал;
\vs Sol 2:12 цветы показались на земле; время пения настало, и голос горлицы слышен в стране нашей;
\vs Sol 2:13 смоковницы распустили свои почки, и виноградные лозы, расцветая, издают благовоние. Встань, возлюбленная моя, прекрасная моя, выйди!
\vs Sol 2:14 Голубица моя в ущелье скалы под кровом утеса! покажи мне лице твое, дай мне услышать голос твой, потому что голос твой сладок и лице твое приятно.
\vs Sol 2:15 Ловите нам лисиц, лисенят, которые портят виноградники, а виноградники наши в цвете.
\rsbpar\vs Sol 2:16 Возлюбленный мой принадлежит мне, а я ему; он пасет между лилиями.
\vs Sol 2:17 Доколе день дышит \bibemph{прохладою}, и убегают тени, возвратись, будь подобен серне или молодому оленю на расселинах гор.
\vs Sol 3:1 На ложе моем ночью искала я того, которого любит душа моя, искала его и не нашла его.
\vs Sol 3:2 Встану же я, пойду по городу, по улицам и площадям, и буду искать того, которого любит душа моя; искала я его и не нашла его.
\vs Sol 3:3 Встретили меня стражи, обходящие город: <<не видали ли вы того, которого любит душа моя?>>
\vs Sol 3:4 Но едва я отошла от них, как нашла того, которого любит душа моя, ухватилась за него, и не отпустила его, доколе не привела его в дом матери моей и во внутренние комнаты родительницы моей.
\rsbpar\vs Sol 3:5 Заклинаю вас, дщери Иерусалимские, сернами или полевыми ланями: не будите и не тревожьте возлюбленной, доколе ей угодно.
\vs Sol 3:6 Кто эта, восходящая от пустыни как бы столбы дыма, окуриваемая миррою и фимиамом, всякими порошками мироварника?
\rsbpar\vs Sol 3:7 Вот одр его~--- Соломона: шестьдесят сильных вокруг него, из сильных Израилевых.
\vs Sol 3:8 Все они держат по мечу, опытны в бою; у каждого меч при бедре его ради страха ночного.
\vs Sol 3:9 Носильный одр сделал себе царь Соломон из дерев Ливанских;
\vs Sol 3:10 столпцы его сделал из серебра, локотники его из золота, седалище его из пурпуровой ткани; внутренность его убрана с любовью дщерями Иерусалимскими.
\vs Sol 3:11 Пойдите и посмотрите, дщери Сионские, на царя Соломона в венце, которым увенчала его мать его в день бракосочетания его, в день, радостный для сердца его.
\vs Sol 4:1 О, ты прекрасна, возлюбленная моя, ты прекрасна! глаза твои голубиные под кудрями твоими; волосы твои~--- как стадо коз, сходящих с горы Галаадской;
\vs Sol 4:2 зубы твои~--- как стадо выстриженных овец, выходящих из купальни, из которых у каждой пара ягнят, и бесплодной нет между ними;
\vs Sol 4:3 как лента алая губы твои, и уста твои любезны; как половинки гранатового яблока~--- ланиты твои под кудрями твоими;
\vs Sol 4:4 шея твоя~--- как столп Давидов, сооруженный для оружий, тысяча щитов висит на нем~--- все щиты сильных;
\vs Sol 4:5 два сосца твои~--- как двойни молодой серны, пасущиеся между лилиями.
\vs Sol 4:6 Доколе день дышит \bibemph{прохладою}, и убегают тени, пойду я на гору мирровую и на холм фимиама.
\rsbpar\vs Sol 4:7 Вся ты прекрасна, возлюбленная моя, и пятна нет на тебе!
\vs Sol 4:8 Со мною с Ливана, невеста! со мною иди с Ливана! спеши с вершины Аманы, с вершины Сенира и Ермона, от логовищ львиных, от гор барсовых!
\vs Sol 4:9 Пленила ты сердце мое, сестра моя, невеста! пленила ты сердце мое одним взглядом очей твоих, одним ожерельем на шее твоей.
\vs Sol 4:10 О, как любезны ласки твои, сестра моя, невеста! о, как много ласки твои лучше вина, и благовоние мастей твоих лучше всех ароматов!
\vs Sol 4:11 Сотовый мед каплет из уст твоих, невеста; мед и молоко под языком твоим, и благоухание одежды твоей подобно благоуханию Ливана!
\vs Sol 4:12 Запертый сад~--- сестра моя, невеста, заключенный колодезь, запечатанный источник:
\vs Sol 4:13 рассадники твои~--- сад с гранатовыми яблоками, с превосходными плодами, киперы с нардами,
\vs Sol 4:14 нард и шафран, аир и корица со всякими благовонными деревами, мирра и алой со всякими лучшими ароматами;
\vs Sol 4:15 садовый источник~--- колодезь живых вод и потоки с Ливана.
\vs Sol 4:16 Поднимись \bibemph{ветер} с севера и принесись с юга, повей на сад мой,~--- и польются ароматы его!~--- Пусть придет возлюбленный мой в сад свой и вкушает сладкие плоды его.
\vs Sol 5:1 Пришел я в сад мой, сестра моя, невеста; набрал мирры моей с ароматами моими, поел сотов моих с медом моим, напился вина моего с молоком моим. Ешьте, друзья, пейте и насыщайтесь, возлюбленные!
\rsbpar\vs Sol 5:2 Я сплю, а сердце мое бодрствует; \bibemph{вот}, голос моего возлюбленного, который стучится: <<отвори мне, сестра моя, возлюбленная моя, голубица моя, чистая моя! потому что голова моя вся покрыта росою, кудри мои~--- ночною влагою>>.
\vs Sol 5:3 Я скинула хитон мой; как же мне опять надевать его? Я вымыла ноги мои; как же мне марать их?
\vs Sol 5:4 Возлюбленный мой протянул руку свою сквозь скважину, и внутренность моя взволновалась от него.
\vs Sol 5:5 Я встала, чтобы отпереть возлюбленному моему, и с рук моих капала мирра, и с перстов моих мирра капала на ручки замка.
\vs Sol 5:6 Отперла я возлюбленному моему, а возлюбленный мой повернулся и ушел. Души во мне не стало, когда он говорил; я искала его и не находила его; звала его, и он не отзывался мне.
\vs Sol 5:7 Встретили меня стражи, обходящие город, избили меня, изранили меня; сняли с меня покрывало стерегущие стены.
\vs Sol 5:8 Заклинаю вас, дщери Иерусалимские: если вы встретите возлюбленного моего, что скажете вы ему? что я изнемогаю от любви.
\vs Sol 5:9 <<Чем возлюбленный твой лучше других возлюбленных, прекраснейшая из женщин? Чем возлюбленный твой лучше других, что ты так заклинаешь нас?>>
\vs Sol 5:10 Возлюбленный мой бел и румян, лучше десяти тысяч других:
\vs Sol 5:11 голова его~--- чистое золото; кудри его волнистые, черные, как ворон;
\vs Sol 5:12 глаза его~--- как голуби при потоках вод, купающиеся в молоке, сидящие в довольстве;
\vs Sol 5:13 щеки его~--- цветник ароматный, гряды благовонных растений; губы его~--- лилии, источают текучую мирру;
\vs Sol 5:14 руки его~--- золотые кругляки, усаженные топазами; живот его~--- как изваяние из слоновой кости, обложенное сапфирами;
\vs Sol 5:15 голени его~--- мраморные столбы, поставленные на золотых подножиях; вид его подобен Ливану, величествен, как кедры;
\vs Sol 5:16 уста его~--- сладость, и весь он~--- любезность. Вот кто возлюбленный мой, и вот кто друг мой, дщери Иерусалимские!
\vs Sol 6:1 <<Куда пошел возлюбленный твой, прекраснейшая из женщин? куда обратился возлюбленный твой? мы поищем его с тобою>>.
\vs Sol 6:2 Мой возлюбленный пошел в сад свой, в цветники ароматные, чтобы пасти в садах и собирать лилии.
\vs Sol 6:3 Я принадлежу возлюбленному моему, а возлюбленный мой~--- мне; он пасет между лилиями.
\rsbpar\vs Sol 6:4 Прекрасна ты, возлюбленная моя, как Фирца, любезна, как Иерусалим, грозна, как полки со знаменами.
\vs Sol 6:5 Уклони очи твои от меня, потому что они волнуют меня.
\vs Sol 6:6 Волосы твои~--- как стадо коз, сходящих с Галаада; зубы твои~--- как стадо овец, выходящих из купальни, из которых у каждой пара ягнят, и бесплодной нет между ними;
\vs Sol 6:7 как половинки гранатового яблока~--- ланиты твои под кудрями твоими.
\vs Sol 6:8 Есть шестьдесят цариц и восемьдесят наложниц и девиц без числа,
\vs Sol 6:9 но единственная~--- она, голубица моя, чистая моя; единственная она у матери своей, отличенная у родительницы своей. Увидели ее девицы, и~--- превознесли ее, царицы и наложницы, и~--- восхвалили ее.
\vs Sol 6:10 Кто эта, блистающая, как заря, прекрасная, как луна, светлая, как солнце, грозная, как полки со знаменами?
\vs Sol 6:11 Я сошла в ореховый сад посмотреть на зелень долины, поглядеть, распустилась ли виноградная лоза, расцвели ли гранатовые яблоки?
\vs Sol 6:12 Не знаю, как душа моя влекла меня к колесницам знатных народа моего.
\vs Sol 7:1 <<Оглянись, оглянись, Суламита! оглянись, оглянись,~--- и мы посмотрим на тебя>>. Что вам смотреть на Суламиту, как на хоровод Манаимский?
\vs Sol 7:2 О, как прекрасны ноги твои в сандалиях, дщерь именитая! Округление бедр твоих, как ожерелье, дело рук искусного художника;
\vs Sol 7:3 живот твой~--- круглая чаша, \bibemph{в которой} не истощается ароматное вино; чрево твое~--- ворох пшеницы, обставленный лилиями;
\vs Sol 7:4 два сосца твои~--- как два козленка, двойни серны;
\vs Sol 7:5 шея твоя~--- как столп из слоновой кости; глаза твои~--- озерки Есевонские, что у ворот Батраббима; нос твой~--- башня Ливанская, обращенная к Дамаску;
\vs Sol 7:6 голова твоя на тебе, как Кармил, и волосы на голове твоей, как пурпур; царь увлечен \bibemph{твоими} кудрями.
\vs Sol 7:7 Как ты прекрасна, как привлекательна, возлюбленная, твоею миловидностью!
\vs Sol 7:8 Этот стан твой похож на пальму, и груди твои на виноградные кисти.
\vs Sol 7:9 Подумал я: влез бы я на пальму, ухватился бы за ветви ее; и груди твои были бы вместо кистей винограда, и запах от ноздрей твоих, как от яблоков;
\vs Sol 7:10 уста твои~--- как отличное вино. Оно течет прямо к другу моему, услаждает уста утомленных.
\vs Sol 7:11 Я принадлежу другу моему, и ко мне \bibemph{обращено} желание его.
\vs Sol 7:12 Приди, возлюбленный мой, выйдем в поле, побудем в селах;
\vs Sol 7:13 поутру пойдем в виноградники, посмотрим, распустилась ли виноградная лоза, раскрылись ли почки, расцвели ли гранатовые яблоки; там я окажу ласки мои тебе.
\vs Sol 7:14 Мандрагоры уже пустили благовоние, и у дверей наших всякие превосходные плоды, новые и старые: \bibemph{это} сберегла я для тебя, мой возлюбленный!
\vs Sol 8:1 О, если бы ты был мне брат, сосавший груди матери моей! тогда я, встретив тебя на улице, целовала бы тебя, и меня не осуждали бы.
\vs Sol 8:2 Повела бы я тебя, привела бы тебя в дом матери моей. Ты учил бы меня, а я поила бы тебя ароматным вином, соком гранатовых яблоков моих.
\vs Sol 8:3 Левая рука его у меня под головою, а правая обнимает меня.
\vs Sol 8:4 Заклинаю вас, дщери Иерусалимские,~--- не будите и не тревожьте возлюбленной, доколе ей угодно.
\rsbpar\vs Sol 8:5 Кто это восходит от пустыни, опираясь на своего возлюбленного? Под яблоней разбудила я тебя: там родила тебя мать твоя, там родила тебя родительница твоя.
\vs Sol 8:6 Положи меня, как печать, на сердце твое, как перстень, на руку твою: ибо крепка, как смерть, любовь; люта, как преисподняя, ревность; стрелы ее~--- стрелы огненные; она пламень весьма сильный.
\vs Sol 8:7 Большие воды не могут потушить любви, и реки не зальют ее. Если бы кто давал все богатство дома своего за любовь, то он был бы отвергнут с презреньем.
\rsbpar\vs Sol 8:8 Есть у нас сестра, которая еще мала, и сосцов нет у нее; что нам будет делать с сестрою нашею, когда будут свататься за нее?
\vs Sol 8:9 Если бы она была стена, то мы построили бы на ней палаты из серебра; если бы она была дверь, то мы обложили бы ее кедровыми досками.
\vs Sol 8:10 Я~--- стена, и сосцы у меня, как башни; потому я буду в глазах его, как достигшая полноты.
\rsbpar\vs Sol 8:11 Виноградник был у Соломона в Ваал-Гамоне; он отдал этот виноградник сторожам; каждый должен был доставлять за плоды его тысячу сребреников.
\vs Sol 8:12 А мой виноградник у меня при себе. Тысяча пусть тебе, Соломон, а двести~--- стерегущим плоды его.
\vs Sol 8:13 Жительница садов! товарищи внимают голосу твоему, дай и мне послушать его.
\rsbpar\vs Sol 8:14 Беги, возлюбленный мой; будь подобен серне или молодому оленю на горах бальзамических!

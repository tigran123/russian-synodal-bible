\bibbookdescr{4Sb}{
  inline={Четвёртая книга Сивилл},
  toc={4-я Сивилл},
  bookmark={4-я Сивилл},
  header={4-я Сивилл},
  abbr={4~Сив}
}
\vs 4Sb 1:1 Слушай, Азийский народ надменный и Европейцы, 

\vs 4Sb 1:2 Все, что намерена я правдиво вам напророчить, 

\vs 4Sb 1:3 Мощные звуки издав из широкоотверстого горла! 

\vs 4Sb 1:4 И не от лживого Феба, которого глупые люди

\vs 4Sb 1:5 Богом назвали, ему приписав, что будто пророк он, 

\vs 4Sb 1:6 Стану вещать, но послушна желанию вечного Бога, 

\vs 4Sb 1:7 Руки кого не слепили людские, подобно тому как 

\vs 4Sb 1:8 Идолов лепят немых и из камня их высекают. 

\vs 4Sb 1:9 В камень не заключен, безмолвно в храме стоящий  

\vs 4Sb 1:10 Глух ко всему, позор жалчайший для рода людского, 

\vs 4Sb 1:11 Бог не видим с земли, глазами Его не измерить,

\vs 4Sb 1:12 Теми, что смертным даны, рукою смертной не создан, 

\vs 4Sb 1:13 Разом всех видя с небес, никем быть увиден не может. 

\vs 4Sb 1:14 Ночь, несущую мрак, светлый день и яркое солнце, 

\vs 4Sb 1:15 Звезды вместе с луной, кишащее рыбами море,

\vs 4Sb 1:16 Землю, реки и устья источников вечнотекущих  

\vs 4Sb 1:17 Все сотворил Он для жизни; дожди, что прольются над пашней,

\vs 4Sb 1:18 Ей обещав урожай, дав деревья, лозу и оливу. 

\vs 4Sb 1:19 Он меня в грудь поразил, бичом полоснул мне по сердцу, 

\vs 4Sb 1:20 Чтобы я людям про то, что есть теперь и что будет

\vs 4Sb 1:21 С ними, от первого рода начав и окончив десятым, 

\vs 4Sb 1:22 Все достоверно сказала. Ведь Тот мне это доверил, 

\vs 4Sb 1:23 Сам Кто причина всему. Народ, послушай Сивиллу, 

\vs 4Sb 1:24 Льющую голос правдивый из уст, благочестия полных!

\vs 4Sb 1:25 Те из людей на земле изведают счастье, что Бога 

\vs 4Sb 1:26 Будут великого чтить и Его прославлять непрестанно, 

\vs 4Sb 1:27 Прежде еды и питья стремясь к благочестию в жизни. 

\vs 4Sb 1:28 Храмы отвергнут они все сразу, лишь только увидят;

\vs 4Sb 1:29 То же и алтари  постройки из мертвого камня 

\vs 4Sb 1:30 Лживых во славу богов, оскверненные кровью животных, 

\vs 4Sb 1:31 Жертвенным дымом. Их люди во имя единого Бога 

\vs 4Sb 1:32 Все забросают камнями, запрет положив на убийство, 

\vs 4Sb 1:33 Тайную мзду не приняв, что стало бы худшим началом. 

\vs 4Sb 1:34 Также не будут искать утех на чужом они ложе, 

\vs 4Sb 1:35 Дерзость мужей им чужда и всегда ненавистна пребудет.

\vs 4Sb 1:36 Не переймут никогда тот характер, нрав и обычай 

\vs 4Sb 1:37 Прочие люди. Они, во всем тяготея к безстыдству, 

\vs 4Sb 1:38 Горло в насмешках сорвав, над этими станут смеяться  

\vs 4Sb 1:39 Глупые дети!  и первым наивно приписывать станут 

\vs 4Sb 1:40 То беззаконье и зло, которое сами свершили.

\vs 4Sb 1:41 Полон неверия род людской. Когда же наступит

\vs 4Sb 1:42 Суд над людьми и всем миром, что Сам вселенной Создатель 

\vs 4Sb 1:43 Будет вершить, на него нечестивых и праведных вместе 

\vs 4Sb 1:44 Вызвав,  Он их разведет: порочных в пламень отправит, 

\vs 4Sb 1:45 В мрак преисподней, чтоб там осознали они, что творили;

\vs 4Sb 1:46 Праведным выпадет жить на равнине, обильной плодами, 

\vs 4Sb 1:47 Вместе с дыханьем Господь им жизнь и радость дарует. 

\vs 4Sb 1:48 Это случиться должно при людях в десятом колене, 

\vs 4Sb 1:49 Что же в первом их ждет и дальше  о том расскажу я.

\vs 4Sb 1:50 Править всеми людьми Ассирийцы будут вначале, 

\vs 4Sb 1:51 Власть над миром держа в пределах шести поколений,  

\vs 4Sb 1:52 После чего Божий гнев на них обрушится с неба, 

\vs 4Sb 1:53 На города и людей, живущих под тем небосводом. 

\vs 4Sb 1:54 Морем тут станет земля из-за вод, что внезапно нахлынут.

\vs 4Sb 1:55 Свергнут Мидийцы их власть и сами возсядут на троны  

\vs 4Sb 1:56 Два поколенья всего будут править. При них совершится 

\vs 4Sb 1:57 Вот что: наступит вдруг ночь среди дня и землю накроет, 

\vs 4Sb 1:58 Звезды с небес пропадут, и Луны круг тоже исчезнет; 

\vs 4Sb 1:59 Почва, вся сотрясаясь от мощных подземных ударов, 

\vs 4Sb 1:60 Много сметет городов и того, что построили люди,  

\vs 4Sb 1:61 Из глубины же морской острова всплывут на поверхность.

\vs 4Sb 1:62 Но когда разольется Евфрат великий от крови, 

\vs 4Sb 1:63 Страшная битва случится тут между Мидийцев и Персов 

\vs 4Sb 1:64 В их друг с другом войне. Под копьями Персов Мидийцы,

\vs 4Sb 1:65 Падая, прочь побегут через воды великого Тигра. 

\vs 4Sb 1:66 Сила же Персов пускай величайшею в мире пребудет, 

\vs 4Sb 1:67 Им предстоит лишь одно поколение счастливо править.

\vs 4Sb 1:68 Многие беды ждут мир, их осыплют проклятьями люди: 

\vs 4Sb 1:69 Кровопролитные войны, убийства, изгнания, распри, 

\vs 4Sb 1:70 Гибель больших городов, падение башен высоких  

\vs 4Sb 1:71 Эллин надменный когда по соленой волне Геллеспонта, 

\vs 4Sb 1:72 Смерть неся Финикийцам и Азии, путь свой направит.

\vs 4Sb 1:73 В хлебном Египте, где вся земля распахана плугом, 

\vs 4Sb 1:74 Голод и неурожай на двадцать лет воцарятся. 

\vs 4Sb 1:75 Нил тому будет причиной, колосьям жизнь приносящий,  

\vs 4Sb 1:76 Темные воды свои упрячет он где-то под землю.

\vs 4Sb 1:77 Будет из Азии царь, что копье большое поднимет, 

\vs 4Sb 1:78 На несметных судах. По влажным дорогам пучины 

\vs 4Sb 1:79 Шагом пройдет, проплывет, разсекши высокую гору. 

\vs 4Sb 1:80 После же бегства с войны его грозная Азия примет.

\vs 4Sb 1:81 Остров Сицилия весь сожжен будет мощным потоком 

\vs 4Sb 1:82 Лавы кипящей, из недр что извергнет с пламенем Этна. 

\vs 4Sb 1:83 Город же славный Кротон погрузится в глубокое море.

\vs 4Sb 1:84 Вспыхнет в Элладе вражда. Совсем обезумев от гнева, 

\vs 4Sb 1:85 Много с землей городов сровняют и многих погубят 

\vs 4Sb 1:86 В битве жестокой. Война принесет всем поровну горя.

\vs 4Sb 1:87 В роде когда же людском поколений сменится десять, 

\vs 4Sb 1:88 Персов рабский ярем тогда ожидает и ужас.

\vs 4Sb 1:89 Слава правителей мира когда отойдет к Македонцам,

\vs 4Sb 1:90 Фивам не избежать позорного будет захвата, 

\vs 4Sb 1:91 Тир населят Карийцы, а жители Тира погибнут.

\vs 4Sb 1:92 Самос засыплет песком, сровняет его с берегами.

\vs 4Sb 1:93 Делос исчезнет из глаз, и все, что на Делосе, тоже.

\vs 4Sb 1:94 Грозный на вид Вавилон, однако слабый в сраженье,

\vs 4Sb 1:95 Будет стоять, возведен на надеждах, не могущих сбыться. 

\vs 4Sb 1:96 Бактры займут Македонцы; их жители, город оставив,

\vs 4Sb 1:97 Так же, как жители Суз, все ринутся в землю Эллады.

\vs 4Sb 1:98 Все это в будущем ждет, когда Пирам среброструйный, 

\vs 4Sb 1:99 Воду в залив вынося, священный остров омоет. 

\vs 4Sb 1:100 В море сползут Сибарис и Кизик, от колебаний 

\vs 4Sb 1:101 Почвы; оба падут под напором подземных ударов. 

\vs 4Sb 1:102 Родос последним постигнет несчастье, но будет сильнейшим.

\vs 4Sb 1:103 Власть Македонцев продлится недолго: там, где заходит 

\vs 4Sb 1:104 Солнце, начнется война Италийская, ей подчинятся 

\vs 4Sb 1:105 Все и под рабским ярмом Италийцам прислуживать станут. 

\vs 4Sb 1:106 Ты же, несчастный Коринф, свое разоренье увидишь. 

\vs 4Sb 1:107 Башни твои, Кархедон, к земле преклонят колено.

\vs 4Sb 1:108 Стойкая Лаодикия, тебя опрокинет однажды 

\vs 4Sb 1:109 Землетрясенье, но ты поднимешься вновь из развалин. 

\vs 4Sb 1:110 О Ликийские Миры, краса городов! Никогда вас 

\vs 4Sb 1:111 Прочно земля не удержит, сама сотрясаясь. В паденье 

\vs 4Sb 1:112 Ниже и ниже клонясь, вы другую страну изберете, 

\vs 4Sb 1:113 Чтобы в ней жизнь продолжать  настоящий метек, а не город.

\vs 4Sb 1:114 Из-за нечестья тогда же затихнет и город Патары, 

\vs 4Sb 1:115 Море его поглотит при землетрясенье и буре.

\vs 4Sb 1:116 Также тебе предстоит, Армения, рабская участь.

\vs 4Sb 1:117 Грозной войны ураган домчит до Иерусалима,

\vs 4Sb 1:118 Путь свой начав с Апеннин, и Храм великий разрушит. 

\vs 4Sb 1:119 Тут, безрассудству отдавшись, когда благочестье отринут 

\vs 4Sb 1:120 И в преддверии храма творить будут жуткую бойню,  

\vs 4Sb 1:121 Царь великий тогда из Италии, словно разбойник,

\vs 4Sb 1:122 Пустится в бегство, невидим, неслышим, за воды Евфрата. 

\vs 4Sb 1:123 После того, как он грех величайший  матери гибель  

\vs 4Sb 1:124 Не побоится принять, и другие свершит преступленья. 

\vs 4Sb 1:125 Многие кровью зальют подножие Римского трона 

\vs 4Sb 1:126 Сразу, как тот убежит через земли Парфянского царства.

\vs 4Sb 1:127 В Сирию воин из Рима придет. Он, Иерусалимский 

\vs 4Sb 1:128 Храм предоставив огню и многих убив Иудеев, 

\vs 4Sb 1:129 Их великую землю, дорогами славную, сгубит.

\vs 4Sb 1:130 Пафос и Саламин уничтожит землетрясенье,

\vs 4Sb 1:131 Кипр, омываем полной, когда черная скроет пучина.

\vs 4Sb 1:132 В час, когда, из глубин разверстой земли Италийской 

\vs 4Sb 1:133 Вырвавшись, огненный столб до широкого неба достанет, 

\vs 4Sb 1:134 Много тут городов он сожжет и многих погубит. 

\vs 4Sb 1:135 Тучи горящего пепла весь воздух собою заполнят, 

\vs 4Sb 1:136 С неба частички его будут падать как красная краска.

\vs 4Sb 1:137 В этом увидеть должны явление Божьего гнева, 

\vs 4Sb 1:138 Благочестивое племя поскольку безвинно страдает. 

\vs 4Sb 1:139 Повод для новой войны появится скоро: на Запад 

\vs 4Sb 1:140 Явится тот, кто бежал из Рима; копье он поднимет, 

\vs 4Sb 1:141 Снова Евфрат перейдя, приведет несметное войско.

\vs 4Sb 1:142 Бедная Антиохия! Ты городом зваться не станешь 

\vs 4Sb 1:143 После того, как падешь под копьями по безразсудству. 

\vs 4Sb 1:144 Голод погубит тогда Киприотов и страшная битва.

\vs 4Sb 1:145 Остров несчастный, о Кипр! Увы тебе! Волны морские 

\vs 4Sb 1:146 Скроют тебя под собой  добычу неистовой бури.

\vs 4Sb 1:147 В Азию груз драгоценный прибудет, что некогда Римом 

\vs 4Sb 1:148 Был добыт на войне и в городе этом хранился. 

\vs 4Sb 1:149 Дважды по столько затем придется еще им отправить 

\vs 4Sb 1:150 Азии в качестве платы за все неудачи в сраженьях.

\vs 4Sb 1:151 Карии все города, что лежат по теченью Меандра  

\vs 4Sb 1:152 Стенами окружены, прекрасны,  свирепый погубит 

\vs 4Sb 1:153 Голод, сокроет когда Меандр свою черную воду.

\vs 4Sb 1:154 Стоит в сердцах человечьих изсякнуть почтению к Богу,

\vs 4Sb 1:155 Вере и праву навеки из мира стоит исчезнуть,

\vs 4Sb 1:156 Как, нетвердые духом в дерзаньях своих нечестивых,

\vs 4Sb 1:157 Люди станут вершить произвол и творить злодеянья. 

\vs 4Sb 1:158 С благочестивым никто к беседе стремиться не будет, 

\vs 4Sb 1:159 Их же, напротив, самих истребят глупцы и безумцы, 

\vs 4Sb 1:160 Наглости собственной рады, с руками, покрытыми кровью. 

\vs 4Sb 1:161 Тут им придется узнать, что милостив дольше не будет

\vs 4Sb 1:162 Бог, но, безудержный в гневе, намерен род погубить их  

\vs 4Sb 1:163 Так, чтобы весь он сгорел во время большого пожара.

\vs 4Sb 1:164 Образ мыслей смените, пустые люди, и кару

\vs 4Sb 1:165 Не вынуждайте Его для вас выбирать. Отказавшись

\vs 4Sb 1:166 От мечей и убийств, от стонов и беззаконья, 

\vs 4Sb 1:167 В реках вечнотекущих омойте все свое тело.

\vs 4Sb 1:168 Руки воздев к небесам, к тому, что прежде свершили, 

\vs 4Sb 1:169 О снисхожденье просите и, Богу хвалу воздавая, 

\vs 4Sb 1:170 Милость Его призывайте к себе, нечестивым. Дарует 

\vs 4Sb 1:171 Он прощение всем, не погубит  снова утихнет 

\vs 4Sb 1:172 Гнев, если в душах своих воспитаете вы благочестье. 

\vs 4Sb 1:173 Если ж не верите мне и нечестие вашему сердцу, 

\vs 4Sb 1:174 Глупые люди, дороже всего, а речи  впустую, 

\vs 4Sb 1:175 Пламя охватит тогда весь мир и знак величайший 

\vs 4Sb 1:176 Меч подаст и труба на восходе дневного светила. 

\vs 4Sb 1:177 Глас тот мощный и рев услышат во всей поднебесной. 

\vs 4Sb 1:178 Выжжена будет земля, человеческий род уничтожен, 

\vs 4Sb 1:179 Вместе же с ним города, пресноводные реки и море. 

\vs 4Sb 1:180 Пеплом все станет, и прах раскаленный ляжет повсюду.

\vs 4Sb 1:181 Но когда, кроме золы, ничего уже в мире не будет, 

\vs 4Sb 1:182 Бог успокоит огонь несказанный, как некогда вызвал.

\vs 4Sb 1:183 Пепел и кости людские вновь Сам соберет и придаст им

\vs 4Sb 1:184 Прежнюю форму. Так род Он смертных людей возстановит.

\vs 4Sb 1:185 После того будет Суд, и Сам Он вершить его станет,

\vs 4Sb 1:186 Мир к ответу призвав: тут тех, кто, живя нечестиво, 

\vs 4Sb 1:187 Истинной веры не знал, земляная толща накроет

\vs 4Sb 1:188 Душного Тартара, пропасть поглотит ужасной геенны.

\vs 4Sb 1:189 Людям же праведным вновь разрешит на земле поселиться,

\vs 4Sb 1:190 Вместе с дыханием жизнь Господь им и радость дарует. 

\vs 4Sb 1:191 Все они тотчас себя увидят при благостном свете

\vs 4Sb 1:192 Солнца, которое впредь уходить с небосвода не будет.

\vs 4Sb 1:193 Счастлив тот человек, кому жить в это время придется.

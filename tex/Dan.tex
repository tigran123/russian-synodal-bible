\bibbookdescr{Dan}{
  inline={\LARGE Книга\\\Huge Пророка Даниила},
  toc={Даниил},
  bookmark={Даниил},
  header={Даниил},
  %headerleft={},
  %headerright={},
  abbr={Дан}
}
\vs Dan 1:1 В третий год царствования Иоакима, царя Иудейского, пришел Навуходоносор, царь Вавилонский, к Иерусалиму и осадил его.
\vs Dan 1:2 И предал Господь в руку его Иоакима, царя Иудейского, и часть сосудов дома Божия, и он отправил их в землю Сеннаар, в дом бога своего, и внес эти сосуды в сокровищницу бога своего.
\vs Dan 1:3 И сказал царь Асфеназу, начальнику евнухов своих, чтобы он из сынов Израилевых, из рода царского и княжеского, привел
\vs Dan 1:4 отроков, у которых нет никакого телесного недостатка, красивых видом, и понятливых для всякой науки, и разумеющих науки, и смышленых и годных служить в чертогах царских, и чтобы научил их книгам и языку Халдейскому.
\vs Dan 1:5 И назначил им царь ежедневную пищу с царского стола и вино, которое сам пил, и велел воспитывать их три года, по истечении которых они должны были предстать пред царя.
\vs Dan 1:6 Между ними были из сынов Иудиных Даниил, Анания, Мисаил и Азария.
\vs Dan 1:7 И переименовал их начальник евнухов~--- Даниила Валтасаром, Ананию Седрахом, Мисаила Мисахом и Азарию Авденаго.
\vs Dan 1:8 Даниил положил в сердце своем не оскверняться яствами со стола царского и вином, какое пьет царь, и потому просил начальника евнухов о том, чтобы не оскверняться ему.
\vs Dan 1:9 Бог даровал Даниилу милость и благорасположение начальника евнухов;
\vs Dan 1:10 и начальник евнухов сказал Даниилу: боюсь я господина моего, царя, который сам назначил вам пищу и питье; если он увидит лица ваши худощавее, нежели у отроков, сверстников ваших, то вы сделаете голову мою виновною перед царем.
\vs Dan 1:11 Тогда сказал Даниил Амелсару, которого начальник евнухов приставил к Даниилу, Анании, Мисаилу и Азарии:
\vs Dan 1:12 сделай опыт над рабами твоими в течение десяти дней; пусть дают нам в пищу овощи и воду для питья;
\vs Dan 1:13 и потом пусть явятся перед тобою лица наши и лица тех отроков, которые питаются царскою пищею, и затем поступай с рабами твоими, как увидишь.
\vs Dan 1:14 Он послушался их в этом и испытывал их десять дней.
\vs Dan 1:15 По истечении же десяти дней лица их оказались красивее, и телом они были полнее всех тех отроков, которые питались царскими яствами.
\vs Dan 1:16 Тогда Амелсар брал их кушанье и вино для питья и давал им овощи.
\vs Dan 1:17 И даровал Бог четырем сим отрокам знание и разумение всякой книги и мудрости, а Даниилу еще даровал разуметь и всякие видения и сны.
\vs Dan 1:18 По окончании тех дней, когда царь приказал представить их, начальник евнухов представил их Навуходоносору.
\vs Dan 1:19 И царь говорил с ними, и из всех \bibemph{отроков} не нашлось подобных Даниилу, Анании, Мисаилу и Азарии, и стали они служить пред царем.
\vs Dan 1:20 И во всяком деле мудрого уразумения, о чем ни спрашивал их царь, он находил их в десять раз выше всех тайноведцев и волхвов, какие были во всем царстве его.
\vs Dan 1:21 И был там Даниил до первого года царя Кира.
\vs Dan 2:1 Во второй год царствования Навуходоносора снились Навуходоносору сны, и возмутился дух его, и сон удалился от него.
\vs Dan 2:2 И велел царь созвать тайноведцев, и гадателей, и чародеев, и Халдеев, чтобы они рассказали царю сновидения его. Они пришли, и стали перед царем.
\vs Dan 2:3 И сказал им царь: сон снился мне, и тревожится дух мой; желаю знать этот сон.
\vs Dan 2:4 И сказали Халдеи царю по-арамейски: царь! вовеки живи! скажи сон рабам твоим, и мы объясним значение его.
\vs Dan 2:5 Отвечал царь и сказал Халдеям: слово отступило от меня; если вы не скажете мне сновидения и значения его, то в куски будете изрублены, и домы ваши обратятся в развалины.
\vs Dan 2:6 Если же расскажете сон и значение его, то получите от меня дары, награду и великую почесть; итак скажите мне сон и значение его.
\vs Dan 2:7 Они вторично отвечали и сказали: да скажет царь рабам своим сновидение, и мы объясним его значение.
\vs Dan 2:8 Отвечал царь и сказал: верно знаю, что вы хотите выиграть время, потому что видите, что слово отступило от меня.
\vs Dan 2:9 Так как вы не объявляете мне сновидения, то у вас один умысел: вы собираетесь сказать мне ложь и обман, пока минет время; итак расскажите мне сон, и тогда я узнаю, что вы можете объяснить мне и значение его.
\vs Dan 2:10 Халдеи отвечали царю и сказали: нет на земле человека, который мог бы открыть это дело царю, и потому ни один царь, великий и могущественный, не требовал подобного ни от какого тайноведца, гадателя и Халдея.
\vs Dan 2:11 Дело, которого царь требует, так трудно, что никто другой не может открыть его царю, кроме богов, которых обитание не с плотью.
\vs Dan 2:12 Рассвирепел царь и сильно разгневался на это, и приказал истребить всех мудрецов Вавилонских.
\rsbpar\vs Dan 2:13 Когда вышло это повеление, чтобы убивать мудрецов, искали Даниила и товарищей его, чтобы умертвить их.
\vs Dan 2:14 Тогда Даниил обратился с советом и мудростью к Ариоху, начальнику царских телохранителей, который вышел убивать мудрецов Вавилонских;
\vs Dan 2:15 и спросил Ариоха, сильного при царе: <<почему такое грозное повеление от царя?>> Тогда Ариох рассказал все дело Даниилу.
\vs Dan 2:16 И Даниил вошел, и упросил царя дать ему время, и он представит царю толкование \bibemph{сна}.
\vs Dan 2:17 Даниил пришел в дом свой, и рассказал дело Анании, Мисаилу и Азарии, товарищам своим,
\vs Dan 2:18 чтобы они просили милости у Бога небесного об этой тайне, дабы Даниил и товарищи его не погибли с прочими мудрецами Вавилонскими.
\vs Dan 2:19 И тогда открыта была тайна Даниилу в ночном видении, и Даниил благословил Бога небесного.
\vs Dan 2:20 И сказал Даниил: да будет благословенно имя Господа от века и до века! ибо у Него мудрость и сила;
\vs Dan 2:21 Он изменяет времена и лета, низлагает царей и поставляет царей; дает мудрость мудрым и разумение разумным;
\vs Dan 2:22 Он открывает глубокое и сокровенное, знает, что во мраке, и свет обитает с Ним.
\vs Dan 2:23 Славлю и величаю Тебя, Боже отцов моих, что Ты даровал мне мудрость и силу и открыл мне то, о чем мы молили Тебя; ибо Ты открыл нам дело царя.
\vs Dan 2:24 После сего Даниил вошел к Ариоху, которому царь повелел умертвить мудрецов Вавилонских, пришел и сказал ему: не убивай мудрецов Вавилонских; введи меня к царю, и я открою значение \bibemph{сна}.
\vs Dan 2:25 Тогда Ариох немедленно привел Даниила к царю и сказал ему: я нашел из пленных сынов Иудеи человека, который может открыть царю значение \bibemph{сна}.
\vs Dan 2:26 Царь сказал Даниилу, который назван был Валтасаром: можешь ли ты сказать мне сон, который я видел, и значение его?
\rsbpar\vs Dan 2:27 Даниил отвечал царю и сказал: тайны, о которой царь спрашивает, не могут открыть царю ни мудрецы, ни обаятели, ни тайноведцы, ни гадатели.
\vs Dan 2:28 Но есть на небесах Бог, открывающий тайны; и Он открыл царю Навуходоносору, что будет в последние дни. Сон твой и видения главы твоей на ложе твоем были такие:
\vs Dan 2:29 ты, царь, на ложе твоем думал о том, что будет после сего? и Открывающий тайны показал тебе то, что будет.
\vs Dan 2:30 А мне тайна сия открыта не потому, чтобы я был мудрее всех живущих, но для того, чтобы открыто было царю разумение и чтобы ты узнал помышления сердца твоего.
\vs Dan 2:31 Тебе, царь, было такое видение: вот, какой-то большой истукан; огромный был этот истукан, в чрезвычайном блеске стоял он пред тобою, и страшен был вид его.
\vs Dan 2:32 У этого истукана голова была из чистого золота, грудь его и руки его~--- из серебра, чрево его и бедра его медные,
\vs Dan 2:33 голени его железные, ноги его частью железные, частью глиняные.
\vs Dan 2:34 Ты видел его, доколе камень не оторвался от горы без содействия рук, ударил в истукана, в железные и глиняные ноги его, и разбил их.
\vs Dan 2:35 Тогда все вместе раздробилось: железо, глина, медь, серебро и золото сделались как прах на летних гумнах, и ветер унес их, и следа не осталось от них; а камень, разбивший истукана, сделался великою горою и наполнил всю землю.
\vs Dan 2:36 Вот сон! Скажем пред царем и значение его.
\vs Dan 2:37 Ты, царь, царь царей, которому Бог небесный даровал царство, власть, силу и славу,
\vs Dan 2:38 и всех сынов человеческих, где бы они ни жили, зверей земных и птиц небесных Он отдал в твои руки и поставил тебя владыкою над всеми ими. Ты~--- это золотая голова!
\vs Dan 2:39 После тебя восстанет другое царство, ниже твоего, и еще третье царство, медное, которое будет владычествовать над всею землею.
\vs Dan 2:40 А четвертое царство будет крепко, как железо; ибо как железо разбивает и раздробляет все, так и оно, подобно всесокрушающему железу, будет раздроблять и сокрушать.
\vs Dan 2:41 А что ты видел ноги и пальцы на ногах частью из глины горшечной, а частью из железа, то будет царство разделенное, и в нем останется несколько крепости железа, так как ты видел железо, смешанное с горшечною глиною.
\vs Dan 2:42 И как персты ног были частью из железа, а частью из глины, так и царство будет частью крепкое, частью хрупкое.
\vs Dan 2:43 А что ты видел железо, смешанное с глиною горшечною, это значит, что они смешаются через семя человеческое, но не сольются одно с другим, как железо не смешивается с глиною.
\vs Dan 2:44 И во дни тех царств Бог небесный воздвигнет царство, которое вовеки не разрушится, и царство это не будет передано другому народу; оно сокрушит и разрушит все царства, а само будет стоять вечно,
\vs Dan 2:45 так как ты видел, что камень отторгнут был от горы не руками и раздробил железо, медь, глину, серебро и золото. Великий Бог дал знать царю, что будет после сего. И верен этот сон, и точно истолкование его!
\vs Dan 2:46 Тогда царь Навуходоносор пал на лице свое и поклонился Даниилу, и велел принести ему дары и благовонные курения.
\vs Dan 2:47 И сказал царь Даниилу: истинно Бог ваш есть Бог богов и Владыка царей, открывающий тайны, когда ты мог открыть эту тайну!
\vs Dan 2:48 Тогда возвысил царь Даниила и дал ему много больших подарков, и поставил его над всею областью Вавилонскою и главным начальником над всеми мудрецами Вавилонскими.
\vs Dan 2:49 Но Даниил просил царя, и он поставил Седраха, Мисаха и Авденаго над делами страны Вавилонской, а Даниил остался при дворе царя.
\vs Dan 3:1 Царь Навуходоносор сделал золотой истукан, вышиною в шестьдесят локтей, шириною в шесть локтей, поставил его на поле Деире, в области Вавилонской.
\vs Dan 3:2 И послал царь Навуходоносор собрать сатрапов, наместников, воевод, верховных судей, казнохранителей, законоведцев, блюстителей суда и всех областных правителей, чтобы они пришли на торжественное открытие истукана, которого поставил царь Навуходоносор.
\vs Dan 3:3 И собрались сатрапы, наместники, военачальники, верховные судьи, казнохранители, законоведцы, блюстители суда и все областные правители на открытие истукана, которого Навуходоносор царь поставил, и стали перед истуканом, которого воздвиг Навуходоносор.
\vs Dan 3:4 Тогда глашатай громко воскликнул: объявляется вам, народы, племена и языки:
\vs Dan 3:5 в то время, как услышите звук трубы, свирели, цитры, цевницы, гуслей и симфонии и всяких музыкальных орудий, падите и поклонитесь золотому истукану, которого поставил царь Навуходоносор.
\vs Dan 3:6 А кто не падет и не поклонится, тотчас брошен будет в печь, раскаленную огнем.
\vs Dan 3:7 Посему, когда все народы услышали звук трубы, свирели, цитры, цевницы, гуслей и всякого рода музыкальных орудий, то пали все народы, племена и языки, и поклонились золотому истукану, которого поставил Навуходоносор царь.
\rsbpar\vs Dan 3:8 В это самое время приступили некоторые из Халдеев и донесли на Иудеев.
\vs Dan 3:9 Они сказали царю Навуходоносору: царь, вовеки живи!
\vs Dan 3:10 Ты, царь, дал повеление, чтобы каждый человек, который услышит звук трубы, свирели, цитры, цевницы, гуслей и симфонии и всякого рода музыкальных орудий, пал и поклонился золотому истукану;
\vs Dan 3:11 а кто не падет и не поклонится, тот должен быть брошен в печь, раскаленную огнем.
\vs Dan 3:12 Есть мужи Иудейские, которых ты поставил над делами страны Вавилонской: Седрах, Мисах и Авденаго; эти мужи не повинуются повелению твоему, царь, богам твоим не служат и золотому истукану, которого ты поставил, не поклоняются.
\vs Dan 3:13 Тогда Навуходоносор во гневе и ярости повелел привести Седраха, Мисаха и Авденаго; и приведены были эти мужи к царю.
\vs Dan 3:14 Навуходоносор сказал им: с умыслом ли вы, Седрах, Мисах и Авденаго, богам моим не служите, и золотому истукану, которого я поставил, не поклоняетесь?
\vs Dan 3:15 Отныне, если вы готовы, как скоро услышите звук трубы, свирели, цитры, цевницы, гуслей, симфонии и всякого рода музыкальных орудий, падите и поклонитесь истукану, которого я сделал; если же не поклонитесь, то в тот же час брошены будете в печь, раскаленную огнем, и тогда какой Бог избавит вас от руки моей?
\vs Dan 3:16 И отвечали Седрах, Мисах и Авденаго, и сказали царю Навуходоносору: нет нужды нам отвечать тебе на это.
\vs Dan 3:17 Бог наш, Которому мы служим, силен спасти нас от печи, раскаленной огнем, и от руки твоей, царь, избавит.
\vs Dan 3:18 Если же и не будет того, то да будет известно тебе, царь, что мы богам твоим служить не будем и золотому истукану, которого ты поставил, не поклонимся.
\vs Dan 3:19 Тогда Навуходоносор исполнился ярости, и вид лица его изменился на Седраха, Мисаха и Авденаго, и он повелел разжечь печь в семь раз сильнее, нежели как обыкновенно разжигали ее,
\vs Dan 3:20 и самым сильным мужам из войска своего приказал связать Седраха, Мисаха и Авденаго и бросить их в печь, раскаленную огнем.
\vs Dan 3:21 Тогда мужи сии связаны были в исподнем и верхнем платье своем, в головных повязках и в прочих одеждах своих, и брошены в печь, раскаленную огнем.
\vs Dan 3:22 И как повеление царя было строго, и печь раскалена была чрезвычайно, то пламя огня убило тех людей, которые бросали Седраха, Мисаха и Авденаго.
\vs Dan 3:23 А сии три мужа, Седрах, Мисах и Авденаго, упали в раскаленную огнем печь связанные.
\vs Dan 3:24 \fns{Стихи с 24-го по 90-й переведены с греческого, потому что в еврейском тексте их нет.}[И ходили посреди пламени, воспевая Бога и благословляя Господа.
\vs Dan 3:25 И став Азария молился и, открыв уста свои среди огня, возгласил:
\rsbpar\vs Dan 3:26 <<Благословен Ты, Господи Боже отцов наших, хвально и прославлено имя Твое вовеки.
\vs Dan 3:27 Ибо праведен Ты во всем, что соделал с нами, и все дела Твои истинны и пути Твои правы, и все суды Твои истинны.
\vs Dan 3:28 Ты совершил истинные суды во всем, что навел на нас и на святый град отцов наших Иерусалим, потому что по истине и по суду навел Ты все это на нас за грехи наши.
\vs Dan 3:29 Ибо согрешили мы, и поступили беззаконно, отступив от Тебя, и во всем согрешили.
\vs Dan 3:30 Заповедей Твоих не слушали и не соблюдали их, и не поступали, как Ты повелел нам, чтобы благо нам было.
\vs Dan 3:31 И все, что Ты навел на нас, и все, что Ты соделал с нами, соделал по истинному суду.
\vs Dan 3:32 И предал нас в руки врагов беззаконных, ненавистнейших отступников, и царю неправосудному и злейшему на всей земле.
\vs Dan 3:33 И ныне мы не можем открыть уст наших; мы сделались стыдом и поношением для рабов Твоих и чтущих Тебя.
\vs Dan 3:34 Но не предай нас навсегда ради имени Твоего, и не разруши завета Твоего.
\vs Dan 3:35 Не отними от нас милости Твоей ради Авраама, возлюбленного Тобою, ради Исаака, раба Твоего, и Израиля, святаго Твоего,
\vs Dan 3:36 которым Ты говорил, что умножишь семя их, как звезды небесные и как песок на берегу моря.
\vs Dan 3:37 Мы умалены, Господи, паче всех народов, и унижены ныне на всей земле за грехи наши,
\vs Dan 3:38 и нет у нас в настоящее время ни князя, ни пророка, ни вождя, ни всесожжения, ни жертвы, ни приношения, ни фимиама, ни места, чтобы нам принести жертву Тебе и обрести милость Твою.
\vs Dan 3:39 Но с сокрушенным сердцем и смиренным духом да будем приняты.
\vs Dan 3:40 Как при всесожжении овнов и тельцов и как при тысячах тучных агнцев, так да будет жертва наша пред Тобою ныне благоугодною Тебе; ибо нет стыда уповающим на Тебя.
\vs Dan 3:41 И ныне мы следуем за Тобою всем сердцем и боимся Тебя и ищем лица Твоего.
\vs Dan 3:42 Не посрами нас, но сотвори с нами по снисхождению Твоему и по множеству милости Твоей
\vs Dan 3:43 и избави нас силою чудес Твоих, и дай славу имени Твоему, Господи,
\vs Dan 3:44 и да постыдятся все, делающие рабам Твоим зло, и да постыдятся со всем могуществом, и сила их да сокрушится,
\vs Dan 3:45 и да познают, что Ты Господь Бог един и славен по всей вселенной>>.
\rsbpar\vs Dan 3:46 А между тем слуги царя, ввергшие их, не переставали разжигать печь нефтью, смолою, паклею и хворостом,
\vs Dan 3:47 и поднимался пламень над печью на сорок девять локтей
\vs Dan 3:48 и вырывался, и сожигал тех из Халдеев, которых достигал около печи.
\vs Dan 3:49 Но Ангел Господень сошел в печь вместе с Азариею и бывшими с ним
\vs Dan 3:50 и выбросил пламень огня из печи, и сделал, что в средине печи был как бы шумящий влажный ветер, и огонь нисколько не прикоснулся к ним, и не повредил им, и не смутил их.
\vs Dan 3:51 Тогда сии трое, как бы одними устами, воспели в печи, и благословили и прославили Бога:
\rsbpar\vs Dan 3:52 <<Благословен Ты, Господи Боже отцов наших, и хвальный и превозносимый во веки, и благословенно имя славы Твоей, святое и прехвальное и превозносимое во веки.
\vs Dan 3:53 Благословен Ты в храме святой славы Твоей, и прехвальный и преславный во веки.
\vs Dan 3:54 Благословен Ты, видящий бездны, восседающий на Херувимах, и прехвальный и превозносимый во веки.
\vs Dan 3:55 Благословен Ты на престоле славы царства Твоего, и прехвальный и превозносимый во веки.
\vs Dan 3:56 Благословен Ты на тверди небесной, и прехвальный и превозносимый во веки.
\vs Dan 3:57 Благословите, все дела Господни, Господа, пойте и превозносите Его во веки.
\vs Dan 3:58 Благословите, Ангелы Господни, Господа, пойте и превозносите Его во веки.
\vs Dan 3:59 Благословите, небеса, Господа, пойте и превозносите Его во веки.
\vs Dan 3:60 Благословите Господа, все воды, которые превыше небес, пойте и превозносите Его во веки.
\vs Dan 3:61 Благословите, все силы Господни, Господа, пойте и превозносите Его во веки.
\vs Dan 3:62 Благословите, солнце и луна, Господа, пойте и превозносите Его во веки.
\vs Dan 3:63 Благословите, звезды небесные, Господа, пойте и превозносите Его во веки.
\vs Dan 3:64 Благословите, всякий дождь и роса, Господа, пойте и превозносите Его во веки.
\vs Dan 3:65 Благословите, все ветры, Господа, пойте и превозносите Его во веки.
\vs Dan 3:66 Благословите, огонь и жар, Господа, пойте и превозносите Его во веки.
\vs Dan 3:67 Благословите, холод и зной, Господа, пойте и превозносите Его во веки.
\vs Dan 3:68 Благословите, росы и инеи, Господа, пойте и превозносите Его во веки.
\vs Dan 3:69 Благословите, ночи и дни, Господа, пойте и превозносите Его во веки.
\vs Dan 3:70 Благословите, свет и тьма, Господа, пойте и превозносите Его во веки.
\vs Dan 3:71 Благословите, лед и мороз, Господа, пойте и превозносите Его во веки.
\vs Dan 3:72 Благословите, иней и снег, Господа, пойте и превозносите Его во веки.
\vs Dan 3:73 Благословите, молнии и облака, Господа, пойте и превозносите Его во веки.
\vs Dan 3:74 Да благословит земля Господа, да поет и превозносит Его во веки.
\vs Dan 3:75 Благословите, горы и холмы, Господа, пойте и превозносите Его во веки.
\vs Dan 3:76 Благословите Господа, все произрастания на земле, пойте и превозносите Его во веки.
\vs Dan 3:77 Благословите, источники, Господа, пойте и превозносите Его во веки.
\vs Dan 3:78 Благословите, моря и реки, Господа, пойте и превозносите Его во веки.
\vs Dan 3:79 Благословите Господа, киты и все, движущееся в водах, пойте и превозносите Его во веки.
\vs Dan 3:80 Благословите, все птицы небесные, Господа, пойте и превозносите Его во веки.
\vs Dan 3:81 Благословите Господа, звери и весь скот, пойте и превозносите Его во веки.
\vs Dan 3:82 Благословите, сыны человеческие, Господа, пойте и превозносите Его во веки.
\vs Dan 3:83 Благослови, Израиль, Господа, пой и превозноси Его во веки.
\vs Dan 3:84 Благословите, священники Господни, Господа, пойте и превозносите Его во веки.
\vs Dan 3:85 Благословите, рабы Господни, Господа, пойте и превозносите Его во веки.
\vs Dan 3:86 Благословите, духи и души праведных, Господа, пойте и превозносите Его во веки.
\vs Dan 3:87 Благословите, праведные и смиренные сердцем, Господа, пойте и превозносите Его во веки.
\vs Dan 3:88 Благословите, Анания, Азария и Мисаил, Господа, пойте и превозносите Его во веки; ибо Он извлек нас из ада и спас нас от руки смерти, и избавил нас из среды печи горящего пламени, и из среды огня избавил нас.
\vs Dan 3:89 Славьте Господа, ибо Он благ, ибо вовек милость Его.
\vs Dan 3:90 Благословите, все чтущие Господа, Бога богов, пойте и славьте, ибо вовек милость Его>>.]
\rsbpar\vs Dan 3:91 Навуходоносор царь, [услышав, что они поют,] изумился, и поспешно встал, и сказал вельможам своим: не троих ли мужей бросили мы в огонь связанными? Они в ответ сказали царю: истинно так, царь!
\vs Dan 3:92 На это он сказал: вот, я вижу четырех мужей несвязанных, ходящих среди огня, и нет им вреда; и вид четвертого подобен сыну Божию.
\vs Dan 3:93 Тогда подошел Навуходоносор к устью печи, раскаленной огнем, и сказал: Седрах, Мисах и Авденаго, рабы Бога Всевышнего! выйдите и подойдите! Тогда Седрах, Мисах и Авденаго вышли из среды огня.
\vs Dan 3:94 И, собравшись, сатрапы, наместники, военачальники и советники царя усмотрели, что над телами мужей сих огонь не имел силы, и волосы на голове не опалены, и одежды их не изменились, и даже запаха огня не было от них.
\vs Dan 3:95 Тогда Навуходоносор сказал: благословен Бог Седраха, Мисаха и Авденаго, Который послал Ангела Своего и избавил рабов Своих, которые надеялись на Него и не послушались царского повеления, и предали тела свои [огню], чтобы не служить и не поклоняться иному богу, кроме Бога своего!
\vs Dan 3:96 И от меня дается повеление, чтобы из всякого народа, племени и языка кто произнесет хулу на Бога Седраха, Мисаха и Авденаго, был изрублен в куски, и дом его обращен в развалины, ибо нет иного бога, который мог бы так спасать.
\vs Dan 3:97 Тогда царь возвысил Седраха, Мисаха и Авденаго в стране Вавилонской [и возвеличил их и удостоил их начальства над прочими Иудеями в его царстве].
\rsbpar\vs Dan 3:98 Навуходоносор царь всем народам, племенам и языкам, живущим по всей земле: мир вам да умножится!
\vs Dan 3:99 Знамения и чудеса, какие совершил надо мною Всевышний Бог, угодно мне возвестить вам.
\vs Dan 3:100 Как велики знамения Его и как могущественны чудеса Его! Царство Его~--- царство вечное, и владычество Его~--- в роды и роды.
\vs Dan 4:1 Я, Навуходоносор, спокоен был в доме моем и благоденствовал в чертогах моих.
\vs Dan 4:2 Но я видел сон, который устрашил меня, и размышления на ложе моем и видения головы моей смутили меня.
\vs Dan 4:3 И дано было мною повеление привести ко мне всех мудрецов Вавилонских, чтобы они сказали мне значение сна.
\vs Dan 4:4 Тогда пришли тайноведцы, обаятели, Халдеи и гадатели; я рассказал им сон, но они не могли мне объяснить значения его.
\vs Dan 4:5 Наконец вошел ко мне Даниил, которому имя было Валтасар, по имени бога моего, и в котором дух святаго Бога; ему рассказал я сон.
\vs Dan 4:6 Валтасар, глава мудрецов! я знаю, что в тебе дух святаго Бога, и никакая тайна не затрудняет тебя; объясни мне видения сна моего, который я видел, и значение его.
\vs Dan 4:7 Видения же головы моей на ложе моем были такие: я видел, вот, среди земли дерево весьма высокое.
\vs Dan 4:8 Большое было это дерево и крепкое, и высота его достигала до неба, и оно видимо было до краев всей земли.
\vs Dan 4:9 Листья его прекрасные, и плодов на нем множество, и пища на нем для всех; под ним находили тень полевые звери, и в ветвях его гнездились птицы небесные, и от него питалась всякая плоть.
\vs Dan 4:10 И видел я в видениях головы моей на ложе моем, и вот, нисшел с небес Бодрствующий и Святый.
\vs Dan 4:11 Воскликнув громко, Он сказал: <<срубите это дерево, обрубите ветви его, стрясите листья с него и разбросайте плоды его; пусть удалятся звери из-под него и птицы с ветвей его;
\vs Dan 4:12 но главный корень его оставьте в земле, и пусть он в узах железных и медных среди полевой травы орошается небесною росою, и с животными пусть будет часть его в траве земной.
\vs Dan 4:13 Сердце человеческое отнимется от него и дастся ему сердце звериное, и пройдут над ним семь времен.
\vs Dan 4:14 Повелением Бодрствующих это определено, и по приговору Святых назначено, дабы знали живущие, что Всевышний владычествует над царством человеческим, и дает его, кому хочет, и поставляет над ним уничиженного между людьми>>.
\vs Dan 4:15 Такой сон видел я, царь Навуходоносор; а ты, Валтасар, скажи значение его, так как никто из мудрецов в моем царстве не мог объяснить его значения, а ты можешь, потому что дух святаго Бога в тебе.
\rsbpar\vs Dan 4:16 Тогда Даниил, которому имя Валтасар, около часа пробыл в изумлении, и мысли его смущали его. Царь начал говорить и сказал: Валтасар! да не смущает тебя этот сон и значение его. Валтасар отвечал и сказал: господин мой! твоим бы ненавистникам этот сон, и врагам твоим значение его!
\vs Dan 4:17 Дерево, которое ты видел, которое было большое и крепкое, высотою своею достигало до небес и видимо было по всей земле,
\vs Dan 4:18 на котором листья были прекрасные и множество плодов и пропитание для всех, под которым обитали звери полевые и в ветвях которого гнездились птицы небесные,
\vs Dan 4:19 это ты, царь, возвеличившийся и укрепившийся, и величие твое возросло и достигло до небес, и власть твоя~--- до краев земли.
\vs Dan 4:20 А что царь видел Бодрствующего и Святаго, сходящего с небес, Который сказал: <<срубите дерево и истребите его, только главный корень его оставьте в земле, и пусть он в узах железных и медных, среди полевой травы, орошается росою небесною, и с полевыми зверями пусть будет часть его, доколе не пройдут над ним семь времен>>,~---
\vs Dan 4:21 то вот значение этого, царь, и вот определение Всевышнего, которое постигнет господина моего, царя:
\vs Dan 4:22 тебя отлучат от людей, и обитание твое будет с полевыми зверями; травою будут кормить тебя, как вола, росою небесною ты будешь орошаем, и семь времен пройдут над тобою, доколе познаешь, что Всевышний владычествует над царством человеческим и дает его, кому хочет.
\vs Dan 4:23 А что повелено было оставить главный корень дерева, это значит, что царство твое останется при тебе, когда ты познаешь власть небесную.
\vs Dan 4:24 Посему, царь, да будет благоугоден тебе совет мой: искупи грехи твои правдою и беззакония твои милосердием к бедным; вот чем может продлиться мир твой.
\vs Dan 4:25 Все это сбылось над царем Навуходоносором.
\rsbpar\vs Dan 4:26 По прошествии двенадцати месяцев, расхаживая по царским чертогам в Вавилоне,
\vs Dan 4:27 царь сказал: это ли не величественный Вавилон, который построил я в дом царства силою моего могущества и в славу моего величия!
\vs Dan 4:28 Еще речь сия была в устах царя, как был с неба голос: <<тебе говорят, царь Навуходоносор: царство отошло от тебя!
\vs Dan 4:29 И отлучат тебя от людей, и будет обитание твое с полевыми зверями; травою будут кормить тебя, как вола, и семь времен пройдут над тобою, доколе познаешь, что Всевышний владычествует над царством человеческим и дает его, кому хочет!>>
\vs Dan 4:30 Тотчас и исполнилось это слово над Навуходоносором, и отлучен он был от людей, ел траву, как вол, и орошалось тело его росою небесною, так что волосы у него выросли как у льва, и ногти у него~--- как у птицы.
\vs Dan 4:31 По окончании же дней тех, я, Навуходоносор, возвел глаза мои к небу, и разум мой возвратился ко мне; и благословил я Всевышнего, восхвалил и прославил Присносущего, Которого владычество~--- владычество вечное, и Которого царство~--- в роды и роды.
\vs Dan 4:32 И все, живущие на земле, ничего не значат; по воле Своей Он действует как в небесном воинстве, так и у живущих на земле; и нет никого, кто мог бы противиться руке Его и сказать Ему: <<что Ты сделал?>>
\vs Dan 4:33 В то время возвратился ко мне разум мой, и к славе царства моего возвратились ко мне сановитость и прежний вид мой; тогда взыскали меня советники мои и вельможи мои, и я восстановлен на царство мое, и величие мое еще более возвысилось.
\vs Dan 4:34 Ныне я, Навуходоносор, славлю, превозношу и величаю Царя Небесного, Которого все дела истинны и пути праведны, и Который силен смирить ходящих гордо.
\vs Dan 5:1 Валтасар царь сделал большое пиршество для тысячи вельмож своих и перед глазами тысячи пил вино.
\vs Dan 5:2 Вкусив вина, Валтасар приказал принести золотые и серебряные сосуды, которые Навуходоносор, отец его, вынес из храма Иерусалимского, чтобы пить из них царю, вельможам его, женам его и наложницам его.
\vs Dan 5:3 Тогда принесли золотые сосуды, которые взяты были из святилища дома Божия в Иерусалиме; и пили из них царь и вельможи его, жены его и наложницы его.
\vs Dan 5:4 Пили вино, и славили богов золотых и серебряных, медных, железных, деревянных и каменных.
\vs Dan 5:5 В тот самый час вышли персты руки человеческой и писали против лампады на извести стены чертога царского, и царь видел кисть руки, которая писала.
\vs Dan 5:6 Тогда царь изменился в лице своем; мысли его смутили его, связи чресл его ослабели, и колени его стали биться одно о другое.
\vs Dan 5:7 Сильно закричал царь, чтобы привели обаятелей, Халдеев и гадателей. Царь начал говорить, и сказал мудрецам Вавилонским: кто прочитает это написанное и объяснит мне значение его, тот будет облечен в багряницу, и золотая цепь будет на шее у него, и третьим властелином будет в царстве.
\vs Dan 5:8 И вошли все мудрецы царя, но не могли прочитать написанного и объяснить царю значения его.
\vs Dan 5:9 Царь Валтасар чрезвычайно встревожился, и вид лица его изменился на нем, и вельможи его смутились.
\vs Dan 5:10 Царица же, по поводу слов царя и вельмож его, вошла в палату пиршества; начала говорить царица и сказала: царь, вовеки живи! да не смущают тебя мысли твои, и да не изменяется вид лица твоего!
\vs Dan 5:11 Есть в царстве твоем муж, в котором дух святаго Бога; во дни отца твоего найдены были в нем свет, разум и мудрость, подобная мудрости богов, и царь Навуходоносор, отец твой, поставил его главою тайноведцев, обаятелей, Халдеев и гадателей,~--- сам отец твой, царь,
\vs Dan 5:12 потому что в нем, в Данииле, которого царь переименовал Валтасаром, оказались высокий дух, ведение и разум, способный изъяснять сны, толковать загадочное и разрешать узлы. Итак пусть призовут Даниила и он объяснит значение.
\vs Dan 5:13 Тогда введен был Даниил пред царя, и царь начал речь и сказал Даниилу: ты ли Даниил, один из пленных сынов Иудейских, которых отец мой, царь, привел из Иудеи?
\vs Dan 5:14 Я слышал о тебе, что дух Божий в тебе и свет, и разум, и высокая мудрость найдена в тебе.
\vs Dan 5:15 Вот, приведены были ко мне мудрецы и обаятели, чтобы прочитать это написанное и объяснить мне значение его; но они не могли объяснить мне этого.
\vs Dan 5:16 А о тебе я слышал, что ты можешь объяснять значение и разрешать узлы; итак, если можешь прочитать это написанное и объяснить мне значение его, то облечен будешь в багряницу, и золотая цепь будет на шее твоей, и третьим властелином будешь в царстве.
\vs Dan 5:17 Тогда отвечал Даниил, и сказал царю: дары твои пусть останутся у тебя, и почести отдай другому; а написанное я прочитаю царю и значение объясню ему.
\rsbpar\vs Dan 5:18 Царь! Всевышний Бог даровал отцу твоему Навуходоносору царство, величие, честь и славу.
\vs Dan 5:19 Пред величием, которое Он дал ему, все народы, племена и языки трепетали и страшились его: кого хотел, он убивал, и кого хотел, оставлял в живых; кого хотел, возвышал, и кого хотел, унижал.
\vs Dan 5:20 Но когда сердце его надмилось и дух его ожесточился до дерзости, он был свержен с царского престола своего и лишен славы своей,
\vs Dan 5:21 и отлучен был от сынов человеческих, и сердце его уподобилось звериному, и жил он с дикими ослами; кормили его травою, как вола, и тело его орошаемо было небесною росою, доколе он познал, что над царством человеческим владычествует Всевышний Бог и поставляет над ним, кого хочет.
\vs Dan 5:22 И ты, сын его Валтасар, не смирил сердца твоего, хотя знал все это,
\vs Dan 5:23 но вознесся против Господа небес, и сосуды дома Его принесли к тебе, и ты и вельможи твои, жены твои и наложницы твои пили из них вино, и ты славил богов серебряных и золотых, медных, железных, деревянных и каменных, которые ни видят, ни слышат, ни разумеют; а Бога, в руке Которого дыхание твое и у Которого все пути твои, ты не прославил.
\vs Dan 5:24 За это и послана от Него кисть руки, и начертано это писание.
\vs Dan 5:25 И вот что начертано: мене, мене, текел, упарсин.
\vs Dan 5:26 Вот и значение слов: мене~--- исчислил Бог царство твое и положил конец ему;
\vs Dan 5:27 Текел~--- ты взвешен на весах и найден очень легким;
\vs Dan 5:28 Перес~--- разделено царство твое и дано Мидянам и Персам.
\rsbpar\vs Dan 5:29 Тогда по повелению Валтасара облекли Даниила в багряницу и возложили золотую цепь на шею его, и провозгласили его третьим властелином в царстве.
\vs Dan 5:30 В ту же самую ночь Валтасар, царь Халдейский, был убит,
\vs Dan 5:31 и Дарий Мидянин принял царство, будучи шестидесяти двух лет.
\vs Dan 6:1 Угодно было Дарию поставить над царством сто двадцать сатрапов, чтобы они были во всем царстве,
\vs Dan 6:2 а над ними трех князей,~--- из которых один был Даниил,~--- чтобы сатрапы давали им отчет и чтобы царю не было никакого обременения.
\vs Dan 6:3 Даниил превосходил прочих князей и сатрапов, потому что в нем был высокий дух, и царь помышлял уже поставить его над всем царством.
\vs Dan 6:4 Тогда князья и сатрапы начали искать предлога к обвинению Даниила по управлению царством; но никакого предлога и погрешностей не могли найти, потому что он был верен, и никакой погрешности или вины не оказывалось в нем.
\vs Dan 6:5 И эти люди сказали: не найти нам предлога против Даниила, если мы не найдем его против него в законе Бога его.
\vs Dan 6:6 Тогда эти князья и сатрапы приступили к царю и так сказали ему: царь Дарий! вовеки живи!
\vs Dan 6:7 Все князья царства, наместники, сатрапы, советники и военачальники согласились между собою, чтобы сделано было царское постановление и издано повеление, чтобы, кто в течение тридцати дней будет просить какого-либо бога или человека, кроме тебя, царь, того бросить в львиный ров.
\vs Dan 6:8 Итак утверди, царь, это определение и подпиши указ, чтобы он был неизменен, как закон Мидийский и Персидский, и чтобы он не был нарушен.
\vs Dan 6:9 Царь Дарий подписал указ и это повеление.
\vs Dan 6:10 Даниил же, узнав, что подписан такой указ, пошел в дом свой; окна же в горнице его были открыты против Иерусалима, и он три раза в день преклонял колени, и молился своему Богу, и славословил Его, как это делал он и прежде того.
\vs Dan 6:11 Тогда эти люди подсмотрели и нашли Даниила молящегося и просящего милости пред Богом своим,
\vs Dan 6:12 потом пришли и сказали царю о царском повелении: не ты ли подписал указ, чтобы всякого человека, который в течение тридцати дней будет просить какого-либо бога или человека, кроме тебя, царь, бросать в львиный ров? Царь отвечал и сказал: это слово твердо, как закон Мидян и Персов, не допускающий изменения.
\vs Dan 6:13 Тогда отвечали они и сказали царю, что Даниил, который из пленных сынов Иудеи, не обращает внимания ни на тебя, царь, ни на указ, тобою подписанный, но три раза в день молится своими молитвами.
\vs Dan 6:14 Царь, услышав это, сильно опечалился и положил в сердце своем спасти Даниила, и даже до захождения солнца усиленно старался избавить его.
\vs Dan 6:15 Но те люди приступили к царю и сказали ему: знай, царь, что по закону Мидян и Персов никакое определение или постановление, утвержденное царем, не может быть изменено.
\vs Dan 6:16 Тогда царь повелел, и привели Даниила, и бросили в ров львиный; при этом царь сказал Даниилу: Бог твой, Которому ты неизменно служишь, Он спасет тебя!
\vs Dan 6:17 И принесен был камень и положен на отверстие рва, и царь запечатал его перстнем своим, и перстнем вельмож своих, чтобы ничто не переменилось в распоряжении о Данииле.
\vs Dan 6:18 Затем царь пошел в свой дворец, лег спать без ужина, и даже не велел вносить к нему пищи, и сон бежал от него.
\vs Dan 6:19 Поутру же царь встал на рассвете и поспешно пошел ко рву львиному,
\vs Dan 6:20 и, подойдя ко рву, жалобным голосом кликнул Даниила, и сказал царь Даниилу: Даниил, раб Бога живаго! Бог твой, Которому ты неизменно служишь, мог ли спасти тебя от львов?
\vs Dan 6:21 Тогда Даниил сказал царю: царь! вовеки живи!
\vs Dan 6:22 Бог мой послал Ангела Своего и заградил пасть львам, и они не повредили мне, потому что я оказался пред Ним чист, да и перед тобою, царь, я не сделал преступления.
\vs Dan 6:23 Тогда царь чрезвычайно возрадовался о нем и повелел поднять Даниила изо рва; и поднят был Даниил изо рва, и никакого повреждения не оказалось на нем, потому что он веровал в Бога своего.
\vs Dan 6:24 И приказал царь, и приведены были те люди, которые обвиняли Даниила, и брошены в львиный ров, как они сами, так и дети их и жены их; и они не достигли до дна рва, как львы овладели ими и сокрушили все кости их.
\rsbpar\vs Dan 6:25 После того царь Дарий написал всем народам, племенам и языкам, живущим по всей земле: <<Мир вам да умножится!
\vs Dan 6:26 Мною дается повеление, чтобы во всякой области царства моего трепетали и благоговели пред Богом Данииловым, потому что Он есть Бог живый и присносущий, и царство Его несокрушимо, и владычество Его бесконечно.
\vs Dan 6:27 Он избавляет и спасает, и совершает чудеса и знамения на небе и на земле; Он избавил Даниила от силы львов>>.
\vs Dan 6:28 И Даниил благоуспевал и в царствование Дария, и в царствование Кира Персидского.
\vs Dan 7:1 В первый год Валтасара, царя Вавилонского, Даниил видел сон и пророческие видения головы своей на ложе своем. Тогда он записал этот сон, изложив сущность дела.
\vs Dan 7:2 Начав речь, Даниил сказал: видел я в ночном видении моем, и вот, четыре ветра небесных боролись на великом море,
\vs Dan 7:3 и четыре больших зверя вышли из моря, непохожие один на другого.
\vs Dan 7:4 Первый~--- как лев, но у него крылья орлиные; я смотрел, доколе не вырваны были у него крылья, и он поднят был от земли, и стал на ноги, как человек, и сердце человеческое дано ему.
\vs Dan 7:5 И вот еще зверь, второй, похожий на медведя, стоял с одной стороны, и три клыка во рту у него, между зубами его; ему сказано так: <<встань, ешь мяса много!>>
\vs Dan 7:6 Затем видел я, вот еще зверь, как барс; на спине у него четыре птичьих крыла, и четыре головы были у зверя сего, и власть дана была ему.
\vs Dan 7:7 После сего видел я в ночных видениях, и вот зверь четвертый, страшный и ужасный и весьма сильный; у него большие железные зубы; он пожирает и сокрушает, остатки же попирает ногами; он отличен был от всех прежних зверей, и десять рогов было у него.
\vs Dan 7:8 Я смотрел на эти рога, и вот, вышел между ними еще небольшой рог, и три из прежних рогов с корнем исторгнуты были перед ним, и вот, в этом роге были глаза, как глаза человеческие, и уста, говорящие высокомерно.
\vs Dan 7:9 Видел я, наконец, что поставлены были престолы, и воссел Ветхий днями; одеяние на Нем было бело, как снег, и волосы главы Его~--- как чистая в\acc{о}лна; престол Его~--- как пламя огня, колеса Его~--- пылающий огонь.
\vs Dan 7:10 Огненная река выходила и проходила пред Ним; тысячи тысяч служили Ему и тьмы тем предстояли пред Ним; судьи сели, и раскрылись книги.
\vs Dan 7:11 Видел я тогда, что за изречение высокомерных слов, какие говорил рог, зверь был убит в глазах моих, и тело его сокрушено и предано на сожжение огню.
\vs Dan 7:12 И у прочих зверей отнята власть их, и продолжение жизни дано им только на время и на срок.
\vs Dan 7:13 Видел я в ночных видениях, вот, с облаками небесными шел как бы Сын человеческий, дошел до Ветхого днями и подведен был к Нему.
\vs Dan 7:14 И Ему дана власть, слава и царство, чтобы все народы, племена и языки служили Ему; владычество Его~--- владычество вечное, которое не прейдет, и царство Его не разрушится.
\vs Dan 7:15 Вострепетал дух мой во мне, Данииле, в теле моем, и видения головы моей смутили меня.
\vs Dan 7:16 Я подошел к одному из предстоящих и спросил у него об истинном значении всего этого, и он стал говорить со мною, и объяснил мне смысл сказанного:
\vs Dan 7:17 <<эти большие звери, которых четыре, \bibemph{означают}, что четыре царя восстанут от земли.
\vs Dan 7:18 Потом примут царство святые Всевышнего и будут владеть царством вовек и во веки веков>>.
\vs Dan 7:19 Тогда пожелал я точного объяснения о четвертом звере, который был отличен от всех и очень страшен, с зубами железными и когтями медными, пожирал и сокрушал, а остатки попирал ногами,
\vs Dan 7:20 и о десяти рогах, которые были на голове у него, и о другом, вновь вышедшем, перед которым выпали три, о том самом роге, у которого были глаза и уста, говорящие высокомерно, и который по виду стал больше прочих.
\vs Dan 7:21 Я видел, как этот рог вел брань со святыми и превозмогал их,
\vs Dan 7:22 доколе не пришел Ветхий днями, и суд дан был святым Всевышнего, и наступило время, чтобы царством овладели святые.
\vs Dan 7:23 Об этом он сказал: зверь четвертый~--- четвертое царство будет на земле, отличное от всех царств, которое будет пожирать всю землю, попирать и сокрушать ее.
\vs Dan 7:24 А десять рогов значат, что из этого царства восстанут десять царей, и после них восстанет иной, отличный от прежних, и уничижит трех царей,
\vs Dan 7:25 и против Всевышнего будет произносить слова и угнетать святых Всевышнего; даже возмечтает отменить у них \bibemph{праздничные} времена и закон, и они преданы будут в руку его до времени и времен и полувремени.
\vs Dan 7:26 Затем воссядут судьи и отнимут у него власть губить и истреблять до конца.
\vs Dan 7:27 Царство же и власть и величие царственное во всей поднебесной дано будет народу святых Всевышнего, Которого царство~--- царство вечное, и все властители будут служить и повиноваться Ему.
\vs Dan 7:28 Здесь конец слова. Меня, Даниила, сильно смущали размышления мои, и лице мое изменилось на мне; но слово я сохранил в сердце моем.
\vs Dan 8:1 В третий год царствования Валтасара царя явилось мне, Даниилу, видение после того, которое явилось мне прежде.
\vs Dan 8:2 И видел я в видении, и когда видел, я был в Сузах, престольном городе в области Еламской, и видел я в видении,~--- как бы я был у реки Улая.
\vs Dan 8:3 Поднял я глаза мои и увидел: вот, один овен стоит у реки; у него два рога, и рога высокие, но один выше другого, и высший поднялся после.
\vs Dan 8:4 Видел я, как этот овен бодал к западу и к северу и к югу, и никакой зверь не мог устоять против него, и никто не мог спасти от него; он делал, что хотел, и величался.
\vs Dan 8:5 Я внимательно смотрел на это, и вот, с запада шел козел по лицу всей земли, не касаясь земли; у этого козла был видный рог между его глазами.
\vs Dan 8:6 Он пошел на того овна, имеющего рога, которого я видел стоящим у реки, и бросился на него в сильной ярости своей.
\vs Dan 8:7 И я видел, как он, приблизившись к овну, рассвирепел на него и поразил овна, и сломил у него оба рога; и недостало силы у овна устоять против него, и он поверг его на землю и растоптал его, и не было никого, кто мог бы спасти овна от него.
\vs Dan 8:8 Тогда козел чрезвычайно возвеличился; но когда он усилился, то сломился большой рог, и на место его вышли четыре, обращенные на четыре ветра небесных.
\vs Dan 8:9 От одного из них вышел небольшой рог, который чрезвычайно разросся к югу и к востоку и к прекрасной стране,
\vs Dan 8:10 и вознесся до воинства небесного, и низринул на землю часть сего воинства и звезд, и попрал их,
\vs Dan 8:11 и даже вознесся на Вождя воинства сего, и отнята была у Него ежедневная жертва, и поругано было место святыни Его.
\vs Dan 8:12 И воинство предано вместе с ежедневною жертвою за нечестие, и он, повергая истину на землю, действовал и успевал.
\vs Dan 8:13 И услышал я одного святого говорящего, и сказал этот святой кому-то, вопрошавшему: <<на сколько времени простирается это видение о ежедневной жертве и об опустошительном нечестии, когда святыня и воинство будут попираемы?>>
\vs Dan 8:14 И сказал мне: <<на две тысячи триста вечеров и утр; и тогда святилище очистится>>.
\vs Dan 8:15 И было: когда я, Даниил, увидел это видение и искал значения его, вот, стал предо мною как облик мужа.
\vs Dan 8:16 И услышал я от средины Улая голос человеческий, который воззвал и сказал: <<Гавриил! объясни ему это видение!>>
\vs Dan 8:17 И он подошел к тому месту, где я стоял, и когда он пришел, я ужаснулся и пал на лице мое; и сказал он мне: <<знай, сын человеческий, что видение относится к концу времени!>>
\vs Dan 8:18 И когда он говорил со мною, я без чувств лежал лицем моим на земле; но он прикоснулся ко мне и поставил меня на место мое,
\vs Dan 8:19 и сказал: <<вот, я открываю тебе, чт\acc{о} будет в последние дни гнева; ибо это относится к концу определенного времени.
\vs Dan 8:20 Овен, которого ты видел с двумя рогами, это цари Мидийский и Персидский.
\vs Dan 8:21 А козел косматый~--- царь Греции, а большой рог, который между глазами его, это первый ее царь;
\vs Dan 8:22 он сломился, и вместо него вышли другие четыре: это~--- четыре царства восстанут из этого народа, но не с его силою.
\vs Dan 8:23 Под конец же царства их, когда отступники исполнят меру беззаконий своих, восстанет царь наглый и искусный в коварстве;
\vs Dan 8:24 и укрепится сила его, хотя и не его силою, и он будет производить удивительные опустошения и успевать и действовать и губить сильных и народ святых,
\vs Dan 8:25 и при уме его и коварство будет иметь успех в руке его, и сердцем своим он превознесется, и среди мира погубит многих, и против Владыки владык восстанет, но будет сокрушен~--- не рукою.
\vs Dan 8:26 Видение же о вечере и утре, о котором сказано, истинно; но ты сокрой это видение, ибо оно относится к отдаленным временам>>.
\rsbpar\vs Dan 8:27 И я, Даниил, изнемог, и болел несколько дней; потом встал и начал заниматься царскими делами; я изумлен был видением сим и не понимал его.
\vs Dan 9:1 В первый год Дария, сына Ассуирова, из рода Мидийского, который поставлен был царем над царством Халдейским,
\vs Dan 9:2 в первый год царствования его я, Даниил, сообразил по книгам число лет, о котором было слово Господне к Иеремии пророку, что семьдесят лет исполнятся над опустошением Иерусалима.
\vs Dan 9:3 И обратил я лице мое к Господу Богу с молитвою и молением, в посте и вретище и пепле.
\vs Dan 9:4 И молился я Господу Богу моему, и исповедовался и сказал: <<Молю Тебя, Господи Боже великий и дивный, хранящий завет и милость любящим Тебя и соблюдающим повеления Твои!
\vs Dan 9:5 Согрешили мы, поступали беззаконно, действовали нечестиво, упорствовали и отступили от заповедей Твоих и от постановлений Твоих;
\vs Dan 9:6 и не слушали рабов Твоих, пророков, которые Твоим именем говорили царям нашим, и вельможам нашим, и отцам нашим, и всему народу страны.
\vs Dan 9:7 У Тебя, Господи, правда, а у нас на лицах стыд, как день сей, у каждого Иудея, у жителей Иерусалима и у всего Израиля, у ближних и дальних, во всех странах, куда Ты изгнал их за отступление их, с каким они отступили от Тебя.
\vs Dan 9:8 Господи! у нас на лицах стыд, у царей наших, у князей наших и у отцов наших, потому что мы согрешили пред Тобою.
\vs Dan 9:9 А у Господа Бога нашего милосердие и прощение, ибо мы возмутились против Него
\vs Dan 9:10 и не слушали гласа Господа Бога нашего, чтобы поступать по законам Его, которые Он дал нам через рабов Своих, пророков.
\vs Dan 9:11 И весь Израиль преступил закон Твой и отвратился, чтобы не слушать гласа Твоего; и за то излились на нас проклятие и клятва, которые написаны в законе Моисея, раба Божия: ибо мы согрешили пред Ним.
\vs Dan 9:12 И Он исполнил слова Свои, которые изрек на нас и на судей наших, судивших нас, наведя на нас великое бедствие, какого не бывало под небесами и какое совершилось над Иерусалимом.
\vs Dan 9:13 Как написано в законе Моисея, так все это бедствие постигло нас; но мы не умоляли Господа Бога нашего, чтобы нам обратиться от беззаконий наших и уразуметь истину Твою.
\vs Dan 9:14 Наблюдал Господь это бедствие и навел его на нас: ибо праведен Господь Бог наш во всех делах Своих, которые совершает, но мы не слушали гласа Его.
\vs Dan 9:15 И ныне, Господи Боже наш, изведший народ Твой из земли Египетской рукою сильною и явивший славу Твою, как день сей! согрешили мы, поступали нечестиво.
\vs Dan 9:16 Господи! по всей правде Твоей да отвратится гнев Твой и негодование Твое от града Твоего, Иерусалима, от святой горы Твоей; ибо за грехи наши и беззакония отцов наших Иерусалим и народ Твой в поругании у всех, окружающих нас.
\vs Dan 9:17 И ныне услыши, Боже наш, молитву раба Твоего и моление его и воззри светлым лицем Твоим на опустошенное святилище Твое, ради Тебя, Господи.
\vs Dan 9:18 Приклони, Боже мой, ухо Твое и услыши, открой очи Твои и воззри на опустошения наши и на город, на котором наречено имя Твое; ибо мы повергаем моления наши пред Тобою, уповая не на праведность нашу, но на Твое великое милосердие.
\vs Dan 9:19 Господи! услыши; Господи! прости; Господи! внемли и соверши, не умедли ради Тебя Самого, Боже мой, ибо Твое имя наречено на городе Твоем и на народе Твоем>>.
\vs Dan 9:20 И когда я еще говорил и молился, и исповедовал грехи мои и грехи народа моего, Израиля, и повергал мольбу мою пред Господом Богом моим о святой горе Бога моего;
\vs Dan 9:21 когда я еще продолжал молитву, муж Гавриил, которого я видел прежде в видении, быстро прилетев, коснулся меня около времени вечерней жертвы
\vs Dan 9:22 и вразумлял меня, говорил со мною и сказал: <<Даниил! теперь я исшел, чтобы научить тебя разумению.
\vs Dan 9:23 В начале моления твоего вышло слово, и я пришел возвестить \bibemph{его тебе}, ибо ты муж желаний; итак вникни в слово и уразумей видение.
\vs Dan 9:24 Семьдесят седмин определены для народа твоего и святаго города твоего, чтобы покрыто было преступление, запечатаны были грехи и заглажены беззакония, и чтобы приведена была правда вечная, и запечатаны были видение и пророк, и помазан был Святый святых.
\vs Dan 9:25 Итак знай и разумей: с того времени, как выйдет повеление о восстановлении Иерусалима, до Христа Владыки семь седмин и шестьдесят две седмины; и возвратится \bibemph{народ} и обстроятся улицы и стены, но в трудные времена.
\vs Dan 9:26 И по истечении шестидесяти двух седмин предан будет смерти Христос, и не будет; а город и святилище разрушены будут народом вождя, который придет, и конец его будет как от наводнения, и до конца войны будут опустошения.
\vs Dan 9:27 И утвердит завет для многих одна седмина, а в половине седмины прекратится жертва и приношение, и на крыле \bibemph{святилища} будет мерзость запустения, и окончательная предопределенная гибель постигнет опустошителя>>.
\vs Dan 10:1 В третий год Кира, царя Персидского, было откровение Даниилу, который назывался именем Валтасара; и истинно было это откровение и великой силы. Он понял это откровение и уразумел это видение.
\vs Dan 10:2 В эти дни я, Даниил, был в сетовании три седмицы дней.
\vs Dan 10:3 Вкусного хлеба я не ел; мясо и вино не входило в уста мои, и мастями я не умащал себя до исполнения трех седмиц дней.
\vs Dan 10:4 А в двадцать четвертый день первого месяца был я на берегу большой реки Тигра,
\vs Dan 10:5 и поднял глаза мои, и увидел: вот один муж, облеченный в льняную одежду, и чресла его опоясаны золотом из Уфаза.
\vs Dan 10:6 Тело его~--- как топаз, лице его~--- как вид молнии; очи его~--- как горящие светильники, руки его и ноги его по виду~--- как блестящая медь, и глас речей его~--- как голос множества людей.
\vs Dan 10:7 И только один я, Даниил, видел это видение, а бывшие со мною люди не видели этого видения; но сильный страх напал на них и они убежали, чтобы скрыться.
\vs Dan 10:8 И остался я один и смотрел на это великое видение, но во мне не осталось крепости, и вид лица моего чрезвычайно изменился, не стало во мне бодрости.
\vs Dan 10:9 И услышал я глас слов его; и как только услышал глас слов его, в оцепенении пал я на лице мое и лежал лицем к земле.
\vs Dan 10:10 Но вот, коснулась меня рука и поставила меня на колени мои и на длани рук моих.
\vs Dan 10:11 И сказал он мне: <<Даниил, муж желаний! вникни в слова, которые я скажу тебе, и стань прямо на ноги твои; ибо к тебе я послан ныне>>. Когда он сказал мне эти слова, я встал с трепетом.
\vs Dan 10:12 Но он сказал мне: <<не бойся, Даниил; с первого дня, как ты расположил сердце твое, чтобы достигнуть разумения и смирить тебя пред Богом твоим, слова твои услышаны, и я пришел бы по словам твоим.
\vs Dan 10:13 Но князь царства Персидского стоял против меня двадцать один день; но вот, Михаил, один из первых князей, пришел помочь мне, и я остался там при царях Персидских.
\vs Dan 10:14 А теперь я пришел возвестить тебе, что будет с народом твоим в последние времена, так как видение относится к отдаленным дням>>.
\vs Dan 10:15 Когда он говорил мне такие слова, я припал лицем моим к земле и онемел.
\vs Dan 10:16 Но вот, некто, по виду похожий на сынов человеческих, коснулся уст моих, и я открыл уста мои, стал говорить и сказал стоящему передо мною: <<господин мой! от этого видения внутренности мои повернулись во мне, и не стало во мне силы.
\vs Dan 10:17 И как может говорить раб такого господина моего с таким господином моим? ибо во мне нет силы, и дыхание замерло во мне>>.
\vs Dan 10:18 Тогда снова прикоснулся ко мне тот человеческий облик и укрепил меня
\vs Dan 10:19 и сказал: <<не бойся, муж желаний! мир тебе; мужайся, мужайся!>> И когда он говорил со мною, я укрепился и сказал: <<говори, господин мой; ибо ты укрепил меня>>.
\vs Dan 10:20 И он сказал: <<знаешь ли, для чего я пришел к тебе? Теперь я возвращусь, чтобы бороться с князем Персидским; а когда я выйду, то вот, придет князь Греции.
\vs Dan 10:21 Впрочем я возвещу тебе, что начертано в истинном писании; и нет никого, кто поддерживал бы меня в том, кроме Михаила, князя вашего.
\vs Dan 11:1 Итак я с первого года Дария Мидянина стал ему подпорою и подкреплением.
\vs Dan 11:2 Теперь возвещу тебе истину: вот, еще три царя восстанут в Персии; потом четвертый превзойдет всех великим богатством, и когда усилится богатством своим, то поднимет всех против царства Греческого.
\vs Dan 11:3 И восстанет царь могущественный, который будет владычествовать с великою властью, и будет действовать по своей воле.
\vs Dan 11:4 Но когда он восстанет, царство его разрушится и разделится по четырем ветрам небесным, и не к его потомкам перейдет, и не с тою властью, с какою он владычествовал; ибо раздробится царство его и достанется другим, кроме этих.
\vs Dan 11:5 И усилится южный царь и один из князей его пересилит его и будет владычествовать, и велико будет владычество его.
\vs Dan 11:6 Но через несколько лет они сблизятся, и дочь южного царя придет к царю северному, чтобы установить правильные отношения между ними; но она не удержит силы в руках своих, не устоит и род ее, но преданы будут как она, так и сопровождавшие ее, и рожденный ею, и помогавшие ей в те времена.
\vs Dan 11:7 Но восстанет отрасль от корня ее, придет к войску и войдет в укрепления царя северного, и будет действовать в них, и усилится.
\vs Dan 11:8 Даже и богов их, истуканы их с драгоценными сосудами их, серебряными и золотыми, увезет в плен в Египет и на несколько лет будет стоять выше царя северного.
\vs Dan 11:9 Хотя этот и сделает нашествие на царство южного царя, но возвратится в свою землю.
\vs Dan 11:10 Потом вооружатся сыновья его и соберут многочисленное войско, и один из них быстро пойдет, наводнит и пройдет, и потом, возвращаясь, будет сражаться с ним до укреплений его.
\vs Dan 11:11 И раздражится южный царь, и выступит, сразится с ним, с царем северным, и выставит большое войско, и предано будет войско в руки его.
\vs Dan 11:12 И ободрится войско, и сердце \bibemph{царя} вознесется; он низложит многие тысячи, но от этого не будет сильнее.
\vs Dan 11:13 Ибо царь северный возвратится и выставит войско больше прежнего, и через несколько лет быстро придет с огромным войском и большим богатством.
\vs Dan 11:14 В те времена многие восстанут против южного царя, и мятежные из сынов твоего народа поднимутся, чтобы исполнилось видение, и падут.
\vs Dan 11:15 И придет царь северный, устроит вал и овладеет укрепленным городом, и не устоят мышцы юга, ни отборное войско его; недостанет силы противостоять.
\vs Dan 11:16 И кто выйдет к нему, будет действовать по воле его, и никто не устоит перед ним; и на славной земле поставит стан свой, и она пострадает от руки его.
\vs Dan 11:17 И вознамерится войти со всеми силами царства своего, и праведные с ним, и совершит это; и дочь жен отдаст ему, на погибель ее, но этот замысел не состоится, и ему не будет пользы из того.
\vs Dan 11:18 Потом обратит лице свое к островам и овладеет многими; но некий вождь прекратит нанесенный им позор и даже свой позор обратит на него.
\vs Dan 11:19 Затем он обратит лице свое на крепости своей земли; но споткнется, падет и не станет его.
\vs Dan 11:20 На место его восстанет некий, который пошлет сборщика податей, пройти по царству славы; но и он после немногих дней погибнет, и не от возмущения и не в сражении.
\vs Dan 11:21 И восстанет на место его презренный, и не воздадут ему царских почестей, но он придет без шума и лестью овладеет царством.
\vs Dan 11:22 И всепотопляющие полчища будут потоплены и сокрушены им, даже и сам вождь завета.
\vs Dan 11:23 Ибо после того, как он вступит в союз с ним, он будет действовать обманом, и взойдет, и одержит верх с малым народом.
\vs Dan 11:24 Он войдет в мирные и плодоносные страны, и совершит то, чего не делали отцы его и отцы отцов его; добычу, награбленное имущество и богатство будет расточать своим и на крепости будет иметь замыслы свои, но только до времени.
\vs Dan 11:25 Потом возбудит силы свои и дух свой с многочисленным войском против царя южного, и южный царь выступит на войну с великим и еще более сильным войском, но не устоит, потому что будет против него коварство.
\vs Dan 11:26 Даже участники трапезы его погубят его, и войско его разольется, и падет много убитых.
\vs Dan 11:27 У обоих царей сих на сердце будет коварство, и за одним столом будут говорить ложь, но успеха не будет, потому что конец еще отложен до времени.
\vs Dan 11:28 И отправится он в землю свою с великим богатством и враждебным намерением против святаго завета, и он исполнит его, и возвратится в свою землю.
\vs Dan 11:29 В назначенное время опять пойдет он на юг; но последний \bibemph{поход} не такой будет, как прежний,
\vs Dan 11:30 ибо в одно время с ним придут корабли Киттимские; и он упадет духом, и возвратится, и озлобится на святый завет, и исполнит свое намерение, и опять войдет в соглашение с отступниками от святаго завета.
\vs Dan 11:31 И поставлена будет им часть войска, которая осквернит святилище могущества, и прекратит ежедневную жертву, и поставит мерзость запустения.
\vs Dan 11:32 Поступающих нечестиво против завета он привлечет к себе лестью; но люди, чтущие своего Бога, усилятся и будут действовать.
\vs Dan 11:33 И разумные из народа вразумят многих, хотя будут несколько времени страдать от меча и огня, от плена и грабежа;
\vs Dan 11:34 и во время страдания своего будут иметь некоторую помощь, и многие присоединятся к ним, но притворно.
\vs Dan 11:35 Пострадают некоторые и из разумных для испытания их, очищения и для убеления к последнему времени; ибо есть еще время до срока.
\vs Dan 11:36 И будет поступать царь тот по своему произволу, и вознесется и возвеличится выше всякого божества, и о Боге богов станет говорить хульное и будет иметь успех, доколе не совершится гнев: ибо, что предопределено, то исполнится.
\vs Dan 11:37 И о богах отцов своих он не помыслит, и ни желания жен, ни даже божества никакого не уважит; ибо возвеличит себя выше всех.
\vs Dan 11:38 Но богу крепостей на месте его будет он воздавать честь, и этого бога, которого не знали отцы его, он будет чествовать золотом и серебром, и дорогими камнями, и разными драгоценностями,
\vs Dan 11:39 и устроит твердую крепость с чужим богом: которые призн\acc{а}ют его, тем увеличит почести и даст власть над многими, и землю раздаст в награду.
\vs Dan 11:40 Под конец же времени сразится с ним царь южный, и царь северный устремится как буря на него с колесницами, всадниками и многочисленными кораблями, и нападет на области, наводнит их, и пройдет через них.
\vs Dan 11:41 И войдет он в прекраснейшую из земель, и многие области пострадают и спасутся от руки его только Едом, Моав и большая часть сынов Аммоновых.
\vs Dan 11:42 И прострет руку свою на разные страны; не спасется и земля Египетская.
\vs Dan 11:43 И завладеет он сокровищами золота и серебра и разными драгоценностями Египта; Ливийцы и Ефиопляне последуют за ним.
\vs Dan 11:44 Но слухи с востока и севера встревожат его, и выйдет он в величайшей ярости, чтобы истреблять и губить многих,
\vs Dan 11:45 и раскинет он царские шатры свои между морем и горою преславного святилища; но придет к своему концу, и никто не поможет ему.
\vs Dan 12:1 И восстанет в то время Михаил, князь великий, стоящий за сынов народа твоего; и наступит время тяжкое, какого не бывало с тех пор, как существуют люди, до сего времени; но спасутся в это время из народа твоего все, которые найдены будут записанными в книге.
\vs Dan 12:2 И многие из спящих в прахе земли пробудятся, одни для жизни вечной, другие на вечное поругание и посрамление.
\vs Dan 12:3 И разумные будут сиять, как светила на тверди, и обратившие многих к правде~--- как звезды, вовеки, навсегда.
\vs Dan 12:4 А ты, Даниил, сокрой слова сии и запечатай книгу сию до последнего времени; многие прочитают ее, и умножится ведение>>.
\vs Dan 12:5 Тогда я, Даниил, посмотрел, и вот, стоят двое других, один на этом берегу реки, другой на том берегу реки.
\vs Dan 12:6 И \bibemph{один} сказал мужу в льняной одежде, который стоял над водами реки: <<когда будет конец этих чудных происшествий?>>
\vs Dan 12:7 И слышал я, как муж в льняной одежде, находившийся над водами реки, подняв правую и левую руку к небу, клялся Живущим вовеки, что к концу времени и времен и полувремени, и по совершенном низложении силы народа святого, все это совершится.
\vs Dan 12:8 Я слышал это, но не понял, и потому сказал: <<господин мой! что же после этого будет?>>
\vs Dan 12:9 И отвечал он: <<иди, Даниил; ибо сокрыты и запечатаны слова сии до последнего времени.
\vs Dan 12:10 Многие очистятся, убелятся и переплавлены будут \bibemph{в искушении}; нечестивые же будут поступать нечестиво, и не уразумеет сего никто из нечестивых, а мудрые уразумеют.
\vs Dan 12:11 Со времени прекращения ежедневной жертвы и поставления мерзости запустения пройдет тысяча двести девяносто дней.
\vs Dan 12:12 Блажен, кто ожидает и достигнет тысячи трехсот тридцати пяти дней.
\vs Dan 12:13 А ты иди к твоему концу и упокоишься, и восстанешь для получения твоего жребия в конце дней>>.
\vs Dan 13:1 \fns{13-я и 14-я главы переведены с греческого, потому что в еврейском тексте их нет.}В Вавилоне жил муж, по имени Иоаким.
\vs Dan 13:2 И взял он жену, по имени Сусанну, дочь Хелкия, очень красивую и богобоязненную.
\vs Dan 13:3 Родители ее были праведные и научили дочь свою закону Моисееву.
\vs Dan 13:4 Иоаким был очень богат, и был у него сад близ дома его; и сходились к нему Иудеи, потому что он был почетнейший из всех.
\vs Dan 13:5 И были поставлены два старца из народа судьями в том году, о которых Господь сказал, что беззаконие вышло из Вавилона от старейшин-судей, которые казались управляющими народом.
\vs Dan 13:6 Они постоянно бывали в доме Иоакима, и к ним приходили все, имевшие спорные дела.
\vs Dan 13:7 Когда народ уходил около полудня, Сусанна входила в сад своего мужа для прогулки.
\vs Dan 13:8 И видели ее оба старейшины всякий день приходящую и прогуливающуюся, и в них родилась похоть к ней,
\vs Dan 13:9 и извратили ум свой, и уклонили глаза свои, чтобы не смотреть на небо и не вспоминать о праведных судах.
\vs Dan 13:10 Оба они были уязвлены похотью к ней, но не открывали друг другу боли своей,
\vs Dan 13:11 потому что стыдились объявить о вожделении своем, что хотели совокупиться с нею.
\vs Dan 13:12 И они прилежно сторожили каждый день, чтобы видеть ее, и говорили друг другу:
\vs Dan 13:13 <<пойдем домой, потому что час обеда>>,~--- и, выйдя, расходились друг от друга,
\vs Dan 13:14 и, возвратившись, приходили на то же самое место, и когда допытывались друг у друга о причине того, признались в похоти своей, и тогда вместе назначили время, когда могли бы найти ее одну.
\vs Dan 13:15 И было, когда они выжидали удобного дня, Сусанна вошла, как вчера и третьего дня, с двумя только служанками и захотела мыться в саду, потому что было жарко.
\vs Dan 13:16 И не было там никого, кроме двух старейшин, которые спрятались и сторожили ее.
\vs Dan 13:17 И сказала она служанкам: принесите мне масла и мыла, и заприте двери сада, чтобы мне помыться.
\vs Dan 13:18 Они так и сделали, как она сказала: заперли двери сада и вышли боковыми дверями, чтобы принести, что приказано было им, и не видали старейшин, потому что они спрятались.
\vs Dan 13:19 И вот, когда служанки вышли, встали оба старейшины, и прибежали к ней, и сказали:
\vs Dan 13:20 Вот, двери сада заперты и никто нас не видит, и мы имеем похотение к тебе, поэтому согласись с нами и побудь с нами.
\vs Dan 13:21 Если же не так, то мы будем свидетельствовать против тебя, что с тобою был юноша, и ты поэтому отослала от себя служанок твоих.
\vs Dan 13:22 Тогда застонала Сусанна и сказала: тесно мне отовсюду; ибо, если я сделаю это, смерть мне, а если не сделаю, то не избегну от рук ваших.
\vs Dan 13:23 Лучше для меня не сделать этого и впасть в руки ваши, нежели согрешить пред Господом.
\vs Dan 13:24 И закричала Сусанна громким голосом; закричали также и оба старейшины против нее,
\vs Dan 13:25 и один побежал и отворил двери сада.
\vs Dan 13:26 Когда же находившиеся в доме услышали крик в саду, вскочили боковыми дверями, чтобы видеть, что случилось с нею.
\vs Dan 13:27 И когда старейшины сказали слова свои, слуги ее чрезвычайно были пристыжены, потому что никогда ничего такого о Сусанне говорено не было.
\vs Dan 13:28 И было на другой день, когда собрался народ к Иоакиму, мужу ее, пришли и оба старейшины, полные беззаконного умысла против Сусанны, чтобы предать ее смерти.
\vs Dan 13:29 И сказали они перед народом: пошлите за Сусанною, дочерью Хелкия, женою Иоакима. И послали.
\vs Dan 13:30 И пришла она, и родители ее, и дети ее, и все родственники ее.
\vs Dan 13:31 Сусанна была очень нежна и красива лицем,
\vs Dan 13:32 и эти беззаконники приказали открыть \bibemph{лице} ее, так как оно было закрыто, чтобы насытиться красотою ее.
\vs Dan 13:33 Родственники же и все, которые смотрели на нее, плакали.
\vs Dan 13:34 А оба старейшины, встав посреди народа, положили руки на голову ее.
\vs Dan 13:35 Она же в слезах смотрела на небо, ибо сердце ее уповало на Господа.
\vs Dan 13:36 И сказали старейшины: когда мы ходили по саду одни, вошла эта с двумя служанками и затворила двери сада, и отослала служанок;
\vs Dan 13:37 и пришел к ней юноша, который скрывался там, и лег с нею.
\vs Dan 13:38 Мы находясь в углу сада и видя такое беззаконие, побежали на них,
\vs Dan 13:39 и увидели их совокупляющимися, и того не могли удержать, потому что он был сильнее нас и, отворив двери, выскочил.
\vs Dan 13:40 Но эту мы схватили и допрашивали: кто был этот юноша? но она не хотела объявить нам. Об этом мы свидетельствуем.
\vs Dan 13:41 И поверило им собрание, как старейшинам народа и судьям, и осудили ее на смерть.
\rsbpar\vs Dan 13:42 Возопила Сусанна громким голосом и сказала: Боже вечный, ведающий сокровенное и знающий все прежде бытия его!
\vs Dan 13:43 Ты знаешь, что они ложно свидетельствовали против меня, и вот, я умираю, не сделав ничего, что эти люди злостно выдумали на меня.
\vs Dan 13:44 И услышал Господь голос ее.
\vs Dan 13:45 И когда она ведена была на смерть, возбудил Бог святой дух молодого юноши, по имени Даниила,
\vs Dan 13:46 и он закричал громким голосом: чист я от крови ее!
\vs Dan 13:47 Тогда обратился к нему весь народ и сказал: что это за слово, которое ты сказал?
\vs Dan 13:48 Тогда он, став посреди них, сказал: так ли вы неразумны, сыны Израиля, что, не исследовав и не узнав истины, осудили дочь Израиля?
\vs Dan 13:49 Возвратитесь в суд, ибо эти ложно против нее засвидетельствовали.
\vs Dan 13:50 И тотчас весь народ возвратился, и сказали ему старейшины: садись посреди нас и объяви нам, потому что Бог дал тебе старейшинство.
\vs Dan 13:51 И сказал им Даниил: отделите их друг от друга подальше, и я допрошу их.
\vs Dan 13:52 Когда же они отделены были один от другого, призвал одного из них и сказал ему: состарившийся в злых днях! ныне обнаружились грехи твои, которые ты делал прежде,
\vs Dan 13:53 производя суды неправедные, осуждая невинных и оправдывая виновных, тогда как Господь говорит: <<невинного и правого не умерщвляй>>.
\vs Dan 13:54 Итак, если ты сию видел, скажи, под каким деревом видел ты их разговаривающими друг с другом? Он сказал: под мастиковым.
\vs Dan 13:55 Даниил сказал: точно, солгал ты на твою голову; ибо вот, Ангел Божий, приняв решение от Бога, рассечет тебя пополам.
\vs Dan 13:56 Удалив его, он приказал привести другого и сказал ему: племя Ханаана, а не Иуды! красота прельстила тебя, и похоть развратила сердце твое.
\vs Dan 13:57 Так поступали вы с дочерями Израиля, и они из страха имели общение с вами; но дочь Иуды не потерпела беззакония вашего.
\vs Dan 13:58 Итак скажи мне: под каким деревом ты застал их разговаривающими между собою? Он сказал: под зеленым дубом.
\vs Dan 13:59 Даниил сказал ему: точно, солгал ты на твою голову; ибо Ангел Божий с мечом ждет, чтобы рассечь тебя пополам, чтобы истребить вас.
\vs Dan 13:60 Тогда все собрание закричало громким голосом, и благословили Бога, спасающего надеющихся на Него,
\vs Dan 13:61 и восстали на обоих старейшин, потому что Даниил их устами обличил их, что они ложно свидетельствовали;
\vs Dan 13:62 и поступили с ними так, как они злоумыслили против ближнего, по закону Моисееву, и умертвили их; и спасена была в тот день кровь невинная.
\vs Dan 13:63 Хелкия же и жена его прославили Бога за дочь свою Сусанну с Иоакимом, мужем ее, и со всеми родственниками, потому что не найдено было в ней постыдного дела.
\vs Dan 13:64 И Даниил стал велик перед народом с того дня и потом.
\vs Dan 14:1 Царь Астиаг приложился к отцам своим, и Кир, Персиянин, принял царство его.
\vs Dan 14:2 И Даниил жил вместе с царем и был славнее всех друзей его.
\vs Dan 14:3 Был у Вавилонян идол, по имени Вил, и издерживали на него каждый день двадцать больших мер пшеничной муки, сорок овец и вина шесть мер.
\vs Dan 14:4 Царь чтил его и ходил каждый день поклоняться ему; Даниил же поклонялся Богу своему. И сказал ему царь: почему ты не поклоняешься Вилу?
\vs Dan 14:5 Он отвечал: потому что я не поклоняюсь идолам, сделанным руками, но \bibemph{поклоняюсь} живому Богу, сотворившему небо и землю и владычествующему над всякою плотью.
\vs Dan 14:6 Царь сказал: не думаешь ли ты, что Вил неживой бог? не видишь ли, сколько он ест и пьет каждый день?
\vs Dan 14:7 Даниил, улыбнувшись, сказал: не обманывайся, царь; ибо он внутри глина, а снаружи медь, и никогда ни ел, ни пил.
\vs Dan 14:8 Тогда царь, разгневавшись, призвал жрецов своих и сказал им: если вы не скажете мне, кто съедает все это, то умрете.
\vs Dan 14:9 Если же вы докажете мне, что съедает это Вил, то умрет Даниил, потому что произнес хулу на Вила. И сказал Даниил царю: да будет по слову твоему.
\vs Dan 14:10 Жрецов Вила было семьдесят, кроме жен и детей.
\vs Dan 14:11 И пришел царь с Даниилом в храм Вила, и сказали жрецы Вила: вот, мы выйдем вон, а ты, царь, поставь пищу и, налив вина, запри двери и запечатай перстнем твоим.
\vs Dan 14:12 И если завтра ты придешь и не найдешь, что все съедено Вилом, мы умрем, или Даниил, который солгал на нас.
\vs Dan 14:13 Они не обращали на это внимания, потому что под столом сделали потаенный вход, и им всегда входили, и съедали это.
\vs Dan 14:14 Когда они вышли, царь поставил пищу перед Вилом, а Даниил приказал слугам своим, и они принесли пепел, и посыпали весь храм в присутствии одного царя, и, выйдя, заперли двери, и запечатали царским перстнем, и отошли.
\vs Dan 14:15 Жрецы же, по обычаю своему, пришли ночью с женами и детьми своими, и все съели и выпили.
\vs Dan 14:16 На другой день царь встал рано и Даниил с ним,
\vs Dan 14:17 и сказал: целы ли печати, Даниил? Он сказал: целы, царь.
\vs Dan 14:18 И как скоро отворены были двери, царь, взглянув на стол, воскликнул громким голосом: велик ты, Вил, и нет никакого обмана в тебе!
\vs Dan 14:19 Даниил, улыбнувшись, удержал царя, чтобы он не входил внутрь, и сказал: посмотри на пол и заметь, чьи это следы.
\vs Dan 14:20 Царь сказал: вижу следы мужчин, женщин и детей.
\vs Dan 14:21 И, разгневавшись, царь приказал схватить жрецов, жен их и детей и они показали потаенные двери, которыми они входили и съедали, что было на столе.
\vs Dan 14:22 Тогда царь повелел умертвить их и отдал Вила Даниилу, и он разрушил его и храм его.
\rsbpar\vs Dan 14:23 Был на том месте большой дракон, и Вавилоняне чтили его.
\vs Dan 14:24 И сказал царь Даниилу: не скажешь ли и об этом, что он медь? вот, он живой, и ест и пьет; ты не можешь сказать, что этот бог неживой; итак поклонись ему.
\vs Dan 14:25 Даниил сказал: Господу Богу моему поклоняюсь, потому что Он Бог живой.
\vs Dan 14:26 Но ты, царь, дай мне позволение, и я умерщвлю дракона без меча и жезла. Царь сказал: даю тебе.
\vs Dan 14:27 Тогда Даниил взял смолы, жира и вол\acc{о}с, сварил это вместе и, сделав из этого ком, бросил его в пасть дракону, и дракон расселся. И сказал \bibemph{Даниил}: вот ваши святыни!
\vs Dan 14:28 Когда же Вавилоняне услышали о том, сильно вознегодовали и восстали против царя, и сказали: царь сделался Иудеем, Вила разрушил и убил дракона, и предал смерти жрецов,
\vs Dan 14:29 и, придя к царю, сказали: предай нам Даниила, иначе мы умертвим тебя и дом твой.
\vs Dan 14:30 И когда царь увидел, что они сильно настаивают, принужден был предать им Даниила,
\vs Dan 14:31 они же бросили его в ров львиный, и он пробыл там шесть дней.
\vs Dan 14:32 Во рве было семь львов, и давалось им каждый день по два тела и по две овцы; в это время им не давали их, чтобы они съели Даниила.
\rsbpar\vs Dan 14:33 Был в Иудее пророк Аввакум, который, сварив похлебку и накрошив хлеба в блюдо, шел на поле, чтобы отнести это жнецам.
\vs Dan 14:34 Но Ангел Господень сказал Аввакуму: отнеси этот обед, который у тебя, в Вавилон к Даниилу, в ров львиный.
\vs Dan 14:35 Аввакум сказал: господин! Вавилона я \bibemph{никогда} не видал и рва не знаю.
\vs Dan 14:36 Тогда Ангел Господень взял его за темя и, подняв его за волосы головы его, поставил его в Вавилоне над рвом силою духа своего.
\vs Dan 14:37 И воззвал Аввакум и сказал: Даниил! Даниил! возьми обед, который Бог послал тебе.
\vs Dan 14:38 Даниил сказал: вспомнил Ты обо мне, Боже, и не оставил любящих Тебя.
\vs Dan 14:39 И встал Даниил и ел; Ангел же Божий мгновенно поставил Аввакума на его место.
\rsbpar\vs Dan 14:40 В седьмой день пришел царь, чтобы поскорбеть о Данииле и, подойдя ко рву, взглянул в него, и вот, Даниил сидел.
\vs Dan 14:41 И воскликнул царь громким голосом, и сказал: велик Ты, Господь Бог Даниилов, и нет иного кроме Тебя!
\vs Dan 14:42 И приказал вынуть \bibemph{Даниила}, а виновников его погубления бросить в ров,~--- и они тотчас были съедены в присутствии его.

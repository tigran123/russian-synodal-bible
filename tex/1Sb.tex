\bibbookdescr{1Sb}{
  inline={Первая книга Сивилл},
  toc={1-я Сивилл},
  bookmark={1-я Сивилл},
  header={1-я Сивилл},
  abbr={1~Сив}
}
\vs 1Sb 1:1 С самых истоков начав, возвещу я судьбу поколений,

\vs 1Sb 1:2 Все по порядку скажу от первого века и дальше,

\vs 1Sb 1:3 То, что случилось уже, что есть и что впредь ожидает

\vs 1Sb 1:4 Смертных людей, преступивших священный закон благочестья.

\vs 1Sb 1:5 Первым мне Бог повелел рассказать правдиво о том, как

\vs 1Sb 1:6 Мир порожден был,  а ты внимай моим песням прилежно,

\vs 1Sb 1:7 Смертный, чтобы из них ни слова зря не пропало.

\vs 1Sb 1:8 Царь, всех превыше стоящий, создал и небо и землю,

\vs 1Sb 1:9 Да зародится,  сказав, и тут же все зародилось.

\vs 1Sb 1:10 Тартаром твердь окружив, Он свет дал миру сладчайший,

\vs 1Sb 1:11 Сверху воздвиг небосвод, простер воды светлого моря,

\vs 1Sb 1:12 Множество ярких созвездий обвил вкруг полюса, землю

\vs 1Sb 1:13 Всю цветами украсил, смешал с потоками море,

\vs 1Sb 1:14 Воздух ветрами смутил и влажные дал ему тучи.

\vs 1Sb 1:15 К тварям живым перейдя, Он рыб глубинам доверил,

\vs 1Sb 1:16 Птиц  воздушным потокам, чащобам  зверей густошерстых,

\vs 1Sb 1:17 Гадов пустил по земле, и все, что ныне мы видим,

\vs 1Sb 1:18 Словом единым создал, и по слову все появилось

\vs 1Sb 1:19 Быстро и точно: сие созерцает теперь Нерожденный,

\vs 1Sb 1:20 С неба на землю смотря,  на том Его труд завершен был.

\vs 1Sb 1:21 И уже после того слепил Он живое творенье 

\vs 1Sb 1:22 Образ Свой запечатлев, человека, прекрасного видом,

\vs 1Sb 1:23 Сходного с Богом. Ему повелел в раю поселиться,

\vs 1Sb 1:24 Чтобы благие дела предметом забот его были.

\vs 1Sb 1:25 Тот, оказавшись один средь цветущего райского сада,

\vs 1Sb 1:26 Стал по беседе скучать и вседневно желаньем томился

\vs 1Sb 1:27 Облик увидеть такой же, как свой. Тут, кость его вынув,

\vs 1Sb 1:28 Бог из нее сотворил супругу законную  Еву,

\vs 1Sb 1:29 Женщину дивной красы, и велел ей в раю с человеком

\vs 1Sb 1:30 Жить совместно. Адам, ее оглядев, удивлен был.

\vs 1Sb 1:31 Радуясь сердцем, смотрел на себе подобную. С речью

\vs 1Sb 1:32 К ней обратился разумной, и сами собой получались

\vs 1Sb 1:33 Те слова у него  предусмотрено все было Богом.

\vs 1Sb 1:34 Похоть не застила ум их, стыда они не знавали,

\vs 1Sb 1:35 Были сердца далеки от всякого зла. Словно звери,

\vs 1Sb 1:36 Тело свое напоказ безпечно они выставляли.

\vs 1Sb 1:37 Сразу же после того, как создал, им указанье

\vs 1Sb 1:38 Дал Господь, чтоб они не трогали древа: на это

\vs 1Sb 1:39 Змей их ужасный подбил, на горе обманом заставив

\vs 1Sb 1:40 Смертную долю принять, а с нею вместе  познанье

\vs 1Sb 1:41 Зла и Добра. Между тем, предательство первой свершила

\vs 1Sb 1:42 Женщина: мужу дала, убедив неразумного словом.

\vs 1Sb 1:43 Он же, речами жены увлечен, позабыл о безсмертном

\vs 1Sb 1:44 Мира Творце и совет без внимания мудрый оставил.

\vs 1Sb 1:45 Так получили они по заслугам, когда им досталось

\vs 1Sb 1:46 Зло вместо блага в удел. Проткнув смоковницы листья,

\vs 1Sb 1:47 Сшили одежды себе и друг на друга надели,

\vs 1Sb 1:48 Чресла листвою прикрыв, ибо стыд у них появился.

\vs 1Sb 1:49 Бог же безсмертный обрушил Свой гнев, их выгнал из Рая

\vs 1Sb 1:50 В смертной юдоли свой век коротать, когда не хранили,

\vs 1Sb 1:51 Раз услыхав, они в памяти слово великого Бога.

\vs 1Sb 1:52 Те, очутившись внезапно среди плодородной равнины,

\vs 1Sb 1:53 Стали слезами ее поливать, непрерывно стеная.

\vs 1Sb 1:54 К ним смягчился Безсмертный тогда и сказал в утешенье:

\vs 1Sb 1:55 Род продолжайте, плодитесь, с землей обращайтесь умело,

\vs 1Sb 1:56 Так, чтобы в поте лица добывали, чем голод насытить.

\vs 1Sb 1:57 Слово такое изрек. В обмане виновного змея

\vs 1Sb 1:58 Землю заставил тереть животом и хвостом, без пощады

\vs 1Sb 1:59 Выгнав из Рая. Вражду тогда между ним поселил Он

\vs 1Sb 1:60 И человеком. Один уберечь свою голову тщится,

\vs 1Sb 1:61 Пятку спасает другой: близка ведь стала отныне

\vs 1Sb 1:62 Смерть и к людям, и к тем, кто злом отравляет советы.

\vs 1Sb 1:63 Начал тут род пополняться людской, как велено было

\vs 1Sb 1:64 Им, всемогущим Владыкой. Одно за другим приходили,

\vs 1Sb 1:65 Множа число, поколенья. Дома они начали строить,

\vs 1Sb 1:66 Стены и города возводить с немалым искусством.

\vs 1Sb 1:67 Долгий и радостный день им сопутствовал в жизни. Не зная

\vs 1Sb 1:68 Горя, смерть принимали они, погружаясь как будто

\vs 1Sb 1:69 В сон. Счастливыми были те люди; могучих героев,

\vs 1Sb 1:70 Бог возлюбил их, безсмертный Спаситель и Царь. Но однако

\vs 1Sb 1:71 Стали и эти в безумье грешить, безстыдно принявшись

\vs 1Sb 1:72 На смех отцов выставлять и над матерями глумиться.

\vs 1Sb 1:73 С близкими начали тут обращаться они, как с чужими,

\vs 1Sb 1:74 Брат поднял руку на брата. Пресытились кровью убитых,

\vs 1Sb 1:75 Ею себя запятнав, вели безразсудные войны.

\vs 1Sb 1:76 Пала за это на них с небес наивысшая кара:

\vs 1Sb 1:77 Люди из жизни теснимы быть начали. Всех их, преступных,

\vs 1Sb 1:78 Принял Аид. Называют его Аидом с тех пор, как

\vs 1Sb 1:79 Первым в нем очутился Адам, чашу смерти пригубив.

\vs 1Sb 1:80 Всюду его обступила земля. И это причиной

\vs 1Sb 1:81 Стало того, что о тех, кто живет на земле, говорится:

\vs 1Sb 1:82 В царство Аида уходят. Однако, и сгинув в Аиде,

\vs 1Sb 1:83 Первые люди почет заслужили, что первым дается.

\vs 1Sb 1:84 Сразу же после того, как их земля поглотила,

\vs 1Sb 1:85 Род сотворил Он другой  из тех, кто еще оставался

\vs 1Sb 1:86 Праведной жизни. Они трудились усердно, прекрасны

\vs 1Sb 1:87 Были дела их, стыдом превзошли остальных и имели

\vs 1Sb 1:88 Разум надежный. Искусства им были знакомы. Искали

\vs 1Sb 1:89 Выход в любом затрудненье и быстро его находили.

\vs 1Sb 1:90 Способ один изобрел, как надо вспахивать плугом

\vs 1Sb 1:91 Землю. Другой размышлял над тем, как строить прочнее,

\vs 1Sb 1:92 Третий  как по морю плавать, по птицам гадать, и на небе

\vs 1Sb 1:93 Звезды четвертый умел наблюдать. Про яды знал пятый.

\vs 1Sb 1:94 Магия делом была еще одного. Все ремесла

\vs 1Sb 1:95 Разным поручены были умельцам. Безсонными звали

\vs 1Sb 1:96 Хлебоедами их, поскольку они отличались

\vs 1Sb 1:97 Вечно ясным умом и ненаполнимым желудком.

\vs 1Sb 1:98 Телом могучие, все отошли, однако, под своды

\vs 1Sb 1:99 Страшного царства Аида. Там, скованны прочно цепями,

\vs 1Sb 1:100 Грех свой должны искупать, пребывая в геенне, где пламя

\vs 1Sb 1:101 Неугасимое жжет и огонь жестокий пылает.

\vs 1Sb 1:102 Вслед за ушедшими племя явилось, мощное духом.

\vs 1Sb 1:103 Третьим было по счету оно. Надменных и дерзких

\vs 1Sb 1:104 Объединяло людей, которые многие беды

\vs 1Sb 1:105 В мир принесли. Очень скоро сражения, войны, убийства

\vs 1Sb 1:106 Их истребили, носивших в груди жестокое сердце.

\vs 1Sb 1:107 Та же причина была, что еще один род прекратился.

\vs 1Sb 1:108 Младшее из четырех людских поколений, и это

\vs 1Sb 1:109 Кровью себя осквернило, повсюду ее проливая.

\vs 1Sb 1:110 Был им страх перед Богом неведом, как друг перед другом 

\vs 1Sb 1:111 Чувство стыда. Наконец, против них же самих обратились

\vs 1Sb 1:112 Гнев, сводящий с ума, совместно с буйным нечестьем.

\vs 1Sb 1:113 Так повергли несчастных убийства, сражения, войны

\vs 1Sb 1:114 В мрак преисподней  мужей, преступивших закон. Их небесный,

\vs 1Sb 1:115 Гневаясь, Бог перенес потом за пределы вселенной,

\vs 1Sb 1:116 Тартаром отгородив под самой земли сердцевиной.

\vs 1Sb 1:117 Был за этим еще один род человеческий создан,

\vs 1Sb 1:118 Много хуже других. Ему злую участь безсмертный

\vs 1Sb 1:119 Бог уготовил, когда творить беззаконие стали.

\vs 1Sb 1:120 Нравом надменнее были они, чем прежние люди, 

\vs 1Sb 1:121 Племя Гигантов, в речах нечестиво хулившее Бога.

\vs 1Sb 1:122 Только один среди всех человек был правдивый и верный 

\vs 1Sb 1:123 Ной, закон почитавший и думавший лишь о хорошем.

\vs 1Sb 1:124 Вот с такими словами с небес к нему Бог обратился:

\vs 1Sb 1:125 Мужество, Ной, собери, тотчас призови к покаянью

\vs 1Sb 1:126 Всех людей на земле, чтоб свои они жизни спасали.

\vs 1Sb 1:127 Дела безстыжим коль нет до того, что повсюду творится,

\vs 1Sb 1:128 Род Я их весь погублю невиданным прежде потопом.

\vs 1Sb 1:129 Ты же на прочной основе, воде не дающей прохода,

\vs 1Sb 1:130 Дом себе быстро построй деревянный, надежно стоящий.

\vs 1Sb 1:131 Знание дам Я тебе для того и умение строить,

\vs 1Sb 1:132 Дам укромное место, размер  обо всем позабочусь,

\vs 1Sb 1:133 Так что спасешься ты сам и все, кто живут с тобой вместе.

\vs 1Sb 1:134 Я же есть Сущий, и ты в своем сердце обдумай такое:

\vs 1Sb 1:135 Небо навлек на Себя, вокруг Себя море раскинул,

\vs 1Sb 1:136 Мне опора для ног  земля, вкруг тела разлился

\vs 1Sb 1:137 Воздух, и звезд хоровод Меня кругом обегает.

\vs 1Sb 1:138 Девять имею Я букв, Меня составляют четыре

\vs 1Sb 1:139 Слога, кто Я  ты пойми: три первых слога содержат

\vs 1Sb 1:140 Каждый две буквы, последний же слог  остальные. Согласных

\vs 1Sb 1:141 Пять. Всего же числа  девятнадцать сотен, десятков

\vs 1Sb 1:142 Три и вдобавок семерка. Узнай, кто Я есть, и ты станешь

\vs 1Sb 1:143 Мудрости высшей Моей чуждым уже не совсем.

\vs 1Sb 1:144 Так сказал. И того, кто все это слышал, великий

\vs 1Sb 1:145 Страх охватил. В уме остальное предвидя, он начал

\vs 1Sb 1:146 Тут людей умолять и такие слова говорил им:

\vs 1Sb 1:147 Веры в вас нет, безумья гонимые жалом! Не спустит

\vs 1Sb 1:148 Бог ничего из того, что вы сделали. Знает Безсмертный

\vs 1Sb 1:149 Все, Спаситель всезрящий, и вам об этом поведать

\vs 1Sb 1:150 Он направил меня, чтоб вы души свои не сгубили.

\vs 1Sb 1:151 Трезво на мир посмотрите, от зла отрекитесь и войны

\vs 1Sb 1:152 Между собой перестаньте вести в исступленье жестоком,

\vs 1Sb 1:153 Щедро землю кругом человеческой кровью питая.

\vs 1Sb 1:154 Люди, побойтесь Того, Кто Сам нерушим и огромен,

\vs 1Sb 1:155 На небе сущего Бога, создавшего все во вселенной.

\vs 1Sb 1:156 Все к Нему обратитесь с мольбами  Он милосердный! 

\vs 1Sb 1:157 Жизнь сохранить городов и всего великого мира,

\vs 1Sb 1:158 Четвероногих и птиц  пусть милостив будет ко всем Он.

\vs 1Sb 1:159 Время наступит, когда безкрайний, людьми населенный

\vs 1Sb 1:160 Мир, от вод погибая, провоет жуткую песню.

\vs 1Sb 1:161 Время наступит, и воздух над вами вдруг всколыхнется,

\vs 1Sb 1:162 Бога великого гнев устремится с неба на землю.

\vs 1Sb 1:163 Истинно время придет, когда на людей опрокинет

\vs 1Sb 1:165 Вечно живущий Спаситель, снискать если вам не удастся

\vs 1Sb 1:166 Милость Его и отныне совсем жить иначе, чем прежде,

\vs 1Sb 1:167 Так, чтоб ни зла, ни обид друг другу преступно не строя,

\vs 1Sb 1:168 Каждый праведной жизнью прикрыт был от Божьего гнева.

\vs 1Sb 1:169 Слыша такие слова, его на смех все поднимали,

\vs 1Sb 1:170 Звали несчастным безумцем, которого разум покинул.

\vs 1Sb 1:171 Ной же к ним вновь и опять обращался с докучливой речью:

\vs 1Sb 1:172 Жалость внушаете вы, постоянства лишенные, сердцем

\vs 1Sb 1:173 Злобные, стыд кто отринул, кого влечет лишь безстыдство,

\vs 1Sb 1:174 Жадные мира владыки, насильники и нечестивцы,

\vs 1Sb 1:175 Те, что неверья полны, злодеи, лжецы, кто ни слова

\vs 1Sb 1:176 Правды вовек не сказал, богохульники, прелюбодеи,

\vs 1Sb 1:177 Бога Всевышнего гнев кому не страшен,  расплата

\vs 1Sb 1:178 Всех вас теперь ожидает до родичей в пятом колене.

\vs 1Sb 1:179 С криком не мечетесь вы, жестокие, только смеетесь:

\vs 1Sb 1:180 Будет язвительный смех на губах, когда вдруг наступит

\vs 1Sb 1:181 То, о чем говорю: невиданный прежде, ужасный

\vs 1Sb 1:182 Хлынет на землю потоп, самим низпосланный Богом.

\vs 1Sb 1:183 Новый род на земле, священный, тут создан водою

\vs 1Sb 1:185 Будет  продолжится он, на корне сухом произросший.

\vs 1Sb 1:186 Сам собою поток в одну ночь исчезнет. Тогда же

\vs 1Sb 1:187 Вместе с людьми города разметает земли Колебатель,

\vs 1Sb 1:188 Их в укромных ущельях достав, и стены разрушит.

\vs 1Sb 1:189 Так погибнет весь мир, и люди исчезнут без счета,

\vs 1Sb 1:190 Те, что его населяют. А мне еще сколько придется

\vs 1Sb 1:191 Горя изведать и скольких еще погибших оплакать

\vs 1Sb 1:192 В доме своем деревянном? С волнами сколько смешаю

\vs 1Sb 1:193 Слез? Ведь только нахлынут по слову Божьему воды,

\vs 1Sb 1:194 Все поплывет  и земля, и горы, и небо над ними.

\vs 1Sb 1:195 Мир весь станет водой и водами будет погублен.

\vs 1Sb 1:195 Ветры дуть прекратят, наступит другая эпоха.

\vs 1Sb 1:196 Фригия! первою ты из воды приподнимешь вершину,

\vs 1Sb 1:197 Первая будешь кормить ты новое племя людское,

\vs 1Sb 1:199 Кончил когда он впустую слова расточать нечестивцам,

\vs 1Sb 1:200 Сам Всевышний явился, и вновь прозвучал Его голос:

\vs 1Sb 1:201 Время настало, о Ной, объявить обо всем по порядку,

\vs 1Sb 1:202 Что Я в тот день обещал тебе привести в исполненье:

\vs 1Sb 1:203 Неисчислимое зло, которое люди свершили,

\vs 1Sb 1:204 Миру без края вернуть за непослушание смертных.

\vs 1Sb 1:205 Ты же прийти поспеши с женой своей и с сыновьями,

\vs 1Sb 1:206 Также их жен позови и тех, кому повелел Я

\vs 1Sb 1:207 Волю Мою объявить: животных, змей и пернатых.

\vs 1Sb 1:208 Этим Сам зароню Я в сердце желанье явиться 

\vs 1Sb 1:209 Всем, кому Я предназначил продолжить дни свои дальше.

\vs 1Sb 1:210 Так было сказано. Ной пошел и громко об этом

\vs 1Sb 1:211 Им возвестил. Тогда жена, сыновья и невестки

\vs 1Sb 1:212 В дом деревянный взошли, и сразу за ними туда же

\vs 1Sb 1:213 Прочие твари, кому Господь повелел это сделать.

\vs 1Sb 1:214 Тут же засов закрепили, надежно дверь закрывавший.

\vs 1Sb 1:215 Косо он приходился в борту, что был гладко оструган.

\vs 1Sb 1:216 Воля небесного Бога тем самым вполне совершилась.

\vs 1Sb 1:217 Тучи собрал Он и скрыл сверкавший ярко диск солнца,

\vs 1Sb 1:218 Звезды вместе с луной и корону, венчавшую небо.

\vs 1Sb 1:219 Тьмою тут все окружив, загремел, людей повергая

\vs 1Sb 1:220 В ужас, наслал ураган  и ветры разом проснулись,

\vs 1Sb 1:221 Вздулись водные жилы и русла покинули, с неба

\vs 1Sb 1:222 Хлынули, вдруг открывшись, огромные водопады,

\vs 1Sb 1:223 Массы воды из трещин, глубоких провалов внезапно

\vs 1Sb 1:224 Вышли на свет, и под ними земля вся безкрайняя скрылась.

\vs 1Sb 1:225 Плавал тогда под дождем ковчег, что по слову был создан

\vs 1Sb 1:226 Бога: удары терпя от волн, подчиняясь порывам

\vs 1Sb 1:227 Ветра, вдруг поднимался он вверх, и множество пены

\vs 1Sb 1:228 Киль разсекал под журчанье воды, что двигалась всюду.

\vs 1Sb 1:229 Тут, когда весь уже мир затопил дождями Всевышний,

\vs 1Sb 1:230 В голову Ною пришло посмотреть, как исполнилась воля

\vs 1Sb 1:231 Господня, и заглянуть самому в морскую пучину.

\vs 1Sb 1:232 Быстро он дверь распахнул в борту, что был гладко оструган,

\vs 1Sb 1:233 Плотно створки которой одна к другой прилегали.

\vs 1Sb 1:234 Только ее он открыл  представилось взору пространство,

\vs 1Sb 1:235 Сплошь покрыто водой, везде, без конца и без края.

\vs 1Sb 1:236 Страх тут и трепет его охватили. В это мгновенье

\vs 1Sb 1:237 Стал редеть понемногу туман, в течение многих

\vs 1Sb 1:238 Дней уставший окутывать мир. Он бледно-кровавый

\vs 1Sb 1:239 Неба вечернего свод показал и усталого солнца

\vs 1Sb 1:240 Огненный диск. Насилу вернулось мужество к Ною.

\vs 1Sb 1:241 Вдаль направив полет, он сизую выпустил птицу,

\vs 1Sb 1:242 Чтобы узнала она, вдруг где-то еще сохранилась

\vs 1Sb 1:243 Твердая почва. Устав бить крыльями воздух, вернулась

\vs 1Sb 1:244 Птица, кругом облетев: вода нигде не спадала,

\vs 1Sb 1:245 Все было ею полно. Через несколько дней он отправил

\vs 1Sb 1:246 Снова голубку узнать, отступили ли воды. Она же,

\vs 1Sb 1:247 Легкая, в дальний опять отправилась путь и достигла

\vs 1Sb 1:248 Влажной земли. Проведя там какое-то время, обратно

\vs 1Sb 1:249 К Ною вернулась, неся засохшую ветку оливы 

\vs 1Sb 1:250 Знак удачи посольства. В сердцах пробудилась отвага,

\vs 1Sb 1:251 Землю увидеть надежда вселила великую радость.

\vs 1Sb 1:252 Сразу же после того еще чернокрылую птицу

\vs 1Sb 1:253 Ной поспешил отпустить. Она, доверившись крыльям,

\vs 1Sb 1:254 Вдаль устремилась охотно  достигнув земли, там осталась.

\vs 1Sb 1:255 Стало тогда очевидно, что ближе придвинулась суша.

\vs 1Sb 1:256 Скоро, плывя среди волн наугад но шумящему понту,

\vs 1Sb 1:257 Горы встречая воды повсюду, нетленное судно

\vs 1Sb 1:258 Дном увязнув, на узкой полоске земли утвердилось.

\vs 1Sb 1:259 Есть во Фригии черной, что без конца и без края,

\vs 1Sb 1:260 Горный обрывистый кряж, называется он Араратом:

\vs 1Sb 1:261 Здесь предстояло спастись всем тем, кто был с Ноем,  и жажду

\vs 1Sb 1:262 В душу вложил им Господь, едва в это место попали,

\vs 1Sb 1:263 Было тут много ключей, от которых питается Марсий.

\vs 1Sb 1:264 После, как спала вода, ковчег на высокой вершине

\vs 1Sb 1:265 Так и остался лежать, и вновь прозвучал тогда с неба

\vs 1Sb 1:266 Голос Великого Бога нетленный. Он слово такое

\vs 1Sb 1:267 Молвил: Ной, избранник судьбы, справедливый и верный!

\vs 1Sb 1:268 Смело покинь свой ковчег с сыновьями вместе, с женою,

\vs 1Sb 1:269 Три пусть выходят невестки: собой наполнить отныне

\vs 1Sb 1:270 Землю должны вы, плодясь и множа свой род, по закону

\vs 1Sb 1:271 Каждому часть уделив, из колена в колено, доколе

\vs 1Sb 1:272 Время суда не придет, который вас всех ожидает.

\vs 1Sb 1:273 Так произнес вечный голос, и Ной, осмелев, из ковчега

\vs 1Sb 1:274 Спрыгнул на землю, а следом  жена, сыновья и невестки,

\vs 1Sb 1:275 Племя пернатых, ползучие гады, и четвероногих

\vs 1Sb 1:276 Разные виды. Все вместе оставили дом деревянный,

\vs 1Sb 1:277 Вместе на землю сошли, и стала она общим домом.

\vs 1Sb 1:278 Ной тогда, всех людей превзошедший праведной жизнью,

\vs 1Sb 1:279 После Адама восьмой, спустился на твердую землю,

\vs 1Sb 1:280 Сорок дней и один проплавав по воле Господней.

\vs 1Sb 1:281 Так поднялся тогда новый род и жизнь свою начал,

\vs 1Sb 1:282 Первый и золотой, шестым был он и наилучшим

\vs 1Sb 1:283 С тех самых пор, как Господь впервые создал человека.

\vs 1Sb 1:284 Буду его называть я небесным, поскольку заботу

\vs 1Sb 1:285 Бог возложил на Себя обо всем, в чем нужда возникала.

\vs 1Sb 1:286 О поколение первое рода шестого! О радость,

\vs 1Sb 1:287 Что ты доставило мне, когда неминуемой смерти

\vs 1Sb 1:288 Я избежала, устав на волнах качаться и страха

\vs 1Sb 1:289 Много перетерпев вместе с мужем и деверьями,

\vs 1Sb 1:290 С женами их, со свекровью и свекром! Достойную славу

\vs 1Sb 1:291 Я тебе пропою: цветок на смоковнице будет

\vs 1Sb 1:292 Пестрый, до середины дойдут века и положат

\vs 1Sb 1:293 Царской власти начало, что носит скипетр, и трое

\vs 1Sb 1:294 Духом могучих царей, справедливейших, земли поделят.

\vs 1Sb 1:295 Многие годы продлится их власть. Делить по закону

\vs 1Sb 1:296 Между людьми они станут заботы и радость. Земля же

\vs 1Sb 1:297 Будет гордиться плодами, что сами собой вызревают,

\vs 1Sb 1:298 Вся расцветет и зерном осыпет счастливое племя.

\vs 1Sb 1:299 Старость с годами к отцам не придет, не зная болезней,

\vs 1Sb 1:300 Смерть сразу многим несущих, и даже озноба, как будто

\vs 1Sb 1:301 В сон погружаясь, умрут, отойдут к берегам Ахеронта,

\vs 1Sb 1:302 В царство Аида, где им будут возданы почести. Ибо

\vs 1Sb 1:303 Род их был родом блаженных и те изведали счастья,

\vs 1Sb 1:304 В головы чьи заложил Саваоф глубокую мудрость 

\vs 1Sb 1:305 С ними всегда обсуждал Он Свою безсмертную волю.

\vs 1Sb 1:306 Но даже этих счастливцев Аид впереди ожидает.

\vs 1Sb 1:307 После на смену придет тяжелое, крепкое племя

\vs 1Sb 1:308 Земнородных людей и будет по счету второе.

\vs 1Sb 1:309 Имя тем людям Титаны, один на другого похожи,

\vs 1Sb 1:310 Каждый ростом, лицом остальных напомнит. Осанка,

\vs 1Sb 1:311 Голос будет один, какой был Богом заложен

\vs 1Sb 1:312 Некогда предкам их в грудь. Однако и эти, имея

\vs 1Sb 1:313 Дерзкий нрав, замахнутся на то, что им не по силам;

\vs 1Sb 1:314 Смерть приближая свою, захотят сразиться со звездным

\vs 1Sb 1:315 Небом. За это на них океана великого воды

\vs 1Sb 1:316 Хлынут бурным потоком  и сам Саваоф, разсердившись,

\vs 1Sb 1:317 Их удерживать будет, мешая тому, чтобы снова

\vs 1Sb 1:318 Из-за злонравия смертных весь мир под водой оказался.

\vs 1Sb 1:319 Но когда Он заставит всех вод безпредельных волненье

\vs 1Sb 1:320 Гнев усмирить свой, сшибая валы и лишая их силы,

\vs 1Sb 1:321 На неглубоких местах, напротив, волну уменьшая

\vs 1Sb 1:322 Тем, что море землей окружит и о берег неровный

\vs 1Sb 1:323 Биться принудит его великий Бог-громовержец

\vs 1Sb 1:324 Сын Его к людям придет, уподобившись обликом смертным,

\vs 1Sb 1:325 В плоть облечен, как и все на земле. Он гласных четыре

\vs 1Sb 1:326 Будет иметь и двойной согласный. Тебе назову я

\vs 1Sb 1:327 Все число целиком: единиц в нем содержится восемь,

\vs 1Sb 1:328 Столько же, сколько десятков; вдобавок к этому сотен

\vs 1Sb 1:329 Тоже восемь предъявит неверящим людям то имя.

\vs 1Sb 1:330 Должен умом ты постичь, что Сын Безсмертного Бога,

\vs 1Sb 1:331 Выше Которого нет,  Христос, Помазанник Божий.

\vs 1Sb 1:332 Он исполнит закон Отца своего, не разрушит;

\vs 1Sb 1:333 Образ Его воплотив, передаст в полноте и ученье.

\vs 1Sb 1:334 Золото в дар принесут волхвы ему, ладан и смирну,

\vs 1Sb 1:335 Ибо он все совершит, что рожденье его предвещало.

\vs 1Sb 1:336 Голос тогда донесется неслыханный через пустыню,

\vs 1Sb 1:337 Чтобы людей известить, и всем повелит приготовить

\vs 1Sb 1:338 Тропы прямые, изгнать пороки с корнем из сердца.

\vs 1Sb 1:339 Также водою омыть велит он каждому тело,

\vs 1Sb 1:340 Свет чтоб оно обрело и чтобы, рожденные свыше,

\vs 1Sb 1:341 Люди больше нигде с благого пути не свернули.

\vs 1Sb 1:342 Этот Божественный голос опутанный пляскою варвар

\vs 1Sb 1:343 Разом отделит от тела, за что понесет наказанье.

\vs 1Sb 1:344 Будет тут знаменье смертным, когда из Египта нежданно

\vs 1Sb 1:345 Камень придет драгоценный, хранимый Богом. Споткнется

\vs 1Sb 1:346 Племя Евреев на Нем, Другие народы, напротив,

\vs 1Sb 1:347 Вместе Его руководству доверятся, ибо познают

\vs 1Sb 1:348 Бога Всевышнего так и дорогу увидят при свете,

\vs 1Sb 1:349 Что возсияет для всех. Ведь вечную жизнь Он укажет

\vs 1Sb 1:350 Избранным и принесет огонь на века нечестивым.

\vs 1Sb 1:351 Станет тогда же лечить больных Он и немощных телом 

\vs 1Sb 1:352 Всех, кто поверил в Него и свои возложил упованья.

\vs 1Sb 1:353 Видеть слепые начнут, хромые пойдут без поддержки,

\vs 1Sb 1:354 Те, кто не слышал, услышат, и вновь залепечут немые.

\vs 1Sb 1:355 Демонов выгонит Он, возстанут из гроба, кто умер.

\vs 1Sb 1:356 Будет ходить по волнам, пять тысяч в пустыне накормит

\vs 1Sb 1:357 Он от пяти хлебов и единой рыбы. Двенадцать

\vs 1Sb 1:358 Трапезы этой остатки корзин собою наполнят.

\vs 1Sb 1:360 Пьяный Израиль тогда ни во что не сможет проникнуть,

\vs 1Sb 1:361 На ухо туг, он никак не ответит, от хмеля тяжелый.

\vs 1Sb 1:362 Но когда на Евреев Всевышний гнев свой обрушит

\vs 1Sb 1:363 Меткоразящий и веру у их народа отнимет,

\vs 1Sb 1:364 Из-за того, что они распяли Божьего Сына,

\vs 1Sb 1:365 Будет Израиль плевать в Него из уст нечестивых

\vs 1Sb 1:366 Яда полной слюной и бить по щекам Его станет.

\vs 1Sb 1:367 Желчь Ему вместо еды и уксус вместо напитка

\vs 1Sb 1:368 Тут нечестиво дадут, побуждаемы тяжким безумьем,

\vs 1Sb 1:369 Ум поразившим и сердце, глазами смотря и не видя 

\vs 1Sb 1:370 Слепы хуже кротов, ужаснее змей ядовитых,

\vs 1Sb 1:371 Ползают что по земле, опутаны сонным дурманом.

\vs 1Sb 1:372 Он же как руки раскинет и все до конца перетерпит,

\vs 1Sb 1:373 На голове понесет венец терновый, и в ребра

\vs 1Sb 1:374 Ткнут Ему острый тростник  среди белого дня воцарится

\vs 1Sb 1:375 Ночь тогда на три часа и тьмою кругом все покроет.

\vs 1Sb 1:376 Знак тут храм Соломонов народам подаст величайший,

\vs 1Sb 1:377 В домы Аида когда отправится Он, возвещая

\vs 1Sb 1:378 Тем, кто умер, что день придет  и из гроба возстанут.

\vs 1Sb 1:379 Через три дня же обратно на свет из Аида вернется,

\vs 1Sb 1:380 Смертным дабы явить Свой образ и научить их.

\vs 1Sb 1:381 После по облакам пройдет Он к жилищу на небе,

\vs 1Sb 1:382 Миру вместо Себя завет Благовестья оставив.

\vs 1Sb 1:383 Здесь во имя Его росток появится новый

\vs 1Sb 1:384 Из народов, что чтут Закон великого Бога.

\vs 1Sb 1:385 Будут тогда на земле мудрецы, что дорогу покажут,

\vs 1Sb 1:386 Всяким пророкам конец после этого в мире настанет.

\vs 1Sb 1:387 С той поры, как Евреи пожнут недобрую жатву,

\vs 1Sb 1:388 Много у них серебра и золота много отнимет

\vs 1Sb 1:389 Римский кесарь. А после другие царства сменяться

\vs 1Sb 1:390 Станут одно за другим со смертью владык и обиды

\vs 1Sb 1:391 Людям чинить. Тогда великие беды придется

\vs 1Sb 1:392 Вынести смертным за то, что гордыми будут не в меру.

\vs 1Sb 1:393 Храм же когда Соломонов в Священной Земле под ударом

\vs 1Sb 1:394 Варварских полчищ падет, одетых в доспехи из меди,

\vs 1Sb 1:395 Изгнаны будут Евреи с Земли, и по миру скитаться

\vs 1Sb 1:396 Им предстоит, претерпев разорение полное, плевел

\vs 1Sb 1:397 В хлеб добавлять. Незавидный удел их всех ожидает.

\vs 1Sb 1:398 Что же до городов, то одних оплачут другие,

\vs 1Sb 1:399 Сами изведав позор  ведь некогда все согрешили,

\vs 1Sb 1:400 Гнев Великого Бога за это приняв в наказанье.

\bibbookdescr{5Sb}{
  inline={Пятая книга Сивилл},
  toc={5-я Сивилл},
  bookmark={5-я Сивилл},
  header={5-я Сивилл},
  abbr={5~Сив}
}
\vs 5Sb 1:1 Слушай, что я расскажу о горестном веке Латинян: 

\vs 5Sb 1:2 Прежде всего, как умрут владыки Египта, которых 

\vs 5Sb 1:3 Всех, одного за другим, земля забрала равнодушно; 

\vs 5Sb 1:4 После рожденного в Пелле, которому под ноги пали

\vs 5Sb 1:5 Все Восточные страны и Запад, безмерно богатый, 

\vs 5Sb 1:6 Кто, посрамлен Вавилоном, был мертвым отправлен к Филиппу,

\vs 5Sb 1:7 Сыном Аммона и Зевса напрасно кого называли; 

\vs 5Sb 1:8 Также и после того, кто плоть и кровь Ассарака, 

\vs 5Sb 1:9 Кто из-под Трои бежал, пройдя сквозь пламени стены;

\vs 5Sb 1:10 После ряда царей  мужей, возлюбивших сраженья, 

\vs 5Sb 1:11 После младенцев, рожденных от зверя, губителя стада, 

\vs 5Sb 1:12 Будет первый властитель, который буквой начальной 

\vs 5Sb 1:13 Увенчает двадцатку. Не знать ему равного в битвах, 

\vs 5Sb 1:14 Имя его начинаться с десятки будет. За этим

\vs 5Sb 1:15 Тот станет править, чей знак  начало всего алфавита. 

\vs 5Sb 1:16 Робость пред ним ощутят Сицилия, Фракия, Мемфис  

\vs 5Sb 1:17 Мемфис, поверженный в прах виною своих полководцев, 

\vs 5Sb 1:18 Из-за упрямства жены, что бросится в волны морские,  

\vs 5Sb 1:19 Он установит народам законы и всех подчинит их. 

\vs 5Sb 1:20 Времени много пройдет  другому власть он оставит, 

\vs 5Sb 1:21 Будет который иметь знак триста на первую букву. 

\vs 5Sb 1:22 Даст ему имя река. Владыкой Персов считаться 

\vs 5Sb 1:23 Станет он и Вавилона, на копья насадит Мидийцев. 

\vs 5Sb 1:24 После же тот примет власть, чьим знаком выпала тройка.

\vs 5Sb 1:25 Царь будет править за ним, которому знак дважды десять. 

\vs 5Sb 1:26 Он доберется до самых окраинных вод Океана 

\vs 5Sb 1:27 И к берегам Авзонийским едва до отлива поспеет. 

\vs 5Sb 1:28 Тот, кому знак пятьдесят назначен судьбой, государем

\vs 5Sb 1:29 Станет, чудовищный змей, войною дышащий тяжкой, 

\vs 5Sb 1:30 Руку который па мать поднимет и смуту посеет,

\vs 5Sb 1:31 Сам нападая, гоня, убивая, творя беззаконье.

\vs 5Sb 1:32 Гору, что между двух волн, разсечет и забрызгает грязью.

\vs 5Sb 1:33 После же смерти исчезнет. Затем вернется обратно,

\vs 5Sb 1:34 С Богом равняясь. Но скоро покажет, что вовсе не Бог он. 

\vs 5Sb 1:35 Трое царей вслед за ним один другого погубят.

\vs 5Sb 1:36 Благочестивых убийца тогда станет править, могучий.

\vs 5Sb 1:37 Семь раз по десять  свой знак  он явит миру. Отнимет

\vs 5Sb 1:38 Власть у него его сын, чьим знаком будут три сотни.

\vs 5Sb 1:39 После судьбой решено быть царю  по знаку четверке. 

\vs 5Sb 1:40 Вслед за этим придет старик, чье число пять десятков.

\vs 5Sb 1:41 Тот же, кто после него воцарится на троне, от века

\vs 5Sb 1:42 Имени первую букву значением триста имеет.

\vs 5Sb 1:43 Горы ногою поправ, спеша на Восточную битву,

\vs 5Sb 1:44 Кельт, он безславную смерть найдет в пути от болезни. 

\vs 5Sb 1:45 Примет мертвого пыль чужеземная; имя Немейский

\vs 5Sb 1:46 Дал ей цветок. А затем  другой, в серебряном шлеме,

\vs 5Sb 1:47 Станет у власти. Ему свое имя море подарит.

\vs 5Sb 1:48 Будет он доблестный муж, с умом, проникающим всюду.

\vs 5Sb 1:49 Так, при тебе, наилучший, что всех превзошел, темнокудрый, 

\vs 5Sb 1:50 И при потомках твоих придет, наконец, это время.

\vs 5Sb 1:51 После него  три царя, из которых третий  не скоро.

\vs 5Sb 1:52 Трижды несчастна, терзаюсь: в груди недоброе слово 

\vs 5Sb 1:53 Давит, Изиды сестра томится пророческой песнью.

\vs 5Sb 1:54 Первым вокруг твоего многослезного храма, Египет,

\vs 5Sb 1:55 Вихрем помчатся менады, и ты попадешь в злые руки

\vs 5Sb 1:56 В тот самый день, когда Нил понесет свои воды однажды

\vs 5Sb 1:57 Через страну Египтян на шестнадцать локтей полноводней,

\vs 5Sb 1:58 Так что омоет всю землю, пустив по ней литься потоки.

\vs 5Sb 1:59 Смолкнет тут радостный смех, чело земли омрачится.

\vs 5Sb 1:60 Мемфис! Ты горше других Египта беды оплачешь,

\vs 5Sb 1:61 Ибо, над всею землей до сих пор величаво царивший,

\vs 5Sb 1:62 Станешь чертогом печали. Тогда призовет тебя с неба

\vs 5Sb 1:63 Зычно сам Повелитель перунов: О, Мемфис могучий,

\vs 5Sb 1:64 Раньше пред жалким народом кичился своею ты славой,

\vs 5Sb 1:65 Ныне в несчастье и скорби заплачешь: придет постиженье

\vs 5Sb 1:66 Вечного Бога к тебе, Безсмертного, сущего в небе. 

\vs 5Sb 1:67 Где теперь воля твоя, что судьбы людские вершила? 

\vs 5Sb 1:68 В диком безумстве детей ты моих, помазанных Богом, 

\vs 5Sb 1:69 Тяжким гоненьям подверг и праведным зло уготовил.

\vs 5Sb 1:70 Будешь за эти дела наказан. Мачеха злая

\vs 5Sb 1:71 Станет уделом твоим, и вовек тебе счастья не видеть: 

\vs 5Sb 1:72 С неба скатившись звездой, обратно не сможешь подняться.

\vs 5Sb 1:73 Вот что Господь мне внушил правдиво Египту поведать 

\vs 5Sb 1:74 В самом исходе времен, когда люди погрязнут в пороках.

\vs 5Sb 1:75 Но продолжают страдать, нечестивцы, в преддверии кары  

\vs 5Sb 1:76 Гнева безсмертного Бога, что тяжко гремит, Небожитель. 

\vs 5Sb 1:77 Вместо Него почитают чудовищ и камни, повсюду 

\vs 5Sb 1:78 Видят священного страха предметы, в которых ни смысла 

\vs 5Sb 1:79 Нет, ни рассудка, ни слуха  о них говорить не пристало

\vs 5Sb 1:80 Мне, называя божков  творения рук человека. 

\vs 5Sb 1:81 Взявшись сами за труд воплотить нечестивые мысли, 

\vs 5Sb 1:82 Люди богов сотворили и каменных, и деревянных, 

\vs 5Sb 1:83 Медных и золотых, серебряных  в коих ни пользы 

\vs 5Sb 1:84 Нет, ни души, глухих, на огне из металла отлитых.

\vs 5Sb 1:85 Сделав же их для себя, впустую на них уповали: 

\vs 5Sb 1:86 Тмуис и Ксуис в беде, конец приходит засилью 

\vs 5Sb 1:87 Зевса, Геракла, Гермеса \ldots

\vs 5Sb 1:88 Славная меть городов, ты тоже, Александрия, 

\vs 5Sb 1:89 Жертвой войны упадешь и то, чем прежде владела,

\vs 5Sb 1:90 Все до остатка отдашь в наказанье за дерзкий характер. 

\vs 5Sb 1:91 Долгим молчание будет, но радостный день возвращенья 

\vs 5Sb 1:92 Больше тебе не нальет напиток нежный \ldots

\vs 5Sb 1:93 Перс наводнит твою землю, подобен жестокому граду, 

\vs 5Sb 1:94 Смерть и разруху неся, людей злонравных погубит.

\vs 5Sb 1:95 Кровью зальет алтари, завалит телами убитых 

\vs 5Sb 1:96 Варвар могучий, свершит он другие безумства, как эти, 

\vs 5Sb 1:97 Словно песчаная буря, замыслив конец твой ускорить. 

\vs 5Sb 1:98 Город счастливый, тогда претерпишь ты многие беды! 

\vs 5Sb 1:99 Вся будет Азия плакать, дары вспоминая, какими

\vs 5Sb 1:100 Голову ты ей венчала  теперь она тоже погибнет. 

\vs 5Sb 1:101 Новый Персидский владыка подвергнет страну разоренью, 

\vs 5Sb 1:102 Всякий им будет убит, и жизнь в тех местах прекратится. 

\vs 5Sb 1:103 Третья лишь часть уцелеет от жалкого племени смертных.

\vs 5Sb 1:104 Он же тут легким прыжком помчится на крыльях к Востоку, 

\vs 5Sb 1:105 Мучая землю войной, в пустыню ее превращая. 

\vs 5Sb 1:106 Власти на гребне своей, хотя и терзаемый страхом, 

\vs 5Sb 1:107 К городу праведных он подойдет, желая разрушить. 

\vs 5Sb 1:108 Посланный Богом тогда некий царь на него ополчится, 

\vs 5Sb 1:109 Что всех великих царей погубит и воинов лучших. 

\vs 5Sb 1:110 Так над людьми приговор исполнит Безсмертный Владыка.

\vs 5Sb 1:111 Гадкое сердце! Зачем ты меня подстрекаешь на это  

\vs 5Sb 1:112 Многих царей предсказать Египту ужасное царство? 

\vs 5Sb 1:113 Лучше вернись на Восток, к потомкам несмысленных

\vs 5Sb 1:114 Персов, Им покажи все как есть, и то, что еще ожидает.

\vs 5Sb 1:115 Воды Евфрата, разлившись, затопят окрестные земли, 

\vs 5Sb 1:116 Персов погубят они, Иберов и Вавилонян, 

\vs 5Sb 1:117 И Массагетов, войну ведущих при помощи луков. 

\vs 5Sb 1:118 Азия до Островов все блеском пожаров осветит. 

\vs 5Sb 1:119 Некогда великолепный, Пергам совсем опустеет.

\vs 5Sb 1:120 Так же, как он, и Питана предстанет безлюдной пустыней. 

\vs 5Sb 1:121 Лесбос опустится весь в пучину бездонную моря. 

\vs 5Sb 1:122 Смирна, с крутых берегов скользнув, заплачет однажды  

\vs 5Sb 1:123 Та, что была столь горда и известна, безславно погибнет. 

\vs 5Sb 1:124 Землю, что стала золой, слезами Вифинцы омоют,

\vs 5Sb 1:125 Сирию всю целиком, многолюдную с ней Финикию. 

\vs 5Sb 1:126 Ликия, горе тебе  столько бед для тебя замышляет 

\vs 5Sb 1:127 Море: однажды само на несчастную землю нахлынув, 

\vs 5Sb 1:128 Скроет в соленых волнах, при страшных подземных ударах 

\vs 5Sb 1:129 Берег Ликийский, где миро растет и где нет его вовсе.

\vs 5Sb 1:130 Гнев на Фригийцев падет ужасный из-за печали, 

\vs 5Sb 1:131 Ради которой пришла сюда Рея и здесь поселилась. 

\vs 5Sb 1:132 Море Таврский народ уничтожит и варваров племя, 

\vs 5Sb 1:133 А Эпидана поток по земле разметает Лапифов, 

\vs 5Sb 1:134 Водовороты крутя, Фессалийскую область погубит.

\vs 5Sb 1:135 Глубоководный Пеней увлечет за собою животных, 

\vs 5Sb 1:136 Тех, что когда-то родил Эпидан, как люди считают.

\vs 5Sb 1:137 Трижды несчастной Эллады поэты участь оплачут, 

\vs 5Sb 1:138 Царь Италийский когда перебьет сухожилие Истма, 

\vs 5Sb 1:139 Богу подобный, могучий, великого Рима властитель.

\vs 5Sb 1:140 Сам его Зевс, говорят, породил и владычица Гера. 

\vs 5Sb 1:141 Кто при стеченье народа поет сладкозвучные гимны

\vs 5Sb 1:142 Голосом нежным, убьет свою мать и многих несчастных.

\vs 5Sb 1:143 Вождь трусливый и наглый, бежит от стен Вавилона 

\vs 5Sb 1:144 Тот, кого среди смертных сильнейшие даже боятся; 

\vs 5Sb 1:145 Многих он жизни лишил, не щадил и матери чрева,

\vs 5Sb 1:146 Грязной любви предавался, вместилище всяких пороков.

\vs 5Sb 1:147 Путь свой к Мидийцам направит и грозным правителям Персов 

\vs 5Sb 1:148 Их он всех раньше призвал и славу им уготовил,

\vs 5Sb 1:149 На неугодный народ замышляя с толпой нечестивцев. 

\vs 5Sb 1:150 Богом поставленный храм захватил он, сжег безпощадно

\vs 5Sb 1:151 Тех, что входили в него, кого я по заслугам воспела.

\vs 5Sb 1:152 В храм он лишь только вступил, как здание все содрогнулось,

\vs 5Sb 1:153 Гибли повсюду цари, а те, кто остался у власти,

\vs 5Sb 1:154 Город великий сгубили с народом праведным вместе.

\vs 5Sb 1:155 В год же четвертый, когда звезда засияет большая, 

\vs 5Sb 1:156 Землю которая всю уничтожит одна ради мести,

\vs 5Sb 1:157 \ldots

\vs 5Sb 1:158 С неба большая звезда упадет в соленые воды, 

\vs 5Sb 1:159 Море она подожжет и с ним Вавилона твердыни, 

\vs 5Sb 1:160 Землю Италии, много виною которой погибло 

\vs 5Sb 1:161 Благочестивых Евреев, угодного Богу народа.

\vs 5Sb 1:162 Между порочных мужей ты себя запятнаешь пороком, 

\vs 5Sb 1:163 Целую вечность потом простоишь совсем опустелым,

\vs 5Sb 1:164 \ldots

\vs 5Sb 1:165 День основанья прокляв, за то, что требовал яда: 

\vs 5Sb 1:166 Ложу измены в тебе, малолетних детей совращенье, 

\vs 5Sb 1:167 Женственный город, дурной, нечестивый и самый несчастный,

\vs 5Sb 1:168 Самый порочный из всех городов земли Италийской, 

\vs 5Sb 1:169 Помесь менады с ехидной, вдовой на холмах ты возляжешь,

\vs 5Sb 1:170 Тибра поток по тебе будет плакать, по милой подруге, 

\vs 5Sb 1:171 В сердце чьем мерзость убийства, а дух отягчен преступленьем,

\vs 5Sb 1:172 Разве не знал ты, что может Господь и что замышляет? 

\vs 5Sb 1:173 Ты говорил: Я один, и никто меня не разрушит!

\vs 5Sb 1:174 Ныне же граждан твоих и тебя вечный Бог уничтожит, 

\vs 5Sb 1:175 Впредь никакое жилье не укажет на то, что здесь было, 

\vs 5Sb 1:176 Как в то время, когда твою славу Господь лишь задумал. 

\vs 5Sb 1:177 Будь же один, безрассудный, и, пламенем жарким охвачен, 

\vs 5Sb 1:178 Рухни в безжалостный мрак забытого Богом Аида.

\vs 5Sb 1:179 Снова теперь о твоем я горюю несчастье, Египет! 

\vs 5Sb 1:180 Мемфис, под гнетом страданий ты первым падешь на колени,

\vs 5Sb 1:181 Даже твои пирамиды ужасные вопли исторгнут.

\vs 5Sb 1:182 Пифон, что некогда прежде Диполисом звался по праву,

\vs 5Sb 1:183 Ты замолчишь навсегда, чтобы впредь не творить злодеяний,

\vs 5Sb 1:184 Город надменный, ларец всевозможных пороков, менадой 

\vs 5Sb 1:185 Жалкой, несчастной вдовой навеки отныне пребудешь 

\vs 5Sb 1:186 Ты, что была рождена править миром долгие годы.

\vs 5Sb 1:187 Но когда на себя кипассий Барка набросит

\vs 5Sb 1:188 Грязного белый поверх, то лучше бы ей не родиться.

\vs 5Sb 1:189 Фивы, великая сила куда ваша делась? Разбойник 

\vs 5Sb 1:190 Сгубит народ ваш, а вы, надев одежды печали,

\vs 5Sb 1:191 В плаче зайдетесь, одни, вину искупая несчастьем 

\vs 5Sb 1:192 Те прегрешенья, что прежде свершили, о, город надменный,

\vs 5Sb 1:193 Мир будет видеть ваш плач  за то, что не чтили закона.

\vs 5Sb 1:194 Царь Эфиопов могучий разрушит город Сиену, 

\vs 5Sb 1:195 Силой Тевхиру населит народ темнокожий Индийцев.

\vs 5Sb 1:196 Слезы, Пентаполь, прольешь: тебя муж многомощный погубит.

\vs 5Sb 1:197 Скорбная Ливия, кто твои беды возьмется исчислить?

\vs 5Sb 1:198 Кто, Кирена, тебя среди смертных достойно оплачет?

\vs 5Sb 1:199 Смолкнут стенанья твои только в час ненавистной кончины.

\vs 5Sb 1:200 В земли Британцев и Галлов, богатых золотом, хлынет 

\vs 5Sb 1:201 Вод Океанских поток, от крови все полноводней. 

\vs 5Sb 1:202 Много ведь горя они доставили детям Господним, 

\vs 5Sb 1:203 В год, когда царь Финикийский в Сидон огромное войско 

\vs 5Sb 1:204 Галлов из Сирии вел. Саму тебя тоже погубит

\vs 5Sb 1:205 Он, Равенна, с собой твоих граждан ведя на убийство.

\vs 5Sb 1:206 Не заноситесь, Индийцы и храбрый народ Эфиопов! 

\vs 5Sb 1:207 Ибо когда колесо небесной оси, Козерога

\vs 5Sb 1:208 Звезды, Телец побегут вкруг центра в созвездии Братьев  

\vs 5Sb 1:209 Дева, на небо взойдя, и Солнце, крутясь непрерывно, 

\vs 5Sb 1:210 Их хоровод поведут по всему небесному своду 

\vs 5Sb 1:211 Будет тут страшный пожар, который охватит всю землю, 

\vs 5Sb 1:212 В битве небесных светил обновится природа, погибнет, 

\vs 5Sb 1:213 Плачем мир огласив, в огне страна Эфиопов!

\vs 5Sb 1:214 Плачь ты тоже, Коринф, над своею судьбою несчастной!

\vs 5Sb 1:215 Мойры когда, три сестры, прядущие нити витые, 

\vs 5Sb 1:216 Вспять беглеца поведут, который тайком с перешейка, 

\vs 5Sb 1:217 Горы минуя, бежал, чтобы снова явить его людям. 

\vs 5Sb 1:218 Кто однажды скалу разсек безудержной медью, 

\vs 5Sb 1:219 Тот сгубит землю твою, разорит, как назначено было,

\vs 5Sb 1:220 Ибо от Бога дана ему сила дерзнуть на такое,

\vs 5Sb 1:221 Что ни один из царей до него не отважился сделать. 

\vs 5Sb 1:222 Прежде всего, отделив от трех голов основанья, 

\vs 5Sb 1:223 Щедро позволит другим голов этих мяса отведать, 

\vs 5Sb 1:224 Так что пожрут они плоть родную царя-нечестивца.

\vs 5Sb 1:225 Прочих людей на земле ожидают убийство и ужас 

\vs 5Sb 1:226 Из-за великого Града и верного Богу народа, 

\vs 5Sb 1:227 Что был спасаем всегда, кого Провиденье избрало.

\vs 5Sb 1:228 Ветреный и безрассудный, в себя все несчастья вобравший, 

\vs 5Sb 1:229 Тяжких страданий исток и их наивысшая степень,

\vs 5Sb 1:230 Город, задержанный в росте, но Мойрами все же спасенный, 

\vs 5Sb 1:231 Дерзкий, зачинщик всех бед, великое горе народам  

\vs 5Sb 1:232 Кто пожелал в тебе жить? Живя в тебе, кто не страдал бы? 

\vs 5Sb 1:233 Кто из царей твоих пал, достойную жизнь завершая? 

\vs 5Sb 1:234 Все ты испортил, что мог, залив всякой мерзостью землю, 

\vs 5Sb 1:235 Мира прекрасные складки тобой изменили свой облик. 

\vs 5Sb 1:236 Может быть,  думаешь ты,  она ищет ссоры со мною? 

\vs 5Sb 1:237 Что за нелепость! Хочу вразумить и вот  упрекаю: 

\vs 5Sb 1:238 Некогда свет возсиял средь людей благодатного Солнца, 

\vs 5Sb 1:239 Лившего те же лучи, что и солнце древних пророков. 

\vs 5Sb 1:240 Мед стекал с языка  напиток сладчайший для смертных; 

\vs 5Sb 1:241 Он обвинял, объяснял  и день на земле продолжался. 

\vs 5Sb 1:242 Из-за того, что был Он  о источник тягчайших пороков!  

\vs 5Sb 1:243 Горе и войны с земли однажды навеки исчезнут. 

\vs 5Sb 1:244 Ты же, начало всех зол и их наивысшая степень, 

\vs 5Sb 1:245 Город, задержанный в росте, но Мойрами все же спасенный, 

\vs 5Sb 1:246 Горькому слову внемли, неприятному, смертных несчастье!

\vs 5Sb 1:247 Люди когда воевать на Персидской земле перестанут,

\vs 5Sb 1:248 Стоны покинут ее и голод, тогда появиться

\vs 5Sb 1:249 Должен в ней будет народ Иудеев блаженных, небесный.

\vs 5Sb 1:250 Он ее среднюю часть вкруг Божьего града заселит, 

\vs 5Sb 1:251 Стену соорудив великую вплоть до Иоппы, 

\vs 5Sb 1:252 Ту, что поднимется ввысь под самые темные тучи. 

\vs 5Sb 1:253 Больше труба никогда не издаст воинственный голос, 

\vs 5Sb 1:254 Люди от вражьей руки перестанут гибнуть в сраженьях

\vs 5Sb 1:255 И установят трофей победе над злом в этом мире.

\vs 5Sb 1:256 Муж на землю с небес сойдет, Кому равных не будет, 

\vs 5Sb 1:257 Руки раскинет Свои на древе, обильном плодами. 

\vs 5Sb 1:258 Лучший среди Иудеев, Он солнца бег остановит 

\vs 5Sb 1:259 Речью прекрасной, что с губ Его безупречных польется.

\vs 5Sb 1:260 Больше не нужно тебе скорбеть душою, блаженный, 

\vs 5Sb 1:261 Богом рожденный цветок, желанный для всех и богатый, 

\vs 5Sb 1:262 Свет благодатный, достойный исход вожделенный, святыня, 

\vs 5Sb 1:263 Город земли Иудейской прекрасный, возвышенный в гимнах! 

\vs 5Sb 1:264 В пляске безумной тебя попирать нечестивой стопою

\vs 5Sb 1:265 Эллины больше не будут, душой исзступленью отдавшись  

\vs 5Sb 1:266 Вместо того окружат почитанием дети Господни  

\vs 5Sb 1:267 Те, что воздвигнут алтарь при звуках священных напевов, 

\vs 5Sb 1:268 Богу многие жертвы неся и молясь непрерывно. 

\vs 5Sb 1:269 Все, кто прежде терпел мучения из-за гонений,

\vs 5Sb 1:270 Радостных дней череду теперь в утешенье получат  

\vs 5Sb 1:271 Те же, кто в небеса нечестиво хулу возносили, 

\vs 5Sb 1:272 Вдруг умолкнут, осыпав друг друга безсмысленной бранью. 

\vs 5Sb 1:273 Скроет в себе их земля, до тех пор пока мир существует. 

\vs 5Sb 1:274 Тут прольется из туч пылающий огненный ливень,

\vs 5Sb 1:275 С пашен отныне собрать не придется блестящих колосьев  

\vs 5Sb 1:276 Все незасеянным впредь и невспаханным будет, доколе 

\vs 5Sb 1:277 Власть не признают над миром Безсмертного, вечного Бога 

\vs 5Sb 1:278 Смертные люди и чтить не забудут земли порожденья  

\vs 5Sb 1:279 Коршунов, также собак, которых дал миру Египет,

\vs 5Sb 1:280 Суетно превозносить, утруждая глупые губы. 

\vs 5Sb 1:281 Родина благочестивых, святая земля принесет им 

\vs 5Sb 1:282 Струи медовые, что из скал и источников льются.

\vs 5Sb 1:283 К чистым душою тогда притечет молоко неземное  

\vs 5Sb 1:284 К тем, что надежды свои на Творца одного возложили, 

\vs 5Sb 1:285 Вышнего Бога, Ему принеся почитанье и веру.

\vs 5Sb 1:286 Ясный мне ум для чего велит поведать такое? 

\vs 5Sb 1:287 Бедная Азия, жалость к тебе мою душу терзает, 

\vs 5Sb 1:288 Скорбь о народе Карийцев, богатых Лидийцев, Ионян. 

\vs 5Sb 1:289 Сарды, увы вам! И вам увы, сердцу милые Траллы! 

\vs 5Sb 1:290 Лаодикия, увы! прекраснейший город  погубит 

\vs 5Sb 1:291 Землетрясение вас и в прах обратит ваши стены.

\vs 5Sb 1:292 В скорбной Азийской земле, в стране богатых Лидийцев 

\vs 5Sb 1:293 Храм Артемиды Эфесской падет под ударами бури; 

\vs 5Sb 1:294 Трещины в почве, толчки  и с берега в море сползет он. 

\vs 5Sb 1:295 Так заливают корабль в непогоду свирепые волны. 

\vs 5Sb 1:296 Навзничь упав, тут Эфес испустит вопль, орошая 

\vs 5Sb 1:297 Берег слезами. Искать будет храм он, что высился прежде.

\vs 5Sb 1:298 Гневом тогда распален, нерушимый небесный Владыка 

\vs 5Sb 1:299 Молнию с силой метнет в преступника из поднебесья 

\vs 5Sb 1:300 Вместо зимы в этот день наступит пора урожая. 

\vs 5Sb 1:301 После того на земле людей ожидают несчастья: 

\vs 5Sb 1:302 В высях Гремящий убьет до единого всех нечестивцев, 

\vs 5Sb 1:303 Громы и молнии в ход пустив, горящие стрелы, 

\vs 5Sb 1:304 Целые тучи врагов  и род истребит их настолько,

\vs 5Sb 1:305 Что мертвых тел на земле будет больше, чем мелких песчинок.

\vs 5Sb 1:306 Смирна тогда же придет своего Ликурга оплакать 

\vs 5Sb 1:307 Под стенами Эфеса и здесь сама же погибнет.

\vs 5Sb 1:308 Глупая Кима с ее священной божественной влагой

\vs 5Sb 1:309 Брошена в руки людей безбожных, неправедных, диких, 

\vs 5Sb 1:310 Впредь возносить в небеса не будет радостных песен,

\vs 5Sb 1:311 Но безжизненным телом в волнах прибрежных качаться.

\vs 5Sb 1:312 Те, кто останутся жить, заплачут в голос от горя.

\vs 5Sb 1:313 Будет им знак  по нему поймут, за что претерпели 

\vs 5Sb 1:314 Кимский злосчастный народ, стыда лишенное племя. 

\vs 5Sb 1:315 После, лишь только они сожженную землю оплачут,

\vs 5Sb 1:316 Лесбос навеки уйдет под воды реки Эридана.

\vs 5Sb 1:317 Горе тебе, Керкира прекрасная! Пляски прервешь ты, 

\vs 5Sb 1:318 И Иераполь, живущий в позорном союзе с богатством! 

\vs 5Sb 1:319 Что пожелал, обретешь, оплаканный многими город  

\vs 5Sb 1:320 Там, где течет Термодонт, засыпан землею ты будешь. 

\vs 5Sb 1:321 Триполь, возросший на скалах близ вод Меандра, который 

\vs 5Sb 1:322 Волнами ночью под берег быть смытым судьбою назначен! 

\vs 5Sb 1:323 До основанья тебя разрушит промысел Божий.

\vs 5Sb 1:324 Пусть не желаю я зла земле, что соседняя Фебу: 

\vs 5Sb 1:325 Пущенный с неба перун роскошный Милет уничтожит 

\vs 5Sb 1:326 Из-за того, что коварным он Феба песням поверил \ldots

\vs 5Sb 1:327 Благоразумный совет и о смертных людях забота.

\vs 5Sb 1:328 Смилуйся, мира Создатель, над тучной землей Иудейской, 

\vs 5Sb 1:329 Щедро несущей плоды, чтоб мы Твои помыслы знали! 

\vs 5Sb 1:330 Ибо Ты первой ее сотворил в Своей милости, Боже, 

\vs 5Sb 1:331 С тем, чтобы даром Твоим она для смертных явилась 

\vs 5Sb 1:332 И могла бы внимать всему, что ей Бог доверяет.

\vs 5Sb 1:333 Трижды несчастная, жажду я видеть творенья Фракийцев,

\vs 5Sb 1:334 Стену промеж двух морей, что вихрем несущейся пыли

\vs 5Sb 1:335 Совлечена, как поток в глубину устремится, где рыбы.

\vs 5Sb 1:336 О Геллеспонт разнесчастный! Тебя запряжет Ассириец, 

\vs 5Sb 1:337 Битва Фракийцев великую силу разрушит. 

\vs 5Sb 1:338 С войском Египетский царь Македонии земли захватит, 

\vs 5Sb 1:339 Варваров область низложит могущество власть предержащих. 

\vs 5Sb 1:340 Там Памфилийцы, Галаты, Лидийцы и Писидийцы 

\vs 5Sb 1:341 Вместе одержат победу, на грозную битву собравшись.

\vs 5Sb 1:342 Трижды несчастная, ляжешь, Италия, мертвой пустыней, 

\vs 5Sb 1:343 Змей доколе в твоей цветущей земле не издохнет.

\vs 5Sb 1:344 В высях заоблачных, в небе широком однажды раздастся 

\vs 5Sb 1:345 Грома раскат, призывая прислушаться к голосу Бога. 

\vs 5Sb 1:346 Больше не явятся миру лучи нетленные солнца, 

\vs 5Sb 1:347 Также сияющий свет луны навеки угаснет  

\vs 5Sb 1:348 В самом исходе времен, когда Божья исполнится воля.

\vs 5Sb 1:349 Тьма тут окутает мир, и мрак по земле расползется, 

\vs 5Sb 1:350 Страшные звери на ней, ослепшие люди и горе.

\vs 5Sb 1:351 Долго продлится тот день, и смертные Бога узнают 

\vs 5Sb 1:352 Сущего на небесах Владыку, чье око всезряще.

\vs 5Sb 1:353 Не пожалеет тогда Он врагов Своих, но уничтожит 

\vs 5Sb 1:354 Тех, что баранов, овец, быков стада и мычащих 

\vs 5Sb 1:355 Телок золоторогих и тучных в жертву приносят

\vs 5Sb 1:356 Гермам бездушным, камням, из которых сделаны боги.

\vs 5Sb 1:357 Мудрый пусть торжествует закон и праведных слава!

\vs 5Sb 1:358 Чтобы нетленный Господь, разгневавшись, смерти не предал

\vs 5Sb 1:359 Весь человеческий род нечестивый, безстыдное племя, 

\vs 5Sb 1:360 Нужно Создателя чтить  Безсмертного Вечного Бога.

\vs 5Sb 1:361 В самом исходе времен, лунный свет когда потускнеет, 

\vs 5Sb 1:362 Мир безумство войны охватит, коварной и подлой.

\vs 5Sb 1:363 С края земли человек придет, что на мать покусился, 

\vs 5Sb 1:364 Бегством спасаясь и в сердце своем замышляя дурное. 

\vs 5Sb 1:365 Он всю землю захватит, и все ему станет подвластно,

\vs 5Sb 1:366 В самые тайные мысли людей он свободно проникнет,

\vs 5Sb 1:367 Из-за которой умрет, саму он, вернувшись, погубит.

\vs 5Sb 1:368 Многих мужей истребит, в том числе и великих тиранов, 

\vs 5Sb 1:369 Всех он огнем будет жечь, что никто доселе не делал, 

\vs 5Sb 1:370 Павших снова подняться, ревнуя к Богу, заставит.

\vs 5Sb 1:371 С Запада будет война грозить великая людям, 

\vs 5Sb 1:372 Крови потоки стекут с берегов в полноводные реки, 

\vs 5Sb 1:373 Желчь будет капать по капле в долинах земли Македонской \ldots

\vs 5Sb 1:374 Помощь с Заката придет, придет и смерть властелину. 

\vs 5Sb 1:375 И вот тогда по земле подует ветер холодный,

\vs 5Sb 1:376 Снова жестокой войной наполнится поле сражений.

\vs 5Sb 1:377 С неба на смертных людей прольется огненный ливень,

\vs 5Sb 1:378 Пламя, кровь и вода, блеск молний, тьма и мрак ночи.

\vs 5Sb 1:379 В битве настигшая смерть, резня под покровом тумана 

\vs 5Sb 1:380 Всех уничтожат царей  а с ними воинов лучших.

\vs 5Sb 1:381 Так прекратится война, и стихнет жуткая бойня.

\vs 5Sb 1:382 Больше никто за мечи и железо рукой не возьмется,

\vs 5Sb 1:383 Копий не тронет никто, что будут теперь под запретом.

\vs 5Sb 1:384 Мир тут получит народ разумный, в живых кто остался, 

\vs 5Sb 1:385 Выдержав пробу войной, чтоб радость вкушать беззаботно.

\vs 5Sb 1:386 Мать кто убил, откажитесь от дерзкой преступной отваги! 

\vs 5Sb 1:387 Те, кто на ложе свое нечестиво детей возводили, 

\vs 5Sb 1:388 И превращали в блудниц под кровом своим непорочных 

\vs 5Sb 1:389 Силой и страхом расправы, разнузданным, наглым безстыдством \ldots

\vs 5Sb 1:390 Мать с порожденьем своим смешалась в тебе беззаконно, 

\vs 5Sb 1:391 Дочь с породившим ее позорный союз заключала, 

\vs 5Sb 1:392 Пачкали в стенах твоих цари покорные губы, 

\vs 5Sb 1:393 Ложе делить со скотом искали в тебе нечестивцы. 

\vs 5Sb 1:394 Смолкни же, мерзостный город, жалчайший, средь праздников шумных,

\vs 5Sb 1:395 Ибо уже никогда горящей легко древесины 

\vs 5Sb 1:396 Чистые девы огонь священный в тебе не увидят. 

\vs 5Sb 1:397 Дом, извечно желанный, с тобою погас, когда снова 

\vs 5Sb 1:398 Видеть мне довелось, как падает он под ударом, 

\vs 5Sb 1:399 Весь охвачен огнем, сраженный рукой нечестивой 

\vs 5Sb 1:400 Вечно цветущий предел, хранящее Бога жилище. 

\vs 5Sb 1:401 Храм, что святыми построен и будет стоять нерушимо, 

\vs 5Sb 1:402 Тот, кому телом и духом поверили смертные люди.

\vs 5Sb 1:403 Он не начал бездумно заморскому богу молиться 

\vs 5Sb 1:404 И его из камней высекать, премудрый строитель.

\vs 5Sb 1:405 Также и золота блеск не чтил он  для душ обольщенье: 

\vs 5Sb 1:406 Богу, вдохнувшему жизнь в тела, Создателю мира 

\vs 5Sb 1:407 Издавна в жертву они овец и быков приносили. 

\vs 5Sb 1:408 Ныне же царь, что пришел невидимым, страшный преступник, 

\vs 5Sb 1:409 Всю их страну разорил и лежать в запустенье оставил,

\vs 5Sb 1:410 С войском явившись большим, с мужами, отважными духом.

\vs 5Sb 1:411 Сам он, на землю вступив безсмертную, жизни лишился. 

\vs 5Sb 1:412 Больше явлено людям такого не было знака, 

\vs 5Sb 1:413 Что и другие придут великий город разрушить.

\vs 5Sb 1:414 Муж с высоких небес сошел блаженный на землю, 

\vs 5Sb 1:415 Руки скиптр держали, что Бог ему вечный доверил. 

\vs 5Sb 1:416 Мощью он всех превзошел и тем справедливо богатство 

\vs 5Sb 1:417 Роздал, кто праведно жил  а прежние лишь отбирали. 

\vs 5Sb 1:418 Все он сжигал города и до основания рушил,

\vs 5Sb 1:419 Жег жилища людей, творивших когда-то злодейства.

\vs 5Sb 1:420 Город же, избранный Богом, блестеть заставил он ярко  

\vs 5Sb 1:421 Ярче сияющих звезд на небе и солнца с луною  

\vs 5Sb 1:422 Пышно украсил, и храм в нем Богу священный поставил, 

\vs 5Sb 1:423 В камень одетый, прекрасный, каких не бывало доселе. 

\vs 5Sb 1:424 Стену построил вокруг на много стадиев, в небо

\vs 5Sb 1:425 Что уходила и туч касалась, видна отовсюду  

\vs 5Sb 1:426 Так что могли созерцать все люди праведной веры 

\vs 5Sb 1:427 Славу Безсмертного Бога, давно желанное чудо. 

\vs 5Sb 1:428 Солнца восход и закат пропели Ему свои гимны. 

\vs 5Sb 1:429 Злу среди рода людского отныне места не будет:

\vs 5Sb 1:430 В браке изменам, с детьми не дозволенным Богом сношеньям,

\vs 5Sb 1:431 Смертоубийству, вражде  лишь законному единоборству. 

\vs 5Sb 1:432 Праведных время придет в конце, тогда и исполнит 

\vs 5Sb 1:433 Все это Бог-Громовержец, Строитель великого храма.

\vs 5Sb 1:434 Горе тебе, Вавилон, златотронный и златообутый, 

\vs 5Sb 1:435 Древний царский чертог, один управляющий миром!

\vs 5Sb 1:436 Тот, что некогда был великим и властным  ты больше,

\vs 5Sb 1:437 Город, в горах золотых у вод Евфрата не ляжешь.

\vs 5Sb 1:438 Но по земле распростершись в смятенье подземных ударов,

\vs 5Sb 1:439 Лишь под властью Парфян всем миром потом овладеешь. 

\vs 5Sb 1:440 Попридержи свой язык, нечестивый потомок Халдеев!

\vs 5Sb 1:441 Слов понапрасну не трать на то, как Персами править

\vs 5Sb 1:442 Станешь, Мидийцами как: ведь и власть, что имел, получил ты,

\vs 5Sb 1:443 Риму заложника дав и Азийских наемников выслав.

\vs 5Sb 1:444 Вот потому-то пойдешь, расчетливый царь, ты в Афины 

\vs 5Sb 1:445 Для выяснения цели: зачем посылал, дескать, выкуп.

\vs 5Sb 1:446 Вместо неискренних слов врагам свой гнев ты покажешь.

\vs 5Sb 1:447 В самом исходе времен однажды высохнет море, 

\vs 5Sb 1:448 Так что не смогут приплыть корабли к берегам Италийским. 

\vs 5Sb 1:449 Азия станет тогда, напротив, водой животворной, 

\vs 5Sb 1:450 То же и Крит. Много бед испытать тут придется и Кипру: 

\vs 5Sb 1:451 Пафос оплакивать будет печальный свой жребий, узнают 

\vs 5Sb 1:452 Все о судьбе Саламина, который постигло несчастье. 

\vs 5Sb 1:453 Больше плодов приносить не станет земля побережья, 

\vs 5Sb 1:454 Мощный набег саранчи погубит страну Киприотов.

\vs 5Sb 1:455 Будете плакать вы, глядя ни Тир, злополучные люди! 

\vs 5Sb 1:456 Гнев тебя ждет, Финикия, ужасный, доколе не рухнешь 

\vs 5Sb 1:457 Тяжко на землю  могли чтобы искренне плакать Сирены.

\vs 5Sb 1:458 В пятом колене людском, когда беды Египта отступят, 

\vs 5Sb 1:459 И цари Египтян друг с другом безстыдно мешаться 

\vs 5Sb 1:460 Станут, в Египте взойдут на трон Памфилийцев потомки. 

\vs 5Sb 1:461 У Македонцев тогда, и в Азии, и у Ликийцев 

\vs 5Sb 1:462 Ужас кровавой войны покроет все пылью и прахом. 

\vs 5Sb 1:463 Римский прервет его царь с владыками Запада вместе.

\vs 5Sb 1:464 Только лишь ветер холодный и снег приносящий подует,

\vs 5Sb 1:465 Только покроются льдом большая река и озера  

\vs 5Sb 1:466 Варварский тотчас народ устремится в Азийскую землю, 

\vs 5Sb 1:467 Словно безсильных, погубит он грозное племя Фракийцев. 

\vs 5Sb 1:468 Будут отцов тут своих поедать несчастные люди, 

\vs 5Sb 1:469 Мучимы голодом, в пищу себе их мясо готовить. 

\vs 5Sb 1:470 Звери же будут кормиться, беря из людского жилища  

\vs 5Sb 1:471 С птицами вместе, они всех смертных людей уничтожат. 

\vs 5Sb 1:472 В ходе жестокой войны прибудет воды в Океане, 

\vs 5Sb 1:473 Примет кровавый он цвет от тел и крови безумцев. 

\vs 5Sb 1:474 В это же время земля настолько уже истощится, 

\vs 5Sb 1:475 Что можно будет в уме мужчин перечислить и женщин.

\vs 5Sb 1:476 Жалкое племя в конце испустит страшные вопли, 

\vs 5Sb 1:477 В час, когда солнце зайдет, чтобы больше уже не подняться, 

\vs 5Sb 1:478 Но, в океанской воде оставаясь, очиститься ею  

\vs 5Sb 1:479 Ибо многих людей пришлось ему видеть нечестье. 

\vs 5Sb 1:480 Темная ночь без луны по небу тогда разольется,

\vs 5Sb 1:481 Мгла, какой прежде не знали, окутает складки земные. 

\vs 5Sb 1:482 Снова, однако, дорогу потом свет Божий укажет 

\vs 5Sb 1:483 Праведным людям, что в гимнах поспели Великого Бога.

\vs 5Sb 1:484 Ты, несчастная трижды Изида! У Нильских потоков 

\vs 5Sb 1:485 Сядешь одна, как менада немая у вод Ахеронта, 

\vs 5Sb 1:486 Память сама о тебе скоро жить на земле перестанет. 

\vs 5Sb 1:487 Ты же, Серапис, мученья претерпишь на каменном ложе, 

\vs 5Sb 1:488 В трижды несчастном Египте руиной падешь величайшей. 

\vs 5Sb 1:489 Все, что тянулись к тебе в стране Египтян, будут скоро

\vs 5Sb 1:490 Плакать, а те, кто вложил в свое сердце разум нетленный, 

\vs 5Sb 1:491 Бога кто в гимнах воспел, поймут, что ты вовсе ничтожен.

\vs 5Sb 1:492 Скажет один из жрецов, одетый в льняные одежды: 

\vs 5Sb 1:493 Люди, построим святыню в честь истинно Сущего Бога! 

\vs 5Sb 1:494 Люди, ужасный обычай, от предков идущий, изменим 

\vs 5Sb 1:495 Тот, по которому деды богам из глины и камня, 

\vs 5Sb 1:496 Шествия, жертвы, обряды творя, потеряли разсудок. 

\vs 5Sb 1:497 Несокрушимого Бога возславив, душой обратимся, 

\vs 5Sb 1:498 Люди, к Нему Самому  Создателю, Сущему вечно, 

\vs 5Sb 1:499 Кто всеми правит, Царю, справедливому мира Владыке,

\vs 5Sb 1:500 Душ Кормильцу, Отцу, Великому, Вечно Живому! 

\vs 5Sb 1:501 Так возведен будет храм в Египте, великий, священный; 

\vs 5Sb 1:502 Жертвы к нему понесет народ, наставленный Богом,  

\vs 5Sb 1:503 Те, кому вечную жизнь Господь на земле уготовил.

\vs 5Sb 1:504 Но лишь только уйдут Эфиопы от дерзких Трибаллов 

\vs 5Sb 1:505 И вознамерятся сами в Египте распахивать земли, 

\vs 5Sb 1:506 Зло они станут творить, чтобы гибель вселенной ускорить, 

\vs 5Sb 1:507 Наземь повергнут и храм великий в Египетском царстве. 

\vs 5Sb 1:508 Бог же за это на них Свой гнев ужасный обрушит, 

\vs 5Sb 1:509 Гибель тем самым неся преступникам и нечестивцам. 

\vs 5Sb 1:510 Больше никто в той земле уже не получит пощады, 

\vs 5Sb 1:511 Ибо сберечь не смогли того, что Господь им доверил.

\vs 5Sb 1:512 Видела я среди звезд сверкавшего Солнца угрозу,

\vs 5Sb 1:513 Гнев Луны величайший при свете блещущих молний.

\vs 5Sb 1:514 Звезды родили войну  Господь повелел им сражаться. 

\vs 5Sb 1:515 Вместо Солнца вовсю бушевало огромное пламя,

\vs 5Sb 1:516 Лунный двурогий изгиб потерял свою прежнюю форму.

\vs 5Sb 1:517 В битву вступила Венера, ко Льву на спину взобравшись;

\vs 5Sb 1:518 Прямо в загривок Тельца Козерог молодого ударил,

\vs 5Sb 1:519 Тот же за это лишил Козерога надежд на спасенье; 

\vs 5Sb 1:520 Дольше на небе сиять Орион Весам не позволил;

\vs 5Sb 1:521 Дева судьбу Близнецов в созвездье Овна изменила;

\vs 5Sb 1:522 Звезды Плеяд не взошли  их пояс Дракон уничтожил;

\vs 5Sb 1:523 В панцирь созвездия Льва наносить стали Рыбы удары;

\vs 5Sb 1:524 Рак не сумел устоять, боясь больше всех Ориона; 

\vs 5Sb 1:525 Встал на свой хвост Скорпион, перед Львом робея ужасным;

\vs 5Sb 1:526 Пес помчался стремглав от огня палящего Солнца;

\vs 5Sb 1:527 Гнев большого Светила заставил пылать Водолея. 

\vs 5Sb 1:528 Начал трястись Небосвод, пока не стряхнул воевавших. 

\vs 5Sb 1:529 Сильно разгневавшись, он с высоты на землю их бросил, 

\vs 5Sb 1:530 Так что, стремительно вниз в океанские воды сорвавшись, 

\vs 5Sb 1:531 Землю спалили огнем, а небо лишилось созвездий.

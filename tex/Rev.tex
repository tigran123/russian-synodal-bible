\bibbookdescr{Rev}{
  inline={Откровение\fns{Апокалипсис (греч.).}\\\LARGE Святого Иоанна Богослова},
  toc={Откровение},
  bookmark={Откровение},
  header={Откровение},
  %headerleft={},
  %headerright={},
  abbr={Откр}
}
\vs Rev 1:1 Откровение Иисуса Христа, которое дал Ему Бог, чтобы показать рабам Своим, чему надлежит быть вскоре. И Он показал, послав \bibemph{оное} через Ангела Своего рабу Своему Иоанну,
\vs Rev 1:2 который свидетельствовал слово Божие и свидетельство Иисуса Христа и что он видел.
\vs Rev 1:3 Блажен читающий и слушающие слова пророчества сего и соблюдающие написанное в нем; ибо время близко.
\rsbpar\vs Rev 1:4 Иоанн семи церквам, находящимся в Асии: благодать вам и мир от Того, Который есть и был и грядет, и от семи духов, находящихся перед престолом Его,
\vs Rev 1:5 и от Иисуса Христа, Который есть свидетель верный, первенец из мертвых и владыка царей земных. Ему, возлюбившему нас и омывшему нас от грехов наших Кровию Своею
\vs Rev 1:6 и соделавшему нас царями и священниками Богу и Отцу Своему, слава и держава во веки веков, аминь.
\vs Rev 1:7 Се, грядет с облаками, и узрит Его всякое око и те, которые пронзили Его; и возрыдают пред Ним все племена земные. Ей, аминь.
\rsbpar\vs Rev 1:8 Я есмь Альфа и Омега, начало и конец, говорит Господь, Который есть и был и грядет, Вседержитель.
\rsbpar\vs Rev 1:9 Я, Иоанн, брат ваш и соучастник в скорби и в царствии и в терпении Иисуса Христа, был на острове, называемом Патмос, за слово Божие и за свидетельство Иисуса Христа.
\vs Rev 1:10 Я был в духе в день воскресный, и слышал позади себя громкий голос, как бы трубный, который говорил: Я есмь Альфа и Омега, Первый и Последний;
\vs Rev 1:11 то, что видишь, напиши в книгу и пошли церквам, находящимся в Асии: в Ефес, и в Смирну, и в Пергам, и в Фиатиру, и в Сардис, и в Филадельфию, и в Лаодикию.
\vs Rev 1:12 Я обратился, чтобы увидеть, чей голос, говоривший со мною; и обратившись, увидел семь золотых светильников
\vs Rev 1:13 и, посреди семи светильников, подобного Сыну Человеческому, облеченного в подир\fns{Подир~--- длинная одежда Иудейских первосвященников и царей.} и по персям опоясанного золотым поясом:
\vs Rev 1:14 глава Его и волосы белы, как белая в\acc{о}лна, как снег; и очи Его, как пламень огненный;
\vs Rev 1:15 и ноги Его подобны халколивану, как раскаленные в печи, и голос Его, как шум вод многих.
\vs Rev 1:16 Он держал в деснице Своей семь звезд, и из уст Его выходил острый с обеих сторон меч; и лице Его, как солнце, сияющее в силе своей.
\vs Rev 1:17 И когда я увидел Его, то пал к ногам Его, как мертвый. И Он положил на меня десницу Свою и сказал мне: не бойся; Я есмь Первый и Последний,
\vs Rev 1:18 и живый; и был мертв, и се, жив во веки веков, аминь; и имею ключи ада и смерти.
\vs Rev 1:19 Итак напиши, что ты видел, и что есть, и что будет после сего.
\vs Rev 1:20 Тайна семи звезд, которые ты видел в деснице Моей, и семи золотых светильников \bibemph{есть сия}: семь звезд суть Ангелы семи церквей; а семь светильников, которые ты видел, суть семь церквей.
\vs Rev 2:1 Ангелу Ефесской церкви напиши: так говорит Держащий семь звезд в деснице Своей, Ходящий посреди семи золотых светильников:
\vs Rev 2:2 знаю дела твои, и труд твой, и терпение твое, и то, что ты не можешь сносить развратных, и испытал тех, которые называют себя апостолами, а они не таковы, и нашел, что они лжецы;
\vs Rev 2:3 ты много переносил и имеешь терпение, и для имени Моего трудился и не изнемогал.
\vs Rev 2:4 Но имею против тебя то, что ты оставил первую любовь твою.
\vs Rev 2:5 Итак вспомни, откуда ты ниспал, и покайся, и твори прежние дела; а если не так, скоро приду к тебе, и сдвину светильник твой с места его, если не покаешься.
\vs Rev 2:6 Впрочем то в тебе \bibemph{хорошо}, что ты ненавидишь дела Николаитов, которые и Я ненавижу.
\vs Rev 2:7 Имеющий ухо да слышит, что Дух говорит церквам: побеждающему дам вкушать от древа жизни, которое посреди рая Божия.
\rsbpar\vs Rev 2:8 И Ангелу Смирнской церкви напиши: так говорит Первый и Последний, Который был мертв, и се, жив:
\vs Rev 2:9 знаю твои дела, и скорбь, и нищету (впрочем ты богат), и злословие от тех, которые говорят о себе, что они Иудеи, а они не таковы, но сборище сатанинское.
\vs Rev 2:10 Не бойся ничего, что тебе надобно будет претерпеть. Вот, диавол будет ввергать из среды вас в темницу, чтобы искусить вас, и будете иметь скорбь дней десять. Будь верен до смерти, и дам тебе венец жизни.
\vs Rev 2:11 Имеющий ухо (слышать) да слышит, что Дух говорит церквам: побеждающий не потерпит вреда от второй смерти.
\rsbpar\vs Rev 2:12 И Ангелу Пергамской церкви напиши: так говорит Имеющий острый с обеих сторон меч:
\vs Rev 2:13 знаю твои дела, и что ты живешь там, где престол сатаны, и что содержишь имя Мое, и не отрекся от веры Моей даже в те дни, в которые у вас, где живет сатана, умерщвлен верный свидетель Мой Антипа.
\vs Rev 2:14 Но имею немного против тебя, потому что есть у тебя там держащиеся учения Валаама, который научил Валака ввести в соблазн сынов Израилевых, чтобы они ели идоложертвенное и любодействовали.
\vs Rev 2:15 Так и у тебя есть держащиеся учения Николаитов, которое Я ненавижу.
\vs Rev 2:16 Покайся; а если не так, скоро приду к тебе и сражусь с ними мечом уст Моих.
\vs Rev 2:17 Имеющий ухо (слышать) да слышит, что Дух говорит церквам: побеждающему дам вкушать сокровенную манну, и дам ему белый камень и на камне написанное новое имя, которого никто не знает, кроме того, кто получает.
\rsbpar\vs Rev 2:18 И Ангелу Фиатирской церкви напиши: так говорит Сын Божий, у Которого очи, как пламень огненный, и ноги подобны халколивану:
\vs Rev 2:19 знаю твои дела и любовь, и служение, и веру, и терпение твое, и то, что последние дела твои больше первых.
\vs Rev 2:20 Но имею немного против тебя, потому что ты попускаешь жене Иезавели, называющей себя пророчицею, учить и вводить в заблуждение рабов Моих, любодействовать и есть идоложертвенное.
\vs Rev 2:21 Я дал ей время покаяться в любодеянии ее, но она не покаялась.
\vs Rev 2:22 Вот, Я повергаю ее на одр и любодействующих с нею в великую скорбь, если не покаются в делах своих.
\vs Rev 2:23 И детей ее поражу смертью, и уразумеют все церкви, что Я есмь испытующий сердца и внутренности; и воздам каждому из вас по делам вашим.
\vs Rev 2:24 Вам же и прочим, находящимся в Фиатире, которые не держат сего учения и которые не знают так называемых глубин сатанинских, сказываю, что не наложу на вас иного бремени;
\vs Rev 2:25 только то, что имеете, держите, пока приду.
\vs Rev 2:26 Кто побеждает и соблюдает дела Мои до конца, тому дам власть над язычниками,
\vs Rev 2:27 и будет пасти их жезлом железным; как сосуды глиняные, они сокрушатся, как и Я получил \bibemph{власть} от Отца Моего;
\vs Rev 2:28 и дам ему звезду утреннюю.
\vs Rev 2:29 Имеющий ухо (слышать) да слышит, что Дух говорит церквам.
\vs Rev 3:1 И Ангелу Сардийской церкви напиши: так говорит Имеющий семь духов Божиих и семь звезд: знаю твои дела; ты носишь имя, будто жив, но ты мертв.
\vs Rev 3:2 Бодрствуй и утверждай прочее близкое к смерти; ибо Я не нахожу, чтобы дела твои были совершенны пред Богом Моим.
\vs Rev 3:3 Вспомни, что ты принял и слышал, и храни и покайся. Если же не будешь бодрствовать, то Я найду на тебя, как тать, и ты не узнаешь, в который час найду на тебя.
\vs Rev 3:4 Впрочем у тебя в Сардисе есть несколько человек, которые не осквернили одежд своих, и будут ходить со Мною в белых \bibemph{одеждах}, ибо они достойны.
\vs Rev 3:5 Побеждающий облечется в белые одежды; и не изглажу имени его из книги жизни, и исповедаю имя его пред Отцем Моим и пред Ангелами Его.
\vs Rev 3:6 Имеющий ухо да слышит, что Дух говорит церквам.
\rsbpar\vs Rev 3:7 И Ангелу Филадельфийской церкви напиши: так говорит Святый, Истинный, имеющий ключ Давидов, Который отворяет~--- и никто не затворит, затворяет~--- и никто не отворит:
\vs Rev 3:8 знаю твои дела; вот, Я отворил перед тобою дверь, и никто не может затворить ее; ты не много имеешь силы, и сохранил слово Мое, и не отрекся имени Моего.
\vs Rev 3:9 Вот, Я сделаю, что из сатанинского сборища, из тех, которые говорят о себе, что они Иудеи, но не суть таковы, а лгут,~--- вот, Я сделаю то, что они придут и поклонятся пред ногами твоими, и познают, что Я возлюбил тебя.
\vs Rev 3:10 И как ты сохранил слово терпения Моего, то и Я сохраню тебя от годины искушения, которая придет на всю вселенную, чтобы испытать живущих на земле.
\vs Rev 3:11 Се, гряду скоро; держи, что имеешь, дабы кто не восхитил венца твоего.
\vs Rev 3:12 Побеждающего сделаю столпом в храме Бога Моего, и он уже не выйдет вон; и напишу на нем имя Бога Моего и имя града Бога Моего, нового Иерусалима, нисходящего с неба от Бога Моего, и имя Мое новое.
\vs Rev 3:13 Имеющий ухо да слышит, что Дух говорит церквам.
\rsbpar\vs Rev 3:14 И Ангелу Лаодикийской церкви напиши: так говорит Аминь, свидетель верный и истинный, начало создания Божия:
\vs Rev 3:15 знаю твои дела; ты ни холоден, ни горяч; о, если бы ты был холоден, или горяч!
\vs Rev 3:16 Но, как ты тепл, а не горяч и не холоден, то извергну тебя из уст Моих.
\vs Rev 3:17 Ибо ты говоришь: <<я богат, разбогател и ни в чем не имею нужды>>; а не знаешь, что ты несчастен, и жалок, и нищ, и слеп, и наг.
\vs Rev 3:18 Советую тебе купить у Меня золото, огнем очищенное, чтобы тебе обогатиться, и белую одежду, чтобы одеться и чтобы не видна была срамота наготы твоей, и глазною мазью помажь глаза твои, чтобы видеть.
\vs Rev 3:19 Кого Я люблю, тех обличаю и наказываю. Итак будь ревностен и покайся.
\vs Rev 3:20 Се, стою у двери и стучу: если кто услышит голос Мой и отворит дверь, войду к нему, и буду вечерять с ним, и он со Мною.
\vs Rev 3:21 Побеждающему дам сесть со Мною на престоле Моем, как и Я победил и сел с Отцем Моим на престоле Его.
\vs Rev 3:22 Имеющий ухо да слышит, что Дух говорит церквам.
\vs Rev 4:1 После сего я взглянул, и вот, дверь отверста на небе, и прежний голос, который я слышал как бы звук трубы, говоривший со мною, сказал: взойди сюда, и покажу тебе, чему надлежит быть после сего.
\vs Rev 4:2 И тотчас я был в духе; и вот, престол стоял на небе, и на престоле был Сидящий;
\vs Rev 4:3 и Сей Сидящий видом был подобен камню яспису и сардису; и радуга вокруг престола, видом подобная смарагду.
\vs Rev 4:4 И вокруг престола двадцать четыре престола; а на престолах видел я сидевших двадцать четыре старца, которые облечены были в белые одежды и имели на головах своих золотые венцы.
\vs Rev 4:5 И от престола исходили молнии и громы и гласы, и семь светильников огненных горели перед престолом, которые суть семь духов Божиих;
\vs Rev 4:6 и перед престолом море стеклянное, подобное кристаллу; и посреди престола и вокруг престола четыре животных, исполненных очей спереди и сзади.
\vs Rev 4:7 И первое животное было подобно льву, и второе животное подобно тельцу, и третье животное имело лице, как человек, и четвертое животное подобно орлу летящему.
\vs Rev 4:8 И каждое из четырех животных имело по шести крыл вокруг, а внутри они исполнены очей; и ни днем, ни ночью не имеют покоя, взывая: свят, свят, свят Господь Бог Вседержитель, Который был, есть и грядет.
\vs Rev 4:9 И когда животные воздают славу и честь и благодарение Сидящему на престоле, Живущему во веки веков,
\vs Rev 4:10 тогда двадцать четыре старца падают пред Сидящим на престоле, и поклоняются Живущему во веки веков, и полагают венцы свои перед престолом, говоря:
\vs Rev 4:11 достоин Ты, Господи, приять славу и честь и силу: ибо Ты сотворил все, и \bibemph{все} по Твоей воле существует и сотворено.
\vs Rev 5:1 И видел я в деснице у Сидящего на престоле книгу, написанную внутри и отвне, запечатанную семью печатями.
\vs Rev 5:2 И видел я Ангела сильного, провозглашающего громким голосом: кто достоин раскрыть сию книгу и снять печати ее?
\vs Rev 5:3 И никто не мог, ни на небе, ни на земле, ни под землею, раскрыть сию книгу, ни посмотреть в нее.
\vs Rev 5:4 И я много плакал о том, что никого не нашлось достойного раскрыть и читать сию книгу, и даже посмотреть в нее.
\vs Rev 5:5 И один из старцев сказал мне: не плачь; вот, лев от колена Иудина, корень Давидов, победил, \bibemph{и может} раскрыть сию книгу и снять семь печатей ее.
\vs Rev 5:6 И я взглянул, и вот, посреди престола и четырех животных и посреди старцев стоял Агнец как бы закланный, имеющий семь рогов и семь очей, которые суть семь духов Божиих, посланных во всю землю.
\vs Rev 5:7 И Он пришел и взял книгу из десницы Сидящего на престоле.
\vs Rev 5:8 И когда Он взял книгу, тогда четыре животных и двадцать четыре старца пали пред Агнцем, имея каждый гусли и золотые чаши, полные фимиама, которые суть молитвы святых.
\vs Rev 5:9 И поют новую песнь, говоря: достоин Ты взять книгу и снять с нее печати, ибо Ты был заклан, и Кровию Своею искупил нас Богу из всякого колена и языка, и народа и племени,
\vs Rev 5:10 и соделал нас царями и священниками Богу нашему; и мы будем царствовать на земле.
\vs Rev 5:11 И я видел, и слышал голос многих Ангелов вокруг престола и животных и старцев, и число их было тьмы тем и тысячи тысяч,
\vs Rev 5:12 которые говорили громким голосом: достоин Агнец закланный принять силу и богатство, и премудрость и крепость, и честь и славу и благословение.
\vs Rev 5:13 И всякое создание, находящееся на небе и на земле, и под землею, и на море, и все, что в них, слышал я, говорило: Сидящему на престоле и Агнцу благословение и честь, и слава и держава во веки веков.
\vs Rev 5:14 И четыре животных говорили: аминь. И двадцать четыре старца пали и поклонились Живущему во веки веков.
\vs Rev 6:1 И я видел, что Агнец снял первую из семи печатей, и я услышал одно из четырех животных, говорящее как бы громовым голосом: иди и смотри.
\vs Rev 6:2 Я взглянул, и вот, конь белый, и на нем всадник, имеющий лук, и дан был ему венец; и вышел он \bibemph{как} победоносный, и чтобы победить.
\rsbpar\vs Rev 6:3 И когда он снял вторую печать, я слышал второе животное, говорящее: иди и смотри.
\vs Rev 6:4 И вышел другой конь, рыжий; и сидящему на нем дано взять мир с земли, и чтобы убивали друг друга; и дан ему большой меч.
\rsbpar\vs Rev 6:5 И когда Он снял третью печать, я слышал третье животное, говорящее: иди и смотри. Я взглянул, и вот, конь вороной, и на нем всадник, имеющий меру в руке своей.
\vs Rev 6:6 И слышал я голос посреди четырех животных, говорящий: хиникс\fns{Хиникс~--- малая хлебная мера.} пшеницы за динарий\fns{Динарий~--- монета, соответствующая дневной плате поденщику.}, и три хиникса ячменя за динарий; елея же и вина не повреждай.
\rsbpar\vs Rev 6:7 И когда Он снял четвертую печать, я слышал голос четвертого животного, говорящий: иди и смотри.
\vs Rev 6:8 И я взглянул, и вот, конь бледный, и на нем всадник, которому имя <<смерть>>; и ад следовал за ним; и дана ему власть над четвертою частью земли~--- умерщвлять мечом и голодом, и мором и зверями земными.
\rsbpar\vs Rev 6:9 И когда Он снял пятую печать, я увидел под жертвенником души убиенных за слово Божие и за свидетельство, которое они имели.
\vs Rev 6:10 И возопили они громким голосом, говоря: доколе, Владыка Святый и Истинный, не судишь и не мстишь живущим на земле за кровь нашу?
\vs Rev 6:11 И даны были каждому из них одежды белые, и сказано им, чтобы они успокоились еще на малое время, пока и сотрудники их и братья их, которые будут убиты, как и они, дополнят число.
\rsbpar\vs Rev 6:12 И когда Он снял шестую печать, я взглянул, и вот, произошло великое землетрясение, и солнце стало мрачно как власяница, и луна сделалась как кровь.
\vs Rev 6:13 И звезды небесные пали на землю, как смоковница, потрясаемая сильным ветром, роняет незрелые смоквы свои.
\vs Rev 6:14 И небо скрылось, свившись как свиток; и всякая гора и остров двинулись с мест своих.
\vs Rev 6:15 И цари земные, и вельможи, и богатые, и тысяченачальники, и сильные, и всякий раб, и всякий свободный скрылись в пещеры и в ущелья гор,
\vs Rev 6:16 и говорят горам и камням: падите на нас и сокройте нас от лица Сидящего на престоле и от гнева Агнца;
\vs Rev 6:17 ибо пришел великий день гнева Его, и кто может устоять?
\vs Rev 7:1 И после сего видел я четырех Ангелов, стоящих на четырех углах земли, держащих четыре ветра земли, чтобы не дул ветер ни на землю, ни на море, ни на какое дерево.
\vs Rev 7:2 И видел я иного Ангела, восходящего от востока солнца и имеющего печать Бога живаго. И воскликнул он громким голосом к четырем Ангелам, которым дано вредить земле и морю, говоря:
\vs Rev 7:3 не делайте вреда ни земле, ни морю, ни деревам, доколе не положим печати на челах рабов Бога нашего.
\vs Rev 7:4 И я слышал число запечатленных: запечатленных было сто сорок четыре тысячи из всех колен сынов Израилевых.
\vs Rev 7:5 Из колена Иудина запечатлено двенадцать тысяч; из колена Рувимова запечатлено двенадцать тысяч; из колена Гадова запечатлено двенадцать тысяч;
\vs Rev 7:6 из колена Асирова запечатлено двенадцать тысяч; из колена Неффалимова запечатлено двенадцать тысяч; из колена Манассиина запечатлено двенадцать тысяч;
\vs Rev 7:7 из колена Симеонова запечатлено двенадцать тысяч; из колена Левиина запечатлено двенадцать тысяч; из колена Иссахарова запечатлено двенадцать тысяч;
\vs Rev 7:8 из колена Завулонова запечатлено двенадцать тысяч; из колена Иосифова запечатлено двенадцать тысяч; из колена Вениаминова запечатлено двенадцать тысяч.
\vs Rev 7:9 После сего взглянул я, и вот, великое множество людей, которого никто не мог перечесть, из всех племен и колен, и народов и языков, стояло пред престолом и пред Агнцем в белых одеждах и с пальмовыми ветвями в руках своих.
\vs Rev 7:10 И восклицали громким голосом, говоря: спасение Богу нашему, сидящему на престоле, и Агнцу!
\vs Rev 7:11 И все Ангелы стояли вокруг престола и старцев и четырех животных, и пали перед престолом на лица свои, и поклонились Богу,
\vs Rev 7:12 говоря: аминь! благословение и слава, и премудрость и благодарение, и честь и сила и крепость Богу нашему во веки веков! Аминь.
\vs Rev 7:13 И, начав речь, один из старцев спросил меня: сии облеченные в белые одежды кто, и откуда пришли?
\vs Rev 7:14 Я сказал ему: ты знаешь, господин. И он сказал мне: это те, которые пришли от великой скорби; они омыли одежды свои и убелили одежды свои Кровию Агнца.
\vs Rev 7:15 За это они пребывают \bibemph{ныне} перед престолом Бога и служат Ему день и ночь в храме Его, и Сидящий на престоле будет обитать в них.
\vs Rev 7:16 Они не будут уже ни алкать, ни жаждать, и не будет палить их солнце и никакой зной:
\vs Rev 7:17 ибо Агнец, Который среди престола, будет пасти их и водить их на живые источники вод; и отрет Бог всякую слезу с очей их.
\vs Rev 8:1 И когда Он снял седьмую печать, сделалось безмолвие на небе, как бы на полчаса.
\vs Rev 8:2 И я видел семь Ангелов, которые стояли пред Богом; и дано им семь труб.
\vs Rev 8:3 И пришел иной Ангел, и стал перед жертвенником, держа золотую кадильницу; и дано было ему множество фимиама, чтобы он с молитвами всех святых возложил его на золотой жертвенник, который перед престолом.
\vs Rev 8:4 И вознесся дым фимиама с молитвами святых от руки Ангела пред Бога.
\vs Rev 8:5 И взял Ангел кадильницу, и наполнил ее огнем с жертвенника, и поверг на землю: и произошли голоса и громы, и молнии и землетрясение.
\rsbpar\vs Rev 8:6 И семь Ангелов, имеющие семь труб, приготовились трубить.
\rsbpar\vs Rev 8:7 Первый Ангел вострубил, и сделались град и огонь, смешанные с кровью, и пали на землю; и третья часть дерев сгорела, и вся трава зеленая сгорела.
\rsbpar\vs Rev 8:8 Второй Ангел вострубил, и как бы большая гора, пылающая огнем, низверглась в море; и третья часть моря сделалась кровью,
\vs Rev 8:9 и умерла третья часть одушевленных тварей, живущих в море, и третья часть судов погибла.
\rsbpar\vs Rev 8:10 Третий Ангел вострубил, и упала с неба большая звезда, горящая подобно светильнику, и пала на третью часть рек и на источники вод.
\vs Rev 8:11 Имя сей звезде <<полынь>>; и третья часть вод сделалась полынью, и многие из людей умерли от вод, потому что они стали горьки.
\rsbpar\vs Rev 8:12 Четвертый Ангел вострубил, и поражена была третья часть солнца и третья часть луны и третья часть звезд, так что затмилась третья часть их, и третья часть дня не светла была~--- так, как и ночи.
\vs Rev 8:13 И видел я и слышал одного Ангела, летящего посреди неба и говорящего громким голосом: горе, горе, горе живущим на земле от остальных трубных голосов трех Ангелов, которые будут трубить!
\vs Rev 9:1 Пятый Ангел вострубил, и я увидел звезду, падшую с неба на землю, и дан был ей ключ от кладязя бездны.
\vs Rev 9:2 Она отворила кладязь бездны, и вышел дым из кладязя, как дым из большой печи; и помрачилось солнце и воздух от дыма из кладязя.
\vs Rev 9:3 И из дыма вышла саранча на землю, и дана была ей власть, какую имеют земные скорпионы.
\vs Rev 9:4 И сказано было ей, чтобы не делала вреда траве земной, и никакой зелени, и никакому дереву, а только одним людям, которые не имеют печати Божией на челах своих.
\vs Rev 9:5 И дано ей не убивать их, а только мучить пять месяцев; и мучение от нее подобно мучению от скорпиона, когда ужалит человека.
\vs Rev 9:6 В те дни люди будут искать смерти, но не найдут ее; пожелают умереть, но смерть убежит от них.
\vs Rev 9:7 По виду своему саранча была подобна коням, приготовленным на войну; и на головах у ней как бы венцы, похожие на золотые, лица же ее~--- как лица человеческие;
\vs Rev 9:8 и волосы у ней~--- как волосы у женщин, а зубы у ней были, как у львов.
\vs Rev 9:9 На ней были брони, как бы брони железные, а шум от крыльев ее~--- как стук от колесниц, когда множество коней бежит на войну;
\vs Rev 9:10 у ней были хвосты, как у скорпионов, и в хвостах ее были жала; власть же ее была~--- вредить людям пять месяцев.
\vs Rev 9:11 Царем над собою она имела ангела бездны; имя ему по-еврейски Аваддон, а по-гречески Аполлион\fns{Губитель.}.
\rsbpar\vs Rev 9:12 Одно горе прошло; вот, идут за ним еще два горя.
\rsbpar\vs Rev 9:13 Шестой Ангел вострубил, и я услышал один голос от четырех рогов золотого жертвенника, стоящего пред Богом,
\vs Rev 9:14 говоривший шестому Ангелу, имевшему трубу: освободи четырех Ангелов, связанных при великой реке Евфрате.
\vs Rev 9:15 И освобождены были четыре Ангела, приготовленные на час и день, и месяц и год, для того, чтобы умертвить третью часть людей.
\vs Rev 9:16 Число конного войска было две тьмы тем; и я слышал число его.
\vs Rev 9:17 Так видел я в видении коней и на них всадников, которые имели на себе брони огненные, гиацинтовые и серные; головы у коней~--- как головы у львов, и изо рта их выходил огонь, дым и сера.
\vs Rev 9:18 От этих трех язв, от огня, дыма и серы, выходящих изо рта их, умерла третья часть людей;
\vs Rev 9:19 ибо сила коней заключалась во рту их и в хвостах их; а хвосты их были подобны змеям, и имели головы, и ими они вредили.
\vs Rev 9:20 Прочие же люди, которые не умерли от этих язв, не раскаялись в делах рук своих, так чтобы не поклоняться бесам и золотым, серебряным, медным, каменным и деревянным идолам, которые не могут ни видеть, ни слышать, ни ходить.
\vs Rev 9:21 И не раскаялись они в убийствах своих, ни в чародействах своих, ни в блудодеянии своем, ни в воровстве своем.
\vs Rev 10:1 И видел я другого Ангела сильного, сходящего с неба, облеченного облаком; над головою его была радуга, и лице его как солнце, и ноги его как столпы огненные,
\vs Rev 10:2 в руке у него была книжка раскрытая. И поставил он правую ногу свою на море, а левую на землю,
\vs Rev 10:3 и воскликнул громким голосом, как рыкает лев; и когда он воскликнул, тогда семь громов проговорили голосами своими.
\vs Rev 10:4 И когда семь громов проговорили голосами своими, я хотел было писать; но услышал голос с неба, говорящий мне: скрой, что говорили семь громов, и не пиши сего.
\vs Rev 10:5 И Ангел, которого я видел стоящим на море и на земле, поднял руку свою к небу
\vs Rev 10:6 и клялся Живущим во веки веков, Который сотворил небо и все, что на нем, землю и все, что на ней, и море и все, что в нем, что времени уже не будет;
\vs Rev 10:7 но в те дни, когда возгласит седьмой Ангел, когда он вострубит, совершится тайна Божия, как Он благовествовал рабам Своим пророкам.
\vs Rev 10:8 И голос, который я слышал с неба, опять стал говорить со мною, и сказал: пойди, возьми раскрытую книжку из руки Ангела, стоящего на море и на земле.
\vs Rev 10:9 И я пошел к Ангелу, и сказал ему: дай мне книжку. Он сказал мне: возьми и съешь ее; она будет горька во чреве твоем, но в устах твоих будет сладка, как мед.
\vs Rev 10:10 И взял я книжку из руки Ангела, и съел ее; и она в устах моих была сладка, как мед; когда же съел ее, то горько стало во чреве моем.
\vs Rev 10:11 И сказал он мне: тебе надлежит опять пророчествовать о народах и племенах, и языках и царях многих.
\vs Rev 11:1 И дана мне трость, подобная жезлу, и сказано: встань и измерь храм Божий и жертвенник, и поклоняющихся в нем.
\vs Rev 11:2 А внешний двор храма исключи и не измеряй его, ибо он дан язычникам: они будут попирать святый город сорок два месяца.
\vs Rev 11:3 И дам двум свидетелям Моим, и они будут пророчествовать тысячу двести шестьдесят дней, будучи облечены во вретище.
\vs Rev 11:4 Это суть две маслины и два светильника, стоящие пред Богом земли.
\vs Rev 11:5 И если кто захочет их обидеть, то огонь выйдет из уст их и пожрет врагов их; если кто захочет их обидеть, тому надлежит быть убиту.
\vs Rev 11:6 Они имеют власть затворить небо, чтобы не шел дождь на землю во дни пророчествования их, и имеют власть над водами, превращать их в кровь, и поражать землю всякою язвою, когда только захотят.
\vs Rev 11:7 И когда кончат они свидетельство свое, зверь, выходящий из бездны, сразится с ними, и победит их, и убьет их,
\vs Rev 11:8 и трупы их оставит на улице великого города, который духовно называется Содом и Египет, где и Господь наш распят.
\vs Rev 11:9 И \bibemph{многие} из народов и колен, и языков и племен будут смотреть на трупы их три дня с половиною, и не позволят положить трупы их во гробы.
\vs Rev 11:10 И живущие на земле будут радоваться сему и веселиться, и пошлют дары друг другу, потому что два пророка сии мучили живущих на земле.
\vs Rev 11:11 Но после трех дней с половиною вошел в них дух жизни от Бога, и они оба стали на ноги свои; и великий страх напал на тех, которые смотрели на них.
\vs Rev 11:12 И услышали они с неба громкий голос, говоривший им: взойдите сюда. И они взошли на небо на облаке; и смотрели на них враги их.
\vs Rev 11:13 И в тот же час произошло великое землетрясение, и десятая часть города пала, и погибло при землетрясении семь тысяч имен человеческих; и прочие объяты были страхом и воздали славу Богу небесному.
\rsbpar\vs Rev 11:14 Второе горе прошло; вот, идет скоро третье горе.
\rsbpar\vs Rev 11:15 И седьмой Ангел вострубил, и раздались на небе громкие голоса, говорящие: царство мира соделалось \bibemph{царством} Господа нашего и Христа Его, и будет царствовать во веки веков.
\vs Rev 11:16 И двадцать четыре старца, сидящие пред Богом на престолах своих, пали на лица свои и поклонились Богу,
\vs Rev 11:17 говоря: благодарим Тебя, Господи Боже Вседержитель, Который еси и был и грядешь, что Ты приял силу Твою великую и воцарился.
\vs Rev 11:18 И рассвирепели язычники; и пришел гнев Твой и время судить мертвых и дать возмездие рабам Твоим, пророкам и святым и боящимся имени Твоего, малым и великим, и погубить губивших землю.
\rsbpar\vs Rev 11:19 И отверзся храм Божий на небе, и явился ковчег завета Его в храме Его; и произошли молнии и голоса, и громы и землетрясение и великий град.
\vs Rev 12:1 И явилось на небе великое знамение: жена, облеченная в солнце; под ногами ее луна, и на главе ее венец из двенадцати звезд.
\vs Rev 12:2 Она имела во чреве, и кричала от болей и мук рождения.
\vs Rev 12:3 И другое знамение явилось на небе: вот, большой красный дракон с семью головами и десятью рогами, и на головах его семь диадим.
\vs Rev 12:4 Хвост его увлек с неба третью часть звезд и поверг их на землю. Дракон сей стал перед женою, которой надлежало родить, дабы, когда она родит, пожрать ее младенца.
\vs Rev 12:5 И родила она младенца мужеского пола, которому надлежит пасти все народы жезлом железным; и восхищено было дитя ее к Богу и престолу Его.
\vs Rev 12:6 А жена убежала в пустыню, где приготовлено было для нее место от Бога, чтобы питали ее там тысячу двести шестьдесят дней.
\rsbpar\vs Rev 12:7 И произошла на небе война: Михаил и Ангелы его воевали против дракона, и дракон и ангелы его воевали \bibemph{против них},
\vs Rev 12:8 но не устояли, и не нашлось уже для них места на небе.
\vs Rev 12:9 И низвержен был великий дракон, древний змий, называемый диаволом и сатаною, обольщающий всю вселенную, низвержен на землю, и ангелы его низвержены с ним.
\vs Rev 12:10 И услышал я громкий голос, говорящий на небе: ныне настало спасение и сила и царство Бога нашего и власть Христа Его, потому что низвержен клеветник братий наших, клеветавший на них пред Богом нашим день и ночь.
\vs Rev 12:11 Они победили его кровию Агнца и словом свидетельства своего, и не возлюбили души своей даже до смерти.
\vs Rev 12:12 Итак веселитесь, небеса и обитающие на них! Горе живущим на земле и на море! потому что к вам сошел диавол в сильной ярости, зная, что немного ему остается времени.
\rsbpar\vs Rev 12:13 Когда же дракон увидел, что низвержен на землю, начал преследовать жену, которая родила младенца мужеского пола.
\vs Rev 12:14 И даны были жене два крыла большого орла, чтобы она летела в пустыню в свое место от лица змия и там питалась в продолжение времени, времен и полвремени.
\vs Rev 12:15 И пустил змий из пасти своей вслед жены воду как реку, дабы увлечь ее рекою.
\vs Rev 12:16 Но земля помогла жене, и разверзла земля уста свои, и поглотила реку, которую пустил дракон из пасти своей.
\vs Rev 12:17 И рассвирепел дракон на жену, и пошел, чтобы вступить в брань с прочими от семени ее, сохраняющими заповеди Божии и имеющими свидетельство Иисуса Христа.
\vs Rev 13:1 И стал я на песке морском, и увидел выходящего из моря зверя с семью головами и десятью рогами: на рогах его было десять диадим, а на головах его имена богохульные.
\vs Rev 13:2 Зверь, которого я видел, был подобен барсу; ноги у него~--- как у медведя, а пасть у него~--- как пасть у льва; и дал ему дракон силу свою и престол свой и великую власть.
\vs Rev 13:3 И видел я, что одна из голов его как бы смертельно была ранена, но эта смертельная рана исцелела. И дивилась вся земля, следя за зверем, и поклонились дракону, который дал власть зверю,
\vs Rev 13:4 и поклонились зверю, говоря: кто подобен зверю сему? и кто может сразиться с ним?
\vs Rev 13:5 И даны были ему уста, говорящие гордо и богохульно, и дана ему власть действовать сорок два месяца.
\vs Rev 13:6 И отверз он уста свои для хулы на Бога, чтобы хулить имя Его, и жилище Его, и живущих на небе.
\vs Rev 13:7 И дано было ему вести войну со святыми и победить их; и дана была ему власть над всяким коленом и народом, и языком и племенем.
\vs Rev 13:8 И поклонятся ему все живущие на земле, которых имена не написаны в книге жизни у Агнца, закланного от создания мира.
\vs Rev 13:9 Кто имеет ухо, да слышит.
\vs Rev 13:10 Кто ведет в плен, тот сам пойдет в плен; кто мечом убивает, тому самому надлежит быть убиту мечом. Здесь терпение и вера святых.
\rsbpar\vs Rev 13:11 И увидел я другого зверя, выходящего из земли; он имел два рога, подобные агнчим, и говорил как дракон.
\vs Rev 13:12 Он действует перед ним со всею властью первого зверя и заставляет всю землю и живущих на ней поклоняться первому зверю, у которого смертельная рана исцелела;
\vs Rev 13:13 и творит великие знамения, так что и огонь низводит с неба на землю перед людьми.
\vs Rev 13:14 И чудесами, которые дано было ему творить перед зверем, он обольщает живущих на земле, говоря живущим на земле, чтобы они сделали образ зверя, который имеет рану от меча и жив.
\vs Rev 13:15 И дано ему было вложить дух в образ зверя, чтобы образ зверя и говорил и действовал так, чтобы убиваем был всякий, кто не будет поклоняться образу зверя.
\vs Rev 13:16 И он сделает то, что всем, малым и великим, богатым и нищим, свободным и рабам, положено будет начертание на правую руку их или на чело их,
\vs Rev 13:17 и что никому нельзя будет ни покупать, ни продавать, кроме того, кто имеет это начертание, или имя зверя, или число имени его.
\vs Rev 13:18 Здесь мудрость. Кто имеет ум, тот сочти число зверя, ибо это число человеческое; число его шестьсот шестьдесят шесть.
\vs Rev 14:1 И взглянул я, и вот, Агнец стоит на горе Сионе, и с Ним сто сорок четыре тысячи, у которых имя Отца Его написано на челах.
\vs Rev 14:2 И услышал я голос с неба, как шум от множества вод и как звук сильного грома; и услышал голос как бы гуслистов, играющих на гуслях своих.
\vs Rev 14:3 Они поют как бы новую песнь пред престолом и пред четырьмя животными и старцами; и никто не мог научиться сей песни, кроме сих ста сорока четырех тысяч, искупленных от земли.
\vs Rev 14:4 Это те, которые не осквернились с женами, ибо они девственники; это те, которые следуют за Агнцем, куда бы Он ни пошел. Они искуплены из людей, как первенцы Богу и Агнцу,
\vs Rev 14:5 и в устах их нет лукавства; они непорочны пред престолом Божиим.
\rsbpar\vs Rev 14:6 И увидел я другого Ангела, летящего по средине неба, который имел вечное Евангелие, чтобы благовествовать живущим на земле и всякому племени и колену, и языку и народу;
\vs Rev 14:7 и говорил он громким голосом: убойтесь Бога и воздайте Ему славу, ибо наступил час суда Его, и поклонитесь Сотворившему небо и землю, и море и источники вод.
\vs Rev 14:8 И другой Ангел следовал за ним, говоря: пал, пал Вавилон, город великий, потому что он яростным вином блуда своего напоил все народы.
\vs Rev 14:9 И третий Ангел последовал за ними, говоря громким голосом: кто поклоняется зверю и образу его и принимает начертание на чело свое, или на руку свою,
\vs Rev 14:10 тот будет пить вино ярости Божией, вино цельное, приготовленное в чаше гнева Его, и будет мучим в огне и сере пред святыми Ангелами и пред Агнцем;
\vs Rev 14:11 и дым мучения их будет восходить во веки веков, и не будут иметь покоя ни днем, ни ночью поклоняющиеся зверю и образу его и принимающие начертание имени его.
\vs Rev 14:12 Здесь терпение святых, соблюдающих заповеди Божии и веру в Иисуса.
\rsbpar\vs Rev 14:13 И услышал я голос с неба, говорящий мне: напиши: отныне блаженны мертвые, умирающие в Господе; ей, говорит Дух, они успокоятся от трудов своих, и дела их идут вслед за ними.
\rsbpar\vs Rev 14:14 И взглянул я, и вот светлое облако, и на облаке сидит подобный Сыну Человеческому; на голове его золотой венец, и в руке его острый серп.
\vs Rev 14:15 И вышел другой Ангел из храма и воскликнул громким голосом к сидящему на облаке: пусти серп твой и пожни, потому что пришло время жатвы, ибо жатва на земле созрела.
\vs Rev 14:16 И поверг сидящий на облаке серп свой на землю, и земля была пожата.
\vs Rev 14:17 И другой Ангел вышел из храма, находящегося на небе, также с острым серпом.
\vs Rev 14:18 И иной Ангел, имеющий власть над огнем, вышел от жертвенника и с великим криком воскликнул к имеющему острый серп, говоря: пусти острый серп твой и обрежь гроздья винограда на земле, потому что созрели на нем ягоды.
\vs Rev 14:19 И поверг Ангел серп свой на землю, и обрезал виноград на земле, и бросил в великое точило гнева Божия.
\vs Rev 14:20 И истоптаны \bibemph{ягоды} в точиле за городом, и потекла кровь из точила даже до узд конских, на тысячу шестьсот стадий.
\vs Rev 15:1 И увидел я иное знамение на небе, великое и чудное: семь Ангелов, имеющих семь последних язв, которыми оканчивалась ярость Божия.
\vs Rev 15:2 И видел я как бы стеклянное море, смешанное с огнем; и победившие зверя и образ его, и начертание его и число имени его, стоят на этом стеклянном море, держа гусли Божии,
\vs Rev 15:3 и поют песнь Моисея, раба Божия, и песнь Агнца, говоря: велики и чудны дела Твои, Господи Боже Вседержитель! праведны и истинны пути Твои, Царь святых!
\vs Rev 15:4 Кто не убоится Тебя, Господи, и не прославит имени Твоего? ибо Ты един свят. Все народы придут и поклонятся пред Тобою, ибо открылись суды Твои.
\rsbpar\vs Rev 15:5 И после сего я взглянул, и вот, отверзся храм скинии свидетельства на небе.
\vs Rev 15:6 И вышли из храма семь Ангелов, имеющие семь язв, облеченные в чистую и светлую льняную одежду и опоясанные по персям золотыми поясами.
\vs Rev 15:7 И одно из четырех животных дало семи Ангелам семь золотых чаш, наполненных гневом Бога, живущего во веки веков.
\vs Rev 15:8 И наполнился храм дымом от славы Божией и от силы Его, и никто не мог войти в храм, доколе не окончились семь язв семи Ангелов.
\vs Rev 16:1 И услышал я из храма громкий голос, говорящий семи Ангелам: идите и вылейте семь чаш гнева Божия на землю.
\vs Rev 16:2 Пошел первый Ангел и вылил чашу свою на землю: и сделались жестокие и отвратительные гнойные раны на людях, имеющих начертание зверя и поклоняющихся образу его.
\rsbpar\vs Rev 16:3 Второй Ангел вылил чашу свою в море: и сделалась кровь, как бы мертвеца, и все одушевленное умерло в море.
\rsbpar\vs Rev 16:4 Третий Ангел вылил чашу свою в реки и источники вод: и сделалась кровь.
\vs Rev 16:5 И услышал я Ангела вод, который говорил: праведен Ты, Господи, Который еси и был, и свят, потому что так судил;
\vs Rev 16:6 за то, что они пролили кровь святых и пророков, Ты дал им пить кровь: они достойны того.
\vs Rev 16:7 И услышал я другого от жертвенника говорящего: ей, Господи Боже Вседержитель, истинны и праведны суды Твои.
\rsbpar\vs Rev 16:8 Четвертый Ангел вылил чашу свою на солнце: и дано было ему жечь людей огнем.
\vs Rev 16:9 И жег людей сильный зной, и они хулили имя Бога, имеющего власть над сими язвами, и не вразумились, чтобы воздать Ему славу.
\rsbpar\vs Rev 16:10 Пятый Ангел вылил чашу свою на престол зверя: и сделалось царство его мрачно, и они кусали языки свои от страдания,
\vs Rev 16:11 и хулили Бога небесного от страданий своих и язв своих; и не раскаялись в делах своих.
\rsbpar\vs Rev 16:12 Шестой Ангел вылил чашу свою в великую реку Евфрат: и высохла в ней вода, чтобы готов был путь царям от восхода солнечного.
\vs Rev 16:13 И видел я \bibemph{выходящих} из уст дракона и из уст зверя и из уст лжепророка трех духов нечистых, подобных жабам:
\vs Rev 16:14 это~--- бесовские духи, творящие знамения; они выходят к царям земли всей вселенной, чтобы собрать их на брань в оный великий день Бога Вседержителя.
\vs Rev 16:15 Се, иду как тать: блажен бодрствующий и хранящий одежду свою, чтобы не ходить ему нагим и чтобы не увидели срамоты его.
\vs Rev 16:16 И он собрал их на место, называемое по-еврейски Армагеддон.
\rsbpar\vs Rev 16:17 Седьмой Ангел вылил чашу свою на воздух: и из храма небесного от престола раздался громкий голос, говорящий: совершилось!
\vs Rev 16:18 И произошли молнии, громы и голоса, и сделалось великое землетрясение, какого не бывало с тех пор, как люди на земле. Такое землетрясение! Так великое!
\vs Rev 16:19 И город великий распался на три части, и города языческие пали, и Вавилон великий воспомянут пред Богом, чтобы дать ему чашу вина ярости гнева Его.
\vs Rev 16:20 И всякий остров убежал, и гор не стало;
\vs Rev 16:21 и град, величиною в талант, пал с неба на людей; и хулили люди Бога за язвы от града, потому что язва от него была весьма тяжкая.
\vs Rev 17:1 И пришел один из семи Ангелов, имеющих семь чаш, и, говоря со мною, сказал мне: подойди, я покажу тебе суд над великою блудницею, сидящею на водах многих;
\vs Rev 17:2 с нею блудодействовали цари земные, и вином ее блудодеяния упивались живущие на земле.
\vs Rev 17:3 И повел меня в духе в пустыню; и я увидел жену, сидящую на звере багряном, преисполненном именами богохульными, с семью головами и десятью рогами.
\vs Rev 17:4 И жена облечена была в порфиру и багряницу, украшена золотом, драгоценными камнями и жемчугом, и держала золотую чашу в руке своей, наполненную мерзостями и нечистотою блудодейства ее;
\vs Rev 17:5 и на челе ее написано имя: тайна, Вавилон великий, мать блудницам и мерзостям земным.
\vs Rev 17:6 Я видел, что жена упоена была кровью святых и кровью свидетелей Иисусовых, и видя ее, дивился удивлением великим.
\vs Rev 17:7 И сказал мне Ангел: что ты дивишься? я скажу тебе тайну жены сей и зверя, носящего ее, имеющего семь голов и десять рогов.
\vs Rev 17:8 Зверь, которого ты видел, был, и нет его, и выйдет из бездны, и пойдет в погибель; и удивятся те из живущих на земле, имена которых не вписаны в книгу жизни от начала мира, видя, что зверь был, и нет его, и явится.
\vs Rev 17:9 Здесь ум, имеющий мудрость. Семь голов суть семь гор, на которых сидит жена,
\vs Rev 17:10 и семь царей, из которых пять пали, один есть, а другой еще не пришел, и когда придет, не долго ему быть.
\vs Rev 17:11 И зверь, который был и которого нет, есть восьмой, и из числа семи, и пойдет в погибель.
\vs Rev 17:12 И десять рогов, которые ты видел, суть десять царей, которые еще не получили царства, но примут власть со зверем, как цари, на один час.
\vs Rev 17:13 Они имеют одни мысли и передадут силу и власть свою зверю.
\vs Rev 17:14 Они будут вести брань с Агнцем, и Агнец победит их; ибо Он есть Господь господствующих и Царь царей, и те, которые с Ним, суть званые и избранные и верные.
\vs Rev 17:15 И говорит мне: в\acc{о}ды, которые ты видел, где сидит блудница, суть люди и народы, и племена и языки.
\vs Rev 17:16 И десять рогов, которые ты видел на звере, сии возненавидят блудницу, и разорят ее, и обнажат, и плоть ее съедят, и сожгут ее в огне;
\vs Rev 17:17 потому что Бог положил им на сердце~--- исполнить волю Его, исполнить одну волю, и отдать царство их зверю, доколе не исполнятся слова Божии.
\vs Rev 17:18 Жена же, которую ты видел, есть великий город, царствующий над земными царями.
\vs Rev 18:1 После сего я увидел иного Ангела, сходящего с неба и имеющего власть великую; земля осветилась от славы его.
\vs Rev 18:2 И воскликнул он сильно, громким голосом говоря: пал, пал Вавилон, великая \bibemph{блудница}, сделался жилищем бесов и пристанищем всякому нечистому духу, пристанищем всякой нечистой и отвратительной птице; ибо яростным вином блудодеяния своего она напоила все народы,
\vs Rev 18:3 и цари земные любодействовали с нею, и купцы земные разбогатели от великой роскоши ее.
\rsbpar\vs Rev 18:4 И услышал я иной голос с неба, говорящий: выйди от нее, народ Мой, чтобы не участвовать вам в грехах ее и не подвергнуться язвам ее;
\vs Rev 18:5 ибо грехи ее дошли до неба, и Бог воспомянул неправды ее.
\vs Rev 18:6 Воздайте ей так, как и она воздала вам, и вдвое воздайте ей по делам ее; в чаше, в которой она приготовляла вам вино, приготовьте ей вдвое.
\vs Rev 18:7 Сколько славилась она и роскошествовала, столько воздайте ей мучений и горестей. Ибо она говорит в сердце своем: <<сижу царицею, я не вдова и не увижу горести!>>
\vs Rev 18:8 За то в один день придут на нее казни, смерть и плач и голод, и будет сожжена огнем, потому что силен Господь Бог, судящий ее.
\vs Rev 18:9 И восплачут и возрыдают о ней цари земные, блудодействовавшие и роскошествовавшие с нею, когда увидят дым от пожара ее,
\vs Rev 18:10 стоя издали от страха мучений ее \bibemph{и} говоря: горе, горе \bibemph{тебе}, великий город Вавилон, город крепкий! ибо в один час пришел суд твой.
\vs Rev 18:11 И купцы земные восплачут и возрыдают о ней, потому что товаров их никто уже не покупает,
\vs Rev 18:12 товаров золотых и серебряных, и камней драгоценных и жемчуга, и виссона и порфиры, и шелка и багряницы, и всякого благовонного дерева, и всяких изделий из слоновой кости, и всяких изделий из дорогих дерев, из меди и железа и мрамора,
\vs Rev 18:13 корицы и фимиама, и мира и ладана, и вина и елея, и муки и пшеницы, и скота и овец, и коней и колесниц, и тел и душ человеческих.
\vs Rev 18:14 И плодов, угодных для души твоей, не стало у тебя, и все тучное и блистательное удалилось от тебя; ты уже не найдешь его.
\vs Rev 18:15 Торговавшие всем сим, обогатившиеся от нее, станут вдали от страха мучений ее, плача и рыдая
\vs Rev 18:16 и говоря: горе, горе \bibemph{тебе}, великий город, одетый в виссон и порфиру и багряницу, украшенный золотом и камнями драгоценными и жемчугом,
\vs Rev 18:17 ибо в один час погибло такое богатство! И все кормчие, и все плывущие на кораблях, и все корабельщики, и все торгующие на море стали вдали
\vs Rev 18:18 и, видя дым от пожара ее, возопили, говоря: какой город подобен городу великому!
\vs Rev 18:19 И пос\acc{ы}пали пеплом головы свои, и вопили, плача и рыдая: горе, горе \bibemph{тебе}, город великий, драгоценностями которого обогатились все, имеющие корабли на море, ибо опустел в один час!
\vs Rev 18:20 Веселись о сем, небо и святые Апостолы и пророки; ибо совершил Бог суд ваш над ним.
\rsbpar\vs Rev 18:21 И один сильный Ангел взял камень, подобный большому жернову, и поверг в море, говоря: с таким стремлением повержен будет Вавилон, великий город, и уже не будет его.
\vs Rev 18:22 И г\acc{о}лоса играющих на гуслях, и поющих, и играющих на свирелях, и трубящих трубами в тебе уже не слышно будет; не будет уже в тебе никакого художника, никакого художества, и шума от жерновов не слышно уже будет в тебе;
\vs Rev 18:23 и свет светильника уже не появится в тебе; и г\acc{о}лоса жениха и невесты не будет уже слышно в тебе: ибо купцы твои были вельможи земли, и волшебством твоим введены в заблуждение все народы.
\vs Rev 18:24 И в нем найдена кровь пророков и святых и всех убитых на земле.
\vs Rev 19:1 После сего я услышал на небе громкий голос как бы многочисленного народа, который говорил: аллилуия! спасение и слава, и честь и сила Господу нашему!
\vs Rev 19:2 Ибо истинны и праведны суды Его: потому что Он осудил ту великую любодейцу, которая растлила землю любодейством своим, и взыскал кровь рабов Своих от руки ее.
\vs Rev 19:3 И вторично сказали: аллилуия! И дым ее восходил во веки веков.
\vs Rev 19:4 Тогда двадцать четыре старца и четыре животных пали и поклонились Богу, сидящему на престоле, говоря: аминь! аллилуия!
\vs Rev 19:5 И голос от престола исшел, говорящий: хвалите Бога нашего, все рабы Его и боящиеся Его, малые и великие.
\vs Rev 19:6 И слышал я как бы голос многочисленного народа, как бы шум вод многих, как бы голос громов сильных, говорящих: аллилуия! ибо воцарился Господь Бог Вседержитель.
\vs Rev 19:7 Возрадуемся и возвеселимся и воздадим Ему славу; ибо наступил брак Агнца, и жена Его приготовила себя.
\vs Rev 19:8 И дано было ей облечься в виссон чистый и светлый; виссон же есть праведность святых.
\vs Rev 19:9 И сказал мне \bibemph{Ангел}: напиши: блаженны званые на брачную вечерю Агнца. И сказал мне: сии суть истинные слова Божии.
\vs Rev 19:10 Я пал к ногам его, чтобы поклониться ему; но он сказал мне: смотри, не делай сего; я сослужитель тебе и братьям твоим, имеющим свидетельство Иисусово; Богу поклонись; ибо свидетельство Иисусово есть дух пророчества.
\rsbpar\vs Rev 19:11 И увидел я отверстое небо, и вот конь белый, и сидящий на нем называется Верный и Истинный, Который праведно судит и воинствует.
\vs Rev 19:12 Очи у Него как пламень огненный, и на голове Его много диадим. \bibemph{Он} имел имя написанное, которого никто не знал, кроме Его Самого.
\vs Rev 19:13 \bibemph{Он был} облечен в одежду, обагренную кровью. Имя Ему: <<Слово Божие>>.
\vs Rev 19:14 И воинства небесные следовали за Ним на конях белых, облеченные в виссон белый и чистый.
\vs Rev 19:15 Из уст же Его исходит острый меч, чтобы им поражать народы. Он пасет их жезлом железным; Он топчет точило вина ярости и гнева Бога Вседержителя.
\vs Rev 19:16 На одежде и на бедре Его написано имя: <<Царь царей и Господь господствующих>>.
\vs Rev 19:17 И увидел я одного Ангела, стоящего на солнце; и он воскликнул громким голосом, говоря всем птицам, летающим по средине неба: летите, собирайтесь на великую вечерю Божию,
\vs Rev 19:18 чтобы пожрать трупы царей, трупы сильных, трупы тысяченачальников, трупы коней и сидящих на них, трупы всех свободных и рабов, и малых и великих.
\rsbpar\vs Rev 19:19 И увидел я зверя и царей земных и воинства их, собранные, чтобы сразиться с Сидящим на коне и с воинством Его.
\vs Rev 19:20 И схвачен был зверь и с ним лжепророк, производивший чудеса пред ним, которыми он обольстил принявших начертание зверя и поклоняющихся его изображению: оба живые брошены в озеро огненное, горящее серою;
\vs Rev 19:21 а прочие убиты мечом Сидящего на коне, исходящим из уст Его, и все птицы напитались их трупами.
\vs Rev 20:1 И увидел я Ангела, сходящего с неба, который имел ключ от бездны и большую цепь в руке своей.
\vs Rev 20:2 Он взял дракона, змия древнего, который есть диавол и сатана, и сковал его на тысячу лет,
\vs Rev 20:3 и низверг его в бездну, и заключил его, и положил над ним печать, дабы не прельщал уже народы, доколе не окончится тысяча лет; после же сего ему должно быть освобожденным на малое время.
\rsbpar\vs Rev 20:4 И увидел я престолы и сидящих на них, которым дано было судить, и души обезглавленных за свидетельство Иисуса и за слово Божие, которые не поклонились зверю, ни образу его, и не приняли начертания на чело свое и на руку свою. Они ожили и царствовали со Христом тысячу лет.
\vs Rev 20:5 Прочие же из умерших не ожили, доколе не окончится тысяча лет. Это~--- первое воскресение.
\vs Rev 20:6 Блажен и свят имеющий участие в воскресении первом: над ними смерть вторая не имеет власти, но они будут священниками Бога и Христа и будут царствовать с Ним тысячу лет.
\rsbpar\vs Rev 20:7 Когда же окончится тысяча лет, сатана будет освобожден из темницы своей и выйдет обольщать народы, находящиеся на четырех углах земли, Гога и Магога, и собирать их на брань; число их как песок морской.
\vs Rev 20:8 И вышли на широту земли, и окружили стан святых и город возлюбленный.
\vs Rev 20:9 И ниспал огонь с неба от Бога и пожрал их;
\vs Rev 20:10 а диавол, прельщавший их, ввержен в озеро огненное и серное, где зверь и лжепророк, и будут мучиться день и ночь во веки веков.
\rsbpar\vs Rev 20:11 И увидел я великий белый престол и Сидящего на нем, от лица Которого бежало небо и земля, и не нашлось им места.
\vs Rev 20:12 И увидел я мертвых, малых и великих, стоящих пред Богом, и книги раскрыты были, и иная книга раскрыта, которая есть книга жизни; и судимы были мертвые по написанному в книгах, сообразно с делами своими.
\vs Rev 20:13 Тогда отдало море мертвых, бывших в нем, и смерть и ад отдали мертвых, которые были в них; и судим был каждый по делам своим.
\vs Rev 20:14 И смерть и ад повержены в озеро огненное. Это смерть вторая.
\vs Rev 20:15 И кто не был записан в книге жизни, тот был брошен в озеро огненное.
\vs Rev 21:1 И увидел я новое небо и новую землю, ибо прежнее небо и прежняя земля миновали, и моря уже нет.
\vs Rev 21:2 И я, Иоанн, увидел святый город Иерусалим, новый, сходящий от Бога с неба, приготовленный как невеста, украшенная для мужа своего.
\vs Rev 21:3 И услышал я громкий голос с неба, говорящий: се, скиния Бога с человеками, и Он будет обитать с ними; они будут Его народом, и Сам Бог с ними будет Богом их.
\vs Rev 21:4 И отрет Бог всякую слезу с очей их, и смерти не будет уже; ни плача, ни вопля, ни болезни уже не будет, ибо прежнее прошло.
\vs Rev 21:5 И сказал Сидящий на престоле: се, творю все новое. И говорит мне: напиши; ибо слова сии истинны и верны.
\vs Rev 21:6 И сказал мне: совершилось! Я есмь Альфа и Омега, начало и конец; жаждущему дам даром от источника воды живой.
\vs Rev 21:7 Побеждающий наследует все, и буду ему Богом, и он будет Мне сыном.
\vs Rev 21:8 Боязливых же и неверных, и скверных и убийц, и любодеев и чародеев, и идолослужителей и всех лжецов участь в озере, горящем огнем и серою. Это смерть вторая.
\rsbpar\vs Rev 21:9 И пришел ко мне один из семи Ангелов, у которых было семь чаш, наполненных семью последними язвами, и сказал мне: пойди, я покажу тебе жену, невесту Агнца.
\vs Rev 21:10 И вознес меня в духе на великую и высокую гору, и показал мне великий город, святый Иерусалим, который нисходил с неба от Бога.
\vs Rev 21:11 Он имеет славу Божию. Светило его подобно драгоценнейшему камню, как бы камню яспису кристалловидному.
\vs Rev 21:12 Он имеет большую и высокую стену, имеет двенадцать ворот и на них двенадцать Ангелов; на воротах написаны имена двенадцати колен сынов Израилевых:
\vs Rev 21:13 с востока трое ворот, с севера трое ворот, с юга трое ворот, с запада трое ворот.
\vs Rev 21:14 Стена города имеет двенадцать оснований, и на них имена двенадцати Апостолов Агнца.
\vs Rev 21:15 Говоривший со мною имел золотую трость для измерения города и ворот его и стен\acc{ы} его.
\vs Rev 21:16 Город расположен четвероугольником, и длина его такая же, как и широта. И измерил он город тростью на двенадцать тысяч стадий; длина и широта и высота его равны.
\vs Rev 21:17 И стену его измерил во сто сорок четыре локтя, мерою человеческою, какова мера и Ангела.
\vs Rev 21:18 Стена его построена из ясписа, а город был чистое золото, подобен чистому стеклу.
\vs Rev 21:19 Основания стены города украшены всякими драгоценными камнями: основание первое яспис, второе сапфир, третье халкидон, четвертое смарагд,
\vs Rev 21:20 пятое сардоникс, шестое сердолик, седьмое хризолит, восьмое вирилл, девятое топаз, десятое хризопрас, одиннадцатое гиацинт, двенадцатое аметист.
\vs Rev 21:21 А двенадцать ворот~--- двенадцать жемчужин: каждые ворота были из одной жемчужины. Улица города~--- чистое золото, как прозрачное стекло.
\vs Rev 21:22 Храма же я не видел в нем, ибо Господь Бог Вседержитель~--- храм его, и Агнец.
\vs Rev 21:23 И город не имеет нужды ни в солнце, ни в луне для освещения своего, ибо слава Божия осветила его, и светильник его~--- Агнец.
\vs Rev 21:24 Спасенные народы будут ходить во свете его, и цари земные принесут в него славу и честь свою.
\vs Rev 21:25 Ворота его не будут запираться днем; а ночи там не будет.
\vs Rev 21:26 И принесут в него славу и честь народов.
\vs Rev 21:27 И не войдет в него ничто нечистое и никто преданный мерзости и лжи, а только те, которые написаны у Агнца в книге жизни.
\vs Rev 22:1 И показал мне чистую реку воды жизни, светлую, как кристалл, исходящую от престола Бога и Агнца.
\vs Rev 22:2 Среди улицы его, и по ту и по другую сторону реки, древо жизни, двенадцать \bibemph{раз} приносящее плоды, дающее на каждый месяц плод свой; и листья дерева~--- для исцеления народов.
\vs Rev 22:3 И ничего уже не будет проклятого; но престол Бога и Агнца будет в нем, и рабы Его будут служить Ему.
\vs Rev 22:4 И узрят лице Его, и имя Его будет на челах их.
\vs Rev 22:5 И ночи не будет там, и не будут иметь нужды ни в светильнике, ни в свете солнечном, ибо Господь Бог освещает их; и будут царствовать во веки веков.
\rsbpar\vs Rev 22:6 И сказал мне: сии слова верны и истинны; и Господь Бог святых пророков послал Ангела Своего показать рабам Своим то, чему надлежит быть вскоре.
\vs Rev 22:7 Се, гряду скоро: блажен соблюдающий слова пророчества книги сей.
\rsbpar\vs Rev 22:8 Я, Иоанн, видел и слышал сие. Когда же услышал и увидел, пал к ногам Ангела, показывающего мне сие, чтобы поклониться \bibemph{ему};
\vs Rev 22:9 но он сказал мне: смотри, не делай сего; ибо я сослужитель тебе и братьям твоим пророкам и соблюдающим слова книги сей; Богу поклонись.
\vs Rev 22:10 И сказал мне: не запечатывай слов пророчества книги сей; ибо время близко.
\vs Rev 22:11 Неправедный пусть еще делает неправду; нечистый пусть еще сквернится; праведный да творит правду еще, и святый да освящается еще.
\vs Rev 22:12 Се, гряду скоро, и возмездие Мое со Мною, чтобы воздать каждому по делам его.
\vs Rev 22:13 Я есмь Альфа и Омега, начало и конец, Первый и Последний.
\vs Rev 22:14 Блаженны те, которые соблюдают заповеди Его, чтобы иметь им право на древо жизни и войти в город воротами.
\vs Rev 22:15 А вне~--- псы и чародеи, и любодеи, и убийцы, и идолослужители, и всякий любящий и делающий неправду.
\rsbpar\vs Rev 22:16 Я, Иисус, послал Ангела Моего засвидетельствовать вам сие в церквах. Я есмь корень и потомок Давида, звезда светлая и утренняя.
\rsbpar\vs Rev 22:17 И Дух и невеста говорят: прииди! И слышавший да скажет: прииди! Жаждущий пусть приходит, и желающий пусть берет воду жизни даром.
\rsbpar\vs Rev 22:18 И я также свидетельствую всякому слышащему слова пророчества книги сей: если кто приложит что к ним, на того наложит Бог язвы, о которых написано в книге сей;
\vs Rev 22:19 и если кто отнимет что от слов книги пророчества сего, у того отнимет Бог участие в книге жизни и в святом граде и в том, что написано в книге сей.
\rsbpar\vs Rev 22:20 Свидетельствующий сие говорит: ей, гряду скоро! Аминь. Ей, гряди, Господи Иисусе!
\rsbpar\vs Rev 22:21 Благодать Господа нашего Иисуса Христа со всеми вами. Аминь.

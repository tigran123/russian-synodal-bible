\begin{center}
\Large\bfseries
БИБЛЕЙСКИЙ КАЛЕНДАРЬ
\end{center}

\makeatletter
\def\bibstrut{%
  \vrule
    \@height\dimexpr\f@size pt*13/10\relax
    \@depth.6\baselineskip
    \@width\z@
}
\makeatother

\newsavebox{\nazvanija}
\newsavebox{\mesjatsi}
\sbox{\nazvanija}{\parbox{9cm}{\centering Названия месяцев {\bfseries ДРЕВНИЕ} и \bibemph{АССИРО-ВАВИЛОНСКИЕ},\\ число дней в месяце и особые дни}}
\sbox{\mesjatsi}{\parbox{2cm}{\centering Соответствует месяцам современного календаря}}

\hspace*{-5mm}
\begin{tabular}{|>{\raggedleft}p{1cm}|>{\raggedleft}p{1cm}|p{9cm}|c|}
\hline
\multicolumn{2}{|c|}{\bibstrut Счет месяцев} & & \\
\cline{1-2}
\multicolumn{1}{|p{1cm}|}{\footnotesize в священном году} & 
\multicolumn{1}{p{1cm}|}{\footnotesize в гражданском году} &
\usebox{\nazvanija} & \usebox{\mesjatsi} \tabularnewline
\hline
\bibstrut I & \bibstrut 7 &
\bibstrut {\bfseries АВИВ}, \bibemph{НИСАН}. 30 дней.
14. Пасха (\bibref[Исх. гл. 12]{Exo 12:1}; \bibref{Lev 23:5}; \bibref{Num 28:16}).
16. Принесение первого снопа жатвы ячменя (\bibref[Лев 23:10--14]{Lev 23:10}).&
\bibstrut март --- апрель\tabularnewline
\hline
\bibstrut II & \bibstrut 8 &
\bibstrut {\bfseries ЗИФ}, \bibemph{ИЯР}. 29 дней.
14. Вторая пасха --- для тех, кто не мог совершить первую (\bibref[Чис 9:10--12]{Num 9:10}).&
\bibstrut апрель --- май\tabularnewline
\hline
\bibstrut III & \bibstrut 9 &
\bibstrut \bibemph{СИВАН}. 30 дней.
6. Пятидесятница (\bibref{Lev 23:16}) или праздник седмиц (\bibref[Втор 16:9--10]{Deu 16:9}).
Принесение начатков жатвы пшеницы (\bibref[Лев 23:15--21]{Lev 23:15}) и начатков всех плодов земли
(\bibref{Num 28:26}; \bibref[Втор 26:2,10]{Deu 26:2}).&
\bibstrut май --- июнь\tabularnewline
\hline
\bibstrut IV & \bibstrut 10 &
\bibstrut \bibemph{ФАММУЗ}. 29 дней.
17. Пост. Взятие Иерусалима (\bibref{Zec 8:19}).&
\bibstrut июнь --- июль\tabularnewline
\hline
\bibstrut V & \bibstrut 11 &
\bibstrut \bibemph{АВ}. 30 дней.
9. Пост. Разрушение храма иерусалимского (\bibref{Zec 8:19}).&
\bibstrut июль --- август\tabularnewline
\hline
\bibstrut VI & \bibstrut 12 &
\bibstrut \bibemph{ЭЛУЛ}. 29 дней.&
\bibstrut август --- сентябрь\tabularnewline
\hline
\bibstrut VII & \bibstrut 1 &
\bibstrut {\bfseries АФАНИМ}, \bibemph{ТИШРИ}. 30 дней.
1. Праздник труб (\bibref{Num 29:1}). Новый год.
10. День очищения (\bibref{Lev 16:29}; \bibref[25:9]{Lev 25:9}).
15-22. Праздник кущей (\bibref[Лев 23:34--36]{Lev 23:34}; \bibref[Чис 29:12--35]{Num 29:12}).&
\bibstrut сентябрь --- октябрь\tabularnewline
\hline
\bibstrut VIII & \bibstrut 2 &
\bibstrut {\bfseries БУЛ}, \bibemph{МАРХЕШВАН}. 29 дней.&
\bibstrut октябрь --- ноябрь\tabularnewline
\hline
\bibstrut IX & \bibstrut 3 &
\bibstrut \bibemph{КИСЛЕВ}. 30 дней.
25. Праздник обновления (\bibref[1~Мак 4:52--59]{1Ma 4:52}; \bibref[Ин 10:22]{Joh 10:22}).&
\bibstrut ноябрь --- декабрь\tabularnewline
\hline
\bibstrut X & \bibstrut 4 &
\bibstrut \bibemph{ТЕБЕФ}. 29 дней.&
\bibstrut декабрь --- январь\tabularnewline
\hline
\bibstrut XI & \bibstrut 5 &
\bibstrut \bibemph{ШЕВАТ}. 30 дней.&
\bibstrut январь --- февраль\tabularnewline
\hline
\bibstrut XII & \bibstrut 6 &
\bibstrut \bibemph{АДАР}. 29 дней.
11. Пост Есфири (\bibref{Est 4:16}).
14--15. Праздник Пурим (\bibref[Есф 9:17--32]{Est 9:17}).&
\bibstrut февраль --- март\tabularnewline
\hline
\end{tabular}

\vspace*{6mm}

%\begin{multicols}{2}
В Библии священный год со времени исхода из Египта начинается с весны, с
месяца авив, что значит месяц зрелого колоса
(\bibref{Exo 13:4}; \bibref[12:2]{Exo 12:2}).
Это был месяц весеннего равноденствия и время созревания ячменя
(\bibref[Лев 23:10--14]{Lev 23:10}).
Позже он стал называться нисаном.
В 14-й день этого месяца, который приходится в полнолуние, праздновали
Пасху (\bibref[Исх. гл. 12]{Exo 12:1}).
Другие месяцы названий не имели, о них говорили: второй месяц, десятый месяц
и т.~д.
Лишь в рассказе о постройке храма Соломона при участии финикийцев три месяца
названы особо:
зиф (месяц цветения) --- \bibref{1Ki 6:1},
афаним (месяц бурных ветров) --- \bibref{1Ki 8:2},
и бул (месяц произрастания) --- \bibref{1Ki 6:38}; это финикийские названия.
После плена вавилонского появились ассиро-вавилонские названия месяцев:
нисан (\bibref{Neh 2:1}),
ияр, сиван (\bibref{Est 8:9}), фаммуз или таммуз, ав, элул (\bibref{Neh 6:15}),
тишри, мархешван, кислев или хаслев (\bibref{Neh 1:1}; \bibref{Zec 7:1}),
тебеф (\bibref{Est 2:16}), шеват (\bibref{Zec 1:7})
и адар (\bibref{Est 3:7}).
Те из них, которые не встречаются в Библии, известны по сочинениям Иосифа Флавия
(I век по Р.~Х.) и другим древним источникам.

Гражданский год начинался и кончался осенью
(ср. \bibref{Exo 23:16}; \bibref[34:22]{Exo 34:22}), после уборки урожая
(в месяце тишри).
В Библии встречается счет месяцев и по священному и по гражданскому году.

Начало месяца определялось по появлению видимого серпа новой луны; этот
день, новомесячие, был праздничным (\bibref{Num 10:10}; \bibref[28:11]{Num 28:11}).
От одного новолуния до другого проходит 29 1/2 суток, поэтому месяцы имели
продолжительность в 29 и 30 дней попеременно.
12 лунных месяцев составляют год в 354 дня, что на 11 дней меньше солнечного
года.
За три года разница между луннным и солнечным годом составит целый месяц,
поэтому примерно раз в три года добавлялся 13-й месяц и получался год
продолжительностью в 384 дня.
Это делалось для того, чтобы авив оставался весенним месяцем.

В прилагаемой таблице указаны названия месяцев в гражданском и священном 
году (т.~е. первый месяц, второй и т.~д.), а также древние (ханаанские и
финикийские) и послепленные (ассиро-вавилонские) названия в том виде, в
каком они приведены в русской Библии. Обозначено количество дней в
месяце, перечислены библейские праздники и посты и показано
приблизительное соответствие библейских месяцев современным.

Дни недели, кроме субботы (шаб\'ат), особых наименований не имели, если не
считать существовавшего в эллинистическую эпоху греческого названия дня
перед субботой --- параскев\'и, что значит <<приготовление>>
(к дню покоя --- субботе).
Неделя завершалась субботой, поэтому <<день первый>> (после субботы, см.
\bibref[Мф 28:1]{Mat 28:1}) соответствует нашему воскресенью,
<<день второй>> --- понедельнику и т.~д.

День (в смысле суток) начинался с захода солнца, т.~е. с позднего
вечера. В древности как ночь, так и день делились на три части: ночь на
первую, вторую и третью стражи (\bibref{Jdg 7:19}), а
день --- на утро, полдень и вечер (см. \bibref{Psa 54:18}).
Позднее, со времен римского владычества, ночь делилась на четыре стражи
(\bibref[Лк 12:38]{Luk 12:38}; \bibref[Мф 14:25]{Mat 14:25}) и вошло в употребление понятие <<час>> ---
двенадцатая часть дня или ночи (\bibref[Мф 20:1--8]{Mat 20:1}; \bibref[Деян 23:23]{Act 23:23}).

%\end{multicols}

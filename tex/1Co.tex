\bibbookdescr{1Co}{
  inline={Первое Послание\\к Коринфянам\\\LARGE Святого Апостола Павла},
  toc={1-е Коринфянам},
  bookmark={1-е Коринфянам},
  header={1-е Коринфянам},
  %headerleft={},
  %headerright={},
  abbr={1~Кор}
}
\vs 1Co 1:1 Павел, волею Божиею призванный Апостол Иисуса Христа, и Сосфен брат,
\vs 1Co 1:2 церкви Божией, находящейся в Коринфе, освященным во Христе Иисусе, призванным святым, со всеми призывающими имя Господа нашего Иисуса Христа, во всяком месте, у них и у нас:
\vs 1Co 1:3 благодать вам и мир от Бога Отца нашего и Господа Иисуса Христа.
\rsbpar\vs 1Co 1:4 Непрестанно благодарю Бога моего за вас, ради благодати Божией, дарованной вам во Христе Иисусе,
\vs 1Co 1:5 потому что в Нем вы обогатились всем, всяким словом и всяким познанием,~---
\vs 1Co 1:6 ибо свидетельство Христово утвердилось в вас,~---
\vs 1Co 1:7 так что вы не имеете недостатка ни в каком даровании, ожидая явления Господа нашего Иисуса Христа,
\vs 1Co 1:8 Который и утвердит вас до конца, \bibemph{чтобы вам быть} неповинными в день Господа нашего Иисуса Христа.
\vs 1Co 1:9 Верен Бог, Которым вы призваны в общение Сына Его Иисуса Христа, Господа нашего.
\rsbpar\vs 1Co 1:10 Умоляю вас, братия, именем Господа нашего Иисуса Христа, чтобы все вы говорили одно, и не было между вами разделений, но чтобы вы соединены были в одном духе и в одних мыслях.
\vs 1Co 1:11 Ибо от \bibemph{домашних} Хлоиных сделалось мне известным о вас, братия мои, что между вами есть споры.
\vs 1Co 1:12 Я разумею то, что у вас говорят: <<я Павлов>>; <<я Аполлосов>>; <<я Кифин>>; <<а я Христов>>.
\vs 1Co 1:13 Разве разделился Христос? разве Павел распялся за вас? или во имя Павла вы крестились?
\vs 1Co 1:14 Благодарю Бога, что я никого из вас не крестил, кроме Криспа и Гаия,
\vs 1Co 1:15 дабы не сказал кто, что я крестил в мое имя.
\vs 1Co 1:16 Крестил я также Стефанов дом; а крестил ли еще кого, не знаю.
\vs 1Co 1:17 Ибо Христос послал меня не крестить, а благовествовать, не в премудрости слова, чтобы не упразднить креста Христова.
\vs 1Co 1:18 Ибо слово о кресте для погибающих юродство есть, а для нас, спасаемых,~--- сила Божия.
\vs 1Co 1:19 Ибо написано: погублю мудрость мудрецов, и разум разумных отвергну.
\vs 1Co 1:20 Где мудрец? где книжник? где совопросник века сего? Не обратил ли Бог мудрость мира сего в безумие?
\vs 1Co 1:21 Ибо когда мир \bibemph{своею} мудростью не познал Бога в премудрости Божией, то благоугодно было Богу юродством проповеди спасти верующих.
\vs 1Co 1:22 Ибо и Иудеи требуют чудес, и Еллины ищут мудрости;
\vs 1Co 1:23 а мы проповедуем Христа распятого, для Иудеев соблазн, а для Еллинов безумие,
\vs 1Co 1:24 для самих же призванных, Иудеев и Еллинов, Христа, Божию силу и Божию премудрость;
\vs 1Co 1:25 потому что немудрое Божие премудрее человеков, и немощное Божие сильнее человеков.
\rsbpar\vs 1Co 1:26 Посмотрите, братия, кто вы, призванные: не много \bibemph{из вас} мудрых по плоти, не много сильных, не много благородных;
\vs 1Co 1:27 но Бог избрал немудрое мира, чтобы посрамить мудрых, и немощное мира избрал Бог, чтобы посрамить сильное;
\vs 1Co 1:28 и незнатное мира и уничиженное и ничего не значащее избрал Бог, чтобы упразднить значащее,~---
\vs 1Co 1:29 для того, чтобы никакая плоть не хвалилась пред Богом.
\vs 1Co 1:30 От Него и вы во Христе Иисусе, Который сделался для нас премудростью от Бога, праведностью и освящением и искуплением,
\vs 1Co 1:31 чтобы \bibemph{было}, как написано: хвалящийся хвались Господом.
\vs 1Co 2:1 И когда я приходил к вам, братия, приходил возвещать вам свидетельство Божие не в превосходстве слова или мудрости,
\vs 1Co 2:2 ибо я рассудил быть у вас незнающим ничего, кроме Иисуса Христа, и притом распятого,
\vs 1Co 2:3 и был я у вас в немощи и в страхе и в великом трепете.
\vs 1Co 2:4 И слово мое и проповедь моя не в убедительных словах человеческой мудрости, но в явлении духа и силы,
\vs 1Co 2:5 чтобы вера ваша \bibemph{утверждалась} не на мудрости человеческой, но на силе Божией.
\rsbpar\vs 1Co 2:6 Мудрость же мы проповедуем между совершенными, но мудрость не века сего и не властей века сего преходящих,
\vs 1Co 2:7 но проповедуем премудрость Божию, тайную, сокровенную, которую предназначил Бог прежде веков к славе нашей,
\vs 1Co 2:8 которой никто из властей века сего не познал; ибо если бы познали, то не распяли бы Господа славы.
\vs 1Co 2:9 Но, как написано: не видел того глаз, не слышало ухо, и не приходило то на сердце человеку, что приготовил Бог любящим Его.
\vs 1Co 2:10 А нам Бог открыл \bibemph{это} Духом Своим; ибо Дух все проницает, и глубины Божии.
\vs 1Co 2:11 Ибо кто из человеков знает, чт\acc{о} в человеке, кроме духа человеческого, живущего в нем? Т\acc{а}к и Божьего никто не знает, кроме Духа Божия.
\vs 1Co 2:12 Но мы приняли не духа мира сего, а Духа от Бога, дабы знать дарованное нам от Бога,
\vs 1Co 2:13 что и возвещаем не от человеческой мудрости изученными словами, но изученными от Духа Святаго, соображая духовное с духовным.
\vs 1Co 2:14 Душевный человек не принимает того, чт\acc{о} от Духа Божия, потому что он почитает это безумием; и не может разуметь, потому что о сем \bibemph{надобно} судить духовно.
\vs 1Co 2:15 Но духовный судит о всем, а о нем судить никто не может.
\vs 1Co 2:16 Ибо кто познал ум Господень, чтобы \bibemph{мог} судить его? А мы имеем ум Христов.
\vs 1Co 3:1 И я не мог говорить с вами, братия, как с духовными, но как с плотскими, как с младенцами во Христе.
\vs 1Co 3:2 Я питал вас молоком, а не \bibemph{твердою} пищею, ибо вы были еще не в силах, да и теперь не в силах,
\vs 1Co 3:3 потому что вы еще плотские. Ибо если между вами зависть, споры и разногласия, то не плотские ли вы? и не по человеческому ли \bibemph{обычаю} поступаете?
\vs 1Co 3:4 Ибо когда один говорит: <<я Павлов>>, а другой: <<я Аполлосов>>, то не плотские ли вы?
\vs 1Co 3:5 Кто Павел? кто Аполлос? Они только служители, через которых вы уверовали, и притом поскольку каждому дал Господь.
\vs 1Co 3:6 Я насадил, Аполлос поливал, но возрастил Бог;
\vs 1Co 3:7 посему и насаждающий и поливающий есть ничто, а \bibemph{все} Бог возращающий.
\vs 1Co 3:8 Насаждающий же и поливающий суть одно; но каждый получит свою награду по своему труду.
\vs 1Co 3:9 Ибо мы соработники у Бога, \bibemph{а} вы Божия нива, Божие строение.
\rsbpar\vs 1Co 3:10 Я, по данной мне от Бога благодати, как мудрый строитель, положил основание, а другой строит на \bibemph{нем}; но каждый смотри, к\acc{а}к строит.
\vs 1Co 3:11 Ибо никто не может положить другого основания, кроме положенного, которое есть Иисус Христос.
\vs 1Co 3:12 Строит ли кто на этом основании из золота, серебра, драгоценных камней, дерева, сена, соломы,~---
\vs 1Co 3:13 каждого дело обнаружится; ибо день покажет, потому что в огне открывается, и огонь испытает дело каждого, каково оно есть.
\vs 1Co 3:14 У кого дело, которое он строил, устоит, тот получит награду.
\vs 1Co 3:15 А у кого дело сгорит, тот потерпит урон; впрочем сам спасется, но т\acc{а}к, как бы из огня.
\rsbpar\vs 1Co 3:16 Разве не знаете, что вы храм Божий, и Дух Божий живет в вас?
\vs 1Co 3:17 Если кто разорит храм Божий, того покарает Бог: ибо храм Божий свят; а этот \bibemph{храм}~--- вы.
\rsbpar\vs 1Co 3:18 Никто не обольщай самого себя. Если кто из вас думает быть мудрым в веке сем, тот будь безумным, чтобы быть мудрым.
\vs 1Co 3:19 Ибо мудрость мира сего есть безумие пред Богом, как написано: уловляет мудрых в лукавстве их.
\vs 1Co 3:20 И еще: Господь знает умствования мудрецов, что они суетны.
\vs 1Co 3:21 Итак никто не хвались человеками, ибо все ваше:
\vs 1Co 3:22 Павел ли, или Аполлос, или Кифа, или мир, или жизнь, или смерть, или настоящее, или будущее,~--- все ваше;
\vs 1Co 3:23 вы же~--- Христовы, а Христос~--- Божий.
\vs 1Co 4:1 Итак каждый должен разуметь нас, как служителей Христовых и домостроителей таин Божиих.
\vs 1Co 4:2 От домостроителей же требуется, чтобы каждый оказался верным.
\vs 1Co 4:3 Для меня очень мало значит, к\acc{а}к судите обо мне вы или \bibemph{к\acc{а}к судят} другие люди; я и сам не сужу о себе.
\vs 1Co 4:4 Ибо \bibemph{хотя} я ничего не знаю за собою, но тем не оправдываюсь; судия же мне Господь.
\vs 1Co 4:5 Посему не суд\acc{и}те никак прежде времени, пока не придет Господь, Который и осветит скрытое во мраке и обнаружит сердечные намерения, и тогда каждому будет похвала от Бога.
\rsbpar\vs 1Co 4:6 Это, братия, приложил я к себе и Аполлосу ради вас, чтобы вы научились от нас не мудрствовать сверх того, что написано, и не превозносились один перед другим.
\vs 1Co 4:7 Ибо кто отличает тебя? Что ты имеешь, чего бы не получил? А если получил, что хвалишься, как будто не получил?
\vs 1Co 4:8 Вы уже пресытились, вы уже обогатились, вы стали царствовать без нас. О, если бы вы \bibemph{и в самом деле} царствовали, чтобы и нам с вами царствовать!
\vs 1Co 4:9 Ибо я думаю, что нам, последним посланникам, Бог судил быть как бы приговоренными к смерти, потому что мы сделались позорищем для мира, для Ангелов и человеков.
\vs 1Co 4:10 Мы безумны Христа ради, а вы мудры во Христе; мы немощны, а вы крепки; вы в славе, а мы в бесчестии.
\vs 1Co 4:11 Даже доныне терпим голод и жажду, и наготу и побои, и скитаемся,
\vs 1Co 4:12 и трудимся, работая своими руками. Злословят нас, мы благословляем; гонят нас, мы терпим;
\vs 1Co 4:13 хулят нас, мы молим; мы как сор для мира, \bibemph{как} прах, всеми \bibemph{попираемый} доныне.
\rsbpar\vs 1Co 4:14 Не к постыжению вашему пишу сие, но вразумляю вас, как возлюбленных детей моих.
\vs 1Co 4:15 Ибо, хотя у вас тысячи наставников во Христе, но не много отцов; я родил вас во Христе Иисусе благовествованием.
\vs 1Co 4:16 Посему умоляю вас: подражайте мне, как я Христу.
\vs 1Co 4:17 Для сего я послал к вам Тимофея, моего возлюбленного и верного в Господе сына, который напомнит вам о путях моих во Христе, как я учу везде во всякой церкви.
\vs 1Co 4:18 Как я не иду к вам, то некоторые \bibemph{у вас} возгордились;
\vs 1Co 4:19 но я скоро приду к вам, если угодно будет Господу, и испытаю не слова возгордившихся, а силу,
\vs 1Co 4:20 ибо Царство Божие не в слове, а в силе.
\vs 1Co 4:21 Чего вы хотите? с жезлом прийти к вам, или с любовью и духом кротости?
\vs 1Co 5:1 Есть верный слух, что у вас \bibemph{появилось} блудодеяние, и притом такое блудодеяние, какого не слышно даже у язычников, что некто \bibemph{вместо жены} имеет жену отца своего.
\vs 1Co 5:2 И вы возгордились, вместо того, чтобы лучше плакать, дабы изъят был из среды вас сделавший такое дело.
\vs 1Co 5:3 А я, отсутствуя телом, но присутствуя \bibemph{у вас} духом, уже решил, как бы находясь у вас: сделавшего такое дело,
\vs 1Co 5:4 в собрании вашем во имя Господа нашего Иисуса Христа, обще с моим духом, силою Господа нашего Иисуса Христа,
\vs 1Co 5:5 предать сатане во измождение плоти, чтобы дух был спасен в день Господа нашего Иисуса Христа.
\vs 1Co 5:6 Нечем вам хвалиться. Разве не знаете, что малая закваска квасит все тесто?
\vs 1Co 5:7 Итак очистите старую закваску, чтобы быть вам новым тестом, так как вы бесквасны, ибо Пасха наша, Христос, заклан за нас.
\vs 1Co 5:8 Посему станем праздновать не со старою закваскою, не с закваскою порока и лукавства, но с опресноками чистоты и истины.
\rsbpar\vs 1Co 5:9 Я писал вам в послании~--- не сообщаться с блудниками;
\vs 1Co 5:10 впрочем не вообще с блудниками мира сего, или лихоимцами, или хищниками, или идолослужителями, ибо иначе надлежало бы вам выйти из мира \bibemph{сего}.
\vs 1Co 5:11 Но я писал вам не сообщаться с тем, кто, называясь братом, остается блудником, или лихоимцем, или идолослужителем, или злоречивым, или пьяницею, или хищником; с таким даже и не есть вместе.
\vs 1Co 5:12 Ибо чт\acc{о} мне судить и внешних? Не внутренних ли вы судите?
\vs 1Co 5:13 Внешних же судит Бог. Итак, извергните развращенного из среды вас.
\vs 1Co 6:1 Как смеет кто у вас, имея дело с другим, судиться у нечестивых, а не у святых?
\vs 1Co 6:2 Разве не знаете, что святые будут судить мир? Если же вами будет судим мир, то неужели вы недостойны судить маловажные \bibemph{дела}?
\vs 1Co 6:3 Разве не знаете, что мы будем судить ангелов, не тем ли более \bibemph{дела} житейские?
\vs 1Co 6:4 А вы, когда имеете житейские тяжбы, поставляете \bibemph{своими судьями} ничего не значащих в церкви.
\vs 1Co 6:5 К стыду вашему говорю: неужели нет между вами ни одного разумного, который мог бы рассудить между братьями своими?
\vs 1Co 6:6 Но брат с братом судится, и притом перед неверными.
\vs 1Co 6:7 И то уже весьма унизительно для вас, что вы имеете тяжбы между собою. Для чего бы вам лучше не оставаться обиженными? для чего бы вам лучше не терпеть лишения?
\vs 1Co 6:8 Но вы \bibemph{сами} обижаете и отнимаете, и притом у братьев.
\vs 1Co 6:9 Или не знаете, что неправедные Царства Божия не наследуют? Не обманывайтесь: ни блудники, ни идолослужители, ни прелюбодеи, ни малакии, ни мужеложники,
\vs 1Co 6:10 ни воры, ни лихоимцы, ни пьяницы, ни злоречивые, ни хищники~--- Царства Божия не наследуют.
\vs 1Co 6:11 И такими были некоторые из вас; но омылись, но освятились, но оправдались именем Господа нашего Иисуса Христа и Духом Бога нашего.
\rsbpar\vs 1Co 6:12 Все мне позволительно, но не все полезно; все мне позволительно, но ничто не должно обладать мною.
\vs 1Co 6:13 Пища для чрева, и чрево для пищи; но Бог уничтожит и то и другое. Тело же не для блуда, но для Господа, и Господь для тела.
\vs 1Co 6:14 Бог воскресил Господа, воскресит и нас силою Своею.
\rsbpar\vs 1Co 6:15 Разве не знаете, что тел\acc{а} ваши суть члены Христовы? Итак отниму ли члены у Христа, чтобы сделать \bibemph{их} членами блудницы? Да не будет!
\vs 1Co 6:16 Или не знаете, что совокупляющийся с блудницею становится одно тело \bibemph{с нею}? ибо сказано: два будут одна плоть.
\vs 1Co 6:17 А соединяющийся с Господом есть один дух с Господом.
\vs 1Co 6:18 Бегайте блуда; всякий грех, какой делает человек, есть вне тела, а блудник грешит против собственного тела.
\vs 1Co 6:19 Не знаете ли, что тел\acc{а} ваши суть храм живущего в вас Святаго Духа, Которого имеете вы от Бога, и вы не свои?
\vs 1Co 6:20 Ибо вы куплены \bibemph{дорогою} ценою. Посему прославляйте Бога и в телах ваших и в душах ваших, которые суть Божии.
\vs 1Co 7:1 А о чем вы писали ко мне, то хорошо человеку не касаться женщины.
\vs 1Co 7:2 Но, \bibemph{во избежание} блуда, каждый имей свою жену, и каждая имей своего мужа.
\vs 1Co 7:3 Муж оказывай жене должное благорасположение; подобно и жена мужу.
\vs 1Co 7:4 Жена не властна над своим телом, но муж; равно и муж не властен над своим телом, но жена.
\vs 1Co 7:5 Не уклоняйтесь друг от друга, разве по согласию, на время, для упражнения в посте и молитве, а \bibemph{потом} опять будьте вместе, чтобы не искушал вас сатана невоздержанием вашим.
\vs 1Co 7:6 Впрочем это сказано мною как позволение, а не как повеление.
\vs 1Co 7:7 Ибо желаю, чтобы все люди были, как и я; но каждый имеет свое дарование от Бога, один так, другой иначе.
\rsbpar\vs 1Co 7:8 Безбрачным же и вдовам говорю: хорошо им оставаться, как я.
\vs 1Co 7:9 Но если не \bibemph{могут} воздержаться, пусть вступают в брак; ибо лучше вступить в брак, нежели разжигаться.
\vs 1Co 7:10 А вступившим в брак не я повелеваю, а Господь: жене не разводиться с мужем,~---
\vs 1Co 7:11 если же разведется, то должна оставаться безбрачною, или примириться с мужем своим,~--- и мужу не оставлять жены \bibemph{своей}.
\vs 1Co 7:12 Прочим же я говорю, а не Господь: если какой брат имеет жену неверующую, и она согласна жить с ним, то он не должен оставлять ее;
\vs 1Co 7:13 и жена, которая имеет мужа неверующего, и он согласен жить с нею, не должна оставлять его.
\vs 1Co 7:14 Ибо неверующий муж освящается женою верующею, и жена неверующая освящается мужем верующим. Иначе дети ваши были бы нечисты, а теперь святы.
\vs 1Co 7:15 Если же неверующий \bibemph{хочет} развестись, пусть разводится; брат или сестра в таких \bibemph{случаях} не связаны; к миру призвал нас Господь.
\vs 1Co 7:16 Почему ты знаешь, жена, не спасешь ли мужа? Или ты, муж, почему знаешь, не спасешь ли жены?
\vs 1Co 7:17 Только каждый поступай так, как Бог ему определил, и каждый, как Господь призвал. Так я повелеваю по всем церквам.
\vs 1Co 7:18 Призван ли кто обрезанным, не скрывайся; призван ли кто необрезанным, не обрезывайся.
\vs 1Co 7:19 Обрезание ничто и необрезание ничто, но \bibemph{всё} в соблюдении заповедей Божиих.
\vs 1Co 7:20 Каждый оставайся в том звании, в котором призван.
\vs 1Co 7:21 Рабом ли ты призван, не смущайся; но если и можешь сделаться свободным, то лучшим воспользуйся.
\vs 1Co 7:22 Ибо раб, призванный в Господе, есть свободный Господа; равно и призванный свободным есть раб Христов.
\vs 1Co 7:23 Вы куплены \bibemph{дорогою} ценою; не делайтесь рабами человеков.
\vs 1Co 7:24 В каком \bibemph{звании} кто призван, братия, в том каждый и оставайся пред Богом.
\rsbpar\vs 1Co 7:25 Относительно девства я не имею повеления Господня, а даю совет, как получивший от Господа милость быть \bibemph{Ему} верным.
\vs 1Co 7:26 По настоящей нужде за лучшее призна\acc{ю}, что хорошо человеку оставаться т\acc{а}к.
\vs 1Co 7:27 Соединен ли ты с женой? не ищи развода. Остался ли без жены? не ищи жены.
\vs 1Co 7:28 Впрочем, если и женишься, не согрешишь; и если девица выйдет замуж, не согрешит. Но таковые будут иметь скорби по плоти; а мне вас жаль.
\rsbpar\vs 1Co 7:29 Я вам сказываю, братия: время уже коротко, так что имеющие жен должны быть, как не имеющие;
\vs 1Co 7:30 и плачущие, как не плачущие; и радующиеся, как не радующиеся; и покупающие, как не приобретающие;
\vs 1Co 7:31 и пользующиеся миром сим, как не пользующиеся; ибо проходит образ мира сего.
\vs 1Co 7:32 А я хочу, чтобы вы были без забот. Неженатый заботится о Господнем, как угодить Господу;
\vs 1Co 7:33 а женатый заботится о мирском, как угодить жене. Есть разность между замужнею и девицею:
\vs 1Co 7:34 незамужняя заботится о Господнем, как угодить Господу, чтобы быть святою и телом и духом; а замужняя заботится о мирском, как угодить мужу.
\vs 1Co 7:35 Говорю это для вашей же пользы, не с тем, чтобы наложить на вас узы, но чтобы вы благочинно и непрестанно \bibemph{служили} Господу без развлечения.
\vs 1Co 7:36 Если же кто почитает неприличным для своей девицы то, чтобы она, будучи в зрелом возрасте, оставалась так, тот пусть делает, как хочет: не согрешит; пусть \bibemph{таковые} выходят замуж.
\vs 1Co 7:37 Но кто непоколебимо тверд в сердце своем и, не будучи стесняем нуждою, но будучи властен в своей воле, решился в сердце своем соблюдать свою деву, тот хорошо поступает.
\vs 1Co 7:38 Посему выдающий замуж свою девицу поступает хорошо; а не выдающий поступает лучше.
\vs 1Co 7:39 Жена связана законом, доколе жив муж ее; если же муж ее умрет, свободна выйти, за кого хочет, только в Господе.
\vs 1Co 7:40 Но она блаженнее, если останется так, по моему совету; а думаю, и я имею Духа Божия.
\vs 1Co 8:1 О идоложертвенных \bibemph{яствах} мы знаем, потому что мы все имеем знание; но знание надмевает, а любовь назидает.
\vs 1Co 8:2 Кто думает, что он знает что-нибудь, тот ничего еще не знает так, как должно знать.
\vs 1Co 8:3 Но кто любит Бога, тому дано знание от Него.
\vs 1Co 8:4 Итак об употреблении в пищу идоложертвенного мы знаем, что идол в мире ничто, и что нет иного Бога, кроме Единого.
\vs 1Co 8:5 Ибо хотя и есть так называемые боги, или на небе, или на земле, так как есть много богов и господ много,~---
\vs 1Co 8:6 но у нас один Бог Отец, из Которого все, и мы для Него, и один Господь Иисус Христос, Которым все, и мы Им.
\vs 1Co 8:7 Но не у всех \bibemph{такое} знание: некоторые и доныне с совестью, \bibemph{признающею} идолов, едят \bibemph{идоложертвенное} как жертвы идольские, и совесть их, будучи немощна, оскверняется.
\vs 1Co 8:8 Пища не приближает нас к Богу: ибо, едим ли мы, ничего не приобретаем; не едим ли, ничего не теряем.
\vs 1Co 8:9 Берегитесь однако же, чтобы эта свобода ваша не послужила соблазном для немощных.
\vs 1Co 8:10 Ибо если кто-нибудь увидит, что ты, имея знание, сидишь за столом в капище, то совесть его, как немощного, не расположит ли и его есть идоложертвенное?
\vs 1Co 8:11 И от знания твоего погибнет немощный брат, за которого умер Христос.
\vs 1Co 8:12 А согрешая таким образом против братьев и уязвляя немощную совесть их, вы согрешаете против Христа.
\vs 1Co 8:13 И потому, если пища соблазняет брата моего, не буду есть мяса вовек, чтобы не соблазнить брата моего.
\vs 1Co 9:1 Не Апостол ли я? Не свободен ли я? Не видел ли я Иисуса Христа, Господа нашего? Не мое ли дело вы в Господе?
\vs 1Co 9:2 Если для других я не Апостол, то для вас \bibemph{Апостол}; ибо печать моего апостольства~--- вы в Господе.
\vs 1Co 9:3 Вот мое защищение против осуждающих меня.
\vs 1Co 9:4 Или мы не имеем власти есть и пить?
\vs 1Co 9:5 Или не имеем власти иметь спутницею сестру жену, как и прочие Апостолы, и братья Господни, и Кифа?
\vs 1Co 9:6 Или один я и Варнава не имеем власти не работать?
\vs 1Co 9:7 Какой воин служит когда-либо на своем содержании? Кто, насадив виноград, не ест плодов его? Кто, пася стадо, не ест молока от стада?
\vs 1Co 9:8 По человеческому ли только \bibemph{рассуждению} я это говорю? Не то же ли говорит и закон?
\vs 1Co 9:9 Ибо в Моисеевом законе написано: не заграждай рта у вола молотящего. О волах ли печется Бог?
\vs 1Co 9:10 Или, конечно, для нас говорится? Так, для нас это написано; ибо, кто пашет, должен пахать с надеждою, и кто молотит, \bibemph{должен молотить} с надеждою получить ожидаемое.
\vs 1Co 9:11 Если мы посеяли в вас духовное, велико ли то, если пожнем у вас телесное?
\vs 1Co 9:12 Если другие имеют у вас власть, не паче ли мы? Однако мы не пользовались сею властью, но все переносим, дабы не поставить какой преграды благовествованию Христову.
\vs 1Co 9:13 Разве не знаете, что священнодействующие питаются от святилища? что служащие жертвеннику берут долю от жертвенника?
\vs 1Co 9:14 Т\acc{а}к и Господь повелел проповедующим Евангелие жить от благовествования.
\vs 1Co 9:15 Но я не пользовался ничем таковым. И написал это не для того, чтобы т\acc{а}к было для меня. Ибо для меня лучше умереть, нежели чтобы кто уничтожил похвалу мою.
\vs 1Co 9:16 Ибо если я благовествую, то нечем мне хвалиться, потому что это необходимая \bibemph{обязанность} моя, и горе мне, если не благовествую!
\vs 1Co 9:17 Ибо если делаю это добровольно, то \bibemph{буду} иметь награду; а если недобровольно, то \bibemph{исполняю только} вверенное мне служение.
\vs 1Co 9:18 За чт\acc{о} же мне награда? За т\acc{о}, что, проповедуя Евангелие, благовествую о Христе безмездно, не пользуясь моею властью в благовествовании.
\vs 1Co 9:19 Ибо, будучи свободен от всех, я всем поработил себя, дабы больше приобрести:
\vs 1Co 9:20 для Иудеев я был как Иудей, чтобы приобрести Иудеев; для подзаконных был как подзаконный, чтобы приобрести подзаконных;
\vs 1Co 9:21 для чуждых закона~--- как чуждый закона,~--- не будучи чужд закона пред Богом, но подзаконен Христу,~--- чтобы приобрести чуждых закона;
\vs 1Co 9:22 для немощных был как немощный, чтобы приобрести немощных. Для всех я сделался всем, чтобы спасти по крайней мере некоторых.
\vs 1Co 9:23 Сие же делаю для Евангелия, чтобы быть соучастником его.
\vs 1Co 9:24 Не знаете ли, что бегущие на ристалище бегут все, но один получает награду? Так бегите, чтобы получить.
\vs 1Co 9:25 Все подвижники воздерживаются от всего: те для получения венца тленного, а мы~--- нетленного.
\vs 1Co 9:26 И потому я бегу не так, как на неверное, бьюсь не так, чтобы только бить воздух;
\vs 1Co 9:27 но усмиряю и порабощаю тело мое, дабы, проповедуя другим, самому не остаться недостойным.
\vs 1Co 10:1 Не хочу оставить вас, братия, в неведении, что отцы наши все были под облаком, и все прошли сквозь море;
\vs 1Co 10:2 и все крестились в Моисея в облаке и в море;
\vs 1Co 10:3 и все ели одну и ту же духовную пищу;
\vs 1Co 10:4 и все пили одно и то же духовное питие: ибо пили из духовного последующего камня; камень же был Христос.
\vs 1Co 10:5 Но не о многих из них благоволил Бог, ибо они поражены были в пустыне.
\vs 1Co 10:6 А это были образы для нас, чтобы мы не были похотливы на злое, как они были похотливы.
\vs 1Co 10:7 Не будьте также идолопоклонниками, как некоторые из них, о которых написано: народ сел есть и пить, и встал играть.
\vs 1Co 10:8 Не станем блудодействовать, как некоторые из них блудодействовали, и в один день погибло их двадцать три тысячи.
\vs 1Co 10:9 Не станем искушать Христа, как некоторые из них искушали и погибли от змей.
\vs 1Co 10:10 Не ропщите, как некоторые из них роптали и погибли от истребителя.
\vs 1Co 10:11 Все это происходило с ними, \bibemph{как} образы; а описано в наставление нам, достигшим последних веков.
\vs 1Co 10:12 Посему, кто думает, что он сто\acc{и}т, берегись, чтобы не упасть.
\vs 1Co 10:13 Вас постигло искушение не иное, как человеческое; и верен Бог, Который не попустит вам быть искушаемыми сверх сил, но при искушении даст и облегчение, так чтобы вы могли перенести.
\rsbpar\vs 1Co 10:14 Итак, возлюбленные мои, убегайте идолослужения.
\vs 1Co 10:15 Я говорю \bibemph{вам} как рассудительным; сами рассуд\acc{и}те о том, что говорю.
\vs 1Co 10:16 Чаша благословения, которую благословляем, не есть ли приобщение Крови Христовой? Хлеб, который преломляем, не есть ли приобщение Тела Христова?
\vs 1Co 10:17 Один хлеб, и мы многие одно тело; ибо все причащаемся от одного хлеба.
\vs 1Co 10:18 Посмотрите на Израиля по плоти: те, которые едят жертвы, не участники ли жертвенника?
\vs 1Co 10:19 Что же я говорю? То ли, что идол есть что-нибудь, или идоложертвенное значит что-нибудь?
\vs 1Co 10:20 \bibemph{Нет}, но что язычники, принося жертвы, приносят бесам, а не Богу. Но я не хочу, чтобы вы были в общении с бесами.
\vs 1Co 10:21 Не можете пить чашу Господню и чашу бесовскую; не можете быть участниками в трапезе Господней и в трапезе бесовской.
\vs 1Co 10:22 Неужели мы \bibemph{решимся} раздражать Господа? Разве мы сильнее Его?
\rsbpar\vs 1Co 10:23 Все мне позволительно, но не все полезно; все мне позволительно, но не все назидает.
\vs 1Co 10:24 Никто не ищи своего, но каждый \bibemph{пользы} другого.
\vs 1Co 10:25 Все, что продается на торгу, ешьте без всякого исследования, для \bibemph{спокойствия} совести;
\vs 1Co 10:26 ибо Господня земля, и чт\acc{о} наполняет ее.
\vs 1Co 10:27 Если кто из неверных позовет вас, и вы захотите пойти, то все, предлагаемое вам, ешьте без всякого исследования, для \bibemph{спокойствия} совести.
\vs 1Co 10:28 Но если кто скажет вам: это идоложертвенное,~--- то не ешьте ради того, кто объявил вам, и ради совести. Ибо Господня земля, и чт\acc{о} наполняет ее.
\vs 1Co 10:29 Совесть же разумею не свою, а другого: ибо для чего моей свободе быть судимой чужою совестью?
\vs 1Co 10:30 Если я с благодарением принимаю \bibemph{пищу}, то для чего порицать меня за то, за что я благодарю?
\vs 1Co 10:31 Итак, едите ли, пьете ли, или иное что делаете, все делайте в славу Божию.
\vs 1Co 10:32 Не подавайте соблазна ни Иудеям, ни Еллинам, ни церкви Божией,
\vs 1Co 10:33 так, как и я угождаю всем во всем, ища не своей пользы, но \bibemph{пользы} многих, чтобы они спаслись.
\vs 1Co 11:1 Будьте подражателями мне, как я Христу.
\rsbpar\vs 1Co 11:2 Хвалю вас, братия, что вы все мое помните и держите предания так, как я передал вам.
\vs 1Co 11:3 Хочу также, чтобы вы знали, что всякому мужу глава Христос, жене глава~--- муж, а Христу глава~--- Бог.
\vs 1Co 11:4 Всякий муж, молящийся или пророчествующий с покрытою головою, постыжает свою голову.
\vs 1Co 11:5 И всякая жена, молящаяся или пророчествующая с открытою головою, постыжает свою голову, ибо \bibemph{это} то же, как если бы она была обритая.
\vs 1Co 11:6 Ибо если жена не хочет покрываться, то пусть и стрижется; а если жене стыдно быть остриженной или обритой, пусть покрывается.
\vs 1Co 11:7 Итак муж не должен покрывать голову, потому что он есть образ и слава Божия; а жена есть слава мужа.
\vs 1Co 11:8 Ибо не муж от жены, но жена от мужа;
\vs 1Co 11:9 и не муж создан для жены, но жена для мужа.
\vs 1Co 11:10 Посему жена и должна иметь на голове своей \bibemph{знак} власти \bibemph{над нею}, для Ангелов.
\vs 1Co 11:11 Впрочем ни муж без жены, ни жена без мужа, в Господе.
\vs 1Co 11:12 Ибо как жена от мужа, так и муж через жену; все же~--- от Бога.
\vs 1Co 11:13 Рассудите сами, прилично ли жене молиться Богу с непокрытою \bibemph{головою}?
\vs 1Co 11:14 Не сама ли природа учит вас, что если муж растит волосы, то это бесчестье для него,
\vs 1Co 11:15 но если жена растит волосы, для нее это честь, так как волосы даны ей вместо покрывала?
\vs 1Co 11:16 А если бы кто захотел спорить, то мы не имеем такого обычая, ни церкви Божии.
\rsbpar\vs 1Co 11:17 Но, предлагая сие, не хвалю \bibemph{вас}, что вы собираетесь не на лучшее, а на худшее.
\vs 1Co 11:18 Ибо, во-первых, слышу, что, когда вы собираетесь в церковь, между вами бывают разделения, чему отчасти и верю.
\vs 1Co 11:19 Ибо надлежит быть и разномыслиям между вами, дабы открылись между вами искусные.
\vs 1Co 11:20 Далее, вы собираетесь, \bibemph{так, что это} не значит вкушать вечерю Господню;
\vs 1Co 11:21 ибо всякий поспешает прежде \bibemph{других} есть свою пищу, \bibemph{так что} иной бывает голоден, а иной упивается.
\vs 1Co 11:22 Разве у вас нет домов на то, чтобы есть и пить? Или пренебрегаете церковь Божию и унижаете неимущих? Чт\acc{о} сказать вам? похвалить ли вас за это? Не похвалю.
\vs 1Co 11:23 Ибо я от \bibemph{Самого} Господа принял т\acc{о}, что и вам передал, что Господь Иисус в ту ночь, в которую предан был, взял хлеб
\vs 1Co 11:24 и, возблагодарив, преломил и сказал: приимите, ядите, сие есть Тело Мое, за вас ломимое; сие творите в Мое воспоминание.
\vs 1Co 11:25 Также и чашу после вечери, и сказал: сия чаша есть новый завет в Моей Крови; сие творите, когда только будете пить, в Мое воспоминание.
\vs 1Co 11:26 Ибо всякий раз, когда вы едите хлеб сей и пьете чашу сию, смерть Господню возвещаете, доколе Он придет.
\vs 1Co 11:27 Посему, кто будет есть хлеб сей или пить чашу Господню недостойно, виновен будет против Тела и Крови Господней.
\vs 1Co 11:28 Да испытывает же себя человек, и таким образом пусть ест от хлеба сего и пьет из чаши сей.
\vs 1Co 11:29 Ибо, кто ест и пьет недостойно, тот ест и пьет осуждение себе, не рассуждая о Теле Господнем.
\vs 1Co 11:30 Оттого многие из вас немощны и больны и немало умирает.
\vs 1Co 11:31 Ибо если бы мы судили сами себя, то не были бы судимы.
\vs 1Co 11:32 Будучи же судимы, наказываемся от Господа, чтобы не быть осужденными с миром.
\vs 1Co 11:33 Посему, братия мои, собираясь на вечерю, друг друга ждите.
\vs 1Co 11:34 А если кто голоден, пусть ест дома, чтобы собираться вам не на осуждение. Прочее устрою, когда приду.
\vs 1Co 12:1 Не хочу оставить вас, братия, в неведении и о \bibemph{дарах} духовных.
\vs 1Co 12:2 Знаете, что когда вы были язычниками, то ходили к безгласным идолам, так, как бы вели вас.
\vs 1Co 12:3 Потому сказываю вам, что никто, говорящий Духом Божиим, не произнесет анафемы на Иисуса, и никто не может назвать Иисуса Господом, как только Духом Святым.
\vs 1Co 12:4 Дары различны, но Дух один и тот же;
\vs 1Co 12:5 и служения различны, а Господь один и тот же;
\vs 1Co 12:6 и действия различны, а Бог один и тот же, производящий все во всех.
\vs 1Co 12:7 Но каждому дается проявление Духа на пользу.
\vs 1Co 12:8 Одному дается Духом слово мудрости, другому слово знания, тем же Духом;
\vs 1Co 12:9 иному вера, тем же Духом; иному дары исцелений, тем же Духом;
\vs 1Co 12:10 иному чудотворения, иному пророчество, иному различение духов, иному разные языки, иному истолкование языков.
\vs 1Co 12:11 Все же сие производит один и тот же Дух, разделяя каждому особо, как Ему угодно.
\vs 1Co 12:12 Ибо, как тело одно, но имеет многие члены, и все члены одного тела, хотя их и много, составляют одно тело,~--- так и Христос.
\vs 1Co 12:13 Ибо все мы одним Духом крестились в одно тело, Иудеи или Еллины, рабы или свободные, и все напоены одним Духом.
\vs 1Co 12:14 Тело же не из одного члена, но из многих.
\vs 1Co 12:15 Если нога скажет: я не принадлежу к телу, потому что я не рука, то неужели она потому не принадлежит к телу?
\vs 1Co 12:16 И если ухо скажет: я не принадлежу к телу, потому что я не глаз, то неужели оно потому не принадлежит к телу?
\vs 1Co 12:17 Если все тело глаз, то где слух? Если все слух, то где обоняние?
\vs 1Co 12:18 Но Бог расположил члены, каждый в \bibemph{составе} тела, как Ему было угодно.
\vs 1Co 12:19 А если бы все были один член, то где \bibemph{было бы} тело?
\vs 1Co 12:20 Но теперь членов много, а тело одно.
\vs 1Co 12:21 Не может глаз сказать руке: ты мне не надобна; или также голова ногам: вы мне не нужны.
\vs 1Co 12:22 Напротив, члены тела, которые кажутся слабейшими, гораздо нужнее,
\vs 1Co 12:23 и которые нам кажутся менее благородными в теле, о тех более прилагаем попечения;
\vs 1Co 12:24 и неблагообразные наши более благовидно покрываются, а благообразные наши не имеют \bibemph{в том} нужды. Но Бог соразмерил тело, внушив о менее совершенном большее попечение,
\vs 1Co 12:25 дабы не было разделения в теле, а все члены одинаково заботились друг о друге.
\vs 1Co 12:26 Посему, страдает ли один член, страдают с ним все члены; славится ли один член, с ним радуются все члены.
\vs 1Co 12:27 И вы~--- тело Христово, а порознь~--- члены.
\vs 1Co 12:28 И иных Бог поставил в Церкви, во-первых, Апостолами, во-вторых, пророками, в-третьих, учителями; далее, \bibemph{иным дал} силы \bibemph{чудодейственные}, также дары исцелений, вспоможения, управления, разные языки.
\vs 1Co 12:29 Все ли Апостолы? Все ли пророки? Все ли учители? Все ли чудотворцы?
\vs 1Co 12:30 Все ли имеют дары исцелений? Все ли говорят языками? Все ли истолкователи?
\vs 1Co 12:31 Ревнуйте о дарах б\acc{о}льших, и я покажу вам путь еще превосходнейший.
\vs 1Co 13:1 Если я говорю языками человеческими и ангельскими, а любви не имею, то я~--- медь звенящая или кимвал звучащий.
\vs 1Co 13:2 Если имею \bibemph{дар} пророчества, и знаю все тайны, и имею всякое познание и всю веру, так что \bibemph{могу} и горы переставлять, а не имею любви,~--- то я ничто.
\vs 1Co 13:3 И если я раздам все имение мое и отдам тело мое на сожжение, а любви не имею, нет мне в том никакой пользы.
\vs 1Co 13:4 Любовь долготерпит, милосердствует, любовь не завидует, любовь не превозносится, не гордится,
\vs 1Co 13:5 не бесчинствует, не ищет своего, не раздражается, не мыслит зла,
\vs 1Co 13:6 не радуется неправде, а сорадуется истине;
\vs 1Co 13:7 все покрывает, всему верит, всего надеется, все переносит.
\vs 1Co 13:8 Любовь никогда не перестает, хотя и пророчества прекратятся, и языки умолкнут, и знание упразднится.
\vs 1Co 13:9 Ибо мы отчасти знаем, и отчасти пророчествуем;
\vs 1Co 13:10 когда же настанет совершенное, тогда то, что отчасти, прекратится.
\vs 1Co 13:11 Когда я был младенцем, то по-младенчески говорил, по-младенчески мыслил, по-младенчески рассуждал; а как стал мужем, то оставил младенческое.
\vs 1Co 13:12 Теперь мы видим как бы сквозь \bibemph{тусклое} стекло, гадательно, тогда же лицем к лицу; теперь знаю я отчасти, а тогда позн\acc{а}ю, подобно как я познан.
\vs 1Co 13:13 А теперь пребывают сии три: вера, надежда, любовь; но любовь из них больше.
\vs 1Co 14:1 Достигайте любви; ревнуйте о \bibemph{дарах} духовных, особенно же о том, чтобы пророчествовать.
\vs 1Co 14:2 Ибо кто говорит на \bibemph{незнакомом} языке, тот говорит не людям, а Богу; потому что никто не понимает \bibemph{его}, он тайны говорит духом;
\vs 1Co 14:3 а кто пророчествует, тот говорит людям в назидание, увещание и утешение.
\vs 1Co 14:4 Кто говорит на \bibemph{незнакомом} языке, тот назидает себя; а кто пророчествует, тот назидает церковь.
\vs 1Co 14:5 Желаю, чтобы вы все говорили языками; но лучше, чтобы вы пророчествовали; ибо пророчествующий превосходнее того, кто говорит языками, разве он притом будет и изъяснять, чтобы церковь получила назидание.
\vs 1Co 14:6 Теперь, если я приду к вам, братия, и стану говорить на \bibemph{незнакомых} языках, то какую принесу вам пользу, когда не изъяснюсь вам или откровением, или познанием, или пророчеством, или учением?
\vs 1Co 14:7 И бездушные \bibemph{вещи}, издающие звук, свирель или гусли, если не производят раздельных тонов, как распознать т\acc{о}, чт\acc{о} играют на свирели или на гуслях?
\vs 1Co 14:8 И если труба будет издавать неопределенный звук, кто станет готовиться к сражению?
\vs 1Co 14:9 Так если и вы языком произносите невразумительные слова, то как узн\acc{а}ют, чт\acc{о} вы говорите? Вы будете говорить на ветер.
\vs 1Co 14:10 Сколько, например, различных слов в мире, и ни одного из них нет без значения.
\vs 1Co 14:11 Но если я не разумею значения слов, то я для говорящего чужестранец, и говорящий для меня чужестранец.
\vs 1Co 14:12 Так и вы, ревнуя о \bibemph{дарах} духовных, старайтесь обогатиться \bibemph{ими} к назиданию церкви.
\vs 1Co 14:13 А потому, говорящий на \bibemph{незнакомом} языке, молись о даре истолкования.
\vs 1Co 14:14 Ибо когда я молюсь на \bibemph{незнакомом} языке, то хотя дух мой и молится, но ум мой остается без плода.
\vs 1Co 14:15 Что же делать? Стану молиться духом, стану молиться и умом; буду петь духом, буду петь и умом.
\vs 1Co 14:16 Ибо если ты будешь благословлять духом, то стоящий на месте простолюдина к\acc{а}к скажет: <<аминь>> при твоем благодарении? Ибо он не понимает, чт\acc{о} ты говоришь.
\vs 1Co 14:17 Ты хорошо благодаришь, но другой не назидается.
\vs 1Co 14:18 Благодарю Бога моего: я более всех вас говорю языками;
\vs 1Co 14:19 но в церкви хочу лучше пять слов сказать умом моим, чтобы и других наставить, нежели тьму слов на \bibemph{незнакомом} языке.
\rsbpar\vs 1Co 14:20 Братия! не будьте дети умом: на злое будьте младенцы, а по уму будьте совершеннолетни.
\vs 1Co 14:21 В законе написано: иными языками и иными устами буду говорить народу сему; но и тогда не послушают Меня, говорит Господь.
\vs 1Co 14:22 Итак языки суть знамение не для верующих, а для неверующих; пророчество же не для неверующих, а для верующих.
\vs 1Co 14:23 Если вся церковь сойдется вместе, и все станут говорить \bibemph{незнакомыми} языками, и войдут к вам незнающие или неверующие, то не скажут ли, что вы беснуетесь?
\vs 1Co 14:24 Но когда все пророчествуют, и войдет кто неверующий или незнающий, то он всеми обличается, всеми судится.
\vs 1Co 14:25 И таким образом тайны сердца его обнаруживаются, и он падет ниц, поклонится Богу и скажет: истинно с вами Бог.
\rsbpar\vs 1Co 14:26 Итак чт\acc{о} же, братия? Когда вы сходитесь, и у каждого из вас есть псалом, есть поучение, есть язык, есть откровение, есть истолкование,~--- все сие да будет к назиданию.
\vs 1Co 14:27 Если кто говорит на \bibemph{незнакомом} языке, \bibemph{говорите} двое, или много трое, и т\acc{о} порознь, а один изъясняй.
\vs 1Co 14:28 Если же не будет истолкователя, то молчи в церкви, а говори себе и Богу.
\vs 1Co 14:29 И пророки пусть говорят двое или трое, а прочие пусть рассуждают.
\vs 1Co 14:30 Если же другому из сидящих будет откровение, то первый молчи.
\vs 1Co 14:31 Ибо все один за другим можете пророчествовать, чтобы всем поучаться и всем получать утешение.
\vs 1Co 14:32 И духи пророческие послушны пророкам,
\vs 1Co 14:33 потому что Бог не есть \bibemph{Бог} неустройства, но мира. Т\acc{а}к \bibemph{бывает} во всех церквах у святых.
\vs 1Co 14:34 Жены ваши в церквах да молчат, ибо не позволено им говорить, а быть в подчинении, как и закон говорит.
\vs 1Co 14:35 Если же они хотят чему научиться, пусть спрашивают \bibemph{о том} дома у мужей своих; ибо неприлично жене говорить в церкви.
\vs 1Co 14:36 Разве от вас вышло слово Божие? Или до вас одних достигло?
\rsbpar\vs 1Co 14:37 Если кто почитает себя пророком или духовным, тот да разумеет, чт\acc{о} я пишу вам, ибо это заповеди Господни.
\vs 1Co 14:38 А кто не разумеет, пусть не разумеет.
\vs 1Co 14:39 Итак, братия, ревнуйте о том, чтобы пророчествовать, но не запрещайте говорить и языками;
\vs 1Co 14:40 только всё должно быть благопристойно и чинно.
\vs 1Co 15:1 Напоминаю вам, братия, Евангелие, которое я благовествовал вам, которое вы и приняли, в котором и утвердились,
\vs 1Co 15:2 которым и спасаетесь, если преподанное удерживаете так, как я благовествовал вам, если только не тщетно уверовали.
\vs 1Co 15:3 Ибо я первоначально преподал вам, что и \bibemph{сам} принял, \bibemph{то есть}, что Христос умер за грехи наши, по Писанию,
\vs 1Co 15:4 и что Он погребен был, и что воскрес в третий день, по Писанию,
\vs 1Co 15:5 и что явился Кифе, потом двенадцати;
\vs 1Co 15:6 потом явился более нежели пятистам братий в одно время, из которых б\acc{о}льшая часть доныне в живых, а некоторые и почили;
\vs 1Co 15:7 потом явился Иакову, также всем Апостолам;
\vs 1Co 15:8 а после всех явился и мне, как некоему извергу.
\vs 1Co 15:9 Ибо я наименьший из Апостолов, и недостоин называться Апостолом, потому что гнал церковь Божию.
\vs 1Co 15:10 Но благодатию Божиею есмь то, что есмь; и благодать Его во мне не была тщетна, но я более всех их потрудился: не я, впрочем, а благодать Божия, которая со мною.
\vs 1Co 15:11 Итак я ли, они ли, мы так проповедуем, и вы так уверовали.
\rsbpar\vs 1Co 15:12 Если же о Христе проповедуется, что Он воскрес из мертвых, то к\acc{а}к некоторые из вас говорят, что нет воскресения мертвых?
\vs 1Co 15:13 Если нет воскресения мертвых, то и Христос не воскрес;
\vs 1Co 15:14 а если Христос не воскрес, то и проповедь наша тщетна, тщетна и вера ваша.
\vs 1Co 15:15 Притом мы оказались бы и лжесвидетелями о Боге, потому что свидетельствовали бы о Боге, что Он воскресил Христа, Которого Он не воскрешал, если, \bibemph{то есть}, мертвые не воскресают;
\vs 1Co 15:16 ибо если мертвые не воскресают, то и Христос не воскрес.
\vs 1Co 15:17 А если Христос не воскрес, то вера ваша тщетна: вы еще во грехах ваших.
\vs 1Co 15:18 Поэтому и умершие во Христе погибли.
\vs 1Co 15:19 И если мы в этой только жизни надеемся на Христа, то мы несчастнее всех человеков.
\vs 1Co 15:20 Но Христос воскрес из мертвых, первенец из умерших.
\vs 1Co 15:21 Ибо, как смерть через человека, \bibemph{так} через человека и воскресение мертвых.
\vs 1Co 15:22 Как в Адаме все умирают, так во Христе все оживут,
\vs 1Co 15:23 каждый в своем порядке: первенец Христос, потом Христовы, в пришествие Его.
\vs 1Co 15:24 А затем конец, когда Он предаст Царство Богу и Отцу, когда упразднит всякое начальство и всякую власть и силу.
\vs 1Co 15:25 Ибо Ему надлежит царствовать, доколе низложит всех врагов под ноги Свои.
\vs 1Co 15:26 Последний же враг истребится~--- смерть,
\vs 1Co 15:27 потому что все покорил под ноги Его. Когда же сказано, что \bibemph{Ему} все покорено, то ясно, что кроме Того, Который покорил Ему все.
\vs 1Co 15:28 Когда же все покорит Ему, тогда и Сам Сын покорится Покорившему все Ему, да будет Бог все во всем.
\vs 1Co 15:29 Иначе, что делают крестящиеся для мертвых? Если мертвые совсем не воскресают, то для чего и крестятся для мертвых?
\vs 1Co 15:30 Для чего и мы ежечасно подвергаемся бедствиям?
\vs 1Co 15:31 Я каждый день умираю: свидетельствуюсь в том похвалою вашею, братия, которую я имею во Христе Иисусе, Господе нашем.
\vs 1Co 15:32 По \bibemph{рассуждению} человеческому, когда я боролся со зверями в Ефесе, какая мне польза, если мертвые не воскресают? Станем есть и пить, ибо завтра умрем!
\vs 1Co 15:33 Не обманывайтесь: худые сообщества развращают добрые нравы.
\vs 1Co 15:34 Отрезвитесь, как должно, и не грешите; ибо, к стыду вашему скажу, некоторые из вас не знают Бога.
\rsbpar\vs 1Co 15:35 Но скажет кто-нибудь: как воскреснут мертвые? и в каком теле придут?
\vs 1Co 15:36 Безрассудный! то, что ты сеешь, не оживет, если не умрет.
\vs 1Co 15:37 И когда ты сеешь, то сеешь не тело будущее, а голое зерно, какое случится, пшеничное или другое какое;
\vs 1Co 15:38 но Бог дает ему тело, как хочет, и каждому семени свое тело.
\vs 1Co 15:39 Не всякая плоть такая же плоть; но иная плоть у человеков, иная плоть у скотов, иная у рыб, иная у птиц.
\vs 1Co 15:40 Есть тела небесные и тела земные; но иная слава небесных, иная земных.
\vs 1Co 15:41 Иная слава солнца, иная слава луны, иная звезд; и звезда от звезды разнится в славе.
\vs 1Co 15:42 Так и при воскресении мертвых: сеется в тлении, восстает в нетлении;
\vs 1Co 15:43 сеется в уничижении, восстает в славе; сеется в немощи, восстает в силе;
\vs 1Co 15:44 сеется тело душевное, восстает тело духовное. Есть тело душевное, есть тело и духовное.
\vs 1Co 15:45 Так и написано: первый человек Адам стал душею живущею; а последний Адам есть дух животворящий.
\vs 1Co 15:46 Но не духовное прежде, а душевное, потом духовное.
\vs 1Co 15:47 Первый человек~--- из земли, перстный; второй человек~--- Господь с неба.
\vs 1Co 15:48 Каков перстный, таковы и перстные; и каков небесный, таковы и небесные.
\vs 1Co 15:49 И как мы носили образ перстного, будем носить и образ небесного.
\rsbpar\vs 1Co 15:50 Но то скажу \bibemph{вам}, братия, что плоть и кровь не могут наследовать Царствия Божия, и тление не наследует нетления.
\vs 1Co 15:51 Говорю вам тайну: не все мы умрем, но все изменимся
\vs 1Co 15:52 вдруг, во мгновение ока, при последней трубе; ибо вострубит, и мертвые воскреснут нетленными, а мы изменимся.
\vs 1Co 15:53 Ибо тленному сему надлежит облечься в нетление, и смертному сему облечься в бессмертие.
\vs 1Co 15:54 Когда же тленное сие облечется в нетление и смертное сие облечется в бессмертие, тогда сбудется слово написанное: поглощена смерть победою.
\vs 1Co 15:55 Смерть! где твое жало? ад! где твоя победа?
\vs 1Co 15:56 Жало же смерти~--- грех; а сила греха~--- закон.
\vs 1Co 15:57 Благодарение Богу, даровавшему нам победу Господом нашим Иисусом Христом!
\vs 1Co 15:58 Итак, братия мои возлюбленные, будьте тверды, непоколебимы, всегда преуспевайте в деле Господнем, зная, что труд ваш не тщетен пред Господом.
\vs 1Co 16:1 При сборе же для святых поступайте так, как я установил в церквах Галатийских.
\vs 1Co 16:2 В первый день недели каждый из вас пусть отлагает у себя и сберегает, сколько позволит ему состояние, чтобы не делать сборов, когда я приду.
\vs 1Co 16:3 Когда же приду, то, которых вы изберете, тех отправлю с письмами, для доставления вашего подаяния в Иерусалим.
\vs 1Co 16:4 А если прилично будет и мне отправиться, то они со мной пойдут.
\rsbpar\vs 1Co 16:5 Я приду к вам, когда пройду Македонию; ибо я иду через Македонию.
\vs 1Co 16:6 У вас же, может быть, поживу, или и перезимую, чтобы вы меня проводили, куда пойду.
\vs 1Co 16:7 Ибо я не хочу видеться с вами теперь мимоходом, а надеюсь пробыть у вас несколько времени, если Господь позволит.
\vs 1Co 16:8 В Ефесе же я пробуду до Пятидесятницы,
\vs 1Co 16:9 ибо для меня отверста великая и широкая дверь, и противников много.
\rsbpar\vs 1Co 16:10 Если же придет к вам Тимофей, смотр\acc{и}те, чтобы он был у вас безопасен; ибо он делает дело Господне, как и я.
\vs 1Co 16:11 Посему никто не пренебрегай его, но провод\acc{и}те его с миром, чтобы он пришел ко мне, ибо я жду его с братиями.
\vs 1Co 16:12 А что до брата Аполлоса, я очень просил его, чтобы он с братиями пошел к вам; но он никак не хотел идти ныне, а придет, когда ему будет удобно.
\rsbpar\vs 1Co 16:13 Бодрствуйте, стойте в вере, будьте мужественны, тверды.
\vs 1Co 16:14 Все у вас да будет с любовью.
\rsbpar\vs 1Co 16:15 Прошу вас, братия (вы знаете семейство Стефаново, что оно есть начаток Ахаии и что они посвятили себя на служение святым),
\vs 1Co 16:16 будьте и вы почтительны к таковым и ко всякому содействующему и трудящемуся.
\vs 1Co 16:17 Я рад прибытию Стефана, Фортуната и Ахаика: они восполнили для меня отсутствие ваше,
\vs 1Co 16:18 ибо они мой и ваш дух успокоили. Почитайте таковых.
\rsbpar\vs 1Co 16:19 Приветствуют вас церкви Асийские; приветствуют вас усердно в Господе Акила и Прискилла с домашнею их церковью.
\vs 1Co 16:20 Приветствуют вас все братия. Приветствуйте друг друга святым целованием.
\rsbpar\vs 1Co 16:21 Мое, Павлово, приветствие собственноручно.
\vs 1Co 16:22 Кто не любит Господа Иисуса Христа, анафема, мар\acc{а}н-аф\acc{а}\fns{Да будет отлучен до пришествия Господа.}.
\vs 1Co 16:23 Благодать Господа нашего Иисуса Христа с вами,
\vs 1Co 16:24 и любовь моя со всеми вами во Христе Иисусе. Аминь.

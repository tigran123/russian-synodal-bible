\bibbookdescr{Tbn}{
  inline={Завещание Вениамина,\\двенадцатого сына Иакова и Рахили},
  toc={Завещание Вениамина},
  bookmark={Завещание Вениамина},
  header={Завещание Вениамина},
  abbr={Внм}
}
\vs Tbn 1:1
Список слов Вениамина, которые сказал он сыновьям своим,
прожив 125 лет.
\vs Tbn 1:2
Поцеловав их, молвил он:
как Исаак родился у Авраама в старости его,
так же и я родился у Иакова.
\vs Tbn 1:3
А Рахиль, мать моя, родив меня, умерла, и я не имел молока.
Потому кормила меня Балла, служанка её.
\vs Tbn 1:4
Рахиль, родив Иосифа, 12 лет была неплодна,
и молила Господа, и постилась, и зачав, родила меня.
\vs Tbn 1:5
Ибо премного любил отец мой Рахиль
и желал видеть двоих сыновей от неё.
\vs Tbn 1:6
Оттого назван был я Вениамин, то есть сын дней.

\vs Tbn 2:1
Когда же пришёл я в Египет, узнал меня брат мой Иосиф,
и спросил он меня: что сказали братья мои отцу,
когда продали меня?
\vs Tbn 2:2
И сказал я ему:
вымазали они хитон твой кровью и отослали его отцу, говоря:
узнай, сына ли твоего этот хитон.
\vs Tbn 2:3
И сказал он мне: да, брат, ибо взяли меня Измаильтяне,
и один из них снял с меня хитон, дал мне какую-то одежду,
ударил бичом и велел бежать.
\vs Tbn 2:4
И пошёл он спрятать одежду мою,
и встретился ему лев и убил того Измаильтянина.
\vs Tbn 2:5
И те, кто был с ним, устрашились и продали меня другим людям.
\vs Tbn 2:6
И не солгали братья мои в словах своих.
Ибо Иосиф желал скрыть от меня дела братьев наших,
и позвав их к себе, сказал им:
\vs Tbn 2:7
не говорите отцу моему, что сделали вы мне,
но так скажите, как рассказал я Вениамину.
\vs Tbn 2:8
И да будут мысли ваши такими же,
и да не дойдут слова эти до сердца отца моего.

\vs Tbn 3:1
И ныне, дети мои, возлюбите вы Господа Бога небес и земли,
и храните заповеди его, уподобляясь доброму и благочестивому
мужу Иосифу.
\vs Tbn 3:2
И да будут помыслы ваши добрыми,
как вы знаете то обо мне.
Ибо имеющий правильные помыслы всё правильно видит.
\vs Tbn 3:3
Бойтесь Господа и любите ближнего;
и если духи Велиаровы во всякую злую печаль ввергнут вас,
да не обретут власти над вами, как не смогли того над Иосифом,
братом моим.
\vs Tbn 3:4
Сколь многие люди желали убить его, и Бог защитил его.
Ибо тот, кто боится Бога и любит ближнего,
не будет сражен духом Велиаровым,
но защитит его страх Божий.
\vs Tbn 3:5
И кознями людей или зверей не может он быть порабощён,
но поможет ему любовь, которую имеет он к ближнему.
И до смерти Иакова не хотел Иосиф говорить о том,
но Иаков, узнав от Господа, сказал ему.
Но и тогда отрицал Иосиф, и едва убедился клятвами Израиля.
\vs Tbn 3:6
И просил Иосиф отца нашего помолиться за братьев его,
дабы не зачёл им Господь грех тот злой,
что совершили против него.
\vs Tbn 3:7
И воскликнул Иаков: О достойное дитя,
победил ты сердце Иакова, отца своего;
и обняв его, целовал 2 часа, говоря:
\vs Tbn 3:8
Исполнится на тебе пророчество небесное об агнце Божием
и Спасителе мира, что безупречный предан будет за беззаконников,
а безгрешный умрёт за нечестивцев в крови Завета во спасение народов
и Израиля, и уничтожит Велиара и слуг его.

\vs Tbn 4:1
Зрите же, дети мои, каков исход доброго мужа.
В доброте уподобляйтесь милосердию его, дабы и вам носить венцы славы.
\vs Tbn 4:2
Ибо у доброго человека око не омрачится, он ведь жалеет всех, если и
грешники это.
\vs Tbn 4:3
Если и недоброго желают ему, всё же творящий добро побеждает зло,
обороняемый Богом.
Праведных же любит он как душу свою.
\vs Tbn 4:4
Если кто славен, не завидует ему; если кто богат, не ревнует;
если мужествен кто, хвалит его; мудрого любит он, бедного жалеет;
слабому сострадает, Бога славит.
\vs Tbn 4:5
Имеющего страх Божий защищает он, любящему Господа помогает;
отвергающего Всевышнего наставляет он и обращает,
а имеющего благодать доброго духа любит как душу свою.

\vs Tbn 5:1
Если и вы будете иметь добрые помыслы,
то даже злые люди примирятся с вами,
и распутные устыдятся вас и обратятся ко благу,
и любостяжатели не только отступят от страсти своей,
но и то, что нажили они алчностью, отдадут страждущим.
\vs Tbn 5:2
Если будете творить добро, то и нечистые духи побегут от вас,
и звери устрашатся вас.
\vs Tbn 5:3
Ибо где свет добрых дел, там и тьма бежит от него.
\vs Tbn 5:4
И тот, кто надменно хулит благочестивого мужа, раскается,
ибо жалеет благочестивый хулителя и молчит.
\vs Tbn 5:5
И если кто предаст праведника, будет молиться тот.
И пусть ненадолго унижен будет, вскоре ещё светлее засияет,
как было то с Иосифом, братом моим.

\vs Tbn 6:1
Помышление доброго мужа~--- не в соблазняющей руке духа Велиарова.
Ибо ангел мира ведёт душу его.
\vs Tbn 6:2
И не взирает он с вожделением на тленное
и не собирает золота из любви к наслаждениям.
\vs Tbn 6:3
Не радуется он наслаждениям, не обижает ближнего,
не наполняется роскошью, не соблазняется взорами очей.
Ибо Господь~--- удел его.
\vs Tbn 6:4
Доброе помышление не внимает ни славе, ни хуле человеческой,
и ни лжи, ни спора, ни хулы не ведает.
Ибо Господь обитает в нём, и освещает душу его,
и радуется он за всех во всякий час.
\vs Tbn 6:5
Благой помысел не имеет двух языков~--- благословения и проклятия,
чести и поругания, покоя и смятения, лицемерия и правды,
бедности и богатства, но обо всех у него чистое и незамутненное суждение.
\vs Tbn 6:6
Нет у такого человека ни зрения двойного, ни слуха, ибо во всём,
что делает и что говорит, знает, что видит Господь душу его.
\vs Tbn 6:7
И очищает он помыслы свои, дабы не осудили его Бог и люди.
У Велиара же всякое дело двойное, и нет в нём простоты.
\vs Tbn 7:1
Потому, дети мои, говорю вам:
убегайте зла Велиарова, ибо нож дает он повинующимся ему.
\vs Tbn 7:2
А нож этот 7 зол порождает, сначала же зачинает мысль от Велиара.
И первое зло~--- убийство,
второе~--- разрушение,
третье~--- угнетение,
четвёртое~--- изгнание,
пятое~--- нужда,
шестое~--- смятение,
седьмое~--- опустошение.
\vs Tbn 7:3
Оттого и Каин 7-ми возмездиям подвергся от Господа,
ибо каждые 100 лет по одному удару наносил ему Господь.
\vs Tbn 7:4
Когда было Каину 200 лет, начал он получать их,
а в 900 был повержен за Авеля, праведного брата его.
7 зол было суждено Каину, а Ламеху~--- 70 раз 7.
\vs Tbn 7:5
Ибо до века будут караться таким судом подражающие Каину
в зависти и ненависти к братьям.
 
\vs Tbn 8:1
Вы же, дети мои, убегайте злобы, зависти и ненависти к братьям,
а прилепитесь к доброте и любви.
\vs Tbn 8:2
Ибо имеющий чистый помысел не взирает на женщину для блуда,
и незапятнано сердце его, ибо почиет на нем дух Божий.
\vs Tbn 8:3
Ибо как солнце не оскверняется, если и видит грязь и нечистоты,
но напротив, оно высушивает их и дурной запах изгоняет,
так же и чистый ум, в мерзостях земных пребывающий,
скорее очищает их, сам же не оскверняется.

\vs Tbn 9:1
Скажу вам, по словам Еноха праведного, и о недобрых делах ваших,
ибо блудить станете вы блудом Содомским, и не останется вас,
кроме немногих.
И вновь с женщинами предадитесь распутству,
и не будет в вас царства Божия,
ибо Господь тотчас заберёт его.
\vs Tbn 9:2
Только в одном уделе вашем возникнет храм Божий,
и будет последний славнее первого, и соберутся туда 12 колен
и все народы до той поры, когда пошлёт Всевышний спасение своё
посещением единородного Пророка.
\vs Tbn 9:3
И войдёт он в 1-ый храм, и там будет поруган Господь и поднят на древо.
\vs Tbn 9:4
И раздерётся завеса в храме, и перейдёт дух Божий к народам,
словно огонь прольётся.
\vs Tbn 9:5
И поднявшись из ада, взойдет он с земли на небо.
Познал я, сколь смирен будет он на земле и сколь прославлен на небе.

\vs Tbn 10:1
Когда же был Иосиф в Египте, желал я видеть лицо его и обличье его,
и по молитвам Иакова, отца моего, узрел я его, бодрствуя днём,
таким, каким был весь вид его.
\vs Tbn 10:2
И сказал им затем: Знайте, дети мои, что я умираю.
\vs Tbn 10:3
Творите же правду каждый ближнему своему, и закон Господа,
и заповеди его храните.
\vs Tbn 10:4
Ибо оставляю вам это вместо всякого наследства,
а вы передайте детям вашим на владение вечное,
ибо делали так Авраам, Исаак и Иаков.
\vs Tbn 10:5
И всё это оставили они нам в наследство, сказав:
Храните заповеди Бога до той поры, когда откроет Господь
спасение своё всем народам.
\vs Tbn 10:6
И тогда узрите вы Еноха, и Ноя, и Сима, и Авраама,
и Исаака, и Иакова восставшими одесную его в радости.
\vs Tbn 10:7
Тогда и мы воскреснем, каждый в уделе власти своей
и преклонимся пред Царём Небесным на землю явившимся
в обличье человеческом смиренно, и те, кто уверует
в него на земле, возрадуются с ним.
\vs Tbn 10:8
И все воскреснут: одни~--- для славы, другие~--- для бесславия,
и будет судить Господь первых Израиля за неправедность
их ибо не уверовали они в Бога, явившегося во плоти.
\vs Tbn 10:9
После же будет судить он все народы ибо не уверовали в него,
явившегося на землю.
\vs Tbn 10:10
И обличит он Израиля через народы избранные,
как обличил он Исава через Мадианитян,
возлюбивших братьев их.
Будьте же, дети мои, в уделе боящихся Господа.
\vs Tbn 10:11
Если пребудете вы в святости, дети мои,
и по заповедям Господа, то в твёрдой надежде будете вновь жить со мною,
и соберётся пред Господом весь Израиль.

\vs Tbn 11:1
И не назовусь я более волком хищным за хищность вашу,
но работником Господним, пищу раздающим тем, кто творит добро.
\vs Tbn 11:2
И восстанет в последние времена возлюбленный Господа от семени Иуды и Левия,
творящий благоволение уст его знанием новым освещая все народы.
Свет знания, придёт он к Израилю во спасение его,
и похитит у них как волк и отдаст собранию народов.
\vs Tbn 11:3
До скончания века пребудет он в собраниях народов и во властителях их,
словно песня сладкозвучная на устах всех.
\vs Tbn 11:4
И записан будет он в книги святые, и дело, и слово его,
и будет он избранником Божиим до века.
\vs Tbn 11:5
И будет ходить он среди них, подобно Иакову, отцу моему, говоря:
Сам восполнит он недостаток племени твоего.

\vs Tbn 12:1
И закончив речи свои, сказал он: Завещаю вам, дети мои,
вынесите кости мои из Египта и похороните меня в Хевроне рядом с отцами моими.
\vs Tbn 12:2
И умер Вениамин, будучи 125-и лет, в старости прекрасной,
и положили его во гроб.
\vs Tbn 12:3
И в 91-ый год прихода сынов Израиля в Египет,
отнесли кости отца своего тайно,
во время войны Ханаанской, в Хеврон и погребли там у ног отцов его.
\vs Tbn 12:4
А сами возвратились они из земли Ханаанской
и пребывали в Египте вплоть до дней исхода их из земли Египетской.

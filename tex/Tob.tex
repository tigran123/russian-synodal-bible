\bibbookdescr{Tob}{
  inline={\LARGE Книга\\\Huge Товита\fns{Переведена с греческого.}},
  toc={Товит*},
  bookmark={Товит},
  header={Товит},
  %headerleft={},
  %headerright={},
  abbr={Тов}
}
\vs Tob 1:1 Книга сказаний Товита, сына Товиилова, Ананиилова, Адуилова, Гаваилова, из племени Асиилова, из колена Неффалимова,
\vs Tob 1:2 который во дни Ассирийского царя Енемессара взят был в плен из Фисвы, находящейся по правую \bibemph{сторону} Кидия Неффалимова, в Галилее, выше Асира. Я, Товит, во все дни жизни моей ходил путями истины и правды
\vs Tob 1:3 и делал много благодеяний братьям моим и народу моему, пришедшим вместе со мною в страну Ассирийскую, в Ниневию.
\vs Tob 1:4 Когда я жил в стране моей, в земле Израиля, будучи еще юношею, тогда все колено Неффалима, отца моего, находилось в отпадении от дома Иерусалима, избранного от всех колен Израиля, чтобы всем им приносить \bibemph{там} жертвы, где освящен храм селения Всевышнего и утвержден во все роды навек.
\vs Tob 1:5 Как все отложившиеся колена приносили жертвы Ваалу, юнице, так и дом Неффалима, отца моего.
\vs Tob 1:6 Я же один часто ходил в Иерусалим на праздники, как предписано всему Израилю установлением вечным, с начатками и десятинами произведений \bibemph{земли} и начатками шерсти овец,
\vs Tob 1:7 и отдавал это священникам, сынам Аароновым, для жертвенника: десятину всех произведений давал сынам Левииным, служащим в Иерусалиме; другую десятину продавал, и каждый год ходил и издерживал ее в Иерусалиме;
\vs Tob 1:8 а третью давал, кому следовало, как заповедала мне Деввора, мать отца моего, когда я после отца моего остался сиротою.
\vs Tob 1:9 Достигнув мужеского возраста, я взял жену Анну из отеческого нашего рода и родил от нее Товию.
\vs Tob 1:10 Когда я отведен был в плен в Ниневию, все братья мои и одноплеменники мои ели от снедей языческих,
\vs Tob 1:11 а я соблюдал душу мою и не ел,
\vs Tob 1:12 ибо я помнил Бога всею душею моею.
\vs Tob 1:13 И даровал мне Всевышний милость и благоволение у Енемессара, и я был у него поставщиком;
\vs Tob 1:14 и ходил в Мидию, и отдал \bibemph{на сохранение} Гаваилу, брату Гаврия, в Рагах Мидийских, десять талантов серебра.
\vs Tob 1:15 Когда же умер Енемессар, вместо него воцарился сын его Сеннахирим, которого пути не были постоянны, и я уже не мог ходить в Мидию.
\vs Tob 1:16 Во дни Енемессара я делал много благодеяний братьям моим:
\vs Tob 1:17 алчущим давал хлеб мой, нагим одежды мои и, если кого из племени моего видел умершим и выброшенным за стену Ниневии, погребал его.
\vs Tob 1:18 Тайно погребал я и тех, которых убивал царь Сеннахирим, когда, обращенный в бегство, возвратился из Иудеи. А он многих умертвил в ярости своей. И отыскивал царь трупы, но их не находили.
\vs Tob 1:19 Один из Ниневитян пошел и донес царю, что я погребаю их; тогда я скрылся. Узнав же, что меня ищут убить, от страха убежал \bibemph{из города}.
\vs Tob 1:20 И было расхищено все имущество мое, и не осталось у меня ничего, кроме Анны, жены моей, и Товии, сына моего.
\vs Tob 1:21 Но не прошло пятидесяти дней, как два сына его убили его и убежали в горы Араратские. И воцарился вместо него сын его Сахердан, который поставил Ахиахара Анаила, сына брата моего, над всею счетною частью царства своего и над всем домоправлением.
\vs Tob 1:22 И ходатайствовал Ахиахар за меня, и я возвратился в Ниневию. Ахиахар же был и виночерпий и хранитель перстня, и домоправитель и казначей; и Сахердан поставил его вторым по себе; он был сын брата моего.
\vs Tob 2:1 Когда я возвратился в дом свой, и отданы мне были Анна, жена моя, и Товия, сын мой, в праздник пятидесятницы, в святую седмицу седмиц, приготовлен у меня был хороший обед, и я возлег есть.
\vs Tob 2:2 Увидев много снедей, я сказал сыну моему: пойди и приведи, кого найдешь, бедного из братьев наших, который помнит Господа, а я подожду тебя.
\vs Tob 2:3 И пришел он и сказал: отец \bibemph{мой}, один из племени нашего удавленный брошен на площади.
\vs Tob 2:4 Тогда я, прежде нежели стал есть, поспешно выйдя, убрал его в одно жилье до захождения солнца.
\vs Tob 2:5 Возвратившись, совершил омовение и ел хлеб мой в скорби.
\vs Tob 2:6 И вспомнил я пророчество Амоса, как он сказал: праздники ваши обратятся в скорбь, и все увеселения ваши~--- в плач.
\vs Tob 2:7 И я плакал. Когда же зашло солнце, я пошел и, выкопав \bibemph{могилу}, похоронил его.
\vs Tob 2:8 Соседи насмехались \bibemph{надо мною} и говорили: еще не боится он быть убитым за это дело; бегал уже, и вот опять погребает мертвых.
\vs Tob 2:9 В эту самую ночь, возвратившись после погребения и будучи нечистым, я лег спать за стеною двора, и лице мое не было покрыто.
\vs Tob 2:10 И не заметил я, что на стене были воробьи. Когда глаза мои были открыты, воробьи испустили теплое на глаза мои, и сделались на глазах моих бельма. И ходил я к врачам, но они не помогли мне. Ахиахар доставлял мне пропитание, доколе не отправился в Елимаиду.
\vs Tob 2:11 А потом жена моя Анна в женских отделениях пряла шерсть
\vs Tob 2:12 и посылала богатым людям, которые давали ей плату и однажды в придачу дали козленка.
\vs Tob 2:13 Когда принесли его ко мне, он начал блеять; и я спросил \bibemph{жену}: откуда этот козленок? не краденый ли? отдай его, кому он принадлежит! ибо непозволительно есть краденое.
\vs Tob 2:14 Она отвечала: это подарили мне сверх платы. Но я не верил ей и настаивал, чтобы отдала его, кому он принадлежит, и разгневался на нее. А она в ответ сказала мне: где же милостыни твои и праведные дела? вот как все они обнаружились на тебе!
\vs Tob 3:1 Опечалившись, я заплакал и молился со скорбью, говоря:
\vs Tob 3:2 праведен Ты, Господи, и все дела Твои и все пути Твои~--- милость и истина, и судом истинным и правым судишь Ты вовек!
\vs Tob 3:3 Воспомяни меня и призри на меня: не наказывай меня за грехи мои и заблуждения мои и отцов моих, которыми они согрешили пред Тобою!
\vs Tob 3:4 Ибо они не послушали заповедей Твоих, и Ты предал нас на расхищение и пленение и смерть, и в притчу поношения пред всеми народами, между которыми мы рассеяны.
\vs Tob 3:5 И, поистине, многи и праведны суды Твои~--- делать со мною по грехам моим и грехам отцов моих, потому что не исполняли заповедей Твоих и не поступали по правде пред Тобою.
\vs Tob 3:6 Итак, твори со мною, что Тебе благоугодно; повели взять дух мой, чтобы я разрешился и обратился в землю, ибо мне лучше умереть, нежели жить, так как я слышу лживые упреки, и глубока скорбь во мне! Повели освободить меня от этой тяготы в обитель вечную и не отврати лица Твоего от меня.
\vs Tob 3:7 В тот самый день случилось и Сарре, дочери Рагуиловой, в Екбатанах Мидийских терпеть укоризны от служанок отца своего
\vs Tob 3:8 за то, что она была отдаваема семи мужьям, но Асмодей, злой дух, умерщвлял их прежде, нежели они были с нею, как с женою. Они говорили ей: разве тебе не совестно, что ты задушила мужей твоих? Уже семерых ты имела, но не назвалась именем ни одного из них.
\vs Tob 3:9 Что нас бить за них? Они умерли: иди и ты за ними, чтобы нам не видеть твоего сына или дочери вовек!
\vs Tob 3:10 Услышав это, она весьма опечалилась, так что решилась было лишить себя жизни, но подумала: я одна у отца моего; если сделаю это, бесчестие ему будет, и я сведу старость его с печалью в преисподнюю.
\vs Tob 3:11 И стала она молиться у окна и говорила: благословен Ты, Господи Боже мой, и благословенно имя Твое святое и славное вовеки: да благословляют Тебя все творения Твои вовек!
\vs Tob 3:12 И ныне к Тебе, Господи, обращаю очи мои и лице мое;
\vs Tob 3:13 молю, возьми меня от земли сей и не дай мне слышать еще укоризны!
\vs Tob 3:14 Ты знаешь, Господи, что я чиста от всякого греха с мужем
\vs Tob 3:15 и не обесчестила имени моего, ни имени отца моего в земле плена моего; я единородная у отца моего, и нет у него сына, который мог бы наследовать ему, ни брата близкого, ни сына братнего, которому я могла бы сберечь себя в жену: уже семеро погибли у меня. Для чего же мне жить? А если не угодно Тебе умертвить меня, то благоволи призреть на меня и помиловать меня, чтобы мне не слышать более укоризны!
\vs Tob 3:16 И услышана была молитва обоих пред славою великого Бога, и послан был Рафаил исцелить обоих:
\vs Tob 3:17 снять бельма у Товита и Сарру, дочь Рагуилову, дать в жену Товии, сыну Товитову, связав Асмодея, злого духа; ибо Товии предназначено наследовать ее.~--- И в одно и то же время Товит, по возвращении, вошел в дом свой, а Сарра, дочь Рагуилова, сошла с горницы своей.
\vs Tob 4:1 В тот день вспомнил Товит о серебре, которое отдал на сохранение Гаваилу в Рагах Мидийских,
\vs Tob 4:2 и сказал сам себе: я просил смерти; что же не позову сына моего Товии, чтобы объявить ему об этом, пока я не умер?
\vs Tob 4:3 И, призвав его, сказал: сын \bibemph{мой}! когда я умру, похорони меня и не покидай матери своей; почитай ее во все дни жизни твоей, делай угодное ей и не причиняй ей огорчения.
\vs Tob 4:4 Помни, сын мой, что она много имела скорбей из-за тебя \bibemph{еще} во время чревоношения. Когда она умрет, похорони ее подле меня в одном гробе.
\vs Tob 4:5 Во все дни помни, сын \bibemph{мой}, Господа Бога нашего и не желай грешить и преступать заповеди Его. Во все дни жизни твоей делай правду и не ходи путями беззакония,
\vs Tob 4:6 ибо, если ты будешь поступать по истине, в делах твоих будет успех, как у всех поступающих по правде.
\vs Tob 4:7 Из имения твоего подавай милостыню, и да не жалеет глаз твой, когда будешь творить милостыню. Ни от какого нищего не отвращай лица твоего, тогда и от тебя не отвратится лице Божие.
\vs Tob 4:8 Когда у тебя будет много, твори из того милостыню, и когда у тебя будет мало, не бойся творить милостыню и понемногу;
\vs Tob 4:9 ты запасешь себе богатое сокровище на день нужды,
\vs Tob 4:10 ибо милостыня избавляет от смерти и не попускает сойти во тьму.
\vs Tob 4:11 Милостыня есть богатый дар для всех, кто творит ее пред Всевышним.
\vs Tob 4:12 Берегись, сын \bibemph{мой}, всякого \bibemph{вида} распутства. Возьми себе жену из племени отцов твоих, но не бери жены иноземной, которая не из колена отца твоего, ибо мы сыны пророков. Издревле отцы наши~--- Ной, Авраам, Исаак и Иаков. Помни, сын \bibemph{мой}, что все они брали жен из \bibemph{среды} братьев своих и были благословенны в детях своих, и потомство их наследует землю.
\vs Tob 4:13 Итак, сын \bibemph{мой}, люби братьев твоих и не превозносись сердцем пред братьями твоими и пред сынами и дочерями народа твоего, чтобы не от них взять тебе жену, потому что от гордости~--- погибель и великое неустройство, а от непотребства~--- оскудение и разорение: непотребство есть мать голода.
\vs Tob 4:14 Плата наемника, который будет работать у тебя, да не переночует у тебя, а отдавай ее тотчас: и тебе воздастся, если будешь служить Богу. Будь осторожен, сын \bibemph{мой}, во всех поступках твоих и будь благоразумен во всем поведении твоем.
\vs Tob 4:15 Что ненавистно тебе самому, того не делай никому. Вина до опьянения не пей, и пьянство да не ходит с тобою в пути твоем.
\vs Tob 4:16 Давай алчущему от хлеба твоего и нагим от одежд твоих; от всего, в чем у тебя избыток, твори милостыни, и да не жалеет глаз твой, когда будешь творить милостыню.
\vs Tob 4:17 Раздавай хлебы твои при гробе праведных, но не давай грешникам.
\vs Tob 4:18 У всякого благоразумного проси совета, и не пренебрегай советом полезным.
\vs Tob 4:19 Благословляй Господа Бога во всякое время и проси у Него, чтобы пути твои были правы и все дела и намерения твои благоуспешны, ибо ни один народ не властен в \bibemph{успехе} начинаний, но Сам Господь ниспосылает все благое и, кого хочет, уничижает по Своей воле. Помни же, сын \bibemph{мой}, заповеди мои, и да не изгладятся они из сердца твоего!
\vs Tob 4:20 Теперь я открою тебе, что я отдал десять талантов серебра на сохранение Гаваилу, сыну Гавриеву, в Рагах Мидийских.
\vs Tob 4:21 Не бойся, сын \bibemph{мой}, что мы обнищали: у тебя много, если ты будешь бояться Господа и, удаляясь от всякого греха, делать угодное пред Ним.
\vs Tob 5:1 И сказал Товия в ответ ему: отец \bibemph{мой}, я исполню все, что ты завещаешь мне;
\vs Tob 5:2 но как я могу получить серебро, не зная того \bibemph{человека}?
\vs Tob 5:3 Тогда \bibemph{отец} дал ему расписку и сказал: найди себе человека, который сопутствовал бы тебе; я дам ему плату, пока еще жив, и ступайте за серебром.
\rsbpar\vs Tob 5:4 И пошел он искать человека и встретил Рафаила. Это был Ангел, но он не знал
\vs Tob 5:5 и сказал ему: можешь ли ты идти со мною в Раги Мидийские и знаешь ли эти места?
\vs Tob 5:6 Ангел отвечал: могу идти с тобою и дорогу знаю; я уже останавливался у Гаваила, брата нашего.
\vs Tob 5:7 И сказал ему Товия: подожди меня, я скажу отцу моему.
\vs Tob 5:8 Тот сказал: ступай, только не медли.
\vs Tob 5:9 Он, придя, сказал отцу: вот я нашел себе спутника. \bibemph{Отец} сказал: пригласи его ко мне; я узнаю, из какого он колена и надежный ли спутник тебе.
\vs Tob 5:10 И позвал его, и он вошел, и приветствовали друг друга.
\vs Tob 5:11 Товит спросил: скажи мне, брат, из какого ты колена и из какого рода?
\vs Tob 5:12 Он отвечал: колена и рода ты ищешь или наемника, который пошел бы с сыном твоим? И сказал ему Товит: брат, мне хочется знать род твой и имя твое.
\vs Tob 5:13 Он сказал: я Азария, \bibemph{из рода} Анании великого, из братьев твоих.
\vs Tob 5:14 Тогда \bibemph{Товит} сказал ему: брат, иди благополучно, и не гневайся на меня за то, что я спросил о колене и роде твоем. Ты доводишься брат мне, из честного и доброго рода. Я знал Ананию и Ионафана, сыновей Семея великого; мы вместе ходили в Иерусалим на поклонение, с первородными и десятинами \bibemph{земных} произведений, ибо не увлекались заблуждением братьев наших: ты, брат, от хорошего корня!
\vs Tob 5:15 Но скажи мне: какую плату я должен буду дать тебе? Я дам тебе драхму на день и все необходимое для тебя и для сына моего,
\vs Tob 5:16 и еще прибавлю тебе сверх этой платы, если благополучно возвратитесь.
\vs Tob 5:17 Так и условились. Тогда он сказал Товии: будь готов в путь, и отправляйтесь благополучно. И приготовил сын его нужное для пути. И сказал ему отец: иди с этим человеком; живущий же на небесах Бог да благоустроит путь ваш, и Ангел Его да сопутствует вам!~--- И отправились оба, и собака юноши с ними.
\rsbpar\vs Tob 5:18 Анна, мать его, заплакала и сказала Товиту: зачем отпустил ты сына нашего? Не он ли был опорою рук наших, когда входил и выходил пред нами?
\vs Tob 5:19 Не предпочитай серебра серебру; пусть оно будет как сор \bibemph{в сравнении} с сыном нашим!
\vs Tob 5:20 Ибо, сколько Господом определено нам жить, на это у нас довольно есть.
\vs Tob 5:21 Товит сказал ей: не печалься, сестра; он придет здоровым, и глаза твои увидят его,
\vs Tob 5:22 ибо ему будет сопутствовать добрый Ангел; путь его будет благоуспешен, и он возвратится здоровым.
\vs Tob 6:1 И перестала она плакать.
\vs Tob 6:2 А путники вечером пришли к реке Тигру и остановились там на ночь.
\vs Tob 6:3 Юноша пошел помыться, но из реки показалась рыба и хотела поглотить юношу.
\vs Tob 6:4 Тогда Ангел сказал ему: возьми эту рыбу. И юноша схватил рыбу и вытащил на землю.
\vs Tob 6:5 И сказал ему Ангел: разрежь рыбу, возьми сердце, печень и желчь, и сбереги \bibemph{их}.
\vs Tob 6:6 Юноша так и сделал, как сказал ему Ангел; рыбу же испекли и съели; и пошли дальше и дошли до Екбатан.
\vs Tob 6:7 И сказал юноша Ангелу: брат Азария, к чему эта печень и сердце и желчь из рыбы?
\vs Tob 6:8 Он отвечал: если кого мучит демон или злой дух, то сердцем и печенью должно курить пред \bibemph{таким} мужчиною или женщиною, и более уже не будет мучиться;
\vs Tob 6:9 а желчью помазать человека, который имеет бельма на глазах, и он исцелится.
\vs Tob 6:10 Когда же приближались к Раге,
\vs Tob 6:11 Ангел сказал юноше: брат, ныне мы переночуем у Рагуила, твоего родственника, у которого есть дочь, по имени Сарра.
\vs Tob 6:12 Я поговорю о ней, чтобы дали ее тебе в жену, ибо тебе предназначено наследство ее, так как ты один из рода ее; а девица прекрасная и умная.
\vs Tob 6:13 Так послушайся меня; я поговорю с ее отцом и, когда мы возвратимся из Раг, совершим брак. Я знаю Рагуила: он никак не даст ее мужу чужому вопреки закону Моисееву; иначе повинен будет смерти, так как наследство следует получить тебе, а не другому кому.
\vs Tob 6:14 Тогда юноша сказал Ангелу: брат Азария, я слышал, что эту девицу отдавали семи мужам, но все они погибли в брачной комнате;
\vs Tob 6:15 а я один у отца и боюсь, как бы, войдя \bibemph{к ней}, не умереть подобно прежним; ее любит демон, который никому не вредит, кроме приближающихся к ней. И потому я боюсь, как бы мне не умереть и не свести жизнь отца моего и матери моей печалью обо мне во гроб их; а другого сына, который похоронил бы их, нет у них.
\vs Tob 6:16 Ангел сказал ему: разве ты забыл слова, которые заповедал тебе отец твой, чтобы ты взял жену из рода твоего? Послушай же меня, брат: ей следует быть твоею женою, а о демоне не беспокойся; в эту же ночь отдадут тебе ее в жену.
\vs Tob 6:17 Только, когда ты войдешь в брачную комнату, возьми курильницу, вложи в нее с\acc{е}рдца и печени рыбы и покури;
\vs Tob 6:18 и демон ощутит запах и удалится, и не возвратится никогда. Когда же тебе надобно будет приблизиться к ней, встаньте оба, воззовите к милосердому Богу, и Он спасет и помилует вас. Не бойся; ибо она предназначена тебе от века, и ты спасешь ее, и она пойдет с тобою, и я знаю, что у тебя будут от нее дети. Выслушав это, Товия полюбил ее, и душа его крепко прилепилась к ней. И пришли они в Екбатаны.
\vs Tob 7:1 И подошли к дому Рагуила. Сарра встретила и приветствовала их, и они ее, и ввела их в дом.
\vs Tob 7:2 И сказал Рагуил Едне, жене своей: как похож этот юноша на Товита, сына брата моего!
\vs Tob 7:3 И спросил их Рагуил: откуда вы, братья? Они отвечали ему: мы из сынов Неффалима, плененных в Ниневию.
\vs Tob 7:4 Еще спросил их: знаете ли брата нашего Товита? Они отвечали: знаем. Потом спросил: здравствует ли он? Они отвечали: жив и здоров.
\vs Tob 7:5 А Товия сказал: это мой отец.
\vs Tob 7:6 И бросился к нему Рагуил и целовал его и плакал.
\vs Tob 7:7 И благословил его и сказал: ты сын честного и доброго человека. Но, услышав, что Товит потерял зрение, опечалился и плакал;
\vs Tob 7:8 плакали и Една, жена его, и Сарра, дочь его. И приняли их весьма радушно,
\vs Tob 7:9 и закололи овна, и предложили обильные снеди. Товия же сказал Рафаилу: брат Азария, переговори, о чем ты говорил на пути; пусть устроится это дело!
\vs Tob 7:10 И он передал эту речь Рагуилу, а Рагуил сказал Товии: ешь, пей и веселись, ибо тебе надлежит взять мою дочь. Впрочем, скажу тебе правду:
\vs Tob 7:11 я отдавал свою дочь семи мужам, и, когда они входили к ней, в ту же ночь умирали. Но ты ныне будь весел! И сказал Товия: я ничего не буду здесь есть до тех пор, пока не сговоритесь и не условитесь со мною. Рагуил сказал: возьми ее теперь же по праву; ты брат ее, и она твоя. Милосердый Бог да устроит вас наилучшим образом!
\vs Tob 7:12 И призвал Сарру, дочь свою, и, взяв руку ее, отдал ее Товии в жену и сказал: вот, по закону Моисееву, возьми ее и веди к отцу твоему. И благословил их.
\vs Tob 7:13 И призвал Едну, жену свою, и, взяв свиток, написал договор и запечатал.
\vs Tob 7:14 И начали есть.
\vs Tob 7:15 И призвал Рагуил Едну, жену свою, и сказал ей: приготовь, сестра, другую спальню и введи ее.
\vs Tob 7:16 И сделала, как он сказал; и ввела ее туда, и заплакала, и приняла взаимно слезы дочери своей, и сказала ей:
\vs Tob 7:17 успокойся, дочь; Господь неба и земли даст тебе радость вместо печали твоей. Успокойся, дочь \bibemph{моя}!
\vs Tob 8:1 Когда окончили ужин, ввели к ней Товию.
\vs Tob 8:2 Он же, идя, вспомнил слова Рафаила, и взял курильницу, и положил сердце и печень рыбы, и курил.
\vs Tob 8:3 Демон, ощутив этот запах, убежал в верхние страны Египта, и связал его Ангел.
\rsbpar\vs Tob 8:4 Когда они остались в комнате вдвоем, Товия встал с постели и сказал: встань, сестра, и помолимся, чтобы Господь помиловал нас.
\vs Tob 8:5 И начал Товия говорить: благословен Ты, Боже отцов наших, и благословенно имя Твое святое и славное вовеки! Да благословляют Тебя небеса и все творения Твои!
\vs Tob 8:6 Ты сотворил Адама и дал ему помощницею Еву, подпорою~--- жену его. От них произошел род человеческий. Ты сказал: нехорошо быть человеку одному, сотворим помощника, подобного ему.
\vs Tob 8:7 И ныне, Господи, я беру сию сестру мою не для удовлетворения похоти, но поистине \bibemph{как жену}: благоволи же помиловать меня и \bibemph{дай} мне состариться с нею!
\vs Tob 8:8 И она сказала с ним: аминь.
\vs Tob 8:9 И оба спокойно спали в эту ночь. Между тем Рагуил, встав, пошел и выкопал могилу,
\vs Tob 8:10 говоря: не умер ли и этот?
\vs Tob 8:11 И пришел Рагуил в дом свой
\vs Tob 8:12 и сказал Едне, жене своей: пошли одну из служанок посмотреть, жив ли он; если нет, похороним его, и никто не будет знать.
\vs Tob 8:13 Служанка, отворив дверь, вошла и увидела, что оба они спят.
\vs Tob 8:14 И, выйдя, объявила им, что он жив.
\rsbpar\vs Tob 8:15 И благословил Рагуил Бога, говоря: благословен Ты, Боже, всяким благословением чистым и святым! Да благословляют Тебя святые Твои, и все создания Твои, и все Ангелы Твои, и все избранные Твои, да благословляют Тебя вовеки!
\vs Tob 8:16 Благословен Ты, что возвеселил меня, и не случилось со мною так, как я думал, но сотворил с нами по великой Твоей милости!
\vs Tob 8:17 Благословен Ты, что помиловал двух единородных! Доверши, Владыка, милость над ними: дай им окончить жизнь во здравии, с весельем и милостью!
\vs Tob 8:18 И приказал рабам своим зарыть могилу.
\vs Tob 8:19 И сделал для них брачный пир на четырнадцать дней.
\vs Tob 8:20 И сказал ему Рагуил с клятвою прежде исполнения дней брачного пира: не уходи, доколе не исполнятся эти четырнадцать дней брачного пира;
\vs Tob 8:21 а тогда, взяв половину имения, благополучно отправляйся к отцу твоему: остальное же \bibemph{получишь}, когда умру я и жена моя.
\vs Tob 9:1 И позвал Товия Рафаила и сказал ему:
\vs Tob 9:2 брат Азария, возьми с собою раба и двух верблюдов и сходи в Раги Мидийские к Гаваилу; принеси мне серебро и самого его приведи ко мне на брак;
\vs Tob 9:3 ибо Рагуил обязал меня клятвою, чтоб я не уходил;
\vs Tob 9:4 между тем отец мой считает дни и, если я много замедлю, он будет очень скорбеть.
\vs Tob 9:5 И пошел Рафаил и остановился у Гаваила и отдал ему расписку; а тот принес мешки за печатями и передал ему.
\vs Tob 9:6 И на утро рано встали они вместе и пришли на брак. И благословил Товия жену свою.
\vs Tob 10:1 Товит, отец его, считал каждый день. И когда исполнились дни путешествия, а он не приходил,
\vs Tob 10:2 Товит сказал: не задержали ли их? или не умер ли Гаваил, и некому отдать им серебра?
\vs Tob 10:3 И очень печалился.
\vs Tob 10:4 Жена же его сказала ему: погиб сын наш, потому и не приходит. И начала плакать по нем и говорила:
\vs Tob 10:5 ничто не занимает меня, сын мой, потому что я отпустила тебя, свет очей моих!
\vs Tob 10:6 Товит говорит ей: молчи, не тревожься, он здоров.
\vs Tob 10:7 А она сказала ему: молчи ты, не обманывай меня; погибло детище мое.~--- И ежедневно ходила за город на дорогу, по которой они отправились; днем не ела хлеба, а по ночам не переставала плакать о сыне своем Товии, пока не окончились четырнадцать дней брачного пира, которые Рагуил заклял его провести там. Тогда Товия сказал Рагуилу: отпусти меня, потому что отец мой и мать моя не надеются уже видеть меня.
\vs Tob 10:8 Тесть же сказал ему: побудь у меня; я пошлю к отцу твоему, и известят его о тебе.
\vs Tob 10:9 А Товия говорит: нет, отпусти меня к отцу моему.
\vs Tob 10:10 И встал Рагуил и отдал ему Сарру, жену его, и половину имения, рабов и скота и серебро,
\vs Tob 10:11 и, благословив их, отпустил и сказал: дети! да благопоспешит вам Бог Небесный, прежде нежели я умру.
\vs Tob 10:12 Потом сказал дочери своей: почитай твоего свекра и свекровь; теперь они~--- родители твои; желаю слышать добрый слух о тебе. И поцеловал ее. И Една сказала Товии: возлюбленный брат, да восставит тебя Господь Небесный и дарует мне видеть детей от Сарры, дочери моей, дабы я возрадовалась пред Господом. И вот, отдаю тебе дочь мою на сохранение; не огорчай ее.
\rsbpar\vs Tob 10:13 После того отправился Товия, благословляя Бога, что Он благоустроил путь его, и благословлял Рагуила и Едну, жену его. И продолжал путь, и приблизились они к Ниневии.
\vs Tob 11:1 И сказал Рафаил Товии: ты знаешь, брат, \bibemph{в} каком \bibemph{положении} ты оставил отца твоего;
\vs Tob 11:2 пойдем вперед, прежде жены твоей, и приготовим помещение;
\vs Tob 11:3 а ты возьми в руку и желчь рыбью. И пошли; за ними побежала и собака.
\rsbpar\vs Tob 11:4 Между тем Анна сидела, высматривая на дороге сына своего,
\vs Tob 11:5 и, заметив, что он идет, сказала отцу его: вот, идет сын твой и человек, отправившийся с ним.
\vs Tob 11:6 Рафаил сказал: я знаю, Товия, что у отца твоего откроются глаза;
\vs Tob 11:7 ты только помажь желчью глаза его, и он, ощутив едкость, оботрет \bibemph{их}, и спадут бельма, и он увидит тебя.
\vs Tob 11:8 Анна, подбежав, бросилась на шею к сыну своему и сказала ему: увидела я тебя, дитя \bibemph{мое},~--- теперь мне хотя умереть. И оба заплакали.
\vs Tob 11:9 А Товит пошел к дверям и споткнулся, но сын его поспешил к нему, и поддержал отца своего,
\vs Tob 11:10 и приложил желчь к глазам отца своего, и сказал: ободрись, отец \bibemph{мой}!
\vs Tob 11:11 Глаза его заело, и он отер их,
\vs Tob 11:12 и снялись с краев глаз его бельма. Увидев сына своего, он пал на шею к нему
\vs Tob 11:13 и заплакал и сказал: благословен Ты, Боже, и благословенно имя Твое вовеки, и благословенны все святые Ангелы Твои!
\vs Tob 11:14 Потому что Ты наказал и помиловал меня. Вот, я вижу Товию, сына моего.~--- И вошел сын его радостно и рассказал отцу своему о чудных \bibemph{делах}, бывших с ним в Мидии.
\vs Tob 11:15 И вышел Товит навстречу невестке своей к воротам Ниневии, радуясь и благословляя Бога. Видевшие, что он идет, удивлялись, как он прозрел.
\vs Tob 11:16 И Товит исповедал пред ними, что Бог помиловал его. Когда подошел Товит к Сарре, невестке своей, благословил ее и сказал: здравствуй, дочь \bibemph{моя}! Благословен Бог, Который привел тебя к нам, и \bibemph{благословенны} отец твой и мать твоя! Обрадовались и все братья его в Ниневии.
\vs Tob 11:17 И пришел Ахиахар и Насвас, племянник его,
\vs Tob 11:18 и весело праздновали брак Товии семь дней.
\vs Tob 12:1 И призвал Товит сына своего Товию и сказал ему: приготовь, сын \bibemph{мой}, плату человеку, который ходил с тобою; ему надобно еще прибавить.
\vs Tob 12:2 Он отвечал: отец \bibemph{мой}, я не буду в убытке, если отдам ему половину всего, что принес;
\vs Tob 12:3 потому что он привел меня к тебе здоровым и жену мою уврачевал, и серебро мое принес, и тебя также исцелил.
\vs Tob 12:4 Старец сказал: так и следует ему.
\vs Tob 12:5 И призвал Ангела и сказал ему: возьми половину всего, что вы принесли, и иди с миром.
\rsbpar\vs Tob 12:6 Тогда, отозвав обоих особо, \bibemph{Ангел} сказал им: благословляйте Бога, прославляйте Его, признавайте величие Его и исповедуйте пред всеми живущими, что Он сделал для вас. Доброе \bibemph{дело}~--- благословлять Бога, превозносить имя Его и благоговейно проповедовать о делах Божиих; и вы не ленитесь прославлять Его.
\vs Tob 12:7 Тайну цареву прилично хранить, а о делах Божиих объявлять похвально. Делайте добро, и зло не постигнет вас.
\vs Tob 12:8 Доброе \bibemph{дело}~--- молитва с постом и милостынею и справедливостью. Лучше малое со справедливостью, нежели многое с неправдою; лучше творить милостыню, нежели собирать золото,
\vs Tob 12:9 ибо милостыня от смерти избавляет и может очищать всякий грех. Творящие милостыни и \bibemph{дела} правды будут долгоденствовать.
\vs Tob 12:10 Грешники же суть враги своей жизни.
\vs Tob 12:11 Не скрою от вас ничего; я сказал уже: тайну цареву прилично хранить, а о делах Божиих объявлять похвально.
\vs Tob 12:12 Когда молился ты и невестка твоя Сарра, я возносил память молитвы вашей пред Святаго, и когда ты хоронил мертвых, я также был с тобою.
\vs Tob 12:13 И когда ты не обленился встать и оставить обед свой, чтобы пойти и убрать мертвого, твоя благотворительность не утаилась от меня, но я был с тобою.
\vs Tob 12:14 И ныне Бог послал меня уврачевать тебя и невестку твою Сарру.
\vs Tob 12:15 Я~--- Рафаил, один из семи святых Ангелов, которые возносят молитвы святых и восходят пред славу Святаго.
\rsbpar\vs Tob 12:16 Тогда оба смутились и пали лицем \bibemph{на землю}, потому что были в страхе.
\vs Tob 12:17 Но он сказал им: не бойтесь, мир будет вам. Благословляйте Бога вовек.
\vs Tob 12:18 Ибо я пришел не по своему произволению, а по воле Бога нашего; потому и благословляйте Его вовек.
\vs Tob 12:19 Все дни я был видим вами, но я не ел и не пил,~--- только взорам вашим представлялось \bibemph{это}.
\vs Tob 12:20 Итак, прославляйте теперь Бога, потому что я восхожу к Пославшему меня, и напишите все совершившееся в книгу.
\vs Tob 12:21 И встали они и более уже не видели его.
\vs Tob 12:22 И стали рассказывать о великих и чудных делах Божиих, и как явился им Ангел Господень.
\vs Tob 13:1 В радости Товит написал молитву \bibemph{в сих} словах: благословен Бог, вечно живущий, и \bibemph{благословенно} царство Его!
\vs Tob 13:2 Ибо Он наказует и милует, низводит до ада и возводит, и нет никого, кто избежал бы от руки Его.
\vs Tob 13:3 Сыны Израилевы! прославляйте Его пред язычниками, ибо Он рассеял нас между ними.
\vs Tob 13:4 Там возвещайте величие Его, превозносите Его пред всем живущим, ибо Он Господь наш и Бог, Отец наш во все веки:
\vs Tob 13:5 накажет нас за неправды наши, и опять помилует и соберет нас из всех народов, где бы вы ни были рассеяны между ними.
\vs Tob 13:6 Если вы будете обращаться к Нему всем сердцем вашим и всею душею вашею, чтобы поступать пред Ним по истине, тогда Он обратится к вам и не скроет от вас лица Своего. Увидите, чт\acc{о} Он сделает с вами. Прославляйте Его всеми \bibemph{глаголами} уст ваших и благословляйте Господа правды и превозносите Царя веков. В земле плена моего я прославляю Его и проповедую силу и величие Его народу грешников. Обратитесь, грешники, и делайте правду пред Ним. Кто знает, может быть, Он возблаговолит о вас и окажет вам милость?
\vs Tob 13:7 Превозношу я Бога моего, и душа моя~--- Небесного Царя, и радуется о величии Его.
\vs Tob 13:8 Пусть все возвещают о Нем и прославляют Его в Иерусалиме.
\vs Tob 13:9 Иерусалим, город святый! Он накажет \bibemph{тебя} за дела сынов твоих и опять помилует сынов праведных.
\vs Tob 13:10 Славь Господа усердно и благословляй Царя веков, чтобы снова сооружена была скиния Его в тебе с радостью, чтобы Он возвеселил среди тебя пленных и возлюбил в тебе несчастных во все роды века.
\vs Tob 13:11 Многие народы издалека придут к имени Господа Бога с дарами в руках, с дарами Царю Небесному; роды родов восхвалят тебя с восклицаниями радостными.
\vs Tob 13:12 Прокляты все ненавидящие тебя, благословенны будут вовек все любящие тебя!
\vs Tob 13:13 Радуйся и веселись о сынах праведных, ибо они соберутся и будут благословлять Господа праведных.
\vs Tob 13:14 О, блаженны любящие тебя! они возрадуются о мире твоем. Блаженны скорбевшие о всех бедствиях твоих, ибо они возрадуются о тебе, когда увидят всю славу твою, и будут веселиться вечно.
\vs Tob 13:15 Да благословляет душа моя Бога, Царя великого,
\vs Tob 13:16 ибо Иерусалим отстроен будет из сапфира и смарагда и из дорогих камней; стены твои, башни и укрепления~--- из чистого золота;
\vs Tob 13:17 и площади Иерусалимские выстланы будут бериллом, анфраксом и камнем из Офира.
\vs Tob 13:18 На всех улицах его будет раздаваться: аллилуия,~--- и будут славословить, говоря: благословен Бог, Который превознес \bibemph{Иерусалим}, на все веки!
\vs Tob 14:1 И окончил Товит славословие.
\rsbpar\vs Tob 14:2 Он был восьмидесяти восьми лет, когда потерял зрение, и чрез восемь лет прозрел. И творил милостыни, и продолжал быть благоговейным пред Господом Богом и прославлять Его.
\vs Tob 14:3 Наконец он очень состарился, и призвал сына своего и шесть сыновей его, и сказал ему: сын \bibemph{мой}, возьми сыновей твоих; вот я состарился и уже на исходе жизни моей.
\vs Tob 14:4 Отправься в Мидию, сын \bibemph{мой}, ибо я уверен, что Ниневия будет разорена, как говорил пророк Иона; а в Мидии будет спокойнее до времени. Братья наши, находящиеся в \bibemph{отечественной} земле, будут рассеяны из сей доброй земли; Иерусалим будет пустынею, и дом Божий в нем будет сожжен и до времени останется пуст.
\vs Tob 14:5 Но опять Бог помилует их и возвратит их в землю; и воздвигнут дом \bibemph{Божий}, не такой, как прежний, доколе не исполнятся времена века. И после того возвратятся из плена и построят Иерусалим великолепно, и дом Божий восстановлен будет в нем на все роды века,~--- здание величественное, как говорили о нем пророки.
\vs Tob 14:6 И все народы обратятся и будут истинно благоговеть пред Господом Богом и ниспровергнут идолов своих;
\vs Tob 14:7 и все народы будут благословлять Господа. И Его народ будет прославлять Бога, и Господь вознесет народ Свой; и все, истинно и праведно любящие Господа Бога, будут радоваться, оказывая милость братьям нашим.
\vs Tob 14:8 Итак, сын \bibemph{мой}, выйди из Ниневии, ибо непременно исполнится то, что говорил пророк Иона.
\vs Tob 14:9 Ты же соблюдай закон и повеления и будь любомилостив и справедлив, чтобы хорошо было тебе.
\vs Tob 14:10 Похорони меня прилично, и мать твою со мною, и потом не оставайтесь в Ниневии.~--- Сын \bibemph{мой}, смотри, чт\acc{о} сделал Аман с Ахиахаром, который воспитал его: как он из света привел его в тьму, и как воздано ему. Ахиахар спасен, а тот получил достойное возмездие~--- сошел во тьму. Манассия творил милостыню, и спасен от смертной сети, которую расставили ему; Аман же пал в сеть и погиб.
\vs Tob 14:11 Итак, дети, знайте, чт\acc{о} делает милостыня и как спасает справедливость.~--- Когда он это сказал, душа его оставила его на ложе; было же ему сто пятьдесят восемь лет, и сын с честью похоронил его.
\rsbpar\vs Tob 14:12 Когда умерла Анна, он похоронил и ее с отцом своим. После того Товия с женою своею и детьми своими отправился в Екбатаны к Рагуилу, тестю своему,
\vs Tob 14:13 и достиг честной старости, и похоронил прилично тестя и тещу своих, и получил в наследство имение их и Товита, отца своего.
\vs Tob 14:14 И умер ста двадцати семи лет в Екбатанах Мидийских.
\vs Tob 14:15 Но прежде нежели умер, он слышал о погибели Ниневии, которую пленил Навуходоносор и Асуир, и возрадовался пред смертью о Ниневии.

\bibbookdescr{Tju}{
  inline={Завещание Иуды,\\четвёртого сына Иакова и Лии},
  toc={Завещание Иуды},
  bookmark={Завещание Иуды},
  header={Завещание Иуды},
  abbr={Ида}
}
\vs Tju 1:1
Список слов Иуды, кои сказал он сыновьям своим, прежде чем умереть.
\vs Tju 1:2
Собравшись, пришли они к нему, и сказал он им:
\vs Tju 1:3
внемлите, дети мои, Иуде, отцу вашему.
Четвёртым сыном был я у отца моего Иакова, и Лия,
мать моя, нарекла меня Иудой, говоря: благодарю Господа за то,
что дал он мне и четвёртого сына.

\vs Tju 1:4
Смышлён был я в юности моей и слушался каждого слова отца моего.
\vs Tju 1:5
И чтил я мать мою и сестру матери моей.
\vs Tju 1:6
И когда настала пора зрелости моей,
благословил меня отец мой, говоря:
царем будешь ты, благим путем идущим во всем.

\vs Tju 2:1
И дал мне Господь милость во всех делах моих: в поле и в доме.
\vs Tju 2:2
Помню, что гнался я за оленем, и взял его,
и приготовил его в пищу отцу моему, и ел он.
\vs Tju 2:3
И серну одолел я в беге, и всё, что было на равнине, ловил я.
\vs Tju 2:4
Льва убил я и спас козленка из пасти его.
Медведицу поймал я за лапы, и бросил её в пропасть, и разбилась она.
\vs Tju 2:5
Дикую свинью нагнал я, и схватил на бегу, и растерзал её.
\vs Tju 2:6
Барс в Хевроне напал на пса моего, и я схватил барса за хвост,
и бросил его о скалу, и разбился он надвое.
\vs Tju 2:7
Быка дикого нашёл я, пасшегося в поле, и за рога поймал его,
и по кругу прогнав его, и помрачив зрение его, бросил и убил его.

\vs Tju 3:1
Когда же пришли два царя Ханаанских вооруженных к пастбищам нашим,
и народ многочисленный с ними, подбежал я один к царю одному,
и, ударив его по голеням, убил его.
\vs Tju 3:2
Другого же царя, Таппуаха, сидящего на коне
[убил я и тем весь народ его рассеял.
\vs Tju 3:3
И царя Ахора,] мужа огромного роста нашёл я,
стрелявшего из лука вперед и назад,
и поднял я камень в 60 фунтов, бросил его в коня и убил его.
\vs Tju 3:4
[И бился я с Ахором 2 часа, и убил его, и рассёк щит его на две части,
и отсёк ноги его.]
\vs Tju 3:5
Когда же снимал я панцирь его, вот, 8 мужей, бывших с ним,
сразились со мною.
\vs Tju 3:6
Намотал я одежду на руку мою, и метал в них камни, как из пращи,
и 4-х убил, а остальные бежали.

\vs Tju 3:7
Отец же мой Иаков убил Велисафа, царя всех царей,
мужа огромного роста, в 12 локтей.
\vs Tju 3:8
И трепет напал на них, и перестали воевать с нами.
\vs Tju 3:9
Оттого не знал беды отец мой в войнах, что с ним был я и братья мои.
\vs Tju 3:10
Ибо узрел он в видении обо мне, что ангел силы следует за мной повсюду,
да не буду побежден.

\vs Tju 4:1
После того произошла у нас война на юге, б\acc{о}льшая бывшей в Сикиме.
И встал я рядом с братьями моими, и преследовали мы 1000-у,
и убили из них 200.
\vs Tju 4:2
И взошёл я на стены и убил царя их.
\vs Tju 4:3
Так освободили мы Хеврон, и взяли всех врагов в плен.

\vs Tju 5:1
На другой день пошли мы в Арету, город могучий и сильный,
грозящий нам смертью.
\vs Tju 5:2
Я и Гад подошли к городу с востока, а Рувим и Левий~--- с запада.
\vs Tju 5:3
И помыслили те, что были на стенах, что мы одни, и пошли на нас.
\vs Tju 5:4
И тут тайно вошли братья наши с других сторон в город.
\vs Tju 5:5
И взяли мы его острием меча, а тех, кто бежал в башню,
огнём сожгли, и так захватили всех и всё имущество их.
\vs Tju 5:6
Когда же уходили мы, мужи из Таппуаха отняли у нас добычу нашу,
и, увидев то, вступили мы в битву с ними.
\vs Tju 5:7
И перебили всех, и обратно взяли добычу.

\vs Tju 6:1
И когда были мы у вод Хозевы, пошли на нас войной люди из Иовеля.
\vs Tju 6:2
И восстав на них, обратили мы их в бегство,
и союзников их из Силома убили,
и не дали им прохода, чтобы идти на нас.
\vs Tju 6:3
И вновь пошли на нас люди из Махира на 5-ый день,
и, восстав на них с мощным ножом,
победили мы их и убили также и их прежде,
нежели выступили они в поход.
\vs Tju 6:4
Когда же подошли мы к городу их,
покатили на нас камни женщины их с высоты горы, где был город.
\vs Tju 6:5
И, укрывшись, я и Симеон сзади взошли на гору и уничтожили и этот город.

\vs Tju 7:1
А на другой день сказали нам,
что царь города Гааш с народом многочисленным идёт на нас.
\vs Tju 7:2
Тогда я и Дан, сделав вид, что мы Амореяне,
как союзники вошли в город их.
\vs Tju 7:3
И глубокой ночью пришли братья наши, мы же открыли им ворота,
и всех жителей перебили и ограбили и 3 стены их разрушили.
\vs Tju 7:4
И подошли к Фамне, где было всё хранилище их.
\vs Tju 7:5
Тут разгневали меня насмешки их, и двинулся я к ним на вершину,
а они метали в меня камни и стреляли из лука.
\vs Tju 7:6
И если бы Дан, брат мой, не вступил в бой вместе со мною, убили бы они меня.
\vs Tju 7:7
И отважно наступили мы на них, и бежали они все, и,
отойдя иным путем к отцу нашему, они умолили его, и он заключил мир с ними.
\vs Tju 7:8
И не сделали мы им никакого зла,
а сделали их данниками нашими и отдали им добытое от них.
\vs Tju 7:9
И восстановили мы города их:
я пострил Фамну, а отец мой построил Рабаэл.
\vs Tju 7:10
Было же мне 20 лет, когда совершилась война эта.
\vs Tju 7:11
И страшились Хананеи меня и братьев моих.

\vs Tju 8:1
Было у меня много скота, и имел я начальником над пастухами
Хиру Одолламитянина.
\vs Tju 8:2
Придя к нему, увидел я Варсаву, царя Одоллама.
И говорил он с нами, и устроил нам пир.
И предложил он, и дал мне в жёны дочь свою, именем Вирсавию.
\vs Tju 8:3
Она родила мне Ира, Онана и Шелу.
И двоих погубил Господь, а Шела остался жить.

\vs Tju 9:1
18 лет был мой отец в мире с братом своим Исавом,
и дети Исава с нами, после того как пришли мы из Месопотамии, от Лавана.
\vs Tju 9:2
Когда же исполнились 18 лет, пришел к нам Исав, брат отца моего,
с народом сильно вооруженным и могучим.
\vs Tju 9:3
И поразил стрелою Иаков Исава,
и тот был унесен раненым на гору Сеир и умер.
\vs Tju 9:4
И мы преследовали сыновей Исава,
а был у них город с железными стенами и медными воротами,
и не могли мы войти в него.
Окружили мы и осадили его.
\vs Tju 9:5
И когда не отворяли они нам 20 дней,
приставил я лестницу на виду у всех и,
держа щит над головой моей и сдерживая удары камней,
взошёл наверх и убил четверых могучих мужей их.
\vs Tju 9:6
Рувим и Гад убили еще шестерых.
\vs Tju 9:7
Тогда просили они нас о мире, и, посоветовавшись с отцом нашим,
приняли мы их в данники.
\vs Tju 9:8
И давали они нам 50 гомеров пшеницы, и масла 50 батов,
и вина 50 мер вплоть до голода, когда пошли мы в Египет.

\vs Tju 10:1
После того взял в жены Ир, сын мой, Фамарь из Месопотамии,
бывшую дочерью Арама.
\vs Tju 10:2
Ир же был недобрым и смущался пред Фамарью,
ибо не была она из Ханаана, и умертвил его ангел Господень.
\vs Tju 10:3
И дал я её Онану, 2-му сыну моему, и его убил Господь.
\vs Tju 10:4
Ибо он не познавал её, хотя прожил с нею год, не желая иметь детей от неё.
\vs Tju 10:5
Когда же пригрозил я ему, сошёлся он с нею,
но излил семя на землю по совету матери своей.
И от греха этого умер и он.
\vs Tju 10:6
Я же хотел дать Фамари и Шелу, но мать его не дозволила.
Злые помыслы имела она,
ибо не была Фамарь из дочерей Хананеев, как она сама.

\vs Tju 11:1
Я же знал, что злой род Хананеи, но мысли юности ослепили разум мой.
\vs Tju 11:2
И, увидев, как она разливает вино, прельстился я
и взял её без воли на то отца моего.
\vs Tju 11:3
Она же в мое отсутствие пошла и взяла Шелу жену из Ханаана.
\vs Tju 11:4
А я, узнав, что сотворила она, проклял ее в скорби души моей.
\vs Tju 11:5
И умерла она от грехов своих вслед за детьми своими.

\vs Tju 12:1
Когда овдовела Фамарь и прошло 2 года, услышала она,
что иду я стричь овец и, нарядившись в наряд свадебный,
села в городе Енаиме у ворот.
\vs Tju 12:2
Был же закон у Амореев, чтобы вдова 7 дней сидела блудницей у ворот.
\vs Tju 12:3
И я, опьянённый вином, не узнал её,
и прельстила меня красота её из-за прекрасного наряда.
\vs Tju 12:4
И свернув с дороги к ней, сказал я: войду к тебе.
А она спросила: а что ты дашь мне?
И дал я ей посох мой, и пояс, и диадему царства моего в залог.
И когда вошёл к ней, зачала она.
\vs Tju 12:5
И не зная, что сам сотворил, хотел я убить Фамарь.
Она же, послав мне тайно всё данное ей, устыдила меня.
\vs Tju 12:6
И призвав её, услышал я те слова тайные, что говорил ей,
когда возлежал с нею в опьянении моём.
И не мог убить её, ибо то было дано Господом.
\vs Tju 12:7
И сказал я: не лукавила она, взяв у другой женщины этот знак.
\vs Tju 12:8
Но не сходился я с ней более до конца жизни моей,
ибо мерзость сотворил я во всём Израиле.
\vs Tju 12:9
А жители города того говорили, что не было блудницы у ворот,
ибо она пришла из другого места и недолго сидела там.
\vs Tju 12:10
И помыслил я, что не видел никто, как вошёл я к ней.

\vs Tju 12:11
После того пошли мы в Египет к Иосифу, так как был голод.
\vs Tju 12:12
Было мне 46 лет, и 73 года провел я в Египте.

\vs Tju 13:1
Ныне завещаю вам, дети мои, послушайте Иуду, отца вашего,
и сохраните слова мои, чтобы делать всё по велениям Господа
и подчиняться заповедям его.
\vs Tju 13:2
Не идите за вожделениями вашими в гордыне сердца своего,
и не похваляйтесь делами и силой молодости вашей,
ибо это злое дело пред лицом Господа.
\vs Tju 13:3
Когда я возгордился, что в войнах не прельстило меня лицо
женщины благообразной, и позорил брата моего Рувима из-за Баллы,
женщины отца моего, тогда стал приступать ко мне дух зависти и блуда,
пока не согрешил я с Вирсавией Хананеянкой и с Фамарью,
невесткой моей.
\vs Tju 13:4
Ибо сказал я тестю моему:
посоветуюсь с отцом моим и тогда возьму дочь твою.
Он же не захотел, но показал мне золота несметное количество,
что было за дочерью его, ибо он был царь.
\vs Tju 13:5
И нарядил он её в золото и жемчуги, велел ей разливать вино на пиру.
\vs Tju 13:6
И совратило вино очи мои, и помрачило мне сердце наслаждением.
\vs Tju 13:7
И возлюбив её, возлёг с нею, и пренебрег заповедью Господа
и заповедью отца моего, и взял её в жены.
\vs Tju 13:8
И воздал мне Господь за помысел души моей, ибо не был я счастлив в детях её.

\vs Tju 14:1
И ныне говорю, дети мои: не опьяняйтесь вином,
ибо вино отвращает разум от истины,
и производит страсть вожделения,
и вводит очи в соблазн.
\vs Tju 14:2
Ведь дух блуда словно слугою имеет вино, дабы услаждать ум,
так что совращают эти два помысла человека.
\vs Tju 14:3
Ибо, если некто пьёт вино до опьянения,
мыслями нечистыми возмущает он ум свой,
и для блуда разгорячает тело свое,
дабы насладиться, и грех совершает, и не стыдится.
\vs Tju 14:4
Таково вино, дети мои, ибо не стыдится опьяневший никого.
\vs Tju 14:5
Вот, и меня оно соблазнило,
так что не устыдился я множества жителей города,
ибо на глазах у всех возлёг с Фамарью,
и совершил грех великий, и раскрыл тайну своей нечистоты сыновьям моим.
\vs Tju 14:6
Пил я вино, и не устыдился заповеди Божией, и взял в жёны Хананеянку.
\vs Tju 14:7
Ибо великое умение нужно пьющему вино, дети мои; это умение винопития,
дабы пить до того времени, пока имеет человек стыд.
\vs Tju 14:8
Когда же перейдёт он предел, входит в ум его дух соблазна
и заставляет пьяного сквернословить,
и творить беззакония, и не стыдиться бесчестия своего,
но кичиться им и мнить себя прекрасным.

\vs Tju 15:1
Блудящий наказания не чувствует и бесчестия не стыдится.
\vs Tju 15:2
Если же царь блудит, лишается он царства,
порабощённый блудом, как и я то претерпел.
\vs Tju 15:3
Отдал я посох мой, который есть опора племени моего,
и пояс мой, который есть сила,
и диадему, которая есть слава царства моего.
\vs Tju 15:4
И раскаявшись в том, не пил я вина и не вкушал мяса до старости моей,
и никакого веселья не видел.
\vs Tju 15:5
И показал мне ангел Божий, что и царем, и нищим правят женщины.
Но не в них преуспеяние жизни.
\vs Tju 15:6
У царя отнимают они славу,
у мужественного~--- силу,
а у нищего~--- самую малую опору в его нищете.

\vs Tju 16:1
Остерегайтесь же, дети мои, преступить предел, положенный вину,
ибо в нём~--- 4 злые духа:
вожделения, жаркой страсти, распутства и алчности.
\vs Tju 16:2
Когда пьёте вино в радости, будьте умеренны, боясь Бога.
Ибо если в радости исчезнет страх Божий, наступит опьянение,
и придет бесстыдство.
\vs Tju 16:3
Если же хотите жить разумно, вовсе не прикасайтесь к вину,
дабы не согрешить в словах надменных, и побоищах, и клевете,
и нарушении заповедей Божиих, и не погибнете не в свой час.
\vs Tju 16:4
Также раскрывает вино тайны Божии и людские,
как и я раскрыл заповеди Божий и тайны Иакова, отца моего,
Хананеянке Вирсавии, чего не велел мне Бог раскрывать.

\vs Tju 17:1
И ныне завещаю вам, дети мои,
не любить серебра и не смотреть на красоту женщин,
ибо и я от серебра и золота, и от красоты соблазнился Вирсавией Хананеянкой.
\vs Tju 17:2
[И знаю, что из-за этих двух предан будет род мой на погибель блуда.
\vs Tju 17:3
Ибо и мудрых мужей из сынов моих собьют они с пути,
и умалят царство Иуды, данное мне Господом за послушание отцу моему.
\vs Tju 17:4
Ведь я никогда не огорчал отца моего Иакова, ибо делал всё,
что говорил он мне.
\vs Tju 17:5
И Авраам, отец деда моего, благословил меня царствовать в Израиле,
и так же благословил меня Иаков.
\vs Tju 17:6
И знаю я, что от меня восстановится царство.

\vs Tju 18:1
И познал я, и читал в книгах Еноха праведного,
какое зло сотворите вы в последние дни.]
\vs Tju 18:2
Остерегайтесь же, дети мои, блуда и сребролюбия, и послушайте Иуду,
отца вашего.
\vs Tju 18:3
Ибо они уводят от закона
Божия и помрачают помысел душевный,
и гордыне научают, и не дают мужу иметь сострадание к ближнему своему.
\vs Tju 18:4
Лишают они душу его всякой доброты
и утесняют его болями и страданием,
сон прогоняют от него и плоть его истребляют.
\vs Tju 18:5
Жертвам Богу он препятствует, о благословении Божием не помнит,
и когда пророк говорит, не слушает, и от слов благочестия отвращается.
\vs Tju 18:6
Ибо двум страстям, противным заповедям Божиим,
рабски служит он и Богу повиноваться не может.
Помрачили они душу его, и днем ходит он словно ночью.

\vs Tju 19:1
Дети мои, сребролюбие ведёт к идолопоклонству,
ибо в соблазне серебра называют богами тех,
кто не есть Бог, а тот, кто имеет серебро, в безумие впадает.
\vs Tju 19:2
От серебра погиб я, дети мои, и если бы не раскаяние моё,
и смирение, и мольбы отца моего, бездетным умер бы я.
\vs Tju 19:3
Но Бог отцов наших смиловался надо мною,
ибо по неведению сотворил я это.
\vs Tju 19:4
Ибо ослепил меня отец обмана и пребывал я в заблуждении
как человек и плоть, грехами сокрушенная, и познал я немощь мою,
когда думал, что непобедим я.

\vs Tju 20:1
Знайте же, дети мои, что 2 духа смотрят
за человеком~--- дух правды и дух лжи.
\vs Tju 20:2
Посредине же~--- дух познания, склоняющего ум туда, куда пожелает.
\vs Tju 20:3
А правдивое и лживое написаны на сердце человека, и всё это известно Господу.
\vs Tju 20:4
И нет часа, в который могли бы укрыться дела людские,
ибо на самом сердце написано пред лицом Господа.
\vs Tju 20:5
А дух правды обличает всё, и жжет грешника огнём в его же сердце,
и не может он поднять лица к Судье.

\vs Tju 21:1
Ныне, дети мои, возвещаю вам:
любите Левия, и пребывайте с ним, и не возноситесь над ним,
да не уничтожитесь вы.
\vs Tju 21:2
Ибо мне дал Бог царство, ему же~--- священство,
и подчинил царство священству.
\vs Tju 21:3
Мне дал он то, что на земле, ему~--- то, что на небесах.
\vs Tju 21:4
Как небеса выше земли,
так священство Божие выше стоит, нежели царство земное,
если согрешив, не отпадёт оно от Господа
и не станет править священством царство земное.
\vs Tju 21:5
Ибо сказал мне ангел Господень: избрал его Господь и поставил выше тебя,
чтобы приблизился ты к нему, и вкушал от трапезы его,
и первенцев от богатств сынов Израиля приносил ему.
Ты же будешь царем над Иаковом.
\vs Tju 21:6
И будешь ты подобен морю.
Ибо, как на море праведные и неправедные попадают в бурю,
и одни попадают в плен, другие же обогащаются,
так и в тебе со всяким родом людей так будет:
одни будут терпеть опасности и пленение,
другие же обогащаться, похищая чужое.
\vs Tju 21:7
Ибо цари китам уподобятся: пожирая людей, словно рыб,
станут они порабощать сыновей и дочерей свободных и грабить дома,
поля, пастбища и всякое добро.
\vs Tju 21:8
И неправедно тела многих отдадут в пищу воронам и цаплям,
и преуспеют во зле, и возвысятся в алчности.

\vs Tju 21:9
И будут лжепророки, словно вихри, и многих праведных будут преследовать.
\vs Tju 22:1
И наведёт на них Господь раздоры друг с другом,
и войны будут в Израиле непрерывные.
\vs Tju 22:2
И к иноплеменным перейдёт царство моё до прихода спасения к Израилю,
до явления Бога справедливого, когда почиет Иаков в мире и все народы.
\vs Tju 22:3
И сам Господь сохранит навеки силу царства моего,
ибо клялся он мне клятвою, что не угаснет царство семени моего до века.

\vs Tju 23:1
Великое горе для меня, дети мои, от нечестия и обмана,
которые сотворите вы в царстве моём,
когда последуете за чревовещателями, прорицателями и бесами соблазна.
\vs Tju 23:2
Дочерей ваших певицами и блудницами сделаете,
и смешаетесь с мерзостью языческой.
\vs Tju 23:3
За то наведёт на вас Господь голод и мор, смерть и меч,
осаду от врагов и позор от друзей, и воспаление очей,
и убийство детей, и похищение имущества, и сожжение Храма Божьего,
и порабощение вас самих язычниками.
\vs Tju 23:4
И оскопят сыновей ваших, чтобы стали они евнухами у жен их.
\vs Tju 23:5
И будет так дотоле, пока не посетит вас Господь,
когда раскаетесь вы и станете жить по всем заповедям его,
и выведет он вас из плена языческого.

\vs Tju 24:1
После того взойдет вам звезда из Иакова в знак мира,
и восстанет человек [от семени моего],
как солнце праведности, и будет жить с людьми в кротости и справедливости,
и не будет на нём никакого греха.
\vs Tju 24:2
И разверзнутся над ним небеса, дабы излить дух благословения
Отца Святого, и сам он изольёт дух милости на вас.
\vs Tju 24:3
И будете ему сыновьями истинными, и жить будете по заветам
его первым и последним.
\vs Tju 24:4
[Он есть отрасль Бога Всевышнего и источник, дающий всем жизнь.]
\vs Tju 24:5
Тогда воссияет скипетр царства моего, и из корня вашего выйдет ствол.
\vs Tju 24:6
А на нём взрастёт жезл праведности для народов,
дабы судить и спасти всех призывающих имя Господа.

\vs Tju 25:1
После того восстанут к жизни Авраам, Исаак и Иаков,
а я и братья мои станем вождями колен Израиля:
1-ый~--- Левий,
2-ой~--- я,
3-ий~--- Иосиф,
4-ый~--- Вениамин,
5-ый~--- Симеон,
6-ой~--- Иссахар,
и так все по порядку.
\vs Tju 25:2
И благословил Господь Левия;
ангел лика Господня~--- меня;
силы славы~--- Симеона;
небо~--- Рувима;
земля~--- Иссахара;
море~--- Завулона;
горы~--- Иосифа;
скиния~--- Вениамина;
светильники~--- Дана;
сад Едемский~--- Неффалима;
солнце~--- Гада;
луна~--- Асира.
\vs Tju 25:3
И будете вы один народ Господень и один язык,
и не будет там духа соблазна Велиарова,
ибо он будет ввержен в огонь навечно.
\vs Tju 25:4
И в скорби скончавшиеся восстанут в радости, а нищие Господом
обогащены будут, а умирающие Господом вдохновлены к жизни будут.
\vs Tju 25:5
И в веселии побегут олени Иакова,
и орлы Израиля полетят в радости, [а нечестивые
восскорбят, и грешники зарыдают], и все народы прославят Господа навеки.

\vs Tju 26:1
Храните же, дети мои, все законы Господни, ибо он есть надежда для всех,
соблюдающих пути его.

\vs Tju 26:2
И сказал Иуда: вот, 118-ти лет умираю я сегодня.
\vs Tju 26:3
Да не погребает меня никто в пышной одежде,
и да не разрезают мне чрево,
что угодно творить царствующим,
но отнесите меня в Хеврон, где и отцы мои.
\vs Tju 26:4
И сказав это, почил он, и сделали сыновья его во всём так, как
завещал он им, и погребли его с отцами его в Хевроне.

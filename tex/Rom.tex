\bibbookdescr{Rom}{
  inline={Послание к Римлянам\\\LARGE Святого Апостола Павла},
  toc={к Римлянам},
  bookmark={к Римлянам},
  header={к Римлянам},
  %headerleft={},
  %headerright={},
  abbr={Рим}
}
\vs Rom 1:1 Павел, раб Иисуса Христа, призванный Апостол, избранный к благовестию Божию,
\vs Rom 1:2 которое Бог прежде обещал через пророков Своих, в святых писаниях,
\vs Rom 1:3 о Сыне Своем, Который родился от семени Давидова по плоти
\vs Rom 1:4 и открылся Сыном Божиим в силе, по духу святыни, через воскресение из мертвых, о Иисусе Христе Господе нашем,
\vs Rom 1:5 через Которого мы получили благодать и апостольство, чтобы во имя Его покорять вере все народы,
\vs Rom 1:6 между которыми находитесь и вы, призванные Иисусом Христом,~---
\vs Rom 1:7 всем находящимся в Риме возлюбленным Божиим, призванным святым: благодать вам и мир от Бога Отца нашего и Господа Иисуса Христа.
\rsbpar\vs Rom 1:8 Прежде всего благодарю Бога моего через Иисуса Христа за всех вас, что вера ваша возвещается во всем мире.
\vs Rom 1:9 Свидетель мне Бог, Которому служу духом моим в благовествовании Сына Его, что непрестанно воспоминаю о вас,
\vs Rom 1:10 всегда прося в молитвах моих, чтобы воля Божия когда-нибудь благопоспешила мне прийти к вам,
\vs Rom 1:11 ибо я весьма желаю увидеть вас, чтобы преподать вам некое дарование духовное к утверждению вашему,
\vs Rom 1:12 то есть утешиться с вами верою общею, вашею и моею.
\vs Rom 1:13 Не хочу, братия, \bibemph{оставить} вас в неведении, что я многократно намеревался прийти к вам (но встречал препятствия даже доныне), чтобы иметь некий плод и у вас, как и у прочих народов.
\vs Rom 1:14 Я должен и Еллинам и варварам, мудрецам и невеждам.
\vs Rom 1:15 Итак, что до меня, я готов благовествовать и вам, находящимся в Риме.
\vs Rom 1:16 Ибо я не стыжусь благовествования Христова, потому что \bibemph{оно} есть сила Божия ко спасению всякому верующему, во-первых, Иудею, \bibemph{потом} и Еллину.
\vs Rom 1:17 В нем открывается правда Божия от веры в веру, как написано: праведный верою жив будет.
\rsbpar\vs Rom 1:18 Ибо открывается гнев Божий с неба на всякое нечестие и неправду человеков, подавляющих истину неправдою.
\vs Rom 1:19 Ибо, чт\acc{о} можно знать о Боге, явно для них, потому что Бог явил им.
\vs Rom 1:20 Ибо невидимое Его, вечная сила Его и Божество, от создания мира через рассматривание творений видимы, так что они безответны.
\vs Rom 1:21 Но как они, познав Бога, не прославили Его, как Бога, и не возблагодарили, но осуетились в умствованиях своих, и омрачилось несмысленное их сердце;
\vs Rom 1:22 называя себя мудрыми, обезумели,
\vs Rom 1:23 и славу нетленного Бога изменили в образ, подобный тленному человеку, и птицам, и четвероногим, и пресмыкающимся,~---
\vs Rom 1:24 то и предал их Бог в похотях сердец их нечистоте, так что они сквернили сами свои тела.
\vs Rom 1:25 Они заменили истину Божию ложью, и поклонялись, и служили твари вместо Творца, Который благословен во веки, аминь.
\vs Rom 1:26 Потому предал их Бог постыдным страстям: женщины их заменили естественное употребление противоестественным;
\vs Rom 1:27 подобно и мужчины, оставив естественное употребление женского пола, разжигались похотью друг на друга, мужчины на мужчинах делая срам и получая в самих себе должное возмездие за свое заблуждение.
\vs Rom 1:28 И как они не заботились иметь Бога в разуме, то предал их Бог превратному уму~--- делать непотребства,
\vs Rom 1:29 так что они исполнены всякой неправды, блуда, лукавства, корыстолюбия, злобы, исполнены зависти, убийства, распрей, обмана, злонравия,
\vs Rom 1:30 злоречивы, клеветники, богоненавистники, обидчики, самохвалы, горды, изобретательны на зло, непослушны родителям,
\vs Rom 1:31 безрассудны, вероломны, нелюбовны, непримиримы, немилостивы.
\vs Rom 1:32 Они знают праведный \bibemph{суд} Божий, что делающие такие \bibemph{дела} достойны смерти; однако не только \bibemph{их} делают, но и делающих одобряют.
\vs Rom 2:1 Итак, неизвинителен ты, всякий человек, судящий \bibemph{другого}, ибо тем же судом, каким судишь другого, осуждаешь себя, потому что, судя \bibemph{другого}, делаешь т\acc{о} же.
\vs Rom 2:2 А мы знаем, что поистине есть суд Божий на делающих такие \bibemph{дела}.
\vs Rom 2:3 Неужели думаешь ты, человек, что избежишь суда Божия, осуждая делающих такие \bibemph{дела} и (сам) делая т\acc{о} же?
\vs Rom 2:4 Или пренебрегаешь богатство благости, кротости и долготерпения Божия, не разумея, что благость Божия ведет тебя к покаянию?
\vs Rom 2:5 Но, по упорству твоему и нераскаянному сердцу, ты сам себе собираешь гнев на день гнева и откровения праведного суда от Бога,
\vs Rom 2:6 Который воздаст каждому по делам его:
\vs Rom 2:7 тем, которые постоянством в добром деле ищут славы, чести и бессмертия,~--- жизнь вечную;
\vs Rom 2:8 а тем, которые упорствуют и не покоряются истине, но предаются неправде,~--- ярость и гнев.
\vs Rom 2:9 Скорбь и теснота всякой душе человека, делающего злое, во-первых, Иудея, \bibemph{потом} и Еллина!
\vs Rom 2:10 Напротив, слава и честь и мир всякому, делающему доброе, во-первых, Иудею, \bibemph{потом} и Еллину!
\vs Rom 2:11 Ибо нет лицеприятия у Бога.
\rsbpar\vs Rom 2:12 Те, которые, не \bibemph{имея} закона, согрешили, вне закона и погибнут; а те, которые под законом согрешили, по закону осудятся
\vs Rom 2:13 (потому что не слушатели закона праведны пред Богом, но исполнители закона оправданы будут,
\vs Rom 2:14 ибо когда язычники, не имеющие закона, по природе законное делают, то, не имея закона, они сами себе закон:
\vs Rom 2:15 они показывают, что дело закона у них написано в сердцах, о чем свидетельствует совесть их и мысли их, то обвиняющие, то оправдывающие одна другую)
\vs Rom 2:16 в день, когда, по благовествованию моему, Бог будет судить тайные \bibemph{дела} человеков через Иисуса Христа.
\rsbpar\vs Rom 2:17 Вот, ты называешься Иудеем, и успокаиваешь себя законом, и хвалишься Богом,
\vs Rom 2:18 и знаешь волю \bibemph{Его}, и разумеешь лучшее, научаясь из закона,
\vs Rom 2:19 и уверен о себе, что ты путеводитель слепых, свет для находящихся во тьме,
\vs Rom 2:20 наставник невежд, учитель младенцев, имеющий в законе образец ведения и истины:
\vs Rom 2:21 как же ты, уча другого, не учишь себя самого?
\vs Rom 2:22 Проповедуя не красть, крадешь? говоря: <<не прелюбодействуй>>, прелюбодействуешь? гнушаясь идолов, святотатствуешь?
\vs Rom 2:23 Хвалишься законом, а преступлением закона бесчестишь Бога?
\vs Rom 2:24 Ибо ради вас, как написано, имя Божие хулится у язычников.
\vs Rom 2:25 Обрезание полезно, если исполняешь закон; а если ты преступник закона, то обрезание твое стало необрезанием.
\vs Rom 2:26 Итак, если необрезанный соблюдает постановления закона, то его необрезание не вменится ли ему в обрезание?
\vs Rom 2:27 И необрезанный по природе, исполняющий закон, не осудит ли тебя, преступника закона при Писании и обрезании?
\vs Rom 2:28 Ибо не тот Иудей, кто \bibemph{таков} по наружности, и не то обрезание, которое наружно, на плоти;
\vs Rom 2:29 но \bibemph{тот} Иудей, кто внутренно \bibemph{таков}, и \bibemph{то} обрезание, \bibemph{которое} в сердце, по духу, \bibemph{а} не по букве: ему и похвала не от людей, но от Бога.
\vs Rom 3:1 Итак, какое преимущество \bibemph{быть} Иудеем, или какая польза от обрезания?
\vs Rom 3:2 Великое преимущество во всех отношениях, а наипаче \bibemph{в том}, что им вверено слово Божие.
\vs Rom 3:3 Ибо что же? если некоторые и неверны были, неверность их уничтожит ли верность Божию?
\vs Rom 3:4 Никак. Бог верен, а всякий человек лжив, как написано: Ты праведен в словах Твоих и победишь в суде Твоем.
\vs Rom 3:5 Если же наша неправда открывает правду Божию, то что скажем? не будет ли Бог несправедлив, когда изъявляет гнев? (говорю по человеческому \bibemph{рассуждению}).
\vs Rom 3:6 Никак. Ибо \bibemph{иначе} как Богу судить мир?
\vs Rom 3:7 Ибо, если верность Божия возвышается моею неверностью к славе Божией, за что еще меня же судить, как грешника?
\vs Rom 3:8 И не делать ли нам зло, чтобы вышло добро, как некоторые злословят нас и говорят, будто мы так учим? Праведен суд на таковых.
\rsbpar\vs Rom 3:9 Итак, что же? имеем ли мы преимущество? Нисколько. Ибо мы уже доказали, что как Иудеи, так и Еллины, все под грехом,
\vs Rom 3:10 как написано: нет праведного ни одного;
\vs Rom 3:11 нет разумевающего; никто не ищет Бога;
\vs Rom 3:12 все совратились с пути, до одного негодны; нет делающего добро, нет ни одного.
\vs Rom 3:13 Гортань их~--- открытый гроб; языком своим обманывают; яд аспидов на губах их.
\vs Rom 3:14 Уста их полны злословия и горечи.
\vs Rom 3:15 Ноги их быстры на пролитие крови;
\vs Rom 3:16 разрушение и пагуба на путях их;
\vs Rom 3:17 они не знают пути мира.
\vs Rom 3:18 Нет страха Божия перед глазами их.
\rsbpar\vs Rom 3:19 Но мы знаем, что закон, если что говорит, говорит к состоящим под законом, так что заграждаются всякие уста, и весь мир становится виновен пред Богом,
\vs Rom 3:20 потому что делами закона не оправдается пред Ним никакая плоть; ибо законом познаётся грех.
\vs Rom 3:21 Но ныне, независимо от закона, явилась правда Божия, о которой свидетельствуют закон и пророки,
\vs Rom 3:22 правда Божия через веру в Иисуса Христа во всех и на всех верующих, ибо нет различия,
\vs Rom 3:23 потому что все согрешили и лишены славы Божией,
\vs Rom 3:24 получая оправдание даром, по благодати Его, искуплением во Христе Иисусе,
\vs Rom 3:25 которого Бог предложил в жертву умилостивления в Крови Его через веру, для показания правды Его в прощении грехов, соделанных прежде,
\vs Rom 3:26 во \bibemph{время} долготерпения Божия, к показанию правды Его в настоящее время, да \bibemph{явится} Он праведным и оправдывающим верующего в Иисуса.
\vs Rom 3:27 Где же то, чем бы хвалиться? уничтожено. Каким законом? \bibemph{законом} дел? Нет, но законом веры.
\vs Rom 3:28 Ибо мы признаём, что человек оправдывается верою, независимо от дел закона.
\vs Rom 3:29 Неужели Бог \bibemph{есть Бог} Иудеев только, а не и язычников? Конечно, и язычников,
\vs Rom 3:30 потому что один Бог, Который оправдает обрезанных по вере и необрезанных через веру.
\vs Rom 3:31 Итак, мы уничтожаем закон верою? Никак; но закон утверждаем.
\vs Rom 4:1 Чт\acc{о} же, скажем, Авраам, отец наш, приобрел по плоти?
\vs Rom 4:2 Если Авраам оправдался делами, он имеет похвалу, но не пред Богом.
\vs Rom 4:3 Ибо чт\acc{о} говорит Писание? Поверил Авраам Богу, и это вменилось ему в праведность.
\vs Rom 4:4 Воздаяние делающему вменяется не по милости, но по долгу.
\vs Rom 4:5 А не делающему, но верующему в Того, Кто оправдывает нечестивого, вера его вменяется в праведность.
\vs Rom 4:6 Так и Давид называет блаженным человека, которому Бог вменяет праведность независимо от дел:
\vs Rom 4:7 Блаженны, чьи беззакония прощены и чьи грехи покрыты.
\vs Rom 4:8 Блажен человек, которому Господь не вменит греха.
\vs Rom 4:9 Блаженство сие \bibemph{относится} к обрезанию, или к необрезанию? Мы говорим, что Аврааму вера вменилась в праведность.
\vs Rom 4:10 Когда вменилась? по обрезании или до обрезания? Не по обрезании, а до обрезания.
\vs Rom 4:11 И знак обрезания он получил, \bibemph{как} печать праведности через веру, которую \bibemph{имел} в необрезании, так что он стал отцом всех верующих в необрезании, чтобы и им вменилась праведность,
\vs Rom 4:12 и отцом обрезанных, не только \bibemph{принявших} обрезание, но и ходящих по следам веры отца нашего Авраама, которую \bibemph{имел он} в необрезании.
\vs Rom 4:13 Ибо не законом \bibemph{даровано} Аврааму, или семени его, обетование~--- быть наследником мира, но праведностью веры.
\vs Rom 4:14 Если утверждающиеся на законе суть наследники, то тщетна вера, бездейственно обетование;
\vs Rom 4:15 ибо закон производит гнев, потому что, где нет закона, нет и преступления.
\vs Rom 4:16 Итак по вере, чтобы \bibemph{было} по милости, дабы обетование было непреложно для всех, не только по закону, но и по вере потомков Авраама, который есть отец всем нам
\vs Rom 4:17 (как написано: Я поставил тебя отцом многих народов) пред Богом, Которому он поверил, животворящим мертвых и называющим несуществующее, как существующее.
\vs Rom 4:18 Он, сверх надежды, поверил с надеждою, через что сделался отцом многих народов, по сказанному: <<так \bibemph{многочисленно} будет семя твое>>.
\vs Rom 4:19 И, не изнемогши в вере, он не помышлял, что тело его, почти столетнего, уже омертвело, и утроба Саррина в омертвении;
\vs Rom 4:20 не поколебался в обетовании Божием неверием, но пребыл тверд в вере, воздав славу Богу
\vs Rom 4:21 и будучи вполне уверен, что Он силен и исполнить обещанное.
\vs Rom 4:22 Потому и вменилось ему в праведность.
\vs Rom 4:23 А впрочем не в отношении к нему одному написано, что вменилось ему,
\vs Rom 4:24 но и в отношении к нам; вменится и нам, верующим в Того, Кто воскресил из мертвых Иисуса Христа, Господа нашего,
\vs Rom 4:25 Который предан за грехи наши и воскрес для оправдания нашего.
\vs Rom 5:1 Итак, оправдавшись верою, мы имеем мир с Богом через Господа нашего Иисуса Христа,
\vs Rom 5:2 через Которого верою и получили мы доступ к той благодати, в которой стоим и хвалимся надеждою славы Божией.
\vs Rom 5:3 И не сим только, но хвалимся и скорбями, зная, что от скорби происходит терпение,
\vs Rom 5:4 от терпения опытность, от опытности надежда,
\vs Rom 5:5 а надежда не постыжает, потому что любовь Божия излилась в сердца наши Духом Святым, данным нам.
\vs Rom 5:6 Ибо Христос, когда еще мы были немощны, в определенное время умер за нечестивых.
\vs Rom 5:7 Ибо едва ли кто умрет за праведника; разве за благодетеля, может быть, кто и решится умереть.
\vs Rom 5:8 Но Бог Свою любовь к нам доказывает тем, что Христос умер за нас, когда мы были еще грешниками.
\vs Rom 5:9 Посему тем более ныне, будучи оправданы Кровию Его, спасемся Им от гнева.
\vs Rom 5:10 Ибо если, будучи врагами, мы примирились с Богом смертью Сына Его, то тем более, примирившись, спасемся жизнью Его.
\vs Rom 5:11 И не довольно сего, но и хвалимся Богом чрез Господа нашего Иисуса Христа, посредством Которого мы получили ныне примирение.
\rsbpar\vs Rom 5:12 Посему, как одним человеком грех вошел в мир, и грехом смерть, так и смерть перешла во всех человеков, \bibemph{потому что} в нем все согрешили.
\vs Rom 5:13 Ибо \bibemph{и} до закона грех был в мире; но грех не вменяется, когда нет закона.
\vs Rom 5:14 Однако же смерть царствовала от Адама до Моисея и над несогрешившими подобно преступлению Адама, который есть образ будущего.
\vs Rom 5:15 Но дар благодати не как преступление. Ибо если преступлением одного подверглись смерти многие, то тем более благодать Божия и дар по благодати одного Человека, Иисуса Христа, преизбыточествуют для многих.
\vs Rom 5:16 И дар не как \bibemph{суд} за одного согрешившего; ибо суд за одно \bibemph{преступление}~--- к осуждению; а дар благодати~--- к оправданию от многих преступлений.
\vs Rom 5:17 Ибо если преступлением одного смерть царствовала посредством одного, то тем более приемлющие обилие благодати и дар праведности будут царствовать в жизни посредством единого Иисуса Христа.
\vs Rom 5:18 Посему, как преступлением одного всем человекам осуждение, так правдою одного всем человекам оправдание к жизни.
\vs Rom 5:19 Ибо, как непослушанием одного человека сделались многие грешными, так и послушанием одного сделаются праведными многие.
\vs Rom 5:20 Закон же пришел после, и таким образом умножилось преступление. А когда умножился грех, стала преизобиловать благодать,
\vs Rom 5:21 дабы, как грех царствовал к смерти, так и благодать воцарилась через праведность к жизни вечной Иисусом Христом, Господом нашим.
\vs Rom 6:1 Что же скажем? оставаться ли нам в грехе, чтобы умножилась благодать? Никак.
\vs Rom 6:2 Мы умерли для греха: как же нам жить в нем?
\vs Rom 6:3 Неужели не знаете, что все мы, крестившиеся во Христа Иисуса, в смерть Его крестились?
\vs Rom 6:4 Итак мы погреблись с Ним крещением в смерть, дабы, как Христос воскрес из мертвых славою Отца, так и нам ходить в обновленной жизни.
\vs Rom 6:5 Ибо если мы соединены с Ним подобием смерти Его, то должны быть \bibemph{соединены} и \bibemph{подобием} воскресения,
\vs Rom 6:6 зная то, что ветхий наш человек распят с Ним, чтобы упразднено было тело греховное, дабы нам не быть уже рабами греху;
\vs Rom 6:7 ибо умерший освободился от греха.
\vs Rom 6:8 Если же мы умерли со Христом, то веруем, что и жить будем с Ним,
\vs Rom 6:9 зная, что Христос, воскреснув из мертвых, уже не умирает: смерть уже не имеет над Ним власти.
\vs Rom 6:10 Ибо, что Он умер, то умер однажды для греха; а что живет, то живет для Бога.
\vs Rom 6:11 Так и вы почитайте себя мертвыми для греха, живыми же для Бога во Христе Иисусе, Господе нашем.
\rsbpar\vs Rom 6:12 Итак да не царствует грех в смертном вашем теле, чтобы вам повиноваться ему в похотях его;
\vs Rom 6:13 и не предавайте членов ваших греху в орудия неправды, но представьте себя Богу, как оживших из мертвых, и члены ваши Богу в орудия праведности.
\vs Rom 6:14 Грех не должен над вами господствовать, ибо вы не под законом, но под благодатью.
\rsbpar\vs Rom 6:15 Что же? станем ли грешить, потому что мы не под законом, а под благодатью? Никак.
\vs Rom 6:16 Неужели вы не знаете, что, кому вы отдаете себя в рабы для послушания, того вы и рабы, кому повинуетесь, или \bibemph{рабы} греха к смерти, или послушания к праведности?
\vs Rom 6:17 Благодарение Богу, что вы, быв прежде рабами греха, от сердца стали послушны тому образу учения, которому предали себя.
\vs Rom 6:18 Освободившись же от греха, вы стали рабами праведности.
\vs Rom 6:19 Говорю по \bibemph{рассуждению} человеческому, ради немощи плоти вашей. Как предавали вы члены ваши в рабы нечистоте и беззаконию на \bibemph{дела} беззаконные, так ныне представьте члены ваши в рабы праведности на \bibemph{дела} святые.
\vs Rom 6:20 Ибо, когда вы были рабами греха, тогда были свободны от праведности.
\vs Rom 6:21 Какой же плод вы имели тогда? \bibemph{Такие дела}, каких ныне сами стыдитесь, потому что конец их~--- смерть.
\vs Rom 6:22 Но ныне, когда вы освободились от греха и стали рабами Богу, плод ваш есть святость, а конец~--- жизнь вечная.
\vs Rom 6:23 Ибо возмездие за грех~--- смерть, а дар Божий~--- жизнь вечная во Христе Иисусе, Господе нашем.
\vs Rom 7:1 Разве вы не знаете, братия (ибо говорю знающим закон), что закон имеет власть над человеком, пока он жив?
\vs Rom 7:2 Замужняя женщина привязана законом к живому мужу; а если умрет муж, она освобождается от закона замужества.
\vs Rom 7:3 Посему, если при живом муже выйдет за другого, называется прелюбодейцею; если же умрет муж, она свободна от закона, и не будет прелюбодейцею, выйдя за другого мужа.
\vs Rom 7:4 Так и вы, братия мои, умерли для закона телом Христовым, чтобы принадлежать другому, Воскресшему из мертвых, да приносим плод Богу.
\vs Rom 7:5 Ибо, когда мы жили по плоти, тогда страсти греховные, \bibemph{обнаруживаемые} законом, действовали в членах наших, чтобы приносить плод смерти;
\vs Rom 7:6 но ныне, умерши для закона, которым были связаны, мы освободились от него, чтобы нам служить Богу в обновлении духа, а не по ветхой букве.
\rsbpar\vs Rom 7:7 Что же скажем? Неужели \bibemph{от} закона грех? Никак. Но я не иначе узнал грех, как посредством закона. Ибо я не понимал бы и пожелания, если бы закон не говорил: не пожелай.
\vs Rom 7:8 Но грех, взяв повод от заповеди, произвел во мне всякое пожелание: ибо без закона грех мертв.
\vs Rom 7:9 Я жил некогда без закона; но когда пришла заповедь, то грех ожил,
\vs Rom 7:10 а я умер; и таким образом заповедь, \bibemph{данная} для жизни, послужила мне к смерти,
\vs Rom 7:11 потому что грех, взяв повод от заповеди, обольстил меня и умертвил ею.
\vs Rom 7:12 Посему закон свят, и заповедь свята и праведна и добра.
\vs Rom 7:13 Итак, неужели доброе сделалось мне смертоносным? Никак; но грех, оказывающийся грехом потому, что посредством доброго причиняет мне смерть, так что грех становится крайне грешен посредством заповеди.
\vs Rom 7:14 Ибо мы знаем, что закон духовен, а я плотян, продан греху.
\vs Rom 7:15 Ибо не понимаю, что делаю: потому что не то делаю, что хочу, а что ненавижу, то делаю.
\vs Rom 7:16 Если же делаю то, чего не хочу, то соглашаюсь с законом, что он добр,
\vs Rom 7:17 а потому уже не я делаю то, но живущий во мне грех.
\vs Rom 7:18 Ибо знаю, что не живет во мне, то есть в плоти моей, доброе; потому что желание добра есть во мне, но чтобы сделать оное, того не нахожу.
\vs Rom 7:19 Доброго, которого хочу, не делаю, а злое, которого не хочу, делаю.
\vs Rom 7:20 Если же делаю т\acc{о}, чего не хочу, уже не я делаю то, но живущий во мне грех.
\vs Rom 7:21 Итак я нахожу закон, что, когда хочу делать доброе, прилежит мне злое.
\vs Rom 7:22 Ибо по внутреннему человеку нахожу удовольствие в законе Божием;
\vs Rom 7:23 но в членах моих вижу иной закон, противоборствующий закону ума моего и делающий меня пленником закона греховного, находящегося в членах моих.
\vs Rom 7:24 Бедный я человек! кто избавит меня от сего тела смерти?
\vs Rom 7:25 Благодарю Бога моего Иисусом Христом, Господом нашим. Итак тот же самый я умом моим служу закону Божию, а плотию закону греха.
\vs Rom 8:1 Итак нет ныне никакого осуждения тем, которые во Христе Иисусе живут не по плоти, но по духу,
\vs Rom 8:2 потому что закон духа жизни во Христе Иисусе освободил меня от закона греха и смерти.
\vs Rom 8:3 Как закон, ослабленный плотию, был бессилен, то Бог послал Сына Своего в подобии плоти греховной \bibemph{в жертву} за грех и осудил грех во плоти,
\vs Rom 8:4 чтобы оправдание закона исполнилось в нас, живущих не по плоти, но по духу.
\vs Rom 8:5 Ибо живущие по плоти о плотском помышляют, а живущие по духу~--- о духовном.
\vs Rom 8:6 Помышления плотские суть смерть, а помышления духовные~--- жизнь и мир,
\vs Rom 8:7 потому что плотские помышления суть вражда против Бога; ибо закону Божию не покоряются, да и не могут.
\vs Rom 8:8 Посему живущие по плоти Богу угодить не могут.
\vs Rom 8:9 Но вы не по плоти живете, а по духу, если только Дух Божий живет в вас. Если же кто Духа Христова не имеет, тот \bibemph{и} не Его.
\vs Rom 8:10 А если Христос в вас, то тело мертво для греха, но дух жив для праведности.
\vs Rom 8:11 Если же Дух Того, Кто воскресил из мертвых Иисуса, живет в вас, то Воскресивший Христа из мертвых оживит и ваши смертные тела Духом Своим, живущим в вас.
\rsbpar\vs Rom 8:12 Итак, братия, мы не должники плоти, чтобы жить по плоти;
\vs Rom 8:13 ибо если живете по плоти, то умрете, а если духом умерщвляете дела плотские, то живы будете.
\vs Rom 8:14 Ибо все, водимые Духом Божиим, суть сыны Божии.
\vs Rom 8:15 Потому что вы не приняли духа рабства, \bibemph{чтобы} опять \bibemph{жить} в страхе, но приняли Духа усыновления, Которым взываем: <<Авва, Отче!>>
\vs Rom 8:16 Сей самый Дух свидетельствует духу нашему, что мы~--- дети Божии.
\vs Rom 8:17 А если дети, то и наследники, наследники Божии, сонаследники же Христу, если только с Ним страдаем, чтобы с Ним и прославиться.
\rsbpar\vs Rom 8:18 Ибо думаю, что нынешние временные страдания ничего не стоят в сравнении с тою славою, которая откроется в нас.
\vs Rom 8:19 Ибо тварь с надеждою ожидает откровения сынов Божиих,
\vs Rom 8:20 потому что тварь покорилась суете не добровольно, но по воле покорившего ее, в надежде,
\vs Rom 8:21 что и сама тварь освобождена будет от рабства тлению в свободу славы детей Божиих.
\vs Rom 8:22 Ибо знаем, что вся тварь совокупно стенает и мучится доныне;
\vs Rom 8:23 и не только \bibemph{она}, но и мы сами, имея начаток Духа, и мы в себе стенаем, ожидая усыновления, искупления тела нашего.
\vs Rom 8:24 Ибо мы спасены в надежде. Надежда же, когда видит, не есть надежда; ибо если кто видит, то чего ему и надеяться?
\vs Rom 8:25 Но когда надеемся того, чего не видим, тогда ожидаем в терпении.
\rsbpar\vs Rom 8:26 Также и Дух подкрепляет нас в немощах наших; ибо мы не знаем, о чем молиться, как должно, но Сам Дух ходатайствует за нас воздыханиями неизреченными.
\vs Rom 8:27 Испытующий же сердца знает, какая мысль у Духа, потому что Он ходатайствует за святых по \bibemph{воле} Божией.
\vs Rom 8:28 Притом знаем, что любящим Бога, призванным по \bibemph{Его} изволению, все содействует ко благу.
\vs Rom 8:29 Ибо кого Он предузнал, тем и предопределил быть подобными образу Сына Своего, дабы Он был первородным между многими братиями.
\vs Rom 8:30 А кого Он предопределил, тех и призвал, а кого призвал, тех и оправдал; а кого оправдал, тех и прославил.
\vs Rom 8:31 Что же сказать на это? Если Бог за нас, кто против нас?
\vs Rom 8:32 Тот, Который Сына Своего не пощадил, но предал Его за всех нас, как с Ним не дарует нам и всего?
\vs Rom 8:33 Кто будет обвинять избранных Божиих? Бог оправдывает \bibemph{их}.
\vs Rom 8:34 Кто осуждает? Христос Иисус умер, но и воскрес: Он и одесную Бога, Он и ходатайствует за нас.
\vs Rom 8:35 Кто отлучит нас от любви Божией: скорбь, или теснота, или гонение, или голод, или нагота, или опасность, или меч? как написано:
\vs Rom 8:36 за Тебя умерщвляют нас всякий день, считают нас за овец, \bibemph{обреченных} на заклание.
\vs Rom 8:37 Но все сие преодолеваем силою Возлюбившего нас.
\vs Rom 8:38 Ибо я уверен, что ни смерть, ни жизнь, ни Ангелы, ни Начала, ни Силы, ни настоящее, ни будущее,
\vs Rom 8:39 ни высота, ни глубина, ни другая какая тварь не может отлучить нас от любви Божией во Христе Иисусе, Господе нашем.
\vs Rom 9:1 Истину говорю во Христе, не лгу, свидетельствует мне совесть моя в Духе Святом,
\vs Rom 9:2 что великая для меня печаль и непрестанное мучение сердцу моему:
\vs Rom 9:3 я желал бы сам быть отлученным от Христа за братьев моих, родных мне по плоти,
\vs Rom 9:4 то есть Израильтян, которым принадлежат усыновление и слава, и заветы, и законоположение, и богослужение, и обетования;
\vs Rom 9:5 их и отцы, и от них Христос по плоти, сущий над всем Бог, благословенный во веки, аминь.
\vs Rom 9:6 Но не т\acc{о}, чтобы слово Божие не сбылось: ибо не все те Израильтяне, которые от Израиля;
\vs Rom 9:7 и не все дети Авраама, которые от семени его, но сказано: в Исааке наречется тебе семя.
\vs Rom 9:8 То есть не плотские дети суть дети Божии, но дети обетования признаются за семя.
\vs Rom 9:9 А слово обетования таково: в это же время приду, и у Сарры будет сын.
\vs Rom 9:10 И не одно это; но \bibemph{так было} и с Ревеккою, когда она зачала в одно время \bibemph{двух сыновей} от Исаака, отца нашего.
\vs Rom 9:11 Ибо, когда они еще не родились и не сделали ничего доброго или худого (дабы изволение Божие в избрании происходило
\vs Rom 9:12 не от дел, но от Призывающего), сказано было ей: больший будет в порабощении у меньшего,
\vs Rom 9:13 как и написано: Иакова Я возлюбил, а Исава возненавидел.
\rsbpar\vs Rom 9:14 Чт\acc{о} же скажем? Неужели неправда у Бога? Никак.
\vs Rom 9:15 Ибо Он говорит Моисею: кого миловать, помилую; кого жалеть, пожалею.
\vs Rom 9:16 Итак \bibemph{помилование зависит} не от желающего и не от подвизающегося, но от Бога милующего.
\vs Rom 9:17 Ибо Писание говорит фараону: для того самого Я и поставил тебя, чтобы показать над тобою силу Мою и чтобы проповедано было имя Мое по всей земле.
\vs Rom 9:18 Итак, кого хочет, милует; а кого хочет, ожесточает.
\rsbpar\vs Rom 9:19 Ты скажешь мне: <<за что же еще обвиняет? Ибо кто противостанет воле Его?>>
\vs Rom 9:20 А ты кто, человек, что споришь с Богом? Изделие скажет ли сделавшему его: <<зачем ты меня так сделал?>>
\vs Rom 9:21 Не властен ли горшечник над глиною, чтобы из той же смеси сделать один сосуд для почетного \bibemph{употребления}, а другой для низкого?
\vs Rom 9:22 Что же, если Бог, желая показать гнев и явить могущество Свое, с великим долготерпением щадил сосуды гнева, готовые к погибели,
\vs Rom 9:23 дабы вместе явить богатство славы Своей над сосудами милосердия, которые Он приготовил к славе,
\vs Rom 9:24 над нами, которых Он призвал не только из Иудеев, но и из язычников?
\vs Rom 9:25 Как и у Осии говорит: не Мой народ назову Моим народом, и не возлюбленную~--- возлюбленною.
\vs Rom 9:26 И на том месте, где сказано им: вы не Мой народ, там названы будут сынами Бога живаго.
\vs Rom 9:27 А Исаия провозглашает об Израиле: хотя бы сыны Израилевы были числом, как песок морской, \bibemph{только} остаток спасется;
\vs Rom 9:28 ибо дело оканчивает и скоро решит по правде, дело решительное совершит Господь на земле.
\vs Rom 9:29 И, как предсказал Исаия: если бы Господь Саваоф не оставил нам семени, то мы сделались бы, как Содом, и были бы подобны Гоморре.
\rsbpar\vs Rom 9:30 Что же скажем? Язычники, не искавшие праведности, получили праведность, праведность от веры.
\vs Rom 9:31 А Израиль, искавший закона праведности, не достиг до закона праведности.
\vs Rom 9:32 Почему? потому что \bibemph{искали} не в вере, а в делах закона. Ибо преткнулись о камень преткновения,
\vs Rom 9:33 как написано: вот, полагаю в Сионе камень преткновения и камень соблазна; но всякий, верующий в Него, не постыдится.
\vs Rom 10:1 Братия! желание моего сердца и молитва к Богу об Израиле во спасение.
\vs Rom 10:2 Ибо свидетельствую им, что имеют ревность по Боге, но не по рассуждению.
\vs Rom 10:3 Ибо, не разумея праведности Божией и усиливаясь поставить собственную праведность, они не покорились праведности Божией,
\vs Rom 10:4 потому что конец закона~--- Христос, к праведности всякого верующего.
\vs Rom 10:5 Моисей пишет о праведности от закона: исполнивший его человек жив будет им.
\vs Rom 10:6 А праведность от веры так говорит: не говори в сердце твоем: кто взойдет на небо? то есть Христа свести.
\vs Rom 10:7 Или кто сойдет в бездну? то есть Христа из мертвых возвести.
\vs Rom 10:8 Но что говорит Писание? Близко к тебе слово, в устах твоих и в сердце твоем, то есть слово веры, которое проповедуем.
\vs Rom 10:9 Ибо если устами твоими будешь исповедовать Иисуса Господом и сердцем твоим веровать, что Бог воскресил Его из мертвых, то спасешься,
\vs Rom 10:10 потому что сердцем веруют к праведности, а устами исповедуют ко спасению.
\vs Rom 10:11 Ибо Писание говорит: всякий, верующий в Него, не постыдится.
\vs Rom 10:12 Здесь нет различия между Иудеем и Еллином, потому что один Господь у всех, богатый для всех, призывающих Его.
\vs Rom 10:13 Ибо всякий, кто призовет имя Господне, спасется.
\rsbpar\vs Rom 10:14 Но как призывать \bibemph{Того}, в Кого не уверовали? как веровать \bibemph{в Того}, о Ком не слыхали? как слышать без проповедующего?
\vs Rom 10:15 И как проповедовать, если не будут посланы? как написано: как прекрасны ноги благовествующих мир, благовествующих благое!
\vs Rom 10:16 Но не все послушались благовествования. Ибо Исаия говорит: Господи! кто поверил слышанному от нас?
\vs Rom 10:17 Итак вера от слышания, а слышание от слова Божия.
\vs Rom 10:18 Но спрашиваю: разве они не слышали? Напротив, по всей земле прошел голос их, и до пределов вселенной слова их.
\vs Rom 10:19 Еще спрашиваю: разве Израиль не знал? Но первый Моисей говорит: Я возбужу в вас ревность не народом, раздражу вас народом несмысленным.
\vs Rom 10:20 А Исаия смело говорит: Меня нашли не искавшие Меня; Я открылся не вопрошавшим о Мне.
\vs Rom 10:21 Об Израиле же говорит: целый день Я простирал руки Мои к народу непослушному и упорному.
\vs Rom 11:1 Итак, спрашиваю: неужели Бог отверг народ Свой? Никак. Ибо и я Израильтянин, от семени Авраамова, из колена Вениаминова.
\vs Rom 11:2 Не отверг Бог народа Своего, который Он наперед знал. Или не знаете, что говорит Писание в \bibemph{повествовании об} Илии? как он жалуется Богу на Израиля, говоря:
\vs Rom 11:3 Господи! пророков Твоих убили, жертвенники Твои разрушили; остался я один, и моей души ищут.
\vs Rom 11:4 Что же говорит ему Божеский ответ? Я соблюл Себе семь тысяч человек, которые не преклонили колени перед Ваалом.
\vs Rom 11:5 Так и в нынешнее время, по избранию благодати, сохранился остаток.
\vs Rom 11:6 Но если по благодати, то не по делам; иначе благодать не была бы уже благодатью. А если по делам, то это уже не благодать; иначе дело не есть уже дело.
\vs Rom 11:7 Что же? Израиль, чего искал, того не получил; избранные же получили, а прочие ожесточились,
\vs Rom 11:8 как написано: Бог дал им дух усыпления, глаза, которыми не видят, и уши, которыми не слышат, даже до сего дня.
\vs Rom 11:9 И Давид говорит: да будет трапеза их сетью, тенетами и петлею в возмездие им;
\vs Rom 11:10 да помрачатся глаза их, чтобы не видеть, и хребет их да будет согбен навсегда.
\rsbpar\vs Rom 11:11 Итак спрашиваю: неужели они преткнулись, чтобы \bibemph{совсем} пасть? Никак. Но от их падения спасение язычникам, чтобы возбудить в них ревность.
\vs Rom 11:12 Если же падение их~--- богатство миру, и оскудение их~--- богатство язычникам, то тем более полнота их.
\rsbpar\vs Rom 11:13 Вам говорю, язычникам. Как Апостол язычников, я прославляю служение мое.
\vs Rom 11:14 Не возбужу ли ревность в \bibemph{сродниках} моих по плоти и не спасу ли некоторых из них?
\vs Rom 11:15 Ибо если отвержение их~--- примирение мира, то что \bibemph{будет} принятие, как не жизнь из мертвых?
\vs Rom 11:16 Если начаток свят, то и целое; и если корень свят, то и ветви.
\vs Rom 11:17 Если же некоторые из ветвей отломились, а ты, дикая маслина, привился на место их и стал общником корня и сока маслины,
\vs Rom 11:18 то не превозносись перед ветвями. Если же превозносишься, \bibemph{то вспомни, что} не ты корень держишь, но корень тебя.
\vs Rom 11:19 Скажешь: <<ветви отломились, чтобы мне привиться>>.
\vs Rom 11:20 Хорошо. Они отломились неверием, а ты держишься верою: не гордись, но бойся.
\vs Rom 11:21 Ибо если Бог не пощадил природных ветвей, то смотри, пощадит ли и тебя.
\vs Rom 11:22 Итак видишь благость и строгость Божию: строгость к отпадшим, а благость к тебе, если пребудешь в благости \bibemph{Божией}; иначе и ты будешь отсечен.
\vs Rom 11:23 Но и те, если не пребудут в неверии, привьются, потому что Бог силен опять привить их.
\vs Rom 11:24 Ибо если ты отсечен от дикой по природе маслины и не по природе привился к хорошей маслине, то тем более сии природные привьются к своей маслине.
\rsbpar\vs Rom 11:25 Ибо не хочу оставить вас, братия, в неведении о тайне сей,~--- чтобы вы не мечтали о себе,~--- что ожесточение произошло в Израиле отчасти, \bibemph{до времени}, пока войдет полное \bibemph{число} язычников;
\vs Rom 11:26 и так весь Израиль спасется, как написано: придет от Сиона Избавитель, и отвратит нечестие от Иакова.
\vs Rom 11:27 И сей завет им от Меня, когда сниму с них грехи их.
\vs Rom 11:28 В отношении к благовестию, они враги ради вас; а в отношении к избранию, возлюбленные \bibemph{Божии} ради отцов.
\vs Rom 11:29 Ибо дары и призвание Божие непреложны.
\vs Rom 11:30 Как и вы некогда были непослушны Богу, а ныне помилованы, по непослушанию их,
\vs Rom 11:31 так и они теперь непослушны для помилования вас, чтобы и сами они были помилованы.
\vs Rom 11:32 Ибо всех заключил Бог в непослушание, чтобы всех помиловать.
\rsbpar\vs Rom 11:33 О, бездна богатства и премудрости и ведения Божия! Как непостижимы судьбы Его и неисследимы пути Его!
\vs Rom 11:34 Ибо кто познал ум Господень? Или кто был советником Ему?
\vs Rom 11:35 Или кто дал Ему наперед, чтобы Он должен был воздать?
\vs Rom 11:36 Ибо все из Него, Им и к Нему. Ему слава во веки, аминь.
\vs Rom 12:1 Итак умоляю вас, братия, милосердием Божиим, представьте тела ваши в жертву живую, святую, благоугодную Богу, \bibemph{для} разумного служения вашего,
\vs Rom 12:2 и не сообразуйтесь с веком сим, но преобразуйтесь обновлением ума вашего, чтобы вам познавать, чт\acc{о} есть воля Божия, благая, угодная и совершенная.
\rsbpar\vs Rom 12:3 По данной мне благодати, всякому из вас говорю: не думайте \bibemph{о себе} более, нежели должно думать; но думайте скромно, по мере веры, какую каждому Бог уделил.
\vs Rom 12:4 Ибо, как в одном теле у нас много членов, но не у всех членов одно и то же дело,
\vs Rom 12:5 так мы, многие, составляем одно тело во Христе, а порознь один для другого члены.
\vs Rom 12:6 И как, по данной нам благодати, имеем различные дарования, \bibemph{то, имеешь ли} пророчество, \bibemph{пророчествуй} по мере веры;
\vs Rom 12:7 \bibemph{имеешь ли} служение, \bibemph{пребывай} в служении; учитель ли,~--- в учении;
\vs Rom 12:8 увещатель ли, увещевай; раздаватель ли, \bibemph{раздавай} в простоте; начальник ли, \bibemph{начальствуй} с усердием; благотворитель ли, \bibemph{благотвори} с радушием.
\vs Rom 12:9 Любовь \bibemph{да будет} непритворна; отвращайтесь зла, прилепляйтесь к добру;
\vs Rom 12:10 будьте братолюбивы друг к другу с нежностью; в почтительности друг друга предупреждайте;
\vs Rom 12:11 в усердии не ослабевайте; духом пламенейте; Господу служите;
\vs Rom 12:12 утешайтесь надеждою; в скорби \bibemph{будьте} терпеливы, в молитве постоянны;
\vs Rom 12:13 в нуждах святых принимайте участие; ревнуйте о странноприимстве.
\vs Rom 12:14 Благословляйте гонителей ваших; благословляйте, а не проклинайте.
\vs Rom 12:15 Радуйтесь с радующимися и плачьте с плачущими.
\vs Rom 12:16 Будьте единомысленны между собою; не высокомудрствуйте, но последуйте смиренным; не мечтайте о себе;
\vs Rom 12:17 никому не воздавайте злом за зло, но пекитесь о добром перед всеми человеками.
\vs Rom 12:18 Если возможно с вашей стороны, будьте в мире со всеми людьми.
\vs Rom 12:19 Не мстите за себя, возлюбленные, но дайте место гневу \bibemph{Божию}. Ибо написано: Мне отмщение, Я воздам, говорит Господь.
\vs Rom 12:20 Итак, если враг твой голоден, накорми его; если жаждет, напой его: ибо, делая сие, ты соберешь ему на голову горящие уголья.
\vs Rom 12:21 Не будь побежден злом, но побеждай зло добром.
\vs Rom 13:1 Всякая душа да будет покорна высшим властям, ибо нет власти не от Бога; существующие же власти от Бога установлены.
\vs Rom 13:2 Посему противящийся власти противится Божию установлению. А противящиеся сами навлекут на себя осуждение.
\vs Rom 13:3 Ибо начальствующие страшны не для добрых дел, но для злых. Хочешь ли не бояться власти? Делай добро, и получишь похвалу от нее,
\vs Rom 13:4 ибо \bibemph{начальник} есть Божий слуга, тебе на добро. Если же делаешь зло, бойся, ибо он не напрасно носит меч: он Божий слуга, отмститель в наказание делающему злое.
\vs Rom 13:5 И потому надобно повиноваться не только из \bibemph{страха} наказания, но и по совести.
\vs Rom 13:6 Для сего вы и подати платите, ибо они Божии служители, сим самым постоянно занятые.
\vs Rom 13:7 Итак отдавайте всякому должное: кому п\acc{о}дать, подать; кому оброк, оброк; кому страх, страх; кому честь, честь.
\rsbpar\vs Rom 13:8 Не оставайтесь должными никому ничем, кроме взаимной любви; ибо любящий другого исполнил закон.
\vs Rom 13:9 Ибо заповеди: не прелюбодействуй, не убивай, не кради, не лжесвидетельствуй, не пожелай \bibemph{чужого} и все другие заключаются в сем слове: люби ближнего твоего, как самого себя.
\vs Rom 13:10 Любовь не делает ближнему зла; итак любовь есть исполнение закона.
\rsbpar\vs Rom 13:11 Так \bibemph{поступайте}, зная время, что наступил уже час пробудиться нам от сна. Ибо ныне ближе к нам спасение, нежели когда мы уверовали.
\vs Rom 13:12 Ночь прошла, а день приблизился: итак отвергнем дела тьмы и облечемся в оружия света.
\vs Rom 13:13 Как днем, будем вести себя благочинно, не \bibemph{предаваясь} ни пированиям и пьянству, ни сладострастию и распутству, ни ссорам и зависти;
\vs Rom 13:14 но облекитесь в Господа нашего Иисуса Христа, и попечения о плоти не превращайте в похоти.
\vs Rom 14:1 Немощного в вере принимайте без споров о мнениях.
\vs Rom 14:2 Ибо иной уверен, \bibemph{что можно} есть все, а немощный ест овощи.
\vs Rom 14:3 Кто ест, не уничижай того, кто не ест; и кто не ест, не осуждай того, кто ест, потому что Бог принял его.
\vs Rom 14:4 Кто ты, осуждающий чужого раба? Перед своим Господом сто\acc{и}т он, или падает. И будет восставлен, ибо силен Бог восставить его.
\vs Rom 14:5 Иной отличает день от дня, а другой судит о всяком дне \bibemph{равно}. Всякий \bibemph{поступай} по удостоверению своего ума.
\vs Rom 14:6 Кто различает дни, для Господа различает; и кто не различает дней, для Господа не различает. Кто ест, для Господа ест, ибо благодарит Бога; и кто не ест, для Господа не ест, и благодарит Бога.
\vs Rom 14:7 Ибо никто из нас не живет для себя, и никто не умирает для себя;
\vs Rom 14:8 а живем ли~--- для Господа живем; умираем ли~--- для Господа умираем: и потому, живем ли или умираем,~--- \bibemph{всегда} Господни.
\vs Rom 14:9 Ибо Христос для того и умер, и воскрес, и ожил, чтобы владычествовать и над мертвыми и над живыми.
\vs Rom 14:10 А ты чт\acc{о} осуждаешь брата твоего? Или и ты, чт\acc{о} унижаешь брата твоего? Все мы предстанем на суд Христов.
\vs Rom 14:11 Ибо написано: живу Я, говорит Господь, предо Мною преклонится всякое колено, и всякий язык будет исповедовать Бога.
\vs Rom 14:12 Итак каждый из нас за себя даст отчет Богу.
\rsbpar\vs Rom 14:13 Не станем же более судить друг друга, а лучше судите о том, как бы не подавать брату \bibemph{случая к} преткновению или соблазну.
\vs Rom 14:14 Я знаю и уверен в Господе Иисусе, что нет ничего в себе самом нечистого; только почитающему что-либо нечистым, тому нечисто.
\vs Rom 14:15 Если же за пищу огорчается брат твой, то ты уже не по любви поступаешь. Не губи твоею пищею того, за кого Христос умер.
\vs Rom 14:16 Да не хулится ваше доброе.
\vs Rom 14:17 Ибо Царствие Божие не пища и питие, но праведность и мир и радость во Святом Духе.
\vs Rom 14:18 Кто сим служит Христу, тот угоден Богу и \bibemph{достоин} одобрения от людей.
\vs Rom 14:19 Итак будем искать того, что служит к миру и ко взаимному назиданию.
\vs Rom 14:20 Ради пищи не разрушай д\acc{е}ла Божия. Все чисто, но худо человеку, который ест на соблазн.
\vs Rom 14:21 Лучше не есть мяса, не пить вина и не \bibemph{делать} ничего \bibemph{такого}, отчего брат твой претыкается, или соблазняется, или изнемогает.
\vs Rom 14:22 Ты имеешь веру? имей ее сам в себе, пред Богом. Блажен, кто не осуждает себя в том, чт\acc{о} избирает.
\vs Rom 14:23 А сомневающийся, если ест, осуждается, потому что не по вере; а все, что не по вере, грех.
\vs Rom 14:24 Могущему же утвердить вас, по благовествованию моему и проповеди Иисуса Христа, по откровению тайны, о которой от вечных времен было умолчано,
\vs Rom 14:25 но которая ныне явлена, и через писания пророческие, по повелению вечного Бога, возвещена всем народам для покорения их вере,
\vs Rom 14:26 Единому Премудрому Богу, через Иисуса Христа, слава во веки. Аминь.
\vs Rom 15:1 Мы, сильные, должны сносить немощи бессильных и не себе угождать.
\vs Rom 15:2 Каждый из нас должен угождать ближнему, во благо, к назиданию.
\vs Rom 15:3 Ибо и Христос не Себе угождал, но, как написано: злословия злословящих Тебя пали на Меня.
\vs Rom 15:4 А все, что писано было прежде, написано нам в наставление, чтобы мы терпением и утешением из Писаний сохраняли надежду.
\vs Rom 15:5 Бог же терпения и утешения да дарует вам быть в единомыслии между собою, по \bibemph{учению} Христа Иисуса,
\vs Rom 15:6 дабы вы единодушно, едиными устами славили Бога и Отца Господа нашего Иисуса Христа.
\vs Rom 15:7 Посему принимайте друг друга, как и Христос принял вас в славу Божию.
\rsbpar\vs Rom 15:8 Разумею то, что Иисус Христос сделался служителем для обрезанных~--- ради истины Божией, чтобы исполнить обещанное отцам,
\vs Rom 15:9 а для язычников~--- из милости, чтобы славили Бога, как написано: за т\acc{о} буду славить Тебя, (Господи,) между язычниками, и буду петь имени Твоему.
\vs Rom 15:10 И еще сказано: возвеселитесь, язычники, с народом Его.
\vs Rom 15:11 И еще: хвал\acc{и}те Господа, все язычники, и прославляйте Его, все народы.
\vs Rom 15:12 Исаия также говорит: будет корень Иессеев, и восстанет владеть народами; на Него язычники надеяться будут.
\vs Rom 15:13 Бог же надежды да исполнит вас всякой радости и мира в вере, дабы вы, силою Духа Святаго, обогатились надеждою.
\rsbpar\vs Rom 15:14 И сам я уверен о вас, братия мои, что и вы полны благости, исполнены всякого познания и можете наставлять друг друга;
\vs Rom 15:15 но писал вам, братия, с некоторою смелостью, отчасти как бы в напоминание вам, по данной мне от Бога благодати
\vs Rom 15:16 быть служителем Иисуса Христа у язычников и \bibemph{совершать} священнодействие благовествования Божия, дабы сие приношение язычников, будучи освящено Духом Святым, было благоприятно \bibemph{Богу}.
\vs Rom 15:17 Итак я могу похвалиться в Иисусе Христе в том, чт\acc{о} \bibemph{относится} к Богу,
\vs Rom 15:18 ибо не осмелюсь сказать что-нибудь такое, чего не совершил Христос через меня, в покорении язычников \bibemph{вере}, словом и делом,
\vs Rom 15:19 силою знамений и чудес, силою Духа Божия, так что благовествование Христово распространено мною от Иерусалима и окрестности до Иллирика.
\vs Rom 15:20 Притом я старался благовествовать не там, где \bibemph{уже} было известно имя Христово, дабы не созидать на чужом основании,
\vs Rom 15:21 но как написано: не имевшие о Нем известия увидят, и не слышавшие узн\acc{а}ют.
\vs Rom 15:22 Сие-то много раз и препятствовало мне прийти к вам.
\vs Rom 15:23 Ныне же, не имея \bibemph{такого} места в сих странах, а с давних лет имея желание прийти к вам,
\vs Rom 15:24 как только предприму путь в Испанию, приду к вам. Ибо надеюсь, что, проходя, увижусь с вами и что вы проводите меня туда, как скоро наслажусь \bibemph{общением} с вами, хотя отчасти.
\vs Rom 15:25 А теперь я иду в Иерусалим, чтобы послужить святым,
\vs Rom 15:26 ибо Македония и Ахаия усердствуют некоторым подаянием для бедных между святыми в Иерусалиме.
\vs Rom 15:27 Усердствуют, да и должники они перед ними. Ибо если язычники сделались участниками в их духовном, то должны и им послужить в телесном.
\vs Rom 15:28 Исполнив это и верно доставив им сей плод \bibemph{усердия}, я отправлюсь через ваши \bibemph{места} в Испанию,
\vs Rom 15:29 и уверен, что когда приду к вам, то приду с полным благословением благовествования Христова.
\rsbpar\vs Rom 15:30 Между тем умоляю вас, братия, Господом нашим Иисусом Христом и любовью Духа, подвизаться со мною в молитвах за меня к Богу,
\vs Rom 15:31 чтобы избавиться мне от неверующих в Иудее и чтобы служение мое для Иерусалима было благоприятно святым,
\vs Rom 15:32 дабы мне в радости, если Богу угодно, прийти к вам и успокоиться с вами.
\vs Rom 15:33 Бог же мира да будет со всеми вами, аминь.
\vs Rom 16:1 Представляю вам Фиву, сестру нашу, диакониссу церкви Кенхрейской.
\vs Rom 16:2 Примите ее для Господа, как прилично святым, и помогите ей, в чем она будет иметь нужду у вас, ибо и она была помощницею многим и мне самому.
\rsbpar\vs Rom 16:3 Приветствуйте Прискиллу и Акилу, сотрудников моих во Христе Иисусе
\vs Rom 16:4 (которые голову свою полагали за мою душу, которых не я один благодарю, но и все церкви из язычников), и домашнюю их церковь.
\vs Rom 16:5 Приветствуйте возлюбленного моего Епенета, который есть начаток Ахаии для Христа.
\vs Rom 16:6 Приветствуйте Мариам, которая много трудилась для нас.
\vs Rom 16:7 Приветствуйте Андроника и Юнию, сродников моих и узников со мною, прославившихся между Апостолами и прежде меня еще уверовавших во Христа.
\vs Rom 16:8 Приветствуйте Амплия, возлюбленного мне в Господе.
\vs Rom 16:9 Приветствуйте Урбана, сотрудника нашего во Христе, и Стахия, возлюбленного мне.
\vs Rom 16:10 Приветствуйте Апеллеса, испытанного во Христе. Приветствуйте \bibemph{верных} из дома Аристовулова.
\vs Rom 16:11 Приветствуйте Иродиона, сродника моего. Приветствуйте из домашних Наркисса тех, которые в Господе.
\vs Rom 16:12 Приветствуйте Трифену и Трифосу, трудящихся о Господе. Приветствуйте Персиду возлюбленную, которая много потрудилась о Господе.
\vs Rom 16:13 Приветствуйте Руфа, избранного в Господе, и матерь его и мою.
\vs Rom 16:14 Приветствуйте Асинкрита, Флегонта, Ерма, Патрова, Ермия и других с ними братьев.
\vs Rom 16:15 Приветствуйте Филолога и Юлию, Нирея и сестру его, и Олимпана, и всех с ними святых.
\vs Rom 16:16 Приветствуйте друг друга с целованием святым. Приветствуют вас все церкви Христовы.
\rsbpar\vs Rom 16:17 Умоляю вас, братия, остерегайтесь производящих разделения и соблазны, вопреки учению, которому вы научились, и уклоняйтесь от них;
\vs Rom 16:18 ибо такие \bibemph{люди} служат не Господу нашему Иисусу Христу, а своему чреву, и ласкательством и красноречием обольщают сердца простодушных.
\vs Rom 16:19 Ваша покорность \bibemph{вере} всем известна; посему я радуюсь за вас, но желаю, чтобы вы были мудры на добро и просты на зло.
\vs Rom 16:20 Бог же мира сокрушит сатану под ногами вашими вскоре. Благодать Господа нашего Иисуса Христа с вами! Аминь.
\rsbpar\vs Rom 16:21 Приветствуют вас Тимофей, сотрудник мой, и Луций, Иасон и Сосипатр, сродники мои.
\vs Rom 16:22 Приветствую вас в Господе и я, Тертий, писавший сие послание.
\vs Rom 16:23 Приветствует вас Гаий, странноприимец мой и всей церкви. Приветствует вас Ераст, городской казнохранитель, и брат Кварт.
\rsbpar\vs Rom 16:24 Благодать Господа нашего Иисуса Христа со всеми вами. Аминь.

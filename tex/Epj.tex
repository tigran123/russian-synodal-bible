\bibbookdescr{Epj}{
  inline={\LARGE Послание\\\Huge Иеремии\fns{Переведена с греческого.}},
  toc={Послание Иеремии*},
  bookmark={Послание Иеремии},
  header={Послание Иеремии},
  %headerleft={},
  %headerright={},
  abbr={Посл~Иер}
}
\vs Epj 1:1 Список послания, которое послал Иеремия к пленникам, отводимым в Вавилон царем Вавилонским, чтобы возвестить им, чт\acc{о} повелено ему Богом.
\rsbpar\vs Epj 1:2 За грехи, которыми вы согрешили пред Богом, будете отведены пленниками в Вавилон Навуходоносором, царем Вавилонским.
\vs Epj 1:3 Войдя в Вавилон, вы пробудете там многие годы и долгое время, даже до семи родов; после же сего Я выведу вас оттуда с миром.
\vs Epj 1:4 Теперь вы увидите в Вавилоне богов серебряных и золотых и деревянных, носимых на плечах, внушающих страх язычникам.
\vs Epj 1:5 Берегитесь же, чтобы и вам не сделаться подобными иноплеменникам, и чтобы страх пред ними не овладел и вами. Видя толпу спереди и сзади их поклоняющеюся перед ними, скажите в уме: <<Тебе должно поклоняться, Владыко!>>
\vs Epj 1:6 Ибо Ангел Мой с вами, и он защитник душ ваших.
\vs Epj 1:7 Язык их выстроган художником, и сами они оправлены в золото и серебро; но они ложные, и не могут говорить.
\vs Epj 1:8 И как бы для девицы, любящей украшение, берут они золото, и приготовляют венцы на головы богов своих.
\vs Epj 1:9 Бывает также, что жрецы похищают у богов своих золото и серебро и употребляют его на себя самих;
\vs Epj 1:10 уделяют из того и блудницам под их кровом; украшают богов золотых и серебряных и деревянных одеждами, как людей.
\vs Epj 1:11 Но они не спасаются от ржавчины и моли, хотя облечены в пурпуровую одежду.
\vs Epj 1:12 Обтирают лице их от пыли в капище, которой на них очень много.
\vs Epj 1:13 Имеет и скипетр, как человек~--- судья страны, но он не может умертвить виновного пред ним.
\vs Epj 1:14 Имеет меч в правой руке и секиру, а себя самого от войска и разбойников не защитит: отсюда познается, что они не боги; итак, не бойтесь их.
\vs Epj 1:15 Ибо, как разбитый сосуд делается бесполезным для человека, так и боги их.
\vs Epj 1:16 После того, как они поставлены в капищах, глаза их полны пыли от ног входящих.
\vs Epj 1:17 И как у нанесшего оскорбление царю заграждаются входы в жилье, когда он отводится на смерть, \bibemph{так} капища их охраняют жрецы их дверями и замками и засовами, чтобы они не были ограблены разбойниками;
\vs Epj 1:18 зажигают для них светильники, и больше, нежели для себя самих, а они ни одного из них не могут видеть.
\vs Epj 1:19 Они как бревно в доме; сердца их, говорят, точат черви земляные, и съедают их самих и одежду их,~--- а они не чувствуют.
\vs Epj 1:20 Лица их черны от курения в капищах.
\vs Epj 1:21 На тело их и на головы их налетают летучие мыши и ласточки и другие птицы, \bibemph{лазают} также по ним и кошки.
\vs Epj 1:22 Из этого уразумеете, что это не боги; итак, не бойтесь их.
\rsbpar\vs Epj 1:23 Если кто не очистит от ржавчины золота, которым они обложены для красы, то они не будут блестеть; и когда выливали их, они не чувствовали.
\vs Epj 1:24 За большую цену они куплены, а духа нет в них.
\vs Epj 1:25 Безногие, они носятся на плечах, показывая чрез то свою ничтожность людям; посрамляются же и служащие им;
\vs Epj 1:26 потому что, в случае падения их на землю, сами собою они не могут встать; также, если бы кто поставил их прямо, не могут сами собою двигаться и, если бы кто наклонил их, не могут выпрямиться; но как перед мертвыми полагают перед ними дары.
\vs Epj 1:27 Жертвы их жрецы продают и злоупотребляют ими; равно и жены их часть из них солят, и ничего не уделяют ни нищему, ни больному.
\vs Epj 1:28 К жертвам их прикасаются женщины нечистые и родильницы. Итак, познав из сего, что они не боги, не бойтесь их.
\vs Epj 1:29 Как же назвать их богами? женщины приносят жертвы этим серебряным и золотым и деревянным богам.
\vs Epj 1:30 И в капищах их сидят жрецы в разодранных одеждах, с обритыми головами и бородами и с непокрытыми головами:
\vs Epj 1:31 ревут они с воплем пред своими богами, как иные на поминках по умершим.
\vs Epj 1:32 Некоторые из одежд их жрецы берут себе и одевают ими своих жен и детей.
\vs Epj 1:33 Если испытывают от кого-либо злое или доброе, не могут воздать; не могут поставить царя, ни низложить его.
\vs Epj 1:34 Равно ни богатства, ни даже мелкой медной монеты они не могут дать. Если кто, обещав им обет, не исполнил бы его, не взыщут.
\vs Epj 1:35 От смерти человека не избавят, ни слабейшего у сильного не отнимут;
\vs Epj 1:36 человеку слепому не возвратят зрения; человеку в нужде не помогут;
\vs Epj 1:37 вдове не окажут сострадания, и сироте не сделают добра.
\vs Epj 1:38 Камням из гор подобны \bibemph{эти боги} деревянные и оправленные в золото и серебро,~--- и служащие им посрамятся.
\vs Epj 1:39 Как же можно подумать или сказать, что они боги?
\vs Epj 1:40 К тому же сами Халдеи обращаются с ними непочтительно: они, когда увидят немого, не могущего говорить, приносят его к Ваалу и требуют, чтобы он говорил, как будто он может чувствовать.
\vs Epj 1:41 И не могут они, заметив это, оставить их, потому что не имеют смысла.
\vs Epj 1:42 Женщины, обвязавшись тростниковым поясом, сидят на улицах, сожигая курение из оливковых зерен.
\vs Epj 1:43 И когда какая-либо из них, увлеченная проходящим, переспит с ним,~--- попрекает своей подруге, что та не удостоена того же, как она, и что перевязь ее не разорвана.
\vs Epj 1:44 Все, совершающееся у них, ложно. Посему как можно думать или говорить, что они боги?
\vs Epj 1:45 Устроены они художниками и плавильщиками золота; не чем иным они не делаются, как тем, чем желали их сделать художники.
\vs Epj 1:46 И те, которые приготовляют их, не бывают долговечны;
\vs Epj 1:47 как же сделанные ими могут быть богами? Они оставили по себе ложь и срам своим потомкам.
\vs Epj 1:48 Когда постигают их война и бедствия, жрецы совещаются между собою, где бы им скрыться с ними.
\vs Epj 1:49 Как же не понять, что те не боги, которые самих себя не спасают ни от войн, ни от бедствий?
\vs Epj 1:50 Так как они деревянные и оправленные в золото и серебро, то можно познать, что они ложь; всем народам и царям сделается ясным, что это не боги, а дела рук человеческих, и в них нет никакого действия божественного.
\vs Epj 1:51 Кому же после сего не понятно, что они не боги?
\vs Epj 1:52 Царя стране они не поставят, дождя людям не дадут;
\vs Epj 1:53 суда не рассудят, обидимого не защитят, будучи бессильны,
\vs Epj 1:54 как вор\acc{о}ны, находящиеся между небом и землею. Ибо и в том случае, когда подверглось бы пожару капище богов деревянных или оправленных в золото и серебро, жрецы их убегут и спасутся,~--- а они сами, как бревна в средине, сгорят.
\vs Epj 1:55 Ни царю, ни врагам они не могут противостать. Как же можно принять или подумать, что они боги?
\vs Epj 1:56 Ни от воров, ни от грабителей не могут охранить самих себя эти боги, деревянные и оправленные в серебро и золото:
\vs Epj 1:57 превосходя их силою, они снимают золото и серебро и одежды, которые на них, и уходят с добычею, а эти себе самим не в силах помочь.
\vs Epj 1:58 Поэтому лучше царь, выказывающий мужество, или полезный в доме сосуд, который употребляет хозяин, нежели ложные боги; или \bibemph{лучше} дверь в доме, охраняющая в нем имущество, нежели ложные боги; или \bibemph{лучше} деревянный столп в царском дворце, нежели ложные боги.
\vs Epj 1:59 Солнце и луна и звезды, будучи светлы и посылаемы ради потребности, благопослушны.
\vs Epj 1:60 Также и молния каждый раз, как является, ясно видима; также ветер во всякой стране веет.
\vs Epj 1:61 И облака, когда повелит им Бог пройти над всею вселенною, исполняют повеление.
\vs Epj 1:62 Тоже огонь, свыше ниспосылаемый для истребления гор и лесов, делает, что назначено; а эти не подобны им ни видом, ни силами.
\vs Epj 1:63 Почему же можно подумать или сказать, что они боги, когда они несильны ни суда рассудить, ни добра делать людям?
\vs Epj 1:64 Итак, зная, что они не боги, не бойтесь их.
\vs Epj 1:65 Царей они ни проклянут, ни благословят;
\vs Epj 1:66 знамений не покажут на небе и пред народами; не осветят, как солнце, и не осияют, как луна.
\vs Epj 1:67 Звери лучше их: они, убегая под кров, могут помочь себе.
\vs Epj 1:68 Итак, ни из чего не видно нам, что они боги; посему не бойтесь их.
\vs Epj 1:69 Как пугало в огороде ничего не сбережет, так и их деревянные, оправленные в золото и серебро боги.
\vs Epj 1:70 Равным образом их деревянные, оправленные в золото и серебро боги подобны терновому кусту в саду, на который садятся всякие птицы, также и трупу, брошенному во тьме.
\vs Epj 1:71 Из пурпура и червленицы, которые истлевают на них, вы можете уразуметь, что они не боги; да и сами они будут наконец съедены и будут позором в стране.
\rsbpar\vs Epj 1:72 Итак, лучше человек праведный, не имеющий идолов, ибо он~--- далеко от позора.

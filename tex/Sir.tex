\bibbookdescr{Sir}{
  inline={\LARGE Книга\\\Huge Премудрости Иисуса,\\сына Сирахова\fns{Переведена с греческого.}},
  toc={Сирах*},
  bookmark={Сирах},
  header={Сирах},
  %headerleft={},
  %headerright={},
  abbr={Сир}
}
\chhdr{Предисловие\fns{Предисловие к греческому переводу, имеющееся у 70-ти и содержащееся в Славянской Библии.}}
\vs Sir 0:0 Многое и великое дано нам через закон, пророков и прочих \bibemph{писателей}, следовавших за ними, за что должно прославлять \bibemph{народ} Израильский за образованность и мудрость; и не только сами изучающие должны делаться разумными, но и находящимся вне [Палестины] усердно занимающиеся [писанием] могут приносить пользу словом и писанием. Поэтому дед мой Иисус, больше других предаваясь изучению закона, пророков и других отеческих книг и приобретя достаточный в них навык, решился и сам написать нечто, относящееся к образованию и мудрости, чтобы любители учения, вникая и в эту [книгу], еще более преуспевали в жизни по закону. Итак, прошу вас, читайте [эту книгу] благосклонно и внимательно и имейте снисхождение к тому, что в некоторых местах мы, может быть, погрешили, трудясь над переводом: ибо неодинаковый смысл имеет то, что читается по-еврейски, когда переведено будет на другой язык,~--- и не только эта [книга], но даже закон, пророчества и остальные книги имеют немалую разницу в смысле, если читать их в подлиннике. Прибыв в Египет в тридцать восьмом году при царе Евергете [Птоломее] и пробыв там, я нашел немалую разницу в образовании [между палестинскими и египетскими евреями], и счел крайне необходимым и самому приложить усердие к тому, чтобы перевести эту книгу. Много бессонного труда и знаний положил я в это время, чтобы довести книгу до конца и сделать ее доступною и тем, которые, находясь на чужбине, желают учиться и приспособляют свои нравы к тому, чтобы жить по закону.
\vs Sir 1:1 Всякая премудрость~--- от Господа и с Ним пребывает вовек.
\vs Sir 1:2 Песок морей и капли дождя и дни вечности кто исчислит?
\vs Sir 1:3 Высоту неба и широту земли, и бездну и премудрость кто исследует?
\vs Sir 1:4 Прежде всего произошла Премудрость, и разумение мудрости~--- от века.
\vs Sir 1:5 Источник премудрости~--- слово Бога Всевышнего, и шествие ее~--- вечные заповеди.
\vs Sir 1:6 Кому открыт корень премудрости? и кто познал искусство ее?
\vs Sir 1:7 Один есть премудрый, весьма страшный, сидящий на престоле Своем, Господь.
\vs Sir 1:8 Он произвел ее и видел и измерил ее
\vs Sir 1:9 и излил ее на все дела Свои
\vs Sir 1:10 и на всякую плоть по дару Своему, и особенно наделил ею любящих Его.
\vs Sir 1:11 Страх Господень~--- слава и честь, и веселие и венец радости.
\vs Sir 1:12 Страх Господень усладит сердце и даст веселие и радость и долгоденствие.
\vs Sir 1:13 Боящемуся Господа благо будет напоследок, и в день смерти своей он получит благословение. Страх Господень~--- дар от Господа и поставляет на стезях любви.
\vs Sir 1:14 Любовь к Господу~--- славная премудрость, и кому благоволит Он, разделяет ее по Своему усмотрению.
\vs Sir 1:15 Начало премудрости~--- бояться Бога, и с верными она образуется вместе во чреве. Среди людей она утвердила себе вечное основание и семени их вверится.
\vs Sir 1:16 Полнота премудрости~--- бояться Господа; она напояет их от плодов своих:
\vs Sir 1:17 весь дом их она наполнит всем, чего желают, и кладовые их~--- произведениями своими.
\vs Sir 1:18 Венец премудрости~--- страх Господень, произращающий мир и невредимое здравие; но то и другое~--- дары Бога, Который распространяет славу любящих Его.
\vs Sir 1:19 Он видел ее и измерил, пролил как дождь в\acc{е}дение и разумное знание и возвысил славу обладающих ею.
\vs Sir 1:20 Корень премудрости~--- бояться Господа, а ветви ее~--- долгоденствие.
\rsbpar\vs Sir 1:21 Страх Господень отгоняет грехи; не имеющий же страха не может оправдаться.
\vs Sir 1:22 Не может быть оправдан несправедливый гнев, ибо \bibemph{самое} движение гнева есть падение для человека.
\vs Sir 1:23 Терпеливый до времени удержится и после вознаграждается веселием.
\vs Sir 1:24 До времени он скроет слова свои, и уста верных расскажут о благоразумии его.
\vs Sir 1:25 В сокровищницах премудрости~--- притчи разума, грешнику же страх Господень ненавистен.
\vs Sir 1:26 Если желаешь премудрости, соблюдай заповеди, и Господь подаст ее тебе,
\vs Sir 1:27 ибо премудрость и знание есть страх пред Господом, и благоугождение Ему~--- вера и кротость.
\vs Sir 1:28 Не будь недоверчивым к страху пред Господом и не приступай к Нему с раздвоенным сердцем.
\vs Sir 1:29 Не лицемерь пред устами других и будь внимателен к устам твоим.
\vs Sir 1:30 Не возноси себя, чтобы не упасть и не навлечь бесчестия на душу твою, ибо Господь откроет тайны твои и уничижит тебя среди собрания за то, что ты не приступил искренно к страху Господню, и сердце твое полно лукавства.
\vs Sir 2:1 Сын мой! если ты приступаешь служить Господу Богу, то приготовь душу твою к искушению:
\vs Sir 2:2 управь сердце твое и будь тверд, и не смущайся во время посещения;
\vs Sir 2:3 прилепись к Нему и не отступай, дабы возвеличиться тебе напоследок.
\vs Sir 2:4 Все, что ни приключится тебе, принимай охотно, и в превратностях твоего уничижения будь долготерпелив,
\vs Sir 2:5 ибо золото испытывается в огне, а люди, угодные Богу,~--- в горниле уничижения.
\vs Sir 2:6 Веруй Ему, и Он защитит тебя; управь пути твои и надейся на Него.
\vs Sir 2:7 Боящиеся Господа! ожидайте милости Его и не уклоняйтесь \bibemph{от Него}, чтобы не упасть.
\vs Sir 2:8 Боящиеся Господа! веруйте Ему, и не погибнет награда ваша.
\vs Sir 2:9 Боящиеся Господа! надейтесь на благое, на радость вечную и милости.
\vs Sir 2:10 Взгляните на древние роды и посмотрите: кто верил Господу~--- и был постыжен? или кто пребывал в страхе Его~--- и был оставлен? или кто взывал к Нему, и Он презрел его?
\vs Sir 2:11 Ибо Господь сострадателен и милостив и прощает грехи, и спасает во время скорби.
\vs Sir 2:12 Горе сердцам боязливым и рукам ослабленным и грешнику, ходящему по двум стезям!
\vs Sir 2:13 Горе сердцу расслабленному! ибо оно не верует, и за то не будет защищено.
\vs Sir 2:14 Горе вам, потерявшим терпение! что будете вы делать, когда Господь посетит?
\vs Sir 2:15 Боящиеся Господа не будут недоверчивы к словам Его, и любящие Его сохранят пути Его.
\vs Sir 2:16 Боящиеся Господа будут искать благоволения Его, и любящие Его насытятся законом.
\vs Sir 2:17 Боящиеся Господа уготовят сердца свои и смирят пред Ним души свои, говоря:
\vs Sir 2:18 впадем в руки Господа, а не в руки людей; ибо, каково величие Его, такова и милость Его.
\vs Sir 3:1 Дети, послушайте меня, отца, и поступайте так, чтобы вам спастись,
\vs Sir 3:2 ибо Господь возвысил отца над детьми и утвердил суд матери над сыновьями.
\vs Sir 3:3 Почитающий отца очистится от грехов,
\vs Sir 3:4 и уважающий мать свою~--- как приобретающий сокровища.
\vs Sir 3:5 Почитающий отца будет иметь радость от детей своих и в день молитвы своей будет услышан.
\vs Sir 3:6 Уважающий отца будет долгоденствовать, и послушный Господу успокоит мать свою.
\vs Sir 3:7 Боящийся Господа почтит отца и, как владыкам, послужит родившим его.
\vs Sir 3:8 Делом и словом почитай отца твоего и мать, чтобы пришло на тебя благословение от них,
\vs Sir 3:9 ибо благословение отца утверждает домы детей, а клятва матери разрушает до основания.
\vs Sir 3:10 Не ищи славы в бесчестии отца твоего, ибо не слава тебе бесчестие отца.
\vs Sir 3:11 Слава человека~--- от чести отца его, и позор детям~--- мать в бесславии.
\vs Sir 3:12 Сын! прими отца твоего в старости \bibemph{его} и не огорчай его в жизни его.
\vs Sir 3:13 Хотя бы он и оскудел разумом, имей снисхождение и не пренебрегай им при полноте силы твоей,
\vs Sir 3:14 ибо милосердие к отцу не будет забыто; несмотря на грехи твои, благосостояние твое умножится.
\vs Sir 3:15 В день скорби твоей воспомянется о тебе: как лед от теплоты, разрешатся грехи твои.
\vs Sir 3:16 Оставляющий отца~--- то же, что богохульник, и проклят от Господа раздражающий мать свою.
\rsbpar\vs Sir 3:17 Сын мой! веди дела твои с кротостью, и будешь любим богоугодным человеком.
\vs Sir 3:18 Сколько ты велик, столько смиряйся, и найдешь благодать у Господа.
\vs Sir 3:19 Много высоких и славных, но тайны открываются смиренным,
\vs Sir 3:20 ибо велико могущество Господа, и Он смиренными прославляется.
\vs Sir 3:21 Чрез меру трудного для тебя не ищи, и, что свыше сил твоих, того не испытывай.
\vs Sir 3:22 Что заповедано тебе, о том размышляй; ибо не нужно тебе, что сокрыто.
\vs Sir 3:23 При многих занятиях твоих, о лишнем не заботься: тебе открыто очень много из человеческого знания;
\vs Sir 3:24 ибо многих ввели в заблуждение их предположения, и лукавые мечты поколебали ум их.
\vs Sir 3:25 Кто любит опасность, тот впадет в нее;
\vs Sir 3:26 упорное сердце напоследок потерпит зло:
\vs Sir 3:27 упорное сердце будет обременено скорбями, и грешник приложит грехи ко грехам.
\vs Sir 3:28 Испытания не служат врачевством для гордого, потому что злое растение укоренилось в нем.
\vs Sir 3:29 Сердце разумного обдумает притчу, и внимательное ухо есть желание мудрого.
\vs Sir 3:30 Вода угасит пламень огня, и милостыня очистит грехи.
\vs Sir 3:31 Кто воздает за благодеяния, тот помышляет о будущем и во время падения найдет опору.
\vs Sir 4:1 Сын мой! не отказывай в пропитании нищему и не утомляй ожиданием очей нуждающихся;
\vs Sir 4:2 не опечаль души алчущей и не огорчай человека в его скудости;
\vs Sir 4:3 не смущай сердца уже огорченного и не откладывай подавать нуждающемуся;
\vs Sir 4:4 не отказывай угнетенному, умоляющему о помощи, и не отвращай лица твоего от нищего;
\vs Sir 4:5 не отвращай очей от просящего и не давай человеку повода проклинать тебя;
\vs Sir 4:6 ибо, когда он в горести души своей будет проклинать тебя, Сотворивший его услышит моление его.
\vs Sir 4:7 В собрании старайся быть приятным и пред высшим наклоняй твою голову;
\vs Sir 4:8 приклоняй ухо твое к нищему и отвечай ему ласково, с кротостью;
\vs Sir 4:9 спасай обижаемого от руки обижающего и не будь малодушен, когда судишь;
\vs Sir 4:10 сиротам будь как отец и матери их~--- вместо мужа:
\vs Sir 4:11 и будешь как сын Вышнего, и Он возлюбит тебя более, нежели мать твоя.
\rsbpar\vs Sir 4:12 Премудрость возвышает сынов своих и поддерживает ищущих ее:
\vs Sir 4:13 любящий ее любит жизнь, и ищущие ее с раннего утра исполнятся радости:
\vs Sir 4:14 обладающий ею наследует славу, и, куда бы ни пошел, Господь благословит его;
\vs Sir 4:15 служащие ей служат Святому, и любящих ее любит Господь;
\vs Sir 4:16 послушный ей будет судить народы, и внимающий ей будет жить надежно;
\vs Sir 4:17 кто вверится ей, тот наследует ее, и потомки его будут обладать ею:
\vs Sir 4:18 ибо сначала она пойдет с ним путями извилистыми, наведет на него страх и боязнь
\vs Sir 4:19 и будет мучить его своим водительством, доколе не уверится в душе его и не искусит его своими уставами;
\vs Sir 4:20 но потом она выйдет к нему на прямом пути и обрадует его
\vs Sir 4:21 и откроет ему тайны свои.
\vs Sir 4:22 Если он совратится с пути, она оставляет его и отдает его в руки падения его.
\rsbpar\vs Sir 4:23 Наблюдай время и храни себя от зла~---
\vs Sir 4:24 и не постыдишься за душу твою:
\vs Sir 4:25 есть стыд, ведущий ко греху, и есть стыд~--- слава и благодать.
\vs Sir 4:26 Не будь лицеприятен против души твоей и не стыдись ко вреду твоему.
\vs Sir 4:27 Не удерживай слова, когда оно может помочь:
\vs Sir 4:28 ибо в слове познается мудрость и в речи языка~--- знание.
\vs Sir 4:29 Не противоречь истине и стыдись твоего невежества.
\vs Sir 4:30 Не стыдись исповедовать грехи твои и не удерживай течения реки.
\vs Sir 4:31 Не подчиняйся человеку глупому и не смотри на сильного.
\vs Sir 4:32 Подвизайся за истину до смерти, и Господь Бог поборет за тебя.
\vs Sir 4:33 Не будь скор языком твоим и ленив и нерадив в делах твоих.
\vs Sir 4:34 Не будь, как лев, в доме твоем и подозрителен к домочадцам твоим.
\vs Sir 4:35 Да не будет рука твоя распростертою к принятию и сжатою при отдании.
\vs Sir 5:1 Не полагайся на имущества твои и не говори: <<станет на жизнь мою>>.
\vs Sir 5:2 Не следуй влечению души твоей и крепости твоей, чтобы ходить в похотях сердца твоего,
\vs Sir 5:3 и не говори: <<кто властен в делах моих?>>, ибо Господь непременно отмстит за дерзость твою.
\vs Sir 5:4 Не говори: <<я грешил, и что мне было?>>, ибо Господь долготерпелив.
\vs Sir 5:5 При мысли об умилостивлении не будь бесстрашен, чтобы прилагать грех ко грехам
\vs Sir 5:6 и не говори: <<милосердие Его велико, Он простит множество грехов моих>>;
\vs Sir 5:7 ибо милосердие и гнев у Него, и на грешниках пребывает ярость Его.
\vs Sir 5:8 Не медли обратиться к Господу и не откладывай со дня на день:
\vs Sir 5:9 ибо внезапно найдет гнев Господа, и ты погибнешь во время отмщения.
\vs Sir 5:10 Не полагайся на имущества неправедные, ибо они не принесут тебе пользы в день посещения.
\vs Sir 5:11 Не вей при всяком ветре и не ходи всякою стезею: таков двоязычный грешник.
\vs Sir 5:12 Будь тверд в твоем убеждении, и одно да будет твое слово.
\vs Sir 5:13 Будь скор к слушанию, и обдуманно давай ответ.
\vs Sir 5:14 Если имеешь знание, то отвечай ближнему, а если нет, то рука твоя да будет на устах твоих.
\vs Sir 5:15 В речах~--- слава и бесчестие, и язык человека бывает падением ему.
\vs Sir 5:16 Не прослыви наушником, и не коварствуй языком твоим:
\vs Sir 5:17 ибо на воре~--- стыд, и на двоязычном~--- злое порицание.
\vs Sir 5:18 Не будь неразумным ни в большом ни в малом.
\vs Sir 6:1 И не делайся врагом из друга, ибо худое имя получает в удел стыд и позор; так~--- и грешник двоязычный.
\vs Sir 6:2 Не возноси себя в помыслах души твоей, чтобы душа твоя не была растерзана, как вол:
\vs Sir 6:3 листья твои ты истребишь и плоды твои погубишь, и останешься, как сухое дерево.
\vs Sir 6:4 Душа лукавая погубит своего обладателя и сделает его посмешищем врагов.
\rsbpar\vs Sir 6:5 Сладкие уста умножат друзей, и доброречивый язык умножит приязнь.
\vs Sir 6:6 Живущих с тобою в мире да будет много, а советником твоим~--- один из тысячи.
\vs Sir 6:7 Если хочешь приобрести друга, приобретай его по испытании и не скоро вверяйся ему.
\vs Sir 6:8 Бывает друг в нужное для него время, и не останется с тобой в день скорби твоей;
\vs Sir 6:9 и бывает друг, который превращается во врага и откроет ссору к поношению твоему.
\vs Sir 6:10 Бывает другом участник в трапезе, и не останется с тобою в день скорби твоей.
\vs Sir 6:11 В имении твоем он будет как ты, и дерзко будет обращаться с домочадцами твоими;
\vs Sir 6:12 но если ты будешь унижен, он будет против тебя и скроется от лица твоего.
\vs Sir 6:13 Отдаляйся от врагов твоих и будь осмотрителен с друзьями твоими.
\vs Sir 6:14 Верный друг~--- крепкая защита: кто нашел его, нашел сокровище.
\vs Sir 6:15 Верному другу нет цены, и нет меры доброте его.
\vs Sir 6:16 Верный друг~--- врачевство для жизни, и боящиеся Господа найдут его.
\vs Sir 6:17 Боящийся Господа направляет дружбу свою так, что, каков он сам, таким делается и друг его.
\rsbpar\vs Sir 6:18 Сын мой! от юности твоей предайся учению, и до седин твоих найдешь мудрость.
\vs Sir 6:19 Приступай к ней как пашущий и сеющий и ожидай добрых плодов ее:
\vs Sir 6:20 ибо малое время потрудишься в возделывании ее, и скоро будешь есть плоды ее.
\vs Sir 6:21 Для невежд она очень сурова, и неразумный не останется с нею:
\vs Sir 6:22 она будет на нем как тяжелый камень испытания, и он не замедлит сбросить ее.
\vs Sir 6:23 Премудрость соответствует имени своему, и немногим открывается.
\vs Sir 6:24 Послушай, сын мой, и прими мнение мое, и не отвергни совета моего.
\vs Sir 6:25 Наложи на ноги твои путы ее и на шею твою цепь ее.
\vs Sir 6:26 Подставь ей плечо твое, и носи ее и не тяготись узами ее.
\vs Sir 6:27 Приблизься к ней всею душею твоею, и всею силою твоею соблюдай пути ее.
\vs Sir 6:28 Исследуй и ищи, и она будет познана тобою и, сделавшись обладателем ее, не покидай ее;
\vs Sir 6:29 ибо наконец ты найдешь в ней успокоение, и она обратится в радость тебе.
\vs Sir 6:30 Путы ее будут тебе крепкою защитою, и цепи ее~--- славным одеянием;
\vs Sir 6:31 ибо на ней украшение золотое, и узы ее~--- гиацинтовые нити.
\vs Sir 6:32 Как одеждою славы ты облечешься ею, и возложишь ее на себя как венец радости.
\vs Sir 6:33 Сын мой! если ты пожелаешь ее, то научишься, и если предашься ей душею твоею, то будешь ко всему способен.
\vs Sir 6:34 Если с любовью будешь слушать \bibemph{ее}, то поймешь ее, и если приклонишь ухо твое, то будешь мудр.
\vs Sir 6:35 Бывай в собрании старцев, и кто мудр, прилепись к тому; люби слушать всякую священную повесть, и притчи разумные да не ускользают от тебя.
\vs Sir 6:36 Если увидишь разумного, ходи к нему с раннего утра, и пусть нога твоя истирает пороги дверей его.
\vs Sir 6:37 Размышляй о повелениях Господа и всегда поучайся в заповедях Его: Он укрепит твое сердце, и желание премудрости дастся тебе.
\vs Sir 7:1 Не делай зла, и тебя не постигнет зло;
\vs Sir 7:2 удаляйся от неправды, и она уклонится от тебя.
\vs Sir 7:3 Сын мой! не сей на бороздах неправды, и не будешь в семь раз более пожинать с них.
\vs Sir 7:4 Не проси у Господа власти, и у царя~--- почетного места.
\vs Sir 7:5 Не оправдывай себя пред Господом, и не мудрствуй пред царем.
\vs Sir 7:6 Не домогайся сделаться судьею, чтобы не оказаться тебе бессильным сокрушить неправду, чтобы не убояться когда-либо лица сильного и не положить тени на правоту твою.
\vs Sir 7:7 Не греши против городского общества, и не роняй себя пред народом.
\vs Sir 7:8 Не прилагай греха ко греху, ибо и за один не останешься ненаказанным.
\vs Sir 7:9 Не говори: <<Он призрит на множество даров моих, и, когда я принесу их Богу Вышнему, Он примет>>.
\vs Sir 7:10 Не малодушествуй в молитве твоей и не пренебрегай подавать милостыню.
\vs Sir 7:11 Не насмехайся над человеком, находящимся в горести души его; ибо есть Смиряющий и Возвышающий.
\vs Sir 7:12 Не выдумывай лжи на брата твоего, и не делай того же против друга.
\vs Sir 7:13 Не желай говорить какую бы то ни было ложь; ибо повторение ее не послужит ко благу.
\vs Sir 7:14 Пред собранием старших не многословь, и не повторяй слова в прошении твоем.
\vs Sir 7:15 Не отвращайся от трудной работы и от земледелия, которое учреждено от Вышнего.
\vs Sir 7:16 Не прилагайся ко множеству грешников.
\vs Sir 7:17 Глубоко смири душу твою.
\vs Sir 7:18 Помни, что гнев не замедлит,
\vs Sir 7:19 что наказание нечестивому~--- огонь и червь.
\vs Sir 7:20 Не меняй друга на сокровище, и брата однокровного~--- на золото Офирское.
\vs Sir 7:21 Не оставляй умной и доброй жены, ибо достоинство ее драгоценнее золота.
\vs Sir 7:22 Не обижай раба, трудящегося усердно, ни наемника, преданного тебе душею.
\vs Sir 7:23 Разумного раба да любит душа твоя, и не откажи ему в свободе.
\vs Sir 7:24 Есть у тебя скот? наблюдай за ним, и если он полезен тебе, пусть остается у тебя.
\vs Sir 7:25 Есть у тебя сыновья? учи их и с юности нагибай шею их.
\vs Sir 7:26 Есть у тебя дочери? имей попечение о теле их и не показывай им веселого лица твоего.
\vs Sir 7:27 Выдай дочь в замужество, и сделаешь великое дело, и подари ее мужу разумному.
\vs Sir 7:28 Есть у тебя жена по душе? не отгоняй ее.
\vs Sir 7:29 Всем сердцем почитай отца твоего и не забывай родильных болезней матери твоей.
\vs Sir 7:30 Помни, что ты рожден от них: и что можешь ты воздать им, как они тебе?
\vs Sir 7:31 Всею душею твоею благоговей пред Господом и уважай священников Его.
\vs Sir 7:32 Всею силою люби Творца твоего, и не оставляй служителей Его.
\vs Sir 7:33 Бойся Господа, и почитай священника, и давай ему часть, как заповедано тебе:
\vs Sir 7:34 начатки, и за грех, и даяние плеч, и жертву освящения, и начатки святых.
\vs Sir 7:35 И к бедному простирай руку твою, дабы благословение твое было совершенно.
\vs Sir 7:36 Милость даяния да будет ко всякому живущему, но и умершего не лишай милости.
\vs Sir 7:37 Не устраняйся от плачущих, и с сетующими сетуй.
\vs Sir 7:38 Не ленись посещать больного, ибо за это ты будешь возлюблен.
\vs Sir 7:39 Во всех делах твоих помни о конце твоем, и вовек не согрешишь.
\vs Sir 8:1 Не ссорься с человеком сильным, чтобы когда-нибудь не впасть в его руки.
\vs Sir 8:2 Не заводи тяжбы с человеком богатым, чтобы он не имел перевеса над тобою;
\vs Sir 8:3 ибо золото многих погубило, и склоняло сердца царей.
\vs Sir 8:4 Не спорь с человеком, дерзким на язык, и не подкладывай дров на огонь его.
\vs Sir 8:5 Не шути с невеждою, чтобы не подверглись бесчестию твои предки.
\vs Sir 8:6 Не укоряй человека, обращающегося от греха: помни, что все мы находимся под эпитимиями.
\vs Sir 8:7 Не пренебрегай человека в старости его, ибо и мы стареем.
\vs Sir 8:8 Не радуйся смерти человека, хотя бы он был самый враждебный тебе: помни, что все мы умрем.
\vs Sir 8:9 Не пренебрегай повестью мудрых и упражняйся в притчах их;
\vs Sir 8:10 ибо от них научишься в\acc{е}дению и~--- как служить сильным.
\vs Sir 8:11 Не удаляйся от повести старцев, ибо и они научились от отцов своих,
\vs Sir 8:12 и ты научишься от них рассудительности и~--- какой в случае надобности дать ответ.
\vs Sir 8:13 Не разжигай углей грешника, чтобы не сгореть от пламени огня его,
\vs Sir 8:14 и не восставай против наглеца, чтобы он не засел засадою в устах твоих.
\vs Sir 8:15 Не давай взаймы человеку, который сильнее тебя; а если дашь, то считай себя потерявшим.
\vs Sir 8:16 Не поручайся сверх силы твоей; а если поручишься, заботься, как обязанный заплатить.
\vs Sir 8:17 Не судись с судьею, потому что его будут судить по его почету.
\vs Sir 8:18 С отважным не пускайся в путь, чтобы он не был тебе в тягость; ибо он будет поступать по своему произволу, и ты можешь погибнуть от его безрассудства.
\vs Sir 8:19 Не заводи ссоры со вспыльчивым и не проходи с ним чрез пустыню; потому что кровь~--- как ничто в глазах его, и где нет помощи, он поразит тебя.
\vs Sir 8:20 Не советуйся с глупым, ибо он не может умолчать о деле.
\vs Sir 8:21 При чужом не делай тайного, ибо не знаешь, что он сделает.
\vs Sir 8:22 Не открывай всякому человеку твоего сердца, чтобы он дурно не отблагодарил тебя.
\vs Sir 9:1 Не будь ревнив к жене сердца твоего и не подавай ей дурного урока против тебя самого.
\vs Sir 9:2 Не отдавай жене души твоей, чтобы она не восстала против власти твоей.
\vs Sir 9:3 Не выходи навстречу развратной женщине, чтобы как-нибудь не попасть в сети ее.
\vs Sir 9:4 Не оставайся долго с певицею, чтобы не плениться тебе искусством ее.
\vs Sir 9:5 Не засматривайся на девицу, чтобы не соблазниться прелестями ее.
\vs Sir 9:6 Не отдавай души твоей блудницам, чтобы не погубить наследства твоего.
\vs Sir 9:7 Не смотри по сторонам на улицах города и не броди по пустым местам его.
\vs Sir 9:8 Отвращай око твое от женщины благообразной и не засматривайся на чужую красоту:
\vs Sir 9:9 многие совратились с пути чрез красоту женскую; от нее, как огонь, загорается любовь.
\vs Sir 9:10 Отнюдь не сиди с женою замужнею и не оставайся с нею на пиру за вином,
\vs Sir 9:11 чтобы не склонилась к ней душа твоя и чтобы ты не поползнулся духом в погибель.
\vs Sir 9:12 Не оставляй старого друга, ибо новый не может сравниться с ним;
\vs Sir 9:13 друг новый~--- то же, что вино новое: когда оно сделается старым, с удовольствием будешь пить его.
\rsbpar\vs Sir 9:14 Не завидуй славе грешника, ибо не знаешь, какой будет конец его.
\vs Sir 9:15 Не одобряй того, что одобряют нечестивые: помни, что они до \bibemph{самого} ада не исправятся.
\vs Sir 9:16 Держи себя дальше от человека, имеющего власть умерщвлять, и ты не будешь смущаться страхом смерти;
\vs Sir 9:17 а если сближаешься с ним, не ошибись, чтобы он не лишил тебя жизни:
\vs Sir 9:18 знай, что ты посреди сетей идешь и по зубцам городских стен проходишь.
\vs Sir 9:19 По силе твоей узнавай ближних и советуйся с мудрыми.
\vs Sir 9:20 Рассуждение твое да будет с разумными, и всякая беседа твоя~--- в законе Вышнего.
\vs Sir 9:21 Да вечеряют с тобою мужи праведные, и слава твоя да будет в страхе Господнем.
\vs Sir 9:22 Изделие хвалится по руке художника, а правитель народа считается мудрым по словам его.
\vs Sir 9:23 Боятся в городе дерзкого на язык, и ненавидят опрометчивого в словах.
\vs Sir 10:1 Мудрый правитель научит народ свой, и правление разумного будет благоустроено.
\vs Sir 10:2 Каков правитель народа, таковы и служащие при нем; и каков начальствующий над городом, таковы и все живущие в нем.
\vs Sir 10:3 Царь ненаученный погубит народ свой, а при благоразумии сильных устроится город.
\vs Sir 10:4 В руке Господа власть над землею, и \bibemph{человека} потребного Он вовремя воздвигнет на ней.
\vs Sir 10:5 В руке Господа благоуспешность человека, и на лице книжника Он отпечатлеет славу Свою.
\vs Sir 10:6 Не гневайся за всякое оскорбление на ближнего, и никого не оскорбляй делом.
\vs Sir 10:7 Гордость ненавистна и Господу и людям и преступна против обоих.
\vs Sir 10:8 Владычество переходит от народа к народу по причине несправедливости, обид и любостяжания.
\vs Sir 10:9 Что гордится земля и пепел?
\vs Sir 10:10 И при жизни извергаются внутренности его.
\vs Sir 10:11 Продолжительною болезнью врач пренебрегает:
\vs Sir 10:12 и вот, ныне царь, а завтра умирает.
\vs Sir 10:13 Когда же человек умрет, то наследием его становятся пресмыкающиеся, звери и черви.
\vs Sir 10:14 Начало гордости~--- удаление человека от Господа и отступление сердца его от Творца его;
\vs Sir 10:15 ибо начало греха~--- гордость, и обладаемый ею изрыгает мерзость;
\vs Sir 10:16 и за это Господь посылает на него страшные наказания и вконец низлагает его.
\vs Sir 10:17 Господь низвергает престолы властителей и посаждает кротких на место их.
\vs Sir 10:18 Господь вырывает с корнем народы и насаждает, вместо них, смиренных.
\vs Sir 10:19 Господь опустошает страны народов и разрушает их до оснований земли.
\vs Sir 10:20 Он иссушает их, и погубляет \bibemph{людей} и истребляет от земли память их.
\vs Sir 10:21 Гордость не сотворена для людей, ни ярость гнева~--- для рождающихся от жен.
\rsbpar\vs Sir 10:22 Семя почтенное какое?~--- Семя человеческое. Семя почтенное какое?~--- Боящиеся Господа.
\vs Sir 10:23 Семя бесчестное какое?~--- Семя человеческое. Семя бесчестное какое?~--- Преступающие заповеди.
\vs Sir 10:24 Старший между братьями~--- в почтении у них, так и боящиеся Господа~--- в очах Его.
\vs Sir 10:25 Богат ли кто и славен, или беден, похвала их~--- страх Господень.
\vs Sir 10:26 Несправедливо~--- бесчестить разумного бедного, и не должно прославлять мужа грешного.
\vs Sir 10:27 Почтенны вельможа, судья и властелин, но нет из них больше боящегося Господа.
\vs Sir 10:28 Рабу мудрому будут служить свободные, и разумный человек, будучи наставляем им, не будет роптать.
\vs Sir 10:29 Не умничай много, чтобы делать дело твое, и не хвались во время нужды.
\vs Sir 10:30 Лучше тот, кто трудится и имеет во всем достаток, нежели кто праздно ходит и хвалится, но нуждается в хлебе.
\vs Sir 10:31 Сын мой! кротостью прославляй душу твою и воздавай ей честь по ее достоинству.
\vs Sir 10:32 Кто будет оправдывать согрешающего против души своей? И кто будет хвалить позорящего жизнь свою?
\vs Sir 10:33 Бедного почитают за познания его, а богатого~--- за его богатство:
\vs Sir 10:34 уважаемый же в бедности насколько больше будет уважаем в богатстве? А бесславный в богатстве насколько будет бесславнее в бедности?
\vs Sir 11:1 Мудрость смиренного вознесет голову его и посадит его среди вельмож.
\vs Sir 11:2 Не хвали человека за красоту его, и не имей отвращения к человеку за наружность его.
\vs Sir 11:3 Мала пчела между летающими, но плод ее~--- лучший из сластей.
\vs Sir 11:4 Не хвались пышностью одежд и не превозносись в день славы: ибо дивны дела Господа, и сокровенны дела Его между людьми.
\vs Sir 11:5 Многие из властелинов сидели на земле, тот же, о ком не думали, носил венец.
\vs Sir 11:6 Многие из сильных подверглись крайнему бесчестию, и славные преданы были в руки других.
\vs Sir 11:7 Прежде, нежели исследуешь, не порицай; узнай прежде, и тогда упрекай.
\vs Sir 11:8 Прежде, нежели выслушаешь, не отвечай, и среди речи не перебивай.
\vs Sir 11:9 Не спорь о деле, для тебя ненужном, и не сиди на суде грешников.
\rsbpar\vs Sir 11:10 Сын мой! не берись за множество дел: при множестве дел не останешься без вины. И если будешь гнаться за ними, не достигнешь, и, убегая, не уйдешь.
\vs Sir 11:11 Иной трудится, напрягает силы, поспешает, и тем более отстает.
\vs Sir 11:12 Иной вял, нуждается в помощи, слабосилен и изобилует нищетою;
\vs Sir 11:13 но очи Господа призрели на него во благо ему, и Он восставил его из унижения его и вознес голову его, и многие изумлялись, смотря на него.
\vs Sir 11:14 Доброе и худое, жизнь и смерть, бедность и богатство~--- от Господа.
\vs Sir 11:15 Даяние Господа предоставлено благочестивым, и благоволение Его будет благопоспешно для них вовек.
\vs Sir 11:16 Иной делается богатым от осмотрительности и бережливости своей, и это часть награды его,
\vs Sir 11:17 когда он скажет: <<я нашел покой и теперь наслаждаюсь моими благами>>.
\vs Sir 11:18 И не знает он, сколько пройдет времени до того, когда он оставит их другим и умрет.
\vs Sir 11:19 Твердо стой в завете твоем и пребывай в нем и состарься в деле твоем.
\vs Sir 11:20 Не удивляйся делам грешника, веруй Господу, и пребывай в труде твоем:
\vs Sir 11:21 ибо легко в очах Господа~--- скоро и внезапно обогатить бедного.
\vs Sir 11:22 Благословение Господа~--- награда благочестивого, и в скором времени процветает он благословением Его.
\vs Sir 11:23 Не говори: <<что мне еще нужно? и какие отныне могу иметь еще блага?>>
\vs Sir 11:24 Не говори: <<довольно у меня, и какое отныне могу я потерпеть зло?>>
\vs Sir 11:25 Во дни счастья бывает забвение о несчастье, и во дни несчастья не вспомнится о счастье.
\vs Sir 11:26 Легко для Господа~--- в день смерти воздать человеку по делам его.
\vs Sir 11:27 Минутное страдание производит забвение утех, и при кончине человека открываются дела его.
\vs Sir 11:28 Прежде смерти не называй никого блаженным; человек познается в детях своих.
\rsbpar\vs Sir 11:29 Не всякого человека вводи в дом твой, ибо много козней у коварного.
\vs Sir 11:30 Как охотничья птица в западне, таково сердце надменного: он, как лазутчик, подсматривает падение;
\vs Sir 11:31 превращая добро во зло, он строит козни и на людей избранных кладет пятно.
\vs Sir 11:32 От искры огня умножаются угли, и человек грешный строит козни на кровь.
\vs Sir 11:33 Остерегайся злодея,~--- ибо он строит зло,~--- чтобы он когда-нибудь не положил на тебе пятна навек.
\vs Sir 11:34 Посели в доме твоем чужого, и он расстроит тебя смутами и сделает тебя чужим для твоих.
\vs Sir 12:1 Если ты делаешь добро, знай, кому делаешь, и будет благодарность за твои благодеяния.
\vs Sir 12:2 Делай добро благочестивому, и получишь воздаяние, и если не от него, то от Всевышнего.
\vs Sir 12:3 Нет добра для того, кто постоянно занимается злом и кто не подает милостыни.
\vs Sir 12:4 Давай благочестивому, и не помогай грешнику.
\vs Sir 12:5 Делай добро смиренному, и не давай нечестивому: запирай от него хлеб и не давай ему, чтобы он чрез то не превозмог тебя;
\vs Sir 12:6 ибо ты получил бы сугубое зло за все добро, которое сделал бы ему; ибо и Всевышний ненавидит грешников и нечестивым воздает отмщением.
\vs Sir 12:7 Давай доброму, и не помогай грешнику.
\vs Sir 12:8 Друг не познается в счастье, и враг не скроется в несчастье.
\vs Sir 12:9 При счастье человека враги его в печали, а в несчастье его и друг разойдется с ним.
\vs Sir 12:10 Не верь врагу твоему вовек, ибо, как ржавеет медь, так и злоба его:
\vs Sir 12:11 хотя бы он смирился и ходил согнувшись, будь внимателен душею твоею и остерегайся его, и будешь пред ним, как чистое зеркало, и узнаешь, что он не до конца очистился от ржавчины;
\vs Sir 12:12 не ставь его подле себя, чтобы он, низринув тебя, не стал на твое место; не сажай его по правую сторону себя, чтобы он когда-нибудь не стал домогаться твоего седалища, и ты наконец поймешь слова мои и со скорбью вспомнишь о наставлениях моих.
\vs Sir 12:13 Кто пожалеет об ужаленном заклинателе змей и обо всех, приближающихся к диким зверям? Так и о сближающемся с грешником и приобщающемся грехам его:
\vs Sir 12:14 на время он останется с тобою, но если ты поколеблешься, он не устоит.
\vs Sir 12:15 Устами своими враг усладит \bibemph{тебя}, но в сердце своем замышляет ввергнуть тебя в яму: глазами своими враг будет плакать, а когда найдет случай, не насытится кровью.
\vs Sir 12:16 Если встретится с тобою несчастье, ты найдешь его там прежде себя,
\vs Sir 12:17 и он, как будто желая помочь, подставит тебе ногу:
\vs Sir 12:18 будет кивать головою и хлопать руками, многое будет шептать, и изменит лицо свое.
\vs Sir 13:1 Кто прикасается к смоле, тот очернится, и кто входит в общение с гордым, сделается подобным ему.
\vs Sir 13:2 Не поднимай тяжести свыше твоей силы, и не входи в общение с тем, кто сильнее и богаче тебя.
\vs Sir 13:3 Какое общение у горшка с котлом? Этот толкнет его, и он разобьется.
\vs Sir 13:4 Богач обидел, и сам же грозит; бедняк обижен, и сам же упрашивает.
\vs Sir 13:5 Если ты выгоден для него, он употребит тебя; а если обеднеешь, он оставит тебя.
\vs Sir 13:6 Если ты достаточен, он будет жить с тобою и истощит тебя, а сам не поболезнует.
\vs Sir 13:7 Возымел он в тебе нужду,~--- будет льстить тебе, будет улыбаться тебе и обнадеживать тебя, ласково будет говорить с тобою и скажет: <<не нужно ли тебе чего?>>
\vs Sir 13:8 Своими угощениями он будет пристыжать тебя, доколе, два или три раза ограбив тебя, не насмеется наконец над тобою.
\vs Sir 13:9 После того он, увидев тебя, уклонится от тебя и будет кивать головою при встрече с тобою.
\vs Sir 13:10 Наблюдай, чтобы тебе не быть обманутым
\vs Sir 13:11 и не быть униженным в твоем веселье.
\vs Sir 13:12 Когда сильный будет приглашать тебя, уклоняйся, и тем более он будет приглашать тебя.
\vs Sir 13:13 Не будь навязчив, чтобы не оттолкнули тебя, и не слишком удаляйся, чтобы не забыли о тебе.
\vs Sir 13:14 Не дозволяй себе говорить с ним, как с равным тебе, и не верь слишком многим словам его; ибо долгим разговором он будет искушать тебя и, как бы шутя, изведывать тебя.
\vs Sir 13:15 Немилостив к себе, кто не удерживает себя в словах своих, и он не убережет себя от оскорбления и от уз.
\vs Sir 13:16 Будь осторожен и весьма внимателен, ибо ты ходишь с падением твоим.
\vs Sir 13:17 Услышав это во сне твоем, не засыпай.
\vs Sir 13:18 Во всю жизнь люби Господа и взывай к Нему о спасении твоем.
\vs Sir 13:19 Всякое животное любит подобное себе, и всякий человек~--- ближнего своего.
\vs Sir 13:20 Всякая плоть соединяется по роду своему, и человек прилепляется к подобному себе.
\vs Sir 13:21 Какое общение у волка с ягненком? Так и у грешника~--- с благочестивым.
\vs Sir 13:22 Какой мир у гиены с собакою? И какой мир у богатого с бедным?
\vs Sir 13:23 Ловля у львов~--- дикие ослы в пустыне, так пастбища богатых~--- бедные.
\vs Sir 13:24 Отвратительно для гордого смирение: так отвратителен для богатого бедный.
\vs Sir 13:25 Когда пошатнется богатый, он поддерживается друзьями; а когда упадет бедный, то отталкивается и друзьями.
\vs Sir 13:26 Когда подвергнется несчастью богатый, у него много помощников; сказал нелепость, и оправдали его.
\vs Sir 13:27 Подвергся несчастью бедняк, и еще бранят его; сказал разумно, и его не слушают.
\vs Sir 13:28 Заговорил богатый,~--- и все замолчали и превознесли речь его до облаков;
\vs Sir 13:29 заговорил бедный, и говорят: <<это кто такой?>> И если он споткнется, то совсем низвергнут его.
\rsbpar\vs Sir 13:30 Хорошо богатство, в котором нет греха, и зла бедность в устах нечестивого.
\vs Sir 13:31 Сердце человека изменяет лицо его или на хорошее, или на худое.
\vs Sir 13:32 Признак сердца в счастье~--- лицо веселое, а изобретение притчей соединено с напряженным размышлением.
\vs Sir 14:1 Блажен человек, который не погрешал устами своими и не уязвлен был печалью греха.
\vs Sir 14:2 Блажен, кого не зазирает душа его и кто не потерял надежды своей.
\vs Sir 14:3 Не добро богатство человеку скупому. И на что имение человеку недоброжелательному?
\vs Sir 14:4 Кто собирает, отнимая у души своей, тот собирает для других, и благами его будут пресыщаться другие.
\vs Sir 14:5 Кто зол для себя, для кого будет добр? И не будет он иметь радости от имения своего.
\vs Sir 14:6 Нет хуже человека, который недоброжелателен к самому себе, и это~--- воздаяние за злобу его.
\vs Sir 14:7 Если он и делает добро, то делает в забывчивости, и после обнаруживает зло свое.
\vs Sir 14:8 Зол, кто имеет завистливые глаза, отвращает лицо и презирает души.
\vs Sir 14:9 Глаза любостяжательного не насыщаются какою-либо частью, и неправда злого иссушает душу.
\vs Sir 14:10 Злой глаз завистлив даже на хлеб и в столе своем терпит скудость.
\vs Sir 14:11 Сын мой! по состоянию твоему делай добро себе и приношения Господу достойно приноси.
\vs Sir 14:12 Помни, что смерть не медлит, и завет ада не открыт тебе:
\vs Sir 14:13 прежде, нежели умрешь, делай добро другу, и по силе твоей простирай твою руку и давай ему.
\vs Sir 14:14 Не лишай себя доброго дня, и часть доброго желания да не пройдет мимо тебя.
\vs Sir 14:15 Не другим ли оставишь ты стяжания твои и плоды усилий твоих для раздела по жребию?
\vs Sir 14:16 Давай и принимай, и утешай душу твою,
\vs Sir 14:17 ибо в аде нельзя найти утех.
\vs Sir 14:18 Всякая плоть, как одежда, ветшает; ибо от века~--- определение: <<смертью умрешь>>.
\vs Sir 14:19 Как зеленеющие листья на густом дереве~--- одни спадают, а другие вырастают: так и род от плоти и крови~--- один умирает, а другой рождается.
\vs Sir 14:20 Всякая вещь, подверженная тлению, исчезает, и сделавший ее умирает с нею.
\rsbpar\vs Sir 14:21 Блажен человек, который упражняется в мудрости и в разуме своем поучается святому.
\vs Sir 14:22 Кто размышляет в сердце своем о путях ее, тот получит разумение и в тайнах ее.
\vs Sir 14:23 Выходи за нею, как ловчий, и строй засаду на путях ее.
\vs Sir 14:24 Кто приклоняется к окнам ее, тот послушает и при дверях ее.
\vs Sir 14:25 Кто обращается вблизи дома ее, тот вобьет гвоздь и в стенах ее, поставит палатку свою подле нее и будет обитать в жилище благ.
\vs Sir 14:26 Он положит детей своих под кровом ее и будет иметь ночлег под сенью ее.
\vs Sir 14:27 Он прикроется ею от зноя и будет жить в славе ее.
\vs Sir 15:1 Боящийся Господа будет поступать так, и твердый в законе овладеет ею.
\vs Sir 15:2 И она встретит его, как мать, и примет его к себе, как целомудренная супруга;
\vs Sir 15:3 напитает его хлебом разума, и водою мудрости напоит его.
\vs Sir 15:4 Он утвердится на ней и не поколеблется; прилепится к ней и не постыдится.
\vs Sir 15:5 И она вознесет его над ближними его, и среди собрания откроет уста его.
\vs Sir 15:6 Веселье и венец радости и вечное имя наследует он.
\vs Sir 15:7 Не постигнут ее люди неразумные, и грешники не увидят ее.
\vs Sir 15:8 Далека она от гордости, и люди лживые не подумают о ней.
\vs Sir 15:9 Неприятна похвала в устах грешника, ибо не от Господа послана она.
\vs Sir 15:10 Будет похвала произнесена мудростью, и Господь благопоспешит ей.
\rsbpar\vs Sir 15:11 Не говори: <<ради Господа я отступил>>; ибо, что Он ненавидит, того ты не должен делать.
\vs Sir 15:12 Не говори: <<Он ввел меня в заблуждение>>, ибо Он не имеет надобности в муже грешном.
\vs Sir 15:13 Всякую мерзость Господь ненавидит, и неприятна она боящимся Его.
\vs Sir 15:14 Он от начала сотворил человека и оставил его в руке произволения его.
\vs Sir 15:15 Если хочешь, соблюдешь заповеди и сохранишь благоугодную верность.
\vs Sir 15:16 Он предложил тебе огонь и воду: на что хочешь, прострешь руку твою.
\vs Sir 15:17 Пред человеком жизнь и смерть, и чего он пожелает, то и дастся ему.
\vs Sir 15:18 Велика премудрость Господа, крепок Он могуществом и видит всё.
\vs Sir 15:19 Очи Его~--- на боящихся Его, и Он знает всякое дело человека.
\vs Sir 15:20 Никому не заповедал Он поступать нечестиво и никому не дал позволения грешить.
\vs Sir 16:1 Не желай множества негодных детей и не радуйся о сыновьях нечестивых. Когда они умножаются, не радуйся о них, если нет в них страха Господня.
\vs Sir 16:2 Не надейся на их жизнь и не опирайся на их множество.
\vs Sir 16:3 Лучше один праведник, нежели тысяча \bibemph{грешников},
\vs Sir 16:4 и лучше умереть бездетным, нежели иметь детей нечестивых,
\vs Sir 16:5 ибо от одного разумного населится город, а племя беззаконных опустеет.
\vs Sir 16:6 Много такого видело око мое, и еще более того слышало ухо мое.
\vs Sir 16:7 В сборище грешников возгорится огонь, как и в народе непокорном возгорался гнев.
\vs Sir 16:8 Не умилостивился Он над древними исполинами, которые в надежде на силу свою сделались отступниками;
\vs Sir 16:9 не пощадил и живших в одном месте с Лотом, которыми возгнушался за их гордость;
\vs Sir 16:10 не помиловал народа погибельного, который надмевался грехами своими,
\vs Sir 16:11 равно как и шестисот тысяч человек, соединившихся в жестокосердии своем. И хотя бы и один был непокорный, было бы удивительно, если б он остался ненаказанным;
\vs Sir 16:12 ибо и милость и гнев~--- во власти Его: силен Он помиловать и излить гнев.
\vs Sir 16:13 Как велика милость Его, так велико и обличение Его. Он судит человека по делам его.
\vs Sir 16:14 Не убежит от Него грешник с хищением, и терпение благочестивого не останется тщетным.
\vs Sir 16:15 Всякой милостыне Он даст место, каждый получит по делам своим.
\vs Sir 16:16 Не говори: <<я скроюсь от Господа; неужели с высоты кто вспомнит обо мне?
\vs Sir 16:17 Во множестве народа меня не заметят; ибо что душа моя в неизмеримом создании?
\vs Sir 16:18 Вот, небо и небо небес~--- Божие, бездна и земля колеблются от посещения Его.
\vs Sir 16:19 Равно сотрясаются от страха горы и основания земли, когда Он взирает.
\vs Sir 16:20 И этого не может понять сердце;
\vs Sir 16:21 а пути Его кто постигнет? Как ветер, которого человек не может видеть, так и большая часть дел Его сокрыты.
\vs Sir 16:22 Кто возвестит о делах правосудия Его? или кто будет ожидать их? ибо далеко это определение>>.
\vs Sir 16:23 Скудный умом думает так, и человек неразумный и заблуждающийся размышляет так глупо.
\rsbpar\vs Sir 16:24 Слушай меня, сын мой, и учись знанию, и внимай сердцем твоим словам моим.
\vs Sir 16:25 Я показываю тебе учение обдуманное и передаю знание точное.
\vs Sir 16:26 По определению Господа дела Его от начала, и от сотворения их Он разделил части их.
\vs Sir 16:27 Навек устроил Он дела Свои, и начала их~--- в роды их. Они не алчут, не утомляются и не прекращают своих действий.
\vs Sir 16:28 Ни одно не стесняет близкого ему,
\vs Sir 16:29 и до века не воспротивятся они слову Его.
\vs Sir 16:30 И потом воззрел Господь на землю и наполнил ее Своими благами.
\vs Sir 16:31 Душа всего живущего покрыла лице ее, и в нее все возвратится.
\vs Sir 17:1 Господь создал человека из земли и опять возвращает его в нее.
\vs Sir 17:2 Определенное число дней и время дал Он им, и дал им власть над всем, что на ней.
\vs Sir 17:3 По природе их, облек их силою и сотворил их по образу Своему,
\vs Sir 17:4 и вложил страх к ним во всякую плоть, чтобы господствовать им над зверями и птицами.
\vs Sir 17:5 Он дал им смысл, язык и глаза, и уши и сердце для рассуждения,
\vs Sir 17:6 исполнил их проницательностью разума и показал им добро и зло.
\vs Sir 17:7 Он положил око Свое на сердца их, чтобы показать им величие дел Своих,
\vs Sir 17:8 да прославляют они святое имя Его и возвещают о величии дел Его.
\vs Sir 17:9 Он приложил им знание и дал им в наследство закон жизни;
\vs Sir 17:10 вечный завет поставил с ними и показал им суды Свои.
\vs Sir 17:11 Величие славы видели глаза их, и славу голоса Его слышало ухо их.
\vs Sir 17:12 И сказал Он им: <<остерегайтесь всякой неправды>>; и заповедал каждому из них обязанность к ближнему.
\vs Sir 17:13 Пути их всегда пред Ним, не скроются от очей Его.
\vs Sir 17:14 Каждому народу поставил Он вождя,
\vs Sir 17:15 а Израиль есть удел Господа.
\vs Sir 17:16 Все дела их~--- как солнце пред Ним, и очи Его всегда на путях их.
\vs Sir 17:17 Не утаились от Него неправды их, и все грехи их~--- пред Господом.
\rsbpar\vs Sir 17:18 Милостыня человека~--- как печать у Него, и благодеяние человека сохранит Он, как зеницу ока.
\vs Sir 17:19 Потом Он восстанет и воздаст им, и даяние их на голову их возвратит.
\vs Sir 17:20 Но кающимся Он давал обращение и ободрял ослабевавших в терпении.
\vs Sir 17:21 Обратись к Господу и оставь грехи;
\vs Sir 17:22 молись пред Ним и уменьши твои преткновения.
\vs Sir 17:23 Возвратись ко Всевышнему, и отвратись от неправды, и сильно возненавидь мерзость.
\vs Sir 17:24 Кто будет восхвалять Всевышнего в аде, вместо живущих и прославляющих Его?
\vs Sir 17:25 От мертвого, как от несуществующего, нет прославления:
\vs Sir 17:26 живый и здоровый восхвалит Господа.
\vs Sir 17:27 Как велико милосердие Господа и примирение с обращающимися к Нему!
\vs Sir 17:28 Не может быть всего в человеке,
\vs Sir 17:29 потому что не бессмертен сын человеческий.
\vs Sir 17:30 Что светлее солнца? но и оно затмевается. И о злом будет помышлять плоть и кровь.
\vs Sir 17:31 За силами высоких небес Он Сам наблюдает, а люди все~--- земля и пепел.
\vs Sir 18:1 Все вообще создал Живущий во веки; Господь один праведен.
\vs Sir 18:2 Никому не предоставил Он изъяснять дел\acc{а} Его.
\vs Sir 18:3 И кто может исследовать великие дела Его?
\vs Sir 18:4 Кто может измерить силу величия Его? и кто может также изречь милости Его?
\vs Sir 18:5 Невозможно ни умалить, ни увеличить, и невозможно исследовать дивных дел Господа.
\vs Sir 18:6 Когда человек окончил бы, тогда он только начинает, и когда перестанет, придет в изумление.
\vs Sir 18:7 Что есть человек и что польза его? что благо его и что зло его?
\vs Sir 18:8 Число дней человека~--- много, если сто лет: как капля воды из моря или крупинка песка, так малы лета его в дне вечности.
\vs Sir 18:9 Посему Господь долготерпелив к \bibemph{людям} и изливает на них милость Свою.
\vs Sir 18:10 Он видит и знает, что конец их очень бедствен,
\vs Sir 18:11 и потому умножает милости Свои.
\vs Sir 18:12 Милость человека~--- к ближнему его, а милость Господа~--- на всякую плоть.
\vs Sir 18:13 Он обличает и вразумляет, и поучает и обращает, как пастырь стадо свое.
\vs Sir 18:14 Он милует принимающих вразумление и усердно обращающихся к закону Его.
\rsbpar\vs Sir 18:15 Сын мой! при благотворениях не делай упреков, и при всяком даре не оскорбляй словами.
\vs Sir 18:16 Роса не охлаждает ли зноя? так слово~--- лучше, нежели даяние.
\vs Sir 18:17 Поэтому не выше ли доброго даяния слово? а у человека доброжелательного и то и другое.
\vs Sir 18:18 Глупый немилосердно укоряет, и подаяние неблагорасположенного иссушает глаза.
\vs Sir 18:19 Прежде, нежели начнешь говорить, обдумывай, и прежде болезни заботься о себе.
\vs Sir 18:20 Испытывай себя прежде суда, и во время посещения найдешь милость.
\vs Sir 18:21 Прежде, нежели почувствуешь слабость, смиряйся, и во время грехов покажи обращение.
\vs Sir 18:22 Ничто да не препятствует тебе исполнить обет благовременно, и не откладывай оправдания до смерти.
\vs Sir 18:23 Прежде, нежели начнешь молиться, приготовь себя, и не будь как человек, искушающий Господа.
\vs Sir 18:24 Припоминай о гневе в день смерти и о времени отмщения, когда Господь отвратит лице Свое.
\vs Sir 18:25 Во время сытости вспоминай о времени голода и во дни богатства~--- о бедности и нужде.
\vs Sir 18:26 От утра до вечера изменяется время, и все скоротечно пред Господом.
\vs Sir 18:27 Человек мудрый во всем будет осторожен и во дни грехов удержится от беспечности.
\vs Sir 18:28 Всякий разумный познает премудрость и нашедшему ее воздаст хвалу.
\vs Sir 18:29 Рассудительные в словах и сами умудряются, и источают основательные притчи.
\rsbpar\vs Sir 18:30 Не ходи вслед похотей твоих и воздерживайся от пожеланий твоих.
\vs Sir 18:31 Если будешь доставлять душе твоей приятное для вожделений, то она сделает тебя потехою для врагов твоих.
\vs Sir 18:32 Не ищи увеселения в большой роскоши и не привязывайся к пиршествам.
\vs Sir 18:33 Не сделайся нищим, пиршествуя на занятые деньги, когда ничего нет у тебя в кошельке.
\vs Sir 19:1 Работник, склонный к пьянству, не обогатится, и ни во что ставящий малое мало-помалу придет в упадок.
\vs Sir 19:2 Вино и женщины развратят разумных, а связывающийся с блудницами сделается еще наглее;
\vs Sir 19:3 гниль и черви наследуют его, и дерзкая душа истребится.
\vs Sir 19:4 Кто скоро доверяет, тот легкомыслен, и согрешающий грешит против души своей.
\vs Sir 19:5 Преданный сердцем удовольствиям будет осужден, а сопротивляющийся вожделениям увенчает жизнь свою.
\vs Sir 19:6 Обуздывающий язык будет жить мирно, и ненавидящий болтливость уменьшит зло.
\vs Sir 19:7 Никогда не повторяй слова, и ничего у тебя не убудет.
\vs Sir 19:8 Ни другу ни недругу не рассказывай и, если это тебе не грех, не открывай;
\vs Sir 19:9 ибо он выслушает тебя, и будет остерегаться тебя, и по времени возненавидит тебя.
\vs Sir 19:10 Выслушал ты слово, пусть умрет оно с тобою: не бойся, не расторгнет оно тебя.
\vs Sir 19:11 Глупый от слова терпит такую же муку, как рождающая~--- от младенца.
\vs Sir 19:12 Что стрела, вонзенная в бедро, то слово в сердце глупого.
\vs Sir 19:13 Расспроси друга \bibemph{твоего}, может быть, не сделал он того; и если сделал, то пусть вперед не делает.
\vs Sir 19:14 Расспроси друга, может быть, не говорил он того; и если сказал, то пусть не повторит того.
\vs Sir 19:15 Расспроси друга, ибо часто бывает клевета.
\vs Sir 19:16 Не всякому слову верь.
\vs Sir 19:17 Иной погрешает \bibemph{словом}, но не от души; и кто не погрешал языком своим?
\vs Sir 19:18 Расспроси ближнего твоего прежде, нежели грозить ему, и дай место закону Всевышнего.\rsbpar Всякая мудрость~--- страх Господень, и во всякой мудрости~--- исполнение закона.
\vs Sir 19:19 И не есть мудрость знание худого. И нет разума, где совет грешников.
\vs Sir 19:20 Есть лукавство, и это мерзость; и есть неразумный, скудный мудростью.
\vs Sir 19:21 Лучше скудный знанием, но богобоязненный, нежели богатый знанием~--- и преступающий закон.
\vs Sir 19:22 Есть хитрость изысканная, но она беззаконна, и есть превращающий \bibemph{суд}, чтобы произнести приговор.
\vs Sir 19:23 Есть лукавый, который ходит согнувшись, в унынии, но внутри он полон коварства.
\vs Sir 19:24 Он поник лицом и притворяется глухим, но он предварит тебя там, где и не думаешь.
\vs Sir 19:25 И если недостаток силы воспрепятствует ему повредить тебе, то он сделает тебе зло, когда найдет случай.
\vs Sir 19:26 По виду узнается человек, и по выражению лица при встрече познается разумный.
\vs Sir 19:27 Одежда и осклабление зубов и походка человека показывают свойство его.
\vs Sir 19:28 Бывает обличение, но не вовремя, и бывает, что иной молчит~--- и он благоразумен.
\vs Sir 20:1 Гораздо лучше обличить, нежели сердиться тайно; и обличаемый наедине предостережется от вреда.
\vs Sir 20:2 Как хорошо обличенному показать раскаяние!
\vs Sir 20:3 Ибо он избежит вольного греха.
\vs Sir 20:4 Что~--- пожелание евнуха растлить девицу, то~--- производящий суд с натяжкою.
\vs Sir 20:5 Иной молчит~--- и оказывается мудрым; а иной бывает ненавистным за многую болтливость.
\vs Sir 20:6 Иной молчит, потому что не имеет, что отвечать; а иной молчит, потому что знает время.
\vs Sir 20:7 Мудрый человек будет молчать до времени; а тщеславный и безрассудный не будет ждать времени.
\vs Sir 20:8 Многоречивый опротивеет, и кто восхищает себе право говорить, будет возненавиден.
\vs Sir 20:9 Бывает успех человеку ко злу, а находка~--- в потерю.
\vs Sir 20:10 Есть даяние, которое не будет тебе на пользу, и есть даяние, за которое бывает сугубое воздаяние.
\vs Sir 20:11 Бывает унижение для славы, а иной от унижения поднимает голову.
\vs Sir 20:12 Иной малым покупает многое и заплатит за то в семь раз больше.
\vs Sir 20:13 Мудрый в слове делается любезным, любезности же глупых останутся напрасными.
\vs Sir 20:14 Даяние безумного не будет тебе на пользу; ибо у него вместо одного много глаз для принятия.
\vs Sir 20:15 Немного даст он, а попрекать будет много, и раскроет уста свои, как глашатай. Ныне он взаем дает, а завтра потребует назад: ненавистен такой человек Господу и людям.
\vs Sir 20:16 Глупый говорит: <<нет у меня друга, и нет благодарности за мои благодеяния. Съедающие хлеб мой льстивы языком>>.
\vs Sir 20:17 Как часто и сколь многие будут насмехаться над ним!
\vs Sir 20:18 Преткновение от земли лучше, нежели от языка. Итак, скоро придет падение злых.
\vs Sir 20:19 Неприятный человек~--- безвременная басня; она всегда будет на устах невежд.
\vs Sir 20:20 Притча из уст глупого отвратительна, ибо он не скажет ее в свое время.
\vs Sir 20:21 Иной удерживается от греха скудостью, и в этом воздержании он не будет сокрушаться.
\vs Sir 20:22 Иной губит душу свою по робости, и губит ее из лицеприятия к безумному.
\vs Sir 20:23 Иной из-за стыда дает обещания другу, и без причины наживает в нем себе врага.
\vs Sir 20:24 Злой порок в человеке~--- ложь; в устах невежд она~--- всегда.
\vs Sir 20:25 Лучше вор, нежели постоянно говорящий ложь; но оба они наследуют погибель.
\vs Sir 20:26 Поведение лживого человека~--- бесчестно, и позор его всегда с ним.
\vs Sir 20:27 Мудрый в словах возвысит себя, и человек разумный понравится вельможам.
\vs Sir 20:28 Возделывающий землю увеличит свой стог, и угождающий вельможам получит помилование в случае неправды.
\vs Sir 20:29 Угощения и подарки ослепляют глаза мудрых и, как бы узда в устах, отвращают обличения.
\vs Sir 20:30 Скрытая мудрость и утаенное сокровище~--- какая польза от обоих?
\vs Sir 20:31 Лучше человек, скрывающий свою глупость, нежели человек, скрывающий свою мудрость.
\vs Sir 21:1 Сын мой! если ты согрешил, не прилагай более грехов и о прежних молись.
\vs Sir 21:2 Беги от греха, как от лица змея; ибо, если подойдешь к нему, он ужалит тебя.
\vs Sir 21:3 Зубы его~--- зубы львиные, которые умерщвляют души людей.
\vs Sir 21:4 Всякое беззаконие как обоюдоострый меч: ране от него нет исцеления.
\vs Sir 21:5 Устрашения и насилия опустошат богатство: так опустеет и дом высокомерного.
\vs Sir 21:6 Моление из уст нищего~--- \bibemph{только} до ушей его; но суд над ним поспешно приближается.
\vs Sir 21:7 Ненавидящий обличение идет по следам грешника, а боящийся Господа обратится сердцем.
\vs Sir 21:8 Издалека узнается сильный языком; но разумный видит, где тот спотыкается.
\vs Sir 21:9 Строящий дом свой на чужие деньги~--- то же, что собирающий камни для своей могилы.
\vs Sir 21:10 Сборище беззаконных~--- куча пакли, и конец их~--- пламень огненный.
\vs Sir 21:11 Путь грешников вымощен камнями, но на конце его~--- пропасть ада.
\vs Sir 21:12 Соблюдающий закон обладает своими мыслями,
\vs Sir 21:13 и совершение страха Господня~--- мудрость.
\vs Sir 21:14 Не научится тот, кто неспособен;
\vs Sir 21:15 но есть способность, умножающая горечь.
\vs Sir 21:16 Знание мудрого увеличивается подобно наводнению, и совет его,~--- как источник жизни.
\vs Sir 21:17 Сердце глупого подобно разбитому сосуду и не удержит в себе никакого знания.
\vs Sir 21:18 Если мудрое слово услышит разумный, то он похвалит его и приложит к себе. Услышал его легкомысленный, и оно не понравилось ему, и он бросил его за себя.
\vs Sir 21:19 Речь глупого~--- как бремя в пути, в устах же разумного находят приятность.
\vs Sir 21:20 Речей разумного будут искать в собрании, и о словах его будут размышлять в сердце.
\vs Sir 21:21 Как разрушенный дом, так мудрость глупому, и знание неразумного~--- бессмысленные слова.
\vs Sir 21:22 Наставление для безумных~--- оковы на ногах и как цепи на правой руке.
\vs Sir 21:23 Глупый в смехе возвышает голос свой, а муж благоразумный едва тихо улыбнется.
\vs Sir 21:24 Как золотой наряд~--- наставление для разумного, и как драгоценное украшение на правой руке.
\vs Sir 21:25 Нога глупого спешит в чужой дом, но человек многоопытный постыдится людей;
\vs Sir 21:26 неразумный сквозь дверь заглядывает в дом, а человек благовоспитанный остановится вне;
\vs Sir 21:27 невежество человека~--- подслушивать у дверей, благоразумный же огорчится таким бесстыдством.
\vs Sir 21:28 Уста многоречивых рассказывают чужое, а слова благоразумных взвешиваются на весах.
\vs Sir 21:29 В устах глупых~--- сердце их, уста же мудрых~--- в сердце их.
\vs Sir 21:30 Когда нечестивый проклинает сатану, то проклинает свою душу.
\vs Sir 21:31 Наушник оскверняет свою душу и будет ненавидим везде, где только жить будет.
\vs Sir 22:1 Грязному камню подобен ленивый: всякий освищет бесславие его.
\vs Sir 22:2 Воловьему помету подобен ленивый: всякий, поднявший его, отряхнет руку.
\vs Sir 22:3 Стыд отцу рождение невоспитанного сына, дочь же \bibemph{невоспитанная} рождается на унижение.
\vs Sir 22:4 Разумная дочь приобретет себе мужа, а бесстыдная~--- печаль родившему.
\vs Sir 22:5 Наглая позорит отца и мужа, и у обоих будет в презрении.
\vs Sir 22:6 Не вовремя рассказ~--- то же, что музыка во время печали; наказание же и учение мудрости прилично всякому времени.
\vs Sir 22:7 Поучающий глупого~--- то же, что склеивающий черепки или пробуждающий спящего от глубокого сна.
\vs Sir 22:8 Рассказывающий что-либо глупому~--- то же, что рассказывающий дремлющему, который по окончании спрашивает: <<что?>>
\vs Sir 22:9 Плачь над умершим, ибо свет исчез для него; плачь и над глупым, ибо разум исчез для него.
\vs Sir 22:10 Меньше плачь над умершим, потому что он успокоился, а злая жизнь глупого~--- хуже смерти.
\vs Sir 22:11 Плачь об умершем~--- семь дней, а о глупом и нечестивом~--- все дни жизни его.
\vs Sir 22:12 С безрассудным много не говори, и к неразумному не ходи;
\vs Sir 22:13 берегись от него, чтобы не иметь неприятности и не замарать себя столкновением с ним;
\vs Sir 22:14 уклонись от него и найдешь покой и не будешь огорчен безумием его.
\vs Sir 22:15 Что тяжелее свинца? и какое имя ему, как не глупый?
\vs Sir 22:16 Легче понести песок и соль и глыбу железа, нежели человека бессмысленного.
\vs Sir 22:17 Как деревянная связь в доме, крепко устроенная, не дает ему распадаться при сотрясении, так сердце, утвержденное на обдуманном совете, не поколеблется во время страха.
\vs Sir 22:18 Сердце, утвержденное на разумном размышлении,~--- как лепное украшение на вытесанной стене.
\vs Sir 22:19 Подпорка, поставленная на высоте, не устоит против ветра:
\vs Sir 22:20 так боязливое сердце, при глупом размышлении, не устоит против страха.
\rsbpar\vs Sir 22:21 Наносящий удар глазу вызывает слезы, а наносящий удар сердцу возбуждает чувство болезненное.
\vs Sir 22:22 Бросающий камень в птиц отгонит их; а поносящий друга расторгнет дружбу.
\vs Sir 22:23 Если ты на друга извлек меч, не отчаивайся, ибо возможно возвращение дружбы.
\vs Sir 22:24 Если ты открыл уста против друга, не бойся, ибо возможно примирение.
\vs Sir 22:25 Только поношение, гордость, обнаружение тайны и коварное злодейство могут отогнать всякого друга.
\vs Sir 22:26 Приобретай доверенность ближнего в нищете его, чтобы радоваться вместе с ним при богатстве его;
\vs Sir 22:27 оставайся с ним во время скорби, чтобы иметь участие в его наследии.
\vs Sir 22:28 Прежде пламени бывает в печи пар и дым: так прежде кровопролития~--- ссоры.
\vs Sir 22:29 Защищать друга я не постыжусь и не скроюсь от лица его;
\vs Sir 22:30 а если приключится мне чрез него зло, то всякий, кто услышит, будет остерегаться его.
\rsbpar\vs Sir 22:31 Кто даст мне стражу к устам моим и печать благоразумия на уста мои, чтобы мне не пасть чрез них и чтобы язык мой не погубил меня!
\vs Sir 23:1 Господи, Отче и Владыко жизни моей! Не оставь меня на волю их и не допусти меня пасть чрез них.
\vs Sir 23:2 Кто приставит бич к помышлениям моим и к сердцу моему наставника в мудрости, чтобы они не щадили проступков моих и не потворствовали заблуждениям их;
\vs Sir 23:3 чтобы не умножались проступки мои и не увеличивались заблуждения мои; чтобы не упасть мне пред противниками, и чтобы не порадовался надо мною враг мой?
\vs Sir 23:4 Господи, Отче и Боже жизни моей! Не дай мне возношения очей и вожделение отврати от меня.
\vs Sir 23:5 Пожелания чрева и сладострастие да не овладеют мною, и не предай меня бесстыдной душе.
\rsbpar\vs Sir 23:6 Выслушайте, дети, наставление для уст: соблюдающий его не будет уловлен своими устами.
\vs Sir 23:7 Уловлен будет ими грешник, и злоречивый и надменный преткнутся чрез них.
\vs Sir 23:8 Не приучай уст твоих к клятве
\vs Sir 23:9 и не обращай в привычку употреблять в клятве имя Святаго.
\vs Sir 23:10 Ибо, как раб, постоянно подвергающийся наказанию, не избавляется от ран, так и клянущийся непрестанно именем Святаго не очистится от греха.
\vs Sir 23:11 Человек, часто клянущийся, исполнится беззакония, и не отступит от дома его бич.
\vs Sir 23:12 Если он согрешит, грех его на нем; и если он вознерадел, то сугубо согрешит;
\vs Sir 23:13 и если он клялся напрасно, то не оправдается, и дом его наполнится несчастьями.
\vs Sir 23:14 Есть речь, облеченная смертью: да не найдется она в наследии Иакова!
\vs Sir 23:15 Ибо от благочестивых все это будет удалено, и они не запутаются во грехах.
\vs Sir 23:16 Не приучай твоих уст к грубой невежливости, ибо при ней бывают греховные слова.
\vs Sir 23:17 Помни об отце и о матери твоей, когда сидишь среди вельмож,
\vs Sir 23:18 чтобы тебе не забыться пред ними и по привычке не сделать глупости, и не пожелать, что лучше бы ты не родился, и не проклясть дня рождения твоего.
\vs Sir 23:19 Человек, привыкающий к бранным словам, во все дни свои не научится.
\vs Sir 23:20 Два качества умножают грехи, а третье навлекает гнев:
\vs Sir 23:21 душа горячая, как пылающий огонь, не угаснет, пока не истощится;
\vs Sir 23:22 человек, блудодействующий в теле плоти своей, не перестанет, пока не прогорит огонь.
\vs Sir 23:23 Блуднику сладок всякий хлеб: он не перестанет, доколе не умрет.
\vs Sir 23:24 Человек, который согрешает против своего ложа, говорит в душе своей: <<кто видит меня?
\vs Sir 23:25 Вокруг меня тьма, и стены закрывают меня, и никто не видит меня: чего мне бояться? Всевышний не воспомянет грехов моих>>.
\vs Sir 23:26 Страх его~--- только глаза человеческие,
\vs Sir 23:27 и не знает он того, что очи Господа в десять тысяч крат светлее солнца
\vs Sir 23:28 и взирают на все пути человеческие, и проникают в места сокровенные.
\vs Sir 23:29 Ему известно было все прежде, нежели сотворено было, равно как и по совершении.
\vs Sir 23:30 Такой \bibemph{человек} будет наказан на улицах города и будет застигнут там, где не думал.
\vs Sir 23:31 Так и жена, оставившая мужа и произведшая наследника от чужого:
\vs Sir 23:32 ибо, во-первых, она не покорилась закону Всевышнего, во-вторых, согрешила против своего мужа и, в-третьих, в блуде прелюбодействовала и произвела детей от чужого мужа.
\vs Sir 23:33 Она будет выведена пред собрание, и о детях ее будет исследование.
\vs Sir 23:34 Дети ее не укоренятся, и ветви ее не дадут плода.
\vs Sir 23:35 Она оставит память о себе на проклятие, и позор ее не изгладится.
\vs Sir 23:36 Оставшиеся познают, что нет ничего лучше страха Господня и нет ничего сладостнее, как внимать заповедям Господним.
\vs Sir 23:37 Великая слава~--- следовать Господу, а быть тебе принятым от Него~--- долгоденствие.
\vs Sir 24:1 Премудрость прославит себя и среди народа своего будет восхвалена.
\vs Sir 24:2 В церкви Всевышнего она откроет уста свои, и пред воинством Его будет прославлять себя:
\vs Sir 24:3 <<я вышла из уст Всевышнего и подобно облаку покрыла землю;
\vs Sir 24:4 я поставила скинию на высоте, и престол мой~--- в столпе облачном;
\vs Sir 24:5 я одна обошла круг небесный и ходила во глубине бездны;
\vs Sir 24:6 в волнах моря и по всей земле и во всяком народе и племени имела я владение:
\vs Sir 24:7 между всеми ими я искала успокоения, и в чьем наследии водвориться мне.
\vs Sir 24:8 Тогда Создатель всех повелел мне, и Произведший меня указал мне покойное жилище и сказал:
\vs Sir 24:9 поселись в Иакове и прими наследие в Израиле.
\vs Sir 24:10 Прежде века от начала Он произвел меня, и я не скончаюсь во веки.
\vs Sir 24:11 Я служила пред Ним во святой скинии и так утвердилась в Сионе.
\vs Sir 24:12 Он дал мне также покой в возлюбленном городе, и в Иерусалиме~--- власть моя.
\vs Sir 24:13 И укоренилась я в прославленном народе, в наследственном уделе Господа.
\vs Sir 24:14 Я возвысилась, как кедр на Ливане и как кипарис на горах Ермонских;
\vs Sir 24:15 я возвысилась, как пальма в Енгадди и как розовые кусты в Иерихоне;
\vs Sir 24:16 я, как красивая маслина в долине и как платан, возвысилась.
\vs Sir 24:17 Как корица и аспалаф, я издала ароматный запах и, как отличная смирна, распространила благоухание,
\vs Sir 24:18 как халвани, оникс и стакти и как благоухание ладана в скинии.
\vs Sir 24:19 Я распростерла свои ветви, как теревинф, и ветви мои~--- ветви славы и благодати.
\vs Sir 24:20 Я~--- как виноградная лоза, произращающая благодать, и цветы мои~--- плод славы и богатства.
\vs Sir 24:21 Приступите ко мне, желающие меня, и насыщайтесь плодами моими;
\vs Sir 24:22 ибо воспоминание обо мне слаще меда и обладание мною приятнее медового сота.
\vs Sir 24:23 Ядущие меня еще будут алкать, и пьющие меня еще будут жаждать.
\vs Sir 24:24 Слушающий меня не постыдится, и трудящиеся со мною не погрешат.
\vs Sir 24:25 Все это~--- книга завета Бога Всевышнего,
\vs Sir 24:26 закон, который заповедал Моисей как наследие сонмам Иаковлевым.
\vs Sir 24:27 Он насыщает мудростью, как Фисон и как Тигр во дни новин;
\vs Sir 24:28 он наполняет разумом, как Евфрат и как Иордан во дни жатвы;
\vs Sir 24:29 он разливает учение, как свет и как Гион во время собирания винограда.
\vs Sir 24:30 Первый человек не достиг полного познания ее; не исследует ее также и последний;
\vs Sir 24:31 ибо мысли ее полнее моря, и намерения ее глубже великой бездны.
\vs Sir 24:32 И я, как канал из реки и как водопровод, вышла в рай.
\vs Sir 24:33 Я сказала: полью мой сад и напою мои гряды.
\vs Sir 24:34 И вот, канал мой сделался рекою, и река моя сделалась морем.
\vs Sir 24:35 И буду я сиять учением, как утренним светом, и далеко проявлю его;
\vs Sir 24:36 и буду я изливать учение, как пророчество, и оставлю его в роды вечные>>.
\vs Sir 24:37 Видите, что я трудился не для себя одного, но для всех, ищущих \bibemph{премудрости}.
\vs Sir 25:1 Тремя я украсилась и стала прекрасною пред Господом и людьми:
\vs Sir 25:2 это~--- единомыслие между братьями и любовь между ближними, и жена и муж, согласно живущие между собою.
\vs Sir 25:3 И три рода \bibemph{людей} возненавидела душа моя, и очень отвратительна для меня жизнь их:
\vs Sir 25:4 надменного нищего, лживого богача и старика-прелюбодея, ослабевающего в рассудке.
\vs Sir 25:5 Чего не собрал ты в юности,~--- как же можешь приобрести в старости твоей?
\vs Sir 25:6 Как прилично сединам судить, и старцам~--- уметь давать совет!
\vs Sir 25:7 Как прекрасна мудрость старцев и как приличны людям почтенным рассудительность и совет!
\vs Sir 25:8 Венец старцев~--- многосторонняя опытность, и хвала их~--- страх Господень.
\vs Sir 25:9 Девять помышлений похвалил я в сердце, а десятое выскажу языком:
\vs Sir 25:10 \bibemph{это} человек, радующийся о детях и при жизни видящий падение врагов.
\vs Sir 25:11 Блажен, кто живет с женою разумною, кто не погрешает языком и не служит недостойному себя.
\vs Sir 25:12 Блажен, кто приобрел мудрость и передает ее в уши слушающих.
\vs Sir 25:13 Как велик тот, кто нашел премудрость! но он не выше того, кто боится Господа.
\vs Sir 25:14 Страх Господень все превосходит, и имеющий его с кем может быть сравнен?
\rsbpar\vs Sir 25:15 \bibemph{Можно перенести} всякую рану, только не рану сердечную, и всякую злость, только не злость женскую,
\vs Sir 25:16 всякое нападение, только не нападение от ненавидящих, и всякое мщение, только не мщение врагов;
\vs Sir 25:17 нет головы ядовитее головы змеиной, и нет ярости сильнее ярости врага.
\vs Sir 25:18 Соглашусь лучше жить со львом и драконом, нежели жить со злою женою.
\vs Sir 25:19 Злость жены изменяет взгляд ее и делает лице ее мрачным, как у медведя.
\vs Sir 25:20 Сядет муж ее среди друзей своих и, услышав \bibemph{о ней}, горько вздохнет.
\vs Sir 25:21 Всякая злость мала в сравнении со злостью жены; жребий грешника да падет на нее.
\vs Sir 25:22 Что восхождение по песку для ног старика, то сварливая жена для тихого мужа.
\vs Sir 25:23 Не засматривайся на красоту женскую и не похотствуй на жену.
\vs Sir 25:24 Досада, стыд и большой срам, когда жена будет преобладать над своим мужем.
\vs Sir 25:25 Сердце унылое и лице печальное и рана сердечная~--- злая жена.
\vs Sir 25:26 Опущенные руки и расслабленные колени~--- жена, которая не счастливит своего мужа.
\vs Sir 25:27 От жены начало греха, и чрез нее все мы умираем.
\vs Sir 25:28 Не давай воде выхода, ни злой жене~--- власти;
\vs Sir 25:29 если она не ходит под рукою твоею, то отсеки ее от плоти твоей.
\vs Sir 26:1 Счастлив муж доброй жены, и число дней его~--- сугубое.
\vs Sir 26:2 Жена добродетельная радует своего мужа и лета его исполнит миром;
\vs Sir 26:3 добрая жена~--- счастливая доля: она дается в удел боящимся Господа;
\vs Sir 26:4 с нею у богатого и бедного~--- сердце довольное и лице во всякое время веселое.
\vs Sir 26:5 Трех страшится сердце мое, а при четвертом я молюсь:
\vs Sir 26:6 городского злословия, возмущения черни и оболгания на смерть,~--- всё это ужасно.
\vs Sir 26:7 Болезнь сердца и печаль~--- жена, ревнивая к \bibemph{другой} жене,
\vs Sir 26:8 и бич языка ее, ко всем приражающийся.
\vs Sir 26:9 Движущееся туда и сюда воловье ярмо~--- злая жена; берущий ее~--- то же, что хватающий скорпиона.
\vs Sir 26:10 Большая досада~--- жена, преданная пьянству, и она не скроет своего срама.
\vs Sir 26:11 Наклонность женщины к блуду узнается по поднятию глаз и век ее.
\vs Sir 26:12 Над бесстыдною дочерью поставь крепкую стражу, чтобы она, улучив послабление, не злоупотребила собою.
\vs Sir 26:13 Берегись бесстыдного глаза, и не удивляйся, если он согрешит против тебя:
\vs Sir 26:14 как томимый жаждою путник открывает уста и пьет всякую близкую воду,
\vs Sir 26:15 так она сядет напротив всякого шатра и пред стрелою откроет колчан.
\vs Sir 26:16 Любезность жены усладит ее мужа, и благоразумие ее утучнит кости его.
\vs Sir 26:17 Кроткая жена~--- дар Господа, и нет цены благовоспитанной душе.
\vs Sir 26:18 Благодать на благодать~--- жена стыдливая,
\vs Sir 26:19 и нет достойной меры для воздержной души.
\vs Sir 26:20 Что солнце, восходящее на высотах Господних,
\vs Sir 26:21 то красота доброй жены в убранстве дома ее;
\vs Sir 26:22 что светильник, сияющий на святом свещнике, то красота лица ее в зрелом возрасте;
\vs Sir 26:23 что золотые столбы на серебряном основании, то прекрасные ноги ее на твердых пятах.
\rsbpar\vs Sir 26:24 От двух скорбело сердце мое, а при третьем возбуждалось во мне негодование:
\vs Sir 26:25 если воин терпит от бедности, и разумные мужи бывают в пренебрежении;
\vs Sir 26:26 и если кто обращается от праведности ко греху, Господь уготовит того на меч.
\vs Sir 26:27 Купец едва может избежать погрешности, а корчемник не спасется от греха.
\vs Sir 27:1 Многие погрешали ради маловажных вещей, и ищущий богатства отвращает глаза.
\vs Sir 27:2 Посреди скреплений камней вбивается гвоздь: так посреди продажи и купли вторгается грех.
\vs Sir 27:3 Если кто не удерживается тщательно в страхе Господнем, то скоро разорится дом его.
\vs Sir 27:4 При трясении решета остается сор: так нечистота человека~--- при рассуждении его.
\vs Sir 27:5 Глиняные сосуды испытываются в печи, а испытание человека~--- в разговоре его.
\vs Sir 27:6 Уход за деревом открывается в плоде его: т\acc{а}к в слове~--- помышления сердца человеческого.
\vs Sir 27:7 Прежде беседы не хвали человека, ибо она есть испытание людей.
\vs Sir 27:8 Если ты усердно будешь искать правды, то найдешь ее и облечешься ею, как подиром славы.
\vs Sir 27:9 Птицы слетаются к подобным себе, и истина обращается к тем, которые упражняются в ней.
\vs Sir 27:10 Как лев подстерегает добычу, так и грехи~--- делающих неправду.
\rsbpar\vs Sir 27:11 Беседа благочестивого~--- всегда мудрость, а безумный изменяется, как луна.
\vs Sir 27:12 Среди неразумных не трать времени, а проводи его постоянно среди благоразумных.
\vs Sir 27:13 Беседа глупых отвратительна, и смех их~--- в забаве грехом.
\vs Sir 27:14 Пустословие много клянущихся поднимет дыбом волосы, а спор их заткнет уши.
\vs Sir 27:15 Ссора надменных~--- кровопролитие, и брань их несносна для слуха.
\vs Sir 27:16 Открывающий тайны потерял доверие и не найдет друга по душе своей.
\vs Sir 27:17 Люби друга и будь верен ему;
\vs Sir 27:18 а если откроешь тайны его, не гонись больше за ним:
\vs Sir 27:19 ибо как человек убивает своего врага, так ты убил дружбу ближнего;
\vs Sir 27:20 и как ты выпустил бы из рук своих птицу, так ты упустил друга и не поймаешь его;
\vs Sir 27:21 не гонись за ним, ибо он далеко ушел и убежал, как серна из сети.
\vs Sir 27:22 Рану можно перевязать, и после ссоры возможно примирение;
\vs Sir 27:23 но кто открыл тайны, тот потерял надежду \bibemph{на примирение}.
\vs Sir 27:24 Кто мигает глазом, тот строит козни, и никто не удержит его от того;
\vs Sir 27:25 пред глазами твоими он будет говорить сладко и будет удивляться словам твоим,
\vs Sir 27:26 а после извратит уста свои и в словах твоих откроет соблазн;
\vs Sir 27:27 многое я ненавижу, но не столько, как его; и Господь возненавидит его.
\rsbpar\vs Sir 27:28 Кто бросает камень вверх, бросает его на свою голову, и коварный удар разделит раны.
\vs Sir 27:29 Кто роет яму, сам упадет в нее, и кто ставит сеть, сам будет уловлен ею.
\vs Sir 27:30 Кто делает зло, на того обратится оно, и он не узн\acc{а}ет, откуда оно пришло к нему;
\vs Sir 27:31 посмеяние и поношение от гордых и мщение, как лев, подстерегут его.
\vs Sir 27:32 Уловлены будут сетью радующиеся о падении благочестивых, и скорбь измождит их прежде смерти их.
\vs Sir 27:33 Злоба и гнев~--- тоже мерзости, и муж грешный будет обладаем ими.
\vs Sir 28:1 Мстительный получит отмщение от Господа, Который не забудет грехов его.
\vs Sir 28:2 Прости ближнему твоему обиду, и тогда по молитве твоей отпустятся грехи твои.
\vs Sir 28:3 Человек питает гнев к человеку, а у Господа просит прощения;
\vs Sir 28:4 к подобному себе человеку не имеет милосердия, и молится о грехах своих;
\vs Sir 28:5 сам, будучи плотию, питает злобу: кто очистит грехи его?
\vs Sir 28:6 Помни последнее и перестань враждовать; помни истление и смерть и соблюдай заповеди;
\vs Sir 28:7 помни заповеди и не злобствуй на ближнего;
\vs Sir 28:8 помни завет Всевышнего и презирай невежество.
\vs Sir 28:9 Удерживайся от ссоры~--- и ты уменьшишь грехи;
\vs Sir 28:10 ибо раздражительный человек возжжет ссору; человек грешник смутит друзей и поселит раздор между живущими в мире.
\vs Sir 28:11 Каково вещество огня, так он и возгорится;
\vs Sir 28:12 и какова сила человека, таков будет и гнев его, и по мере богатства усилится ярость его.
\vs Sir 28:13 Жаркий спор возжигает огонь, а жаркая ссора проливает кровь.
\vs Sir 28:14 Если подуешь на искру, она разгорится, а если плюнешь на нее, угаснет: то и другое выходит из уст твоих.
\rsbpar\vs Sir 28:15 Наушник и двоязычный да будут прокляты, ибо они погубили многих, живших в тишине;
\vs Sir 28:16 язык третий многих поколебал и изгонял их от народа к народу,
\vs Sir 28:17 и разорял укрепленные города и ниспровергал домы вельмож;
\vs Sir 28:18 язык третий изгнал доблестных жен и лишил их трудов их;
\vs Sir 28:19 внимающий ему не найдет покоя и не будет жить в тишине.
\vs Sir 28:20 Удар бича делает рубцы, а удар языка сокрушит кости;
\vs Sir 28:21 многие пали от острия меча, но не столько, сколько павших от языка;
\vs Sir 28:22 счастлив, кто укрылся от него, кто не испытал ярости его, кто не влачил ярма его и не связан был узами его;
\vs Sir 28:23 ибо ярмо его~--- ярмо железное, и узы его~--- узы медные,
\vs Sir 28:24 смерть лютая~--- смерть его, и самый ад лучше его.
\vs Sir 28:25 Не овладеет он благочестивыми, и не сгорят они в пламени его;
\vs Sir 28:26 оставляющие Господа впадут в него; в них возгорится он и не угаснет: он будет послан на них, как лев, и, как барс, будет истреблять их.
\vs Sir 28:27 Смотри, огради владение твое терновником,
\vs Sir 28:28 свяжи серебро твое и золото,
\vs Sir 28:29 и для слов твоих сделай вес и меру, и для уст твоих~--- дверь и запор.
\vs Sir 28:30 Берегись, чтобы не споткнуться ими и не пасть пред злоумышляющим.
\vs Sir 29:1 Кто оказывает милость, тот дает взаем ближнему, и кто поддерживает его своею рукою, тот соблюдает заповеди.
\vs Sir 29:2 Давай взаймы ближнему во время нужды его и сам в свое время возвращай ближнему.
\vs Sir 29:3 Твердо держи слово и будь верен ему~--- и ты во всякое время найдешь нужное для тебя.
\vs Sir 29:4 Многие считали заем находкою и причинили огорчение тем, которые помогли им.
\vs Sir 29:5 Доколе не получит, он будет целовать руку его и из-за денег ближнего смирит голос;
\vs Sir 29:6 а в срок отдачи он будет протягивать время и будет отвечать уныло и жаловаться на время.
\vs Sir 29:7 Если он будет в состоянии, то едва половину принесет~--- и это вменит ему в находку;
\vs Sir 29:8 а если будет не в состоянии, то заимодавец лишился своих денег и без причины приобрел себе врага в нем:
\vs Sir 29:9 он воздаст ему проклятиями и бранью и вместо почтения воздаст бесчестием.
\vs Sir 29:10 Многие по причине такого лукавства уклоняются \bibemph{от ссуды}, опасаясь напрасно потерпеть утрату.
\vs Sir 29:11 Но к бедному ты будь снисходителен и милостынею ему не медли;
\vs Sir 29:12 ради заповеди помоги бедному и в нужде его не отпускай его ни с чем.
\vs Sir 29:13 Трать серебро для брата и друга и не давай ему заржаветь под камнем на погибель;
\vs Sir 29:14 располагай сокровищем твоим по заповедям Всевышнего, и оно принесет тебе более пользы, нежели золото;
\vs Sir 29:15 заключи в кладовых твоих милостыню, и она избавит тебя от всякого несчастья:
\vs Sir 29:16 лучше крепкого щита и твердого копья она защитит тебя против врага.
\vs Sir 29:17 Добрый человек поручится за ближнего, а потерявший стыд оставит его.
\vs Sir 29:18 Не забывай благодеяний поручителя; ибо он дал душу свою за тебя.
\vs Sir 29:19 Грешник расстроит состояние поручителя, и неблагодарный в душе оставит своего избавителя.
\vs Sir 29:20 Поручительство привело в разорение многих достаточных людей и пошатнуло их, как волна морская;
\vs Sir 29:21 мужей могущественных изгнало из домов, и они блуждали между чужими народами.
\vs Sir 29:22 Грешник, принимающий на себя поручительство и ищущий корысти, впадет в тяжбу.
\vs Sir 29:23 Помогай ближнему по силе твоей и берегись, чтобы тебе не впасть \bibemph{в то же}.
\vs Sir 29:24 Главная потребность для жизни~--- вода и хлеб, и одежда и дом, прикрывающий наготу.
\vs Sir 29:25 Лучше жизнь бедного под дощатым кровом, нежели роскошные пиршества в чужих \bibemph{домах}.
\vs Sir 29:26 Будь доволен малым, как и многим.
\vs Sir 29:27 Худая жизнь~--- \bibemph{скитаться} из дома в дом, и где водворишься, не посмеешь и рта открыть;
\vs Sir 29:28 будешь подавать пищу и питье без благодарности, да и сверх того еще услышишь горькое:
\vs Sir 29:29 <<пойди сюда, пришлец, приготовь стол и, если есть что у тебя, накорми меня>>;
\vs Sir 29:30 <<удались, пришлец, ради почетного лица: брат пришел ко мне в гости, дом нужен>>.
\vs Sir 29:31 Тяжел для человека с чувством упрек за приют в доме и порицание за одолжение.
\vs Sir 30:1 Кто любит своего сына, тот пусть чаще наказывает его, чтобы впоследствии утешаться им.
\vs Sir 30:2 Кто наставляет своего сына, тот будет иметь помощь от него и среди знакомых будет хвалиться им.
\vs Sir 30:3 Кто учит своего сына, тот возбуждает зависть во враге, а пред друзьями будет радоваться о нем.
\vs Sir 30:4 Умер отец его~--- и как будто не умирал, ибо оставил по себе подобного себе;
\vs Sir 30:5 при жизни своей он смотрел на него и утешался, и при смерти своей не опечалился;
\vs Sir 30:6 для врагов он оставил в нем мстителя, а для друзей~--- воздающего благодарность.
\vs Sir 30:7 Поблажающий сыну будет перевязывать раны его, и при всяком крике его будет тревожиться сердце его.
\vs Sir 30:8 Необъезженный конь бывает упрям, а сын, оставленный на свою волю, делается дерзким.
\vs Sir 30:9 Лелей дитя, и оно устрашит тебя; играй с ним, и оно опечалит тебя.
\vs Sir 30:10 Не смейся с ним, чтобы не горевать с ним и после не скрежетать зубами своими.
\vs Sir 30:11 Не давай ему воли в юности и не потворствуй неразумию его.
\vs Sir 30:12 Нагибай выю его в юности и сокрушай рёбра его, доколе оно молодо, дабы, сделавшись упорным, оно не вышло из повиновения тебе.
\vs Sir 30:13 Учи сына твоего и трудись над ним, чтобы не иметь тебе огорчения от непристойных поступков его.
\rsbpar\vs Sir 30:14 Лучше бедняк здоровый и крепкий силами, нежели богач с изможденным телом;
\vs Sir 30:15 здоровье и благосостояние тела дороже всякого золота, и крепкое тело лучше несметного богатства;
\vs Sir 30:16 нет богатства лучше телесного здоровья, и нет радости выше радости сердечной;
\vs Sir 30:17 лучше смерть, нежели горестная жизнь или постоянно продолжающаяся болезнь.
\vs Sir 30:18 Сласти, поднесенные к сомкнутым устам, то же, что снеди, поставленные на могиле.
\vs Sir 30:19 Какая польза идолу от жертвы? он ни есть, ни обонять не может:
\vs Sir 30:20 так преследуемый от Господа,
\vs Sir 30:21 смотря глазами и стеная, подобен евнуху, который обнимает девицу и вздыхает.
\vs Sir 30:22 Не предавайся печали душею твоею и не мучь себя своею мнительностью;
\vs Sir 30:23 веселье сердца~--- жизнь человека, и радость мужа~--- долгоденствие;
\vs Sir 30:24 люби душу твою и утешай сердце твое и удаляй от себя печаль,
\vs Sir 30:25 ибо печаль многих убила, а пользы в ней нет.
\vs Sir 30:26 Ревность и гнев сокращают дни, а забота~--- прежде времени приводит старость.
\vs Sir 30:27 Открытое и доброе сердце заботится и о снедях своих.
\vs Sir 31:1 Бдительность над богатством изнуряет тело, и забота о нем отгоняет сон.
\vs Sir 31:2 Бдительная забота не дает дремать, и тяжкая болезнь отнимает сон.
\vs Sir 31:3 Потрудился богатый при умножении имуществ~--- и в покое насыщается своими благами.
\vs Sir 31:4 Потрудился бедный при недостатках в жизни~--- и в покое остается скудным.
\vs Sir 31:5 Любящий золото не будет прав, и кто гоняется за тлением, наполнится им.
\vs Sir 31:6 Многие ради золота подверглись падению, и погибель их была пред лицем их;
\vs Sir 31:7 оно~--- дерево преткновения для приносящих ему жертвы, и всякий несмысленный будет уловлен им.
\vs Sir 31:8 Счастлив богач, который оказался безукоризненным и который не гонялся за золотом.
\vs Sir 31:9 Кто он? и мы прославим его; ибо он сделал чудо в народе своем.
\vs Sir 31:10 Кто был искушаем \bibemph{золотом}~--- и остался непорочным? Да будет это в похвалу ему.
\vs Sir 31:11 Кто мог погрешить~--- и не погрешил, сделать зло~--- и не сделал?
\vs Sir 31:12 Прочно будет богатство его, и о милостынях его будет возвещать собрание.
\rsbpar\vs Sir 31:13 Когда ты сядешь за богатый стол, не раскрывай на него гортани твоей
\vs Sir 31:14 и не говори: <<много же на нем!>> Помни, что алчный глаз~--- злая вещь.
\vs Sir 31:15 Что из сотворенного завистливее глаза? Потому он плачет о всем, что видит.
\vs Sir 31:16 Куда он посмотрит, не протягивай руки, и не сталкивайся с ним в блюде.
\vs Sir 31:17 Суди о ближнем по себе и о всяком действии рассуждай.
\vs Sir 31:18 Ешь, как человек, что тебе предложено, и не пресыщайся, чтобы не возненавидели тебя;
\vs Sir 31:19 переставай \bibemph{есть} первый из вежливости и не будь алчен, чтобы не послужить соблазном;
\vs Sir 31:20 и если ты сядешь посреди многих, то не протягивай руки твоей прежде них.
\vs Sir 31:21 Немногим довольствуется человек благовоспитанный, и потому он не страдает одышкою на своем ложе.
\vs Sir 31:22 Здоровый сон бывает при умеренности желудка: встал рано, и душа его с ним;
\vs Sir 31:23 страдание бессонницею и холера и резь в животе бывают у человека ненасытного.
\vs Sir 31:24 Если ты обременил себя яствами, то встань из-за стола и отдохни.
\vs Sir 31:25 Послушай меня, сын мой, и не пренебреги мною, и впоследствии ты поймешь слова мои.
\vs Sir 31:26 Во всех делах твоих будь осмотрителен, и никакая болезнь не приключится тебе.
\vs Sir 31:27 Щедрого на хлебы будут благословлять уста, и свидетельство о доброте его~--- верно;
\vs Sir 31:28 против скупого на хлеб будет роптать город, и свидетельство о скупости его~--- справедливо.
\vs Sir 31:29 Против вина не показывай себя храбрым, ибо многих погубило вино.
\vs Sir 31:30 Печь испытывает крепость лезвия закалкою; так вино испытывает сердца гордых~--- пьянством.
\vs Sir 31:31 Вино полезно для жизни человека, если будешь пить его умеренно.
\vs Sir 31:32 Что за жизнь без вина? оно сотворено на веселие людям.
\vs Sir 31:33 Отрада сердцу и утешение душе~--- вино, умеренно употребляемое вовремя;
\vs Sir 31:34 горесть для души~--- вино, когда пьют его много, при раздражении и ссоре.
\vs Sir 31:35 Излишнее употребление вина увеличивает ярость неразумного до преткновения, умаляя крепость его и причиняя раны.
\vs Sir 31:36 На пиру за вином не упрекай ближнего и не унижай его во время его веселья;
\vs Sir 31:37 не говори ему оскорбительных слов и не обременяй его требованиями.
\vs Sir 32:1 Если поставили тебя старшим \bibemph{на пиру}, не возносись; будь между другими как один из них:
\vs Sir 32:2 позаботься о них и потом садись. И когда всё твое дело исполнишь, тогда займи твое место,
\vs Sir 32:3 чтобы порадоваться на них и за хорошее распоряжение получить венок.
\vs Sir 32:4 Разговор веди ты, старший,~--- ибо это прилично тебе,~---
\vs Sir 32:5 с основательным знанием, и не возбраняй музыки.
\vs Sir 32:6 Когда слушают, не размножай разговора и безвременно не мудрствуй.
\vs Sir 32:7 Что рубиновая печать в золотом украшении, то благозвучие музыки в пиру за вином;
\vs Sir 32:8 что смарагдовая печать в золотой оправе, то приятность песней за вкусным вином.
\vs Sir 32:9 Говори, юноша, если нужно тебе, едва слова два, когда будешь спрошен,
\vs Sir 32:10 говори главное, многое в немногих словах. Будь как знающий и, вместе, как умеющий молчать.
\vs Sir 32:11 Среди вельмож не равняйся с ними, и, когда говорит другой, ты много не говори.
\vs Sir 32:12 Грому предшествует молния, а стыдливого предваряет благорасположение.
\vs Sir 32:13 Вставай вовремя и не будь последним; поспешай домой и не останавливайся.
\vs Sir 32:14 Там забавляйся и делай, что тебе нравится; но не согрешай гордым словом.
\vs Sir 32:15 И за это благословляй Сотворившего тебя и Насыщающего тебя Своими благами.
\rsbpar\vs Sir 32:16 Боящийся Господа примет наставление, и с раннего утра обращающиеся к Нему приобретут благоволение Его.
\vs Sir 32:17 Ищущий закона насытится им, а лицемер преткнется в нем.
\vs Sir 32:18 Боящиеся Господа найдут суд и, как свет, возжгут правосудие.
\vs Sir 32:19 Человек грешный уклоняется от обличения и находит извинение, согласно желанию своему.
\vs Sir 32:20 Человек рассудительный не пренебрегает размышлением, а безрассудный и гордый не содрогается от страха и после того, как сделал что-либо без размышления.
\vs Sir 32:21 Без рассуждения не делай ничего, и когда сделаешь, не раскаивайся.
\vs Sir 32:22 Не ходи по пути, где развалины, чтобы не споткнуться о камень;
\vs Sir 32:23 не полагайся и на ровный путь; остерегайся даже детей твоих.
\vs Sir 32:24 Во всяком деле верь душе твоей: и это есть соблюдение заповедей.
\vs Sir 32:25 Верующий закону внимателен к заповедям, и надеющийся на Господа не потерпит вреда.
\vs Sir 33:1 Боящемуся Господа не приключится зла, но и в искушении Он избавит его.
\vs Sir 33:2 Мудрый муж не возненавидит закона, а притворно держащийся его~--- как корабль в бурю.
\vs Sir 33:3 Разумный человек верит закону, и закон для него верен, как ответ урима.
\vs Sir 33:4 Приготовь слово~--- и будешь выслушан; собери наставления~--- и отвечай.
\vs Sir 33:5 Колесо в колеснице~--- сердце глупого, и как вертящаяся ось~--- мысль его.
\vs Sir 33:6 Насмешливый друг то же, что ярый конь, который под всяким седоком ржет.
\rsbpar\vs Sir 33:7 Почему один день лучше другого, тогда как каждый дневной свет в году \bibemph{исходит} от солнца?
\vs Sir 33:8 Они разделены премудростью Господа; Он отличил времена и празднества:
\vs Sir 33:9 некоторые из них Он возвысил и освятил, а прочие положил в числе обыкновенных дней.
\vs Sir 33:10 И все люди из праха, и Адам был создан из земли;
\vs Sir 33:11 но по всеведению Своему Господь положил различие между ними и назначил им разные пути:
\vs Sir 33:12 одних из них благословил и возвысил, других освятил и приблизил к Себе, а иных проклял и унизил и сдвинул с места их.
\vs Sir 33:13 Как глина у горшечника в руке его и все судьбы ее в его произволе, так люди~--- в руке Сотворившего их, и Он воздает им по суду Своему.
\vs Sir 33:14 Как напротив зла~--- добро и напротив смерти~--- жизнь, так напротив благочестивого~--- грешник. Так смотри и на все дела Всевышнего: их по два, одно напротив другого.
\vs Sir 33:15 И я последний бодрственно потрудился, как подбиравший позади собирателей винограда,
\vs Sir 33:16 и по благословению Господа успел и наполнил точило, как собиратель винограда.
\vs Sir 33:17 Поймите, что я трудился не для себя одного, но для всех ищущих наставления.
\rsbpar\vs Sir 33:18 Послушайте меня, князья народа, и внимайте, начальники собрания:
\vs Sir 33:19 ни сыну, ни жене, ни брату, ни другу не давай власти над тобою при жизни твоей;
\vs Sir 33:20 и не отдавай другому имения твоего, чтобы, раскаявшись, не умолять о нем.
\vs Sir 33:21 Доколе ты жив и дыхание в тебе, не заменяй себя никем;
\vs Sir 33:22 ибо лучше, чтобы дети просили тебя, нежели тебе смотреть в руки сыновей твоих.
\vs Sir 33:23 Во всех делах твоих будь главным, и не клади пятна на честь твою.
\vs Sir 33:24 При скончании дней жизни твоей и при смерти передай наследство.
\vs Sir 33:25 Корм, палка и бремя~--- для осла; хлеб, наказание и дело~--- для раба.
\vs Sir 33:26 Занимай раба работою~--- и будешь иметь покой; ослабь руки ему~--- и он будет искать свободы.
\vs Sir 33:27 Ярмо и ремень согнут выю \bibemph{вола}, а для лукавого раба~--- узы и раны;
\vs Sir 33:28 употребляй его на работу, чтобы он не оставался в праздности, ибо праздность научила многому худому;
\vs Sir 33:29 приставь его к делу, как ему следует, и если он не будет повиноваться, наложи на него тяжкие оковы.
\vs Sir 33:30 Но ни на кого не налагай лишнего и ничего не делай без рассуждения.
\vs Sir 33:31 Если есть у тебя раб, то да будет он как ты, ибо ты приобрел его кровью;
\vs Sir 33:32 если есть у тебя раб, то поступай с ним, как с братом, ибо ты будешь нуждаться в нем, как в душе твоей;
\vs Sir 33:33 если ты будешь обижать его, и он встанет и убежит от тебя, то на какой дороге ты будешь искать его?
\vs Sir 34:1 Пустые и ложные надежды~--- у человека безрассудного, и сонные грезы окрыляют глупых.
\vs Sir 34:2 Как обнимающий тень или гонящийся за ветром, так верящий сновидениям.
\vs Sir 34:3 Сновидения совершенно то же, что подобие лица против лица.
\vs Sir 34:4 От нечистого что может быть чистого, и от ложного что может быть истинного?
\vs Sir 34:5 Гадания и приметы и сновидения~--- суета, и сердце наполняется мечтами, как у рождающей.
\vs Sir 34:6 Если они не будут посланы от Всевышнего для вразумления, не прилагай к ним сердца твоего.
\vs Sir 34:7 Сновидения ввели многих в заблуждение, и надеявшиеся на них подверглись падению.
\vs Sir 34:8 Закон исполняется без обмана, и мудрость в устах верных совершается.
\vs Sir 34:9 Человек ученый знает много, и многоопытный выскажет знание.
\vs Sir 34:10 Кто не имел опытов, тот мало знает; а кто странствовал, тот умножил знание.
\vs Sir 34:11 Многое я видел в моем странствовании, и я знаю больше, нежели сколько говорю.
\vs Sir 34:12 Много раз был я в опасности смерти, и спасался при помощи \bibemph{опыта}.
\vs Sir 34:13 Дух боящихся Господа поживет, ибо надежда их~--- на Спасающего их.
\vs Sir 34:14 Боящийся Господа ничего не устрашится и не убоится, ибо Он~--- надежда его.
\vs Sir 34:15 Блаженна душа боящегося Господа! кем он держится, и кто опора его?
\vs Sir 34:16 Очи Господа~--- на любящих Его. Он~--- могущественная защита и крепкая опора, покров от зноя и покров от полуденного жара, охранение от преткновения и защита от падения;
\vs Sir 34:17 Он возвышает душу и просвещает очи, дает врачевство, жизнь и благословение.
\rsbpar\vs Sir 34:18 Кто приносит жертву от неправедного \bibemph{стяжания}, того приношение насмешливое, и дары беззаконных неблагоугодны;
\vs Sir 34:19 не благоволит Всевышний к приношениям нечестивых и множеством жертв не умилостивляется о грехах их.
\vs Sir 34:20 Что заколающий на жертву сына пред отцем его, то приносящий жертву из имения бедных.
\vs Sir 34:21 Хлеб нуждающихся есть жизнь бедных: отнимающий его есть кровопийца.
\vs Sir 34:22 Убивает ближнего, кто отнимает у него пропитание, и проливает кровь, кто лишает наемника платы.
\vs Sir 34:23 Когда один строит, а другой разрушает, то что они получат для себя кроме утомления?
\vs Sir 34:24 Когда один молится, а другой проклинает, чей голос услышит Владыка?
\vs Sir 34:25 Когда кто омывается от осквернения мертвым и опять прикасается к нему, какая польза от его омовения?
\vs Sir 34:26 Так человек, который постится за грехи свои и опять идет и делает то же самое: кто услышит молитву его? и какую пользу получит он оттого, что смирялся?
\vs Sir 35:1 Кто соблюдает закон, тот умножает приношения; кто держится заповедей, тот приносит жертву спасения.
\vs Sir 35:2 Кто воздает благодарность, тот приносит семидал; а подающий милостыню приносит жертву хвалы.
\vs Sir 35:3 Благоугождение Господу~--- отступление от зла, и умилостивление \bibemph{Его}~--- уклонение от неправды.
\vs Sir 35:4 Не являйся пред лице Господа с пустыми руками, ибо всё это~--- по заповеди.
\vs Sir 35:5 Приношение праведного утучняет алтарь, и благоухание его~--- пред Всевышним;
\vs Sir 35:6 жертва праведного мужа благоприятна, и память о ней незабвенна будет.
\vs Sir 35:7 С веселым оком прославляй Господа и не умаляй начатков трудов твоих;
\vs Sir 35:8 при всяком даре имей лице веселое и в радости посвящай десятину.
\vs Sir 35:9 Давай Всевышнему по даянию Его, и с веселым оком~--- по мере приобретения рукою твоею,
\vs Sir 35:10 ибо Господь есть воздаятель и воздаст тебе всемеро.
\vs Sir 35:11 Не уменьшай даров, ибо Он не примет их: и не надейся на неправедную жертву,
\vs Sir 35:12 ибо Господь есть судия, и нет у Него лицеприятия:
\vs Sir 35:13 Он не уважит лица пред бедным и молитву обиженного услышит;
\vs Sir 35:14 Он не презрит моления сироты, ни вдовы, когда она будет изливать прошение \bibemph{свое}.
\vs Sir 35:15 Не слезы ли вдовы льются по щекам, и не вопиет ли она против того, кто вынуждает их?
\vs Sir 35:16 Служащий \bibemph{Богу} будет принят с благоволением, и молитва его дойдет до облаков.
\vs Sir 35:17 Молитва смиренного проникнет сквозь облака, и он не утешится, доколе она не приблизится \bibemph{к Богу},
\vs Sir 35:18 и не отступит, доколе Всевышний не призрит и не рассудит справедливо и не произнесет решения.
\vs Sir 35:19 И Господь не замедлит и не потерпит, доколе не сокрушит чресл немилосердых;
\vs Sir 35:20 Он будет воздавать отмщение и народам, доколе не истребит сонма притеснителей и не сокрушит скипетров неправедных,
\vs Sir 35:21 доколе не воздаст человеку по делам его, и за дела людей~--- по намерениям их,
\vs Sir 35:22 доколе не совершит суда над народом Своим и не обрадует их Своею милостью.
\vs Sir 35:23 Благовременна милость во время скорби, как дождевые облака во время засухи.
\vs Sir 36:1 Помилуй нас, Владыко, Боже всех, и призри,
\vs Sir 36:2 и наведи на все народы страх Твой.
\vs Sir 36:3 Воздвигни руку Твою на чужие народы, и да позн\acc{а}ют они могущество Твое.
\vs Sir 36:4 Как пред ними Ты явил святость Твою в нас, так пред нами яви величие Твое в них,~---
\vs Sir 36:5 и да познают они Тебя, как мы познали, что нет Бога, кроме Тебя, Господи.
\vs Sir 36:6 Возобнови знамения и сотвори новые чудеса;
\vs Sir 36:7 прославь руку и правую мышцу \bibemph{Твою}; воздвигни ярость и пролей гнев;
\vs Sir 36:8 истреби противника и уничтожь врага;
\vs Sir 36:9 ускори время и вспомни клятву, и да возвестят о великих делах Твоих.
\vs Sir 36:10 Яростью огня да будет истреблен убегающий \bibemph{от меча}, и угнетающие народ Твой да найдут погибель.
\vs Sir 36:11 Сокруши головы начальников вражеских, которые говорят: <<никого нет, кроме нас!>>
\vs Sir 36:12 Собери все колена Иакова и соделай их наследием Твоим, как было сначала.
\vs Sir 36:13 Помилуй, Господи, народ, названный по имени Твоему, и Израиля, которого Ты нарек первенцем.
\vs Sir 36:14 Умилосердись над городом святыни Твоей, над Иерусалимом, местом покоя Твоего.
\vs Sir 36:15 Наполни Сион хвалою обетований Твоих, и Твоею славою~--- народ Твой.
\vs Sir 36:16 Даруй свидетельство тем, которые от начала были достоянием Твоим, и воздвигни пророчества от имени Твоего.
\vs Sir 36:17 Даруй награду надеющимся на Тебя, и да веруют пророкам Твоим.
\vs Sir 36:18 Услышь, Господи, молитву рабов Твоих, по благословению Аарона, о народе Твоем,~---
\vs Sir 36:19 и познают все живущие на земле, что Ты~--- Господь, Бог веков.
\rsbpar\vs Sir 36:20 Желудок принимает в себя всякую пищу, но пища пищи лучше:
\vs Sir 36:21 гортань отличает пищу из дичи, так разумное сердце~--- слова ложные.
\vs Sir 36:22 Лукавое сердце причинит печаль, но человек многоопытный воздаст ему.
\vs Sir 36:23 Женщина примет всякого мужа, но девица девицы лучше:
\vs Sir 36:24 красота жены веселит лице и всего вожделеннее для мужа;
\vs Sir 36:25 если есть на языке ее приветливость и кротость, то муж ее выходит из ряда сынов человеческих.
\vs Sir 36:26 Приобретающий жену полагает начало стяжанию, приобретает соответственно ему помощника, опору спокойствия его.
\vs Sir 36:27 Где нет ограды, \bibemph{там} расхитится имение; а у кого нет жены, тот будет вздыхать скитаясь:
\vs Sir 36:28 ибо кто поверит вооруженному разбойнику, скитающемуся из города в город?
\vs Sir 36:29 Так и человеку, не имеющему оседлости и останавливающемуся для ночлега там, где он запоздает.
\vs Sir 37:1 Всякий друг может сказать: <<и я подружился с ним>>. Но бывает друг по имени только другом.
\vs Sir 37:2 Не есть ли это скорбь до смерти, когда приятель и друг обращается во врага?
\vs Sir 37:3 О, злая мысль! откуда вторглась ты, чтобы покрыть землю коварством?
\vs Sir 37:4 Приятель радуется при веселии друга, а во время скорби его будет против него.
\vs Sir 37:5 Приятель помогает другу в трудах его ради чрева, а в случае войны возьмется за щит.
\vs Sir 37:6 Не забывай друга в душе твоей и не забывай его в имении твоем.
\vs Sir 37:7 Всякий советник хвалит \bibemph{свой} совет, но иной советует в свою пользу;
\vs Sir 37:8 от советника охраняй душу твою и наперед узнай, что ему нужно; ибо, может быть, он будет советовать для самого себя;
\vs Sir 37:9 может быть, он бросит на тебя жребий и скажет тебе: <<путь твой хорош>>; а сам станет напротив тебя, чтобы посмотреть, что случится с тобою.
\vs Sir 37:10 Не советуйся с недоброжелателем твоим и от завистников твоих скрывай намерения.
\vs Sir 37:11 Не советуйся с женою о сопернице ее и с боязливым~--- о войне, с продавцом~--- о мене, с покупщиком~--- о продаже, с завистливым~--- о благодарности,
\vs Sir 37:12 с немилосердым~--- о благотворительности, с ленивым~--- о всяком деле,
\vs Sir 37:13 с годовым наемником~--- об окончании работы, с ленивым рабом~--- о большой работе:
\vs Sir 37:14 не полагайся на таких ни при каком совещании,
\vs Sir 37:15 но обращайся всегда только с мужем благочестивым, о котором узн\acc{а}ешь, что он соблюдает заповеди Господни,
\vs Sir 37:16 который своею душею~--- по душе тебе и, в случае падения твоего, поскорбит вместе с тобою.
\vs Sir 37:17 Держись совета сердца твоего, ибо нет никого для тебя вернее его;
\vs Sir 37:18 душа человека иногда более скажет, нежели семь наблюдателей, сидящих на высоком месте для наблюдения.
\vs Sir 37:19 Но при всем этом молись Всевышнему, чтобы Он управил путь твой в истине.
\rsbpar\vs Sir 37:20 Начало всякого дела~--- размышление, а прежде всякого действия~--- совет.
\vs Sir 37:21 Выражение сердечного изменения~--- лице. Четыре состояния выражаются на нем: добро и зло, жизнь и смерть, а господствует всегда язык.
\vs Sir 37:22 Иной человек искусен и многих учит, а для своей души бесполезен.
\vs Sir 37:23 Иной ухищряется в речах, а \bibemph{бывает} ненавистен,~--- такой останется без всякого пропитания;
\vs Sir 37:24 ибо не дана ему от Господа благодать, и он лишен всякой мудрости.
\vs Sir 37:25 Иной мудр для души своей, и плоды знания на устах его верны.
\vs Sir 37:26 Мудрый муж поучает народ свой, и плоды знания его верны.
\vs Sir 37:27 Мудрый муж будет изобиловать благословением, и все видящие его будут называть его блаженным.
\vs Sir 37:28 Жизнь человека определяется числом дней, а дни Израиля бесчисленны.
\vs Sir 37:29 Мудрый приобретет доверие у своего народа, и имя его будет жить вовек.
\rsbpar\vs Sir 37:30 Сын мой! в продолжение жизни испытывай твою душу и наблюдай, что для нее вредно, и не давай ей того;
\vs Sir 37:31 ибо не всё полезно для всех, и не всякая душа ко всему расположена.
\vs Sir 37:32 Не пресыщайся всякою сластью и не бросайся на разные снеди,
\vs Sir 37:33 ибо от многоядения бывает болезнь, и пресыщение доводит до холеры;
\vs Sir 37:34 от пресыщения многие умерли, а воздержный прибавит себе жизни.
\vs Sir 38:1 Почитай врача честью по надобности в нем, ибо Господь создал его,
\vs Sir 38:2 и от Вышнего~--- врачевание, и от царя получает он дар.
\vs Sir 38:3 Знание врача возвысит его голову, и между вельможами он будет в почете.
\vs Sir 38:4 Господь создал из земли врачевства, и благоразумный человек не будет пренебрегать ими.
\vs Sir 38:5 Не от дерева ли вода сделалась сладкою, чтобы познана была сила Его?
\vs Sir 38:6 Для того Он и дал людям знание, чтобы прославляли Его в чудных делах Его:
\vs Sir 38:7 ими он врачует \bibemph{человека} и уничтожает болезнь его.
\vs Sir 38:8 Приготовляющий лекарства делает из них смесь, и занятия его не оканчиваются, и чрез него бывает благо на лице земли.
\vs Sir 38:9 Сын мой! в болезни твоей не будь небрежен, но молись Господу, и Он исцелит тебя.
\vs Sir 38:10 Оставь греховную жизнь и исправь руки твои, и от всякого греха очисти сердце.
\vs Sir 38:11 Вознеси благоухание и из семидала памятную жертву и сделай приношение тучное, как бы уже умирающий;
\vs Sir 38:12 и дай место врачу, ибо и его создал Господь, и да не удаляется он от тебя, ибо он нужен.
\vs Sir 38:13 В иное время и в их руках бывает успех;
\vs Sir 38:14 ибо и они молятся Господу, чтобы Он помог им подать \bibemph{больному} облегчение и исцеление к продолжению жизни.
\vs Sir 38:15 Но кто согрешает пред Сотворившим его, да впадет в руки врача!
\vs Sir 38:16 Сын мой! над умершим пролей слезы и, как бы подвергшийся жестокому несчастию, начни плач; прилично облеки тело его и не пренебреги погребением его;
\vs Sir 38:17 горький да будет плач и рыдание теплое, и продолжи сетование о нем, по достоинству его, день или два, для избежания осуждения, и тогда утешься от печали;
\vs Sir 38:18 ибо от печали бывает смерть, и печаль сердечная истощит силу.
\vs Sir 38:19 С несчастьем пребывает и печаль, и жизнь нищего тяжела для сердца.
\vs Sir 38:20 Не предавай сердца твоего печали; отдаляй ее \bibemph{от себя}, вспоминая о конце.
\vs Sir 38:21 Не забывай о сем, ибо нет возвращения; и ему ты не принесешь пользы, а себе повредишь.
\vs Sir 38:22 <<Вспоминай о приговоре надо мною, потому что он также и над тобою; мне вчера, а тебе сегодня>>.
\vs Sir 38:23 С упокоением умершего успокой и память о нем, и утешься о нем по исходе души его.
\rsbpar\vs Sir 38:24 Мудрость книжная приобретается в благоприятное время досуга, и кто мало имеет своих занятий, может приобрести мудрость.
\vs Sir 38:25 Как может сделаться мудрым тот, кто правит плугом и хвалится бичом, гоняет волов и занят работами их, и которого разговор \bibemph{только} о молодых волах?
\vs Sir 38:26 Сердце его занято тем, чтобы проводить борозды, и забота его~--- о корме для телиц.
\vs Sir 38:27 Так и всякий плотник и зодчий, который проводит ночь, как день: кто занимается резьбою, того прилежание в том, чтобы оразнообразить форму;
\vs Sir 38:28 сердце свое он устремляет на то, чтобы изображение было похоже, и забота его~--- о том, чтоб окончить дело в совершенстве.
\vs Sir 38:29 Так и ковач, который сидит у наковальни и думает об изделии из железа: дым от огня изнуряет его тело, и с жаром от печи борется он;
\vs Sir 38:30 звук молота оглушает его слух, и глаза его устремлены на модель сосуда;
\vs Sir 38:31 сердце его устремлено на окончание дела, и попечение его~--- о том, чтобы отделать его в совершенстве.
\vs Sir 38:32 Так и горшечник, который сидит над своим делом и ногами своими вертит колесо,
\vs Sir 38:33 который постоянно в заботе о деле своем и у которого исчислена вся работа его:
\vs Sir 38:34 рукою своею он дает форму глине, а ногами умягчает ее жесткость;
\vs Sir 38:35 он устремляет сердце к тому, чтобы хорошо окончить сосуд, и забота его~--- о том, чтоб очистить печь.
\vs Sir 38:36 Все они надеются на свои руки, и каждый умудряется в своем деле;
\vs Sir 38:37 без них ни город не построится, ни жители не населятся и не будут жить в нем;
\vs Sir 38:38 и однако ж они в собрание не приглашаются, на судейском седалище не сидят и не рассуждают о судебных постановлениях, не произносят оправдания и осуждения и не занимаются притчами;
\vs Sir 38:39 но поддерживают быт житейский, и молитва их~--- об успехе художества их.
\vs Sir 39:1 Только тот, кто посвящает свою душу размышлению о законе Всевышнего, будет искать мудрости всех древних и упражняться в пророчествах:
\vs Sir 39:2 он будет замечать сказания мужей именитых и углубляться в тонкие обороты притчей;
\vs Sir 39:3 будет исследовать сокровенный смысл изречений и заниматься загадками притчей.
\vs Sir 39:4 Он будет проходить служение среди вельмож и являться пред правителем;
\vs Sir 39:5 будет путешествовать по земле чужих народов, ибо испытал доброе и злое между людьми.
\vs Sir 39:6 Сердце свое он направит к тому, чтобы с раннего утра обращаться к Господу, сотворившему его, и будет молиться пред Всевышним; откроет в молитве уста свои и будет молиться о грехах своих.
\vs Sir 39:7 Если Господу великому угодно будет, он исполнится духом разума,
\vs Sir 39:8 будет источать слова мудрости своей и в молитве прославлять Господа;
\vs Sir 39:9 благоуправит свою волю и ум и будет размышлять о тайнах Господа;
\vs Sir 39:10 он покажет мудрость своего учения и будет хвалиться законом завета Господня.
\vs Sir 39:11 Многие будут прославлять знание его, и он не будет забыт вовек;
\vs Sir 39:12 память о нем не погибнет, и имя его будет жить в роды родов.
\vs Sir 39:13 Народы будут прославлять его мудрость, и общество будет возвещать хвалу его;
\vs Sir 39:14 доколе будет жить, он приобретет б\acc{о}льшую славу, нежели тысячи; а когда почиет, увеличит ее.
\rsbpar\vs Sir 39:15 Еще размыслив, расскажу, ибо я полон, как луна в полноте своей.
\vs Sir 39:16 Выслушайте меня, благочестивые дети, и растите, как роза, растущая на поле при потоке;
\vs Sir 39:17 издавайте благоухание, как ливан;
\vs Sir 39:18 цветите, как лилия, распространяйте благовоние и пойте песнь;
\vs Sir 39:19 благословляйте Господа во всех делах; величайте имя Его и прославляйте Его хвалою Его,
\vs Sir 39:20 песнями уст и гуслями и, прославляя, говорите так:
\vs Sir 39:21 все дела Господа весьма благотворны, и всякое повеление Его в свое время исполнится;
\vs Sir 39:22 и нельзя сказать: <<что это? для чего это?>>, ибо все в свое время откроется.
\vs Sir 39:23 По слову Его стала вода, как стог, и по изречению уст Его \bibemph{явились} вместилища вод.
\vs Sir 39:24 В повелениях Его~--- всё Его благоволение, и никто не может умалить спасительность их.
\vs Sir 39:25 Пред Ним дела всякой плоти, и невозможно укрыться от очей Его.
\vs Sir 39:26 Он прозирает из века в век, и ничего нет дивного пред Ним.
\vs Sir 39:27 Нельзя сказать: <<что это? для чего это?>>, ибо все создано для своего употребления.
\vs Sir 39:28 Благословение Его покрывает, как река, и, как потоп, напояет сушу.
\vs Sir 39:29 Но и гнев Его испытывают народы, как некогда Он превратил воды в солончаки.
\vs Sir 39:30 Пути Его для святых прямы, а для беззаконных они~--- преткновения.
\vs Sir 39:31 От начала для добрых создано доброе, как для грешников~--- злое.
\vs Sir 39:32 Главное из всех потребностей для жизни человека~--- вода, огонь, железо, соль, пшеничная мука, мед, молоко, виноградный сок, масло и одежда:
\vs Sir 39:33 все это благочестивым служит в пользу, а грешникам может обратиться во вред.
\vs Sir 39:34 Есть ветры, которые созданы для отмщения и в ярости своей усиливают удары свои,
\vs Sir 39:35 во время устремления своего изливают силу и удовлетворяют ярости Сотворившего их.
\vs Sir 39:36 Огонь и град, голод и смерть~--- все это создано для отмщения;
\vs Sir 39:37 зубы зверей, и скорпионы, и змеи, и меч, мстящий нечестивым погибелью,~---
\vs Sir 39:38 обрадуются повелению Его и готовы будут на земле, когда потребуются, и в свое время не преступят слова Его.
\vs Sir 39:39 Посему я с самого начала решил, обдумал и оставил в писании,
\vs Sir 39:40 что все дела Господа прекрасны, и Он дарует все потребное в свое время;
\vs Sir 39:41 и нельзя сказать: <<это хуже того>>, ибо все в свое время признано будет хорошим.
\vs Sir 39:42 Итак, всем сердцем и устами пойте и благословляйте имя Господа.
\vs Sir 40:1 Много трудов предназначено каждому человеку, и тяжело иго на сынах Адама со дня исхода из чрева матери их до дня возвращения к матери всех.
\vs Sir 40:2 Мысль об ожидаемом и день смерти производит в них размышления и страх сердца.
\vs Sir 40:3 От сидящего на славном престоле и до поверженного на земле и во прахе,
\vs Sir 40:4 от носящего порфиру и венец и до одетого в рубище,~---
\vs Sir 40:5 \bibemph{у всякого} досада и ревность, и смущение, и беспокойство, и страх смерти, и негодование, и распря, и во время успокоения на ложе ночной сон расстраивает ум его.
\vs Sir 40:6 Мало, почти совсем не имеет он покоя, и потому и во сне он, как днем, на страже:
\vs Sir 40:7 будучи смущен сердечными своими мечтами, как бежавший с поля брани, во время безопасности своей он пробуждается и не может надивиться, что ничего не было страшного.
\vs Sir 40:8 Хотя \bibemph{это бывает} со всякою плотью, от человека до скота, но у грешников в семь крат более сего.
\vs Sir 40:9 Смерть, убийство, ссора, меч, бедствия, голод, сокрушение и удары,~---
\vs Sir 40:10 все это~--- для беззаконных; и потоп был для них.
\vs Sir 40:11 Все, что от земли, обращается в землю, и что из воды, возвращается в море.
\vs Sir 40:12 Всякий подарок и несправедливость будут истреблены, а верность будет стоять вовек.
\vs Sir 40:13 Имения неправедных, как поток, иссохнут и, как сильный гром при проливном дожде, прогремят.
\vs Sir 40:14 Кто открывает руку, тот бывает весел; а преступники вконец погибнут.
\vs Sir 40:15 Потомки нечестивых не умножат ветвей, и нечистые корни~--- на утесистой скале:
\vs Sir 40:16 осока при всякой воде и на берегу реки скашивается прежде всякой другой травы.
\vs Sir 40:17 Благотворительность, как рай, полна благословений, и милостыня пребывает вовек.
\vs Sir 40:18 Жизнь довольного своею участью \bibemph{и} труженика сладостна; но превосходит обоих тот, кто находит сокровище.
\vs Sir 40:19 Дети и построение города увековечивают имя, но превосходнее того и другого считается безукоризненная жена.
\vs Sir 40:20 Вино и музыка веселят сердце, но лучше того и другого~--- любовь к мудрости.
\vs Sir 40:21 Свирель и гусли делают приятным пение, но лучше их~--- приятный язык.
\vs Sir 40:22 Приятность и красота вожделенны для очей твоих, но более той и другой~--- зелень посева.
\vs Sir 40:23 Друг и приятель сходятся по временам, но жена с мужем~--- всегда.
\vs Sir 40:24 Братья и покровители~--- во время скорби, но вернее тех и других спасает милостыня.
\vs Sir 40:25 Золото и серебро утверждают стопы, но надежнее того и другого признаётся \bibemph{добрый} совет.
\vs Sir 40:26 Богатство и сила возвышают сердце, но выше того~--- страх Господень:
\vs Sir 40:27 в страхе Господнем нет недостатка, и нет надобности искать при нем помощи;
\vs Sir 40:28 страх Господень~--- как благословенный рай, и облекает его всякою славою.
\rsbpar\vs Sir 40:29 Сын мой! не живи жизнью нищенскою: лучше умереть, нежели просить милостыни.
\vs Sir 40:30 Кто засматривается на чужой стол, того жизнь~--- не жизнь: он унижает душу свою чужими яствами;
\vs Sir 40:31 но человек разумный и благовоспитанный предостережет себя от того.
\vs Sir 40:32 В устах бесстыдного сладким покажется прошение милостыни, но в утробе его огонь возгорится.
\vs Sir 41:1 О, смерть! как горько воспоминание о тебе для человека, который спокойно живет в своих владениях,
\vs Sir 41:2 для человека, который ничем не озабочен и во всем счастлив и еще в силах принимать пищу.
\vs Sir 41:3 О, смерть! отраден твой приговор для человека, нуждающегося и изнемогающего в силах,
\vs Sir 41:4 для престарелого и обремененного заботами обо всем, для не имеющего надежды и потерявшего терпение.
\vs Sir 41:5 Не бойся смертного приговора: вспомни о предках твоих и потомках. Это приговор от Господа над всякою плотью.
\vs Sir 41:6 Итак, для чего ты отвращаешься от того, что благоугодно Всевышнему? десять ли, сто ли, или тысяча лет,~---
\vs Sir 41:7 в аде нет исследования о \bibemph{времени} жизни.
\vs Sir 41:8 Дети грешников бывают дети отвратительные и общаются с нечестивыми.
\vs Sir 41:9 Наследие детей грешников погибнет, и вместе с племенем их будет распространяться бесславие.
\vs Sir 41:10 Нечестивого отца будут укорять дети, потому что за него они терпят бесславие.
\vs Sir 41:11 Горе вам, люди нечестивые, которые оставили закон Бога Всевышнего!
\vs Sir 41:12 Когда вы рождаетесь, то рождаетесь на проклятие; и когда умираете, то получаете в удел свой проклятие.
\vs Sir 41:13 Все, что из земли, возвратится в землю: так нечестивые~--- от проклятия в погибель.
\vs Sir 41:14 Плач людей бывает о телах их, но грешников и имя недоброе изгладится.
\vs Sir 41:15 Заботься об имени, ибо оно пребудет с тобою долее, нежели многие тысячи золота:
\vs Sir 41:16 дням доброй жизни есть число, но доброе имя пребывает вовек.
\rsbpar\vs Sir 41:17 Соблюдайте, дети, наставление в мире; а сокрытая мудрость и сокровище невидимое~--- какая в них польза?
\vs Sir 41:18 Лучше человек, скрывающий свою глупость, нежели человек, скрывающий свою мудрость.
\vs Sir 41:19 Итак, стыдитесь того, о чем я скажу,
\vs Sir 41:20 ибо не всякую стыдливость хорошо соблюдать и не всё всеми одобряется по истине.
\vs Sir 41:21 Стыдитесь пред отцом и матерью блуда, пред начальником и властелином~--- лжи;
\vs Sir 41:22 пред судьею и князем~--- преступления, пред собранием и народом~--- беззакония;
\vs Sir 41:23 пред товарищем и другом~--- неправды, пред соседями~--- кражи:
\vs Sir 41:24 стыдитесь сего и пред истиною Бога и завета Его. Стыдись и облокачивания на стол, обмана при займе и отдаче;
\vs Sir 41:25 стыдись молчания пред приветствующими, смотрения на распутную женщину, отвращения лица от родственника,
\vs Sir 41:26 отнятия доли и дара, помысла на замужнюю женщину, ухаживания за своею служанкою,
\vs Sir 41:27 и не подходи к постели ее;
\vs Sir 41:28 пред друзьями стыдись слов укорительных,~--- и после того, как ты дал, не попрекай,~---
\vs Sir 41:29 повторения слухов и разглашения слов тайных. И будешь истинно стыдлив и приобретешь благорасположение всякого человека.
\vs Sir 42:1 Не стыдись вот чего, и из лицеприятия не греши:
\vs Sir 42:2 не стыдись \bibemph{точного исполнения} закона Всевышнего и завета, и суда, чтобы оказать правосудие нечестивому,
\vs Sir 42:3 спора между товарищем и посторонними и предоставления наследства друзьям,
\vs Sir 42:4 точности в весах и мерах,~--- много ли, мало ли приобретаешь,~---
\vs Sir 42:5 беспристрастия в купле и продаже и строгого воспитания детей, и~--- окровавить ребро худому рабу.
\vs Sir 42:6 При худой жене хорошо иметь печать, и, где много рук, там запирай.
\vs Sir 42:7 Если что выдаешь, \bibemph{выдавай} счетом и весом и делай всякую выдачу и прием по записи.
\vs Sir 42:8 Не стыдись вразумлять неразумного и глупого, и престарелого, состязающегося с молодыми: и будешь истинно благовоспитанным и заслужишь одобрение от всякого человека.
\rsbpar\vs Sir 42:9 Дочь для отца~--- тайная постоянная забота, и попечение о ней отгоняет сон: в юности ее~--- как бы не отцвела, а в замужестве~--- как бы не опротивела;
\vs Sir 42:10 в девстве~--- как бы не осквернилась и не сделалась беременною в отцовском доме, в замужестве~--- чтобы не нарушила супружеской верности и в сожительстве с мужем не осталась бесплодною.
\vs Sir 42:11 Над бесстыдною дочерью усиль надзор, чтобы она не сделала тебя посмешищем для врагов, притчею в городе и упреком в народе и не осрамила тебя пред обществом.
\vs Sir 42:12 Не смотри на красоту человека и не сиди среди женщин:
\vs Sir 42:13 ибо как из одежд выходит моль, так от женщины~--- лукавство женское.
\vs Sir 42:14 Лучше злой мужчина, нежели ласковая женщина,~--- женщина, которая стыдит до поношения.
\rsbpar\vs Sir 42:15 Воспомяну теперь о делах Господа и расскажу о том, что я видел. По слову Господа \bibemph{явились} дела Его:
\vs Sir 42:16 сияющее солнце смотрит на все, и все дело его полно славы Господней.
\vs Sir 42:17 И святым не предоставил Господь провозвестить о всех чудесах Его, которые утвердил Господь Вседержитель, чтобы вселенная стояла твердо во славу Его.
\vs Sir 42:18 Он проникает бездну и сердце и видит все изгибы их; ибо Господь знает всякое в\acc{е}дение и прозирает в знамения века,
\vs Sir 42:19 возвещая прошедшее и будущее и открывая следы сокровенного;
\vs Sir 42:20 не минует Его никакое помышление и не утаится от Него ни одно слово.
\vs Sir 42:21 Он устроил великие дела Своей премудрости и пребывает прежде века и вовек;
\vs Sir 42:22 Он не увеличился и не умалился и не требовал никакого советника.
\vs Sir 42:23 Как вожделенны все дела Его, хотя мы можем видеть их как только искры!
\vs Sir 42:24 Все они живут и пребывают вовек для всяких потребностей, и все повинуются \bibemph{Ему}.
\vs Sir 42:25 Все они~--- вдвойне, одно напротив другого, и ничего не сотворил Он несовершенным:
\vs Sir 42:26 одно поддерживает благо другого,~--- и кто насытится зрением славы Его?
\vs Sir 43:1 Величие высоты, твердь чистоты, вид неба в славном явлении!
\vs Sir 43:2 Солнце, когда оно является, возвещает о них при восходе: чудное создание, дело Всевышнего!
\vs Sir 43:3 В полдень свой оно иссушает землю, и пред жаром его кто устоит?
\vs Sir 43:4 Распаляют горн для работ плавильных, но втрое сильнее солнце палит горы: дыша пламенем огня и блистая лучами, оно ослепляет глаза.
\vs Sir 43:5 Велик Господь, Который сотворил его, и по слову Его оно поспешно пробегает путь свой.
\vs Sir 43:6 И луна всем в свое время служит указанием времен и знамением века:
\vs Sir 43:7 от луны~--- указание праздника; свет ее умаляется по достижении ею полноты;
\vs Sir 43:8 месяц называется по имени ее; она дивно возрастает в своем изменении;
\vs Sir 43:9 это~--- глава вышних строев; она сияет на тверди небесной;
\vs Sir 43:10 красота неба, слава звезд, блестящее украшение, владыка на высотах!
\vs Sir 43:11 По слову Святаго \bibemph{звезды} стоят по чину и не устают на страже своей.
\vs Sir 43:12 Взгляни на радугу, и прославь Сотворившего ее: прекрасна она в сиянии своем!
\vs Sir 43:13 Величественным кругом своим она обнимает небо; руки Всевышнего распростерли ее.
\vs Sir 43:14 Повелением Его скоро сыплется снег, и быстро сверкают молнии суда Его.
\vs Sir 43:15 Отверзаются сокровищницы и вылетают из них облака, как птицы.
\vs Sir 43:16 Могуществом Своим Он укрепляет облака, и разбиваются камни града;
\vs Sir 43:17 от взора Его потрясаются горы, и по изволению Его веет южный ветер.
\vs Sir 43:18 Голос грома Его приводит в трепет землю, и северная буря и вихрь.
\vs Sir 43:19 Он сыплет снег подобно летящим вниз крылатым, и ниспадение его~--- как опускающаяся саранча;
\vs Sir 43:20 красоте белизны его удивляется глаз, и ниспадению его изумляется сердце.
\vs Sir 43:21 И как соль, рассыпает Он по земле иней, который, замерзая, делается остроконечным.
\vs Sir 43:22 Подует северный холодный ветер,~--- и из воды делается лед: он расстилается на всяком вместилище вод, и вода облекается как бы в латы;
\vs Sir 43:23 поядает горы, и пожигает пустыню, и, как огонь, опаляет траву.
\vs Sir 43:24 Но скорым исцелением всему служит туман; появляющаяся роса прохлаждает от зноя.
\vs Sir 43:25 Повелением Своим Господь укрощает бездну и насаждает на ней острова.
\vs Sir 43:26 Плавающие по морю рассказывают об опасностях на нем, и мы дивимся тому, что слышим ушами нашими:
\vs Sir 43:27 ибо там необычайные и чудные дела, разнообразие всяких животных, роды чудовищ.
\vs Sir 43:28 Чрез Него все успешно достигает своего назначения, и все держится словом Его.
\vs Sir 43:29 Многое можем мы сказать, и, однако же, не постигнем Его, и конец слов: Он есть всё.
\vs Sir 43:30 Где возьмем силу, чтобы прославить Его? ибо Он превыше всех дел Своих.
\vs Sir 43:31 Страшен Господь и весьма велик, и дивно могущество Его!
\vs Sir 43:32 Прославляя Господа, превозносите Его, сколько можете, но и затем Он будет превосходнее;
\vs Sir 43:33 и, величая Его, прибавьте силы: но не труд\acc{и}тесь, ибо не постигнете.
\vs Sir 43:34 Кто видел Его, и объяснит? и кто прославит Его, как Он есть?
\vs Sir 43:35 Много сокрыто, что гораздо больше сего; ибо мы видим малую часть дел Его.
\vs Sir 43:36 Всё сотворил Господь, и благочестивым даровал мудрость.
\vs Sir 44:1 Теперь восхвалим славных мужей и отцов нашего рода:
\vs Sir 44:2 много славного Господь являл \bibemph{чрез них}, величие Свое от века;
\vs Sir 44:3 это были господствующие в царствах своих и мужи, именитые силою; они давали разумные советы, возвещали в пророчествах;
\vs Sir 44:4 они были руководителями народа при совещаниях и в книжном обучении.
\vs Sir 44:5 Мудрые слова были в учении их; они изобрели музыкальные строи и гимны предали писанию;
\vs Sir 44:6 люди богатые, одаренные силою, они мирно обитали в жилищах своих.
\vs Sir 44:7 Все они были уважаемы между племенами своими и во дни свои были славою.
\vs Sir 44:8 Есть между ними такие, которые оставили по себе имя для возвещения хвалы их,~--- и есть такие, о которых не осталось памяти, которые исчезли, как будто не существовали, и сделались как бы небывшими, и дети их после них.
\vs Sir 44:9 Но те были мужи милости, которых праведные дела не забываются;
\vs Sir 44:10 в семени их пребывает доброе наследство; потомки их~--- в заветах;
\vs Sir 44:11 семя их будет твердо, и дети их~--- ради них;
\vs Sir 44:12 семя их пребудет до века, и слава их не истребится;
\vs Sir 44:13 тела их погребены в мире, и имена их живут в роды;
\vs Sir 44:14 народы будут рассказывать о их мудрости, а церковь будет возвещать их хвалу.
\rsbpar\vs Sir 44:15 Енох угодил Господу и был взят на небо,~--- образ покаяния для \bibemph{всех} родов.
\vs Sir 44:16 Ной оказался совершенным, праведным; во время гнева он был умилостивлением;
\vs Sir 44:17 посему сделался остатком на земле, когда был потоп;
\vs Sir 44:18 с ним заключен был вечный завет, что никакая плоть не истребится более потопом.
\vs Sir 44:19 Авраам~--- великий отец множества народов, и не было подобного ему в славе;
\vs Sir 44:20 он сохранил закон Всевышнего и был в завете с Ним,
\vs Sir 44:21 и на своей плоти утвердил завет и в испытании оказался верным;
\vs Sir 44:22 поэтому Господь с клятвою обещал ему, что в семени его благословятся все народы;
\vs Sir 44:23 обещал умножить его, как прах земли, и возвысить семя его, как звезды, и дать им наследство от моря до моря и от реки до края земли.
\vs Sir 44:24 И Исааку ради Авраама, отца его, Он также подтвердил благословение всех людей и завет;
\vs Sir 44:25 и оно же почило на голове Иакова:
\vs Sir 44:26 Он ущедрил его Своими благословениями, и дал ему в наследие \bibemph{землю}, и отделил участки ее, и разделил между двенадцатью коленами.
\rsbpar\vs Sir 44:27 И произвел от него мужа милости, который приобрел любовь в глазах всякой плоти,
\vs Sir 45:1 возлюбленного Богом и людьми Моисея, которого память благословенна.
\vs Sir 45:2 Он сравнял его в славе со святыми и возвеличил его делами на страх врагам;
\vs Sir 45:3 Он его словом прекращал чудесные знамения, прославил его пред лицем царей, давал чрез него повеления к народу его и показал ему от славы Своей.
\vs Sir 45:4 За верность и кротость его Он освятил его, избрал Себе из всех людей,
\vs Sir 45:5 сподобил его слышать голос Его, ввел его во мглу
\vs Sir 45:6 и дал ему лицем к лицу заповеди, закон жизни и в\acc{е}дения, чтобы он научил Иакова завету и Израиля~--- постановлениям Его.
\vs Sir 45:7 Он возвысил Аарона, подобного ему святого, брата его из колена Левиина,~---
\vs Sir 45:8 постановил с ним вечный завет и дал ему священство в народе; Он благословил его особым украшением и опоясал его поясом славы;
\vs Sir 45:9 Он облек его высшим украшением и облачил его в богатые одежды:
\vs Sir 45:10 в исподнюю одежду, в подир и ефод;
\vs Sir 45:11 и окружил его золотыми яблоками и весьма многими позвонками, чтобы при хождении его они издавали звук, чтобы сделать слышным в храме звон для напоминания сынам народа Его;
\vs Sir 45:12 облек его одеждою святою из золота и гиацинтовой шерсти и крученого виссона художественной работы, словом суда, уримом и туммимом,
\vs Sir 45:13 червленым тканьем искусной работы, многоценными камнями, вырезанными как на печати, в золотой оправе гранильной работы, с вырезанными на память начертаниями \bibemph{имен} по числу колен Израилевых;
\vs Sir 45:14 на кидаре его~--- золотой венец, знамение святыни, слава достоинства: величественное украшение, дело искусства, вожделенное для глаз.
\vs Sir 45:15 Прежде него не было сего от века:
\vs Sir 45:16 непринадлежащий к его племени не одевался так, только сыновья его и потомки его во все времена.
\vs Sir 45:17 Жертвы их приносятся каждый день, всегда по два раза.
\vs Sir 45:18 Моисей наполнил руки его и помазал его святым елеем:
\vs Sir 45:19 ему постановлено в вечный завет и семени его на дни неба, чтобы они служили Ему и вместе священнодействовали и благословляли народ Его именем Его;
\vs Sir 45:20 Он избрал его из всех живущих, чтобы приносить Господу жертву, курение и благоухание в память умилостивления о народе своем;
\vs Sir 45:21 Он дал ему Свои заповеди и власть в постановлениях судебных, чтобы учить Иакова откровениям и наставлять Израиля в законе Его.
\vs Sir 45:22 Восстали против него чужие, и позавидовали ему в пустыне люди, приставшие к Дафану и Авирону, и скопище Корея в ярости и гневе;
\vs Sir 45:23 Господь увидел, и Ему неугодно было это,~--- и они погибли от ярости гнева.
\vs Sir 45:24 Он сотворил над ними чудо, истребив их пламенем огня Своего.
\vs Sir 45:25 И умножил славу Аарона и дал ему наследие~--- отделил им начатки плодов:
\vs Sir 45:26 прежде всего уготовил им хлеб в насыщение, ибо они едят и жертвы Господни, которые Он дал ему и семени его;
\vs Sir 45:27 но он не должен иметь наследия в земле народа и нет ему участка между народом, ибо Он Сам удел и наследие его.
\vs Sir 45:28 Также и Финеес, сын Елеазара, третий по славе, потому что он ревновал о страхе Господнем и, при отпадении народа, устоял в добром расположении души своей и умилостивил Господа к Израилю;
\vs Sir 45:29 посему постановлен с ним завет мира, чтобы быть ему предстоятелем святых и народа своего, чтобы ему и семени его принадлежало достоинство священства навеки.
\vs Sir 45:30 Как по завету с Давидом, сыном Иессея из колена Иудина, царское наследие переходило от сына к сыну, так наследие священства \bibemph{принадлежало} Аарону и семени его.
\vs Sir 45:31 Да даст нам Бог мудрость в нашем сердце~--- судить народ Его справедливо, дабы не погибли блага их и слава их пребыла в роды их.
\vs Sir 46:1 Силен был в бранях Иисус Навин и был преемником Моисея в пророчествах.
\vs Sir 46:2 Соответственно имени своему, он был велик в спасении избранных Божиих, когда мстил восставшим врагам, чтобы ввести Израиля в наследие \bibemph{его}.
\vs Sir 46:3 Как он прославился, когда поднял руки свои и простер меч на города!
\vs Sir 46:4 Кто прежде него так стоял? Ибо он вел брани Господни.
\vs Sir 46:5 Не его ли рукою остановлено было солнце, и один день был как бы два?
\vs Sir 46:6 Он воззвал ко Всевышнему Владыке, когда со всех сторон стеснили его враги, и великий Господь услышал его:
\vs Sir 46:7 камнями града с могущественною силою бросил Он на враждебный народ и погубил противников на склоне горы,
\vs Sir 46:8 дабы язычники познали всеоружие \bibemph{его}, что война его была пред Господом, а он \bibemph{только} следовал за Всемогущим.
\vs Sir 46:9 И во дни Моисея он оказал благодеяние, он и Халев, сын Иефоннии,~--- тем, что они противостояли враждующим, удерживали народ от греха и утишали злой ропот.
\vs Sir 46:10 И они только двое из шестисот тысяч путешествовавших были спасены, чтобы ввести \bibemph{народ} в наследие~--- в землю, текущую молоком и медом.
\vs Sir 46:11 И дал Господь Халеву крепость, которая сохранилась в нем до старости, взойти на высоту земли, и семя его получило наследие,
\vs Sir 46:12 дабы видели все сыны Израилевы, что благо следовать Господу.
\vs Sir 46:13 Также и судии, каждый по своему имени, которых сердце не заблуждалось и которые не отвращались от Господа,~--- да будет память их во благословениях!
\vs Sir 46:14 Да процветут кости их от места своего,
\vs Sir 46:15 и имя их да перейдет к сынам их в прославлении их!
\vs Sir 46:16 Возлюбленный Господом своим Самуил, пророк Господень, учредил царство и помазал царей народу своему;
\vs Sir 46:17 он судил народ по закону Господню, и Господь призирал на Иакова;
\vs Sir 46:18 по вере своей он был истинным пророком, и в словах его дознана верность видения.
\vs Sir 46:19 Он воззвал ко Всемогущему Господу, когда отвсюду теснили его враги, и принес в жертву молодого агнца,~---
\vs Sir 46:20 и Господь возгремел с неба и в сильном шуме слышным сделал голос Свой,
\vs Sir 46:21 и истребил вождей Тирских и всех князей Филистимских.
\vs Sir 46:22 Еще прежде времени вечного успокоения своего он свидетельствовался пред Господом и помазанником \bibemph{Его}: <<имущества, ни даже обуви, я не брал ни от кого>>, и никто не укорил его.
\vs Sir 46:23 Он пророчествовал и по смерти своей, и предсказал царю смерть его, и в пророчестве возвысил из земли голос свой, что беззаконный народ истребится.
\vs Sir 47:1 После сего явился Нафан, чтобы пророчествовать во дни Давида.
\vs Sir 47:2 Как тук, отделенный от мирной жертвы, так Давид от сынов Израилевых.
\vs Sir 47:3 Он играл со львами, как с козлятами, и с медведями, как с ягнятами.
\vs Sir 47:4 В юности своей не убил ли он исполина, не снял ли поношение с народа,
\vs Sir 47:5 когда поднял руку с пращным камнем и низложил гордыню Голиафа?
\vs Sir 47:6 Ибо он воззвал к Господу Всевышнему, и Он дал крепость правой руке его~--- поразить человека, сильного в войне, и возвысить рог народа своего.
\vs Sir 47:7 Так прославил народ его тьмами и восхвалил его в благословениях Господа, как достойного венца славы,
\vs Sir 47:8 ибо он истребил окрестных врагов и смирил враждебных Филистимлян,~--- даже доныне сокрушил рог их.
\vs Sir 47:9 После каждого дела своего он приносил благодарение Святому Всевышнему словом хвалы;
\vs Sir 47:10 от всего сердца он воспевал и любил Создателя своего.
\vs Sir 47:11 И поставил пред жертвенником песнопевцев, чтобы голосом их услаждать песнопение.
\vs Sir 47:12 Он дал праздникам благолепие и с точностью определил времена, чтобы они хвалили святое имя Его и с раннего утра оглашали святилище.
\vs Sir 47:13 И Господь отпустил ему грехи и навеки вознес рог его и даровал ему завет царственный и престол славы в Израиле.
\vs Sir 47:14 После него восстал мудрый сын его и ради \bibemph{отца} жил счастливо.
\vs Sir 47:15 Соломон царствовал в мирные дни, потому что Бог успокоил его со всех сторон, дабы он построил дом во имя Его и приготовил святилище навеки.
\vs Sir 47:16 Как мудр был ты в юности твоей и, подобно реке, полон разума!
\vs Sir 47:17 Душа твоя покрыла землю, и ты наполнил ее загадочными притчами;
\vs Sir 47:18 имя твое пронеслось до отдаленных островов, и ты был любим за мир твой;
\vs Sir 47:19 за песни и изречения, за притчи и изъяснения тебе удивлялись страны.
\vs Sir 47:20 Во имя Господа Бога, наименованного Богом Израиля,
\vs Sir 47:21 ты собрал золото, как медь, и умножил серебро, как свинец.
\vs Sir 47:22 Но ты наклонил чресла твои к женщинам и поработился им телом твоим;
\vs Sir 47:23 ты положил пятно на славу твою и осквернил семя твое так, что навел гнев на детей твоих,~--- и они горько оплакивали твое безумие,~--- что власть разделилась надвое, и от Ефрема произошло непокорное царство.
\vs Sir 47:24 Но Господь не оставит Своей милости и не разрушит ни одного из дел Своих, не истребит потомков избранного Своего и не искоренит семени возлюбившего Его.
\vs Sir 47:25 И Он дал Иакову остаток, и Давиду~--- корень от него.
\rsbpar\vs Sir 47:26 И почил Соломон с отцами своими,
\vs Sir 47:27 и оставил по себе от семени своего безумие народу,
\vs Sir 47:28 скудного разумом Ровоама, который отвратил от себя народ чрез свое совещание,
\vs Sir 47:29 и Иеровоама, сына Наватова, который ввел в грех Израиля и Ефрему указал путь греха.
\vs Sir 47:30 И весьма умножились грехи их, так что они изгнаны были из земли своей;
\vs Sir 47:31 и посягали они на всякое зло, доколе не пришло на них мщение.
\vs Sir 48:1 И восстал Илия пророк, как огонь, и слово его горело, как светильник.
\vs Sir 48:2 Он навел на них голод и ревностью своею умалил \bibemph{число} их;
\vs Sir 48:3 словом Господним он заключил небо и три раза низводил огонь.
\vs Sir 48:4 Как прославился ты, Илия, чудесами твоими, и кто может сравниться с тобою в славе!
\vs Sir 48:5 Ты воздвиг мертвого от смерти и из ада словом Всевышнего;
\vs Sir 48:6 ты низводил в погибель царей и знатных с ложа их;
\vs Sir 48:7 ты слышал на Синае обличение \bibemph{на них} и на Хориве суды мщения;
\vs Sir 48:8 ты помазал царей на воздаяние и пророков~--- в преемники себе;
\vs Sir 48:9 ты восх\acc{и}щен был огненным вихрем на колеснице с огненными конями;
\vs Sir 48:10 ты предназначен был на обличения в свои времена, чтобы утишить гнев, прежде нежели обратится он в ярость,~--- обратить сердце отца к сыну и восстановить колена Иакова.
\vs Sir 48:11 Блаженны видевшие тебя и украшенные любовью,~--- и мы жизнью поживем.
\vs Sir 48:12 Илия сокрыт был вихрем,~--- и Елисей исполнился духом его
\vs Sir 48:13 и во дни свои не трепетал пред князем, и никто не превозмог его;
\vs Sir 48:14 ничто не одолело его, и по успении его пророчествовало тело его.
\vs Sir 48:15 И при жизни своей совершал он чудеса, и по смерти дивны были дела его.
\rsbpar\vs Sir 48:16 При всем том народ не покаялся, и не отступили от грехов своих, доколе не были пленены из земли своей и рассеяны по всей земле.
\vs Sir 48:17 И осталось весьма мало народа и князь из дома Давидова.
\vs Sir 48:18 Некоторые из них делали угодное Богу, а некоторые умножали грехи.
\vs Sir 48:19 Езекия укрепил город свой и провел внутрь его воду, пробил железом скалу и устроил хранилища для воды.
\vs Sir 48:20 Во дни его сделал нашествие Сеннахирим и послал к нему Рабсака, который поднял руку свою на Сион и много величался в гордости своей.
\vs Sir 48:21 Тогда затрепетали сердца и руки их, и они мучились, как родильницы;
\vs Sir 48:22 и воззвали они к Господу милосердому, простерши к Нему руки свои,
\vs Sir 48:23 и Святый скоро услышал их с неба и избавил их рукою Исаии;
\vs Sir 48:24 Он поразил войско Ассириян, и Ангел Его истребил их,
\vs Sir 48:25 ибо Езекия делал угодное Господу и крепко держался путей Давида, отца своего, как заповедал пророк Исаия, великий и верный в видениях своих.
\vs Sir 48:26 В его дни солнце отступило назад, и он прибавил жизни царю.
\vs Sir 48:27 Великим духом своим он провидел отдаленное будущее и утешал сетующих в Сионе;
\vs Sir 48:28 до века возвещал он будущее и сокровенное, прежде нежели оно исполнилось.
\vs Sir 49:1 Память Иосии~--- как состав фимиама, приготовленный искусством мироварника:
\vs Sir 49:2 во всяких устах она будет сладка, как мед и как музыка при угощении вином.
\vs Sir 49:3 Он успешно действовал в обращении народа и истребил мерзости беззакония;
\vs Sir 49:4 он направил к Господу сердце свое и во дни беззаконных утвердил благочестие.
\vs Sir 49:5 Кроме Давида, Езекии и Иосии, все тяжко согрешили,
\vs Sir 49:6 ибо оставили закон Всевышнего; цари Иудейские престали,
\vs Sir 49:7 ибо предали рог свой другим и славу свою~--- чужому народу.
\vs Sir 49:8 Избранный город святыни сожжен, и улицы его опустошены, как предсказал Иеремия,
\vs Sir 49:9 которого они оскорбляли, хотя он еще во чреве освящен был в пророка, чтобы искоренять, поражать и погублять, равно как строить и насаждать.
\rsbpar\vs Sir 49:10 Иезекииль видел явление славы, которую \bibemph{Бог} показал ему в херувимской колеснице;
\vs Sir 49:11 он напоминал о врагах под образом дождя и возвещал доброе тем, которые исправляли пути свои.
\vs Sir 49:12 И двенадцать пророков~--- да процветут кости их от места своего!~--- утешали Иакова и спасали их верною надеждою.
\vs Sir 49:13 Как возвеличим Зоровавеля? И он~--- как перстень на правой руке;
\vs Sir 49:14 также Иисус, сын Иоседека: они во дни свои построили дом и восстановили святый храм Господу, предназначенный к вечной славе.
\vs Sir 49:15 Велика память и Неемии, который воздвиг нам павшие стены, поставил ворота и запоры и возобновил разрушенные домы наши.
\vs Sir 49:16 Не было на земле никого из сотворенных, подобного Еноху,~--- ибо он был восх\acc{и}щен от земли,~---
\vs Sir 49:17 и не родился такой муж, как Иосиф, глава братьев, опора народа,~--- и кости его были почтены.
\vs Sir 49:18 Прославились между людьми Сим и Сиф, но выше всего живущего в творении~--- Адам.
\vs Sir 50:1 Симон, сын Онии, великий священник, при жизни своей исправил дом и во дни свои укрепил храм:
\vs Sir 50:2 им положено основание двойного возвышения~--- возведение высокой ограды храма;
\vs Sir 50:3 во дни его уменьшено водохранилище, окружность медного моря;
\vs Sir 50:4 чтобы предохранить народ свой от бедствия, он укрепил город против осады.
\vs Sir 50:5 Как величествен был он среди народа, при выходе из завесы храма!
\vs Sir 50:6 Как утренняя звезда среди облаков, как луна полная во днях,
\vs Sir 50:7 как солнце, сияющее над храмом Всевышнего, и как радуга, сияющая в величественных облаках,
\vs Sir 50:8 как цвет роз в весенние дни, как лилии при источниках вод, как ветвь ливана в летние дни,
\vs Sir 50:9 как огонь с ладаном в кадильнице,
\vs Sir 50:10 как кованый золотой сосуд, украшенный всякими драгоценными камнями,
\vs Sir 50:11 как маслина с плодами и как возвышающийся до облаков кипарис.
\vs Sir 50:12 Когда он принимал великолепную одежду и облекался во все величественное украшение, то, при восхождении к святому жертвеннику, освещал блеском окружность святилища.
\vs Sir 50:13 Также, когда он принимал \bibemph{жертвенные} части из рук священников, стоя у огня жертвенника,~---
\vs Sir 50:14 вокруг него был венец братьев, как отрасли кедра на Ливане, и они окружали его как финиковые ветви,
\vs Sir 50:15 и все сыны Аарона в славе своей, и приношение Господу в руках их пред всем собранием Израиля.
\vs Sir 50:16 В довершение служб на алтаре, чтобы увенчать приношение Всевышнему Вседержителю,
\vs Sir 50:17 он простирал свою руку к жертвенной чаше, лил в нее из винограда кровь и выливал ее к подножию жертвенника в вон\acc{ю} благоухания Вышнему Всецарю.
\vs Sir 50:18 Тогда сыны Аароновы восклицали, трубили коваными трубами и издавали громкий голос в напоминание пред Всевышним.
\vs Sir 50:19 Тогда весь народ вместе спешил падать лицем на землю, чтобы поклониться Господу своему, Вседержителю, Богу Вышнему;
\vs Sir 50:20 а песнопевцы восхваляли Его своими голосами; в пространном храме раздавалось сладостное пение,
\vs Sir 50:21 и народ молился Господу Всевышнему молитвою пред Милосердым, доколе совершалось славословие Господа,~--- и так оканчивали они службу Ему.
\vs Sir 50:22 Тогда он, сойдя, поднимал руки свои на все собрание сынов Израилевых, чтобы устами своими преподать благословение Господа и похвалиться именем Его;
\vs Sir 50:23 народ повторял поклонение, чтобы принять благословение от Всевышнего.
\vs Sir 50:24 И ныне все благословляйте Бога, Который везде совершает великие дела, Который продлил дни наши от утробы и поступает с нами по милости Своей:
\vs Sir 50:25 да даст Он нам веселие сердца, и да будет во дни наши мир в Израиле до дней века;
\vs Sir 50:26 да сохранит милость Свою к нам и в свое время да избавит нас!
\vs Sir 50:27 Двумя народами гнушается душа моя, а третий не есть народ:
\vs Sir 50:28 \bibemph{это} сидящие на горе Сеир, Филистимляне и глупый народ, живущий в Сикимах.
\vs Sir 50:29 Учение мудрости и благоразумия начертал в книге сей я, Иисус, сын Сирахов, Иерусалимлянин, который излил мудрость от сердца своего.
\vs Sir 50:30 Блажен, кто будет упражняться в сих \bibemph{наставлениях},~--- и кто положит их на сердце, тот сделается мудрым;
\vs Sir 50:31 а если будет исполнять, то все возможет; ибо свет Господень~--- путь его.
\chhdr{Молитва Иисуса, сына Сирахова.}
\vs Sir 51:1 Прославлю Тебя, Господи Царю, и восхвалю Тебя, Бога, Спасителя моего; прославляю имя Твое,
\vs Sir 51:2 ибо Ты был мне покровителем и помощником
\vs Sir 51:3 и избавил тело мое от погибели и от сети клеветнического языка, от уст сплетающих ложь; и против восставших на меня Ты был мне помощником
\vs Sir 51:4 и избавил меня, по множеству милости и ради имени Твоего, от скрежета зубов, готовых пожрать меня,
\vs Sir 51:5 от руки искавших души моей, от многих скорбей, которые я имел,
\vs Sir 51:6 от удушающего со всех сторон огня и из среды пламени, в котором я не сгорел,
\vs Sir 51:7 из глубины чрева адова, от языка нечистого и слова ложного, от клеветы пред царем языка неправедного.
\vs Sir 51:8 Душа моя близка была к смерти,
\vs Sir 51:9 и жизнь моя была близ ада преисподнего:
\vs Sir 51:10 со всех сторон окружали меня, и не было помогающего; искал я глазами заступления от людей,~--- и не было его.
\vs Sir 51:11 И вспомнил я о Твоей, Господи, милости и о делах Твоих от века,
\vs Sir 51:12 что Ты избавляешь надеющихся на Тебя и спасаешь их от руки врагов.
\vs Sir 51:13 И я вознес от земли моление мое и молился о избавлении от смерти:
\vs Sir 51:14 воззвал я к Господу, Отцу Господа моего, чтобы Он не оставил меня во дни скорби, когда не было помощи от людей надменных.
\vs Sir 51:15 Буду хвалить имя Твое непрестанно и воспевать в славословии, ибо молитва моя была услышана;
\vs Sir 51:16 Ты спас меня от погибели и избавил меня от злого времени.
\vs Sir 51:17 За это я буду прославлять и хвалить Тебя и благословлять имя Господа.
\rsbpar\vs Sir 51:18 Будучи еще юношею, прежде нежели пошел я странствовать, открыто искал я мудрости в молитве моей:
\vs Sir 51:19 пред храмом я молился о ней, и до конца буду искать ее; как бы от цвета зреющего винограда,
\vs Sir 51:20 сердце мое радуется о ней; нога моя шла прямым путем, я следил за нею от юности моей.
\vs Sir 51:21 Понемногу наклонял я ухо мое и принимал ее, и находил в ней много наставлений для себя:
\vs Sir 51:22 мне был успех в ней.
\vs Sir 51:23 Воздам славу Дающему мне мудрость.
\vs Sir 51:24 Я решился следовать ей, ревновал о добром, и не постыжусь.
\vs Sir 51:25 Душа моя подвизалась ради нее, и в делах моих я был точен;
\vs Sir 51:26 простирал руки мои к высоте и сознавал мое невежество.
\vs Sir 51:27 Я направил к ней душу мою, и сердце мое предал ей с самого начала~---
\vs Sir 51:28 и при чистоте достиг ее; посему не буду оставлен ею.
\vs Sir 51:29 И подвиглась внутренность моя, чтобы искать ее; посему я приобрел доброе приобретение.
\vs Sir 51:30 В награду мне Бог дал язык, и им я буду хвалить Его.
\vs Sir 51:31 Приблизьтесь ко мне, ненаученные, и водворитесь в доме учения,
\vs Sir 51:32 ибо вы нуждаетесь в этом и души ваши сильно жаждут.
\vs Sir 51:33 Я отверзаю уста мои и говорю: приобретайте ее себе без серебра;
\vs Sir 51:34 подклоните выю вашу под иго ее, и пусть душа ваша принимает учение; его можно найти близко.
\vs Sir 51:35 Видите своими глазами: я немного потрудился~--- и нашел себе великое успокоение.
\vs Sir 51:36 Приобретайте учение и за большое количество серебра,~--- и вы приобретете много золота.
\vs Sir 51:37 Да радуется душа ваша о милости Его, и не стыдитесь хвалить Его;
\vs Sir 51:38 делайте свое дело заблаговременно, и Он в свое время отдаст вашу награду.

\bibbookdescr{Tjs}{
  inline={Завещание Иосифа,\\одиннадцатого сына Иакова и Рахили},
  toc={Завещание Иосифа},
  bookmark={Завещание Иосифа},
  header={Завещание Иосифа},
  abbr={Исф}
}
\vs Tjs 1:1
Список завещания Иосифа.
Когда собрался он умирать, то,
призвав сыновей и братьев своих,
сказал им:
\vs Tjs 1:2
братья мои и дети мои,
послушайте Иосифа, возлюбленного Израиля, внемлите речам уст моих.
\vs Tjs 1:3
Видел я в жизни моей зависть и смерть.
И не соблазнился, но пребывал в правде Господней.
\vs Tjs 1:4
Братья мои возненавидели меня, Господь же возлюбил меня.
Они желали меня убить, но Бог отцов моих сохранил меня.
В колодец меня бросили, но Всевышний вывел меня оттуда.
\vs Tjs 1:5
Продан я был в рабство, но Владыка над всеми освободил меня.
Был я в плену, но могучая рука его помогла мне.
Голод мучил меня, но сам Господь накормил меня.
\vs Tjs 1:6
Одинок я был, и Бог утешил меня;
занемог, и Господь посетил меня,
в темнице был я, и Бог мой смиловался надо мною;
горькие слова слышал от Египтян, и избавил меня;
рабом был и возвысил меня.

\vs Tjs 2:1
И главный повар фараона доверил мне дом свой.
\vs Tjs 2:2
И боролся я с женщиной бесстыдной, склонявшей меня согрешить с нею,
но Бог отцов моих избавил меня от огня пылающего.
\vs Tjs 2:3
Заключили меня в темницу, били и насмехались надо мною,
но дал мне Господь благорасположение тюремщика.
\vs Tjs 2:4
Ибо не оставляет Господь боящихся его ни во тьме, ни в оковах,
ни в скорби, ни в нужде.
\vs Tjs 2:5
Ведь не стыдится Бог подобно человеку, и не робеет подобно сыну
человеческому, и не бежит в страхе подобно землеродному.
\vs Tjs 2:6
Но во всех этих печалях помогает он и различными способами утешает,
и лишь ненадолго отступает от человека,
дабы испытать помышление души его.
\vs Tjs 2:7
10-ти испытаниям подверг он меня,
и все их выдержал я терпеливо.
Ибо великое средство~--- долготерпение,
и много благого даёт стойкость.

\vs Tjs 3:1
Сколько раз угрожала мне смертью Египтянка!
Сколь часто, предав меня пыткам, звала к себе,
а когда не хотел я сойтись с нею, говорила мне:
\vs Tjs 3:2
будешь владыкою надо мною и надо всем, что есть в доме моём,
если предашь себя мне, и будешь ты как хозяин наш.
\vs Tjs 3:3
Я же памятовал о словах отца моего и,
войдя в комнату, плача молил Господа.
\vs Tjs 3:4
И постился я тогда 7 лет,
а Египтянам казалось, что живу я в роскоши.
Ибо постящиеся ради Господа радость на лице являют.
\vs Tjs 3:5
Когда же отсутствовал господин мой, не пил я вина
и раз в 3 дня принимал пищу,
а остальное отдавал бедным и слабым.
\vs Tjs 3:6
И на рассвете обращался я к Господу
и плакал о Египтянке из Мемфиса,
ибо непрестанно и премного беспокоила она меня.
Ибо и ночью подходила она ко мне под тем предлогом,
что желает проведать меня.
\vs Tjs 3:7
И поскольку не было у неё ребёнка мужского пола,
делала она так, будто я~--- сын её.
\vs Tjs 3:8
И до времени как сына меня обнимала, а я не знал того.
После же захотела она во блуд вовлечь меня.
\vs Tjs 3:9
И когда понял, опечалился я до смерти.
И когда удалилась она, пошёл я к себе
и горевал о ней многие дни,
ибо познал я хитрость её и соблазн.
\vs Tjs 3:10
И говорил я ей слова Всевышнего,
дабы отвратилась она от страсти злой.

\vs Tjs 4:1
И часто льстила она мне речами своими как святому мужу,
и хитрыми словами хвалила чистоту мою пред лицом мужа своего,
желая ввести меня в искушение,
когда будем мы одни.
\vs Tjs 4:2
Явно прославляла она целомудрие мое,
а втайне говорила мне: не бойся мужа моего, ибо он убеждён
в целомудрии твоём;
и если кто скажет ему о нас, не поверит он.
\vs Tjs 4:3
Тогда я, пав на землю, молил Бога,
чтобы избавил он меня от коварства её.
\vs Tjs 4:4
Когда же ничего не достигла она,
снова приходила ко мне как бы наставления ради,
дабы слушать слово Божие.
\vs Tjs 4:5
И говорила мне:
если хочешь, чтобы оставила я идолов,
сойдись со мною, а я сумею убедить мужа моего отречься от них,
и будем жить пред лицом Господа твоего.
\vs Tjs 4:6
Я же отвечал ей, что не хочет Господь,
чтобы в нечистоте почитали его,
и не развратникам благоволит он,
но только тем, кто с чистым сердцем 
и устами незапятнанными приходит к нему.
\vs Tjs 4:7
Она же была рассержена и желала исполнить желание своё.
\vs Tjs 4:8
А я предался посту и молитве, дабы избавил меня Господь от неё.

\vs Tjs 5:1
И вновь, в иное время сказала она мне:
если блудить не желаешь,
тогда убью я мужа моего ядом и возьму тебя в мужья.
\vs Tjs 5:2
Я же, услышав это, разодрал одежды мои и сказал ей:
женщина, постыдись Бога и не сотвори дела этого злого,
дабы не погибнуть тебе.
Ибо знай, что я разглашу всем этот твой умысел.
\vs Tjs 5:3
Она же, убоявшись, молила меня,
чтобы не разглашал я замысла того.
\vs Tjs 5:4
И удалилась она, ублажив меня дарами и услаждениями всяческими.

\vs Tjs 6:1
А после того послала мне кушанья, намешав в них колдовское зелье.
\vs Tjs 6:2
Но когда пришел евнух и принес кушанья, взглянул я и увидел
испуганного мужа, подающего мне блюдо и нож;
и понял я, что делается это, дабы соблазнить меня.
\vs Tjs 6:3
И когда вышел он, плакал я и не испробовал ни этого,
ни другого какого-либо из кушаний её.
\vs Tjs 6:4
Через день же пришла она ко мне и, увидев, сказала мне:
отчего не отведал ты кушанья?
\vs Tjs 6:5
И отвечал я ей:
оттого, что наполнила ты его зельем смертельным;
и как говорила ты, что, мол, не приближусь я к идолам,
а к одному только Господу?
\vs Tjs 6:6
Ныне же знай, что Бог отца моего открыл мне
через ангела своего зло твоё, и сохранил я кушанье это,
дабы обличить тебя, и, увидев то, быть может, покаешься ты.
\vs Tjs 6:7
Но дабы узнала ты,
что против чтящих Бога в целомудрии не имеет
силы зло нечестивцев,
вот, возьму я от кушанья и съем пред тобою.
И сказав это, помолился я так:
да будет со мною Бог отцов моих и ангел Авраама.
И вкусил я.
\vs Tjs 6:8
Она же, узрев это, пала с плачем на лицо своё к ногам моим,
и поднял я её и вразумлял.
\vs Tjs 6:9
Она же обещала мне не творить никогда нечестия такого.

\vs Tjs 7:1
Но сердце её лежало ещё во зле, и смотрела она,
каким бы способом поймать меня в западню.
И стеная непрестанно, чахла она,
хоть и не была больна.
\vs Tjs 7:2
Увидев же это, сказал ей муж её:
отчего исхудало лицо твоё?
Она же отвечала ему:
страдаю я болью сердечной, и стенание духа мучает меня.
И утешал он её словами своими.
\vs Tjs 7:3
Она же, улучив удобное время, вбежала ко мне,
когда уже ушёл муж её, и сказала мне:
терзаюсь я, и если не возляжешь со мною, брошусь я со скалы.
\vs Tjs 7:4
Я же, поняв, что дух Велиаров мучит её,
обратился с мольбою к Господу и сказал ей:
\vs Tjs 7:5
Что ты, несчастная женщина, терзаешься и мятёшься,
ослеплённая грехом?
помни, что если убьёшь ты себя,
то Астифо, наложница мужа твоего и соперница твоя,
перебъёт всех детей твоих,
и исчезнет память о тебе на земле.
\vs Tjs 7:6
И сказала она мне: вот, всё же ты любишь меня.
Да будет мне довольно этого.
Только вступись за жизнь мою и детей моих,
а я буду ожидать, пока не услажу страсти моей.
\vs Tjs 7:7
Ибо не знала она, что ради Господа моего сказал я так,
а не ради неё.
\vs Tjs 7:8
Но кто одержим страстью желания и рабски служит ей,
как эта женщина, тот, если и доброе что услышит
ко страданию своему, относит это к страсти злой.

\vs Tjs 8:1
И вот, говорю, дети мои, что было около 6-го часа,
когда вышла она от меня.
И преклонив колени к Господу,
стоял я так весь день и всю ночь,
а на рассвете восстал,
плача и моля избавить меня от Египтянки.
\vs Tjs 8:2
И тогда, наконец, схватила она меня за одежды,
силою желая принудить меня сойтись с нею.
\vs Tjs 8:3
И увидев, что в безумии схватила меня за хитон,
оставил его ей и убежал нагим.
\vs Tjs 8:4
Она же, взяв хитон, ложно донесла на меня.
И муж её, придя, заключил меня под стражу
в доме своём и, побив бичами, отослал в темницу фараонову.
\vs Tjs 8:5
И когда был я в оковах, терзалась Египтянка от горя.
И, приходя, внимала она тому, как благодарил
я Господа и пел хвалы ему в доме тьмы и ликовал,
радостным голосом славя Бога моего, ибо избавил
он меня от Египтянки.

\vs Tjs 9:1
Она же часто посылала ко мне, говоря:
благоволи исполнить желание моё,
и я освобожу тебя из оков и от тьмы избавлю.
\vs Tjs 9:2
А я даже мыслию не склонился к ней.
Ведь больше любит Бог целомудренного,
который терпит тьму во рву,
нежели распутника, который роскошествует в царских палатах.
\vs Tjs 9:3
Если же тот, кто живёт в целомудрии, желает и славы,
и знает Всевышний, что это полезно ему,
подаст он, как подал и мне.
\vs Tjs 9:4
Сколько раз она, и будучи больной,
сходила ко мне по вечерам и слушала голос мой,
когда молился я;
я же, слыша стенания её, молчал.
\vs Tjs 9:5
И когда был я в доме её, обнажала она руки и бёдра свои,
дабы возлёг я с нею;
ибо она была прекрасна весьма и украшалась премного,
чтобы соблазнить меня.
И уберёг меня Господь от злых умыслов её.

\vs Tjs 10:1
Зрите же, дети мои, что творит терпение и молитва с постом.
\vs Tjs 10:2
Так и вы, если к целомудрию и чистоте стремиться будете
в терпении и молитве с постом,
в смирении сердечном, поселится в вас Господь,
ибо он любит целомудрие.
\vs Tjs 10:3
А там, где живёт Всевышний, если и зависть приступит,
или рабство, или клевета, Господь, живущий в том человеке,
за целомудрие не только избавит его от бед,
но и возвысит, и прославит, как и меня.
\vs Tjs 10:4
Ибо всякий человек прельщается или в делах, или в словах,
или в помыслах своих.

\vs Tjs 10:5
Знают братья мои, как возлюбил меня отец мой,
но нисколько не возносился я в мыслях моих,
и хоть был ещё ребёнком, страх Божий имел
в сердце моём, ибо знал, что всё прейдет.
\vs Tjs 10:6
И не восстал я в злобе против них,
но почтил их; и уважая их,
даже когда продали меня, умолчал пред Измаильтянами,
что я сын Иакова, мужа великого и праведного.

\vs Tjs 11:1
Так и вы, дети мои, имейте во всяком деле вашем
страх Божий перед очами и чтите братьев ваших,
ибо всякий творящий закон Божий возлюблен им будет.
\vs Tjs 11:2
И когда шёл я с Измаильтянами,
вопрошали они меня: раб ли ты?
И говорил я, что раб домашний,
дабы не опозорить братьев моих.
\vs Tjs 11:3
Говорил же мне старший из них:
ты не раб, ибо видно это по тебе.
Я же сказал им: я ваш раб.
\vs Tjs 11:4
Когда же пришли в Египет, спорили из-за меня,
кто из них даст золото и возьмёт меня.
\vs Tjs 11:5
И решили все, что должен я остаться в Египте
с перекупщиком товаров их, пока не вернутся они с товарами своими.
\vs Tjs 11:6
Господь же даровал мне милость в очах перекупщика того,
и доверил он мне дом свой.
\vs Tjs 11:7
И благословил его Бог рукою моей и обогатил его
золотом, серебром и имуществом.
\vs Tjs 11:8
И был я у него 3 месяца.

\vs Tjs 12:1
А в то время прибыла в пышности великой Мемфиянка,
жена Пентефриса, ибо услышала она обо мне от евнухов своих.
\vs Tjs 12:2
И сказала она мужу своему, что разбогател тот купец
руками некоего юного Еврея, и говорят, что украли его
из земли Ханаанской.
\vs Tjs 12:3
Сотвори же ныне суд и забери юношу в дом наш,
и благословит тебя Бог Еврейский,
ибо благодать небесная на юноше том.

\vs Tjs 13:1
Послушался Пентефрис слов её, и призвал к себе купца,
и сказал ему: что это слышу я о тебе, что крадёшь
ты души из земли Ханаанской,
и в рабы перепродаёшь их?
\vs Tjs 13:2
А купец пал к ногам его и стал умолять:
прошу тебя, господин, не знаю я, что ты говоришь.
\vs Tjs 13:3
И сказал ему Пентефрис: откуда же этот Еврейский юноша?
И отвечал тот: Измаильтяне отдали мне его до той поры,
когда возвратятся они.
\vs Tjs 13:4
И не поверил ему Пентефрис, но приказал раздеть его донага и бить.
Когда же оставался тот при словах своих, сказал Пентефрис:
да будет приведён юноша.
\vs Tjs 13:5
И войдя, поклонился я Пентефрису,
ибо он был 3-им в ряду владык после фараона.
\vs Tjs 13:6
И отведя меня в сторону, вопросил он:
раб ты или свободный?
Я же ответил: раб.
\vs Tjs 13:7
И вопросил он: чей?
И сказал я: Измаильтян.
\vs Tjs 13:8
Он же вопросил: как сделался ты рабом их?
И отвечал я: в земле Ханаанской купили они меня.
\vs Tjs 13:9
И сказал он мне: ты лжёшь.
И тотчас приказал бить и меня нагого.

\vs Tjs 14:1
А Мемфиянка видела через окно,
как били меня, ибо рядом был дом её,
и послала к Пентефрису, говоря:
неправеден суд твой, ибо свободного
и украденного наказываешь ты как преступника.
\vs Tjs 14:2
А я не отказывался от слов моих, хотя и били меня,
и приказал он охранять меня, пока, сказал он,
не придут хозяева юноши.
\vs Tjs 14:3
И сказала ему жена его: за что мучаешь ты и держишь
в оковах юношу, попавшего в плен, коего лучше
бы было освободить, дабы служил он тебе?
\vs Tjs 14:4
Ибо желала она видеть меня, чтобы совершить грех,
а я не знал ничего об этом.
\vs Tjs 14:5
И сказал ей муж её: не отнимают чужого Египтяне,
пока не совершится разбирательство.
\vs Tjs 14:6
А затем сказал купцу: юноша должен быть заключен в тюрьму.

\vs Tjs 15:1
Спустя же 24 дня пришли Измаильтяне;
ибо услышали они, что Иаков, отец мой, премного печалится обо мне.
И придя, сказали они мне: 
\vs Tjs 15:2
что же это ты назвал себя рабом?
И вот, узнали мы, что ты сын человека великого в земле Ханаанской,
и печалится о тебе отец твой во вретище и пепле.
\vs Tjs 15:3
Когда услышал я это, размягчилось и растаяло сердце моё,
и хотел я заплакать громко, но сдержал себя,
дабы не опозорить братьев моих, и сказал им:
ничего не знаю, раб я.
\vs Tjs 15:4
Тогда решили они продать меня, дабы не был я найден
в руках у них.
\vs Tjs 15:5
Ибо они страшились отца моего, как бы не пришёл он,
дабы отомстить им ужасно.
Слышали они, что велик он пред Богом и людьми.
\vs Tjs 15:6
Тут сказал им купец: избавьте меня от суда Пентефриса.
\vs Tjs 15:7
И пошли они и просили меня:
скажи, что за серебро был продан ты нам,
и он освободит нас от ответственности.

\vs Tjs 16:1
А Мемфиянка сказала мужу своему:
купи этого юношу, ибо я слышу, говорят,
что продают его.
\vs Tjs 16:2
И послала она евнуха к Измаильтянам с просьбой купить меня.
\vs Tjs 16:3
А евнух не купил меня, но возвратился и сказал госпоже своей,
что большую цену просят они за юношу.
\vs Tjs 16:4
И послала она евнуха обратно, говоря:
если и 2 мины просят они, дай им,
не жалей золота;
только купи юношу и приведи его ко мне.
\vs Tjs 16:5
И пошёл евнух и, отдав им 80 золотых, взял меня;
Египтянке же сказал он, что отдал 100.
\vs Tjs 16:6
А я знал о том, но промолчал, дабы не опозорить евнуха.

\vs Tjs 17:1
Смотрите же, дети мои, сколько пришлось перенести мне,
дабы не опозорить братьев моих.
\vs Tjs 17:2
И вы любите друг друга, и в долготерпении скрывайте
прегрешения друг друга.
\vs Tjs 17:3
Ибо радуется Бог единомыслию братьев и помыслу сердца благого,
стремящегося к добру.
\vs Tjs 17:4
Когда же пришли братья мои в Египет, знают они,
что возвратил я им серебро, и не укорял их,
и утешил их.
\vs Tjs 17:5
А после смерти Иакова, отца моего, ещё более возлюбил их и всё,
чего желали они, в изобилии делал им.
\vs Tjs 17:6
И не допускал я, чтобы горевали они хотя бы из-за самого малого,
и всё, что было в руке моей, давал им.
\vs Tjs 17:7
И сыновья их~--- мои сыновья, а мои сыновья~--- как рабы их,
и душа их~--- моя душа, и всякая боль их~--- моя боль,
и всякая истома их~--- моя болезнь,
и воля их~--- моя воля.
\vs Tjs 17:8
И не превозносился я среди них, хвалясь славой моей в мире,
но был среди них как один из малейших.

\vs Tjs 18:1
Если и вы, дети мои, жить будете по заповедям Господним,
возвысит вас Бог вовеки.
\vs Tjs 18:2
И если кто-либо пожелает зло сделать вам,
сотворите доброе дело и помолитесь за него,
и ото всякого зла избавлены будете вы Господом.
\vs Tjs 18:3
Ибо вот, видите вы, что за смирение и долготерпение
моё взял я в жёны себе дочь жреца Гелиопольского,
и 100 талантов золота дали мне с нею,
и сделал их Господь мой рабами моими.
\vs Tjs 18:4
И обличье прекрасное дал он мне превыше прекрасных в Израиле,
и до старости во здравии и в красоте хранил меня.
Ибо во всём был я подобен Иакову.

\vs Tjs 19:1
Услышьте же, дети мои, также и о сне, который видел я.
\vs Tjs 19:2
Видел я 12 оленей, которые паслись,
и 9 из них были рассеяны по всей земле,
3 же спаслись, но на следующий день и они были рассеяны.
\vs Tjs 19:3
И узрел я, что 3 оленя сделались 3-мя агнцами и возопили к Господу,
и привёл их Господь на место цветущее и водою обильное
и вывел из тьмы на свет.
\vs Tjs 19:4
И тут возопили к Господу 9 оленей,
потом собрались они и стали как 12 овец
и в недолгом времени увеличились
и стали многими стадами.
\vs Tjs 19:5
После того взглянул я, и вот, явилось 12 быков,
сосущих одну телицу, которая море молока давала,
и пили от неё 12 стад и бесчисленные стада.
\vs Tjs 19:6
И у 4-го быка выросли рога до неба и стали
как стена для стад, а между двух рогов вырос иной рог.
\vs Tjs 19:7
И узрел я тельца, который двенадцатикратно окружил их,
и подал он помощь всем быкам.
\vs Tjs 19:8
И увидел я среди рогов некую деву,
имеющую пёструю одежду,
и от неё произошёл агнец,
и слева от него~--- лев,
и пошли против него все звери и все гады,
и победил их агнец, и погубил их.
\vs Tjs 19:9
И радовались ему быки, и телица, и ангелы, и вся земля.
\vs Tjs 19:10
И должно тому быть в последние дни.

\vs Tjs 19:11
Вы же, дети мои, храните заповеди Господа и чтите Левия и Иуду,
ибо от семени их придёт агнец Божий, дабы принять на себя грех мира,
Спаситель всех народов и Израиля.
\vs Tjs 19:12
Ибо царствие его будет вечным, и не прейдёт оно.
Моего же царства, которое в вас, не станет,
словно сторожки в саду, что уничтожается по прошествии лета.

\vs Tjs 20:1
Знаю я, что после кончины моей притеснять будут вас Египтяне,
и Бог отомстит за вас и приведёт вас к обещанному отцам моим.
\vs Tjs 20:2
Вы же возьмите с собою кости мои, ибо,
когда понесёте вы туда эти кости,
будет с вами Господь в свете, а Велиар во тьме будет с Египтянами.
\vs Tjs 20:3
А мать свою Асинефу отведите к Ипподрому и похороните её рядом с Рахилью,
матерью моей.

\vs Tjs 20:4
И сказав это, вытянул он ноги свои и почил сном прекрасным.
\vs Tjs 20:5
И оплакал его весь Израиль и весь Египет в скорби великой.
Ибо и для Египтян был он как соплеменник их,
и добро им творил, помогая во всем и советом, и делом своим.
\vs Tjs 20:6
А когда вышли сыны Израиля из Египта,
взяли они с собою кости Иосифа и погребли их
в Хевроне с отцами его.
И было лет жизни его 110.

\bibbookdescr{Tas}{
  inline={Завещание Асира,\\десятого сына Иакова и Зелфы},
  toc={Завещание Асира},
  bookmark={Завещание Асира},
  header={Завещание Асира},
  abbr={Аср}
}
\vs Tas 1:1
Список завещания Асира, данного им сыновьям его в 125-ый год жизни его.
\vs Tas 1:2
Будучи здоров, говорил он им:
послушайте, дети Асира, отца вашего,
и всё прямое пред лицом Господа покажу вам.

\vs Tas 1:3
2 пути дал Бог сынам человеческим,
и 2 помысла, и 2 дела, и 2 способа, и 2 исхода.
\vs Tas 1:4
Оттого все по 2 одно против другого.
\vs Tas 1:5
Ибо есть 2 пути доброго и злого, и 2 помышления о них в груди нашей,
различающие их.
\vs Tas 1:6
Если желает душа быть доброй,
все дела свои творит она в справедливости,
а если и согрешит, тотчас же кается.
\vs Tas 1:7
Помышляя праведное и отвергая худое,
тотчас же истребляет она зло и с корнем вырывает грех.
\vs Tas 1:8
Если же к худому клонится помышление души,
всякое дело её во зле, и отвергает она добро,
и прилепляется ко злу, и властвует над нею Велиар;
а если и доброе творит, во зло его обращает.
\vs Tas 1:9
Когда начинает творить добро,
исход дела того злым бывает,
ибо сокровище помышления злым духом наполняется.

\vs Tas 2:1
Бывает, что душа на словах доброе выше злого ставит,
но исход дела ее злой.
\vs Tas 2:2
Бывает, что человек не щадит тех,
кто в недобром ему помогает,
и это двулико, но всё в целом~--- зло.
\vs Tas 2:3
Бывает, что человек возлюбит делающего зло,
так что и умереть во зле согласится ради него, и ясно,
что это двулико, но всё в целом~--- злое дело.
\vs Tas 2:4
И если и есть любовь, во зле тот,
кто скрывает злое под именем доброго;
исход же дела недобрый.

\vs Tas 2:5
Иной крадёт, обижает, грабит, корыстолюбив,
но бедных жалеет; и это двулико, но всё в целом~--- зло.
\vs Tas 2:6
Отнимающий у ближнего своего гневит Бога,
ложно клянётся Всевышним,
а нищего жалеет.
Наставляющего в законе Господнем гонит и хулит,
а бедняку подаёт помощь.
\vs Tas 2:7
Душу пятнает он, а тело украшает, многих убивает,
а немногих жалеет, и это двулико, а всё в целом~--- зло.

\vs Tas 2:8
Иной предаётся блуду и разврату, а от пищи воздерживается;
и в посте злые дела творит, и силою богатства многих притесняет,
а наставления даёт несмотря на великое зло своё;
и это двулико, всё же вместе~--- зло.
\vs Tas 2:9
Такие люди~--- как зайцы, ибо наполовину чисты они,
но по правде нечисты.
\vs Tas 2:10
Ибо так сказал Бог на скрижалях заповедей.

\vs Tas 3:1
Вы же, дети мои, сами не будьте двуликими~--- и добрыми,
и злыми вместе, но к одной доброте прилепитесь,
ибо в ней обитает Господь Бог,
и люди её желают.
\vs Tas 3:2
А зла убегайте, убивая помышление злое делами добрыми,
ибо двуликие служат не Богу, но страстям своим,
дабы угодить Велиару и людям, подобным себе.

\vs Tas 4:1
А люди добрые и одноликие праведны пред Богом,
если и говорят двуликие, что согрешают они.
\vs Tas 4:2
Многие убивающие злых 2 дела совершают~--- доброе и злое,
но всё в целом~--- добро, ибо гибнет вырванное с корнем зло.
\vs Tas 4:3
Ненавидящий того, кто и милостив и неправеден вместе,
и блудит и постится вместе, также двуликое совершает,
но всё дело его~--- доброе;
ибо он уподобляется Господу, не принимая за истинное добро то,
что добрым только кажется.
\vs Tas 4:4
Иной же не хочет видеть дня праздничного с распутными,
дабы не осрамить тела своего и не запятнать души своей,
и это двулико, но в целом~--- добро.
\vs Tas 4:5
Такие люди оленям и ланям подобны, ибо они,
имея обличье диких зверей, кажутся нечистыми,
но в целом~--- чисты.
Ведь в ревности Господней живут они, удаляясь от того,
что и Бог возненавидел и запретил заповедями своими,
отделяя доброе от злого.

\vs Tas 5:1
Смотрите, дети, что во всём есть
2 стороны~--- одна противоположна другой,
и одна за другой сокрыта:
в приобретении~--- любостяжательство,
в радости~--- опьянение,
в веселии~--- скорбь,
в браке~--- распутство.
\vs Tas 5:2
Жизни следует смерть,
славе~--- бесчестие,
дню~--- ночь,
свету~--- тьма
и всё под днём, под жизнью~--- праведное, а под смертью~--- неправедное.
Оттого и за смертью грядёт жизнь вечная.
\vs Tas 5:3
И нельзя назвать правду ложью,
или праведное~--- неправедным,
ибо всякая правда~--- в свете, как всё~--- под Богом.
\vs Tas 5:4
Всё это испытал я в жизни моей,
и не уклонялся от правды Господней,
и заповеди Всевышнего изучал,
и был одноликим, всею силою души моей стремясь к добру.
\vs Tas 6:1
Следуйте и вы, дети мои, заповедям Господа,
и будьте одноликими, следуя правде.
\vs Tas 6:2
Ибо двуликие двоякий грех совершают,
ибо и делают злое, и одобряют делающих,
подражая духам соблазна и борясь против людей.
\vs Tas 6:3
Вы же, дети мои, храните закон Господа,
и не внимайте злу, схожему с добром,
а взирайте на то, что сутью своей благо,
и его блюдите по всем заповедям Господним,
в нём пребывая и почивая.
\vs Tas 6:4
Ибо конец жизни человека являет праведность его,
и встречает он либо ангелов Господних, либо Велиаровых.
\vs Tas 6:5
Когда смятенная душа отходит,
обличается она злым духом,
ибо человек тот был рабом страстей и дурных дел.
\vs Tas 6:6
Если же спокойна душа,
в радости узнаёт она ангела мира,
и ведёт он её в жизнь вечную.

\vs Tas 7:1
Не уподобляйтесь Содому,
не узнавшему ангелов Господних
и погибшему навечно.
\vs Tas 7:2
Ибо знаю я, что согрешите вы и преданы
будете в руки врагов ваших,
и земля ваша запустеет, и святыни ваши разрушатся,
вы же рассеяны будете по 4-ём углам земли,
и будете в рассеянии презираемы как вода бесполезная.
\vs Tas 7:3
До той поры будет это, когда посмотрит Всевышний на землю,
и сам придёт как человек, с людьми вкушающий и пьющий,
и снесёт голову дракона в воде, и избавит он Израиля и все народы
Бог, в человека облёкшийся.
\vs Tas 7:4
Скажите же, дети мои, и вы детям вашим об этом,
дабы не ослушались его.
\vs Tas 7:5
Ибо узнал я, что ослушаетесь вы и пребудете в нечестии,
внимая не закону Божию, но советам людским,
совращаясь во зле.
\vs Tas 7:6
И за то разделены будете вы, подобно Гаду и Дану, братьям моим,
чьей земли, рода и языка не узн\acc{а}ете.
\vs Tas 7:7
Но вновь восставит он вас в вере милосердием своим
и ради Авраама, Исаака и Иакова.

\vs Tas 8:1
И сказав это, завещал им: похороните меня в Хевроне.
И умер, почив сном прекрасным.
\vs Tas 8:2
И сделали сыновья его, как завещал он им,
и отнесли его в Хеврон, и погребли там с отцами его.

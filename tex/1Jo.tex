\bibbookdescr{1Jo}{
  inline={Первое Соборное Послание\\\LARGE Святого Апостола Иоанна Богослова},
  toc={1-е Иоанна},
  bookmark={1-е Иоанна},
  header={1-е Иоанна},
  %headerleft={},
  %headerright={},
  abbr={1~Ин}
}
\vs 1Jo 1:1 О том, что было от начала, что мы слышали, что видели своими очами, что рассматривали и что осязали руки наши, о Слове жизни,~---
\vs 1Jo 1:2 ибо жизнь явилась, и мы видели и свидетельствуем, и возвещаем вам сию вечную жизнь, которая была у Отца и явилась нам,~---
\vs 1Jo 1:3 о том, что мы видели и слышали, возвещаем вам, чтобы и вы имели общение с нами: а наше общение~--- с Отцем и Сыном Его, Иисусом Христом.
\vs 1Jo 1:4 И сие пишем вам, чтобы радость ваша была совершенна.
\rsbpar\vs 1Jo 1:5 И вот благовестие, которое мы слышали от Него и возвещаем вам: Бог есть свет, и нет в Нем никакой тьмы.
\vs 1Jo 1:6 Если мы говорим, что имеем общение с Ним, а ходим во тьме, то мы лжем и не поступаем по истине;
\vs 1Jo 1:7 если же ходим во свете, подобно как Он во свете, то имеем общение друг с другом, и Кровь Иисуса Христа, Сына Его, очищает нас от всякого греха.
\vs 1Jo 1:8 Если говорим, что не имеем греха,~--- обманываем самих себя, и истины нет в нас.
\vs 1Jo 1:9 Если исповедуем грехи наши, то Он, будучи верен и праведен, простит нам грехи наши и очистит нас от всякой неправды.
\vs 1Jo 1:10 Если говорим, что мы не согрешили, то представляем Его лживым, и сл\acc{о}ва Его нет в нас.
\vs 1Jo 2:1 Дети мои! сие пишу вам, чтобы вы не согрешали; а если бы кто согрешил, то мы имеем ходатая пред Отцем, Иисуса Христа, праведника;
\vs 1Jo 2:2 Он есть умилостивление за грехи наши, и не только за наши, но и за \bibemph{грехи} всего мира.
\rsbpar\vs 1Jo 2:3 А что мы познали Его, узнаём из того, что соблюдаем Его заповеди.
\vs 1Jo 2:4 Кто говорит: <<я познал Его>>, но заповедей Его не соблюдает, тот лжец, и нет в нем истины;
\vs 1Jo 2:5 а кто соблюдает слово Его, в том истинно любовь Божия совершилась: из сего узнаём, что мы в Нем.
\vs 1Jo 2:6 Кто говорит, что пребывает в Нем, тот должен поступать так, как Он поступал.
\rsbpar\vs 1Jo 2:7 Возлюбленные! пишу вам не новую заповедь, но заповедь древнюю, которую вы имели от начала. Заповедь древняя есть слово, которое вы слышали от начала.
\vs 1Jo 2:8 Но притом и новую заповедь пишу вам, чт\acc{о} есть истинно и в Нем и в вас: потому что тьма проходит и истинный свет уже светит.
\vs 1Jo 2:9 Кто говорит, что он во свете, а ненавидит брата своего, тот еще во тьме.
\vs 1Jo 2:10 Кто любит брата своего, тот пребывает во свете, и нет в нем соблазна.
\vs 1Jo 2:11 А кто ненавидит брата своего, тот находится во тьме, и во тьме ходит, и не знает, куда идет, потому что тьма ослепила ему глаза.
\rsbpar\vs 1Jo 2:12 Пишу вам, дети, потому что прощены вам грехи ради имени Его.
\vs 1Jo 2:13 Пишу вам, отцы, потому что вы познали Сущего от начала. Пишу вам, юноши, потому что вы победили лукавого. Пишу вам, отроки, потому что вы познали Отца.
\vs 1Jo 2:14 Я написал вам, отцы, потому что вы познали Безначального. Я написал вам, юноши, потому что вы сильны, и слово Божие пребывает в вас, и вы победили лукавого.
\vs 1Jo 2:15 Не люб\acc{и}те мира, ни того, что в мире: кто любит мир, в том нет любви Отчей.
\vs 1Jo 2:16 Ибо всё, что в мире: похоть плоти, похоть очей и гордость житейская, не есть от Отца, но от мира сего.
\vs 1Jo 2:17 И мир проходит, и похоть его, а исполняющий волю Божию пребывает вовек.
\rsbpar\vs 1Jo 2:18 Дети! последнее время. И как вы слышали, что придет антихрист, и теперь появилось много антихристов, то мы и познаём из того, что последнее время.
\vs 1Jo 2:19 Они вышли от нас, но не были наши: ибо если бы они были наши, то остались бы с нами; но \bibemph{они вышли, и} через т\acc{о} открылось, что не все наши.
\vs 1Jo 2:20 Впрочем, вы имеете помазание от Святаго и знаете всё.
\vs 1Jo 2:21 Я написал вам не потому, чтобы вы не знали истины, но потому, что вы знаете ее, \bibemph{равно как} и т\acc{о}, что всякая ложь не от истины.
\vs 1Jo 2:22 Кто лжец, если не тот, кто отвергает, что Иисус есть Христос? Это антихрист, отвергающий Отца и Сына.
\vs 1Jo 2:23 Всякий, отвергающий Сына, не имеет и Отца; а исповедующий Сына имеет и Отца.
\vs 1Jo 2:24 Итак, что вы слышали от начала, то и да пребывает в вас; если пребудет в вас то, что вы слышали от начала, то и вы пребудете в Сыне и в Отце.
\vs 1Jo 2:25 Обетование же, которое Он обещал нам, есть жизнь вечная.
\rsbpar\vs 1Jo 2:26 Это я написал вам об обольщающих вас.
\vs 1Jo 2:27 Впрочем, помазание, которое вы получили от Него, в вас пребывает, и вы не имеете нужды, чтобы кто учил вас; но как самое сие помазание учит вас всему, и оно истинно и неложно, то, чему оно научило вас, в том пребывайте.
\rsbpar\vs 1Jo 2:28 Итак, дети, пребывайте в Нем, чтобы, когда Он явится, иметь нам дерзновение и не постыдиться пред Ним в пришествие Его.
\vs 1Jo 2:29 Если вы знаете, что Он праведник, знайте и т\acc{о}, что всякий, делающий правду, рожден от Него.
\vs 1Jo 3:1 Смотр\acc{и}те, какую любовь дал нам Отец, чтобы нам называться и быть детьми Божиими. Мир потому не знает нас, что не познал Его.
\rsbpar\vs 1Jo 3:2 Возлюбленные! мы теперь дети Божии; но еще не открылось, чт\acc{о} будем. Знаем только, что, когда откроется, будем подобны Ему, потому что увидим Его, как Он есть.
\vs 1Jo 3:3 И всякий, имеющий сию надежду на Него, очищает себя так, как Он чист.
\vs 1Jo 3:4 Всякий, делающий грех, делает и беззаконие; и грех есть беззаконие.
\vs 1Jo 3:5 И вы знаете, что Он явился для того, чтобы взять грехи наши, и что в Нем нет греха.
\vs 1Jo 3:6 Всякий, пребывающий в Нем, не согрешает; всякий согрешающий не видел Его и не познал Его.
\rsbpar\vs 1Jo 3:7 Дети! да не обольщает вас никто. Кто делает правду, тот праведен, подобно как Он праведен.
\vs 1Jo 3:8 Кто делает грех, тот от диавола, потому что сначала диавол согрешил. Для сего-то и явился Сын Божий, чтобы разрушить дела диавола.
\vs 1Jo 3:9 Всякий, рожденный от Бога, не делает греха, потому что семя Его пребывает в нем; и он не может грешить, потому что рожден от Бога.
\vs 1Jo 3:10 Дети Божии и дети диавола узна\acc{ю}тся так: всякий, не делающий правды, не есть от Бога, равно и не любящий брата своего.
\vs 1Jo 3:11 Ибо таково благовествование, которое вы слышали от начала, чтобы мы любили друг друга,
\vs 1Jo 3:12 не т\acc{а}к, к\acc{а}к Каин, \bibemph{который} был от лукавого и убил брата своего. А за что убил его? За то, что дела его были злы, а дела брата его праведны.
\vs 1Jo 3:13 Не дивитесь, братия мои, если мир ненавидит вас.
\vs 1Jo 3:14 Мы знаем, что мы перешли из смерти в жизнь, потому что любим братьев; не любящий брата пребывает в смерти.
\vs 1Jo 3:15 Всякий, ненавидящий брата своего, есть человекоубийца; а вы знаете, что никакой человекоубийца не имеет жизни вечной, в нем пребывающей.
\vs 1Jo 3:16 Любовь познали мы в том, что Он положил за нас душу Свою: и мы должны полагать души свои за братьев.
\vs 1Jo 3:17 А кто имеет достаток в мире, но, видя брата своего в нужде, затворяет от него сердце свое,~--- как пребывает в том любовь Божия?
\rsbpar\vs 1Jo 3:18 Дети мои! станем любить не словом или языком, но делом и истиною.
\vs 1Jo 3:19 И вот по чему узнаём, что мы от истины, и успокаиваем пред Ним сердца наши;
\vs 1Jo 3:20 ибо если сердце наше осуждает нас, то \bibemph{кольми паче Бог}, потому что Бог больше сердца нашего и знает всё.
\vs 1Jo 3:21 Возлюбленные! если сердце наше не осуждает нас, то мы имеем дерзновение к Богу,
\vs 1Jo 3:22 и, чего ни попросим, получим от Него, потому что соблюдаем заповеди Его и делаем благоугодное пред Ним.
\vs 1Jo 3:23 А заповедь Его та, чтобы мы веровали во имя Сына Его Иисуса Христа и любили друг друга, как Он заповедал нам.
\vs 1Jo 3:24 И кто сохраняет заповеди Его, тот пребывает в Нем, и Он в том. А что Он пребывает в нас, узнаём по духу, который Он дал нам.
\vs 1Jo 4:1 Возлюбленные! не всякому духу верьте, но испытывайте духов, от Бога ли они, потому что много лжепророков появилось в мире.
\vs 1Jo 4:2 Духа Божия (и духа заблуждения) узнавайте так: всякий дух, который исповедует Иисуса Христа, пришедшего во плоти, есть от Бога;
\vs 1Jo 4:3 а всякий дух, который не исповедует Иисуса Христа, пришедшего во плоти, не есть от Бога, но это дух антихриста, о котором вы слышали, что он придет и теперь есть уже в мире.
\rsbpar\vs 1Jo 4:4 Дети! вы от Бога, и победили их; ибо Тот, Кто в вас, больше того, кто в мире.
\vs 1Jo 4:5 Они от мира, потому и говорят по-мирски, и мир слушает их.
\vs 1Jo 4:6 Мы от Бога; знающий Бога слушает нас; кто не от Бога, тот не слушает нас. По сему-то узнаём духа истины и духа заблуждения.
\rsbpar\vs 1Jo 4:7 Возлюбленные! будем любить друг друга, потому что любовь от Бога, и всякий любящий рожден от Бога и знает Бога.
\vs 1Jo 4:8 Кто не любит, тот не познал Бога, потому что Бог есть любовь.
\vs 1Jo 4:9 Любовь Божия к нам открылась в том, что Бог послал в мир Единородного Сына Своего, чтобы мы получили жизнь через Него.
\vs 1Jo 4:10 В том любовь, что не мы возлюбили Бога, но Он возлюбил нас и послал Сына Своего в умилостивление за грехи наши.
\rsbpar\vs 1Jo 4:11 Возлюбленные! если так возлюбил нас Бог, то и мы должны любить друг друга.
\vs 1Jo 4:12 Бога никто никогда не видел. Если мы любим друг друга, то Бог в нас пребывает, и любовь Его совершенна есть в нас.
\vs 1Jo 4:13 Что мы пребываем в Нем и Он в нас, узнаём из того, что Он дал нам от Духа Своего.
\vs 1Jo 4:14 И мы видели и свидетельствуем, что Отец послал Сына Спасителем миру.
\vs 1Jo 4:15 Кто исповедует, что Иисус есть Сын Божий, в том пребывает Бог, и он в Боге.
\vs 1Jo 4:16 И мы познали любовь, которую имеет к нам Бог, и уверовали в нее. Бог есть любовь, и пребывающий в любви пребывает в Боге, и Бог в нем.
\vs 1Jo 4:17 Любовь до того совершенства достигает в нас, что мы имеем дерзновение в день суда, потому что поступаем в мире сем, как Он.
\vs 1Jo 4:18 В любви нет страха, но совершенная любовь изгоняет страх, потому что в страхе есть мучение. Боящийся несовершен в любви.
\vs 1Jo 4:19 Будем любить Его, потому что Он прежде возлюбил нас.
\vs 1Jo 4:20 Кто говорит: <<я люблю Бога>>, а брата своего ненавидит, тот лжец: ибо не любящий брата своего, которого видит, как может любить Бога, Которого не видит?
\vs 1Jo 4:21 И мы имеем от Него такую заповедь, чтобы любящий Бога любил и брата своего.
\vs 1Jo 5:1 Всякий верующий, что Иисус есть Христос, от Бога рожден, и всякий, любящий Родившего, любит и Рожденного от Него.
\vs 1Jo 5:2 Что мы любим детей Божиих, узнаём из того, когда любим Бога и соблюдаем заповеди Его.
\vs 1Jo 5:3 Ибо это есть любовь к Богу, чтобы мы соблюдали заповеди Его; и заповеди Его нетяжки.
\vs 1Jo 5:4 Ибо всякий, рожденный от Бога, побеждает мир; и сия есть победа, победившая мир, вера наша.
\vs 1Jo 5:5 Кто побеждает мир, как не тот, кто верует, что Иисус есть Сын Божий?
\vs 1Jo 5:6 Сей есть Иисус Христос, пришедший водою и кровию и Духом, не водою только, но водою и кровию, и Дух свидетельствует о \bibemph{Нем}, потому что Дух есть истина.
\vs 1Jo 5:7 Ибо три свидетельствуют на небе: Отец, Слово и Святый Дух; и Сии три суть едино.
\vs 1Jo 5:8 И три свидетельствуют на земле: дух, вода и кровь; и сии три об одном.
\vs 1Jo 5:9 Если мы принимаем свидетельство человеческое, свидетельство Божие~--- больше, ибо это есть свидетельство Божие, которым Бог свидетельствовал о Сыне Своем.
\vs 1Jo 5:10 Верующий в Сына Божия имеет свидетельство в себе самом; не верующий Богу представляет Его лживым, потому что не верует в свидетельство, которым Бог свидетельствовал о Сыне Своем.
\vs 1Jo 5:11 Свидетельство сие состоит в том, что Бог даровал нам жизнь вечную, и сия жизнь в Сыне Его.
\vs 1Jo 5:12 Имеющий Сына (Божия) имеет жизнь; не имеющий Сына Божия не имеет жизни.
\rsbpar\vs 1Jo 5:13 Сие написал я вам, верующим во имя Сына Божия, дабы вы знали, что вы, веруя в Сына Божия, имеете жизнь вечную.
\vs 1Jo 5:14 И вот какое дерзновение мы имеем к Нему, что, когда просим чего по воле Его, Он слушает нас.
\vs 1Jo 5:15 А когда мы знаем, что Он слушает нас во всем, чего бы мы ни просили,~--- знаем и то, что получаем просимое от Него.
\vs 1Jo 5:16 Если кто видит брата своего согрешающего грехом не к смерти, то пусть молится, и \bibemph{Бог} даст ему жизнь, \bibemph{то есть} согрешающему \bibemph{грехом} не к смерти. Есть грех к смерти: не о том говорю, чтобы он молился.
\vs 1Jo 5:17 Всякая неправда есть грех; но есть грех не к смерти.
\rsbpar\vs 1Jo 5:18 Мы знаем, что всякий, рожденный от Бога, не грешит; но рожденный от Бога хранит себя, и лукавый не прикасается к нему.
\vs 1Jo 5:19 Мы знаем, что мы от Бога и что весь мир лежит во зле.
\vs 1Jo 5:20 Знаем также, что Сын Божий пришел и дал нам свет и разум, да позн\acc{а}ем Бога истинного и да будем в истинном Сыне Его Иисусе Христе. Сей есть истинный Бог и жизнь вечная.
\rsbpar\vs 1Jo 5:21 Дети! храните себя от идолов. Аминь.

\bibbookdescr{Tis}{
  inline={Завещание Иссахара,\\пятого сына Иакова и Лии},
  toc={Завещание Иссахара},
  bookmark={Завещание Иссахара},
  header={Завещание Иссахара},
  abbr={Исс}
}
\vs Tis 1:1
Список слов Иссахара.
Ибо он призвал сыновей своих и сказал им:
выслушайте, дети, Иссахара, отца вашего;
внемлите словам возлюбленного Господом.

\vs Tis 1:2
Родился я пятым сыном Иакова, платою за мандрагоры.
\vs Tis 1:3
Ибо Рувим, брат мой, принёс с поля мандрагоры,
и Рахиль, встретив его, взяла их.
\vs Tis 1:4
И плакал Рувим, и на голос его вышла Лия, мать моя.
\vs Tis 1:5
А были то яблоки благовонные, которые рождаются в земле Харана
на дне ложбин водных.
\vs Tis 1:6
И сказала Рахиль: не дам я тебе их, но мне самой нужны они,
дабы иметь детей.
Ведь обошёл меня Господь, и не рождала я сыновей Иакову.
\vs Tis 1:7
А яблок было 2.
И сказала Лия Рахили: да будет довольно тебе,
что взяла ты мужа моего, так возьмёшь ещё и это у меня?
\vs Tis 1:8
Отвечала ей Рахиль:
да будет Иаков с тобою в ночь эту за мандрагоры сына твоего.
\vs Tis 1:9
Сказала же ей Лия: мой Иаков, ибо я~--- жена юности его.
\vs Tis 1:10
И сказала Рахиль:
не возносись и не похваляйся,
ибо ко мне первой прилепился он и ради меня служил отцу моему 14 лет.
\vs Tis 1:11
И если бы не возросла хитрость на земле,
и злоба человеческая не преуспевала бы,
не была бы ты тою, что узрела лицо Иакова.
\vs Tis 1:12
Ибо ты не жена его, но хитростью опередила меня.
\vs Tis 1:13
И обманул меня отец мой, и удалил в ту ночь,
и не позволил мне видеть Иакова, ибо, если бы я там была,
не случилось бы того.
\vs Tis 1:14
Но за мандрагоры уступлю тебе на одну ночь Иакова.
\vs Tis 1:15
И познал Иаков Лию, и, зачав, родила она меня,
и от этой платы наречен я был Иссахаром.

\vs Tis 2:1
Тогда явился Иакову ангел Господень, говоря:
родит двоих детей Рахиль,
ибо пренебрегла она сообщением с мужем своим и воздержание избрала.
\vs Tis 2:2
И если бы мать моя Лия за сообщение с Иаковом не отдала 2 яблока,
то родила бы 8-ых сыновей, но из-за того родила 6-ых,
а Рахиль двоих, ибо в мандрагорах призрел на неё Господь.
\vs Tis 2:3
Ибо видел он, что ради детей желала она сойтись с Иаковом,
а не ради любострастия.
\vs Tis 2:4
И на другой день отдала она Иакова, чтобы взять и другую мандрагору.
Ибо в мандрагорах услышал Господь Рахиль.
\vs Tis 2:5
А она, возжелав их, не вкусила, но отнесла их в дом Господень
и отдала священнику, бывшему в то время.

\vs Tis 3:1
Когда же возмужал я, дети мои, жил я в прямоте сердечной,
и стал земледельцем отцу моему и братьям моим,
и приносил плоды с полей.
\vs Tis 3:2
И благословил меня отец мой, видя, что в простоте живу я.
\vs Tis 3:3
И не был я суетным в делах моих, ни завистником, ни клеветником
ближнему моему.
\vs Tis 3:4
Не наговаривал я никогда ни на кого, и не хулил жизнь никакого человека.
\vs Tis 3:5
45-и лет взял я себе жену, ибо труд снедал силы мои,
и не помышлял я о наслаждении от женщины, но от усталости засыпал я.
\vs Tis 3:6
И радовался простоте моей отец мой, ибо всякое первородное через
священника приносил я Господу, а после и отцу моему.
\vs Tis 3:7
И Господь утысячерял добро моё в руках моих,
и знал Иаков, отец мой, что Бог помогает простоте моей.
\vs Tis 3:8
Ибо всем бедным и страждущим уделял я от благ земли в простоте сердца моего.

\vs Tis 4:1
И ныне, внемлите мне, дети мои, и живите в простоте сердец
ваших, ибо узрел я, что в этом всякое благоугождение Господу.
\vs Tis 4:2
Простосердечный не стремится к золоту, и не хочет превзойти ближнего
богатством, и не домогается многообразных яств, и не желает разных одежд.
\vs Tis 4:3
Не хочет он приписать многих лет к своей жизни, но приемлет одну лишь
волю Божию.
\vs Tis 4:4
И духи соблазна ничего не могут против него,
ибо не воззрел он на красоту женскую, дабы не осквернить порчею ума своего.
\vs Tis 4:5
Не завидует он в помыслах своих, и клевета не изнуряет души его,
ни желание ненасытное ума его.
\vs Tis 4:6
Живет он в простоте души, всё зрит в прямоте сердца,
но оберегает очи свои от соблазна мирского,
дабы не видеть уклонений от заповедей Господних.

\vs Tis 5:1
Храните же, дети мои, закон Божий, и простоту обретайте, и в
беззлобии живите, не заботясь излишне о делах ближнего.
\vs Tis 5:2
Но возлюбите Господа и ближнего, а бедного и слабого жалейте.
\vs Tis 5:3
Склоните спины ваши к земледелию и утруждайте себя всяким земледелием,
принося Господу плоды с благодарностью.
\vs Tis 5:4
Ибо в первенцах плодов земных благословит вас Господь,
как благословил он всех святых от Авеля и доныне.
\vs Tis 5:5
Ибо не дастся вам иной удел, кроме тучности земли в трудах плодородия.
\vs Tis 5:6
Так и отец мой Иаков благословениями земли
и первенцев плодов благословил меня.
\vs Tis 5:7
А Левий и Иуда прославлены у Господа и в сынах Иакова.
И дал им наследие Господь:
Левию дал он священство, а Иуде~--- царство.
\vs Tis 5:8
Вы же слушайтесь их и пребывайте в прямодушии отца вашего.

\vs Tis 6:1
Знайте же, дети мои, что в последние времена
оставят сыновья ваши простоту, и погрязнут в алчности,
и отринут беззлобие, и совершат злодеяния,
и оставят заповеди Господа,
и прилепятся к Велиару.
\vs Tis 6:2
И оставят они земледелие, и последуют злым помыслам своим,
и рассеются среди народов, и рабами будут врагам своим.
\vs Tis 6:3
И вы скажите это детям вашим, дабы, если согрешат, тотчас обращались
вновь к Господу.
\vs Tis 6:4
Ибо он милостив, и пожалеет их, и вернёт в землю их.

\vs Tis 7:1
Вот, как видите вы, живу я 126 лет и не знал греха смертного.
\vs Tis 7:2
И кроме жены моей, не познавал я другой и не совершал блуда,
взирая очами моими.
\vs Tis 7:3
Вина соблазняющего не пил, не желал ничего из имущества ближнего моего.
\vs Tis 7:4
Хитрость не рождалась в сердце моём, ложь не входила на уста мои.
\vs Tis 7:5
Сострадал я всякому человеку скорбящему, и с нищим делил хлеб мой.
Благочестие творил я во все дни мои и правду хранил.
\vs Tis 7:6
Господа любил я и всякого человека всем сердцем моим.
\vs Tis 7:7
Так и вы делайте, дети мои, и всякий дух Велиаров бежит от вас,
и никакое дело злых людей не возобладает над вами,
и всякого зверя дикого усмирите,
если с вами будет Бог небес и земли, помогающий людям простосердечным.

\vs Tis 7:8
И сказав это сыновьям своим, завещал им, дабы отнесли его в Хеврон и
погребли там с отцами его.
\vs Tis 7:9
И вытянув ноги свои, почил он в старости прекрасной сном вечным.

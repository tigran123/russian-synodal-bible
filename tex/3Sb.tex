\bibbookdescr{3Sb}{
  inline={Третья книга Сивилл},
  toc={3-я Сивилл},
  bookmark={3-я Сивилл},
  header={3-я Сивилл},
  abbr={3~Сив}
}
\vs 3Sb 1:1 В небе на троне Cидящий превыше самих херувимов, 

\vs 3Sb 1:2 О Громовержец Блаженный, молю Тебя  дай мне покоя! 

\vs 3Sb 1:3 Вестница истины всей, я устала вещать непрестанно. 

\vs 3Sb 1:4 Но отчего мое сердце трепещет все снова и снова? 

\vs 3Sb 1:5 Бич меня нудит какой устами правдивое пенье

\vs 3Sb 1:6 Смертным открыто излить? Опять обо всем расскажу я, 

\vs 3Sb 1:7 Что бы Господь ни велел мне людям ясно поведать.

\vs 3Sb 1:8 Люди, в облике вашем творение Божие зримо, 

\vs 3Sb 1:9 Что ж вы блуждаете зря, отнюдь не желая тропою

\vs 3Sb 1:10 В жизни прямою идти, о Безсмертном Создателе помня? 

\vs 3Sb 1:11 Только Единый есть Бог  в небесах, никем не рожденный,

\vs 3Sb 1:12 Неизречен и невидим, Он видит все, что есть в мире. 

\vs 3Sb 1:13 Бог не был создан ничьею рукой никогда, ни из камня 

\vs 3Sb 1:14 И ни из золота и ни из кости слоновьей твореньем

\vs 3Sb 1:15 Не был. Извечность Свою Он сам доказал непреложно  

\vs 3Sb 1:16 Сущий ныне, был раньше и впредь всегда Он пребудет. 

\vs 3Sb 1:17 Смертным дано ли очам Всевышнего Бога увидеть? 

\vs 3Sb 1:18 Разве вместит кто-нибудь одно только имя услышать 

\vs 3Sb 1:19 Бога Великого, в небе Живущего, мира Владыки?

\vs 3Sb 1:20 Сущее все Он создал Своим словом  и небо, и море, 

\vs 3Sb 1:21 Неутомимое солнце, луну, что растет постепенно, 

\vs 3Sb 1:22 Множество звезд светоносных и матерь могучую Тефис, 

\vs 3Sb 1:23 Дни с ночами, источники рек, огонь негасимый. 

\vs 3Sb 1:24 Бог сотворил человека, который был первым из смертных,

\vs 3Sb 1:25 Имя ему  Адам, и эти буквы четыре

\vs 3Sb 1:26 Север, и Юг, и Восток, и Запад собой заполняют. 

\vs 3Sb 1:27 Сам Господь утвердил людские вид и обличье, 

\vs 3Sb 1:28 Сделал зверей Он, и гадов, и птиц, летающих в небе.

\vs 3Sb 1:29 Бога не чтите вы и не боитесь в своем заблужденье,

\vs 3Sb 1:30 Вы поклоняетесь змеям и жертвы приносите кошкам, 

\vs 3Sb 1:31 Также и всяким кумирам, людей изваяньям из камня, 

\vs 3Sb 1:32 И перед входами в храмы безбожные вечно сидите. 

\vs 3Sb 1:33 Сущего Бога побойтесь, Который все наблюдает, 

\vs 3Sb 1:34 О почитатели мерзости каменной, как позабыли

\vs 3Sb 1:35 Вы о Мессии, что создал, Безсмертный, и небо и землю? 

\vs 3Sb 1:36 Род людей кровожадных, лукавых, дурных, нечестивых, 

\vs 3Sb 1:37 Племя злонравное с полными лжи языками двойными, 

\vs 3Sb 1:38 Идолов чтите, прелюбы творите и злое коварство 

\vs 3Sb 1:39 В сердце своем замышляете вы, друг у друга крадете \ldots

\vs 3Sb 1:40 В мыслях безстыдство у вас, в груди  свирепое жало! 

\vs 3Sb 1:41 Тот, кто владеет богатством, ничем не поделится с бедным; 

\vs 3Sb 1:42 Злобы ужасной полны, все люди про верность забудут; 

\vs 3Sb 1:43 Многие вдовы и с ними замужние женщины даже 

\vs 3Sb 1:44 Тайно станут любить других, желая наживы,

\vs 3Sb 1:45 После ж и вовсе в открытую будут греху предаваться.

\vs 3Sb 1:46 Рим пока еще медлит, но время настанет  Египтом 

\vs 3Sb 1:47 Он овладеет. Тогда величайшее царство на землю 

\vs 3Sb 1:48 Скоро к людям сойдет, им Царь будет править Безсмертный, 

\vs 3Sb 1:49 Вождь священный придет, держащий скиптры земные,

\vs 3Sb 1:50 Сколько б времен ни прошло, все ж нет конца Его Царству. 

\vs 3Sb 1:51 Вспыхнет тогда у латинских мужей великая ярость; 

\vs 3Sb 1:52 Трое разрушат Рим, когда бросят жребий несчастный. 

\vs 3Sb 1:53 Люди погибнут все под гнетом собственных кровель, 

\vs 3Sb 1:54 Ибо огненный ливень с небес на землю прольется.

\vs 3Sb 1:55 О, я несчастная, страшный тот день  когда ж он настанет, 

\vs 3Sb 1:56 День, когда призовет на суд Властитель Небесный? 

\vs 3Sb 1:57 Стройтесь до времени, о города, и еще украшайтесь 

\vs 3Sb 1:58 Пышностью храмов, рынков, ристалищ, кумиров из камня, 

\vs 3Sb 1:59 Золота и серебра, и все это так сохранится

\vs 3Sb 1:60 Вплоть до горького дня  в парах удушливой серы 

\vs 3Sb 1:61 Люди тогда задохнутся \ldots\ Но лучше все по порядку 

\vs 3Sb 1:62 Я о несчастьях скажу, в каких городах они будут \ldots

\vs 3Sb 1:63 Явится вслед за тем Велиал, он придет из Себасты, 

\vs 3Sb 1:64 Станет горы сдвигать, усмирит и бурное море, 

\vs 3Sb 1:65 Солнце с луной светоносные он в небесах остановит,

\vs 3Sb 1:66 Тех, кто усоп, воскресит и много знамений чудных 

\vs 3Sb 1:67 Людям он явит, но мира конец еще не наступит  

\vs 3Sb 1:68 Будет все только соблазн, хоть, конечно, немало обманет 

\vs 3Sb 1:69 Верных сей Велиал Евреев и множество прочих

\vs 3Sb 1:70 Смертных мужей, что Закона и Божьего Слова не знают. 

\vs 3Sb 1:71 Но лишь начнут исполняться угрозы великого Бога, 

\vs 3Sb 1:72 Пламень, сжигающий все, потоками хлынет на землю; 

\vs 3Sb 1:73 Сгинут в пламени том Велиал и надменные люди  

\vs 3Sb 1:74 Все, кто веру речам и делам его даровали.

\vs 3Sb 1:75 Женщине миром всецело тогда завладеет, и станет 

\vs 3Sb 1:76 Он ей во всем подчиняться и слушаться безпрекословно. 

\vs 3Sb 1:77 После того вдова окажется мира царицей; 

\vs 3Sb 1:78 Бросит в море она серебро и злато людское, 

\vs 3Sb 1:79 Также всю медь и железо утопит о соленой пучине;

\vs 3Sb 1:80 Все элементы тогда с лица земного исчезнут. 

\vs 3Sb 1:81 Руки могучие Бог из чертогов эфирных протянет, 

\vs 3Sb 1:82 Свод небесный свернет, как будто свиток прочтенный; 

\vs 3Sb 1:83 Весь небосвод многовидный обрушится наземь и в море, 

\vs 3Sb 1:84 Огненный дождь будет лить и все сжигать непрестанно 

\vs 3Sb 1:85 Землю и воду спалит и небесную ось уничтожит. 

\vs 3Sb 1:86 Так творение Божье окажется сплавом единым, 

\vs 3Sb 1:87 После же снова на части разнимется для очищенья. 

\vs 3Sb 1:88 Больше не будут с небес никогда смеяться светила, 

\vs 3Sb 1:89 Ночь и заря упразднятся, и дней, заботами полных,

\vs 3Sb 1:90 Также не станет, исчезнут четыре времени года. 

\vs 3Sb 1:91 Век начнется великий, и Суд Всемощного Бога 

\vs 3Sb 1:92 Будет над миром, когда реченное все совершится.

\vs 3Sb 1:93 О судоходные воды, о суша вся от Востока 

\vs 3Sb 1:94 И до Заката  хоть больше уже не закатится солнце  

\vs 3Sb 1:95 Все подчинится Ему, в этот мир пришедшему снова, 

\vs 3Sb 1:96 Ибо сам Он познал Свою силу могучую первым.

\vs 3Sb 1:97 Все угрозы привел Безсмертный Бог в исполненье, 

\vs 3Sb 1:98 Коими людям грозил  они в земле Ассирийской 

\vs 3Sb 1:99 Башню построили  все меж собою согласными были  

\vs 3Sb 1:100 Страстно желали до звезд по этой башне добраться. 

\vs 3Sb 1:101 Тут Безсмертный ветрам повелел лететь что есть силы 

\vs 3Sb 1:102 К месту тому  ветра повергли огромное зданье,

\vs 3Sb 1:103 Ссору тогда меж собой учинили строители башни; 

\vs 3Sb 1:104 Вот почему с тех пор это место зовут Вавилоном.

\vs 3Sb 1:105 После крушения башни язык людской разделился 

\vs 3Sb 1:106 И превратился в обилье наречий разных, а дальше 

\vs 3Sb 1:107 Смертные, землю заполнив, ее поделили на царства. 

\vs 3Sb 1:108 То поколение было десятым с тех пор, как всемирный 

\vs 3Sb 1:109 Залил землю Потоп и первых людей уничтожил.

\vs 3Sb 1:110 Кронос, Титан и Япет над миром стали царями, 

\vs 3Sb 1:111 Люди их называли сынами Урана и Геи, 

\vs 3Sb 1:112 Имя земли и небес потому к царям прилагая, 

\vs 3Sb 1:113 Что наилучшими были они средь того поколенья. 

\vs 3Sb 1:114 Землю натрое всю разделили и бросили жребий,

\vs 3Sb 1:115 Каждый стал управлять в удел полученной частью; 

\vs 3Sb 1:116 Все отцу дали клятвы, и правильным было деленье, 

\vs 3Sb 1:117 Так что меж ними вражды не возникло. Но время настало 

\vs 3Sb 1:118 Умер старый отец. И, клятвы нарушив преступно, 

\vs 3Sb 1:119 Дети тогда меж собой учинили раздор величайший:

\vs 3Sb 1:120 Стали спорить, кому быть царем над всею землею. 

\vs 3Sb 1:121 Тут вражда началась у Титана и Кроноса злая. 

\vs 3Sb 1:122 Рея  сестра, мать Гея, Деметра и Афродита, 

\vs 3Sb 1:123 Та, что венки сплетать мастерица, и Гестия с ними, 

\vs 3Sb 1:124 Также Диона прекрасноволосая их помирили;

\vs 3Sb 1:125 Вместе собрали царей, их братьев и родичей разных, 

\vs 3Sb 1:126 Всех отцов и потомков, кто кровью был близок, созвали. 

\vs 3Sb 1:127 Те же держали совет и решили, что Кронос над всеми 

\vs 3Sb 1:128 Царствовать должен, поскольку он старше, благообразней. 

\vs 3Sb 1:129 Должен был Кронос Титану поклясться страшною клятвой

\vs 3Sb 1:130 В том, что мужского потомства иметь никогда он не будет, 

\vs 3Sb 1:131 Чтоб после смерти отца не мог его сын воцариться. 

\vs 3Sb 1:132 И когда срок наступал разрешиться от бремени Рее, 

\vs 3Sb 1:133 Подле Титаны садились и мальчиков всех разрывали 

\vs 3Sb 1:134 В клочья, а девочек всех у сосцов оставляли кормиться.

\vs 3Sb 1:135 Третьи роды настали у Реи, и первая Гера

\vs 3Sb 1:136 Вышла на свет, и, увидев младенца своими глазами, 

\vs 3Sb 1:137 Злые встали Титаны и все ушли восвояси. 

\vs 3Sb 1:138 Только затем появился ребенок пола мужского, 

\vs 3Sb 1:139 Рея тайком отослала его, чтобы спасся и вырос,

\vs 3Sb 1:140 Через трех жителей Крита во Фригию, клятвой связав их; 

\vs 3Sb 1:141 То, что сына вот так переслала, дало ему имя.

\vs 3Sb 1:142 Позже Рея спасла и другое дитя  Посейдона. 

\vs 3Sb 1:143 Третьим сыном Плутон был у этой женщины чудной  

\vs 3Sb 1:144 Недалеко от Додоны от бремени им разрешилась,

\vs 3Sb 1:145 Там, где несет свои воды Эвроп, который, с Пенеем 

\vs 3Sb 1:146 Слившись, в море течет и зовется Стигийской рекою. 

\vs 3Sb 1:147 Стало известно Титанам, что тайно смерти избегли 

\vs 3Sb 1:148 Дети, рожденные Реей от Кроноса. Тут возмутился 

\vs 3Sb 1:149 Сам Титан. Шестьдесят сыновей призвавши на помощь,

\vs 3Sb 1:150 Брата цепями сковал, а с ним и жену его Рею, 

\vs 3Sb 1:151 В землю упрятал обоих и там в оковах держал их. 

\vs 3Sb 1:152 Только узнали о том могучего Кроноса дети, 

\vs 3Sb 1:153 Подняли шум боевой и затеяли жаркую битву, 

\vs 3Sb 1:154 Битва великая та положила всем войнам начало,

\vs 3Sb 1:155 Первоначало всех войн среди смертных собою явила. 

\vs 3Sb 1:156 Вот за это наслал Господь на Титанов несчастье, 

\vs 3Sb 1:157 Сгинули все их потомки, но племя Кроноса  тоже. 

\vs 3Sb 1:158 Время затем совершило свой круг, и царство Египта , 

\vs 3Sb 1:159 Было воздвигнуто, вслед появились новые царства 

\vs 3Sb 1:160 Персов, Мидян, Эфиопов, в Ассирии вкруг Вавилона,

\vs 3Sb 1:161 У Македонцев и снова в Египте, в конце же  у Римлян. 

\vs 3Sb 1:162 Бог Всемогущий тогда вложил мне пророчество в душу, 

\vs 3Sb 1:163 И возвестить повелел по всей земле это слово, 

\vs 3Sb 1:164 Дабы властителям стало известно, что будет в грядущем.

\vs 3Sb 1:165 Первое то мне открыл Господь Единый, какие

\vs 3Sb 1:166 Царства людские возникнут и сколько их будет на свете. 

\vs 3Sb 1:167 Первым дом Соломонов над Азией всей воцарится, 

\vs 3Sb 1:168 Персии, Фригии станет владыкою и Финикии, 

\vs 3Sb 1:169 Островитяне, Карийцы, Мизийцы ему подчинятся,

\vs 3Sb 1:170 Он покорит и Лидийцев, богатое золотом племя... 

\vs 3Sb 1:171 Эллинов род, злодеев надменных, господствовать будет 

\vs 3Sb 1:172 Дальше, а после него  великое пестрое племя 

\vs 3Sb 1:173 Тех Македонцев, что тучи войны надвинут на смертных; 

\vs 3Sb 1:174 Бог небесный, однако, их всех уничтожит под корень.

\vs 3Sb 1:175 Но вслед за ними грядет другого царства начало: 

\vs 3Sb 1:176 Белый, могучий народ с берегом Гесперийского моря 

\vs 3Sb 1:177 Выйдет, разные страны захватит и в ужас повергнет 

\vs 3Sb 1:178 Многих, а в душах царей он страх надолго поселит. 

\vs 3Sb 1:179 Золота и серебра в городах награблено будет

\vs 3Sb 1:180 Тут немало, но вновь появится золото в мире

\vs 3Sb 1:181 И серебро, а потом и других украшений в достатке. 

\vs 3Sb 1:182 Смертные много тогда претерпят, но в наказанье

\vs 3Sb 1:183 Низко падут нечестивцы надменные, мерзостью жуткой 

\vs 3Sb 1:184 Жизнь их наполнится вся, мужчина с мужчиною станут

\vs 3Sb 1:185 Здесь предаваться разврату, а малых детей на продажу 

\vs 3Sb 1:186 Будут в позорных домах выставлять. Великое горе 

\vs 3Sb 1:187 К людям придет в те дни и посеет страшную смуту, 

\vs 3Sb 1:188 Все устои разрушит и злом это царство наполнит; 

\vs 3Sb 1:189 Страсть к наживе лихой, позорная алчность охватят

\vs 3Sb 1:190 Многие страны, а больше других  Македонскую землю. 

\vs 3Sb 1:191 Долго у них в чести коварство и ненависть будут, 

\vs 3Sb 1:192 Это продлится до царства седьмого по счету в Египте  

\vs 3Sb 1:193 Родом должен быть Эллин в то время Египта владыка  

\vs 3Sb 1:194 Сила появится вновь у народа великого Бога:

\vs 3Sb 1:195 Праведной жизни пути он смертным указывать станет...

\vs 3Sb 1:196 Но отчего же Господь вещать меня заставляет,

\vs 3Sb 1:197 Что будет первым несчастьем для всех человеков, что дальше

\vs 3Sb 1:198 С ними случится, где бедствий конец и где их источник?

\vs 3Sb 1:199 Первыми примут Титаны от Бога жестокую кару: 

\vs 3Sb 1:200 Мощного Кроноса им сыновья отомстят по заслугам, 

\vs 3Sb 1:201 Ибо сковали Титаны отца их и мать вероломно. 

\vs 3Sb 1:202 Позже у Эллинов власть захватят злые тираны, 

\vs 3Sb 1:203 И воцарятся у них надменные прелюбодеи 

\vs 3Sb 1:204 И нечестивцы, которым все доброе чуждо, а войнам 

\vs 3Sb 1:205 Впредь не будет конца. Фригийцы грозные сгинут, 

\vs 3Sb 1:206 Бедствий черные дни для Троянского града настанут. 

\vs 3Sb 1:207 Горе придет не замедлив и к Персам и к Ассирийцам, 

\vs 3Sb 1:208 В Ливию и к Эфиопам прошествует через Египет, 

\vs 3Sb 1:209 Быть ему и у Карийцев, в Памфилии также... да что я 

\vs 3Sb 1:210 Перечисляю народы?  у всех людей будет горе. 

\vs 3Sb 1:211 Только конец одному, как вскоре второе нагрянет 

\vs 3Sb 1:212 К людям несчастье, но я вначале скажу о первейшем.

\vs 3Sb 1:213 Горе постигнет и тех, кто возле великого храма, 

\vs 3Sb 1:214 Что Соломон воздвиг, живут в благочестье  и предки 

\vs 3Sb 1:215 Праведны были у них, и вот теперь поведу я 

\vs 3Sb 1:216 Речь об этом народе, земле его, предках и ясно 

\vs 3Sb 1:217 Все опишу для тебя, коварный и суетный смертный!

\vs 3Sb 1:218 Город есть на Востоке, зовется он Уром Халдейским, 

\vs 3Sb 1:219 Праведной жизни народ происходит оттуда, те люди

\vs 3Sb 1:220 Мыслили здраво всегда и много благого творили. 

\vs 3Sb 1:221 Их не заботит ничуть светил небесных вращенье, 

\vs 3Sb 1:222 Не помышляют они о земных чудовищах жутких. 

\vs 3Sb 1:223 Ни о манящих глубинах соленых вод Океана. 

\vs 3Sb 1:224 То, как птицы клюют, иль то, как люди чихают,

\vs 3Sb 1:225 Их не волнует, они чародеям и магам не верят,

\vs 3Sb 1:226 Чревовещатели ложью своей соблазнить их не в силах. 

\vs 3Sb 1:227 Не признают они там ни халдейских гаданий по звездам, 

\vs 3Sb 1:228 Ни астрономии. Нет, все то почитают обманом, 

\vs 3Sb 1:229 Чем занимаются изо дня в день неразумные люди,

\vs 3Sb 1:230 Души свои упражняя в вещах совершенно ненужных. 

\vs 3Sb 1:231 Этим своим заблужденьям они еще обучают 

\vs 3Sb 1:232 Разных глупцов, оттого много бед бывает, ведь люди, 

\vs 3Sb 1:233 Сбившись с благого пути, забывают о праведной жизни. 

\vs 3Sb 1:234 Те же, о ком говорю, почитают все справедливость

\vs 3Sb 1:235 И добродетель, не думают, как бы им стать побогаче 

\vs 3Sb 1:236 (Смертным нажива несет лишь зло, и голод, и войны), 

\vs 3Sb 1:237 Верная мера у них во всем в городах и в селеньях. 

\vs 3Sb 1:238 Здесь никто по ночам ничего у других не ворует, 

\vs 3Sb 1:239 Коз, овец и волов не бывает, чтоб тут угоняли,

\vs 3Sb 1:240 В поле земли никогда не отнимет сосед у соседа, 

\vs 3Sb 1:241 Самый богатый у них не обидит того, кто беднее, 

\vs 3Sb 1:242 Горя не причинит вдове, а напротив, поможет 

\vs 3Sb 1:243 Хлебом в нужде, вином и оливками  не поскупится. 

\vs 3Sb 1:244 Есть тут немало счастливцев, но, если кто-то несчастен,

\vs 3Sb 1:245 С бедным своим урожаем поделится летом имущий. 

\vs 3Sb 1:246 Ибо послушны они реченью великого Бога  

\vs 3Sb 1:247 Общей создал для всех небесный Царь эту землю.

\vs 3Sb 1:248 В дни, как Египет покинут и двинутся в путь по пустыне 

\vs 3Sb 1:249 Эти двенадцать колен, от Господа сопровождены;

\vs 3Sb 1:250 Будет дано им: в ночи столп огня озарит их дорогу, 

\vs 3Sb 1:251 Скрытые обликом, днем пойдут они безопасно. 

\vs 3Sb 1:252 Посланный Богом народу, его предводителем станет 

\vs 3Sb 1:253 Славный муж Моисей, который ребенком в болоте 

\vs 3Sb 1:254 Был царицею найден  она его воспитала,

\vs 3Sb 1:255 Сыном назван; и вот, с ним вышел народ из Египта. 

\vs 3Sb 1:256 Бог к Синайской горе привел их и с неба народу 

\vs 3Sb 1:257 Дал закон благочестья, на двух записав его досках. 

\vs 3Sb 1:258 И повелел: того, кто не станет блюсти предписаний,

\vs 3Sb 1:259 Или закон покарает, иль руки накажут людские,

\vs 3Sb 1:260 Если ж и скрыться сумеет, расплата его не минует. 

\vs 3Sb 1:261 [Общей создал для всех небесный Царь эту землю, 

\vs 3Sb 1:262 В сердце им всем Господь благое вложил помышленье.] 

\vs 3Sb 1:263 Только добрым сторицей воздаст хлебодарная пашня, 

\vs 3Sb 1:264 Так отмерил сам Бог. Но и добрых людей ожидают

\vs 3Sb 1:265 Беды, им не избегнуть никак ужасного мора.

\vs 3Sb 1:266 И побежишь ты тогда, покинув храм свой чудесный, 

\vs 3Sb 1:267 Ибо священную землю оставить велят тебе судьбы. 

\vs 3Sb 1:268 Жить придется тебе в земле Ассирийской, увидишь 

\vs 3Sb 1:269 Жен и малых детей рабами, людям враждебным.

\vs 3Sb 1:270 Все тут богатство погибнет, не сможешь добыть пропитанья; 

\vs 3Sb 1:271 Будут тобою полны все земли и воды морские, 

\vs 3Sb 1:272 Но не полюбит никто обычай твой и законы. 

\vs 3Sb 1:273 Вся же твоя страна опустеет: холм укрепленный, 

\vs 3Sb 1:274 Храм великого Бога и длинные мощные стены

\vs 3Sb 1:275 Рухнут тогда во прах, а причиною  то, что не чтил ты 

\vs 3Sb 1:276 Господом данный закон священный, но в заблужденье 

\vs 3Sb 1:277 Идолам мерзким служил и ничуть Того не боялся, 

\vs 3Sb 1:278 Кто породил всех богов и людей  Безсмертного Бога; 

\vs 3Sb 1:279 Чтить ты Его не желал, почитал изваяния смертных.

\vs 3Sb 1:280 Вот за это земля плодородная будет пустыней 

\vs 3Sb 1:281 Семь десятков времен, и во храме чудес не увидят. 

\vs 3Sb 1:282 Но в конце тебя ждут великая радость и слава: 

\vs 3Sb 1:283 Все исполнят Господь и смертный, когда не предашь ты 

\vs 3Sb 1:284 Веры в священный закон, полученный некогда свыше, 

\vs 3Sb 1:285 Ноги устанут твои, но светлого дня ты достигнешь.

\vs 3Sb 1:286 Царь будет послан от Бога с высокого неба на землю,

\vs 3Sb 1:287 Каждого станет судить в крови и в пламенном свете. 

\vs 3Sb 1:288 Только один парод, одно лишь царское племя 

\vs 3Sb 1:289 Не поколеблется тут. Ему предназначено править 

\vs 3Sb 1:290 В смене времен и начать строительство нового храма. 

\vs 3Sb 1:291 Всякий владыка Персидский тут помощь оказывать станет 

\vs 3Sb 1:292 Бронзою, кованым прочным железом и золотом даже, 

\vs 3Sb 1:293 Ибо Господь сновиденья священные ночью пошлет им. 

\vs 3Sb 1:294 Так, воздвигнувшись вновь, святыня пребудет, как прежде.

\vs 3Sb 1:295 В сердце утихло моем звучанье божественной песни, 

\vs 3Sb 1:296 И обратила мольбы я к Творцу, чтоб Он дал мне покоя.

\vs 3Sb 1:297 Но Всемогущий опять вложил пророчество в душу

\vs 3Sb 1:298 И возвестить повелел по всей земле это слово,

\vs 3Sb 1:299 Дабы властителям стало известно, что будет в грядущем.

\vs 3Sb 1:300 Первое, что наказал мне Господь Единый поведать,  

\vs 3Sb 1:301 Сколько горьких напастей отмерил Он Вавилону 

\vs 3Sb 1:302 Карою за разграбленье великого Божьего храма. 

\vs 3Sb 1:303 Горе тебе, Вавилон и племя мужей Ассирийских! 

\vs 3Sb 1:304 Шум ужасный услышат родившие грешников земли,

\vs 3Sb 1:305 Клич боевой принесет погибель внезапную людям, 

\vs 3Sb 1:306 Бог, моих песен владыка, сразит их могучим ударом. 

\vs 3Sb 1:307 Бог к тебе, Вавилон, сойдет из высей воздушных, 

\vs 3Sb 1:308 Спустится Он со святых небес на грешную землю.

\vs 3Sb 1:309 Гнев Господень сулит сынам твоим вечную гибель.

\vs 3Sb 1:310 Станешь тогда ты, как если б и не было вовсе на свете 

\vs 3Sb 1:311 Города никогда такого, наполнишься кровью, 

\vs 3Sb 1:312 Вспомнишь, как сам проливал кровь добрых и справедливых, 

\vs 3Sb 1:313 Что и доселе еще вопиет к высокому небу.

\vs 3Sb 1:314 Страшный удар потрясет твои жилища, Египет, 

\vs 3Sb 1:315 Ты никогда и представить не мог, что случится такое! 

\vs 3Sb 1:316 Меч тяжелый тебя пронзит посредине, а следом 

\vs 3Sb 1:317 Голод и мор и рассеянье будут, идя чередою, 

\vs 3Sb 1:318 В царство седьмое губить страну, и так ты исчезнешь.

\vs 3Sb 1:319 Гог и Магог, увы, увы тебе, край Эфиопский! 

\vs 3Sb 1:320 Реки текут по тебе сейчас, но в будущем хлынет 

\vs 3Sb 1:321 Кровь потоками здесь, и, затоплена черною кровью, 

\vs 3Sb 1:322 Судным местом тогда среди смертных ты прозовешься.

\vs 3Sb 1:323 Ливия, горе тебе, и землям горе и водам!

\vs 3Sb 1:324 Дочери Запада, вас настигнет день несчастливый, 

\vs 3Sb 1:325 Вам не уйти от борьбы тяжелой, она неотступно

\vs 3Sb 1:326 Будет преследовать вас, и суд наступит ужасный.

\vs 3Sb 1:327 И поневоле придется погибнуть вам всем в это время.

\vs 3Sb 1:328 Ибо бессмертного Бога жилища вы источили

\vs 3Sb 1:329 И растерзали его вконец зубами стальными, 

\vs 3Sb 1:330 Край свой увидишь тогда в страну мертвецов превращенным:

\vs 3Sb 1:331 Сгинут одни от войны и по воле несчастного рока, 

\vs 3Sb 1:332 Голод иных уничтожит, чума и бешенство вражье, 

\vs 3Sb 1:333 Все твои города, все земли пустынею станут. 

\vs 3Sb 1:334 Вспыхнет звезда на Заходе  она наречется кометой  

\vs 3Sb 1:335 Вестницей станет она сражений, голода, смерти, 

\vs 3Sb 1:336 Гибели славных вождей и прочих людей знаменитых.

\vs 3Sb 1:337 Знаменья будут тогда даны величайшие смертным: 

\vs 3Sb 1:338 Течь прекратит Танаис в Меотиду струей многоводной,

\vs 3Sb 1:339 Высохнув, русло его плодородною пашнею станет,

\vs 3Sb 1:340 В озеро воды польются по множеству малых протоков. 

\vs 3Sb 1:341 Много в почве возникнет провалов и пропастей, рухнут 

\vs 3Sb 1:342 В них города со всеми людьми. Эта страшная участь 

\vs 3Sb 1:343 В Азии Смирну постигнет, Иас, Кебрен, Пандонию, 

\vs 3Sb 1:344 Антиохию, Эфес, Колофон, Никею, Скиагру,

\vs 3Sb 1:345 Астипалею, Синоп, Иераполь, счастливую Газу. 

\vs 3Sb 1:346 Так же погибнут в Европе Танагра и Меропея, 

\vs 3Sb 1:347 Так пропадут Микены, Магнезия и Антигона. 

\vs 3Sb 1:348 Знайте тогда, что вскоре конец настанет Египту, 

\vs 3Sb 1:349 Прежнее лето всегда будет лучшим для Александрийцев.

\vs 3Sb 1:350 Сколько бы Рим ни взял с покоренной Азии дани, 

\vs 3Sb 1:351 Втрое больше ему возвратить сокровищ придется 

\vs 3Sb 1:352 Азии, ибо надменным она победителем станет. 

\vs 3Sb 1:353 Много богатств возьмет с Азиатов народ Италийский,

\vs 3Sb 1:354 Двадцатикратно, однако, он собственной рабскою службой

\vs 3Sb 1:355 Должен будет вернуть, в нищете пребывая великой. 

\vs 3Sb 1:356 В золоте, в роскоши ты, о дочь Латинского Рима, 

\vs 3Sb 1:357 С множеством женихов сколь часто вином упивалась!  

\vs 3Sb 1:358 В жены тебя отдадут не в пышном наряде  служанкой, 

\vs 3Sb 1:359 Срежет тебе госпожа копну волос твоих пышных.

\vs 3Sb 1:360 Восторжествует тогда справедливость, и с неба на землю

\vs 3Sb 1:361 Сброшено будет одно, из праха восстанет другое  

\vs 3Sb 1:362 Слишком уж люди погрязли в пороке и жизни нечестной.

\vs 3Sb 1:363 Делос невидимым станет, а Самос в песок превратится, 

\vs 3Sb 1:364 Рим руинами будет  исполнятся все предсказанья. 

\vs 3Sb 1:365 Больше ни слова о Смирне  пускай себе погибает 

\vs 3Sb 1:366 От преступлений вождей, неразумных и несправедливых.

\vs 3Sb 1:367 В Азии тихий покой воцарится, счастливою станет 

\vs 3Sb 1:368 В те времена и Европа: блаженную жизнь и здоровье 

\vs 3Sb 1:369 Небо людям пошлет вместо злого снега и града,

\vs 3Sb 1:370 Даст оно много зверей и птиц и ползучих в достатке.

\vs 3Sb 1:371 О, сколь счастливы те мужи и жены, которым 

\vs 3Sb 1:372 Жить доведется в тот век, похожий на дивную сказку.

\vs 3Sb 1:373 Благозаконие и справедливость со звездного неба 

\vs 3Sb 1:374 К людям придут, и тогда воцарится всем смертным на пользу

\vs 3Sb 1:375 Мудрое мыслей единство, а с ним  любовь и доверье,

\vs 3Sb 1:376 Гостеприимства законы блюсти станут люди; при этом 

\vs 3Sb 1:377 Вовсе исчезнут нужда и насилие, больше не будет 

\vs 3Sb 1:378 Зависти, гнева, насмешек, безумства и преступлений; 

\vs 3Sb 1:379 Ссоры, жестокая брань, грабеж по ночам и убийства 

\vs 3Sb 1:380 В общем, всякое зло в те дни на земле прекратится. 

\vs 3Sb 1:381 Но Македонцы сулят всей Азии тяжкие беды; 

\vs 3Sb 1:382 Вырастет и для Европы еще великое горе  

\vs 3Sb 1:383 Горе от племени мнимых Кронидов и рабского рода; 

\vs 3Sb 1:384 И Вавилон, хорошо укрепленный, захватит их войско.

\vs 3Sb 1:385 Этих людей назовут владыками целого света,

\vs 3Sb 1:386 Но погибнет их царство от страшных бед, не оставив 

\vs 3Sb 1:387 Даже законов потомкам, по разным разбредшимся странам.

\vs 3Sb 1:388 Муж коварный в то время в счастливую Азию вступит, 

\vs 3Sb 1:389 Будет носить на плечах порфирное он одеянье.

\vs 3Sb 1:390 Молния в мир его принесет. Потому-то, свирепый, 

\vs 3Sb 1:391 Дикий и непостоянный, ярмо для Азии злое 

\vs 3Sb 1:392 Этот муж приготовит, и кровью убийства сырая 

\vs 3Sb 1:393 Здесь упьется земля, но Аид усмирит кровопийцу. 

\vs 3Sb 1:394 Род, который под корень хотелось ему уничтожить,

\vs 3Sb 1:395 Сам погубит потом его, а единственный корень  

\vs 3Sb 1:396 Срубит после один из десятка рогов кровожадный, 

\vs 3Sb 1:397 После же новый побег он с прежними рядом насадит. 

\vs 3Sb 1:398 Но, погубив отца порфироносного рода, 

\vs 3Sb 1:399 Тоже погибнет от рук детей, заговорщиков дерзких,

\vs 3Sb 1:400 И воцарится затем тот рог, что посажен был рядом.

\vs 3Sb 1:401 Знаменье явится вскоре Фригийской земле плодоносной: 

\vs 3Sb 1:402 Род огромный и злой  потомки матери Реи  

\vs 3Sb 1:403 Вечным мнивший себя, ибо вырос из корня сухого, 

\vs 3Sb 1:404 В ночь исчезнет одну. В эту ночь земли Колебатель

\vs 3Sb 1:405 Почву разверзнет в том граде, которому люди позднее 

\vs 3Sb 1:406 Имя дадут Дорилейон. Все это в древней случится 

\vs 3Sb 1:407 Черной Фригийской земле, не раз слезами политой. 

\vs 3Sb 1:408 Времени этому люди дадут Колебателя имя, 

\vs 3Sb 1:409 Ибо Он щели разверзнет земные и стены разрушит.

\vs 3Sb 1:410 Знаки все это дурные  за ними беды начнутся.

\vs 3Sb 1:411 Множество разных народов придет со своими вождями, 

\vs 3Sb 1:412 Чтобы в земле воевать, где предки живут Энеадов. 

\vs 3Sb 1:413 Станут добычей они снедаемым жадностью людям.

\vs 3Sb 1:414 Горе тебе, Илион! Эриния вырастит в Спарте

\vs 3Sb 1:415 Ветвь чудесную, чья красота всем известною станет. 

\vs 3Sb 1:416 Но породит она бурю над Азией и над Европой, 

\vs 3Sb 1:417 Ты, Илион, больше всех услышишь тут плача и стонов, 

\vs 3Sb 1:418 Вечно будут, однако, виновницу помнить потомки.

\vs 3Sb 1:419 Явится старец затем и напишет много неправды, 

\vs 3Sb 1:420 Ложно и город родной назовет. Хоть света не узрят 

\vs 3Sb 1:421 Очи его, но с великим умом и, мысль облекая 

\vs 3Sb 1:422 Ясно в слова, он напишет  но, Хиос своим называя, 

\vs 3Sb 1:423 Речь о том поведет, что у стен Илиона свершилось. 

\vs 3Sb 1:424 Ложь его будет правдивой казаться: слова и размеры 

\vs 3Sb 1:425 Он ведь из книг моих почерпнет, сперва прочитав их.

\vs 3Sb 1:426 Очень красиво опишет воителей подвиги старец,

\vs 3Sb 1:427 Гектора, сына Приама, и сына Пелея, Ахилла,

\vs 3Sb 1:428 Также и прочих, кто в этой войне подвизался отважно.

\vs 3Sb 1:429 Изобразит он, что боги сражавшимся там помогали, 

\vs 3Sb 1:430 Самую разную ложь сочинит для людей скудоумных.

\vs 3Sb 1:431 Павшим у стен Илиона причтется великая слава:

\vs 3Sb 1:432 Поочередно певец про оба войска расскажет.

\vs 3Sb 1:433 Много зла причинит Ликийцам Локра потомство;

\vs 3Sb 1:434 Ты, Халкидон, у морской теснины лежащий, погибнешь 

\vs 3Sb 1:435 В час, как придет к тебе дитя земли Этолийской.

\vs 3Sb 1:436 Кизик, море отнимет твое богатство и счастье;

\vs 3Sb 1:437 Плохо придется тебе, Византий, стоящий напротив

\vs 3Sb 1:438 Азии; стоном и кровью до края ты будешь наполнен.

\vs 3Sb 1:439 С той вершины горы, что над Ликией высится, воды 

\vs 3Sb 1:440 Хлынут потоками вниз, из твердого выбившись камня.

\vs 3Sb 1:441 Чтобы утихли они, надо сбыться отцов предсказаньям.

\vs 3Sb 1:442 Город обильной вином Пропонтиды, увы тебе, Кизик! 

\vs 3Sb 1:443 Бурной Риндакской волною ты будешь залит и потоплен.

\vs 3Sb 1:444 Родос, и твой век недолог, хотя и немалое время 

\vs 3Sb 1:445 Рабства ты не познаешь и славиться будешь богатством, 

\vs 3Sb 1:446 И не оспорит никто твоего господства над морем. 

\vs 3Sb 1:447 Все же станешь добычей снедаемым жадностью людям, 

\vs 3Sb 1:448 За красоту и богатство  ужасное иго претерпишь.

\vs 3Sb 1:449 В Лидии вздрогнет земля, и Персия вся сокрушится;

\vs 3Sb 1:450 Сколько несчастий Европу и Азию тут ожидают!

\vs 3Sb 1:451 Царь Сидонский и много других владык кровожадных 

\vs 3Sb 1:452 За море смерть понесут с собою  на Самос и дальше. 

\vs 3Sb 1:453 В море много земли потоки кровавые смоют, 

\vs 3Sb 1:454 Жены и девы в красивых нарядах горько заплачут;

\vs 3Sb 1:455 Жалкую долю свою проклянут они, ибо навеки 

\vs 3Sb 1:456 Эти любимых отцов, а те  сыновей потеряют.

\vs 3Sb 1:457 Будет для Кипра знак  ужасное землетрясенье, 

\vs 3Sb 1:458 Множество душ оно Аиду отдаст в одночасье!

\vs 3Sb 1:459 Рухнут и мощные стены с Эфесом соседнего Тралла 

\vs 3Sb 1:460 За преступления их обитателей, злых и жестоких. 

\vs 3Sb 1:461 Воды горячие с неба на землю польются, и станет 

\vs 3Sb 1:462 Впитывать почва ту влагу и запах удушливой серы.

\vs 3Sb 1:463 Самос царский дворец построит, как сроки настанут.

\vs 3Sb 1:464 Не чужеземный Арей твоим, Италия, бедам 

\vs 3Sb 1:465 Будет причиной, но кровь, с которою тяжко бороться, 

\vs 3Sb 1:466 Кровь родных сыновей разорит тебя без пощады! 

\vs 3Sb 1:467 Все об этом позоре узнают, и, в пепле простершись, 

\vs 3Sb 1:468 Ты погибнешь  и раньше могла все это предвидеть: 

\vs 3Sb 1:469 Матерь добрых людей, зверенышей диких вскормила.

\vs 3Sb 1:470 Муж-paзоритель когда придет из Италии новый, 

\vs 3Sb 1:471 Лаодикия, во прах преклонишь тогда ты колени. 

\vs 3Sb 1:472 Славный город Карийский, у струй прекрасного Лика, 

\vs 3Sb 1:473 Горько оплакав надменного предка, навеки умолкнешь.

\vs 3Sb 1:474 Племя Фракийцев взойдет на вершины высокого Гема.

\vs 3Sb 1:475 Лихо придет и к Кампанцам  опустошительный голод; 

\vs 3Sb 1:476 Древний день своего основания также оплачут 

\vs 3Sb 1:477 Кирн и Сардиния, их удары холодного ветра, 

\vs 3Sb 1:478 Посланы Богом святым, в пучину соленую сбросят, 

\vs 3Sb 1:479 Станут они в волнах морским обитальцам добычей.

\vs 3Sb 1:480 Горе! сколько Аид невест прекрасных добудет,

\vs 3Sb 1:481 Сколько непогребенных юнцов не отпустят глубины! 

\vs 3Sb 1:482 О, невинные дети! О, тяжкое злато в пучине!

\vs 3Sb 1:483 Царский возникнет род в земле Мизийцев счастливой.

\vs 3Sb 1:484 Жизнь Кархедона, однако, не долго вовсе продлится. 

\vs 3Sb 1:485 Жалобный стон разнесется среди Галатов, и будет 

\vs 3Sb 1:486 На Тенедосе несчастье последнее самым ужасным. 

\vs 3Sb 1:487 И Сикион, и Коринф возгордятся лаем доспехов, 

\vs 3Sb 1:488 Но не минует их участь вести бесславные войны.

\vs 3Sb 1:489 В сердце утихло моем звучанье божественной песни,

\vs 3Sb 1:490 Но Всемогущий опять вложил мне пророчество в душу 

\vs 3Sb 1:491 И возвестить повелел по всей земле это слово.

\vs 3Sb 1:492 Горе вам всем, Финикийцы, мужчинам и женщинам горе! 

\vs 3Sb 1:493 Также и всем городам Побережья морского  не сможет 

\vs 3Sb 1:494 Светлой дорогой никто из вас до солнца достигнуть.

\vs 3Sb 1:495 Больше семей не возникнет и новых детей не родится  

\vs 3Sb 1:496 За дерзновенный язык и жизнь порочную смертных, 

\vs 3Sb 1:497 Наглых и беззаконных хулителей и святотатцев. 

\vs 3Sb 1:498 Страшные лживые речи вели преступники эти, 

\vs 3Sb 1:499 И против Господа мерзкий мятеж они учиняли;

\vs 3Sb 1:500 Карой за все злодеянья Господь бичевые удары

\vs 3Sb 1:501 Страшно обрушит на смертных от края земли и до края. 

\vs 3Sb 1:502 Горькая ждет их судьба, когда города и жилища 

\vs 3Sb 1:503 До основанья огнем небесным выжжены будут.

\vs 3Sb 1:504 Горе, о горе тебе, печали и скорби обитель, 

\vs 3Sb 1:505 Крит, от удара ты рухнешь ужасного, сгинешь навеки. 

\vs 3Sb 1:506 Ты задымишься тогда на глазах у целого света, 

\vs 3Sb 1:507 И не оставит уж пламя тебя, пока не исчезнешь.

\vs 3Sb 1:508 Фракия, горе тебе  рабынею жалкою станешь: 

\vs 3Sb 1:509 Время настанет, Галаты набег совершат на Элладу 

\vs 3Sb 1:510 Вместе с Дарданцами, тут-то несчастье тебя ожидает  

\vs 3Sb 1:511 Беды несла ты другим, теперь же возмездие примешь.

\vs 3Sb 1:512 Горе вам, Гог и Магог и все племена по соседству  

\vs 3Sb 1:513 Марсы, Анги, иные  вас ждет ужасная участь.

\vs 3Sb 1:514 Много крушений грядут к Ликийцам, Мизийцам, Фригийцам,

\vs 3Sb 1:515 К жителям Лидии и Памфилийцам, а также и к людям 

\vs 3Sb 1:516 Варварской речи  ко всем Эфиопам, Маврам, Арабам, 

\vs 3Sb 1:517 Каппадокийцам. Но что я пророчить каждому стану 

\vs 3Sb 1:518 Жребий его? Ибо всем племенам, населяющим землю, 

\vs 3Sb 1:519 Страшный удар ниспошлет и тяжкую кару Всевышний.

\vs 3Sb 1:520 Варварский, чуждый народ появится в Эллинском крае  

\vs 3Sb 1:521 Многим голов не сносить в то время мужам знаменитым, 

\vs 3Sb 1:522 Много жирных овец у смертных угнано будет, 

\vs 3Sb 1:523 Зычно ревущих быков, коней и мулов без счета. 

\vs 3Sb 1:524 Крепкой постройки дома предав огню беззаконно,

\vs 3Sb 1:525 Жителей в рабство насильно угонят враги на чужбину. 

\vs 3Sb 1:526 Эллин, увидишь картины ужасные: жен беззащитных 

\vs 3Sb 1:527 Вместе с детьми из покоев выбрасывать станут на землю 

\vs 3Sb 1:528 Нежным коленом, и жены в оковах вражьих претерпят 

\vs 3Sb 1:529 Весь позор униженья у варваров. Нет им защиты

\vs 3Sb 1:530 Здесь от расправы, никто им в страшной беде не поможет! 

\vs 3Sb 1:531 Враг все богатство твое и все достоянье присвоит, 

\vs 3Sb 1:532 И затрясутся колени твои, если это увидишь. 

\vs 3Sb 1:533 Сто человек побегут, чтоб спастись, но один всех погубит. 

\vs 3Sb 1:534 Пятеро гневом ужасным тогда вскипят, но позорно

\vs 3Sb 1:535 Между собой препираться начнут и оружье подымут 

\vs 3Sb 1:536 Друг против друга на радость врагам, а Элладе  на горе. 

\vs 3Sb 1:537 В рабстве придется Элладе сносить тяжелое иго; 

\vs 3Sb 1:538 Всех будут мучить война и с нею губительный голод. 

\vs 3Sb 1:539 Небо высокое медью Господь покроет, а землю

\vs 3Sb 1:540 Высушит всю, и в железо поверхность ее превратится. 

\vs 3Sb 1:541 И, убедившись, что больше нельзя ни вспахать, ни посеять, 

\vs 3Sb 1:542 Горько люди заплачут. Но Бог великий, что создал 

\vs 3Sb 1:543 Все на свете, теперь обрушит жестокое пламя 

\vs 3Sb 1:544 Вниз, и тогда лишь треть людей на земле уцелеет.

\vs 3Sb 1:545 О, для чего ты, Эллада, на смертных вождей полагалась? 

\vs 3Sb 1:546 Им не дано ведь никак избегнуть конца рокового; 

\vs 3Sb 1:547 Что ж ублажаешь дарами никчемными тех, кто погибнет, 

\vs 3Sb 1:548 А изваяниям жертвы приносишь? Отколь научилась 

\vs 3Sb 1:549 Делать такое, презрев Лицо всемогущего Бога?

\vs 3Sb 1:550 Имя Родителя общего чти, не оставь в небреженье! 

\vs 3Sb 1:551 Правили тысячу лет и еще пять сотен вдобавок 

\vs 3Sb 1:552 Много надменных царей в Элладе, и вот от них-то 

\vs 3Sb 1:553 Первых учиться злу неразумные смертные стали: 

\vs 3Sb 1:554 Идолов мертвых воздвигли для тех, кто сами невечны;

\vs 3Sb 1:555 Этим в умы вам вложили пустые ложные мысли. 

\vs 3Sb 1:556 Но когда Божий гнев разразится над вами внезапно, 

\vs 3Sb 1:557 Сразу узнаете тут Лицо всемогущего Бога. 

\vs 3Sb 1:558 Тут же все души людские, наполнив воздух стенаньем, 

\vs 3Sb 1:559 Руки к широкому небу с мольбою протягивать станут,

\vs 3Sb 1:560 И о защите молить Царя великого в небе,

\vs 3Sb 1:561 И вопрошать: кто же их от страшного гнева избавит?

\vs 3Sb 1:562 Должен еще ты узнать, и в уме твоем пусть сохранится, 

\vs 3Sb 1:563 Сколько несчастий несут с собою бегущие годы

\vs 3Sb 1:564 Зычно ревущих быков и коров соберет в изобилье 

\vs 3Sb 1:565 К храму великого Бога Эллада, и больше не станет 

\vs 3Sb 1:566 Злобных побоищ на землях ее и гнетущего страха, 

\vs 3Sb 1:567 Голод и рабское иго тогда же вскоре исчезнут. 

\vs 3Sb 1:568 Род нечестивцев, однако, дотоле продлится, покуда 

\vs 3Sb 1:569 Срок судеб истечет и день настанет реченный. 

\vs 3Sb 1:570 Жертвуйте Богу тогда лишь, когда все исполнится, ибо 

\vs 3Sb 1:571 То, чего Он желает, небывшим остаться не может, 

\vs 3Sb 1:572 Все заставит Господь по воле Его совершиться.

\vs 3Sb 1:573 Явится племя святое людей, благочестия полных, 

\vs 3Sb 1:574 Господу истинно верных и волей и помыслом всяким.

\vs 3Sb 1:575 Храм великого Бога почтят кроплением влаги 

\vs 3Sb 1:576 И возжиганием дыма от тучных жертв, гекатомбой 

\vs 3Sb 1:577 Тою священной, когда закалают быков превосходных, 

\vs 3Sb 1:578 Жирных баранов, овец и едва родившихся агнцев; 

\vs 3Sb 1:579 Многое с мыслью благой на алтарь великий возложат.

\vs 3Sb 1:580 И по законам великого Бога, храня справедливость,

\vs 3Sb 1:581 Будут счастливую жизнь проводить в городах и на пашнях. 

\vs 3Sb 1:582 Их Бессмертный возвысит, пророками сделав, и станут 

\vs 3Sb 1:583 Радость большую нести всем смертным, на свете живущим. 

\vs 3Sb 1:584 Только им даровал разуменье благое Всевышний,

\vs 3Sb 1:585 Веру и лучшие чувства людские вложил Он в их души. 

\vs 3Sb 1:586 Делу рук человечьих они молиться не станут, 

\vs 3Sb 1:587 Золото, медь, серебро, слоновья кость не прельстят их, 

\vs 3Sb 1:588 Крашеным идолам шатким из дерева, камня и глины, 

\vs 3Sb 1:589 Изображеньям зверей и всему, что смертных бездумных

\vs 3Sb 1:590 Вводит легко в соблазн, не будут они поклоняться. 

\vs 3Sb 1:591 Но поутру ото сна пробудясь, омывают водою 

\vs 3Sb 1:592 Руки и чистыми их всегда к небесам воздевают. 

\vs 3Sb 1:593 Чтут одного лишь Владыку  Безсмертного, Вечного Бога, 

\vs 3Sb 1:594 Мать и отца вслед за Ним; и больше, чем прочие люди,

\vs 3Sb 1:595 В мысли имеют они сохранять целомудрие ложа, 

\vs 3Sb 1:596 С детями пола мужского не водят позорную дружбу, 

\vs 3Sb 1:597 Как Египтяне и как Финикийцы, а также Латины, 

\vs 3Sb 1:598 Эллины разных племен и множество прочих народов: 

\vs 3Sb 1:599 Персы, Галаты и целая Азия  все, кто забыли

\vs 3Sb 1:600 И преступили священный закон Безсмертного Бога. 

\vs 3Sb 1:601 Людям пошлет Господь за эти грехи наказанье: 

\vs 3Sb 1:602 Пагубу разную, голод, страдания, жалкие стоны, 

\vs 3Sb 1:603 Войны жестокие, мор и слезы от боли ужасной. 

\vs 3Sb 1:604 К вечному ибо Отцу всех тех, кто мир населяет,

\vs 3Sb 1:605 Не пожелали почтенья иметь, а идолов чтили; 

\vs 3Sb 1:606 Будет, однако, пора, и дело рук своих сами 

\vs 3Sb 1:607 Сбросят в расселины гор, дабы скрыть позор величайший. 

\vs 3Sb 1:608 Новый владыка тогда воцарится в Египте, и станет 

\vs 3Sb 1:609 Он седьмым от начала правления Эллинов, то есть

\vs 3Sb 1:610 С той поры, как начнется здесь власть мужей Македонских. 

\vs 3Sb 1:611 Тут горящим орлом великий царь Азиатский 

\vs 3Sb 1:612 Явится, землю покрыв и пешим войском и конным; 

\vs 3Sb 1:613 Все на пути своем уничтожит и злом переполнит. 

\vs 3Sb 1:614 Царская власть сокрушится и Египте тогда, а захватчик,

\vs 3Sb 1:615 Всю добычу забрав, уплывет за широкое море.

\vs 3Sb 1:616 Люди пред Господом Богом, великим и вечным, колена 

\vs 3Sb 1:617 Белые тут преклонят, опустясь на кормилицу-землю. 

\vs 3Sb 1:618 Рухнут и сгинут в пожаре творения рук человечьих; 

\vs 3Sb 1:619 Но получат взамен от Бога великую радость

\vs 3Sb 1:620 Смертные, ибо земля, деревья и пастбища будут

\vs 3Sb 1:621 Истинный плод приносить, и тогда появится вдоволь 

\vs 3Sb 1:622 Сладкого меда, вина, молока белоснежного, хлеба  

\vs 3Sb 1:623 Главное, хлеба, ведь он  наивысшее благо для смертных.

\vs 3Sb 1:624 Только медлить не смей, злонравный и суетный смертный,

\vs 3Sb 1:625 Но обратись покаянно, моли прощенья у Бога! 

\vs 3Sb 1:626 В жертву Ему приноси козлят и ягнят первородных, 

\vs 3Sb 1:627 Сотни быков приноси, пока сменяются годы. 

\vs 3Sb 1:628 Господа ты умоляй низойти и явить Свою милость, 

\vs 3Sb 1:629 Ибо единый Он Бог, и быть другого не может.

\vs 3Sb 1:630 Чти справедливость всегда, никому не делай обиды  

\vs 3Sb 1:631 Это  Безсмертного Бога веление людям несчастным. 

\vs 3Sb 1:632 Но берегись пробужденья всевышнего Божьего гнева 

\vs 3Sb 1:633 В час, как на смертных чума нагрянет, несущая гибель, 

\vs 3Sb 1:634 И не уйдет человек тогда от расплаты ужасной.

\vs 3Sb 1:635 Встанет царь на царя, победит и землю отнимет, 

\vs 3Sb 1:636 Племя на племя пойдет, правители многих погубят, 

\vs 3Sb 1:637 Все вожди разбегутся по разным странам в то время. 

\vs 3Sb 1:638 Облик изменит земля, засилье варваров диких 

\vs 3Sb 1:639 Опустошит всю Элладу, ее плодородная почва

\vs 3Sb 1:640 Всяких лишится богатств; но вражды причиною будут 

\vs 3Sb 1:641 Золото и серебро  готовит многие беды 

\vs 3Sb 1:642 Любостяжательство людям, нет пастыря хуже, чем алчность.

\vs 3Sb 1:643 \ldots\ и на чужбине они останутся без погребенья, 

\vs 3Sb 1:644 Дикие звери и злые стервятники здесь растерзают

\vs 3Sb 1:645 Их тела, а когда реченное все совершится, 

\vs 3Sb 1:646 Этих усопших останки земля широкая скроет. 

\vs 3Sb 1:647 Но уж не вспашет ту землю никто, и никто не засеет; 

\vs 3Sb 1:648 Скажет несчастьем своим о позоре множества смертных \ldots

\vs 3Sb 1:649 В долгой смене времен и годов обращенья не станет 

\vs 3Sb 1:650 Копий, щитов и другого оружья, привычного людям; 

\vs 3Sb 1:651 Чтобы разжечь костры, железом древес не коснутся.

\vs 3Sb 1:652 Бог на землю пошлет царя, что придет от Восхода, 

\vs 3Sb 1:653 Злую войну прекратит этот царь по всей поднебесной, 

\vs 3Sb 1:654 Жизни лишая одних и клятвы другим выполняя. 

\vs 3Sb 1:655 Но не по воле своей совершит он деяния эти, 

\vs 3Sb 1:656 А подчиняясь благим веленьям великого Бога \ldots\

\vs 3Sb 1:657 Свой же народ Господь одарит чудесным богатством  

\vs 3Sb 1:658 Золотом и серебром и прекрасной одеждой пурпурной; 

\vs 3Sb 1:659 И плодородная почва и даже соленое море

\vs 3Sb 1:660 Много благ принесут. Но снова цари друг на друга 

\vs 3Sb 1:661 Примутся зло замышлять и творить его в гневе великом; 

\vs 3Sb 1:662 Будет недобрая зависть в обычае смертных несчастных. 

\vs 3Sb 1:663 Против все той же земли цари поднимут народы, 

\vs 3Sb 1:664 Только в поход соберутся они себе на погибель.

\vs 3Sb 1:665 Храм великого Бога святой и мужей наилучших 

\vs 3Sb 1:666 Всех истребить захотят; и, в землю эту явившись, 

\vs 3Sb 1:667 Город цари окружат и средь войск своих непокорных 

\vs 3Sb 1:668 Сядут на трон и начнут приносить нечестивые жертвы. 

\vs 3Sb 1:669 В этот-то час и раздастся с небес голос Бога могучий

\vs 3Sb 1:670 К диким и глупым народам, и суд начнется над ними, 

\vs 3Sb 1:671 Суд великого Бога, Который бессмертной рукою 

\vs 3Sb 1:672 Их умертвит. С высоты мечи огневые на землю 

\vs 3Sb 1:673 Он обрушит тогда; и огромные факелы будут 

\vs 3Sb 1:674 Всех людей освещать, внезапно средь них появившись.

\vs 3Sb 1:675 И от Господней руки земля, что все порождает, 

\vs 3Sb 1:676 Тут сотрясется, и все затрепещут  рыбы морские, 

\vs 3Sb 1:677 Звери земные и птиц несчетные стаи и воды, 

\vs 3Sb 1:678 Также и души людей содрогнутся, как только увидят 

\vs 3Sb 1:679 Лик Безсмертного Бога  и ужас будет великий.

\vs 3Sb 1:680 Гор высоких вершины, холмы и крутые обрывы 

\vs 3Sb 1:681 Он сокрушит, и черный всем взорам явится Тартар. 

\vs 3Sb 1:682 Полными трупов предстанут ущелья туманные в скалах, 

\vs 3Sb 1:683 Брызнет кровь из камней и вниз потоками хлынет 

\vs 3Sb 1:684 С гор по теснинам и быстро долины собою затопит.

\vs 3Sb 1:685 Рухнут крепкие стены, ведь их неразумные люди 

\vs 3Sb 1:686 Строили, вовсе не зная закона великого Бога, 

\vs 3Sb 1:687 Ни суда, что их ждет. Потому-то резню учинили 

\vs 3Sb 1:688 И в безумии вы на Святыню подняли копья. 

\vs 3Sb 1:689 Будет судить Господь вас всех мечом и войною,

\vs 3Sb 1:690 Пламенем и дождем, затопляющим землю, и серой, 

\vs 3Sb 1:691 С неба летящей, камнями огромными, градом ужасным 

\vs 3Sb 1:692 И умерщвленьем повсюду животных четвероногих. 

\vs 3Sb 1:693 Ясно люди поймут, что суд Безсмертного Бога 

\vs 3Sb 1:694 К ним пришел, и тогда умирающих стоны и вопли

\vs 3Sb 1:695 Землю всю огласят, и, лишаясь речи от страха,

\vs 3Sb 1:696 Кровью умывшись своей, погибнут. И почва впитает 

\vs 3Sb 1:697 Кровь, а тела мертвецов разорвут ненасытные звери.

\vs 3Sb 1:698 Все это мне повелел Господь великий и вечный 

\vs 3Sb 1:699 Так предсказать. И не может реченное мною не сбыться 

\vs 3Sb 1:700 Иль не исполниться в чем-то, ведь все задумано Богом  

\vs 3Sb 1:701 Чуждый обмана, витает Господний Дух в поднебесной.

\vs 3Sb 1:702 Дети великого Бога вокруг Святыни в покое 

\vs 3Sb 1:703 Будут жизнь проводить, Господним деяниям рады, 

\vs 3Sb 1:704 Ибо Творец, справедливый Судья и Владыка, дарует

\vs 3Sb 1:705 Многое: Он на защиту народа встанет, могучий, 

\vs 3Sb 1:706 Словно высокую стену, огонь кругом воздвигая. 

\vs 3Sb 1:707 Ни в городах, ни в селеньях война грозить им не будет; 

\vs 3Sb 1:708 Злую руку вражды отразит святая десница  

\vs 3Sb 1:709 Ведь Безсмертный Боец обороной им станет надежной.

\vs 3Sb 1:710 Скажут все города и все острова, что Всевышний 

\vs 3Sb 1:711 Этих мужей возлюбил великой любовью, и будут 

\vs 3Sb 1:712 И небосвод, и луна, и солнце, водимое Богом, 

\vs 3Sb 1:713 Им помогать во всем и печься о них неустанно. 

\vs 3Sb 1:714 И сотрясется в те дни земля, что все порождает.

\vs 3Sb 1:715 И людские уста воспоют сладкогласные гимны:

\vs 3Sb 1:716 Все к земле припадем, обратимся с молитвою к Богу! 

\vs 3Sb 1:717 Он  Безсмертный Царь, Владыка великий и вечный, 

\vs 3Sb 1:718 Он  наш единый Господь, пошлем же к Господнему храму, 

\vs 3Sb 1:719 Будем все вместе внимать законам Всевышнего Бога,

\vs 3Sb 1:720 Ибо нет ничего справедливей, чем эти законы. 

\vs 3Sb 1:721 Мы заблудились, свернув с дороги, указанной Богом, 

\vs 3Sb 1:722 Стали творения рук человеческих чтить неразумно, 

\vs 3Sb 1:723 Статуям смертных людей деревянным кланяться стали. 

\vs 3Sb 1:724 Веру обретшие души такие крики исторгнут.

\vs 3Sb 1:725 Люди Господа, все падем устами на землю,

\vs 3Sb 1:726 В каждом доме Творцу воспоем прекрасные гимны! 

\vs 3Sb 1:727 Все оружие вражье по миру всему соберем мы. 

\vs 3Sb 1:728 Смогут семь лет совершить свое обращенье по кругу 

\vs 3Sb 1:729 Копья, шлемы, щиты и множество разных доспехов,

\vs 3Sb 1:730 Луки и стрелы. Все это уйдет из рук нечестивцев, 

\vs 3Sb 1:731 Чтобы разжечь костры, железом древес не коснутся.

\vs 3Sb 1:732 Ты же в гордыне своей перестань возноситься, Эллада! 

\vs 3Sb 1:733 Жалкая, остерегись, моли милосердного Бога,

\vs 3Sb 1:734 Свой неразумный народ войной не веди в этот город  

\vs 3Sb 1:735 Пусть спокойно живут и земле великого Бога.

\vs 3Sb 1:736 А Камарину не трогай  ей лучше быть неподвижной, 

\vs 3Sb 1:737 И не буди леопарда, не то может зло приключиться. 

\vs 3Sb 1:738 Будь же воздержна, и пусть в груди твоей не проснется 

\vs 3Sb 1:739 Гордый дух и надменный, стремящийся в жаркую битву. 

\vs 3Sb 1:740 Господу верно служи и радости станешь причастна: 

\vs 3Sb 1:741 Ибо, как срок совершится, отмеренный точно судьбою,

\vs 3Sb 1:742 Суд Безсмертного Бога в тот день настанет для смертных.

\vs 3Sb 1:743 Власть Господня в тот день на добрых людей обратится. 

\vs 3Sb 1:744 Плод наилучший земля, которая все порождает, 

\vs 3Sb 1:745 Смертным даст  изобилье пшеницы, вина и оливок. 

\vs 3Sb 1:746 Множество сладкого меда пошлют небеса человеку, 

\vs 3Sb 1:747 Будет древесных плодов в достатке и тучной скотины: 

\vs 3Sb 1:748 Коз, и коров, и овец с ягнятами малыми вместе. 

\vs 3Sb 1:749 Вырвутся из-под земли молока белоснежного струи.

\vs 3Sb 1:750 Будут полны богатств города, а поля плодоносны;

\vs 3Sb 1:751 Шум боевой и резня ужасная вовсе исчезнут,

\vs 3Sb 1:752 С тяжким стоном земля уж больше не содрогнется,

\vs 3Sb 1:753 Войны и засуха миру угрозою быть перестанут,

\vs 3Sb 1:754 С ними же голод и град, что бьет урожай, упразднятся.

\vs 3Sb 1:755 Мир на землю сойдет великий, неведомый прежде, 

\vs 3Sb 1:756 Станут друзьями теперь цари до скончания века, 

\vs 3Sb 1:757 Люди по всей земле одним жить будут законом, 

\vs 3Sb 1:758 Что установит Господь, на небе правящий звездном; 

\vs 3Sb 1:759 Этим законом Безсмертный дела людские измерит,

\vs 3Sb 1:760 Ибо единый Он Бог, и быть другого не может, 

\vs 3Sb 1:761 И сожжет Он огнем человеческий род нечестивый.

\vs 3Sb 1:762 Так поспешите же, люди, слова мои сердцем усвоить: 

\vs 3Sb 1:763 Идолов мерзких оставьте, живому Богу служите; 

\vs 3Sb 1:764 Остерегайтеся блуда и грязного ложа мужского, 

\vs 3Sb 1:765 Если дети родятся, растите их, не убивайте  

\vs 3Sb 1:766 Все прегрешения эти влекут Божий гнев за собою.

\vs 3Sb 1:767 Бог ниспошлет наконец всем людям вечное царство: 

\vs 3Sb 1:768 Дав священный закон Его почитавшим как должно, 

\vs 3Sb 1:769 Пообещал Он, что мир и землю для них Он откроет 

\vs 3Sb 1:770 И распахнет им врата блаженства  великая радость, 

\vs 3Sb 1:771 Вечно здравый рассудок и мысли светлые станут 

\vs 3Sb 1:772 Их достояньем. Тогда к жилищу великого Бога

\vs 3Sb 1:773 С целого света дары принесут и ладан воскурят. 

\vs 3Sb 1:774 Люди не спросят уже к другому дому дороги,

\vs 3Sb 1:775 Кроме того, что велел Господь почитать Своим верным. 

\vs 3Sb 1:776 Сыном великого Бога те люди его называют.

\vs 3Sb 1:777 Все пути по равнинам и все обрывы крутые, 

\vs 3Sb 1:778 Горные выси и волны, что на море дико бушуют,  

\vs 3Sb 1:779 Станет все в эти дни легко человеку доступным.

\vs 3Sb 1:780 Ибо у добрых людей покой и мир воцарятся, 

\vs 3Sb 1:781 Меч упразднят пророки великого Бога, и сами 

\vs 3Sb 1:782 Смертных станут они судить и царить справедливо. 

\vs 3Sb 1:783 Праведным станет тогда и все богатство людское. 

\vs 3Sb 1:784 Вот каковы будут суд и власть Всевышнего Бога.

\vs 3Sb 1:785 Возвеселись и ликуй, о дева! Вечную радость 

\vs 3Sb 1:786 Тот даровал тебе, Кто создал небо и землю. 

\vs 3Sb 1:787 Он, в тебе поселившись, твоим станет светом безсмертным. 

\vs 3Sb 1:788 Овцы вместе с волками в горах травою питаться 

\vs 3Sb 1:789 Будут, а дикие барсы  пастись с козлятами вместе.

\vs 3Sb 1:790 Сможет теленок с медведем в загоне быть безопасно, 

\vs 3Sb 1:791 Лев плотоядный мякиной, как вол, насытится в яслях; 

\vs 3Sb 1:792 Малые дети его, связав, поведут за собою, 

\vs 3Sb 1:793 Зверь этот станет ручным по воле великого Бога. 

\vs 3Sb 1:794 Змей с младенцем уснет в одной постели спокойно

\vs 3Sb 1:795 И не сделает зла  Господня рука не позволит.

\vs 3Sb 1:796 Ясное знаменье я укажу тебе, чтобы узнал ты

\vs 3Sb 1:797 Время, когда конец всему земному настанет.

\vs 3Sb 1:798 Звездный свод озарять мечи огромные будут 

\vs 3Sb 1:799 В небе встанут они с Востока и с Запада ночью.

\vs 3Sb 1:800 Пепел внезапно и пыль посыплются сверху на землю 

\vs 3Sb 1:801 В целом мире, и днем угаснет солнца сиянье, 

\vs 3Sb 1:802 Вместо того луна в небесах появится тотчас, 

\vs 3Sb 1:803 Бледным лучом своим осветив земную поверхность. 

\vs 3Sb 1:804 Кровь, просочившись из камня, вам тоже даст несомненный

\vs 3Sb 1:805 Знак, а в тумане предстанет сражение пеших и конных, 

\vs 3Sb 1:806 Схожее с травлей зверей и во мгле виденью подобно. 

\vs 3Sb 1:807 Значит, вскоре Господь, живущий в небе, положит 

\vs 3Sb 1:808 Войнам конец. Но каждый пусть жертвы Богу приносит.

\vs 3Sb 1:809 Из Ассирийской земли, от мощных стен Вавилона 

\vs 3Sb 1:810 Я в Элладу явилась, как некий огонь, в исступленье,

\vs 3Sb 1:811 Чтобы всем смертным изречь, чем Бог угрожает во гневе \ldots\

\vs 3Sb 1:812 Смертным все предскажу в загадках, внушенных мне Богом.

\vs 3Sb 1:813 Станут тогда говорить в Элладе, что я  чужеземка, 

\vs 3Sb 1:814 Что родилась я в Эритрах, безстыдная; и нарекут мне

\vs 3Sb 1:815 Имя Сивиллы и скажут, как будто Гностом и Киркой 

\vs 3Sb 1:816 Мать и отца моих звали, а я  безумная лгунья. 

\vs 3Sb 1:817 Но когда все свершится, слова мои вспомните сразу 

\vs 3Sb 1:818 И не безумной сочтете  великой пророчицей Бога. 

\vs 3Sb 1:819 Мне Господь не открыл того, что родителям прежде

\vs 3Sb 1:820 Он поведал моим, но то, что было вначале, 

\vs 3Sb 1:821 И грядущее все вложил мне в душу Всевышний, 

\vs 3Sb 1:822 Чтобы пророчить могла я о бывшем и будущем людям. 

\vs 3Sb 1:823 Ибо, покрыли когда всю землю воды Потопа, 

\vs 3Sb 1:824 Славный муж лишь один в то время спасся от смерти:

\vs 3Sb 1:825 Дом деревянный построив, по водам проплыл и с собою 

\vs 3Sb 1:826 Взял он птиц и зверей, чтобы мир наполнился снова. 

\vs 3Sb 1:827 Мне же стать довелось невесткой этого мужа; 

\vs 3Sb 1:828 Первое с ним совершилось, ему же последнее ясным 

\vs 3Sb 1:829 Сделал Господь  и во всем уста мои будут правдивы.

\thispagestyle{empty}
\begin{center}
\normalsize\bfseries
О КНИГАХ КАНОНИЧЕСКИХ И НЕКАНОНИЧЕСКИХ
\end{center}

%\begin{multicols}{2}
Христианская Библия состоит из двух частей: Ветхого Завета и Нового Завета,
Книги Ветхого Завета писались на протяжении более тысячи лет до Рождества
Христова на древнееврейском языке, книги Нового Завета написаны на греческом
языке в I в. по Р.~Х.

В Ветхом Завете есть книги канонические и неканонические.
Основное различие между ними в том, что книги канонические более древние,
написаны в XV--V вв.~до Р.~Х., а книги неканонические, т.~е. не вошедшие в
канон, в собрание священных книг, написаны позже, в IV--I вв.~до Р.~Х.
Ветхозаветный канон создавался постепенно.
Первым собирателем священных книг воедино считают Ездру (V в.~до Р.~Х.).
В III в.~до Р.~Х. -- I в. по Р.~Х. ветхозаветный канон приобрел тот вид,
который существует в современной еврейской, так называемой массоретской,
Библии (она содержит лишь Ветхий Завет; массореты, хранители предания,
закончили работу над ней в VIII в. по Р.~Х.).
В ней 39 книг, которые разделены на три отдела: \bibemph{закон},
\bibemph{пророки} и \bibemph{писания} (этими словами в древности называли
Ветхий Завет,-- см. \bibref{Mat 7:12}; \bibref{Luk 24:44}).
\bibemph{Закон} (по-еврейски тор\'а) содержит Пятикнижие Моисея: Бытие, Исход, Левит,
Числа и Второзаконие.
\bibemph{Пророки} (неби\'им) делятся на первых или старших, которым принадлежат книги
Иисуса Навина, Судей, две книги Самуила (в нашей Библии это 1 и 2 Царств) и
две книги Царей (наши 3 и 4 Царств; в христианской Церкви книги старших
пророков, а также Руфь, Есфирь, Ездры, Неемии и Паралипоменон принято
считать историческими книгами), и на последних или младших, которые в
свою очередь подразделяются на великих пророков и малых.
Книги трёх великих пророков: Исайя, Иеремия, Иезекииль;
двенадцати малых: Осия.  Иоиль, Амос, Авдий, Иона, Михей, Наум, Аввакум,
Софония, Аггей, Захария и Малахия.
\bibemph{Писания} (кетуб\'им) составляют: Псалмы, Притчи, Иов, Песнь Песней, Руфь,
Плач Иеремии, Екклезиаст, Есфирь, Даниил. Ездра, Неемия и Летописи
($=$ 1 и 2 Паралипоменон).

После возвращения евреев из плена вавилонского, т.~е. после V в.~до Р.~Х.,
было составлено и написано еще несколько книг на еврейском и греческом языках.
В канон еврейских священных книг их уже не включили, но они вошли, как
полезные и назидательные, в Септуагинту, т.~е. греческий перевод Библии.  

Этот перевод был сделан в III--II вв.~до Р.~Х. для александрийских
евреев-эллинистов и иудеев рассеяния, т.~е. живущих вне Палестины, которые
уже забывали родной язык и говорили по-гречески
(см. предисловие к Книге Иисуса сына Сирахова).
Древнее предание говорит о 70-ти (или 72-х) толковниках, т.~е. переводчиках,
которые перевели священные книги с еврейского языка на греческий, поэтому и
перевод этот называется <<переводом семидесяти>> или по-гречески <<Септуагинта>>.
Он в некоторых деталях отличается от масоретского текста, так как массореты и
переводчики на греческий пользовались разными списками (рукописями) древнего текста.

Библейские книги на латинском языке были известны уже в конце II века по Р.~Х.
Блаж. Иероним перевел их заново в конце IV -- начале V в., и этот перевод, известный
под названием <<Вульгата>>, получил широкое распространение в Католической Церкви.

Перевод книг Священного Писания на славянский язык начат был святыми равноапостольными Кириллом
и Мефодием в IX в.
Современная славянская Библия используемая в церковном служении представляет собой
перепечатку Елизаветинского издания 1751--1756~гг., в котором текст Ветхого Завета
был выверен по греческой Библии.

На русский язык Библия переведена в середине XIX века.
Канонические книги переводились с еврейского массоретского текста с
дополнениями и вариантами из Септуагинты, а неканонические --- с греческого,
за исключением Третьей книги Ездры, переведенной с латинского, так как этой
книги нет ни в еврейской, ни в греческой Библии.
Русская православная Библия, как и славянская, содержит все 39 канонических и
11 неканонических книг Ветхого Завета.

Что касается деления книг на главы, то оно было введено уже в XIII веке
двумя западными исследователями Библии --- кардиналом Стефаном Лангтоном и
доминиканцем Гуго де Сен-Шира.

Деление глав на стихи ввел в середине XVI века парижский типограф
Робертус Стефанус. \bibemph{(Новая толковая Библия, 1990, Ленинград)}
%\end{multicols}

\bibbookdescr{2En}{
  inline={Вторая книга Еноха},
  toc={2-я Еноха},
  bookmark={2-я Еноха},
  header={2-я Еноха},
  abbr={2~Ено}
}
\vs 2En 1:1
Мужа мудрого, великого книжника, которого взял Господь, дабы он увидел и возлюбил высшее житие, непреходящее царство премудрого и великого Бога Вседержителя,
\vs 2En 1:2
дабы стал он свидетелем превеликого, многоочитого и непоколебимого престола Господа, пресветлого предстояния слуг Господа и степеней их господства,
\vs 2En 1:3
геенны огненной, неисчислимого состава воинства небесного, многого множества стихий и различных видений, несказанного пения воинства Херувов, и света безмерного.
\vs 2En 1:4
Когда мне исполнилось сто шестьдесят пять лет, сказал Енох, у меня родился сын Мафусаил. Затем я прожил еще двести лет. А вся моя жизнь продолжалась триста шестьдесят пять лет.
\vs 2En 1:5
В первый месяц, в известный день первого месяца я, Енох, был в доме своем один.
\vs 2En 1:6
И когда лежал я на ложе своем и спал, обильная скорбь охватила сердце мое, и я сказал: Вот, очи мои испускают слезы (ибо во сне я не мог понять, что означает сия скорбь). Что будет со мною?"
\vs 2En 1:7
И явились мне два мужа столь великих, каких никогда не видел я на земле: лица их сияли подобно солнцу, а очи их были словно свечи горящие,
\vs 2En 1:8
из уст их исходил как бы огонь, и одеяния их были как струящаяся пена, светлее золота крылья их, и руки их белее снега.
\vs 2En 1:9
И стали они у изголовья моего и позвали меня по имени.
\vs 2En 1:10
И пробудился я от сна моего, и вот, мужи те стоят предо мною наяву.
\vs 2En 1:11
И встал я поспешно и поклонился им, и вспыхнуло лицо мое от страха перед увиденным.
\vs 2En 1:12
И сказали мне мужи те: Ободрись, Енох, не бойся, Господь вечный послал нас к тебе, в день сей восходишь ты с нами на небо.
\vs 2En 1:13
Скажи же сынам своим все, что нужно им сделать на земле; и пусть никто из дома твоего не ищет тебя до тех пор, пока не возвратит тебя к ним Господь".

\vs 2En 2:1
И послушался я их, и пошел, и призвал сыновей своих Мафуселу и Ригима, и поведал им то, что сказали мне мужи те:
\vs 2En 2:2
И вот я знаю, дети, что я не знаю, куда иду и что встретит меня.
\vs 2En 2:3
Вы же, дети мои, не отступайте от Бога, и пред лицем Господним ходите, и соблюдайте суды Его,
\vs 2En 2:4
не отвергайте жертв спасения вашего и не отвергнет Господь труд рук ваших;
\vs 2En 2:5
не лишайте даров Господа и не лишит Господь приращений Своих в хранилищах ваших!
\vs 2En 2:6
Благословляйте Господа первенцами от стад скота вашего и будете благословенны перед Господом во веки.
\vs 2En 2:7
Не отступайте от Господа и не поклоняйтесь богам суетным, не сотворившим ни небес, ни земли;
\vs 2En 2:8
и они погибнут, и те, что поклонятся им.
\vs 2En 2:9
Да утвердит Господь сердца ваши в страхе своем!
\vs 2En 2:10
И ныне, дети мои, пусть никто не ищет меня, доколе Господь не возвратит меня к вам.

\vs 2En 3:1
И было, когда говорил я сыновьям своим, позвали меня мужи те и взяли на крылья свои.
\vs 2En 3:2
И вознесли меня на первое небо, и поставили меня там.
\vs 2En 3:3
И привели пред лице мое верховных владык чинов звездных, и показали мне путь и движение их от года до года.
\vs 2En 3:4
И показали мне двести ангелов, которые управляют звездами и составом небес.
\vs 2En 3:5
И показали мне там море огромное, большее моря земного.
\vs 2En 3:6
И вокруг ангелы летали на крыльях своих.
\vs 2En 3:7
И показали мне хранилища снега и льда и грозных ангелов, стражей хранилищ тех.
\vs 2En 3:8
И показали мне там хранилища облаков, откуда они выходят и куда входят.
\vs 2En 3:9
И показали мне хранилища росы, подобной елею масличному; и ангелов, стерегущих сокровища те, и вид их как все цветы земные.

\vs 2En 4:1
И взяли меня мужи те, и поставили меня на втором небе, и показали мне узников, соблюдаемых для суда безмерного.
\vs 2En 4:2
И там видел я ангелов осужденных, плачущих, и спросил я мужей, которые были со мною: За что они мучимы?
\vs 2En 4:3
И отвечали мне мужи те: Это отступники от Господа, не послушавшиеся гласа Господня, но своею волею державшие совет.
\vs 2En 4:4
И опечалился я о них. И ангелы те поклонились мне, и сказали: Муж Божий, помолись бы о нас ко Господу.
\vs 2En 4:5
И я отвечал им, и сказал: Кто я, человек смертный, чтобы молиться об ангелах; кто знает, куда иду или что встретит меня, или кто помолится обо мне?

\vs 2En 5:1
И взяли меня оттуда мужи те, и возвели на третье небо, и поставили меня посреди рая.
\vs 2En 5:2
И место то невыразимо красотою вида его: всякое дерево цветами украшено, и всякий плод зрел, и всякие яства вечно изобилуют, всякое дуновение благовонно.
\vs 2En 5:3
И четыре реки протекают там покойным течением.
\vs 2En 5:4
И всякий злак, который рождается в пищу, прекрасен.
\vs 2En 5:5
И древо жизни на месте том, и на нем почивает Господь, когда входит Господь в рай, и древо то несказанно прекрасно благоуханием.
\vs 2En 5:6
И рядом другое древо масличное, постоянно источающее елей.
\vs 2En 5:7
И всякое дерево благоплодно, и нет там дерева безплодного; и все место то благовонно.
\vs 2En 5:8
И Ангелы, охраняющие рай, светлы весьма, непрестанным гласом сладкопения своего служат Богу во все дни.
\vs 2En 5:9
И сказал я: Сколь благо это место весьма!
\vs 2En 5:10
Отвечали мне мужи те: Место это, Енох, уготовано праведникам, которые претерпят напасти в этой жизни, и душам которых причинят зло, и которые отвратят очи свои от неправды и сотворят суд праведный
\vs 2En 5:11
чтобы дать хлеб алчущим и покрыть нагого одеждой, и поднять падшего, и помочь обиженным;
\vs 2En 5:12
которые пред лицем Господа ходят и Ему одному служат, тем уготовано сие в наследие вечное.

\vs 2En 6:1
И взяли меня оттуда мужи те, и вознесли меня на север неба, и показали мне там место весьма страшное:
\vs 2En 6:2
всякое томление и мучение на месте том, и тьма, и мгла, и нет там света, но огонь мрачный разгорается всегда на месте том, и река огненная растекается на все места те;
\vs 2En 6:3
лед холодный, и темницы, и ангелы лютые и неистовые, носящие оружие и мучающие без милости.
\vs 2En 6:4
И сказал я: Как страшно место это весьма!
\vs 2En 6:5
И отвечали мне мужи те: Это место, Енох, уготовано нечестивым, творящим безбожное на земле,
\vs 2En 6:6
тем, которые творят колдовство и волхвование, и похваляются делами своими, и тайно крадут души, и разрешают бремя связанное,
\vs 2En 6:7
тем, которые богатеют в ущерб имуществу чужому, и умерщвляют голодом алчущего, дабы самим насытиться; и имея возможность одеть нагих, раздевают;
\vs 2En 6:8
тем, которые не познали Творца своего, но поклонялись богам суетным, создавая идолов и поклоняясь творению рук своих.
\vs 2En 6:9
И всем тем уготовано это место в удел вечный.

\vs 2En 7:1
И взяли меня оттуда мужи те и подняли на четвертое небо, и показали мне там все движение солнца и луны и все лучи их.
\vs 2En 7:2
И измерил я путь их, и рассчитал свет их,
\vs 2En 7:3
и видел я: солнце имеет свет, в семь раз больший луны.
\vs 2En 7:4
И видел я круг их и колесницы, на которых ездит каждый из них, как ходит ветер,
\vs 2En 7:5
и нет им покоя, день и ночь ходящим и возвращающимся.
\vs 2En 7:6
И я видел четыре звезды великих, висящих справа от колесницы солнца, и четыре слева от солнца всегда.
\vs 2En 7:7
И ангелы движутся перед колесницей солнечной, духи летающие;
\vs 2En 7:8
двенадцать крыльев у каждого ангела, что мчат колесницу солнца, неся росу и зной, когда повелит им Господь сойти на землю с лучами солнечными.

\vs 2En 8:1
И отнесли меня мужи те на восток неба, и показали мне врата, из которых выходит солнце в положенные времена и по обращениям луны всего года,
\vs 2En 8:2
и при убавлении, и при возрастании дня, сообразно уменьшению и возрастанию дня и ночи:
\vs 2En 8:3
шесть ворот одинаковых, отверстых в тридцать одну стадию ровно, и я измерил величину их, и не мог постичь величины их.
\vs 2En 8:4
И те врата ими восходит солнце и идет на запад:
\vs 2En 8:5
первыми вратами выходит оно сорок два дня, вторыми тридцать пять дней,
\vs 2En 8:6
третьими тридцать пять дней, четвертыми тридцать пять дней,
\vs 2En 8:7
пятыми тридцать пять дней, шестыми сорок два дня,
\vs 2En 8:8
и снова возвращается шестыми вратами по истечению срока своего.
\vs 2En 8:9
И оно входит пятыми воротами тридцать пять дней, четвертыми воротами тридцать пять дней,
\vs 2En 8:10
третьими воротами тридцать пять дней, вторыми тридцать пять дней,
\vs 2En 8:11
и заканчиваются дни года по обращению времен.

\vs 2En 9:1
И возвели меня мужи те на запад неба, и показали мне там шесть великих ворот отверстых, поставленных против ворот восточных.
\vs 2En 9:2
И ими заходит солнце по обращении по небу из восточных ворот: по восходу из восточных ворот и по числу дней так же заходит в западные ворота.
\vs 2En 9:3
И когда изойдет оно из западных ворот, берут четыре ангела венец его и возносят ко Господу,
\vs 2En 9:4
а солнце поворачивает колесницу свою и идет без света, и там возлагают на него венец.
\vs 2En 9:5
И вот, показали они мне порядок солнца и ворот, которыми оно восходит и заходит, ибо эти ворота сотворил Господь. И солнце смену времен года указывает.
\vs 2En 9:7
А лунный порядок другой. И показали они мне весь путь ее и все обращения ее,
\vs 2En 9:8
и показали мне мужи те врата, и показали мне двенадцать ворот на востоке и двенадцать ворот на западе,
\vs 2En 9:9
и круги, по которым восходит и заходит луна по установленному времени:
\vs 2En 9:10
первыми воротами на востоке тридцать один день точно, вторыми тридцать пять дней точно,
\vs 2En 9:11
третьими тридцать один день ровно, четвертыми тридцать дней точно,
\vs 2En 9:12
пятыми тридцать один день особо, шестыми тридцать один день точно,
\vs 2En 9:13
седьмыми тридцать дней точно, восьмыми тридцать один день особо,
\vs 2En 9:14
девятыми тридцать один день определенно, десятыми тридцать дней точно,
\vs 2En 9:15
одиннадцатыми тридцать один день ровно, двенадцатыми воротами входит двадцать два дня точно.
\vs 2En 9:16
Также и западными воротамим по обращению и по числу восточных ворот.
\vs 2En 9:17
Так входит она и западными воротами и совершает год в триста шестьдесят пять дней \ldots
\vs 2En 9:18
Когда заканчиваются западные ворота, возвращается и идет к восточным со светом своим.
\vs 2En 9:19
И так ходит кругом день и ночь, и круг ее подобен небу.
\vs 2En 9:20
И колесницу, на которую она восходит, влечет ветер.
\vs 2En 9:21
И движется колесница ее летящими духами, у каждого из ангелов по шесть крыльев. Таков порядок лунный.
\vs 2En 9:22
И видел я посреди неба воинов вооруженных, служащих Богу непрестанным гласом тимпанов и органов, и я наслаждался, слушая их.

\vs 2En 10:1
И взяли меня оттуда мужи те, и вознесли меня на пятое небо.
\vs 2En 10:2
И я видел там множество Бодрствующих, видел я двести.
\vs 2En 10:3
Видом своим как люди, величиной же больше чудес великих, лица их печальны, уста их молчат, и не было там служения.
\vs 2En 10:4
И сказал я у мужам, бывшим со мною: Почему столь печальны и унылы лица их, и уста их молчат, и нет службы на небе этом?
\vs 2En 10:5
И отвечали мне мужи те: Это Бодрствующие, которые отпали от Господа: двое князей и двести ходящих вслед князей тех;
\vs 2En 10:6
и сошли они на землю и исполнили обет свой на хребте горы Ермон, чтобы оскверниться с женами человеческими.
\vs 2En 10:7
И за осквернение то осудил их Господь, и вот, рыдают они о братии своей, и о бывшей укоризне.
\vs 2En 10:8
И сказал я Бодрствующим: Я видел братию вашу и дела их познал, и вот мольбу их видел, и молился о них.
\vs 2En 10:9
И вот осудил их Господь под землю, доколе не придет конец неба и земли.
\vs 2En 10:10
Зачем же ждете вы братьев своих и не служите пред лицем Господа?
\vs 2En 10:11
Восстановите прежнее служение, служите во имя Господне! Ведь если разгневаете Господа, Бога вашего, свергнет Он вас с места этого.
\vs 2En 10:12
И вняли они увещанию моему и стали на небе по четырем чинам;
\vs 2En 10:13
и пока я стоял там, вострубили одновременно в четыре трубы и стали служить Бодрствующие, и поднялся голос их единым гласом к лицу Господа.

\vs 2En 11:1
И подняли меня оттуда мужи те, и вознесли меня на шестое небо.
\vs 2En 11:2
И увидел я там семь ангелов, стоящих вместе, светлых и славных весьма: лица их как лучи солнечные блистают, и нет различия ни в лицах, ни во власти, ни в содержании власти их.
\vs 2En 11:3
Они устрояют и преподают благой порядок миру: движению звезд, солнца и луны, и ангелов, возящих их, и небесным гласам, и умиротворяют все бытие небес;
\vs 2En 11:4
и устрояют заповеди и поучения, и сладкогласное пение, и всякую хвалу и славу.
\vs 2En 11:5
И есть ангелы над временами и годами, и ангелы над реками и морями, и ангелы над плодами и травою, и над всем прозябающим;
\vs 2En 11:6
и ангелы всех народов, управляющие всею жизнью их и записывающие ее перед лицом Господа.
\vs 2En 11:7
И среди них семь Фениксов, и семь Херувов и семь Серафов, единогласных голосами своими и пением своим, и неизъяснимо пение их.
\vs 2En 11:8
И радуется Господь подножию Своему.

\vs 2En 12:1
И подняли меня оттуда мужи те и вознесли меня на седьмое небо.
\vs 2En 12:2
И увидел я свет великий и все огненное воинство безплотных: архангелов, ангелов, и светозарное стояние Офанов.
\vs 2En 12:3
И я устрашился, и вострепетал, и взяли меня иужи те, и поставили среди них, говоря мне: Ободрись, Енох, не бойся!
\vs 2En 12:4
И показали мне издалека Господа, сидящего на престоле Своем, и все воинства небесные, соединенные по чину, приступали и кланялись Господу,
\vs 2En 12:5
и снова отходили, и шли на места свои в радости и веселии, и в свете безмерном.
\vs 2En 12:6
И Славные служат ему неотступно день и ночь, стоя перед лицем Господа и творя волю Его.
\vs 2En 12:7
И все воинство Херувов вокруг престола Его неотступно, и Серафы покрывают престол Его, воспевая пред лицем Господа.
\vs 2En 12:8
И когда я увидел все это, то отошли от меня мужи те, и больше я не видел их.
\vs 2En 12:9
И они поставили меня одного на краю неба, и испугался я, и пал на лице свое.
\vs 2En 12:10
И послал Господь одного из Славных своих ко мне, Гавриила, и он сказал мне. Дерзай, Енох, не бойся! Встань и пойди со мной, и стань перед лицем Господним во веки.
\vs 2En 12:11
Я же отвечал ему, говоря: Увы мне, господин, душа моя отступила из меня от страха.
\vs 2En 12:12
Позови ко мне мужей, приведших меня на место сие, ибо к ним я имею доверие и с ними пришел я пред лице Господне.

\vs 2En 13:1
И восхитил меня Гавриил, как восхищается лист ветром, и погнал меня, и поставил меня пред лицем Господним.
\vs 2En 13:2
И увидел я Господа, и лицо его могущественно, и преславно и страшно.
\vs 2En 13:3
Но кто я, чтобы поведать, объять подлинное лице Господа, могущественное и весьма страшное,
\vs 2En 13:4
или изречь о хорах окрест Его, многоочитых и многогласных, о престоле Его, весьма великом и нерукотворенном,
\vs 2En 13:5
или стояние, которое пред Ним, воинства Херувов и Серафов, или неизменное, неисповедимое, неумолкающее славное служение Ему?
\vs 2En 13:6
И пал я на лице свое, и поклонился Господу.
\vs 2En 13:7
И Господь устами Своими воззвал ко мне: Дерзай, Енох, не бойся! Восстань и стань пред лицем Моим во веки.
\vs 2En 13:8
И поднял меня Михаил, архангел Господень, и привел меня пред лице Господа.
\vs 2En 13:9
И испытал Господь слуг своих, сказав им: Да вступит Енох, чтобы стоять пред лицем Моим во веки.
\vs 2En 13:10
И Славные поклонились Ему и сказали: Да вступит.
\vs 2En 13:11
И сказал Господь Михаилу: Возьми Еноха, и сними с него земные одежды, и помажь елеем многоценным, и облеки в ризы славы.
\vs 2En 13:12
И снял Михаил одежды мои с меня, и помазал меня елеем благим.
\vs 2En 13:13
И вид елея ярче света великого, и умащение им словно роса добрая, и благоухание его подобно мирре, и лучи его как солнечные.
\vs 2En 13:14
И оглядел я всего себя: и стал я, как один из Славных, и не было на вид различия.

\vs 2En 14:1
И призвал Господь Веревеила, одного из архангелов Своих, который был мудр и записывал все дела Господни.
\vs 2En 14:2
И сказал Господь Веревеилу: Возьми книги из хранилищ и дай Еноху трость и прочти ему книги.
\vs 2En 14:3
И поспешил Веревеил и принес мне книги, изукрашенные смирной.
\vs 2En 14:4
И дал мне трость из руки своей, и стал рассказывать мне все дела Господни: о земле, о море, о всех стихиях, о движении всех планет и бытии их,
\vs 2En 14:5
о смене лет и движении дней, о земных заповедях и наставлениях, о сладкогласном пении, о входах облаков и исходах ветров,
\vs 2En 14:6
о всяком народе, и о новой песне вооруженного воинства все, чему следовало меня научить, поведал мне Веревеил.
\vs 2En 14:7
Тридцать дней и тридцать ночей говорили уста его, не умолкая.
\vs 2En 14:8
И я не спал тридцать дней и тридцать ночей, записывая все знаками.
\vs 2En 14:9
Когда же закончил, сказал мне Веревеил: Сядь, напиши то, что я поведал тебе.
\vs 2En 14:10
И я, просидев еще тридцать дней и тридцать ночей, записал все подробно и отчетливо, и поведал это в трехсот и шестидесяти книгах.

\vs 2En 15:1
И призвал меня Господь, и поставил меня слева от Себя рядом с Гавриилом, и я поклонился Господу.
\vs 2En 15:2
И сказал мне Господь: Все, что ты видел, Енох, неподвижное и движущееся, сотворено Мною, и Я возвещу тебе о том от начала.
\vs 2En 15:3
Прежде, когда не было всего, что Я привел из небытия в бытие, и из невидимого в видимое, и ангелам Моим не возвестил Я тайны Моей, и не поведал им, как сотворил их, и не постигли они бесконечного Моего и непостижимого творения, тебе же Я возвещаю сегодня.
\vs 2En 15:4
Прежде, когда не было всего видимого, открылся свет, и Я среди света, будучи невидим, один проезжал, как ездит солнце от востока до запада и от запада на восток,
\vs 2En 15:5
но солнце находит покой, Я же не обрел покоя, поскольку все было несотворенным.
\vs 2En 15:6
И помыслил Я поставить основание, сотворить тварь видимую. И повелел Я в глубине, да взойдет одно из невидимых в видимое.
\vs 2En 15:7
И вышла Божественная вечность, весьма великая, и вот, имела она во чреве своем великий век.
\vs 2En 15:8
И Я сказал ей: Разрешись от бремени, о Божественная вечность, и да будет видимо разрешаемое из тебя.
\vs 2En 15:9
И разрешилась она, и вышел из нее великий век, и таким образом изнесло все творение, которое Я восхотел сотворить. И увидел Я, что это хорошо.
\vs 2En 15:10
И поставил Я Себе престол, и сел на нем, свету же сказал: Взойди ввысь, и утвердись, и будь основанием горнему. И нет превыше света ничего иного.
\vs 2En 15:11
И увидев это, Я восклонился с престола Моего, и воззвал в глубине во второй раз, и сказал: Да произыдет из невидимого твердь, и да станет видима!
\vs 2En 15:12
И произошло основание тверди, тяжелое и весьма мрачное. И увидел Я, что это хорошо.
\vs 2En 15:13
И Я сказал ему: Сойди вниз, и утвердись, и будь основанием дольнему.
\vs 2En 15:14
И сошло оно, и утвердилось, и стало основанием дольнему. И ниже тьмы нет ничего иного.
\vs 2En 15:15
И облек Я эфир светом, уплотнил Я его и простер Я его над тьмою, а из вод утвердил камни великие.
\vs 2En 15:16
Водам же бездны повелел Я высохнуть, а впадины Я назвал безднами.
\vs 2En 15:17
Море Я собрал в одно место, связал его узами, и дал морю границу вечную, и не исторгнется из вод.
\vs 2En 15:18
И поставил Я твердь и основал ее поверх вод.
\vs 2En 15:19
И помимо всего воинства небесного, образовал Я на небесах солнце от света великого, и поставил его на небе, дабы светило оно на землю.
\vs 2En 15:20
Из камня Я высек огонь великий и из огня сотворил все воинства безплотные и все воинства звезд. И Херувов, и Серафов, и Офанов и это все Я высек из огня.
\vs 2En 15:21
Земле же Я повелел произрастить всякое дерево, и всякую гору, и всякую траву живую, и всякое семя живое, сеющее семя, прежде, чем сотворить души живые, Я приготовил пищу им.
\vs 2En 15:22
И морю повелел Я породить в себе рыб и всяких гадов, ползающих по земле, и всякую птицу парящую.
\vs 2En 15:23
И когда закончил все, повелел Я Премудрости Своей сотворить человека.

\vs 2En 16:1
И ныне, то, что Я рассказал тебе, и то, что ты видел на небесах, и то, что видел на земле, и то, что ты написал в книгах, создал Я Премудростью Своею и искусством Своим, и сотворил от нижнего основания до высшего.
\vs 2En 16:2
И до скончания их нет Мне советника, ни помощника.
\vs 2En 16:3
Сам Я вечен, нерукотворен. Неизменна мысль моя, советник Мой, и слово Мое есть дело, и очи Мои следят за всем: если отверну лице Мое все погибнут, если же призираю утверждаются.
\vs 2En 16:4
Положи, Енох, ум свой, и познай Говорящего с тобою, и возьми книги, которые ты написал.
\vs 2En 16:5
И даю тебе Семеила и Рагуила, возведших тебя ко Мне, и сойди на землю и расскажи сынам своим то, что Я говорил тебе, и то, что видел ты от нижнего неба и до престола Моего,
\vs 2En 16:6
все воинства Я сотворил, и нет противящегося Мне или не покоряющегося: все покоряются Моему единовластию и работают одной Моей власти.
\vs 2En 16:7
И дай им книги, соделанные рукою твоею, и прочтут они и познают Творца своего, и уразумеют и они, что нет иного Творца, кроме Меня.
\vs 2En 16:8
И передай книги, соделанные рукою твоею, детям и детям детей твоих, и дай наставления ближним из рода в род.
\vs 2En 16:9
Ибо Я дам тебе ходатая, о Енох, архистратига моего Михаила, чтобы написанное рукою твоею и написанное рукою отцов твоих, Адамом и Сифом, не погибло до века последнего, 10 Ибо Я заповедал ангелам Моим, Ариоху и Мариоху, которых поставил Я над землею, дабы хранили ее и повелевали временами,
\vs 2En 16:11
дабы соблюли они написанное рукою твоею и написанное рукою отцов твоих, и не погибло это в грядущий потоп, который Я сотворю в роде твоем.
\vs 2En 16:12
Я знаю злобу человеческую, что они не вынесут бремени, которое Я возложил на них, и не будут сеять семя, которое Я дал им, но отвергнут бремя Мое, и иное бремя примут,
\vs 2En 16:13
и посеют семена пустые, и поклонятся богам суетным, и отринут единовластие Мое, и вся земля согрешит неправдами, и обидами, и прелюбодейством, и идолослужением.
\vs 2En 16:14
Тогда наведу Я потоп на землю, и земля сама сокрушится в бездну великую.
\vs 2En 16:15
И Я оставлю мужа праведного из племени твоего, со всем домом его, который сотворит по воле Моей.
\vs 2En 16:16
И от семени их востанет род последний, многочисленный и весьма ненасытный.
\vs 2En 16:17
Тогда при исходе рода того явятся книги, написанные рукою твоею и отцов твоих, которые стражи земные покажут мужам верным, и расскажут роду тому и будут они почитаемы впоследствии больше, чем прежде.

\vs 2En 17:1
Ныне же, Енох, даю тебе срок ожидания тридцать дней, чтобы сотворил ты в доме твоем и рассказал сыновьям твоим и домочадцам твоим от Меня.
\vs 2En 17:2
И всякий, кто блюдет сердце свое, да прочтет и уразумеет, что нет никого, кроме Меня.
\vs 2En 17:3
И спустя тридцать дней Я пошлю ангелов за тобою, и возьмут тебя ко Мне от земли и от сыновей твоих;
\vs 2En 17:4
возьмут тебя ко Мне, ибо уготовано тебе место, и ты будешь перед лицем Моим отныне и до века.
\vs 2En 17:5
И будешь созерцать тайны Мои, и будешь книжником над рабами Моими,
\vs 2En 17:6
ибо будешь записывать все дела земные и о пребывающих на земле и на небесах, и будешь Мне свидетелем Суда Великого Века.
\vs 2En 17:7
Все сие говорил мне Господь, как говорит муж ближнему своему.

\vs 2En 18:1
И ныне, чада мои, услышьте голос отца вашего и то, что я заповедаю вам сегодня:
\vs 2En 18:2
ходите пред лицем Господним, и все, чему должно произойти по воле Господа.
\vs 2En 18:3
Ибо я послан от уст Господа к вам, дабы сказать вам, что есть и что будет до Дня Судного.
\vs 2En 18:4
И ныне, чада мои, не от уст моих вещаю вам сегодня, но от уст Господа, пославшего меня к вам.
\vs 2En 18:5
И вы слышите слова мои из уст моих, подобно вам созданного человека; я же слышал из уст Господа, огненных, ибо уста Господа как печь огненная, и слова его как пламя огненное исходят.
\vs 2En 18:6
Вы, чада мои, видите лице мое, подобно вам созданного человека, я же видел лице Господа, как железо, раскаленное огнем, испускающее искры.
\vs 2En 18:7
И вы видите глаза, подобно вам созданного человека, я же видел очи Господа, светящиеся, как лучи солнца, ужасающие глаза человеческие.
\vs 2En 18:8
И вы, дети, видите десницу мою, подающую вам знаки, подобно вам сотворенного человека, я же видел руку Господа, подающую знак мне, наполняющую небо.
\vs 2En 18:9
И вы видите охват тела моего, подобного вашему, я же видел объятие Господа, безграничное и несравненное, которому нет конца.
\vs 2En 18:10
И вы слышите слова из уст моих, я же слышал глаголы Господа, как гром великий, приводящие в непрестанное движение облака.
\vs 2En 18:11
Теперь, чада мои, услышьте беседующего о царе земном страшно и трепетно стоять перед лицем царя земного, страшно, потому что воля царя смерть, и воля царя жизнь.
\vs 2En 18:12
Каково же стоять перед лицем Царя Небесного? кто выдержит этот безмерный страх и жар великий?
\vs 2En 18:13
Но призвал Господь одного из ангелов своих верховных, грозного, и поставил рядом со мною,
\vs 2En 18:14
видом же ангел был, как снег, а руки его как лед, и он остудил лице мое, потому что не стерпел бы я страха и жара огненного.
\vs 2En 18:15
И тогда сказал мне Господь все слова Свои.

\vs 2En 19:1
И ныне, чада мои, я знаю все: одно из уст Господа, другое глаза мои видели; от начала и до конца, и от конца до нового обращения все я узнал.
\vs 2En 19:2
И записал я в книгах обо всем наполняющем небеса до краев их, я измерил путь их, и воинство их я узнал, и записал звезд многое множество бесчисленное.
\vs 2En 19:3
Кто из людей знает о круговом движении их и пути их, и обращении их, или о том, кто ведет их, или о ведомых?
\vs 2En 19:4
Ангелы не знают числа их, я же имена их записал.
\vs 2En 19:5
И солнечный круг я измерил, и лучи сосчитал, и весь путь его, и входы его и исходы его, и имена их записал.
\vs 2En 19:6
И лунный круг я измерил, и движение ее во все дни исчислил, и свет ее на всякий день и час, и в книгах имена ее записал.
\vs 2En 19:7
И жилища облаков, и устав их, и крылья их, и дождь их, и капли их я изследовал.
\vs 2En 19:8
И описал грохот грома и блеск молнии, и показали мне хранителей ключей их, и восходы их;
\vs 2En 19:9
ходят же по мере: узами поднимаются и узами опускаются, дабы тяжелым напором не обрушили облака, и не погубили то, что на земле.
\vs 2En 19:10
Я написал о сокровищницах снега и хранилищах льда и воздуха холодного;
\vs 2En 19:11
наблюдал я, как время от времени хранители ключей их наполняют ими облака, но не истощаются сокровищницы.
\vs 2En 19:12
Написал я о ложе ветров, смотрел я и увидел, как хранители ключей их, носящие весы с собой и меры, на одну чашу весов кладут сокровища, во вторую же меру, и по мере отпускают их на всю землю, дабы чрезмерным ветром не поколебать землю.

\vs 2En 20:1
Оттуда сведен я был вниз, и пришел на место судное, и я видел ад отверстый, и видел там тех, кому хуже, чем узникам, суд безмерный.
\vs 2En 20:2
И спустился я, и написал о всяком суде осужденных, и все вопрошения их увидел, и воздохнул, и заплакал о погибели нечестивых.
\vs 2En 20:3
И сказал я в сердце своем: Блажен, кто еще не родился, или родившийся, но не согрешивший перед лицем Господа, дабы не попал на место это и дабы не понес бремени места этого.
\vs 2En 20:4
И видел я хранителей ключей ада, стоящих у весьма великих ворот, подобных аспидам огромным,
\vs 2En 20:5
лица их как свечи потухшие, глаза их как пламя померкшее, и зубы их обнажены до персей их.
\vs 2En 20:6
И сказал я в лице их: Ушел бы я и не видел вас, ибо вы здесь за деяния ваши. И да не придет никто из племени моего к вам.

\vs 2En 21:1
И оттуда взошел я в рай праведных, и видел там место благословенное, и вся тварь благословенна, и все живут в радости и веселии, и в свете безмерном, и в жизни вечной.
\vs 2En 21:2
Тогда сказал я: Чада мои, говорю вам: блажен, кто боится имени Господа, и перед лицем Его будет служить всегда, и приготовит дары и приношения \ldots и жизнию поживет, и умрет.
\vs 2En 21:3
Блажен, кто будет творить суд праведный: нагого оденет в одежды и голодному даст хлеб.
\vs 2En 21:4
Блажен, кто судит суд праведный: сироте, и вдовице, и всякому обиженному поможет.
\vs 2En 21:5
Блажен, кто сойдет с пути временного и пойдет путями праведными.
\vs 2En 21:6
Блажен, кто сеет семена праведные, ибо и пожнет их седмерицею.
\vs 2En 21:7
Блажен, в ком есть истина, да говорит истину ближнему своему.
\vs 2En 21:8
Блажен, у кого в устах его милость истинная и кротость.
\vs 2En 21:9
Блажен, кто разумеет дела Господа, ибо по делам Его познается Создатель.
\vs 2En 21:10
И вот, дети мои, я обозрел землю до краев ее, и записал я все: все года сложил, и из лет разделил месяцы, и в месяце рассчитал дни, дни разделил на часы, часы же измерил.
\vs 2En 21:11
И описал всякое семя на земле, и разделил каждую меру, и каждые весы правильные я измерил, и описал.
\vs 2En 21:12
И как год от года разнится в достоинстве, так и человек от человека в чести:
\vs 2En 21:13
кто благодаря большому богатству, кто благодаря сердечной мудрости, а кто благодаря остроте ума или молчанию уст.
\vs 2En 21:14
Но нет никого более боящегося Господа, ибо боящиеся Господа славны будут во век.
\vs 2En 21:15
Господь руками Своими создал человека в подобие лица Своего, малого и великого сотворил Господь.
\vs 2En 21:16
Оскорбляющий лице человеческое оскорбляет лице Господа, гнушающийся лица человеческого гнушается лица Господа,
\vs 2En 21:17
презирающий лице человека презирает лице Господа; гнев и Суд Великий тем, кто плюет в лицо человеку.

\vs 2En 22:1
Блажен, кто приготовит себя всякому человеку: кто помогает осуждаемому, и кто поднимает упавшего, и кто подает просящему,
\vs 2En 22:2
ибо в день Суда Великого всякое дело человека обновится писанием.
\vs 2En 22:3
Блажен, у кого будет мера праведная, и весы праведные, и гири праведные,
\vs 2En 22:4
так как в день Суда Великого каждая мера, и каждые весы, и каждая гиря словно при покупке приложатся, и узнает каждый меру свою и по ней примет мзду.
\vs 2En 22:5
Тому, кто всегда творит пред лицем Господа, управит Господь приобретения его.
\vs 2En 22:6
Тому, кто умножает светильники пред лицем Господа, умножит Господь хранилища его.
\vs 2En 22:7
Разве нужны Господу хлеб или свеча, или овен, или телец? но этим испытывает Господь сердце человека.
\vs 2En 22:8
Ибо когда Господь пошлет свой свет великий во тьму и будет Суд, кто тогда утаиться?
\vs 2En 22:9
Ныне же, чада мои, положите помышление на сердцах ваших и внемлите словам отца вашего, ибо то, что я вещаю вам от уст Господних.
\vs 2En 22:10
Возьмите книги эти книги, написанные рукою отца вашего, и прочтите их, и из них узнаете дела Господа, и что нет никого, кроме Господа единого,
\vs 2En 22:11
Который поставил основания на неведомом, и простер небеса на невидимом, землю поставил, на водах ее основав непостоянных,
\vs 2En 22:12
Который безчисленную тварь сотворил один (а кто исчислил прах земной или песок морской, или капли облаков?),
\vs 2En 22:13
Который землю и море соединил нерушимыми узами,
\vs 2En 22:14
Который немыслимую красоту из огня высек и украсил небо,
\vs 2En 22:15
Который из невидимого в видимое все сотворил, Сам будучи невидимым.
\vs 2En 22:16
И раздайте эти книги детям вашим и детям детей ваших.
\vs 2En 22:17
И все ближние ваши, и все сродники ваши, которые знают и боятся Господа, да примут их, и да будут они им нужнее всякой пищи доброй, и да прочтут и прилепятся к ним!
\vs 2En 22:18
А неразумные, не знающие Господа, не примут их, но отвергнут, ибо отягчат они бремя их.
\vs 2En 22:19
Блажен, кто понесет бремя их, примет его, ибо обретет его в день Суда Великого.
\vs 2En 22:20
Ибо я клянусь вам, чада мои, что еще прежде, чем быть человеку, место Судное уготовано ему, и мера, и весы, которыми будет испытан человек, там прежде уготованы.
\vs 2En 22:21
Я же дело всякого человека в письменах изложу, и никто не сможет укрыться.

\vs 2En 23:1
И ныне, чада мои, пребывайте в терпении и кротости число дней ваших, да наследуете безконечный век будущий.
\vs 2En 23:2
И всякое бедствие, и всякое страдание, и зной, и всякое слово злое, если найдет на вас, потерпите Господа ради.
\vs 2En 23:3
И хотя можете отплатить расплатой, не воздавайте ближнему, ибо один Господь воздает, и будет отмщающим за вас в День Суда Великого.
\vs 2En 23:4
Золотом и серебром пожертвуйте ради брата, дабы принять сокровище плоти в День Судный.
\vs 2En 23:5
И к сироте и вдове прострите руки ваши, и по силе помогите бедному, и обретете покровительство во время всякого труда.
\vs 2En 23:6
Если найдет на вас скорбь и печаль, ради Господа отриньте, и обретете воздаяние в День Судный.
\vs 2En 23:7
Утром, и в полдень, и вечером благо есть ходить в дом Господень прославлять Творца всего.
\vs 2En 23:8
Блажен, кто раскрывает сердце свое для хвалы и хвалит Господа.
\vs 2En 23:9
Проклят раскрывающий сердце свое для хулы и клеветы на ближнего.
\vs 2En 23:10
Блажен, кто раскрывает уста свои, благословляя и прославляя Господа.
\vs 2En 23:11
Проклят раскрывающий уста свои для проклятия и хулы в лице Господа.
\vs 2En 23:12
Блажен прославляющий все дела Господни.
\vs 2En 23:13
Проклят оскорбляющий творение Господа.
\vs 2En 23:14
Блажен созидающий и воздвигающий трудом рук своих.
\vs 2En 23:15
Проклят стремящийся уничтожить труды чужие.
\vs 2En 23:16
Блажен хранящий устои отцов до конца.
\vs 2En 23:17
Проклят нарушающий установления и законы отцов своих.
\vs 2En 23:18
Благословен насаждающий мир.
\vs 2En 23:19
Проклят уничтожающий мирное.
\vs 2En 23:20
Благословен говорящий мир и имеющий мир в сердце своем.
\vs 2En 23:21
Проклят говорящий то, но не имеющий мира в сердце своем.
\vs 2En 23:22
Все это на весах и в книгах изобличится в День Суда Высшего.

\vs 2En 24:1
И ныне, чада мои, блюдите сердца ваши от всякой неправды, да унаследуете подножие света во веки.
\vs 2En 24:2
Не говорите, дети мои: отец наш с Господом и умолит нас о грехе.
\vs 2En 24:3
Знайте, что все дела всякого человека я записываю, и никто не может уничтожить написанного рукою моею, потому что Господь все видит.
\vs 2En 24:4
И теперь, чада мои, усвойте все слова отца вашего, которые я говорю вам, да будут они вам в достояние покоя.
\vs 2En 24:5
И книги, которые я дал вам, не отриньте их, но всем желающим растолкуйте их, может быть узнают дела Господа.
\vs 2En 24:6
И вот, чада мои, приближается назначенный день года, и время подходит установленное, и ангелы, которые идут со мною, стоят пред лицем моим.
\vs 2En 24:7
И утром я поднимусь на небо высшее, в вечное мое достояние, и потому заповедаю вам, дети мои, делайте все то, что благословенно пред лицем Господа.

\vs 2En 25:1
И отвечал Мафусела отцу своему Еноху: Что угодно очам твоим, отец? Да приготовим пищу пред лицем твоим;
\vs 2En 25:2
да благословишь дома наши и сыновей своих, и всех домочадцев своих, и прославишь народ свой, и после этого уйдешь.
\vs 2En 25:3
И ответил Енох сыну своему, говоря: Слушай, сын мой, от того дня, как помазал меня Господь елеем славы Своей, вострепетал я, и не услаждает меня пища, и не хочется мне ничего из земных яств.
\vs 2En 25:4
Но позови братьев своих и всех домочадцев наших, и старейшин народа, дабы я говорил с ними, и тогда отойду.
\vs 2En 25:5
И поспешил Мафусела, и позвал братьев своих Регима, и Ариима, и Ахазухана, и Харимиона, и старейшин народа, и привел их пред лице отца своего Еноха.
\vs 2En 25:6
И поклонились ему, и принял их Енох, и благословил их, и отвечал ним, говоря:
\vs 2En 25:7
Послушайте, чада! Во дни отца вашего Адама сошел Господь на землю, чтобы посетить ее и все сотворенное Им, которое Сам создал.
\vs 2En 25:8
И призвал Господь всех зверей земных и всех гадов земных, и всех птиц пернатых, и привел их пред лице отца вашего Адама, чтобы нарек он имена всем на земле.
\vs 2En 25:9
И оставил их Господь у него, и подчинил ему всех, сделав вторыми по меньшинству, и притупил весь разум их, дабы повиновались человеку.
\vs 2En 25:10
Ибо Господь сотворил человека над всем владением Своим, и за это не будет Суда никакой душе живой, но одному человеку.
\vs 2En 25:11
Всем душам скотов уготовано одно место, и предел один, и пастбище одно в Веке Великом.
\vs 2En 25:12
И не укроется ни одна душа живая, которую сотворил Господь, до Суда, и все души, которых оклевещут до Суда.
\vs 2En 25:13
И тот, кто плохо заботится о душе своей, сделает свою душу беззаконною.
\vs 2En 25:14
А приносящий жертву из чистого скота, имеет исцеление, он исцеляет душу свою.
\vs 2En 25:15
Умерщвляющий же скот всякий, не связав его, преступает закон, он предает душу свою беззаконию.
\vs 2En 25:16
И творящий злое животным в тайне, преступает закон, он предает душу свою беззаконию.
\vs 2En 25:17
Творящий злое душе человеческой, творит злое душе своей, и нет ему исцеления во веки.
\vs 2En 25:18
Толкающий человека в сеть, сам в ней увязнет, и нет ему исцеления во веки.
\vs 2En 25:19
И подвергающий человека осуждению, непременно будет осужден во веки.

\vs 2En 26:1
И ныне, чада мои, храните сердца ваши от всякой неправды, которую возненавидит Господь, более же всего по отношению ко всякой душе живой, которую создал Господь.
\vs 2En 26:2
И что просит человек для своей души от Господа, пусть тоже сотворит всякой душе живой.
\vs 2En 26:3
Потому что в Веке Великом многие обители уготованы человеку: обители добрые весьма и бесчисленные обители злые.
\vs 2En 26:4
Блажен, кто отойдет в благословенные обители, ибо из злых нет возвращения.
\vs 2En 26:5
Когда положит человек слово в сердце своем принести дар пред лицем Господа, а руки его того не сделают, тогда отвергнет Господь труд рук его, и не обретет ничего.
\vs 2En 26:6
И если сотворят руки его, но сердце его будет роптать, то не прекратится болезнь сердца его, ибо роптание его поспешно.
\vs 2En 26:7
Блажен человек, который в терпении своем принесет дар пред лицем Господа, ибо обретет воздаяние.
\vs 2En 26:8
И если человек назначит устами своими определенное время для принесения дара пред лицем Господа и совершит это то обретет воздаяние;
\vs 2En 26:9
если же пройдет назначенное время, и возвратит слова свои, то даже если раскается, не будет ему благословения.
\vs 2En 26:10
Потому что всякое промедление порождает искушение.
\vs 2En 26:11
Человек, который прикроет нагого и алчущему даст хлеб, обретет воздаяние.
\vs 2En 26:12
Если же станет роптать сердце его, то погубит себя и не будет ему воздаяния.
\vs 2En 26:13
И если нищий, когда насытится сердце его, возгордится, то погубит все добрые дела свои и не обретет воздаяния, ибо мерзок Господу всякий муж возгордившийся.

\vs 2En 27:1
И было, когда говорил Енох сыновьям своим и князьям народа, услышал его весь народ и все близкие его, что призывает Еноха Господь,
\vs 2En 27:2
и, посовещавшись, сказали: Идем и приветствуем Еноха.
\vs 2En 27:3
И собралось около двух тысяч мужей, и пришли на место Азухань, где был Енох и сыновья его, и старейшины народа, и приветствовали Еноха, говоря:
\vs 2En 27:4
Благословен ты у Господа, Царя Вечного, ныне же благослови народ свой и прославь его пред лицем Господа, ибо тебя избрал Господь в пророки, чтобы ты отнял грехи наши.
\vs 2En 27:5
И отвечал Енох народу своему, говоря: Слушайте, чада мои. Сначала, когда ничего не было, прежде, чем появилось все творение, создал Господь век сотворенный,
\vs 2En 27:6
и после этого сотворил все творение Свое, видимое и невидимое, и после всего этого создал человека по образу Своему,
\vs 2En 27:7
и вложил ему глаза, чтобы видеть, и уши, чтобы слышать, и сердце, чтобы разуметь, и ум, чтобы размышлять.
\vs 2En 27:8
Тогда освободил Господь век ради человека, и разделил его на времена и часы,
\vs 2En 27:9
да размышляет человек о смене времен, и о конце и начале лет, и окончании месяцев, и дней, и часов, да предаст ему свою жизнь и смерть.
\vs 2En 27:10
Когда же перестанет существовать все творение, которое сотворил Господь, и всякий человек придет на Суд Господа Великий,
\vs 2En 27:11
тогда исчезнут времена, и лет больше не будет, и ни месяцы, ни дни, ни часы более не будут сосчитываться, но настанет век единый.
\vs 2En 27:12
И все праведники, которые избегнут Суда Господня Великого, соединятся в Веке Великом, вместе соединятся праведники, и будут пребывать вечно.
\vs 2En 27:13
И более не будет у них ни труда, ни болезни, ни скорби, ни ожидания невзгод, ни тягот, ни ночи, ни тьмы; но свет великий будет для них всегда.
\vs 2En 27:14
И стена неразрушимая в раю великом будет защитой их жилища вечного.
\vs 2En 27:15
Блаженны праведники, которые избегнут Суда Господня Великого, ибо озарятся лица их подобно солнцу.
\vs 2En 27:16
Ныне же, чада мои, оберегайте души ваши от всякой неправды, которую возненавидел Господь;
\vs 2En 27:17
пред лицем Господа ходите и Ему одному служите, и всякое приношение приносите пред лице Господа.
\vs 2En 27:18
Если и посмотрит человек ввысь то там Господь, ибо Господь сотворил небеса;
\vs 2En 27:19
если посмотрит на землю и на море и подумает о том, что под землей, то и там Господь, ибо Господь сотворил все, и не скроется ни какое дело от лица Господня.
\vs 2En 27:20
Вы же с долготерпением и с кротостью, сквозь страдания и мучения, пройдете болезненный век сей.

\vs 2En 28:1
И когда говорил это Енох народу своему, послал Господь мрак на землю, и была тьма, и покрыла тьма стоящих с Енохом мужей.
\vs 2En 28:2
И поспешили ангелы, и взяли ангелы Еноха, и вознесли его на небо вышнее.
\vs 2En 28:3
И принял его Господь, и поставил его пред лицем Своим во веки.
\vs 2En 28:4
И отступила тьма от земли, и стал свет.
\vs 2En 28:5
И увидел народ, и понял, что взят был Енох, и, прославив Бога, пошли в дома свои.
\vs 2En 28:6
И поспешил Мафусела и братья его, сыновья Еноха, и сделали жертвенник на месте Азухань, откуда взят был Енох, и, взяв овнов и тельцов, принесли их в жертву пред лицем Господа.
\vs 2En 28:7
И созвали всех людей, дабы пришли к ним на пир.
\vs 2En 28:8
И принесли люди дары сыновьям Еноха.
\vs 2En 28:9
И радовались и веселились три дня.

\vs 2En 29:1
И в третий день, во время вечернее, сказали старейшины народа Мафуселе, говоря:
\vs 2En 29:2
Иди и встань пред лицем Господа и пред лицем народа своего пред алтарем Господним и будешь славен в народе твоем.
\vs 2En 29:3
И отвечал Мафусела народу своему: Господь Бог отца моего Еноха, Сам изберет священника над народом Своим.
\vs 2En 29:4
И ждал народ всю ночь ту на месте Азухань.
\vs 2En 29:5
И пребывал Мафусела у алтаря, и молился Господу, говоря: Господь всего века, сего ли сына отца нашего Еноха избрал ты?
\vs 2En 29:6
Господи, яви священника народу Своему, да в неразумии сердца их боятся славы Твоей, и сотвори все по воле Твоей!
\vs 2En 29:7
И уснул Мафусела, и явился ему Господь в видении ночном, и сказал ему:
\vs 2En 29:8
Слушай Мафусела, Я Господь Бог отца твоего Еноха, услышал Я глас народа Своего.
\vs 2En 29:9
Встань же пред ними и перед алтарем Моим, и прославлю тебя перед этим народом Моим во все дни жизни твоей.
\vs 2En 29:10
И востал Мафусела от сна своего, и благословил Явившегося ему.
\vs 2En 29:11
И утром пришли старейшины народа к Мафуселе, и направил Господь Бог сердце Мафуселы послушаться голоса народа,
\vs 2En 29:12
и сказал им: Да сотворит Господь Бог наш благое в глазах Его для этого народа Своего.
\vs 2En 29:13
И поспешили Сарсан, и Хармий, и Заза, старейшины народа, и облекли Мафуселу в одежду превосходную, и возложили венец светлый на голову его.
\vs 2En 29:14
И поспешно привел народ овнов и тельцов, и из птиц все, что положено, чтобы принес Мафусела жертву пред лицем Господа и пред лицем народа.
\vs 2En 29:15
И поднялся Мафусела к жертвеннику Господню, подобно восходящей деннице, и весь народ шел за ним.
\vs 2En 29:16
И стал Мафусела у алтаря, и весь народ вокруг алтаря.
\vs 2En 29:17
И старейшины народа, взяв овнов и тельцов, связали их по четыре ноги и положили на возглавие алтаря.
\vs 2En 29:18
И сказал народ Мафуселе: Возьми нож и заколи назначенное перед лицем Господа.
\vs 2En 29:19
И простер Мафусела руки свои к небу и призвал Господа, говоря:
\vs 2En 29:20
Увы мне, Господи, кто я, чтобы стоять у возглавия жертвенника Твоего, во главе всего народа Твоего, и для всего познания?
\vs 2En 29:21
Яви благодать рабу Твоему пред лицем народа сего, да знают, что это Ты! Назначь священника народу Своему!
\vs 2En 29:22
И было, когда молился Мафусела, сотрясся алтарь, и поднялся нож с алтаря, и вскочил нож в руки Мафуселе перед лицем всего народа.
\vs 2En 29:23
И объял всех людей трепет, и прославили Господа.
\vs 2En 29:24
И был почитаем Мафусела в глазах Господа и в глазах всего народа с того дня.
\vs 2En 29:25
И взял Мафусела нож, и совершил заклание всех приношений народа.
\vs 2En 29:26
И возрадовался народ, и возвеселился пред лицем Господа и пред лицем Мафуселы в тот день, и после этого разошлись по домам своим.
\vs 2En 29:27
И стоял Мафусела у возглавия алтаря и во главе всего народа с того дня триста девяносто два года.
\vs 2En 29:28
И благословил Господь Мафуселу в жертвах, и в дарах его, и во всем служении, которое совершал он пред лицем Господа.

\vs 2En 30:1
И когда приблизились к концу дни Мафуселы, явился ему Господь в видении ночном и сказал ему:
\vs 2En 30:2
Слушай, Мафусела, Я Бог отца твоего Еноха. Познай волю Мою, ибо кончились дни жизни твоей, и приблизился день отдохновения твоего.
\vs 2En 30:3
Ибо приблизились времена погибели всей земли, и всякого человека, и всего, что движется по земле, ибо во дни сии велико нестроение на земле.
\vs 2En 30:4
Ибо возненавидел человек ближнего своего, и люди на людей нападают, и народ против народа возбуждает брань, и наполнилась земля кровью и пагубным безпорядком.
\vs 2En 30:5
И ко всему этому они оставили Творца Своего, и стали поклоняться тверди небесной, и ходящим по земле, и волнам морским.
\vs 2En 30:6
И возвеличился Сатана, и радуется делам их.
\vs 2En 30:7
И к негодованию Моему, вся земля восприняла перемену устройства своего, и всякий плод, и всякая трава изменили пору свою, ибо предчувствуют время погибели.
\vs 2En 30:8
И все народы изменились на земле, к сожалению Моему.
\vs 2En 30:9
И тогда Я повелю бездне низринуться на землю, и запасы вод небесных устремятся на землю.
\vs 2En 30:10
И погибнет весь состав земной, и сотрясется вся земля, и лишится силы своей с того дня.
\vs 2En 30:11
Тогда Я сохраню Ноя, первородного сына Ламеха, сына твоего, и воссоздам от семени его иной мир, и семя его пребудет во веки.
\vs 2En 30:12
И пробудился Мафусела от сна своего, и весьма опечалился о сне этом;
\vs 2En 30:13
и призвал всех старейшин народа и поведал им все, что сказал ему Господь, и о видении, явившемся ему от Господа.
\vs 2En 30:14
И опечалились люди из-за видения его, и ответили ему: Господь властен творить по воле Своей.
\vs 2En 30:15
И когда говорил Мафусела народу, взволновался дух его, и преклонил он колени свои, и простер руки свои к небу, и молился Господу, и когда молился он, отошел дух его.

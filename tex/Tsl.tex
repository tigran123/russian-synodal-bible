\bibbookdescr{Tsl}{
  inline={Завещание Соломона},
  toc={Завещание Соломона},
  bookmark={Завещание Соломона},
  header={Завещание Соломона},
  abbr={Зав~Сол}
}
\vs Tsl 0:0
Завещание Соломона, сына Давидова, который был царем в Иерусалиме, и господствовал и управлял всеми духами воздуха на земле и под землей. С их помощью он также совершал все сверхъестественные работы храма. Здесь же говориться о той власти, которой они обладают над людьми, и о том, какие ангелы могут противостоять этим демонам.
\vs Tsl 1:1 
Из премудрости Соломона. Благословен Ты, о Адонаи ЯХВЕ, Который дал Соломону такую власть! Слава тебе и могущество во веки веков. Аминь.
\vs Tsl 1:2 
И вот, когда храм города Иерусалима строился и когда там работали ремесленники, приходил к ним на закате демон Орния\fnote{Орния}{$=$ Птичий.}; и он забирал себе половину платы начальника работ, юного отрока, как половину его пищи. Также он продолжал каждый день сосать большой палец его правой руки. И отрок исхудал, хотя царь его очень любил.
\vs Tsl 1:3 
Итак царь Соломон однажды призвал к себе отрока и стал расспрашивать, говоря ему: "Разве я не люблю тебя больше, чем всех ремесленников, работающих в храме Божием? Не даю ли тебе двойную плату и двойное даяние пищи? Почему же тогда ты день за днём и час за часом худеешь?"
\vs Tsl 1:4 
Но отрок сказал царю: "Я прошу тебя, о царь, выслушай, что происходит с твоим дитя. После того, как мы прекращаем работать на строительстве храма Божия, после заката, когда я ложусь, чтобы отдохнуть, приходит один из злых демонов и забирает у меня одну половину моей платы и одну половину моей пищи. Потом он хватает меня за правую руку и сосет мой большой палец. И вот, моя душа страдает, а тело с каждым днем становится тоньше".
\vs Tsl 1:5 
Тотчас, как я, Соломон, услышал это, я вошел в храм Божий и от всей души молился ночью и днем, чтобы демон попал мне в руки и я получил бы власть над ним. И по моим молитвам это произошло~--- милость ЯХВЕ Саваофа была дана мне через Его архангела Михаила. (Он принес мне) маленькое кольцо с печатью, вырезанной на камне, и сказал мне: "Возьми, о Соломон, царь, сын Давида, дар, который ЯХВЕ Саваоф, Бог Всевышний, послал тебе. Им ты свяжешь всех демонов земли, мужского и женского рода; и с его помощью ты построишь Иерусалим. Ты [должен] носить эту печать Бога. И эта вырезанная печать на кольце, которое тебе послано,~--- пентальфа.
\vs Tsl 1:6 
И я, Соломон, был очень обрадован, и восхвалял и прославлял Бога неба и земли. И наутро я призвал отрока, и дал ему кольцо, и сказал: "Возьми это, и в час, когда демон придет к тебе, брось в грудь демону это кольцо и скажи ему: "Именем Бога, царь Соломон призывает тебя сюда"; и затем спеши ко мне без какого-либо опасения обо всем, что ты можешь услышать от демона".
\vs Tsl 1:7 
Итак отрок взял кольцо и ушел; и вот, в обычный час жестокий демон Орния пришел подобно пылающему огню, чтобы взять плату у отрока. Но тот, согласно советам, полученным от царя, бросил кольцо в грудь демону и сказал: "Царь Соломон призывает тебя сюда"; и затем побежал к царю. Но демон громко возопил, говоря: "Дитя, почему ты сделал это со мною? Убери от меня кольцо, и я воздам тебе золотом земли. Только убери его от меня, и не веди меня к Соломону".
\vs Tsl 1:8 
Но отрок сказал демону: "Жив ЯХВЕ, Бог Израилев! я не потерплю тебя. Итак приди сюда". И отрок, радуясь, поспешил к царю и сказал: "Я привел демона, о царь, как ты мне приказал, о мой господин. И вот, он стоит перед воротами двора твоего дома, громко плача и умоляя, обещая мне серебро и золото земли, если только я не стану приводить его к тебе".
\vs Tsl 1:9 
И когда Соломон услышал это, он встал со своего трона и вышел в притвор двора своего дома; и вот он увидел демона, дрожащего и трепещущего. И он сказал ему: "Кто ты?" И демон ответил: "Меня зовут Орния".
\vs Tsl 1:10 
И Соломон сказал ему: "Скажи мне, о демон, какому зодиакальному знаку ты подчиняешься?" И тот ответил: "Водолею. И тех, кто охвачен страстью к благородным девам на земле, таких я удавляю.\fnote{Водолею. И тех, \ldots\ я удавляю.}{\vsep\ я удавляю тех, кто принадлежит Водолею из-за их страсти к женщинам, родившимся под знаком Девы.} Но в случае, когда я не расположен ко сну\fnote{не расположен ко сну}{\vsep\ не нахожусь в трансе.}, я превращаюсь в три формы. Всякий раз, когда мужи собираются увлечься женщинами, я превращаюсь в миловидную женщину; и я овладеваю мужами во сне их и играю с ними. И потом я снова обретаю мои крылья и спешу к небесным областям. Я также являюсь как лев, и через меня повелевают все демоны. Я~--- семя архангела Уриила, силы Божией".
\vs Tsl 1:11 
Я, Соломон, услышав имя архангела, молился и прославлял Бога, Господа неба и земли. И я запечатал демона и заставил его работать в каменоломне, чтобы он мог обтесывать для храма камни, лежащие на побережье, принесенные Аравийским морем. Но он, испугавшись железа, продолжал и сказал мне: "Я умоляю тебя, царь Соломон, позволь мне идти на свободу; и я приведу к тебе всех демонов". И поскольку он не желал подчиняться мне, я помолился, чтобы пришел архангел Уриил и помог мне; и я тотчас увидел архангела Уриила, сходящего ко мне с небес.
\vs Tsl 1:12 
И ангел приказал морским чудовищам выйти из бездны. И он поверг своё определение на землю, и оно подчинило огромного демона\fnote{он поверг \ldots\ демона}{\vsep\ он изсушил их род и поверг его судьбу на землю.}. И он заставил огромного и дерзкого демона Орнию тесать камни для храма (и приносить для завершения строительства Храма). И так я, Соломон, прославлял Бога, Творца неба и земли. И он приказал Орнии идти своим предназначением, и дал ему печать, сказав: "Удались и принеси мне сюда князя всех демонов".
\vs Tsl 1:13 
И тогда Орния взял перстень и отправился к Ваал-Зевулу, царствующему над демонами. Он сказал ему: "Соломон зовет тебя сюда". Но Ваал-Зевул, услышав, сказал ему: "Поведай мне, кто этот Соломон, о котором ты мне говоришь?" Тогда Орния бросил в грудь Ваал-Зевула кольцо, говоря: "Царь Соломон зовет тебя". Но Ваал-Зевул громко завопил сильным голосом и изверг великое огненное пламя; и он поднялся и последовал за Орнией и пришел к Соломону.
\vs Tsl 1:14 
И когда я увидел князя демонов, я прославил Адонаи ЯХВЕ, Творца неба и земли, и я сказал: "Благословен Ты, ЯХВЕ, Бог Всемогущий, Который дал твоему рабу, Соломону, мудрость, Советник мудрых, и подчинил мне всю власть дьявола".
\vs Tsl 1:15 
И я вопрошал его, и сказал: "Кто ты?" Демон ответил: "Я~--- Ваал-Зевул, начальник демонов. И все демоны имеют свои главные седалища возле меня. И я~--- тот, кто делает появление каждого демона явным".\fnote{"Я~--- Ваал-Зевул, \ldots\ явным"}{\vsep\ "Я~--- Ваал-Зевул, правитель демонов". Я потребовал, чтобы он тотчас сел рядом со мною и объяснил, как можно заставить демонов появиться.} И он пообещал принести ко мне в узах всех нечистых духов. И я снова прославил Бога неба и земли, ибо я всегда благодарю Его.
\vs Tsl 1:16 
Тогда я спросил демона, есть ли среди них женского пола. И когда он сказал мне, что есть, я сказал, что я хотел бы их увидеть. Итак Ваал-Зевул ушел весьма скоро и принес ко мне Оноскелию\fnote{Оноскелию}{$=$ Ослоногая.}, которая имела очень красивый вид и кожу\fnote{кожу}{\vsep\ лицо.} цвета светлой женщины (но ноги, как у осла); и она трясла своей головой.
\vs Tsl 1:17 
И когда она пришла, я сказал ей: "Поведай мне, кто ты?" Но она сказала мне: "Меня зовут Оноскелия, дух действующий, скрывающийся в земле\fnote{дух действующий, скрывающийся в земле}{\vsep\ Я~--- дух, который создается в трупах.}. Есть золотая пещера, где я лежу. Но я имею место, всегда меняющееся. В одно время я удавляю мужчин петлей; в другое~--- я подкрадываюсь от природы к рукам\fnote{рукам}{\vsep\ червям.}. Но чаще всего местом моего жительства являются пропасти, пещеры, овраги. Однако, я часто сопроводжаю мужчин в образе женщины, и особенно~--- тех, у кого темная кожа. Поскольку они разделяют мою звезду со мною,~--- так как они те, кто тайно или явно поклоняется моей звезде, не зная, что они вредят сами себе,~--- то лишь возбуждают моё влечение для дальнейшего вреда. Поскольку они желают обезпечить себя деньгами посредством поминания [меня], я даю немного тем, кто правильно поклоняется мне".
\vs Tsl 1:18 
И я, Соломон, спросил о её рождении; и она ответила: "Я рождена безвременным гласом, так называемым эхом человеческого праха, падшего через древо".
\vs Tsl 1:19 
И я сказал ей: "Под какой звездой ты проходишь?" И она ответила мне: "Под звездой полной луны, потому что луна движется над большинством вещей". Тогда я сказал ей: "А какой ангел упраздняет тебя?" И она сказала мне: "Тот, что в тебе управляет". И я подумал, что она насмехается надо мною, и приказал воину ударить её. Но она громко закричала и сказала: "Я [подчиняюсь] тебе, о царь, мудростью Бога, данной тебе, и ангелу Иоилю".
\vs Tsl 1:20 
Тогда я приказал, чтобы она свивала коноплю для веревок, используемых на строительстве дома Божия; и таким образом, когда я запечатал и связал её, она покорилась и смирилась настолько, что постоянно вила веревки день и ночь.
\vs Tsl 1:21 
И я тотчас же приказал привести ко мне другого демона; и тотчас приблизился ко мне связанный демон Асмодей; и я спросил его: "Кто ты?" Но он пронзил меня гневным и яростным взглядом и сказал: "А ты кто?" И я сказал ему: "Так как наказан ты, отвечай мне, кто ты". Но он с гневом сказал мне: "Но как я буду отвечать тебе, сыну человека, тогда как я был рожден от семени ангела человеческой дочерью, так что никакое слово нашего небесного образа не может быть обращено высокомерному человеку, который родился на земле? Также поэтому моя звезда сияет на небе, и люди называют её~--- некоторые Колесницей, а некоторые Сыном дракона. Я соблюдаюсь около этой звезды. Так что не спрашивай меня о многих вещах; твоё царство после краткого времени также будет разрушено, а твоя слава~--- лишь на год. И недолгим будет твоё насилие над нами; и потом мы снова будем иметь свободу власти над людьми, так что они станут почитать нас, как если бы мы были богами, потому что они, люди, не знают имен ангелов, поставленных над нами".
\vs Tsl 1:22 
И я Соломон, услышав это, связал его еще сильнее и приказал, чтобы его выпороли плетью из воловьей шкуры, и [чтобы он] поведал мне смиренно, как его имя и какое его дело. И он ответил мне так: "Среди смертных меня называют Асмодеем, а моё дело замышлять против новобрачных, так чтобы они не смогли познать друг друга. И я совсем разлучаю их многими бедами, и я напрочь истощаю красоту девственниц и отчуждаю их сердца".
\vs Tsl 1:23 
И я сказал ему: "Это единственное, чем ты занимаешься?" И он ответил мне: "Я довожу мужей до приступов безумия и страсти, так чтобы они, имея своих жен, покидали и уходили ночью и днем к тем, которые принадлежат другим мужам; и таким образом они совершают грех и впадают в смертные дела"\fnote{"Я довожу \ldots\ дела"}{\vsep\ я простираю безумие на женщин посредством звезд, и я часто совершал стремительные убийства.}.
\vs Tsl 1:24 
И я заклинал его именем ЯХВЕ Саваофа, говоря: "Убойся Бога, Асмодей, и скажи, какой ангел упраздняет тебя?" А он сказал: "Рафаил, архангел, который стоит перед престолом Божиим. Но желчь и печень рыбы обращают меня в бегство, если воскурять на пепле тамариска\fnote{на пепле тамариска}{\vsep\ на древесных углях.}". Я снова спросил его и сказал: "Не скрывай от меня ничего, ибо я~--- Соломон, сын Давида, царь Израилев. Поведай мне имя рыбы, которая тебя поражает". И он ответил: "Это имя~--- Глан\fnote{Глан}{$=$ большая зубатка.}, и находится в реках Ассирии; ибо я блуждаю возле тех пределов".
\vs Tsl 1:25 
И я сказал ему: "Нет ли еще чего у тебя, Aсмодей?" И он ответил: "Сила Божия, которой [ты] связал меня нерасторжимыми узами вот той печати, знает; поэтому всё, что я говорю тебе~--- правда. Я прошу тебя, царь Соломон, не осуждай меня в воду". Но я улыбнулся и сказал ему: "Жив ЯХВЕ, Бог отцов моих! я наложу на тебя носить железо. Но ты сделаешь также глину для строительства всего храма, утаптывая её твоими ногами". И я приказал, чтобы ему дали десять сосудов, чтобы носить в них воду. И демон ужасно стонал и делал работу, которую я приказал ему делать. И я сделал так, потому что этот жестокий демон Асмодей знал даже будущее. И я, Соломон, прославлял Бога, Который дал мудрость мне, Соломону, Его рабу. А печень и желчь рыбы я подвесил на колос тростника\fnote{на колос тростника}{\vsep\ на ветке стиракса.} и поджег это над Асмодеем, ибо он был очень сильным, и так была упразднена его невыносимая злоба.
\vs Tsl 1:26 
И я снова вызвал Ваал-Зевула, князя демонов, предстать предо мною, и я посадил его на возвышенном почетном месте и сказал ему: "Почему ты одинок, князь демонов?" И он ответил мне: "Потому что я один остался из небесных ангелов, которые сошли вниз. Ибо я был первым ангелом на первом небе, будучи наречен Ваал-Зевулом. И теперь я стерегу всех тех, кто заключен в тартар. Но у меня есть ребенок\fnote{у меня есть ребенок}{\vsep\ меня сопровождает другой безбожный.}, который является в море Суф. И при каждом удобном случае он приходит ко мне снова, будучи подчинен мне, и открывает мне, что бы он сделал, и я помогаю ему".
\vs Tsl 1:27 
Я, Соломон, сказал ему: "Ваал-Зевул, чем ты занимаешься?" И он ответил мне: "Я истребляю царей\fnote{Я истребляю царей}{\vsep\ Я приношу гибель с помощью властителей.}, объединяясь с иноземными властителями. И моих собственных демонов я помещаю в людей, чтобы последние веровали в них и заблуждались. И избранных рабов Божиих, священников и верных людей, я возбуждаю к дурным грехам и злой ереси и беззаконным делам; и они повинуются мне, а я веду их к гибели. И я внушаю людям зависть и убийства, войны, мужеложества и другие злые вещи. И я уничтожу мир".
\vs Tsl 1:28 
Итак я сказал ему: "Принеси ко мне своего ребенка, который, как ты говоришь, в море Суф". Но он сказал мне: "Я не стану приносить его к тебе. Но вот придет ко мне другой демон по имени Эфиппа\fnote{Эфиппа}{$=$ всадник.}. Я свяжу его и он принесет его ко мне из бездны". И я сказал ему: "Как пришел на дно моря твой сын и как его имя?" И он ответил мне: "Не спрашивай меня, ибо ты не можешь научиться от меня. Однако, он придет к тебе по любому повелению и скажет тебе открыто". (Потом я спросил его: "Поведай мне, в какой звезде ты живешь?"~--- "В той, которую люди называют Вечерней Звездой").
\vs Tsl 1:29 
Я сказал ему: "Поведай мне, каким ангелом ты упраздняешься?" И он ответил: "Святым и драгоценным Именем Всемогущего Бога. И если кто-либо заклинает меня этим великим Именем, я сразу исчезаю".
\vs Tsl 1:30 
Я, Соломон, был изумлен, когда услышал это; и я приказал ему распиливать фиванский мрамор. И когда он начал пилить его, другие демоны возопили громким голосом, воя из-за их царя Ваал-Зевула.
\vs Tsl 1:31 
Но я, Соломон, спросил его, говоря: "Ты получишь облегчение, если расскажешь мне о том, что происходит на небесах". И Ваал-Зевул сказал: "Послушай, о царь, если ты будешь жечь смолу и ладан, и морской бульб, с нардом и шафраном, и зажжешь последовательно\fnote{последовательно}{\vsep\ в землетрясение.} семь лампад, ты прочно установишь твой дом. И если, будучи чистым, ты зажжешь их на рассвете под лучами солнца, тогда ты увидишь небесных драконов, как они извиваются и тащат колесницу солнца".
\vs Tsl 1:32 
И я, Соломон, услышав это, упрекнул его и сказал: "Теперь замолчи и продолжай распиливать мрамор, как я тебе приказал". И я, Соломон, прославил Бога и велел, чтобы другой демон стал предо мною. И он пришел предо мною, с лицом, которое находилось высоко над землей, но остальное тело духа было закручено подобно спирали. И он прорвался между несколькими воинами и также поднял ужасную пыль с земли, вознеся её вверх, а потом снова швырнув её на нас, чтобы испугать, и спросил, какие вопросы я обычно задаю. И я встал, и плюнул в то место на земле, и запечатал кольцом Бога. И тотчас кружение пыли прекратилось. Потом я спросил его, сказав: "Кто ты, о дух?" Тогда он еще раз возмутил пыль и ответил мне: "Что хочешь ты получить, царь Соломон?" Я ответил ему: "Поведай мне, как тебя зовут, и я охотно задам тебе вопрос. А пока я воздам благодарение Богу, Который сделал меня мудрым, чтобы отвечать на эти злые козни".
\vs Tsl 1:33 
Но [демон] ответил мне: "Я~--- дух праха". И я сказал ему: "Чем ты занимаешься?" И он ответил: "Я навожу тьму на людей и поджигаю поля; и я довожу поместья до ничтожества. Но больше всего я занят летом. Однако, если есть возможность, я заползаю в углы стен, ночью и днем. Ибо я~--- семя великого, и никак не меньше". Тогда я сказал ему: "Какой звезде ты подлежишь?" И он ответил: "На самом конце рога луны, когда она находится на юге,~--- там моя звезда. Поскольку я повелеваю ослабить судороги полутородневной лихорадки, поэтому люди много молятся, [чтобы вылечиться от] полуторадневной лихорадки, употребляя эти три имени: Бультала, Фаллаль, Meлхаль,~--- и я излечиваю их". И я сказал ему: "Я~--- Соломон; посему [ответь], когда ты вредишь, чьей властью ты это делаешь?" А он сказал мне: "Тем ангелом, который также может успокоить лихорадку на третий день". Тогда я спросил его и сказал: "Как его имя?" И он ответил: "Это архангел Азаил". И я призвал [имя] архангела Азаила и запечатал демона, и приказал ему брать большие камни и бросать их рабочим, которые находятся на самых верхних местах храма. И понуждаемый, демон начал делать то, что ему приказали делать.
\vs Tsl 1:34 
И я опять прославил Бога, Который даровал мне эту власть, и повелел, чтобы другой демон стал предо мною. И пришли семь духов женского пола, связанные и соединенные вместе, прекрасные наружностью и миловидные. И я, Соломон, увидев их, спросил и сказал: "Кто вы?" А они согласно сказали в один голос: "Мы~--- тридцать три стихии миродержца тьмы". И первая сказала: "я~--- Ложь"; вторая сказала: "я~--- Раздор"; третья: "я~--- Клофод, то есть сражение"; четвертая: "я~--- Ревность"; пятая: "я~--- Сила"; шестая: "я~--- Грех"; седьмая: "я~--- самая худшая из всех; и наши звезды находятся на небесах. Семь звезд слабо мерцают, и все вместе. И мы названы, как если бы мы были богини. Мы меняем наши места все вместе, и вместе мы живем, иногда~--- в Лидии, иногда~--- на Олимпе, иногда~--- на большой горе".
\vs Tsl 1:35 
Тогда я, Соломон, стал спрашивать их одну за другой, начиная с первой и дойдя до седьмой. Первая сказала: "Я~--- Ложь, я обманываю и плету сети здесь и там. Я источаю и возбуждаю ересь. Но я имею ангела, который упраздняет меня~--- Ламехалала".
\vs Tsl 1:36 
Также вторая сказала: "Я~--- Раздор, раздор раздоров. Я тотчас приношу бревна, камни, крюки,~--- моё оружие. Но я имею ангела, который меня упраздняет~--- Варахиила".
\vs Tsl 1:37 
Также третья сказала: "Меня называют Клофод, или Сражение, и я заставляю благонравного стать рассеянным и сталкиваться с другими. И почему я говорю так много? Я имею ангела, который упраздняет меня~--- Мармарафа".
\vs Tsl 1:38 
Также четвертая сказала: "Я заставляю людей забывать об умеренности и выдержке. Я разделяю их и раскалываю на части; ради раздора [они] следуют за мной рука об руку. Я отрываю мужа от брачного ложа, и детей~--- от родителей, и братьев~--- от сестер. Но зачем сообщать так много о моей злобе? Я имею ангела, который упраздняет меня~--- великого Бальтиала".
\vs Tsl 1:39 
Также пятая сказала: "Я~--- Сила. Силою я возвожу властелинов и свергаю царей. Я снабжаю силой всех мятежников. Я имею ангела, который упраздняет меня~--- Аштерофа".
\vs Tsl 1:40 
Также шестая сказала: "Я~--- Грех, о царь Соломон. И я сделаю так, чтобы ты грешил, как я делала прежде, когда я заставила тебя убить собственного брата. Я введу тебя в грех, чтобы раскапывать могилы; и я научу тех, кто роет, и уведу грешные души далеко от всякого благочестия, и во мне есть много других злых свойств. Но я имею ангела, который упраздняет меня~--- Уриила".
\vs Tsl 1:41 
Также ответила седьмая: "Я~--- самая худшая, и я сделаю тебя еще хуже, чем ты был; потому что я налагаю узы Артемиды. Но саранча освободит меня, ибо это значит, что тебе суждено выполнить мою волю. Ибо, если кто-либо был бы мудр, он не стал бы обращать свои стопы ко мне".
\vs Tsl 1:42 
И я, Соломон, слушая и удивляясь, запечатал их моим кольцом; и так как их было так много, я приказал им копать основания для храма Божия. Ибо длина его была двести пятьдесят локтей. И я повелел им быть трудолюбивыми, и с ропотом, все вместе противясь, они начали выполнять поставленные задачи.
\vs Tsl 1:43 
А я, Соломон, прославил Бога и приказал привести предо мною другого демона. И был принесен ко мне демон, имеющий все члены человека, но без головы. И я, увидев его, сказал ему: "Поведай мне, кто ты?" И он ответил: "Я~--- демон". Тогда я сказал ему: "Какой?" И он ответил мне: "Я зовусь Завистью. Ибо я наслаждаюсь пожиранием голов, желая получить себе голову; но я не насыщаюсь вдоволь, а стремлюсь иметь такую голову, как у тебя".
\vs Tsl 1:44 
Я, Соломон, услышав это, запечатал его, протянув свою руку к его груди, отчего демон подскочил и низвергся, и застонал, говоря: "Горе мне! Куда я пришел? O предатель Орния, я не могу видеть!" Тогда я сказал ему: "Я~--- Соломон. Поведай мне, как ты управляешь зрением?" И он ответил мне: "Посредством моих чувств". И тогда я, Соломон, слыша его голос доходящим до меня, спросил его, как он может говорить. И он ответил мне: "Я, о царь Соломон,~--- и есть целиком голос, потому что я унаследовал голоса множества людей. Ибо для всех людей безразлично, которых называют немыми, я~--- тот, кто разбивает их головы, когда они были детьми и достигли своего восьмого дня. Когда ребенок плачет ночью, я становлюсь духом и подкрадываюсь посредством его голоса. На перепутьях также я исполняю множество служений и столкновение со мной исполнено вреда. Ибо я мгновенно хватаю человеческую голову, и своими руками, как мечом, отсекаю её и приставляю её к себе. И потом, посредством огня, который находится во мне, я глотаю это через шею. Я~--- тот, кто насылает тяжелые и неизлечимые увечья человеческим ногам и причиняет язвы".
\vs Tsl 1:45 
И я, Соломон, услышав это, сказал ему: "Поведай мне, как ты извергаешь спереди огонь? Из какого источника ты испускаешь его?" И дух сказал мне: "От утренней звезды. Ибо вот еще не обнаружили Элвуриона, которому люди возносят молитвы и возжигают огни. И его имя призывают семь демонов предо мною. И он дорожит ими".
\vs Tsl 1:46 
Но я сказал ему: "Поведай мне его имя". А он ответил: "Я не могу поведать тебе. Ибо если я сообщу его имя, я окажусь неисцельным. А он придет в ответ на своё имя". И услышав это, я, Соломон, сказал ему: "Сообщи мне тогда, какой ангел упраздняет тебя". И он ответил: "Огненная вспышка молнии". И я преклонил колени перед ЯХВЕ, Богом Израилевым, и приказал ему оставаться под присмотром Ваал-Зевула, пока не придет Иакс.
\vs Tsl 1:47 
Потом я повелел другому демону придти предо мною, и вот вошла в моём присутствии собака очень большого вида и так проговорила громким голосом, и сказала: "Привет, господин, царь Соломон!" И я, Соломон, был изумлен. Я сказал этому: "Кто ты, о собака?" И тот ответил: "Я действительно кажусь тебе собакой, но прежде, нежели ты появился, о царь Соломон, я уже был человеком, который сотворил много нечестивых дел на земле. Я так превосходно научился в письменах и стал настолько могуществен, что мог бы повернуть звезды небесные вспять. И много колдовских дел я подготовил. Поскольку я причиняю вред людям, которые следуют за нашей звездой, и обращаю их к \ldots\ И я хватаю неистовых людей за горло, и так уничтожаю их".
\vs Tsl 1:48 
Я, Соломон, сказал ему: "Как твоё имя?" И он ответил: "Жезл". И я спросил: "Чем ты занимаешься? И какого результата ты можешь достичь?" И он ответил: "Дай мне своего человека, и я поведу его далеко в гористую местность и покажу ему зеленые камни, разбросанные здесь и там, которыми ты сможешь украсить храм Адонаи ЯХВЕ".
\vs Tsl 1:49 
И я, Соломон, услышав это, приказал своему слуге отправиться с ним и взять перстень, неся с собою печать Бога. И я сказал ему: "Того, кто покажет тебе зеленые камни, запечатай этим кольцом. И тщательно заметь это место и принеси мне демона сюда". И демон показал ему зеленые камни, и слуга запечатал его и принес демона ко мне. И я, Соломон, решил заключить моей печатью на моей деснице сих двух~--- безголового демона [и] подобного собаке, который был так огромен, [что] его следовало хорошо связать. И я повелел собаке надежно держать огненного духа, так чтобы он мог, как светильники, днем и ночью отбрасывать свет через свою утробу ремесленникам на работе.
\vs Tsl 1:50 
И я, Соломон, взял из шахты камень в двести сиклей для опор стола воскурений, который был похож по виду. И я, Соломон, прославил Адонаи ЯХВЕ, а потом скрыл обратно тот драгоценный камень. И я снова приказал демонам тесать мрамор для строительства дома Божия. И я, Соломон, помолился ЯХВЕ и спросил собаку, говоря: "Какой ангел упраздняет тебя?" И демон ответил: "Великий Брий".
\vs Tsl 1:51 
И я восхвалил ЯХВЕ, Господа неба и земли, и повелел другому демону придти предо мною; и вот ко мне явился некто в виде ревущего льва. И он стал и ответил мне, говоря: "О царь, в том виде, который я имею, я~--- дух, весьма неспособный к восприятию. Подкравшись, я набрасываюсь на всех людей, которые обезсилены болезнью. И я исполняю человека слабостью, так чтобы его обычное тело было расслаблено. Но я также имею другую славу, о Царь. Я низвергаю демонов, и я имею легионы под моим управлением. И я способен к похищению в местах моего пребывания наряду со всеми демонами подчиненных мне легионов". Но я, Соломон, услышав это, спросил: "Как твоё имя?" А он ответил: "Я~--- Львоносец, по природе стремительный". И я сказал ему: "Как упразднить тебя вместе с твоими легионами? Какой ангел упраздняет вас?" И он ответил: "Если я сообщу тебе его имя, я свяжу не только одного себя, но также легионы демонов, подчиненных мне".
\vs Tsl 1:52 
И тогда я сказал ему: "Заклинаю тебя Именем Бога Саваофа, назови мне имя того, кто упраздняет тебя вместе с твоим множеством". И дух ответил мне: "Величайший из людей, Который много пострадает от рук человеческих, Чье имя~--- Эммануил; Он~--- Тот, Который пленит нас, и Который придет и потопит нас с крутизны под водой. Он возвестит повсюду в трех письменах, которые низвергнут нас".
\vs Tsl 1:53 
И я, Соломон, услышав это, прославил Бога и осудил его легион носить бревна из рощи. И я осудил львообразного распиливать молодые деревья своими клыками, чтобы сжигать их в неугасимом пламени печи храма Божия.
\vs Tsl 1:54 
И я поклонился ЯХВЕ, Богу Израилеву, и приказал другому демону придти предо мною. И вот пришел трехглавый дракон устрашающего цвета. И я спросил его: "Кто ты?" И он ответил мне: "Я~--- иглоподобный дух, чье действие в трех направлениях. Я ослепляю детей в женских утробах и выкручиваю их уши. И я делаю их глухими и немыми. И, кроме того, посредством своей третьей головы я могу кружить. И я поражаю людей в солнечное сплетение и вызываю у них падучую и пену и скрежет зубов. Но я имею свой собственный путь, который упразднится,~--- Иерусалим, означенный в Писании, в том месте, которое названо "Черепом". Ибо там предопределен Ангел Великого Совета. И тогда Он будет явно пребывать на кресте. Он упразднит меня, и я подчинюсь Ему.
\vs Tsl 1:55 
Но на том месте, где ты возседаешь, о царь Соломон, будет возвышаться столп в воздухе пурпурного цвета. Демон именем Эфиппа принесет его из моря Суф, из внутренней Аравии. Он тот, кто должен быть закрыт в мехе и принесен пред тобою. А при входе в храм, который ты начал строить, о царь Соломон, спрятано много золота, которое ты выкопаешь и унесешь". И я, Соломон, послал своего слугу и нашел то, о чем сказал демон. И я запечатал его моим кольцом и восхвалил Адонаи ЯХВЕ.
\vs Tsl 1:56 
Тогда я сказал ему: "Как тебя зовут?" И демон ответил: "Я~--- Драконий гребень". И я велел ему делать кирпичи для храма. Он имел человеческие руки.
\vs Tsl 1:57 
И я поклонился ЯХВЕ, Богу Израилеву, и приказал явиться другому демону. И вот пришел предо мной дух в виде женщины, которая имела только голову без каких-либо конечностей, и её волосы были всклокочены. И я сказал ей: "Кто ты?" А она ответила: "Нет, кто ты? И почему ты хочешь услышать обо мне? Но да будет тебе известно, вот я стою связанная пред лицом твоим. Теперь иди в свои царские чертоги и умой свои руки. Потом сядь снова пред твоим судилищем и задавай мне вопросы; и ты узнаешь, о царь, кто я".
\vs Tsl 1:58 
И я, Соломон, сделал так, как она потребовала от меня, и потому что сдержался по мудрости, обитающей во мне, чтобы я смог услышать о её делах, и осудить их, и объявить их людям. И я сел и сказал демону: "Кто ты?" И она сказала: "Я называюсь среди людей Обизуфь; и по ночам я не сплю, а брожу своими кругами по всему миру и посещаю женщин при деторождении. И в назначенный час я занимаю своё место, и если мне везет, я удавляю ребенка. Но если нет~--- я удаляюсь в другое место, потому что я не могу за одну ночь остаться безсупешной. Ибо я~--- жестокий дух с мириадами имен и множеством образов. И ныне я блуждаю туда-сюда. И я отправляюсь на закате моими кругами. Но ныне, как бы ты не запечатывал меня окрест кольцом Бога, ты ничего не сделаешь. Я не остановлюсь перед тобою, и ты безсилен командовать мною. Ибо я не имею никакой иной работы, кроме как уничтожаю детей, делаю их уши глухими, творю вред их глазам и спутываю их уста узами, и гублю их разум, причиняю боль их телам.
\vs Tsl 1:59 
Когда я, Соломон, услышал это, я поразился её внешностью, ибо я видел, что всё её тело было из тьмы. Но её отражение было вполне ярким и зеленеющим, и её волосы были безпорядочно растрепаны как у дракона; а всё остальное тело было невидимым. И у неё был очень отчетливый голос, как он доносился до меня. И я коварно сказал: "Поведай мне, какой ангел упраздняет тебя, о злой дух?" Она ответила мне: "Ангел Божий по имени Афарот (который значит Рафаил) тот, кто упраздняет меня ныне и во все времена. Если это имя знает какой-либо муж, и его же напишет на женщине при деторождении, тогда мне будет невозможно проникнуть в неё. У этого имени число~--- 640". И я, Соломон, услышав это, прославил Бога, и повелел, чтобы ей связали волосы и подвесили её на фасаде храма Божия; чтобы все дети Израилевы, проходя мимо, могли видеть её и прославлять ЯХВЕ, Бога Израилева, Который дал мне посредством этой печати такую власть с мудростью и силой Божией.
\vs Tsl 1:60 
И я снова приказал другому демону придти предо мною. И пришел, катясь вперед, один, видом подобный дракону, но имеющий лицо и руки человеческие. И все его члены, кроме ног, были как у дракона; и у него были крылья на спине. И когда я увидел его, то изумился и сказал: "Кто ты, демон, и как тебя зовут? И откуда ты прибыл, поведай мне?"
\vs Tsl 1:61 
И дух ответил и сказал: "Вот, первый раз я сам предстал, о царь Соломон. Я~--- дух, сделанный богом среди людей, но теперь сведенный к ничтожеству кольцом и мудростью, которыми удостоил тебя Бог. Ныне я~--- так называемый Крылатый дракон, и я сожительствую не с многими женщинами, но лишь с некоторыми, которые красивы видом, которые обладают именем древа, от этой звезды. И я соединяюсь с ними в облике крылатого по виду духа, совокупляясь с ними через зад. И та, на которую я набросился, тяжело вынашивает ребенка, и тот, родившись от неё, становится страстным. Но так как такое потомство нельзя пронести мимо людей, женщина бьется в сомнении духа. Такова моя роль. Итак только допустим, что я убежден; и все другие демоны, которым ты досаждаешь и безпокоишь, скажут тебе всю правду. Но те, которые состоят из огня, станут поводом для сожжения огнем дeревянного материала, который будет собран ими для постройки храма".
\vs Tsl 1:62 
И как только демон сказал это, я увидел духа, вышедшего наружу из его уст, и он потребил древесину ладанного дерева и сжег все бревна, которые мы поместили в храме Божием. И я, Соломон, увидел, что сделал дух, и я изумился.
\vs Tsl 1:63 
И, прославив Бога, я спросил драконообразного демона, и сказал: "Поведай мне, какой ангел упраздняет тебя?" И он ответил: "Великий ангел, который имеет седалище на втором небе, который называется по-еврейски Базазеф. И я, Соломон, услышав это и призвав к себе его ангела, осудил его распиливать мрамор для строительства храма Божия; и я прославил Бога, и велел другому демону придти предо мною.
\vs Tsl 1:64 
И вот пред лицо моё пришел другой дух~--- как бы в виде женщины. Но на своих плечах она имела еще две другие головы с руками. И я спросил её, и сказал: "Поведай мне, кто ты?" И она ответила: "Я~--- Энепсиг, имеющая также мириады имен". И я сказал ей: "Какой ангел упраздняет тебя?" Но она сказала мне: "Чего ты ищешь, о чем спрашиваешь? Я подвержена изменчивости, подобно богине названа я. И я меняюсь снова и перехожу во власть другой формы. И поэтому не желай узнать всё, что касается меня. Но поскольку ты передо мною именно из-за этого многого,~--- слушай. Я имею свою обитель на луне, и по этой причине я обладаю тремя формами. Время от времени меня магически вызывает мудрый, как Крон. В другое время, из-за тех, кто приносит меня вниз, я низхожу и являюсь в другой форме. Мера стихии непостижима и неприступна, и не может быть упразднена. Тогда я, изменяясь в эти формы, низхожу и становлюсь такой, какой ты меня видишь; но меня упраздняет ангел Рафанаил, который возседает на третьем небе. Вот почему я говорю с тобой. Вон тот храм не может сдержать меня".
\vs Tsl 1:65 
Поэтому я, Соломон, помолился моему Богу, и призвал [имя] ангела, о котором говорила мне Энепсиг, и применил свою печать. И я запечатал её тройной цепью, и внизу её [поместил] замок цепи. Я использовал печать Бога, и дух прорицал мне, говоря: "Вот как ты, царь Соломон, обращаешься с нами! Но по времени твоё царство прекратится, и этот храм будет разломан на части; и весь Иерусалим будет уничтожен царем Персов и Мидян и Халдеев. И сосуды сего храма, который ты соделываешь, будут преданы в рабское услужение богам; и вместе с ними все сосуды, в которых ты нас заключил, будут разбиты человеческими руками. И потом мы пойдем дальше в великой силе туда и сюда. И мы возглавим вселенную на долгие годы, пока Сын Божий не будет распят на кресте. Ибо никогда прежде не возставал царь, подобный Ему, единственный, упраздняющий нас всех, мать Которого не прикоснется к мужу. Кто еще может получить такую власть над духами, кроме Него~--- Того, Кого дьявол изначально будет искать соблазнить, но не превозможет (число его имени~--- 644), Который есть~--- Эммануил. Вот почему, о царь Соломон, твоё время~--- злое, и годы твои~--- кратки и злы, и твоему слуге будет отдано твоё царство".
\vs Tsl 1:66 
И я, Соломон, услышав это, прославил Бога. И хотя я удивлялся оправданию демонов, я не верил этому, пока это не произошло. И я не верил их словам; но когда они осуществились, тогда я уразумел, и перед своей смертью написал это [Завещание] для сынов Израилевых и дал его им, чтобы они смогли познать могущество демонов и их виды, и имена ангелов, которыми эти ангелы упраздняются. И я прославлял ЯХВЕ, Бога Израилева, и приказал, чтобы духи были связаны нерушимыми узами.
\vs Tsl 1:67 
И восхвалив Бога, я повелел другому духу придти предо мною; и вот пришел пред лицом моим другой демон, имеющий спереди вид лошади, а сзади~--- рыбы. И он имел могущественный глас и сказал мне: "O царь Соломон! я~--- жестокий дух моря, и я алчный до золота и серебра. Я~--- такой дух, который вертится и пересекает водные пространства моря, и я преследую людей, которые проплывают по нему. Ибо я кружусь в волнах и превращаюсь, и потом бросаюсь на корабли и вхожу в них. И это моё занятие и мой путь~--- завладеть деньгами и людьми. Ибо я хватаю людей и, вертясь, кружу их с собою и выбрасываю их вне моря. Ибо я не жажду человеческих тел, но выбрасываю их вне моря столь далеко. Но с тех пор Ваал-Зевул, повелитель духов воздуха и тех, что под землей, и господин земных, имеет единое царствование с нами в отношении действий каждого из нас, поэтому я вышел из моря получить некоторую перспективу в его сообществе.
\vs Tsl 1:68 
Но я имею также другое свойство и роль. Я преображаюсь в волны и восхожу из моря. И я показываюсь людям, так что на земле они называют меня Кинопастон, потому что я принимаю человеческую форму. И имя моё~--- истинное. Ибо своим превращением в человека я вызываю определенное отвращение. Затем я пошел получить совет у князя Ваал-Зевула; и он связал меня и доставил меня в твои руки. И я здесь перед тобой из-за этой печати, и ты ныне мучаешь меня. Вот теперь, через два или три дня, дух, который с тобой разговаривает, ослабеет, потому что я останусь без воды".
\vs Tsl 1:69 
И я сказал ему: "Поведай мне, какой ангел упраздняет тебя?" И он ответил: "Ямеф". И я прославил Бога и приказал, чтобы дух был ввергнут в чашу с десятью кувшинами морской воды по две меры каждый. И я запечатал их кругом сверху мрамором и асфальтом и смолою на горлышке сосуда. И запечатав это моим кольцом, я приказал, чтобы это было положено в храме Божием. И я повелел другому духу придти предо мною.
\vs Tsl 1:70 
И вот пришел пред моё лицо другой порабощенный дух, имеющий смутную человеческую форму, со сверкающими глазами, и неся в руке лезвие. И я спросил: "Кто ты?" А он ответил: "Я~--- похотливый дух, порожденный человеком-гигантом, который погиб в битве во времена гигантов". Я сказал ему: "Поведай мне, чем ты занимаешься на земле и где твоё жилище".
\vs Tsl 1:71 
И он сказал: "Моё жилище находится в плодородных местах, а мой образ действий таков: я сажусь возле людей, которые проходят среди могил, и в неурочное время я принимаю вид мертвеца; и если я поймаю кого-нибудь, я тотчас уничтожаю его моим мечом. Но если я не могу уничтожить его, я причиняю ему обладать демоном и пожирать свою собственную плоть и волосы, пока не отвалится его подбородок". Но я сказал ему: "Убойся Бога неба и земли и поведай мне [имя] ангела, упраздняющего тебя". И он ответил: "Меня уничтожает Тот, Кто должен стать Спасителем, Человек, Чье число, если кто-нибудь напишет его у себя на челе, он победит меня, и я в страхе быстро удалюсь. И, действительно, если кто-либо напишет этот знак на себе, я буду в страхе". И я, Соломон, услышав это и прославив Адонаи ЯХВЕ, заключил этого демона, подобно остальным.
\vs Tsl 1:72 
И я приказал другому демону придти предо мною. И вот пришли пред лицом моим тридцать шесть духов: их головы безформенны, как у собак, но в самих себе они имели человеческий вид; с лицами ослов, лицами волов и лицами птиц. И я, Соломон, слушая их и наблюдая за ними, изумился, и я спросил их и сказал: "Кто вы?" А они единодушно, в один голос сказали: "Мы~--- тридцать шесть стихий, мироправители тьмы. Но, о царь Соломон, ты безсилен навредить нам, ни заключить нас, ни возложить обязательств на нас; но поскольку Адонаи ЯХВЕ дал тебе власть над каждым духом, в воздухе и на земле и под землей, поэтому мы тоже явились пред тобою подобно другим духам, от Овна и Тельца, от Близнецов и Рака, от Льва и Девы, от Весов и Скорпиона, Стрельца, Козерога, Водолея и Рыбы.
\vs Tsl 1:73 
Тогда я Соломон призвал имя ЯХВЕ Саваофа, и расспросил каждого по очереди о том, каково было их свойство. И я повелел каждому выступать вперед и сообщать о своих служениях. Тогда первый выступил вперед и сказал: "Я~--- первый десятник зодиакального круга, и меня называют Овном, и со мной эти двое". Тогда я задал им вопрос: "Как вас зовут?" Первый сказал: "Меня, о господин, зовут Руакс; я причиняю головам людей становиться праздными, и я лишаю их бровей. Но стоит мне услышать слова: "Михаил, заключи Руакса", и я сразу удаляюсь".
\vs Tsl 1:74 
И второй сказал: "Меня зовут Варсафаил, и я причиняю тем, кто подчинен моему часу, чувствовать боль мигрени. Если только я слышу слова: "Гавриил, заключи Варсафаила", сразу удаляюсь".
\vs Tsl 1:75 
Третий сказал: "Меня зовут Аротосаил. Я врежу глазам и тяжко повреждаю их. Только стоит мне услышать слова: "Уриил, заключи Аратосаила", сразу удаляюсь".
\vs Tsl 1:76 
Пятый сказал: "Меня зовут Юдал, и я приношу пробку в уши и глухоту слуху. Если я услышу: "Уриил, [заключи] Юдала", я сразу удаляюсь".
\vs Tsl 1:77 
Шестой сказал: "Меня зовут Сфендонаил\fnote{Сфендонаил}{$=$ праща.}. Я причиняю опухоли околоушных желез и воспаление миндалин и заворот кишок. Если я слышу: "Сабраил, заключи Сфендонаила", сразу удаляюсь".
\vs Tsl 1:78 
И седьмой сказал: "Меня зовут Сфандор, и я расслабляю силу плеч и причиняю им дрожать; и я парализую нервы рук, и я ломаю и дроблю шейные кости. И я, я высасываю костный мозг. Но если я слышу слова: "Араил, заключи Сфандора", я сразу удалаюсь".
\vs Tsl 1:79 
И восьмой сказал: "Меня зовут Белбел. Я развращаю сердца и помыслы людей. Если я слышу слова: "Араил, заключи Белбела", я сразу удаляюсь".
\vs Tsl 1:80 
И девятый сказал: "Меня зовут Куртаил. Я насылаю кишечные колики. Я вызываю боль. Если я слышу слова: "Иаот, заключи Куртаила", я сразу удаляюсь".
\vs Tsl 1:81 
Десятый сказал: "Меня зовут Метафиакс. Я вызываю боль в почках. Если я слышу слова: "Адонаил, заключи Метафиакса", я сразу удаляюсь".
\vs Tsl 1:82 
Одиннадцатый сказал: "Меня зовут Катаникотаил. Я творю ссоры и обиды в человеческих домах и насылаю на них тяжелое настроение. Если кто-либо желает мира в своём доме, пусть он напишет на семи лавровых листьях имя ангела, который упраздняет меня, вместе с этими именами: "Иае, Иео, сыновья Саваофа, во имя великого Бога, да заключат Катаникотаила". Потом пусть промоет лавровые листья в воде и окропит свой дом водой изнутри и снаружи. И сразу я удаляюсь".
\vs Tsl 1:83 
Двенадцатый сказал: "Меня зовут Сафатораил, и я вдохновляю привязанность к людям и восторг по причине их преткновения. Если кто-то напишет на бумаге эти имена ангелов~--- Иасо, Иеало, Иоелет, Саваоф, Ифоф, Бае, и свернув это, будет носить вокруг шеи или на своём ухе, я сразу удаляюсь и рассеиваю пьяный припадок".
\vs Tsl 1:84 
Тринадцатый сказал: "Меня зовут Бобел, и моими нападениями я причиняю болезнь нервов. Если я услышу великое имя: "Адонаил, заключи Бофофела", я сразу удаляюсь".
\vs Tsl 1:85 
Четырнадцатый сказал: "Меня зовут Кумеател, и я наношу приступы дрожи и оцепенения. Как только я услышу слова: "Зороил, заключи Кумеатела", я сразу удаляюсь".
\vs Tsl 1:86 
Пятнадцатый сказал: "Меня зовут Роелед. Я причиняю простуду и озноб, и боль в желудке. Стоит мне только услышать слова: "Иакс, не жди, не горячись, ибо Соломон справедливее двенадцати патриархов", я с[разу] удаляюсь".
\vs Tsl 1:87 
Шестнадцатый сказал: "Меня зовут Атракс. Я навожу на людей лихорадку, неисцелимую и пагубную. Если ты хочешь заключить меня, измельчи кориандр и смажь им губы, произнеся следующее заклинание: "Лихорадка, которая бывает от грязи, я изгоняю тебя престолом Всевышнего Бога, отступи от грязи и отойди от создания, образованного Богом". И сразу я удаляюсь".
\vs Tsl 1:88 
Семнадцатый сказал: "Меня зовут Иеропаил. Я сижу на человеческих животах и причиняю судороги в бане и в дороге; и везде, где я бы ни находился, или обрел человека, я низвергаю его. Но если кто-нибудь скажет бедствующему в ухо, три раза в правое ухо, эти имена: "Юдаризе, Сабуне, Деное", я сразу удаляюсь".
\vs Tsl 1:89 
Восемнадцатый сказал: "Меня зовут Булдумех. Я разлучаю жену от мужа и вызываю ревность между ними. Если кто-нибудь изобразит имена своих праотцов, Соломон, на папирусе и положит его в прихожей своего дома, я отступаю оттуда. И написанная надпись должна быть следующая: "Бог Авраама, и Бог Исаака, и Бог Иакова повелевает тебе: удались из этого дома с миром". И я сразу удаляюсь".
\vs Tsl 1:90 
Девятнадцатый сказал: "Меня зовут Нафаф, и я занимаю своё место на коленях людей. Если кто-нибудь напишет на папирусе: "Фнунобоеол, уведи Нафафа и не трогай шею", я сразу удаляюсь".
\vs Tsl 1:91 
Двадцатый сказал: "Меня зовут Мардеро. Я насылаю на людей неизлечимую лихорадку. Если кто-либо напишет на листе свитка: "Сфенер, Рафаил, изгоните, не обременяйте меня снова, не сдирайте с меня кожу", и повяжет это вокруг своей шеи, я сразу отступаю".
\vs Tsl 1:92 
Двадцать первый сказал: "Меня зовут Алаф, и я причиняю кашель и удушье детям. Если кто-либо напишет на бумаге: "Рорекс, ты преследуй Алафа", и закрепит это на своей шее, я сразу удаляюсь".
\vs Tsl 1:93 
Двадцать третий сказал: "Меня зовут Нефтада. Я причиняю болезнь почек и я приношу недержание мочи. Если кто-либо напишет на оловянной табличке: "Иафоф, Уруел, Нефтада" и закрепит это вокруг чресел, я сразу отступаю".
\vs Tsl 1:94 
Двадцать четвертый сказал: "Меня зовут Актон. Я причиняю болезнь ребер и поясничных мышц. Если кто выгравирует на кусочке меди, взятом с корабля, который сорвался с якоря, следующее: "Мармараоф, Саваоф, преследуйте Актона", и закрепит это вокруг поясницы, я сразу отступаю".
\vs Tsl 1:95 
Двадцать пятый сказал: "Меня зовут Анатреф и я раздираю внутренности жаром и лихорадкой. Но если я слышу: "Арара, Харара", я сразу отступаю".
\vs Tsl 1:96 
Двадцать шестой сказал: "Меня зовут Эненуф. Я окрадываю людей разумом и извращаю их сердца, и делаю человека беззубым. Если кто напишет: "Аллазоол, преследуй Эненуфа", и обвяжет папирус вокруг себя, я сразу отступаю".
\vs Tsl 1:97 
Двадцать седьмой сказал: "Меня зовут Фет. Я делаю человека чахоточным и причиняю кровотечение. Если кто будет изгонять меня вином, ароматным и несмешанным, и скажет: "Я изгоняю тебя вином, я требую остановиться, Фет", потом даст это выпить терпящему, и я сразу отступаю".
\vs Tsl 1:98 
Двадцать восьмой сказал: "Меня зовут Харпакс, и я насылаю на людей безсонницу. Если кто напишет: "Кокфенедисм", и обвяжет это вокруг висков, я сразу удаляюсь".
\vs Tsl 1:99 
Двадцать девятый сказал: "Меня зовут Аностер. Я порождаю бешенство матки и боли в мочевом пузыре. Если кто истолчет в чистый елей три семечка лавра и намажет этим, говоря: "Я изгоняю тебя, Аностер. Остановись [именем] Мармарао", и я сразу отступаю".
\vs Tsl 1:100 
Тридцатый сказал: "Меня зовут Аллебориф. Если кто вкушающий рыбу проглотил кость, ему нужно взять [другую] кость из рыбы и прокашляться, и я сразу отступаю".
\vs Tsl 1:101 
Тридцать первый сказал: "Меня зовут Эфесимиреф, и [я] причиняю затяжную болезнь. Если ты бросишь соль, перетертую руками, в елей и намажешь этим терпящего, говоря: "Серафы, Херувы, помогите мне!", я сразу удаляюсь".
\vs Tsl 1:102 
Тридцать второй сказал: "Меня зовут Ихфион. Я парализую мышцы и повреждаю их. Если я услышу: "Адонаеф, помоги!", я сразу удаляюсь".
\vs Tsl 1:103 
Тридцать третий сказал: "Меня зовут Агхонион. Я ложусь среди пеленок и в пропасти. И если кто-либо напишет на фиговых листьях: "Ликург", убирая последовательно по одной букве, и запишет эти буквы в обратном порядке, я сразу удаляюсь".
\vs Tsl 1:104 
Тридцать четвертый сказал: "Меня зовут Аутофит. Я причиняю зависть и борьбу. Поэтому меня упраздняют Алеф и Тав, если [это] записано".
\vs Tsl 1:105 
Тридцать пятый сказал: "Меня зовут Птеноф. Я бросаю дурной глаз на всякого человека. Поэтому многострадальное око, если оно открыто, упраздняет меня".
\vs Tsl 1:106 
Тридцать шестой сказал: "Меня зовут Бианакиф. Я имею ненависть против тела. Я полагаю дома опустошенными, я причиняю плоти разложение, и еще всякое такое подобное. Если человек напишет на входной двери своего дома: "Мелто, Арду, Анааф", я убегаю из того места".
\vs Tsl 1:107 
И я, Соломон, когда я услышал это, прославил Бога неба и земли. И я приказал им, чтобы они принесли воду для храма Божия. И я еще помолился Адонаи ЯХВЕ, чтобы вызвать демонов, которые противятся роду человеческому, наружу и связать и привести к храму Божьему. Некоторых из этих демонов я осудил выполнять тяжелую работу на строительстве храма. Других я заключил в темницы. Иным я приказал бороться с огнем на [изготовлении] золота и серебра, сидя у котла и черпала, и приготовить места для других демонов, в которые они будут заключены.
\vs Tsl 1:108 
И я, Соломон, имел много покоя во всей земле и проводил свою жизнь в глубоком мире, почитаемый всеми людьми и всей поднебесной. И я построил целиком храм Адонаи ЯХВЕ. И моё царство процветало, и моё войско было со мною. И для остальных город Иерусалим имел покой, радость и наслаждение. И все цари земные приходили ко мне от концов земли, чтобы увидеть храм, который я создал для Адонаи ЯХВЕ. И, прослышав о мудрости, данной мне, они творили мне поклон в храме, принося золото и серебро и драгоценные камни, многочисленные и изысканные, бронзу и железо и свинец и кедровые бревна. Но гнилой древесины они не приносили мне для отделки храма Божия.
\vs Tsl 1:109 
И среди них также царица Юга, будучи очаровательной, пришла с большим интересом и низко поклонилась предо мною к земле. И слушая мою мудрость, она прославляла Бога Израилева, и она сотворила строгий суд над всей моей мудростью, от всей любви, в которой я наставлял её согласно мудрости, данной мне. И все сыны Израилевы прославляли Бога.
\vs Tsl 1:110 
И вот в те дни один из рабочих, пожилого возраста, пал ниц предо мною и сказал: "Царь Соломон, пожалей меня, ибо я стар". Тогда я попросил его встать и сказал: "Поведай мне, старик, всё, что ты желаешь". И он ответил: "Я прошу тебя, царь, я имею единородного сына, и он оскорбляет и бьет меня открыто, и вырывает волосы моей головы, и угрожает мне мучительной смертью. Поэтому я прошу тебя, отомсти за меня".
\vs Tsl 1:111 
И я, Соломон, выслушав это, почувствовал сожаление, как только посмотрел на его преклонный возраст; и я велел дитя привести ко мне. И когда он был приведен, я спросил, правда ли это. И юноша сказал: "Я не настолько безумен, чтобы бить своего отца рукою моею. Будь милостив ко мне, о царь. Ибо я, негодный, не смел совершить такого нечестия". А я, Соломон, выслушав это от юноши, увещевал старика отреагировать на дело и принять оправдание сына. Однако, он не желал этого, а сказал, что скорее позволит себе умереть. И так как старик не уступал, я собирался было объявить приговор юноше, как увидел смеющегося демона Орнию. Я весьма разгневался на смех демона в моем присутствии; и я приказал моим людям удалить другие стороны и выдвинуть Орнию перед моим судом. И когда он был приведен предо мною, я сказал ему: "Проклятый, почему ты смотришь на меня и смеёшься?" И демон ответил: "Прошу, царь, я смеялся не над тобой, а я смеялся над этим несчастным стариком и бедным юношей, его сыном. Ибо спустя три дня его сын внезапно умрет; и вот~--- старик стремится жестоко покончить с собою".
\vs Tsl 1:112 
Но я, Соломон, услышав это, сказал демону: "Правда ли то, что ты говоришь?" И он ответил: "Да, это правда, о царь". И я, выслушав это, велел им увести в сторону демона и [сказал], что они должны снова привести предо мною старика с его сыном. Я предложил им опять помириться друг с другом, и я снабдил их пищей. И потом я велел старику через три дня снова привести его сына ко мне сюда. "И,~--- сказал я,~--- я позабочусь о нем". И они приветствовали меня и ушли своей дорогой.
\vs Tsl 1:113 
И когда они ушли, я приказал Орнии предстать предо мною и сказал ему: "Поведай мне, откуда ты знаешь об этом?" И он ответил: "Мы, демоны, восходим к тверди небесной и летаем среди звезд. И мы слышим о наказаниях, которые идут свыше на души человеческие, и тотчас мы отправляемся [на землю], и либо силой воздействия, или огнем, или мечом, или каким-нибудь несчастным случаем мы незаметно [творим] наше разрушительное действие; и если человек не умирает из-за какого-то неожиданного бедствия или насилия, то мы, демоны, преобразуемся таким образом, чтобы являться людям и быть почитаемыми [от них] в нашей [мнимой] человеческой природе".
\vs Tsl 1:114 
Тогда я, услышав это, восхвалил Адонаи ЯХВЕ, и снова я спросил демона, говоря: "Поведай мне, как вы, будучи демонами, можете взойти на небеса и смешиваться среди звезд и святых ангелов?" И он ответил: "Как вещи совершаются на небесах, так и на земле [совершаются] образы всех из них. Ибо там есть начала, власти, мироправители, и мы, демоны, летаем везде в воздухе; и мы слышим голоса небожителей и обозреваем все силы. И как не имеющие основания, на которое можно сойти и отдохнуть, мы теряем силу и падаем словно листья с дерева. И люди, видя нас, воображают, что звезды падают с небес. Но на самом деле это не так, о царь; но мы падаем из-за нашей усталости и потому что мы не имеем возможности где-нибудь удержаться; и так мы падаем подобно молнии глубокой ночью и внезапно. И мы оставляем города в огне и поджигаем поля. Ибо звезды [в отличие от нас] имеют прочные основы на небесах подобно солнцу и луне".
\vs Tsl 1:115 
И я, Соломон, услышав это, приказал охранять демона в течение пяти дней. И спустя пять дней я вспомнил о старике и намеревался допросить его. Но он пришел ко мне в печали и с почерневшим лицом. И я сказал ему: "Поведай мне, старик, где сын твой? И что значит твоя одежда?" И он ответил: "Вот, я стал бездетным и сижу у могилы моего сына в отчаянии. Ибо вот уже два дня как он мертв". А я, Соломон, услышав это и поняв, что демон Орния поведал мне правду, прославил Бога Израилева.
\vs Tsl 1:116 
И царица Юга увидела всё это и изумилась, восхваляя Бога Израилева; и она созерцала храм ЯХВЕ будучи построенным. И она дала золотой сикль и сто мириад серебра и лучшей бронзы, и она вошла в храм. И [она увидела] алтарь для благовоний и медные опоры этого алтаря, и драгоценные камни светильников, сверкающие многоразличными цветами, и подсвечник каменный, и изумрудный, и гиацинтовый, и сапфировый; и она увидела сосуды из золота, и серебра, и бронзы, и дерева, и кожаную обивку, окрашенную в красный с мареной цвет. И она увидела основания столпов храма ЯХВЕ: все были из чистого золота [и к ним никого не подпускали] кроме демонов, которым я приказал работать. И вот был мир вокруг моего царства и по всей земле.
\vs Tsl 1:117 
И спустя некоторое время, которое я был в своем царстве, царь Аравии Адарий, прислал мне письмо, и в письме было написано следующее: "Царю Соломону~--- всякое приветствие! Вот, мы слышали,~--- и это слышат во всех концах земли,~--- о мудрости, которой ты был удостоен, и что ты~--- человек, который получил милость от ЯХВЕ. И [тебе] было даровано знание обо всех духах в воздухе и на земле и под землей. Ныне, ввиду того, что здесь на аравийской земле существует дух нижеследующего рода, рано на рассвете здесь начинает дуть некий ветер до третьего часа. И его порыв резок и ужасен, и это убивает и людей и животных. И нет духа, могущего на земле противостоять этому демону. Я прошу тебя, поскольку [этот] дух~--- ветер, придумай что-нибудь по мудрости, данной тебе ЯХВЕ, Богом твоим, и благоволи послать человека, способного его схватить. И вот, царь Соломон, я и мои люди, и вся моя земля будем служить тебе здесь до смерти. И вся Аравия будет в мире с тобой, если ты окажешь благодеяние к нам. Ибо мы просим тебя, не презри нашу скромную просьбу и не позволь довести до крайнего опустошения область, подчиненную твоей власти. Потому что мы~--- просители, и я, и мои люди, и вся моя земля. Прощай, мой господин. Всякого благополучия!"
\vs Tsl 1:118 
И я, Соломон, прочитал это послание; и я сложил его и отдал его моим людям и сказал им: "Спустя семь дней напомните мне об этом послании". И Иерусалим был отстроен, и храм завершен. И вот был камень, краеугольный камень, огромный, избранный, единственный, который я хотел поместить во главу угла для завершения храма. И все рабочие, и все демоны, помогавшие им, пришли на то же место, чтобы поднять камень и установить его на возглавие святого храма, и не были достаточно сильны, чтобы сдвинуть его и положить его в угол, определенный ему. Ибо этот камень был очень большой и необходимый для угла храма".
\vs Tsl 1:119 
И спустя семь дней, когда мне напомнили о послании Адария, царя Аравии, я призвал моего слугу и сказал ему: "Снаряди твоего верблюда и возьми для себя мех, и возьми также эту печать. И отправляйся в Аравию, в то место, где дует злой дух; и там возьми мех и кольцо с печатью перед отверстием меха, и [держи их] к порывам ветра. И когда мех надуется, ты поймешь, что демон [в нем]. Тогда быстро завяжи отверстие меха и надежно запечатай его кольцом с печатью, и аккуратно уложи его на верблюда и привези его ко мне сюда. И если по дороге он предложит тебе золото или серебро или сокровища за позволение ему идти, смотри, чтобы не быть тебе убежденным. Но сговорись, не давая клятвы, что ты выпустишь его. И потом, если он покажет места, где золото или серебро, отметь [эти] места и запечатай их этой печатью. И принеси демона ко мне. И теперь отправляйся и пусть будет тебе благо".
\vs Tsl 1:120 
Тогда юноша сделал как ему было велено. Он снарядил своего верблюда, положил на него мех и отправился в Аравию. И люди той страны не верили, что он способен поймать злого духа. И когда настал рассвет, слуга стал против порывов ветра и положил мех на землю и кольцо с печатью на отверстие меха. И демон подул сквозь перстень в отверстие меха и, войдя, стал выдувать мех. Но муж поспешно воспрепятствовал этому и туго стянул своею рукою отверстие фляги во имя ЯХВЕ, Бога Саваофа. И демон остался внутри меха. И после этого юноша оставался в той стране три дня, чтобы сделать испытание [надежно ли он заключил духа]. И дух больше не дул против того города. И все арабы узнали, что он надежно заключил духа.
\vs Tsl 1:121 
Тогда юноша закрепил мех на верблюде, и арабы отправили его дальше в путь с великой честью и дорогими дарами, восхваляя и превознося Бога Израилева. Итак юноша принес [мех] в мешке и положил посреди храма. И на следующий день я, царь Соломон, вошел в храм Божий и сидел в глубокой печали о краеугольном камне. И когда я входил в храм, мех поднялся и прошел вокруг около семи шагов, а потом упал на отверстие и преклонился предо мною. И я был поражен тем, что даже вместе с бутылью демон все же имел силу и мог идти; и я приказал, чтобы он встал. И мех поднялся и стоял на его ногах, весь раздуваясь. И я спросил его, говоря: "Поведай мне, кто ты?" И дух внутри сказал: "Я~--- демон, названный Эфиппой, из Аравии". И я сказал ему: "Это твоё имя?" И он ответил: "Да; где бы я ни находился, я воспламеняюсь и поджигаю и умерщвляю".
\vs Tsl 1:122 
И я сказал ему: "Какой ангел тебя упраздняет?" И он ответил: "Единодержавный Бог, Который имеет власть надо мною, как только услышу [Его Имя]. Он, Тот, Кто будет рожден девой и замучен иудеями на кресте, Кому служат ангелы и архангелы. Он упраздняет меня и ослабляет меня от моей великой силы, которая была дана мне моим отцом дьяволом". И я сказал ему: "Что ты можешь делать?" И он ответил: "Я могу передвигать горы, [заставлять] царей нарушать клятвы. Я изсушаю деревья и делаю их листья опадающими". И я спросил: "Можешь ли ты поднять этот камень и положить его во главу этого угла, который находится в прекрасном плане храма?" И он сказал: "Не только поднять его, о царь; но также, с помощью демона, который господствует над морем Суф, я перенесу по воздуху столп и установлю его там, где ты пожелаешь, в Иерусалиме".
\vs Tsl 1:123 
Сказав это, я надавил на него, и мех стал как будто испустившим воздух. И я положил его под камень, и [дух] переместился к верху [меха] и поднял его над верхом меха. И мех пошел, неся камень, и опустил его у самого входа в храм. И я, Соломон, видя, как камень был поднят вверх и помещен в основание [храма], сказал: "Истинно исполнилось священное писание, которое говорит: "Камень, который строители [признали] негодным при испытании, тот самый станет во главу угла". Ибо это~--- не моё обетование, а Бога, что демон будет достаточно сильным поднять столь большой камень и поместить его в том месте, где я хотел".
\vs Tsl 1:124 
И Эфиппа привел демона моря Суф со столпом. И они оба взяли столп и вознесли его вверх от земли. И я перехитрил этих двух духов, так чтобы они через мгновение не смогли поколебать всю землю. И затем я запечатал вокруг моим кольцом, с той и другой стороны, и сказал: "Стерегите!" И духи остались его опорой до сего дня в доказательство мудрости, которой я удостоен. И вот столп огромного размера был подвешен посреди воздуха, поддерживаемый духами. И таким образом духи оказались внизу, подобно воздуху удерживая его. И если пристально посмотреть, столп, поддерживаемый духами, немного наклонен; и так~--- до сего дня.
\vs Tsl 1:125 
И я, Соломон, спросил другого духа, который пришел со столпом из глубины моря Суф. И я сказал ему: "Кто ты и как тебя зовут? И каковы твои дела? Ибо я слышал много вещей о тебе". И демон ответил: "Меня, о царь Соломон, зовут Авезифивод. Я~--- семя архангела (?). Когда-то я возседал на первом небе, имя которому~--- Амелеуф; потом я [стал] жестоким духом и крылатым, и с единственным крылом, интригующим против каждого поднебесного духа. Я присутствовал, когда Моисей вошел перед Фараоном, царем Мицрейским, и я ожесточил его сердце. Я~--- тот, кого Ианний и Иамврий призывали, возвращаясь с Моисеем в Мицру. Я~--- тот, кто боролся с Моисеем чудесами и знамениями".
\vs Tsl 1:126 
И поэтому я сказал ему: "Как ты оказался в море Суф?" И он ответил: "При исходе сынов Израилевых я ожесточил сердце Фараона. И я побудил его сердце и тех его слуг. И я заставил их преследовать детей Израилевых после. И Фараон последовал со [мною] и всеми мицрейцами. Тогда я присутствовал там, и мы следовали вместе. И мы все подошли к морю Суф. И они подошли к переходу, когда дети Израилевы пересекали [море]: вода вернулась и накрыла всё множество мицрейцев и всё их могущество. И я остался в море, соблюдаемый под этим столпом. Но когда Эфиппа пришел, посланный тобою, заключенный в мех, он привел меня к тебе".
\vs Tsl 1:127 
Поэтому я, Соломон, услышав всё это, прославил Бога и заклинал демонов не неповиноваться мне, но остаться поддерживать столп. И они оба поклялись, сказав: "Жив ЯХВЕ, Бог твой! Мы не отпустим этот столп до кончины мира. Но в некий день этот камень упадет, тогда будет конец мира". \ldots\ 
\vs Tsl 1:128 
И я, Соломон, прославил Бога и украсил храм ЯХВЕ со всем великолепием. И в душе я был счастлив в моем царстве, и был мир в мои дни. И я брал себе жен из всякой земли, которые были безчисленны. И я пошел против иевусеев, и там я увидел иевусеянку, дщерь человеческую: и очень сильно влюбился в неё, и пожелал взять её в жены вместе с другими моими женами. И я сказал их жрецам: "Дайте мне Сунамитянку в жены". Но жрецы Молоха сказали мне: "Если ты любишь эту девушку, приди и поклонись нашим богам, великому богу Рафану\fnote{Рафан}{$=$ капуста.} и богу, называемому Молохом". От этого я испугался славы Божией, и не пошел поклоняться. И я сказал им: "Я не буду поклоняться чужому богу. Что это за предложение, которым вы заставляете меня сделать так много?" Но они сказали: "\ldots\ нашими отцами".
\vs Tsl 1:129 
И когда я ответил, что не буду ни при каких условиях поклоняться чужим богам, они велели девушке не спать со мною, пока я не уступлю и не принесу жертвы богам. Тогда я стал волноваться, но коварная Эр принесла и положила для меня пять её саранчей, сказав: "Возьми этих саранчей и раздави их вместе во имя бога Молоха; и тогда я буду спать с тобою". И тотчас я сделал это. И сразу же Дух Божий отступил от меня, и я стал слабым, также как и глупым, в словах моих. И после этого я был обязан построить ей храм идолам Ваала, и Раф[ан]а, и Молоха и другим идолам.
\vs Tsl 1:130 
Тогда я, сей несчастный человек, последовал за её советом, и слава Божия совершенно отступила от меня; и мой дух померк, и я стал забавой для идолов и демонов. Поэтому я написал это Завещание, чтобы вы, которые получите власть эту, смогли сожалеть и внимать последним вещам, а не первым, для того, чтобы вы смогли обрести благодать во веки веков. Аминь.

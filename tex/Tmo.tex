\bibbookdescr{Tmo}{
  inline={Завещание Моисея},
  toc={Завещание Моисея},
  bookmark={Завещание Моисея},
  header={Завещание Моисея},
  abbr={Зав~Мо}
}
\vs Tmo 1:1
Завещание Моисея, данное им в сто двадцатый год жизни его, который есть четырехсотый по отправлении из Ханаана, когда вышел народ с Моисеем и дошел до Бен-Амми за Иорданом по пророчеству Моисееву.
\vs Tmo 1:2
Призвал Моисей к себе Иесуа, сына Нуна, человека, угодного Яхве, дабы стал он преемником народа и ковчега Завета со всеми святынями его и дабы ввел народ в землю, данную коленам его, дать им ее по завету и по клятве, которую произнес он в скинии, что даст ее через Иесуа;
\vs Tmo 1:3
И сказал Моисей к Иесуа такое слово: "Обещай, что все сотворишь, что поручено тебе, сотворишь со старанием, в точности и без ропота, ибо так говорит Владыка мира.
\vs Tmo 1:4
Создал Он мир ради народа Своего и не сделал начала творения ясно видимым от начала мира, дабы обличились тем народы и низкими речами своими обличили себя.
\vs Tmo 1:5
Так Он измыслил и изобрел меня, от начала мира готового стать судьею завета Его.
\vs Tmo 1:6
И ныне открою тебе, что совершилось время лет жизни моей и отхожу я в успение отцов моих.
\vs Tmo 1:7
Предо всем народом прими писание сие, дабы не забывал ты хранить книги, кои передам тебе,
\vs Tmo 1:8
Ты же их расположишь в порядке и запечатаешь и положишь в сосудах глиняных в месте, созданном от начала мира, дабы призывалось имя Его вплоть до дня покаяния с почитанием, коим почтил их Яхве на исходе дней.

\vs Tmo 2:1
Войдут они с тобою в землю, которую назначил и обещал Он дать отцам их.
\vs Tmo 2:2
В ней благословишь ты и дашь каждому и установишь им жребий мой и утвердишь им царство и управление на местах определишь им по тому, как угодно будет Яхве, Богу их, по суду и справедливости.
\vs Tmo 2:3
После того как войдут в землю свою, пройдет пять лет, и будет власть вождей и тираннов восемнадцать лет, и на девятнадцать лет отделятся десять колен, ибо отойдут два колена и перенесут ковчег Завета.
\vs Tmo 2:4
Тогда Бог небесный явит ковчег Свой и башню святилища Своего. И утвердятся два колена святости, десять же колен установят себе по законам своим царства.
\vs Tmo 2:5
И будут приносить жертвы двадцать лет: за семь лет соорудят стены, и ограждать их буду девять лет, и нападать будут на завет Яхве четыре года, и, наконец, осквернят договор, который сотворил с ними Яхве.
\vs Tmo 2:6
И принесут детей своих в жертву чужеземным богам, и установят в скинии идолов, служа им, и в доме Яхве будут вершить преступления, и многих идолов всех животных сделают.

\vs Tmo 3:1
В те времена придет к ним с востока царь, и покроет конница землю их, и сожжет он огнем поселения их со святым храмом Яхве, и все святые сосуды он истребит, и весь народ изгонит, и уведет их в землю отчизны своей, и два колена уведет с собой.
\vs Tmo 3:2
Тогда воззовут два колена к десяти коленам, и лягут словно львица, покрытые пылью в полях, алчущие и жаждущие с детьми нашими, и возопиют: "Праведен и свят Яхве! Отчего вы грешили, а мы так же уведены с вами?"
\vs Tmo 3:3
Тогда восплачут десять колен, слыша слова упрека от двух колен и скажут: "Что сделали мы вам, братья? Не во весь ли дом Израилев вошло горе это?"
\vs Tmo 3:4
И все колена восплачут, вопия к небу и говоря: "Бог Авраама, Бог Исаака, Бог Иакова, воспомни завет Твой, который заключил Ты с ними, и клятву, которою клялся Ты им, что никогда не упразднится семя их от земли, которую Ты дал им".
\vs Tmo 3:5
И в тот день воспомнят имя мое, говоря колено к колену, и всякий человек к ближнему своему: "Не то ли это, в чем удостоверял нас Моисей в пророчествах своих, он, претерпевший многое в Египте и в море Суф, и в пустыне в продолжение сорока лет, свидетельствуя и призывая в свидетели небо и землю, да не преступим мы заповедей Его, в коих был он нам судьею;
\vs Tmo 3:6
И так случилось с нами по словам Его, и по уверению Его, как свидетельствовал он нам в те времена, и вышло, что ведут нас, плененных, в Восточную землю, где и будем мы рабами около семидесяти семи лет".

\vs Tmo 4:1
Тогда войдет один, стоящий над ними, и прострет руки, и преклонит колени свои, и станет молиться за них, говоря:
\vs Tmo 4:2
"Яхве, Царь всех, на высоком престоле властвующий над миром, возжелавший, дабы сей народ был народом Твоим избранным, тогда хотел Ты называться их Богом по завету, который заключил Ты с отцами их.
\vs Tmo 4:3
И пошли они, плененные, в землю чуждую с женами и детьми своими, и пребывают у врат иноплеменных и там. Где же величие великое? Призри и смилуйся над ними, Господь небесный!"
\vs Tmo 4:4
Тогда воспомнит о них Бог по завету, что сотворил Он с отцами их, и явит Он милосердие Свое, и вложит в те времена в душу царя, дабы смиловался над ними, и отпустит их царь в землю и область их.
\vs Tmo 4:5
Тогда поднимутся некоторые части колен и пойдут в свое место установленное и обновят укрепления его.
\vs Tmo 4:6
Два же колена пребудут в вере своей, печальные и плачущие, ибо не смогут принести жертв Яхве, Богу отцов своих, десять же колен увеличатся и умножатся среди племен во времена пленения их.

\vs Tmo 5:1
Когда же приблизятся времена обличения, мщение наступит от царей, соучастников преступлений, кои разделятся воистину.
\vs Tmo 5:2
Потому и было сказано: "Уклонятся от праведности, и перейдут к неправедности, и осквернят нечестиями дом служения своего, и осквернятся служением чужим богам.
\vs Tmo 5:3
И не последуют истине Божией, но осквернят алтарь дарами, кои воздадут Яхве не жрецы, но рожденные рабами от рабов.
\vs Tmo 5:4
Ибо те, которые суть ученые учителя их, будут взирать в те времена на лица страстей, и, принимая дары, продадут праведность в наказаниях.
\vs Tmo 5:5
И настолько наполнится население и предел обитания их преступлениями и обидами Бога, что те, кто творил беззаконие пред лицем Яхве, судьями станут и судить будут, кто как пожелает".

\vs Tmo 6:1
Тогда возстанут у них цари властные. Назовутся они священниками великого Бога и удалят творящих нечестие от святая святых.
\vs Tmo 6:2
И придет вслед за ними царь дерзновенный, который не будет из рода священнического. Сей человек безрассудный и злой, и будет судить он их, как они того достойны.
\vs Tmo 6:3
Истребит он вождей их мечом, и в неизвестные места порознь положит тела их, дабы не ведал никто, где тела их. Погубит он старших возрастом и юношей не пощадит.
\vs Tmo 6:4
Тогда страх пред ним будет великий в земле их, и станет он вершить суд над ними, как вершили его Египтяне, тридцать четыре года, и покарает их.
\vs Tmo 6:5
И породит он сыновей, кои будут царствовать не столь долго, и придут в землю их когорты мощного царя Западного, и одолеет он их и уведет в плен и часть храма их огнем сожжет, некоторых же распнет вокруг поселения их.

\vs Tmo 7:1
После того совершатся времена и будут править ими люди погибельные и нечестивые, называющие себя праведными,
\vs Tmo 7:2
И возбудят они гнев душ своих, будучи людьми коварными, себе угождающими, лживыми во всех делах своих и во всякий час дня, любящими пиры, чревоугодниками, пожиратели имущества бедных,
\vs Tmo 7:3
Скажут, что творили это по милосердию и истребляя стяжателей.
\vs Tmo 7:4
Будут обманывать, скрываясь, дабы не уличили их, нечестивцев, в преступлении, исполненные неправедности, от восхода до заката говоря: "Будут у нас роскошные ложа, и станем есть и пить на них. И помыслили мы, что будем, словно князья".
\vs Tmo 7:5
И руки их, и умы творят нечистое, и уста их полны слов надутых. И скажут они: "Не касайся, да не осквернишь места моего"

\vs Tmo 8:1
Придет к ним мщение и гнев, какого не бывало у них от века до того времени.
\vs Tmo 8:2
Тогда возставит им Яхве царя из царей земли и мощь из мощи великой, что распнет на кресте исповедующих обрезание.
\vs Tmo 8:3
И предаст он пыткам тех, кто откажется, и повелит в оковах отвести в темницу. А жен их отдадут богам языческим; сыновья же их, мальчики, обрезанные врачами, принуждены будут принять необрезание.
\vs Tmo 8:4
Будут карать их пытками, огнем и железом, заставят их носить идолов своих оскверненных, как и те, кто им служит, и заставят их мучители войти в тайное место свое, и понудят их стрекалами произнести слова хульные, а потом и законы похулить, положенные на алтаре их.

\vs Tmo 9:1
Тогда возстанет муж из колена Левиева, имя коему будет Таксо. Имея семерых сыновей, обратится к ним с просьбою:
\vs Tmo 9:2
"Смотрите, сыны, вот, свершилось второе отмщение народу жестокое и нечестивое, и пленение безжалостное, и превосходят они бывшие доселе. Какое племя, какая земля, какой народ, нечестивых, творивших преступление в доме своем, столько бед претерпел, сколько нас обступило?
\vs Tmo 9:3
Ныне, послушайте меня, дети, ибо видите и знаете, что никогда не испытывали Бога ни родители наши, ни праотцы, преступая заповеди Его.
\vs Tmo 9:4
Ибо знаете: в этом сила наша. Сделаем так: будем поститься три дня, а на четвертый день войдем в пещеру, которая на поле, и лучше умрем, чем преступим заповеди Бога богов, Господа отцов наших.
\vs Tmo 9:5
Если так сотворим и умрем, кровь наша отомщена будет пред Яхве".

\vs Tmo 10:1
И тогда явится царствие Его во всяком творении Его. И тогда диавол обретет конец, и скорбь с ним отойдет.
\vs Tmo 10:2
Тогда наполнится рука ангела, утвержденного на небесах, и тотчас избавит он их от врагов их.
\vs Tmo 10:3
Ибо поднимется Небесный с престола царствия Своего и выйдет из святого жилища Своего с негодованием и гневом на сынов Своих.
\vs Tmo 10:4
И задрожит земля и до пределов своих сотрясется, и высокие горы понизятся и сотрясутся, и долины падут, солнце не даст света, и во мрак обратятся рога луны и сокрушатся, и все обратится в кровь, и круг звезд смешается, и море отступит до бездны, и источники вод изсякнут, и реки высохнут.
\vs Tmo 10:5
Ибо возстанет Великий Бог, Единый и Вечный, и явится всем и отомстит народам и уничтожит всех идолов их.
\vs Tmo 10:6
Тогда блажен будешь ты, Израиль, и поднимешься ты на головы и на крылья орлиные, и наполнятся они воздухом, и возвысит тебя Бог и утвердит тебя в небе звездном в месте пребывания звезд.
\vs Tmo 10:7
И воззришь с высоты, и увидишь врагов своих на земле, и узнаешь их, и возрадуешься, и возблагодаришь, и хвалу вознесешь Создателю твоему.
\vs Tmo 10:8
Ты же, Иесуа Нун, сбереги слова сии и книгу сию. Ибо от того дня, когда приму я смерть, пройдет до пришествия Его двести пятьдесят времен. Столько времени пройдет, пока не прейдут времена.
\vs Tmo 10:9
Я же отхожу к успению отцов моих. Итак, ты, Иесуа Нун, мужайся, тебя избрал Бог быть мне преемником в завете сем.

\vs Tmo 11:1
Когда услышал Иесуа слова Моисея, записанные в писании его, и все, что предрек он, разодрал одежды свои, пал к ногам его, и утешал его Моисей и плакал с ним.
\vs Tmo 11:2
И отвечал ему Иесуа и сказал: Утешишь ты меня, господин мой Моисей, и как утешить меня в том, что сказано голосом горьким, что вышел из уст твоих, и полон слез и рыданий, ибо уходишь ты от народа Израилева.
\vs Tmo 11:3
Какое место примет тебя, каков будет памятник могильный, кто осмелится перенести тело твое из одного места в другое?
\vs Tmo 11:4
Ибо у всех, кто умирает в свое время, есть могилы свои на земле, твоя же могила от восхода солнца до заката, и от юга до севера весь мир есть могила твоя, господин мой.
\vs Tmo 11:5
Уходишь ты, и кто будет питать народ сей, и кто сжалится над ними, и кто вождем будет им в пути, и кто молиться станет за них? Не смогу я и одного дня вести их в земле предков.
\vs Tmo 11:6
Как же буду я народу сему словно отец для единого сына или мать для дочери-девицы, что готовит ее для славного мужа, оберегает, боясь, тело ее от солнца и старается, дабы не поранила та ног своих, бегая по земле?
\vs Tmo 11:7
Как дам им пищу по желанию их и насыщу их? Ведь их шестьсот тысяч возросло их число молитвами твоими, господин мой Моисей.
\vs Tmo 11:8
Какая же у меня мудрость и какое разумение в доме Божием словами судить и давать ответы?
\vs Tmo 11:9
Но и цари Аморрейские, когда услышат об этом, помыслят, что одолеют нас,
\vs Tmo 11:10
Ибо нет больше с нами Духа Святого, достойного Яхве, слову многоликого и непонятного Яхве верного во всем, божественного пророка всего мира, ведь умер он и нет более в веке сем учителя, и скажут они тогда:
\vs Tmo 11:11
"Пойдем на них, если нечестивое совершили они единожды Господу Своему, нет у них заступника, который бы вознес молитвы Яхве, каков был Моисей, великий ангел.
\vs Tmo 11:12
Он по целым часам стоял днем и ночью коленами своими на земле, молясь и взирая на мир и всех людей с милосердием и праведностью, памятуя о завете предков своих и клятвами умилостивляя Яхве".
\vs Tmo 11:13
Итак, скажут они: "Не с ними Бог. Пойдем же и сотрем их с лица земли". Вот как будет с народом сим, господин мой Моисей".

\vs Tmo 12:1
И, закончив слова свои, вновь пал Иесуа к стопам Моисеевым. И взял Моисей его за руку и посадил пред собою на седалище.
\vs Tmo 12:2
И отвечал и сказал ему Моисей: Иесуа, не бойся за себя, но будь уверен и внемли словам моим.
\vs Tmo 12:3
Все народы, какие есть в мире, создал Бог, и предусмотрел Он о них и о нас от начала творения всего мира.
\vs Tmo 12:4
И до скончания века нет ничего, чего бы не усмотрел Он до самой малой вещи, но все Он предусмотрел и устроил.
\vs Tmo 12:5
Все в этом мире предусмотрел Он, и вот \ldots
\vs Tmo 12:6
Меня поставил Он молиться за них и за грехи их и заступником быть им не по добродетели моей и не по немощи, но в меру милосердия Его и терпения Его.
\vs Tmo 12:7
Говорю же тебе, Иесуа: не по благочестию народа сего истребишь ты язычников; все тверди небесные Богом созданы и одобрены, и лишь под Его десницею они.
\vs Tmo 12:8
Итак, творящие и совершающие заповеди Божии возрастают и по доброму пути продвигаются, а согрешающим и пренебрегающим, нет им добрых заповедей в том, что предсказано.
\vs Tmo 12:9
И будут они покараны язычниками и преданы многим пыткам. Но не может быть того, чтобы совершенно уничтожил их Он и оставил.
\vs Tmo 12:10
Ибо выйдет Бог, провидящий все вовеки, и тверд Завет Его и клятва в том, что \ldots

\bibbookdescr{Eze}{
  inline={\LARGE Книга\\\Huge Пророка Иезекииля},
  toc={Иезекииль},
  bookmark={Иезекииль},
  header={Иезекииль},
  %headerleft={},
  %headerright={},
  abbr={Иез}
}
\vs Eze 1:1 И было в тридцатый год, в четвертый \bibemph{месяц}, в пятый \bibemph{день} месяца, когда я находился среди переселенцев при реке Ховаре, отверзлись небеса, и я видел видения Божии.
\vs Eze 1:2 В пятый \bibemph{день} месяца (это был пятый год от пленения царя Иоакима),
\vs Eze 1:3 было слово Господне к Иезекиилю, сыну Вузия, священнику, в земле Халдейской, при реке Ховаре; и была на нем там рука Господня.
\rsbpar\vs Eze 1:4 И я видел, и вот, бурный ветер шел от севера, великое облако и клубящийся огонь, и сияние вокруг него,
\vs Eze 1:5 а из средины его как бы свет пламени из средины огня; и из средины его видно было подобие четырех животных,~--- и таков был вид их: облик их был, как у человека;
\vs Eze 1:6 и у каждого четыре лица, и у каждого из них четыре крыла;
\vs Eze 1:7 а ноги их~--- ноги прямые, и ступни ног их~--- как ступня ноги у тельца, и сверкали, как блестящая медь, [и крылья их легкие].
\vs Eze 1:8 И руки человеческие были под крыльями их, на четырех сторонах их;
\vs Eze 1:9 и лица у них и крылья у них~--- у всех четырех; крылья их соприкасались одно к другому; во время шествия своего они не оборачивались, а шли каждое по направлению лица своего.
\vs Eze 1:10 Подобие лиц их~--- лице человека и лице льва с правой стороны у всех их четырех; а с левой стороны лице тельца у всех четырех и лице орла у всех четырех.
\vs Eze 1:11 И лица их и крылья их сверху были разделены, но у каждого два крыла соприкасались одно к другому, а два покрывали тела их.
\vs Eze 1:12 И шли они, каждое в ту сторону, которая пред лицем его; куда дух хотел идти, туда и шли; во время шествия своего не оборачивались.
\vs Eze 1:13 И вид этих животных был как вид горящих углей, как вид лампад; \bibemph{огонь} ходил между животными, и сияние от огня и молния исходила из огня.
\vs Eze 1:14 И животные быстро двигались туда и сюда, как сверкает молния.
\vs Eze 1:15 И смотрел я на животных, и вот, на земле подле этих животных по одному колесу перед четырьмя лицами их.
\vs Eze 1:16 Вид колес и устроение их~--- как вид топаза, и подобие у всех четырех одно; и по виду их и по устроению их казалось, будто колесо находилось в колесе.
\vs Eze 1:17 Когда они шли, шли на четыре свои стороны; во время шествия не оборачивались.
\vs Eze 1:18 А ободья их~--- высоки и страшны были они; ободья их у всех четырех вокруг полны были глаз.
\vs Eze 1:19 И когда шли животные, шли и колеса подле \bibemph{них}; а когда животные поднимались от земли, тогда поднимались и колеса.
\vs Eze 1:20 Куда дух хотел идти, туда шли и они; куда бы ни пошел дух, и колеса поднимались наравне с ними, ибо дух животных \bibemph{был} в колесах.
\vs Eze 1:21 Когда шли те, шли и они; и когда те стояли, стояли и они; и когда те поднимались от земли, тогда наравне с ними поднимались и колеса, ибо дух животных \bibemph{был} в колесах.
\vs Eze 1:22 Над головами животных было подобие свода, как вид изумительного кристалла, простертого сверху над головами их.
\vs Eze 1:23 А под сводом простирались крылья их прямо одно к другому, и у каждого были два крыла, которые покрывали их, у каждого два крыла покрывали тела их.
\vs Eze 1:24 И когда они шли, я слышал шум крыльев их, как бы шум многих вод, как бы глас Всемогущего, сильный шум, как бы шум в воинском стане; \bibemph{а} когда они останавливались, опускали крылья свои.
\vs Eze 1:25 И голос был со свода, который над головами их; когда они останавливались, тогда опускали крылья свои.
\vs Eze 1:26 А над сводом, который над головами их, \bibemph{было} подобие престола по виду как бы из камня сапфира; а над подобием престола было как бы подобие человека вверху на нем.
\vs Eze 1:27 И видел я как бы пылающий металл, как бы вид огня внутри него вокруг; от вида чресл его и выше и от вида чресл его и ниже я видел как бы некий огонь, и сияние \bibemph{было} вокруг него.
\vs Eze 1:28 В каком виде бывает радуга на облаках во время дождя, такой вид имело это сияние кругом.
\vs Eze 2:1 Такое было видение подобия славы Господней. Увидев это, я пал на лице свое, и слышал глас Глаголющего, и Он сказал мне: сын человеческий! стань на ноги твои, и Я буду говорить с тобою.
\vs Eze 2:2 И когда Он говорил мне, вошел в меня дух и поставил меня на ноги мои, и я слышал Говорящего мне.
\vs Eze 2:3 И Он сказал мне: сын человеческий! Я посылаю тебя к сынам Израилевым, к людям непокорным, которые возмутились против Меня; они и отцы их изменники предо Мною до сего самого дня.
\vs Eze 2:4 И эти сыны с огрубелым лицем и с жестоким сердцем; к ним Я посылаю тебя, и ты скажешь им: <<так говорит Господь Бог!>>
\vs Eze 2:5 Будут ли они слушать, или не будут, ибо они мятежный дом; но пусть знают, что был пророк среди них.
\vs Eze 2:6 А ты, сын человеческий, не бойся их и не бойся речей их, если они волчцами и тернами будут для тебя, и ты будешь жить у скорпионов; не бойся речей их и не страшись лица их, ибо они мятежный дом;
\vs Eze 2:7 и говори им слова Мои, будут ли они слушать, или не будут, ибо они упрямы.
\vs Eze 2:8 Ты же, сын человеческий, слушай, что Я буду говорить тебе; не будь упрям, как этот мятежный дом; открой уста твои и съешь, что Я дам тебе.
\vs Eze 2:9 И увидел я, и вот, рука простерта ко мне, и вот, в ней книжный свиток.
\vs Eze 2:10 И Он развернул его передо мною, и вот, свиток исписан был внутри и снаружи, и написано на нем: <<плач, и стон, и горе>>.
\vs Eze 3:1 И сказал мне: сын человеческий! съешь, что перед тобою, съешь этот свиток, и иди, говори дому Израилеву.
\vs Eze 3:2 Тогда я открыл уста мои, и Он дал мне съесть этот свиток;
\vs Eze 3:3 и сказал мне: сын человеческий! напитай чрево твое и наполни внутренность твою этим свитком, который Я даю тебе; и я съел, и было в устах моих сладко, как мед.
\vs Eze 3:4 И Он сказал мне: сын человеческий! встань и иди к дому Израилеву, и говори им Моими словами;
\vs Eze 3:5 ибо не к народу с речью невнятною и с непонятным языком ты посылаешься, но к дому Израилеву,
\vs Eze 3:6 не к народам многим с невнятною речью и с непонятным языком, которых слов ты не разумел бы; да если бы Я послал тебя и к ним, то они послушались бы тебя;
\vs Eze 3:7 а дом Израилев не захочет слушать тебя; ибо они не хотят слушать Меня, потому что весь дом Израилев с крепким лбом и жестоким сердцем.
\vs Eze 3:8 Вот, Я сделал и твое лице крепким против лиц их, и твое чело крепким против их лба.
\vs Eze 3:9 Как алмаз, который крепче камня, сделал Я чело твое; не бойся их и не страшись перед лицем их, ибо они мятежный дом.
\vs Eze 3:10 И сказал мне: сын человеческий! все слова Мои, которые буду говорить тебе, прими сердцем твоим и выслушай ушами твоими;
\vs Eze 3:11 встань и пойди к переселенным, к сынам народа твоего, и говори к ним, и скажи им: <<так говорит Господь Бог!>> будут ли они слушать, или не будут.
\vs Eze 3:12 И поднял меня дух; и я слышал позади себя великий громовой голос: <<благословенна слава Господа от места своего!>>
\vs Eze 3:13 и также шум крыльев животных, соприкасающихся одно к другому, и стук колес подле них, и звук сильного грома.
\vs Eze 3:14 И дух поднял меня, и взял меня. И шел я в огорчении, с встревоженным духом; и рука Господня была крепко на мне.
\vs Eze 3:15 И пришел я к переселенным в Тел-Авив, живущим при реке Ховаре, и остановился там, где они жили, и провел среди них семь дней в изумлении.
\rsbpar\vs Eze 3:16 По прошествии же семи дней было ко мне слово Господне:
\vs Eze 3:17 сын человеческий! Я поставил тебя стражем дому Израилеву, и ты будешь слушать слово из уст Моих, и будешь вразумлять их от Меня.
\vs Eze 3:18 Когда Я скажу беззаконнику: <<смертью умрешь!>>, а ты не будешь вразумлять его и говорить, чтобы остеречь беззаконника от беззаконного пути его, чтобы он жив был, то беззаконник тот умрет в беззаконии своем, и Я взыщу кровь его от рук твоих.
\vs Eze 3:19 Но если ты вразумлял беззаконника, а он не обратился от беззакония своего и от беззаконного пути своего, то он умрет в беззаконии своем, а ты спас душу твою.
\vs Eze 3:20 И если праведник отступит от правды своей и поступит беззаконно, когда Я положу пред ним преткновение, и он умрет, то, если ты не вразумлял его, он умрет за грех свой, и не припомнятся ему праведные дела его, какие делал он; и Я взыщу кровь его от рук твоих.
\vs Eze 3:21 Если же ты будешь вразумлять праведника, чтобы праведник не согрешил, и он не согрешит, то и он жив будет, потому что был вразумлен, и ты спас душу твою.
\rsbpar\vs Eze 3:22 И была на мне там рука Господа, и Он сказал мне: встань и выйди в поле, и Я буду говорить там с тобою.
\vs Eze 3:23 И встал я, и вышел в поле; и вот, там стояла слава Господня, как слава, которую видел я при реке Ховаре; и пал я на лице свое.
\vs Eze 3:24 И вошел в меня дух, и поставил меня на ноги мои, и Он говорил со мною, и сказал мне: иди и запрись в доме твоем.
\vs Eze 3:25 И ты, сын человеческий,~--- вот, возложат на тебя узы, и свяжут тебя ими, и не будешь ходить среди них.
\vs Eze 3:26 И язык твой Я прилеплю к гортани твоей, и ты онемеешь, и не будешь обличителем их, ибо они мятежный дом.
\vs Eze 3:27 А когда Я буду говорить с тобою, тогда открою уста твои, и ты будешь говорить им: <<так говорит Господь Бог!>> кто хочет слушать, слушай; а кто не хочет слушать, не слушай: ибо они мятежный дом.
\vs Eze 4:1 И ты, сын человеческий, возьми себе кирпич и положи его перед собою, и начертай на нем город Иерусалим;
\vs Eze 4:2 и устрой осаду против него, и сделай укрепление против него, и насыпь вал вокруг него, и расположи стан против него, и расставь кругом против него стенобитные машины;
\vs Eze 4:3 и возьми себе железную доску, и поставь ее \bibemph{как бы} железную стену между тобою и городом, и обрати на него лице твое, и он будет в осаде, и ты осаждай его. Это будет знамением дому Израилеву.
\vs Eze 4:4 Ты же ложись на левый бок твой и положи на него беззаконие дома Израилева: по числу дней, в которые будешь лежать на нем, ты будешь нести беззаконие их.
\vs Eze 4:5 И Я определил тебе годы беззакония их числом дней: триста девяносто дней ты будешь нести беззаконие дома Израилева.
\vs Eze 4:6 И когда исполнишь это, то вторично ложись уже на правый бок, и сорок дней неси на себе беззаконие дома Иудина, день за год, день за год Я определил тебе.
\vs Eze 4:7 И обрати лице твое и обнаженную правую руку твою на осаду Иерусалима, и пророчествуй против него.
\vs Eze 4:8 Вот, Я возложил на тебя узы, и ты не повернешься с одного бока на другой, доколе не исполнишь дней осады твоей.
\vs Eze 4:9 Возьми себе пшеницы и ячменя, и бобов, и чечевицы, и пшена, и полбы, и всыпь их в один сосуд, и сделай себе из них хлебы, по числу дней, в которые ты будешь лежать на боку твоем; триста девяносто дней ты будешь есть их.
\vs Eze 4:10 И пищу твою, которою будешь питаться, ешь весом по двадцати сиклей в день; от времени до времени ешь это.
\vs Eze 4:11 И воду пей мерою, по шестой части гина пей; от времени до времени пей так.
\vs Eze 4:12 И ешь, как ячменные лепешки, и пеки их при глазах их на человеческом кале.
\vs Eze 4:13 И сказал Господь: так сыны Израилевы будут есть нечистый хлеб свой среди тех народов, к которым Я изгоню их.
\vs Eze 4:14 Тогда сказал я: о, Господи Боже! душа моя никогда не осквернялась, и мертвечины и растерзанного зверем я не ел от юности моей доныне; и никакое нечистое мясо не входило в уста мои.
\vs Eze 4:15 И сказал Он мне: вот, Я дозволяю тебе, вместо человеческого кала, коровий помет, и на нем приготовляй хлеб твой.
\vs Eze 4:16 И сказал мне: сын человеческий! вот, Я сокрушу в Иерусалиме опору хлебную, и будут есть хлеб весом и в печали, и воду будут пить мерою и в унынии,
\vs Eze 4:17 потому что у них будет недостаток в хлебе и воде; и они с ужасом будут смотреть друг на друга, и исчахнут в беззаконии своем.
\vs Eze 5:1 А ты, сын человеческий, возьми себе острый нож, бритву брадобреев возьми себе, и води ею по голове твоей и по бороде твоей, и возьми себе весы, и раздели волосы на части.
\vs Eze 5:2 Третью часть сожги огнем посреди города, когда исполнятся дни осады; третью часть возьми и изруби ножом в окрестностях его; и третью часть развей по ветру; а Я обнажу меч вслед за ними.
\vs Eze 5:3 И возьми из этого небольшое число, и завяжи их у себя в полы.
\vs Eze 5:4 Но и из этого еще возьми, и брось в огонь, и сожги это в огне. Оттуда выйдет огонь на весь дом Израилев.
\rsbpar\vs Eze 5:5 Так говорит Господь Бог: это Иерусалим! Я поставил его среди народов, и вокруг него~--- земли.
\vs Eze 5:6 А он поступил против постановлений Моих нечестивее язычников, и против уставов Моих~--- хуже, нежели земли вокруг него; ибо они отвергли постановления Мои и по уставам Моим не поступают.
\vs Eze 5:7 Посему так говорит Господь Бог: за то, что вы умножили беззакония ваши более, нежели язычники, которые вокруг вас, по уставам Моим не поступаете и постановлений Моих не исполняете, и даже не поступаете и по постановлениям язычников, которые вокруг вас,~---
\vs Eze 5:8 посему так говорит Господь Бог: вот и Я против тебя, Я Сам, и произведу среди тебя суд перед глазами язычников.
\vs Eze 5:9 И сделаю над тобою то, чего Я никогда не делал и чему подобного впредь не буду делать, за все твои мерзости.
\vs Eze 5:10 За то отцы будут есть сыновей среди тебя, и сыновья будут есть отцов своих; и произведу над тобою суд, и весь остаток твой развею по всем ветрам.
\vs Eze 5:11 Посему,~--- живу Я, говорит Господь Бог,~--- за то, что ты осквернил святилище Мое всеми мерзостями твоими и всеми гнусностями твоими, Я умалю тебя, и не пожалеет око Мое, и Я не помилую тебя.
\vs Eze 5:12 Третья часть у тебя умрет от язвы и погибнет от голода среди тебя; третья часть падет от меча в окрестностях твоих; а третью часть развею по всем ветрам, и обнажу меч вслед за ними.
\vs Eze 5:13 И совершится гнев Мой, и утолю ярость Мою над ними, и удовлетворюсь; и узнают, что Я, Господь, говорил в ревности Моей, когда совершится над ними ярость Моя.
\vs Eze 5:14 И сделаю тебя пустынею и поруганием среди народов, которые вокруг тебя, перед глазами всякого мимоходящего.
\vs Eze 5:15 И будешь посмеянием и поруганием, примером и ужасом у народов, которые вокруг тебя, когда Я произведу над тобою суд во гневе и ярости, и в яростных казнях;~--- Я, Господь, изрек сие;~---
\vs Eze 5:16 и когда пошлю на них лютые стрелы голода, которые будут губить, когда пошлю их на погибель вашу, и усилю голод между вами, и сокрушу хлебную опору у вас,
\vs Eze 5:17 и пошлю на вас голод и лютых зверей, и обесчадят тебя; и язва и кровь пройдет по тебе, и меч наведу на тебя; Я, Господь, изрек сие.
\vs Eze 6:1 И было ко мне слово Господне:
\vs Eze 6:2 сын человеческий! обрати лице твое к горам Израилевым и прореки на них,
\vs Eze 6:3 и скажи: горы Израилевы! слушайте слово Господа Бога. Так говорит Господь Бог горам и холмам, долинам и лощинам: вот, Я наведу на вас меч, и разрушу высоты ваши;
\vs Eze 6:4 и жертвенники ваши будут опустошены, столбы ваши в честь солнца будут разбиты, и повергну убитых ваших перед идолами вашими;
\vs Eze 6:5 и положу трупы сынов Израилевых перед идолами их, и рассыплю кости ваши вокруг жертвенников ваших.
\vs Eze 6:6 Во всех местах вашего жительства города будут опустошены и высоты разрушены, для того, чтобы опустошены и разрушены были жертвенники ваши, чтобы сокрушены и уничтожены были идолы ваши, и разбиты солнечные столбы ваши, и изгладились произведения ваши.
\vs Eze 6:7 И будут падать среди вас убитые, и узнаете, что Я Господь.
\vs Eze 6:8 Но Я сберегу остаток, так что будут у вас среди народов уцелевшие от меча, когда вы будете рассеяны по землям.
\vs Eze 6:9 И вспомнят о Мне уцелевшие ваши среди народов, куда будут отведены в плен, когда Я приведу в сокрушение блудное сердце их, отпавшее от Меня, и глаза их, блудившие вслед идолов; и они к самим себе почувствуют отвращение за то зло, какое они делали во всех мерзостях своих;
\vs Eze 6:10 и узнают, что Я Господь; не напрасно говорил Я, что наведу на них такое бедствие.
\vs Eze 6:11 Так говорит Господь Бог: всплесни руками твоими и топни ногою твоею, и скажи: горе за все гнусные злодеяния дома Израилева! падут они от меча, голода и моровой язвы.
\vs Eze 6:12 Кто вдали, тот умрет от моровой язвы; а кто близко, тот падет от меча; а оставшийся и уцелевший умрет от голода; так совершу над ними гнев Мой.
\vs Eze 6:13 И узнаете, что Я Господь, когда пораженные будут \bibemph{лежать} между идолами своими вокруг жертвенников их, на всяком высоком холме, на всех вершинах гор и под всяким зеленеющим деревом, и под всяким ветвистым дубом, на том месте, где они приносили благовонные курения всем идолам своим.
\vs Eze 6:14 И простру на них руку Мою, и сделаю землю пустынею и степью, от пустыни Дивлаф, во всех местах жительства их, и узнают, что Я Господь.
\vs Eze 7:1 И было ко мне слово Господне:
\vs Eze 7:2 и ты, сын человеческий, [скажи]: так говорит Господь Бог; земле Израилевой конец,~--- конец пришел на четыре края земли.
\vs Eze 7:3 Вот конец тебе; и пошлю на тебя гнев Мой, и буду судить тебя по путям твоим, и возложу на тебя все мерзости твои.
\vs Eze 7:4 И не пощадит тебя око Мое, и не помилую, и воздам тебе по путям твоим, и мерзости твои с тобою будут, и узнаете, что Я Господь.
\vs Eze 7:5 Так говорит Господь Бог: беда единственная, вот, идет беда.
\vs Eze 7:6 Конец пришел, пришел конец, встал на тебя; вот дошла,
\vs Eze 7:7 дошла напасть до тебя, житель земли! приходит время, приближается день смятения, а не веселых восклицаний на горах.
\vs Eze 7:8 Вот, скоро изолью на тебя ярость Мою и совершу над тобою гнев Мой, и буду судить тебя по путям твоим, и возложу на тебя все мерзости твои.
\vs Eze 7:9 И не пощадит тебя око Мое, и не помилую. По путям твоим воздам тебе, и мерзости твои с тобою будут; и узнаете, что Я Господь каратель.
\vs Eze 7:10 Вот день! вот пришла, наступила напасть! жезл вырос, гордость разрослась.
\vs Eze 7:11 Восстает сила на жезл нечестия; ничего \bibemph{не останется} от них, и от богатства их, и от шума их, и от пышности их.
\vs Eze 7:12 Пришло время, наступил день; купивший не радуйся, и продавший не плачь; ибо гнев над всем множеством их.
\vs Eze 7:13 Ибо продавший не возвратится к проданному, хотя бы и остались они в живых; ибо пророческое видение о всем множестве их не отменится, и никто своим беззаконием не укрепит своей жизни.
\vs Eze 7:14 Затрубят в трубу, и все готовится, но никто не идет на войну: ибо гнев Мой над всем множеством их.
\vs Eze 7:15 Вне дома меч, а в доме мор и голод. Кто в поле, тот умрет от меча; а кто в городе, того пожрут голод и моровая язва.
\vs Eze 7:16 А уцелевшие из них убегут и будут на горах, как голуби долин; все они будут стонать, каждый за свое беззаконие.
\vs Eze 7:17 У всех руки опустятся, и у всех колени задрожат, \bibemph{как} вода.
\vs Eze 7:18 Тогда они препояшутся вретищем, и обоймет их трепет; и у всех на лицах будет стыд, и у всех на головах плешь.
\vs Eze 7:19 Серебро свое они выбросят на улицы, и золото у них будет в пренебрежении. Серебро их и золото их не сильно будет спасти их в день ярости Господа. Они не насытят ими душ своих и не наполнят утроб своих; ибо оно было поводом к беззаконию их.
\vs Eze 7:20 И в красных нарядах своих они превращали его в гордость, и делали из него изображения гнусных своих истуканов; за то и сделаю его нечистым для них;
\vs Eze 7:21 и отдам его в руки чужим в добычу и беззаконникам земли на расхищение, и они осквернят его.
\vs Eze 7:22 И отвращу от них лице Мое, и осквернят сокровенное Мое; и придут туда грабители, и осквернят его.
\vs Eze 7:23 Сделай цепь, ибо земля эта наполнена кровавыми злодеяниями, и город полон насилий.
\vs Eze 7:24 Я приведу злейших из народов, и завладеют домами их. И положу конец надменности сильных, и будут осквернены святыни их.
\vs Eze 7:25 Идет пагуба; будут искать мира, и не найдут.
\vs Eze 7:26 Беда пойдет за бедою и весть за вестью; и будут просить у пророка видения, и не станет учения у священника и совета у старцев.
\vs Eze 7:27 Царь будет сетовать, и князь облечется в ужас, и у народа земли будут дрожать руки. Поступлю с ними по путям их, и по судам их буду судить их; и узнают, что Я Господь.
\vs Eze 8:1 И было в шестом году, в шестом \bibemph{месяце}, в пятый день месяца, сидел я в доме моем, и старейшины Иудейские сидели перед лицем моим, и низошла на меня там рука Господа Бога.
\vs Eze 8:2 И увидел я: и вот подобие [мужа], как бы огненное, и от чресл его и ниже~--- огонь, и от чресл его и выше~--- как бы сияние, как бы свет пламени.
\vs Eze 8:3 И простер Он как бы руку, и взял меня за волоса головы моей, и поднял меня дух между землею и небом, и принес меня в видениях Божиих в Иерусалим ко входу внутренних ворот, обращенных к северу, где поставлен был идол ревности, возбуждающий ревнование.
\vs Eze 8:4 И вот, там была слава Бога Израилева, подобная той, какую я видел на поле.
\vs Eze 8:5 И сказал мне: сын человеческий! подними глаза твои к северу. И я поднял глаза мои к северу, и вот, с северной стороны у ворот жертвенника~--- тот идол ревности при входе.
\vs Eze 8:6 И сказал Он мне: сын человеческий! видишь ли ты, что они делают? великие мерзости, какие делает дом Израилев здесь, чтобы Я удалился от святилища Моего? но обратись, и ты увидишь еще б\acc{о}льшие мерзости.
\vs Eze 8:7 И привел меня ко входу во двор, и я взглянул, и вот в стене скважина.
\vs Eze 8:8 И сказал мне: сын человеческий! прокопай стену; и я прокопал стену, и вот какая-то дверь.
\vs Eze 8:9 И сказал мне: войди и посмотри на отвратительные мерзости, какие они делают здесь.
\vs Eze 8:10 И вошел я, и вижу, и вот всякие изображения пресмыкающихся и нечистых животных и всякие идолы дома Израилева, написанные по стенам кругом.
\vs Eze 8:11 И семьдесят мужей из старейшин дома Израилева стоят перед ними, и Иезания, сын Сафанов, среди них; и у каждого в руке свое кадило, и густое облако курений возносится кверху.
\vs Eze 8:12 И сказал мне: видишь ли, сын человеческий, что делают старейшины дома Израилева в темноте, каждый в расписанной своей комнате? ибо говорят: <<не видит нас Господь, оставил Господь землю сию>>.
\vs Eze 8:13 И сказал мне: обратись, и увидишь еще б\acc{о}льшие мерзости, какие они делают.
\vs Eze 8:14 И привел меня ко входу в ворота дома Господня, которые к северу, и вот, там сидят женщины, плачущие по Фаммузе,
\vs Eze 8:15 и сказал мне: видишь ли, сын человеческий? обратись, и еще увидишь б\acc{о}льшие мерзости.
\vs Eze 8:16 И ввел меня во внутренний двор дома Господня, и вот у дверей храма Господня, между притвором и жертвенником, около двадцати пяти мужей \bibemph{стоят} спинами своими ко храму Господню, а лицами своими на восток, и кланяются на восток солнцу.
\vs Eze 8:17 И сказал мне: видишь ли, сын человеческий? мало ли дому Иудину, чтобы делать такие мерзости, какие они делают здесь? но они еще землю наполнили нечестием, и сугубо прогневляют Меня; и вот, они ветви подносят к носам своим.
\vs Eze 8:18 За то и Я стану действовать с яростью; не пожалеет око Мое, и не помилую; и хотя бы они взывали в уши Мои громким голосом, не услышу их.
\vs Eze 9:1 И возгласил в уши мои великим гласом, говоря: пусть приблизятся каратели города, каждый со своим губительным орудием в руке своей.
\vs Eze 9:2 И вот, шесть человек идут от верхних ворот, обращенных к северу, и у каждого в руке губительное орудие его, и между ними один, одетый в льняную одежду, у которого при поясе его прибор писца. И пришли и стали подле медного жертвенника.
\vs Eze 9:3 И слава Бога Израилева сошла с Херувима, на котором была, к порогу дома. И призвал Он человека, одетого в льняную одежду, у которого при поясе прибор писца.
\vs Eze 9:4 И сказал ему Господь: пройди посреди города, посреди Иерусалима, и на челах людей скорбящих, воздыхающих о всех мерзостях, совершающихся среди него, сделай знак.
\vs Eze 9:5 А тем сказал в слух мой: идите за ним по городу и поражайте; пусть не жалеет око ваше, и не щадите;
\vs Eze 9:6 старика, юношу и девицу, и младенца и жен бейте до смерти, но не троньте ни одного человека, на котором знак, и начните от святилища Моего. И начали они с тех старейшин, которые были перед домом.
\vs Eze 9:7 И сказал им: оскверните дом, и наполните дворы убитыми, и выйдите. И вышли, и стали убивать в городе.
\vs Eze 9:8 И когда они их убили, а я остался, тогда я пал на лице свое и возопил, и сказал: о, Господи Боже! неужели Ты погубишь весь остаток Израиля, изливая гнев Твой на Иерусалим?
\vs Eze 9:9 И сказал Он мне: нечестие дома Израилева и Иудина велико, весьма велико; и земля сия полна крови, и город исполнен неправды; ибо они говорят: <<оставил Господь землю сию, и не видит Господь>>.
\vs Eze 9:10 За то и Мое око не пощадит, и не помилую; обращу поведение их на их голову.
\vs Eze 9:11 И вот человек, одетый в льняную одежду, у которого при поясе прибор писца, дал ответ и сказал: я сделал, как Ты повелел мне.
\vs Eze 10:1 И видел я, и вот на своде, который над главами Херувимов, как бы камень сапфир, как бы нечто, похожее на престол, видимо было над ними.
\vs Eze 10:2 И говорил Он человеку, одетому в льняную одежду, и сказал: войди между колесами под Херувимов и возьми полные пригоршни горящих угольев между Херувимами, и брось на город; и он вошел в моих глазах.
\vs Eze 10:3 Херувимы же стояли по правую сторону дома, когда вошел тот человек, и облако наполняло внутренний двор.
\vs Eze 10:4 И поднялась слава Господня с Херувима к порогу дома, и дом наполнился облаком, и двор наполнился сиянием славы Господа.
\vs Eze 10:5 И шум от крыльев Херувимов слышен был даже на внешнем дворе, как бы глас Бога Всемогущего, когда Он говорит.
\vs Eze 10:6 И когда Он дал повеление человеку, одетому в льняную одежду, сказав: <<возьми огня между колесами, между Херувимами>>, и когда он вошел и стал у колеса,~---
\vs Eze 10:7 тогда из среды Херувимов один Херувим простер руку свою к огню, который между Херувимами, и взял и дал в пригоршни одетому в льняную одежду. Он взял и вышел.
\vs Eze 10:8 И видно было у Херувимов подобие рук человеческих под крыльями их.
\vs Eze 10:9 И видел я: и вот четыре колеса подле Херувимов, по одному колесу подле каждого Херувима, и колеса по виду как бы из камня топаза.
\vs Eze 10:10 И по виду все четыре сходны, как будто бы колесо находилось в колесе.
\vs Eze 10:11 Когда шли они, то шли на четыре свои стороны; во время шествия своего не оборачивались, но к тому месту, куда обращена была голова, и они туда шли; во время шествия своего не оборачивались.
\vs Eze 10:12 И все тело их, и спина их, и руки их, и крылья их, и колеса кругом были полны очей, все четыре колеса их.
\vs Eze 10:13 К колесам сим, как я слышал, сказано было: <<галгал>>\fns{Вихрь.}.
\vs Eze 10:14 И у каждого \bibemph{из} животных четыре лица: первое лице~--- лице херувимово, второе лице~--- лице человеческое, третье лице львиное и четвертое лице орлиное.
\vs Eze 10:15 Херувимы поднялись. Это были те же животные, которых видел я при реке Ховаре.
\vs Eze 10:16 И когда шли Херувимы, тогда шли подле них и колеса; и когда Херувимы поднимали крылья свои, чтобы подняться от земли, и колеса не отделялись, но были при них.
\vs Eze 10:17 Когда те стояли, стояли и они; когда те поднимались, поднимались и они; ибо в них \bibemph{был} дух животных.
\vs Eze 10:18 И отошла слава Господня от порога дома и стала над Херувимами.
\vs Eze 10:19 И подняли Херувимы крылья свои, и поднялись в глазах моих от земли; когда они уходили, то и колеса подле них; и стали у входа в восточные врата дома Господня, и слава Бога Израилева вверху над ними.
\vs Eze 10:20 Это были те же животные, которых видел я в подножии Бога Израилева при реке Ховаре. И я узнал, что это Херувимы.
\vs Eze 10:21 У каждого по четыре лица, и у каждого по четыре крыла, и под крыльями их подобие рук человеческих.
\vs Eze 10:22 А подобие лиц их то же, какие лица видел я при реке Ховаре,~--- и вид их, и сами они. Каждый шел прямо в ту сторону, которая была перед лицем его.
\vs Eze 11:1 И поднял меня дух, и привел меня к восточным воротам дома Господня, которые обращены к востоку. И вот, у входа в ворота двадцать пять человек; и между ними я видел Иазанию, сына Азурова, и Фалтию, сына Ванеева, князей народа.
\vs Eze 11:2 И Он сказал мне: сын человеческий! вот люди, у которых на уме беззаконие и которые дают худой совет в городе сем,
\vs Eze 11:3 говоря: <<еще не близко; будем строить домы; он\fns{Город.} котел, а мы мясо>>.
\vs Eze 11:4 Посему изреки на них пророчество, пророчествуй, сын человеческий.
\vs Eze 11:5 И нисшел на меня Дух Господень и сказал мне: скажи, так говорит Господь: что говорите вы, дом Израилев, и что на ум вам приходит, это Я знаю.
\vs Eze 11:6 Много убитых ваших вы положили в сем городе и улицы его наполнили трупами.
\vs Eze 11:7 Посему так говорит Господь Бог: убитые ваши, которых вы положили среди него, суть мясо, а он~--- котел; но вас Я выведу из него.
\vs Eze 11:8 Вы боитесь меча, и Я наведу на вас меч, говорит Господь Бог.
\vs Eze 11:9 И выведу вас из него, и отдам вас в руку чужих, и произведу над вами суд.
\vs Eze 11:10 От меча падете; на пределах Израилевых будут судить вас, и узнаете, что Я Господь.
\vs Eze 11:11 Он не будет для вас котлом, и вы не будете мясом в нем; на пределах Израилевых буду судить вас.
\vs Eze 11:12 И узнаете, что Я Господь; ибо по заповедям Моим вы не ходили и уставов Моих не выполняли, а поступали по уставам народов, окружающих вас.
\vs Eze 11:13 И было, когда я пророчествовал, Фалтия, сын Ванеев, умер. И пал я на лице, и возопил громким голосом, и сказал: о, Господи Боже! неужели Ты хочешь до конца истребить остаток Израиля?
\rsbpar\vs Eze 11:14 И было ко мне слово Господне:
\vs Eze 11:15 сын человеческий! твоим братьям, твоим братьям, твоим единокровным и всему дому Израилеву, всем им говорят живущие в Иерусалиме: <<живите вдали от Господа; нам во владение отдана эта земля>>.
\vs Eze 11:16 На это скажи: так говорит Господь Бог: хотя Я и удалил их к народам и хотя рассеял их по землям, но Я буду для них некоторым святилищем в тех землях, куда пошли они.
\vs Eze 11:17 Затем скажи: так говорит Господь Бог: Я соберу вас из народов, и возвращу вас из земель, в которые вы рассеяны; и дам вам землю Израилеву.
\vs Eze 11:18 И придут туда, и извергнут из нее все гнусности ее и все мерзости ее.
\vs Eze 11:19 И дам им сердце единое, и дух новый вложу в них, и возьму из плоти их сердце каменное, и дам им сердце плотяное,
\vs Eze 11:20 чтобы они ходили по заповедям Моим, и соблюдали уставы Мои, и выполняли их; и будут Моим народом, а Я буду их Богом.
\vs Eze 11:21 А чье сердце увлечется вслед гнусностей их и мерзостей их, поведение тех обращу на их голову, говорит Господь Бог.
\vs Eze 11:22 Тогда Херувимы подняли крылья свои, и колеса подле них; и слава Бога Израилева вверху над ними.
\vs Eze 11:23 И поднялась слава Господа из среды города и остановилась над горою, которая на восток от города.
\rsbpar\vs Eze 11:24 И дух поднял меня и перенес меня в Халдею, к переселенцам, в видении, Духом Божиим. И отошло от меня видение, которое я видел.
\vs Eze 11:25 И я пересказал переселенцам все слова Господа, которые Он открыл мне.
\vs Eze 12:1 И было ко мне слово Господне:
\vs Eze 12:2 сын человеческий! ты живешь среди дома мятежного; у них есть глаза, чтобы видеть, а не видят; у них есть уши, чтобы слышать, а не слышат; потому что они~--- мятежный дом.
\vs Eze 12:3 Ты же, сын человеческий, изготовь себе нужное для переселения, и среди дня переселяйся перед глазами их, и переселяйся с места твоего в другое место перед глазами их; может быть, они уразумеют, хотя они~--- дом мятежный;
\vs Eze 12:4 и вещи твои вынеси, как вещи нужные при переселении, днем, перед глазами их, и сам выйди вечером перед глазами их, как выходят для переселения.
\vs Eze 12:5 Перед глазами их проломай себе отверстие в стене, и вынеси через него.
\vs Eze 12:6 Перед глазами их возьми ношу на плечо, впотьмах вынеси ее, лице твое закрой, чтобы не видеть земли; ибо Я поставил тебя знамением дому Израилеву.
\vs Eze 12:7 И сделал я, как повелено было мне; вещи мои, как вещи нужные при переселении, вынес днем, а вечером проломал себе рукою отверстие в стене, впотьмах вынес ношу и поднял на плечо перед глазами их.
\vs Eze 12:8 И было ко мне слово Господне поутру:
\vs Eze 12:9 сын человеческий! не говорил ли тебе дом Израилев, дом мятежный: <<что ты делаешь?>>
\vs Eze 12:10 Скажи им: так говорит Господь Бог: это~--- предвещание для начальствующего в Иерусалиме и для всего дома Израилева, который находится там.
\vs Eze 12:11 Скажи: я знамение для вас; что делаю я, то будет с ними,~--- в переселение, в плен пойдут они.
\vs Eze 12:12 И начальствующий, который среди них, впотьмах поднимет \bibemph{ношу} на плечо и выйдет. Стену проломают, чтобы отправить \bibemph{его} через нее; он закроет лице свое, так что не увидит глазами земли сей.
\vs Eze 12:13 И раскину на него сеть Мою, и будет пойман в тенета Мои, и отведу его в Вавилон, в землю Халдейскую, но он не увидит ее, и там умрет.
\vs Eze 12:14 А всех, которые вокруг него, споборников его и все войско его развею по всем ветрам, и обнажу вслед их меч.
\vs Eze 12:15 И узнают, что Я Господь, когда рассею их по народам и развею их по землям.
\vs Eze 12:16 Но небольшое число их Я сохраню от меча, голода и язвы, чтобы они рассказали у народов, к которым пойдут, о всех своих мерзостях; и узнают, что Я Господь.
\vs Eze 12:17 И было ко мне слово Господне:
\vs Eze 12:18 сын человеческий! хлеб твой ешь с трепетом, и воду твою пей с дрожанием и печалью.
\vs Eze 12:19 И скажи народу земли: так говорит Господь Бог о жителях Иерусалима, о земле Израилевой: они хлеб свой будут есть с печалью и воду свою будут пить в унынии, потому что земля его будет лишена всего изобилия своего за неправды всех живущих на ней.
\vs Eze 12:20 И будут разорены населенные города, и земля сделается пустою, и узнаете, что Я Господь.
\vs Eze 12:21 И было ко мне слово Господне:
\vs Eze 12:22 сын человеческий! что за поговорка у вас, в земле Израилевой: <<много дней пройдет, и всякое пророческое видение исчезнет>>?
\vs Eze 12:23 Посему скажи им: так говорит Господь Бог: уничтожу эту поговорку, и не будут уже употреблять такой поговорки у Израиля; но скажи им: близки дни и исполнение всякого видения пророческого.
\vs Eze 12:24 Ибо уже не останется втуне никакое видение пророческое, и ни одно предвещание не будет ложным в доме Израилевом.
\vs Eze 12:25 Ибо Я Господь, Я говорю; и слово, которое Я говорю, исполнится, и не будет отложено; в ваши дни, мятежный дом, Я изрек слово, и исполню его, говорит Господь Бог.
\vs Eze 12:26 И было ко мне слово Господне:
\vs Eze 12:27 сын человеческий! вот, дом Израилев говорит: <<пророческое видение, которое видел он, \bibemph{сбудется} после многих дней, и он пророчествует об отдаленных временах>>.
\vs Eze 12:28 Посему скажи им: так говорит Господь Бог: ни одно из слов Моих уже не будет отсрочено, но слово, которое Я скажу, сбудется, говорит Господь Бог.
\vs Eze 13:1 И было ко мне слово Господне:
\vs Eze 13:2 сын человеческий! изреки пророчество на пророков Израилевых пророчествующих, и скажи пророкам от собственного сердца: слушайте слово Господне!
\vs Eze 13:3 Так говорит Господь Бог: горе безумным пророкам, которые водятся своим духом и ничего не видели!
\vs Eze 13:4 Пророки твои, Израиль, как лисицы в развалинах.
\vs Eze 13:5 В проломы вы не вх\acc{о}дите и не ограждаете стеною дома Израилева, чтобы твердо стоять в сражении в день Господа.
\vs Eze 13:6 Они видят пустое и предвещают ложь, говоря: <<Господь сказал>>; а Господь не посылал их; и обнадеживают, что слово сбудется.
\vs Eze 13:7 Не пустое ли видение видели вы? и не лживое ли предвещание изрекаете, говоря: <<Господь сказал>>, а Я не говорил?
\vs Eze 13:8 Посему так говорит Господь Бог: так как вы говорите пустое и видите в видениях ложь, за то вот Я~--- на вас, говорит Господь Бог.
\vs Eze 13:9 И будет рука Моя против этих пророков, видящих пустое и предвещающих ложь; в совете народа Моего они не будут, и в список дома Израилева не впишутся, и в землю Израилеву не войдут; и узнаете, что Я Господь Бог.
\vs Eze 13:10 За то, что они вводят народ Мой в заблуждение, говоря: <<мир>>, тогда как нет мира; и когда он строит стену, они обмазывают ее грязью,
\vs Eze 13:11 скажи обмазывающим стену грязью, что она упадет. Пойдет проливной дождь, и вы, каменные градины, падете, и бурный ветер разорвет ее.
\vs Eze 13:12 И вот, падет стена; тогда не скажут ли вам: <<где та обмазка, которою вы обмазывали?>>
\vs Eze 13:13 Посему так говорит Господь Бог: Я пущу бурный ветер во гневе Моем, и пойдет проливной дождь в ярости Моей, и камни града в негодовании Моем, для истребления.
\vs Eze 13:14 И разрушу стену, которую вы обмазывали грязью, и повергну ее на землю, и откроется основание ее, и падет, и вы вместе с нею погибнете; и узнаете, что Я Господь.
\vs Eze 13:15 И истощу ярость Мою на стене и на обмазывающих ее грязью, и скажу вам: нет стены, и нет обмазывавших ее,
\vs Eze 13:16 пророков Израилевых, которые пророчествовали Иерусалиму и возвещали ему видения мира, тогда как нет мира, говорит Господь Бог.
\vs Eze 13:17 Ты же, сын человеческий, обрати лице твое к дщерям народа твоего, пророчествующим от собственного своего сердца, и изреки на них пророчество,
\vs Eze 13:18 и скажи: так говорит Господь Бог: горе сшивающим чародейные мешочки под мышки и делающим покрывала для головы всякого роста, чтобы уловлять души! Неужели, уловляя души народа Моего, вы спасете ваши души?
\vs Eze 13:19 И бесславите Меня пред народом Моим за горсти ячменя и за куски хлеба, умерщвляя души, которые не должны умереть, и оставляя жизнь душам, которые не должны жить, обманывая народ, который слушает ложь.
\vs Eze 13:20 Посему так говорит Господь Бог: вот, Я~--- на ваши чародейные мешочки, которыми вы там уловляете души, чтобы они прилетали, и вырву их из-под мышц ваших, и пущу на свободу души, которые вы уловляете, чтобы прилетали к вам.
\vs Eze 13:21 И раздеру покрывала ваши, и избавлю народ Мой от рук ваших, и не будут уже в ваших руках добычею, и узнаете, что Я Господь.
\vs Eze 13:22 За то, что вы ложью опечаливаете сердце праведника, которое Я не хотел опечаливать, и поддерживаете руки беззаконника, чтобы он не обратился от порочного пути своего и не сохранил жизни своей,~---
\vs Eze 13:23 за это уже не будете иметь пустых видений и впредь не будете предугадывать; и Я избавлю народ Мой от рук ваших, и узнаете, что Я Господь.
\vs Eze 14:1 И пришли ко мне несколько человек из старейшин Израилевых и сели перед лицем моим.
\vs Eze 14:2 И было ко мне слово Господне:
\vs Eze 14:3 сын человеческий! Сии люди допустили идолов своих в сердце свое и поставили соблазн нечестия своего перед лицем своим: могу ли Я отвечать им?
\vs Eze 14:4 Посему говори с ними и скажи им: так говорит Господь Бог: если кто из дома Израилева допустит идолов своих в сердце свое и поставит соблазн нечестия своего перед лицем своим, и придет к пророку,~--- то Я, Господь, могу ли, при множестве идолов его, дать ему ответ?
\vs Eze 14:5 Пусть дом Израилев поймет в сердце своем, что все они через своих идолов сделались чужими для Меня.
\vs Eze 14:6 Посему скажи дому Израилеву: так говорит Господь Бог: обратитесь и отвратитесь от идолов ваших, и от всех мерзостей ваших отвратите лице ваше.
\vs Eze 14:7 Ибо если кто из дома Израилева и из пришельцев, которые живут у Израиля, отложится от Меня и допустит идолов своих в сердце свое, и поставит соблазн нечестия своего перед лицем своим, и придет к пророку вопросить Меня через него,~--- то Я, Господь, дам ли ему ответ от Себя?
\vs Eze 14:8 Я обращу лице Мое против того человека и сокрушу его в знамение и притчу, и истреблю его из народа Моего, и узнаете, что Я Господь.
\vs Eze 14:9 А если пророк допустит обольстить себя и скажет слово так, как бы Я, Господь, научил этого пророка, то Я простру на него руку Мою и истреблю его из народа Моего, Израиля.
\vs Eze 14:10 И понесут вину беззакония своего: какова вина вопрошающего, такова будет вина и пророка,
\vs Eze 14:11 чтобы впредь дом Израилев не уклонялся от Меня и чтобы более не оскверняли себя всякими беззакониями своими, но чтобы были Моим народом, и Я был их Богом, говорит Господь Бог.
\vs Eze 14:12 И было ко мне слово Господне:
\vs Eze 14:13 сын человеческий! если бы какая земля согрешила предо Мною, вероломно отступив от Меня, и Я простер на нее руку Мою, и истребил в ней хлебную опору, и послал на нее голод, и стал губить на ней людей и скот;
\vs Eze 14:14 и если бы нашлись в ней сии три мужа: Ной, Даниил и Иов,~--- то они праведностью своею спасли бы только свои души, говорит Господь Бог.
\vs Eze 14:15 Или, если бы Я послал на эту землю лютых зверей, которые осиротили бы ее, и она по причине зверей сделалась пустою и непроходимою:
\vs Eze 14:16 то сии три мужа среди нее,~--- живу Я, говорит Господь Бог,~--- не спасли бы ни сыновей, ни дочерей, а они, только они спаслись бы, земля же сделалась бы пустынею.
\vs Eze 14:17 Или, если бы Я навел на ту землю меч и сказал: <<меч, пройди по земле!>>, и стал истреблять на ней людей и скот,
\vs Eze 14:18 то сии три мужа среди нее,~--- живу Я, говорит Господь Бог,~--- не спасли бы ни сыновей, ни дочерей, а они только спаслись бы.
\vs Eze 14:19 Или, если бы Я послал на ту землю моровую язву и излил на нее ярость Мою в кровопролитии, чтобы истребить на ней людей и скот:
\vs Eze 14:20 то Ной, Даниил и Иов среди нее,~--- живу Я, говорит Господь Бог,~--- не спасли бы ни сыновей, ни дочерей; праведностью своею они спасли бы только свои души.
\vs Eze 14:21 Ибо так говорит Господь Бог: если и четыре тяжкие казни Мои: меч, и голод, и лютых зверей, и моровую язву пошлю на Иерусалим, чтобы истребить в нем людей и скот,
\vs Eze 14:22 и тогда останется в нем остаток, сыновья и дочери, которые будут выведены оттуда; вот, они выйдут к вам, и вы увидите поведение их и дела их, и утешитесь о том бедствии, которое Я навел на Иерусалим, о всем, что Я навел на него.
\vs Eze 14:23 Они утешат вас, когда вы увидите поведение их и дела их; и узнаете, что Я не напрасно сделал все то, что сделал в нем, говорит Господь Бог.
\vs Eze 15:1 И было ко мне слово Господне:
\vs Eze 15:2 сын человеческий! какое преимущество имеет дерево виноградной лозы перед всяким другим деревом и ветви виноградной лозы~--- между деревами в лесу?
\vs Eze 15:3 Берут ли от него кусок на какое-либо изделие? Берут ли от него хотя на гвоздь, чтобы вешать на нем какую-либо вещь?
\vs Eze 15:4 Вот, оно отдается огню на съедение; оба конца его огонь поел, и обгорела середина его: годится ли оно на какое-нибудь изделие?
\vs Eze 15:5 И тогда, как оно было цело, не годилось ни на какое изделие; тем паче, когда огонь поел его, и оно обгорело, годится ли оно на какое-нибудь изделие?
\vs Eze 15:6 Посему так говорит Господь Бог: как дерево виноградной лозы между деревами лесными Я отдал огню на съедение, так отдам ему и жителей Иерусалима.
\vs Eze 15:7 И обращу лице Мое против них; из одного огня выйдут, и другой огонь пожрет их,~--- и узнаете, что Я Господь, когда обращу против них лице Мое.
\vs Eze 15:8 И сделаю эту землю пустынею за то, что они вероломно поступали, говорит Господь Бог.
\vs Eze 16:1 И было ко мне слово Господне:
\vs Eze 16:2 сын человеческий! выскажи Иерусалиму мерзости его
\vs Eze 16:3 и скажи: так говорит Господь Бог \bibemph{дщери} Иерусалима: твой корень и твоя родина в земле Ханаанской; отец твой Аморрей, и мать твоя Хеттеянка;
\vs Eze 16:4 при рождении твоем, в день, когда ты родилась, пупа твоего не отрезали, и водою ты не была омыта для очищения, и солью не была осолена, и пеленами не повита.
\vs Eze 16:5 Ничей глаз не сжалился над тобою, чтобы из милости к тебе сделать тебе что-нибудь из этого; но ты выброшена была на поле, по презрению к жизни твоей, в день рождения твоего.
\vs Eze 16:6 И проходил Я мимо тебя, и увидел тебя, брошенную на попрание в кровях твоих, и сказал тебе: <<в кровях твоих живи!>> Так, Я сказал тебе: <<в кровях твоих живи!>>
\vs Eze 16:7 Умножил тебя как полевые растения; ты выросла и стала большая, и достигла превосходной красоты: поднялись груди, и волоса у тебя выросли; но ты была нага и непокрыта.
\vs Eze 16:8 И проходил Я мимо тебя, и увидел тебя, и вот, это было время твое, время любви; и простер Я воскрилия \bibemph{риз} Моих на тебя, и покрыл наготу твою; и поклялся тебе и вступил в союз с тобою, говорит Господь Бог,~--- и ты стала Моею.
\vs Eze 16:9 Омыл Я тебя водою и смыл с тебя кровь твою и помазал тебя елеем.
\vs Eze 16:10 И надел на тебя узорчатое платье, и обул тебя в сафьянные сандалии, и опоясал тебя виссоном, и покрыл тебя шелковым покрывалом.
\vs Eze 16:11 И нарядил тебя в наряды, и положил на руки твои запястья и на шею твою ожерелье.
\vs Eze 16:12 И дал тебе кольцо на твой нос и серьги к ушам твоим и на голову твою прекрасный венец.
\vs Eze 16:13 Так украшалась ты золотом и серебром, и одежда твоя \bibemph{была} виссон и шелк и узорчатые ткани; питалась ты хлебом из лучшей пшеничной муки, медом и елеем, и была чрезвычайно красива, и достигла царственного величия.
\vs Eze 16:14 И пронеслась по народам слава твоя ради красоты твоей, потому что она была вполне совершенна при том великолепном наряде, который Я возложил на тебя, говорит Господь Бог.
\vs Eze 16:15 Но ты понадеялась на красоту твою, и, пользуясь славою твоею, стала блудить и расточала блудодейство твое на всякого мимоходящего, отдаваясь ему.
\vs Eze 16:16 И взяла из одежд твоих, и сделала себе разноцветные высоты, и блудодействовала на них, как никогда не случится и не будет.
\vs Eze 16:17 И взяла нарядные твои вещи из Моего золота и из Моего серебра, которые Я дал тебе, и сделала себе мужские изображения, и блудодействовала с ними.
\vs Eze 16:18 И взяла узорчатые платья твои, и одела их ими, и ставила перед ними елей Мой и фимиам Мой,
\vs Eze 16:19 и хлеб Мой, который Я давал тебе, пшеничную муку, и елей, и мед, которыми Я питал тебя, ты поставляла перед ними в приятное благовоние; и это было, говорит Господь Бог.
\vs Eze 16:20 И взяла сыновей твоих и дочерей твоих, которых ты родила Мне, и приносила в жертву на снедение им. Мало ли тебе было блудодействовать?
\vs Eze 16:21 Но ты и сыновей Моих заколала и отдавала им, проводя их \bibemph{через огонь}.
\vs Eze 16:22 И при всех твоих мерзостях и блудодеяниях твоих ты не вспомнила о днях юности твоей, когда ты была нага и непокрыта и брошена в крови твоей на попрание.
\vs Eze 16:23 И после всех злодеяний твоих,~--- горе, горе тебе! говорит Господь Бог,~---
\vs Eze 16:24 ты построила себе блудилища и наделала себе возвышений на всякой площади;
\vs Eze 16:25 при начале всякой дороги устроила себе возвышения, позорила красоту твою и раскидывала ноги твои для всякого мимоходящего, и умножила блудодеяния твои.
\vs Eze 16:26 Блудила с сыновьями Египта, соседями твоими, людьми великорослыми, и умножала блудодеяния твои, прогневляя Меня.
\vs Eze 16:27 И вот, Я простер на тебя руку Мою, и уменьшил назначенное тебе, и отдал тебя на произвол ненавидящим тебя дочерям Филистимским, которые устыдились срамного поведения твоего.
\vs Eze 16:28 И блудила ты с сынами Ассура и не насытилась; блудила с ними, но тем не удовольствовалась;
\vs Eze 16:29 и умножила блудодеяния твои в земле Ханаанской до Халдеи, но и тем не удовольствовалась.
\vs Eze 16:30 Как истомлено должно быть сердце твое, говорит Господь Бог, когда ты все это делала, как необузданная блудница!
\vs Eze 16:31 Когда ты строила себе блудилища при начале всякой дороги и делала себе возвышения на всякой площади, ты была не как блудница, потому что отвергала подарки,
\vs Eze 16:32 но как прелюбодейная жена, принимающая вместо своего мужа чужих.
\vs Eze 16:33 Всем блудницам дают подарки, а ты сама давала подарки всем любовникам твоим и подкупала их, чтобы они со всех сторон приходили к тебе блудить с тобою.
\vs Eze 16:34 У тебя в блудодеяниях твоих было противное тому, что бывает с женщинами: не за тобою гонялись, но ты давала подарки, а тебе не давали подарков; и потому ты поступала в противность другим.
\vs Eze 16:35 Посему выслушай, блудница, слово Господне!
\vs Eze 16:36 Так говорит Господь Бог: за то, что ты так сыпала деньги твои, и в блудодеяниях твоих раскрываема была нагота твоя перед любовниками твоими и перед всеми мерзкими идолами твоими, и за кровь сыновей твоих, которых ты отдавала им,~---
\vs Eze 16:37 за то вот, Я соберу всех любовников твоих, которыми ты услаждалась и которых ты любила, со всеми теми, которых ненавидела, и соберу их отовсюду против тебя, и раскрою перед ними наготу твою, и увидят весь срам твой.
\vs Eze 16:38 Я буду судить тебя судом прелюбодейц и проливающих кровь,~--- и предам тебя кровавой ярости и ревности;
\vs Eze 16:39 предам тебя в руки их и они разорят блудилища твои, и раскидают возвышения твои, и сорвут с тебя одежды твои, и возьмут наряды твои, и оставят тебя нагою и непокрытою.
\vs Eze 16:40 И созовут на тебя собрание, и побьют тебя камнями, и разрубят тебя мечами своими.
\vs Eze 16:41 Сожгут домы твои огнем и совершат над тобою суд перед глазами многих жен; и положу конец блуду твоему, и не будешь уже давать подарков.
\vs Eze 16:42 И утолю над тобою гнев Мой, и отступит от тебя негодование Мое, и успокоюсь, и уже не буду гневаться.
\vs Eze 16:43 За то, что ты не вспомнила о днях юности твоей и всем этим раздражала Меня, вот, и Я поведение твое обращу на \bibemph{твою} голову, говорит Господь Бог, чтобы ты не предавалась более разврату после всех твоих мерзостей.
\vs Eze 16:44 Вот, всякий, кто говорит притчами, может сказать о тебе: <<какова мать, такова и дочь>>.
\vs Eze 16:45 Ты дочь в мать твою, которая бросила мужа своего и детей своих,~--- и ты сестра в сестер твоих, которые бросили мужей своих и детей своих. Мать ваша Хеттеянка, и отец ваш Аморрей.
\vs Eze 16:46 Б\acc{о}льшая же сестра твоя~--- Самария, с дочерями своими живущая влево от тебя; а меньшая сестра твоя, живущая от тебя вправо, есть Содома с дочерями ее.
\vs Eze 16:47 Но ты и не их путями ходила и не по их мерзостям поступала; этого было мало: ты поступала развратнее их на всех путях твоих.
\vs Eze 16:48 Живу Я, говорит Господь Бог; Содома, сестра твоя, не делала того сама и ее дочери, что делала ты и дочери твои.
\vs Eze 16:49 Вот в чем было беззаконие Содомы, сестры твоей и дочерей ее: в гордости, пресыщении и праздности, и она руки бедного и нищего не поддерживала.
\vs Eze 16:50 И возгордились они, и делали мерзости пред лицем Моим, и, увидев это, Я отверг их.
\vs Eze 16:51 И Самария половины грехов твоих не нагрешила; ты превзошла их мерзостями твоими, и через твои мерзости, какие делала ты, сестры твои оказались правее тебя.
\vs Eze 16:52 Неси же посрамление твое и ты, которая осуждала сестер твоих; по грехам твоим, какими ты опозорила себя более их, они правее тебя. Красней же от стыда и ты, и неси посрамление твое, так оправдав сестер твоих.
\vs Eze 16:53 Но Я возвращу плен их, плен Содомы и дочерей ее, плен Самарии и дочерей ее, и между ними плен плененных твоих,
\vs Eze 16:54 дабы ты несла посрамление твое и стыдилась всего того, что делала, служа для них утешением.
\vs Eze 16:55 И сестры твои, Содома и дочери ее, возвратятся в прежнее состояние свое; и Самария и дочери ее возвратятся в прежнее состояние свое, и ты и дочери твои возвратитесь в прежнее состояние ваше.
\vs Eze 16:56 О сестре твоей Содоме и помина не было в устах твоих во дни гордыни твоей,
\vs Eze 16:57 доколе еще не открыто было нечестие твое, как во время посрамления от дочерей Сирии и всех окружавших ее, от дочерей Филистимы, смотревших на тебя с презрением со всех сторон.
\vs Eze 16:58 За разврат твой и за мерзости твои терпишь ты, говорит Господь.
\vs Eze 16:59 Ибо так говорит Господь Бог: Я поступлю с тобою, как поступила ты, презрев клятву нарушением союза.
\vs Eze 16:60 Но Я вспомню союз Мой с тобою во дни юности твоей, и восстановлю с тобою вечный союз.
\vs Eze 16:61 И ты вспомнишь о путях твоих, и будет стыдно тебе, когда станешь принимать к себе сестер твоих, б\acc{о}льших тебя, как и меньших тебя, и когда Я буду давать тебе их в дочерей, но не от твоего союза.
\vs Eze 16:62 Я восстановлю союз Мой с тобою, и узнаешь, что Я Господь,
\vs Eze 16:63 для того, чтобы ты помнила и стыдилась, и чтобы вперед нельзя было тебе и рта открыть от стыда, когда Я прощу тебе все, что ты делала, говорит Господь Бог.
\vs Eze 17:1 И было ко мне слово Господне:
\vs Eze 17:2 сын человеческий! предложи загадку и скажи притчу к дому Израилеву.
\vs Eze 17:3 Скажи: так говорит Господь Бог: большой орел с большими крыльями, с длинными перьями, пушистый, пестрый, прилетел на Ливан и снял с кедра верхушку,
\vs Eze 17:4 сорвал верхний из молодых побегов его и принес его в землю Ханаанскую, в городе торговцев положил его;
\vs Eze 17:5 и взял от семени этой земли, и посадил на земле семени, поместил у больших вод, как сажают иву.
\vs Eze 17:6 И оно выросло, и сделалось виноградною лозою, широкою, низкою ростом, которой ветви клонились к ней, и корни ее были под нею же, и стало виноградною лозою, и дало отрасли, и пустило ветви.
\vs Eze 17:7 И еще был орел с большими крыльями и пушистый; и вот, эта виноградная лоза потянулась к нему своими корнями и простерла к нему ветви свои, чтобы он поливал ее из борозд рассадника своего.
\vs Eze 17:8 Она была посажена на хорошем поле, у больших вод, так что могла пускать ветви и приносить плод, сделаться лозою великолепною.
\vs Eze 17:9 Скажи: так говорит Господь Бог: будет ли ей успех? Не вырвут ли корней ее, и не оборвут ли плодов ее, так что она засохнет? все молодые ветви, отросшие от нее, засохнут. И не с большою силою и не со многими людьми сорвут ее с корней ее.
\vs Eze 17:10 И вот, хотя она посажена, но будет ли успех? Не иссохнет ли она, как скоро коснется ее восточный ветер? иссохнет на грядах, где выросла.
\vs Eze 17:11 И было ко мне слово Господне:
\vs Eze 17:12 скажи мятежному дому: разве не знаете, что это значит?~--- Скажи: вот, пришел царь Вавилонский в Иерусалим, и взял царя его и князей его, и привел их к себе в Вавилон.
\vs Eze 17:13 И взял \bibemph{другого} из царского рода, и заключил с ним союз, и обязал его клятвою, и взял сильных земли той с собою,
\vs Eze 17:14 чтобы царство было покорное, чтобы не могло подняться, чтобы сохраняем был союз и стоял твердо.
\vs Eze 17:15 Но тот отложился от него, послав послов своих в Египет, чтобы дали ему коней и много людей. Будет ли ему успех? Уцелеет ли тот, кто это делает? Он нарушил союз и уцелеет ли?
\vs Eze 17:16 Живу Я, говорит Господь Бог: в местопребывании царя, который поставил его царем, и которому данную клятву он презрел, и нарушил союз свой с ним, он умрет у него в Вавилоне.
\vs Eze 17:17 С великою силою и с многочисленным народом фараон ничего не сделает для него в этой войне, когда будет насыпан вал и построены будут осадные башни на погибель многих душ.
\vs Eze 17:18 Он презрел клятву, чтобы нарушить союз, и вот, дал руку свою и сделал все это; он не уцелеет.
\vs Eze 17:19 Посему так говорит Господь Бог: живу Я! клятву Мою, которую он презрел, и союз Мой, который он нарушил, Я обращу на его голову.
\vs Eze 17:20 И закину на него сеть Мою, и пойман будет в тенета Мои; и приведу его в Вавилон, и там буду судиться с ним за вероломство его против Меня.
\vs Eze 17:21 А все беглецы его из всех полков его падут от меча, а оставшиеся развеяны будут по всем ветрам; и узнаете, что Я, Господь, сказал это.
\vs Eze 17:22 Так говорит Господь Бог: и возьму Я с вершины высокого кедра, и посажу; с верхних побегов его оторву нежную отрасль и посажу на высокой и величественной горе.
\vs Eze 17:23 На высокой горе Израилевой посажу его, и пустит ветви, и принесет плод, и сделается величественным кедром, и будут обитать под ним всякие птицы, всякие пернатые будут обитать в тени ветвей его.
\vs Eze 17:24 И узнают все дерева полевые, что Я, Господь, высокое дерево понижаю, низкое дерево повышаю, зеленеющее дерево иссушаю, а сухое дерево делаю цветущим: Я, Господь, сказал, и сделаю.
\vs Eze 18:1 И было ко мне слово Господне:
\vs Eze 18:2 зачем вы употребляете в земле Израилевой эту пословицу, говоря: <<отцы ели кислый виноград, а у детей на зубах оскомина>>?
\vs Eze 18:3 Живу Я! говорит Господь Бог,~--- не будут вперед говорить пословицу эту в Израиле.
\vs Eze 18:4 Ибо вот, все души~--- Мои: как душа отца, так и душа сына~--- Мои: душа согрешающая, та умрет.
\vs Eze 18:5 Если кто праведен и творит суд и правду,
\vs Eze 18:6 на горах жертвенного не ест и к идолам дома Израилева не обращает глаз своих, жены ближнего своего не оскверняет и к своей жене во время очищения нечистот ее не приближается,
\vs Eze 18:7 никого не притесняет, должнику возвращает залог его, хищения не производит, хлеб свой дает голодному и нагого покрывает одеждою,
\vs Eze 18:8 в рост не отдает и лихвы не берет, от неправды удерживает руку свою, суд человеку с человеком производит правильный,
\vs Eze 18:9 поступает по заповедям Моим и соблюдает постановления Мои искренно: то он праведник, он непременно будет жив, говорит Господь Бог.
\vs Eze 18:10 Но если у него родился сын разбойник, проливающий кровь, и делает что-нибудь из всего того,
\vs Eze 18:11 чего он сам не делал совсем, и на горах ест жертвенное, и жену ближнего своего оскверняет,
\vs Eze 18:12 бедного и нищего притесняет, насильно отнимает, залога не возвращает, и к идолам обращает глаза свои, делает мерзость,
\vs Eze 18:13 в рост дает, и берет лихву; то будет ли он жив? \bibemph{Нет}, он не будет жив. Кто делает все такие мерзости, тот непременно умрет, кровь его будет на нем.
\vs Eze 18:14 Но если у кого родился сын, который, видя все грехи отца своего, какие он делает, видит и не делает подобного им:
\vs Eze 18:15 на горах жертвенного не ест, к идолам дома Израилева не обращает глаз своих, жены ближнего своего не оскверняет,
\vs Eze 18:16 и человека не притесняет, залога не берет, и насильно не отнимает, хлеб свой дает голодному, и нагого покрывает одеждою,
\vs Eze 18:17 от \bibemph{обиды} бедному удерживает руку свою, роста и лихвы не берет, исполняет Мои повеления и поступает по заповедям Моим,~--- то сей не умрет за беззаконие отца своего; он будет жив.
\vs Eze 18:18 А отец его, так как он жестоко притеснял, грабил брата и недоброе делал среди народа своего, вот, он умрет за свое беззаконие.
\vs Eze 18:19 Вы говорите: <<почему же сын не несет вины отца своего?>> Потому что сын поступает законно и праведно, все уставы Мои соблюдает и исполняет их; он будет жив.
\vs Eze 18:20 Душа согрешающая, она умрет; сын не понесет вины отца, и отец не понесет вины сына, правда праведного при нем и остается, и беззаконие беззаконного при нем и остается.
\vs Eze 18:21 И беззаконник, если обратится от всех грехов своих, какие делал, и будет соблюдать все уставы Мои и поступать законно и праведно, жив будет, не умрет.
\vs Eze 18:22 Все преступления его, какие делал он, не припомнятся ему: в правде своей, которую будет делать, он жив будет.
\vs Eze 18:23 Разве Я хочу смерти беззаконника? говорит Господь Бог. Не того ли, чтобы он обратился от путей своих и был жив?
\vs Eze 18:24 И праведник, если отступит от правды своей и будет поступать неправедно, будет делать все те мерзости, какие делает беззаконник, будет ли он жив? все добрые дела его, какие он делал, не припомнятся; за беззаконие свое, какое делает, и за грехи свои, в каких грешен, он умрет.
\vs Eze 18:25 Но вы говорите: <<неправ путь Господа!>> Послушайте, дом Израилев! Мой ли путь неправ? не ваши ли пути неправы?
\vs Eze 18:26 Если праведник отступает от правды своей и делает беззаконие и за то умирает, то он умирает за беззаконие свое, которое сделал.
\vs Eze 18:27 И беззаконник, если обращается от беззакония своего, какое делал, и творит суд и правду,~--- к жизни возвратит душу свою.
\vs Eze 18:28 Ибо он увидел и обратился от всех преступлений своих, какие делал; он будет жив, не умрет.
\vs Eze 18:29 А дом Израилев говорит: <<неправ путь Господа!>> Мои ли пути неправы, дом Израилев? не ваши ли пути неправы?
\vs Eze 18:30 Посему Я буду судить вас, дом Израилев, каждого по путям его, говорит Господь Бог; покайтесь и обратитесь от всех преступлений ваших, чтобы нечестие не было вам преткновением.
\vs Eze 18:31 Отвергните от себя все грехи ваши, которыми согрешали вы, и сотворите себе новое сердце и новый дух; и зачем вам умирать, дом Израилев?
\vs Eze 18:32 Ибо Я не хочу смерти умирающего, говорит Господь Бог; но обратитесь, и живите!
\vs Eze 19:1 А ты подними плач о князьях Израиля
\vs Eze 19:2 и скажи: что за львица мать твоя? расположилась среди львов, между молодыми львами растила львенков своих.
\vs Eze 19:3 И вскормила одного из львенков своих; он сделался молодым львом и научился ловить добычу, ел людей.
\vs Eze 19:4 И услышали о нем народы; он пойман был в яму их, и в цепях отвели его в землю Египетскую.
\vs Eze 19:5 И когда, пождав, увидела она, что надежда ее пропала, тогда взяла другого из львенков своих и сделала его молодым львом.
\vs Eze 19:6 И, сделавшись молодым львом, он стал ходить между львами и научился ловить добычу, ел людей
\vs Eze 19:7 и осквернял вдов их и города их опустошал; и опустела земля и все селения ее от рыкания его.
\vs Eze 19:8 Тогда восстали на него народы из окрестных областей и раскинули на него сеть свою; он пойман был в яму их.
\vs Eze 19:9 И посадили его в клетку на цепи и отвели его к царю Вавилонскому; отвели его в крепость, чтобы не слышен уже был голос его на горах Израилевых.
\vs Eze 19:10 Твоя мать была, как виноградная лоза, посаженная у воды; плодовита и ветвиста была она от обилия воды.
\vs Eze 19:11 И были у нее ветви крепкие для скипетров властителей, и высоко поднялся ствол ее между густыми ветвями; и выдавалась она высотою своею со множеством ветвей своих.
\vs Eze 19:12 Но во гневе вырвана, брошена на землю, и восточный ветер иссушил плод ее; отторжены и иссохли крепкие ветви ее, огонь пожрал их.
\vs Eze 19:13 А теперь она пересажена в пустыню, в землю сухую и жаждущую.
\vs Eze 19:14 И вышел огонь из ствола ветвей ее, пожрал плоды ее и не осталось на ней ветвей крепких для скипетра властителя. Это плачевная песнь, и останется для плача.
\vs Eze 20:1 В седьмом году, в пятом \bibemph{месяце}, в десятый день месяца, пришли мужи из старейшин Израилевых вопросить Господа и сели перед лицем моим.
\vs Eze 20:2 И было ко мне слово Господне:
\vs Eze 20:3 сын человеческий! говори со старейшинами Израилевыми и скажи им: так говорит Господь Бог: вы пришли вопросить Меня? Живу Я, не дам вам ответа, говорит Господь Бог.
\vs Eze 20:4 Хочешь ли судиться с ними, хочешь ли судиться, сын человеческий? выскажи им мерзости отцов их
\vs Eze 20:5 и скажи им: так говорит Господь Бог: в тот день, когда Я избрал Израиля и, подняв руку Мою, \bibemph{поклялся} племени дома Иаковлева, и открыл Себя им в земле Египетской, и, подняв руку, сказал им: <<Я Господь Бог ваш!>>~---
\vs Eze 20:6 в тот день, подняв руку Мою, Я поклялся им вывести их из земли Египетской в землю, которую Я усмотрел для них, текущую молоком и медом, красу всех земель,
\vs Eze 20:7 и сказал им: отвергните каждый мерзости от очей ваших и не оскверняйте себя идолами Египетскими: Я Господь Бог ваш.
\vs Eze 20:8 Но они возмутились против Меня и не хотели слушать Меня; никто не отверг мерзостей от очей своих и не оставил идолов Египетских. И Я сказал: изолью на них гнев Мой, истощу на них ярость Мою среди земли Египетской.
\vs Eze 20:9 Но Я поступил ради имени Моего, чтобы оно не хулилось перед народами, среди которых находились они и перед глазами которых Я открыл Себя им, чтобы вывести их из земли Египетской.
\vs Eze 20:10 И Я вывел их из земли Египетской и привел их в пустыню,
\vs Eze 20:11 и дал им заповеди Мои, и объявил им Мои постановления, исполняя которые человек жив был бы через них;
\vs Eze 20:12 дал им также субботы Мои, чтобы они были знамением между Мною и ими, чтобы знали, что Я Господь, освящающий их.
\vs Eze 20:13 Но дом Израилев возмутился против Меня в пустыне: по заповедям Моим не поступали и отвергли постановления Мои, исполняя которые человек жив был бы через них, и субботы Мои нарушали, и Я сказал: изолью на них ярость Мою в пустыне, чтобы истребить их.
\vs Eze 20:14 Но Я поступил ради имени Моего, чтобы оно не хулилось перед народами, в глазах которых Я вывел их.
\vs Eze 20:15 Даже Я, подняв руку Мою против них в пустыне, \bibemph{поклялся}, что не введу их в землю, которую Я назначил,~--- текущую молоком и медом, красу всех земель,~---
\vs Eze 20:16 за то, что они отвергли постановления Мои, и не поступали по заповедям Моим, и нарушали субботы Мои; ибо сердце их стремилось к идолам их.
\vs Eze 20:17 Но око Мое пожалело погубить их; и Я не истребил их в пустыне.
\vs Eze 20:18 И говорил Я сыновьям их в пустыне: не ходите по правилам отцов ваших, и не соблюдайте установлений их, и не оскверняйте себя идолами их.
\vs Eze 20:19 Я Господь Бог ваш: по Моим заповедям поступайте, и Мои уставы соблюдайте, и исполняйте их.
\vs Eze 20:20 И святите субботы Мои, чтобы они были знамением между Мною и вами, дабы вы знали, что Я Господь Бог ваш.
\vs Eze 20:21 Но и сыновья возмутились против Меня: по заповедям Моим не поступали и уставов Моих не соблюдали, не исполняли того, что исполняя, человек был бы жив, нарушали субботы Мои,~--- и Я сказал: изолью на них гнев Мой, истощу над ними ярость Мою в пустыне;
\vs Eze 20:22 но Я отклонил руку Мою и поступил ради имени Моего, чтобы оно не хулилось перед народами, перед глазами которых Я вывел их.
\vs Eze 20:23 Также, подняв руку Мою в пустыне, Я \bibemph{поклялся} рассеять их по народам и развеять их по землям
\vs Eze 20:24 за то, что они постановлений Моих не исполняли и заповеди Мои отвергли, и нарушали субботы мои, и глаза их обращались к идолам отцов их.
\vs Eze 20:25 И попустил им учреждения недобрые и постановления, от которых они не могли быть живы,
\vs Eze 20:26 и попустил им оскверниться жертвоприношениями их, когда они стали проводить через огонь всякий первый плод утробы, чтобы разорить их, дабы знали, что Я Господь.
\vs Eze 20:27 Посему говори дому Израилеву, сын человеческий, и скажи им: так говорит Господь Бог: вот чем еще хулили Меня отцы ваши, вероломно поступая против Меня:
\vs Eze 20:28 Я привел их в землю, которую клятвенно обещал дать им, подняв руку Мою,~--- а они, высмотрев себе всякий высокий холм и всякое ветвистое дерево, стали заколать там жертвы свои, и ставили там оскорбительные для Меня приношения свои и благовонные курения свои, и возливали там возлияния свои.
\vs Eze 20:29 И Я говорил им: что это за высота, куда ходите вы? поэтому именем Бама называется она и до сего дня.
\vs Eze 20:30 Посему скажи дому Израилеву: так говорит Господь Бог: не оскверняете ли вы себя по примеру отцов ваших и не блудодействуете ли вслед мерзостей их?
\vs Eze 20:31 Принося дары ваши и проводя сыновей ваших через огонь, вы оскверняете себя всеми идолами вашими до сего дня, и хотите вопросить Меня, дом Израилев? живу Я, говорит Господь Бог, не дам вам ответа.
\vs Eze 20:32 И что приходит вам на ум, совсем не сбудется. Вы говорите: <<будем, как язычники, как племена иноземные, служить дереву и камню>>.
\vs Eze 20:33 Живу Я, говорит Господь Бог: рукою крепкою и мышцею простертою и излиянием ярости буду господствовать над вами.
\vs Eze 20:34 И выведу вас из народов и из стран, по которым вы рассеяны, и соберу вас рукою крепкою и мышцею простертою и излиянием ярости.
\vs Eze 20:35 И приведу вас в пустыню народов, и там буду судиться с вами лицом к лицу.
\vs Eze 20:36 Как Я судился с отцами вашими в пустыне земли Египетской, так буду судиться с вами, говорит Господь Бог.
\vs Eze 20:37 И проведу вас под жезлом и введу вас в узы завета.
\vs Eze 20:38 И выделю из вас мятежников и непокорных Мне. Из земли пребывания их выведу их, но в землю Израилеву они не войдут, и узнаете, что Я Господь.
\vs Eze 20:39 А вы, дом Израилев,~--- так говорит Господь Бог,~--- идите каждый к своим идолам и служите им, если Меня не слушаете, но не оскверняйте более святаго имени Моего дарами вашими и идолами вашими,
\vs Eze 20:40 потому что на Моей святой горе, на горе высокой Израилевой,~--- говорит Господь Бог,~--- там будет служить Мне весь дом Израилев,~--- весь, сколько ни есть его на земле; там Я с благоволением приму их, и там потребую приношений ваших и начатков ваших со всеми святынями вашими.
\vs Eze 20:41 Приму вас, как благовонное курение, когда выведу вас из народов и соберу вас из стран, по которым вы рассеяны, и буду святиться в вас перед глазами народов.
\vs Eze 20:42 И узнаете, что Я Господь, когда введу вас в землю Израилеву,~--- в землю, которую Я \bibemph{клялся} дать отцам вашим, подняв руку Мою.
\vs Eze 20:43 И вспомните там о путях ваших и обо всех делах ваших, какими вы оскверняли себя, и возгнушаетесь самими собою за все злодеяния ваши, какие вы делали.
\vs Eze 20:44 И узнаете, что Я Господь, когда буду поступать с вами ради имени Моего, не по злым вашим путям и вашим делам развратным, дом Израилев,~--- говорит Господь Бог.
\rsbpar\vs Eze 20:45 И было ко мне слово Господне:
\vs Eze 20:46 сын человеческий! обрати лице твое на путь к полудню, и произнеси слово на полдень, и изреки пророчество на лес южного поля.
\vs Eze 20:47 И скажи южному лесу: слушай слово Господа; так говорит Господь Бог: вот, Я зажгу в тебе огонь, и он пожрет в тебе всякое дерево зеленеющее и всякое дерево сухое; не погаснет пылающий пламень, и все будет опалено им от юга до севера.
\vs Eze 20:48 И увидит всякая плоть, что Я, Господь, зажег его, и он не погаснет.
\vs Eze 20:49 И сказал я: о, Господи Боже! они говорят обо мне: <<не говорит ли он притчи?>>
\vs Eze 21:1 И было ко мне слово Господне:
\vs Eze 21:2 сын человеческий! обрати лице твое к Иерусалиму и произнеси слово на святилища, и изреки пророчество на землю Израилеву,
\vs Eze 21:3 и скажи земле Израилевой: так говорит Господь Бог: вот, Я~--- на тебя, и извлеку меч Мой из ножен его и истреблю у тебя праведного и нечестивого.
\vs Eze 21:4 А для того, чтобы истребить у тебя праведного и нечестивого, меч Мой из ножен своих пойдет на всякую плоть от юга до севера.
\vs Eze 21:5 И узнает всякая плоть, что Я, Господь, извлек меч Мой из ножен его, и он уже не возвратится.
\vs Eze 21:6 Ты же, сын человеческий, стенай, сокрушая бедра твои, и в горести стенай перед глазами их.
\vs Eze 21:7 И когда скажут тебе: <<отчего ты стенаешь?>>, скажи: <<от слуха, что идет>>,~--- и растает всякое сердце, и все руки опустятся, и всякий дух изнеможет, и все колени задрожат, как вода. Вот, это придет и сбудется, говорит Господь Бог.
\rsbpar\vs Eze 21:8 И было ко мне слово Господне:
\vs Eze 21:9 сын человеческий! изреки пророчество и скажи: так говорит Господь Бог: скажи: меч, меч наострен и вычищен;
\vs Eze 21:10 наострен для того, чтобы больше заколать; вычищен, чтобы сверкал, как молния. Радоваться ли нам, что жезл сына Моего презирает всякое дерево?
\vs Eze 21:11 Я дал его вычистить, чтобы взять в руку; уже наострен этот меч и вычищен, чтобы отдать его в руку убийцы.
\vs Eze 21:12 Стенай и рыдай, сын человеческий, ибо он~--- на народ Мой, на всех князей Израиля; они отданы будут под меч с народом Моим; посему ударяй себя по бедрам.
\vs Eze 21:13 Ибо он уже испытан. И что, если он презирает и жезл? сей не устоит, говорит Господь Бог.
\vs Eze 21:14 Ты же, сын человеческий, пророчествуй и ударяй рукою об руку; и удвоится меч и утроится, меч на поражаемых, меч на поражение великого, проникающий во внутренность жилищ их.
\vs Eze 21:15 Чтобы растаяли сердца и чтобы павших было более, Я у всех ворот их поставлю грозный меч, увы! сверкающий, как молния, наостренный для заклания.
\vs Eze 21:16 Соберись и иди направо или иди налево, куда бы ни обратилось лице твое.
\vs Eze 21:17 И Я буду рукоплескать и утолю гнев Мой; Я, Господь, сказал.
\vs Eze 21:18 И было ко мне слово Господне:
\vs Eze 21:19 и ты, сын человеческий, представь себе две дороги, по которым должно идти мечу царя Вавилонского,~--- обе они должны выходить из одной земли; и начертай руку, начертай при начале дорог в города.
\vs Eze 21:20 Представь дорогу, по которой меч шел бы в Равву сынов Аммоновых и в Иудею, в укрепленный Иерусалим;
\vs Eze 21:21 потому что царь Вавилонский остановился на распутье, при начале двух дорог, для гаданья: трясет стрелы, вопрошает терафимов, рассматривает печень.
\vs Eze 21:22 В правой руке у него гаданье: <<в Иерусалим>>, где должно поставить тараны, открыть для побоища уста, возвысить голос для военного крика, подвести тараны к воротам, насыпать вал, построить осадные башни.
\vs Eze 21:23 Это гаданье показалось в глазах их лживым; но так как они клялись клятвою, то он, вспомнив о таком их вероломстве, положил взять его.
\vs Eze 21:24 Посему так говорит Господь Бог: так как вы сами приводите на память беззаконие ваше, делая явными преступления ваши, выставляя на вид грехи ваши во всех делах ваших, и сами приводите это на память, то вы будете взяты руками.
\vs Eze 21:25 И ты, недостойный, преступный вождь Израиля, которого день наступил ныне, когда нечестию его положен будет конец!
\vs Eze 21:26 так говорит Господь Бог: сними с себя диадему и сложи венец; этого уже не будет; униженное возвысится и высокое унизится.
\vs Eze 21:27 Низложу, низложу, низложу и его не будет, доколе не придет Тот, Кому \bibemph{принадлежит} он, и Я дам Ему.
\vs Eze 21:28 И ты, сын человеческий, изреки пророчество и скажи: так говорит Господь Бог о сынах Аммона и о поношении их; и скажи: меч, меч обнажен для заклания, вычищен для истребления, чтобы сверкал, как молния,
\vs Eze 21:29 чтобы, тогда как представляют тебе пустые видения и ложно гадают тебе, и тебя приложил к обезглавленным нечестивцам, которых день наступил, когда нечестию их положен будет конец.
\vs Eze 21:30 Возвратить ли его в ножны его?~--- на месте, где ты сотворен, на земле происхождения твоего буду судить тебя:
\vs Eze 21:31 и изолью на тебя негодование Мое, дохну на тебя огнем ярости Моей и отдам тебя в руки людей свирепых, опытных в убийстве.
\vs Eze 21:32 Ты будешь пищею огню, кровь твоя останется на земле; не будут и вспоминать о тебе; ибо Я, Господь, сказал это.
\vs Eze 22:1 И было ко мне слово Господне:
\vs Eze 22:2 и ты, сын человеческий, хочешь ли судить, судить город кровей? выскажи ему все мерзости его.
\vs Eze 22:3 И скажи: так говорит Господь Бог: о, город, проливающий кровь среди себя, чтобы наступило время твое, и делающий у себя идолов, чтобы осквернять себя!
\vs Eze 22:4 Кровью, которую ты пролил, ты сделал себя виновным, и идолами, каких ты наделал, ты осквернил себя, и приблизил дни твои и достиг годины твоей. За это отдам тебя на посмеяние народам, на поругание всем землям.
\vs Eze 22:5 Близкие и далекие от тебя будут ругаться над тобою, осквернившим имя твое, прославившимся буйством.
\vs Eze 22:6 Вот, начальствующие у Израиля, каждый по мере сил своих, были у тебя, чтобы проливать кровь.
\vs Eze 22:7 У тебя отца и мать злословят, пришельцу делают обиду среди тебя, сироту и вдову притесняют у тебя.
\vs Eze 22:8 Святынь Моих ты не уважаешь и субботы Мои нарушаешь.
\vs Eze 22:9 Клеветники находятся в тебе, чтобы проливать кровь, и на горах едят у тебя \bibemph{идоложертвенное}, среди тебя производят гнусность.
\vs Eze 22:10 Наготу отца открывают у тебя, жену во время очищения нечистот ее насилуют у тебя.
\vs Eze 22:11 Иной делает мерзость с женою ближнего своего, иной оскверняет сноху свою, иной насилует сестру свою, дочь отца своего.
\vs Eze 22:12 Взятки берут у тебя, чтобы проливать кровь; ты берешь рост и лихву и насилием вымогаешь корысть у ближнего твоего, а Меня забыл, говорит Господь Бог.
\vs Eze 22:13 И вот, Я всплеснул руками Моими о корыстолюбии твоем, какое обнаруживается у тебя, и о кровопролитии, которое совершается среди тебя.
\vs Eze 22:14 Устоит ли сердце твое, будут ли тверды руки твои в те дни, в которые буду действовать против тебя? Я, Господь, сказал и сделаю.
\vs Eze 22:15 И рассею тебя по народам, и развею тебя по землям, и положу конец мерзостям твоим среди тебя.
\vs Eze 22:16 И сделаешь сам себя презренным перед глазами народов, и узнаешь, что Я Господь.
\vs Eze 22:17 И было ко мне слово Господне:
\vs Eze 22:18 сын человеческий! дом Израилев сделался у Меня изгарью; все они~--- олово, медь и железо и свинец в горниле, сделались, как изгарь серебра.
\vs Eze 22:19 Посему так говорит Господь Бог: так как все вы сделались изгарью, за то вот, Я соберу вас в Иерусалим.
\vs Eze 22:20 Как в горнило кладут вместе серебро, и медь, и железо, и свинец, и олово, чтобы раздуть на них огонь и расплавить; так Я во гневе Моем и в ярости Моей соберу, и положу, и расплавлю вас.
\vs Eze 22:21 Соберу вас и дохну на вас огнем негодования Моего, и расплавитесь среди него.
\vs Eze 22:22 Как серебро расплавляется в горниле, так расплавитесь и вы среди него, и узнаете, что Я, Господь, излил ярость Мою на вас.
\vs Eze 22:23 И было ко мне слово Господне:
\vs Eze 22:24 сын человеческий! скажи ему: ты~--- земля неочищенная, не орошаемая дождем в день гнева!
\vs Eze 22:25 Заговор пророков ее среди нее~--- как лев рыкающий, терзающий добычу; съедают души, обирают имущество и драгоценности, и умножают число вдов.
\vs Eze 22:26 Священники ее нарушают закон Мой и оскверняют святыни Мои, не отделяют святаго от несвятаго и не указывают различия между чистым и нечистым, и от суббот Моих они закрыли глаза свои, и Я уничижен у них.
\vs Eze 22:27 Князья у нее как волки, похищающие добычу; проливают кровь, губят души, чтобы приобрести корысть.
\vs Eze 22:28 А пророки ее всё замазывают грязью, видят пустое и предсказывают им ложное, говоря: <<так говорит Господь Бог>>, тогда как не говорил Господь.
\vs Eze 22:29 А в народе угнетают друг друга, грабят и притесняют бедного и нищего, и пришельца угнетают несправедливо.
\vs Eze 22:30 Искал Я у них человека, который поставил бы стену и стал бы предо Мною в проломе за сию землю, чтобы Я не погубил ее, но не нашел.
\vs Eze 22:31 Итак изолью на них негодование Мое, огнем ярости Моей истреблю их, поведение их обращу им на голову, говорит Господь Бог.
\vs Eze 23:1 И было ко мне слово Господне:
\vs Eze 23:2 сын человеческий! были две женщины, дочери одной матери,
\vs Eze 23:3 и блудили они в Египте, блудили в своей молодости; там измяты груди их, и там растлили девственные сосцы их.
\vs Eze 23:4 Имена им: большой~--- Огола, а сестре ее~--- Оголива. И были они Моими, и рождали сыновей и дочерей; и именовались~--- Огола Самариею, а Оголива Иерусалимом.
\vs Eze 23:5 И стала Огола блудить от Меня и пристрастилась к своим любовникам, к Ассириянам, к соседям своим,
\vs Eze 23:6 к одевавшимся в ткани яхонтового цвета, к областеначальникам и градоправителям, ко всем красивым юношам, всадникам, ездящим на конях;
\vs Eze 23:7 и расточала блудодеяния свои со всеми отборными из сынов Ассура, и оскверняла себя всеми идолами тех, к кому ни пристращалась;
\vs Eze 23:8 не переставала блудить и с Египтянами, потому что они с нею спали в молодости ее и растлевали девственные сосцы ее, и изливали на нее похоть свою.
\vs Eze 23:9 За то Я и отдал ее в руки любовников ее, в руки сынов Ассура, к которым она пристрастилась.
\vs Eze 23:10 Они открыли наготу ее, взяли сыновей ее и дочерей ее, а ее убили мечом. И она сделалась позором между женщинами, когда совершили над нею казнь.
\vs Eze 23:11 Сестра ее, Оголива, видела это, и еще развращеннее была в любви своей, и блужение ее превзошло блужение сестры ее.
\vs Eze 23:12 Она пристрастилась к сынам Ассуровым, к областеначальникам и градоправителям, соседям ее, пышно одетым, к всадникам, ездящим на конях, ко всем отборным юношам.
\vs Eze 23:13 И Я видел, что она осквернила себя, \bibemph{и что} у обеих их одна дорога.
\vs Eze 23:14 Но эта еще умножила блудодеяния свои, потому что, увидев вырезанных на стене мужчин, красками нарисованные изображения Халдеев,
\vs Eze 23:15 опоясанных по чреслам своим поясом, с роскошными на голове их повязками, имеющих вид военачальников, похожих на сынов Вавилона, которых родина земля Халдейская,
\vs Eze 23:16 она влюбилась в них по одному взгляду очей своих и послала к ним в Халдею послов.
\vs Eze 23:17 И пришли к ней сыны Вавилона на любовное ложе, и осквернили ее блудодейством своим, и она осквернила себя ими; и отвратилась от них душа ее.
\vs Eze 23:18 Когда же она явно предалась блудодеяниям своим и открыла наготу свою, тогда и от нее отвратилась душа Моя, как отвратилась душа Моя от сестры ее.
\vs Eze 23:19 И она умножала блудодеяния свои, вспоминая дни молодости своей, когда блудила в земле Египетской;
\vs Eze 23:20 и пристрастилась к любовникам своим, у которых плоть~--- плоть ослиная, и похоть, как у жеребцов.
\vs Eze 23:21 Так ты вспомнила распутство молодости твоей, когда Египтяне жали сосцы твои из-за девственных грудей твоих.
\vs Eze 23:22 Посему, Оголива, так говорит Господь Бог: вот, Я возбужу против тебя любовников твоих, от которых отвратилась душа твоя, и приведу их против тебя со всех сторон:
\vs Eze 23:23 сынов Вавилона и всех Халдеев, из Пехода, из Шоа и Коа, и с ними всех сынов Ассура, красивых юношей, областеначальников и градоправителей, сановных и именитых, всех искусных наездников.
\vs Eze 23:24 И придут на тебя с оружием, с конями и колесницами и с множеством народа, и обступят тебя кругом в латах, со щитами и в шлемах, и отдам им тебя на суд, и будут судить тебя своим судом.
\vs Eze 23:25 И обращу ревность Мою против тебя, и поступят с тобою яростно: отрежут у тебя нос и уши, а остальное твое от меча падет; возьмут сыновей твоих и дочерей твоих, а остальное твое огнем будет пожрано;
\vs Eze 23:26 и снимут с тебя одежды твои, возьмут наряды твои.
\vs Eze 23:27 И положу конец распутству твоему и блужению твоему, принесенному из земли Египетской, и не будешь обращать к ним глаз твоих, и о Египте уже не вспомнишь.
\vs Eze 23:28 Ибо так говорит Господь Бог: вот, Я предаю тебя в руки тех, которых ты возненавидела, в руки тех, от которых отвратилась душа твоя.
\vs Eze 23:29 И поступят с тобою жестоко, и возьмут у тебя все, нажитое трудами, и оставят тебя нагою и непокрытою, и открыта будет срамная нагота твоя, и распутство твое, и блудодейство твое.
\vs Eze 23:30 Это будет сделано с тобою за блудодейство твое с народами, которых идолами ты осквернила себя.
\vs Eze 23:31 Ты ходила дорогою сестры твоей; за то и дам в руку тебе чашу ее.
\vs Eze 23:32 Так говорит Господь Бог: ты будешь пить чашу сестры твоей, глубокую и широкую, и подвергнешься посмеянию и позору, по огромной вместительности ее.
\vs Eze 23:33 Опьянения и горести будешь исполнена: чаша ужаса и опустошения~--- чаша сестры твоей, Самарии!
\vs Eze 23:34 И выпьешь ее, и осушишь, и черепки ее оближешь, и груди твои истерзаешь: ибо Я сказал это, говорит Господь Бог.
\vs Eze 23:35 Посему так говорит Господь Бог: так как ты забыла Меня и отвратилась от Меня, то и терпи за беззаконие твое и за блудодейство твое.
\vs Eze 23:36 И сказал мне Господь: сын человеческий! хочешь ли судить Оголу и Оголиву? выскажи им мерзости их;
\vs Eze 23:37 ибо они прелюбодействовали, и кровь на руках их, и с идолами своими прелюбодействовали, и сыновей своих, которых родили Мне, через огонь проводили в пищу им.
\vs Eze 23:38 Еще вот что они делали Мне: оскверняли святилище Мое в тот же день, и нарушали субботы Мои;
\vs Eze 23:39 потому что, когда они заколали детей своих для идолов своих, в тот же день приходили в святилище Мое, чтобы осквернять его: вот как поступали они в доме Моем!
\vs Eze 23:40 Кроме сего посылали за людьми, приходившими издалека; к ним отправляли послов, и вот, они приходили, и ты для них умывалась, сурьмила глаза твои и украшалась нарядами,
\vs Eze 23:41 и садились на великолепное ложе, перед которым приготовляем был стол и на котором предлагала ты благовонные курения Мои и елей Мой.
\vs Eze 23:42 И раздавался голос народа, ликовавшего у нее, и к людям из толпы народной вводимы были пьяницы из пустыни; и они возлагали на руки их запястья и на головы их красивые венки.
\vs Eze 23:43 Тогда сказал Я об одряхлевшей в прелюбодействе: теперь кончатся блудодеяния ее вместе с нею.
\vs Eze 23:44 Но приходили к ней, как приходят к жене блуднице, так приходили к Оголе и Оголиве, к распутным женам.
\vs Eze 23:45 Но мужи праведные будут судить их; они будут судить их судом прелюбодейц и судом проливающих кровь, потому что они прелюбодейки, и у них кровь на руках.
\vs Eze 23:46 Ибо так сказал Господь Бог: созвать на них собрание и предать их озлоблению и грабежу.
\vs Eze 23:47 И собрание побьет их камнями, и изрубит их мечами своими, и убьет сыновей их и дочерей их, и домы их сожжет огнем.
\vs Eze 23:48 Так положу конец распутству на сей земле, и все женщины примут урок, и не будут делать срамных дел подобно вам;
\vs Eze 23:49 и возложат на вас ваше распутство, и понесете наказание за грехи с идолами вашими, и узнаете, что Я Господь Бог.
\vs Eze 24:1 И было ко мне слово Господне в девятом году, в десятом месяце, в десятый день месяца:
\vs Eze 24:2 сын человеческий! запиши себе имя этого дня, этого самого дня: в этот самый день царь Вавилонский подступит к Иерусалиму.
\vs Eze 24:3 И произнеси на мятежный дом притчу, и скажи им: так говорит Господь Бог: поставь котел, поставь и налей в него воды;
\vs Eze 24:4 сложи в него куски мяса, все лучшие куски, бедра и плеча, и наполни отборными костями;
\vs Eze 24:5 отборных овец возьми, и \bibemph{разожги} под ним кости, и кипяти до того, чтобы и кости разварились в нем.
\vs Eze 24:6 Посему так говорит Господь Бог: горе городу кровей! горе котлу, в котором есть накипь и с которого накипь его не сходит! кусок за куском его выбрасывайте из него, не выбирая по жребию.
\vs Eze 24:7 Ибо кровь его среди него; он оставил ее на голой скале; не на землю проливал ее, где она могла бы покрыться пылью.
\vs Eze 24:8 Чтобы возбудить гнев для совершения мщения, Я оставил кровь его на голой скале, чтобы она не скрылась.
\vs Eze 24:9 Посему так говорит Господь Бог: горе городу кровей! и Я разложу большой костер.
\vs Eze 24:10 Прибавь дров, разведи огонь, вывари мясо; пусть все сгустится, и кости перегорят.
\vs Eze 24:11 И когда котел будет пуст, поставь его на уголья, чтобы он разгорелся, и чтобы медь его раскалилась, и расплавилась в нем нечистота его, и вся накипь его исчезла.
\vs Eze 24:12 Труд будет тяжелый; но большая накипь его не сойдет с него; и в огне \bibemph{останется} на нем накипь его.
\vs Eze 24:13 В нечистоте твоей такая мерзость, что, сколько Я ни чищу тебя, ты все нечист; от нечистоты твоей ты и впредь не очистишься, доколе ярости Моей Я не утолю над тобою.
\vs Eze 24:14 Я Господь, Я говорю: это придет и Я сделаю; не отменю и не пощажу, и не помилую. По путям твоим и по делам твоим будут судить тебя, говорит Господь Бог.
\rsbpar\vs Eze 24:15 И было ко мне слово Господне:
\vs Eze 24:16 сын человеческий! вот, Я возьму у тебя язвою утеху очей твоих; но ты не сетуй и не плачь, и слезы да не выступают у тебя;
\vs Eze 24:17 вздыхай в безмолвии, плача по умершим не совершай; но обвязывай себя повязкою и обувай ноги твои в обувь твою, и бород\acc{ы} не закрывай, и хлеба от чужих не ешь.
\vs Eze 24:18 И после того, как говорил я поутру слово к народу, вечером умерла жена моя, и на другой день я сделал так, как повелено было мне.
\vs Eze 24:19 И сказал мне народ: не скажешь ли нам, какое для нас значение в том, что ты делаешь?
\vs Eze 24:20 И сказал я им: ко мне было слово Господне:
\vs Eze 24:21 скажи дому Израилеву: так говорит Господь Бог: вот, Я отдам на поругание святилище Мое, опору силы вашей, утеху очей ваших и отраду души вашей, а сыновья ваши и дочери ваши, которых вы оставили, падут от меча.
\vs Eze 24:22 И вы будете делать то же, что делал я; бород\acc{ы} не будете закрывать, и хлеба от чужих не будете есть;
\vs Eze 24:23 и повязки ваши будут на головах ваших, и обувь ваша на ногах ваших; не будете сетовать и плакать, но будете истаявать от грехов ваших и воздыхать друг перед другом.
\vs Eze 24:24 И будет для вас Иезекииль знамением: все, что он делал, и вы будете делать; и когда это сбудется, узнаете, что Я Господь Бог.
\vs Eze 24:25 А что до тебя, сын человеческий, то в тот день, когда Я возьму у них украшение славы их, утеху очей их и отраду души их, сыновей их и дочерей их,~---
\vs Eze 24:26 в тот день придет к тебе спасшийся \bibemph{оттуда}, чтобы подать весть в уши твои.
\vs Eze 24:27 В тот день при этом спасшемся откроются уста твои, и ты будешь говорить, и не останешься уже безмолвным, и будешь знамением для них, и узнают, что Я Господь.
\vs Eze 25:1 И было ко мне слово Господне:
\vs Eze 25:2 сын человеческий! обрати лице твое к сынам Аммоновым и изреки на них пророчество,
\vs Eze 25:3 и скажи сынам Аммоновым: слушайте слово Господа Бога: так говорит Господь Бог: за то, что ты о святилище Моем говоришь: <<а! а!>>, потому что оно поругано,~--- и о земле Израилевой, потому что она опустошена, и о доме Иудином, потому что они пошли в плен,~---
\vs Eze 25:4 за то вот, Я отдам тебя в наследие сынам востока, и построят у тебя овчарни свои, и поставят у тебя шатры свои, и будут есть плоды твои и пить молоко твое.
\vs Eze 25:5 Я сделаю Равву стойлом для верблюдов, и сынов Аммоновых~--- пастухами овец, и узнаете, что Я Господь.
\vs Eze 25:6 Ибо так говорит Господь Бог: за то, что ты рукоплескал и топал ногою, и со всем презрением к земле Израилевой душевно радовался,~---
\vs Eze 25:7 за то вот, Я простру руку Мою на тебя и отдам тебя на расхищение народам, и истреблю тебя из числа народов, и изглажу тебя из числа земель; сокрушу тебя, и узнаешь, что Я Господь.
\vs Eze 25:8 Так говорит Господь Бог: за то, что Моав и Сеир говорят: <<вот и дом Иудин, как все народы!>>,
\vs Eze 25:9 за то вот, Я, \bibemph{начиная} от городов, от всех пограничных городов его, красы земли, от Беф-Иешимофа, Ваалмеона и Кириафаима, открою бок Моава
\vs Eze 25:10 для сынов востока и отдам его в наследие \bibemph{им}, вместе с сынами Аммоновыми, чтобы сыны Аммона не упоминались более среди народов.
\vs Eze 25:11 И над Моавом произведу суд, и узнают, что Я Господь.
\vs Eze 25:12 Так говорит Господь Бог: за то, что Едом жестоко мстил дому Иудину и тяжко согрешил, совершая над ним мщение,
\vs Eze 25:13 за то, так говорит Господь Бог: простру руку Мою на Едома и истреблю у него людей и скот, и сделаю его пустынею; от Фемана до Дедана все падут от меча.
\vs Eze 25:14 И совершу мщение Мое над Едомом рукою народа Моего, Израиля; и они будут действовать в Идумее по Моему гневу и Моему негодованию, и узнают мщение Мое, говорит Господь Бог.
\vs Eze 25:15 Так говорит Господь Бог: за то, что Филистимляне поступили мстительно и мстили с презрением в душе, на погибель, по вечной неприязни,
\vs Eze 25:16 за то, так говорит Господь Бог: вот, Я простру руку Мою на Филистимлян, и истреблю Критян, и уничтожу остаток их на берегу моря;
\vs Eze 25:17 и совершу над ними великое мщение наказаниями яростными; и узнают, что Я Господь, когда совершу над ними Мое мщение.
\vs Eze 26:1 В одиннадцатом году, в первый день первого месяца, было ко мне слово Господне:
\vs Eze 26:2 сын человеческий! за то, что Тир говорит о Иерусалиме: <<а! а! он сокрушен~--- врата народов; он обращается ко мне; наполнюсь; он опустошен>>,~---
\vs Eze 26:3 за то, так говорит Господь Бог: вот, Я~--- на тебя, Тир, и подниму на тебя многие народы, как море поднимает волны свои.
\vs Eze 26:4 И разобьют стены Тира и разрушат башни его; и вымету из него прах его и сделаю его голою скалою.
\vs Eze 26:5 Местом для расстилания сетей будет он среди моря; ибо Я сказал это, говорит Господь Бог: и будет он на расхищение народам.
\vs Eze 26:6 А дочери его, которые на земле, убиты будут мечом, и узнают, что Я Господь.
\vs Eze 26:7 Ибо так говорит Господь Бог: вот, Я приведу против Тира от севера Навуходоносора, царя Вавилонского, царя царей, с конями и с колесницами, и со всадниками, и с войском, и с многочисленным народом.
\vs Eze 26:8 Дочерей твоих на земле он побьет мечом и устроит против тебя осадные башни, и насыплет против тебя вал, и поставит против тебя щиты;
\vs Eze 26:9 и к стенам твоим придвинет стенобитные машины и башни твои разрушит секирами своими.
\vs Eze 26:10 От множества коней его покроет тебя пыль, от шума всадников и колес и колесниц потрясутся стены твои, когда он будет входить в ворота твои, как входят в разбитый город.
\vs Eze 26:11 Копытами коней своих он истопчет все улицы твои, народ твой побьет мечом и памятники могущества твоего повергнет на землю.
\vs Eze 26:12 И разграбят богатство твое, и расхитят товары твои, и разрушат стены твои, и разобьют красивые домы твои, и камни твои и дерева твои, и землю твою бросят в воду.
\vs Eze 26:13 И прекращу шум песней твоих, и звук цитр твоих уже не будет слышен.
\vs Eze 26:14 И сделаю тебя голою скалою, будешь местом для расстилания сетей; не будешь вновь построен: ибо Я, Господь, сказал это, говорит Господь Бог.
\vs Eze 26:15 Так говорит Господь Бог Тиру: от шума падения твоего, от стона раненых, когда будет производимо среди тебя избиение, не содрогнутся ли острова?
\vs Eze 26:16 И сойдут все князья моря с престолов своих, и сложат с себя мантии свои, и снимут с себя узорчатые одежды свои, облекутся в трепет, сядут на землю, и ежеминутно будут содрогаться и изумляться о тебе.
\vs Eze 26:17 И поднимут плач о тебе и скажут тебе: как погиб ты, населенный мореходцами, город знаменитый, который был силен на море, сам и жители его, наводившие страх на всех обитателей его!
\vs Eze 26:18 Ныне, в день падения твоего, содрогнулись острова; острова на море приведены в смятение погибелью твоею.
\vs Eze 26:19 Ибо так говорит Господь Бог: когда Я сделаю тебя городом опустелым, подобным городам необитаемым, когда подниму на тебя пучину, и покроют тебя большие воды;
\vs Eze 26:20 тогда низведу тебя с отходящими в могилу к народу давно бывшему, и помещу тебя в преисподних земли, в пустынях вечных, с отшедшими в могилу, чтобы ты не был более населен; и явлю Я славу на земле живых.
\vs Eze 26:21 Ужасом сделаю тебя, и не будет тебя, и будут искать тебя, но уже не найдут тебя во веки, говорит Господь Бог.
\vs Eze 27:1 И было ко мне слово Господне:
\vs Eze 27:2 и ты, сын человеческий, подними плач о Тире
\vs Eze 27:3 и скажи Тиру, поселившемуся на выступах в море, торгующему с народами на многих островах: так говорит Господь Бог: Тир! ты говоришь: <<я совершенство красоты!>>
\vs Eze 27:4 Пределы твои в сердце морей; строители твои усовершили красоту твою:
\vs Eze 27:5 из Сенирских кипарисов устроили все помосты твои; брали с Ливана кедр, чтобы сделать на тебе мачты;
\vs Eze 27:6 из дубов Васанских делали весла твои; скамьи твои делали из букового дерева, с оправою из слоновой кости с островов Киттимских;
\vs Eze 27:7 узорчатые полотна из Египта употреблялись на паруса твои и служили флагом; голубого и пурпурового цвета ткани с островов Елисы были покрывалом твоим.
\vs Eze 27:8 Жители Сидона и Арвада были у тебя гребцами; свои знатоки были у тебя, Тир; они были у тебя кормчими.
\vs Eze 27:9 Старшие из Гевала и знатоки его были у тебя, чтобы заделывать пробоины твои. Всякие морские корабли и корабельщики их находились у тебя для производства торговли твоей.
\vs Eze 27:10 Перс и Лидиянин и Ливиец находились в войске твоем и были у тебя ратниками, вешали на тебе щит и шлем; они придавали тебе величие.
\vs Eze 27:11 Сыны Арвада с собственным твоим войском стояли кругом на стенах твоих, и Гамадимы были на башнях твоих; кругом по стенам твоим они вешали колчаны свои; они довершали красу твою.
\vs Eze 27:12 Фарсис, торговец твой, по множеству всякого богатства, платил за товары твои серебром, железом, свинцом и оловом.
\vs Eze 27:13 Иаван, Фувал и Мешех торговали с тобою, выменивая товары твои на души человеческие и медную посуду.
\vs Eze 27:14 Из дома Фогарма за товары твои доставляли тебе лошадей и строевых коней и лошаков.
\vs Eze 27:15 Сыны Дедана торговали с тобою; многие острова производили с тобою мену, в уплату тебе доставляли слоновую кость и черное дерево.
\vs Eze 27:16 По причине большого торгового производства твоего торговали с тобою Арамеяне; за товары твои они платили карбункулами, тканями пурпуровыми, узорчатыми, и виссонами, и кораллами, и рубинами.
\vs Eze 27:17 Иудея и земля Израилева торговали с тобою; за товар твой платили пшеницею Миннифскою и сластями, и медом, и деревянным маслом, и бальзамом.
\vs Eze 27:18 Дамаск, по причине большого торгового производства твоего, по изобилию всякого богатства, торговал с тобою вином Хелбонским и белою шерстью.
\vs Eze 27:19 Дан и Иаван из Узала платили тебе за товары твои выделанным железом; кассия и благовонная трость шли на обмен тебе.
\vs Eze 27:20 Дедан торговал с тобою драгоценными попонами для верховой езды.
\vs Eze 27:21 Аравия и все князья Кидарские производили мену с тобою: ягнят и баранов и козлов променивали тебе.
\vs Eze 27:22 Купцы из Савы и Раемы торговали с тобою всякими лучшими благовониями и всякими дорогими камнями, и золотом платили за товары твои.
\vs Eze 27:23 Харан и Хане и Еден, купцы Савейские, Ассур и Хилмад торговали с тобою.
\vs Eze 27:24 Они торговали с тобою драгоценными одеждами, шелковыми и узорчатыми материями, которые они привозили на твои рынки в дорогих ящиках, сделанных из кедра и хорошо упакованных.
\vs Eze 27:25 Фарсисские корабли были твоими караванами в твоей торговле, и ты сделался богатым и весьма славным среди морей.
\vs Eze 27:26 Гребцы твои завели тебя в большие воды; восточный ветер разбил тебя среди морей.
\vs Eze 27:27 Богатство твое и товары твои, все склады твои, корабельщики твои и кормчие твои, заделывавшие пробоины твои и распоряжавшиеся торговлею твоею, и все ратники твои, какие у тебя были, и все множество народа в тебе, в день падения твоего упадет в сердце морей.
\vs Eze 27:28 От вопля кормчих твоих содрогнутся окрестности.
\vs Eze 27:29 И с кораблей своих сойдут все гребцы, корабельщики, все кормчие моря, и станут на землю;
\vs Eze 27:30 и зарыдают о тебе громким голосом, и горько застенают, посыпав пеплом головы свои и валяясь во прахе;
\vs Eze 27:31 и остригут по тебе волосы догола, и опояшутся вретищами, и заплачут о тебе от душевной скорби горьким плачем;
\vs Eze 27:32 и в сетовании своем поднимут плачевную песнь о тебе, и так зарыдают о тебе: <<кто как Тир, так разрушенный посреди моря!
\vs Eze 27:33 Когда приходили с морей товары твои, ты насыщал многие народы; множеством богатства твоего и торговлею твоею обогащал царей земли.
\vs Eze 27:34 А когда ты разбит морями в пучине вод, товары твои и все толпившееся в тебе упало.
\vs Eze 27:35 Все обитатели островов ужаснулись о тебе, и цари их содрогнулись, изменились в лицах.
\vs Eze 27:36 Торговцы других народов свистнули о тебе; ты сделался ужасом,~--- и не будет тебя во веки>>.
\vs Eze 28:1 И было ко мне слово Господне:
\vs Eze 28:2 сын человеческий! скажи начальствующему в Тире: так говорит Господь Бог: за то, что вознеслось сердце твое и ты говоришь: <<я бог, восседаю на седалище божием, в сердце морей>>, и будучи человеком, а не Богом, ставишь ум твой наравне с умом Божиим,~---
\vs Eze 28:3 вот, ты премудрее Даниила, нет тайны, сокрытой от тебя;
\vs Eze 28:4 твоею мудростью и твоим разумом ты приобрел себе богатство и в сокровищницы твои собрал золота и серебра;
\vs Eze 28:5 большою мудростью твоею, посредством торговли твоей, ты умножил богатство твое, и ум твой возгордился богатством твоим,~---
\vs Eze 28:6 за то так говорит Господь Бог: так как ты ум твой ставишь наравне с умом Божиим,
\vs Eze 28:7 вот, Я приведу на тебя иноземцев, лютейших из народов, и они обнажат мечи свои против красы твоей мудрости и помрачат блеск твой;
\vs Eze 28:8 низведут тебя в могилу, и умрешь в сердце морей смертью убитых.
\vs Eze 28:9 Скажешь ли тогда перед твоим убийцею: <<я бог>>, тогда как в руке поражающего тебя ты будешь человек, а не бог?
\vs Eze 28:10 Ты умрешь от руки иноземцев смертью необрезанных; ибо Я сказал это, говорит Господь Бог.
\vs Eze 28:11 И было ко мне слово Господне:
\vs Eze 28:12 сын человеческий! плачь о царе Тирском и скажи ему: так говорит Господь Бог: ты печать совершенства, полнота мудрости и венец красоты.
\vs Eze 28:13 Ты находился в Едеме, в саду Божием; твои одежды были украшены всякими драгоценными камнями; рубин, топаз и алмаз, хризолит, оникс, яспис, сапфир, карбункул и изумруд и золото, все, искусно усаженное у тебя в гнездышках и нанизанное на тебе, приготовлено было в день сотворения твоего.
\vs Eze 28:14 Ты был помазанным херувимом, чтобы осенять, и Я поставил тебя на то; ты был на святой горе Божией, ходил среди огнистых камней.
\vs Eze 28:15 Ты совершен был в путях твоих со дня сотворения твоего, доколе не нашлось в тебе беззакония.
\vs Eze 28:16 От обширности торговли твоей внутреннее твое исполнилось неправды, и ты согрешил; и Я низвергнул тебя, как нечистого, с горы Божией, изгнал тебя, херувим осеняющий, из среды огнистых камней.
\vs Eze 28:17 От красоты твоей возгордилось сердце твое, от тщеславия твоего ты погубил мудрость твою; за то Я повергну тебя на землю, перед царями отдам тебя на позор.
\vs Eze 28:18 Множеством беззаконий твоих в неправедной торговле твоей ты осквернил святилища твои; и Я извлеку из среды тебя огонь, который и пожрет тебя: и Я превращу тебя в пепел на земле перед глазами всех, видящих тебя.
\vs Eze 28:19 Все, знавшие тебя среди народов, изумятся о тебе; ты сделаешься ужасом, и не будет тебя во веки.
\rsbpar\vs Eze 28:20 И было ко мне слово Господне:
\vs Eze 28:21 сын человеческий! обрати лице твое к Сидону и изреки на него пророчество,
\vs Eze 28:22 и скажи: вот, Я~--- на тебя, Сидон, и прославлюсь среди тебя, и узнают, что Я Господь, когда произведу суд над ним и явлю в нем святость Мою;
\vs Eze 28:23 и пошлю на него моровую язву и кровопролитие на улицы его, и падут среди него убитые мечом, пожирающим его отовсюду; и узнают, что Я Господь.
\vs Eze 28:24 И не будет он впредь для дома Израилева колючим терном и причиняющим боль волчцом, более всех соседей зложелательствующим ему, и узнают, что Я Господь Бог.
\vs Eze 28:25 Так говорит Господь Бог: когда Я соберу дом Израилев из народов, между которыми они рассеяны, и явлю в них святость Мою перед глазами племен, и они будут жить на земле своей, которую Я дал рабу Моему Иакову:
\vs Eze 28:26 тогда они будут жить на ней безопасно, и построят домы, и насадят виноградники, и будут жить в безопасности, потому что Я произведу суд над всеми зложелателями их вокруг них, и узнают, что Я Господь Бог их.
\vs Eze 29:1 В десятом году, в десятом \bibemph{месяце}, в двенадцатый \bibemph{день} месяца, было ко мне слово Господне:
\vs Eze 29:2 сын человеческий! обрати лице твое к фараону, царю Египетскому, и изреки пророчество на него и на весь Египет.
\vs Eze 29:3 Говори и скажи: так говорит Господь Бог: вот, Я~--- на тебя, фараон, царь Египетский, большой крокодил, который, лежа среди рек своих, говоришь: <<моя река, и я создал ее для себя>>.
\vs Eze 29:4 Но Я вложу крюк в челюсти твои и к чешуе твоей прилеплю рыб из рек твоих, и вытащу тебя из рек твоих со всею рыбою рек твоих, прилипшею к чешуе твоей;
\vs Eze 29:5 и брошу тебя в пустыне, тебя и всю рыбу из рек твоих, ты упадешь на открытое поле, не уберут и не подберут тебя; отдам тебя на съедение зверям земным и птицам небесным.
\vs Eze 29:6 И узнают все обитатели Египта, что Я Господь; потому что они дому Израилеву были подпорою тростниковою.
\vs Eze 29:7 Когда они ухватились за тебя рукою, ты расщепился и все плечо исколол им; и когда они оперлись о тебя, ты сломился и изранил все чресла им.
\vs Eze 29:8 Посему так говорит Господь Бог: вот, Я наведу на тебя меч, и истреблю у тебя людей и скот.
\vs Eze 29:9 И сделается земля Египетская пустынею и степью; и узнают, что Я Господь. Так как он говорит: <<моя река, и я создал ее>>;
\vs Eze 29:10 то вот, Я~--- на реки твои, и сделаю землю Египетскую пустынею из пустынь от Мигдола до Сиены, до самого предела Ефиопии.
\vs Eze 29:11 Не будет проходить по ней нога человеческая, и нога скотов не будет проходить по ней, и не будут обитать на ней сорок лет.
\vs Eze 29:12 И сделаю землю Египетскую пустынею среди земель опустошенных; и города ее среди опустелых городов будут пустыми сорок лет, и рассею Египтян по народам, и развею их по землям.
\vs Eze 29:13 Ибо так говорит Господь Бог: по окончании сорока лет Я соберу Египтян из народов, между которыми они будут рассеяны;
\vs Eze 29:14 и возвращу плен Египта, и обратно приведу их в землю Пафрос, в землю происхождения их, и там они будут царством слабым.
\vs Eze 29:15 Оно будет слабее \bibemph{других} царств, и не будет более возноситься над народами; Я умалю их, чтобы они не господствовали над народами.
\vs Eze 29:16 И не будут впредь дому Израилеву опорою, припоминающею беззаконие их, когда они обращались к нему; и узнают, что Я Господь Бог.
\rsbpar\vs Eze 29:17 В двадцать седьмом году, в первом \bibemph{месяце}, в первый \bibemph{день} месяца, было ко мне слово Господне:
\vs Eze 29:18 сын человеческий! Навуходоносор, царь Вавилонский, утомил свое войско большими работами при Тире; все головы оплешивели и все плечи стерты; а ни ему, ни войску его нет вознаграждения от Тира за работы, которые он употребил против него.
\vs Eze 29:19 Посему так говорит Господь Бог: вот, Я Навуходоносору, царю Вавилонскому, даю землю Египетскую, чтобы он обобрал богатство ее и произвел грабеж в ней, и ограбил награбленное ею, и это будет вознаграждением войску его.
\vs Eze 29:20 В награду за дело, которое он произвел в нем, Я отдаю ему землю Египетскую, потому что они делали это для Меня, сказал Господь Бог.
\vs Eze 29:21 В тот день возвращу рог дому Израилеву, и тебе открою уста среди них, и узнают, что Я Господь.
\vs Eze 30:1 И было ко мне слово Господне:
\vs Eze 30:2 сын человеческий! изреки пророчество и скажи: так говорит Господь Бог: рыдайте! о, злосчастный день!
\vs Eze 30:3 Ибо близок день, так! близок день Господа, день мрачный; година народов наступает.
\vs Eze 30:4 И пойдет меч на Египет, и ужас распространится в Ефиопии, когда в Египте будут падать пораженные, когда возьмут богатство его, и основания его будут разрушены;
\vs Eze 30:5 Ефиопия и Ливия, и Лидия, и весь смешанный народ, и Хуб, и сыны земли завета вместе с ними падут от меча.
\vs Eze 30:6 Так говорит Господь: падут подпоры Египта, и упадет гордыня могущества его; от Мигдола до Сиены будут падать в нем от меча, сказал Господь Бог.
\vs Eze 30:7 И опустеет он среди опустошенных земель, и города его будут среди опустошенных городов.
\vs Eze 30:8 И узнают, что Я Господь, когда пошлю огонь на Египет, и все подпоры его будут сокрушены.
\vs Eze 30:9 В тот день пойдут от Меня вестники на кораблях, чтобы устрашить беспечных Ефиоплян, и распространится у них ужас, как в день Египта; ибо вот, он идет.
\vs Eze 30:10 Так говорит Господь Бог: положу конец многолюдству Египта рукою Навуходоносора, царя Вавилонского.
\vs Eze 30:11 Он и с ним народ его, лютейший из народов, приведены будут на погибель сей земли, и обнажат мечи свои на Египет, и наполнят землю пораженными.
\vs Eze 30:12 И реки сделаю сушею и предам землю в руки злым, и рукою иноземцев опустошу землю и все, наполняющее ее. Я, Господь, сказал это.
\vs Eze 30:13 Так говорит Господь Бог: истреблю идолов и уничтожу лжебогов в Мемфисе, и из земли Египетской не будет уже властителя, и наведу страх на землю Египетскую.
\vs Eze 30:14 И опустошу Пафрос и пошлю огонь на Цоан, и произведу суд над Но.
\vs Eze 30:15 И изолью ярость Мою на Син, крепость Египта, и истреблю многолюдие в Но.
\vs Eze 30:16 И пошлю огонь на Египет; вострепещет Син, и Но рушится, и на Мемфис нападут враги среди дня.
\vs Eze 30:17 Молодые люди Она и Бубаста падут от меча, а прочие пойдут в плен.
\vs Eze 30:18 И в Тафнисе померкнет день, когда Я сокрушу там ярмо Египта, и прекратится в нем гордое могущество его. Облако закроет его, и дочери его пойдут в плен.
\vs Eze 30:19 Так произведу Я суд над Египтом, и узнают, что Я Господь.
\rsbpar\vs Eze 30:20 В одиннадцатом году, в первом месяце, в седьмой день \bibemph{месяца}, было ко мне слово Господне:
\vs Eze 30:21 сын человеческий! Я уже сокрушил мышцу фараону, царю Египетскому; и вот, она еще не обвязана для излечения ее и не обвита врачебными перевязками, от которых она получила бы силу держать меч.
\vs Eze 30:22 Посему так говорит Господь Бог: вот, Я~--- на фараона, царя Египетского, и сокрушу мышцы его, здоровую и переломленную, так что меч выпадет из руки его.
\vs Eze 30:23 И рассею Египтян по народам, и развею их по землям.
\vs Eze 30:24 А мышцы царя Вавилонского сделаю крепкими и дам ему меч Мой в руку, мышцы же фараона сокрушу, и он изъязвленный будет сильно стонать перед ним.
\vs Eze 30:25 Укреплю мышцы царя Вавилонского, а мышцы у фараона опустятся; и узнают, что Я Господь, когда меч Мой дам в руку царю Вавилонскому, и он прострет его на землю Египетскую.
\vs Eze 30:26 И рассею Египтян по народам, и развею их по землям, и узнают, что Я Господь.
\vs Eze 31:1 В одиннадцатом году, в третьем \bibemph{месяце}, в первый день месяца, было ко мне слово Господне:
\vs Eze 31:2 сын человеческий! скажи фараону, царю Египетскому, и народу его: кому ты равняешь себя в величии твоем?
\vs Eze 31:3 Вот, Ассур был кедр на Ливане, с красивыми ветвями и тенистою листвою, и высокий ростом; вершина его находилась среди толстых сучьев.
\vs Eze 31:4 Воды растили его, бездна поднимала его, реки ее окружали питомник его, и она протоки свои посылала ко всем деревам полевым.
\vs Eze 31:5 Оттого высота его перевысила все дерева полевые, и сучьев на нем было много, и ветви его умножались, и сучья его становились длинными от множества вод, когда он разрастался.
\vs Eze 31:6 На сучьях его вили гнезда всякие птицы небесные, под ветвями его выводили детей всякие звери полевые, и под тенью его жили всякие многочисленные народы.
\vs Eze 31:7 Он красовался высотою роста своего, длиною ветвей своих, ибо корень его был у великих вод.
\vs Eze 31:8 Кедры в саду Божием не затемняли его; кипарисы не равнялись сучьям его, и каштаны не были величиною с ветви его, ни одно дерево в саду Божием не равнялось с ним красотою своею.
\vs Eze 31:9 Я украсил его множеством ветвей его, так что все дерева Едемские в саду Божием завидовали ему.
\vs Eze 31:10 Посему так сказал Господь Бог: за то, что ты высок стал ростом и вершину твою выставил среди толстых сучьев, и сердце его возгордилось величием его,~---
\vs Eze 31:11 за то Я отдал его в руки властителю народов; он поступил с ним, как надобно; за беззаконие его Я отверг его.
\vs Eze 31:12 И срубили его чужеземцы, лютейшие из народов, и повергли его на горы; и на все долины упали ветви его; и сучья его сокрушились на всех лощинах земли, и из-под тени его ушли все народы земли, и оставили его.
\vs Eze 31:13 На обломках его поместились всякие птицы небесные, и в сучьях были всякие полевые звери.
\vs Eze 31:14 Это для того, чтобы никакие дерева при водах не величались высоким ростом своим и не поднимали вершины своей из среды толстых сучьев, и чтобы не прилеплялись к ним из-за высоты их дерева, пьющие воду; ибо все они будут преданы смерти, в преисподнюю страну вместе с сынами человеческими, отшедшими в могилу.
\vs Eze 31:15 Так говорит Господь Бог: в тот день, когда он сошел в могилу, Я сделал сетование о нем, затворил ради него бездну и остановил реки ее, и задержал большие воды и омрачил по нем Ливан, и все дерева полевые были в унынии по нем.
\vs Eze 31:16 Шумом падения его Я привел в трепет народы, когда низвел его в преисподнюю, к отшедшим в могилу, и обрадовались в преисподней стране все дерева Едема, отличные и наилучшие Ливанские, все, пьющие воду;
\vs Eze 31:17 ибо и они с ним отошли в преисподнюю, к пораженным мечом, и союзники его, жившие под тенью его, среди народов.
\vs Eze 31:18 Итак которому из дерев Едемских равнялся ты в славе и величии? Но теперь наравне с деревами Едемскими ты будешь низведен в преисподнюю, будешь лежать среди необрезанных, с пораженными мечом. Это фараон и все множество народа его, говорит Господь Бог.
\vs Eze 32:1 В двенадцатом году, в двенадцатом месяце, в первый \bibemph{день} месяца, было ко мне слово Господне:
\vs Eze 32:2 сын человеческий! подними плач о фараоне, царе Египетском, и скажи ему: ты, как молодой лев между народами и как чудовище в морях, кидаешься в реках твоих, и мутишь ногами твоими воды, и попираешь потоки их.
\vs Eze 32:3 Так говорит Господь Бог: Я закину на тебя сеть Мою в собрании многих народов, и они вытащат тебя Моею мрежею.
\vs Eze 32:4 И выкину тебя на землю, на открытом поле брошу тебя, и будут садиться на тебя всякие небесные птицы, и насыщаться тобою звери всей земли.
\vs Eze 32:5 И раскидаю мясо твое по горам, и долины наполню твоими трупами.
\vs Eze 32:6 И землю плавания твоего напою кровью твоею до самых гор; и рытвины будут наполнены тобою.
\vs Eze 32:7 И когда ты угаснешь, закрою небеса и звезды их помрачу, солнце закрою облаком, и луна не будет светить светом своим.
\vs Eze 32:8 Все светила, светящиеся на небе, помрачу над тобою и на землю твою наведу тьму, говорит Господь Бог.
\vs Eze 32:9 Приведу в смущение сердце многих народов, когда разглашу о падении твоем между народами, по землям, которых ты не знал.
\vs Eze 32:10 И приведу тобою в ужас многие народы, и цари их содрогнутся о тебе в страхе, когда мечом Моим потрясу перед лицем их, и поминутно будут трепетать каждый за душу свою в день падения твоего.
\vs Eze 32:11 Ибо так говорит Господь Бог: меч царя Вавилонского придет на тебя.
\vs Eze 32:12 От мечей сильных падет народ твой; все они~--- лютейшие из народов, и уничтожат гордость Египта, и погибнет все множество его.
\vs Eze 32:13 И истреблю весь скот его при великих водах, и вперед не будет мутить их нога человеческая, и копыта скота не будут мутить их.
\vs Eze 32:14 Тогда дам покой водам их, и сделаю, что реки их потекут, как масло, говорит Господь Бог.
\vs Eze 32:15 Когда сделаю землю Египетскую пустынею, и когда лишится земля всего, наполняющего ее; когда поражу всех живущих на ней, тогда узнают, что Я Господь.
\vs Eze 32:16 Вот плачевная песнь, которую будут петь; дочери народов будут петь ее; о Египте и обо всем множестве его будут петь ее, говорит Господь Бог.
\rsbpar\vs Eze 32:17 В двенадцатом году, в пятнадцатый \bibemph{день того же} месяца, было ко мне слово Господне:
\vs Eze 32:18 сын человеческий! оплачь народ Египетский, и низринь его, его и дочерей знаменитых народов в преисподнюю, с отходящими в могилу.
\vs Eze 32:19 Кого ты превосходишь? сойди, и лежи с необрезанными.
\vs Eze 32:20 Те падут среди убитых мечом, и он отдан мечу; влеките его и все множество его.
\vs Eze 32:21 Среди преисподней будут говорить о нем и о союзниках его первые из героев; они пали и лежат там между необрезанными, сраженные мечом.
\vs Eze 32:22 Там Ассур и все полчище его, вокруг него гробы их, все пораженные, павшие от меча.
\vs Eze 32:23 Гробы его поставлены в самой глубине преисподней, и полчище его вокруг гробницы его, все пораженные, павшие от меча, те, которые распространяли ужас на земле живых.
\vs Eze 32:24 Там Елам со всем множеством своим вокруг гробницы его, все они пораженные, павшие от меча, которые необрезанными сошли в преисподнюю, которые распространили собою ужас на земле живых и несут позор свой с отшедшими в могилу.
\vs Eze 32:25 Среди пораженных дали ложе ему со всем множеством его; вокруг него гробы их, все необрезанные, пораженные мечом; и как они распространяли ужас на земле живых, то и несут на себе позор наравне с отшедшими в могилу и положены среди пораженных.
\vs Eze 32:26 Там Мешех и Фувал со всем множеством своим; вокруг него гробы их, все необрезанные, пораженные мечом, потому что они распространяли ужас на земле живых.
\vs Eze 32:27 Не должны ли \bibemph{и} они лежать с павшими героями необрезанными, которые с воинским оружием своим сошли в преисподнюю и мечи свои положили себе под головы, и осталось беззаконие их на костях их, потому что они, как сильные, были ужасом на земле живых.
\vs Eze 32:28 И ты будешь сокрушен среди необрезанных и лежать с пораженными мечом.
\vs Eze 32:29 Там Едом и цари его и все князья его, которые при всей своей храбрости положены среди пораженных мечом; они лежат с необрезанными и сошедшими в могилу.
\vs Eze 32:30 Там властелины севера, все они и все Сидоняне, которые сошли туда с пораженными, быв посрамлены в могуществе своем, наводившем ужас, и лежат они с необрезанными, пораженными мечом, и несут позор свой с отшедшими в могилу.
\vs Eze 32:31 Увидит их фараон и утешится о всем множестве своем, пораженном мечом, фараон и все войско его, говорит Господь Бог.
\vs Eze 32:32 Ибо Я распространю страх Мой на земле живых, и положен будет фараон и все множество его среди необрезанных с пораженными мечом, говорит Господь Бог.
\vs Eze 33:1 И было ко мне слово Господне:
\vs Eze 33:2 сын человеческий! изреки слово к сынам народа твоего и скажи им: если Я на какую-либо землю наведу меч, и народ той земли возьмет из среды себя человека и поставит его у себя стражем;
\vs Eze 33:3 и он, увидев меч, идущий на землю, затрубит в трубу и предостережет народ;
\vs Eze 33:4 и если кто будет слушать голос трубы, но не остережет себя,~--- то, когда меч придет и захватит его, кровь его будет на его голове.
\vs Eze 33:5 Голос трубы он слышал, но не остерег себя, кровь его на нем будет; а кто остерегся, тот спас жизнь свою.
\vs Eze 33:6 Если же страж видел идущий меч и не затрубил в трубу, и народ не был предостережен,~--- то, когда придет меч и отнимет у кого из них жизнь, сей схвачен будет за грех свой, но кровь его взыщу от руки стража.
\vs Eze 33:7 И тебя, сын человеческий, Я поставил стражем дому Израилеву, и ты будешь слышать из уст Моих слово и вразумлять их от Меня.
\vs Eze 33:8 Когда Я скажу беззаконнику: <<беззаконник! ты смертью умрешь>>, а ты не будешь ничего говорить, чтобы предостеречь беззаконника от пути его,~--- то беззаконник тот умрет за грех свой, но кровь его взыщу от руки твоей.
\vs Eze 33:9 Если же ты остерегал беззаконника от пути его, чтобы он обратился от него, но он от пути своего не обратился,~--- то он умирает за грех свой, а ты спас душу твою.
\vs Eze 33:10 И ты, сын человеческий, скажи дому Израилеву: вы говорите так: <<преступления наши и грехи наши на нас, и мы истаеваем в них: как же можем мы жить?>>
\vs Eze 33:11 Скажи им: живу Я, говорит Господь Бог: не хочу смерти грешника, но чтобы грешник обратился от пути своего и жив был. Обратитесь, обратитесь от злых путей ваших; для чего умирать вам, дом Израилев?
\vs Eze 33:12 И ты, сын человеческий, скажи сынам народа твоего: праведность праведника не спасет в день преступления его, и беззаконник за беззаконие свое не падет в день обращения от беззакония своего, равно как и праведник в день согрешения своего не может остаться в живых за свою праведность.
\vs Eze 33:13 Когда Я скажу праведнику, что он будет жив, а он понадеется на свою праведность и сделает неправду,~--- то все праведные дела его не помянутся, и он умрет от неправды своей, какую сделал.
\vs Eze 33:14 А когда скажу беззаконнику: <<ты смертью умрешь>>, и он обратится от грехов своих и будет творить суд и правду,
\vs Eze 33:15 \bibemph{если} этот беззаконник возвратит залог, за похищенное заплатит, будет ходить по законам жизни, не делая ничего худого,~--- то он будет жив, не умрет.
\vs Eze 33:16 Ни один из грехов его, какие он сделал, не помянется ему; он стал творить суд и правду, он будет жив.
\vs Eze 33:17 А сыны народа твоего говорят: <<неправ путь Господа>>, тогда как их путь неправ.
\vs Eze 33:18 Когда праведник отступил от праведности своей и начал делать беззаконие,~--- то он умрет за то.
\vs Eze 33:19 И когда беззаконник обратился от беззакония своего и стал творить суд и правду, он будет за то жив.
\vs Eze 33:20 А вы говорите: <<неправ путь Господа!>> Я буду судить вас, дом Израилев, каждого по путям его.
\rsbpar\vs Eze 33:21 В двенадцатом году нашего переселения, в десятом \bibemph{месяце}, в пятый \bibemph{день} месяца, пришел ко мне один из спасшихся из Иерусалима и сказал: <<разрушен город!>>
\vs Eze 33:22 Но еще до прихода сего спасшегося вечером была на мне рука Господа, и Он открыл мне уста, прежде нежели тот пришел ко мне поутру. И открылись уста мои, и я уже не был безмолвен.
\vs Eze 33:23 И было ко мне слово Господне:
\vs Eze 33:24 сын человеческий! живущие на опустелых местах в земле Израилевой говорят: <<Авраам был один, и получил во владение землю сию, а нас много; \bibemph{итак} нам дана земля сия во владение>>.
\vs Eze 33:25 Посему скажи им: так говорит Господь Бог: вы едите с кровью и поднимаете глаза ваши к идолам вашим, и проливаете кровь; и хотите владеть землею?
\vs Eze 33:26 Вы опираетесь на меч ваш, делаете мерзости, оскверняете один жену другого, и хотите владеть землею?
\vs Eze 33:27 Вот что скажи им: так говорит Господь Бог: живу Я! те, которые на местах разоренных, падут от меча; а кто в поле, того отдам зверям на съедение; а которые в укреплениях и пещерах, те умрут от моровой язвы.
\vs Eze 33:28 И сделаю землю пустынею из пустынь, и гордое могущество ее престанет, и горы Израилевы опустеют, так что не будет проходящих.
\vs Eze 33:29 И узнают, что Я Господь, когда сделаю землю пустынею из пустынь за все мерзости их, какие они делали.
\vs Eze 33:30 А о тебе, сын человеческий, сыны народа твоего разговаривают у стен и в дверях домов и говорят один другому, брат брату: <<пойдите и послушайте, какое слово вышло от Господа>>.
\vs Eze 33:31 И они приходят к тебе, как на народное сходбище, и садится перед лицем твоим народ Мой, и слушают слова твои, но не исполняют их; ибо они в устах своих делают из этого забаву, сердце их увлекается за корыстью их.
\vs Eze 33:32 И вот, ты для них~--- как забавный певец с приятным голосом и хорошо играющий; они слушают слова твои, но не исполняют их.
\vs Eze 33:33 Но когда сбудется,~--- вот, уже и сбывается,~--- тогда узнают, что среди них был пророк.
\vs Eze 34:1 И было ко мне слово Господне:
\vs Eze 34:2 сын человеческий! изреки пророчество на пастырей Израилевых, изреки пророчество и скажи им, пастырям: так говорит Господь Бог: горе пастырям Израилевым, которые пасли себя самих! не стадо ли должны пасти пастыри?
\vs Eze 34:3 Вы ели тук и в\acc{о}лною одевались, откормленных овец заколали, \bibemph{а} стада не пасли.
\vs Eze 34:4 Слабых не укрепляли, и больной овцы не врачевали, и пораненной не перевязывали, и угнанной не возвращали, и потерянной не искали, а правили ими с насилием и жестокостью.
\vs Eze 34:5 И рассеялись они без пастыря и, рассеявшись, сделались пищею всякому зверю полевому.
\vs Eze 34:6 Блуждают овцы Мои по всем горам и по всякому высокому холму, и по всему лицу земли рассеялись овцы Мои, и никто не разведывает о них, и никто не ищет их.
\vs Eze 34:7 Посему, пастыри, выслушайте слово Господне.
\vs Eze 34:8 Живу Я! говорит Господь Бог; за то, что овцы Мои оставлены были на расхищение и без пастыря сделались овцы Мои пищею всякого зверя полевого, и пастыри Мои не искали овец Моих, и пасли пастыри самих себя, а овец Моих не пасли,~---
\vs Eze 34:9 за то, пастыри, выслушайте слово Господне.
\vs Eze 34:10 Так говорит Господь Бог: вот, Я~--- на пастырей, и взыщу овец Моих от руки их, и не дам им более пасти овец, и не будут более пастыри пасти самих себя, и исторгну овец Моих из челюстей их, и не будут они пищею их.
\vs Eze 34:11 Ибо так говорит Господь Бог: вот, Я Сам отыщу овец Моих и осмотрю их.
\vs Eze 34:12 Как пастух поверяет стадо свое в тот день, когда находится среди стада своего рассеянного, так Я пересмотрю овец Моих и высвобожу их из всех мест, в которые они были рассеяны в день облачный и мрачный.
\vs Eze 34:13 И выведу их из народов, и соберу их из стран, и приведу их в землю их, и буду пасти их на горах Израилевых, при потоках и на всех обитаемых местах земли сей.
\vs Eze 34:14 Буду пасти их на хорошей пажити, и загон их будет на высоких горах Израилевых; там они будут отдыхать в хорошем загоне и будут пастись на тучной пажити, на горах Израилевых.
\vs Eze 34:15 Я буду пасти овец Моих и Я буду покоить их, говорит Господь Бог.
\vs Eze 34:16 Потерявшуюся отыщу и угнанную возвращу, и пораненную перевяжу, и больную укреплю, а разжиревшую и буйную истреблю; буду пасти их по правде.
\vs Eze 34:17 Вас же, овцы Мои,~--- так говорит Господь Бог,~--- вот, Я буду судить между овцою и овцою, между бараном и козлом.
\vs Eze 34:18 Разве мало вам того, что пасетесь на хорошей пажити, а между тем остальное на пажити вашей топчете ногами вашими, пьете чистую воду, а оставшуюся мутите ногами вашими,
\vs Eze 34:19 так что овцы Мои должны питаться тем, что потоптано ногами вашими, и пить то, что возмущено ногами вашими?
\vs Eze 34:20 Посему так говорит им Господь Бог: вот, Я Сам буду судить между овцою тучною и овцою тощею.
\vs Eze 34:21 Так как вы толкаете боком и плечом, и рогами своими бодаете всех слабых, доколе не вытолкаете их вон,~---
\vs Eze 34:22 то Я спасу овец Моих, и они не будут уже расхищаемы, и рассужу между овцою и овцою.
\vs Eze 34:23 И поставлю над ними одного пастыря, который будет пасти их, раба Моего Давида; он будет пасти их и он будет у них пастырем.
\vs Eze 34:24 И Я, Господь, буду их Богом, и раб Мой Давид будет князем среди них. Я, Господь, сказал это.
\vs Eze 34:25 И заключу с ними завет мира и удалю с земли лютых зверей, так что безопасно будут жить в степи и спать в лесах.
\vs Eze 34:26 Дарую им и окрестностям холма Моего благословение, и дождь буду ниспосылать в свое время; это будут дожди благословения.
\vs Eze 34:27 И полевое дерево будет давать плод свой, и земля будет давать произведения свои; и будут они безопасны на земле своей, и узнают, что Я Господь, когда сокрушу связи ярма их и освобожу их из руки поработителей их.
\vs Eze 34:28 Они не будут уже добычею для народов, и полевые звери не будут пожирать их; они будут жить безопасно, и никто не будет устрашать \bibemph{их}.
\vs Eze 34:29 И произведу у них насаждение славное, и не будут уже погибать от голода на земле и терпеть посрамления от народов.
\vs Eze 34:30 И узнают, что Я, Господь Бог их, с ними, и они, дом Израилев, Мой народ, говорит Господь Бог,
\vs Eze 34:31 и что вы~--- овцы Мои, овцы паствы Моей; вы~--- человеки, \bibemph{а} Я Бог ваш, говорит Господь Бог.
\vs Eze 35:1 И было ко мне слово Господне:
\vs Eze 35:2 сын человеческий! обрати лице твое к горе Сеир и изреки на нее пророчество
\vs Eze 35:3 и скажи ей: так говорит Господь Бог: вот, Я~--- на тебя, гора Сеир! и простру на тебя руку Мою и сделаю тебя пустою и необитаемою.
\vs Eze 35:4 Города твои превращу в развалины, и ты сама опустеешь и узнаешь, что Я Господь.
\vs Eze 35:5 Так как у тебя вечная вражда, и ты предавала сынов Израилевых в руки мечу во время несчастья их, во время окончательной гибели:
\vs Eze 35:6 за это~--- живу Я! говорит Господь Бог~--- сделаю тебя кровью, и кровь будет преследовать тебя; так как ты не ненавидела крови, то кровь и будет преследовать тебя.
\vs Eze 35:7 И сделаю гору Сеир пустою и безлюдною степью и истреблю на ней приходящего и возвращающегося.
\vs Eze 35:8 И наполню высоты ее убитыми ее; на холмах твоих и в долинах твоих, и во всех рытвинах твоих будут падать сраженные мечом.
\vs Eze 35:9 Сделаю тебя пустынею вечною, и в городах твоих не будут жить, и узнаете, что Я Господь.
\vs Eze 35:10 Так как ты говорила: <<эти два народа и эти две земли будут мои, и мы завладеем ими, хотя и Господь был там>>:
\vs Eze 35:11 за то,~--- живу Я! говорит Господь Бог,~--- поступлю с тобою по мере ненависти твоей и зависти твоей, какую ты выказала из ненависти твоей к ним, и явлю Себя им, когда буду судить тебя.
\vs Eze 35:12 И узнаешь, что Я, Господь, слышал все глумления твои, какие ты произносила на горы Израилевы, говоря: <<опустели! нам отданы на съедение!>>
\vs Eze 35:13 Вы величались предо Мною языком вашим и умножали речи ваши против Меня; Я слышал это.
\vs Eze 35:14 Так говорит Господь Бог: когда вся земля будет радоваться, Я сделаю тебя пустынею.
\vs Eze 35:15 Как ты радовалась тому, что удел дома Израилева опустел, так сделаю Я и с тобою: опустошена будешь, гора Сеир, и вся Идумея вместе, и узнают, что Я Господь.
\vs Eze 36:1 И ты, сын человеческий, изреки пророчество на горы Израилевы и скажи: горы Израилевы! слушайте слово Господне.
\vs Eze 36:2 Так говорит Господь Бог: так как враг говорит о вас: <<а! а! и вечные высоты достались нам в удел>>,
\vs Eze 36:3 то изреки пророчество и скажи: так говорит Господь Бог: за то, именно за то, что опустошают вас и поглощают вас со всех сторон, чтобы вы сделались достоянием прочих народов и подверглись злоречию и пересудам людей,~---
\vs Eze 36:4 за это, горы Израилевы, выслушайте слово Господа Бога: так говорит Господь Бог горам и холмам, лощинам и долинам, и опустелым развалинам, и оставленным городам, которые сделались добычею и посмеянием прочим окрестным народам;
\vs Eze 36:5 за это так говорит Господь Бог: в огне ревности Моей Я изрек слово на прочие народы и на всю Идумею, которые назначили землю Мою во владение себе, с сердечною радостью и с презрением в душе обрекая ее в добычу себе.
\vs Eze 36:6 Посему изреки пророчество о земле Израилевой и скажи горам и холмам, лощинам и долинам: так говорит Господь Бог: вот, Я изрек сие в ревности Моей и в ярости Моей, потому что вы несете на себе посмеяние от народов.
\vs Eze 36:7 Посему так говорит Господь Бог: Я поднял руку Мою с клятвою, что народы, которые вокруг вас, сами понесут срам свой.
\vs Eze 36:8 А вы, горы Израилевы, распустите ветви ваши и будете приносить плоды ваши народу Моему Израилю; ибо они скоро придут.
\vs Eze 36:9 Ибо вот, Я к вам обращусь, и вы будете возделываемы и засеваемы.
\vs Eze 36:10 И поселю на вас множество людей, весь дом Израилев, весь, и заселены будут города и застроены развалины.
\vs Eze 36:11 И умножу на вас людей и скот, и они будут плодиться и размножаться, и заселю вас, как было в прежние времена ваши, и буду благотворить вам больше, нежели в прежние времена ваши, и узнаете, что Я Господь.
\vs Eze 36:12 И приведу на вас людей, народ Мой, Израиля, и они будут владеть тобою, \bibemph{земля}! и ты будешь наследием их и не будешь более делать их бездетными.
\vs Eze 36:13 Так говорит Господь Бог: за то, что говорят о вас: <<ты~--- \bibemph{земля}, поедающая людей и делающая народ твой бездетным>>:
\vs Eze 36:14 за то уже не будешь поедать людей и народа твоего не будешь вперед делать бездетным, говорит Господь Бог.
\vs Eze 36:15 И не будешь более слышать посмеяния от народов, и поругания от племен не понесешь уже на себе, и народа твоего вперед не будешь делать бездетным, говорит Господь Бог.
\rsbpar\vs Eze 36:16 И было ко мне слово Господне:
\vs Eze 36:17 сын человеческий! когда дом Израилев жил на земле своей, он осквернял ее поведением своим и делами своими; путь их пред лицем Моим был как нечистота женщины во время очищения ее.
\vs Eze 36:18 И Я излил на них гнев Мой за кровь, которую они проливали на этой земле, и за то, что они оскверняли ее идолами своими.
\vs Eze 36:19 И Я рассеял их по народам, и они развеяны по землям; Я судил их по путям их и по делам их.
\vs Eze 36:20 И пришли они к народам, куда пошли, и обесславили святое имя Мое, потому что о них говорят: <<они~--- народ Господа, и вышли из земли Его>>.
\vs Eze 36:21 И пожалел Я святое имя Мое, которое обесславил дом Израилев у народов, куда пришел.
\vs Eze 36:22 Посему скажи дому Израилеву: так говорит Господь Бог: не для вас Я сделаю это, дом Израилев, а ради святаго имени Моего, которое вы обесславили у народов, куда пришли.
\vs Eze 36:23 И освящу великое имя Мое, бесславимое у народов, среди которых вы обесславили его, и узнают народы, что Я Господь, говорит Господь Бог, когда явлю на вас святость Мою перед глазами их.
\vs Eze 36:24 И возьму вас из народов, и соберу вас из всех стран, и приведу вас в землю вашу.
\vs Eze 36:25 И окроплю вас чистою водою, и вы очиститесь от всех скверн ваших, и от всех идолов ваших очищу вас.
\vs Eze 36:26 И дам вам сердце новое, и дух новый дам вам; и возьму из плоти вашей сердце каменное, и дам вам сердце плотяное.
\vs Eze 36:27 Вложу внутрь вас дух Мой и сделаю то, что вы будете ходить в заповедях Моих и уставы Мои будете соблюдать и выполнять.
\vs Eze 36:28 И будете жить на земле, которую Я дал отцам вашим, и будете Моим народом, и Я буду вашим Богом.
\vs Eze 36:29 И освобожу вас от всех нечистот ваших, и призову хлеб, и умножу его, и не дам вам терпеть голода.
\vs Eze 36:30 И умножу плоды на деревах и произведения полей, чтобы вперед не терпеть вам поношения от народов из-за голода.
\vs Eze 36:31 Тогда вспомните о злых путях ваших и недобрых делах ваших и почувствуете отвращение к самим себе за беззакония ваши и за мерзости ваши.
\vs Eze 36:32 Не ради вас Я сделаю это, говорит Господь Бог, да будет вам известно. Краснейте и стыдитесь путей ваших, дом Израилев.
\vs Eze 36:33 Так говорит Господь Бог: в тот день, когда очищу вас от всех беззаконий ваших и населю города, и обстроены будут развалины,
\vs Eze 36:34 и опустошенная земля будет возделываема, быв пустынею в глазах всякого мимоходящего,
\vs Eze 36:35 тогда скажут: <<эта опустелая земля сделалась, как сад Едемский; и эти развалившиеся и опустелые и разоренные города укреплены и населены>>.
\vs Eze 36:36 И узнают народы, которые останутся вокруг вас, что Я, Господь, вновь созидаю разрушенное, засаждаю опустелое. Я, Господь, сказал~--- и сделал.
\vs Eze 36:37 Так говорит Господь Бог: вот, еще и в том явлю милость Мою дому Израилеву, умножу их людьми как стадо.
\vs Eze 36:38 Как много бывает жертвенных овец в Иерусалиме во время праздников его, так полны будут людьми опустелые города, и узнают, что Я Господь.
\vs Eze 37:1 Была на мне рука Господа, и Господь вывел меня духом и поставил меня среди поля, и оно было полно костей,
\vs Eze 37:2 и обвел меня кругом около них, и вот весьма много их на поверхности поля, и вот они весьма сухи.
\vs Eze 37:3 И сказал мне: сын человеческий! оживут ли кости сии? Я сказал: Господи Боже! Ты знаешь это.
\vs Eze 37:4 И сказал мне: изреки пророчество на кости сии и скажи им: <<кости сухие! слушайте слово Господне!>>
\vs Eze 37:5 Так говорит Господь Бог костям сим: вот, Я введу дух в вас, и оживете.
\vs Eze 37:6 И обложу вас жилами, и выращу на вас плоть, и покрою вас кожею, и введу в вас дух, и оживете, и узнаете, что Я Господь.
\vs Eze 37:7 Я изрек пророчество, как повелено было мне; и когда я пророчествовал, произошел шум, и вот движение, и стали сближаться кости, кость с костью своею.
\vs Eze 37:8 И видел я: и вот, жилы были на них, и плоть выросла, и кожа покрыла их сверху, а духа не было в них.
\vs Eze 37:9 Тогда сказал Он мне: изреки пророчество духу, изреки пророчество, сын человеческий, и скажи духу: так говорит Господь Бог: от четырех ветров приди, дух, и дохни на этих убитых, и они оживут.
\vs Eze 37:10 И я изрек пророчество, как Он повелел мне, и вошел в них дух, и они ожили, и стали на ноги свои~--- весьма, весьма великое полчище.
\vs Eze 37:11 И сказал Он мне: сын человеческий! кости сии~--- весь дом Израилев. Вот, они говорят: <<иссохли кости наши, и погибла надежда наша, мы оторваны от корня>>.
\vs Eze 37:12 Посему изреки пророчество и скажи им: так говорит Господь Бог: вот, Я открою гробы ваши и выведу вас, народ Мой, из гробов ваших и введу вас в землю Израилеву.
\vs Eze 37:13 И узнаете, что Я Господь, когда открою гробы ваши и выведу вас, народ Мой, из гробов ваших,
\vs Eze 37:14 и вложу в вас дух Мой, и оживете, и помещу вас на земле вашей, и узнаете, что Я, Господь, сказал это~--- и сделал, говорит Господь.
\rsbpar\vs Eze 37:15 И было ко мне слово Господне:
\vs Eze 37:16 ты же, сын человеческий, возьми себе один жезл и напиши на нем: <<Иуде и сынам Израилевым, союзным с ним>>; и еще возьми жезл и напиши на нем: <<Иосифу>>; это жезл Ефрема и всего дома Израилева, союзного с ним.
\vs Eze 37:17 И сложи их у себя один с другим в один жезл, чтобы они в руке твоей были одно.
\vs Eze 37:18 И когда спросят у тебя сыны народа твоего: <<не объяснишь ли нам, что это у тебя?>>,
\vs Eze 37:19 тогда скажи им: так говорит Господь Бог: вот, Я возьму жезл Иосифов, который в руке Ефрема и союзных с ним колен Израилевых, и приложу их к нему, к жезлу Иуды, и сделаю их одним жезлом, и будут одно в руке Моей.
\vs Eze 37:20 Когда же оба жезла, на которых ты напишешь, будут в руке твоей перед глазами их,
\vs Eze 37:21 то скажи им: так говорит Господь Бог: вот, Я возьму сынов Израилевых из среды народов, между которыми они находятся, и соберу их отовсюду и приведу их в землю их.
\vs Eze 37:22 На этой земле, на горах Израиля Я сделаю их одним народом, и один Царь будет царем у всех их, и не будут более двумя народами, и уже не будут вперед разделяться на два царства.
\vs Eze 37:23 И не будут уже осквернять себя идолами своими и мерзостями своими и всякими пороками своими, и освобожу их из всех мест жительства их, где они грешили, и очищу их, и будут Моим народом, и Я буду их Богом.
\vs Eze 37:24 А раб Мой Давид будет Царем над ними и Пастырем всех их, и они будут ходить в заповедях Моих, и уставы Мои будут соблюдать и выполнять их.
\vs Eze 37:25 И будут жить на земле, которую Я дал рабу Моему Иакову, на которой жили отцы их; там будут жить они и дети их, и дети детей их во веки; и раб Мой Давид будет князем у них вечно.
\vs Eze 37:26 И заключу с ними завет мира, завет вечный будет с ними. И устрою их, и размножу их, и поставлю среди них святилище Мое на веки.
\vs Eze 37:27 И будет у них жилище Мое, и буду их Богом, а они будут Моим народом.
\vs Eze 37:28 И узнают народы, что Я Господь, освящающий Израиля, когда святилище Мое будет среди них во веки.
\vs Eze 38:1 И было ко мне слово Господне:
\vs Eze 38:2 сын человеческий! обрати лице твое к Гогу в земле Магог, князю Роша, Мешеха и Фувала, и изреки на него пророчество
\vs Eze 38:3 и скажи: так говорит Господь Бог: вот, Я~--- на тебя, Гог, князь Роша, Мешеха и Фувала!
\vs Eze 38:4 И поверну тебя, и вложу удила в челюсти твои, и выведу тебя и все войско твое, коней и всадников, всех в полном вооружении, большое полчище, в бронях и со щитами, всех вооруженных мечами,
\vs Eze 38:5 Персов, Ефиоплян и Ливийцев с ними, всех со щитами и в шлемах,
\vs Eze 38:6 Гомера со всеми отрядами его, дом Фогарма, от пределов севера, со всеми отрядами его, многие народы с тобою.
\vs Eze 38:7 Готовься и снаряжайся, ты и все полчища твои, собравшиеся к тебе, и будь им вождем.
\vs Eze 38:8 После многих дней ты понадобишься; в последние годы ты придешь в землю, избавленную от меча, собранную из многих народов, на горы Израилевы, которые были в постоянном запустении, но теперь жители ее будут возвращены из народов, и все они будут жить безопасно.
\vs Eze 38:9 И поднимешься, как буря, пойдешь, как туча, чтобы покрыть землю, ты и все полчища твои и многие народы с тобою.
\vs Eze 38:10 Так говорит Господь Бог: в тот день придут тебе на сердце мысли, и ты задумаешь злое предприятие
\vs Eze 38:11 и скажешь: <<поднимусь я на землю неогражденную, пойду на беззаботных, живущих беспечно,~--- все они живут без стен, и нет у них ни запоров, ни дверей,~---
\vs Eze 38:12 чтобы произвести грабеж и набрать добычи, наложить руку на вновь заселенные развалины и на народ, собранный из народов, занимающийся хозяйством и торговлею, живущий на вершине земли>>.
\vs Eze 38:13 Сава и Дедан и купцы Фарсисские со всеми молодыми львами их скажут тебе: <<ты пришел, чтобы произвести грабеж, собрал полчище твое, чтобы набрать добычи, взять серебро и золото, отнять скот и имущество, захватить большую добычу?>>
\vs Eze 38:14 Посему изреки пророчество, сын человеческий, и скажи Гогу: так говорит Господь Бог: не так ли? в тот день, когда народ Мой Израиль будет жить безопасно, ты узнаешь это;
\vs Eze 38:15 и пойдешь с места твоего, от пределов севера, ты и многие народы с тобою, все сидящие на конях, сборище великое и войско многочисленное.
\vs Eze 38:16 И поднимешься на народ Мой, на Израиля, как туча, чтобы покрыть землю: это будет в последние дни, и Я приведу тебя на землю Мою, чтобы народы узнали Меня, когда Я над тобою, Гог, явлю святость Мою пред глазами их.
\vs Eze 38:17 Так говорит Господь Бог: не ты ли тот самый, о котором Я говорил в древние дни чрез рабов Моих, пророков Израилевых, которые пророчествовали в те времена, что Я приведу тебя на них?
\vs Eze 38:18 И будет в тот день, когда Гог придет на землю Израилеву, говорит Господь Бог, гнев Мой воспылает в ярости Моей.
\vs Eze 38:19 И в ревности Моей, в огне негодования Моего Я сказал: истинно в тот день произойдет великое потрясение на земле Израилевой.
\vs Eze 38:20 И вострепещут от лица Моего рыбы морские и птицы небесные, и звери полевые и все пресмыкающееся, ползающее по земле, и все люди, которые на лице земли, и обрушатся горы, и упадут утесы, и все стены падут на землю.
\vs Eze 38:21 И по всем горам Моим призову меч против него, говорит Господь Бог; меч каждого человека будет против брата его.
\vs Eze 38:22 И буду судиться с ним моровою язвою и кровопролитием, и пролью на него и на полки его и на многие народы, которые с ним, всепотопляющий дождь и каменный град, огонь и серу;
\vs Eze 38:23 и покажу Мое величие и святость Мою, и явлю Себя пред глазами многих народов, и узнают, что Я Господь.
\vs Eze 39:1 Ты же, сын человеческий, изреки пророчество на Гога и скажи: так говорит Господь Бог: вот, Я~--- на тебя, Гог, князь Роша, Мешеха и Фувала!
\vs Eze 39:2 И поверну тебя, и поведу тебя, и выведу тебя от краев севера, и приведу тебя на горы Израилевы.
\vs Eze 39:3 И выбью лук твой из левой руки твоей, и выброшу стрелы твои из правой руки твоей.
\vs Eze 39:4 Падешь ты на горах Израилевых, ты и все полки твои, и народы, которые с тобою; отдам тебя на съедение всякого рода хищным птицам и зверям полевым.
\vs Eze 39:5 На открытом поле падешь; ибо Я сказал это, говорит Господь Бог.
\vs Eze 39:6 И пошлю огонь на землю Магог и на жителей островов, живущих беспечно, и узнают, что Я Господь.
\vs Eze 39:7 И явлю святое имя Мое среди народа Моего, Израиля, и не дам вперед бесславить святаго имени Моего, и узнают народы, что Я Господь, Святый в Израиле.
\vs Eze 39:8 Вот, это придет и сбудется, говорит Господь Бог,~--- это тот день, о котором Я сказал.
\vs Eze 39:9 Тогда жители городов Израилевых выйдут, и разведут огонь, и будут сожигать оружие, щиты и латы, луки и стрелы, и булавы и копья; семь лет буду жечь их.
\vs Eze 39:10 И не будут носить дров с поля, ни рубить из лесов, но будут жечь только оружие; и ограбят грабителей своих, и оберут обирателей своих, говорит Господь Бог.
\vs Eze 39:11 И будет в тот день: дам Гогу место для могилы в Израиле, долину прохожих на восток от моря, и она будет задерживать прохожих; и похоронят там Гога и все полчище его, и будут называть ее долиною полчища Гогова.
\vs Eze 39:12 И дом Израилев семь месяцев будет хоронить их, чтобы очистить землю.
\vs Eze 39:13 И весь народ земли будет хоронить \bibemph{их}, и знаменит будет у них день, в который Я прославлю Себя, говорит Господь Бог.
\vs Eze 39:14 И назначат людей, которые постоянно обходили бы землю и с помощью прохожих погребали бы оставшихся на поверхности земли, для очищения ее; по прошествии семи месяцев они начнут делать поиски;
\vs Eze 39:15 и когда кто из обходящих землю увидит кость человеческую, то поставит возле нее знак, доколе погребатели не похоронят ее в долине полчища Гогова.
\vs Eze 39:16 И будет имя городу: Гамона [полчище]. И так очистят они землю.
\vs Eze 39:17 Ты же, сын человеческий, так говорит Господь Бог, скажи всякого рода птицам и всем зверям полевым: собирайтесь и идите, со всех сторон сходитесь к жертве Моей, которую Я заколю для вас, к великой жертве на горах Израилевых; и будете есть мясо и пить кровь.
\vs Eze 39:18 Мясо мужей сильных будете есть, и будете пить кровь князей земли, баранов, ягнят, козлов и тельцов, всех откормленных на Васане;
\vs Eze 39:19 и будете есть жир до сытости и пить кровь до опьянения от жертвы Моей, которую Я заколю для вас.
\vs Eze 39:20 И насытитесь за столом Моим конями и всадниками, мужами сильными и всякими людьми военными, говорит Господь Бог.
\vs Eze 39:21 И явлю славу Мою между народами, и все народы увидят суд Мой, который Я произведу, и руку Мою, которую Я наложу на них.
\vs Eze 39:22 И будет знать дом Израилев, что Я Господь Бог их, от сего дня и далее.
\vs Eze 39:23 И узнают народы, что дом Израилев был переселен за неправду свою; за то, что они поступали вероломно предо Мною, Я сокрыл от них лице Мое и отдал их в руки врагов их, и все они пали от меча.
\vs Eze 39:24 За нечистоты их и за их беззаконие Я сделал это с ними, и сокрыл от них лице Мое.
\vs Eze 39:25 Посему так говорит Господь Бог: ныне возвращу плен Иакова, и помилую весь дом Израиля, и возревную по святом имени Моем.
\vs Eze 39:26 И почувствуют они бесчестие свое и все беззакония свои, какие делали предо Мною, когда будут жить на земле своей безопасно, и никто не будет устрашать их,
\vs Eze 39:27 когда Я возвращу их из народов, и соберу их из земель врагов их, и явлю в них святость Мою пред глазами многих народов.
\vs Eze 39:28 И узнают, что Я Господь Бог их, когда, рассеяв их между народами, опять соберу их в землю их и не оставлю уже там ни одного из них;
\vs Eze 39:29 и не буду уже скрывать от них лица Моего, потому что Я изолью дух Мой на дом Израилев, говорит Господь Бог.
\vs Eze 40:1 В двадцать пятом году по переселении нашем, в начале года, в десятый \bibemph{день} месяца, в четырнадцатом году по разрушении города, в тот самый день была на мне рука Господа, и Он повел меня туда.
\vs Eze 40:2 В видениях Божиих привел Он меня в землю Израилеву и поставил меня на весьма высокой горе, и на ней, с южной стороны, были как бы городские здания;
\vs Eze 40:3 и привел меня туда. И вот муж, которого вид как бы вид блестящей меди, и льняная вервь в руке его и трость измерения, и стоял он у ворот.
\vs Eze 40:4 И сказал мне этот муж: <<сын человеческий! смотри глазами твоими и слушай ушами твоими, и прилагай сердце твое ко всему, что я буду показывать тебе, ибо ты для того и приведен сюда, чтоб я показал тебе \bibemph{это}; все, что увидишь, возвести дому Израилеву>>.
\vs Eze 40:5 И вот, вне храма стена со всех сторон \bibemph{его}, и в руке того мужа трость измерения в шесть локтей, \bibemph{считая каждый локоть} в локоть с ладонью; и намерил он в этом здании одну трость толщины и одну трость вышины.
\vs Eze 40:6 Потом пошел к воротам, обращенным лицом к востоку, и взошел по ступеням их, и нашел меры в одном пороге ворот одну трость ширины и в другом пороге одну трость ширины.
\vs Eze 40:7 И в каждой боковой комнате одна трость длины и одна трость ширины, а между комнатами пять локтей, и в пороге ворот у притвора ворот внутри одна же трость.
\vs Eze 40:8 И смерил он в притворе ворот внутри одну трость,
\vs Eze 40:9 а в притворе у ворот намерил восемь локтей и два локтя в столбах. Этот притвор у ворот со стороны храма.
\vs Eze 40:10 Боковых комнат у восточных ворот три~--- с одной стороны и три~--- с другой; одна мера во всех трех и одна мера в столбах с той и другой стороны.
\vs Eze 40:11 Ширины в отверстии ворот он намерил десять локтей, а длины ворот тринадцать локтей.
\vs Eze 40:12 А перед комнатами выступ в один локоть, и в один же локоть с другой стороны выступ; эти комнаты с одной стороны \bibemph{имели} шесть локтей и шесть же локтей с другой стороны.
\vs Eze 40:13 Потом намерил он в воротах от крыши одной комнаты до крыши другой двадцать пять локтей ширины; дверь была против двери.
\vs Eze 40:14 А в столбах он насчитал шестьдесят локтей, в каждом столбе около двора и у ворот,
\vs Eze 40:15 и от передней стороны входа в ворота до передней стороны внутренних ворот пятьдесят локтей.
\vs Eze 40:16 Решетчатые окна были и в боковых комнатах и в столбах их, внутрь ворот кругом, также и в притворах окна были кругом на внутреннюю сторону, и на столбах~--- пальмы.
\vs Eze 40:17 И привел он меня на внешний двор, и вот там комнаты, и каменный помост кругом двора был сделан; тридцать комнат на том помосте.
\vs Eze 40:18 И помост этот был по бокам ворот, соответственно длине ворот; этот помост был ниже.
\vs Eze 40:19 И намерил он в ширину от нижних ворот до внешнего края внутреннего двора сто локтей, к востоку и к северу.
\vs Eze 40:20 Он измерил также длину и ширину ворот внешнего двора, обращенных лицом к северу,
\vs Eze 40:21 и боковые комнаты при них, три с одной стороны и три с другой; и столбы их, и выступы их были такой же меры, как у прежних ворот: длина их пятьдесят локтей, а ширина двадцать пять локтей.
\vs Eze 40:22 И окна их, и выступы их, и пальмы их~--- той же меры, как у ворот, обращенных лицом к востоку; и входят к ним семью ступенями, и перед ними выступы.
\vs Eze 40:23 И во внутренний двор есть ворота против ворот северных и восточных; и намерил он от ворот до ворот сто локтей.
\vs Eze 40:24 И повел меня на юг, и вот там ворота южные; и намерил он в столбах и выступах такую же меру.
\vs Eze 40:25 И окна в них и в преддвериях их такие же, как те окна: длины пятьдесят локтей, а ширины двадцать пять локтей.
\vs Eze 40:26 Подъем к ним~--- в семь ступеней, и преддверия перед ними; и пальмовые украшения~--- одно с той стороны и одно с другой на столбах их.
\vs Eze 40:27 И во внутренний двор были южные ворота; и намерил он от ворот до ворот южных сто локтей.
\vs Eze 40:28 И привел он меня через южные ворота во внутренний двор; и намерил в южных воротах ту же меру.
\vs Eze 40:29 И боковые комнаты их, и столбы их, и притворы их~--- той же меры, и окна в них в притворах их были кругом; всего в длину пятьдесят локтей, а в ширину двадцать пять локтей.
\vs Eze 40:30 Притворы были кругом длиною в двадцать пять локтей, а шириною в пять локтей.
\vs Eze 40:31 И притворы были у них на внешний двор, и пальмы были на столбах их; подъем к ним~--- в восемь ступеней.
\vs Eze 40:32 И повел меня восточными воротами на внутренний двор; и намерил в этих воротах ту же меру.
\vs Eze 40:33 И боковые комнаты их, и столбы их, и притворы их были той же меры; и окна в них и притворах их были кругом; длина пятьдесят локтей, а ширина двадцать пять локтей.
\vs Eze 40:34 Притворы у них были на внешний двор, и пальмы на столбах их с той и другой стороны; подъем к ним~--- в восемь ступеней.
\vs Eze 40:35 Потом привел меня к северным воротам, и намерил в них ту же меру.
\vs Eze 40:36 Боковые комнаты при них, столбы их и притворы их, и окна в них были кругом; всего в длину пятьдесят локтей, и в ширину двадцать пять локтей.
\vs Eze 40:37 Притворы у них были на внешний двор, и пальмы на столбах их с той и с другой стороны; подъем к ним~--- в восемь ступеней.
\vs Eze 40:38 Была также комната, со входом в нее, у столбов ворот: там омывают жертвы всесожжения.
\vs Eze 40:39 А в притворе у ворот два стола с одной стороны и два с другой стороны, чтобы заколать на них жертвы всесожжения и жертвы за грех и жертвы за преступление.
\vs Eze 40:40 И у наружного бока при входе в отверстие северных ворот были два стола, и у другого бока, подле притвора у ворот, два стола.
\vs Eze 40:41 Четыре стола с одной стороны и четыре стола с другой стороны, по бокам ворот: \bibemph{всего} восемь столов, на которых заколают \bibemph{жертвы}.
\vs Eze 40:42 И четыре стола для приготовления всесожжения были из тесаных камней, длиною в полтора локтя, и шириною в полтора локтя, а вышиною в один локоть; на них кладут орудия для заклания жертвы всесожжения и \bibemph{других} жертв.
\vs Eze 40:43 И крюки в одну ладонь приделаны были к стенам здания кругом, а на столах клали жертвенное мясо.
\vs Eze 40:44 Снаружи внутренних ворот были комнаты для певцов; на внутреннем дворе, сбоку северных ворот, одна обращена лицом к югу, а другая, сбоку южных ворот, обращена лицом к северу.
\vs Eze 40:45 И сказал он мне: <<эта комната, которая лицом к югу, для священников, бодрствующих на страже храма;
\vs Eze 40:46 а комната, которая лицом к северу, для священников, бодрствующих на страже жертвенника: это сыны Садока, которые одни из сынов Левия приближаются к Господу, чтобы служить Ему>>.
\vs Eze 40:47 И намерил он во дворе сто локтей длины и сто локтей ширины: \bibemph{он} был четыреугольный; а перед храмом стоял жертвенник.
\vs Eze 40:48 И привел он меня к притвору храма, и намерил в столбах притвора пять локтей с одной стороны и пять локтей с другой; а в воротах три локтя ширины с одной стороны и три локтя с другой.
\vs Eze 40:49 Длина притвора~--- в двадцать локтей, а ширина~--- в одиннадцать локтей, и всходят в него по десяти ступеням; и были подпоры у столбов, одна с одной стороны, а другая с другой.
\vs Eze 41:1 Потом ввел меня в храм и намерил в столбах шесть локтей ширины с одной стороны и шесть локтей ширины с другой стороны, в ширину скинии.
\vs Eze 41:2 В дверях десять локтей ширины, и по бокам дверей пять локтей с одной стороны и пять локтей с другой стороны; и намерил длины в храме сорок локтей, а ширины двадцать локтей.
\vs Eze 41:3 И пошел внутрь, и намерил в столбах у дверей два локтя и в дверях шесть локтей, а ширина двери~--- в семь локтей.
\vs Eze 41:4 И отмерил в нем двадцать локтей в длину и двадцать локтей в ширину храма, и сказал мне: <<это~--- Святое Святых>>.
\vs Eze 41:5 И намерил в стене храма шесть локтей, а ширины в боковых комнатах, кругом храма, по четыре локтя.
\vs Eze 41:6 Боковых комнат было тридцать три, комната подле комнаты; они вдаются в стену, которая у храма для комнат кругом, так что они в связи с нею, но стен\acc{ы} самого храма не касаются.
\vs Eze 41:7 И он более и более расширялся кругом вверх боковыми комнатами, потому что окружность храма восходила выше и выше вокруг храма, и потому храм имел б\acc{о}льшую ширину вверху, и из нижнего этажа восходили в верхний через средний.
\vs Eze 41:8 И я видел верх дома во всю окружность; боковые комнаты в основании имели там меры цельную трость, шесть полных локтей.
\vs Eze 41:9 Ширина стены боковых комнат, выходящих наружу, пять локтей, и открытое пространство есть подле боковых комнат храма.
\vs Eze 41:10 И между комнатами расстояние двадцать локтей кругом всего храма.
\vs Eze 41:11 Двери боковых комнат \bibemph{ведут} на открытое пространство, одни двери~--- на северную сторону, а другие двери~--- на южную сторону; а ширина этого открытого пространства~--- пять локтей кругом.
\vs Eze 41:12 Здание перед площадью на западной стороне~--- шириною в семьдесят локтей; стена же этого здания~--- в пять локтей ширины кругом, а длина ее~--- девяносто локтей.
\vs Eze 41:13 И намерил он в храме сто локтей длины, и в площади и в пристройке, и в стенах его также сто локтей длины.
\vs Eze 41:14 И ширина храма по лицевой стороне и площади к востоку сто же локтей.
\vs Eze 41:15 И в длине здания перед площадью на задней стороне ее с боковыми комнатами его по ту и другую сторону он намерил сто локтей, со внутренностью храма и притворами двора.
\vs Eze 41:16 Дверные брусья и решетчатые окна, и боковые комнаты кругом, во всех трех \bibemph{ярусах}, против порогов обшиты деревом и от пола по окна; окна были закрыты.
\vs Eze 41:17 От верха дверей как внутри храма, так и снаружи, и по всей стене кругом, внутри и снаружи, были резные изображения,
\vs Eze 41:18 сделаны были херувимы и пальмы: пальма между двумя херувимами, и у каждого херувима два лица.
\vs Eze 41:19 С одной стороны к пальме обращено лицо человеческое, а с другой стороны к пальме~--- лице львиное; так сделано во всем храме кругом.
\vs Eze 41:20 От пола до верха дверей сделаны были херувимы и пальмы, также и по стене храма.
\vs Eze 41:21 В храме были четырехугольные дверные косяки, и святилище имело такой же вид, как я видел.
\vs Eze 41:22 Жертвенник был деревянный в три локтя вышины и в два локтя длины; и углы его, и подножие его, и стенки его~--- из дерева. И сказал он мне: <<это трапеза, которая пред Господом>>.
\vs Eze 41:23 В храме и во святилище по две двери,
\vs Eze 41:24 и двери сии о двух досках, обе доски подвижные, две у одной двери и две доски у другой;
\vs Eze 41:25 и сделаны на них, на дверях храма, херувимы и пальмы такие же, какие сделаны по стенам; а перед притвором снаружи был деревянный помост.
\vs Eze 41:26 И решетчатые окна с пальмами, по ту и другую сторону, были по бокам притвора и в боковых комнатах храма и на деревянной обшивке.
\vs Eze 42:1 И вывел меня ко внешнему двору северною дорогою, и привел меня к комнатам, которые против площади и против здания на севере,
\vs Eze 42:2 к тому месту, которое у северных дверей имеет в длину сто локтей, а в ширину пятьдесят локтей.
\vs Eze 42:3 Напротив двадцати \bibemph{локтей} внутреннего двора и напротив помоста, который на внешнем дворе, были галерея против галереи в три яруса.
\vs Eze 42:4 А перед комнатами ход в десять локтей ширины, а внутрь в один локоть; двери их лицом к северу.
\vs Eze 42:5 Верхние комнаты \acc{у}же, потому что галереи отнимают у них несколько против нижних и средних \bibemph{комнат} этого здания.
\vs Eze 42:6 Они в три яруса, и таких столбов, какие на дворах, нет у них; потому они и сделаны \acc{у}же против нижних и средних комнат, начиная от пола.
\vs Eze 42:7 А наружная стена напротив этих комнат от внешнего двора, составляющая лицевую сторону комнат, имеет длины пятьдесят локтей;
\vs Eze 42:8 потому что \bibemph{и} комнаты на внешнем дворе занимают длины только пятьдесят локтей, и вот перед храмом сто локтей.
\vs Eze 42:9 А снизу ход к этим комнатам с восточной стороны, когда подходят к ним со внешнего двора.
\vs Eze 42:10 В ширину стены двора к востоку перед площадью и перед зданием были комнаты.
\vs Eze 42:11 И ход перед ними такой же, как и у тех комнат, которые обращены к северу, такая же длина, как и у тех, и такая же ширина, и все выходы их, и устройство их, и двери их такие же, как и у тех.
\vs Eze 42:12 Такие же двери, как и у комнат, которые на юг, и для входа в них дверь у самой дороги, которая шла прямо вдоль стены на восток.
\vs Eze 42:13 И сказал он мне: <<комнаты на север \bibemph{и} комнаты на юг, которые перед площадью, суть комнаты священные, в которых священники, приближающиеся к Господу, съедают священнейшие жертвы; там же они кладут священнейшие жертвы, и хлебное приношение, и жертву за грех, и жертву за преступление, ибо это место святое.
\vs Eze 42:14 Когда войдут \bibemph{туда} священники, то они не должны выходить из этого святаго места на внешний двор, доколе не оставят там одежд своих, в которых служили, ибо они священны; они должны надеть на себя другие одежды и тогда выходить к народу>>.
\vs Eze 42:15 Когда кончил он измерения внутреннего храма, то вывел меня воротами, обращенными лицом к востоку, и стал измерять его кругом.
\vs Eze 42:16 Он измерил восточную сторону тростью измерения и \bibemph{намерил} тростью измерения всего пятьсот тростей;
\vs Eze 42:17 в северной стороне той же тростью измерения намерил всего пятьсот тростей;
\vs Eze 42:18 в южной стороне намерил тростью измерения также пятьсот тростей.
\vs Eze 42:19 Поворотив к западной стороне, намерил тростью измерения пятьсот тростей.
\vs Eze 42:20 Со всех четырех сторон он измерил его; кругом него была стена длиною в пятьсот \bibemph{тростей} и в пятьсот \bibemph{тростей} шириною, чтобы отделить святое место от несвятого.
\vs Eze 43:1 И привел меня к воротам, к тем воротам, которые обращены лицом к востоку.
\vs Eze 43:2 И вот, слава Бога Израилева шла от востока, и глас Его~--- как шум вод многих, и земля осветилась от славы Его.
\vs Eze 43:3 Это видение было такое же, какое я видел прежде, точно такое, какое я видел, когда приходил возвестить гибель городу, и видения, подобные видениям, какие видел я у реки Ховара. И я пал на лице мое.
\vs Eze 43:4 И слава Господа вошла в храм путем ворот, обращенных лицом к востоку.
\rsbpar\vs Eze 43:5 И поднял меня дух, и ввел меня во внутренний двор, и вот, слава Господа наполнила весь храм.
\vs Eze 43:6 И я слышал кого-то, говорящего мне из храма, а тот муж стоял подле меня,
\vs Eze 43:7 и сказал мне: сын человеческий! это место престола Моего и место стопам ног Моих, где Я буду жить среди сынов Израилевых во веки; и дом Израилев не будет более осквернять святаго имени Моего, ни они, ни цари их, блужением своим и трупами царей своих на высотах их.
\vs Eze 43:8 Они ставили порог свой у порога Моего и вереи дверей своих подле Моих верей, так что одна стена \bibemph{была} между Мною и ими, и оскверняли святое имя Мое мерзостями своими, какие делали, и за то Я погубил их во гневе Моем.
\vs Eze 43:9 А теперь они удалят от Меня блужение свое и трупы царей своих, и Я буду жить среди них во веки.
\vs Eze 43:10 Ты, сын человеческий, возвести дому Израилеву о храме сем, чтобы они устыдились беззаконий своих и чтобы сняли с него меру.
\vs Eze 43:11 И если они устыдятся всего того, что делали, то покажи им вид храма и расположение его, и выходы его, и входы его, и все очертания его, и все уставы его, и все образы его, и все законы его, и напиши при глазах их, чтобы они сохраняли все очертания его и все уставы его и поступали по ним.
\vs Eze 43:12 Вот закон храма: на вершине горы все пространство его вокруг~--- Святое Святых; вот закон храма!
\vs Eze 43:13 И вот размеры жертвенника локтями, \bibemph{считая} локоть в локоть с ладонью: основание в локоть, ширина в локоть же, и пояс по всем краям его в одну пядень; и вот задняя сторона жертвенника.
\vs Eze 43:14 От основания, что в земле, до нижнего выступа два локтя, а шириною он в один локоть; от малого выступа до большого выступа четыре локтя, а ширина его~--- в один локоть.
\vs Eze 43:15 Самый жертвенник вышиною в четыре локтя; и из жертвенника \bibemph{поднимаются} вверх четыре рога.
\vs Eze 43:16 Жертвенник имеет двенадцать \bibemph{локтей} длины \bibemph{и} двенадцать ширины; он четырехугольный на все свои четыре стороны.
\vs Eze 43:17 А в площадке четырнадцать \bibemph{локтей} длины и четырнадцать ширины на все четыре стороны ее, и вокруг нее пояс в пол-локтя, а основание ее в локоть вокруг, ступени же к нему~--- с востока.
\vs Eze 43:18 И сказал он мне: сын человеческий! так говорит Господь Бог: вот уставы жертвенника к тому дню, когда он будет сделан для приношения на нем всесожжений и для кропления на него кровью.
\vs Eze 43:19 Священникам от колена Левиина, которые из племени Садока, приближающимся ко Мне, чтобы служить Мне, говорит Господь Бог, дай тельца из стада волов, в жертву за грех.
\vs Eze 43:20 И возьми крови его, и покропи на четыре рога его, и на четыре угла площадки, и на пояс кругом, и так очисти его и освяти его.
\vs Eze 43:21 И возьми тельца, \bibemph{в жертву} за грех, и сожги его на назначенном месте дома вне святилища.
\vs Eze 43:22 А на другой день в жертву за грех принеси из козьего стада козла без порока, и пусть очистят жертвенник так же, как очищали тельцом.
\vs Eze 43:23 Когда же кончишь очищение, приведи из стада волов тельца без порока и из стада овец овна без порока;
\vs Eze 43:24 и принеси их пред лице Господа; и священники бросят на них соли, и вознесут их во всесожжение Господу.
\vs Eze 43:25 Семь дней приноси в жертву за грех по козлу в день; также пусть приносят в жертву по тельцу из стада волов и по овну из стада овец без порока.
\vs Eze 43:26 Семь дней они должны очищать жертвенник и освящать его и наполнять руки свои.
\vs Eze 43:27 По окончании же сих дней, в восьмой день и далее, священники будут возносить на жертвеннике ваши всесожжения и благодарственные жертвы; и Я буду милостив к вам, говорит Господь Бог.
\vs Eze 44:1 И привел он меня обратно ко внешним воротам святилища, обращенным лицом на восток, и они были затворены.
\vs Eze 44:2 И сказал мне Господь: ворота сии будут затворены, не отворятся, и никакой человек не войдет ими, ибо Господь, Бог Израилев, вошел ими, и они будут затворены.
\vs Eze 44:3 Что до князя, он, \bibemph{как} князь, сядет в них, чтобы есть хлеб пред Господом; войдет путем притвора этих ворот, и тем же путем выйдет.
\vs Eze 44:4 Потом привел меня путем ворот северных перед лице храма, и я видел, и вот, слава Господа наполняла дом Господа, и пал я на лице мое.
\vs Eze 44:5 И сказал мне Господь: сын человеческий! прилагай сердце твое \bibemph{ко всему}, и смотри глазами твоими, и слушай ушами твоими все, что Я говорю тебе о всех постановлениях дома Господа и всех законах его; и прилагай сердце твое ко входу в храм и ко всем выходам из святилища.
\vs Eze 44:6 И скажи мятежному дому Израилеву: так говорит Господь Бог: довольно вам, дом Израилев, делать все мерзости ваши,
\vs Eze 44:7 вводить сынов чужой, необрезанных сердцем и необрезанных плотью, чтобы они были в Моем святилище и оскверняли храм Мой, подносить хлеб Мой, тук и кровь, и разрушать завет Мой всякими мерзостями вашими.
\vs Eze 44:8 Вы не исполняли стражи у святынь Моих, а ставили вместо себя их для стражи в Моем святилище.
\vs Eze 44:9 Так говорит Господь Бог: никакой сын чужой, необрезанный сердцем и необрезанный плотью, не должен входить во святилище Мое, даже и тот сын чужой, который \bibemph{живет} среди сынов Израиля.
\vs Eze 44:10 Равно и левиты, которые удалились от Меня во время отступничества Израилева, которые, оставив Меня, блуждали вслед идолов своих, понесут наказание за вину свою.
\vs Eze 44:11 Они будут служить во святилище Моем, как сторожа у ворот храма и прислужники у храма; они будут заколать для народа всесожжение и другие жертвы, и будут стоять пред ними для служения им.
\vs Eze 44:12 За то, что они служили им пред идолами их и были для дома Израилева соблазном к нечестию, Я поднял на них руку Мою, говорит Господь Бог, и они понесут наказание за вину свою;
\vs Eze 44:13 они не будут приближаться ко Мне, чтобы священнодействовать предо Мною и приступать ко всем святыням Моим, к Святому Святых, но будут нести на себе бесславие свое и мерзости свои, какие делали.
\vs Eze 44:14 Сделаю их стражами храма для всех служб его и для всего, что производится в нем.
\vs Eze 44:15 А священники из колена Левиина, сыны Садока, которые, во время отступления сынов Израилевых от Меня, постоянно стояли на страже святилища Моего, те будут приближаться ко Мне, чтобы служить Мне, и будут предстоять пред лицем Моим, чтобы приносить Мне тук и кровь, говорит Господь Бог.
\vs Eze 44:16 Они будут входить во святилище Мое и приближаться к трапезе Моей, чтобы служить Мне и соблюдать стражу Мою.
\vs Eze 44:17 Когда придут к воротам внутреннего двора, тогда оденутся в одежды льняные, а шерстяное не должно быть на них во время служения их в воротах внутреннего двора и внутри храма.
\vs Eze 44:18 Увясла на головах их должны быть также льняные; и исподняя одежда на чреслах их должна быть также льняная; в поту они не должны опоясываться.
\vs Eze 44:19 А когда надобно будет выйти на внешний двор, на внешний двор к народу, тогда они должны будут снять одежды свои, в которых они служили, и оставить их в священных комнатах, и одеться в другие одежды, чтобы священными одеждами своими не прикасаться к народу.
\vs Eze 44:20 И головы своей они не должны брить, и не должны отпускать волос, а пусть непременно стригут головы свои.
\vs Eze 44:21 И вина не должен пить ни один священник, когда идет во внутренний двор.
\vs Eze 44:22 Ни вдовы, ни разведенной с мужем они не должны брать себе в жены, а только могут брать себе девиц из племени дома Израилева и вдову, оставшуюся вдовою от священника.
\vs Eze 44:23 Они должны учить народ Мой отличать священное от несвященного и объяснять им, чт\acc{о} нечисто и чт\acc{о} чисто.
\vs Eze 44:24 При спорных делах они должны присутствовать в суде, и по уставам Моим судить их, и наблюдать законы Мои и постановления Мои о всех праздниках Моих, и свято хранить субботы Мои.
\vs Eze 44:25 К мертвому человеку никто из них не должен подходить, чтобы не сделаться нечистым; только ради отца и матери, ради сына и дочери, брата и сестры, которая не была замужем, можно им сделать себя нечистыми.
\vs Eze 44:26 По очищении же такого, еще семь дней надлежит отсчитать ему.
\vs Eze 44:27 И в тот день, когда ему надобно будет приступать ко святыне во внутреннем дворе, чтобы служить при святыне, он должен принести жертву за грех, говорит Господь Бог.
\vs Eze 44:28 А что до удела их, то Я их удел. И владения не давайте им в Израиле: Я их владение.
\vs Eze 44:29 Они будут есть от хлебного приношения, от жертвы за грех и жертвы за преступление; и все заклятое у Израиля им же принадлежит.
\vs Eze 44:30 И начатки из всех плодов ваших и всякого рода приношения, из чего ни состояли бы приношения ваши, принадлежат священникам; и начатки молотого вами отдавайте священнику, чтобы над домом твоим почивало благословение.
\vs Eze 44:31 Никакой мертвечины и ничего, растерзанного зверем, ни из птиц, ни из скота, не должны есть священники.
\vs Eze 45:1 Когда будете по жребию делить землю на уделы, тогда отделите священный участок Господу в двадцать пять тысяч \bibemph{тростей} длины и десять тысяч ширины; да будет свято это место во всем объеме своем, кругом.
\vs Eze 45:2 От него к святилищу отойдет четырехугольник по пятисот \bibemph{тростей} кругом, и кругом него площадь в пятьдесят локтей.
\vs Eze 45:3 Из этой меры отмерь двадцать пять тысяч \bibemph{тростей} в длину и десять тысяч в ширину, где будет находиться святилище, Святое Святых.
\vs Eze 45:4 Эта священная часть земли принадлежать будет священникам, служителям святилища, приступающим к служению Господу: это будет для них местом для домов и святынею для святилища.
\vs Eze 45:5 Двадцать пять тысяч \bibemph{тростей} длины и десять тысяч ширины будут принадлежать левитам, служителям храма, как их владение для обитания их.
\vs Eze 45:6 И во владение городу дайте пять тысяч ширины и двадцать пять тысяч длины, против священного места, отделенного Господу; это принадлежать должно всему дому Израилеву.
\vs Eze 45:7 И князю \bibemph{дайте} долю по ту и другую сторону, как подле священного места, отделенного \bibemph{Господу}, так и подле городского владения, к западу с западной стороны и к востоку с восточной стороны, длиною наравне с одним из оных уделов от западного предела до восточного.
\vs Eze 45:8 Это его земля, его владение в Израиле, чтобы князья Мои вперед не теснили народа Моего и чтобы предоставили землю дому Израилеву по коленам его.
\vs Eze 45:9 Так говорит Господь Бог: довольно вам, князья Израилевы! отложите обиды и угнетения и творите суд и правду, перестаньте вытеснять народ Мой из владения его, говорит Господь Бог.
\vs Eze 45:10 Да будут у вас правильные весы и правильная ефа и правильный бат.
\vs Eze 45:11 Ефа и бат должны быть одинаковой меры, так чтобы бат вмещал в себе десятую часть хомера и ефа десятую часть хомера; мера их должна определяться по хомеру.
\vs Eze 45:12 В сикле двадцать гер; а двадцать сиклей, двадцать пять сиклей и пятнадцать сиклей составлять будут у вас мину.
\vs Eze 45:13 Вот дань, какую вы должны давать \bibemph{князю}: шестую часть ефы от хомера пшеницы и шестую часть ефы от хомера ячменя;
\vs Eze 45:14 постановление об елее: от кора елея десятую часть бата; десять батов \bibemph{составят} хомер, потому что в хомере десять батов;
\vs Eze 45:15 одну овцу от стада в двести овец с тучной пажити Израиля: все это для хлебного приношения и всесожжения, и благодарственной жертвы, в очищение их, говорит Господь Бог.
\vs Eze 45:16 Весь народ земли обязывается делать сие приношение князю в Израиле.
\vs Eze 45:17 А на обязанности князя будут лежать всесожжение и хлебное приношение, и возлияние в праздники и в новомесячия, и в субботы, во все торжества дома Израилева; он должен будет приносить жертву за грех и хлебное приношение, и всесожжение, и жертву благодарственную для очищения дома Израилева.
\vs Eze 45:18 Так говорит Господь Бог: в первом \bibemph{месяце}, в первый \bibemph{день} месяца, возьми из стада волов тельца без порока, и очисти святилище.
\vs Eze 45:19 Священник пусть возьмет крови от этой жертвы за грех и покропит ею на вереи храма и на четыре угла площадки у жертвенника и на вереи ворот внутреннего двора.
\vs Eze 45:20 То же сделай и в седьмой \bibemph{день} месяца за согрешающих умышленно и по простоте, и так очищайте храм.
\vs Eze 45:21 В первом \bibemph{месяце}, в четырнадцатый день месяца, должна быть у вас Пасха, праздник семидневный, когда должно есть опресноки.
\vs Eze 45:22 В этот день князь за себя и за весь народ земли принесет тельца в жертву за грех.
\vs Eze 45:23 И в эти семь дней праздника он должен приносить во всесожжение Господу каждый день по семи тельцов и по семи овнов без порока, и в жертву за грех каждый день по козлу из козьего стада.
\vs Eze 45:24 Хлебного приношения он должен приносить по ефе на тельца и по ефе на овна и по гину елея на ефу.
\vs Eze 45:25 В седьмом \bibemph{месяце}, в пятнадцатый день месяца, в праздник, в течение семи дней он должен приносить то же: такую же жертву за грех, такое же всесожжение, и столько же хлебного приношения и столько же елея.
\vs Eze 46:1 Так говорит Господь Бог: ворота внутреннего двора, обращенные лицом к востоку, должны быть заперты в продолжение шести рабочих дней, а в субботний день они должны быть отворены и в день новомесячия должны быть отворены.
\vs Eze 46:2 Князь пойдет через внешний притвор ворот и станет у вереи этих ворот; и священники совершат его всесожжение и его благодарственную жертву; и он у порога ворот поклонится \bibemph{Господу}, и выйдет, а ворота остаются незапертыми до вечера.
\vs Eze 46:3 И народ земли будет поклоняться пред Господом, при входе в ворота, в субботы и новомесячия.
\vs Eze 46:4 Всесожжение, которое князь принесет Господу в субботний день, должно быть из шести агнцев без порока и из овна без порока;
\vs Eze 46:5 хлебного приношения ефа на овна, а на агнцев хлебного приношения, сколько рука его подаст, а елея гин на ефу.
\vs Eze 46:6 В день новомесячия будут приносимы им из стада волов телец без порока, также шесть агнцев и овен без порока.
\vs Eze 46:7 Хлебного приношения он принесет ефу на тельца и ефу на овна, а на агнцев, сколько рука его подаст, и елея гин на ефу.
\vs Eze 46:8 И когда приходить будет князь, то должен входить через притвор ворот и тем же путем выходить.
\vs Eze 46:9 А когда народ земли будет приходить пред лице Господа в праздники, то вошедший северными воротами для поклонения должен выходить воротами южными, а вошедший южными воротами должен выходить воротами северными; он не должен выходить теми же воротами, которыми вошел, а должен выходить противоположными.
\vs Eze 46:10 И князь должен находиться среди них; когда они входят, входит и он; и когда они выходят, выходит и он.
\vs Eze 46:11 И в праздники и в торжественные дни хлебного приношения \bibemph{от него} должно быть по ефе на тельца и по ефе на овна, а на агнцев, сколько подаст рука его, и елея по гину на ефу.
\vs Eze 46:12 А если князь, по усердию своему, захочет принести всесожжение или благодарственную жертву Господу, то должны отворить ему ворота, обращенные к востоку, и он совершит свое всесожжение и свою благодарственную жертву так же, как совершил в субботний день, и после сего он выйдет, и по выходе его ворота запрутся.
\vs Eze 46:13 Каждый день приноси Господу во всесожжение однолетнего агнца без порока; каждое утро приноси его.
\vs Eze 46:14 А хлебного приношения прилагай к нему каждое утро шестую часть ефы и елея третью часть гина, чтобы растворить муку; таково вечное постановление о хлебном приношении Господу, навсегда.
\vs Eze 46:15 Пусть приносят во всесожжение агнца и хлебное приношение и елей каждое утро постоянно.
\vs Eze 46:16 Так говорит Господь Бог: если князь дает кому из сыновей своих подарок, то это должно пойти в наследство и его сыновьям; это владение их должно быть наследственным.
\vs Eze 46:17 Если же он даст из наследия своего кому-либо из рабов своих подарок, то это будет принадлежать ему только до года освобождения, и тогда возвратится к князю. Только к сыновьям его должно переходить наследие его.
\vs Eze 46:18 Но князь не может брать из наследственного участка народа, вытесняя их из владения их; из своего только владения он может уделять детям своим, чтобы никто из народа Моего не был изгоняем из своего владения.
\vs Eze 46:19 И привел он меня тем ходом, который сбоку ворот, к священным комнатам для священников, обращенным к северу, и вот там одно место на краю к западу.
\vs Eze 46:20 И сказал мне: <<это~--- место, где священники должны варить жертву за преступление и жертву за грех, где должны печь хлебное приношение, не вынося его на внешний двор, для освящения народа>>.
\vs Eze 46:21 И вывел меня на внешний двор, и провел меня по четырем углам двора, и вот, в каждом углу двора еще двор.
\vs Eze 46:22 Во всех четырех углах двора были покрытые дворы в сорок \bibemph{локтей} длины и тридцать ширины, одной меры во всех четырех углах.
\vs Eze 46:23 И кругом всех их четырех~--- стены, а у стен сделаны очаги кругом.
\vs Eze 46:24 И сказал мне: <<вот поварни, в которых служители храма варят жертвы народные>>.
\vs Eze 47:1 Потом привел он меня обратно к дверям храма, и вот, из-под порога храма течет вода на восток, ибо храм стоял лицом на восток, и вода текла из-под правого бока храма, по южную сторону жертвенника.
\vs Eze 47:2 И вывел меня северными воротами, и внешним путем обвел меня к внешним воротам, путем, обращенным к востоку; и вот, вода течет по правую сторону.
\vs Eze 47:3 Когда тот муж пошел на восток, то в руке держал шнур, и отмерил тысячу локтей, и повел меня по воде; воды было по лодыжку.
\vs Eze 47:4 И \bibemph{еще} отмерил тысячу, и повел меня по воде; воды было по колено. И еще отмерил тысячу, и повел меня; воды было по поясницу.
\vs Eze 47:5 И еще отмерил тысячу, и уже тут был такой поток, через который я не мог идти, потому что вода была так высока, что надлежало плыть, а переходить нельзя было этот поток.
\vs Eze 47:6 И сказал мне: <<видел, сын человеческий?>> и повел меня обратно к берегу этого потока.
\vs Eze 47:7 И когда я пришел назад, и вот, на берегах потока много было дерев по ту и другую сторону.
\vs Eze 47:8 И сказал мне: эта вода течет в восточную сторону земли, сойдет на равнину и войдет в море; и воды его сделаются здоровыми.
\vs Eze 47:9 И всякое живущее существо, пресмыкающееся там, где войдут две струи, будет живо; и рыбы будет весьма много, потому что войдет туда эта вода, и в\acc{о}ды \bibemph{в море} сделаются здоровыми, и, куда войдет этот поток, все будет живо там.
\vs Eze 47:10 И будут стоять подле него рыболовы от Ен-Гадди до Эглаима, будут закидывать сети. Рыба будет в своем виде и, как в большом море, рыбы будет весьма много.
\vs Eze 47:11 Болота его и лужи его, которые не сделаются здоровыми, будут оставлены для соли.
\vs Eze 47:12 У потока по берегам его, с той и другой стороны, будут расти всякие дерева, доставляющие пищу: листья их не будут увядать, и плоды на них не будут истощаться; каждый месяц будут созревать новые, потому что вода для них течет из святилища; плоды их будут употребляемы в пищу, а листья на врачевание.
\rsbpar\vs Eze 47:13 Так говорит Господь Бог: вот распределение, по которому вы должны разделить землю в наследие двенадцати коленам Израилевым: Иосифу два удела.
\vs Eze 47:14 И наследуйте ее, как один, так и другой; так как Я, подняв руку Мою, клялся отдать ее отцам вашим, то и будет земля сия наследием вашим.
\vs Eze 47:15 И вот предел земли: на северном конце, начиная от великого моря, через Хетлон, по дороге в Цедад,
\vs Eze 47:16 Емаф, Берот, Сивраим, находящийся между Дамасскою и Емафскою областями Гацар-Тихон, который на границе Аврана.
\vs Eze 47:17 И будет граница от моря до Гацар-Енон, граница с Дамаском, и далее на севере область Емаф; и вот северный край.
\vs Eze 47:18 Черту восточного края ведите между Авраном и Дамаском, между Галаадом и землею Израильскою, по Иордану, от северного края до восточного моря; это восточный край.
\vs Eze 47:19 А южный край с полуденной стороны от Тамары до вод пререкания при Кадисе, и по течению потока до великого моря; это полуденный край на юге.
\vs Eze 47:20 Западный же предел~--- великое море, от южной границы до места против Емафа; это западный край.
\vs Eze 47:21 И разделите себе землю сию на уделы по коленам Израилевым.
\vs Eze 47:22 И разделите ее по жребию в наследие себе и иноземцам, живущим у вас, которые родили у вас детей; и они среди сынов Израилевых должны считаться наравне с природными жителями, и они с вами войдут в долю среди колен Израилевых.
\vs Eze 47:23 В котором колене живет иноземец, в том и дайте ему наследие его, говорит Господь Бог.
\vs Eze 48:1 Вот имена колен. На северном краю по дороге от Хетлона, ведущей в Емаф, Гацар-Енон, от северной границы Дамаска по пути к Емафу: все это от востока до моря один удел Дану.
\vs Eze 48:2 Подле границы Дана, от восточного края до западного, это один удел Асиру.
\vs Eze 48:3 Подле границы Асира, от восточного края до западного, это один удел Неффалиму.
\vs Eze 48:4 Подле границы Неффалима, от восточного края до западного, это один удел Манассии.
\vs Eze 48:5 Подле границы Манассии, от восточного края до западного, это один удел Ефрему.
\vs Eze 48:6 Подле границы Ефрема, от восточного края до западного, это один удел Рувиму.
\vs Eze 48:7 Подле границы Рувима, от восточного края до западного, это один удел Иуде.
\vs Eze 48:8 А подле границы Иуды, от восточного края до западного, священный участок, шириною в двадцать пять тысяч \bibemph{тростей}, а длиною наравне с другими уделами, от восточного края до западного; среди него будет святилище.
\vs Eze 48:9 Участок, который вы посвятите Господу, длиною будет в двадцать пять тысяч, а шириною в десять тысяч \bibemph{тростей}.
\vs Eze 48:10 И этот священный участок должен принадлежать священникам, к северу двадцать пять тысяч и к морю в ширину десять тысяч, и к востоку в ширину десять тысяч, а к югу в длину двадцать пять тысяч \bibemph{тростей}, и среди него будет святилище Господне.
\vs Eze 48:11 Это посвятите священникам из сынов Садока, которые стояли на страже Моей, которые во время отступничества сынов Израилевых не отступили от Меня, как отступили \bibemph{другие} левиты.
\vs Eze 48:12 Им будет принадлежать эта часть земли из священного участка, святыня из святынь, у предела левитов.
\vs Eze 48:13 И левиты получат также у священнического предела двадцать пять тысяч в длину и десять тысяч \bibemph{тростей} в ширину; вся длина двадцать пять тысяч, а ширина десять тысяч \bibemph{тростей}.
\vs Eze 48:14 И из этой части они не могут ни продать, ни променять; и начатки земли не могут переходить к другим, потому что это святыня Господня.
\vs Eze 48:15 А остальные пять тысяч в ширину с двадцатью пятью тысячами \bibemph{в длину} назначаются для города в общее употребление, на заселение и на предместья; город будет в средине.
\vs Eze 48:16 И вот размеры его: северная сторона четыре тысячи пятьсот и южная сторона четыре тысячи пятьсот, восточная сторона четыре тысячи пятьсот и западная сторона четыре тысячи пятьсот \bibemph{тростей}.
\vs Eze 48:17 А предместья города к северу двести пятьдесят, и к востоку двести пятьдесят, и к югу двести пятьдесят, и к западу двести пятьдесят \bibemph{тростей}.
\vs Eze 48:18 А что остается из длины против священного участка, десять тысяч к востоку и десять тысяч к западу, против священного участка, произведения с этой земли должны быть для продовольствия работающих в городе.
\vs Eze 48:19 Работать же в городе могут работники из всех колен Израилевых.
\vs Eze 48:20 Весь отделенный участок в двадцать пять тысяч длины и в двадцать пять тысяч ширины, четырехугольный, выделите в священный удел, со включением владений города;
\vs Eze 48:21 а остальное князю. Как со стороны священного участка, так и со стороны владений города, против двадцати пяти тысяч \bibemph{тростей} до восточной границы участка, и на запад против двадцати пяти тысяч у западной границы соразмерно с сими уделами, удел князю, так что священный участок и святилище будет в средине его.
\vs Eze 48:22 И то, что от владений левитских \bibemph{и} от владений города остается в промежутке, принадлежит также князю; промежуток между границею Иуды и между границею Вениамина будет принадлежать князю.
\vs Eze 48:23 Остальное же от колен, от восточного края до западного~--- один удел Вениамину.
\vs Eze 48:24 Подле границы Вениамина, от восточного края до западного~--- один удел Симеону.
\vs Eze 48:25 Подле границы Симеона, от восточного края до западного~--- один удел Иссахару.
\vs Eze 48:26 Подле границы Иссахара, от восточного края до западного~--- один удел Завулону.
\vs Eze 48:27 Подле границы Завулона, от восточного края до западного~--- один удел Гаду.
\vs Eze 48:28 А подле границы Гада на южной стороне идет южный предел от Тамары к водам пререкания при Кадисе, вдоль потока до великого моря.
\vs Eze 48:29 Вот земля, которую вы по жребию разделите коленам Израилевым, и вот участки их, говорит Господь Бог.
\vs Eze 48:30 И вот выходы города: с северной стороны меры четыре тысячи пятьсот;
\vs Eze 48:31 и ворота города называются именами колен Израилевых; к северу трое ворот: ворота Рувимовы одни, ворота Иудины одни, ворота Левиины одни.
\vs Eze 48:32 И с восточной стороны \bibemph{меры} четыре тысячи пятьсот, и трое ворот: ворота Иосифовы одни, ворота Вениаминовы одни, ворота Дановы одни;
\vs Eze 48:33 и с южной стороны меры четыре тысячи пятьсот, и трое ворот: ворота Симеоновы одни, ворота Иссахаровы одни, ворота Завулоновы одни.
\vs Eze 48:34 С морской стороны \bibemph{меры} четыре тысячи пятьсот, ворот здесь трое же: ворота Гадовы одни, ворота Асировы одни, ворота Неффалимовы одни.
\vs Eze 48:35 Всего кругом восемнадцать тысяч. А имя городу с того дня будет: <<Господь там>>.

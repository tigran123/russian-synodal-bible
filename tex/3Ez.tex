\bibbookdescr{3Ez}{
  inline={\LARGE Третья книга\\\Huge Ездры\fns{Книги этой нет ни на еврейском, ни на греческом языках. Как славянский, так и русский переводы сделаны с Вульгаты. В последней она разделена на две книги: первую составляют главы 3--14 по славянскому переводу, а вторая заключает в себе главы 1, 2, 15 и 16. В русском переводе удержан порядок глав славянского перевода.}},
  toc={3-я Ездры*},
  bookmark={3-я Ездры},
  header={3-я Ездры},
  %headerleft={},
  %headerright={},
  abbr={3~Езд}
}
\vs 3Ez 1:1 Вторая книга Ездры пророка, сына Сераии, сына Азарии, сына Хелкии, сына Шаллума, сына Садока, сына Ахитува,
\vs 3Ez 1:2 сына Ахии, сына Финееса, сына Илия, сына Амарии, сына Асиела, сына Мерайофа, сына Арна, сына Уззия, сына Ворифа, сына Авишуя, сына Финееса, сына Елеазара,
\vs 3Ez 1:3 сына Аарона от колена Левиина, который был пленником в стране Мидийской, в царствование Артаксеркса, царя Персидского.
\rsbpar\vs 3Ez 1:4 Было слово Господне ко мне:
\vs 3Ez 1:5 иди и возвести народу Моему злые дела их и сыновьям их~--- беззакония, которые они совершили против Меня, чтобы они возвестили сынам сынов своих;
\vs 3Ez 1:6 ибо грехи родителей их возросли в них; забыв Меня, они приносили жертвы богам чужим.
\vs 3Ez 1:7 Не Я ли вывел их из земли Египетской, из дома рабства? а они прогневали Меня и советы Мои презрели.
\vs 3Ez 1:8 Ты остриги волосы головы твоей, и брось на них все злое, ибо они не слушались закона моего~--- народ необузданный!
\vs 3Ez 1:9 Доколе Я буду терпеть их, которым сделал столько благодеяний?
\vs 3Ez 1:10 Ради них Я многих царей низложил; поразил фараона с рабами его и со всем войском его;
\vs 3Ez 1:11 всех язычников от лица их погубил, и на востоке народ двух областей, Тира и Сидона, рассеял и всех врагов их истребил.
\vs 3Ez 1:12 Ты же так скажи им: так говорит Господь:
\vs 3Ez 1:13 именно Я провел вас через море и по дну его проложил вам огражденную улицу, дал вам вождя Моисея и Аарона священника,
\vs 3Ez 1:14 дал вам свет в столпе огненном, и многие чудеса сотворил среди вас; а вы Меня забыли, говорит Господь.
\rsbpar\vs 3Ez 1:15 Так говорит Господь Вседержитель: перепелы были вам в знамение. Я дал вам станы для защиты, но вы и там роптали
\vs 3Ez 1:16 и не радовались во имя Мое о погибели врагов ваших, но даже доныне еще ропщете.
\vs 3Ez 1:17 Где те благодеяния, которые Я сделал вам? Не в пустыне ли, когда вы, взалкав, вопияли ко Мне,
\vs 3Ez 1:18 говоря: <<зачем Ты привел нас в эту пустыню? уморить нас? лучше нам было служить Египтянам, нежели умереть в этой пустыне>>?
\vs 3Ez 1:19 Я сжалился на стенания ваши, и дал вам манну в пищу: вы ели хлеб ангельский.
\vs 3Ez 1:20 Когда вы жаждали, не рассек ли Я камень, и потекли воды до сытости? от зноя покрывал вас листьями древесными.
\vs 3Ez 1:21 Разделил вам земли тучные; Хананеев, Ферезеев и Филистимлян изгнал от лица вашего. Что еще сделаю вам? говорит Господь.
\vs 3Ez 1:22 Так говорит Господь Вседержитель: когда вы были в пустыне, на реке Мерры, и жаждущие хулили имя Мое,
\vs 3Ez 1:23 не огонь послал Я на вас за богохульства, но вложил дерево в воду и реку сделал сладкою.
\vs 3Ez 1:24 Что сделаю тебе, Иаков? Не хотел ты повиноваться, Иуда. Переселюсь к другим народам и дам им имя Мое, чтобы соблюдали законы Мои.
\vs 3Ez 1:25 Так как вы Меня оставили, то и Я оставлю вас; просящих у Меня милости не помилую.
\vs 3Ez 1:26 Когда будете призывать Меня, Я не услышу вас, ибо вы осквернили руки ваши кровью, и ноги ваши быстры на совершение человекоубийства.
\vs 3Ez 1:27 Вы как бы не Меня оставили, а вас самих, говорит Господь.
\vs 3Ez 1:28 Так говорит Господь Вседержитель: не Я ли умолял вас, как отец сыновей и как мать дочерей и как кормилица питомцев своих,
\vs 3Ez 1:29 чтобы вы были Мне народом и Я вам Богом, чтобы вы были Мне сынами и Я вам Отцом?
\vs 3Ez 1:30 Я собрал вас, как курица птенцов своих под крылья свои. Что ныне сделаю вам? Отвергну вас от лица Моего.
\vs 3Ez 1:31 Когда принесете Мне приношение, отвращу лице Мое от вас; ибо ваши дни праздничные и новомесячия и обрезания Я отринул.
\vs 3Ez 1:32 Я послал к вам рабов Моих, пророков; вы, схватив их, умертвили и растерзали тела их. Кровь их Я взыщу, говорит Господь.
\rsbpar\vs 3Ez 1:33 Так говорит Господь Вседержитель: дом ваш пуст. Развею вас, как ветер мякину,
\vs 3Ez 1:34 и сыновья не будут иметь потомства, потому что заповедь Мою презрели и делали то, что зло предо Мною.
\vs 3Ez 1:35 Предам домы ваши людям грядущим, которые, не слышав Меня, уверуют, которые, хотя Я не показывал им знамений, исполнят то, что Я заповедал,
\vs 3Ez 1:36 не видев пророков, воспомянут о своих беззакониях.
\vs 3Ez 1:37 Завещеваю благодать людям грядущим, дети которых, не видев Меня очами плотскими, но духом веруя тому, что Я сказал, торжествуют с весельем.
\vs 3Ez 1:38 Итак теперь смотри, брат, какая слава,~--- смотри на людей, грядущих с востока,
\vs 3Ez 1:39 которым Я дам в вожди Авраама, Исаака и Иакова, и Осию, и Амоса, и Михея, и Иоиля, и Авдия, и Иону,
\vs 3Ez 1:40 и Наума, и Аввакума, Софонию, Аггея, Захарию и Малахию, который наречен и Ангелом Господним.
\vs 3Ez 2:1 Так говорит Господь: Я вывел народ сей из работы, дал им повеление через рабов Моих, пророков, которых они не захотели слушать, но отвергли Мои советы.
\vs 3Ez 2:2 Мать, которая родила их, говорит им: <<идите, дети; ибо я вдова и оставлена.
\vs 3Ez 2:3 Я воспитала вас с радостью, и отпустила с плачем и горестью, потому что вы согрешили пред Господом Богом вашим, и сделали злое пред Ним.
\vs 3Ez 2:4 Ныне же что сделаю для вас? Я вдова и оставлена: идите, дети, и просите у Господа милости>>.
\vs 3Ez 2:5 Тебя, Отче, призываю во свидетеля на мать сыновей, которые не захотели хранить завета моего.
\vs 3Ez 2:6 Предай их посрамлению и мать их~--- на расхищение, чтобы не было рода их.
\vs 3Ez 2:7 Пусть рассеются имена их по народам и изгладятся от земли, ибо они презрели завет мой.
\vs 3Ez 2:8 Горе тебе, Ассур, скрывающий у себя нечестивых! Род лукавый! вспомни, что Я сделал Содому и Гоморре.
\vs 3Ez 2:9 Земля их лежит в смоляных глыбах и холмах пепельных. Так поступлю Я с теми, которые Меня не слушались, говорит Господь Вседержитель.
\vs 3Ez 2:10 Так говорит Господь к Ездре: возвести народу Моему, что Я дам им царство Иерусалимское, которое обещал Израилю,
\vs 3Ez 2:11 и прииму славу от них и дам им обители вечные, которые приготовил для них.
\vs 3Ez 2:12 Древо жизни будет для них мастью благовонною; не будут изнуряемы трудом и не изнемогут.
\vs 3Ez 2:13 Идите и получ\acc{и}те; прос\acc{и}те себе дней малых, дабы они не замедлили. Уже готово для вас царство: бодрствуйте.
\vs 3Ez 2:14 Свидетельствуй, небо и земля, ибо Я стер злое и сотворил доброе. Живу Я! говорит Господь.
\vs 3Ez 2:15 Мать! обними сыновей твоих, воспитывай их с радостью; как голубица укрепляй ноги их, ибо Я избрал тебя, говорит Господь.
\vs 3Ez 2:16 И воскрешу мертвых от мест их и из гробов выведу их, потому что Я познал имя Мое в Израиле.
\vs 3Ez 2:17 Не бойся, мать сынов, ибо Я избрал тебя, говорит Господь.
\vs 3Ez 2:18 Я пошлю тебе в помощь рабов Моих Исаию и Иеремию, по совету которых Я освятил и приготовил тебе двенадцать дерев, обремененных различными плодами,
\vs 3Ez 2:19 и столько же источников, текущих молоком и медом, и семь гор величайших, произращающих розу и лилию, через которые исполню радостью сынов твоих.
\vs 3Ez 2:20 Оправдай вдову, дай суд бедному, помоги нищему, защити сироту, одень нагого,
\vs 3Ez 2:21 о расслабленном и немощном попекись, над хромым не смейся, безрукого защити, и слепого приведи к видению света Моего,
\vs 3Ez 2:22 старца и юношу в стенах твоих сохрани,
\vs 3Ez 2:23 мертвых, где найдешь, запечатлев, предай гробу, и Я дам тебе первое место в Моем воскресении.
\vs 3Ez 2:24 Отдыхай и покойся, народ Мой, ибо придет покой твой.
\vs 3Ez 2:25 Корми сынов твоих, добрая кормилица, укрепляй ноги их.
\vs 3Ez 2:26 Из рабов, которых Я дал тебе, никто да не погибнет, ибо Я взыщу их от тебя.
\vs 3Ez 2:27 Не ослабевай. Когда придет день печали и тесноты, другие будут плакать и сокрушаться, а ты будешь весела и изобильна.
\vs 3Ez 2:28 Язычники будут завидовать тебе, но ничего против тебя сделать не могут, говорит Господь.
\vs 3Ez 2:29 Руки Мои покроют тебя, чтобы сыны твои не видели геенны.
\vs 3Ez 2:30 Утешайся, мать, с сынами твоими, ибо Я спасу тебя.
\vs 3Ez 2:31 Помни о сынах твоих почивающих. Я выведу их от краев земли и окажу им милость, ибо Я милостив, говорит Господь Вседержитель.
\vs 3Ez 2:32 Обними детей твоих, доколе Я приду и сделаю им милость; ибо источники Мои обильны и благодать Моя не оскудеет.
\rsbpar\vs 3Ez 2:33 Я, Ездра, получил на горе Орив повеление от Господа идти к Израилю. Когда я пришел к ним, они отвергли меня и презрели заповедь Господню.
\vs 3Ez 2:34 Посему вам говорю, язычники, которые можете слышать и понимать: ожидайте Пастыря вашего, Он даст вам покой вечный, ибо близко Тот, Который придет в скончание века.
\vs 3Ez 2:35 Будьте готовы к воздаянию царствия, ибо свет немерцающий воссияет вам на вечное время.
\vs 3Ez 2:36 Избегайте тени века сего; приимите сладость славы вашей. Я открыто свидетельствую о Спасителе моем.
\vs 3Ez 2:37 Вверенный дар приимите, и наслаждайтесь, благодаря Того, Кто призвал вас в небесное царство.
\vs 3Ez 2:38 Встаньте и стойте, и смотрите, какое число знаменованных на вечери Господней,
\vs 3Ez 2:39 которые, переселившись от тени века сего, получили от Господа светлые одежды.
\vs 3Ez 2:40 Приими число твое, Сион, и заключи твоих, одетых в белые одеяния, которые исполнили закон Господень.
\vs 3Ez 2:41 Число желанных сынов твоих полно. Проси державу Господа, чтобы освятился народ твой, призванный от начала.
\rsbpar\vs 3Ez 2:42 Я, Ездра, видел на горе Сионской сонм великий, которого не мог исчислить, и все они песнями прославляли Господа.
\vs 3Ez 2:43 Посреди них был юноша величественный, превосходящий всех их, и возлагал венцы на главу каждого из них и тем более возвышался; я поражен был удивлением.
\vs 3Ez 2:44 Тогда я спросил Ангела: кто сии, господин мой?
\vs 3Ez 2:45 Он в ответ мне сказал: это те, которые сложили смертную одежду и облеклись в бессмертную и исповедали имя Божие; они теперь увенчиваются и принимают победные пальмы.
\vs 3Ez 2:46 Я спросил: а кто сей юноша, который возлагает на них венцы и вручает им пальмы?
\vs 3Ez 2:47 Он отвечал мне: Сам Сын Божий, Которого они прославляли в веке сем. И я начал славить их, мужественно стоявших за имя Господне.
\vs 3Ez 2:48 Тогда Ангел сказал мне: иди и возвести народу моему, какие видел ты дивные дела Господа Бога.
\vs 3Ez 3:1 В тридцатом году по разорении города был я в Вавилоне, и смущался, лежа на постели моей, и помышления всходили на сердце мое,
\vs 3Ez 3:2 ибо я видел опустошение Сиона и богатство живущих в Вавилоне.
\vs 3Ez 3:3 И возмутился дух мой, и я начал со страхом говорить ко Всевышнему,
\vs 3Ez 3:4 и сказал: Владыко Господи! Ты сказал от начала, когда един основал землю, и повелел персти,
\vs 3Ez 3:5 и дал Адаму тело смертное, которое было также создание рук Твоих, и вдохнул в него дух жизни, и он сделался живым пред Тобою,
\vs 3Ez 3:6 и ввел его в рай, который насадила десница Твоя, прежде нежели земля произрастила плоды;
\vs 3Ez 3:7 Ты повелел ему хранить заповедь Твою, но он нарушил ее, и Ты осудил его на смерть, и род его и происшедшие от него поколения и племена, народы и отрасли их, которым нет числа.
\vs 3Ez 3:8 Каждый народ стал ходить по своему хотению, делал пред Тобою дела неразумные и презирал заповеди Твои.
\vs 3Ez 3:9 По времени, Ты навел потоп на обитателей земли и истребил их,
\vs 3Ez 3:10 и исполнилось на каждом из них,~--- как на Адаме смерть, так на сих потоп.
\vs 3Ez 3:11 Одного из них Ты оставил~--- Ноя с семейством его, и от него произошли все праведные.
\vs 3Ez 3:12 Когда начали размножаться обитающие на земле, и умножились сыны и народы и поколения многие, и опять начали предаваться нечестию, более нежели прежние,
\vs 3Ez 3:13 когда начали делать пред Тобою беззаконие: Ты избрал Себе из них мужа, которому имя Авраам,
\vs 3Ez 3:14 и возлюбил его и открыл ему одному волю Твою,
\vs 3Ez 3:15 и положил ему завет вечный, и сказал ему, что никогда не оставишь семени его. И дал ему Исаака, и Исааку дал Иакова и Исава;
\vs 3Ez 3:16 Ты избрал Себе Иакова, Исава же отринул. И умножился Иаков чрезвычайно.
\vs 3Ez 3:17 Когда Ты вывел из Египта семя его и привел к горе Синайской,
\vs 3Ez 3:18 тогда преклонил небеса, уставил землю, поколебал вселенную, привел в трепет бездны и весь мир в смятение.
\vs 3Ez 3:19 И прошла слава Твоя в четырех \bibemph{явлениях}: в огне, землетрясении, бурном ветре и морозе, чтобы дать закон семени Иакова и радение роду Израиля,
\vs 3Ez 3:20 но не отнял у них сердца лукавого, чтобы закон Твой принес в них плод.
\vs 3Ez 3:21 С сердцем лукавым первый Адам преступил заповедь, и побежден был; так и все, от него происшедшие.
\vs 3Ez 3:22 Осталась немощь и закон в сердце народа с корнем зла, и отступило доброе, и осталось злое.
\vs 3Ez 3:23 Прошли времена и окончились лета,~--- и Ты воздвиг Себе раба, именем Давида;
\vs 3Ez 3:24 повелел ему построить город имени Твоему и в нем приносить Тебе фимиам и жертвы.
\vs 3Ez 3:25 Много лет это исполнялось, и потом согрешили населяющие город,
\vs 3Ez 3:26 во всем поступая так, как поступил Адам и все его потомки; ибо и у них было сердце лукавое.
\vs 3Ez 3:27 И Ты предал город Твой в руки врагов Твоих.
\vs 3Ez 3:28 Неужели лучше живут обитатели Вавилона и за это владеют Сионом?
\vs 3Ez 3:29 Когда я пришел сюда, видел нечестия, которым нет числа, и в этом тридцатом году пленения видит душа моя многих грешников,~--- и изныло сердце мое,
\vs 3Ez 3:30 ибо я видел, как Ты поддерживаешь сих грешников и щадишь нечестивцев, а народ Твой погубил, врагов же Твоих сохранил и не явил о том никакого знамения.
\vs 3Ez 3:31 Не понимаю, как этот путь мог измениться. Неужели Вавилон поступает лучше, нежели Сион?
\vs 3Ez 3:32 Или иной народ познал Тебя, кроме Израиля? или какие племена веровали заветам Твоим, как Иаков?
\vs 3Ez 3:33 Ни воздаяние им не равномерно, ни труд их не принес плода, ибо я прошел среди народов, и видел, что они живут в изобилии, хотя и не вспоминают о заповедях Твоих.
\vs 3Ez 3:34 Итак взвесь на весах и наши беззакония и дела живущих на земле, и нигде не найдется имя Твое, как только у Израиля.
\vs 3Ez 3:35 Когда не грешили пред Тобою живущие на земле? или какой народ так сохранил заповеди Твои?
\vs 3Ez 3:36 Между сими хотя по именам найдешь хранящих заповеди Твои, а у других народов не найдешь.
\vs 3Ez 4:1 Тогда отвечал мне посланный ко мне Ангел, которому имя Уриил,
\vs 3Ez 4:2 и сказал: сердце твое слишком далеко зашло в этом веке, что ты помышляешь постигнуть путь Всевышнего.
\vs 3Ez 4:3 Я отвечал: так, господин мой. Он же сказал мне: три пути послан я показать тебе и три подобия предложить тебе.
\vs 3Ez 4:4 Если ты одно из них объяснишь мне, то и я покажу тебе путь, который желаешь ты видеть, и научу тебя, откуда произошло сердце лукавое.
\vs 3Ez 4:5 Тогда я сказал: говори, господин мой. Он же сказал мне: иди и взвесь тяжесть огня, или измерь мне дуновение ветра, или возврати мне день, который уже прошел.
\vs 3Ez 4:6 Какой человек, отвечал я, может сделать то, чего ты требуешь от меня?
\vs 3Ez 4:7 А он сказал мне: если бы я спросил тебя, сколько обиталищ в сердце морском, или сколько источников в самом основании бездны, или сколько жил над твердью, или какие пределы у рая,
\vs 3Ez 4:8 ты, может быть, сказал бы мне: <<в бездну я не сходил, и в ад также, и на небо никогда не восходил>>.
\vs 3Ez 4:9 Теперь же я спросил тебя только об огне, ветре и дне, который ты пережил, и о том, без чего ты быть не можешь, и на это ты не отвечал мне.
\vs 3Ez 4:10 И сказал мне: ты и того, что твое и с тобою от юности, не можешь познать;
\vs 3Ez 4:11 как же сосуд твой мог бы вместить в себе путь Всевышнего и в этом уже заметно растленном веке понять растление, которое очевидно в глазах моих?
\vs 3Ez 4:12 На это сказал я: лучше было бы нам вовсе не быть, нежели жить в нечестиях и страдать, не зная, почему.
\vs 3Ez 4:13 Он же в ответ сказал мне: вот, я отправился в полевой лес, и застал дерева держащими совет.
\vs 3Ez 4:14 Они говорили: <<придите, и пойдем и объявим войну морю, чтобы оно отступило перед нами, и мы там возрастим для себя другие леса>>.
\vs 3Ez 4:15 Подобным образом и волны морские имели совещание: <<придите>>, говорили они, <<поднимемся и завоюем леса полевые, чтобы и там приобрести для себя другое место>>.
\vs 3Ez 4:16 Но замысел леса оказался тщетным, ибо пришел огонь и сжег его.
\vs 3Ez 4:17 Подобным образом кончился и замысел волн морских, ибо стал песок, и воспрепятствовал им.
\vs 3Ez 4:18 Если бы ты был судьею их, кого бы ты стал оправдывать или кого обвинять?
\vs 3Ez 4:19 Подлинно, отвечал я, замыслы их были суетны, ибо земля дана лесу, дано место и морю, чтобы носить свои волны.
\vs 3Ez 4:20 Он же в ответ сказал мне: справедливо рассудил ты; почему же ты не судил таким же образом себя самого?
\vs 3Ez 4:21 Ибо как земля дана лесу, а море волнам его, так обитающие на земле могут разуметь только то, что на земле; а обитающие на небесах могут разуметь, что на высоте небес.
\vs 3Ez 4:22 И отвечал я, и сказал: молю Тебя, Господи, да дастся мне смысл разумения.
\vs 3Ez 4:23 Не хотел я вопрошать Тебя о высшем, а о том, что ежедневно бывает у нас: почему Израиль предан на поругание язычникам? почему народ, который Ты возлюбил, отдан нечестивым племенам, и закон отцов наших доведен до ничтожества, и писанных постановлений нигде нет?
\vs 3Ez 4:24 Переходим из века сего, как саранча, жизнь наша проходит в страхе и ужасе, и мы сделались недостойными милосердия.
\vs 3Ez 4:25 Но что сделает Он с именем Своим, которое наречено на нас? вот о чем я вопрошал.
\vs 3Ez 4:26 Он же отвечал мне: чем больше будешь испытывать, тем больше будешь удивляться; потому что быстро спешит век сей к своему исходу,
\vs 3Ez 4:27 и не может вместить того, что обещано праведным в будущие времена, потому что век сей исполнен неправдою и немощами.
\vs 3Ez 4:28 А о том, о чем ты спрашивал меня, скажу тебе: посеяно зло, а еще не пришло время искоренения его.
\vs 3Ez 4:29 Посему, доколе посеянное не исторгнется, и место, на котором насеяно зло, не упразднится,~--- не придет место, на котором всеяно добро.
\vs 3Ez 4:30 Ибо зерно злого семени посеяно в сердце Адама изначала, и сколько нечестия народило оно доселе и будет рождать до тех пор, пока не настанет молотьба!
\vs 3Ez 4:31 Рассуди с собою, сколько зерно злого семени народило плодов нечестия!
\vs 3Ez 4:32 Когда будут пожаты бесчисленные колосья его, какое огромное понадобится для сего гумно!
\vs 3Ez 4:33 Как же и когда это будет? спросил я его; почему наши лета малы и несчастны?
\vs 3Ez 4:34 Не спеши подниматься, отвечал он, выше Всевышнего; ибо напрасно спешишь быть выше Его: слишком далеко заходишь.
\vs 3Ez 4:35 Не о том же ли вопрошали души праведных в затворах своих, говоря: <<доколе таким образом будем мы надеяться? И когда плод нашего возмездия?>>
\vs 3Ez 4:36 На это отвечал мне Иеремиил Архангел: <<когда исполнится число семян в вас, ибо Всевышний на весах взвесил век сей,
\vs 3Ez 4:37 и мерою измерил времена, и числом исчислил часы, и не подвинет и не ускорит до тех пор, доколе не исполнится определенная мера>>.
\vs 3Ez 4:38 Я же в ответ на это сказал ему: о, Владыко Господи! а мы все преисполнены нечестием.
\vs 3Ez 4:39 И, может быть, из-за нас не наполняются житницы праведных, и ради грехов живущих на земле.
\vs 3Ez 4:40 На это он отвечал мне: пойди, спроси беременную женщину, могут ли, по исполнении девятимесячного срока, ложесна ее удержать в себе плод?
\vs 3Ez 4:41 Я сказал: не могут. Тогда он сказал мне: подобны ложеснам и обиталища душ в преисподней.
\vs 3Ez 4:42 Как рождающая спешит родить, чтобы освободиться от болезней рождения, так и эти спешат отдать вверенное им.
\vs 3Ez 4:43 Сначала будет показано тебе то, что ты желаешь видеть.
\vs 3Ez 4:44 Если я обрел благодать пред очами твоими, отвечал я, и если это возможно и я способен к тому,
\vs 3Ez 4:45 покажи мне: имеющее прийти более ли того, что прошло, или сбывшееся более того, что будет?
\vs 3Ez 4:46 Что прошло, я это знаю, а что придет, не ведаю.
\vs 3Ez 4:47 Он сказал мне: стань на правую сторону, и я объясню тебе значение подобием.
\vs 3Ez 4:48 И я стал, и увидел: вот горящая печь проходит передо мною; и когда пламя прошло, я увидел: остался дым.
\vs 3Ez 4:49 После сего прошло предо мною облако, наполненное водою, и пролился из него сильный дождь; но как скоро стремительность дождя остановилась, остались капли.
\vs 3Ez 4:50 Тогда он сказал мне: размышляй себе: как дождь более капель, а огонь больше дыма, так мера прошедшего превысила, а остались капли и дым.
\vs 3Ez 4:51 Тогда я умолял его и сказал: думаешь ли ты, что я доживу до этих дней? и что будет в эти дни?
\vs 3Ez 4:52 На это отвечал он, и сказал: о знамениях, о которых ты спрашиваешь меня, я отчасти могу сказать тебе, а о жизни твоей я не послан говорить с тобою, да и не знаю.
\vs 3Ez 5:1 О знамениях: вот, настанут дни, в которые многие из живущих на земле, обладающие в\acc{е}дением, будут вос\-х\acc{и}\-ще\-ны, и путь истины сокроется, и вселенная оскудеет верою,
\vs 3Ez 5:2 и умножится неправда, которую теперь ты видишь и о которой издавна слышал.
\vs 3Ez 5:3 И будет, что страна, которую ты теперь видишь господствующею, подвергнется опустошению.
\vs 3Ez 5:4 А если Всевышний даст тебе дожить, то увидишь, что после третьей трубы внезапно воссияет среди ночи солнце и луна трижды в день;
\vs 3Ez 5:5 и с дерева будет капать кровь, камень даст голос свой, и народы поколеблются.
\vs 3Ez 5:6 Тогда будет царствовать тот, которого живущие на земле не ожидают, и птицы перелетят на другие места.
\vs 3Ez 5:7 Море Содомское извергнет рыб, будет издавать ночью голос, неведомый для многих; однако же все услышат голос его.
\vs 3Ez 5:8 Будет смятение во многих местах, часто будет посылаем с неба огонь; дикие звери переменят места свои, и нечистые женщины будут рождать чудовищ.
\vs 3Ez 5:9 Сладкие воды сделаются солеными, и все друзья ополчатся друг против друга; тогда сокроется ум, и разум удалится в свое хранилище.
\vs 3Ez 5:10 Многие будут искать его, но не найдут, и умножится на земле неправда и невоздержание.
\vs 3Ez 5:11 Одна область будет спрашивать другую соседнюю: <<не проходила ли по тебе правда, делающая праведным?>> И та скажет: <<нет>>.
\vs 3Ez 5:12 Люди в то время будут надеяться, и не достигнут желаемого, будут трудиться, и не управятся пути их.
\vs 3Ez 5:13 Об этих знамениях мне дозволено сказать тебе, и если снова помолишься и поплачешь, как теперь, и попостишься семь дней, то услышишь еще больше того.
\vs 3Ez 5:14 И я пришел в себя, и тело мое сильно дрожало, и душа моя изнемогла, как будто исчезала.
\vs 3Ez 5:15 Но пришедший ко мне Ангел поддержал меня и укрепил меня, и поставил на ноги.
\vs 3Ez 5:16 И было, во вторую ночь пришел ко мне Салафиил, вождь народа, и спросил меня: где ты был, и отчего лице твое так печально?
\vs 3Ez 5:17 Разве не знаешь, что тебе вверен Израиль в стране преселения его?
\vs 3Ez 5:18 Итак встань и вкуси хлеба, и не оставляй нас, как пастырь своего стада, в руках лукавых волков.
\vs 3Ez 5:19 Тогда сказал я ему: отойди от меня, и не приближайся ко мне. И он, услышав это, удалился от меня.
\vs 3Ez 5:20 А я семь дней постился, стеная и плача, как повелел мне Ангел Уриил.
\vs 3Ez 5:21 И после семи дней помышления сердца моего опять были для меня крайне тягостны;
\vs 3Ez 5:22 но душа моя прияла дух разумения, и я снова начал говорить пред Всевышним
\vs 3Ez 5:23 и сказал: о, Владыко Господи! Ты из всех лесов на земле и из всех дерев на ней избрал только одну виноградную лозу;
\vs 3Ez 5:24 Ты из всего круга земного избрал Себе одну пещеру, и из всех цветов во вселенной Ты избрал Себе одну лилию;
\vs 3Ez 5:25 Ты из всех пучин морских наполнил для Себя один источник, а из всех построенных городов освятил для Себя один Сион.
\vs 3Ez 5:26 Из всех сотворенных птиц Ты наименовал Себе одну голубицу, и из всех сотворенных скотов Ты избрал Себе одну овцу;
\vs 3Ez 5:27 из всех многочисленных народов Ты приобрел Себе один народ, и возлюбил его, дал ему закон совершенный.
\vs 3Ez 5:28 Но ныне, Господи, отчего же Ты предал одного многим, и на одном корне Ты насадил другие отрасли и рассеял Твой единственный народ между многими народами?
\vs 3Ez 5:29 И попрали его противники обетованиям Твоим и заветам Твоим не веровавшие.
\vs 3Ez 5:30 И если уже Ты сильно возненавидел народ Твой, то пусть бы он Твоими руками наказывался.
\rsbpar\vs 3Ez 5:31 Когда я произносил слова сии, послан был ко мне Ангел, который приходил ко мне прежде ночью,
\vs 3Ez 5:32 и сказал мне: послушай меня, и я научу тебя; внимай мне, и я скажу тебе еще более.
\vs 3Ez 5:33 Говори, сказал я, господин мой. И он сказал мне: ты слишком далеко зашел пытливостью ума твоего об Израиле; неужели ты больше любишь его, нежели Тот, Который сотворил его?
\vs 3Ez 5:34 Нет, господин мой, отвечал я, но говорил от великой скорби. Внутренность моя мучает меня всякий час, когда я стараюсь постигнуть путь Всевышнего и исследовать хотя часть суда Его.
\vs 3Ez 5:35 Он отвечал: не можешь. Почему же, господин мой? спросил я. Лучше бы я не родился, и утроба матерняя сделалась для меня гробом, нежели видеть угнетение Иакова и изнурение рода Израильского.
\vs 3Ez 5:36 И он сказал мне: исчисли мне, что еще не пришло, и собери мне рассеянные капли, и оживи иссохшие цветы;
\vs 3Ez 5:37 открой заключенные хранилища и выведи мне заключенные в них ветры, и покажи мне образ голоса: и тогда я покажу тебе то, что ты усиливаешься видеть.
\vs 3Ez 5:38 Владыко Господи! отвечал я, кто может знать это, разве только тот, кто не живет с человеками?
\vs 3Ez 5:39 А я безумен, и как могу говорить о том, о чем Ты спросил меня?
\vs 3Ez 5:40 Тогда Он сказал мне: как ты не можешь сделать ничего из сказанного, так не можешь познать судеб Моих, ни предела любви, которую обещал Я народу.
\vs 3Ez 5:41 Но вот, Господи, Ты близок к тем, которые к концу близятся, и что будут делать те, которые прежде меня были, или мы, или которые после нас будут?
\vs 3Ez 5:42 Он сказал мне: венцу уподоблю я суд Мой; как нет запоздания последних, так и ускорения первых.
\vs 3Ez 5:43 Отвечал я и сказал: не мог ли бы Ты соединить воедино как тех, которые сотворены были прежде, так и тех, которые существуют и которые будут, дабы скорее объявить им суд Твой?
\vs 3Ez 5:44 Он отвечал мне: не может ускорить творение Творца своего, ни век сей не может вместить в себе всех вместе, которые должны быть сотворены.
\vs 3Ez 5:45 И сказал я: как же Ты сказал рабу Твоему, что Ты дал жизнь созданному творению вкупе, и однако творение выдержало это; посему могли бы понести и ныне существующие вкупе.
\vs 3Ez 5:46 Он сказал мне: спроси женщину, и скажи ей: <<если ты рождаешь десять, то почему рождаешь по временам?>>, и проси ее, чтобы она родила десять вдруг.
\vs 3Ez 5:47 Я же сказал Ему: невозможно это, но должно быть по времени.
\vs 3Ez 5:48 Тогда Он сказал мне: и Я дал недрам земли способность посеянное на ней возращать по временам.
\vs 3Ez 5:49 Как младенец не может производить того, что свойственно старцам, так Я устроил созданный Мною век.
\vs 3Ez 5:50 Тогда я вопросил Его и сказал: когда Ты открыл мне путь, то позволь мне сказать Тебе: мать наша, о которой Ты говорил Мне, молода ли еще, или приближается к старости?
\vs 3Ez 5:51 Спроси об этом рождающую, и она скажет тебе.
\vs 3Ez 5:52 Скажи ей: <<почему рождаемые тобою ныне не подобны тем, которые рождены были прежде, но меньше их ростом?>>
\vs 3Ez 5:53 И она скажет тебе: <<одни рождены мною в крепости молодой силы, а другие рождены под старость, когда ложесна начали терять свою силу>>.
\vs 3Ez 5:54 Рассуди же ты: вы теперь меньше станом, нежели те, которые были прежде вас;
\vs 3Ez 5:55 и те, которые после вас родятся, будут еще меньше вас, так как творения, уже состаривающиеся, и крепость юноши уже миновала.
\vs 3Ez 5:56 И сказал я: если я приобрел благоволение пред очами Твоими, покажи рабу Твоему, через кого Ты посещаешь творение Твое?
\vs 3Ez 6:1 И сказал Он мне: от начала творения круга земного и прежде нежели установлены были пределы века, и прежде нежели подули ветры;
\vs 3Ez 6:2 прежде нежели услышаны были гласы громов, прежде нежели возблистали молнии, прежде нежели утвердились основания рая;
\vs 3Ez 6:3 прежде нежели показались прекрасные цветы, прежде нежели утвердились силы подвижные, и прежде нежели собрались бесчисленные воинства Ангелов;
\vs 3Ez 6:4 прежде нежели поднялись высоты воздушные, прежде нежели определились меры твердей, прежде нежели возгорелись огни на Сионе;
\vs 3Ez 6:5 прежде нежели исследованы были лета, и отделены те, которые грешат ныне, и запечатлены те, которые хранили веру, как сокровище:
\vs 3Ez 6:6 тогда Я помыслил, и сотворено было все Мною одним, а не чрез кого-либо иного; от Меня также последует и конец, а не от кого-либо иного.
\vs 3Ez 6:7 Тогда я отвечал: какое разделение времен, и когда будет конец первого и начало последнего?
\vs 3Ez 6:8 От Авраама даже до Исаака, когда родились от него Иаков и Исав, рука Иакова держала от начала пяту Исава.
\vs 3Ez 6:9 Конец сего века~--- Исав, а начало следующего~--- Иаков.
\vs 3Ez 6:10 Рука человека~--- начало его, а конец~--- пята его. О другом, Ездра, не спрашивай Меня.
\vs 3Ez 6:11 Я же в ответ сказал Ему: о, Владыко Господи! если я обрел благодать пред очами Твоими,
\vs 3Ez 6:12 молю Тебя, покажи рабу Твоему конец знамений Твоих, которых часть показал Ты мне в прошедшую ночь.
\vs 3Ez 6:13 Он отвечал мне и сказал: встань на ноги твои, и слушай голос, исполненный шума,
\vs 3Ez 6:14 и будет как бы землетрясение, но место, на котором ты стоишь, не поколеблется.
\vs 3Ez 6:15 Посему, когда будет говорить, ты не ужасайся; ибо о конце будет слово, и основания земли разумеются.
\vs 3Ez 6:16 А как речь идет о них самих, то земля вострепещет и поколеблется, ибо знает, что конец их должен измениться.
\rsbpar\vs 3Ez 6:17 И было, когда я услышал голос, встал на ноги мои, и слышал, и вот голос говорящий, и шум его, как шум вод многих,
\vs 3Ez 6:18 и сказал: вот, наступают дни, когда Я начну приближаться, чтобы посетить живущих на земле,
\vs 3Ez 6:19 когда начну Я взыскивать с тех, которые неправдою своею произвели неправедно великий вред, и когда исполнится мера уничижения Сиона.
\vs 3Ez 6:20 А когда назнаменается век, который начнет проходить, то вот знамения, которые Я покажу: книги раскроются пред лицем тверди, и все вместе увидят;
\vs 3Ez 6:21 и однолетние младенцы заговорят своими голосами, и беременные женщины будут рождать недозрелых младенцев через три и четыре месяца, и они останутся живыми и укрепятся;
\vs 3Ez 6:22 засеянные поля внезапно явятся как незасеянные, и полные житницы окажутся пустыми;
\vs 3Ez 6:23 затем вострубит труба с шумом, и когда услышат ее, все внезапно ужаснутся.
\vs 3Ez 6:24 И будет в то время, вооружатся друзья против друзей, как враги, и устрашится земля с живущими на ней, и жилы источников остановятся и три часа не будут течь.
\vs 3Ez 6:25 Всякий, кто после всего этого, о чем Я предсказал тебе, останется в живых, сам спасется, и увидит спасение Мое и конец вашего века.
\vs 3Ez 6:26 И увидят люди избранные, которые не испытали смерти от рождения своего, и изменится сердце живущих и обратится в чувство иное.
\vs 3Ez 6:27 Ибо зло истребится, и исчезнет лукавство;
\vs 3Ez 6:28 процветет вера, побеждено будет растление, явится истина, которая столько времени оставалась без плода.
\vs 3Ez 6:29 Когда Он говорил, я взглянул на того, пред которым стоял.
\vs 3Ez 6:30 И он сказал мне: я пришел показать тебе время грядущей ночи.
\vs 3Ez 6:31 Итак, если ты опять помолишься и опять семь дней попостишься, то я покажу тебе больше в день, в который я услышал тебя.
\vs 3Ez 6:32 Голос твой услышан у Всевышнего; увидел Крепкий правильное действие, увидел и чистоту, которую хранил ты от юности твоей.
\vs 3Ez 6:33 Посему Он послал меня показать тебе все это и сказать: уповай и не бойся;
\vs 3Ez 6:34 не спеши с первыми временами помышлять суетное, дабы не судить тебе с такою же поспешностью о временах последних.
\rsbpar\vs 3Ez 6:35 После сего я снова со слезами молился, и также постился семь дней, чтобы исполнить три седмицы, заповеданные мне.
\vs 3Ez 6:36 В восьмую же ночь сердце мое пришло снова в возбуждение, и я начал говорить пред Всевышним,
\vs 3Ez 6:37 ибо дух мой воспламенялся сильно, и душа моя томилась.
\vs 3Ez 6:38 И сказал я: Господи! Ты от начала творения говорил; в первый день сказал: <<да будет небо и земля>>, и слово Твое было совершившимся делом.
\vs 3Ez 6:39 Тогда носился Дух, и тьма облегала вокруг и молчание: звука человеческого голоса еще не было.
\vs 3Ez 6:40 Тогда повелел Ты из сокровищниц Твоих выйти обильному свету, чтобы явилось дело Твое.
\vs 3Ez 6:41 Во второй день сотворил Ты дух тверди и повелел ему отделить и произвести разделение между водами, чтобы некоторая часть их поднялась вверх, а прочая осталась внизу.
\vs 3Ez 6:42 В третий день Ты повелел водам собраться на седьмой части земли, а шесть частей осушил, чтобы они служили пред Тобою к обсеменению и обработанию.
\vs 3Ez 6:43 Слово Твое исходило, и тотчас являлось дело;
\vs 3Ez 6:44 вдруг явилось безмерное множество плодов и многоразличные приятности для вкуса, цветы в виде своем неизменные, с запахом, несказанно благоуханным: все это совершено было в третий день.
\vs 3Ez 6:45 В четвертый день Ты повелел быть сиянию солнца, свету луны, расположению звезд
\vs 3Ez 6:46 и повелел, чтобы они служили имеющему быть созданным человеку.
\vs 3Ez 6:47 В пятый день Ты сказал седьмой части, в которой была собрана вода, чтобы она произвела животных, летающих и рыб, что и сделалось.
\vs 3Ez 6:48 Вода немая и бездушная, по мановению Божию, произвела животных, чтобы все роды возвещали дивные дела Твои.
\vs 3Ez 6:49 Тогда Ты сохранил двух животных: одно называлось бегемотом, а другое левиафаном.
\vs 3Ez 6:50 И Ты отделил их друг от друга, потому что седьмая часть, где была собрана вода, не могла принять их вместе.
\vs 3Ez 6:51 Бегемоту Ты дал одну часть из земли, осушенной в третий день, да обитает в ней, в которой тысячи гор.
\vs 3Ez 6:52 Левиафану дал седьмую часть водяную, и сохранил его, чтобы он был пищею тем, кому Ты хочешь, и когда хочешь.
\vs 3Ez 6:53 В шестый же день повелел Ты земле произвести пред Тобою скотов, зверей и пресмыкающихся;
\vs 3Ez 6:54 а после них Ты сотворил Адама, которого поставил властелином над всеми Твоими тварями и от которого происходим все мы и народ, который Ты избрал.
\vs 3Ez 6:55 Все это сказал я пред Тобою, Господи, потому что для нас создал Ты век сей.
\vs 3Ez 6:56 О прочих же народах, происшедших от Адама, Ты сказал, что они ничто, но подобны слюне, и все множество их Ты уподобил каплям, каплющим из сосуда.
\vs 3Ez 6:57 И ныне, Господи, вот, эти народы, за ничто Тобою признанные, начали владычествовать над нами и пожирать нас.
\vs 3Ez 6:58 Мы же, народ Твой, который Ты назвал Твоим первенцем, единородным, возлюбленным Твоим, преданы в руки их.
\vs 3Ez 6:59 Если для нас создан век сей, то почему не получаем мы наследия с веком? И доколе это?
\vs 3Ez 7:1 Когда я окончил говорить эти слова, послан был ко мне Ангел, который посылаем был ко мне в прежние ночи,
\vs 3Ez 7:2 и сказал мне: встань, Ездра, и слушай слов\acc{а}, которые я пришел говорить тебе.
\vs 3Ez 7:3 Я сказал: говори, господин мой. И он сказал мне: море расположено в пространном месте, чтобы быть глубоким и безмерным;
\vs 3Ez 7:4 но вход в него находится в тесном месте, так что подобен рекам.
\vs 3Ez 7:5 Кто пожелал бы войти в море и видеть его, или господствовать над ним, тот, если не пройдет тесноты, как может дойти до широты?
\vs 3Ez 7:6 Или иное подобие: город построен и расположен на равнине, и наполнен всеми благами;
\vs 3Ez 7:7 но вход в него тесен и расположен на крутизне так, что по правую сторону огонь, а по левую глубокая вода.
\vs 3Ez 7:8 Между ними, то есть между огнем и водою, лежит лишь одна стезя, на которой может поместиться не более, как только ступень человека.
\vs 3Ez 7:9 Если город этот будет дан в наследство человеку, то как он получит свое наследство, если никогда не перейдет лежащей на пути опасности?
\vs 3Ez 7:10 Я сказал: так, Господи. И Он сказал мне: такова и доля Израиля.
\vs 3Ez 7:11 Для них Я сотворил век; но когда Адам нарушил Мои постановления, определено быть тому, что сделано.
\vs 3Ez 7:12 И сделались входы века сего тесными, болезненными, утомительными, также узкими, лукавыми, исполненными бедствий и требующими великого труда.
\vs 3Ez 7:13 А входы будущего века пространны, безопасны, и приносят плод бессмертия.
\vs 3Ez 7:14 Итак, если входящие, которые живут, не войдут в это тесное и бедственное, они не могут получить, что уготовано.
\vs 3Ez 7:15 Зачем же смущаешься, когда ты тленен, и что мятешься, когда смертен?
\vs 3Ez 7:16 Зачем не принял ты в сердце твоем того, что будущее, а принял то, что в настоящем?
\vs 3Ez 7:17 Я отвечал и сказал: Владыко Господи! вот, Ты определил законом Твоим, что праведники наследуют это, а грешники погибнут.
\vs 3Ez 7:18 Праведники потерпят тесноту, надеясь пространного, а нечестиво жившие, хотя потерпели тесноту, не увидят пространного.
\vs 3Ez 7:19 И Он сказал мне: нет судии выше Бога, нет разумеющего более Всевышнего.
\vs 3Ez 7:20 Погибают многие в этой жизни, потому что нерадят о предложенном им законе Божием.
\vs 3Ez 7:21 Ибо строго повелел Бог приходящим, когда они пришли, что делая, они будут живы, и что соблюдая, не будут наказаны.
\vs 3Ez 7:22 А они не послушались, и воспротивились Ему, утвердили в себе помышление суетное.
\vs 3Ez 7:23 Увлеклись греховными обольщениями, сказали о Всевышнем, что \bibemph{Его} нет, не познали путей Его,
\vs 3Ez 7:24 презрели закон Его, отвергли обетования Его, не имели веры к обрядовым установлениям Его, не совершали дел Его.
\vs 3Ez 7:25 И потому, Ездра, пустым пустое, а полным полное.
\vs 3Ez 7:26 Вот, придет время, когда придут знамения, которые Я предсказал тебе, и явится невеста, и являясь покажется,~--- скрываемая ныне землею.
\vs 3Ez 7:27 И всякий, кто избавится от прежде исчисленных зол, сам увидит чудеса Мои.
\vs 3Ez 7:28 Ибо откроется Сын Мой Иисус с теми, которые с Ним, и оставшиеся будут наслаждаться четыреста лет.
\vs 3Ez 7:29 А после этих лет умрет Сын Мой Христос и все люди, имеющие дыхание.
\vs 3Ez 7:30 И обратится век в древнее молчание на семь дней, подобно тому, как было прежде, так что не останется никого.
\vs 3Ez 7:31 После же семи дней восстанет век усыпленный, и умрет поврежденный.
\vs 3Ez 7:32 И отдаст земля тех, которые в ней спят, и прах тех, которые молчаливо в нем обитают, а хранилища отдадут вверенные им души.
\vs 3Ez 7:33 Тогда явится Всевышний на престоле суда, и пройдут беды, и окончится долготерпение.
\vs 3Ez 7:34 Суд будет один, истина утвердится, вера укрепится.
\vs 3Ez 7:35 Затем последует дело, откроется воздаяние, восстанет правда, перестанет господствовать неправда.
\vs 3Ez 7:(36) \fns{70 стихов, находящихся между 35 и 36 стихами 7-й главы, имеются в русском переводе в <<Толковой Библии>> А.П.Лопухина (Петербург, 1913) и в т.н. <<Брюссельской>> Библии (Брюссель, 1973). В Синодальной Библии их нет.}И откроется озеро мучения, а против него место покоя; видна будет печь геенны, а против нее рай сладости.
\vs 3Ez 7:(37) И скажет тогда Всевышний пробудившимся народам: <<посмотрите и поймите, Кого вы отвергли, Кому вы не служили и Чьи заповеди вы презрели.
\vs 3Ez 7:(38) Взгляните прямо пред собою и напротив: там сладость и покой, а тут огонь и мучения>>. Вот что скажешь Ты им в день суда.
\vs 3Ez 7:(39) Этот день таков, что не имеет ни солнца, ни луны, ни звезд,
\vs 3Ez 7:(40) ни облака, ни грома, ни молнии, ни ветра, ни дождя, ни тумана, ни мрака, ни вечера, ни утра,
\vs 3Ez 7:(41) ни лета, ни весны, ни жары, ни зимы, ни мороза, ни холода, ни града, ни дождя, ни росы,
\vs 3Ez 7:(42) ни полдня, ни ночи, ни предрассветных сумерек, ни блеска, ни ясности, ни света, кроме одного лишь сияния светлости Всевышнего, вследствие чего все могут видеть то, что пред ними.
\vs 3Ez 7:(43) Его длительность будет такая же, как седьмины лет.
\vs 3Ez 7:(44) Таков суд Мой и его порядок. Одному тебе Я открыл это.
\vs 3Ez 7:(45) И я отвечал: <<я говорил уже, и теперь скажу: блаженны живущие и исполняющие заповеданное Тобою.
\vs 3Ez 7:(46) Но я молил о следующем: найдется ли кто из живущих, чтобы не грешил, или найдется ли кто из родившихся, чтобы не нарушал Твоего завета?
\vs 3Ez 7:(47) И теперь я вижу, что будущий век принесет сладость немногим, а мучения многим.
\vs 3Ez 7:(48) Ибо внутри нас выросло сердце злое, которое удалило нас от Него и привело нас к тлению и путям смерти, показало нам тропинки погибели и удалило нас от жизни, притом не малое количество, но почти всех, кто был сотворен>>.
\vs 3Ez 7:(49) И Он отвечал мне и сказал: выслушай Меня, и Я наставлю тебя и вразумлю тебя относительно имеющего быть.
\vs 3Ez 7:(50) В виду этого Бог и сотворил не один век, а два.
\vs 3Ez 7:(51) Что же касается твоих слов, что праведных не много, но мало, тогда как нечестивых множество, то выслушай на это вот что:
\vs 3Ez 7:(52) <<если у тебя будет весьма немного драгоценных камней, то ты станешь складывать их у себя по числу их; свинца же и глины изобилие>>.
\vs 3Ez 7:(53) И я сказал: <<как же это возможно?>>
\vs 3Ez 7:(54) И Он сказал мне: <<не только это, но спроси землю, и та скажет тебе, подойди к ней с лестью, и та поведает тебе.
\vs 3Ez 7:(55) Ты скажешь ей: ты производишь золото, серебро и медь, а также железо, свинец и глину.
\vs 3Ez 7:(56) Серебра же больше, чем золота, меди больше, чем серебра, железа больше, чем меди, свинца больше, чем железа, и глины больше, чем свинца.
\vs 3Ez 7:(57) Посуди теперь сам, что драгоценно и влечет к себе, то ли, чего много, или то, что является редкостью>>.
\vs 3Ez 7:(58) И я сказал: <<Владыка Господи! Что встречается в избытке, то хуже, а что попадается реже, то драгоценнее>>.
\vs 3Ez 7:(59) И Он отвечал мне и сказал: <<взвесь про себя то, что ты подумал: кто владеет тем, что с трудом добывается, бывает рад больше того, кто обладает тем, что встречается в избытке.
\vs 3Ez 7:(60) Так обстоит дело и с обещанною Мною тварью. Я рад буду немногим спасшимся, потому что они утвердили ныне владычество Моей славы и на них наречено ныне же Мое имя.
\vs 3Ez 7:(61) Меня не будет огорчать множество погибших: ведь это те самые, которые теперь уже уподоблены пару и приравнены к огню и дыму. Вот они вспыхнули, запылали и погасли>>.
\vs 3Ez 7:(62) И я отвечал и сказал: <<о, земля! что же ты породила, если разум произошел из праха, как и остальная тварь?
\vs 3Ez 7:(63) Лучше было бы не появляться самому праху, чтобы из него не возник разум.
\vs 3Ez 7:(64) А теперь, разум возрастает вместе с нами, и из-за этого мы мучимся, так как сознательно идем к гибели.
\vs 3Ez 7:(65) Пусть рыдает род человеческий, и радуются полевые звери; пусть рыдают все, кто родился, и веселятся четвероногие и скоты.
\vs 3Ez 7:(66) Ибо им гораздо лучше, чем нам, так как они не ждут суда; им неведомы ни мучения, ни блаженство, обещанные им после смерти.
\vs 3Ez 7:(67) Что нам пользы в том, что мы будем снова жить, но будем жестоко мучиться?
\vs 3Ez 7:(68) Ведь все, кто родился, пропитаны беззакониями, полны грехов и отягчены преступлениями.
\vs 3Ez 7:(69) И быть может, лучше было бы нам, если бы нам не нужно было идти на суд>>.
\vs 3Ez 7:(70) И Он отвечал мне и сказал: <<раньше, чем Всевышний сотворил век с Адамом и всеми, происшедшими от него, Он приготовил суд и то, что относится к суду.
\vs 3Ez 7:(71) Теперь же уразумей на основании своих собственных слов; ведь ты сказал, что разум возрастает с нами.
\vs 3Ez 7:(72) Поэтому те, кто живет на земле, терпят здесь мучения, потому что, имея разум, они совершали беззакония и, получая заповеди, не исполняли их, и, будучи последователями закона, отвергали закон, полученный ими.
\vs 3Ez 7:(73) Что же имеют они сказать на суде или какой ответ дадут они в ближайшее время?
\vs 3Ez 7:(74) В самом деле, сколько времени Всевышний проявлял долготерпение к тем, кто населяет век, и не ради их самих, а ради исполнения предусмотренного Им срока>>.
\vs 3Ez 7:(75) И я отвечал и сказал: <<если я нашел благодать пред Тобою, Господи, то покажи рабу Твоему еще следующее. Будем ли мы после смерти, то есть когда каждый из нас отдаст душу свою, пребывать в покое, пока не наступят те времена, когда Ты начнешь обновлять тварь, или же тотчас будем терпеть мучения?>>
\vs 3Ez 7:(76) И Он отвечал мне и сказал: <<покажу тебе и это. Но ты не смешивай себя с теми, кто презирал, и не причисляй себя к тем, которые терпят мучения,
\vs 3Ez 7:(77) ибо у тебя есть сокровище дел, сохраняемое у Всевышнего; но оно не будет пока дано тебе до наступления последнего времени.
\vs 3Ez 7:(78) Теперь будет речь о смерти, когда выйдет от Всевышнего приговор относительно срока, чтобы умереть человеку, и когда дух выйдет из тела, чтобы снова вернуться к Тому, Кто дал его, для поклонения прежде всего славе Всевышнего.
\vs 3Ez 7:(79) И если это будут души тех, кто презирал и не сохранял путей Всевышнего, пренебрегал Его законом и ненавидел боящихся Бога,
\vs 3Ez 7:(80) то таковые души не войдут в обители, но немедленно начнут в мучениях, в постоянной скорби и печали блуждать по семи путям.
\vs 3Ez 7:(81) Первый путь это то, что они презрели закон Всевышнего.
\vs 3Ez 7:(82) Второй путь: они уже не могут принести доброе раскаяние, чтобы жить.
\vs 3Ez 7:(83) Третий путь: они увидят награду, сохраняемую для тех, кто верен заветам Всевышнего.
\vs 3Ez 7:(84) Четвертый путь: они увидят мучения, сохраняемые для них на самое последнее время.
\vs 3Ez 7:(85) Пятый путь: они видят жилища других, охраняемые в глубочайшем молчании ангелами.
\vs 3Ez 7:(86) Шестой путь: они видят, что немедленно же отсюда они перейдут на мучения.
\vs 3Ez 7:(87) Седьмой путь, превосходящий все названные выше пути, состоит в том, что они тают от смятения, их снедает стыд, они изнемогают от страха, при виде славы Всевышнего, пред которой они грешили при жизни и пред которой им предстоит суд в последние времена.
\vs 3Ez 7:(88) Что же касается тех, кто сохранял пути Всевышнего, то удел их по разлучению с тленным сосудом будет следующий:
\vs 3Ez 7:(89) во время пребывания в нем они с трудностями служили Всевышнему и каждый час подвергались опасностям, лишь бы всецело сохранить закон Законодателя.
\vs 3Ez 7:(90) Поэтому приговор о них будет такой:
\vs 3Ez 7:(91) прежде всего они увидят с великою радостью славу Того, Кто принимает их к Себе; покой же они будут вкушать семи видов.
\vs 3Ez 7:(92) Первый вид это то, что они с великим трудом вели борьбу, с целью преодолеть помышление злое, созданное вместе с ними, чтобы оно не могло отвлекать их от жизни к смерти.
\vs 3Ez 7:(93) Второй вид: они созерцают смятение, в каком блуждают души нечестивых, и наказание, предстоящее им.
\vs 3Ez 7:(94) Третий вид: они созерцают данное им их Создателем свидетельство, что они при жизни сохранили закон, вверенный им.
\vs 3Ez 7:(95) Четвертый вид: они сознают свой покой, которым они наслаждаются ныне, собравшись в своих хранилищах и оберегаемые в глубоком молчании ангелами, и прославление, ожидающее их в последние времена.
\vs 3Ez 7:(96) Пятый вид: они ликуют по поводу того, что покинули ныне тленное и получат будущее наследие; они видят кроме того ту тесноту, полную тягостей, от которой они освободились, и начинают чувствовать простор, блаженные и бессмертные.
\vs 3Ez 7:(97) Шестой вид: им показано будет, как лицо их засияет подобно солнцу и они уподобятся по блеску звездам, став тотчас же нетленными.
\vs 3Ez 7:(98) Седьмой вид, превосходящий все ранее названные: они будут ликовать с уверенностью, надеяться без посрамления и радоваться без страха, так как они спешат увидеть лицо Того, Кому они служили при жизни, и от Кого они должны получить награду, состоящую в прославлении.
\vs 3Ez 7:(99) Таков удел душ праведников, возвещаемый им тотчас же. Ранее были названы пути тех мучений, которые терпят немедленно же грешники>>.
\vs 3Ez 7:(100) И я отвечал и сказал: <<значит, душам по разлучении их с телом будет дано время, чтобы видеть то, о чем Ты мне сказал>>.
\vs 3Ez 7:(101) И Он сказал мне: <<семь дней будет длиться их свобода, чтобы они за семь дней увидели то, о чем была выше речь, а после этого они соберутся в свои жилища>>.
\vs 3Ez 7:(102) И я отвечал и сказал: <<если я нашел милость пред очами Твоими, то покажи мне, рабу Твоему, кроме того, могут ли в день суда праведники достигнуть оправдания нечестивых или молить за них Всевышнего,
\vs 3Ez 7:(103) отцы за сыновей, сыновья за родителей, братья за братьев, родственники за своих близких, или друзья за дорогих для них лиц>>.
\vs 3Ez 7:(104) Он отвечал мне и сказал: <<так как ты нашел милость пред очами Моими, то Я покажу тебе и это. День суда решительный и являет всем печать истины. Подобно тому, как ныне отец не посылает сына или сын отца, или господин раба, или друг самого дорогого для него человека с тем, чтобы тот думал за него, или спал, или ел, или лечился,
\vs 3Ez 7:(105) так никогда никто не будет за кого-либо ходатайствовать, но каждый принесет тогда свои правды или неправды>>.
\vs 3Ez 7:36 Я сказал: Авраам первый молился о Содомлянах; Моисей~--- за отцов, согрешивших в пустыне;
\vs 3Ez 7:37 Иисус после него~--- за Израиля во дни Ахана;
\vs 3Ez 7:38 Самуил и Давид~--- за погубляемых, Соломон~--- за тех, которые пришли на освящение;
\vs 3Ez 7:39 Илия~--- за тех, которые приняли дождь, и за мертвеца, чтобы он ожил;
\vs 3Ez 7:40 Езекия~--- за народ во дни Сеннахирима, и многие~--- за многих.
\vs 3Ez 7:41 Итак, если тогда, когда усилилось растление и умножилась неправда, праведные молились за неправедных, то почему же не быть тому и ныне?
\vs 3Ez 7:42 Он отвечал мне и сказал: настоящий век не есть конец; славы в нем часто не бывает, потому молились за немощных.
\vs 3Ez 7:43 День же суда будет концом времени сего и началом времени будущего бессмертия, когда пройдет тление,
\vs 3Ez 7:44 прекратится невоздержание, пресечется неверие, а возрастет правда, воссияет истина.
\vs 3Ez 7:45 Тогда никто не возможет спасти погибшего, ни погубить победившего.
\vs 3Ez 7:46 Я отвечал и сказал: вот мое слово первое и последнее: лучше было не давать земли Адаму, или, когда уже дана, удержать его, чтобы не согрешил.
\vs 3Ez 7:47 Что пользы людям~--- в настоящем веке жить в печали, а по смерти ожидать наказания?
\vs 3Ez 7:48 О, что сделал ты, Адам? Когда ты согрешил, то совершилось падение не тебя только одного, но и нас, которые от тебя происходим.
\vs 3Ez 7:49 Что пользы нам, если нам обещано бессмертное время, а мы делали смертные дела?
\vs 3Ez 7:50 Нам предсказана вечная надежда, а мы, непотребные, сделались суетными.
\vs 3Ez 7:51 Нам уготованы жилища здоровья и покоя, а мы жили худо;
\vs 3Ez 7:52 уготована слава Всевышнего, чтобы покрыть тех, которые жили кротко, а мы ходили по путям злым.
\vs 3Ez 7:53 Показан будет рай, плод которого пребывает нетленным и в котором покой и врачевство;
\vs 3Ez 7:54 но мы не войдем \bibemph{в него}, потому что обращались в местах неплодных.
\vs 3Ez 7:55 Светлее звезд воссияют лица тех, которые имели воздержание, а наши лица~--- чернее тьмы.
\vs 3Ez 7:56 Мы не помышляли в жизни, когда делали беззаконие, что по смерти будем страдать.
\vs 3Ez 7:57 Он отвечал и сказал: это~--- помышление о борьбе, которую должен вести на земле родившийся человек,
\vs 3Ez 7:58 чтобы, если будет побежден, потерпеть то, о чем ты сказал, а если победит, получить то, о чем Я говорю.
\vs 3Ez 7:59 Это та жизнь, о которой сказал Моисей, когда жил, к народу, говоря: <<избери себе жизнь, чтобы жить>>.
\vs 3Ez 7:60 Но они не поверили ему, ни пророкам после него, ни Мне, говорившему к ним,
\vs 3Ez 7:61 что не будет скорби о погибели их, как будет радость о тех, которым уготовано спасение.
\vs 3Ez 7:62 Я отвечал и сказал: знаю, Господи, что Всевышний называется милосердым, потому что помилует тех, которые еще не пришли в мир,
\vs 3Ez 7:63 и милует тех, которые провождают жизнь в законе Его.
\vs 3Ez 7:64 Он долготерпелив, ибо оказывает долготерпение к согрешившим, как к Своему творению.
\vs 3Ez 7:65 Он щедр, ибо готов давать по надобности,
\vs 3Ez 7:66 и многомилостив, ибо умножает милости Свои к живущим ныне и к жившим и к тем, которые будут жить.
\vs 3Ez 7:67 Ибо, если бы не умножал Он Своих милостей, то не мог бы век продолжать жить с теми, которые обитают в нем.
\vs 3Ez 7:68 Он подает дары; ибо если бы не даровал по благости Своей, да облегчатся совершившие нечестие от своих беззаконий, то не могла бы оставаться в живых десятитысячная часть людей.
\vs 3Ez 7:69 Он судия, и если бы не прощал тех, которые сотворены словом Его, и не истребил множества преступлений,
\vs 3Ez 7:70 может быть, из бесчисленного множества остались бы только весьма немногие.
\vs 3Ez 8:1 Он отвечал мне и сказал: этот век Всевышний сотворил для многих, а будущий для немногих.
\vs 3Ez 8:2 Скажу тебе, Ездра, подобие. Как если спросишь землю, она скажет тебе, что дает очень много вещества, из которого делаются глиняные вещи, а не много праха, из которого бывает золото, так и дела настоящего века.
\vs 3Ez 8:3 Многие сотворены, но немногие спасутся.
\vs 3Ez 8:4 Я отвечал и сказал: душа! пожри смысл и поглоти мудрость.
\vs 3Ez 8:5 Ибо ты обещала слушать, и пожелала пророчествовать, а тебе дано время только, чтобы жить.
\vs 3Ez 8:6 О, Господи! неужели Ты не позволишь рабу Твоему, чтобы мы молились пред Тобою о даровании сердцу нашему семени и разуму возделания, чтобы произошел плод, которым мог бы жить всякий растленный, кто будет носить имя человека?
\vs 3Ez 8:7 Ты един, и мы единое творение рук Твоих, как сказал Ты.
\vs 3Ez 8:8 И как же ныне во чреве матернем образуется тело, и Ты даешь члены, как сохраняется Твое творение в огне и воде, и как девять месяцев терпит в себе Твое же создание Твою тварь, которая в нем сотворена?
\vs 3Ez 8:9 И хранящее и хранимое, и то и другое сохраняются, и чрево матери в свое время отдает то сохраненное, что в нем произросло.
\vs 3Ez 8:10 Ты повелел из самих членов, то есть из сосцов, давать молоко, плод сосцов,
\vs 3Ez 8:11 да питается созданное до некоторого времени, а после передашь его Твоему милосердию.
\vs 3Ez 8:12 Ты воспитал его Твоею правдою, научил его Твоему закону, наставил его Твоим разумом,
\vs 3Ez 8:13 и умертвишь его, как Твое творение, и опять оживишь, как Твое дело.
\vs 3Ez 8:14 Если Ты погубишь созданного с таким попечением, то повелению Твоему легко устроить, чтобы и сохранялось то, что было создано.
\vs 3Ez 8:15 И ныне, Господи, я скажу: о всяком человеке Ты больше знаешь; но \bibemph{скажу} о народе Твоем, о котором болезную,
\vs 3Ez 8:16 о наследии Твоем, о котором проливаю слезы, об Израиле, о котором скорблю, об Иакове, о котором сокрушаюсь.
\vs 3Ez 8:17 Начну молиться пред Тобою за себя и за них, ибо вижу грехопадения нас, обитающих на земле.
\vs 3Ez 8:18 Но я слышал, что скоро придет Судия.
\vs 3Ez 8:19 Посему услышь мой голос, вонми словам моим, и я буду говорить пред Тобою. [Начало слов Ездры, прежде нежели он был взят.]
\vs 3Ez 8:20 Я сказал: Господи, живущий вечно, Которого очи обращены на выспреннее и небесное,
\vs 3Ez 8:21 Которого престол неоценим и слава непостижима, Которому с трепетом предстоят воинства Ангелов, служащих в ветре и огне, Которого слово истинно и глаголы непреложны,
\vs 3Ez 8:22 повеление сильно и правление страшно, Которого взор иссушает бездны, гнев расплавляет горы и истина пребывает во веки!
\vs 3Ez 8:23 Услышь молитву раба Твоего, и вонми молению создания Твоего.
\vs 3Ez 8:24 Доколе живу, буду говорить, и доколе разумею, буду отвечать. Не взирай на грехи народа Твоего, но на тех, которые Тебе в истине служат;
\vs 3Ez 8:25 не обращай внимания на нечестивые дела язычников, но на тех, которые заветы Твои сохранили среди бедствий;
\vs 3Ez 8:26 не помышляй о тех, которые пред Тобою лживо поступали, но помяни тех, которые, по воле Твоей, познали страх;
\vs 3Ez 8:27 не погубляй тех, которые жили по-скотски, но воззри на тех, которые ясно учили закону Твоему;
\vs 3Ez 8:28 не прогневайся на тех, которые признаны худшими зверей;
\vs 3Ez 8:29 но возлюби тех, которые всегда надеются на правду Твою и славу.
\vs 3Ez 8:30 Ибо мы и отцы наши такими болезнями страдаем;
\vs 3Ez 8:31 а Ты, ради нас~--- грешных, назовешься милосердым.
\vs 3Ez 8:32 Если Ты пожелаешь помиловать нас, то назовешься милосердым, потому что мы не имеем дел правды.
\vs 3Ez 8:33 Праведники же, у которых много дел приобретено, по собственным делам получат воздаяние.
\vs 3Ez 8:34 Что есть человек, чтобы Ты гневался на него, и род растленный, чтобы Ты столько огорчался им?
\vs 3Ez 8:35 Поистине, нет никого из рожденных, кто не поступил бы нечестиво, и из исповедающих \bibemph{Тебя} нет никого, кто не согрешил бы.
\vs 3Ez 8:36 В том-то и возвестится правда Твоя и благость Твоя, Господи, когда помилуешь тех, которые не имеют существа добрых дел.
\vs 3Ez 8:37 Он отвечал мне и сказал: справедливо ты сказал нечто, и по словам твоим так и будет.
\vs 3Ez 8:38 Ибо истинно не помышляю Я о делах тех созданий, которые согрешили, прежде смерти, прежде суда, прежде погибели;
\vs 3Ez 8:39 но услаждаюсь подвигами праведных, и воспоминаю, как они странствовали, как спасались и старались заслужить награду.
\vs 3Ez 8:40 Как сказал Я, так и есть.
\vs 3Ez 8:41 Как земледелец сеет на земле многие семена и садит многие растения, но не все посеянное сохранится со временем, и не все посаженное укоренится, так и те, которые посеяны в веке \bibemph{сем}, не все спасутся.
\vs 3Ez 8:42 Я отвечал и сказал: если я обрел благодать, то буду говорить.
\vs 3Ez 8:43 Как семя земледельца, если не взойдет, или не примет вовремя дождя Твоего, или повредится от множества дождя, погибает:
\vs 3Ez 8:44 так и человек, созданный руками Твоими,~--- и Ты называешься его первообразом, потому что Ты подобен ему, для которого создал все и которого Ты уподобил семени земледельца.
\vs 3Ez 8:45 Не гневайся на нас, но пощади народ Твой и помилуй наследие Твое,~--- а Ты милосерд к созданию Твоему.
\vs 3Ez 8:46 Он отвечал мне и сказал: настоящее настоящим и будущее будущим.
\vs 3Ez 8:47 Многого недостает тебе, чтобы ты мог возлюбить создание Мое более Меня, хотя Я часто приближался к тебе самому, а к неправедным никогда.
\vs 3Ez 8:48 Но и в том дивен ты пред Всевышним,
\vs 3Ez 8:49 что смирил себя, как прилично тебе, и не судил о себе так, чтобы много славиться между праведными.
\vs 3Ez 8:50 Многие и горестные бедствия постигнут тех, которые населяют век, в последнее время, потому что они ходили в великой гордыне.
\vs 3Ez 8:51 А ты заботься о себе, и подобным тебе ищи славы;
\vs 3Ez 8:52 ибо вам открыт рай, насаждено древо жизни, предназначено будущее время, готово изобилие, построен город, приготовлен покой, совершенная благость и совершенная премудрость.
\vs 3Ez 8:53 Корень зла запечатан от вас, немощь и тля сокрыты от вас, и растление бежит в ад в забвение.
\vs 3Ez 8:54 Прошли болезни, и в конце показалось сокровище бессмертия.
\vs 3Ez 8:55 Не старайся более испытывать о множестве погибающих.
\vs 3Ez 8:56 Ибо они, получив свободу, презрели Всевышнего, пренебрегли закон Его и оставили пути Его,
\vs 3Ez 8:57 а еще и праведных Его попрали,
\vs 3Ez 8:58 и говорили в сердце своем: <<нет Бога>>, хотя и знали, что они смертны.
\vs 3Ez 8:59 Как вас ожидает то, о чем сказано прежде, так и их~--- жажда и мучение, которые приготовлены. Бог не хотел погубить человека,
\vs 3Ez 8:60 но сами сотворенные обесславили имя Того, Кто сотворил их, и были неблагодарными к Тому, Кто предуготовил им жизнь.
\vs 3Ez 8:61 Посему суд Мой ныне приближается,~---
\vs 3Ez 8:62 о чем Я не всем открыл, а только тебе и немногим, тебе подобным. Я отвечал и сказал:
\vs 3Ez 8:63 вот ныне, Господи, Ты показал мне множество знамений, которые Ты начнешь творить при кончине, но не показал, в какое время.
\vs 3Ez 9:1 Он отвечал мне и сказал: измеряя измеряй время в себе самом, и когда увидишь, что прошла некоторая часть знамений, прежде указанных,
\vs 3Ez 9:2 тогда уразумеешь, что это и есть то время, в которое начнет Всевышний посещать век, Им созданный.
\vs 3Ez 9:3 Когда обнаружится в веке колебание мест, смятение народов,
\vs 3Ez 9:4 тогда уразумеешь, что об этом говорил Всевышний от дней, бывших прежде тебя, от начала.
\vs 3Ez 9:5 Как все, сотворенное в веке, имеет начало, равно и конец, и окончание бывает явно:
\vs 3Ez 9:6 так и времена Всевышнего имеют начала, открывающиеся чудесами и силами, и окончания, являемые действиями и знамениями.
\vs 3Ez 9:7 Всякий, кто спасется и возможет делами своими и верою, которою веруете, избежать от преждесказанных бед,
\vs 3Ez 9:8 останется, и увидит спасение Мое на земле Моей и в пределах Моих, которые Я освятил Себе от века.
\vs 3Ez 9:9 Тогда пожалеют отступившие ныне от путей Моих, и отвергшие их с презрением пребудут в муках.
\vs 3Ez 9:10 Те, которые не познали Меня, получая при жизни благодеяния,
\vs 3Ez 9:11 и возгнушались законом Моим, не уразумели его, но презрели, когда еще имели свободу и когда еще отверсто было им место для покаяния,
\vs 3Ez 9:12 те познают Меня по смерти в мучении.
\vs 3Ez 9:13 Ты не любопытствуй более, как нечестивые будут мучиться, но исследуй, как спасутся праведные, которым принадлежит век и ради которых век, и когда.
\vs 3Ez 9:14 Я отвечал и сказал:
\vs 3Ez 9:15 я прежде говорил, и теперь говорю, и после буду говорить, что больше тех, которые погибнут, нежели тех, которые спасутся, как волна больше капли.
\vs 3Ez 9:16 Он отвечал мне и сказал:
\vs 3Ez 9:17 какова нива, таковы и семена; каковы цветы, таковы и краски; каков делатель, таково и дело; каков земледелец, таково и возделывание; ибо то было время века.
\vs 3Ez 9:18 Когда Я уготовлял век, прежде нежели он был, для обитания тех, которые живут ныне в нем, никто Мне не противоречил.
\vs 3Ez 9:19 А ныне, когда век сей был создан, нравы сотворенных повредились при неоскудевающей жатве, при неисследимом законе.
\vs 3Ez 9:20 И рассмотрел Я век, и вот, оказалась опасность от замыслов, которые появились в нем.
\vs 3Ez 9:21 Я увидел и пощадил его, и сохранил для Себя одну ягоду из виноградной кисти и одно насаждение из множества.
\vs 3Ez 9:22 Пусть погибнет множество, которое напрасно родилось, и сохранится ягода Моя и насаждение Мое, которое Я вырастил с большим трудом.
\vs 3Ez 9:23 А ты, когда по прошествии семи дней иных, не постясь однако в них,
\vs 3Ez 9:24 выйдешь на цветущее поле, где нет построенного дома, и станешь питаться только от полевых цветов и не вкушать мяса, ни пить вина, а только цветы,
\vs 3Ez 9:25 молись ко Всевышнему непрестанно, и Я приду и буду говорить с тобою.
\vs 3Ez 9:26 И пошел я, как Он сказал мне, на поле, которое называется Ардаф, и сел там в цветах и вкушал от полевых трав, и была мне пища от них в насыщение.
\vs 3Ez 9:27 После семи дней лежал я на траве, и сердце мое опять смущалось, как прежде.
\vs 3Ez 9:28 И отверзлись уста мои, и я начал говорить пред Всевышним и сказал:
\vs 3Ez 9:29 о, Господи! являя Себя нам, Ты явился отцам нашим в пустыне непроходимой и бесплодной, когда они вышли из Египта,
\vs 3Ez 9:30 и сказал: <<слушай Меня, Израиль, и внимай словам Моим, семя Иакова.
\vs 3Ez 9:31 Вот, Я сею в вас закон Мой, и принесет в вас плод, и вы будете славиться в нем вечно>>.
\vs 3Ez 9:32 Но отцы наши, приняв закон, не исполнили его и постановлений Твоих не сохранили, и хотя плод закона Твоего не погиб и не мог погибнуть, потому что был Твой,
\vs 3Ez 9:33 но принявшие \bibemph{закон} погибли, не сохранив того, что в нем было посеяно.
\vs 3Ez 9:34 Обыкновенно бывает, что если земля приняла семя, или море корабль, или какой-либо сосуд пищу или питье, и если будет повреждено то, в чем посеяно, или то, в чем помещено,
\vs 3Ez 9:35 в таком случае погибает вместе и самое посеянное, или помещенное, или принятое, и принятого уже не остается пред нами. Но с нами не так.
\vs 3Ez 9:36 Мы, принявшие закон, согрешая, погибли, равно и сердце наше, которое приняло его;
\vs 3Ez 9:37 но закон не погиб, и остается в своей силе.
\vs 3Ez 9:38 Когда я говорил это в сердце моем, я воззрел глазами моими, и увидел на правой стороне женщину; и вот, она плакала и рыдала с великим воплем, и сильно болела душею; одежда ее была разодрана, а на голове ее пепел.
\vs 3Ez 9:39 Тогда оставил я размышления, которыми был занят, и, обратившись к ней, сказал ей:
\vs 3Ez 9:40 о чем плачешь ты, и о чем так скорбишь душею?
\vs 3Ez 9:41 Она сказала: оставь меня, господин мой, да плачу о себе и усугублю скорбь, ибо я весьма огорчена душею и весьма унижена.
\vs 3Ez 9:42 Я спросил ее: что потерпела ты? скажи мне. И она отвечала мне:
\vs 3Ez 9:43 я была неплодна, раба твоя, и не рождала, имея мужа, тридцать лет.
\vs 3Ez 9:44 Каждый час, каждый день в эти тридцать лет я молила Всевышнего непрестанно,
\vs 3Ez 9:45 и услышал меня Бог, рабу твою, после тридцати лет, увидел смирение мое, внял скорби моей и дал мне сына, и я сильно обрадовалась ему, и муж мой, и все сограждане мои, и мы много прославляли Всевышнего.
\vs 3Ez 9:46 Я вскормила его с великим трудом,
\vs 3Ez 9:47 и когда он возрос и пошел взять себе жену, я устроила день пиршества.
\vs 3Ez 10:1 Но когда сын мой вошел в брачный чертог свой, он упал, и умер.
\vs 3Ez 10:2 И опрокинули все мы светильники, и все сограждане мои поднялись утешать меня, и я почила до ночи другого дня.
\vs 3Ez 10:3 Когда же все перестали утешать меня, чтобы оставить меня в покое, я, встав ночью, побежала и пришла, как видишь, на это поле.
\vs 3Ez 10:4 И думаю уже не возвращаться в город, но оставаться здесь, ни есть, ни пить, но непрестанно плакать и поститься, доколе не умру.
\vs 3Ez 10:5 Оставив размышления, которыми занимался, я с гневом отвечал ей и сказал:
\vs 3Ez 10:6 о, безумнейшая из всех жен! не видишь ли скорби нашей и приключившегося нам,~---
\vs 3Ez 10:7 что Сион, мать наша, печалится безмерно, крайне унижена, и плачет горько?
\vs 3Ez 10:8 И теперь, когда все мы скорбим и печалимся, потому что все опечалены, будешь ли ты печалиться об одном сыне твоем?
\vs 3Ez 10:9 Спроси землю, и она скажет тебе, что ей-то должно оплакивать падение столь многих рождающихся на ней;
\vs 3Ez 10:10 ибо все рожденные из нее от начала и другие, которые имеют произойти, едва не все погибают, и толикое множество их предаются истреблению.
\vs 3Ez 10:11 Итак кто должен более печалиться, как не та, которая потеряла толикое множество, а не ты, скорбящая об одном?
\vs 3Ez 10:12 Если ты скажешь мне: <<плач мой не подобен плачу земли, ибо я лишилась плода чрева моего, который я носила с печалью и родила с болезнью;
\vs 3Ez 10:13 а земля~--- по свойству земли; на ней настоящее множество как отходит, так и приходит>>:
\vs 3Ez 10:14 и я скажу тебе, что как ты с трудом родила, так и земля дает плод свой человеку, который от начала возделывает ее.
\vs 3Ez 10:15 Посему воздержись теперь от скорби твоей и мужественно переноси случившуюся тебе потерю.
\vs 3Ez 10:16 Ибо если ты признаешь праведным определение Божие, то в свое время получишь сына, и между женами будешь прославлена.
\vs 3Ez 10:17 Итак возвратись в город к мужу твоему.
\vs 3Ez 10:18 Но она сказала: не сделаю так, не возвращусь в город, но здесь умру.
\vs 3Ez 10:19 Продолжая говорить с нею, я сказал:
\vs 3Ez 10:20 не делай этого, но послушай совета моего. Ибо сколько бед Сиону? Утешься ради скорби Иерусалима.
\vs 3Ez 10:21 Ибо ты видишь, что святилище наше опустошено, алтарь наш ниспровергнут, храм наш разрушен,
\vs 3Ez 10:22 псалтирь наш уничижен, песни умолкли, радость наша исчезла, свет светильника нашего угас, ковчег завета нашего расхищен, Святое наше осквернено, и имя, которое наречено на нас, едва не поругано, дети наши потерпели позор, священники наши избиты, левиты наши отведены в плен, девицы наши осквернены, жены наши потерпели насилие, праведники наши увлечены, отроки наши погибли, юноши наши в рабстве, крепкие наши изнемогли;
\vs 3Ez 10:23 и что всего тяжелее, знамя Сиона лишено славы своей, потому что предано в руки ненавидящих нас.
\vs 3Ez 10:24 Посему оставь великую печаль твою, и отложи множество скорбей, чтобы помиловал тебя Крепкий, и Всевышний даровал тебе успокоение и облегчение трудов.
\vs 3Ez 10:25 При сих словах моих к ней, внезапно просияло лице и взор ее, и вот, вид сделался блистающим, так что я, устрашенный ею, помышлял, что бы это было.
\vs 3Ez 10:26 И вот, она внезапно испустила столь громкий и столь страшный звук голоса, что от сего звука жены поколебалась земля.
\vs 3Ez 10:27 И я видел, и вот, жена более не являлась мне, но созидался город, и место его обозначалось на обширных основаниях, и я устрашенный громко воскликнул и сказал:
\vs 3Ez 10:28 где Ангел Уриил, который вначале приходил ко мне? ибо он привел меня в такое исступление ума, в котором цель моего стремления исчезла, и молитва моя обратилась в поношение.
\vs 3Ez 10:29 Когда я говорил это, он пришел ко мне;
\vs 3Ez 10:30 и увидел меня, и вот, я лежал, как мертвый и в бессознательном состоянии; он взял меня за правую руку, укрепил меня и, поставив на ноги, сказал мне:
\vs 3Ez 10:31 что с тобою? отчего смущены разум твой и чувства сердца твоего? отчего смущаешься?
\vs 3Ez 10:32 Оттого, отвечал я ему, что ты оставил меня, и я, поступая по словам твоим, вышел на поле, и вот увидел и еще вижу то, о чем не могу рассказать.
\vs 3Ez 10:33 А он сказал мне: стой мужественно, и я объясню тебе.
\vs 3Ez 10:34 Говори мне, господин мой, сказал я, только не оставляй меня, чтобы я не умер напрасно;
\vs 3Ez 10:35 ибо я видел, чего не знал, и слышал, чего не знаю.
\vs 3Ez 10:36 Чувство ли мое обманывает меня, или душа моя грезит во сне?
\vs 3Ez 10:37 Посему прошу тебя объяснить мне, рабу твоему, это исступление ума моего. Отвечая мне, сказал он:
\vs 3Ez 10:38 внимай мне, и я научу тебя, и изъясню тебе то, что устрашило тебя: ибо Всевышний откроет тебе многие тайны.
\vs 3Ez 10:39 Он видит правый путь твой, что ты непрестанно скорбишь о народе твоем и сильно печалишься о Сионе.
\vs 3Ez 10:40 Таково значение видения, которое пред сим явилось тебе:
\vs 3Ez 10:41 жена, которую ты видел плачущею и старался утешать,
\vs 3Ez 10:42 которая потом сделалась невидима, но явился тебе город созидаемый,
\vs 3Ez 10:43 и которая тебе рассказала о смерти сына своего, вот что значит:
\vs 3Ez 10:44 жена, которую ты видел, это Сион. А что сказала тебе та, которую ты видел, как город только что созидаемый,
\vs 3Ez 10:45 что она тридцать лет была неплодна, этим указывается на то, что в продолжение тридцати лет в Сионе еще не была приносима жертва.
\vs 3Ez 10:46 По истечении тридцати лет неплодная родила сына: это было тогда, когда Соломон создал город и принес жертвы.
\vs 3Ez 10:47 А что она сказала тебе, что с трудом воспитала его, это было обитание в Иерусалиме.
\vs 3Ez 10:48 А что сын ее, как она сказала тебе, входя в чертог свой, упал и умер, это было падение Иерусалима.
\vs 3Ez 10:49 И вот, ты видел подобие ее, и как она скорбела о сыне, старался утешать ее в случившемся: то надлежало открыть тебе о сем.
\vs 3Ez 10:50 Ныне же Всевышний, видя, что ты скорбишь душею и всем сердцем болезнуешь о нем, показал тебе светлость славы его и красоту его.
\vs 3Ez 10:51 Для сего-то я повелел тебе жить в поле, где нет дома.
\vs 3Ez 10:52 Я знал, что Всевышний покажет тебе это;
\vs 3Ez 10:53 для того и повелел, чтобы ты пришел на поле, где не положено основания здания.
\vs 3Ez 10:54 Ибо не могло дело человеческого созидания существовать там, где начинал показываться город Всевышнего.
\vs 3Ez 10:55 Итак не бойся, и да не страшится сердце твое, но войди и посмотри на светлость и великолепие созидания, сколько могут видеть глаза твои.
\vs 3Ez 10:56 После того услышишь, сколько могут слышать уши твои.
\vs 3Ez 10:57 Ты блаженнее многих и призван к Всевышнему, как немногие.
\vs 3Ez 10:58 На завтрашнюю ночь оставайся здесь,
\vs 3Ez 10:59 и Всевышний покажет тебе видение величайших дел, которые Он сотворит для обитателей земли в последние дни.
\vs 3Ez 10:60 И спал я в ту ночь и в следующую, как он повелел мне.
\vs 3Ez 11:1 И видел я сон, и вот, поднялся с моря орел, у которого было двенадцать крыльев пернатых и три головы.
\vs 3Ez 11:2 И видел я: вот, он распростирал крылья свои над всею землею, и все ветры небесные дули на него и собирались облака.
\vs 3Ez 11:3 И видел я, что из перьев его выходили другие малые перья, и из тех выходили еще меньшие и короткие.
\vs 3Ez 11:4 Головы его покоились, и средняя голова была больше других голов, но также покоилась с ними.
\vs 3Ez 11:5 И видел я: вот орел летал на крыльях своих и царствовал над землею и над всеми обитателями ее.
\vs 3Ez 11:6 И видел я, что все поднебесное было покорно ему, и никто не сопротивлялся ему, ни одна из тварей, существующих на земле.
\vs 3Ez 11:7 И вот, орел стал на когти свои и испустил голос к перьям своим и сказал:
\vs 3Ez 11:8 не бодрствуйте все вместе; спите каждое на своем месте, и бодрствуйте поочередно,
\vs 3Ez 11:9 а головы пусть сохраняются на последнее время.
\vs 3Ez 11:10 Видел я, что голос его исходил не из голов его, но из средины тела его.
\vs 3Ez 11:11 Я сосчитал малые перья его; их было восемь.
\vs 3Ez 11:12 И вот, с правой стороны поднялось одно перо и воцарилось над всею землею.
\vs 3Ez 11:13 И когда воцарилось, пришел конец его, и не видно стало места его; потом поднялось другое перо и царствовало; это владычествовало долгое время.
\vs 3Ez 11:14 Когда оно царствовало и приблизился конец его, чтобы оно так же исчезло, как и первое,
\vs 3Ez 11:15 и вот, слышен был голос, говорящий ему:
\vs 3Ez 11:16 слушай ты, которое столько времени обладало землею! вот что я возвещаю тебе, прежде нежели начнешь исчезать:
\vs 3Ez 11:17 никто после тебя не будет владычествовать столько времени, как ты, и даже половины того.
\vs 3Ez 11:18 И поднялось третье перо, и владычествовало, как и прежние, но исчезло и оно.
\vs 3Ez 11:19 Так было и со всеми другими: они владычествовали и потом исчезали навсегда.
\vs 3Ez 11:20 Я видел, что по времени с правой стороны поднимались следующие перья, чтобы и им иметь начальство, и некоторые из них начальствовали, но тотчас исчезали;
\vs 3Ez 11:21 иные же из них поднимались, но не получали начальства.
\vs 3Ez 11:22 После сего не являлись более двенадцать перьев, ни два малых пера;
\vs 3Ez 11:23 и не осталось в теле орла ничего, кроме двух голов покоящихся и шести малых перьев.
\vs 3Ez 11:24 Я видел, и вот, из шести малых перьев отделились два и остались под головою, которая была с правой стороны, а четыре оставались на своем месте.
\vs 3Ez 11:25 Потом подкрыльные перья покушались подняться и начальствовать;
\vs 3Ez 11:26 и вот, одно поднялось, но тотчас исчезло;
\vs 3Ez 11:27 а следующие исчезали еще скорее, нежели прежние.
\vs 3Ez 11:28 И видел я: вот, два остававшиеся пера покушались также царствовать.
\vs 3Ez 11:29 Когда они покушались, одна из покоящихся голов, которая была средняя, пробудилась, и она была более других двух голов.
\vs 3Ez 11:30 И видел я, что две другие головы соединились с нею.
\vs 3Ez 11:31 И эта голова, обратившись с теми, которые были соединены с нею, пожрала два подкрыльных пера, которые покушались царствовать.
\vs 3Ez 11:32 Эта голова устрашила всю землю и владычествовала над обитателями земли с великим угнетением, и удерживала власть на земном шаре более всех крыльев, которые были.
\vs 3Ez 11:33 После того я видел, что и средняя голова внезапно исчезла, как и крылья;
\vs 3Ez 11:34 оставались две головы, которые подобным образом царствовали на земле и над ее обитателями.
\vs 3Ez 11:35 И вот, голова с правой стороны пожрала ту, которая была с левой.
\vs 3Ez 11:36 И слышал я голос, говорящий мне: смотри перед собою, и размышляй о том, что видишь.
\vs 3Ez 11:37 И видел я: вот, как бы лев, выбежавший из леса и рыкающий, испустил человеческий голос к орлу и сказал:
\vs 3Ez 11:38 слушай, что я буду говорить тебе и что скажет тебе Всевышний:
\vs 3Ez 11:39 не ты ли оставшийся из числа четырех животных, которых Я поставил царствовать в веке Моем, чтобы через них пришел конец времен тех?
\vs 3Ez 11:40 И четвертое из них пришло, победило всех прежде бывших животных и держало век в большом трепете и всю вселенную в лютом угнетении, и с тягостнейшим утеснением подвластных, и столь долгое время обитало на земле с коварством.
\vs 3Ez 11:41 Ты судил землю не по правде;
\vs 3Ez 11:42 ты утеснял кротких, обижал миролюбивых, любил лжецов, разорял жилища тех, которые приносили пользу, и разрушал стены тех, которые не делали тебе вреда.
\vs 3Ez 11:43 И взошла ко Всевышнему обида твоя, и гордыня твоя~--- к Крепкому.
\vs 3Ez 11:44 И воззрел Всевышний на времена гордыни, и вот, они кончились, и исполнилась мера злодейств ее.
\vs 3Ez 11:45 Поэтому исчезни ты, орел, с страшными крыльями твоими, с гнусными перьями твоими, со злыми головами твоими, с жестокими когтями твоими и со всем негодным телом твоим,
\vs 3Ez 11:46 чтобы отдохнула вся земля и освободилась от твоего насилия, и надеялась на суд и милосердие своего Создателя.
\vs 3Ez 12:1 Когда лев говорил к орлу эти слова, я увидел,
\vs 3Ez 12:2 что не являлась более голова, которая оставалась вместе с четырьмя крыльями, которые перешли к ней и поднимались, чтобы царствовать, но которых царство было слабо и исполнено возмущений.
\vs 3Ez 12:3 И я видел, и вот они исчезли, и все тело орла сгорало, и ужаснулась земля, и я от тревоги, исступления ума и от великого страха пробудился и сказал духу моему:
\vs 3Ez 12:4 вот, ты причинил мне это тем, что испытываешь пути Всевышнего.
\vs 3Ez 12:5 Вот, я еще трепещу сердцем и весьма изнемог духом моим, и нет во мне нисколько силы от великого страха, которым я поражен в эту ночь.
\vs 3Ez 12:6 Итак ныне я помолюсь Всевышнему, чтобы Он укрепил меня до конца.
\vs 3Ez 12:7 И сказал я: Владыко Господи! если я обрел благодать пред очами Твоими, если Ты нашел меня праведным пред многими, и если молитва моя подлинно взошла пред лице Твое,
\vs 3Ez 12:8 укрепи меня и покажи мне, рабу Твоему, значение сего страшного видения, чтобы вполне успокоить душу мою:
\vs 3Ez 12:9 ибо Ты судил меня достойным, чтобы показать мне последние времена. И Он сказал мне:
\vs 3Ez 12:10 Таково значение видения сего:
\vs 3Ez 12:11 орел, которого ты видел восходящим от моря, есть царство, показанное в видении Даниилу, брату твоему;
\vs 3Ez 12:12 но ему не было изъяснено то, что ныне Я изъясню тебе.
\vs 3Ez 12:13 Вот, приходят дни, когда восстанет на земле царство более страшное, нежели все царства, бывшие прежде него.
\vs 3Ez 12:14 В нем будут царствовать, один после другого, двенадцать царей.
\vs 3Ez 12:15 Второй из них начнет царствовать, и удержит власть более продолжительное время, нежели прочие двенадцать.
\vs 3Ez 12:16 Таково значение двенадцати крыльев, виденных тобою.
\vs 3Ez 12:17 А что ты слышал говоривший голос, исходящий не от голов орла, но из средины тела его,
\vs 3Ez 12:18 это означает, что после времени того царства произойдут немалые распри, и царство подвергнется опасности падения; но оно не падет тогда и восстановится в первоначальное состояние свое.
\vs 3Ez 12:19 А что ты видел восемь малых подкрыльных перьев, соединенных с крыльями, это означает,
\vs 3Ez 12:20 что восстанут в царстве восемь царей, которых времена будут легки и годы скоротечны, и два из них погибнут.
\vs 3Ez 12:21 Когда будет приближаться среднее время, четыре сохранятся до того времени, когда будет близок конец его; а два сохранятся до конца.
\vs 3Ez 12:22 А что ты видел три головы покоящиеся, это означает,
\vs 3Ez 12:23 что в последние дни царства Всевышний воздвигнет три царства и покорит им многие другие, и они будут владычествовать над землею и обитателями ее
\vs 3Ez 12:24 с б\acc{о}льшим утеснением, нежели все прежде бывшие; поэтому они и названы головами орла,
\vs 3Ez 12:25 ибо они-то довершат беззакония его и положат конец ему.
\vs 3Ez 12:26 А что ты видел, что большая голова не являлась более, это означает, что один из царей умрет на постели своей, впрочем с мучением,
\vs 3Ez 12:27 а двух остальных пожрет меч;
\vs 3Ez 12:28 меч одного пожрет того, который с ним, но и он в последствие времени умрет от меча.
\vs 3Ez 12:29 А что ты видел, два подкрыльных пера перешли на голову, находящуюся с правой стороны,
\vs 3Ez 12:30 это те, которых Всевышний сохранил к концу царства, то есть царство скудное и исполненное беспокойств.
\vs 3Ez 12:31 Лев, которого ты видел поднявшимся из леса и рыкающим, говорящим к орлу и обличающим его в неправдах его всеми словами его, которые ты слышал,
\vs 3Ez 12:32 это~--- Помазанник, сохраненный Всевышним к концу против них и нечестий их, Который обличит их и представит пред ними притеснения их.
\vs 3Ez 12:33 Он поставит их на суд живых и, обличив их, накажет их.
\vs 3Ez 12:34 Он по милосердию избавит остаток народа Моего, тех, которые сохранились в пределах Моих, и обрадует их, доколе не придет конец, день суда, о котором Я сказал тебе вначале.
\vs 3Ez 12:35 Таков сон, виденный тобою, и таково значение его.
\vs 3Ez 12:36 Ты один был достоин знать эту тайну Всевышнего.
\vs 3Ez 12:37 Все это, виденное тобою, напиши в книге и положи в сокровенном месте;
\vs 3Ez 12:38 и научи этому мудрых из народа твоего, которых сердц\acc{а} призн\acc{а}ешь способными принять и хранить сии тайны.
\vs 3Ez 12:39 А ты пребудь здесь еще семь дней, чтобы тебе показано было, что Всевышнему угодно будет показать тебе. И отошел от меня.
\rsbpar\vs 3Ez 12:40 Когда по истечении семи дней весь народ услышал, что я не возвратился в город, собрались все от малого до большого и, придя ко мне, говорили мне:
\vs 3Ez 12:41 чем согрешили мы против тебя? И чем обидели тебя, что ты, оставив нас, сидишь на этом месте?
\vs 3Ez 12:42 Ты один из всего народа остался нам, как гроздь от винограда, как светильник в темном месте и как пристань и корабль, спасенный от бури.
\vs 3Ez 12:43 Неужели мало бедствий, приключившихся нам?
\vs 3Ez 12:44 Если ты оставишь нас, то лучше было бы для нас сгореть, когда горел Сион.
\vs 3Ez 12:45 Ибо мы не лучше тех, которые умерли там. И плакали они с громким воплем. Отвечая им, я сказал:
\vs 3Ez 12:46 надейся, Израиль, и не скорби, дом Иакова;
\vs 3Ez 12:47 ибо помнит о вас Всевышний, и Крепкий не забыл вас в напасти.
\vs 3Ez 12:48 И я не оставил вас и не ушел от вас, но пришел на это место, чтобы помолиться о разоренном Сионе и просить милосердия уничиженной святыне вашей.
\vs 3Ez 12:49 Теперь идите каждый в дом свой, и я приду к вам после сих дней.
\vs 3Ez 12:50 И пошел народ, как я сказал ему, в город,
\vs 3Ez 12:51 а я оставался в поле в продолжение семи дней, как повелено мне, и питался в те дни только цветами полевыми, и трава была мне пищею.
\vs 3Ez 13:1 И было после семи дней, я видел ночью сон:
\vs 3Ez 13:2 вот, поднялся ветер с моря, чтобы возмутить все волны его.
\vs 3Ez 13:3 Я смотрел, и вот, вышел крепкий муж с воинством небесным, и куда он ни обращал лице свое, чтобы взглянуть, все трепетало, что виднелось под ним;
\vs 3Ez 13:4 и куда ни выходил голос из уст его, загорались все, которые слышали голос его, подобно тому, как тает воск, когда почувствует огонь.
\vs 3Ez 13:5 И после этого видел я: вот, собралось множество людей, которым не было числа, от четырех ветров небесных, чтобы преодолеть этого мужа, который поднялся с моря.
\vs 3Ez 13:6 Видел я, и вот, он изваял себе большую гору и взлетел на нее.
\vs 3Ez 13:7 Я старался увидеть ту страну или место, откуда изваяна была эта гора, но не мог.
\vs 3Ez 13:8 После сего видел я, что все, которые собрались победить его, очень испугались и однако же осмелились воевать.
\vs 3Ez 13:9 Он же, когда увидел устремление идущего множества, не поднял руки своей, ни копья не держал и никакого оружия воинского;
\vs 3Ez 13:10 но только, как я видел, он испускал из уст своих как бы дуновение огня и из губ своих~--- как бы дыхание пламени и с языка своего пускал искры и бури, и все это смешалось вместе: и дуновение огня и дыхание пламени и сильная буря.
\vs 3Ez 13:11 И стремительно напал он на это множество, которое приготовилось сразиться, и сжег всех, так что ничего не видно было из бесчисленного множества, кроме праха, и только был запах от дыма; увидел я это, и устрашился.
\vs 3Ez 13:12 После сего я видел того мужа сходящим с горы и призывающим к себе другое множество, мирное.
\vs 3Ez 13:13 И многие приступали к нему, иные с лицами веселыми, а иные с печальными, иные были связаны, иных приносили,~--- и я изнемог от великого страха, пробудился и сказал:
\vs 3Ez 13:14 Ты от начала показал рабу Твоему чудеса сии и судил меня достойным, чтобы принять молитву мою;
\vs 3Ez 13:15 покажи же мне и значение сна сего,
\vs 3Ez 13:16 потому что, как я понимаю разумом моим, горе тем, которые оставлены будут до тех дней, а еще более горе тем, которые не оставлены.
\vs 3Ez 13:17 Ибо те, которые не оставлены, были печальны.
\vs 3Ez 13:18 Теперь я понимаю, что то, что отложено на последние дни, встретит их, но и тех, которые оставлены.
\vs 3Ez 13:19 Поэтому они пришли в большие опасности и большие затруднения, как показывают эти сны.
\vs 3Ez 13:20 Но легче находящемуся в опасности потерпеть это, нежели перейти подобно облаку из мира сего и не видеть того, что будет в последние времена. Он отвечал мне и сказал:
\vs 3Ez 13:21 И значение видения Я скажу тебе, и о чем ты говорил, открою тебе.
\vs 3Ez 13:22 Так как ты говорил о тех, которые оставлены, то вот объяснение:
\vs 3Ez 13:23 кто выдержит опасность в то время, тот сохранил себя, а которые впадут в опасность, это те, которые имеют дела и веру во Всемогущего.
\vs 3Ez 13:24 Итак знай, что те, которые оставлены, блаженнее умерших.
\vs 3Ez 13:25 Вот объяснение видения: так как ты видел мужа, восходящего из средины моря,
\vs 3Ez 13:26 это тот, которого Всевышний хранит многие времена, который самим собою избавит творение свое и управит тех, которые оставлены.
\vs 3Ez 13:27 А что ты видел исходивший из уст его как бы ветер, огонь и бурю,
\vs 3Ez 13:28 и что он не держал ни копья и никакого воинского оружия, но устремление его поразило множество, которое пришло, чтобы победить его, то вот объяснение:
\vs 3Ez 13:29 вот, наступают дни, когда Всевышний начнет избавлять тех, которые на земле,
\vs 3Ez 13:30 и приведет в изумление живущих на земле.
\vs 3Ez 13:31 И будут предпринимать войны одни против других, город против города, одно место против другого, народ против народа, царство против царства.
\vs 3Ez 13:32 Когда это будет и явятся знамения, которые Я показал тебе прежде, тогда откроется Сын Мой, Которого ты видел, как мужа восходящего.
\vs 3Ez 13:33 И когда все народы услышат глас Его, каждый оставит войну в своей собственной стране, которую они имеют между собою.
\vs 3Ez 13:34 И соберется в одно собрание множество бесчисленное, как бы желая идти и победить Его.
\vs 3Ez 13:35 Он же станет на верху горы Сиона.
\vs 3Ez 13:36 И Сион придет и покажется всем приготовленный и устроенный, как ты видел гору, изваянную без рук.
\vs 3Ez 13:37 Сын же Мой обличит нечестия, изобретенные этими народами, которые своими злыми помышлениями приблизили бурю и мучения, которыми они начнут мучиться,
\vs 3Ez 13:38 и которые подобны огню; и Он истребит их без труда законом, который подобен огню.
\vs 3Ez 13:39 А что ты видел, что Он собирал к себе другое, мирное общество:
\vs 3Ez 13:40 это десять колен, которые отведены были пленными из земли своей во дни царя Осии, которого отвел в плен Салманассар, царь Ассирийский, и перевел их за реку, и переведены были в землю иную.
\vs 3Ez 13:41 Они же положили в совете своем, чтобы оставить множество язычников и отправиться в дальнюю страну, где никогда не обитал род человеческий,
\vs 3Ez 13:42 чтобы там соблюдать законы свои, которых они не соблюдали в стране своей.
\vs 3Ez 13:43 Тесными входами подошли они к реке Евфрату;
\vs 3Ez 13:44 ибо Всевышний сотворил тогда для них чудеса и остановил жилы реки, доколе они проходили;
\vs 3Ez 13:45 ибо через эту страну шли они долго, полтора года; эта страна называется Арсареф.
\vs 3Ez 13:46 Там жили они до последнего времени. И ныне, когда они начнут приходить,
\vs 3Ez 13:47 Всевышний снова остановит жилы реки, чтобы они могли пройти; поэтому ты видел множество мирное.
\vs 3Ez 13:48 Но которые оставлены от народа твоего, это те, которые находятся внутри пределов Моих.
\vs 3Ez 13:49 Ибо, когда начнет Он истреблять множество собравшихся вместе народов, Он защитит народ Свой, который останется.
\vs 3Ez 13:50 И тогда покажет им множество чудес.
\vs 3Ez 13:51 Я сказал: Владыко Господи! Объясни мне это, для чего видел я мужа, восходящего из средины моря?
\vs 3Ez 13:52 И Он сказал мне: как не можешь ты исследовать и познать того, что во глубине моря, так никто не может на земле видеть Сына Моего, ни тех, которые с Ним, разве только во время дня Его.
\vs 3Ez 13:53 Вот истолкование сна, который ты видел и которым ты один здесь просвещен.
\vs 3Ez 13:54 Ты оставил дела твои и упражнялся в законе Моем, и взыскал его,
\vs 3Ez 13:55 ибо жизнь твою ты устроил в мудрости и рассудительность назвал твоею матерью.
\vs 3Ez 13:56 Поэтому Я показал тебе воздаяния у Всевышнего; после трех дней Я покажу тебе другое и открою тебе важное и чудное.
\vs 3Ez 13:57 Тогда я пошел и вышел в поле, много славя и благодаря Всевышнего за чудеса, которые Он совершал по временам,
\vs 3Ez 13:58 и что Он управляет настоящим и тем, что произойдет во времена,~--- и там я сидел три дня.
\vs 3Ez 14:1 И было после трех дней, я сидел под дубом, и вот, голос вышел из куста против меня и сказал: Ездра, Ездра!
\vs 3Ez 14:2 Я сказал: вот я, Господи. И встал на ноги мои.
\vs 3Ez 14:3 Тогда сказал Он мне: в кусте Я открылся и говорил Моисею, когда народ Мой был рабом в Египте;
\vs 3Ez 14:4 и послал его и вывел народ Мой из Египта, и привел его к горе Синаю и держал его у Себя много дней,
\vs 3Ez 14:5 и открыл ему много чудес и показал тайны времен и конец, и заповедал ему, сказав:
\vs 3Ez 14:6 <<Эти слова объяви, а прочие скрой>>.
\vs 3Ez 14:7 И ныне тебе говорю:
\vs 3Ez 14:8 знамения, которые Я показал тебе, и сны, которые ты видел, и толкования, которые слышал, положи в сердце твоем;
\vs 3Ez 14:9 потому что ты взят будешь от людей и будешь обращаться с Сыном Моим и с подобными тебе, доколе не окончатся времена.
\vs 3Ez 14:10 Ибо век потерял свою юность, и времена приближаются к старости,
\vs 3Ez 14:11 так как век разделен на двенадцать частей, и девять частей его и половина десятой части уже прошли,
\vs 3Ez 14:12 и остается то, что после половины десятой части.
\vs 3Ez 14:13 Итак ныне устрой дом твой и вразуми народ твой, утешь уничиженных и отрекись тления,
\vs 3Ez 14:14 и отпусти от себя смертные помышления, отбрось тягости людские, сними с себя немощи естества и отложи в сторону тягостные для тебя помыслы, и готовься переселиться от времен сих.
\vs 3Ez 14:15 Ибо после больше будет бедствий, нежели сколько ты видел ныне.
\vs 3Ez 14:16 Сколько будет слабеть век от старости, столько будет умножаться зло для живущих.
\vs 3Ez 14:17 Еще дальше удалится истина, и приблизится ложь; уже поспешает прийти видение, которое ты видел.
\vs 3Ez 14:18 Тогда отвечал я и сказал: вот, я~--- пред Тобою, Господи;
\vs 3Ez 14:19 я пойду, как Ты повелел мне, и вразумлю нынешний народ; но кто научит тех, которые потом родятся?
\vs 3Ez 14:20 Ибо век во тьме лежит, и живущие в нем~--- без света;
\vs 3Ez 14:21 потому что закон Твой сожжен, и оттого никто не знает, что соделано Тобою или что должно им делать.
\vs 3Ez 14:22 Но если я приобрел милость у Тебя, ниспошли на меня Духа Святаго, чтобы я написал все, что было соделано в мире от начала, что было написано в законе Твоем, дабы люди могли найти стезю и дабы те, которые захотят жить в последние времена, могли жить.
\vs 3Ez 14:23 И Он в ответ сказал мне: иди, собери народ и скажи ему, чтобы он не искал тебя в продолжение сорока дней.
\vs 3Ez 14:24 Ты же приготовь себе побольше дощечек и возьми с собою Сария, Даврия, Салемия, Ехана и Асиеля, этих пять, способных писать скоро.
\vs 3Ez 14:25 И приди сюда, и Я возжгу в сердце твоем светильник разума, который не угаснет, доколе не окончится то, что ты начнешь писать.
\vs 3Ez 14:26 И когда ты совершишь это, то иное объяви, а иное тайно передай мудрым. Завтра в этот час ты начнешь писать.
\vs 3Ez 14:27 Тогда я пошел, как Он повелел мне, и собрал весь народ и сказал:
\vs 3Ez 14:28 слушай, Израиль, слова сии:
\vs 3Ez 14:29 отцы наши были странниками в Египте, и освобождены были оттуда,
\vs 3Ez 14:30 и приняли закон жизни, которого не сохранили, который и вы после них нарушили.
\vs 3Ez 14:31 И дана была вам земля в наследие и земля Сион; но отцы ваши и вы делали беззаконие и не держались тех путей, которые Всевышний заповедал вам.
\vs 3Ez 14:32 И Он, как праведный судия, отнял у вас ныне, что даровал вам.
\vs 3Ez 14:33 И ныне вы здесь и братья ваши между вами.
\vs 3Ez 14:34 Если вы будете управлять чувством вашим и образуете сердце ваше, то сохраните жизнь и по смерти пол\acc{у}чите милость.
\vs 3Ez 14:35 Ибо по смерти настанет суд, когда мы оживем; и тогда имена праведных будут объявлены и показаны дела нечестивых.
\vs 3Ez 14:36 Никто не приходи ко мне ныне и не ищи меня до сорока дней.
\vs 3Ez 14:37 И взял я пять мужей, как Он заповедал мне, и пошли мы в поле и остались там.
\vs 3Ez 14:38 И вот, на другой день голос воззвал ко мне: Ездра! открой уста твои и выпей то, чем Я напою тебя.
\vs 3Ez 14:39 Я открыл уста мои, и вот полная чаша подана была мне, которая была наполнена как бы водою, но цвет того был подобен огню.
\vs 3Ez 14:40 И взял я и пил; и когда я пил, сердце мое дышало разумом и в груди моей возрастала мудрость, ибо дух мой подкреплялся памятью;
\vs 3Ez 14:41 уста мои были открыты и больше не закрывались.
\vs 3Ez 14:42 Всевышний даровал разум пяти мужам, и они ночью писали по порядку, что было говорено им и чего они не знали.
\vs 3Ez 14:43 Ночью они ели хлеб; а я говорил днем и не молчал ночью.
\vs 3Ez 14:44 Написаны же были в сорок дней девяносто четыре книги.
\vs 3Ez 14:45 И когда исполнилось сорок дней,
\vs 3Ez 14:46 Всевышний сказал: первые, которые ты написал, положи открыто, чтобы могли читать и достойные и недостойные,
\vs 3Ez 14:47 но последние семьдесят сбереги, чтобы передать их мудрым из народа;
\vs 3Ez 14:48 потому что в них проводник разума, источник мудрости и река знания. Так я и сделал.
\vs 3Ez 15:1 Говори вслух народа Моего слова пророчества, которые вложу Я в уста твои, говорит Господь;
\vs 3Ez 15:2 и сделай, чтобы они написаны были на хартии, потому что они верны и истинны.
\vs 3Ez 15:3 Не бойся, что будут замышлять против тебя, и да не смущает тебя неверие тех, которые будут говорить против тебя,
\vs 3Ez 15:4 ибо всякий неверующий в неверии своем умрет.
\vs 3Ez 15:5 Вот, Я наведу, говорит Господь, на круг земной бедствия: меч и голод, и смерть и пагубу
\vs 3Ez 15:6 за то, что нечестие \bibemph{людей} осквернило всю землю, и пагубные дела их переполнились.
\vs 3Ez 15:7 Посему говорит Господь:
\vs 3Ez 15:8 Я уже не буду молчать о беззакониях, которые совершают они нечестиво, и не буду терпеть в них того, что они делают преступно: вот, кровь неповинная и праведная вопиет ко Мне, и души праведных вопиют непрестанно.
\vs 3Ez 15:9 Отмщу им, говорит Господь, и возьму от них к Себе всякую кровь неповинную.
\vs 3Ez 15:10 Вот, народ Мой ведется как стадо на заклание; не потерплю более, чтобы он жил в Египте,
\vs 3Ez 15:11 но выведу его рукою сильною и мышцею высокою, и поражу Египет казнью, как прежде, и погублю всю землю его.
\vs 3Ez 15:12 Восплачет Египет и основания его, пораженные казнью и мщением, которое наведет на него Бог.
\vs 3Ez 15:13 Восплачут земледельцы, возделывающие землю, потому что оскудеют у них семена от ржавчины и от града и от страшной звезды.
\vs 3Ez 15:14 Горе веку и тем, которые живут в нем,
\vs 3Ez 15:15 ибо приблизился меч и истребление их, и восстанет народ на народ для войны, и мечи в руках их.
\vs 3Ez 15:16 Люди сделаются непостоянными и, одни других одолевая, вознерадят о царе своем, и начальники~--- о ходе дел своих в пределах своей власти.
\vs 3Ez 15:17 Пожелает человек идти в город, и не возможет,
\vs 3Ez 15:18 ибо, по причине их гордости, города возмутятся, домы будут разорены, на людей нападет страх.
\vs 3Ez 15:19 Не сжалится человек над ближним своим, предавая домы их на разорение оружием, расхищая имущество их по причине голода и многих бед.
\vs 3Ez 15:20 Вот, Я созываю, говорит Бог, всех царей земли, от востока и юга, от севера и Ливана, чтобы благоговели предо Мною и обратились к себе самим, и чтобы воздать им, что они делали тем.
\vs 3Ez 15:21 Как поступают они даже доселе с избранными Моими, так поступлю \bibemph{с ними} и воздам в недро их, говорит Господь Бог.
\vs 3Ez 15:22 Не пощадит десница Моя грешников, и меч не перестанет поражать проливающих на землю неповинную кровь.
\vs 3Ez 15:23 Исшел огонь из гнева Его и истребил основания земли и грешников, как зажженную солому.
\vs 3Ez 15:24 Горе грешникам и не соблюдающим заповедей Моих! говорит Господь.
\vs 3Ez 15:25 Не пощажу их. Удалитесь, сыновья отступников, не оскверняйте святыни Моей.
\vs 3Ez 15:26 Господь знает всех, которые грешат против Него; потому предал их на смерть и на убиение.
\vs 3Ez 15:27 На круг земной пришли уже бедствия, и вы пребудете в них. Бог не избавит вас, потому что вы согрешили против Него.
\vs 3Ez 15:28 Вот, видение грозное, и лице его от востока.
\vs 3Ez 15:29 Выступят порождения драконов Аравийских на многих колесницах и с быстротою ветра понесутся по земле, так что наведут страх и трепет на всех, которые услышат о них.
\vs 3Ez 15:30 Выйдут, как вепри из леса, Кармоняне, неистовствующие в ярости, и придут в великой силе, вступят в борьбу с ними и опустошат часть земли Ассирийской.
\vs 3Ez 15:31 Потом драконы, помнящие происхождение свое, одержат верх и, обладая великою силою, обратятся преследовать тех.
\vs 3Ez 15:32 Те смутятся, умолкнут перед силою их и обратят ноги свои в бегство.
\vs 3Ez 15:33 Но находящийся в засаде со стороны Ассириян окружит их и умертвит одного из них; в войске их произойдет страх и трепет и ропот на царей их.
\vs 3Ez 15:34 Вот, облака от востока и от севера до юга, и вид их весьма грозен, исполнен свирепости и бури.
\vs 3Ez 15:35 Они столкнутся между собою, и свергнут много звезд на землю и звезду их; и будет кровь от меча до чрева,
\vs 3Ez 15:36 и помет человеческий~--- до седла верблюда; страх и трепет великий будет на земле.
\vs 3Ez 15:37 Ужаснутся \bibemph{все}, которые увидят эту свирепость, и вострепещут.
\vs 3Ez 15:38 После того много раз будут подниматься бури от юга и севера и частью от запада,
\vs 3Ez 15:39 и ветры сильные поднимутся от востока и откроют его и облако, которое Я подвигнул во гневе; а звезда, назначенная для устрашения при восточном и западном ветре, повредится.
\vs 3Ez 15:40 И поднимутся облака, великие и сильные, полные свирепости, и звезда, чтобы устрашить всю землю и жителей ее; и прольют на всякое место, высокое и возвышенное, страшную звезду,
\vs 3Ez 15:41 огонь и град, мечи летающие и многие воды, чтобы наполнить все поля и все источники множеством вод.
\vs 3Ez 15:42 И затопят город, и стены, и горы, и холмы, и дерева в лесах, и траву в лугах, и хлебные растения их;
\vs 3Ez 15:43 и пройдут безостановочно до Вавилона и сокрушат его;
\vs 3Ez 15:44 соберутся к нему и окружат его; прольют звезду и ярость на него. И поднимется пыль и дым до самого неба, и все кругом будут оплакивать его,
\vs 3Ez 15:45 а те, которые останутся подвластными ему, будут служить тем, которые навели страх.
\vs 3Ez 15:46 И ты, Асия, соучастница в надежде Вавилона и в славе его:
\vs 3Ez 15:47 горе тебе, бедная, за то, что уподоблялась ему и украшала дочерей твоих в блудодеянии, чтобы они нравились и славились у любовников твоих, которые желали всегда блудодействовать с тобою.
\vs 3Ez 15:48 Ты подражала ненавистному во всех делах и предприятиях его.
\vs 3Ez 15:49 За то, говорит Бог, пошлю на тебя бедствия: вдовство, нищету, и голод, и меч, и язву, чтобы опустошить домы твои насилием и смертью.
\vs 3Ez 15:50 И слава могущества твоего засохнет, как цвет, когда настанет зной, посланный на тебя.
\vs 3Ez 15:51 Ты изнеможешь, как нищая, избитая и израненная женщинами, чтобы люди знатные и любовники не могли принимать тебя.
\vs 3Ez 15:52 Стал ли бы Я так ненавидеть тебя, говорит Господь,
\vs 3Ez 15:53 если бы ты не убивала избранных Моих во всякое время, поднимая руки на поражение их и глумясь над смертью их, когда ты была в опьянении?
\vs 3Ez 15:54 Украшай твое лице.
\vs 3Ez 15:55 Мзда блудодеяния твоего в недре твоем; за то и получишь ты воздаяние.
\vs 3Ez 15:56 Как поступала ты с избранными Моими, говорит Господь, так с тобою поступит Бог, и подвергнет тебя бедствиям.
\vs 3Ez 15:57 Дети твои погибнут от голода, ты падешь от меча, города твои будут разрушены, и все твои падут в поле от меча.
\vs 3Ez 15:58 А которые на горах, те погибнут от голода, и будут есть плоть свою по недостатку хлеба и пить кровь по недостатку воды.
\vs 3Ez 15:59 В несчастии пойдешь по морям,~--- и там встретишь беды.
\vs 3Ez 15:60 Во время переходов твоих они бросятся на опустошенный город, и истребят часть земли твоей, и часть славы твоей уничтожат.
\vs 3Ez 15:61 Разоренная, ты послужишь для них соломою, а они для тебя будут огнем;
\vs 3Ez 15:62 и истребят тебя, и города твои, землю твою, горы твои, все леса твои и дерева плодоносные сожгут огнем.
\vs 3Ez 15:63 Сыновей твоих уведут в плен, имущество твое захватят в добычу, и славу твою истребят.
\vs 3Ez 16:1 Горе тебе, Вавилон и Асия, горе тебе, Египет и Сирия!
\vs 3Ez 16:2 Препояшьтесь вретищем и власяницами, оплакивайте сыновей ваших, и болезнуйте, потому что приблизилась ваша погибель.
\vs 3Ez 16:3 Послан на вас меч,~--- и кто отклонит его?
\vs 3Ez 16:4 Послан на вас огонь,~--- и кто угасит его?
\vs 3Ez 16:5 Посланы на вас бедствия,~--- и кто отвратит их?
\vs 3Ez 16:6 Прогонит ли кто голодного льва в лесу, или угасит ли мгновенно огонь в соломе, когда он начнет разгораться?
\vs 3Ez 16:7 Отразит ли кто стрелу, пущенную стрелком сильным?
\vs 3Ez 16:8 Господь сильный посылает бедствия,~--- и кто отвратит их?
\vs 3Ez 16:9 Исшел огонь от гнева Его,~--- и кто угасит его?
\vs 3Ez 16:10 Он блеснет молнией,~--- и кто не убоится? Возгремит,~--- и кто не ужаснется?
\vs 3Ez 16:11 Господь воззрит грозно,~--- и кто не сокрушится до основания от лица Его?
\vs 3Ez 16:12 Содрогнулась земля и основания ее; море волнуется со дна, и волны его возмущаются и рыбы его от лица Господа и от величия силы Его.
\vs 3Ez 16:13 Ибо сильна Его десница, напрягающая лук, остры Его стрелы, пускаемые Им, не ослабеют, когда будут посылаемы до концов земли.
\vs 3Ez 16:14 Вот, посылаются бедствия, и не возвратятся, доколе не придут на землю.
\vs 3Ez 16:15 Возгорается огонь, и не угаснет, доколе не попалит основания земли.
\vs 3Ez 16:16 Как стрела, пущенная сильным стрелком, не возвращается, так не возвратятся бедствия, которые будут посланы на землю.
\vs 3Ez 16:17 Горе мне, горе мне! Кто избавит меня в те дни?
\vs 3Ez 16:18 Начнутся болезни,~--- и многие восстенают; начнется голод,~--- и многие будут гибнуть; начнутся войны,~--- и начальствующими овладеет страх; начнутся бедствия,~--- и все вострепещут.
\vs 3Ez 16:19 Что мне делать тогда, когда придут бедствия?
\vs 3Ez 16:20 Вот, голод и язва, и скорбь и теснота посланы как бичи для исправления:
\vs 3Ez 16:21 но при всем этом \bibemph{люди} не обратятся от беззаконий своих и о бичах не всегда будут помнить.
\vs 3Ez 16:22 Вот, на земле будет дешевизна во всем, и подумают, что настал мир; но тогда-то и постигнут землю бедствия~--- меч, голод и великое смятение.
\vs 3Ez 16:23 От голода погибнут очень многие жители земли, а прочие, которые перенесут голод, падут от меча.
\vs 3Ez 16:24 И трупы, как навоз, будут выбрасываемы, и некому будет оплакивать их, ибо земля опустеет, и города ее будут разрушены.
\vs 3Ez 16:25 Не останется никого, кто возделывал бы землю и сеял на ней.
\vs 3Ez 16:26 Дерева дадут плоды, и кто будет собирать их?
\vs 3Ez 16:27 Виноград созреет, и кто будет топтать его? Ибо повсюду будет великое запустение.
\vs 3Ez 16:28 Трудно будет человеку увидеть человека, или услышать голос его,
\vs 3Ez 16:29 ибо из жителей города останется не более десяти, и из поселян~--- человека два, которые скроются в густых рощах и расселинах скал.
\vs 3Ez 16:30 Как в масличном саду остаются иногда на деревах три или четыре маслины,
\vs 3Ez 16:31 или в винограднике обобранном не досмотрят несколько гроздей те, которые внимательно обирают виноград:
\vs 3Ez 16:32 так в те дни останутся трое или четверо при обыске домов их с мечом.
\vs 3Ez 16:33 Земля останется в запустении, поля ее заглохнут, дороги ее и все тропинки ее зарастут терном, потому что некому будет ходить по ним.
\vs 3Ez 16:34 Плакать будут девицы, не имея женихов; плакать будут жены, не имея мужей; плакать будут дочери их, не имея помощи.
\vs 3Ez 16:35 Женихов их убьют на войне, и мужья их погибнут от голода.
\vs 3Ez 16:36 Слушайте это, и вразумляйтесь, рабы Господни!
\vs 3Ez 16:37 Это~--- слово Господа: внимайте ему, и не верьте богам, о которых говорит Господь.
\vs 3Ez 16:38 Вот, приближаются бедствия, и не замедлят.
\vs 3Ez 16:39 Как у беременной женщины, когда в девятый месяц настанет ей пора родить сына, часа за два или за три до рождения, боли охватывают чрево ее и, при выходе младенца из чрева, не замедлят ни на одну минуту:
\vs 3Ez 16:40 так не замедлят прийти на землю бедствия, и люди того времени восстенают; боли охватят их.
\vs 3Ez 16:41 Слушай слово, народ мой: готовьтесь на брань, и среди бедствий будьте как пришельцы земли.
\vs 3Ez 16:42 Продающий пусть будет, как собирающийся в бегство, и покупающий~--- как готовящийся на погибель;
\vs 3Ez 16:43 торгующий~--- как не ожидающий никакой прибыли, и строящий дом~--- как не надеющийся жить в нем.
\vs 3Ez 16:44 Сеятель пусть думает, что не пожнет, и виноградарь,~--- что не соберет винограда;
\vs 3Ez 16:45 вступающие в брак,~--- что не будут рождать детей, и не вступающие,~--- как вдовцы.
\vs 3Ez 16:46 Посему все трудящиеся без пользы трудятся,
\vs 3Ez 16:47 ибо плодами трудов их воспользуются чужеземцы, и имущество их расхитят, домы их разрушат и сыновей их поработят, потому что в плену и в голоде они рождают детей своих.
\vs 3Ez 16:48 Кто занимается хищничеством, тех, чем дольше украшают они города и домы свои, владения и лица свои,
\vs 3Ez 16:49 тем более возненавижу за грехи их, говорит Господь.
\vs 3Ez 16:50 Как блудница ненавидит женщину честную и весьма благонравную,
\vs 3Ez 16:51 так правда возненавидит неправду, украшающую себя, и обвинит ее в лице, когда придет Тот, Кто будет защищать преследующего всякий грех на земле.
\vs 3Ez 16:52 Потому не подражайте неправде и делам ее,
\vs 3Ez 16:53 ибо еще немного, и неправда будет удалена с земли, а правда воцарится над вами.
\vs 3Ez 16:54 Пусть не говорит грешник, что он не согрешил, потому что горящие угли возгорятся на голове того, кто говорит: я не согрешил пред Господом Богом и славою Его.
\vs 3Ez 16:55 Господь знает все дела людей и начинания их, и помышления их и сердца их.
\vs 3Ez 16:56 Он сказал: <<да будет земля>>,~--- и земля явилась; <<да будет небо>>,~--- и было.
\vs 3Ez 16:57 Словом Его сотворены звезды, и Он знает число звезд.
\vs 3Ez 16:58 Он созерцает бездны и сокровенное в них, измерил море и что в нем.
\vs 3Ez 16:59 Словом Своим Он заключил море среди вод и землю повесил на водах.
\vs 3Ez 16:60 Он простер небо, как шатер, на водах основал его.
\vs 3Ez 16:61 Он поместил в пустыне источники вод и озера на вершинах гор, для низведения рек с высоких скал, чтобы напоять землю.
\vs 3Ez 16:62 Он сотворил человека и положил сердце его в средине тела, и вложил в него дух, жизнь и разум
\vs 3Ez 16:63 и дыхание Бога всемогущего, Который сотворил все и созерцает все сокровенное в сокровенных земли.
\vs 3Ez 16:64 Он знает намерение ваше и что помышляете вы в сердцах ваших, когда грешите и хотите скрыть грехи ваши.
\vs 3Ez 16:65 Потому Господь совершенно ясно видит все дела ваши, и обличит всех вас;
\vs 3Ez 16:66 и вы будете посрамлены, когда грехи ваши откроются перед людьми, и беззакония предстанут обвинителями в тот день.
\vs 3Ez 16:67 Что вы сделаете и как скроете грехи ваши пред Богом и Ангелами Его?
\vs 3Ez 16:68 Вот, Бог~--- Судия; бойтесь Его; оставьте грехи ваши и навсегда перестаньте делать беззакония, и Бог изведет вас и избавит от всякой скорби.
\vs 3Ez 16:69 Ибо вот, возгорается на вас ярость многочисленного полчища, и схватят некоторых из вас и умертвят для принесения в жертву идолам.
\vs 3Ez 16:70 Кто будет единомыслен с ними, тех подвергнут они посмеянию, поношению и попранию.
\vs 3Ez 16:71 Ибо по всем местам и в соседних городах многие восстанут против боящихся Господа.
\vs 3Ez 16:72 Будут, как исступленные, без пощады расхищать и опустошать все у боящихся Господа.
\vs 3Ez 16:73 Опустошат и расхитят имущество их, и из домов их изгонят их.
\vs 3Ez 16:74 Тогда настанет испытание избранным Моим, как золото испытывается огнем.
\vs 3Ez 16:75 Слушайте, возлюбленные Мои, говорит Господь: вот перед вами дни скорби, и от них Я избавлю вас.
\vs 3Ez 16:76 Не бойтесь и не сомневайтесь, ибо вождь ваш~--- Бог.
\vs 3Ez 16:77 Если будете исполнять заповеди и повеления Мои, говорит Господь Бог, то грехи ваши не будут бременем, подавляющим вас, и беззакония ваши не превозмогут вас.
\vs 3Ez 16:78 Горе тем, которые связаны грехами своими и покрыты беззакониями своими! Это~--- поле, которое заросло кустарником и через которое путь покрыт терном, так что человек проходить не может: оно оставляется, и обрекается огню на истребление.

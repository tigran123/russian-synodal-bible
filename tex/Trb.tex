\bibbookdescr{Trb}{
  inline={Завещание Рувима,\\первородного сына Иакова и Лии\fns{В греч. тексте $+$ ``о мыслях''.}},
  toc={Завещание Рувима},
  bookmark={Завещание Рувима},
  header={Завещание Рувима},
  abbr={Рув}
}
\vs Trb 1:1
Список завета Рувима, которое он завещал сыновьям своим,
прежде чем умереть, в 125-ый год жизни своей.
\vs Trb 1:2
Два года спустя после кончины Иосифа, брата его, занемог Рувим,
и собрались проведать его дети и дети детей его.
\vs Trb 1:3
И сказал он им: дети мои, я умираю и отправляюсь дорогою отцов моих.
\vs Trb 1:4
Увидев же там Иуду, Гада и Асира,
братьев своих, сказал им:
поднимите меня, братья, дабы говорить мне к братьям моим
и детям моим о том, что сокрыто в сердце у меня.
Ибо я отхожу от вас отныне.
\vs Trb 1:5
И поднявшись, поцеловал их и, рыдая, сказал им:
слушайте, братья мои и дети мои, внемлите Рувиму,
отцу вашему, что завещаю вам.

\vs Trb 1:6
Вот, я заклинаю вас Богом небесным, да не совершите проступка
по незнанию юности, как я предался пороку и осквернил
ложе отца моего Иакова.
\vs Trb 1:7
Говорю же вам, что заполнила болезнь великая
поясницу мою на 7 месяцев, и, если бы не просил отец
наш Иаков обо мне Господа, желал убить меня Господь.
\vs Trb 1:8
Было мне 30 лет, когда совершил я злое пред лицом Господа,
и слабость смертная охватила меня на 7 месяцев.
\vs Trb 1:9
После же этого каялся я пред лицом Господа 7 лет,
ибо возжелала того душа моя.
\vs Trb 1:10
И не пил я вина и сикера, и мясо не входило в уста мои,
и никакого хлеба вожделенного не пробовал я,
но пребывал в печали о согрешении моём, ибо оно было велико,
и не было подобного ему в Израиле.

\vs Trb 2:1
А теперь, выслушайте меня, дети мои, что увидел я в раскаянии моём о
7-и духах соблазна.
\vs Trb 2:2
Ибо 7 духов поставлены против человека Велиаром, и они суть источники дел юношеских.
\vs Trb 2:3
И \bibemph{иные} 7 духов даны ему по сотворении его,
дабы в них было всякое дело человеческое.
\vs Trb 2:4
Первый~--- дух жизни, на коем зиждется его существование.
Второй~--- дух зрения, из коего происходит желание.
\vs Trb 2:5
Третий~--- дух слуха, из коего происходит научение.
Четвёртый~--- дух обоняния, из коего происходит вкус
от втягивания воздуха и вдыхания.
\vs Trb 2:6
Пятый~--- дух речи, из коей происходит знание.
\vs Trb 2:7
Шестой~--- дух вкуса, из коего происходит вкушение пищи и питья,
и сила на нём зиждется, ибо в пище~---  основание силы.
\vs Trb 2:8
Седьмой~--- дух деторождения и плотского сообщения,
из коего от любви к наслаждениям происходит собрание грехов.
\vs Trb 2:9
Оттого этот дух~--- последний в сотворении и первый в юности,
ибо исполнен неразумия, он же ведет юношу, словно слепого в яму
и словно стадо к пропасти.

\vs Trb 3:1
Ко всем же этим есть ещё восьмой дух~--- дух сна,
на коем основан экстаз естества и образ смерти.
\vs Trb 3:2
С этими духами соединяется дух обмана.]
\vs Trb 3:3
Первый~--- дух блуда, содержится он в естестве и чувствах.
Второй~--- дух ненасытности желудка.
\vs Trb 3:4
Третий~--- дух борьбы, что в печени и в желчи.
Четвёртый~--- дух угождения и магии, дабы казаться прекрасным,
в чём нет никакой пользы.
\vs Trb 3:5
Пятый~--- дух гордыни, дабы похваляться и кичиться.
Шестой~--- дух лжи губительной и пристрастной,
дабы измышлять речи и скрывать дела даже от родичей и ближних.
\vs Trb 3:6
Седьмой~--- дух несправедливости, от коего воровство и грабежи
для услаждения сердца своего.
Ибо несправедливость содействует прочим духам,
когда отнимается нечто у других людей.
\vs Trb 3:7
[Со всеми же этими соседствует дух сна, восьмой дух,
он же~--- дух обмана и фантазии.]
\vs Trb 3:8
И так гибнет всякий юноша, затемняющий ум свой от истины,
не входящий в закон Бога, не слушающий наставлений отцов своих,
каков и я был в юности моей.
\vs Trb 3:9
Ныне, дети мои, возлюбите истину, и она убережет вас.
Я учу вас, внемлите словам Рувима, отца вашего.
\vs Trb 3:10
Не взирайте на женщин, не сходитесь с женщиной иного мужа,
не имейте дел ненужных с женщинами.
\vs Trb 3:11
Ибо, если бы не увидел я Баллу, когда купалась она в скрытом месте,
не впал бы я в беззаконие великое.
\vs Trb 3:12
Захватила меня мысль о наготе женской и не давала мне уснуть,
пока не совершил я мерзость.
\vs Trb 3:13
Когда Иаков, отец мой, ушёл к Исааку~--- а были мы в Гадере
близ Ефрафы в Вифлееме опьянела Балла и лежала непокрытая
в спальне.
\vs Trb 3:14
И я, вошедши и увидевши наготу её, сотворил нечестивое,
[а она не чувствовала,] и я отошёл, оставив её спящей.
\vs Trb 3:15
И тотчас ангел Божий открыл нечестивое дело моё отцу моему.
Придя, сетовал он на меня, более не прикасаясь к ней.

\vs Trb 4:1
Так не смотрите же, дети мои, на красоту женскую
и не помышляйте о делах женщин,
но живите в простоте сердечной, в страхе Господнем,
трудитесь, творя добрые дела, и отвлекаясь грамматикой,
и на пастбищах ваших дотоле, пока не даст вам Господь супругу,
какую он пожелает, дабы не претерпеть вам того же, что мне.
\vs Trb 4:2
Ибо вплоть до кончины отца моего не хватало смелости мне посмотреть
в глаза ему или говорить с кем-либо из братьев моих, из-за укоризны.
\vs Trb 4:3
И доныне мучит меня совесть из-за греха моего.
\vs Trb 4:4
И много утешал меня отец мой, и просил за меня Господа,
да отойдёт от меня гнев его, и так поступил со мной Господь.
С тех пор и поныне остерегался я и не грешил.
\vs Trb 4:5
Потому говорю вам, дети мои, сохраните всё,
что внушаю вам, и не грешите.
\vs Trb 4:6
Ибо грех блуда есть пропасть душевная, отделяющая от Бога
и приближающая к идолам, ведь он помрачает ум и помыслы
и сводит юношей в Ад прежде времени.
\vs Trb 4:7
Блуд сгубил многих, ибо стар ли кто, знатен ли, богат или беден,
одинаково порицание обретает он у сынов человеческих и даёт повод
Велиару создать преткновение ему.

\vs Trb 4:8
Слышали же вы об Иосифе, как остерегался он всякой женщины
и хранил помыслы в чистоте ото всякого блуда и обрёл
благодать у Господа и у человеков.
\vs Trb 4:9
А ведь многое творила ему Египтянка, и колдунов призывая
и снадобье ему поднося, но не впал помысел души его
в вожделение злое.
\vs Trb 4:10
За это избавил его Бог отцов наших от всякого зла видимой
и таящейся смерти.
\vs Trb 4:11
Если же не овладеет блуд помыслами вашими, не сможет одолеть вас и Велиар.

\vs Trb 5:1
Злы женщины, дети мои, и, не имея власти и силы над мужами,
коварно действуют своими чарами, дабы привлечь их к себе.
\vs Trb 5:2
Кого же такими чарами не могут приворожить, того обманом покоряют.
\vs Trb 5:3
Говорил же мне ангел Божий и учил меня, что уступают женщины тому духу
блуда больше, нежели мужи.
И замышляют они в сердце своём против мужей,
и украшениями соблазняют помыслы их,
и через очи подсыпают им яд, и так порабощают их.
\vs Trb 5:4
Ибо не может женщина прямо принудить мужа,
но совершает это злодейство своими чарами блудными.
\vs Trb 5:5
Итак, убегайте блуда, дети мои,
и приказывайте женам вашим и дочерям вашим,
да не украшают голов и лиц своих для обмана здравых помыслов мужчин.
Ибо всякая женщина, прибегающая к этим ухищрениям,
обречена на муку вечную.
\vs Trb 5:6
Ибо так обольстили они Стражей, бывших до
Катаклизма\fnote{Катаклизма}{Потопа или расстворения предыдущего мира.}.
Те постоянно смотрели на них, и возжелали их,
и замыслили дело: приняв человеческое обличье, сошлись с
женщинами в образе мужей их.
\vs Trb 5:7
А те, вообразив в вожделении своем, породили Гигантов,
ведь показались им Стражи достигающими небес.

\vs Trb 6:1
Так остерегайтесь же блуда.
И если желаете очистить разум, то сдерживайте чувства
свои от женщин.
\vs Trb 6:2
А женщинам внушайте не иметь общения с мужами,
дабы и они очищали \bibemph{свой} разум.
\vs Trb 6:3
Ибо постоянное общение,
если и не совершится нечестивое,
для них есть болезнь неисцелимая,
для нас же погибель Велиарова и позор вечный.
\vs Trb 6:4
Нет ни совести, ни благочестия в блуде,
и всякая ревность живет в вожделении его.
\vs Trb 6:5
Потому и говорю вам:
будете вы ревновать и стремиться превзойти сыновей Левия,
но не сможете.
\vs Trb 6:6
Потому что Бог отомстит за них, вы же умрёте смертью злою.
\vs Trb 6:7
Ибо Левию дал Бог власть
[и с ним Иуде, и мне, и Дану, и Иосифу, дабы мы были вождями].
\vs Trb 6:8
Посему завещаю вам слушать Левия, ибо он познает закон Господа
и установит суд и будет приносить жертвы за Израиля вплоть
до конца времен~--- первосвященник помазанный,
коего призвал Господь.
\vs Trb 6:9
Хочу, чтобы поклялись вы Богом небесным, что будете творить правду,
каждый ближнему своему, и иметь любовь, каждый к брату своему.
\vs Trb 6:10
А к Левию подойдите в смирении сердца вашего,
да примете благословение из уст его.
\vs Trb 6:11
Ибо он благословит Израиля и Иуду,
ибо в нём избрал Господь царствовать над всеми народами.
\vs Trb 6:12
И поклонитесь семени его, ибо за вас будет умирать оно в
войнах зримых и незримых.
И будет он над вaми царём вечным.

\vs Trb 7:1
И умер Рувим, завещав это сыновьям своим.
\vs Trb 7:2
И положили его во гроб, а после вынесли его из Египта
и погребли в Хевроне в пещере, где погребён был и отец его.

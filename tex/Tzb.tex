\bibbookdescr{Tzb}{
  inline={Завещание Завулона,\\шестого сына Иакова и Лии},
  toc={Завещание Завулона},
  bookmark={Завещание Завулона},
  header={Завещание Завулона},
  abbr={Зав}
}
\vs Tzb 1:1
Список слов Завулона, речённых им к сыновьям своим,
прежде чем умер он в 114-ый год жизни своей,
спустя 2 года после смерти Иосифа.
\vs Tzb 1:2
Сказал он им: слушайте меня, сыновья Завулона, внемлите речам отца вашего.

\vs Tzb 1:3
Я, Завулон, даром прекрасным родился у отца и матери моих.
Ибо когда родился я, премного возрос отец мой мелким и крупным
скотом, когда пеструю скотину получил он в удел.
\vs Tzb 1:4
Не знал я греха за собою во все дни мои, кроме только мысленного.
\vs Tzb 1:5
Не вспомню, чтобы совершил я несправедливость,
кроме греха неведения, сотворенного мною против Иосифа,
когда сговорился я с братьями моими не возвещать отцу моему о случившемся.
\vs Tzb 1:6
Но плакал я много дней из-за Иосифа, втайне, ибо страшился я братьев моих,
ибо положили они, что выдавший тайну будет убит.
\vs Tzb 1:7
Но когда желали убить Иосифа, молил я их со слезами не делать греха этого.

\vs Tzb 2:1
Ибо Симеон, Дан и Гад приступили к Иосифу, чтобы убить его,
и говорил он им со слезами, пав на лицо своё:
\vs Tzb 2:2
пощадите меня, братья мои; пожалейте сердце Иакова, отца нашего.
Не поднимайте рук ваших, дабы пролить кровь невинную,
ибо не согрешил я против вас.
\vs Tzb 2:3
Если же и согрешил, наказанием накажите меня,
но не поднимайте руки,
чтобы убить брата вашего ради отца нашего Иакова.
\vs Tzb 2:4
Когда же говорил он, скорбя, эти речи,
не вынес я стонов его и начал плакать,
и сотряслись внутренности мои, и всё во мне ослабло.
\vs Tzb 2:5
И заплакал я с Иосифом и забилось громко сердце моё,
и задрожали суставы тела моего, и не был я в силах стоять.
\vs Tzb 2:6
Когда же увидел Иосиф, что плачу вместе с ним,
а они подступили и хотят убить его, спрятался за спину мою,
умоляя помочь ему.
\vs Tzb 2:7
Тогда встал Рувим посредине и сказал:
братья мои, не будем убивать его,
но бросим его в один из сухих колодцев,
которые рыли отцы наши и не находили там воды.
\vs Tzb 2:8
Ибо для того не дал Господь подняться туда воде,
чтобы спасся Иосиф.
\vs Tzb 2:9
И сделали они так до той поры,
когда продали его Измаильтянам.

\vs Tzb 3:1
От платы за Иосифа не взял я своей части, дети мои.
\vs Tzb 3:2
Но Симеон, Дан, Гад и другие братья наши,
взяв плату за него, купили обувь себе и жёнам и детям своим,
говоря:
\vs Tzb 3:3
пропитания не купим, ибо это цена крови брата нашего,
но ногами потопчем её за то, что говорил он,
будто станет властвовать над нами;
и увидим, что будет из его снов.
\vs Tzb 3:4
Оттого записано в законе Моисеевом:
с нежелающего возместить семя брату своему
да снимут обувь его и плюнут в лицо ему.
\vs Tzb 3:5
А братья Иосифа не хотели, чтобы жил он,
и Господь снял с них обувь,
которую носили они за брата своего Иосифа.
\vs Tzb 3:6
И когда пришли они в Египет, за воротами сняли с них
обувь дети Иосифа,
и так преклонились они пред Иосифом, словно пред фараоном.
\vs Tzb 3:7
И не только преклонились пред ним,
но и оплёваны были в тот же час,
пав пред ним, и опозорены были пред Египтянами.
\vs Tzb 3:8
Ибо Египтяне слышали обо всех злых делах,
которые совершили они против Иосифа.

\vs Tzb 4:1
И сделав это, сели братья мои есть и пить.
\vs Tzb 4:2
Я же, терзаясь из-за Иосифа, не ел,
но смотрел на колодец, ибо опасался,
как бы Симеон, Дан и Гад не пошли и не убили Иосифа.
\vs Tzb 4:3
Увидев, что я не ем, они оставили меня стеречь его,
\vs Tzb 4:4
пока не продали Измаильтянам.

\vs Tzb 4:5
Затем пришел Рувим и, услышав,
что продали Иосифа в его отсутствие,
разодрал хитон свой и сказал плача:
как посмотрю я в лицо отцу моему Иакову?
\vs Tzb 4:6
И взяв серебро,
побежал вслед за купцами и, не найдя их,
вернулся опечаленный.
Купцы же, сойдя с широкой дороги,
пошли кратчайшим путём через землю Троглодитов.
\vs Tzb 4:7
И скорбел Рувим, и не ел хлеба в тот день.
И подойдя к нему, сказал Дан:
\vs Tzb 4:8
не плачь и не скорби; мы найдем, что сказать отцу нашему.
\vs Tzb 4:9
Зарежем козла, и вымараем хитон Иосифа,
и пошлём его Иакову, говоря:
узнай, сына ли твоего этот хитон?
Так они и сделали.
\vs Tzb 4:10
Ибо, продавая Иосифа, сняли с него хитон и одели на него плащ рабский.
\vs Tzb 4:11
Симеон же взял хитон и не хотел отдать его, ибо он желал убить Иосифа
и гневался, что не убили его.
\vs Tzb 4:12
И встав, сказали все мы ему:
если не отдашь хитон, скажем отцу нашему,
что ты один сотворил это зло в Израиле.
\vs Tzb 4:13
И отдал он им хитон. И сделали так, как сказал Дан.

\vs Tzb 5:1
Ныне, дети мои, завещаю вам хранить заповеди Господа,
и творить милость ближнему,
и добросердечными быть не только к людям,
но и к бессловесным животным.
\vs Tzb 5:2
За это и благословил меня Господь,
и когда занемогли все братья мои,
оставался я здоров; ибо знал Господь помыслы каждого.
\vs Tzb 5:3
Имейте же милость в сердцах ваших, ибо что сделает человек ближнему
своему, то сделает Господь с ним самим.
\vs Tzb 5:4
И болели, и умирали сыновья братьев моих из-за Иосифа,
ибо не имели милости в сердцах своих.
А мои сыновья сохранялись в здравии,
как вам то известно.
\vs Tzb 5:5
И когда были мы в земле Ханаанской,
ловил я рыб для отца моего Иакова,
и многие утонули в море,
а я невредим остался.

\vs Tzb 6:1
Первым я был, кто сделал лодку, чтобы плавать по морю,
ибо дал мне Господь для того знание и мудрость.
\vs Tzb 6:2
И приладил я весло деревянное сзади у неё,
а на другом прямом куске дерева натянул парус посредине.
\vs Tzb 6:3
И плавал я в лодке той по морским водам,
и ловил рыбу для дома отца моего,
пока не пошли мы в Египет.
\vs Tzb 6:4
[И из добычи моей всякому человеку чужому уделял я от доброты сердца.
\vs Tzb 6:5
Был ли кто чужестранец, или больной, или старый, готовил я рыбу,
делал её хорошо и давал всякому по надобности его,
соболезнуя и сострадая.
\vs Tzb 6:6
За это много рыб давал мне Господь, когда ловил я.
Ибо тот, кто делится с ближним,
получает многократно от Господа.]
\vs Tzb 6:7
5 лет ловил я рыбу [давая всякому человеку,
какого видел, и вдоволь отдавая дому отца моего].
\vs Tzb 6:8
Летом ловил я, а зимою пас стада вместе с братьями моими.

\vs Tzb 7:1
[Ныне возвещу вам, что сделал я.
Увидел я зимою страждущего от наготы,
и сжалился над ним, и, украв плащ из дома отца моего,
тайно дал страждущему.
\vs Tzb 7:2
Так и вы, дети мои, милостиво уделяйте из того, что даёт вам Господь,
всем без различия, и давайте всякому человеку в доброте сердечной.
\vs Tzb 7:3
Если же не имеете, что подать нуждающемуся,
сострадайте ему сердцем своим.
\vs Tzb 7:4
Помню, как не нашла рука моя, что подать нуждающемуся,
и прошёл я с ним семь стадиев и плакал с ним вместе,
и сердце моё сотрясалось от сострадания к нему.
\vs Tzb 8:1
Так и вы, дети мои, добросердечны будьте со всяким человеком в милости,
дабы и Господь, сжалившись, помиловал вас.
\vs Tzb 8:2
Ибо в последние дни пошлёт Бог сердце своё на землю,
и где найдет сердце милостивое, поселится в нём.
\vs Tzb 8:3
Ибо как человек жалеет ближнего своего, так сжалится и Господь над ним.]

\vs Tzb 8:4
Когда же пришли мы в Египет, не вспомнил нам зла Иосиф.
\vs Tzb 8:5
Воззрев на него, дети мои, возлюбите и вы друг друга, и не замышляйте
зла каждый на брата своего.
\vs Tzb 8:6
Ибо это разделяет единое, и всякое родство уничтожает,
и душу возмущает, и лицо искажает.

\vs Tzb 9:1
Посмотрите на воды и узрите, что когда текут они вместе,
то камни, деревья, землю и иное сметают они.
\vs Tzb 9:2
Если же на много частей разделятся, земля поглотит их,
и станут они ничтожными.
\vs Tzb 9:3
И вы, если разделитесь, будете таковыми.
\vs Tzb 9:4
Не разделяйтесь же на две головы, ибо всё, что сотворил Господь,
одну голову имеет, а два плеча, две руки, две ноги
и все другие члены слушаются одной головы.
\vs Tzb 9:5
Узнал же я из Писаний отцов наших,
что разделитесь вы в Израиле,
и за двумя царями последуете, и всякую мерзость сотворите.

\vs Tzb 9:6
И возьмут вас в плен враги ваши,
и зло будет вам от язычников среди многой скорби и бессилия.
\vs Tzb 9:7
После того, вспомнив о Господе, обратитесь вы, и помилует он вас,
ибо он милостив и добр сердцем,
и не мыслит зла против сынов человеческих,
ведь они~--- плоть и блуждают во злых делах своих.
\vs Tzb 9:8
И после того взойдёт для вас сам Господь,
свет справедливости, и возвратитесь вы в землю вашу,
и узрите его в Иерусалиме, избранном ради имени его святого.
\vs Tzb 9:9
И вновь злобою дел ваших прогневаете его,
и отвержены будете им вплоть до конца времен.

\vs Tzb 10:1
И ныне, дети мои, не скорбите, что умираю я,
и не унывайте, когда отойду я.
\vs Tzb 10:2
Ибо снова восстану я среди вас,
как предводитель среди сынов своих,
и возрадуюсь среди тех из рода моего,
кто сохранит закон Господень и заповеди Завулона, отца своего.
\vs Tzb 10:3
На нечестивых же наведёт Господь огонь вечный,
и погубит их до потомства потомков их.
\vs Tzb 10:4
Я же ныне отхожу к покою, как и отцы мои отошли.
\vs Tzb 10:5
Вы же бойтесь Господа Бога нашего всеми силами вашими во все дни жизни вашей.

\vs Tzb 10:6
И сказав это, почил он сном прекрасным,
и положили его сыновья во гроб деревянный.
\vs Tzb 10:7
После же отнесли его и погребли в Хевроне с отцами его.

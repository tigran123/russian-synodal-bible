\bibbookdescr{Tjb}{
  inline={Завещание Иова,\\непорочного, жертвы, завоевателя во многих соревнованиях},
  toc={Завещание Иова},
  bookmark={Завещание Иова},
  header={Завещание Иова},
  abbr={Зав~Иов}
}
\vs Tjb 1:1
Однажды он стал больным, и, зная, что он должен будет оставить свою телесную обитель, он призвал своих семерых сыновей (их имена: Терси, Хор, Гион, Никэ, Фор, Фиф, Фруон) и своих трех дочерей вместе и сказал им так:
\vs Tjb 1:2
Составьте круг около меня, дети, и послушайте, и я разскажу вам, что Господь сделал для меня, и всё, что происходило со мною.
\vs Tjb 1:3
Ибо я~--- Иов, ваш отец.
\vs Tjb 1:4
К тому же знайте, мои дети, что вы~--- род избранный, и примите во внимание ваше благородное рождение.
\vs Tjb 1:5
Ибо я~--- из сынов Исава. Мой брат Нерос, и ваша мать Дина. Чрез неё я стал отцом вашим.
\vs Tjb 1:6
Ибо моя первая жена умерла с моими другими десятью детьми горькой смертью.
\vs Tjb 1:7
Послушайте теперь, дети, и я открою вам, что происходило со мною.
\vs Tjb 1:8
Я был очень богатым человеком, живущим на Востоке в земле Уц, и прежде, чем Господь назвал меня Иовом, я был назван Иовав.
\vs Tjb 1:9
Начало моего испытания было таким. Возле моего дома был идол одного поклоняющегося [ему] народа; и я постоянно видел жертвоприношения ему как богу.
\vs Tjb 1:10
Тогда я подумал и сказал в себе: Тот ли это, который сотворил небо и землю, море и нас всех? Как я узнаю истину?
\vs Tjb 1:11
И в ту ночь, когда я лег спать, пришел глас и позвал: Иовав! Иовав! Встань, и я поведаю тебе, Кто~--- Тот, Кого ты пожелал узнать.
\vs Tjb 1:12
Тот, впрочем, кому люди приносят всесожжения и возлияния,~--- не Бог, но это~--- сила и действо Соблазнителя, которыми он обманывает людей.
\vs Tjb 1:13
И когда я слушал это, я пал на землю и я простерся, говоря:
\vs Tjb 1:14
O господин мой, который говорит для спасения души моей! Я прошу тебя, если это~--- идол Сатаны, я прошу тебя, позволь мне пойти отсюда и уничтожить его и очистить это место.
\vs Tjb 1:15
Ибо вот нет ни одного, кто может запретить мне сделать это, потому что я~--- царь этой земли; так чтобы те, что живут в ней, больше не вводились в заблуждение.
\vs Tjb 1:16
И глас, который говорил из пламени, ответил мне: Ты можешь очистить это место.
\vs Tjb 1:17
Но вот, я возвещаю тебе то, что Господь повелел мне, чтобы я сообщил тебе, ибо я~--- архангел Божий.
\vs Tjb 1:18
И я сказал: То, что будет сказано Его рабу, я буду слушать.
\vs Tjb 1:19
И архангел сказал мне: Так говорит Господь: если ты решишься уничтожить и убрать образ Сатаны, он примется с гневом вести войну против тебя, и он покажет на тебе всю свою злобу.
\vs Tjb 1:20
Он принесет тебе много жестоких бедствий и отнимет у тебя всё, что ты имел.
\vs Tjb 1:21
Он отнимет твоих детей и причинит много зла тебе.
\vs Tjb 1:22
Тогда ты должен бороться подобно борцу и сопротивляться боли, уверенный в своей награде, преодолевать испытания и бедствия.
\vs Tjb 1:23
Но когда ты претерпишь, Я сделаю твое имя известным среди всех поколений земли до кончины мира.
\vs Tjb 1:24
И Я возвращу тебе всё, что ты имел; и вдвойне от того, что ты потерял, дам тебе, чтобы ты мог знать, что Бог нелицеприятен, но дает каждому, кто заслужил, благо.
\vs Tjb 1:25
И тебе также будет дано оно, и ты наденешь диадему амарантовую,
\vs Tjb 1:26
и в воскресение ты пробудишься для жизни вечной. Тогда ты узнаешь, что Он~--- Господь праведный и истинный и Всемогущий.
\vs Tjb 1:27
После чего, дети мои, я ответил: Я буду из любви Божьей терпеть до смерти всё, что снизойдет на меня, и я не уклонюсь вспять.
\vs Tjb 1:28
Тогда ангел положил свою печать на мне и покинул меня.

\vs Tjb 2:1
После этого я встал ночью и взял пятьдесят рабов, и пошел к храму идола и разрушил его до основания.
\vs Tjb 2:2
И так я возвратился в мой дом и дал наказы, чтобы дверь его крепко затворили, говоря моим привратницам:
\vs Tjb 2:3
Если кто-нибудь будет спрашивать обо мне, не доносите никакого сообщения ко мне, но скажите ему: Он занят срочными делами, он~--- внутри.
\vs Tjb 2:4
Тогда Сатана притворился нищим и сильно стучал в двери, говоря привратнице:
\vs Tjb 2:5
Сообщите Иову и скажите, что я желаю встретиться с ним.
\vs Tjb 2:6
И привратница пошла и сказала мне это, но услышала от меня, что я занят.
\vs Tjb 2:7
Велиал, потерпев неудачу в этом, ушел и взял на своё плечо старую порванную корзину, и пришел и сказал привратнице, говоря: Скажите Иову: дай мне хлеба от рук твоих, чтобы я мог поесть.
\vs Tjb 2:8
И когда я услышал это, я дал ей подгорелого хлеба, чтобы дать его ему, и я известил его: Не жди есть моего хлеба, ибо это запрещено тебе.
\vs Tjb 2:9
Но привратница, устыдившись вручить ему подгорелый и сожженный хлеб (ибо она не знала, что это был Сатана), взяла своего хорошего хлеба и дала его ему.
\vs Tjb 2:10
Но он [не] взял его и, зная, что произошло, сказал деве: Иди прочь, плохая служанка, и принеси мне хлеб, который дали тебе, чтобы вручить мне.
\vs Tjb 2:11
И служанка воскликнула и сказала в печали: Ты говоришь истину, говоря, что я являюсь плохой служанкой, ибо я не сделала так, как была научена моим владыкой.
\vs Tjb 2:12
И она возвратилась и принесла ему горелого хлеба и сказала ему: Так говорит мой господин: тебе не есть от моего хлеба больше, ибо это запрещено тебе.
\vs Tjb 2:13
И это он дал мне, [говоря: это я дал] с тем, чтобы не могло быть навлечено против меня обвинение, что я не подал врагу, который просил.
\vs Tjb 2:14
И когда Сатана услышал это, он отослал служанку обратно ко мне, говоря: Как ты видишь этот хлеб весь сожженным, так буду я скоро жечь твоё тело, чтобы сделать его подобным тому.
\vs Tjb 2:15
И я ответил: Делай, что ты желаешь делать и исполни всё, что ты замыслил. Ибо я готов претерпеть всё, что бы ты ни навел на меня.
\vs Tjb 2:16
И когда диавол услышал это, он оставил меня, и, взойдя на [самое высокое] поднебесье, он взял от Господа клятву, что он сможет иметь власть над всем моим имением.
\vs Tjb 2:17
И после получения власти он пошел и тотчас взял всё мое богатство.

\vs Tjb 3:1
И я имел сто и тридцать тысяч овец, и из них я отделял семь тысяч на одежду сиротам и вдовам, и нуждающимся и больным.
\vs Tjb 3:2
Я имел загон из восьмисот псов, которые стерегли моих овец, и помимо них~--- двести, чтобы охраняли мой дом.
\vs Tjb 3:3
И я имел девять мельниц, работающих для всего города, и корабли, чтобы перевозить товары, и я доставлял их во всякий город и в селения немощному и больному и тем, которые были несчастны.
\vs Tjb 3:4
И я имел триста и сорок тысяч вьючных ослов, и из них я отбирал пятьсот, и потомство их я определял на продажу, а доходы отдавал нищим и нуждающимся.
\vs Tjb 3:5
Ибо со всех стран нищие приходили, чтобы встретиться со мною.
\vs Tjb 3:6
Ибо четыре двери моего дома были отверсты, каждая, будучи под ответственностью сторожа, который наблюдал, каждый ли из приходящих людей получал милостыню, и видел ли меня сидящим у одной двери, так чтобы они могли выходить через другую и получать всё, в чем они нуждались.
\vs Tjb 3:7
Также я имел тридцать столовых наборов, предназначенных на всякое время для одиноких странников, и я также имел двенадцать просторных столов для вдов.
\vs Tjb 3:8
И каждый, кто бы ни приходил, прося о милостыне, он находил пищу на моем столе, получая всё, в чем он нуждался, и я не позволял никому покинуть мою дверь с пустым животом.
\vs Tjb 3:9
Я также имел три тысячи пятьсот пар волов, и я отбирал из них пятьсот и отводил их пахарям.
\vs Tjb 3:10
И с ними я делал всю работу на всяком поле, у тех, кто хотел бы взять его на попечение, и доход их урожаев я откладывал для нищих на их столе.
\vs Tjb 3:11
Я также имел пятьдесят пекарен, от которых я посылал [хлеб] к столу для бедняков.
\vs Tjb 3:12
И я имел рабов, избранных для служения им.
\vs Tjb 3:13
Имелись также некоторые странники, которые видели мою добрую волю; они желали послужить как служители сами.
\vs Tjb 3:14
Другие, будучи в бедствии и неспособные приобрести средства к существованию, приходили с просьбой, говоря:
\vs Tjb 3:15
Мы просим тебя: поскольку мы тоже можем выполнять эту обязанность служителей и не имеем никакого стяжания, сжалься над нами и ссуди денег нам, с тем чтобы мы могли пойти в большие города и продавать товары.
\vs Tjb 3:16
И излишек от нашей прибыли мы можем отдавать в помощь нищим, и затем мы возвратим тебе твою собственность.
\vs Tjb 3:17
И когда я слышал это, я был доволен тем, что они возьмут это совместно со мною ради бережливости и милосердия к нищим.
\vs Tjb 3:18
И по желанию сердца я давал им что они хотели, и я принимал их расписки, но не брал никакого иного залога от них, кроме письменного свидетельства.
\vs Tjb 3:19
И они ходили повсюду и подавали вовремя бедным, насколько они преуспевали.
\vs Tjb 3:20
Часто, однако, некоторые из их товаров были теряемы на пути или на море, или их он отнимал у них.
\vs Tjb 3:21
Тогда они придут и скажут: Мы просим тебя: будь великодушен к нам, с тем чтобы мы могли подумать, как мы можем возвратить тебе твою собственность.
\vs Tjb 3:22
И когда я слышал это, я сочувствовал им и вручал им их расписку, и часто читавший её перед ними разрывал её, и прощал им из их долга, говоря им:
\vs Tjb 3:23
Что я посвятил для помощи бедным, я не буду брать от тебя.
\vs Tjb 3:24
И так я ничего не брал от моего должника.
\vs Tjb 3:25
И тогда муж с весёлым сердцем приходил ко мне, говоря: Мне не нужен трудовой заработок бедняка, необходимый ему;
\vs Tjb 3:26
но я желаю служить нуждающемуся за вашим столом. И он соглашался работать, и он ел свою долю.
\vs Tjb 3:27
Однако же я давал ему его плату, и я\fnote{я}{он(?)} уходил домой, радуясь.
\vs Tjb 3:28
А когда он не хотел брать это, я вынуждал его, чтобы делать так, говоря: Я знаю, что ты~--- муж труда, который надеется и ждет своей платы, и ты должен брать это.
\vs Tjb 3:29
Никогда я не отсрочиваю плату мзды наемнику или любому другому, ни задерживаю в моём доме в течение одного вечера из его найма, который был должен ему.
\vs Tjb 3:30
Те, которые доили коров и овец, сообщали проходящим, что они должны взять свою долю.
\vs Tjb 3:31
Ибо молоко текло в таком множестве, что оно свертывалось в масло на склонах и на краю дороги; и у камней и холмов скот ложился, чтобы родить своё потомство.
\vs Tjb 3:32
Ибо мои слуги утомлялись хранением мяса вдов и нищих и делили его [себе] на маленькие кусочки.
\vs Tjb 3:33
Ибо они, бывало, ругались и говорили: О, что мы имеем от его мяса, чем мы могли бы быть удовлетворены!, хотя я был очень любезен к ним.
\vs Tjb 3:34
Я также имел шесть арф [и шесть рабов, играющих на арфах], и также лиру десятиструнную, и я ударял по ним в течение дня.
\vs Tjb 3:35
И я брал лиру, и вдовы подпевали [мне] после их еды.
\vs Tjb 3:36
И с музыкальным орудием я напоминал им о Боге, что они должны воздавать хвалу Господу.
\vs Tjb 3:37
И когда мои рабыни, бывало, роптали, тогда я брал музыкальные орудия и играл столько, сколько они сделали бы за их плату, и давал им облегчение от их труда и воздыханий.

\vs Tjb 4:1
И мои дети, взяв ответственность за служение, брали своё ежедневное пропитание вместе с их тремя сестрами, начиная со старшего брата, и делали праздник.
\vs Tjb 4:2
И я вставал утром и предлагал, как искупительную жертву за них, пятьдесят овнов и девятнадцать овец, и что оставалось, как остаток было посвящаемо бедным.
\vs Tjb 4:3
И я говорил им: Берите это как остаток, и молитесь за моих детей.
\vs Tjb 4:4
Возможно, мои сыновья грешили перед Господом, говоря в надменности духа: Мы~--- дети этого богатого человека. Наше~--- всё это добро; почему мы должны быть слугами нищих?
\vs Tjb 4:5
И говоря так в надменности духа, они, возможно, вызывали гнев Бога, ибо гордое превозношение~--- мерзость пред Господом.
\vs Tjb 4:6
И так я приносил волов как жертву священнику на алтарь, говоря: Не хулили ли мои дети когда-либо Бога в своих сердцах?
\vs Tjb 4:7
Пока я жил таким образом, Соблазнитель не мог спокойно смотреть на добро [творимое мною], и он испросил у Бога войну против меня.
\vs Tjb 4:8
И он напал на меня безжалостно.
\vs Tjb 4:9
Сначала он сжег большое количество овец, потом верблюдов, затем он сжег волов и всё моё стадо; или же они были захвачены не только врагами, но также и теми, кто был облагодетельствован мною.
\vs Tjb 4:10
И пастухи пришли и возвестили это мне.
\vs Tjb 4:11
Но когда я услышал это, я воздал хвалу Богу и не богохульствовал.
\vs Tjb 4:12
И когда Соблазнитель познал моё терпение, он замыслил новое дело против меня.
\vs Tjb 4:13
Он вошел в царя Фираса и осадил мой город, и после того, как он увел всё, что было там, он сказал им в злобе, говоря хвастливой речью:
\vs Tjb 4:14
Этот муж Иов~--- тот, который получил все блага земли и не оставил ничего для других, он разрушил и низверг храм бога.
\vs Tjb 4:15
Поэтому я воздам ему тем же, что он сделал дому великого бога.
\vs Tjb 4:16
Ныне идите со мною, и мы будем грабить всё, что осталось в его доме.
\vs Tjb 4:17
И они ответили и сказали ему: Он имеет семь сыновей и трех дочерей.
\vs Tjb 4:18
Позаботься, чтобы они не убежали в другие страны, и не стали бы нашими мучителями, и тогда они превозмогут нас силою и убьют нас.
\vs Tjb 4:19
И он сказал: Ничего не бойтесь. Его стада и его богатство я уничтожил огнем, и остальное я расхитил, и вот, его детей я убью.
\vs Tjb 4:20
И говоря так, он пошел и обрушил дом на моих детей и убил их.
\vs Tjb 4:21
И мои сограждане, видя то, что сказанное им действительно совершилось, пришли и преследовали меня, и отняли у меня всё, что было в моем доме.
\vs Tjb 4:22
И я увидел моими глазами грабеж моего дома, и люди невоспитанные и безчестные сидели за моим столом и на моих ложах, и я не мог возражать против них.
\vs Tjb 4:23
Ибо я был истощен подобно женщине с её ложеснами, освободившимися от множества болей, помня главное~--- что эта война была предсказана мне Господом через Его ангела.
\vs Tjb 4:24
И я стал подобным тому, кто, видя бурное море и противные ветры, в то время как груз судна посреди океана слишком тяжел, сбрасывает тяжесть в море, говоря:
\vs Tjb 4:25
Я хочу уничтожить всё это для того, чтобы благополучно прибыть в город, так чтобы я мог взять как прибыль спасенное судно и лучшее из моих вещей.
\vs Tjb 4:26
Так я управлял моими делами.
\vs Tjb 4:27
Но вот пришел другой вестник и возвестил мне о гибели моих детей, и я был потрясен ужасом.
\vs Tjb 4:28
И я разодрал мою одежду и сказал: Господь дал, Господь и взял. Как это было угодно Господу, так это и сделалось. Да будет имя Господне благословенно.

\vs Tjb 5:1
И когда Сатана увидел, что он не смог возбудить во мне отчаяние, он пошел и выпросил моё тело у Господа, дабы причинить язву мне, ибо Велиал не мог вынести моего терпения.
\vs Tjb 5:2
Тогда Господь предал меня в его руки, чтобы использовать моё тело, как он хотел, но Он не дал ему власти над моей душой.
\vs Tjb 5:3
И он пришел ко мне, я же был сидящим на моём троне, всё еще печалясь по моим детям.
\vs Tjb 5:4
И он был подобен великому урагану и перевернул мой трон и бросил меня оземь.
\vs Tjb 5:5
И я долго лежал на полу в течение трех часов. И он поразил меня тяжкой проказой от темени головы моей до кончиков ног моих.
\vs Tjb 5:6
И я покинул город в великом ужасе и горе и сел на навозную кучу моим телом червоточивым.
\vs Tjb 5:7
И я орошал землю мокротою моего воспаленного тела, ибо гной стекал с моего тела, и множество червей покрывало его.
\vs Tjb 5:8
И когда один [какой-нибудь] червь сползал с моего тела, я клал его назад, говоря: Останься на том месте, где ты находился, пока Тот, Кто послал тебя, не направит тебя куда-нибудь еще.
\vs Tjb 5:9
Так я претерпевал в течение семи лет, сидя на навозной куче вне города, будучи поражен проказой.
\vs Tjb 5:10
И я увидел своими глазами моих томящихся детей
\vs Tjb 5:11
и мою унижающуюся жену, которая [некогда] была приведена в её свадебный чертог в такой великой роскоши и с копьеносцами как телохранителями. Я видел её выполняющею работу носильщика воды, подобно рабу, в доме простого человека, для того чтобы заработать немного хлеба и принести его мне.
\vs Tjb 5:12
И в моем лютом бедствии я сказал: О, что [значат] эти хвастливые правители города, которые будут теперь нанимать мою жену как служанку, которых душу я не подумаю сравнить [даже] с моими сторожевыми псами!
\vs Tjb 5:13
И после этого я обрел храбрость вновь.
\vs Tjb 5:14
Однако, впоследствии они отказывали [ей] даже в хлебе [для меня], чтобы она имела только её собственное пропитание.
\vs Tjb 5:15
Но она брала это и разделяла это между собою и мною, говоря скорбно: Горе мне! Отныне он больше не сможет питаться хлебом, и он не может пойти на торжище попросить хлеба у хлеботорговцев для того, чтобы принести его мне [и] чтобы он мог есть.
\vs Tjb 5:16
И когда Сатана узнал это, он принял облик хлеботорговца; и это было как будто случайным, что моя жена встретила его и спросила его о хлебе, думая, что это был его человеческий вид.
\vs Tjb 5:17
Но Сатана сказал ей: Дай мне цену, и потом бери, что ты пожелаешь.
\vs Tjb 5:18
Тогда она ответила, говоря: Где я возьму денег? Разве ты не знаешь, какая беда произошла со мною? Если ты имеешь жалость, яви её мне; если нет, ты смотри.
\vs Tjb 5:19
И он ответил, говоря: Если бы ты не заслуживала этой беды, ты бы не испытала всё это.
\vs Tjb 5:20
Ныне, если нет сребренника в руке твоей, дай мне волосы головы твоей и возьми три буханки хлеба за это, так что ты сможешь прожить на них три дня.
\vs Tjb 5:21
Тогда она сказала в себе: Что есть волосы головы моей по сравнению с моим голодающим мужем?
\vs Tjb 5:22
И так, подумав над вопросом, она сказала ему: Встань и отрежь мои волосы.
\vs Tjb 5:23
Тогда он взял ножницы и отнял волосы её головы в присутствии всех и дал ей три буханки хлеба.
\vs Tjb 5:24
Тогда она взяла их и принесла их мне. И Сатана последовал за ней по дороге, притаившись когда он шел и весьма безпокоя её сердце.

\vs Tjb 6:1
И тотчас же моя жена пришла ко мне и, вопия громко и плача, она сказала: Иов, Иов! Как долго ты сидишь на навозной куче вне города, размышляя уже на протяжении [столького] времени и ожидая получить твоё желанное спасение!
\vs Tjb 6:2
И я должна была блуждать с места на место, скитаясь повсюду как наемная служанка, [и слышать:] вот их память уже исчезла от земли.
\vs Tjb 6:3
И мои сыновья и дочери, которых я носила на моей груди, и труды и муки, которые я выдержала, были напрасны?
\vs Tjb 6:4
И ты сидишь в смраде болезни и червях, проводя ночи на холодном воздухе.
\vs Tjb 6:5
И я подвергалась всяким испытаниям и скорбям и мукам, днем и ночью, пока я не преуспевала в снабжении тебя хлебом.
\vs Tjb 6:6
Поскольку твоего излишка хлеба больше не позволили мне [брать]; и поскольку я едва могу брать мою собственную пищу и делить её между нами, я размышляла в моем сердце, что это несправедливо, что ты должен находиться в болезни и голоде из-за [отсутствия] хлеба.
\vs Tjb 6:7
И тогда я решилась идти на торжище без робости. И когда хлеботорговец сказал мне: Дай мне деньги и ты получишь хлеб, я открыла ему наше бедственное положение.
\vs Tjb 6:8
Тогда я услышала, как он сказал: Если ты не имеешь никаких денег, вручи мне волосы твоей головы, и возьми три буханки хлеба для того, чтобы ты могла жить на них три дня.
\vs Tjb 6:9
И я уступила несправедливости и сказала ему: Встань и отрежь мои волосы! И он встал, и публично отрезал ножницами волосы моей головы на рыночной площади, в то время как толпа стояла рядом и удивлялась.
\vs Tjb 6:10
Кто тогда не удивлялся, говоря: Это ли Сифь, жена Иова, которая имела четырнадцать занавесов, чтобы закрывать её сокровенный чертог, и двери за дверями, так что тот был весьма польщен, кто был приведен подле него; и ныне, вот, она обменивает свои волосы на хлеб!
\vs Tjb 6:11
\ldots кто имел верблюдов, нагруженных товарами, и они отводились в отдаленные страны к нищим; и ныне она продает свои волосы за хлеб!
\vs Tjb 6:12
Смотрите на неё, кто имела семь неподвижных столовых наборов в её доме, за которыми всякий бедный человек и всякий странник ел; и ныне она продает свои волосы за хлеб!
\vs Tjb 6:13
Смотрите на неё, кто имела купальню, чтобы омывать свои ноги, сделанную из золота и серебра; и ныне она ходит по земле [босая], и [продает свои волосы за хлеб!]
\vs Tjb 6:14
Смотрите на неё, кто имела одеяние, сделанное из виссона, вышитое золотом; и ныне она обменивает свои волосы на хлеб!
\vs Tjb 6:15
Смотрите на неё, кто имела ложа из золота и серебра; и ныне она продает свои волосы за хлеб!
\vs Tjb 6:16
Затем вкратце, Иов, после стольких вещей, которые были сказаны мне, я ныне скажу тебе одним словом:
\vs Tjb 6:17
Так как слабость моего сердца сокрушает мои кости, встань и возьми эти буханки хлеба и насладись ими, и потом прокляни Господа и умри!
\vs Tjb 6:18
Ибо я тоже заменила бы оковы смерти за хлеб насущный моему телу.
\vs Tjb 6:19
Но я ответил ей: Вот, я был в течение этих семи лет пораженным проказой, и я терпел червей в моем теле, и я не был отягощен в моей душе всеми этими мучениями.
\vs Tjb 6:20
И как [за] слово, которое ты говоришь: Прокляни Бога и умри, вместе с тобою я выдержу зло, которое ты видишь? И позволь нам перенести разорение всего, что мы имеем.
\vs Tjb 6:21
Всё же ты хочешь, чтобы мы прокляли Бога и чтобы Он был заменен на великого Плутона.
\vs Tjb 6:22
Почему ты не помнишь тех великих благ, которыми мы обладали? Если эти блага исходят из уделов Господних, не должны ли мы также претерпевать [от Него и] зло и быть премудрыми во всем, пока Господь не помилует [нас] снова и окажет жалость к нам?
\vs Tjb 6:23
Ты не видишь Соблазнителя, ставшего позади тебя и спутавшего твои мысли, чтобы ты обманывала меня.
\vs Tjb 6:24
И он обратился к Сатане и сказал: Почему же ты не приходишь ко мне явно, не перестанешь скрывать себя? Ты~--- жалкий
\vs Tjb 6:25
лев, показывающий свою силу в удобной клетке, или птица, летающая в корзине. Ныне я говорю тебе: выходи и веди твою войну против меня.
\vs Tjb 6:26
Тогда он вышел из-за спины моей жены и поставил себя предо мною, вопия; и он сказал: Вот, Иов, я сдаюсь и уступаю дорогу тебе, который искусен, но~--- плоть, тогда как я~--- дух.
\vs Tjb 6:27
Ты поражен проказой, но я в великой печали.
\vs Tjb 6:28
Ибо я подобен соревнующемуся с борцом борцу, который в бою одной рукою низверг своего соперника и скрыл его в прахе и сокрушил каждый член его, тогда как тот, который лежит внизу, являя свою храбрость, издает звуки торжества, свидетельствующие о его великом превосходстве.
\vs Tjb 6:29
Так и ты, о Иов, унижен и поражен проказой и мукою, и все же ты вынес победу в соревновании со мною, и вот, я уступаю тебе.
\vs Tjb 6:30
Тогда он покинул меня смущенный.
\vs Tjb 6:31
Ныне, мои дети, вы делайте также, являя твердость сердца во всяком зле, которое происходит с вами, ибо твердость сердца~--- больше всех дел.

\vs Tjb 7:1
В это время цари услышали о том, что произошло со мною, и они встали и пришли ко мне, каждый от его земли, чтобы посетить меня и утешить меня.
\vs Tjb 7:2
И когда они подошли ко мне, они возопили громким голосом, и каждый разодрал свою одежду.
\vs Tjb 7:3
И после того, как они поклонились, касаясь земли своими головами, они сидели рядом со мною семь дней и семь ночей, и ни один не сказал ни слова.
\vs Tjb 7:4
Их было числом четверо: Елифаз, царь Фемана, и Вилдад, и Софар, и Елиуй.
\vs Tjb 7:5
И когда они заняли своё место, они беседовали о том, что произошло со мною.
\vs Tjb 7:6
В то время, когда они в первый раз приходили ко мне и я показывал им мои драгоценные камни, они были удивлены и сказали:
\vs Tjb 7:7
Если бы от нас, трех царей, всё наше имущество было бы соединено в одно, оно не сравнилось бы с драгоценными камнями царства Иовава. Ибо твое превосходство больше, чем всех людей Востока.
\vs Tjb 7:8
И поэтому, когда они ныне пришли в землю Уц, чтобы посетить меня, они спросили в городе: Где~--- Иовав, правитель этой всей земли?
\vs Tjb 7:9
И они сказали им обо мне: Он сидит на навозной куче вне города, ибо он не входит в город в течение семи лет.
\vs Tjb 7:10
И тогда они снова спросили о моём имуществе, и вот было показано им всё, что произошло со мною.
\vs Tjb 7:11
И когда они узнали это, они вышли из города с жителями, и мой согражданин показал меня им.
\vs Tjb 7:12
Но они возражали и говорили: Конечно, это~--- не Иовав.
\vs Tjb 7:13
И пока они колебались, вот Елифаз, царь Фемана, говорит: Давайте, подойдем ближе и посмотрим.
\vs Tjb 7:14
И когда они подошли ближе, я вспомнил их, и я сильно плакал, когда я узнал о цели их путешествия.
\vs Tjb 7:15
И я посыпал прах на мою голову, и пока отряхивал свою голову, я открыл им, кто я был.
\vs Tjb 7:16
И когда они увидели меня, трясущего своей головою, они поверглись ниц до земли, все охваченные волнением.
\vs Tjb 7:17
И пока толпа стояла вокруг, я видел этих трех царей лежащими на земле в течение трех часов подобно мертвым.
\vs Tjb 7:18
Тогда они встали и сказали друг другу: Мы не можем поверить, что это~--- Иовав.
\vs Tjb 7:19
И, наконец, после того, как они на седьмой день узнали всё обо мне и искали [и не нашли] мои стада и другое имущество, они сказали:
\vs Tjb 7:20
Разве мы не знаем, сколько товаров посылал он городам и селениям, повсюду подавая нищим, кроме всего, что было отдано им внутри его собственного дома? Как же мог он впасть в таковое состояние погибели и горя!
\vs Tjb 7:21
И после семи дней Елиуй сказал царям: Давайте подойдем ближе и рассмотрим его тщательно, истинно ли он Иовав или нет?
\vs Tjb 7:22
И они, будучи на расстоянии стадии от его зловонного тела, встали и шагнули ближе, неся благовония в их руках, а их воины пошли с ними и бросали ароматные шарики ладана к ним так, чтобы они могли приблизиться ко мне.
\vs Tjb 7:23
И после того, как они так прошли три часа, покрывая путь ароматом, они почти достигли.
\vs Tjb 7:24
И Елифаз начал и сказал: Ты ли, воистину, Иов, соцарствующий нам? Ты ли тот, кто имел великую славу?
\vs Tjb 7:25
Ты ли тот, кто когда-то сиял подобно дневному солнцу на всю землю? Ты ли тот, кто когда-то походил на луну и звезды, сияющие всю ночь?
\vs Tjb 7:26
И я ответил ему и сказал: Это я. И затем все плакали и стенали, и они воспели царскую плачевную песнь, [и] всё их войско соединилось с ними в хоре.
\vs Tjb 7:27
И опять Елифаз сказал мне: Ты ли тот, кто приказал раздать семь тысяч овец для одежды нищим? Поблекла, значит, преходящая слава твоего престола!
\vs Tjb 7:28
Ты ли тот, кто повелел трем тысячам волов пахать поле для бедных? Поблекла, значит, твоя преходящая слава!
\vs Tjb 7:29
Ты ли тот, кто имел золотые ложа, и ныне ты сидишь на навозной куче? [Поблекла, значит, твоя преходящая слава!]
\vs Tjb 7:30
Ты ли тот, кто имел шестьдесят столовых набора для нищих? Ты ли тот, кто имел кадило для прекрасных благовоний, отделанное драгоценными камнями, и ныне ты в зловонии? Поблекла, значит, твоя преходящая слава!
\vs Tjb 7:31
Ты ли тот, кто имел золотой набор подсвечников на серебряных подставках; и ныне должен ты тосковать из-за естественного отражения луны? [Поблекла, значит, твоя преходящая слава!]
\vs Tjb 7:32
Ты ли тот самый, кто делал притирание из смеси ладана, и ныне ты в мерзости! [Поблекла, значит, твоя преходящая слава!]
\vs Tjb 7:33
Ты ли тот, кто высмеивал неправедных делателей и презирал грешников, и ныне ты стал посмешищем у всех! [Поблекла, значит, твоя преходящая слава!]
\vs Tjb 7:34
И пока Eлифаз много времени вопиял и стенал, а все остальные соединились с ним, так что смятение было весьма великим, я сказал им:
\vs Tjb 7:35
Умолкните и я покажу вам мой престол и славу его великолепия: моя слава будет вечной.
\vs Tjb 7:36
Весь мир погибнет и его слава исчезнет, и все те, кто прилепляются к нему, будут в преисподней, но мой престол пребывает в вышнем мире, и его слава и великолепие будут одесную от Искупителя [моего] на небесах.
\vs Tjb 7:37
Мой престол существует в обществе святых и их славы в нетленном мире.
\vs Tjb 7:38
Ибо реки высохнут, и их надменность будет унижена до глубины бездны, но потоки моей земли, в которой мой престол воздвигнут, не высохнут, но останутся нерушимыми в силе.
\vs Tjb 7:39
Цари погибают и князья исчезают, и их слава и гордость~--- как отражение в зеркале; но моё царство продлится всегда и вечно, и его слава и красота пребывают в колеснице моего Отца.

\vs Tjb 8:1
Когда я говорил им так, Елифаз разгневался и сказал другим друзьям: Для этой ли цели мы пришли сюда с нашим войском утешать его? Вот, он поносит нас. Поэтому давайте мы возвратимся в наши страны.
\vs Tjb 8:2
Этот человек сидит здесь в червоточивом страдании среди невыносимого гниения, и все же он испытывает своё спасение: Погибнут царства и их правители, но моё царство, говорит он, продлится вовек.
\vs Tjb 8:3
Eлифаз затем встал в большом смятении, и, отвернувшись от них в великой ярости, сказал: Я ухожу отсюда. Воистину мы пришли утешить его, но он объявляет войну нам ввиду наших войск.
\vs Tjb 8:4
Но тогда Вилдад схватил его за руку и сказал: Не так должно говорить со страдающим человеком, и особенно с пораженным таковыми многими бедствиями.
\vs Tjb 8:5
Вот, мы, будучи в добром здравии, не осмеливались приблизиться к нему по причине зловония, кроме как с помощью множества ароматных благовоний. Но ты, Елифаз, забываешь обо всем этом.
\vs Tjb 8:6
Скажи мне прямо: позволишь ли нам быть великодушными и узнать, какова причина [его бедствия]? Не мог же он при воспоминании его прежних дней счастья стать безумным в его разуме?
\vs Tjb 8:7
Кто не был бы в совершенном недоумении, видя себя впадшим в таковое несчастье и беду? Но позвольте мне приступить к нему, чтобы я смог выяснить, в чем суть его дела.
\vs Tjb 8:8
И Вилдад встал и приблизился ко мне, говоря: Ты ли Иов? И [еще] он сказал: Находится ли твоё сердце в добром расположении?
\vs Tjb 8:9
И я сказал: Я не прилепляюсь к земным делам, тогда как земля со всем, что обитает на ней~--- непостоянна. Но моё сердце прилепляется к небу, ибо там, в небесах, нет печали.
\vs Tjb 8:10
Тогда Вилдад возразил и сказал: Мы знаем, что земля непостоянна, ибо она изменяется по сезонам. По временам она в состоянии мира, и по временам она в состоянии войны. Но о небе мы слышим, что оно совершенно неизменно.
\vs Tjb 8:11
Но истинно ли ты в покое? Поэтому позволь мне спрашивать и говорить; и когда ты ответишь мне на моё первое слово, я задам второй вопрос, и если вновь ты ответишь словами хорошо подобранными, станет очевидно, что твоё сердце не пребывает неуравновешенным.
\vs Tjb 8:12
И я сказал\fnote{я сказал}{он сказал(?)}: На что ты полагаешь твою надежду? И я ответил: На Бога живого.
\vs Tjb 8:13
И он сказал мне: Кто лишил тебя всего, чем ты обладал? И кто причинил тебе эти несчастья? И я сказал: Бог.
\vs Tjb 8:14
И он сказал: Если ты всё еще полагаешь свою надежду на Бога, то как Он может творить неправедный суд, нанося тебе эти несчастья и беды, и отняв у тебя все твои владения?
\vs Tjb 8:15
И так как Он забирает их, то ясно, что Он не дает тебе ничего. Царь станет ли безчестить своего воина, который хорошо служит ему как телохранитель?
\vs Tjb 8:16
[И я ответил на притчу]: Кто уразумеет глубины Господни и Его мудрость, чтобы быть способным обвинить Бога в несправедливости?
\vs Tjb 8:17
[И Вилдад сказал]: Ответь мне, о Иов, на это. Снова я скажу тебе: если ты~--- в здравом разсудке, вразуми меня, если ты имеешь мудрость:
\vs Tjb 8:18
почему мы видим восход солнца на Востоке, а закат на Западе? И опять, когда встаем утром, [почему] мы находим его восходящим на Востоке? Сообщи мне, что ты думаешь об этом?
\vs Tjb 8:19
Тогда сказал я: Зачем я буду выдавать величайшие тайны Божии и мои уста должны преткнуться при раскрытии дел, принадлежащих Владыке? Никогда!
\vs Tjb 8:20
Кто мы, что мы будем вникать в дела вышнего мира, тогда как мы~--- только из плоти; нет~--- земля и прах!
\vs Tjb 8:21
Чтобы ты знал, что моё сердце здраво, послушай, что я спрошу тебя:
\vs Tjb 8:22
Через утробу проходит пища, и воду ты пьешь через уста, и затем это течет через одно и то же горло, и когда оба опускаются, становясь выделением, они снова разделяются; кто производит это разделение?
\vs Tjb 8:23
И Вилдад сказал: Я не знаю. И я возразил и сказал ему: Если ты не понимаешь даже испражнений тела, как можешь ты понимать небесные круговращения?
\vs Tjb 8:24
Тогда Софар возразил и сказал: Мы не спрашиваем о своих делах, но мы желаем знать,~--- в здравом ли ты состоянии, и вот, мы видим, что твой разсудок не поколеблен.
\vs Tjb 8:25
Что ныне ты хочешь, чтобы мы сделали для тебя? Вот, мы пришли сюда и привели лекарей трех царей, и если ты желаешь, ты можешь вылечиться у них.
\vs Tjb 8:26
Но я ответил и сказал: Моё лекарство и моё возстановление происходят от Бога, Творца врачей.

\vs Tjb 9:1
И когда я говорил им так, вот, туда прибежала моя жена Сифь, одетая в лохмотья, от служения тому хозяину, которым она была нанята как рабыня, хотя ей запретили покидать [его], чтобы цари, видя её, не могли взять её как пленницу.
\vs Tjb 9:2
И когда она пришла, она простерлась обезсиленная у их ног, рыдая и говоря: Вспомнили, Елифаз и вы, остальные друзья, сколь я одинока с вами, и как я изменилась, как ныне я одета, чтобы встретить вас!
\vs Tjb 9:3
Тогда цари упали навзничь в великом плаче и, будучи в крайнем недоумении, они хранили молчание. Но Елифаз взял свою пурпурную мантию и бросил её ей, чтобы оделась в это.
\vs Tjb 9:4
Но она попросила его, говоря: Я прошу как милость у вас, моих господ, чтобы вы приказали вашим воинам копать среди развалин нашего дома, который упал на моих детей, так чтобы их кости могли быть принесены целыми к могилам.
\vs Tjb 9:5
Это первое, в чем мы имеем нужду в нашем несчастье, обезсилев совсем, и таким образом мы сможем по крайней мере увидеть их кости.
\vs Tjb 9:6
Ибо я имела безотчетное материнское чувство как у диких зверей, что мои десять детей должны погибнуть в один день; и ни одному из них я не могла бы дать достойное погребение?
\vs Tjb 9:7
И цари дали повеление, чтобы руины моего дома были выкопаны. Но я запретил это, предостерегая:
\vs Tjb 9:8
Не ходите в напрасной скорби; ибо мои дети не будут найдены, потому что они соблюдаются их Творцом и Владыкой.
\vs Tjb 9:9
И цари ответили и сказали: Кто станет сему противоречить? Он [вышел] из своего ума и бредит.
\vs Tjb 9:10
Ибо в то время как мы хотим принести кости его детей обратно, он запрещает нам делать [это], говоря так: Они были взяты и помещены на хранение их Творцом. Поэтому докажи нам [твою] истину.
\vs Tjb 9:11
Но я сказал им: Поднимите меня, чтобы я мог встать; и они подняли меня, поддерживая мои руки с обоих сторон.
\vs Tjb 9:12
И я стал прямо и объявил сначала похвалу Богу, а после молитвы я сказал им: Посмотрите глазами вашими на Восток.
\vs Tjb 9:13
И они посмотрели и увидели моих детей с венцами подле Царя славы, Владыки небес.
\vs Tjb 9:14
И когда моя жена Сифь увидела это, она пала на землю и простерла[сь] пред Богом, говоря: Теперь я знаю, что моя память останется у Господа.
\vs Tjb 9:15
И после того, как она сказала [это] и настал вечер, она пошла в город, обратно к хозяину, которому она служила как рабыня, и легла в воловьих яслях и умерла там от истощения.
\vs Tjb 9:16
И когда её жестокий хозяин искал её и не находил её, он пришел к загону своего стада, и там он увидел её простершейся в яслях мертвой, в то время как все животные вокруг плакали о ней.
\vs Tjb 9:17
И все, кто видели её, плакали и стенали, и вопль распространился повсюду во всем городе.
\vs Tjb 9:18
И люди унесли её и обернули её и похоронили её в доме, который упал на её детей.
\vs Tjb 9:19
И городские нищие сотворили великий плачь по ней и говорили: Вот, это Сифь, подобной кому в благородстве и в славе не найдется среди жен. Увы,~--- она не обрела достойной могилы!
\vs Tjb 9:20
Погребальную песнь по ней вы найдете в записи.

\vs Tjb 10:1
Но Eлифаз и те, что были с ним, были удивлены этим вещам, и они сидели со мною и отвечали мне, говоря в хвастливых словах обо мне в течение двадцати семи дней.
\vs Tjb 10:2
Они повторяли это снова и снова: что я страдал так по заслугам, потому что совершил много грехов, и что, вот, надежды не осталось мне; но я опровергал этих мужей в пылу спора.
\vs Tjb 10:3
И они встали в раздражении, готовые разстаться в гневном духе. Но Елиуй заклинал их остаться еще немного до тех пор, пока он не покажет им, как это было.
\vs Tjb 10:4
Ибо,~--- сказал он,~--- так много дней вы проводите, позволяя Иову хвалиться, что он праведен. Но я больше не буду терпеть этого.
\vs Tjb 10:5
Ибо изначально я продолжаю плакать по нему, помня его прежнее счастье. Но теперь он говорит хвастливо, и в заносчивой гордыне он говорит, что он имеет свой престол на небесах.
\vs Tjb 10:6
Поэтому, послушайте меня, и я поведаю вам, какова причина [такой] его судьбы.
\vs Tjb 10:7
Тогда вдохновляемый духом Сатаны Елиуй сказал жестокие слова, которые записаны в записях, оставленных Елиуем.
\vs Tjb 10:8
И когда он закончил, Бог явился мне в буре и мраке, и говорил, осуждая Елиуя и показывая мне, что тот, кто говорил, был не человек, а бешеное животное.
\vs Tjb 10:9
И когда Бог закончил говорить со мною, Господь сказал Eлифазу: Ты и твои друзья согрешили в том, что вы не говорили истины о Моем рабе Иове.
\vs Tjb 10:10
Поэтому поднимитесь и побудите его принести искупительную жертву за вас, чтобы ваши грехи могли быть прощены; ибо если бы не он, Я уничтожил бы вас.
\vs Tjb 10:11
И так они принесли мне всё, что необходимо для жертвы, и я взял это и принес за них искупительную жертву, и Господь принял её благосклонно и простил им их неправду.
\vs Tjb 10:12
После того, как Елифаз, Вилдад и Софар увидели, что Бог милостиво простил их грех через Его раба Иова, но что Он не соизволил простить Елиуя, тогда Eлифаз начал петь гимн, в то время как остальные подпевали, [и] их воины также присоединились при водруженном алтаре.
\vs Tjb 10:13
И Eлифаз говорил так: Отпущен грех и наша несправедливость омыта;
\vs Tjb 10:14
но Елиуй, оный Велиал, не будет иметь памяти среди живущих; его светило гаснет и теряет свой свет.
\vs Tjb 10:15
Слава его светильника явится ему, ибо он~--- сын тьмы, а не света.
\vs Tjb 10:16
Привратники обиталища тьмы дадут ему их славу и красоту в удел; его царство исчезло, его престол разсыпался, и честь его стати~--- в Шеоле.
\vs Tjb 10:17
Ибо он возлюбил лесть змеи и кожу дракона, его желчь и его яд~--- аспида.
\vs Tjb 10:18
Ибо он не стремился к Господу, ни боялся он Его, но он ненавидел тех, кого Он избрал.
\vs Tjb 10:19
Оттого Бог забыл его, и святые оставили его, его гнев и раздражение будут ему запустением, и он не обретет ни милосердия в его сердце, ни мира, ибо он имел яд аспида на его языке.
\vs Tjb 10:20
Праведен Господь, и Его суды~--- истинны. У Него нет лицеприятия, ибо Он судит всех одинаково.
\vs Tjb 10:21
Вот, Господь грядет! Вот, святые приготовились: венцы и награды победителей предшествуют им!
\vs Tjb 10:22
Да возрадуются святые, и да возликуют сердца их в веселии; ибо они получат славу, которая соблюдается для них.
\vs Tjb 10:23
Хор: Наши грехи прощены, наша несправедливость очищена, но Елиую нет памяти среди живущих.
\vs Tjb 10:24
После того, как Елифаз окончил гимн, мы встали и возвратились в город, каждый к дому, где он жил.
\vs Tjb 10:25
И народ сотворил пир для меня в благодарность и восхищение Богу, и все мои друзья возвратились ко мне.
\vs Tjb 10:26
И все те, кто видели меня в моем прежнем счастье, спросили меня, говоря: Что это за три вещи здесь среди нас? \ldots

\vs Tjb 11:1
Но я, желая взяться снова за мой труд благотворительности для бедных, просил их, говоря:
\vs Tjb 11:2
Дайте мне каждый агнца для одежды нищим в их наготе, и четыре драхмы серебра или золота.
\vs Tjb 11:3
Тогда Господь благословил всё, что было отложено мне, и после немногих дней я снова стал богат имением, в стадах и всех делах, которые я потерял, и я вновь получил всё вдвойне.
\vs Tjb 11:4
Потом я также взял в жену вашу мать и стал отцом вам десятерым вместо десяти детей, которые умерли.
\vs Tjb 11:5
И ныне, дети мои, позвольте мне предупредить вас: Вот, я умираю [и] вы получите моё жилище,
\vs Tjb 11:6
только не оставляйте Господа. Будьте милосердны к нищим; не презирайте немощных; не берите себе жен из иноплеменников.
\vs Tjb 11:7
Вот, дети мои, я разделю среди вас, чем я обладаю, так чтобы каждый мог иметь власть над его собственностью и полную силу делать благое с его долей.
\vs Tjb 11:8
И после того, как он сказал так, он принес всё своё стяжание и разделил его между его семью сыновьями, но он не дал ничего из своего имущества его дочерям.
\vs Tjb 11:9
Тогда они сказали своему отцу: Наш господин и отец! Не являемся ли мы тоже твоими детьми? Почему тогда ты не даешь также и нам часть твоего имения?
\vs Tjb 11:10
Тогда сказал Иов своим дочерям: Не гневайтесь, мои дочери. Я не забыл вас. Вот, я приготовил для вас имение, лучшее чем то, которое взяли ваши братья.
\vs Tjb 11:11
И он призвал свою дочь, имя которой Йемима, и сказал ей: Возьми это витое кольцо, используемое как ключ, и иди к сокровищнице, и принеси мне золотой ковчежец, чтобы я дал вам ваше имение.
\vs Tjb 11:12
И она пошла и принесла его ему, и он открыл его и взял трехслойные опоясания, вид которых человек не может изъяснить.
\vs Tjb 11:13
Ибо они были не земной работы, но небесные искры света сверкали через них подобно лучам солнца.
\vs Tjb 11:14
И он дал по одному поясу каждой из его дочерей и сказал: Оденьте их как опоясания ваши, чтобы все дни жизни вашей они могли окутывать вас и наделять каждую из вас добродетелью.
\vs Tjb 11:15
И другая дочь, имя которой было Касия, сказала: Это достояние, о котором ты говоришь, разве оно лучше, чем таковое же у наших братьев? Что теперь? Сможем ли мы жить на это?
\vs Tjb 11:16
И их отец сказал им: Не только здесь вам хватит на жизнь, но они приведут вас в лучший мир жительства~--- на небеса.
\vs Tjb 11:17
Или вы не знаете, мои дети, значение этих вещей здесь? Тогда послушайте! Когда Господь посчитал меня достойным иметь сострадание ко мне и удалить от моего тела проказу и червей, Он призвал меня и вручил мне эти три пояса.
\vs Tjb 11:18
И Он сказал мне: Встань и препояшь чресла твои, как подобает мужу: Я взыщу тебя и провозглашу тебя Моим.
\vs Tjb 11:19
И я взял их и обвязал их вокруг моих чресл, и тотчас черви оставили моё тело, также и проказа, и всё мое тело обрело новую силу от Господа. И так я пошел, как если бы я никогда не страдал,
\vs Tjb 11:20
но также и в моем сердце я забыл муки. Тогда говорил Господь со мною в Его великом могуществе и показал мне всё, что было и будет.
\vs Tjb 11:21
Теперь, когда вы, дети мои, под охраной их,~--- не будете иметь замышляющего против вас врага, ни [злых] намерений в вашем разуме, потому что этот филактерий от Господа.
\vs Tjb 11:22
Так встаньте же и опояшьте их вокруг себя, прежде чем я умру, дабы вы могли увидеть ангелов, грядущих на моё разлучение, так чтобы вы могли видеть с удивлением силы Божии.
\vs Tjb 11:23
Тогда встала та, чье имя было Йемима, и опоясалась; и тотчас она отлучилась от своего тела, как сказал её отец, и она обрела иное сердце, как будто она никогда не заботилась о земных делах.
\vs Tjb 11:24
И она пела ангельские песни голосом ангелов, и она воспевала ангельскую похвалу Богу в танце.
\vs Tjb 11:25
Тогда другая дочь, по имени Касия, надела пояс, и её сердце преобразилось так, что она больше не желала мирских дел.
\vs Tjb 11:26
И её уста претворились в речь небесных Начал, и она пела благодарственные славословия творения вышней обители, и если кто-либо желал познать творение небес, он мог найти постижение в гимнах Касии.
\vs Tjb 11:27
Тогда другая дочь, по имени Керенгаппух, опоясалась и её уста заговорили на языке том высоком; ибо её сердце преобразилось, восхищаясь превыше мирских дел.
\vs Tjb 11:28
Она говорила речами Керубов, воспевающих похвалу Владыке вселенских сил и превознося их славу.
\vs Tjb 11:29
И тот, кто желает следовать остаткам славы отца, найдет их записанными в молитвах Керенгаппух.

\vs Tjb 12:1
После того, как сии три окончили петь гимны, я, Нерос, брат Иова, сел рядом с ним, когда он возлег.
\vs Tjb 12:2
И я слышал изумительные дела о трех дочерях моего брата: одно всегда сопутствовало другому среди благоговейного безмолвия.
\vs Tjb 12:3
И я написал эту книгу, содержащую гимны, помимо гимнов и знамений [святого] слова, ибо они были великими делами Бога.
\vs Tjb 12:4
И Иов возлег от болезни на его ложе, однако без боли и страдания, потому что его боль не одолела силу разума его по причине чудесного действия пояса, который он обернул вокруг себя.
\vs Tjb 12:5
Но по прошествии трех дней Иов увидел, что святые ангелы пришли за его душой, и тотчас он встал и взял лиру и дал её своей дочери Йемиме.
\vs Tjb 12:6
И Касии он дал касию, а Керенгаппух он дал тимпан, чтобы они могли благословлять святых ангелов, которые пришли за его душою.
\vs Tjb 12:7
И они взяли их, и пели, и играли на арфе и восхваляли и прославляли Бога в святой речи.
\vs Tjb 12:8
И после этого Он пришел~--- Тот, Который возседает на большой колеснице, и облобызал Иова, в то время как его три дочери смотрели, но другие не видели этого.
\vs Tjb 12:9
И Он взял душу Иова, и Он вознесся ввысь, взяв её рукою и перенеся её на колесницу, и Он пошел на Восток.
\vs Tjb 12:10
Его тело, однако, было принесено к могиле, в то время как три дочери шествовали впереди, надев свои пояса и воспевая гимны в похвалу Богу.
\vs Tjb 12:11
Тогда провели Нерос, его брат, и его семь сыновей с остальным народом и нищими, вдовами и немощными, великую скорбь по нему, говоря:
\vs Tjb 12:12
Горе нам, ибо сегодня была взята от нас сила немощных, свет слепых, отец сирот,
\vs Tjb 12:13
приют странников; уведен руководитель заблудших, покров нагих, защита вдов. Кто не будет сетовать по мужу Божию!
\vs Tjb 12:14
И поскольку они скорбели таким и таковым образом, они не хотели перенести его, чтобы положить в могилу.
\vs Tjb 12:15
После трех дней, однако, он был, наконец, положен в могилу, как бы в приятном сне, и он наследовал доброе имя, которое станет прославляться повсюду всеми поколениями мира.
\vs Tjb 12:16
Он оставил семь сыновей и трех дочерей, и вот не нашлось дочерей на земле, столь же прекрасных как дочери Иова.
\vs Tjb 12:17
Имя Иова было прежде Иовав, и он был назван Иовом у Господа.
\vs Tjb 12:18
Он жил прежде его бедствия восемьдесят пять лет, а после бедствия он взял двойную долю всего; следовательно его годы также удвоились, то есть~--- сто семьдесят лет. Таким образом он жил всего двести пятьдесят пять лет.
\vs Tjb 12:19
И он увидел сыновей его сыновей до четвертого поколения.
\vs Tjb 12:20
Так написано: он опять возстанет с теми, кого Господь пробудит.
\vs Tjb 12:21
Нашему Господу слава. Аминь.

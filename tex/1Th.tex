\bibbookdescr{1Th}{
  inline={Первое Послание\\к Фессалоникийцам\\\LARGE Святого Апостола Павла},
  toc={1-е Фессалоникийцам},
  bookmark={1-е Фессалоникийцам},
  header={1-е Фессалоникийцам},
  %headerleft={},
  %headerright={},
  abbr={1~Фес}
}
\vs 1Th 1:1 Павел и Силуан и Тимофей~--- церкви Фессалоникской в Боге Отце и Господе Иисусе Христе: благодать вам и мир от Бога Отца нашего и Господа Иисуса Христа.
\rsbpar\vs 1Th 1:2 Всегда благодарим Бога за всех вас, вспоминая о вас в молитвах наших,
\vs 1Th 1:3 непрестанно памятуя ваше дело веры и труд любви и терпение упования на Господа нашего Иисуса Христа пред Богом и Отцем нашим,
\vs 1Th 1:4 зная избрание ваше, возлюбленные Богом братия;
\vs 1Th 1:5 потому что наше благовествование у вас было не в слове только, но и в силе и во Святом Духе, и со многим удостоверением, как вы \bibemph{сами} знаете, каковы были мы для вас между вами.
\vs 1Th 1:6 И вы сделались подражателями нам и Господу, приняв слово при многих скорбях с радостью Духа Святаго,
\vs 1Th 1:7 так что вы стали образцом для всех верующих в Македонии и Ахаии.
\vs 1Th 1:8 Ибо от вас пронеслось слово Господне не только в Македонии и Ахаии, но и во всяком месте прошла \bibemph{слава} о вере вашей в Бога, так что нам ни о чем не нужно рассказывать.
\vs 1Th 1:9 Ибо сами они сказывают о нас, какой вход имели мы к вам, и как вы обратились к Богу от идолов, \bibemph{чтобы} служить Богу живому и истинному
\vs 1Th 1:10 и ожидать с небес Сына Его, Которого Он воскресил из мертвых, Иисуса, избавляющего нас от грядущего гнева.
\vs 1Th 2:1 Вы сами знаете, братия, о нашем входе к вам, что он был не бездейственный;
\vs 1Th 2:2 но, прежде пострадав и быв поруганы в Филиппах, как вы знаете, мы дерзнули в Боге нашем проповедать вам благовестие Божие с великим подвигом.
\vs 1Th 2:3 Ибо в учении нашем нет ни заблуждения, ни нечистых \bibemph{побуждений}, ни лукавства;
\vs 1Th 2:4 но, как Бог удостоил нас того, чтобы вверить \bibemph{нам} благовестие, так мы и говорим, угождая не человекам, но Богу, испытующему сердца наши.
\vs 1Th 2:5 Ибо никогда не было у нас перед вами ни слов ласкательства, как вы знаете, ни видов корысти: Бог свидетель!
\vs 1Th 2:6 Не ищем славы человеческой ни от вас, ни от других:
\vs 1Th 2:7 мы могли явиться с важностью, как Апостолы Христовы, но были тихи среди вас, подобно как кормилица нежно обходится с детьми своими.
\vs 1Th 2:8 Так мы, из усердия к вам, восхотели передать вам не только благовестие Божие, но и души наши, потому что вы стали нам любезны.
\vs 1Th 2:9 Ибо вы помните, братия, труд наш и изнурение: ночью и днем работая, чтобы не отяготить кого из вас, мы проповедовали у вас благовестие Божие.
\vs 1Th 2:10 Свидетели вы и Бог, как свято и праведно и безукоризненно поступали мы перед вами, верующими,
\vs 1Th 2:11 потому что вы знаете, как каждого из вас, как отец детей своих,
\vs 1Th 2:12 мы просили и убеждали и умоляли поступать достойно Бога, призвавшего вас в Свое Царство и славу.
\rsbpar\vs 1Th 2:13 Посему и мы непрестанно благодарим Бога, что, приняв от нас слышанное слово Божие, вы приняли не \bibemph{к\acc{а}к} слово человеческое, но \bibemph{как} слово Божие,~--- каково оно есть по истине,~--- которое и действует в вас, верующих.
\vs 1Th 2:14 Ибо вы, братия, сделались подражателями церквам Божиим во Христе Иисусе, находящимся в Иудее, потому что и вы то же претерпели от своих единоплеменников, что и те от Иудеев,
\vs 1Th 2:15 которые убили и Господа Иисуса и Его пророков, и нас изгнали, и Богу не угождают, и всем человекам противятся,
\vs 1Th 2:16 которые препятствуют нам говорить язычникам, чтобы спаслись, и через это всегда наполняют меру грехов своих; но приближается на них гнев до конца.
\rsbpar\vs 1Th 2:17 Мы же, братия, быв разлучены с вами на короткое время лицем, а не сердцем, тем с б\acc{о}льшим желанием старались увидеть лице ваше.
\vs 1Th 2:18 И потому мы, я Павел, и раз и два хотели прийти к вам, но воспрепятствовал нам сатана.
\vs 1Th 2:19 Ибо кто наша надежда, или радость, или венец похвалы? Не и вы ли пред Господом нашим Иисусом Христом в пришествие Его?
\vs 1Th 2:20 Ибо вы~--- слава наша и радость.
\vs 1Th 3:1 И потому, не терпя более, мы восхотели остаться в Афинах одни,
\vs 1Th 3:2 и послали Тимофея, брата нашего и служителя Божия и сотрудника нашего в благовествовании Христовом, чтобы утвердить вас и утешить в вере вашей,
\vs 1Th 3:3 чтобы никто не поколебался в скорбях сих: ибо вы сами знаете, что так нам суждено.
\vs 1Th 3:4 Ибо мы и тогда, как были у вас, предсказывали вам, что будем страдать, как и случилось, и вы знаете.
\vs 1Th 3:5 Посему и я, не терпя более, послал узнать о вере вашей, чтобы как не искусил вас искуситель и не сделался тщетным труд наш.
\vs 1Th 3:6 Теперь же, когда пришел к нам от вас Тимофей и принес нам добрую весть о вере и любви вашей, и что вы всегда имеете добрую память о нас, желая нас видеть, как и мы вас,
\vs 1Th 3:7 то мы, при всей скорби и нужде нашей, утешились вами, братия, ради вашей веры;
\vs 1Th 3:8 ибо теперь мы живы, когда вы стоите в Господе.
\vs 1Th 3:9 Какую благодарность можем мы воздать Богу за вас, за всю радость, которою радуемся о вас пред Богом нашим,
\vs 1Th 3:10 ночь и день всеусердно молясь о том, чтобы видеть лице ваше и дополнить, чего недоставало вере вашей?
\vs 1Th 3:11 Сам же Бог и Отец наш и Господь наш Иисус Христос да управит путь наш к вам.
\vs 1Th 3:12 А вас Господь да исполнит и преисполнит любовью друг к другу и ко всем, какою мы исполнены к вам,
\vs 1Th 3:13 чтобы утвердить сердца ваши непорочными во святыне пред Богом и Отцем нашим в пришествие Господа нашего Иисуса Христа со всеми святыми Его. Аминь.
\vs 1Th 4:1 За сим, братия, просим и умоляем вас Христом Иисусом, чтобы вы, приняв от нас, как должно вам поступать и угождать Богу, более в том преуспевали,
\vs 1Th 4:2 ибо вы знаете, какие мы дали вам заповеди от Господа Иисуса.
\vs 1Th 4:3 Ибо воля Божия есть освящение ваше, чтобы вы воздерживались от блуда;
\vs 1Th 4:4 чтобы каждый из вас умел соблюдать свой сосуд в святости и чести,
\vs 1Th 4:5 а не в страсти похотения, как и язычники, не знающие Бога;
\vs 1Th 4:6 чтобы вы ни в чем не поступали с братом своим противозаконно и корыстолюбиво: потому что Господь~--- мститель за все это, как и прежде мы говорили вам и свидетельствовали.
\vs 1Th 4:7 Ибо призвал нас Бог не к нечистоте, но к святости.
\vs 1Th 4:8 Итак непокорный непокорен не человеку, но Богу, Который и дал нам Духа Своего Святаго.
\rsbpar\vs 1Th 4:9 О братолюбии же нет нужды писать к вам; ибо вы сами научены Богом любить друг друга,
\vs 1Th 4:10 ибо вы так и поступаете со всеми братиями по всей Македонии. Умоляем же вас, братия, более преуспевать
\vs 1Th 4:11 и усердно стараться о том, чтобы жить тихо, делать свое \bibemph{дело} и работать своими собственными руками, как мы заповедовали вам;
\vs 1Th 4:12 чтобы вы поступали благоприлично перед внешними и ни в чем не нуждались.
\rsbpar\vs 1Th 4:13 Не хочу же оставить вас, братия, в неведении об умерших, дабы вы не скорбели, как прочие, не имеющие надежды.
\vs 1Th 4:14 Ибо, если мы веруем, что Иисус умер и воскрес, то и умерших в Иисусе Бог приведет с Ним.
\vs 1Th 4:15 Ибо сие говорим вам словом Господним, что мы живущие, оставшиеся до пришествия Господня, не предупредим умерших,
\vs 1Th 4:16 потому что Сам Господь при возвещении, при гласе Архангела и трубе Божией, сойдет с неба, и мертвые во Христе воскреснут прежде;
\vs 1Th 4:17 потом мы, оставшиеся в живых, вместе с ними восхищены будем на облаках в сретение Господу на воздухе, и так всегда с Господом будем.
\vs 1Th 4:18 Итак утешайте друг друга сими словами.
\vs 1Th 5:1 О временах же и сроках нет нужды писать к вам, братия,
\vs 1Th 5:2 ибо сами вы достоверно знаете, что день Господень так придет, как тать ночью.
\vs 1Th 5:3 Ибо, когда будут говорить: <<мир и безопасность>>, тогда внезапно постигнет их пагуба, подобно как мука родами \bibemph{постигает} имеющую во чреве, и не избегнут.
\vs 1Th 5:4 Но вы, братия, не во тьме, чтобы день застал вас, как тать.
\vs 1Th 5:5 Ибо все вы~--- сыны света и сыны дня: мы~--- не \bibemph{сыны} ночи, ни тьмы.
\vs 1Th 5:6 Итак, не будем спать, как и прочие, но будем бодрствовать и трезвиться.
\vs 1Th 5:7 Ибо спящие спят ночью, и упивающиеся упиваются ночью.
\vs 1Th 5:8 Мы же, будучи \bibemph{сынами} дня, да трезвимся, облекшись в броню веры и любви и в шлем надежды спасения,
\vs 1Th 5:9 потому что Бог определил нас не на гнев, но к получению спасения через Господа нашего Иисуса Христа,
\vs 1Th 5:10 умершего за нас, чтобы мы, бодрствуем ли, или спим, жили вместе с Ним.
\vs 1Th 5:11 Посему увещавайте друг друга и назидайте один другого, как вы и делаете.
\rsbpar\vs 1Th 5:12 Просим же вас, братия, уважать трудящихся у вас, и предстоятелей ваших в Господе, и вразумляющих вас,
\vs 1Th 5:13 и почитать их преимущественно с любовью за дело их; будьте в мире между собою.
\vs 1Th 5:14 Умоляем также вас, братия, вразумляйте бесчинных, утешайте малодушных, поддерживайте слабых, будьте долготерпеливы ко всем.
\vs 1Th 5:15 Смотрите, чтобы кто кому не воздавал злом за зло; но всегда ищите добра и друг другу и всем.
\vs 1Th 5:16 Всегда радуйтесь.
\vs 1Th 5:17 Непрестанно молитесь.
\vs 1Th 5:18 За все благодарите: ибо такова о вас воля Божия во Христе Иисусе.
\vs 1Th 5:19 Духа не угашайте.
\vs 1Th 5:20 Пророчества не уничижайте.
\vs 1Th 5:21 Все испытывайте, хорошего держитесь.
\vs 1Th 5:22 Удерживайтесь от всякого рода зла.
\vs 1Th 5:23 Сам же Бог мира да освятит вас во всей полноте, и ваш дух и душа и тело во всей целости да сохранится без порока в пришествие Господа нашего Иисуса Христа.
\vs 1Th 5:24 Верен Призывающий вас, Который и сотворит \bibemph{сие}.
\vs 1Th 5:25 Братия! молитесь о нас.
\rsbpar\vs 1Th 5:26 Приветствуйте всех братьев лобзанием святым.
\vs 1Th 5:27 Заклинаю вас Господом прочитать сие послание всем святым братиям.
\rsbpar\vs 1Th 5:28 Благодать Господа нашего Иисуса Христа с вами. Аминь.

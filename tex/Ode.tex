\bibbookdescr{Ode}{
  inline={Оды Соломона},
  toc={Оды Соломона},
  bookmark={Оды Соломона},
  header={Оды Соломона},
  abbr={Оды}
}
\vs Ode 1:1
Яхве на
главе моей подобен венцу, и да не пребуду никогда без Него.
\vs Ode 1:2
Сплетенный
для меня~--- истинный венец, и он побудил ветви Твои прорасти во мне.
\vs Ode 1:3
Ибо не похож
он на увядший венец, который не цветет;
\vs Ode 1:4
ибо Ты
обитаешь над моею главой и расцвел на мне.
\vs Ode 1:5
Полны и
спелы плоды Твои; они исполнены спасения Твоего \ldots

\vs Ode 2:1
\bibemph{не сохранилась.}

\vs Ode 3:1
\ldots\ полагаюсь я на любовь Яхве.
\vs Ode 3:2
И чресла Его
пребывают с Ним, и зависим я от них; и Он любит меня.
\vs Ode 3:3
Ибо мне бы
не стоило узнавать о том, как любит Яхве, если бы постоянно не любил Он меня.
\vs Ode 3:4
Кто же
способен различать любовь, как не тот, кто любим?
\vs Ode 3:5
Люблю я
Возлюбленного и сам я люблю Его, ибо, где бы ни был покой Его, также и я там.
\vs Ode 3:6
И не
сделаюсь чужим я, ибо нет ревности между Яхве Всевышним и Милостивым.
\vs Ode 3:7
Я соединился
с Ним, ибо любящий отыскал Возлюбленного; поскольку же люблю я того, кто есть
Сын, я сделаюсь Сыном.
\vs Ode 3:8
Истинно,
тот, кто соединится с бессмертным, воистину станет бессмертен.
\vs Ode 3:9
И тот, кто
восторгается Жизни, оживет.
\vs Ode 3:10
Вот Дух Яхве, не обманчивый, учащий сынов человеческих познавать пути Его.
\vs Ode 3:11
Будь же
мудрым, и понимающим, и пробужденным.
Аллилуйя.

\vs Ode 4:1
Ни один
человек не может осквернить святое место Твоё, о, Боже мой, как не сможет он и
изменить его и поместить его в иное место,
\vs Ode 4:2
ибо нет у
него власти над ним; ибо Святилище Свое создал Ты прежде, чем создал Ты особые
места.
\vs Ode 4:3
Древний же
не извратится тем, кто ниже него. Дал ты сердце Своё, о, Яхве, верующим в
Тебя.
\vs Ode 4:4
Не будешь ты
ни праздным, ни бесплодным;
\vs Ode 4:5
ибо один
день веры Твоей дивнее всех дней и лет.
\vs Ode 4:6
Ибо кто
положится на милость Твою и отвергнут будет?
\vs Ode 4:7
Ибо известна
печать Твоя; и творения Твои известны ей,
\vs Ode 4:8
и воинство
Твоё одержимо ею, и архангелы избранные облачены ею.
\vs Ode 4:9
Ты воздал
нам сопричастностью Твоею, не оттого, что Ты нуждался в нас, но чтобы мы
всегда нуждались в Тебе.
\vs Ode 4:10
Излей же на
нас живительный дождь Свой, и раскрой обильные источники Свои, щедро дающие
нам молоко и мёд.
\vs Ode 4:11
Ибо не
пребывает с Тобою печаль; чтобы не сожалел Ты ни о чем обещанном Тобою,
\vs Ode 4:12
ибо
результат был явлен Тебе.
\vs Ode 4:13
Ибо
отданное Тобой Ты отдал свободно, так что не передумаешь и снова не заберешь,
\vs Ode 4:14
ибо всё
было явлено Тебе как Богу и восставлено пред Тобою от начала.
\vs Ode 4:15
И Ты, о,
Яхве, создал всё.
Аллилуйя.

\vs Ode 5:1
Я молю Тебя,
о Яхве, ибо я люблю Тебя.
\vs Ode 5:2
О,
Всевышний, не отрекись от меня, ибо Ты~--- надежда моя.
\vs Ode 5:3
Да получу я
свободно милость Твою, и да буду жить ею.
4.
Преследователи мои придут, но да не увидят они меня.
\vs Ode 5:5
Да ниспадет
на очи их облако тьмы; да окутает их воздух тьмы густой.
\vs Ode 5:6
И да не
будет у них света, чтобы видеть, дабы им было не схватить меня.
\vs Ode 5:7
Да замрут
все их поползновения, дабы, где бы ни спрятались они, пасть на их собственные
головы.
\vs Ode 5:8
Ибо
замыслили они нечто, что не для них.
\vs Ode 5:9
Приуготовили
себя они злонамеренно, но найдут их бессильными.
\vs Ode 5:10
Истинно
надеюсь на Яхве, да не убоюсь.
\vs Ode 5:11
А раз Яхве спасение моё, да не убоюсь.
\vs Ode 5:12
И подобен
Ты сотканному венцу на главе моей, и да не дрогну я.
\vs Ode 5:13
Даже если
всё дрогнет, я устою неколебимо.
\vs Ode 5:14
И даже если
погибнет всё видимое, я не умру;
\vs Ode 5:15
Ибо со мною
Яхве, а я~--- с Ним.
Аллилуйя.

\vs Ode 6:1
Как ветер
проскальзывает сквозь арфу и струны говорят,
\vs Ode 6:2
так и Дух Яхве
говорит через чресла мои, а я глаголю через любовь Его,
\vs Ode 6:3
ибо сокрушает
Он всё, что чуждо, и всё сущее~--- от Яхве,
\vs Ode 6:4
ибо так было
от начала и пребудет до самого конца,
\vs Ode 6:5
чтобы не было
ничего вопреки и ничто не восстало бы на Него.
\vs Ode 6:6
Умножил Яхве
знание Своё, и был Он усерден, дабы должное стать известным по милости Его,
воздалось бы нам.
\vs Ode 6:7
И похвалу свою
воздал Он нам именем Своим, духи же наши восхваляли Его Святой Дух.
\vs Ode 6:8
И изошел
Поток, и стал рекой~--- великой и широкой; истинно, смыла она всё, и разрушила
всё, и вынесла к Храму.
\vs Ode 6:9
И преграды,
возведенные людьми, не смогли сдержать её, как не смогли даже умения тех, кто
обыкновенно сдерживает воду.
\vs Ode 6:10
Ибо разлилась
она по поверхности всей земли и заполонила всё.
\vs Ode 6:11
Когда пьют
все жаждущие на земле, и жажда облегчается и утоляется;
\vs Ode 6:12
ибо питие
дано от Всевышнего.
\vs Ode 6:13
Потому
блаженны служащие этого пития, которым вверили воду Его.
\vs Ode 6:14
Освежили они
уста пересохшие и воспрянули увядшей было волей.
\vs Ode 6:15
Даже живые,
близкие к угасанию, восстали из смерти.
\vs Ode 6:16
И омертвевшие
члены воспрянули и восстановились.
\vs Ode 6:17
Они дали силу
идти и свет глазам их.
\vs Ode 6:18
Ибо всякий
узнал их как (принадлежащих) Яхве и ожил живой водою вечности.
Аллилуйя.

\vs Ode 7:1
Как гневаются
над нечестивостью, так же и радуются Возлюбленному, и вкушают свободно от плодов
этих.
\vs Ode 7:2
Радость же моя
Яхве, и путь мой~--- к Нему, и этот путь мой превосходен.
\vs Ode 7:3
Ибо есть у
меня Помощник~--- Яхве. Он щедро явил Себя мне в простоте Своей, ибо благость Его
умалила суровость Его.
\vs Ode 7:4
Сделался Он
подобным мне, чтобы я заполучил Его. Он решил пребывать в облике, подобном
моему, чтобы я положился на Него.
\vs Ode 7:5
И не трепетал
я, когда увидел Его, ибо Он был милостив ко мне.
\vs Ode 7:6
Подобным
природе моей сделался Он, чтобы мог я понять Его. И подобным облику моему, дабы
не отвратился я от Него.
\vs Ode 7:7
Отец же знания
Слово знания.
\vs Ode 7:8
Он,
сотворивший мудрость, мудрее трудов Своих.
\vs Ode 7:9
И Он,
сотворивший меня, когда я еще не знал, что мне следовало делать, когда я начал
быть.
\vs Ode 7:10
Оттого был Он
милостив ко мне Своей щедрой милостью и позволил мне просить у Него и извлечь
пользу из жертвы Его.
\vs Ode 7:11
Ибо именно Он
нетленный, совершенство миров и их Отца.
\vs Ode 7:12
Он позволил
им явиться тем, которые Его; для того, чтобы они опознали Его, сотворившего их и
не думали, что они родились сами собой.
\vs Ode 7:13
Ибо к знанию
направил Он путь Свой, Он расширил его, и удлинил его, и привел его к полному
совершенству.
\vs Ode 7:14
И наставил
над ним следы Света Своего, и продолжалось это от начала до конца.
\vs Ode 7:15
Ибо Сам по
Себе служил Он, и Сыном наслаждался Он.
\vs Ode 7:16
И в силу
спасения Своего всем овладеет Он. И узнают Всевышнего святые Его:
\vs Ode 7:17
Возгласит
тем, кто поет песни явления Яхве, чтобы вышли они встречать Его и пели Ему,
радостно и с арфой, берущей многие ноты.
\vs Ode 7:18
Грядут
пророки прежде Него, и узрят их пред Ним.
\vs Ode 7:19
И восхвалят
они Яхве в любви Его, ибо близок Он и видит.
\vs Ode 7:20
И ненависть
исчезнет с земли, и вместе с ревностью утонет она.
\vs Ode 7:21
Ибо
невежество рушилось на ней, ибо знание Яхве пребывало на ней.
\vs Ode 7:22
Да воспоют
певцы милость Всевышнего Яхве, и да привнесут песнопения свои.
\vs Ode 7:23
И да пребудет
сердце их подобным дню, а бархатные голоса их подобными волшебной красе Яхве.
\vs Ode 7:24
Да не
пребудет там никто дышащий, в ком нет знания или гласа.
\vs Ode 7:25
Ибо дал Он
уста тварям Своим, чтобы открыли голос уст навстречу Ему и чтобы воспеть Его.
\vs Ode 7:26
Веруйте же в
силу Его и возгласите милость Его.
Аллилуйя.

\vs Ode 8:1
Откройте же,
откройте сердца ваши ликованию Яхве, и да пребудет ваша любовь обильной от
сердца к устам,
\vs Ode 8:2
чтобы принести
плоды Яхве, святую жизнь, и чтобы говорить со смирением в свете Его.
\vs Ode 8:3
Восстаньте же
и стойте неподвижно, вы, кого порой ниспосылают вниз.
\vs Ode 8:4
Вы,
пребывавшие в безмолвии, говорите же, ибо отверзлись уста ваши.
\vs Ode 8:5
Вы, бывшие
презираемыми, отныне возвыситесь, ибо возвеличена была Правда ваша.
\vs Ode 8:6
Ибо с вами
десница Яхве, и будет Он помощником вам.
\vs Ode 8:7
И уготован вам
мир прежде того, что может быть войной вашей.
\vs Ode 8:8
Слушайте же
слово истины, и получайте знание Всевышнего.
\vs Ode 8:9
Ни плоть вашу
не следует понимать так, как возвещу я вам, ни одежду вашу, что явлю я вам.
\vs Ode 8:10
Храните же
тайну мою, вы, хранимые ею, храните веру мою, вы, хранимые ею.
\vs Ode 8:11
И понимайте
знание моё, вы, понимающие меня в истине, любите меня с нежностью, вы, любящие;
\vs Ode 8:12
ибо не
отвращу лица моего от тех, кто мои, ибо я знаю их.
\vs Ode 8:13
И прежде, чем
появились они, распознал я их и наложил печать на лица их.
\vs Ode 8:14
Я создал
чресла их, и Мои собственные груди приуготовил Я для них, чтобы могли они испить
Моё святое молоко и жить им.
\vs Ode 8:15
Я любезен им,
и не пристыжён Я ими.
\vs Ode 8:16
Ибо
мастерство Моё~--- они, и сила мыслей Моих.
\vs Ode 8:17
Затем кто
восстанет против труда Моего? И кто не подвержен им?
\vs Ode 8:18
Я возжелал и
создал разум и сердце, и они~--- Мои. И одесную посадил Я избранных Моих.
\vs Ode 8:19
И правда Моя
шествует перед ними, и не лишатся они имени Моего, ибо с ними оно.
\vs Ode 8:20
Молитесь же и
возрастайте, и блюдите себя в любви Яхве.
\vs Ode 8:21
И вы, бывшие
возлюбленными в Возлюбленном, и вы, хранимые в Том, Кто жив, и вы, спасенные в
Том, Кто спасен.
\vs Ode 8:22
И найдут вас
непорочными в любые времена, во имя Отца вашего.
Аллилуйя.

\vs Ode 9:1
Навострите же
уши ваши, и скажу я вам.
\vs Ode 9:2
Отдайтесь же
мне, чтобы я также отдался вам.
\vs Ode 9:3
(Чтобы отдал я
вам) Слово Яхве и страсти Его, святую мысль, которую думал Он о Помазаннике
Своем.
\vs Ode 9:4
Ибо в воле
Яхве~--- жизнь ваша, и цель Его~--- вечная жизнь, и совершенство ваше~--- непорочно.
\vs Ode 9:5
Обогатитесь же
в Боге-Отце, и стяжайте цель Всевышнего. Станьте же сильными и искупленными
милостью Его.
\vs Ode 9:6
Ибо возвещаю я
мир вам, святым Его, дабы никто из слышащих не впал в войну.
\vs Ode 9:7
А также дабы
познавшие Его не погибли, и дабы стяжавшие Его не устыдились.
\vs Ode 9:8
Вечный же
венец суть Истина; блаженны носящие ее на главах своих.
\vs Ode 9:9
Это~--- камень
драгоценный, ибо из-за венца этого велись войны.
\vs Ode 9:10
Но взяла его
Правда и отдала вам.
\vs Ode 9:11
Возложите же
венец этот в истинном согласии с Яхве, и всех побежденных впишут в книгу Его.
\vs Ode 9:12
Ибо книга их
награда победы вашей, и видит она вас пред собою и желает, чтобы вы спасены
были.
Аллилуйя.

\vs Ode 10:1
Яхве наставил
уста мои Словом Своим и открыл сердце моё Светом Своим.
\vs Ode 10:2
И велел Он мне
остаться в бессмертной жизни Его и позволил мне возвестить о плоде покоя Его,
\vs Ode 10:3
преобразить
жизни жаждущих прийти к Нему и вести пленных к свободе.
\vs Ode 10:4
Я же осмелел и
стал сильным, и захватил мир этот, и стала неволя Моей во славу Всевышнего и
Бога, Отца моего.
\vs Ode 10:5
И кротких,
бывших рассеянными, собрали вместе, но не осквернился я любовью своей к ним, ибо
они благодарили меня в вышних местах.
\vs Ode 10:6
И следы света
легли на сердце их, и шли они как по жизни моей, и спасены были, и сделались они
народом моим вовеки веков.
Аллилуйя.

\vs Ode 11:1
Моё сердце
было обрезано и появился цветок у него, затем же милость проросла в нем, и дало
плоды сердце моё ради Яхве.
\vs Ode 11:2
Ибо Всевышний
обрезал его своим Духом Святым, и он открыл мою внутреннюю жизнь навстречу Ему и
наполнил меня любовью Своей.
\vs Ode 11:3
И обрезание
Его сделалось спасением моим, и взошел я на путь, на покой Его, на путь истины.
\vs Ode 11:4
От начала до
конца стяжал я знание Его.
\vs Ode 11:5
И встал я на
скале истины, где Он оставил меня.
\vs Ode 11:6
И говорящие
воды щедро коснулись уст моих из фонтана Яхве.
\vs Ode 11:7
И так пил я и
пьянел от живой воды бессмертной.
\vs Ode 11:8
И опьянение
моё не привело к невежеству, но отрекся я от спеси,
\vs Ode 11:9
и обратился ко
Всевышнему, Богу моему, и обогатился пользой Его.
\vs Ode 11:10
И отверг я
глупость, павшую на землю, и разоблачил её и отбросил прочь от себя.
\vs Ode 11:11
И Яхве
обновил меня одеянием Своим и овладел мною светом Своим.
\vs Ode 11:12
И воздал Он
мне свыше бессмертным покоем, и сделался я подобным земле цветущей и радостной
плодами своими.
\vs Ode 11:13
И Яхве
подобен солнцу над лицом земли.
\vs Ode 11:14
Мне
просветлили очи, и лицо моё окропилось росой;
\vs Ode 11:15
и освежилось
дыхание моё благоуханным ароматом Яхве.
\vs Ode 11:16
И взял Он
меня в Рай Свой, где богатство удовольствия Яхве. Я узрел цветущие и
плодоносящие деревья, и самовозросшей была крона их. Прорастали ветви их и сияли
плоды их. Из бессмертной земли были корни их. И река радости орошала их и
окружала их в земле вечной жизни.
\vs Ode 11:17
Затем
поклонился я Яхве за великолепие Его.
\vs Ode 11:18
И сказал я:
Блаженны, о Яхве, возросшие в земле Твоей, и те, кому есть место в Раю Твоем,
\vs Ode 11:19
и кто растет
ростом деревьев Твоих и перебрался из тьмы в свет.
\vs Ode 11:20
Узри же: все
работники Твои чисты, они, делающие добрые дела, и обращающиеся из дикости в
приязнь Твою.
\vs Ode 11:21
Ибо резкий
запах деревьев сих изменился в земле Твоей,
\vs Ode 11:22
и всё
делается частичкой Тебя. Блаженны же труженики вод Твоих и вечные памятники
набожных слуг Твоих.
\vs Ode 11:23
Истинно, в
Раю Твоем обителей много. И нет там ничего пустого, но всё исполнено плодами.
\vs Ode 11:24
Славься же
Ты, Боже, (и да пребудет) райское ликование вовеки.
Аллилуйя.

\vs Ode 12:1
Он наполнил
меня словами истины, дабы я проповедовал Его.
\vs Ode 12:2
И подобно
течению вод, истина вытекает из уст моих, и уста мои возвещают плоды Его.
\vs Ode 12:3
И побудил Он
знание Своё изобиловать во мне, ибо уста Яхве суть Слово истинное и врата Света
Его.
\vs Ode 12:4
И Всевышний
воздал Его коленам Его, которые суть толковники красы Его, и толковники славы
Его, и исповедники цели Его, и проповедники разума Его, и учителя дел Его.
\vs Ode 12:5
Ибо невыразима
тонкость Слова этого; и каково изречение Его, таковы и быстрота Его, и острота
Его, ибо беспредельность Его суть развитие Его.
\vs Ode 12:6
Никогда не
падает Он, но выстаивает, и никто не может понять нисхождение Его или же путь
Его.
\vs Ode 12:7
Ибо каково
дело Его, такова же надежда Его, ибо Он суть свет и заря мысли.
\vs Ode 12:8
И через Него
колена говорят друг с другом, и безмолвные речь обрели.
\vs Ode 12:9
И из Него
изошли любовь и равенство, и друг другу говорили они, что это принадлежало им.
\vs Ode 12:10
И подтолкнуло
их Слово, и познали Того, Кто создал их, ибо они пребывали в гармонии.
\vs Ode 12:11
Ибо уста
Всевышнего говорили им, и объяснение Его расцвело через Него.
\vs Ode 12:12
Ибо жилище
Слова~--- человек, а истина Его~--- любовь.
\vs Ode 12:13
Блаженны же
воспринявшие всё через Него и познавшие Яхве в истине Его.
Аллилуйя.

\vs Ode 13:1
Узрите же,
Яхве~--- зеркало наше. Откройте очи ваши и узрите их в Нем.
\vs Ode 13:2
И изучите вид
лица вашего, вознося хвалы Духу Святому,
\vs Ode 13:3
и сотрите
краску с лица вашего, и возлюбите святость Его и положитесь на неё.
\vs Ode 13:4
Тогда будете
вы безупречными с Ним во все времена.
Аллилуйя.

\vs Ode 14:1
Как очи сына
на отца его, также и мои очи, о, Яхве, к Тебе во все времена.
\vs Ode 14:2
Ибо сердце моё
и радость моя~--- с Тобой.
\vs Ode 14:3
Не отврати же
милостей Своих от меня, о Яхве, и не отними доброты Своей у меня.
\vs Ode 14:4
Протяни же ко
мне, мой Господь, на все времена, десницу Свою, и веди меня до самого конца,
согласно воле Твоей.
\vs Ode 14:5
Позволь же мне
быть угодным Тебе, о Яхве, во славу Твою и во имя Твоё позволь мне спастись от
Нечистого.
\vs Ode 14:6
И снисхождение
Твое, о, Яхве, да снизойдет на меня, и плоды любви Твоей.
\vs Ode 14:7
Обучи же меня
одам истины Твоей, дабы плодоносил я в Тебе.
\vs Ode 14:8
И открой мне
арфу Твоего Святого Духа, чтобы с каждой нотой восхвалял я Тебя, о Яхве.
\vs Ode 14:9
И по многим
милостям Твоим дари мне и спеши дарить по просьбам нашим.
\vs Ode 14:10
Ибо хватит
Тебя на все нужды наши.
Аллилуйя.

\vs Ode 15:1
Как солнце~---
радость алчущих восхода его, так же моя радость~--- Яхве.
\vs Ode 15:2
Ибо Он~---
Солнце моё, и лучи Его вознесли меня, и свет Его рассеял всю тьму с лица моего.
\vs Ode 15:3
Глаза обрел я
в Нем и увидел святой день Его.
\vs Ode 15:4
Уши обрел я и
услышал истину Его.
\vs Ode 15:5
Мысль знания
обрел я и преисполнился восторгом всецело через Него.
\vs Ode 15:6
Отрекся я от
пути ложного и пришел к Нему и премного стяжал спасения у Него.
\vs Ode 15:7
И по щедрости
Своей воздал Он мне, и по образу превосходной красоты Своей создал Он меня.
\vs Ode 15:8
Облекся я
бессмертием именем Его и очистился от тлена милостью Его.
\vs Ode 15:9
Повержена
смерть перед лицом моим, и повержен Шеол словом моим.
\vs Ode 15:10
И взошла в
земле Яхве жизнь вечная, и возвестили её верным Его и беспредельно дана была
всем верующим в Него.
Аллилуйя.

\vs Ode 16:1
Как дело
пахаря~--- пахота, а дело рулевого~--- править кораблем, так и моё дело~--- псалом
Яхве в гимнах Его.
\vs Ode 16:2
Искусство моё
и служение моё~--- в гимнах Его, ибо любовь Его питала сердце мо, а плоды Свои
излил Он на уста мои.
\vs Ode 16:3
Ибо любовь моя
Сам Яхве, затем же стану я петь о Нем.
\vs Ode 16:4
Ибо силен я
похвалами Его и веру имею в Нем.
\vs Ode 16:5
Открою я уста
мои, а Дух Его возвестит через меня славу Яхве и красу Его,
\vs Ode 16:6
работу рук Его
и труд перст Его
\vs Ode 16:7
ради многих
милостей Его и силы Слова Его.
\vs Ode 16:8
Ибо Слово Яхве
изучает невидимое и открывает мысль Его.
\vs Ode 16:9
Ибо видит око
труды Его, а ухо слышит мысль Его.
\vs Ode 16:10
Ибо именно Он
создал землю широкой и налил воды в море,
\vs Ode 16:11
Он расширил
небо и зажег звезды,
\vs Ode 16:12
и Он создал
творение и наставил его, а затем почил Он от трудов Своих.
\vs Ode 16:13
И сотворил
все вещи бегущими путями своими и делающими дела свои, ибо никогда не могут они
ни перестать быть, ни потерпеть неудачу.
\vs Ode 16:14
И светила~---
подданные Слова Его.
\vs Ode 16:15
Сосуд же
света~--- солнце, а сосуд тьмы~--- ночь.
\vs Ode 16:16
Ибо создал Он
солнце во имя дня, дабы был свет, ночь же приносит тьму на лицо земли,
\vs Ode 16:17
и по частичке
друг от друга составляют они красоту Божью.
\vs Ode 16:18
И нет ничего
вне Яхве, ибо Он был прежде, чем что-либо начало быть.
\vs Ode 16:19
И эти миры~---
по слову Его и по мысли сердца Его.
\vs Ode 16:20
Восхваляйте
же и чтите имя Его.
Аллилуйя.

\vs Ode 17:1
Затем
увенчался я Богом моим, и венец мой был живым.
\vs Ode 17:2
И оправдался я
Господом моим, ибо спасение моё нетленно.
\vs Ode 17:3
Освободился я
от гордыни, и не осужден.
\vs Ode 17:4
Узы мои были
разрублены руками Его, стяжал я образ и подобие новой личности, и я ходил под
Ним и был спасен.
\vs Ode 17:5
И водила мной
мысль истины, и я следовал за ней и не блуждал я.
\vs Ode 17:6
И все видевшие
меня были изумлены, и незнакомцем казался я им.
\vs Ode 17:7
А Тот, Кто
знал и возвысил меня, суть Всевышний во всем совершенстве Своем.
\vs Ode 17:8
И прославил Он
меня добротой Своей и вознес понимание моё до вершин истины.
\vs Ode 17:9
И оттуда
указал мне путь шагов Своих, и отверз я двери закрытые.
\vs Ode 17:10
И сокрушил я
засовы железные, ибо собственные кандалы мои сделались горячи и расплавились
предо мною.
\vs Ode 17:11
И ничто не
являлось мне закрытым, ибо всё открывал я.
\vs Ode 17:12
И шел я ко
всем узам моим, чтобы избавиться от них, дабы не оставить никого скованным или
же связанным.
\vs Ode 17:13
И щедро
раздавал я знание своё и восстание своё через любовь свою.
\vs Ode 17:14
И посеял я
плоды свои в сердцах и преобразил их собою.
\vs Ode 17:15
Затем стяжали
они благодать мою и жили, и собирались подле меня и спасались.
\vs Ode 17:16
Ибо сделались
они чреслами моими, а я был главой их.
\vs Ode 17:17
Слава Тебе,
Глава наша, о Яхве, Помазанник.
Аллилуйя.

\vs Ode 18:1
Сердце моё
воспрянуло и обогатилось в любви Всевышнего, дабы под именем моим восхвалял я
Его.
\vs Ode 18:2
Чресла же мои
усилены были, дабы не выпасть из-под власти Его.
\vs Ode 18:3
Немощи вышли
из тела моего, и стояло оно твердо, ради Яхве, по воле Его, ибо твердо Царство
Его.
\vs Ode 18:4
О, Яхве, ради
нуждающихся, не отпускай Слово Своё от меня.
\vs Ode 18:5
И ради трудов
их, удержи при мне совершенство Своё.
\vs Ode 18:6
Да не
низвергнется свет тьмою, а истина да отделится от лжи.
\vs Ode 18:7
Да приведет к
победе десница Твоя спасение наше, и да придет она из всякой области и да
утвердится на берегу всякого, снедаемого горестями.
\vs Ode 18:8
Ты~--- Бог мой,
не в Твоих устах ложь и смерть, только совершенство~--- воля Твоя.
\vs Ode 18:9
И не ведаешь
Ты гордыни, ибо никто из творящих её не ведает Тебя.
\vs Ode 18:10
И не ведаешь
Ты ошибки, ибо никто из творящих её не ведает Тебя.
\vs Ode 18:11
И явилось
невежество словно пыль и словно пена морская.
\vs Ode 18:12
И пустые люди
думали, что величественно оно, и сделались они подобными типу его и были они
истощены.
\vs Ode 18:13
Но те, кто
ведали, поняли и осмыслили и не осквернились мыслями своими,
\vs Ode 18:14
ибо пребывали
они в разуме Всевышнего и высмеяли ходивших ложными путями.
\vs Ode 18:15
Затем же
изрекали они истину от дыхания, которое вдохнул в них Всевышний.
\vs Ode 18:16
Хвала и честь
великая имени Его.
Аллилуйя.

\vs Ode 19:1
Чашку молока
предложили мне, и пил я его во сладости доброты Яхве.
\vs Ode 19:2
Сын~--- чаша
эта, а Отец~--- Тот, кто доил, а Дух Святой~--- Та, которая доила Его;
\vs Ode 19:3
ибо груди Его
были полны, и не хотелось бы, чтобы млеко Его пропало без толку.
\vs Ode 19:4
Дух Святой
обнажил грудь Её, и смешал молоко из двух грудей Отца.
\vs Ode 19:5
Затем дала Она
смесь колену без ведома их, и получившие её пребывают в совершенстве одесную.
\vs Ode 19:6
Чрево девы
поглотило её, и обрела она идею и порождала.
\vs Ode 19:7
Так дева стала
матерью с милосердием превеликим.
\vs Ode 19:8
И трудилась
она и родила Сына, но без боли, ибо не было это бесцельно.
\vs Ode 19:9
И не надо было
ей повитухи, ибо Он побудил её к порождению жизни.
\vs Ode 19:10
Родила же
она, подобно сильному человеку, со страстью, и родила она согласно проявлению, и
приобрела она согласно Великой Силе.
\vs Ode 19:11
И любила она
искупительной (любовью), и оберегала с добротой, и вещала великолепно.
Аллилуйя.

\vs Ode 20:1
Я~--- священник
Яхве, и ему служу я священником;
\vs Ode 20:2
и Ему
предлагаю я подношение мысли Его.
\vs Ode 20:3
Ибо мысль Его
не подобна ни миру сему, ни плоти, ни поклоняющимся по законам плоти.
\vs Ode 20:4
Приношение же
Яхве~--- правда и чистота сердца и уст.
\vs Ode 20:5
Жертвуйте же
безупречно внутренней жизнью вашей, и да не погасится сострадание состраданием
вашим, и да не станете вы угнетать себя.
\vs Ode 20:6
Не подкупайте
иноземца, ибо он не похож на вас, и не следует также пытаться обмануть ближнего
вашего или же лишить его одежды, дабы обнажить его.
\vs Ode 20:7
Но щедро
положитесь на милость Яхве, и придите в Рай Его, и сделайте себе гирлянду из
древа Его.
\vs Ode 20:8
Затем положите
её на главу вашу, и будьте радостны, и положитесь на покой Его.
\vs Ode 20:9
ибо слава Его
проследует пред вами, и стяжаете вы от доброты Его и от милости Его, и помажут
вас в истине, с хвалою святости Его.
\vs Ode 20:10
Хвала и честь
имени Его.
Аллилуйя.

\vs Ode 21:1
Воздел я руки
ввысь во имя сострадания Яхве.
\vs Ode 21:2
Ибо совлек Он
с меня узы мои, а Помощник мой вознес меня соответственно состраданию Его и
спасению Его.
\vs Ode 21:3
И совлек я
тьму и облекся светом
\vs Ode 21:4
и даже сам
обрел чресла. В них не было болезни, или несчастья, или страдания.
\vs Ode 21:5
И щедро
помогала мне мысль Яхве, и Его вечное братство.
\vs Ode 21:6
И был я поднят
в свет, и я проследовал перед Ним.
\vs Ode 21:7
И постоянно
пребывал я подле Него, хваля и исповедуя Его.
\vs Ode 21:8
Подвиг Он
сердце моё переполниться, и нашли его в устах моих; и вросло оно в уста мои.
\vs Ode 21:9
Затем же
сделалось чертой лица моего ликование Яхве и похвала Его.
Аллилуйя.

\vs Ode 22:1
Он, подвигший
меня снизойти свыше и вознестись из мест нижних,
\vs Ode 22:2
и Он,
собирающий тех, что в середине, и сбрасывающий их ко мне,
\vs Ode 22:3
Он,
раскидавший врагов моих и соперников моих,
\vs Ode 22:4
Он, давший мне
власть над узами, дабы мог я развязать их,
\vs Ode 22:5
Он, моими
руками свергнувший дракона семиглавого и поставивший меня на корни его, дабы
сокрушил я семя его,
\vs Ode 22:6
Ты был там и
помогал мне, и в каждом месте имя Твоё окружало меня.
\vs Ode 22:7
Десница Твоя
сокрушила едкий яд его, и рука Твоя указала путь верующим в Тебя.
\vs Ode 22:8
И вызволила
она их из могил и отделила от мертвых.
\vs Ode 22:9
Она взяла
мертвые кости и покрыла их плотью.
\vs Ode 22:10
Но были они
неподвижны, поэтому дала она им жизненную силу.
\vs Ode 22:11
Непорочен был
путь Твой и лицо Твоё; привел Ты мир Свой к тлену, чтобы всё распалось и
обновилось.
\vs Ode 22:12
И основание
всего~--- скала Твоя. И на ней воздвиг Ты Царство Своё, и сделалось оно местом
обитания святых.
Аллилуйя.

\vs Ode 23:1
Радость~---
святым. И кто же облечется в неё, как не сами они?
\vs Ode 23:2
Милость~---
избранным. И кому же стяжать её, как не верующим в нее от начала?
\vs Ode 23:3
Любовь~---
избранным. И кто же облечется в неё, как не одержимые ею от начала?
\vs Ode 23:4
Ходите в
знании Яхве, и щедро познаете вы милость Яхве; как ради ликования Его, так и во
имя совершенства знания Его.
\vs Ode 23:5
И мысль Его
уподобилась письменам, и воля Его снизошла свыше
\vs Ode 23:6
и послана была
она подобно стреле из лука, выстрелившей с усилием.
\vs Ode 23:7
И много рук
поспешили к письму, дабы похитить его, а затем взять и прочесть его.
\vs Ode 23:8
Но ускользнуло
оно из пальцев их, и испугались они его, и печати, бывшей на нем.
\vs Ode 23:9
Ибо не
дозволялось им терять печать его, ибо власть печати этой была величественнее их.
\vs Ode 23:10
Но видевшие
письмо пришли за ним, чтобы изучить, где его должно бросить, и кто должен
прочесть его, и кто должен услышать его.
\vs Ode 23:11
Но колесу
досталось оно, и вошло оно чрез него.
\vs Ode 23:12
И знак был с
ним~--- Царства и Провидения.
\vs Ode 23:13
И всё,
мешавшее колесу, скосило оно и обрубило.
\vs Ode 23:14
И сдержало
оно множество недругов, и вымостило реки.
\vs Ode 23:15
И испещрило и
выкорчевало оно многие леса и расчистило путь.
\vs Ode 23:16
Пала глава к
ногам, ибо к ногам прикатилось колесо, и всё пришедшее с ним.
\vs Ode 23:17
Письмо же
было одним из повелений, и поэтому все области собрались воедино.
\vs Ode 23:18
И виден был
на главе его, на открывшейся главе, даже Сын Истины от Всевышнего Отца.
\vs Ode 23:19
И Он
наследовал всё и обладал всем, и прекратились затем происки многих.
\vs Ode 23:20
Затем же все
соблазнители заупрямились и сбежали, а преследователи увяли и были уничтожены.
\vs Ode 23:21
А письмо
сделалось большим томом, полностью начертанным перстом Божьим.
\vs Ode 23:22
И имя Отца
было на нем, а также Сына и Святого Духа, дабы властвовать вовеки веков.
Аллилуйя.

\vs Ode 24:1
Парила голубка
над главой нашего Яхве Помазанника, ибо был Он главой её,
\vs Ode 24:2
и пела она над
Ним, и услышан был глас её.
\vs Ode 24:3
Затем убоялись
живущие, и обеспокоились чужестранцы.
\vs Ode 24:4
Птица же стала
взлетать, и всякая тварь ползучая подохла в норе своей.
\vs Ode 24:5
И отверзались
и захлопывались бездны, и искали Яхве они, подобно готовым родить.
\vs Ode 24:6
Но не был Он
отдан им на съедение, ведь Он не принадлежал им.
\vs Ode 24:7
Но пропасти
погрузились в печать Яхве, и погибли они в той мысли, с которой оставались от
начала.
\vs Ode 24:8
Ибо от начала
пребывали в трудах они, и концом их тяжкого труда была жизнь.
\vs Ode 24:9
И все они,
пребывавшие в лишениях, погибли, ибо неспособны были они сказать такое слово,
чтобы суметь остаться.
\vs Ode 24:10
И сокрушил
Яхве символы всех не нашедших правды в них.
\vs Ode 24:11
Ибо их
обделили мудростью, их, занимавшихся самовосхвалением в разуме своем.
\vs Ode 24:12
Так их
отвергли, ибо правда не была с ними.
\vs Ode 24:13
Ибо Яхве
открыл путь Свой и широко распространил милость Свою.
\vs Ode 24:14
И те, кто
поняли это, познали святость Его.
Аллилуйя.

\vs Ode 25:1
Спасен я был
от уз моих и бежал к Тебе, о мой Господь.
\vs Ode 25:2
Ибо Ты~---
десница спасения и Спаситель мой.
\vs Ode 25:3
Ты сдержал
восставших против меня, и больше их не видели.
\vs Ode 25:4
Ибо лицо Твоё
пребывало со мной, спасая меня милостью Твоей.
\vs Ode 25:5
Я же был
презираемым и отверженным в глазах многих, и был я в их глазах свинцу подобен.
\vs Ode 25:6
И обрел я силу
от Тебя, и помощь.
\vs Ode 25:7
Светильник
водрузил ты ради меня и справа, и слева, чтобы не было во мне ничего
неосвещенного.
\vs Ode 25:8
И облекся я
покровом Духа Твоего и отбросил прочь от себя одежды кожаные.
\vs Ode 25:9
Ибо десница
Твоя вознесла меня и принудила болезнь оставить меня.
\vs Ode 25:10
И стал я
могучим истиной Твоей и святым правдой Твоей.
\vs Ode 25:11
И все недруги
мои убоялись меня, и сделался я (человеком) Яхве во имя Яхве.
\vs Ode 25:12
И оправдался
я добротой Его и покоем Его во веки веков.
Аллилуйя.

\vs Ode 26:1
Излил я хвалу
Яхве, ибо я~--- Он Сам.
\vs Ode 26:2
И зачитаю я
святую оду Его, ибо сердце моё~--- с Ним.
\vs Ode 26:3
Ибо арфа его в
руке моей, и не утихнут оды покоя Его.
\vs Ode 26:4
Желаю я
воззвать к Нему всем сердцем моим, желаю я восхвалять и возвышать Его всеми
чреслами моими.
\vs Ode 26:5
Ибо от востока
до запада пребывает хвала Его,
\vs Ode 26:6
а также с юга
на север простирается благодарение Его,
\vs Ode 26:7
даже с пиков
вершин и до края их в совершенстве Его.
\vs Ode 26:8
Кто же сумеет
записать оды Яхве и кто сумеет прочесть их?
\vs Ode 26:9
И кто сумеет
приуготовить себя к жизни, чтобы спастись самому?
\vs Ode 26:10
И кто сумеет
вынудить Всевышнего, чтобы Собственными устами зачитал Он?
\vs Ode 26:11
Кто же сумеет
истолковать чудеса Яхве? Убьют толкующего~--- истолкованное еще останется.
\vs Ode 26:12
Ибо
достаточно воспринять и удовольствоваться, ведь сочинители од стоят спокойные;
\vs Ode 26:13
подобно реке
с усиленно бьющим истоком и текущей на смену им, дабы отыскать это.
Аллилуйя.

\vs Ode 27:1
Простер я руки
свои и освятил Господа Моего,
\vs Ode 27:2
ибо
простирание рук моих~--- знак Его,
\vs Ode 27:3
а простирание
моё~--- вертикальный крест.
Аллилуйя.

\vs Ode 28:1
Как крылья
голубей распростерты над птенцами их, а клювики птенцов смотрят в клювы их, так
же и крылья Духа распростерты над сердцем моим.
\vs Ode 28:2
Сердце моё
непрерывно освежается и прыгает от радости, как малыш, прыгающий от радости во
чреве матери своей.
\vs Ode 28:3
Я веровал, а
значит пребывал я в покое, ибо верующий~--- Тот, в кого я уверовал.
\vs Ode 28:4
Он с чувством
благословил меня, и глава моя пребывает с Ним.
\vs Ode 28:5
Ни кинжал не
разделит меня с Ним, ни меч,
\vs Ode 28:6
ибо готов я
прежде, чем настанет погибель, и восстал в Его бессмертном уделе.
\vs Ode 28:7
И объяла меня
жизнь бессмертная, и облобызала меня.
\vs Ode 28:8
И от жизни
этой исходит Дух, Который во мне. И не может Он умереть, ибо Он~--- Жизнь.
\vs Ode 28:9
Видевшие же
меня изумились; ибо притесняли меня.
\vs Ode 28:10
И думали они,
что поглощали меня, ибо казался я им одним из пропавших.
\vs Ode 28:11
Но
несправедливость моя сделалась спасением моим.
\vs Ode 28:12
И сделался я
отвращением их, ибо не было во мне ревности.
\vs Ode 28:13
Ибо долго
делал я добро всякому, ненавидящему меня.
\vs Ode 28:14
И окружили
они меня словно собаки, что по глупости нападают на хозяев своих.
\vs Ode 28:15
Ибо мысль их
развращена и разум их извращен.
\vs Ode 28:16
Но несу воду
я в деснице моей, а горечь их продлил я приязнью своей.
\vs Ode 28:17
И не погиб я,
ибо ни братом их не был я, ни рождение моё не было подобным их.
\vs Ode 28:18
И искали они
смерти моей, но не нашли ее возможной, ибо был я старше, чем память их; и во
тщете массами набросились они на меня.
\vs Ode 28:19
И бывшие
после меня тщетно пытались сокрушить памятник Тому, Кто был пред ними.
\vs Ode 28:20
Ибо мысль
Всевышнего не могла быть предрассудком, а сердце Его превыше всякой мудрости.
Аллилуйя.

\vs Ode 29:1
Яхве~--- надежда
моя, и да не устыжусь Его.
\vs Ode 29:2
Ибо во хвале
Своей создал Он меня, и милостью Своей даже её воздал Он мне.
\vs Ode 29:3
И милостями
Своими возвеличил Он меня, и великой честью Своей возвысил Он меня.
\vs Ode 29:4
И побудил Он
меня восстать из глубин Шеола, и из уст смерти вызволил Он меня.
\vs Ode 29:5
И умалил я
врагов своих, и оправдал меня Он милостью Своей.
\vs Ode 29:6
Ибо уверовал я
в Помазанника Яхве и счел, что Он и есть Бог.
\vs Ode 29:7
И явил Он мне
знак, и водил меня светом Своим.
\vs Ode 29:8
И дал Он мне
скипетр власти Своей, чтобы подчинил я машины людские и умалил силу могущества,
\vs Ode 29:9
воевал словом
Его и одержал победу силой Его.
\vs Ode 29:10
И низверг
Яхве врага моего словом Своим, и тот уподобился пыли, сдуваемой ветром.
\vs Ode 29:11
И воздал я
хвалу Всевышнему, ибо возвеличил Он слугу Своего и Сына служанки Своей.
Аллилуйя.

\vs Ode 30:1
Наполнитесь же
водою из живого источника Яхве, ибо он открылся вам,
\vs Ode 30:2
и придите все
алчущие и напейтесь, и отдохните подле источника Яхве,
\vs Ode 30:3
ибо приятен и
сверкающ он и вечно освежает.
\vs Ode 30:4
Ибо много
слаще вода Его, нежели мёд, и соты пчелиные не сравнятся с ним;
\vs Ode 30:5
ибо исходил он
из уст Яхве, а вызван был из сердца Яхве.
\vs Ode 30:6
И пришла она
бесконечной и невидимой, и пока не налилась в середине, они не узнали её.
\vs Ode 30:7
Блаженны же
пившие оттуда и освежившиеся ею.
Аллилуйя.

\vs Ode 31:1
Пред Яхве
исчезали пропасти, а Тьма рассеивалась пред появлением Его.
\vs Ode 31:2
Ошибка же
ошиблась и погибла из-за Него, а неуважению не дали тропу, ибо была она
затоплена правдой Яхве.
\vs Ode 31:3
Отверз Он уста
Свои и изрек милость и ликование, и зачитал новый гимн имени Своему.
\vs Ode 31:4
Затем же
возвысил Он глас Свой ко Всевышнему и вверил ему ставших Сыновьями из-за Него.
\vs Ode 31:5
И лицо Его
признано было, ибо так воздал Ему Святой Отец Его.
\vs Ode 31:6
Придите,
обиженные, и возрадуйтесь.
\vs Ode 31:7
И владейте
собою милостиво и вберите в себя жизнь бессмертную.
\vs Ode 31:8
И осудили меня
они, когда восстал я,~--- меня, не осужденного.
\vs Ode 31:9
Затем же
разделили они добро моё, хотя ничего не причиталось им.
\vs Ode 31:10
Но вытерпел
я, и держал себя в руках, и безмолвен был, дабы не быть умерщвленным ими.
\vs Ode 31:11
Но
непоколебимо стоял я, словно твердая скала, долгое время побиваемая накатами
волн, и терпел.
\vs Ode 31:12
И желчность
их смиренно вынес я, дабы искупить народ мой и наставить его,
\vs Ode 31:13
и дабы не
отречься от обетов патриархам, которым был обещан я во спасение потомков их.
Аллилуйя.

\vs Ode 32:1
Блаженным~---
радость сердец их и свет Того, Кто пребывает в них,
\vs Ode 32:2
и Слово правды
само родившееся,
\vs Ode 32:3
ибо усилен был
Он Святою Силою Всевышнего, и не поколеблен Он вовеки веков.
Аллилуйя.

\vs Ode 33:1
Но вновь
поспешила милость и отвергла Искусителя и снизошла в него, дабы низвергнуть его.
\vs Ode 33:2
И вызвал он
сплошное разрушение перед собой и испортил весь труд свой.
\vs Ode 33:3
И стоял он на
пике горной вершины и громко вопил из конца в конец земли.
\vs Ode 33:4
Затем приволок
он к нему всех подчиненных ему, ибо не как грешник явился он.
\vs Ode 33:5
Однако именно
совершенная дева стояла, проповедуя, и взывая, и говоря:
\vs Ode 33:6
О, вы, сыновья
человеческие, вернитесь, и вы, дщери их, придите,
\vs Ode 33:7
и оставьте
пути Искусителя этого, и приблизьтесь ко мне.
\vs Ode 33:8
И войду я в
вас, и выведу из разрушения вас, и сделаю мудрыми вас на путях правды.
\vs Ode 33:9
Не будьте же
ни грешными, ни гибнущими.
\vs Ode 33:10
Повинуйтесь
мне, и спасены будете, ибо возглашу я вам милость Божью.
\vs Ode 33:11
И чрез меня
спасетесь вы и сделаетесь блаженными. Я~--- суд ваш;
\vs Ode 33:12
и
положившихся на меня не обвинят неправедно, но нетленность стяжают они в новом
мире.
\vs Ode 33:13
Избранные мои
пошли за мною, и пути мои сделаю я известными тем, кто взыскивает меня; и
завещаю я им имя своё.
Аллилуйя.

\vs Ode 34:1
Нет ни
трудного пути там, где есть простодушие, ни преграды для прямодушия,
\vs Ode 34:2
ни урагана в
глубине просветленной мысли.
\vs Ode 34:3
Там, где
окружен некто со всех сторон приятной страной, там ничто не разделено в нем.
\vs Ode 34:4
То, что внизу,
подобно тому, что вверху.
\vs Ode 34:5
Ибо всё~---
свыше, а снизу~--- ничего, но спорят с этим те, в ком нет понимания.
\vs Ode 34:6
Милость же
явлена во спасение ваше.
Аллилуйя.

\vs Ode 35:1
Живительный
ливень Яхве преспокойно накрыл меня и облако покоя: побудили они взойти над
головой моей,
\vs Ode 35:2
чтобы могло
оно оберегать меня во все времена. И стало оно спасением мне.
\vs Ode 35:3
Всякий
обеспокоился и убоялся, и изошли из них дым и судилище.
\vs Ode 35:4
Я же спокоен
был в воинстве Яхве; больше, нежели тенью был он для меня, и больше, нежели
основанием.
\vs Ode 35:5
И носили меня,
как мать дитя своё, Он же дал мне молоко, росу Яхве.
\vs Ode 35:6
И обогатился я
пользою Его, и покоился в совершенстве Его.
\vs Ode 35:7
И распростер я
в вознесении руки свои, и направился ко Всевышнему, и искуплен был пред Ним.
Аллилуйя.

\vs Ode 36:1
Покоился я в
Духе Яхве, и Он вознес меня к небесам;
\vs Ode 36:2
и велел мне
стать на ноги в вышнем месте Яхве, перед совершенством Его и славой Его, где и
продолжил я славить Его сочинением од Его.
\vs Ode 36:3
Дух же принес
меня к лицу Яхве, а поскольку был я Сыном человека, меня назвали Светом, Сыном
Божьим;
\vs Ode 36:4
ибо был я
достославным среди славных и величайшим среди великих.
\vs Ode 36:5
Ибо к величию
Всевышнего сделал меня Он, и по новизне Своей обновил Он меня.
\vs Ode 36:6
И помазал Он
меня совершенством Своим, и сделался я одним из тех, кто подле Него.
\vs Ode 36:7
И открылись
уста мои, словно облако росы, и хлынуло сердце моё как скважина праведности.
\vs Ode 36:8
И приближение
моё было мирным, и поставили меня в Духе Провидения.
Аллилуйя.

\vs Ode 37:1
Воздел я руки
свои к Яхве, и ко Всевышнему возвысил свой глас я.
\vs Ode 37:2
И говорил я
устами сердца своего, и слышал Он меня, когда глас мой достигал Его.
\vs Ode 37:3
Его же Слово
изошло ко мне, дабы воздались мне плоды трудов моих,
\vs Ode 37:4
и воздался бы
покой мне милостью Яхве.
Аллилуйя.

\vs Ode 38:1
Взошел я во
Свет Истины словно в колесницу, и Истина вела меня и велела прийти,
\vs Ode 38:2
и велела мне
пройти через пропасти и заливы и спасала меня от скал и долин,
\vs Ode 38:3
и стала для
меня небесами спасения, и водрузила меня в месте бессмертной жизни.
\vs Ode 38:4
И шел Он со
мною и велел мне отдыхать и не позволял мне оступаться, ибо был Он и является
Истиной.
\vs Ode 38:5
Мне было
безопасно, ибо я всегда шел с Ним, и не оступался ни в чем, ибо я слушался Его;
\vs Ode 38:6
ибо ошибка
сбежало от Него и никогда не сталкивалась с Ним.
\vs Ode 38:7
Но Истина шла
прямым путем, и всё, чего не смыслил я, являл Он мне:
\vs Ode 38:8
все яды Ошибки
и смертельные болезни, считавшиеся сладостными.
\vs Ode 38:9
И поражая
Искусителя, я видел, как украшена была порочная невеста, и жениха, совращающего
и совращаемого.
\vs Ode 38:10
И спросил я
Истину: Кто они? И сказала Она мне: Это обманщик и ошибка,
\vs Ode 38:11
и подражают
они Возлюбленному и Невесте Его, и понуждают мир сей оступиться и совращают его.
\vs Ode 38:12
И зовут они
многих на пир брачный, и позволяют им пить вино отравы своей,
\vs Ode 38:13
дабы вынудить
их вытошнить мудростью и знанием их, и готовят для них бессмыслицу.
\vs Ode 38:14
Затем же они
покидают их, и так они спотыкаются, словно безумные и растленные люди,
\vs Ode 38:15
ведь в них
нет ведения, да и не ищут они его.
\vs Ode 38:16
Но я сделался
мудрым, дабы не пасть в руки обманщиков, и возрадовался внутри себя, ибо истина
шла со мной.
\vs Ode 38:17
Ибо я был
создан, и жил, и был искуплен, и начала мои легли из-за руки Яхве, ибо посадил
Он меня.
\vs Ode 38:18
Ибо посадил
Он корень, и полил его, и ухаживал за ним, и благословлял его, и плоды его
пребудут вовеки.
\vs Ode 38:19
Он глубоко
врос, и пророс, и развился, и полнился, и ширился,
\vs Ode 38:20
и Яхве одним
славился, посадкой Его и выращиванием Его,
\vs Ode 38:21
и заботой
Его, и благословением уст Его, в чудесном саду одесную Его,
\vs Ode 38:22
и в знаниях
сада Его, и в понимании разума Его.
Аллилуйя.

\vs Ode 39:1
Свирепые реки
сила Яхве, они бросают головой вниз презирающих Его,
\vs Ode 39:2
и путают тропы
их, и разрушают переправы их,
\vs Ode 39:3
и хватают тела
их, и растлевают естества их.
\vs Ode 39:4
Ибо они~---
быстрее молний, даже скорее.
\vs Ode 39:5
Но не помешают
тем, кто переправляется через них с верою
\vs Ode 39:6
и не выбросят
тех, кто безупречно сплавляется по ним.
\vs Ode 39:7
Ибо знак на
них~--- Сам Яхве, и знак этот~--- путь для переправляющихся во имя Яхве.
\vs Ode 39:8
Затем же
положитесь на имя Всевышнего и познайте Его, и вы переправитесь безопасно, ибо
реки станут послушны вам.
\vs Ode 39:9
Яхве же
вымостил их Словом Своим, и он ходил и пересекал их как посуху.
\vs Ode 39:10
И твердо
отпечатывались на водах следы Его, и не стирались, но подобны были бруску древа,
на Истине выстроенного.
\vs Ode 39:11
С обеих
сторон вздымались волны, но следы нашего Яхве Помазанника оставались тверды,
\vs Ode 39:12
и ни намокали
они, ни разрушались.
\vs Ode 39:13
И путь этот
был предначертан переправляющимся следом за Ним, и строго следующим стезею веры
Его, и чтящим имя Его.
Аллилуйя.

\vs Ode 40:1
Как истекает
мёд из сот пчелиных, а молоко~--- из жены, любящей детей своих, так и надежда на
Тебя, Боже мой.
\vs Ode 40:2
Как хлынет
вода из фонтана, так и сердце моё хлынет восхвалением Яхве, и вознесут хвалу Ему
уста мои.
\vs Ode 40:3
И усладится
язык мой гимнами Его, и чресла мои помажутся одами Его.
\vs Ode 40:4
Лицо же моё
веселится ликованием Его, и дух мой ликует в любви Его, и естество моё светится
в Нем.
\vs Ode 40:5
И уверует в
Него убоявшийся, и искупления достигнет в Нем.
\vs Ode 40:6
И владения Его
жизнь бессмертная, и стяжавшие её непорочны.
Аллилуйя.

\vs Ode 41:1
Да восхвалят
Яхве все чада Яхве, да стяжаем мы истину веры Его.
\vs Ode 41:2
И дети его да
утвердятся в Нем, а поэтому воспоем же любви Его.
\vs Ode 41:3
Живем мы в
Яхве милостью Его, а жизнь стяжали мы у Помазанника Его.
\vs Ode 41:4
Ибо великий
день засветил нам, и чудесен Тот, кто воздал нам славою Своей.
\vs Ode 41:5
Пребудем же
поэтому все мы в согласии во имя Яхве и воздадим почести в доброте Его.
\vs Ode 41:6
И да воссияют
лица наши во свете Его, и да сосредоточатся сердца наши на любви Его и днем, и
ночью.
\vs Ode 41:7
Так возликуем
же ликованием Яхве!
\vs Ode 41:8
Все видящие
меня да изумятся, ибо я~--- из иного рода.
\vs Ode 41:9
Ибо Отец
Истины вспомнил обо мне, Он, от начала владеющий мною.
\vs Ode 41:10
Ибо богатства
Его породили меня, и мысль сердца Его.
\vs Ode 41:12
И пребывает с
нами Слово Его на всем пути нашем~--- Спаситель, дающий жизнь и не отвергающий
нас.
\vs Ode 41:13
Сын же
Всевышнего явился в совершенстве Отца Своего.
\vs Ode 41:14
И забрезжил
из Слова свет, пребывавший в Нем прежде времени.
\vs Ode 41:15
Помазанник же
один в истине. И знали Его прежде основ мира сего, чтобы истиной имени Своего
вовеки наделял он жизнью людей.
\vs Ode 41:16
Новая же
песнь~--- Яхве, от любящих Его.
Аллилуйя.

\vs Ode 42:1
Распростер я
руки и приблизился к Яхве, ибо распростертые руки мои~--- знак Его.
\vs Ode 42:2
И
распростертые мои~--- вертикальный крест, поднявшийся на стезе Праведного.
\vs Ode 42:3
И сделался я
бесполезным для не знавших меня, ибо скроюсь я от тех, кто не одержим мною,
\vs Ode 42:4
а пребуду с
любящими меня.
\vs Ode 42:5
Все
преследовавшие меня мертвы, искавшие меня, злословившие против меня,~--- ибо жив
я.
\vs Ode 42:6
К тому же
воскрес я, и я с ними, и буду говорят устами их.
\vs Ode 42:7
Ибо отвергли
они притеснявших их, и запечатлел их клеймом любви моей.
\vs Ode 42:8
Подобно клейму
жениха на невесте клеймо моё на знающих меня.
\vs Ode 42:9
И как чертог
брачный выстраивается родными брачной пары, так и любовь моя (живет) верующими в
меня.
\vs Ode 42:10
Меня не
отвергли, хотя и хотели, и не погиб я, хотя они и считали меня (погибшим).
\vs Ode 42:11
Шеол же
увидел меня и поколебался, и Смерть низвергла меня и многих вместе со мною.
\vs Ode 42:12
Я был уксусом
и горечью его, и спустился я с нею вниз на глубину его.
\vs Ode 42:13
Затем же
отпустил он ноги и голову, ибо был он неспособен выносить лицо моё.
\vs Ode 42:14
И создал я
собрание живых среди его мертвых и говорил с ними устами живыми, дабы слово моё
не было бесполезным.
\vs Ode 42:15
И ринулись ко
мне умершие, и возопили, и заголосили:
<<Сын Божий, смилуйся над нами,
\vs Ode 42:16
и поступи с
нами по доброте Твоей, и вызволи нас из оков тьмы,
\vs Ode 42:17
и открой нам
дверь, в которую мы сможем выйти к Тебе, ибо осознали мы, что смерть наша не
коснулась Тебя.
\vs Ode 42:18
А еще да
спасемся с Тобою, ибо Ты~--- Спаситель наш.>>
\vs Ode 42:19
Затем услышал
я голос их, и вложил веру их в сердце моё.
\vs Ode 42:20
И вложил я
имя своё в головы их, ибо свободны они и принадлежат мне.
Аллилуйя.

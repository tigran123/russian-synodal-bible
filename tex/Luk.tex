\bibbookdescr{Luk}{
  inline={От Луки\\\LARGE святое благовествование},
  toc={От Луки},
  bookmark={От Луки},
  header={От Луки},
  %headerleft={},
  %headerright={},
  abbr={Лк}
}
\vs Luk 1:1 Как уже многие начали составлять повествования о совершенно известных между нами событиях,
\vs Luk 1:2 как передали нам т\acc{о} бывшие с самого начала очевидцами и служителями Слова,
\vs Luk 1:3 то рассудилось и мне, по тщательном исследовании всего сначала, по порядку описать тебе, достопочтенный Феофил,
\vs Luk 1:4 чтобы ты узнал твердое основание того учения, в котором был наставлен.
\rsbpar\vs Luk 1:5 Во дни Ирода, царя Иудейского, был священник из Авиевой чреды, именем Захария, и жена его из рода Ааронова, имя ей Елисавета.
\vs Luk 1:6 Оба они были праведны пред Богом, поступая по всем заповедям и уставам Господним беспорочно.
\vs Luk 1:7 У них не было детей, ибо Елисавета была неплодна, и оба были уже в летах преклонных.
\vs Luk 1:8 Однажды, когда он в порядке своей чреды служил пред Богом,
\vs Luk 1:9 по жребию, как обыкновенно было у священников, досталось ему войти в храм Господень для каждения,
\vs Luk 1:10 а всё множество народа молилось вне во время каждения,~---
\vs Luk 1:11 тогда явился ему Ангел Господень, стоя по правую сторону жертвенника кадильного.
\vs Luk 1:12 Захария, увидев его, смутился, и страх напал на него.
\vs Luk 1:13 Ангел же сказал ему: не бойся, Захария, ибо услышана молитва твоя, и жена твоя Елисавета родит тебе сына, и наречешь ему имя: Иоанн;
\vs Luk 1:14 и будет тебе радость и веселие, и многие о рождении его возрадуются,
\vs Luk 1:15 ибо он будет велик пред Господом; не будет пить вина и сикера, и Духа Святаго исполнится еще от чрева матери своей;
\vs Luk 1:16 и многих из сынов Израилевых обратит к Господу Богу их;
\vs Luk 1:17 и предъидет пред Ним в духе и силе Илии, чтобы возвратить сердца отцов детям, и непокоривым образ мыслей праведников, дабы представить Господу народ приготовленный.
\vs Luk 1:18 И сказал Захария Ангелу: по чему я узн\acc{а}ю это? ибо я стар, и жена моя в летах преклонных.
\vs Luk 1:19 Ангел сказал ему в ответ: я Гавриил, предстоящий пред Богом, и послан говорить с тобою и благовестить тебе сие;
\vs Luk 1:20 и вот, ты будешь молчать и не будешь иметь возможности говорить до того дня, как это сбудется, за т\acc{о}, что ты не поверил словам моим, которые сбудутся в свое время.
\vs Luk 1:21 Между тем народ ожидал Захарию и дивился, что он медлит в храме.
\vs Luk 1:22 Он же, выйдя, не мог говорить к ним; и они поняли, что он видел видение в храме; и он объяснялся с ними знаками, и оставался нем.
\vs Luk 1:23 А когда окончились дни службы его, возвратился в дом свой.
\vs Luk 1:24 После сих дней зачала Елисавета, жена его, и таилась пять месяцев и говорила:
\vs Luk 1:25 так сотворил мне Господь во дни сии, в которые призрел на меня, чтобы снять с меня поношение между людьми.
\rsbpar\vs Luk 1:26 В шестой же месяц послан был Ангел Гавриил от Бога в город Галилейский, называемый Назарет,
\vs Luk 1:27 к Деве, обрученной мужу, именем Иосифу, из дома Давидова; имя же Деве: Мария.
\vs Luk 1:28 Ангел, войдя к Ней, сказал: радуйся, Благодатная! Господь с Тобою; благословенна Ты между женами.
\vs Luk 1:29 Она же, увидев его, смутилась от слов его и размышляла, чт\acc{о} бы это было за приветствие.
\vs Luk 1:30 И сказал Ей Ангел: не бойся, Мария, ибо Ты обрела благодать у Бога;
\vs Luk 1:31 и вот, зачнешь во чреве, и родишь Сына, и наречешь Ему имя: Иисус.
\vs Luk 1:32 Он будет велик и наречется Сыном Всевышнего, и даст Ему Господь Бог престол Давида, отца Его;
\vs Luk 1:33 и будет царствовать над домом Иакова во веки, и Царству Его не будет конца.
\vs Luk 1:34 Мария же сказала Ангелу: к\acc{а}к будет это, когда Я мужа не знаю?
\vs Luk 1:35 Ангел сказал Ей в ответ: Дух Святый найдет на Тебя, и сила Всевышнего осенит Тебя; посему и рождаемое Святое наречется Сыном Божиим.
\vs Luk 1:36 Вот и Елисавета, родственница Твоя, называемая неплодною, и она зачала сына в старости своей, и ей уже шестой месяц,
\vs Luk 1:37 ибо у Бога не останется бессильным никакое слово.
\vs Luk 1:38 Тогда Мария сказала: се, Раба Господня; да будет Мне по слову твоему. И отошел от Нее Ангел.
\rsbpar\vs Luk 1:39 Встав же Мария во дни сии, с поспешностью пошла в нагорную страну, в город Иудин,
\vs Luk 1:40 и вошла в дом Захарии, и приветствовала Елисавету.
\vs Luk 1:41 Когда Елисавета услышала приветствие Марии, взыграл младенец во чреве ее; и Елисавета исполнилась Святаго Духа,
\vs Luk 1:42 и воскликнула громким голосом, и сказала: благословенна Ты между женами, и благословен плод чрева Твоего!
\vs Luk 1:43 И откуда это мне, что пришла Матерь Господа моего ко мне?
\vs Luk 1:44 Ибо когда голос приветствия Твоего дошел до слуха моего, взыграл младенец радостно во чреве моем.
\vs Luk 1:45 И блаженна Уверовавшая, потому что совершится сказанное Ей от Господа.
\vs Luk 1:46 И сказала Мария: величит душа Моя Господа,
\vs Luk 1:47 и возрадовался дух Мой о Боге, Спасителе Моем,
\vs Luk 1:48 что призрел Он на смирение Рабы Своей, ибо отныне будут ублажать Меня все роды;
\vs Luk 1:49 что сотворил Мне величие Сильный, и свято имя Его;
\vs Luk 1:50 и милость Его в роды родов к боящимся Его;
\vs Luk 1:51 явил силу мышцы Своей; рассеял надменных помышлениями с\acc{е}рдца их;
\vs Luk 1:52 низложил сильных с престолов, и вознес смиренных;
\vs Luk 1:53 алчущих исполнил благ, и богатящихся отпустил ни с чем;
\vs Luk 1:54 воспринял Израиля, отрока Своего, воспомянув милость,
\vs Luk 1:55 к\acc{а}к говорил отцам нашим, к Аврааму и семени его до века.
\vs Luk 1:56 Пребыла же Мария с нею около трех месяцев, и возвратилась в дом свой.
\rsbpar\vs Luk 1:57 Елисавете же настало время родить, и она родила сына.
\vs Luk 1:58 И услышали соседи и родственники ее, что возвеличил Господь милость Свою над нею, и радовались с нею.
\vs Luk 1:59 В восьмой день пришли обрезать младенца и хотели назвать его, по имени отца его, Захариею.
\vs Luk 1:60 На это мать его сказала: нет, а назвать его Иоанном.
\vs Luk 1:61 И сказали ей: никого нет в родстве твоем, кто назывался бы сим именем.
\vs Luk 1:62 И спрашивали знаками у отца его, к\acc{а}к бы он хотел назвать его.
\vs Luk 1:63 Он потребовал дощечку и написал: Иоанн имя ему. И все удивились.
\vs Luk 1:64 И тотчас разрешились уста его и язык его, и он стал говорить, благословляя Бога.
\vs Luk 1:65 И был страх на всех живущих вокруг них; и рассказывали обо всем этом по всей нагорной стране Иудейской.
\vs Luk 1:66 Все слышавшие положили это на сердце своем и говорили: чт\acc{о} будет младенец сей? И рука Господня была с ним.
\vs Luk 1:67 И Захария, отец его, исполнился Святаго Духа и пророчествовал, говоря:
\vs Luk 1:68 благословен Господь Бог Израилев, что посетил народ Свой и сотворил избавление ему,
\vs Luk 1:69 и воздвиг рог спасения нам в дому Давида, отрока Своего,
\vs Luk 1:70 к\acc{а}к возвестил устами бывших от века святых пророков Своих,
\vs Luk 1:71 что спасет нас от врагов наших и от руки всех ненавидящих нас;
\vs Luk 1:72 сотворит милость с отцами нашими и помянет святой завет Свой,
\vs Luk 1:73 клятву, которою клялся Он Аврааму, отцу нашему, дать нам,
\vs Luk 1:74 небоязненно, по избавлении от руки врагов наших,
\vs Luk 1:75 служить Ему в святости и правде пред Ним, во все дни жизни нашей.
\vs Luk 1:76 И ты, младенец, наречешься пророком Всевышнего, ибо предъидешь пред лицем Господа приготовить пути Ему,
\vs Luk 1:77 дать уразуметь народу Его спасение в прощении грехов их,
\vs Luk 1:78 по благоутробному милосердию Бога нашего, которым посетил нас Восток свыше,
\vs Luk 1:79 просветить сидящих во тьме и тени смертной, направить ноги наши на путь мира.
\rsbpar\vs Luk 1:80 Младенец же возрастал и укреплялся духом, и был в пустынях до дня явления своего Израилю.
\vs Luk 2:1 В те дни вышло от кесаря Августа повеление сделать перепись по всей земле.
\vs Luk 2:2 Эта перепись была первая в правление Квириния Сириею.
\vs Luk 2:3 И пошли все записываться, каждый в свой город.
\vs Luk 2:4 Пошел также и Иосиф из Галилеи, из города Назарета, в Иудею, в город Давидов, называемый Вифлеем, потому что он был из дома и рода Давидова,
\vs Luk 2:5 записаться с Мариею, обрученною ему женою, которая была беременна.
\vs Luk 2:6 Когда же они были там, наступило время родить Ей;
\vs Luk 2:7 и родила Сына своего Первенца, и спеленала Его, и положила Его в ясли, потому что не было им места в гостинице.
\rsbpar\vs Luk 2:8 В той стране были на поле пастухи, которые содержали ночную стражу у стада своего.
\vs Luk 2:9 Вдруг предстал им Ангел Господень, и слава Господня осияла их; и убоялись страхом великим.
\vs Luk 2:10 И сказал им Ангел: не бойтесь; я возвещаю вам великую радость, которая будет всем людям:
\vs Luk 2:11 ибо ныне родился вам в городе Давидовом Спаситель, Который есть Христос Господь;
\vs Luk 2:12 и вот вам знак: вы найдете Младенца в пеленах, лежащего в яслях.
\vs Luk 2:13 И внезапно явилось с Ангелом многочисленное воинство небесное, славящее Бога и взывающее:
\vs Luk 2:14 слава в вышних Богу, и на земле мир, в человеках благоволение!
\vs Luk 2:15 Когда Ангелы отошли от них на небо, пастухи сказали друг другу: пойдем в Вифлеем и посмотрим, чт\acc{о} там случилось, о чем возвестил нам Господь.
\vs Luk 2:16 И, поспешив, пришли и нашли Марию и Иосифа, и Младенца, лежащего в яслях.
\vs Luk 2:17 Увидев же, рассказали о том, чт\acc{о} было возвещено им о Младенце Сем.
\vs Luk 2:18 И все слышавшие дивились тому, чт\acc{о} рассказывали им пастухи.
\vs Luk 2:19 А Мария сохраняла все слова сии, слагая в сердце Своем.
\vs Luk 2:20 И возвратились пастухи, славя и хваля Бога за всё т\acc{о}, что слышали и видели, к\acc{а}к им сказано было.
\rsbpar\vs Luk 2:21 По прошествии восьми дней, когда надлежало обрезать \bibemph{Младенца}, дали Ему имя Иисус, нареченное Ангелом прежде зачатия Его во чреве.
\rsbpar\vs Luk 2:22 А когда исполнились дни очищения их по закону Моисееву, принесли Его в Иерусалим, чтобы представить пред Господа,
\vs Luk 2:23 как предписано в законе Господнем, чтобы всякий младенец мужеского пола, разверзающий ложесна, был посвящен Господу,
\vs Luk 2:24 и чтобы принести в жертву, по реченному в законе Господнем, две горлицы или двух птенцов голубиных.
\vs Luk 2:25 Тогда был в Иерусалиме человек, именем Симеон. Он был муж праведный и благочестивый, чающий утешения Израилева; и Дух Святый был на нем.
\vs Luk 2:26 Ему было предсказано Духом Святым, что он не увидит смерти, доколе не увидит Христа Господня.
\vs Luk 2:27 И пришел он по вдохновению в храм. И, когда родители принесли Младенца Иисуса, чтобы совершить над Ним законный обряд,
\vs Luk 2:28 он взял Его на руки, благословил Бога и сказал:
\rsbpar\vs Luk 2:29 Ныне отпускаешь раба Твоего, Владыко, по слову Твоему, с миром,
\vs Luk 2:30 ибо видели очи мои спасение Твое,
\vs Luk 2:31 которое Ты уготовал пред лицем всех народов,
\vs Luk 2:32 свет к просвещению язычников и славу народа Твоего Израиля.
\rsbpar\vs Luk 2:33 Иосиф же и Матерь Его дивились сказанному о Нем.
\vs Luk 2:34 И благословил их Симеон и сказал Марии, Матери Его: се, лежит Сей на падение и на восстание многих в Израиле и в предмет пререканий,~---
\vs Luk 2:35 и Тебе Самой оружие пройдет душу,~--- да откроются помышления многих сердец.
\vs Luk 2:36 Тут была также Анна пророчица, дочь Фануилова, от колена Асирова, достигшая глубокой старости, прожив с мужем от девства своего семь лет,
\vs Luk 2:37 вдова лет восьмидесяти четырех, которая не отходила от храма, постом и молитвою служа Богу день и ночь.
\vs Luk 2:38 И она в то время, подойдя, славила Господа и говорила о Нем всем, ожидавшим избавления в Иерусалиме.
\rsbpar\vs Luk 2:39 И когда они совершили всё по закону Господню, возвратились в Галилею, в город свой Назарет.
\vs Luk 2:40 Младенец же возрастал и укреплялся духом, исполняясь премудрости, и благодать Божия была на Нем.
\vs Luk 2:41 Каждый год родители Его ходили в Иерусалим на праздник Пасхи.
\vs Luk 2:42 И когда Он был двенадцати лет, пришли они также по обычаю в Иерусалим на праздник.
\vs Luk 2:43 Когда же, по окончании дней \bibemph{праздника}, возвращались, остался Отрок Иисус в Иерусалиме; и не заметили того Иосиф и Матерь Его,
\vs Luk 2:44 но думали, что Он идет с другими. Пройдя же дневной путь, стали искать Его между родственниками и знакомыми
\vs Luk 2:45 и, не найдя Его, возвратились в Иерусалим, ища Его.
\vs Luk 2:46 Через три дня нашли Его в храме, сидящего посреди учителей, слушающего их и спрашивающего их;
\vs Luk 2:47 все слушавшие Его дивились разуму и ответам Его.
\vs Luk 2:48 И, увидев Его, удивились; и Матерь Его сказала Ему: Чадо! чт\acc{о} Ты сделал с нами? Вот, отец Твой и Я с великою скорбью искали Тебя.
\vs Luk 2:49 Он сказал им: зачем было вам искать Меня? или вы не знали, что Мне должно быть в том, чт\acc{о} принадлежит Отцу Моему?
\vs Luk 2:50 Но они не поняли сказанных Им слов.
\vs Luk 2:51 И Он пошел с ними и пришел в Назарет; и был в повиновении у них. И Матерь Его сохраняла все слова сии в сердце Своем.
\vs Luk 2:52 Иисус же преуспевал в премудрости и возрасте и в любви у Бога и человеков.
\vs Luk 3:1 В пятнадцатый же год правления Тиверия кесаря, когда Понтий Пилат начальствовал в Иудее, Ирод был четвертовластником в Галилее, Филипп, брат его, четвертовластником в Итурее и Трахонитской области, а Лисаний четвертовластником в Авилинее,
\vs Luk 3:2 при первосвященниках Анне и Каиафе, был глагол Божий к Иоанну, сыну Захарии, в пустыне.
\vs Luk 3:3 И он проходил по всей окрестной стране Иорданской, проповедуя крещение покаяния для прощения грехов,
\vs Luk 3:4 как написано в книге слов пророка Исаии, который говорит: глас вопиющего в пустыне: приготовьте путь Господу, прямыми сделайте стези Ему;
\vs Luk 3:5 всякий дол да наполнится, и всякая гора и холм да понизятся, кривизны выпрямятся и неровные пути сделаются гладкими;
\vs Luk 3:6 и узрит всякая плоть спасение Божие.
\vs Luk 3:7 \bibemph{Иоанн} приходившему креститься от него народу говорил: порождения ехиднины! кто внушил вам бежать от будущего гнева?
\vs Luk 3:8 Сотворите же достойные плоды покаяния и не думайте говорить в себе: отец у нас Авраам, ибо говорю вам, что Бог может из камней сих воздвигнуть детей Аврааму.
\vs Luk 3:9 Уже и секира при корне дерев лежит: всякое дерево, не приносящее доброго плода, срубают и бросают в огонь.
\vs Luk 3:10 И спрашивал его народ: что же нам делать?
\vs Luk 3:11 Он сказал им в ответ: у кого две одежды, тот дай неимущему, и у кого есть пища, делай то же.
\vs Luk 3:12 Пришли и мытари креститься, и сказали ему: учитель! что нам делать?
\vs Luk 3:13 Он отвечал им: ничего не требуйте более определенного вам.
\vs Luk 3:14 Спрашивали его также и воины: а нам что делать? И сказал им: никого не обижайте, не клевещите, и довольствуйтесь своим жалованьем.
\vs Luk 3:15 Когда же народ был в ожидании, и все помышляли в сердцах своих об Иоанне, не Христос ли он,~---
\vs Luk 3:16 Иоанн всем отвечал: я крещу вас водою, но идёт Сильнейший меня, у Которого я недостоин развязать ремень обуви; Он будет крестить вас Духом Святым и огнем.
\vs Luk 3:17 Лопата Его в руке Его, и Он очистит гумно Свое и соберет пшеницу в житницу Свою, а солому сожжет огнем неугасимым.
\vs Luk 3:18 Многое и другое благовествовал он народу, поучая его.
\rsbpar\vs Luk 3:19 Ирод же четвертовластник, обличаемый от него за Иродиаду, жену брата своего, и за всё, что сделал Ирод худого,
\vs Luk 3:20 прибавил ко всему прочему и т\acc{о}, что заключил Иоанна в темницу.
\rsbpar\vs Luk 3:21 Когда же крестился весь народ, и Иисус, крестившись, молился: отверзлось небо,
\vs Luk 3:22 и Дух Святый нисшел на Него в телесном виде, как голубь, и был глас с небес, глаголющий: Ты Сын Мой Возлюбленный; в Тебе Мое благоволение!
\rsbpar\vs Luk 3:23 Иисус, начиная \bibemph{Своё служение}, был лет тридцати, и был, как думали, Сын Иосифов, Илиев,
\vs Luk 3:24 Матфатов, Левиин, Мелхиев, Ианнаев, Иосифов,
\vs Luk 3:25 Маттафиев, Амосов, Наумов, Еслимов, Наггеев,
\vs Luk 3:26 Маафов, Маттафиев, Семеиев, Иосифов, Иудин,
\vs Luk 3:27 Иоаннанов, Рисаев, Зоровавелев, Салафиилев, Нириев,
\vs Luk 3:28 Мелхиев, Аддиев, Косамов, Елмодамов, Иров,
\vs Luk 3:29 Иосиев, Елиезеров, Иоримов, Матфатов, Левиин,
\vs Luk 3:30 Симеонов, Иудин, Иосифов, Ионанов, Елиакимов,
\vs Luk 3:31 Мелеаев, Маинанов, Маттафаев, Нафанов, Давидов,
\vs Luk 3:32 Иессеев, Овидов, Воозов, Салмонов, Наассонов,
\vs Luk 3:33 Аминадавов, Арамов, Есромов, Фаресов, Иудин,
\vs Luk 3:34 Иаковлев, Исааков, Авраамов, Фаррин, Нахоров,
\vs Luk 3:35 Серухов, Рагавов, Фалеков, Еверов, Салин,
\vs Luk 3:36 Каинанов, Арфаксадов, Симов, Ноев, Ламехов,
\vs Luk 3:37 Мафусалов, Енохов, Иаредов, Малелеилов, Каинанов,
\vs Luk 3:38 Еносов, Сифов, Адамов, Божий.
\vs Luk 4:1 Иисус, исполненный Духа Святаго, возвратился от Иордана и поведен был Духом в пустыню.
\vs Luk 4:2 Там сорок дней Он был искушаем от диавола и ничего не ел в эти дни, а по прошествии их напоследок взалкал.
\vs Luk 4:3 И сказал Ему диавол: если Ты Сын Божий, то вели этому камню сделаться хлебом.
\vs Luk 4:4 Иисус сказал ему в ответ: написано, что не хлебом одним будет жить человек, но всяким словом Божиим.
\vs Luk 4:5 И, возведя Его на высокую гору, диавол показал Ему все царства вселенной во мгновение времени,
\vs Luk 4:6 и сказал Ему диавол: Тебе дам власть над всеми сими \bibemph{царствами} и славу их, ибо она предана мне, и я, кому хочу, даю ее;
\vs Luk 4:7 итак, если Ты поклонишься мне, то всё будет Твое.
\vs Luk 4:8 Иисус сказал ему в ответ: отойди от Меня, сатана; написано: Господу Богу твоему поклоняйся, и Ему одному служи.
\vs Luk 4:9 И повел Его в Иерусалим, и поставил Его на крыле храма, и сказал Ему: если Ты Сын Божий, бросься отсюда вниз,
\vs Luk 4:10 ибо написано: Ангелам Своим заповедает о Тебе сохранить Тебя;
\vs Luk 4:11 и на руках понесут Тебя, да не преткнешься о камень ногою Твоею.
\vs Luk 4:12 Иисус сказал ему в ответ: сказано: не искушай Господа Бога твоего.
\vs Luk 4:13 И, окончив всё искушение, диавол отошел от Него до времени.
\rsbpar\vs Luk 4:14 И возвратился Иисус в силе Духа в Галилею; и разнеслась молва о Нем по всей окрестной стране.
\vs Luk 4:15 Он учил в синагогах их, и от всех был прославляем.
\rsbpar\vs Luk 4:16 И пришел в Назарет, где был воспитан, и вошел, по обыкновению Своему, в день субботний в синагогу, и встал читать.
\vs Luk 4:17 Ему подали книгу пророка Исаии; и Он, раскрыв книгу, нашел место, где было написано:
\vs Luk 4:18 Дух Господень на Мне; ибо Он помазал Меня благовествовать нищим, и послал Меня исцелять сокрушенных сердцем, проповедовать пленным освобождение, слепым прозрение, отпустить измученных на свободу,
\vs Luk 4:19 проповедовать лето Господне благоприятное.
\vs Luk 4:20 И, закрыв книгу и отдав служителю, сел; и глаза всех в синагоге были устремлены на Него.
\vs Luk 4:21 И Он начал говорить им: ныне исполнилось писание сие, слышанное вами.
\vs Luk 4:22 И все засвидетельствовали Ему это, и дивились словам благодати, исходившим из уст Его, и говорили: не Иосифов ли это сын?
\vs Luk 4:23 Он сказал им: конечно, вы скажете Мне присловие: врач! исцели Самого Себя; сделай и здесь, в Твоем отечестве, т\acc{о}, чт\acc{о}, мы слышали, было в Капернауме.
\vs Luk 4:24 И сказал: истинно говорю вам: никакой пророк не принимается в своем отечестве.
\vs Luk 4:25 Поистине говорю вам: много вдов было в Израиле во дни Илии, когда заключено было небо три года и шесть месяцев, так что сделался большой голод по всей земле,
\vs Luk 4:26 и ни к одной из них не был послан Илия, а только ко вдове в Сарепту Сидонскую;
\vs Luk 4:27 много также было прокаженных в Израиле при пророке Елисее, и ни один из них не очистился, кроме Неемана Сириянина.
\vs Luk 4:28 Услышав это, все в синагоге исполнились ярости
\vs Luk 4:29 и, встав, выгнали Его вон из города и повели на вершину горы, на которой город их был построен, чтобы свергнуть Его;
\vs Luk 4:30 но Он, пройдя посреди них, удалился.
\rsbpar\vs Luk 4:31 И пришел в Капернаум, город Галилейский, и учил их в дни субботние.
\vs Luk 4:32 И дивились учению Его, ибо слово Его было со властью.
\vs Luk 4:33 Был в синагоге человек, имевший нечистого духа бесовского, и он закричал громким голосом:
\vs Luk 4:34 оставь; чт\acc{о} Тебе до нас, Иисус Назарянин? Ты пришел погубить нас; знаю Тебя, кто Ты, Святый Божий.
\vs Luk 4:35 Иисус запретил ему, сказав: замолчи и выйди из него. И бес, повергнув его посреди \bibemph{синагоги}, вышел из него, нимало не повредив ему.
\vs Luk 4:36 И напал на всех ужас, и рассуждали между собою: что это значит, что Он со властью и силою повелевает нечистым духам, и они выходят?
\vs Luk 4:37 И разнесся слух о Нем по всем окрестным местам.
\rsbpar\vs Luk 4:38 Выйдя из синагоги, Он вошел в дом Симона; тёща же Симонова была одержима сильною горячкою; и просили Его о ней.
\vs Luk 4:39 Подойдя к ней, Он запретил горячке; и оставила ее. Она тотчас встала и служила им.
\vs Luk 4:40 При захождении же солнца все, имевшие больных различными болезнями, приводили их к Нему и Он, возлагая на каждого из них руки, исцелял их.
\vs Luk 4:41 Выходили также и бесы из многих с криком и говорили: Ты Христос, Сын Божий. А Он запрещал им сказывать, что они знают, что Он Христос.
\rsbpar\vs Luk 4:42 Когда же настал день, Он, выйдя \bibemph{из дома}, пошел в пустынное место, и народ искал Его и, придя к Нему, удерживал Его, чтобы не уходил от них.
\vs Luk 4:43 Но Он сказал им: и другим городам благовествовать Я должен Царствие Божие, ибо на то Я послан.
\vs Luk 4:44 И проповедовал в синагогах галилейских.
\vs Luk 5:1 Однажды, когда народ теснился к Нему, чтобы слышать слово Божие, а Он стоял у озера Геннисаретского,
\vs Luk 5:2 увидел Он две лодки, стоящие на озере; а рыболовы, выйдя из них, вымывали сети.
\vs Luk 5:3 Войдя в одну лодку, которая была Симонова, Он просил его отплыть несколько от берега и, сев, учил народ из лодки.
\vs Luk 5:4 Когда же перестал учить, сказал Симону: отплыви на глубину и закиньте сети свои для лова.
\vs Luk 5:5 Симон сказал Ему в ответ: Наставник! мы трудились всю ночь и ничего не поймали, но по слову Твоему закину сеть.
\vs Luk 5:6 Сделав это, они поймали великое множество рыбы, и даже сеть у них прорывалась.
\vs Luk 5:7 И дали знак товарищам, находившимся на другой лодке, чтобы пришли помочь им; и пришли, и наполнили обе лодки, так что они начинали тонуть.
\vs Luk 5:8 Увидев это, Симон Петр припал к коленям Иисуса и сказал: выйди от меня, Господи! потому что я человек грешный.
\vs Luk 5:9 Ибо ужас объял его и всех, бывших с ним, от этого лова рыб, ими пойманных;
\vs Luk 5:10 также и Иакова и Иоанна, сыновей Зеведеевых, бывших товарищами Симону. И сказал Симону Иисус: не бойся; отныне будешь ловить человеков.
\vs Luk 5:11 И, вытащив обе лодки на берег, оставили всё и последовали за Ним.
\rsbpar\vs Luk 5:12 Когда Иисус был в одном городе, пришел человек весь в проказе и, увидев Иисуса, пал ниц, умоляя Его и говоря: Господи! если хочешь, можешь меня очистить.
\vs Luk 5:13 Он простер руку, прикоснулся к нему и сказал: хочу, очистись. И тотчас проказа сошла с него.
\vs Luk 5:14 И Он повелел ему никому не сказывать, а пойти показаться священнику и принести \bibemph{жертву} за очищение свое, к\acc{а}к повелел Моисей, во свидетельство им.
\vs Luk 5:15 Но тем более распространялась молва о Нём, и великое множество народа стекалось к Нему слушать и врачеваться у Него от болезней своих.
\vs Luk 5:16 Но Он уходил в пустынные места и молился.
\rsbpar\vs Luk 5:17 В один день, когда Он учил, и сидели тут фарисеи и законоучители, пришедшие из всех мест Галилеи и Иудеи и из Иерусалима, и сила Господня являлась в исцелении \bibemph{больных},~---
\vs Luk 5:18 вот, принесли некоторые на постели человека, который был расслаблен, и старались внести его \bibemph{в дом} и положить перед Иисусом;
\vs Luk 5:19 и, не найдя, где пронести его за многолюдством, влезли на верх дома и сквозь кровлю спустили его с постелью на средину пред Иисуса.
\vs Luk 5:20 И Он, видя веру их, сказал человеку тому: прощаются тебе грехи твои.
\vs Luk 5:21 Книжники и фарисеи начали рассуждать, говоря: кто это, который богохульствует? кто может прощать грехи, кроме одного Бога?
\vs Luk 5:22 Иисус, уразумев помышления их, сказал им в ответ: чт\acc{о} вы помышляете в сердцах ваших?
\vs Luk 5:23 Чт\acc{о} легче сказать: прощаются тебе грехи твои, или сказать: встань и ходи?
\vs Luk 5:24 Но чтобы вы знали, что Сын Человеческий имеет власть на земле прощать грехи,~--- сказал Он расслабленному: тебе говорю: встань, возьми постель твою и иди в дом твой.
\vs Luk 5:25 И он тотчас встал перед ними, взял, на чём лежал, и пошел в дом свой, славя Бога.
\vs Luk 5:26 И ужас объял всех, и славили Бога и, быв исполнены страха, говорили: ч\acc{у}дные дела видели мы ныне.
\rsbpar\vs Luk 5:27 После сего \bibemph{Иисус} вышел и увидел мытаря, именем Левия, сидящего у сбора пошлин, и говорит ему: следуй за Мною.
\vs Luk 5:28 И он, оставив всё, встал и последовал за Ним.
\vs Luk 5:29 И сделал для Него Левий в доме своем большое угощение; и там было множество мытарей и других, которые возлежали с ними.
\vs Luk 5:30 Книжники же и фарисеи роптали и говорили ученикам Его: зачем вы едите и пьете с мытарями и грешниками?
\vs Luk 5:31 Иисус же сказал им в ответ: не здоровые имеют нужду во враче, но больные;
\vs Luk 5:32 Я пришел призвать не праведников, а грешников к покаянию.
\vs Luk 5:33 Они же сказали Ему: почему ученики Иоанновы постятся часто и молитвы творят, также и фарисейские, а Твои едят и пьют?
\vs Luk 5:34 Он сказал им: можете ли заставить сынов чертога брачного поститься, когда с ними жених?
\vs Luk 5:35 Но придут дни, когда отнимется у них жених, и тогда будут поститься в те дни.
\vs Luk 5:36 При сем сказал им притчу: никто не приставляет заплаты к ветхой одежде, отодрав от новой одежды; а иначе и новую раздерет, и к старой не подойдет заплата от новой.
\vs Luk 5:37 И никто не вливает молодого вина в мехи ветхие; а иначе молодое вино прорвет мехи, и само вытечет, и мехи пропадут;
\vs Luk 5:38 но молодое вино должно вливать в мехи новые; тогда сбережется и т\acc{о} и другое.
\vs Luk 5:39 И никто, пив старое \bibemph{вино}, не захочет тотчас молодого, ибо говорит: старое лучше.
\vs Luk 6:1 В субботу, первую по втором дне Пасхи, случилось Ему проходить засеянными полями, и ученики Его срывали колосья и ели, растирая руками.
\vs Luk 6:2 Некоторые же из фарисеев сказали им: зачем вы делаете то, чего не должно делать в субботы?
\vs Luk 6:3 Иисус сказал им в ответ: разве вы не читали, что сделал Давид, когда взалкал сам и бывшие с ним?
\vs Luk 6:4 К\acc{а}к он вошел в дом Божий, взял хлебы предложения, которых не должно было есть никому, кроме одних священников, и ел, и дал бывшим с ним?
\vs Luk 6:5 И сказал им: Сын Человеческий есть господин и субботы.
\rsbpar\vs Luk 6:6 Случилось же и в другую субботу войти Ему в синагогу и учить. Там был человек, у которого правая рука была сухая.
\vs Luk 6:7 Книжники же и фарисеи наблюдали за Ним, не исцелит ли в субботу, чтобы найти обвинение против Него.
\vs Luk 6:8 Но Он, зная помышления их, сказал человеку, имеющему сухую руку: встань и выступи на средину. И он встал и выступил.
\vs Luk 6:9 Тогда сказал им Иисус: спрошу Я вас: чт\acc{о} должно делать в субботу? добро, или зло? спасти душу, или погубить? Они молчали.
\vs Luk 6:10 И, посмотрев на всех их, сказал тому человеку: протяни руку твою. Он так и сделал; и стала рука его здорова, как другая.
\vs Luk 6:11 Они же пришли в бешенство и говорили между собою, чт\acc{о} бы им сделать с Иисусом.
\rsbpar\vs Luk 6:12 В те дни взошел Он на гору помолиться и пробыл всю ночь в молитве к Богу.
\vs Luk 6:13 Когда же настал день, призвал учеников Своих и избрал из них двенадцать, которых и наименовал Апостолами:
\vs Luk 6:14 Симона, которого и назвал Петром, и Андрея, брата его, Иакова и Иоанна, Филиппа и Варфоломея,
\vs Luk 6:15 Матфея и Фому, Иакова Алфеева и Симона, прозываемого Зилотом,
\vs Luk 6:16 Иуду Иаковлева и Иуду Искариота, который потом сделался предателем.
\rsbpar\vs Luk 6:17 И, сойдя с ними, стал Он на ровном месте, и множество учеников Его, и много народа из всей Иудеи и Иерусалима и приморских мест Тирских и Сидонских,
\vs Luk 6:18 которые пришли послушать Его и исцелиться от болезней своих, также и страждущие от нечистых духов; и исцелялись.
\vs Luk 6:19 И весь народ искал прикасаться к Нему, потому что от Него исходила сила и исцеляла всех.
\vs Luk 6:20 И Он, возведя очи Свои на учеников Своих, говорил:\rsbpar Блаженны нищие духом, ибо ваше есть Царствие Божие.
\rsbpar\vs Luk 6:21 Блаженны алчущие ныне, ибо насытитесь.\rsbpar Блаженны плачущие ныне, ибо воссмеетесь.
\rsbpar\vs Luk 6:22 Блаженны вы, когда возненавидят вас люди и когда отлучат вас, и будут поносить, и пронесут имя ваше, как бесчестное, за Сына Человеческого.
\vs Luk 6:23 Возрадуйтесь в тот день и возвеселитесь, ибо велика вам награда на небесах. Так поступали с пророками отцы их.
\rsbpar\vs Luk 6:24 Напротив, горе вам, богатые! ибо вы уже получили свое утешение.
\vs Luk 6:25 Горе вам, пресыщенные ныне! ибо взалчете. Горе вам, смеющиеся ныне! ибо восплачете и возрыдаете.
\vs Luk 6:26 Горе вам, когда все люди будут говорить о вас хорошо! ибо так поступали с лжепророками отцы их.
\rsbpar\vs Luk 6:27 Но вам, слушающим, говорю: люб\acc{и}те врагов ваших, благотворите ненавидящим вас,
\vs Luk 6:28 благословляйте проклинающих вас и молитесь за обижающих вас.
\vs Luk 6:29 Ударившему тебя по щеке подставь и другую, и отнимающему у тебя верхнюю одежду не препятствуй взять и рубашку.
\vs Luk 6:30 Всякому, просящему у тебя, давай, и от взявшего твое не требуй назад.
\vs Luk 6:31 И к\acc{а}к хотите, чтобы с вами поступали люди, т\acc{а}к и вы поступайте с ними.
\vs Luk 6:32 И если любите любящих вас, какая вам за то благодарность? ибо и грешники любящих их любят.
\vs Luk 6:33 И если делаете добро тем, которые вам делают добро, какая вам за то благодарность? ибо и грешники т\acc{о} же делают.
\vs Luk 6:34 И если взаймы даёте тем, от которых надеетесь получить обратно, какая вам за то благодарность? ибо и грешники дают взаймы грешникам, чтобы получить обратно столько же.
\vs Luk 6:35 Но вы люб\acc{и}те врагов ваших, и благотворите, и взаймы давайте, не ожидая ничего; и будет вам награда великая, и будете сынами Всевышнего; ибо Он благ и к неблагодарным и злым.
\vs Luk 6:36 Итак, будьте милосерды, как и Отец ваш милосерд.
\rsbpar\vs Luk 6:37 Не суд\acc{и}те, и не будете судимы; не осуждайте, и не будете осуждены; прощайте, и прощены будете;
\vs Luk 6:38 давайте, и дастся вам: мерою доброю, утрясенною, нагнетенною и переполненною отсыплют вам в лоно ваше; ибо, какою мерою мерите, такою же отмерится и вам.
\vs Luk 6:39 Сказал также им притчу: может ли слепой водить слепого? не оба ли упадут в яму?
\vs Luk 6:40 Ученик не бывает выше своего учителя; но, и усовершенствовавшись, будет всякий, как учитель его.
\vs Luk 6:41 Что ты смотришь на сучок в глазе брата твоего, а бревна в твоем глазе не чувствуешь?
\vs Luk 6:42 Или, как можешь сказать брату твоему: брат! дай, я выну сучок из глаза твоего, когда сам не видишь бревна в твоем глазе? Лицемер! вынь прежде бревно из твоего глаза, и тогда увидишь, как вынуть сучок из глаза брата твоего.
\vs Luk 6:43 Нет доброго дерева, которое приносило бы худой плод; и нет худого дерева, которое приносило бы плод добрый,
\vs Luk 6:44 ибо всякое дерево познаётся по плоду своему, потому что не собирают смокв с терновника и не снимают винограда с кустарника.
\vs Luk 6:45 Добрый человек из доброго сокровища сердца своего выносит доброе, а злой человек из злого сокровища сердца своего выносит злое, ибо от избытка сердца говорят уста его.
\rsbpar\vs Luk 6:46 Чт\acc{о} вы зовете Меня: Господи! Господи!~--- и не делаете того, чт\acc{о} Я говорю?
\vs Luk 6:47 Всякий, приходящий ко Мне и слушающий слова Мои и исполняющий их, скажу вам, кому подобен.
\vs Luk 6:48 Он подобен человеку, строящему дом, который копал, углубился и положил основание на камне; почему, когда случилось наводнение и вода напёрла на этот дом, то не могла поколебать его, потому что он основан был на камне.
\vs Luk 6:49 А слушающий и неисполняющий подобен человеку, построившему дом на земле без основания, который, когда напёрла на него вода, тотчас обрушился; и разрушение дома сего было великое.
\vs Luk 7:1 Когда Он окончил все слова Свои к слушавшему народу, то вошел в Капернаум.
\vs Luk 7:2 У одного сотника слуга, которым он дорожил, был болен при смерти.
\vs Luk 7:3 Услышав об Иисусе, он послал к Нему Иудейских старейшин просить Его, чтобы пришел исцелить слугу его.
\vs Luk 7:4 И они, придя к Иисусу, просили Его убедительно, говоря: он достоин, чтобы Ты сделал для него это,
\vs Luk 7:5 ибо он любит народ наш и построил нам синагогу.
\vs Luk 7:6 Иисус пошел с ними. И когда Он недалеко уже был от дома, сотник прислал к Нему друзей сказать Ему: не трудись, Господи! ибо я недостоин, чтобы Ты вошел под кров мой;
\vs Luk 7:7 потому и себя самого не почел я достойным прийти к Тебе; но скажи слово, и выздоровеет слуга мой.
\vs Luk 7:8 Ибо я и подвластный человек, но, имея у себя в подчинении воинов, говорю одному: пойди, и идет; и другому: приди, и приходит; и слуге моему: сделай т\acc{о}, и делает.
\vs Luk 7:9 Услышав сие, Иисус удивился ему и, обратившись, сказал идущему за Ним народу: сказываю вам, что и в Израиле не нашел Я такой веры.
\vs Luk 7:10 Посланные, возвратившись в дом, нашли больного слугу выздоровевшим.
\rsbpar\vs Luk 7:11 После сего Иисус пошел в город, называемый Наин; и с Ним шли многие из учеников Его и множество народа.
\vs Luk 7:12 Когда же Он приблизился к городским воротам, тут выносили умершего, единственного сына у матери, а она была вдова; и много народа шло с нею из города.
\vs Luk 7:13 Увидев ее, Господь сжалился над нею и сказал ей: не плачь.
\vs Luk 7:14 И, подойдя, прикоснулся к одру; несшие остановились, и Он сказал: юноша! тебе говорю, встань!
\vs Luk 7:15 Мертвый, поднявшись, сел и стал говорить; и отдал его \bibemph{Иисус} матери его.
\vs Luk 7:16 И всех объял страх, и славили Бога, говоря: великий пророк восстал между нами, и Бог посетил народ Свой.
\vs Luk 7:17 Такое мнение о Нём распространилось по всей Иудее и по всей окрестности.
\rsbpar\vs Luk 7:18 И возвестили Иоанну ученики его о всём том.
\vs Luk 7:19 Иоанн, призвав двоих из учеников своих, послал к Иисусу спросить: Ты ли Тот, Который должен прийти, или ожидать нам другого?
\vs Luk 7:20 Они, придя к \bibemph{Иисусу}, сказали: Иоанн Креститель послал нас к Тебе спросить: Ты ли Тот, Которому должно прийти, или другого ожидать нам?
\vs Luk 7:21 А в это время Он многих исцелил от болезней и недугов и от злых духов, и многим слепым даровал зрение.
\vs Luk 7:22 И сказал им Иисус в ответ: пойдите, скажите Иоанну, чт\acc{о} вы видели и слышали: слепые прозревают, хромые ходят, прокаженные очищаются, глухие слышат, мертвые воскресают, нищие благовествуют;
\vs Luk 7:23 и блажен, кто не соблазнится о Мне!
\rsbpar\vs Luk 7:24 По отшествии же посланных Иоанном, начал говорить к народу об Иоанне: чт\acc{о} смотреть ходили вы в пустыню? трость ли, ветром колеблемую?
\vs Luk 7:25 Чт\acc{о} же смотреть ходили вы? человека ли, одетого в мягкие одежды? Но одевающиеся пышно и роскошно живущие находятся при дворах царских.
\vs Luk 7:26 Чт\acc{о} же смотреть ходили вы? пророка ли? Да, говорю вам, и больше пророка.
\vs Luk 7:27 Сей есть, о котором написано: вот, Я посылаю Ангела Моего пред лицем Твоим, который приготовит путь Твой пред Тобою.
\vs Luk 7:28 Ибо говорю вам: из рожденных женами нет ни одного пророка больше Иоанна Крестителя; но меньший в Царствии Божием больше его.
\vs Luk 7:29 И весь народ, слушавший \bibemph{Его}, и мытари воздали славу Богу, крестившись крещением Иоанновым;
\vs Luk 7:30 а фарисеи и законники отвергли волю Божию о себе, не крестившись от него.
\vs Luk 7:31 Тогда Господь сказал: с кем сравню людей рода сего? и кому они подобны?
\vs Luk 7:32 Они подобны детям, которые сидят на улице, кличут друг друга и говорят: мы играли вам на свирели, и вы не плясали; мы пели вам плачевные песни, и вы не плакали.
\vs Luk 7:33 Ибо пришел Иоанн Креститель: ни хлеба не ест, ни вина не пьет; и говорите: в нем бес.
\vs Luk 7:34 Пришел Сын Человеческий: ест и пьет; и говорите: вот человек, который любит есть и пить вино, друг мытарям и грешникам.
\vs Luk 7:35 И оправдана премудрость всеми чадами ее.
\rsbpar\vs Luk 7:36 Некто из фарисеев просил Его вкусить с ним пищи; и Он, войдя в дом фарисея, возлег.
\vs Luk 7:37 И вот, женщина того города, которая была грешница, узнав, что Он возлежит в доме фарисея, принесла алавастровый сосуд с миром
\vs Luk 7:38 и, став позади у ног Его и плача, начала обливать ноги Его слезами и отирать волосами головы своей, и целовала ноги Его, и мазала миром.
\vs Luk 7:39 Видя это, фарисей, пригласивший Его, сказал сам в себе: если бы Он был пророк, то знал бы, кто и какая женщина прикасается к Нему, ибо она грешница.
\vs Luk 7:40 Обратившись к нему, Иисус сказал: Симон! Я имею нечто сказать тебе. Он говорит: скажи, Учитель.
\vs Luk 7:41 Иисус сказал: у одного заимодавца было два должника: один должен был пятьсот динариев, а другой пятьдесят,
\vs Luk 7:42 но как они не имели чем заплатить, он простил обоим. Скажи же, который из них более возлюбит его?
\vs Luk 7:43 Симон отвечал: думаю, тот, которому более простил. Он сказал ему: правильно ты рассудил.
\vs Luk 7:44 И, обратившись к женщине, сказал Симону: видишь ли ты эту женщину? Я пришел в дом твой, и ты воды Мне на ноги не дал, а она слезами облила Мне ноги и волосами головы своей отёрла;
\vs Luk 7:45 ты целования Мне не дал, а она, с тех пор как Я пришел, не перестает целовать у Меня ноги;
\vs Luk 7:46 ты головы Мне маслом не помазал, а она миром помазала Мне ноги.
\vs Luk 7:47 А потому сказываю тебе: прощаются грехи её многие за то, что она возлюбила много, а кому мало прощается, тот мало любит.
\vs Luk 7:48 Ей же сказал: прощаются тебе грехи.
\vs Luk 7:49 И возлежавшие с Ним начали говорить про себя: кто это, что и грехи прощает?
\vs Luk 7:50 Он же сказал женщине: вера твоя спасла тебя, иди с миром.
\vs Luk 8:1 После сего Он проходил по городам и селениям, проповедуя и благовествуя Царствие Божие, и с Ним двенадцать,
\vs Luk 8:2 и некоторые женщины, которых Он исцелил от злых духов и болезней: Мария, называемая Магдалиною, из которой вышли семь бесов,
\vs Luk 8:3 и Иоанна, жена Хузы, домоправителя Иродова, и Сусанна, и многие другие, которые служили Ему имением своим.
\rsbpar\vs Luk 8:4 Когда же собралось множество народа, и из всех городов жители сходились к Нему, Он начал говорить притчею:
\vs Luk 8:5 вышел сеятель сеять семя свое, и когда он сеял, иное упало при дороге и было потоптано, и птицы небесные поклевали его;
\vs Luk 8:6 а иное упало на камень и, взойдя, засохло, потому что не имело влаги;
\vs Luk 8:7 а иное упало между тернием, и выросло терние и заглушило его;
\vs Luk 8:8 а иное упало на добрую землю и, взойдя, принесло плод сторичный. Сказав сие, возгласил: кто имеет уши слышать, да слышит!
\vs Luk 8:9 Ученики же Его спросили у Него: что бы значила притча сия?
\vs Luk 8:10 Он сказал: вам дано знать тайны Царствия Божия, а прочим в притчах, так что они видя не видят и слыша не разумеют.
\vs Luk 8:11 Вот что значит притча сия: семя есть слово Божие;
\vs Luk 8:12 а упавшее при пути, это суть слушающие, к которым пот\acc{о}м приходит диавол и уносит слово из сердца их, чтобы они не уверовали и не спаслись;
\vs Luk 8:13 а упавшее на камень, это те, которые, когда услышат слово, с радостью принимают, но которые не имеют корня, и временем веруют, а во время искушения отпадают;
\vs Luk 8:14 а упавшее в терние, это те, которые слушают слово, но, отходя, заботами, богатством и наслаждениями житейскими подавляются и не приносят плода;
\vs Luk 8:15 а упавшее на добрую землю, это те, которые, услышав слово, хранят его в добром и чистом сердце и приносят плод в терпении. Сказав это, Он возгласил: кто имеет уши слышать, да слышит!
\vs Luk 8:16 Никто, зажегши свечу, не покрывает ее сосудом, или не ставит под кровать, а ставит на подсвечник, чтобы входящие видели свет.
\vs Luk 8:17 Ибо нет ничего тайного, чт\acc{о} не сделалось бы явным, ни сокровенного, чт\acc{о} не сделалось бы известным и не обнаружилось бы.
\vs Luk 8:18 Итак, наблюдайте, как вы слушаете: ибо, кто имеет, тому дано будет, а кто не имеет, у того отнимется и т\acc{о}, чт\acc{о} он думает иметь.
\rsbpar\vs Luk 8:19 И пришли к Нему Матерь и братья Его, и не могли подойти к Нему по причине народа.
\vs Luk 8:20 И дали знать Ему: Матерь и братья Твои стоят вне, желая видеть Тебя.
\vs Luk 8:21 Он сказал им в ответ: матерь Моя и братья Мои суть слушающие слово Божие и исполняющие его.
\rsbpar\vs Luk 8:22 В один день Он вошел с учениками Своими в лодку и сказал им: переправимся на ту сторону озера. И отправились.
\vs Luk 8:23 Во время плавания их Он заснул. На озере поднялся бурный ветер, и заливало их \bibemph{волнами}, и они были в опасности.
\vs Luk 8:24 И, подойдя, разбудили Его и сказали: Наставник! Наставник! погибаем. Но Он, встав, запретил ветру и волнению воды; и перестали, и сделалась тишина.
\vs Luk 8:25 Тогда Он сказал им: где вера ваша? Они же в страхе и удивлении говорили друг другу: кто же это, что и ветрам повелевает и воде, и повинуются Ему?
\rsbpar\vs Luk 8:26 И приплыли в страну Гадаринскую, лежащую против Галилеи.
\vs Luk 8:27 Когда же вышел Он на берег, встретил Его один человек из города, одержимый бесами с давнего времени, и в одежду не одевавшийся, и живший не в доме, а в гробах.
\vs Luk 8:28 Он, увидев Иисуса, вскричал, пал пред Ним и громким голосом сказал: чт\acc{о} Тебе до меня, Иисус, Сын Бога Всевышнего? умоляю Тебя, не мучь меня.
\vs Luk 8:29 Ибо \bibemph{Иисус} повелел нечистому духу выйти из сего человека, потому что он долгое время мучил его, так что его связывали цепями и узами, сберегая его; но он разрывал узы и был гоним бесом в пустыни.
\vs Luk 8:30 Иисус спросил его: как тебе имя? Он сказал: легион,~--- потому что много бесов вошло в него.
\vs Luk 8:31 И они просили Иисуса, чтобы не повелел им идти в бездну.
\vs Luk 8:32 Тут же на горе паслось большое стадо свиней; и \bibemph{бесы} просили Его, чтобы позволил им войти в них. Он позволил им.
\vs Luk 8:33 Бесы, выйдя из человека, вошли в свиней, и бросилось стадо с крутизны в озеро и потонуло.
\vs Luk 8:34 Пастухи, видя происшедшее, побежали и рассказали в городе и в селениях.
\vs Luk 8:35 И вышли видеть происшедшее; и, придя к Иисусу, нашли человека, из которого вышли бесы, сидящего у ног Иисуса, одетого и в здравом уме; и ужаснулись.
\vs Luk 8:36 Видевшие же рассказали им, как исцелился бесновавшийся.
\vs Luk 8:37 И просил Его весь народ Гадаринской окрестности удалиться от них, потому что они объяты были великим страхом. Он вошел в лодку и возвратился.
\vs Luk 8:38 Человек же, из которого вышли бесы, просил Его, чтобы быть с Ним. Но Иисус отпустил его, сказав:
\vs Luk 8:39 возвратись в дом твой и расскажи, чт\acc{о} сотворил тебе Бог. Он пошел и проповедовал по всему городу, что сотворил ему Иисус.
\rsbpar\vs Luk 8:40 Когда же возвратился Иисус, народ принял Его, потому что все ожидали Его.
\vs Luk 8:41 И вот, пришел человек, именем Иаир, который был начальником синагоги; и, пав к ногам Иисуса, просил Его войти к нему в дом,
\vs Luk 8:42 потому что у него была одна дочь, лет двенадцати, и та была при смерти. Когда же Он шел, народ теснил Его.
\vs Luk 8:43 И женщина, страдавшая кровотечением двенадцать лет, которая, издержав на врачей всё имение, ни одним не могла быть вылечена,
\vs Luk 8:44 подойдя сзади, коснулась края одежды Его; и тотчас течение крови у ней остановилось.
\vs Luk 8:45 И сказал Иисус: кто прикоснулся ко Мне? Когда же все отрицались, Петр сказал и бывшие с Ним: Наставник! народ окружает Тебя и теснит,~--- и Ты говоришь: кто прикоснулся ко Мне?
\vs Luk 8:46 Но Иисус сказал: прикоснулся ко Мне некто, ибо Я чувствовал силу, исшедшую из Меня.
\vs Luk 8:47 Женщина, видя, что она не утаилась, с трепетом подошла и, пав пред Ним, объявила Ему перед всем народом, по какой причине прикоснулась к Нему и как тотчас исцелилась.
\vs Luk 8:48 Он сказал ей: дерзай, дщерь! вера твоя спасла тебя; иди с миром.
\vs Luk 8:49 Когда Он еще говорил это, приходит некто из дома начальника синагоги и говорит ему: дочь твоя умерла; не утруждай Учителя.
\vs Luk 8:50 Но Иисус, услышав это, сказал ему: не бойся, только веруй, и спасена будет.
\vs Luk 8:51 Придя же в дом, не позволил войти никому, кроме Петра, Иоанна и Иакова, и отца девицы, и матери.
\vs Luk 8:52 Все плакали и рыдали о ней. Но Он сказал: не плачьте; она не умерла, но спит.
\vs Luk 8:53 И смеялись над Ним, зная, что она умерла.
\vs Luk 8:54 Он же, выслав всех вон и взяв ее за руку, возгласил: девица! встань.
\vs Luk 8:55 И возвратился дух ее; она тотчас встала, и Он велел дать ей есть.
\vs Luk 8:56 И удивились родители ее. Он же повелел им не сказывать никому о происшедшем.
\vs Luk 9:1 Созвав же двенадцать, дал силу и власть над всеми бесами и врачевать от болезней,
\vs Luk 9:2 и послал их проповедовать Царствие Божие и исцелять больных.
\vs Luk 9:3 И сказал им: ничего не берите на дорогу: ни посоха, ни сум\acc{ы}, ни хлеба, ни серебра, и не имейте по две одежды;
\vs Luk 9:4 и в какой дом войдете, там оставайтесь и оттуда отправляйтесь \bibemph{в путь}.
\vs Luk 9:5 А если где не примут вас, то, выходя из того города, отрясите и прах от ног ваших во свидетельство на них.
\vs Luk 9:6 Они пошли и проходили по селениям, благовествуя и исцеляя повсюду.
\rsbpar\vs Luk 9:7 Услышал Ирод четвертовластник о всём, что делал \bibemph{Иисус}, и недоумевал: ибо одни говорили, что это Иоанн восстал из мертвых;
\vs Luk 9:8 другие, что Илия явился, а иные, что один из древних пророков воскрес.
\vs Luk 9:9 И сказал Ирод: Иоанна я обезглавил; кто же Этот, о Котором я слышу такое? И искал увидеть Его.
\rsbpar\vs Luk 9:10 Апостолы, возвратившись, рассказали Ему, чт\acc{о} они сделали; и Он, взяв их с Собою, удалился особо в пустое место, близ города, называемого Вифсаидою.
\rsbpar\vs Luk 9:11 Но народ, узнав, пошел за Ним; и Он, приняв их, беседовал с ними о Царствии Божием и требовавших исцеления исцелял.
\vs Luk 9:12 День же начал склоняться к вечеру. И, приступив к Нему, двенадцать говорили Ему: отпусти народ, чтобы они пошли в окрестные селения и деревни ночевать и достали пищи; потому что мы здесь в пустом месте.
\vs Luk 9:13 Но Он сказал им: вы дайте им есть. Они сказали: у нас нет более пяти хлебов и двух рыб; разве нам пойти купить пищи для всех сих людей?
\vs Luk 9:14 Ибо их было около пяти тысяч человек. Но Он сказал ученикам Своим: рассадите их рядами по пятидесяти.
\vs Luk 9:15 И сделали так, и рассадили всех.
\vs Luk 9:16 Он же, взяв пять хлебов и две рыбы и воззрев на небо, благословил их, преломил и дал ученикам, чтобы раздать народу.
\vs Luk 9:17 И ели, и насытились все; и оставшихся у них кусков набрано двенадцать коробов.
\rsbpar\vs Luk 9:18 В одно время, когда Он молился в уединенном месте, и ученики были с Ним, Он спросил их: за кого почитает Меня народ?
\vs Luk 9:19 Они сказали в ответ: за Иоанна Крестителя, а иные за Илию; другие же \bibemph{говорят}, что один из древних пророков воскрес.
\vs Luk 9:20 Он же спросил их: а вы за кого почитаете Меня? Отвечал Петр: за Христа Божия.
\vs Luk 9:21 Но Он строго приказал им никому не говорить о сем,
\vs Luk 9:22 сказав, что Сыну Человеческому должно много пострадать, и быть отвержену старейшинами, первосвященниками и книжниками, и быть убиту, и в третий день воскреснуть.
\rsbpar\vs Luk 9:23 Ко всем же сказал: если кто хочет идти за Мною, отвергнись себя, и возьми крест свой, и следуй за Мною.
\vs Luk 9:24 Ибо кто хочет душу свою сберечь, тот потеряет ее; а кто потеряет душу свою ради Меня, тот сбережет ее.
\vs Luk 9:25 Ибо что пользы человеку приобрести весь мир, а себя самого погубить или повредить себе?
\vs Luk 9:26 Ибо кто постыдится Меня и Моих слов, того Сын Человеческий постыдится, когда приидет во славе Своей и Отца и святых Ангелов.
\vs Luk 9:27 Говорю же вам истинно: есть некоторые из стоящих здесь, которые не вкусят смерти, как уже увидят Царствие Божие.
\rsbpar\vs Luk 9:28 После сих слов, дней через восемь, взяв Петра, Иоанна и Иакова, взошел Он на гору помолиться.
\vs Luk 9:29 И когда молился, вид лица Его изменился, и одежда Его сделалась белою, блистающею.
\vs Luk 9:30 И вот, два мужа беседовали с Ним, которые были Моисей и Илия;
\vs Luk 9:31 явившись во славе, они говорили об исходе Его, который Ему надлежало совершить в Иерусалиме.
\vs Luk 9:32 Петр же и бывшие с ним отягчены были сном; но, пробудившись, увидели славу Его и двух мужей, стоявших с Ним.
\vs Luk 9:33 И когда они отходили от Него, сказал Петр Иисусу: Наставник! хорошо нам здесь быть; сделаем три кущи: одну Тебе, одну Моисею и одну Илии,~--- не зная, чт\acc{о} говорил.
\vs Luk 9:34 Когда же он говорил это, явилось облако и осенило их; и устрашились, когда вошли в облако.
\vs Luk 9:35 И был из облака глас, глаголющий: Сей есть Сын Мой Возлюбленный, Его слушайте.
\vs Luk 9:36 Когда был глас сей, остался Иисус один. И они умолчали, и никому не говорили в те дни о том, что видели.
\rsbpar\vs Luk 9:37 В следующий же день, когда они сошли с горы, встретило Его много народа.
\vs Luk 9:38 Вдруг некто из народа воскликнул: Учитель! умоляю Тебя взглянуть на сына моего, он один у меня:
\vs Luk 9:39 его схватывает дух, и он внезапно вскрикивает, и терзает его, так что он испускает пену; и насилу отступает от него, измучив его.
\vs Luk 9:40 Я просил учеников Твоих изгнать его, и они не могли.
\vs Luk 9:41 Иисус же, отвечая, сказал: о, род неверный и развращенный! доколе буду с вами и буду терпеть вас? приведи сюда сына твоего.
\vs Luk 9:42 Когда же тот еще шел, бес поверг его и стал бить; но Иисус запретил нечистому духу, и исцелил отрока, и отдал его отцу его.
\vs Luk 9:43 И все удивлялись величию Божию.\rsbpar Когда же все дивились всему, что творил Иисус, Он сказал ученикам Своим:
\vs Luk 9:44 вложите вы себе в уши слова сии: Сын Человеческий будет предан в руки человеческие.
\vs Luk 9:45 Но они не поняли сл\acc{о}ва сего, и оно было закрыто от них, так что они не постигли его, а спросить Его о сем слове боялись.
\vs Luk 9:46 Пришла же им мысль: кто бы из них был больше?
\vs Luk 9:47 Иисус же, видя помышление сердца их, взяв дитя, поставил его пред Собою
\vs Luk 9:48 и сказал им: кто примет сие дитя во имя Мое, тот Меня принимает; а кто примет Меня, тот принимает Пославшего Меня; ибо кто из вас меньше всех, тот будет велик.
\vs Luk 9:49 При сем Иоанн сказал: Наставник! мы видели человека, именем Твоим изгоняющего бесов, и запретили ему, потому что он не ходит с нами.
\vs Luk 9:50 Иисус сказал ему: не запрещайте, ибо кто не против вас, тот за вас.
\rsbpar\vs Luk 9:51 Когда же приближались дни взятия Его \bibemph{от мира}, Он восхотел идти в Иерусалим;
\vs Luk 9:52 и послал вестников пред лицем Своим; и они пошли и вошли в селение Самарянское; чтобы приготовить для Него;
\vs Luk 9:53 но \bibemph{там} не приняли Его, потому что Он имел вид путешествующего в Иерусалим.
\vs Luk 9:54 Видя т\acc{о}, ученики Его, Иаков и Иоанн, сказали: Господи! хочешь ли, мы скажем, чтобы огонь сошел с неба и истребил их, как и Илия сделал?
\vs Luk 9:55 Но Он, обратившись к ним, запретил им и сказал: не знаете, какого вы духа;
\vs Luk 9:56 ибо Сын Человеческий пришел не губить души человеческие, а спасать. И пошли в другое селение.
\rsbpar\vs Luk 9:57 Случилось, что когда они были в пути, некто сказал Ему: Господи! я пойду за Тобою, куда бы Ты ни пошел.
\vs Luk 9:58 Иисус сказал ему: лисицы имеют норы, и птицы небесные~--- гнезда; а Сын Человеческий не имеет, где приклонить голову.
\vs Luk 9:59 А другому сказал: следуй за Мною. Тот сказал: Господи! позволь мне прежде пойти и похоронить отца моего.
\vs Luk 9:60 Но Иисус сказал ему: предоставь мертвым погребать своих мертвецов, а ты иди, благовествуй Царствие Божие.
\vs Luk 9:61 Еще другой сказал: я пойду за Тобою, Господи! но прежде позволь мне проститься с домашними моими.
\vs Luk 9:62 Но Иисус сказал ему: никто, возложивший руку свою на плуг и озирающийся назад, не благонадежен для Царствия Божия.
\vs Luk 10:1 После сего избрал Господь и других семьдесят \bibemph{учеников}, и послал их по два пред лицем Своим во всякий город и место, куда Сам хотел идти,
\vs Luk 10:2 и сказал им: жатвы много, а делателей мало; итак, молите Господина жатвы, чтобы выслал делателей на жатву Свою.
\vs Luk 10:3 Идите! Я посылаю вас, как агнцев среди волков.
\vs Luk 10:4 Не берите ни мешка, ни сум\acc{ы}, ни обуви, и никого на дороге не приветствуйте.
\vs Luk 10:5 В какой дом войдете, сперва говорите: мир дому сему;
\vs Luk 10:6 и если будет там сын мира, то почиет на нём мир ваш, а если нет, то к вам возвратится.
\vs Luk 10:7 В доме же том оставайтесь, ешьте и пейте, что у них есть, ибо трудящийся достоин награды за труды свои; не переходите из дома в дом.
\vs Luk 10:8 И если придёте в какой город и примут вас, ешьте, что вам предложат,
\vs Luk 10:9 и исцеляйте находящихся в нём больных, и говорите им: приблизилось к вам Царствие Божие.
\vs Luk 10:10 Если же придете в какой город и не примут вас, то, выйдя на улицу, скажите:
\vs Luk 10:11 и прах, прилипший к нам от вашего города, отрясаем вам; однако же знайте, что приблизилось к вам Царствие Божие.
\vs Luk 10:12 Сказываю вам, что Содому в день оный будет отраднее, нежели городу тому.
\vs Luk 10:13 Горе тебе, Хоразин! горе тебе, Вифсаида! ибо если бы в Тире и Сидоне явлены были силы, явленные в вас, то давно бы они, сидя во вретище и пепле, покаялись;
\vs Luk 10:14 но и Тиру и Сидону отраднее будет на суде, нежели вам.
\vs Luk 10:15 И ты, Капернаум, до неба вознесшийся, до ада низвергнешься.
\vs Luk 10:16 Слушающий вас Меня слушает, и отвергающийся вас Меня отвергается; а отвергающийся Меня отвергается Пославшего Меня.
\rsbpar\vs Luk 10:17 Семьдесят \bibemph{учеников} возвратились с радостью и говорили: Господи! и бесы повинуются нам о имени Твоем.
\vs Luk 10:18 Он же сказал им: Я видел сатану, спадшего с неба, как молнию;
\vs Luk 10:19 се, даю вам власть наступать на змей и скорпионов и на всю силу вражью, и ничто не повредит вам;
\vs Luk 10:20 однако ж тому не радуйтесь, что духи вам повинуются, но радуйтесь тому, что имена ваши написаны на небесах.
\vs Luk 10:21 В тот час возрадовался духом Иисус и сказал: славлю Тебя, Отче, Господи неба и земли, что Ты утаил сие от мудрых и разумных и открыл младенцам. Ей, Отче! Ибо таково было Твое благоволение.
\vs Luk 10:22 И, обратившись к ученикам, сказал: всё предано Мне Отцем Моим; и кто есть Сын, не знает никто, кроме Отца, и кто есть Отец, \bibemph{не знает никто}, кроме Сына, и кому Сын хочет открыть.
\vs Luk 10:23 И, обратившись к ученикам, сказал им особо: блаженны очи, видящие то, что вы видите!
\vs Luk 10:24 ибо сказываю вам, что многие пророки и цари желали видеть, чт\acc{о} вы видите, и не видели, и слышать, чт\acc{о} вы слышите, и не слышали.
\rsbpar\vs Luk 10:25 И вот, один законник встал и, искушая Его, сказал: Учитель! чт\acc{о} мне делать, чтобы наследовать жизнь вечную?
\vs Luk 10:26 Он же сказал ему: в законе чт\acc{о} написано? к\acc{а}к читаешь?
\vs Luk 10:27 Он сказал в ответ: возлюби Господа Бога твоего всем сердцем твоим, и всею душею твоею, и всею крепостию твоею, и всем разумением твоим, и ближнего твоего, как самого себя.
\vs Luk 10:28 \bibemph{Иисус} сказал ему: правильно ты отвечал; так поступай, и будешь жить.
\vs Luk 10:29 Но он, желая оправдать себя, сказал Иисусу: а кто мой ближний?
\vs Luk 10:30 На это сказал Иисус: некоторый человек шел из Иерусалима в Иерихон и попался разбойникам, которые сняли с него одежду, изранили его и ушли, оставив его едва живым.
\vs Luk 10:31 По случаю один священник шел тою дорогою и, увидев его, прошел мимо.
\vs Luk 10:32 Также и левит, быв на том месте, подошел, посмотрел и прошел мимо.
\vs Luk 10:33 Самарянин же некто, проезжая, нашел на него и, увидев его, сжалился
\vs Luk 10:34 и, подойдя, перевязал ему раны, возливая масло и вино; и, посадив его на своего осла, привез его в гостиницу и позаботился о нем;
\vs Luk 10:35 а на другой день, отъезжая, вынул два динария, дал содержателю гостиницы и сказал ему: позаботься о нем; и если издержишь что более, я, когда возвращусь, отдам тебе.
\vs Luk 10:36 Кто из этих троих, думаешь ты, был ближний попавшемуся разбойникам?
\vs Luk 10:37 Он сказал: оказавший ему милость. Тогда Иисус сказал ему: иди, и ты поступай так же.
\rsbpar\vs Luk 10:38 В продолжение пути их пришел Он в одно селение; здесь женщина, именем Марфа, приняла Его в дом свой;
\vs Luk 10:39 у неё была сестра, именем Мария, которая села у ног Иисуса и слушала слово Его.
\vs Luk 10:40 Марфа же заботилась о большом угощении и, подойдя, сказала: Господи! или Тебе нужды нет, что сестра моя одну меня оставила служить? скажи ей, чтобы помогла мне.
\vs Luk 10:41 Иисус же сказал ей в ответ: Марфа! Марфа! ты заботишься и суетишься о многом,
\vs Luk 10:42 а одно только нужно; Мария же избрала благую часть, которая не отнимется у неё.
\vs Luk 11:1 Случилось, что когда Он в одном месте молился, и перестал, один из учеников Его сказал Ему: Господи! научи нас молиться, как и Иоанн научил учеников своих.
\vs Luk 11:2 Он сказал им: когда м\acc{о}литесь, говорите:\rsbpar Отче наш, сущий на небесах! да святится имя Твое; да приидет Царствие Твое; да будет воля Твоя и на земле, как на небе;
\vs Luk 11:3 хлеб наш насущный подавай нам на каждый день;
\vs Luk 11:4 и прости нам грехи наши, ибо и мы прощаем всякому должнику нашему; и не введи нас в искушение, но избавь нас от лукавого.
\rsbpar\vs Luk 11:5 И сказал им: \bibemph{положим, что} кто-нибудь из вас, имея друга, придёт к нему в полночь и скажет ему: друг! дай мне взаймы три хлеба,
\vs Luk 11:6 ибо друг мой с дороги зашел ко мне, и мне нечего предложить ему;
\vs Luk 11:7 а тот изнутри скажет ему в ответ: не беспокой меня, двери уже заперты, и дети мои со мною на постели; не могу встать и дать тебе.
\vs Luk 11:8 Если, говорю вам, он не встанет и не даст ему по дружбе с ним, то по неотступности его, встав, даст ему, сколько просит.
\vs Luk 11:9 И Я скажу вам: прос\acc{и}те, и дано будет вам; ищите, и найдете; стучите, и отворят вам,
\vs Luk 11:10 ибо всякий просящий получает, и ищущий находит, и стучащему отворят.
\vs Luk 11:11 Какой из вас отец, \bibemph{когда} сын попросит у него хлеба, подаст ему камень? или, \bibemph{когда попросит} рыбы, подаст ему змею вместо рыбы?
\vs Luk 11:12 Или, если попросит яйца, подаст ему скорпиона?
\vs Luk 11:13 Итак, если вы, будучи злы, умеете даяния благие давать детям вашим, тем более Отец Небесный даст Духа Святаго просящим у Него.
\rsbpar\vs Luk 11:14 Однажды изгнал Он беса, который был нем; и когда бес вышел, немой стал говорить; и народ удивился.
\vs Luk 11:15 Некоторые же из них говорили: Он изгоняет бесов силою веельзевула, князя бесовского.
\vs Luk 11:16 А другие, искушая, требовали от Него знамения с неба.
\vs Luk 11:17 Но Он, зная помышления их, сказал им: всякое царство, разделившееся само в себе, опустеет, и дом, \bibemph{разделившийся} сам в себе, падет;
\vs Luk 11:18 если же и сатана разделится сам в себе, то к\acc{а}к устоит царство его? а вы говорите, что Я силою веельзевула изгоняю бесов;
\vs Luk 11:19 и если Я силою веельзевула изгоняю бесов, то сыновья ваши чьею силою изгоняют их? Посему они будут вам судьями.
\vs Luk 11:20 Если же Я перстом Божиим изгоняю бесов, то, конечно, достигло до вас Царствие Божие.
\vs Luk 11:21 Когда сильный с оружием охраняет свой дом, тогда в безопасности его имение;
\vs Luk 11:22 когда же сильнейший его нападет на него и победит его, тогда возьмет всё оружие его, на которое он надеялся, и разделит похищенное у него.
\vs Luk 11:23 Кто не со Мною, тот против Меня; и кто не собирает со Мною, тот расточает.
\vs Luk 11:24 Когда нечистый дух выйдет из человека, то ходит по безводным местам, ища покоя, и, не находя, говорит: возвращусь в дом мой, откуда вышел;
\vs Luk 11:25 и, придя, находит его выметенным и убранным;
\vs Luk 11:26 тогда идет и берет с собою семь других духов, злейших себя, и, войдя, живут там,~--- и бывает для человека того последнее хуже первого.
\vs Luk 11:27 Когда же Он говорил это, одна женщина, возвысив голос из народа, сказала Ему: блаженно чрево, носившее Тебя, и сосцы, Тебя питавшие!
\vs Luk 11:28 А Он сказал: блаженны слышащие слово Божие и соблюдающие его.
\rsbpar\vs Luk 11:29 Когда же народ стал сходиться во множестве, Он начал говорить: род сей лукав, он ищет знамения, и знамение не дастся ему, кроме знамения Ионы пророка;
\vs Luk 11:30 ибо к\acc{а}к Иона был знамением для Ниневитян, т\acc{а}к будет и Сын Человеческий для рода сего.
\vs Luk 11:31 Царица южная восстанет на суд с людьми рода сего и осудит их, ибо она приходила от пределов земли послушать мудрости Соломоновой; и вот, здесь больше Соломона.
\vs Luk 11:32 Ниневитяне восстанут на суд с родом сим и осудят его, ибо они покаялись от проповеди Иониной, и вот, здесь больше Ионы.
\rsbpar\vs Luk 11:33 Никто, зажегши свечу, не ставит ее в сокровенном месте, ни под сосудом, но на подсвечнике, чтобы входящие видели свет.
\vs Luk 11:34 Светильник тела есть око; итак, если око твое будет чисто, то и все тело твое будет светло; а если оно будет худо, то и тело твое будет темно.
\vs Luk 11:35 Итак, смотри: свет, который в тебе, не есть ли тьма?
\vs Luk 11:36 Если же тело твое всё светло и не имеет ни одной темной части, то будет светло всё т\acc{а}к, как бы светильник освещал тебя сиянием.
\rsbpar\vs Luk 11:37 Когда Он говорил это, один фарисей просил Его к себе обедать. Он пришел и возлег.
\vs Luk 11:38 Фарисей же удивился, увидев, что Он не умыл \bibemph{рук} перед обедом.
\vs Luk 11:39 Но Господь сказал ему: ныне вы, фарисеи, внешность чаши и блюда очищаете, а внутренность ваша исполнена хищения и лукавства.
\vs Luk 11:40 Неразумные! не Тот же ли, Кто сотворил внешнее, сотворил и внутреннее?
\vs Luk 11:41 Подавайте лучше милостыню из того, чт\acc{о} у вас есть, тогда всё будет у вас чисто.
\vs Luk 11:42 Но горе вам, фарисеям, что даете десятину с мяты, руты и всяких овощей, и нерадите о суде и любви Божией: сие надлежало делать, и того не оставлять.
\vs Luk 11:43 Горе вам, фарисеям, что любите председания в синагогах и приветствия в народных собраниях.
\vs Luk 11:44 Горе вам, книжники и фарисеи, лицемеры, что вы~--- как гробы скрытые, над которыми люди ходят и не знают того.
\vs Luk 11:45 На это некто из законников сказал Ему: Учитель! говоря это, Ты и нас обижаешь.
\vs Luk 11:46 Но Он сказал: и вам, законникам, горе, что налагаете на людей бремена неудобоносимые, а сами и одним перстом своим не дотрагиваетесь до них.
\vs Luk 11:47 Горе вам, что строите гробницы пророкам, которых избили отцы ваши:
\vs Luk 11:48 сим вы свидетельствуете о делах отцов ваших и соглашаетесь с ними, ибо они избили пророков, а вы строите им гробницы.
\vs Luk 11:49 Потому и премудрость Божия сказала: пошлю к ним пророков и Апостолов, и из них одних убьют, а других изгонят,
\vs Luk 11:50 да взыщется от рода сего кровь всех пророков, пролитая от создания мира,
\vs Luk 11:51 от крови Авеля до крови Захарии, убитого между жертвенником и храмом. Ей, говорю вам, взыщется от рода сего.
\vs Luk 11:52 Горе вам, законникам, что вы взяли ключ разумения: сами не вошли, и входящим воспрепятствовали.
\vs Luk 11:53 Когда Он говорил им это, книжники и фарисеи начали сильно приступать к Нему, вынуждая у Него ответы на многое,
\vs Luk 11:54 подыскиваясь под Него и стараясь уловить что-нибудь из уст Его, чтобы обвинить Его.
\vs Luk 12:1 Между тем, когда собрались тысячи народа, так что теснили друг друга, Он начал говорить сперва ученикам Своим: берегитесь закваски фарисейской, которая есть лицемерие.
\vs Luk 12:2 Нет ничего сокровенного, что не открылось бы, и тайного, чего не узнали бы.
\vs Luk 12:3 Посему, чт\acc{о} вы сказали в темноте, т\acc{о} услышится во свете; и чт\acc{о} говорили на ухо внутри дома, т\acc{о} будет провозглашено на кровлях.
\vs Luk 12:4 Говорю же вам, друзьям Моим: не бойтесь убивающих тело и потом не могущих ничего более сделать;
\vs Luk 12:5 но скажу вам, кого бояться: бойтесь Того, Кто, по убиении, может ввергнуть в геенну: ей, говорю вам, Того бойтесь.
\vs Luk 12:6 Не пять ли малых птиц продаются за два ассария? и ни одна из них не забыта у Бога.
\vs Luk 12:7 А у вас и волосы на голове все сочтены. Итак не бойтесь: вы дороже многих малых птиц.
\vs Luk 12:8 Сказываю же вам: всякого, кто исповедает Меня пред человеками, и Сын Человеческий исповедает пред Ангелами Божиими;
\vs Luk 12:9 а кто отвергнется Меня пред человеками, тот отвержен будет пред Ангелами Божиими.
\vs Luk 12:10 И всякому, кто скажет слово на Сына Человеческого, прощено будет; а кто скажет хулу на Святаго Духа, тому не простится.
\vs Luk 12:11 Когда же приведут вас в синагоги, к начальствам и властям, не заботьтесь, к\acc{а}к или чт\acc{о} отвечать, или чт\acc{о} говорить,
\vs Luk 12:12 ибо Святый Дух научит вас в тот час, чт\acc{о} должно говорить.
\rsbpar\vs Luk 12:13 Некто из народа сказал Ему: Учитель! скажи брату моему, чтобы он разделил со мною наследство.
\vs Luk 12:14 Он же сказал человеку тому: кто поставил Меня судить или делить вас?
\vs Luk 12:15 При этом сказал им: смотрите, берегитесь любостяжания, ибо жизнь человека не зависит от изобилия его имения.
\vs Luk 12:16 И сказал им притчу: у одного богатого человека был хороший урожай в поле;
\vs Luk 12:17 и он рассуждал сам с собою: что мне делать? некуда мне собрать плодов моих?
\vs Luk 12:18 И сказал: вот что сделаю: сломаю житницы мои и построю б\acc{о}льшие, и соберу туда весь хлеб мой и всё добро мое,
\vs Luk 12:19 и скажу душе моей: душа! много добра лежит у тебя на многие годы: покойся, ешь, пей, веселись.
\vs Luk 12:20 Но Бог сказал ему: безумный! в сию ночь душу твою возьмут у тебя; кому же достанется то, что ты заготовил?
\vs Luk 12:21 Так \bibemph{бывает с тем}, кто собирает сокровища для себя, а не в Бога богатеет.
\vs Luk 12:22 И сказал ученикам Своим: посему говорю вам,~--- не заботьтесь для души вашей, что вам есть, ни для тела, во что одеться:
\vs Luk 12:23 душа больше пищи, и тело~--- одежды.
\vs Luk 12:24 Посмотрите на воронов: они не сеют, не жнут; нет у них ни хранилищ, ни житниц, и Бог питает их; сколько же вы лучше птиц?
\vs Luk 12:25 Да и кто из вас, заботясь, может прибавить себе роста хотя на один локоть?
\vs Luk 12:26 Итак, если и малейшего сделать не можете, чт\acc{о} заботитесь о прочем?
\vs Luk 12:27 Посмотрите на лилии, как они растут: не трудятся, не прядут; но говорю вам, что и Соломон во всей славе своей не одевался так, как всякая из них.
\vs Luk 12:28 Если же траву на поле, которая сегодня есть, а завтра будет брошена в печь, Бог так одевает, то кольми паче вас, маловеры!
\vs Luk 12:29 Итак, не ищите, чт\acc{о} вам есть, или чт\acc{о} пить, и не беспокойтесь,
\vs Luk 12:30 потому что всего этого ищут люди мира сего; ваш же Отец знает, что вы имеете нужду в том;
\vs Luk 12:31 наипаче ищите Царствия Божия, и это всё приложится вам.
\vs Luk 12:32 Не бойся, малое стадо! ибо Отец ваш благоволил дать вам Царство.
\vs Luk 12:33 Продавайте имения ваши и давайте милостыню. Приготовляйте себе влагалища не ветшающие, сокровище неоскудевающее на небесах, куда вор не приближается и где моль не съедает,
\vs Luk 12:34 ибо где сокровище ваше, там и сердце ваше будет.
\rsbpar\vs Luk 12:35 Да будут чресла ваши препоясаны и светильники горящи.
\vs Luk 12:36 И вы будьте подобны людям, ожидающим возвращения господина своего с брака, дабы, когда придёт и постучит, тотчас отворить ему.
\vs Luk 12:37 Блаженны рабы те, которых господин, придя, найдёт бодрствующими; истинно говорю вам, он препояшется и посадит их, и, подходя, станет служить им.
\vs Luk 12:38 И если придет во вторую стражу, и в третью стражу придет, и найдет их так, то блаженны рабы те.
\vs Luk 12:39 Вы знаете, что если бы ведал хозяин дома, в который час придет вор, то бодрствовал бы и не допустил бы подкопать дом свой.
\vs Luk 12:40 Будьте же и вы готовы, ибо, в который час не думаете, приидет Сын Человеческий.
\vs Luk 12:41 Тогда сказал Ему Петр: Господи! к нам ли притчу сию говоришь, или и ко всем?
\vs Luk 12:42 Господь же сказал: кт\acc{о} верный и благоразумный домоправитель, которого господин поставил над слугами своими раздавать им в своё время меру хлеба?
\vs Luk 12:43 Блажен раб тот, которого господин его, придя, найдет поступающим так.
\vs Luk 12:44 Истинно говорю вам, что над всем имением своим поставит его.
\vs Luk 12:45 Если же раб тот скажет в сердце своем: не скоро придет господин мой, и начнет бить слуг и служанок, есть и пить и напиваться,~---
\vs Luk 12:46 то придет господин раба того в день, в который он не ожидает, и в час, в который не думает, и рассечет его, и подвергнет его одной участи с неверными.
\vs Luk 12:47 Раб же тот, который знал волю господина своего, и не был готов, и не делал по воле его, бит будет много;
\vs Luk 12:48 а который не знал, и сделал достойное наказания, бит будет меньше. И от всякого, кому дано много, много и потребуется, и кому много вверено, с того больше взыщут.
\vs Luk 12:49 Огонь пришел Я низвести на землю, и как желал бы, чтобы он уже возгорелся!
\vs Luk 12:50 Крещением должен Я креститься; и как Я томлюсь, пока сие совершится!
\vs Luk 12:51 Думаете ли вы, что Я пришел дать мир земле? Нет, говорю вам, но разделение;
\vs Luk 12:52 ибо отныне пятеро в одном доме станут разделяться, трое против двух, и двое против трех:
\vs Luk 12:53 отец будет против сына, и сын против отца; мать против дочери, и дочь против матери; свекровь против невестки своей, и невестка против свекрови своей.
\vs Luk 12:54 Сказал же и народу: когда вы видите облако, поднимающееся с запада, тотчас говорите: дождь будет, и бывает так;
\vs Luk 12:55 и когда дует южный ветер, говорите: зной будет, и бывает.
\vs Luk 12:56 Лицемеры! лице земли и неба распознавать умеете, как же времени сего не узнаете?
\vs Luk 12:57 Зачем же вы и по самим себе не судите, чему быть должно?
\vs Luk 12:58 Когда ты идешь с соперником своим к начальству, то на дороге постарайся освободиться от него, чтобы он не привел тебя к судье, а судья не отдал тебя истязателю, а истязатель не вверг тебя в темницу.
\vs Luk 12:59 Сказываю тебе: не выйдешь оттуда, пока не отдашь и последней полушки.
\vs Luk 13:1 В это время пришли некоторые и рассказали Ему о Галилеянах, которых кровь Пилат смешал с жертвами их.
\vs Luk 13:2 Иисус сказал им на это: думаете ли вы, что эти Галилеяне были грешнее всех Галилеян, что так пострадали?
\vs Luk 13:3 Нет, говорю вам, но, если не покаетесь, все т\acc{а}к же погибнете.
\vs Luk 13:4 Или думаете ли, что те восемнадцать человек, на которых упала башня Силоамская и побила их, виновнее были всех, живущих в Иерусалиме?
\vs Luk 13:5 Нет, говорю вам, но, если не покаетесь, все т\acc{а}к же погибнете.
\vs Luk 13:6 И сказал сию притчу: некто имел в винограднике своем посаженную смоковницу, и пришел искать плода на ней, и не нашел;
\vs Luk 13:7 и сказал виноградарю: вот, я третий год прихожу искать плода на этой смоковнице и не нахожу; сруби ее: на что она и землю занимает?
\vs Luk 13:8 Но он сказал ему в ответ: господин! оставь ее и на этот год, пока я окопаю ее и обложу навозом,~---
\vs Luk 13:9 не принесет ли плода; если же нет, то в следующий \bibemph{год} срубишь ее.
\rsbpar\vs Luk 13:10 В одной из синагог учил Он в субботу.
\vs Luk 13:11 Там была женщина, восемнадцать лет имевшая духа немощи: она была скорчена и не могла выпрямиться.
\vs Luk 13:12 Иисус, увидев ее, подозвал и сказал ей: женщина! ты освобождаешься от недуга твоего.
\vs Luk 13:13 И возложил на нее руки, и она тотчас выпрямилась и стала славить Бога.
\vs Luk 13:14 При этом начальник синагоги, негодуя, что Иисус исцелил в субботу, сказал народу: есть шесть дней, в которые должно делать; в те и приход\acc{и}те исцеляться, а не в день субботний.
\vs Luk 13:15 Господь сказал ему в ответ: лицемер! не отвязывает ли каждый из вас вола своего или осла от яслей в субботу и не ведет ли поить?
\vs Luk 13:16 сию же дочь Авраамову, которую связал сатана вот уже восемнадцать лет, не надлежало ли освободить от уз сих в день субботний?
\vs Luk 13:17 И когда говорил Он это, все противившиеся Ему стыдились; и весь народ радовался о всех славных делах Его.
\rsbpar\vs Luk 13:18 Он же сказал: чему подобно Царствие Божие? и чему уподоблю его?
\vs Luk 13:19 Оно подобно зерну горчичному, которое, взяв, человек посадил в саду своем; и выросло, и стало большим деревом, и птицы небесные укрывались в ветвях его.
\vs Luk 13:20 Ещё сказал: чему уподоблю Царствие Божие?
\vs Luk 13:21 Оно подобно закваске, которую женщина, взяв, положила в три меры муки, доколе не вскисло всё.
\rsbpar\vs Luk 13:22 И проходил по городам и селениям, уча и направляя путь к Иерусалиму.
\rsbpar\vs Luk 13:23 Некто сказал Ему: Господи! неужели мало спасающихся? Он же сказал им:
\vs Luk 13:24 подвизайтесь войти сквозь тесные врата, ибо, сказываю вам, многие поищут войти, и не возмогут.
\vs Luk 13:25 Когда хозяин дома встанет и затворит двери, тогда вы, стоя вне, станете стучать в двери и говорить: Господи! Господи! отвори нам; но Он скажет вам в ответ: не знаю вас, откуда вы.
\vs Luk 13:26 Тогда станете говорить: мы ели и пили пред Тобою, и на улицах наших учил Ты.
\vs Luk 13:27 Но Он скажет: говорю вам: не знаю вас, откуда вы; отойдите от Меня все делатели неправды.
\vs Luk 13:28 Там будет плач и скрежет зубов, когда увидите Авраама, Исаака и Иакова и всех пророков в Царствии Божием, а себя изгоняемыми вон.
\vs Luk 13:29 И придут от востока и запада, и севера и юга, и возлягут в Царствии Божием.
\vs Luk 13:30 И вот, есть последние, которые будут первыми, и есть первые, которые будут последними.
\rsbpar\vs Luk 13:31 В тот день пришли некоторые из фарисеев и говорили Ему: выйди и удались отсюда, ибо Ирод хочет убить Тебя.
\vs Luk 13:32 И сказал им: пойдите, скажите этой лисице: се, изгоняю бесов и совершаю исцеления сегодня и завтра, и в третий \bibemph{день} кончу;
\vs Luk 13:33 а впрочем, Мне должно ходить сегодня, завтра и в последующий день, потому что не бывает, чтобы пророк погиб вне Иерусалима.
\vs Luk 13:34 Иерусалим! Иерусалим! избивающий пророков и камнями побивающий посланных к тебе! сколько раз хотел Я собрать чад твоих, как птица птенцов своих под крылья, и вы не захотели!
\vs Luk 13:35 Се, оставляется вам дом ваш пуст. Сказываю же вам, что вы не увидите Меня, пока не придет время, когда скажете: благословен Грядый во имя Господне!
\vs Luk 14:1 Случилось Ему в субботу прийти в дом одного из начальников фарисейских вкусить хлеба, и они наблюдали за Ним.
\vs Luk 14:2 И вот, предстал пред Него человек, страждущий водяною болезнью.
\vs Luk 14:3 По сему случаю Иисус спросил законников и фарисеев: позволительно ли врачевать в субботу?
\vs Luk 14:4 Они молчали. И, прикоснувшись, исцелил его и отпустил.
\vs Luk 14:5 При сем сказал им: если у кого из вас осёл или вол упадет в колодезь, не тотчас ли вытащит его и в субботу?
\vs Luk 14:6 И не могли отвечать Ему на это.
\rsbpar\vs Luk 14:7 Замечая же, как званые выбирали первые места, сказал им притчу:
\vs Luk 14:8 когда ты будешь позван кем на брак, не садись на первое место, чтобы не случился кто из званых им почетнее тебя,
\vs Luk 14:9 и звавший тебя и его, подойдя, не сказал бы тебе: уступи ему место; и тогда со стыдом должен будешь занять последнее место.
\vs Luk 14:10 Но когда зван будешь, придя, садись на последнее место, чтобы звавший тебя, подойдя, сказал: друг! пересядь выше; тогда будет тебе честь пред сидящими с тобою,
\vs Luk 14:11 ибо всякий возвышающий сам себя унижен будет, а унижающий себя возвысится.
\vs Luk 14:12 Сказал же и позвавшему Его: когда делаешь обед или ужин, не зови друзей твоих, ни братьев твоих, ни родственников твоих, ни соседей богатых, чтобы и они тебя когда не позвали, и не получил ты воздаяния.
\vs Luk 14:13 Но, когда делаешь пир, зови нищих, увечных, хромых, слепых,
\vs Luk 14:14 и блажен будешь, что они не могут воздать тебе, ибо воздастся тебе в воскресение праведных.
\vs Luk 14:15 Услышав это, некто из возлежащих с Ним сказал Ему: блажен, кто вкусит хлеба в Царствии Божием!
\vs Luk 14:16 Он же сказал ему: один человек сделал большой ужин и звал многих,
\vs Luk 14:17 и когда наступило время ужина, послал раба своего сказать званым: идите, ибо уже всё готово.
\vs Luk 14:18 И начали все, как бы сговорившись, извиняться. Первый сказал ему: я купил землю и мне нужно пойти посмотреть ее; прошу тебя, извини меня.
\vs Luk 14:19 Другой сказал: я купил пять пар волов и иду испытать их; прошу тебя, извини меня.
\vs Luk 14:20 Третий сказал: я женился и потому не могу прийти.
\vs Luk 14:21 И, возвратившись, раб тот донес о сем господину своему. Тогда, разгневавшись, хозяин дома сказал рабу своему: пойди скорее по улицам и переулкам города и приведи сюда нищих, увечных, хромых и слепых.
\vs Luk 14:22 И сказал раб: господин! исполнено, как приказал ты, и еще есть место.
\vs Luk 14:23 Господин сказал рабу: пойди по дорогам и изгородям и убеди прийти, чтобы наполнился дом мой.
\vs Luk 14:24 Ибо сказываю вам, что никто из тех званых не вкусит моего ужина, ибо много званых, но мало избранных.
\rsbpar\vs Luk 14:25 С Ним шло множество народа; и Он, обратившись, сказал им:
\vs Luk 14:26 если кто приходит ко Мне и не возненавидит отца своего и матери, и жены и детей, и братьев и сестер, а притом и самой жизни своей, тот не может быть Моим учеником;
\vs Luk 14:27 и кто не несет креста своего и идёт за Мною, не может быть Моим учеником.
\vs Luk 14:28 Ибо кто из вас, желая построить башню, не сядет прежде и не вычислит издержек, имеет ли он, что нужно для совершения ее,
\vs Luk 14:29 дабы, когда положит основание и не возможет совершить, все видящие не стали смеяться над ним,
\vs Luk 14:30 говоря: этот человек начал строить и не мог окончить?
\vs Luk 14:31 Или какой царь, идя на войну против другого царя, не сядет и не посоветуется прежде, силен ли он с десятью тысячами противостать идущему на него с двадцатью тысячами?
\vs Luk 14:32 Иначе, пока тот еще далеко, он пошлет к нему посольство просить о мире.
\vs Luk 14:33 Так всякий из вас, кто не отрешится от всего, что имеет, не может быть Моим учеником.
\vs Luk 14:34 Соль~--- добрая вещь; но если соль потеряет силу, чем исправить ее?
\vs Luk 14:35 ни в землю, ни в навоз не годится; вон выбрасывают ее. Кто имеет уши слышать, да слышит!
\vs Luk 15:1 Приближались к Нему все мытари и грешники слушать Его.
\vs Luk 15:2 Фарисеи же и книжники роптали, говоря: Он принимает грешников и ест с ними.
\vs Luk 15:3 Но Он сказал им следующую притчу:
\vs Luk 15:4 кто из вас, имея сто овец и потеряв одну из них, не оставит девяноста девяти в пустыне и не пойдет за пропавшею, пока не найдет ее?
\vs Luk 15:5 А найдя, возьмет ее на плечи свои с радостью
\vs Luk 15:6 и, придя домой, созовет друзей и соседей и скажет им: порадуйтесь со мною: я нашел мою пропавшую овцу.
\vs Luk 15:7 Сказываю вам, что так на небесах более радости будет об одном грешнике кающемся, нежели о девяноста девяти праведниках, не имеющих нужды в покаянии.
\vs Luk 15:8 Или какая женщина, имея десять драхм, если потеряет одну драхму, не зажжет свеч\acc{и} и не станет мести комнату и искать тщательно, пока не найдет,
\vs Luk 15:9 а найдя, созовет подруг и соседок и скажет: порадуйтесь со мною: я нашла потерянную драхму.
\vs Luk 15:10 Так, говорю вам, бывает радость у Ангелов Божиих и об одном грешнике кающемся.
\rsbpar\vs Luk 15:11 Еще сказал: у некоторого человека было два сына;
\vs Luk 15:12 и сказал младший из них отцу: отче! дай мне следующую \bibemph{мне} часть имения. И \bibemph{отец} разделил им имение.
\vs Luk 15:13 По прошествии немногих дней младший сын, собрав всё, пошел в дальнюю сторону и там расточил имение свое, живя распутно.
\vs Luk 15:14 Когда же он прожил всё, настал великий голод в той стране, и он начал нуждаться;
\vs Luk 15:15 и пошел, пристал к одному из жителей страны той, а тот послал его на поля свои пасти свиней;
\vs Luk 15:16 и он рад был наполнить чрево свое рожк\acc{а}ми, которые ели свиньи, но никто не давал ему.
\vs Luk 15:17 Придя же в себя, сказал: сколько наемников у отца моего избыточествуют хлебом, а я умираю от голода;
\vs Luk 15:18 встану, пойду к отцу моему и скажу ему: отче! я согрешил против неба и пред тобою
\vs Luk 15:19 и уже недостоин называться сыном твоим; прими меня в число наемников твоих.
\vs Luk 15:20 Встал и пошел к отцу своему. И когда он был еще далеко, увидел его отец его и сжалился; и, побежав, пал ему на шею и целовал его.
\vs Luk 15:21 Сын же сказал ему: отче! я согрешил против неба и пред тобою и уже недостоин называться сыном твоим.
\vs Luk 15:22 А отец сказал рабам своим: принесите лучшую одежду и оденьте его, и дайте перстень на руку его и обувь на ноги;
\vs Luk 15:23 и приведите откормленного теленка, и заколите; станем есть и веселиться!
\vs Luk 15:24 ибо этот сын мой был мертв и ожил, пропадал и нашелся. И начали веселиться.
\vs Luk 15:25 Старший же сын его был на поле; и возвращаясь, когда приблизился к дому, услышал пение и ликование;
\vs Luk 15:26 и, призвав одного из слуг, спросил: что это такое?
\vs Luk 15:27 Он сказал ему: брат твой пришел, и отец твой заколол откормленного теленка, потому что принял его здоровым.
\vs Luk 15:28 Он осердился и не хотел войти. Отец же его, выйдя, звал его.
\vs Luk 15:29 Но он сказал в ответ отцу: вот, я столько лет служу тебе и никогда не преступал приказания твоего, но ты никогда не дал мне и козлёнка, чтобы мне повеселиться с друзьями моими;
\vs Luk 15:30 а когда этот сын твой, расточивший имение своё с блудницами, пришел, ты заколол для него откормленного теленка.
\vs Luk 15:31 Он же сказал ему: сын мой! ты всегда со мною, и всё мое твое,
\vs Luk 15:32 а о том надобно было радоваться и веселиться, что брат твой сей был мертв и ожил, пропадал и нашелся.
\vs Luk 16:1 Сказал же и к ученикам Своим: один человек был богат и имел управителя, на которого донесено было ему, что расточает имение его;
\vs Luk 16:2 и, призвав его, сказал ему: что это я слышу о тебе? дай отчет в управлении твоем, ибо ты не можешь более управлять.
\vs Luk 16:3 Тогда управитель сказал сам в себе: что мне делать? господин мой отнимает у меня управление домом; копать не могу, просить стыжусь;
\vs Luk 16:4 знаю, что сделать, чтобы приняли меня в домы свои, когда отставлен буду от управления домом.
\vs Luk 16:5 И, призвав должников господина своего, каждого порознь, сказал первому: сколько ты должен господину моему?
\vs Luk 16:6 Он сказал: сто мер масла. И сказал ему: возьми твою расписку и садись скорее, напиши: пятьдесят.
\vs Luk 16:7 Потом другому сказал: а ты сколько должен? Он отвечал: сто мер пшеницы. И сказал ему: возьми твою расписку и напиши: восемьдесят.
\vs Luk 16:8 И похвалил господин управителя неверного, что догадливо поступил; ибо сыны века сего догадливее сынов света в своем роде.
\vs Luk 16:9 И Я говорю вам: приобретайте себе друзей богатством неправедным, чтобы они, когда обнищаете, приняли вас в вечные обители.
\vs Luk 16:10 Верный в малом и во многом верен, а неверный в малом неверен и во многом.
\vs Luk 16:11 Итак, если вы в неправедном богатстве не были верны, кто поверит вам истинное?
\vs Luk 16:12 И если в чужом не были верны, кто даст вам ваше?
\vs Luk 16:13 Никакой слуга не может служить двум господам, ибо или одного будет ненавидеть, а другого любить, или одному станет усердствовать, а о другом нерадеть. Не можете служить Богу и маммоне.
\rsbpar\vs Luk 16:14 Слышали всё это и фарисеи, которые были сребролюбивы, и они смеялись над Ним.
\vs Luk 16:15 Он сказал им: вы выказываете себя праведниками пред людьми, но Бог знает сердц\acc{а} ваши, ибо что высоко у людей, т\acc{о} мерзость пред Богом.
\vs Luk 16:16 Закон и пророки до Иоанна; с сего времени Царствие Божие благовествуется, и всякий усилием входит в него.
\vs Luk 16:17 Но скорее небо и земля прейдут, нежели одна черта из закона пропадет.
\vs Luk 16:18 Всякий, разводящийся с женою своею и женящийся на другой, прелюбодействует, и всякий, женящийся на разведенной с мужем, прелюбодействует.
\rsbpar\vs Luk 16:19 Некоторый человек был богат, одевался в порфиру и виссон и каждый день пиршествовал блистательно.
\vs Luk 16:20 Был также некоторый нищий, именем Лазарь, который лежал у ворот его в струпьях
\vs Luk 16:21 и желал напитаться крошками, падающими со стола богача, и псы, приходя, лизали струпья его.
\vs Luk 16:22 Умер нищий и отнесен был Ангелами на лоно Авраамово. Умер и богач, и похоронили его.
\vs Luk 16:23 И в аде, будучи в муках, он поднял глаза свои, увидел вдали Авраама и Лазаря на лоне его
\vs Luk 16:24 и, возопив, сказал: отче Аврааме! умилосердись надо мною и пошли Лазаря, чтобы омочил конец перста своего в воде и прохладил язык мой, ибо я мучаюсь в пламени сем.
\vs Luk 16:25 Но Авраам сказал: чадо! вспомни, что ты получил уже доброе твое в жизни твоей, а Лазарь~--- злое; ныне же он здесь утешается, а ты страдаешь;
\vs Luk 16:26 и сверх всего того между нами и вами утверждена великая пропасть, так что хотящие перейти отсюда к вам не могут, также и оттуда к нам не переходят.
\vs Luk 16:27 Тогда сказал он: так прошу тебя, отче, пошли его в дом отца моего,
\vs Luk 16:28 ибо у меня пять братьев; пусть он засвидетельствует им, чтобы и они не пришли в это место мучения.
\vs Luk 16:29 Авраам сказал ему: у них есть Моисей и пророки; пусть слушают их.
\vs Luk 16:30 Он же сказал: нет, отче Аврааме, но если кто из мертвых придет к ним, покаются.
\vs Luk 16:31 Тогда \bibemph{Авраам} сказал ему: если Моисея и пророков не слушают, то если бы кто и из мертвых воскрес, не поверят.
\vs Luk 17:1 Сказал также \bibemph{Иисус} ученикам: невозможно не прийти соблазнам, но горе тому, через кого они приходят;
\vs Luk 17:2 лучше было бы ему, если бы мельничный жернов повесили ему на шею и бросили его в море, нежели чтобы он соблазнил одного из малых сих.
\vs Luk 17:3 Наблюдайте за собою. Если же согрешит против тебя брат твой, выговори ему; и если покается, прости ему;
\vs Luk 17:4 и если семь раз в день согрешит против тебя и семь раз в день обратится, и скажет: каюсь,~--- прости ему.
\rsbpar\vs Luk 17:5 И сказали Апостолы Господу: умножь в нас веру.
\vs Luk 17:6 Господь сказал: если бы вы имели веру с зерно горчичное и сказали смоковнице сей: исторгнись и пересадись в море, то она послушалась бы вас.
\vs Luk 17:7 Кто из вас, имея раба п\acc{а}шущего или пасущего, по возвращении его с поля, скажет ему: пойди скорее, садись за стол?
\vs Luk 17:8 Напротив, не скажет ли ему: приготовь мне поужинать и, подпоясавшись, служи мне, пока буду есть и пить, и потом ешь и пей сам?
\vs Luk 17:9 Станет ли он благодарить раба сего за то, что он исполнил приказание? Не думаю.
\vs Luk 17:10 Так и вы, когда исполните всё повеленное вам, говорите: мы рабы ничего не стоящие, потому что сделали, чт\acc{о} должны были сделать.
\rsbpar\vs Luk 17:11 Идя в Иерусалим, Он проходил между Самариею и Галилеею.
\vs Luk 17:12 И когда входил Он в одно селение, встретили Его десять человек прокаженных, которые остановились вдали
\vs Luk 17:13 и громким голосом говорили: Иисус Наставник! помилуй нас.
\vs Luk 17:14 Увидев \bibemph{их}, Он сказал им: пойдите, покажитесь священникам. И когда они шли, очистились.
\vs Luk 17:15 Один же из них, видя, что исцелен, возвратился, громким голосом прославляя Бога,
\vs Luk 17:16 и пал ниц к ногам Его, благодаря Его; и это был Самарянин.
\vs Luk 17:17 Тогда Иисус сказал: не десять ли очистились? где же девять?
\vs Luk 17:18 как они не возвратились воздать славу Богу, кроме сего иноплеменника?
\vs Luk 17:19 И сказал ему: встань, иди; вера твоя спасла тебя.
\rsbpar\vs Luk 17:20 Быв же спрошен фарисеями, когда придет Царствие Божие, отвечал им: не придет Царствие Божие приметным образом,
\vs Luk 17:21 и не скажут: вот, оно здесь, или: вот, там. Ибо вот, Царствие Божие внутрь вас есть.
\vs Luk 17:22 Сказал также ученикам: придут дни, когда пожелаете видеть хотя один из дней Сына Человеческого, и не увидите;
\vs Luk 17:23 и скажут вам: вот, здесь, или: вот, там,~--- не ходите и не гоняйтесь,
\vs Luk 17:24 ибо, как молния, сверкнувшая от одного края неба, блистает до другого края неба, так будет Сын Человеческий в день Свой.
\vs Luk 17:25 Но прежде надлежит Ему много пострадать и быть отвержену родом сим.
\vs Luk 17:26 И как было во дни Ноя, так будет и во дни Сына Человеческого:
\vs Luk 17:27 ели, пили, женились, выходили замуж, до того дня, как вошел Ной в ковчег, и пришел потоп и погубил всех.
\vs Luk 17:28 Т\acc{а}к же, к\acc{а}к было и во дни Лота: ели, пили, покупали, продавали, садили, строили;
\vs Luk 17:29 но в день, в который Лот вышел из Содома, пролился с неба дождь огненный и серный и истребил всех;
\vs Luk 17:30 так будет и в тот день, когда Сын Человеческий явится.
\vs Luk 17:31 В тот день, кто будет на кровле, а вещи его в доме, тот не сходи взять их; и кто будет на поле, также не обращайся назад.
\vs Luk 17:32 Вспоминайте жену Лотову.
\vs Luk 17:33 Кто станет сберегать душу свою, тот погубит ее; а кто погубит ее, тот оживит ее.
\vs Luk 17:34 Сказываю вам: в ту ночь будут двое на одной постели: один возьмется, а другой оставится;
\vs Luk 17:35 две будут молоть вместе: одна возьмется, а другая оставится;
\vs Luk 17:36 двое будут на поле: один возьмется, а другой оставится.
\vs Luk 17:37 На это сказали Ему: где, Господи? Он же сказал им: где труп, там соберутся и орлы.
\vs Luk 18:1 Сказал также им притчу о том, что должно всегда молиться и не унывать,
\vs Luk 18:2 говоря: в одном городе был судья, который Бога не боялся и людей не стыдился.
\vs Luk 18:3 В том же городе была одна вдова, и она, приходя к нему, говорила: защити меня от соперника моего.
\vs Luk 18:4 Но он долгое время не хотел. А после сказал сам в себе: хотя я и Бога не боюсь и людей не стыжусь,
\vs Luk 18:5 но, как эта вдова не дает мне покоя, защищу ее, чтобы она не приходила больше докучать мне.
\vs Luk 18:6 И сказал Господь: слышите, что говорит судья неправедный?
\vs Luk 18:7 Бог ли не защитит избранных Своих, вопиющих к Нему день и ночь, хотя и медлит защищать их?
\vs Luk 18:8 сказываю вам, что подаст им защиту вскоре. Но Сын Человеческий, придя, найдет ли веру на земле?
\rsbpar\vs Luk 18:9 Сказал также к некоторым, которые уверены были о себе, что они праведны, и уничижали других, следующую притчу:
\vs Luk 18:10 два человека вошли в храм помолиться: один фарисей, а другой мытарь.
\vs Luk 18:11 Фарисей, став, молился сам в себе так: Боже! благодарю Тебя, что я не таков, как прочие люди, грабители, обидчики, прелюбодеи, или как этот мытарь:
\vs Luk 18:12 пощусь два раза в неделю, даю десятую часть из всего, чт\acc{о} приобретаю.
\vs Luk 18:13 Мытарь же, стоя вдали, не смел даже поднять глаз на небо; но, ударяя себя в грудь, говорил: Боже! будь милостив ко мне грешнику!
\vs Luk 18:14 Сказываю вам, что сей пошел оправданным в дом свой более, нежели тот: ибо всякий, возвышающий сам себя, унижен будет, а унижающий себя возвысится.
\rsbpar\vs Luk 18:15 Приносили к Нему и младенцев, чтобы Он прикоснулся к ним; ученики же, видя то, возбраняли им.
\vs Luk 18:16 Но Иисус, подозвав их, сказал: пустите детей приходить ко Мне и не возбраняйте им, ибо таковых есть Царствие Божие.
\vs Luk 18:17 Истинно говорю вам: кто не примет Царствия Божия, как дитя, тот не войдет в него.
\rsbpar\vs Luk 18:18 И спросил Его некто из начальствующих: Учитель благий! что мне делать, чтобы наследовать жизнь вечную?
\vs Luk 18:19 Иисус сказал ему: что ты называешь Меня благим? никто не благ, как только один Бог;
\vs Luk 18:20 знаешь заповеди: не прелюбодействуй, не убивай, не кради, не лжесвидетельствуй, почитай отца твоего и матерь твою.
\vs Luk 18:21 Он же сказал: все это сохранил я от юности моей.
\vs Luk 18:22 Услышав это, Иисус сказал ему: еще одного недостает тебе: все, что имеешь, продай и раздай нищим, и будешь иметь сокровище на небесах, и приходи, следуй за Мною.
\vs Luk 18:23 Он же, услышав сие, опечалился, потому что был очень богат.
\vs Luk 18:24 Иисус, видя, что он опечалился, сказал: как трудно имеющим богатство войти в Царствие Божие!
\vs Luk 18:25 ибо удобнее верблюду пройти сквозь игольные уши, нежели богатому войти в Царствие Божие.
\vs Luk 18:26 Слышавшие сие сказали: кто же может спастись?
\vs Luk 18:27 Но Он сказал: невозможное человекам возможно Богу.
\rsbpar\vs Luk 18:28 Петр же сказал: вот, мы оставили все и последовали за Тобою.
\vs Luk 18:29 Он сказал им: истинно говорю вам: нет никого, кто оставил бы дом, или родителей, или братьев, или сестер, или жену, или детей для Царствия Божия,
\vs Luk 18:30 и не получил бы гораздо более в сие время, и в век будущий жизни вечной.
\rsbpar\vs Luk 18:31 Отозвав же двенадцать учеников Своих, сказал им: вот, мы восходим в Иерусалим, и совершится все, написанное через пророков о Сыне Человеческом,
\vs Luk 18:32 ибо предадут Его язычникам, и поругаются над Ним, и оскорбят Его, и оплюют Его,
\vs Luk 18:33 и будут бить, и убьют Его: и в третий день воскреснет.
\vs Luk 18:34 Но они ничего из этого не поняли; слова сии были для них сокровенны, и они не разумели сказанного.
\rsbpar\vs Luk 18:35 Когда же подходил Он к Иерихону, один слепой сидел у дороги, прося милостыни,
\vs Luk 18:36 и, услышав, что мимо него проходит народ, спросил: что это такое?
\vs Luk 18:37 Ему сказали, что Иисус Назорей идет.
\vs Luk 18:38 Тогда он закричал: Иисус, Сын Давидов! помилуй меня.
\vs Luk 18:39 Шедшие впереди заставляли его молчать; но он еще громче кричал: Сын Давидов! помилуй меня.
\vs Luk 18:40 Иисус, остановившись, велел привести его к Себе: и, когда тот подошел к Нему, спросил его:
\vs Luk 18:41 чего ты хочешь от Меня? Он сказал: Господи! чтобы мне прозреть.
\vs Luk 18:42 Иисус сказал ему: прозри! вера твоя спасла тебя.
\vs Luk 18:43 И он тотчас прозрел и пошел за Ним, славя Бога; и весь народ, видя это, воздал хвалу Богу.
\vs Luk 19:1 Потом \bibemph{Иисус} вошел в Иерихон и проходил через него.
\vs Luk 19:2 И вот, некто, именем Закхей, начальник мытарей и человек богатый,
\vs Luk 19:3 искал видеть Иисуса, кто Он, но не мог за народом, потому что мал был ростом,
\vs Luk 19:4 и, забежав вперед, взлез на смоковницу, чтобы увидеть Его, потому что Ему надлежало проходить мимо нее.
\vs Luk 19:5 Иисус, когда пришел на это место, взглянув, увидел его и сказал ему: Закхей! сойди скорее, ибо сегодня надобно Мне быть у тебя в доме.
\vs Luk 19:6 И он поспешно сошел и принял Его с радостью.
\vs Luk 19:7 И все, видя то, начали роптать, и говорили, что Он зашел к грешному человеку;
\vs Luk 19:8 Закхей же, став, сказал Господу: Господи! половину имения моего я отдам нищим, и, если кого чем обидел, воздам вчетверо.
\vs Luk 19:9 Иисус сказал ему: ныне пришло спасение дому сему, потому что и он сын Авраама,
\vs Luk 19:10 ибо Сын Человеческий пришел взыскать и спасти погибшее.
\rsbpar\vs Luk 19:11 Когда же они слушали это, присовокупил притчу: ибо Он был близ Иерусалима, и они думали, что скоро должно открыться Царствие Божие.
\vs Luk 19:12 Итак сказал: некоторый человек высокого рода отправлялся в дальнюю страну, чтобы получить себе царство и возвратиться;
\vs Luk 19:13 призвав же десять рабов своих, дал им десять мин\fns{Фунтов серебра.} и сказал им: употребляйте их в оборот, пока я возвращусь.
\vs Luk 19:14 Но граждане ненавидели его и отправили вслед за ним посольство, сказав: не хотим, чтобы он царствовал над нами.
\vs Luk 19:15 И когда возвратился, получив царство, велел призвать к себе рабов тех, которым дал серебро, чтобы узнать, кто что приобрел.
\vs Luk 19:16 Пришел первый и сказал: господин! мина твоя принесла десять мин.
\vs Luk 19:17 И сказал ему: хорошо, добрый раб! за то, что ты в малом был верен, возьми в управление десять городов.
\vs Luk 19:18 Пришел второй и сказал: господин! мина твоя принесла пять мин.
\vs Luk 19:19 Сказал и этому: и ты будь над пятью городами.
\vs Luk 19:20 Пришел третий и сказал: господин! вот твоя мина, которую я хранил, завернув в платок,
\vs Luk 19:21 ибо я боялся тебя, потому что ты человек жестокий: берешь, чего не клал, и жнешь, чего не сеял.
\vs Luk 19:22 \bibemph{Господин} сказал ему: твоими устами буду судить тебя, лукавый раб! ты знал, что я человек жестокий, беру, чего не клал, и жну, чего не сеял;
\vs Luk 19:23 для чего же ты не отдал серебра моего в оборот, чтобы я, придя, получил его с прибылью?
\vs Luk 19:24 И сказал предстоящим: возьмите у него мину и дайте имеющему десять мин.
\vs Luk 19:25 И сказали ему: господин! у него есть десять мин.
\vs Luk 19:26 Сказываю вам, что всякому имеющему дано будет, а у неимеющего отнимется и то, что имеет;
\vs Luk 19:27 врагов же моих тех, которые не хотели, чтобы я царствовал над ними, приведите сюда и избейте предо мною.
\vs Luk 19:28 Сказав это, Он пошел далее, восходя в Иерусалим.
\rsbpar\vs Luk 19:29 И когда приблизился к Виффагии и Вифании, к горе, называемой Елеонскою, послал двух учеников Своих,
\vs Luk 19:30 сказав: пойдите в противолежащее селение; войдя в него, найдете молодого осла привязанного, на которого никто из людей никогда не садился; отвязав его, приведите;
\vs Luk 19:31 и если кто спросит вас: зачем отвязываете? скажите ему так: он надобен Господу.
\vs Luk 19:32 Посланные пошли и нашли, как Он сказал им.
\vs Luk 19:33 Когда же они отвязывали молодого осла, хозяева его сказали им: зачем отвязываете осленка?
\vs Luk 19:34 Они отвечали: он надобен Господу.
\vs Luk 19:35 И привели его к Иисусу, и, накинув одежды свои на осленка, посадили на него Иисуса.
\vs Luk 19:36 И, когда Он ехал, постилали одежды свои по дороге.
\vs Luk 19:37 А когда Он приблизился к спуску с горы Елеонской, все множество учеников начало в радости велегласно славить Бога за все чудеса, какие видели они,
\vs Luk 19:38 говоря: благословен Царь, грядущий во имя Господне! мир на небесах и слава в вышних!
\vs Luk 19:39 И некоторые фарисеи из среды народа сказали Ему: Учитель! запрети ученикам Твоим.
\vs Luk 19:40 Но Он сказал им в ответ: сказываю вам, что если они умолкнут, то камни возопиют.
\vs Luk 19:41 И когда приблизился к городу, то, смотря на него, заплакал о нем
\vs Luk 19:42 и сказал: о, если бы и ты хотя в сей твой день узнал, что служит к миру твоему! Но это сокрыто ныне от глаз твоих,
\vs Luk 19:43 ибо придут на тебя дни, когда враги твои обложат тебя окопами и окружат тебя, и стеснят тебя отовсюду,
\vs Luk 19:44 и разорят тебя, и побьют детей твоих в тебе, и не оставят в тебе камня на камне за т\acc{о}, что ты не узнал времени посещения твоего.
\vs Luk 19:45 И, войдя в храм, начал выгонять продающих в нем и покупающих,
\vs Luk 19:46 говоря им: написано: дом Мой есть дом молитвы, а вы сделали его вертепом разбойников.
\vs Luk 19:47 И учил каждый день в храме. Первосвященники же и книжники и старейшины народа искали погубить Его,
\vs Luk 19:48 и не находили, что бы сделать с Ним; потому что весь народ неотступно слушал Его.
\vs Luk 20:1 В один из тех дней, когда Он учил народ в храме и благовествовал, приступили первосвященники и книжники со старейшинами,
\vs Luk 20:2 и сказали Ему: скажи нам, какою властью Ты это делаешь, или кто дал Тебе власть сию?
\vs Luk 20:3 Он сказал им в ответ: спрошу и Я вас об одном, и скажите Мне:
\vs Luk 20:4 крещение Иоанново с небес было, или от человеков?
\vs Luk 20:5 Они же, рассуждая между собою, говорили: если скажем: с небес, то скажет: почему же вы не поверили ему?
\vs Luk 20:6 а если скажем: от человеков, то весь народ побьет нас камнями, ибо он уверен, что Иоанн есть пророк.
\vs Luk 20:7 И отвечали: не знаем откуда.
\vs Luk 20:8 Иисус сказал им: и Я не скажу вам, какою властью это делаю.
\rsbpar\vs Luk 20:9 И начал Он говорить к народу притчу сию: один человек насадил виноградник и отдал его виноградарям, и отлучился на долгое время;
\vs Luk 20:10 и в свое время послал к виноградарям раба, чтобы они дали ему плодов из виноградника; но виноградари, прибив его, отослали ни с чем.
\vs Luk 20:11 Еще послал другого раба; но они и этого, прибив и обругав, отослали ни с чем.
\vs Luk 20:12 И еще послал третьего; но они и того, изранив, выгнали.
\vs Luk 20:13 Тогда сказал господин виноградника: что мне делать? Пошлю сына моего возлюбленного; может быть, увидев его, постыдятся.
\vs Luk 20:14 Но виноградари, увидев его, рассуждали между собою, говоря: это наследник; пойдем, убьем его, и наследство его будет наше.
\vs Luk 20:15 И, выведя его вон из виноградника, убили. Что же сделает с ними господин виноградника?
\vs Luk 20:16 Придет и погубит виноградарей тех, и отдаст виноградник другим. Слышавшие же это сказали: да не будет!
\vs Luk 20:17 Но Он, взглянув на них, сказал: что значит сие написанное: камень, который отвергли строители, тот самый сделался главою угла?
\vs Luk 20:18 Всякий, кто упадет на тот камень, разобьется, а на кого он упадет, того раздавит.
\vs Luk 20:19 И искали в это время первосвященники и книжники, чтобы наложить на Него руки, но побоялись народа, ибо поняли, что о них сказал Он эту притчу.
\vs Luk 20:20 И, наблюдая за Ним, подослали лукавых людей, которые, притворившись благочестивыми, уловили бы Его в каком-либо слове, чтобы предать Его начальству и власти правителя.
\vs Luk 20:21 И они спросили Его: Учитель! мы знаем, что Ты правдиво говоришь и учишь и не смотришь на лице, но истинно пути Божию учишь;
\vs Luk 20:22 позволительно ли нам давать подать кесарю, или нет?
\vs Luk 20:23 Он же, уразумев лукавство их, сказал им: что вы Меня искушаете?
\vs Luk 20:24 Покажите Мне динарий: чье на нем изображение и надпись? Они отвечали: кесаревы.
\vs Luk 20:25 Он сказал им: итак, отдавайте кесарево кесарю, а Божие Богу.
\vs Luk 20:26 И не могли уловить Его в слове перед народом, и, удивившись ответу Его, замолчали.
\rsbpar\vs Luk 20:27 Тогда пришли некоторые из саддукеев, отвергающих воскресение, и спросили Его:
\vs Luk 20:28 Учитель! Моисей написал нам, что если у кого умрет брат, имевший жену, и умрет бездетным, то брат его должен взять его жену и восставить семя брату своему.
\vs Luk 20:29 Было семь братьев, первый, взяв жену, умер бездетным;
\vs Luk 20:30 взял ту жену второй, и тот умер бездетным;
\vs Luk 20:31 взял ее третий; также и все семеро, и умерли, не оставив детей;
\vs Luk 20:32 после всех умерла и жена;
\vs Luk 20:33 итак, в воскресение которого из них будет она женою, ибо семеро имели ее женою?
\vs Luk 20:34 Иисус сказал им в ответ: чада века сего женятся и выходят замуж;
\vs Luk 20:35 а сподобившиеся достигнуть того века и воскресения из мертвых ни женятся, ни замуж не выходят,
\vs Luk 20:36 и умереть уже не могут, ибо они равны Ангелам и суть сыны Божии, будучи сынами воскресения.
\vs Luk 20:37 А что мертвые воскреснут, и Моисей показал при купине, когда назвал Господа Богом Авраама и Богом Исаака и Богом Иакова.
\vs Luk 20:38 Бог же не есть \bibemph{Бог} мертвых, но живых, ибо у Него все живы.
\vs Luk 20:39 На это некоторые из книжников сказали: Учитель! Ты хорошо сказал.
\vs Luk 20:40 И уже не смели спрашивать Его ни о чем. Он же сказал им:
\vs Luk 20:41 к\acc{а}к говорят, что Христос есть Сын Давидов,
\vs Luk 20:42 а сам Давид говорит в книге псалмов: сказал Господь Господу моему: седи одесную Меня,
\vs Luk 20:43 доколе положу врагов Твоих в подножие ног Твоих?
\vs Luk 20:44 Итак, Давид Господом называет Его; как же Он Сын ему?
\vs Luk 20:45 И когда слушал весь народ, Он сказал ученикам Своим:
\vs Luk 20:46 остерегайтесь книжников, которые любят ходить в длинных одеждах и любят приветствия в народных собраниях, председания в синагогах и предвозлежания на пиршествах,
\vs Luk 20:47 которые поедают д\acc{о}мы вдов и лицемерно долго молятся; они примут тем большее осуждение.
\vs Luk 21:1 Взглянув же, Он увидел богатых, клавших дары свои в сокровищницу;
\vs Luk 21:2 увидел также и бедную вдову, положившую туда две лепты,
\vs Luk 21:3 и сказал: истинно говорю вам, что эта бедная вдова больше всех положила;
\vs Luk 21:4 ибо все те от избытка своего положили в дар Богу, а она от скудости своей положила все пропитание свое, какое имела.
\rsbpar\vs Luk 21:5 И когда некоторые говорили о храме, что он украшен дорогими камнями и вкладами, Он сказал:
\vs Luk 21:6 придут дни, в которые из того, что вы здесь видите, не останется камня на камне; все будет разрушено.
\vs Luk 21:7 И спросили Его: Учитель! когда же это будет? и какой признак, когда это должно произойти?
\vs Luk 21:8 Он сказал: берегитесь, чтобы вас не ввели в заблуждение, ибо многие придут под именем Моим, говоря, что это Я; и это время близко: не ходите вслед их.
\vs Luk 21:9 Когда же услышите о войнах и смятениях, не ужасайтесь, ибо этому надлежит быть прежде; но не тотчас конец.
\vs Luk 21:10 Тогда сказал им: восстанет народ на народ, и царство на царство;
\vs Luk 21:11 будут большие землетрясения по местам, и глады, и моры, и ужасные явления, и великие знамения с неба.
\vs Luk 21:12 Прежде же всего того возложат на вас руки и будут гнать \bibemph{вас}, предавая в синагоги и в темницы, и поведут пред царей и правителей за имя Мое;
\vs Luk 21:13 будет же это вам для свидетельства.
\vs Luk 21:14 Итак положите себе на сердце не обдумывать заранее, что отвечать,
\vs Luk 21:15 ибо Я дам вам уста и премудрость, которой не возмогут противоречить ни противостоять все, противящиеся вам.
\vs Luk 21:16 Преданы также будете и родителями, и братьями, и родственниками, и друзьями, и некоторых из вас умертвят;
\vs Luk 21:17 и будете ненавидимы всеми за имя Мое,
\vs Luk 21:18 но и волос с головы вашей не пропадет,~---
\vs Luk 21:19 терпением вашим спасайте души ваши.
\vs Luk 21:20 Когда же увидите Иерусалим, окруженный войсками, тогда знайте, что приблизилось запустение его:
\vs Luk 21:21 тогда находящиеся в Иудее да бегут в горы; и кто в городе, выходи из него; и кто в окрестностях, не входи в него,
\vs Luk 21:22 потому что это дни отмщения, да исполнится все написанное.
\vs Luk 21:23 Горе же беременным и питающим сосцами в те дни; ибо великое будет бедствие на земле и гнев на народ сей:
\vs Luk 21:24 и падут от острия меча, и отведутся в плен во все народы; и Иерусалим будет попираем язычниками, доколе не окончатся времена язычников.
\vs Luk 21:25 И будут знамения в солнце и луне и звездах, а на земле уныние народов и недоумение; и море восшумит и возмутится;
\vs Luk 21:26 люди будут издыхать от страха и ожидания \bibemph{бедствий}, грядущих на вселенную, ибо силы небесные поколеблются,
\vs Luk 21:27 и тогда увидят Сына Человеческого, грядущего на облаке с силою и славою великою.
\vs Luk 21:28 Когда же начнет это сбываться, тогда восклонитесь и поднимите головы ваши, потому что приближается избавление ваше.
\vs Luk 21:29 И сказал им притчу: посмотрите на смоковницу и на все деревья:
\vs Luk 21:30 когда они уже распускаются, то, видя это, знаете сами, что уже близко лето.
\vs Luk 21:31 Так, и когда вы увидите то сбывающимся, знайте, что близко Царствие Божие.
\vs Luk 21:32 Истинно говорю вам: не прейдет род сей, как все это будет;
\vs Luk 21:33 небо и земля прейдут, но слова Мои не прейдут.
\vs Luk 21:34 Смотрите же за собою, чтобы сердца ваши не отягчались объядением и пьянством и заботами житейскими, и чтобы день тот не постиг вас внезапно,
\vs Luk 21:35 ибо он, как сеть, найдет на всех живущих по всему лицу земному;
\vs Luk 21:36 итак бодрствуйте на всякое время и молитесь, да сподобитесь избежать всех сих будущих \bibemph{бедствий} и предстать пред Сына Человеческого.
\rsbpar\vs Luk 21:37 Днем Он учил в храме, а ночи, выходя, проводил на горе, называемой Елеонскою.
\vs Luk 21:38 И весь народ с утра приходил к Нему в храм слушать Его.
\vs Luk 22:1 Приближался праздник опресноков, называемый Пасхою,
\vs Luk 22:2 и искали первосвященники и книжники, как бы погубить Его, потому что боялись народа.
\vs Luk 22:3 Вошел же сатана в Иуду, прозванного Искариотом, одного из числа двенадцати,
\vs Luk 22:4 и он пошел, и говорил с первосвященниками и начальниками, как Его предать им.
\vs Luk 22:5 Они обрадовались и согласились дать ему денег;
\vs Luk 22:6 и он обещал, и искал удобного времени, чтобы предать Его им не при народе.
\rsbpar\vs Luk 22:7 Настал же день опресноков, в который надлежало заколать пасхального \bibemph{агнца},
\vs Luk 22:8 и послал \bibemph{Иисус} Петра и Иоанна, сказав: пойдите, приготовьте нам есть пасху.
\vs Luk 22:9 Они же сказали Ему: где велишь нам приготовить?
\vs Luk 22:10 Он сказал им: вот, при входе вашем в город, встретится с вами человек, несущий кувшин воды; последуйте за ним в дом, в который войдет он,
\vs Luk 22:11 и скажите хозяину дома: Учитель говорит тебе: где комната, в которой бы Мне есть пасху с учениками Моими?
\vs Luk 22:12 И он покажет вам горницу большую устланную; там приготовьте.
\vs Luk 22:13 Они пошли, и нашли, как сказал им, и приготовили пасху.
\rsbpar\vs Luk 22:14 И когда настал час, Он возлег, и двенадцать Апостолов с Ним,
\vs Luk 22:15 и сказал им: очень желал Я есть с вами сию пасху прежде Моего страдания,
\vs Luk 22:16 ибо сказываю вам, что уже не буду есть ее, пока она не совершится в Царствии Божием.
\vs Luk 22:17 И, взяв чашу и благодарив, сказал: приимите ее и разделите между собою,
\vs Luk 22:18 ибо сказываю вам, что не буду пить от плода виноградного, доколе не придет Царствие Божие.
\vs Luk 22:19 И, взяв хлеб и благодарив, преломил и подал им, говоря: сие есть тело Мое, которое за вас предается; сие творите в Мое воспоминание.
\vs Luk 22:20 Также и чашу после вечери, говоря: сия чаша \bibemph{есть} Новый Завет в Моей крови, которая за вас проливается.
\vs Luk 22:21 И вот, рука предающего Меня со Мною за столом;
\vs Luk 22:22 впрочем, Сын Человеческий идет по предназначению, но горе тому человеку, которым Он предается.
\vs Luk 22:23 И они начали спрашивать друг друга, кто бы из них был, который это сделает.
\vs Luk 22:24 Был же и спор между ними, кто из них должен почитаться б\acc{о}льшим.
\vs Luk 22:25 Он же сказал им: цари господствуют над народами, и владеющие ими благодетелями называются,
\vs Luk 22:26 а вы не так: но кто из вас больше, будь как меньший, и начальствующий~--- как служащий.
\vs Luk 22:27 Ибо кто больше: возлежащий, или служащий? не возлежащий ли? А Я посреди вас, как служащий.
\vs Luk 22:28 Но вы пребыли со Мною в напастях Моих,
\vs Luk 22:29 и Я завещаваю вам, как завещал Мне Отец Мой, Царство,
\vs Luk 22:30 да ядите и пиете за трапезою Моею в Царстве Моем, и сядете на престолах судить двенадцать колен Израилевых.
\vs Luk 22:31 И сказал Господь: Симон! Симон! се, сатана просил, чтобы сеять вас как пшеницу,
\vs Luk 22:32 но Я молился о тебе, чтобы не оскудела вера твоя; и ты некогда, обратившись, утверди братьев твоих.
\vs Luk 22:33 Он отвечал Ему: Господи! с Тобою я готов и в темницу и на смерть идти.
\vs Luk 22:34 Но Он сказал: говорю тебе, Петр, не пропоет петух сегодня, как ты трижды отречешься, что не знаешь Меня.
\vs Luk 22:35 И сказал им: когда Я посылал вас без мешка и без сум\acc{ы} и без обуви, имели ли вы в чем недостаток? Они отвечали: ни в чем.
\vs Luk 22:36 Тогда Он сказал им: но теперь, кто имеет мешок, тот возьми его, также и сум\acc{у}; а у кого нет, продай одежду свою и купи меч;
\vs Luk 22:37 ибо сказываю вам, что должно исполниться на Мне и сему написанному: и к злодеям причтен. Ибо то, что о Мне, приходит к концу.
\vs Luk 22:38 Они сказали: Господи! вот, здесь два меча. Он сказал им: довольно.
\rsbpar\vs Luk 22:39 И, выйдя, пошел по обыкновению на гору Елеонскую, за Ним последовали и ученики Его.
\vs Luk 22:40 Придя же на место, сказал им: молитесь, чтобы не впасть в искушение.
\vs Luk 22:41 И Сам отошел от них на вержение камня, и, преклонив колени, молился,
\vs Luk 22:42 говоря: Отче! о, если бы Ты благоволил пронести чашу сию мимо Меня! впрочем не Моя воля, но Твоя да будет.
\vs Luk 22:43 Явился же Ему Ангел с небес и укреплял Его.
\vs Luk 22:44 И, находясь в борении, прилежнее молился, и был пот Его, как капли крови, падающие на землю.
\vs Luk 22:45 Встав от молитвы, Он пришел к ученикам, и нашел их спящими от печали
\vs Luk 22:46 и сказал им: что вы спите? встаньте и молитесь, чтобы не впасть в искушение.
\rsbpar\vs Luk 22:47 Когда Он еще говорил это, появился народ, а впереди его шел один из двенадцати, называемый Иуда, и он подошел к Иисусу, чтобы поцеловать Его. Ибо он такой им дал знак: Кого я поцелую, Тот и есть.
\vs Luk 22:48 Иисус же сказал ему: Иуда! целованием ли предаешь Сына Человеческого?
\vs Luk 22:49 Бывшие же с Ним, видя, к чему идет дело, сказали Ему: Господи! не ударить ли нам мечом?
\vs Luk 22:50 И один из них ударил раба первосвященникова, и отсек ему правое ухо.
\vs Luk 22:51 Тогда Иисус сказал: оставьте, довольно. И, коснувшись уха его, исцелил его.
\vs Luk 22:52 Первосвященникам же и начальникам храма и старейшинам, собравшимся против Него, сказал Иисус: как будто на разбойника вышли вы с мечами и кольями, чтобы взять Меня?
\vs Luk 22:53 Каждый день бывал Я с вами в храме, и вы не поднимали на Меня рук, но теперь ваше время и власть тьмы.
\rsbpar\vs Luk 22:54 Взяв Его, повели и привели в дом первосвященника. Петр же следовал издали.
\vs Luk 22:55 Когда они развели огонь среди двора и сели вместе, сел и Петр между ними.
\vs Luk 22:56 Одна служанка, увидев его сидящего у огня и всмотревшись в него, сказала: и этот был с Ним.
\vs Luk 22:57 Но он отрекся от Него, сказав женщине: я не знаю Его.
\vs Luk 22:58 Вскоре потом другой, увидев его, сказал: и ты из них. Но Петр сказал этому человеку: нет!
\vs Luk 22:59 Прошло с час времени, еще некто настоятельно говорил: точно и этот был с Ним, ибо он Галилеянин.
\vs Luk 22:60 Но Петр сказал тому человеку: не знаю, что ты говоришь. И тотчас, когда еще говорил он, запел петух.
\vs Luk 22:61 Тогда Господь, обратившись, взглянул на Петра, и Петр вспомнил слово Господа, как Он сказал ему: прежде нежели пропоет петух, отречешься от Меня трижды.
\vs Luk 22:62 И, выйдя вон, горько заплакал.
\rsbpar\vs Luk 22:63 Люди, державшие Иисуса, ругались над Ним и били Его;
\vs Luk 22:64 и, закрыв Его, ударяли Его по лицу и спрашивали Его: прореки, кто ударил Тебя?
\vs Luk 22:65 И много иных хулений произносили против Него.
\rsbpar\vs Luk 22:66 И как настал день, собрались старейшины народа, первосвященники и книжники, и ввели Его в свой синедрион
\vs Luk 22:67 и сказали: Ты ли Христос? скажи нам. Он сказал им: если скажу вам, вы не поверите;
\vs Luk 22:68 если же и спрошу вас, не будете отвечать Мне и не отпустите \bibemph{Меня};
\vs Luk 22:69 отныне Сын Человеческий воссядет одесную силы Божией.
\vs Luk 22:70 И сказали все: итак, Ты Сын Божий? Он отвечал им: вы говорите, что Я.
\vs Luk 22:71 Они же сказали: какое еще нужно нам свидетельство? ибо мы сами слышали из уст Его.
\vs Luk 23:1 И поднялось все множество их, и повели Его к Пилату,
\vs Luk 23:2 и начали обвинять Его, говоря: мы нашли, что Он развращает народ наш и запрещает давать подать кесарю, называя Себя Христом Царем.
\vs Luk 23:3 Пилат спросил Его: Ты Царь Иудейский? Он сказал ему в ответ: ты говоришь.
\vs Luk 23:4 Пилат сказал первосвященникам и народу: я не нахожу никакой вины в этом человеке.
\vs Luk 23:5 Но они настаивали, говоря, что Он возмущает народ, уча по всей Иудее, начиная от Галилеи до сего места.
\vs Luk 23:6 Пилат, услышав о Галилее, спросил: разве Он Галилеянин?
\vs Luk 23:7 И, узнав, что Он из области Иродовой, послал Его к Ироду, который в эти дни был также в Иерусалиме.
\vs Luk 23:8 Ирод, увидев Иисуса, очень обрадовался, ибо давно желал видеть Его, потому что много слышал о Нем, и надеялся увидеть от Него какое-нибудь чудо,
\vs Luk 23:9 и предлагал Ему многие вопросы, но Он ничего не отвечал ему.
\vs Luk 23:10 Первосвященники же и книжники стояли и усильно обвиняли Его.
\vs Luk 23:11 Но Ирод со своими воинами, уничижив Его и насмеявшись над Ним, одел Его в светлую одежду и отослал обратно к Пилату.
\vs Luk 23:12 И сделались в тот день Пилат и Ирод друзьями между собою, ибо прежде были во вражде друг с другом.
\vs Luk 23:13 Пилат же, созвав первосвященников и начальников и народ,
\vs Luk 23:14 сказал им: вы привели ко мне человека сего, как развращающего народ; и вот, я при вас исследовал и не нашел человека сего виновным ни в чем том, в чем вы обвиняете Его;
\vs Luk 23:15 и Ирод также, ибо я посылал Его к нему; и ничего не найдено в Нем достойного смерти;
\vs Luk 23:16 итак, наказав Его, отпущу.
\vs Luk 23:17 А ему и нужно было для праздника отпустить им одного \bibemph{узника}.
\vs Luk 23:18 Но весь народ стал кричать: смерть Ему! а отпусти нам Варавву.
\vs Luk 23:19 Варавва был посажен в темницу за произведенное в городе возмущение и убийство.
\vs Luk 23:20 Пилат снова возвысил голос, желая отпустить Иисуса.
\vs Luk 23:21 Но они кричали: распни, распни Его!
\vs Luk 23:22 Он в третий раз сказал им: какое же зло сделал Он? я ничего достойного смерти не нашел в Нем; итак, наказав Его, отпущу.
\vs Luk 23:23 Но они продолжали с великим криком требовать, чтобы Он был распят; и превозмог крик их и первосвященников.
\vs Luk 23:24 И Пилат решил быть по прошению их,
\vs Luk 23:25 и отпустил им посаженного за возмущение и убийство в темницу, которого они просили; а Иисуса предал в их волю.
\rsbpar\vs Luk 23:26 И когда повели Его, то, захватив некоего Симона Киринеянина, шедшего с поля, возложили на него крест, чтобы нес за Иисусом.
\vs Luk 23:27 И шло за Ним великое множество народа и женщин, которые плакали и рыдали о Нем.
\vs Luk 23:28 Иисус же, обратившись к ним, сказал: дщери Иерусалимские! не плачьте обо Мне, но плачьте о себе и о детях ваших,
\vs Luk 23:29 ибо приходят дни, в которые скажут: блаженны неплодные, и утробы неродившие, и сосцы непитавшие!
\vs Luk 23:30 тогда начнут говорить горам: падите на нас! и холмам: покройте нас!
\vs Luk 23:31 Ибо если с зеленеющим деревом это делают, то с сухим что будет?
\rsbpar\vs Luk 23:32 Вели с Ним на смерть и двух злодеев.
\vs Luk 23:33 И когда пришли на место, называемое Лобное, там распяли Его и злодеев, одного по правую, а другого по левую сторону.
\vs Luk 23:34 Иисус же говорил: Отче! прости им, ибо не знают, что делают. И делили одежды Его, бросая жребий.
\vs Luk 23:35 И стоял народ и смотрел. Насмехались же вместе с ними и начальники, говоря: других спасал; пусть спасет Себя Самого, если Он Христос, избранный Божий.
\vs Luk 23:36 Также и воины ругались над Ним, подходя и поднося Ему уксус
\vs Luk 23:37 и говоря: если Ты Царь Иудейский, спаси Себя Самого.
\vs Luk 23:38 И была над Ним надпись, написанная словами греческими, римскими и еврейскими: Сей есть Царь Иудейский.
\vs Luk 23:39 Один из повешенных злодеев злословил Его и говорил: если Ты Христос, спаси Себя и нас.
\vs Luk 23:40 Другой же, напротив, унимал его и говорил: или ты не боишься Бога, когда и сам осужден на то же?
\vs Luk 23:41 и мы \bibemph{осуждены} справедливо, потому что достойное по делам нашим приняли, а Он ничего худого не сделал.
\vs Luk 23:42 И сказал Иисусу: помяни меня, Господи, когда приидешь в Царствие Твое!
\vs Luk 23:43 И сказал ему Иисус: истинно говорю тебе, ныне же будешь со Мною в раю.
\rsbpar\vs Luk 23:44 Было же около шестого часа дня, и сделалась тьма по всей земле до часа девятого:
\vs Luk 23:45 и померкло солнце, и завеса в храме раздралась по средине.
\vs Luk 23:46 Иисус, возгласив громким голосом, сказал: Отче! в руки Твои предаю дух Мой. И, сие сказав, испустил дух.
\vs Luk 23:47 Сотник же, видев происходившее, прославил Бога и сказал: истинно человек этот был праведник.
\vs Luk 23:48 И весь народ, сшедшийся на сие зрелище, видя происходившее, возвращался, бия себя в грудь.
\vs Luk 23:49 Все же, знавшие Его, и женщины, следовавшие за Ним из Галилеи, стояли вдали и смотрели на это.
\rsbpar\vs Luk 23:50 Тогда некто, именем Иосиф, член совета, человек добрый и правдивый,
\vs Luk 23:51 не участвовавший в совете и в деле их; из Аримафеи, города Иудейского, ожидавший также Царствия Божия,
\vs Luk 23:52 пришел к Пилату и просил тела Иисусова;
\vs Luk 23:53 и, сняв его, обвил плащаницею и положил его в гробе, высеченном \bibemph{в скале}, где еще никто не был положен.
\vs Luk 23:54 День тот был пятница, и наступала суббота.
\vs Luk 23:55 Последовали также и женщины, пришедшие с Иисусом из Галилеи, и смотрели гроб, и как полагалось тело Его;
\vs Luk 23:56 возвратившись же, приготовили благовония и масти; и в субботу остались в покое по заповеди.
\vs Luk 24:1 В первый же день недели, очень рано, неся приготовленные ароматы, пришли они ко гробу, и вместе с ними некоторые другие;
\vs Luk 24:2 но нашли камень отваленным от гроба.
\vs Luk 24:3 И, войдя, не нашли тела Господа Иисуса.
\vs Luk 24:4 Когда же недоумевали они о сем, вдруг предстали перед ними два мужа в одеждах блистающих.
\vs Luk 24:5 И когда они были в страхе и наклонили лица \bibemph{свои} к земле, сказали им: что вы ищете живого между мертвыми?
\vs Luk 24:6 Его нет здесь: Он воскрес; вспомните, как Он говорил вам, когда был еще в Галилее,
\vs Luk 24:7 сказывая, что Сыну Человеческому надлежит быть предану в руки человеков грешников, и быть распяту, и в третий день воскреснуть.
\vs Luk 24:8 И вспомнили они слова Его;
\vs Luk 24:9 и, возвратившись от гроба, возвестили всё это одиннадцати и всем прочим.
\vs Luk 24:10 То были Магдалина Мария, и Иоанна, и Мария, \bibemph{мать} Иакова, и другие с ними, которые сказали о сем Апостолам.
\vs Luk 24:11 И показались им слова их пустыми, и не поверили им.
\vs Luk 24:12 Но Петр, встав, побежал ко гробу и, наклонившись, увидел только пелены лежащие, и пошел назад, дивясь сам в себе происшедшему.
\rsbpar\vs Luk 24:13 В тот же день двое из них шли в селение, отстоящее стадий на шестьдесят от Иерусалима, называемое Эммаус;
\vs Luk 24:14 и разговаривали между собою о всех сих событиях.
\vs Luk 24:15 И когда они разговаривали и рассуждали между собою, и Сам Иисус, приблизившись, пошел с ними.
\vs Luk 24:16 Но глаза их были удержаны, так что они не узнали Его.
\vs Luk 24:17 Он же сказал им: о чем это вы, идя, рассуждаете между собою, и отчего вы печальны?
\vs Luk 24:18 Один из них, именем Клеопа, сказал Ему в ответ: неужели Ты один из пришедших в Иерусалим не знаешь о происшедшем в нем в эти дни?
\vs Luk 24:19 И сказал им: о чем? Они сказали Ему: что было с Иисусом Назарянином, Который был пророк, сильный в деле и слове пред Богом и всем народом;
\vs Luk 24:20 как предали Его первосвященники и начальники наши для осуждения на смерть и распяли Его.
\vs Luk 24:21 А мы надеялись было, что Он есть Тот, Который должен избавить Израиля; но со всем тем, уже третий день ныне, как это произошло.
\vs Luk 24:22 Но и некоторые женщины из наших изумили нас: они были рано у гроба
\vs Luk 24:23 и не нашли тела Его и, придя, сказывали, что они видели и явление Ангелов, которые говорят, что Он жив.
\vs Luk 24:24 И пошли некоторые из наших ко гробу и нашли так, как и женщины говорили, но Его не видели.
\vs Luk 24:25 Тогда Он сказал им: о, несмысленные и медлительные сердцем, чтобы веровать всему, что предсказывали пророки!
\vs Luk 24:26 Не так ли надлежало пострадать Христу и войти в славу Свою?
\vs Luk 24:27 И, начав от Моисея, из всех пророков изъяснял им сказанное о Нем во всем Писании.
\vs Luk 24:28 И приблизились они к тому селению, в которое шли; и Он показывал им вид, что хочет идти далее.
\vs Luk 24:29 Но они удерживали Его, говоря: останься с нами, потому что день уже склонился к вечеру. И Он вошел и остался с ними.
\vs Luk 24:30 И когда Он возлежал с ними, то, взяв хлеб, благословил, преломил и подал им.
\vs Luk 24:31 Тогда открылись у них глаза, и они узнали Его. Но Он стал невидим для них.
\vs Luk 24:32 И они сказали друг другу: не горело ли в нас сердце наше, когда Он говорил нам на дороге и когда изъяснял нам Писание?
\vs Luk 24:33 И, встав в тот же час, возвратились в Иерусалим и нашли вместе одиннадцать \bibemph{Апостолов} и бывших с ними,
\vs Luk 24:34 которые говорили, что Господь истинно воскрес и явился Симону.
\vs Luk 24:35 И они рассказывали о происшедшем на пути, и как Он был узнан ими в преломлении хлеба.
\rsbpar\vs Luk 24:36 Когда они говорили о сем, Сам Иисус стал посреди них и сказал им: мир вам.
\vs Luk 24:37 Они, смутившись и испугавшись, подумали, что видят духа.
\vs Luk 24:38 Но Он сказал им: что смущаетесь, и для чего такие мысли входят в сердца ваши?
\vs Luk 24:39 Посмотрите на руки Мои и на ноги Мои; это Я Сам; осяжите Меня и рассмотр\acc{и}те; ибо дух плоти и костей не имеет, как видите у Меня.
\vs Luk 24:40 И, сказав это, показал им руки и ноги.
\vs Luk 24:41 Когда же они от радости еще не верили и дивились, Он сказал им: есть ли у вас здесь какая пища?
\vs Luk 24:42 Они подали Ему часть печеной рыбы и сотового меда.
\vs Luk 24:43 И, взяв, ел пред ними.
\vs Luk 24:44 И сказал им: вот то, о чем Я вам говорил, еще быв с вами, что надлежит исполниться всему, написанному о Мне в законе Моисеевом и в пророках и псалмах.
\vs Luk 24:45 Тогда отверз им ум к уразумению Писаний.
\vs Luk 24:46 И сказал им: так написано, и так надлежало пострадать Христу, и воскреснуть из мертвых в третий день,
\vs Luk 24:47 и проповедану быть во имя Его покаянию и прощению грехов во всех народах, начиная с Иерусалима.
\vs Luk 24:48 Вы же свидетели сему.
\vs Luk 24:49 И Я пошлю обетование Отца Моего на вас; вы же оставайтесь в городе Иерусалиме, доколе не облечетесь силою свыше.
\rsbpar\vs Luk 24:50 И вывел их вон \bibemph{из города} до Вифании и, подняв руки Свои, благословил их.
\vs Luk 24:51 И, когда благословлял их, стал отдаляться от них и возноситься на небо.
\vs Luk 24:52 Они поклонились Ему и возвратились в Иерусалим с великою радостью.
\vs Luk 24:53 И пребывали всегда в храме, прославляя и благословляя Бога. Аминь.

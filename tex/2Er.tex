\bibbookdescr{2Er}{
  inline={Пастырь Ермы. Книга 2. Заповеди},
  toc={2-я Ермы},
  bookmark={2-я Ермы},
  header={2-я Ермы},
  abbr={2~Ермы}
}
\chhdr{Видение 5-е.}
\vs 2Er 1:1
Когда я
помолился дома и сидел на ложе, вошел ко мне человек почтенного вида, в
пастушеской одежде: на нем был белый плащ, сума за плечами и посох в руке.
\vs 2Er 1:2
Он приветствовал меня, и я
ответил ему также приветствием. Тотчас же он сел подле меня и говорит:
\vs 2Er 1:3
Я послан от
достопоклоняемого ангела, чтобы жить с тобою остальные дни твоей жизни.
\vs 2Er 1:4
Мне показалось, что он
искушает меня, и сказал я ему: кто же ты? Я знаю того, кому препоручен я.
\vs 2Er 1:5
Не узнаешь меня? Нет.
\vs 2Er 1:6
Я тот самый пастырь,
которому препоручен ты.
\vs 2Er 1:7
Пока он говорил, вид его
изменился, и я узнал, что это тот, которому я препоручен.
\vs 2Er 1:8
Тотчас я смутился, объял
меня страх, и весь я разрывался от скорби, что отвечал ему так лукаво и
неразумно.
\vs 2Er 1:9
Он же сказал мне: не
смущайся, но укрепись заповедями, которые услышишь от меня.
\vs 2Er 1:10
Ибо я послан для того,
чтобы снова показать тебе все, что видел ты прежде, и особенно то, что полезно
для вас.
\vs 2Er 1:11
Итак, я приказываю тебе
сперва записать заповеди мои и притчи, чтобы перечитывать их время от времени,
так удобнее будет тебе выполнять их.
\vs 2Er 1:12
Поэтому я записал
заповеди и притчи так, как повелел он мне.
\vs 2Er 1:13
Если, услышав их, вы
будете поступать по ним и исполните их с чистым сердцем, то получите от
Господа то, что обещал Он вам.
\vs 2Er 1:14
Если же, услышав их, не
покаетесь, но обратитесь к грехам вашим, то воспримите от Господа наказание.
\vs 2Er 1:15
Все это повелел мне
записать этот пастырь, ангел покаяния.
 
\chhdr{Заповедь 1-я.}
\vs 2Er 2:1
Прежде
всего веруй, что един есть Бог, все сотворивший и совершивший, приведший все
не из чего в бытие.
\vs 2Er 2:2
Он все объемлет, Сам
будучи необъятен, и не может быть ни словом определен, ни умом постигнут.
\vs 2Er 2:3
Итак, веруй в Него, бойся
Его и, боясь, соблюдай воздержание.
\vs 2Er 2:4
Храни это и отринешь от
себя всякую похоть и беззаконие, и облечешься во всякую добродетель и правду и
будешь жить с Богом, если сохранишь эту заповедь.

\chhdr{Заповедь 2-я.}
\vs 2Er 3:1
Пастырь
сказал мне: имей простоту и будь незлобив, будь как дитя, которое не знает
лукавства, губящего жизнь людей.
\vs 2Er 3:2
Ни о ком не говори худо и
не стремись слушать того, кто говорит худо.
\vs 2Er 3:3
Если же будешь слушать, то
будешь причастен греху злословящего; веря ему, ты будешь подобен ему, потому
что поверил злословящему на брата твоего.
\vs 2Er 3:4
Гибельно злословие: это~--- дух беспокойный,
который никогда не находится в мире, но всегда живёт в несогласии.
\vs 2Er 3:5
Удерживайся от него и
всегда имей мир с братом твоим.
\vs 2Er 3:6
Облекись
благопристойностью, в которой нет ничего оскорбительного, но все ровно и
приятно.
\vs 2Er 3:7
Делай добро и от плода
трудов твоих, который дарует тебе Бог, давай всем бедным просто, нимало не
сомневаясь, кому даешь.
\vs 2Er 3:8
Давай всем, ибо Бог хочет,
чтобы всем досталось от Его даров.
\vs 2Er 3:9
Берущие дадут отчет Богу
почему и на что брали. Берущие по нужде не будут осуждены, а берущие притворно
подвергнутся суду.
\vs 2Er 3:10
Дающий же не будет
виноват: ибо он исполнил служение, назначенное Богом, не разбирая, кому дать и
кому не давать, и исполнил с похвалою пред Богом.
\vs 2Er 3:11
Итак, соблюдай эту
заповедь, как я сказал тебе, чтобы покаяние твоё и семейства твоего было в
простоте и сердце твое явилось чистым и непорочным пред Богом.

\chhdr{Заповедь 3-я.}
\vs 2Er 4:1
Также
сказал он мне: люби истину, и пусть исходит из уст твоих всякая истина,
\vs 2Er 4:2
чтобы дух, который Господь
поселил в этом теле, предстал истинным пред всеми людьми, и чтобы прославлялся
Господь, который дал тебе дух,
\vs 2Er 4:3
потому что Бог истинен во
всяком слове и никакой лжи нет в Нем.
\vs 2Er 4:4
И те, которые лгут,
отвергают Бога и не возвращают Ему залога, который получили; а они получили от
Него дух нелживый.
\vs 2Er 4:5
Если они возвращают его
лживым, то бесчестят заповедь Господа и становятся похитителями.
\vs 2Er 4:6
Услышав это, я горько
заплакал. Видя скорбь мою, он спросил: о чем ты плачешь?
\vs 2Er 4:7
Не знаю, господин, могу ли
спастись я. Почему?
\vs 2Er 4:8
Потому, что никогда в
жизни, господин, не произнес я слова правдивого, но всегда говорил коварно и
выдавал пред всеми ложь за истину; и никто не прекословил мне, потому что
доверяли моему слову.
\vs 2Er 4:9
Как же я могу жить, когда
поступал таким образом?
\vs 2Er 4:10
Он отвечал: ты
рассуждаешь хорошо и верно, ибо следовало тебе, как рабу Божьему, ходить в
истине, и не соединять лукавой совести с духом истины, и не оскорблять Духа
Божьего Святого и истинного.
\vs 2Er 4:11
И я сказал ему: никогда,
господин, я не слышал таких слов.
\vs 2Er 4:12
Услышав сейчас, впредь
соблюдай их и старайся, чтобы и те лживые слова, которые прежде говорил ты,
стали верными от последующих речей твоих, если они окажутся правдивыми.
\vs 2Er 4:13
Ибо и те могут сделаться
верными, если отныне будешь говорить правду; и если будешь соблюдать истину,
можешь получить себе жизнь.
\vs 2Er 4:14
И всякий, кто только
послушается этой заповеди, будет исполнять ее и удаляться от лжи, будет жить
с Богом.

\chhdr{Заповедь 4-я.}
\vs 2Er 5:1
Заповедую тебе, говорил пастырь, соблюдать целомудрие.
\vs 2Er 5:2
И да не войдет тебе в
сердце помысел о чужой жене, или о любодеянии, или о каком-либо подобном
дурном деле, ибо все это великий грех.
\vs 2Er 5:3
А ты помни о Господе во
все часы и никогда не согрешишь.
\vs 2Er 5:4
Если какой низкий помысел
взойдет на твое сердце, то совершишь великий грех; и кто творит такое
преступное дело, обрекает себя на смерть.
\vs 2Er 5:5
Итак, смотри ты,
воздерживайся от таких помыслов.
\vs 2Er 5:6
Ибо где обитает
целомудрие, там никогда не должен возникать худой помысел в сердце человека
праведного.
\vs 2Er 5:7
И попросил я: позволь мне,
господин, спросить тебя немного. Спрашивай.
\vs 2Er 5:8
Если, господин, сказал
я, муж имеет жену верную в Господе и заметит ее в прелюбодеянии, то будет ли
он грешен, живя с нею?
\vs 2Er 5:9
И ответил он мне: доколе
муж не знает греха своей жены, он не грешит, если живет с нею.
\vs 2Er 5:10
Если же узнает муж о
грехе ее, а она не покается в своем прелюбодеянии, то муж согрешит, живя с
нею, и сделается участником в ее прелюбодеянии.
\vs 2Er 5:11
Что же делать, спросил
я, если жена будет оставаться в своем пороке?
\vs 2Er 5:12
Пусть муж отпустит ее и
останется один. Если же, отпустивши свою жену, возьмет другую, то и сам примет
грех прелюбодеяния.
\vs 2Er 5:13
Что же, господин, если
жена отпущенная покается и пожелает возвратиться к мужу своему, то должна ли
она быть принята мужем?
\vs 2Er 5:14
Если не примет ее муж, он
совершит грех великий, он ответил мне.
\vs 2Er 5:15
Должно принимать
грешницу, которая раскаивается, но не много раз. Ибо для рабов Божьих покаяние
положено одно.
\vs 2Er 5:16
Поэтому ради раскаяния не
должен муж, отпустив жену свою, брать себе другую. Так же делать надлежит и
жене.
\vs 2Er 5:17
Но прелюбодейство не
только в осквернении плоти своей: прелюбодействует и тот, кто поступает
подобно народам.
\vs 2Er 5:18
Избегай общения с тем,
кто совершает такие дела и не кается, иначе и ты будешь причастен греху его.
\vs 2Er 5:19
Итак, заповедуется вам,
чтобы вы оставались одинокими как муж, так и жена, ибо в этом случае может
иметь место покаяние.
\vs 2Er 5:20
Но я не даю повода к тому
чтобы так делалось: пусть не грешит более тот, кто уже согрешил.
\vs 2Er 5:21
Что касается прежних
грехов его, то есть Бог, который может дать исцеление, ибо Он имеет власть над
всем.

\vs 2Er 6:1
И я опять просил его:
поскольку Господь удостоил меня того, чтобы ты всегда жил со мною, то дозволь
сказать мне еще несколько слов,
\vs 2Er 6:2
потому что я ничего не
понимаю и сердце мое омрачено прежними делами моими. Вразуми меня, так как я
совершенно ничего не смыслю.
\vs 2Er 6:3
И он в ответ сказал мне: я
приставник покаяния и всем кающимся даю разум. Или самое покаяние, ты думаешь,
не есть великий разум?
\vs 2Er 6:4
Грешник кающийся уразумел,
что он согрешил пред Господом, он осудил всем сердцем содеянные им дела и,
раскаявшись, более уже не делает зла, но совершает добро, и смиряет душу и
мучит ее за то, что согрешила. Итак, понимаешь, что покаяние есть великий
разум?
\vs 2Er 6:5
Потому-то, господин, я и
расспрашиваю тебя подробно обо всем, что я грешник и желаю узнать, что мне
делать, чтобы жить, ибо грехи мои многочисленны и разнообразны.
\vs 2Er 6:6
Ты будешь жив, сказал
он, если сохранишь мои заповеди и будешь поступать по ним; и всякий, кто
только услышит и исполнит эти заповеди, будет жить с Богом.

\vs 2Er 7:1
Я сказал ему: господин, я
слышал от некоторых учителей, что нет иного покаяния, кроме того, когда сходим
в воду и получаем отпущение прежних грехов наших.
\vs 2Er 7:2
Справедливо ты слышал. Ибо
получившему отпущение грехов не должно более грешить, но жить в чистоте.
\vs 2Er 7:3
И так как ты обо всем
расспрашиваешь, объясню тебе это, не давая повода к заблуждению тем, которые
собираются уверовать или только что уверовали в Господа.
\vs 2Er 7:4
Они не имеют покаяния во
грехах, но имеют отпущение прежних грехов своих.
\vs 2Er 7:5
Тем же, которые призваны
прежде, положил Господь покаяние, ибо Он сердцеведец, провидящий все, знал
слабость людей и великое коварство дьявола, который будет сеять вред и злобу
среди рабов Божьих.
\vs 2Er 7:6
Поэтому милосердный
Господь сжалился над своим созданием и положил покаяние, над которым и дана
мне власть.
\vs 2Er 7:7
Итак, я говорю тебе, после
этого великого и святого призвания, если кто, будучи искушен дьяволом,
согрешил, пусть покается.
\vs 2Er 7:8
Если же часто он будет
грешить и творить покаяние, не принесет ему покаяние пользы, ибо с трудом он
будет жить с Богом.
\vs 2Er 7:9
И я сказал: господин, я
обновился, когда услышал об этом так обстоятельно. Ибо я знаю, что спасусь,
если еще не присовокуплю ничего к грехам своим.
\vs 2Er 7:10
Спасешься, ответил он,
ты и все, которые сделают то же.

\vs 2Er 8:1
И опять я попросил его:
господин, так как ты терпеливо меня выслушиваешь, объясни мне еще вот что.
\vs 2Er 8:2
Если муж или жена умрет и
один из них вступит в брак согрешает ли вступающий в брак?
\vs 2Er 8:3
Не согрешает, но если
останется сам по себе, то приобретет себе большую славу у Господа.
\vs 2Er 8:4
Поэтому храни чистоту и
целомудрие и будешь жить с Богом.
\vs 2Er 8:5
То, что я говорю и
собираюсь сказать тебе после, соблюдай с этого самого дня, ибо ты поручен мне
и живу в твоем доме.
\vs 2Er 8:6
И прежним грехам твоим
будет отпущение, если сохранишь мои заповеди; и все, кто сохранит их и будет
ходить в чистоте, получит отпущение.

\chhdr{Заповедь 5-я.}
\vs 2Er 9:1
Будь
великодушен и терпелив, сказал пастырь, и будешь господствовать над всеми
злыми делами и сотворишь всякую правду.
\vs 2Er 9:2
Если будешь великодушен,
то Дух Святой, в тебе обитающий, останется чист и не омрачится от какого-либо
злого духа, но, ликуя, расширится, и вместе с сосудом, в котором обитает,
будет радостно служить Господу.
\vs 2Er 9:3
Если же найдет какой-либо
гнев, то Дух Святой, сущий в тебе, тотчас же будет стеснен и постарается
удалиться,
\vs 2Er 9:4
ибо подавляется злым духом
и, оскорбляемый гневом, не имеет возможности служить Господу, как желает.
\vs 2Er 9:5
Поэтому, когда оба духа
обитают вместе, плохо бывает человеку.
\vs 2Er 9:6
Так, если взять немножко
полыни и положить в сосуд с медом, не весь ли мед испортится?
\vs 2Er 9:7
И столько меда пропадает
от незначительного количества полыни, теряет сладость и уже не имеет
приятности для своего владельца, потому что делается горьким и негодным к
употреблению. Но если в мед не класть полынь, он останется сладок.
\vs 2Er 9:8
Сам видишь, великодушие
слаще меда, и оно угодно Богу и Господь обитает в нем, а гнев горек и
неугоден.
\vs 2Er 9:9
Итак, если к великодушию
примешивается гнев, то дух возмущается, и неприятна Богу молитва его.
\vs 2Er 9:10
И я сказал ему: желал бы
я узнать, господин, действие гнева, чтобы уберечь себя от него.
\vs 2Er 9:11
Если ты и твои домочадцы
не будете удерживаться от него, то потеряете всякую надежду спасения.
\vs 2Er 9:12
Но воздерживайся от
гнева, ибо я с тобою; и от него воздержатся все, которые покаются от всего
сердца своего, ибо я буду с ними и сохраню их.
\vs 2Er 9:13
Все такие принимаются
святейшим ангелом в число праведных.

\vs 2Er 10:1
Послушай теперь и о
действии гнева, как он вреден и как губит рабов Божьих и отвращает их от
правды.
\vs 2Er 10:2
Он не может вредить людям,
исполненным веры, потому что с ними пребывает сила Божья; совращает же
сомневающихся и не имеющих ее.
\vs 2Er 10:3
Как скоро он увидит таких
людей спокойными проникает в сердце их, и муж или жена сердятся друг на
друга по каким-нибудь житейским делам:
\vs 2Er 10:4
или из-за пищи, или
пустого слова, или какого приятеля, или долга, или из-за подобных мелочных
вещей. Все это глупо, пусто и неприлично рабам Божьим.
\vs 2Er 10:5
Но великодушие твердо и
мужественно, имеет крепкую силу и пребывает в великой широте, весело и
беззаботно радуясь, и прославляет Господа во всякое время чуждое всякой
горечи, всегда мирное и кроткое.
\vs 2Er 10:6
Это великодушие живет с
имеющими полную веру. А гнев безрассуден, пуст и легкомыслен.
\vs 2Er 10:7
От безрассудства рождается
огорчение, от огорчения раздражение, от раздражения гнев, от гнева же
неистовство.
\vs 2Er 10:8
Неистовство, происшедшее
от стольких зол, есть великий и неискупимый грех.
\vs 2Er 10:9
И когда все это находится
в одном сосуде, где обитает и Дух Святой, то сосуд не вмещает их в себе, но
переполняется:
\vs 2Er 10:10
добрый дух не может жить
вместе со злым духом, а удаляется от такого человека и ищет себе пристанища в
кротости и тишине.
\vs 2Er 10:11
Когда он отступит от
человека, в котором обитал, человек, исполненный духами злыми, делается чужд
Святого Духа и закрыт для благой мысли. Так бывает со всеми гневливыми.
\vs 2Er 10:12
Итак, ты удаляйся гнева,
но облекись в великодушие и противься всякому огорчению и будешь в чистоте и
святости, любезной Богу.
\vs 2Er 10:13
Смотри поэтому, чтобы
как-нибудь не пренебречь тебе этой заповедью, ибо если соблюдешь эту заповедь,
то можешь исполнить и прочие мои заповеди, которые хочу тебе преподать.
\vs 2Er 10:14
Итак, теперь утверждайся
в этих заповедях, чтобы тебе жить с Богом, равно и все, кто соблюдет их,
будут жить с Богом.

\chhdr{Заповедь 6-я.}
\vs 2Er 11:1
Я повелел тебе, сказал пастырь, в первой заповеди, чтобы хранил ты веру,
страх и воздержание.
\vs 2Er 11:2
Да, господин, подтвердил
я.
\vs 2Er 11:3
А теперь я хочу объяснить
тебе силу этих добродетелей, чтобы знал ты, как каждая из них действует и
какую имеет власть.
\vs 2Er 11:4
Двоякого рода их действия
и состоят в праведном и неправедном.
\vs 2Er 11:5
Ты веруй праведному,
неправедному нисколько не веруй.
\vs 2Er 11:6
Ибо правда имеет путь
прямой, а неправда кривой.
\vs 2Er 11:7
Но ты иди путем прямым, а
кривой оставь.
\vs 2Er 11:8
Кривой путь неровен, имеет
множество преткновений, скалист и тернист и ведет к погибели идущих по нему.
\vs 2Er 11:9
А те, которые следуют
прямому пути, идут ровно и без препятствий, потому что он не скалист и не
тернист. Итак, видишь, что лучше идти этим путем.
\vs 2Er 11:10
Я сказал: мне нравится
идти этим путем.
\vs 2Er 11:11
И пойдешь ты, равно как
пойдут по нему и все, которые от всего сердца обратятся к Господу.

\vs 2Er 12:1
Послушай теперь,
продолжал он, о вере. Два ангела с человеком: один добрый, а другой злой.
\vs 2Er 12:2
Я спросил его: каким
образом, господин, я могу распознать их, если оба ангела живут со мною?
\vs 2Er 12:3
Слушай и разумей. Добрый
ангел тих и скромен, кроток и мирен.
\vs 2Er 12:4
Поэтому войдя в твое
сердце, постоянно будет внушать он тебе справедливость, целомудрие, чистоту
ласковость, снисходительность, любовь и благочестие.
\vs 2Er 12:5
Когда все это вселится в
твое сердце, знай, что добрый ангел с тобою: верь этому ангелу и следуй делам
его.
\vs 2Er 12:6
Послушай и о действиях
ангела злого. Прежде всего он злобен, гневлив и безрассуден, и действия его
злы и развращают рабов Божьих.
\vs 2Er 12:7
Поэтому когда войдет он в
твое сердце, из действий его разумей, что это ангел злой.
\vs 2Er 12:8
Каким образом, спросил
я, узнаю его, господин?
\vs 2Er 12:9
Слушай. Когда овладеют
тобой гнев или досада, знай, что он в тебе;
\vs 2Er 12:10
также когда возникнет в
сердце твоем пожелание разных и роскошных яств, и напитков, и чужих жен, то
вселяются в него гордость, хвастовство, надменность и тому подобное тогда
знай, что с тобою злой ангел.
\vs 2Er 12:11
Поэтому ты, зная его
дела, избегай и не верь ему: дела его злы и не свойственны рабам Божьим.
\vs 2Er 12:12
Таковы действия того и
другого ангела. Разумей их, верь ангелу доброму и удаляйся от ангела злого,
потому что внушение его во всяком деле злое.
\vs 2Er 12:13
Даже если в сердце
человека верующего войдет помысел злого ангела, то он непременно согрешит.
\vs 2Er 12:14
Если же злые люди откроют
сердце свое делам ангела доброго, то обязательно он сделает что-нибудь доброе.
\vs 2Er 12:15
Итак, видишь, что хорошо
следовать ангелу доброму. Если станешь повиноваться ему и творить его дела, то
будешь жить с Богом;
\vs 2Er 12:16
равно как и все, которые
будут следовать его делам, будут жить с Богом.

\chhdr{Заповедь 7-я.}
\vs 2Er 13:1
Бойся, говорил пастырь, Господа и соблюдай заповеди его,
\vs 2Er 13:2
ибо, соблюдая заповеди
Божьи, будешь тверд в любом деле и преуспеешь в нем.
\vs 2Er 13:3
Боясь Господа, будешь все
делать хорошо. Вот страх, которым должно страшиться, чтобы спастись.
\vs 2Er 13:4
Дьявола же не бойся: боясь
Господа, ты будешь господствовать над дьяволом, потому что в нем нет никакой
силы.
\vs 2Er 13:5
А в ком нет силы, того не
должно бояться.
\vs 2Er 13:6
В ком есть превосходная
сила, того и должно бояться.
\vs 2Er 13:7
Ибо всякий, имеющий силу,
внушает страх; а кто не имеет силы, всеми презирается.
\vs 2Er 13:8
Бойся, впрочем, дел
дьявола, потому что они злы; боясь Господа, ты не совершишь дел дьявола, но
удержишься от них.
\vs 2Er 13:9
Двоякий есть страх. Если
ты захотел сделать злое, то бойся Бога и не сделаешь этого.
\vs 2Er 13:10
Равно если бы захотел ты
сделать доброе, то опять бойся Бога и сделаешь его.
\vs 2Er 13:11
Подлинно, страх Божий
велик, силен и славен.
\vs 2Er 13:12
Итак, бойся Бога, и
будешь жить. И все те, которые будут бояться Его, соблюдая Его заповеди, будут
жить с Богом;
\vs 2Er 13:13
а которые не соблюдают
Его заповедей, в тех нет жизни.

\chhdr{Заповедь 8-я.}
\vs 2Er 14:1
Я сказал тебе, продолжал поучения пастырь, что творения Божьи двояки, двояко
и воздержание.
\vs 2Er 14:2
Поэтому от некоторых
следует воздерживаться, а от иных не следует.
\vs 2Er 14:3
Открой мне, господин,
попросил я, от чего следует воздерживаться и от чего не следует.
\vs 2Er 14:4
Воздерживайся, отвечал
он, от зла и не делай его, а от доброго не воздерживайся, но делай его.
\vs 2Er 14:5
Ибо если будешь
удерживаться от доброго и не будешь его делать, согрешишь.
\vs 2Er 14:6
Итак, удерживайся от
всякого зла и делай всякое добро.
\vs 2Er 14:7
От какого зла, спросил
я, должно удерживаться?
\vs 2Er 14:8
От прелюбодеяния, пьянства
и чрезмерных пиршеств, от излишеств в яствах, от роскоши и тщеславия,
\vs 2Er 14:9
от гордости, от лжи и
клеветы, от лицемерия, злопамятства и всякого оскорбления чести другого.
\vs 2Er 14:10
Таковы дела злые, от
которых должно воздерживаться рабу Божьему. Кто не воздерживается от них, тот
не может жить с Богом.
\vs 2Er 14:11
Послушай теперь и о
делах, следующих за ними.
\vs 2Er 14:12
Разве и еще есть,
господин, дела злые?
\vs 2Er 14:13
И подлинно есть еще много
такого, от чего должен воздерживаться раб Божий. Это воровство,
лжесвидетельство, пожелание чужого, надменность и тому подобное.
\vs 2Er 14:14
Не почитаешь ли всего
этого злым? Подлинно, это есть зло рабов Божьих и от всего этого должен
воздерживаться раб Божий, чтобы жить с Богом и быть вместе с теми, которые
воздерживаются от злых дел.
\vs 2Er 14:15
А теперь слушай о тех
добрых делах, которые положено творить, чтобы спастись.
\vs 2Er 14:16
Прежде всего это вера,
страх Божий, любовь, согласие, справедливость, истина, терпение лучше их
ничего нет в жизни человеческой:
\vs 2Er 14:17
кто соблюдает их и во все
дни не станет избегать, тот блажен в своей жизни.
\vs 2Er 14:18
Затем следуют добрые
дела, состоящие в том, чтобы служить вдовам, печься о сиротах и бедных,
избавлять от нужды рабов Божьих,
\vs 2Er 14:19
быть гостеприимным, не
прекословить, быть уравновешенным, считать себя ниже всех людей,
\vs 2Er 14:20
почитать старших
возрастом, соблюдать правду, хранить братство, переносить обиды, быть
великодушным,
\vs 2Er 14:21
не отвергать отпадших от
веры, но обращать и успокаивать их, вразумлять согрешающих, не притеснять
должников и тому подобное.
\vs 2Er 14:22
Не почитаешь ли это
добром?
\vs 2Er 14:23
Нет ничего лучше и
достойнее этого! воскликнул я.
\vs 2Er 14:24
Вот и твори эти дела и не
воздерживайся жить с Богом, равно как и все, которые соблюдут эту заповедь,
будут жить с Богом.

\chhdr{Заповедь 9-я.}
\vs 2Er 15:1
Далее говорил мне пастырь:
отринь от себя сомнения и нисколько не колеблись просить
чего-либо у Господа,
\vs 2Er 15:2
говоря себе: каким образом
могу я просить у Господа и получить, столько согрешив пред Ним?
\vs 2Er 15:3
Не помышляй этого, но от
всего сердца обращайся к Господу и проси без сомнения и познаешь великую
благость Его,
\vs 2Er 15:4
потому что Он не презрит
тебя, но исполнит прошение души твоей.
\vs 2Er 15:5
Ибо Бог не как люди,
которые помнят обиды, Он не помнит зла и милосерден к своему созданию.
\vs 2Er 15:6
Итак, очисти сердце свое
от всех сует настоящего века и прежде всего выполняй данные тебе от Бога
наказы
\vs 2Er 15:7
и получишь все блага,
которых просишь, и все прошения твои не будут оставлены, если будешь просить у
Господа без сомнения.
\vs 2Er 15:8
Те же, которые
сомневаются, совсем ничего не получают из того, о чем просят.
\vs 2Er 15:9
Исполненные веры всего
просят с упованием и получают от Господа, ибо просят без сомнения.
\vs 2Er 15:10
Всякий колеблющийся
человек с трудом спасется, если только не покается.
\vs 2Er 15:11
Поэтому очисти сердце
свое от сомнения, облекись в веру и, веруя Господу, получишь все, о чем
просишь.
\vs 2Er 15:12
Но если иногда, прося о
чем-либо Господа, долго не получаешь, не колеблись оттого, что сразу не
выполняются прошения души твоей.
\vs 2Er 15:13
Ибо, может быть, для
испытания или за грех твой, которого не знаешь, позднее получишь то, что
просишь.
\vs 2Er 15:14
Но ты не переставай
высказывать желание души своей и будешь вознагражден.
\vs 2Er 15:15
Если же придешь в уныние
и перестанешь просить, то жалуйся на себя, а не на Бога, что Он не дает тебе.
\vs 2Er 15:16
Итак, видишь, как
гибельно и ужасно сомнение, и многих даже твердых в вере совсем отторгает от
веры.
\vs 2Er 15:17
Ибо сомнение это дочь
дьявола и сильно злоумышляет на рабов Божьих.
\vs 2Er 15:18
Итак, отвергни сомнение и
одолей его во всяком деле, вооружившись сильной и могущественной верой.
\vs 2Er 15:19
Ибо вера все обещает и
все совершает, сомнение же ни в чем не доверяет себе и оттого не имеет успеха
в делах своих.
\vs 2Er 15:20
Итак, видишь, что вера
исходит свыше от Бога и имеет великую силу.
\vs 2Er 15:21
Сомнение же есть земной
дух, от дьявола, и силы не имеет.
\vs 2Er 15:22
Поэтому служи вере,
имеющей силу, и удаляйся от сомнения, которое бессильно,
\vs 2Er 15:23
и будете жить с Богом
ты и все люди, поступающие так же.

\chhdr{Заповедь 10-я.}
\vs 2Er 16:1
Удаляй от себя всякую печаль, потому что она сестра сомнения и гнева.
\vs 2Er 16:2
Каким образом, господин,
удивился я, она сестра их? Мне кажется, печаль это одно, другое гнев, и
сомнение само по себе.
\vs 2Er 16:3
И он ответил: неразумен
ты. Неужели не понимаешь, что печаль самый злой из всех духов и самый
вредный для рабов Божьих?
\vs 2Er 16:4
Она губит человека как
ничто другое и изгоняет из него Святого Духа и опять спасает.
\vs 2Er 16:5
Господин, не могу я
постичь смысла этих притчей и не понимаю, каким образом печаль может погубить
и опять спасти.
\vs 2Er 16:6
Слушай, сказал он, и
разумей. Кто никогда не изыскивал истины и не исследовал Божество, но только
уверовал и потом предался разным языческим занятиям и другим делам сего мира,
\vs 2Er 16:7
тот не понимает притчей
божественных, потому что помрачается от таких дел, повреждается и загрубевает
разумом.
\vs 2Er 16:8
Как хорошие виноградные
лозы, оставленные без ухода, подавляются и заглушаются разными сорняками и
терниями,
\vs 2Er 16:9
так и люди, которые только
уверовали и вдались в дела этого мира, лишаются своего смысла и, думая о
богатствах, совершенно ничего не понимают, и разум их, занятый мирской суетой,
глух к Господу.
\vs 2Er 16:10
Но те, которые живут в
страхе Божьем, тщательно исследуют истину и божественное и сердцем обращены к
Господу, они легко принимают и разумеют все, что говорится им.
\vs 2Er 16:11
Ибо, где обитает Господь,
там много разума.
\vs 2Er 16:12
Поэтому прилепись к
Господу и все поймешь и уразумеешь.

\vs 2Er 17:1
Послушай теперь,
неразумный, каким образом печаль изгоняет Духа Святого и как опять спасает.
\vs 2Er 17:2
Когда сомневающийся не
обретает успеха в каком-либо деле из-за своего сомнения, то печаль входит в
сердце такого человека, омрачает Духа Святого и изгоняет его.
\vs 2Er 17:3
И когда охватывают
человека гнев и сильное раздражение по какому-нибудь поводу, то опять печаль
входит в сердце, он скорбит о своем поступке, раскаивается, что разгневался.
\vs 2Er 17:4
Эта печаль кажется
спасительною, потому что влечет раскаянье.
\vs 2Er 17:5
Но и в том и в другом
случае печаль оскорбляет Святого Духа.
\vs 2Er 17:6
Печаль, вызванная
сомнением или тем, что не удалось человеку его дело, печаль неправедная.
\vs 2Er 17:7
Печаль же от досады на
дурной поступок не плохая печаль, но и она оскорбляет Святого Духа.
\vs 2Er 17:8
Посему удаляй от себя
печаль и не оскорбляй Святого Духа, в тебе живущего, чтобы он не возроптал на
тебя к Господу и не удалился от тебя.
\vs 2Er 17:9
Ибо Дух Божий, обитающий в
этом теле, не терпит печали.
\vs 2Er 17:10
Итак, облекись ты в
радость, которая всегда имеет благодать пред Господом и угодна Ему и утешайся
ею.
\vs 2Er 17:11
Всякий радующийся человек
совершает добро и помышляет о добре, презирая печаль.
\vs 2Er 17:12
А человек печальный
всегда зол, во-первых, потому, что оскорбляет Святого Духа, который дан
человеку радостным;
\vs 2Er 17:13
и, во-вторых, потому, что
он творит беззаконие, не обращаясь к Господу и не исповедуясь перед Ним.
\vs 2Er 17:14
Молитва печального
человека никогда не достигает престола Божьего.
\vs 2Er 17:15
И я спросил его: почему
же, господин, молитва печального человека не восходит к престолу Господню?
\vs 2Er 17:16
Потому, ответил он,
что печаль пребывает в его сердце.
\vs 2Er 17:17
Печаль, смешанная с
молитвою, не допускает молитву чистою взойти к престолу Божьему.
\vs 2Er 17:18
Как вино с добавлением
уксуса уже не имеет прежней приятности, так и печаль, примешанная к Святому
Духу, не имеет той же чистой молитвы.
\vs 2Er 17:19
Посему очищайся от злой
печали и будешь жить с Богом,
\vs 2Er 17:20
и все будут жить с Богом,
если только отбросят от себя печаль и облекутся в радость.

\chhdr{Заповедь 11-я.}
\vs 2Er 18:1
Пастырь показал мне людей, сидящих на скамьях, и одного стоящего на кафедре,
\vs 2Er 18:2
и сказал: посмотри на них.
Те, которые сидят на скамьях, верующие, а стоящий на кафедре лжепророк,
погубляющий смысл рабов Божьих тех, которые двоедушествуют, а не истинно
верующих.
\vs 2Er 18:3
Эти двоедушные приходят к
нему как к пророку и спрашивают его о том, что станет с ними,
\vs 2Er 18:4
и он, не имея в себе силы
Духа Божественного, отвечает им, говоря то, что хотят они услышать, и
наполняет души их лживыми обещаниями.
\vs 2Er 18:5
Будучи суетен, он суетно и
отвечает суетным людям.
\vs 2Er 18:6
Впрочем, он говорит и
кое-что справедливое, потому что дьявол вселяет в него свой дух, дабы привлечь
кого-либо из праведных.
\vs 2Er 18:7
Но сильные в вере,
облеченные в истину не присоединяются к таким духам, но удаляются от них.
\vs 2Er 18:8
Двоедушные же и часто
кающиеся обращаются за прорицаниями, как и народы, и навлекают на себя великий
грех своим идолопоклонством,
\vs 2Er 18:9
потому что спрашивающий
лжепророка является идолопоклонником, он чужд истины и неразумен.
\vs 2Er 18:10
А всякий дух, Богом
данный, не дожидается расспросов, но, имея силу Божественную, говорит все сам,
потому что он свыше, от силы Духа Божьего.
\vs 2Er 18:11
Дух, который отвечает на
вопросы согласно желаниям человеческим, есть дух земной, легкомысленный, не
имеющий силы: он совсем не говорит, если его не спрашивают.
\vs 2Er 18:12
И я сказал: как же можно
распознать, кто истинный пророк и кто лжепророк?
\vs 2Er 18:13
Выслушай, говорит, об
обоих пророках; и по тому, что я скажу тебе, отличишь пророка Божьего от
ложного пророка.
\vs 2Er 18:14
По делам узнавай
человека, который имеет Дух Божий.
\vs 2Er 18:15
Во-первых, он спокоен,
кроток и смирен, удаляется от всякого зла и суетного желания этого века,
\vs 2Er 18:16
ставит себя ниже всех
людей и никому не отвечает на вопросы, не говорит наедине;
\vs 2Er 18:16
Дух Божий говорит не
тогда, когда человек того желает, но когда угодно Богу.
\vs 2Er 18:17
Поэтому когда человек,
имеющий Дух Божий, придет в церковь праведных, имеющих веру, там совершается
молитва к Господу;
\vs 2Er 18:18
тогда ангел пророческого
духа, приставленный к нему, исполняет этого человека Духом Святым, и он
говорит к собранию, как угодно Богу. Так проявляется Дух Божественный и сила
его.
\vs 2Er 18:19
Слушай теперь и о духе
земном, суетном, неразумном и не имеющем силы.
\vs 2Er 18:20
Прежде всего человек,
кажущийся исполненным духа, возвышает себя, стремится к власти, нагл и
многословен,
\vs 2Er 18:21
живет среди роскоши и
многих удовольствий, берет мзду за свое прорицание, без вознаграждения не
пророчествует.
\vs 2Er 18:22
Может ли Дух Божий брать
мзду и пророчествовать?
\vs 2Er 18:23
Это не свойственно
пророку Божьему, и в поступающих таким образом обитает дух земной.
\vs 2Er 18:24
Далее, он не входит в
собрание мужей праведных, но избегает их
\vs 2Er 18:25
и, наоборот, общается с
людьми двоедушными и пустыми, пророчествует в местах потаенных и обманывает
речами, которые хотят услышать, и говорит суетное людям суетным:
\vs 2Er 18:26
так пустая посуда, когда
складывается с другими пустыми же, не разбивается, но они хорошо приходятся
одна к другой.
\vs 2Er 18:27
А когда он оказывается
среди людей праведных, исполненных Духа Божественного, возносящих молитву,
тогда и обнаруживается его пустота:
\vs 2Er 18:28
земной дух от страха
покидает его, и он, совершенно поверженный, ничего не может говорить.
\vs 2Er 18:29
Если в кладовую поместить
вино или масло и туда же поставить пустой сосуд, а после брать запасы из
кладовой, то сосуд, который поставил пустым, пустым и найдешь.
\vs 2Er 18:30
И пустые пророки, какими
приходят к людям, имеющим Святого Духа, такими и остаются. Вот образ пророка
истинного и ложного.
\vs 2Er 18:31
Итак, испытывай по делам
и по жизни того человека, который говорит, что он имеет Святого Духа.
\vs 2Er 18:32
Верь Духу, приходящему от
Бога и имеющему силу; духу же земному и пустому, в котором нет силы, не верь,
ибо он приходит от дьявола.
\vs 2Er 18:33
Задумайся над примером,
который приведу я тебе. Если взять камень и бросить в небо, то сможешь ли
докинуть до него?
\vs 2Er 18:34
Или же если взять трубу с
водою, направить струю в небо, то сможешь ли ты пробить небо?
\vs 2Er 18:35
Что ты, господин,
воскликнул я, все это невозможно!
\vs 2Er 18:36
Вот, сказал он, как
этого не может быть, так точно дух земной бессилен и недейственен.
\vs 2Er 18:37
Осознай теперь силу,
свыше приходящую. Град крупинка очень малая, но, попадая в голову человека,
какую причиняет боль?
\vs 2Er 18:38
Или еще пример: дождевая
капля, которая, с крыши скатываясь вниз, источает камень.
\vs 2Er 18:39
Видишь, и самое малое,
что сверху падает на землю, имеет великую силу: так силен и Дух Божественный,
приходящий свыше.
\vs 2Er 18:40
Этому Духу ты верь, а от
другого удаляйся.

\chhdr{Заповедь 12-я.}
\vs 2Er 19:1
Пастырь сказал мне: удали от себя всякую похоть злую и облекись в желание
доброе и святое.
\vs 2Er 19:2
Ибо, облекшись в желание
доброе, ты возненавидишь зло и будешь управлять им, как захочешь.
\vs 2Er 19:3
Похоть злая люта и с
трудом усмиряется: она страшна и своею лютостью сокрушает людей.
\vs 2Er 19:4
Но сокрушает тех людей,
которые не имеют стремления доброго и погрузились в дела этого века: их-то она
предает смерти.
\vs 2Er 19:5
Какие действия, господин,
спросил я, злой похоти обрекают людей на смерть? Объясни мне, чтобы я мог
избегать их.
\vs 2Er 19:6
Послушай, посредством
каких действий злая похоть умерщвляет рабов Божьих.

\vs 2Er 20:1
Злая похоть состоит в том,
чтобы желать чужой жены, или жене желать чужого мужа, желать великого
богатства, множества роскошных яств и питий и других наслаждений:
\vs 2Er 20:2
ибо всякое наслаждение
бессмысленно и суетно для рабов Божьих.
\vs 2Er 20:3
Таковы пожелания злые,
умерщвляющие рабов Божьих.
\vs 2Er 20:4
Злая похоть есть дочь
дьявола. Поэтому должно удаляться злой похоти, чтобы жить с Богом.
\vs 2Er 20:5
А те, которые поддадутся
злой похоти и не воспротивятся ей, погибнут, потому что она смертоносна.
\vs 2Er 20:6
Итак, ты стремись к правде
и, вооружившись страхом Господним, противостой злой похоти. Ибо страх Божий
обитает в добрых пожеланиях.
\vs 2Er 20:7
И злая похоть, видя тебя
вооруженным страхом Господним и противящимся ей, убежит от тебя далеко и не
явится к тебе, боясь твоего оружия;
\vs 2Er 20:8
и одержавши победу и
увенчанный за нее, предайся стремлению к правде и, воздавши Ему за полученную
тобою победу, служи Ему по Его воле.
\vs 2Er 20:9
И если послужишь доброму
началу и покоришься Ему, то можешь владычествовать над злою похотью и
управлять ею, как тебе угодною.
\vs 2Er 20:10
Желал бы я услышать,
господин, сказал я, как должно служить доброму желанию?
\vs 2Er 20:11
Слушай. Имей страх Божий
и веру в Бога, люби истину, твори правду и подобные добрые дела.
\vs 2Er 20:12
Делая это, ты будешь
угодным рабом Божьим и будешь жить с Богом; и все, которые будут служить
стремлению доброму, будут жить с Богом.

\vs 2Er 21:1
И так окончил он
двенадцать заповедей и сказал мне:
\vs 2Er 21:2
вот тебе заповеди,
поступай по ним и к тому же убеждай людей слушать тебя, чтобы покаяние их было
чисто в остальные дни их жизни.
\vs 2Er 21:3
И это служение, которое
поручаю тебе, исполняй тщательно и получишь великий плод,
\vs 2Er 21:4
ибо найдешь любовь у всех,
которые покаются и послушаются слов твоих.
\vs 2Er 21:5
Я буду с тобою и буду
побуждать их слушаться тебя.
\vs 2Er 21:6
И я сказал ему: господин,
эти заповеди величественны, прекрасны и могут возвеселить сердце человека,
который исполнит их.
\vs 2Er 21:7
Но не знаю, господин,
способен ли человек соблюдать эти заповеди, потому что они очень трудны.
\vs 2Er 21:8
Он отвечал мне: эти
заповеди легко соблюсти, и не покажутся они трудными, если будешь убежден, что
их можно соблюсти;
\vs 2Er 21:9
но если закралось в сердце
твое сомнение, что не по силам человеку, то не соблюдешь их.
\vs 2Er 21:10
Теперь же говорю тебе:
если не соблюдешь этих заповедей и пренебрежешь ими, то не спасешься ты и дети
твои, и весь дом твой,
\vs 2Er 21:11
потому что ты сам себе
присудил, что этих заповедей нельзя соблюсти человеку.

\vs 2Er 22:1
Произносил он это с
большим гневом, и я очень смутился и испугался.
\vs 2Er 22:2
Лицо его изменилось так,
что вид его стал невыносим для человека.
\vs 2Er 22:3
Но, видя, что я весь в
смущении и страхе, начал он говорить умереннее и ласковее:
\vs 2Er 22:4
неразумный и непостоянный,
не видишь ли славу Божью, не понимаешь, как велик и дивен Тот, который
сотворил мир для человека,
\vs 2Er 22:5
и все творение покорил
человеку, и дал ему всю власть господствовать над всем поднебесным?
\vs 2Er 22:6
Если человек есть владыка
тварей Божьих и над всем господствует, то ужели он не может господствовать и
над этими заповедями?
\vs 2Er 22:7
Это по силам человеку,
имеющему Господа в сердце своем.
\vs 2Er 22:8
Кто же имеет Господа
только в устах своих, огрубел сердцем и далек от Господа, для того эти
заповеди тяжки и неисполнимы.
\vs 2Er 22:9
Итак вы, слабые и
нетвердые в вере, положите себе Господа вашего в сердце и узнаете, что ничего
нет легче этих заповедей, ничего приятнее и доступнее их.
\vs 2Er 22:10
Обратитесь к Господу,
оставьте дьяволу его удовольствия, которые злы и горьки, и не бойтесь дьявола,
потому что над вами он не имеет силы.
\vs 2Er 22:11
Ибо я с вами, ангел
покаяния, и я господствую над ним.
\vs 2Er 22:12
Дьявол наводит страх, но
страх его не имеет силы.
\vs 2Er 22:13
Посему не бойтесь его, и
он покинет вас.

\vs 2Er 23:1
И я попросил его:
господин, выслушай несколько слов моих.
\vs 2Er 23:2
Говори, разрешил он.
\vs 2Er 23:3
Всякий человек желает
исполнять Божьи заповеди, и нет такого, который бы не просил у Бога силы
соблюдать Его заповеди;
\vs 2Er 23:4
но дьявол упорен и своею
силою противодействует рабам Божьим.
\vs 2Er 23:5
Не может дьявол,
возразил он, пересилить рабов Божьих, которые веруют в Господа от всего
сердца.
\vs 2Er 23:6
Дьявол может
противоборствовать, но победить не может.
\vs 2Er 23:7
Если воспротивитесь ему,
то, побежденный, он с позором покинет вас.
\vs 2Er 23:8
Боятся дьявола, как будто
имеющего власть, те, которые не тверды в вере.
\vs 2Er 23:9
Дьявол искушает рабов
Божьих и, если найдет слабых, губит их.
\vs 2Er 23:10
Когда человек наполняет
сосуды хорошим вином и между ними ставит несколько сосудов неполных,
\vs 2Er 23:11
то, приходя попробовать
вино, не думает о полных, ибо знает, что они хороши, а отведывает из неполных,
не скисло ли в них вино,
\vs 2Er 23:12
потому что в неполных
сосудах вино скоро скисает и теряет вкус.
\vs 2Er 23:13
Так и дьявол приходит к
рабам Божьим, чтобы искусить их.
\vs 2Er 23:14
И все те, которые полны
веры, мужественно противятся ему; и он удаляется от них, потому что негде
войти ему.
\vs 2Er 23:15
Тогда он подступает к
тем, которые не полны веры, и, имея возможность, вселяется в них, делает с
ними что хочет, и они становятся его рабами.

\vs 2Er 24:1
Но, говорю вам я, ангел
покаяния: не бойтесь дьявола, ибо я послан для того, чтобы быть с вами,
кающимися от всего сердца, и утвердить вас в вере.
\vs 2Er 24:2
Посему верьте вы, которые
по грехам своим отчаялись в спасении, и, прилагая грехи к грехам, отягощаете
жизнь свою:
\vs 2Er 24:3
если обратитесь к Господу
от всего сердца вашего и будете творить правду в остальные дни своей жизни и
служить Ему по воле Его,
\vs 2Er 24:4
то Он простит прежние
грехи ваши, и обретете власть над делами дьявола.
\vs 2Er 24:5
Угроз же дьявола вовсе не
бойтесь, потому что они бессильны, как нервы человека мертвого.
\vs 2Er 24:6
Итак, слушайте меня и
бойтесь Господа, Который может спасти и погубить: соблюдайте заповеди Его и
будете жить с Богом.
\vs 2Er 24:7
И я сказал ему: господин,
теперь я проникся всеми заповедями Господа, потому что ты со мною;
\vs 2Er 24:8
знаю, что сокрушишь всю
силу дьявола, и мы восторжествуем над ним;
\vs 2Er 24:9
и надеюсь, что могу
соблюсти при помощи Божьей заповеди, которые ты передал.
\vs 2Er 24:10
Соблюдешь, сказал он,
если сердце твое будет чисто пред Господом, и все соблюдут, которые очистят
сердца свои от суетных похотей этого века, и будут жить с Богом.

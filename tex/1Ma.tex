\bibbookdescr{1Ma}{
  inline={\LARGE Первая книга\\\Huge Маккавейская\fns{Книги Маккавейские переведены с греческого, потому что в еврейском тексте их нет.}},
  toc={1-я Маккавейская*},
  bookmark={1-я Маккавейская},
  header={1-я Маккавейская},
  %headerleft={},
  %headerright={},
  abbr={1~Мак}
}
\vs 1Ma 1:1 После того как Александр, сын Филиппа, Македонянин, который вышел из земли Киттим, поразил Дария, царя Персидского и Мидийского, и воцарился вместо него прежде над Елладою,~---
\vs 1Ma 1:2 он произвел много войн и овладел многими укрепленными местами, и убивал царей земли.
\vs 1Ma 1:3 И прошел до пределов земли и взял добычу от множества народов; и умолкла земля пред ним, и он возвысился, и вознеслось сердце его.
\vs 1Ma 1:4 Он собрал весьма сильное войско и господствовал над областями и народами и властителями, и они сделались его данниками.
\vs 1Ma 1:5 После того он слег в постель и, почувствовав, что умирает,
\vs 1Ma 1:6 призвал знатных из слуг своих, которые были воспитаны с ним от юности, и разделил им свое царство еще при жизни своей.
\rsbpar\vs 1Ma 1:7 Александр царствовал двенадцать лет и умер.
\vs 1Ma 1:8 И владычествовали слуги его каждый в своем месте.
\vs 1Ma 1:9 И по смерти его все они возложили на себя венцы, а после них и сыновья их в течение многих лет; и умножили зло на земле.
\vs 1Ma 1:10 И вышел от них корень греха~--- Антиох Епифан, сын царя Антиоха, который был заложником в Риме, и воцарился в сто тридцать седьмом году царства Еллинского.
\rsbpar\vs 1Ma 1:11 В те дни вышли из Израиля сыны беззаконные и убеждали многих, говоря: пойдем и заключим союз с народами, окружающими нас, ибо с тех пор, как мы отделились от них, постигли нас многие бедствия.
\vs 1Ma 1:12 И добрым показалось это слово в глазах их.
\vs 1Ma 1:13 Некоторые из народа изъявили желание и отправились к царю; и он дал им право исполнять установления языческие.
\vs 1Ma 1:14 Они построили в Иерусалиме училище по обычаю языческому
\vs 1Ma 1:15 и установили у себя необрезание, и отступили от святаго завета, и соединились с язычниками, и продались, чтобы делать зло.
\rsbpar\vs 1Ma 1:16 Когда Антиох увидел, что царство укрепилось, предпринял воцариться над Египтом, чтобы царствовать над двумя царствами,
\vs 1Ma 1:17 и вошел он в Египет с сильным ополчением, с колесницами, и слонами, и всадниками, и множеством кораблей;
\vs 1Ma 1:18 и вступил в сражение с Птоломеем, царем Египетским; и убоялся Птоломей от лица его и обратился в бегство, и много пало раненых.
\vs 1Ma 1:19 И овладели они укрепленными городами в земле Египетской, и взял он добычу из земли Египетской.
\rsbpar\vs 1Ma 1:20 После поражения Египта Антиох возвратился в сто сорок третьем году и пошел против Израиля, и вступил в Иерусалим с сильным ополчением;
\vs 1Ma 1:21 вошел во святилище с надменностью и взял золотой жертвенник, светильник и все сосуды его,
\vs 1Ma 1:22 и трапезу предложения, и возлияльники, и чаши, и кадильницы золотые, и завесу, и венцы, и золотое украшение, бывшее снаружи храма, и всё обобрал.
\vs 1Ma 1:23 Взял и серебро, и золото, и драгоценные сосуды, и взял скрытые сокровища, какие отыскал.
\vs 1Ma 1:24 И, взяв всё, отправился в землю свою и совершил убийства, и говорил с великою надменностью.
\vs 1Ma 1:25 Посему был великий плач в Израиле, во всех местах его.
\vs 1Ma 1:26 Стенали начальники и старейшины, изнемогали девы и юноши, и изменилась красота женская.
\vs 1Ma 1:27 Всякий жених предавался плачу, и сидящая в брачном чертоге была в скорби.
\vs 1Ma 1:28 Вострепетала земля за обитающих на ней, и весь дом Иакова облекся стыдом.
\rsbpar\vs 1Ma 1:29 По прошествии двух лет послал царь начальника податей в города Иуды, и он пришел в Иерусалим с большою толпою;
\vs 1Ma 1:30 коварно говорил им слова мира, и они поверили ему; но он внезапно напал на город и поразил его великим поражением, и погубил множество народа Израильского;
\vs 1Ma 1:31 взял добычи из города и сожег его огнем, и разрушил домы его и стены его кругом;
\vs 1Ma 1:32 и увели в плен жен и детей, и овладели скотом.
\vs 1Ma 1:33 Оградили город Давидов большою и крепкою стеною и крепкими башнями, и сделался он для них крепостью.
\vs 1Ma 1:34 И поместили там народ нечестивый, людей беззаконных, и они укрепились в ней;
\vs 1Ma 1:35 запаслись оружием и продовольствием и, собрав добычи Иерусалимские, сложили там, и сделались большою сетью.
\vs 1Ma 1:36 И было это постоянною засадою для святилища и злым диаволом для Израиля.
\vs 1Ma 1:37 Они проливали невинную кровь вокруг святилища и оскверняли святилище.
\vs 1Ma 1:38 Жители же Иерусалима разбежались ради них, и он сделался жилищем чужих и стал чужим для своего рода, и дети его оставили его.
\vs 1Ma 1:39 Святилище его запустело, как пустыня, праздники его обратились в плач, субботы его~--- в поношение, честь его~--- в уничижение.
\vs 1Ma 1:40 По мере славы его увеличилось бесчестие его, и высота его обратилась в печаль.
\rsbpar\vs 1Ma 1:41 Царь Антиох написал всему царству своему, чтобы все были одним народом
\vs 1Ma 1:42 и чтобы каждый оставил свой закон. И согласились все народы по слову царя.
\vs 1Ma 1:43 И многие из Израиля приняли идолослужение его и принесли жертвы идолам, и осквернили субботу.
\vs 1Ma 1:44 Царь послал через вестников грамоты в Иерусалим и в города Иудейские, чтобы они следовали узаконениям, чужим для сей земли,
\vs 1Ma 1:45 и чтобы не допускались всесожжения и жертвоприношения, и возлияние в святилище, чтобы ругались над субботами и праздниками
\vs 1Ma 1:46 и оскверняли святилище и святых,
\vs 1Ma 1:47 чтобы строили жертвенники, храмы и капища идольские, и приносили в жертву свиные мяса и скотов нечистых,
\vs 1Ma 1:48 и оставляли сыновей своих необрезанными, и оскверняли души их всякою нечистотою и мерзостью,
\vs 1Ma 1:49 для того, чтобы забыли закон и изменили все постановления.
\vs 1Ma 1:50 А если кто не сделает по слову царя, да будет предан смерти.
\vs 1Ma 1:51 Согласно этому писал он всему царству своему и поставил надзирателей над всем народом, и повелел городам Иудейским приносить жертвы во всяком городе.
\vs 1Ma 1:52 И собрались к ним многие из народа, все, которые оставили закон,~--- и совершили зло в земле;
\vs 1Ma 1:53 и заставили Израиля укрываться во всяком убежище его.
\rsbpar\vs 1Ma 1:54 В пятнадцатый день Хаслева, сто сорок пятого года, устроили на жертвеннике мерзость запустения, и в городах Иудейских вокруг построили жертвенники,
\vs 1Ma 1:55 и перед дверями домов и на улицах совершали курения,
\vs 1Ma 1:56 и книги закона, какие находили, разрывали и сожигали огнем;
\vs 1Ma 1:57 у кого находили книгу завета и кто держался закона, того, по повелению царя, предавали смерти.
\vs 1Ma 1:58 С таким насилием поступали они с Израильтянами, приходившими каждый месяц в города.
\vs 1Ma 1:59 И в двадцать пятый день месяца, принося жертвы на жертвеннике, который был над алтарем,
\vs 1Ma 1:60 они, по данному повелению, убивали жен, обрезавших детей своих,
\vs 1Ma 1:61 а младенцев вешали за шеи их, домы их расхищали и совершавших над ними обрезание убивали.
\vs 1Ma 1:62 Но многие в Израиле остались твердыми и укрепились, чтобы не есть нечистого,
\vs 1Ma 1:63 и предпочли умереть, чтобы не оскверниться пищею и не поругать святаго завета,~--- и умирали.
\vs 1Ma 1:64 И был весьма великий гнев над Израилем.
\vs 1Ma 2:1 В те дни восстал Маттафия, сын Иоанна, сына Симеонова, священник из сынов Иоарива из Иерусалима; жил он в Модине.
\vs 1Ma 2:2 У него было пять сыновей: Иоанн, прозываемый Гаддис,
\vs 1Ma 2:3 Симон, называемый Фасси,
\vs 1Ma 2:4 Иуда, прозываемый Маккавей,
\vs 1Ma 2:5 Елеазар, прозываемый Аваран, Ионафан, прозываемый Апфус.
\vs 1Ma 2:6 Видя богохульства, происходившие в Иудее и Иерусалиме,
\vs 1Ma 2:7 он сказал: горе мне! для чего родился я видеть разорение народа моего и разорение святаго города и оставаться здесь, когда он предан в руки врагов и святилище~--- в руки чужих?
\vs 1Ma 2:8 Храм его сделался, как муж бесславный,
\vs 1Ma 2:9 драгоценные сосуды его унесены в плен, младенцы его избиты на улицах, юноши его пали от меча врага.
\vs 1Ma 2:10 Какой народ не занимал царства его и не овладевал добычами его?
\vs 1Ma 2:11 Все украшение его отнято; из свободного он сделался рабом.
\vs 1Ma 2:12 И вот святыни наши, и благолепие наше, и слава наша опустели, и язычники осквернили их.
\vs 1Ma 2:13 Для чего нам еще жить?
\vs 1Ma 2:14 И разодрал Маттафия и сыновья его одежды свои, и облеклись во вретища, и горько плакали.
\vs 1Ma 2:15 И пришли от царя в город Модин принуждавшие к отступничеству, чтобы приносить жертвы.
\vs 1Ma 2:16 И многие из Израиля пристали к ним; а Маттафия и сыновья его устояли.
\vs 1Ma 2:17 И отвечали пришедшие от царя и сказали Маттафии: ты вождь, ты славен и велик в этом городе и имеешь опору в сыновьях и братьях.
\vs 1Ma 2:18 Итак, приступи теперь первый и исполни повеление царя, как сделали это все народы и мужи Иудейские и оставшиеся в Иерусалиме, и будешь ты и дом твой в числе друзей царских, и ты и сыновья твои будете почтены и серебром, и золотом, и многими дарами.
\vs 1Ma 2:19 И отвечал Маттафия и сказал громким голосом: если и все народы в области царства царя послушают его и отступят каждый от богослужения отцов своих, и согласятся на повеления его,
\vs 1Ma 2:20 то я и сыновья мои и братья мои будем поступать по завету отцов наших.
\vs 1Ma 2:21 Помилуй нас Бог, чтобы оставить закон и постановления!
\vs 1Ma 2:22 Не послушаем мы слов царя, чтобы отступить нам от нашего богослужения вправо или влево.
\vs 1Ma 2:23 Когда перестал он говорить эти слова, подошел муж Иудеянин пред глазами всех, чтобы принести по повелению царя идольскую жертву на жертвеннике, который был в Модине.
\vs 1Ma 2:24 Увидев это, Маттафия возревновал, и затрепетала внутренность его, и воспламенилась ярость его по законе, и он, подбежав, убил его при жертвеннике.
\vs 1Ma 2:25 И в то же время убил мужа царского, принуждавшего приносить жертву, и разрушил жертвенник.
\vs 1Ma 2:26 И возревновал он по законе, как это сделал Финеес с Замврием, сыном Салома.
\vs 1Ma 2:27 И воскликнул Маттафия в городе громким голосом: всякий, кто ревнует по законе и стоит в завете, да идет вслед за мною!
\vs 1Ma 2:28 И убежал сам и сыновья его в горы, оставив всё, что имели в городе.
\vs 1Ma 2:29 Тогда многие, преданные правде и закону, ушли в пустыню и оставались там,
\vs 1Ma 2:30 сами и сыновья их, и жены их, и скоты их, потому что умножились беды над ними.
\vs 1Ma 2:31 И возвещено было мужам царским и войску, находившемуся в Иерусалиме, городе Давидовом, что некоторые мужи, нарушив царское повеление, ушли в сокровенные места в пустыне.
\vs 1Ma 2:32 И погнались за ними многие и, настигнув их, ополчились, и выстроились к сражению против них в день субботний,
\vs 1Ma 2:33 и сказали им: теперь еще можно; выходите и сделайте по слову царя, и останетесь живы.
\vs 1Ma 2:34 Но они отвечали: не выйдем и не сделаем по слову царя, не оскверним дня субботнего.
\vs 1Ma 2:35 Тогда поспешили начать сражение против них.
\vs 1Ma 2:36 Но они не отвечали им, ни даже камня не бросили на них, ни заградили тайных убежищ своих,
\vs 1Ma 2:37 и сказали: мы все умрем в невинности нашей; небо и земля свидетели за нас, что вы несправедливо губите нас.
\vs 1Ma 2:38 Нападали на них по субботам, и умерло их, и жен их, и детей их со скотом их, до тысячи душ.
\vs 1Ma 2:39 Когда узнал о том Маттафия и друзья его, горько плакали о них;
\vs 1Ma 2:40 и говорили друг другу: если все мы будем поступать так, как поступали эти братья наши, и не будем сражаться с язычниками за жизнь нашу и постановления наши, то они скоро истребят нас с земли.
\vs 1Ma 2:41 И решили они в тот день и сказали: кто бы ни пошел на войну против нас в день субботний, будем сражаться против него, дабы нам не умереть всем, как умерли братья наши в тайных убежищах.
\vs 1Ma 2:42 Тогда собрались к ним множество Иудеев, крепкие силою из Израиля, все верные закону.
\vs 1Ma 2:43 И все, бежавшие от бедствия, присоединились к ним и сделались подкреплением для них.
\vs 1Ma 2:44 Так составили они войско и поражали в гневе своем нечестивых и в ярости своей мужей беззаконных; остальные же бежали для спасения к язычникам.
\vs 1Ma 2:45 И обходил вокруг Маттафия и друзья его, и разрушали жертвенники,
\vs 1Ma 2:46 и небоязненно обрезывали необрезанных детей, сколько находили в пределах Израильских,
\vs 1Ma 2:47 и преследовали сынов гордыни, и дело успешно шло в руках их.
\vs 1Ma 2:48 Так защищали они закон от руки язычников и от руки царей и не дали восторжествовать грешнику.
\rsbpar\vs 1Ma 2:49 Приблизились дни смерти Маттафии, и он сказал сыновьям своим: ныне усилилась гордость и испытание, ныне время переворота и гнев ярости.
\vs 1Ma 2:50 Итак, дети, возревнуйте о законе и отдайте жизнь вашу за завет отцов наших.
\vs 1Ma 2:51 Вспомните о делах отцов наших, которые они совершили во времена свои, и вы приобретете великую славу и вечное имя.
\vs 1Ma 2:52 Авраам не в искушении ли найден был верным? и это вменилось ему в праведность.
\vs 1Ma 2:53 Иосиф в стесненном положении своем сохранил заповедь и сделался господином Египта.
\vs 1Ma 2:54 Финеес, отец наш, за то, что возревновал ревностью, получил завет вечного священства.
\vs 1Ma 2:55 Иисус за исполнение слова сделался судьею над Израилем.
\vs 1Ma 2:56 Халев за свидетельство перед собранием получил в наследие землю.
\vs 1Ma 2:57 Давид за свое милосердие наследовал престол царства навеки.
\vs 1Ma 2:58 Илия за великую ревность по законе взят даже на небо.
\vs 1Ma 2:59 Анания, Азария, Мисаил верою спаслись от пламени.
\vs 1Ma 2:60 Даниил за свою невинность избавлен от челюстей львов.
\vs 1Ma 2:61 Итак, припоминайте от рода до рода, что все, надеющиеся на Него, не изнемогут.
\vs 1Ma 2:62 Не убойтесь слов мужа грешного, ибо слава его обратится в навоз и в червей.
\vs 1Ma 2:63 Сегодня он превозносится, а завтра не найдут его, ибо он обратился в прах свой, и замысел его погиб.
\vs 1Ma 2:64 Но вы, дети мои, крепитесь и мужественно стойте в законе, ибо чрез него вы прославитесь.
\vs 1Ma 2:65 Вот~--- Симон, брат ваш: знаю, что он~--- муж совета, слушайтесь его во все дни; он будет вам вместо отца.
\vs 1Ma 2:66 А Иуда Маккавей, крепкий силою от юности своей, да будет у вас начальником войска, и будет вести войну с народами.
\vs 1Ma 2:67 Итак, соберите к себе всех исполнителей закона и отмщайте за обиды народа вашего;
\vs 1Ma 2:68 воздайте воздаяние язычникам и будьте внимательны к повелениям закона.
\vs 1Ma 2:69 И благословил их и приложился к отцам своим.
\vs 1Ma 2:70 Умер же он на сто сорок шестом году; и сыновья его похоронили его в гробе отцов своих в Модине, и весь Израиль оплакивал его горьким плачем.
\vs 1Ma 3:1 И восстал вместо него Иуда, называемый Маккавей, сын его.
\vs 1Ma 3:2 И помогали ему все братья его и все, которые были привержены к отцу его, и вели войну Израиля с радостью.
\vs 1Ma 3:3 Он распространил славу народа своего; он облекался бронею, как исполин, опоясывался воинскими доспехами своими и вел войну, защищая ополчение мечом;
\vs 1Ma 3:4 он уподоблялся льву в делах своих и был как скимен, рыкающий на добычу;
\vs 1Ma 3:5 он преследовал беззаконных, отыскивая их, и возмущающих народ его сожигал.
\vs 1Ma 3:6 И смирились беззаконные из страха пред ним, и все делатели беззакония смутились пред ним, и благоуспешно было спасение рукою его.
\vs 1Ma 3:7 Он огорчил многих царей и возвеселил Иакова делами своими, и память его до века в благословении;
\vs 1Ma 3:8 прошел по городам Иудеи и истребил в ней нечестивых, и отвратил гнев от Израиля,
\vs 1Ma 3:9 и сделался именитым до последних пределов земли, и собрал погибавших.
\rsbpar\vs 1Ma 3:10 Тогда Аполлоний собрал язычников и из Самарии многочисленное войско, чтобы воевать против Израиля.
\vs 1Ma 3:11 Иуда узнал о том и вышел к нему навстречу, и поразил, и убил его; и много пало пораженных, а остальные убежали.
\vs 1Ma 3:12 И взял Иуда добычу их, и взял меч Аполлония, и сражался им во все дни.
\rsbpar\vs 1Ma 3:13 И услышал Сирон, военачальник Сирии, что Иуда собрал вокруг себя людей и сонм верных, выступающих с ним на войну,
\vs 1Ma 3:14 и сказал: сделаю себе имя и прославлюсь в царстве, и сражусь с Иудою и с теми, которые вместе с ним и которые презирают слово царское.
\vs 1Ma 3:15 И решился он идти, и пошло с ним сильное полчище нечестивых помогать ему и сделать отмщение на сынах Израиля.
\vs 1Ma 3:16 Когда они приблизились к возвышенности Вефорона, Иуда вышел к ним навстречу с очень немногими,
\vs 1Ma 3:17 которые, когда увидели идущее навстречу им войско, сказали Иуде: как можем мы в таком малом числе сражаться против такого сильного множества? И мы же совсем ослабели, еще не евши ныне.
\vs 1Ma 3:18 Но Иуда сказал им: легко и многим попасть в руки немногих, и у Бога небесного нет различия, многими ли спасти, или немногими;
\vs 1Ma 3:19 ибо не от множества войска бывает победа на войне, но с неба приходит сила.
\vs 1Ma 3:20 Они идут против нас во множестве надменности и нечестия, чтобы истребить нас и жен наших и детей наших, чтобы ограбить нас;
\vs 1Ma 3:21 а мы сражаемся за души наши и законы наши.
\vs 1Ma 3:22 Он Сам сокрушит их пред лицем нашим; вы же не страшитесь их.
\vs 1Ma 3:23 Перестав говорить, он внезапно бросился на них, и поражен был Сирон и войско его перед ним.
\vs 1Ma 3:24 И они преследовали его по спуску Вефорона до самой равнины; и пало из них до восьмисот мужей, прочие же убежали в землю Филистимскую.
\vs 1Ma 3:25 И начал страх перед Иудою и братьями его и боязнь нападать на всех окрестных язычников.
\vs 1Ma 3:26 Дошло и до царя имя его, и все народы рассказывали о битвах Иуды.
\rsbpar\vs 1Ma 3:27 Когда же услышал эти речи царь Антиох, то воспылал гневом и, послав, собрал все силы царства своего, весьма сильное ополчение;
\vs 1Ma 3:28 и открыл казнохранилище свое, и выдал войскам своим годовое жалованье, и приказал им быть готовыми на всякую надобность.
\vs 1Ma 3:29 Но увидел, что истощилось серебро в казнохранилищах, а подати страны скудны по причине волнения и разорения, которое он произвел в земле той, уничтожая законы, существовавшие от дней древних.
\vs 1Ma 3:30 И начал он опасаться, что у него недостанет, разве только на раз или два, на издержки и подарки, которые прежде раздавал щедрою рукою и превзошел в том прежних царей.
\vs 1Ma 3:31 Сильно озабоченный в душе своей, он решился идти в Персию и взять подати со стран и собрать побольше серебра.
\vs 1Ma 3:32 А дела царские от реки Евфрата до пределов Египта предоставил Лисию, человеку знаменитому, происходившему от рода царского,
\vs 1Ma 3:33 также и воспитание сына своего, Антиоха, до его возвращения;
\vs 1Ma 3:34 и передал ему половину войск и слонов, дав ему приказания о всем, чего хотел, и о жителях Иудеи и Иерусалима,
\vs 1Ma 3:35 чтобы он послал против них войско сокрушить и уничтожить могущество Израиля и остаток Иерусалима, и истребить память их от места того,
\vs 1Ma 3:36 и поселить во всех пределах их сынов иноплеменных, и разделить по жребию землю их.
\vs 1Ma 3:37 Царь же взял остальную половину войска и отправился из Антиохии, престольного города своего, в сто сорок седьмом году и, перейдя реку Евфрат, прошел верхние страны.
\vs 1Ma 3:38 Лисий избрал Птоломея, сына Дорименова, и Никанора и Горгия, мужей сильных из друзей царя,
\vs 1Ma 3:39 и послал с ними сорок тысяч мужей и семь тысяч всадников, чтобы идти в землю Иудейскую и разорить ее по слову царя.
\vs 1Ma 3:40 Они отправились со всем войском своим и, придя, расположились на равнине близ Еммаума.
\vs 1Ma 3:41 Купцы этой страны услышали имя их и, взяв весьма много серебра и золота и слуг, пришли в стан покупать сынов Израиля в рабы; к ним присоединилось и войско Сирии и земл\acc{и} иноплеменных.
\rsbpar\vs 1Ma 3:42 Увидел Иуда и братья его, что умножились бедствия и войска расположились станом в пределах их; узнали и о повелении царя, которое он приказал исполнить над народом к погублению и истреблению его.
\vs 1Ma 3:43 И говорили каждый ближнему своему: восставим низверженный народ наш и сразимся за народ наш и за святыню.
\vs 1Ma 3:44 И собрался сонм, чтобы быть готовыми к войне и помолиться, и испросить милости и сожаления.
\vs 1Ma 3:45 Иерусалим был необитаем, как пустыня; не было ни входящего в него, ни выходящего из него из природных жителей его; святилище было попрано, и сыновья инородных были в крепости его; он стал жилищем язычников; и отнято веселье у Иакова, и не слышно стало свирели и цитры.
\vs 1Ma 3:46 Итак, они собрались и пошли в Массифу, напротив Иерусалима, ибо место молитвы у Израильтян было прежде в Массифе.
\vs 1Ma 3:47 И постились в этот день, и возложили на себя вретища и пепел на головы свои, и разодрали одежды свои,
\vs 1Ma 3:48 раскрыли книгу закона из тех, которые язычники отыскивали, чтобы сделать на них изображения своих идолов,
\vs 1Ma 3:49 и принесли священнические облачения и первородных и десятины; и созвали назореев, исполнивших дни свои,
\vs 1Ma 3:50 и громко возопили к небу: что нам делать с ними и куда отвести их?
\vs 1Ma 3:51 Святилище Твое попрано и осквернено, и священники Твои в скорби и уничижении.
\vs 1Ma 3:52 И вот, собрались против нас язычники, чтобы истребить нас. Ты знаешь, что умышляют они против нас.
\vs 1Ma 3:53 Как можем мы устоять пред лицем их, если Ты не поможешь нам?
\vs 1Ma 3:54 И вострубили трубами и воскликнули громким голосом.
\rsbpar\vs 1Ma 3:55 После сего Иуда поставил вождей для народа~--- тысяченачальников, стоначальников, пятидесятиначальников и десятиначальников.
\vs 1Ma 3:56 И сказали тем, которые строили дома, обручились с женами, насадили виноградники, и людям боязливым, чтобы каждый из них, по закону, возвратился в свой дом.
\vs 1Ma 3:57 Тогда двинулось ополчение и расположилось станом на юге от Еммаума.
\vs 1Ma 3:58 И сказал Иуда: опояшьтесь и будьте мужественны и готовы к утру сразиться с этими язычниками, которые собрались против нас, чтобы погубить нас и святыню нашу.
\vs 1Ma 3:59 Ибо лучше нам умереть в сражении, нежели видеть бедствия нашего народа и святыни.
\vs 1Ma 3:60 А какая будет воля на небе, так да сотворит!
\vs 1Ma 4:1 И взял Горгий пять тысяч мужей и тысячу отборных всадников, и двинулось ополчение ночью,
\vs 1Ma 4:2 чтобы напасть на ополчение Иудеев и поразить их внезапно, а жившие в крепости служили ему проводниками.
\vs 1Ma 4:3 И услышал Иуда и выступил сам и храбрые мужи, чтобы поразить войско царя в Еммауме,
\vs 1Ma 4:4 доколе силы неприятельские были еще в отдаленности от стана.
\vs 1Ma 4:5 И пришел Горгий в стан Иуды ночью, и никого не нашел, и искал их по горам, ибо говорил: они бегут от нас.
\vs 1Ma 4:6 Но с рассветом дня Иуда явился на равнине с тремя тысячами мужей, но они не имели ни щитов, ни мечей, как того желали.
\vs 1Ma 4:7 Когда увидели они крепкое и вооруженное ополчение язычников и окружающую его конницу, обученных для войны,
\vs 1Ma 4:8 Иуда сказал бывшим с ним мужам: не бойтесь множества их и не страшитесь нападения их.
\vs 1Ma 4:9 Вспомните, как спасены были отцы наши в Чермном море, когда фараон преследовал их с войском.
\vs 1Ma 4:10 И ныне возопием на небо; может быть, Он умилосердится над нами, воспомянув завет с отцами нашими, и сокрушит ныне это ополчение перед лицем нашим;
\vs 1Ma 4:11 и все язычники познают, что есть Избавляющий и Спасающий Израиля.
\vs 1Ma 4:12 Иноплеменники, подняв глаза свои, увидели, что идут против них,
\vs 1Ma 4:13 и вышли из стана на сражение, а бывшие с Иудою затрубили,
\vs 1Ma 4:14 и сошлись, и разбиты были язычники, и побежали на равнину,
\vs 1Ma 4:15 а все остальные пали от меча; и преследовали их до Газера и до равнин Идумеи, Азота и Иамнии, и пали из них до трех тысяч мужей.
\vs 1Ma 4:16 И возвратился Иуда и войско его от преследования их
\vs 1Ma 4:17 и сказал народу: не бросайтесь на добычу, ибо война еще предстоит нам;
\vs 1Ma 4:18 Горгий и войско его на горе близ нас; станьте теперь против врагов наших и сражайтесь с ними, а после смело возьмете добычу.
\vs 1Ma 4:19 Когда еще говорил это Иуда, показалась некоторая толпа, выступавшая с горы.
\vs 1Ma 4:20 И увидел он, что их обратили в бегство и жгут лагерь; ибо поднимающийся дым показывал, что произошло.
\vs 1Ma 4:21 Когда они увидели это, очень испугались; увидев же и войско Иуды на равнине, готовое к сражению,
\vs 1Ma 4:22 все побежали в землю иноплеменников.
\vs 1Ma 4:23 Тогда Иуда обратился на добычу стана, и захватили много золота и серебра, гиацинтовых и багряных одежд и великое богатство.
\vs 1Ma 4:24 И, возвращаясь, воспевали и благословляли Господа небесного, потому что Он благ и что вовек милость Его.
\vs 1Ma 4:25 И было в тот день великое спасение Израилю.
\vs 1Ma 4:26 Уцелевшие же из иноплеменников пришли к Лисию и возвестили о всем случившемся.
\vs 1Ma 4:27 Он, услышав, уныл и опечалился, что не то случилось с Израилем, чего он хотел, и не то вышло, что повелел ему царь.
\vs 1Ma 4:28 И на следующий год Лисий собрал шестьдесят тысяч избранных мужей и пять тысяч всадников, чтобы победить их.
\vs 1Ma 4:29 И пришли они в Идумею и расположились станом в Вефсурах; а Иуда встретил их с десятью тысячами мужей.
\vs 1Ma 4:30 Увидев сильное ополчение, он молился и говорил: благословен Ты, Спаситель Израиля, сокрушивший нападение сильного рукою раба Твоего Давида и предавший полк иноплеменников в руки Ионафана, сына Саулова, и оруженосца его.
\vs 1Ma 4:31 Предай войско сие в руки народа Твоего~--- Израиля, и да будут они постыжены в силе и коннице их;
\vs 1Ma 4:32 наведи на них страх и сокруши дерзость силы их; да будут они потрясены поражением своим;
\vs 1Ma 4:33 низложи их мечом любящих Тебя, и да прославят Тебя в песнях все знающие имя Твое.
\vs 1Ma 4:34 И сразились они, и пало из войска Лисия до пяти тысяч мужей, пали перед ними.
\vs 1Ma 4:35 Лисий, увидев бегство войска своего и храбрость воинов Иуды и что они готовы или жить, или умереть отважно, отправился в Антиохию, набрал чужеземцев и, увеличив бывшее войско, думал снова идти в Иудею.
\rsbpar\vs 1Ma 4:36 Иуда же и братья его сказали: вот, враги наши сокрушены, взойдем очистить и обновить святилище.
\vs 1Ma 4:37 И собралось все ополчение, и взошли на гору Сион.
\vs 1Ma 4:38 И увидели, что святилище опустошено, жертвенник осквернен, ворота сожжены, и в притворах, как в лесу или на какой-либо горе, поросл\acc{и} растения, и хранилища разрушены,
\vs 1Ma 4:39 и разодрали они одежды свои, плакали горьким плачем и сыпали пепел на свои головы,
\vs 1Ma 4:40 и падали лицом на землю и трубили вестовыми трубами, и вопили к небу.
\vs 1Ma 4:41 Тогда отрядил Иуда мужей воевать против находившихся в крепости, доколе он очистит святилище.
\vs 1Ma 4:42 И избрал священников беспорочных, ревнителей закона.
\vs 1Ma 4:43 Они очистили святилище и оскверненные камни вынесли в нечистое место.
\vs 1Ma 4:44 Потом они рассуждали об оскверненном жертвеннике всесожжения, как поступить с ним.
\vs 1Ma 4:45 И пришла им добрая мысль разрушить его, чтобы он когда-нибудь не послужил им в поношение, так как язычники осквернили его; и разрушили они жертвенник,
\vs 1Ma 4:46 и камни сложили на горе храма в приличном месте, пока придет пророк и даст ответ о них.
\vs 1Ma 4:47 Взяли камни целые, по закону, и построили новый жертвенник по-прежнему;
\vs 1Ma 4:48 потом устроили святыни и внутренние части храма и освятили притворы;
\vs 1Ma 4:49 устроили новую священную утварь и внесли в храм свещник и алтарь всесожжений и фимиамов и трапезу;
\vs 1Ma 4:50 и воскурили на алтаре фимиам и зажгли светильники на свещнике, и осветили храм;
\vs 1Ma 4:51 и положили на трапезу хлебы, и развесили завесы, и окончили все дела, которые предприняли.
\rsbpar\vs 1Ma 4:52 В двадцать пятый день девятого месяца~--- это месяц Хаслев~--- сто сорок восьмого года встали весьма рано
\vs 1Ma 4:53 и принесли жертву по закону на новоустроенном жертвеннике всесожжений.
\vs 1Ma 4:54 В то время, в тот самый день, в который язычники осквернили жертвенник, обновлен он с песнями, с цитрами, гуслями и кимвалами.
\vs 1Ma 4:55 И весь народ падал на лицо свое, и молились и воссылали благодарение на небо Благопоспешившему им.
\vs 1Ma 4:56 Так совершали обновление жертвенника восемь дней с весельем, принося всесожжения и вознося жертву спасения и хвалы.
\vs 1Ma 4:57 И украсили переднюю сторону храма золотыми венцами и щитами и возобновили ворота и хранилища, и сделали для них двери.
\vs 1Ma 4:58 И была весьма великая радость в народе, и отвращено было поношение язычников.
\vs 1Ma 4:59 И установил Иуда и братья его и все собрание Израиля, чтобы дни обновления жертвенника празднуемы были с веселием и радостью в свое время, каждый год восемь дней, от двадцатого дня месяца Хаслева.
\vs 1Ma 4:60 В то же время обстроили гору Сион вокруг высокими стенами и крепкими башнями, чтобы язычники, придя когда-нибудь, не попрали их, как сделали это прежде.
\vs 1Ma 4:61 И расположил там Иуда войско стеречь гору, и укрепили для охранения ее Вефсуру, чтобы народ имел крепость против Идумеи.
\vs 1Ma 5:1 Когда окрестные народы услышали, что построен жертвенник и возобновлено святилище, как прежде, сильно вознегодовали;
\vs 1Ma 5:2 и решились истребить род Иакова, живший среди них, и начали убивать и истреблять людей в этом народе.
\vs 1Ma 5:3 Тогда Иуда ополчился против сынов Исава в Идумее, в Акравиме, так как они держали в осаде Израиля, и поразил их великим поражением, и смирил их, и взял добычи их.
\vs 1Ma 5:4 Вспомнил он и о злобе сынов Веана, которые были для народа сетью и претыканием, строя ему засады на дорогах.
\vs 1Ma 5:5 Хотя они заперлись от него в башнях, но он ополчился против них, предал их заклятию и сожег огнем башни их со всеми, бывшими в них.
\vs 1Ma 5:6 Потом он перешел к сынам Аммона и встретил сильное войско и многочисленный народ и Тимофея, предводителя их.
\vs 1Ma 5:7 Он имел с ними много сражений, и они были разбиты пред лицем его; он поразил их;
\vs 1Ma 5:8 взял Иазер и селения его и возвратился в Иудею.
\vs 1Ma 5:9 Тогда собрались язычники, жившие в Галааде, против Израильтян, находившихся в пределах их, чтобы истребить их; но они бежали в крепость Дафему.
\vs 1Ma 5:10 И послали письма к Иуде и братьям его и сказали: собрались против нас окружающие нас язычники, чтобы истребить нас,
\vs 1Ma 5:11 и готовятся идти и сделать нападение на крепость, в которую мы убежали, и Тимофей предводительствует войском их.
\vs 1Ma 5:12 Итак, приди и избавь нас от руки их, ибо множество из нас погибло;
\vs 1Ma 5:13 и все братья наши, бывшие в пределах Това, преданы смерти, а жен их и детей их и имущество взяли в плен, и погубили там около тысячи мужей.
\vs 1Ma 5:14 Еще читались эти письма, как вот, пришли другие вестники из Галилеи в разодранных одеждах с таким извещением:
\vs 1Ma 5:15 собрались против нас из Птолемаиды и из Тира и Сидона, и из всей Галилеи языческой, чтобы погубить нас.
\vs 1Ma 5:16 Когда услышал эти слова Иуда и народ, то собралось великое собрание для совещания, что сделать для сих братьев, находящихся в бедствии и угрожаемых войною от тех язычников?
\vs 1Ma 5:17 Тогда Иуда сказал Симону, брату своему: выбери себе мужей и иди и защити братьев твоих, находящихся в Галилее; а я и Ионафан, брат мой, пойдем в Галаад.
\vs 1Ma 5:18 И оставил он Иосифа, сына Захарии, и Азарию начальниками над народом с остатком войска в Иудее на охранение.
\vs 1Ma 5:19 И дал им повеление, сказав: управляйте народом сим, но не начинайте войны против язычников до нашего возвращения.
\vs 1Ma 5:20 Симону отделены для похода в Галилею три тысячи мужей, Иуде же~--- в Галаад восемь тысяч мужей.
\vs 1Ma 5:21 И отправился Симон в Галилею и произвел много сражений с язычниками, и разбиты им язычники.
\vs 1Ma 5:22 Он преследовал их до ворот Птолемаиды, и пало из язычников до трех тысяч мужей, и он взял добычи их.
\vs 1Ma 5:23 Также взял он с собою находившихся в Галилее и Арваттах \bibemph{Иудеев} с женами и детьми и со всем имением их и привел в Иудею с великою радостью.
\vs 1Ma 5:24 А Иуда Маккавей и Ионафан, брат его, перешли Иордан и совершили трехдневный путь в пустыне.
\vs 1Ma 5:25 Их встретили Навуфеи и приняли мирно, и рассказали им все, случившееся с братьями их в Галааде,
\vs 1Ma 5:26 и что многие из них заперты в Васаре и Восоре, в Алемах, Хасфоре, Македе и Карнаине~--- все сии города укреплены и велики~---
\vs 1Ma 5:27 и в прочих городах Галаада находятся в осаде, и что завтра назначено напасть на эти укрепления и взять их и погубить всех их в один день.
\vs 1Ma 5:28 Посему Иуда со своим войском вдруг направил путь свой в пустыню к Восору и взял этот город, и избил весь мужеский пол острием меча, и взял все добычи их, и сожег его огнем;
\vs 1Ma 5:29 а оттуда отправился ночью и шел до укрепления.
\vs 1Ma 5:30 Когда наступало утро, и подняли глаза, и вот, народ многочисленный, которому числа не было, поднимают лестницы и машины, чтобы взять укрепление, и осаждают бывших в нем.
\vs 1Ma 5:31 Увидел Иуда, что началась битва и вопль города восходил на небо трубами и громким криком,
\vs 1Ma 5:32 и сказал воинам: сражайтесь теперь за братьев ваших.
\vs 1Ma 5:33 Он обошел врагов с тыла с тремя отрядами, и затрубили трубами и воскликнули с молитвою;
\vs 1Ma 5:34 и узнало войско Тимофея, что это~--- Маккавей, и побежали от лица его, и он поразил их великим поражением, и пало из них в этот день до восьми тысяч мужей.
\vs 1Ma 5:35 Тогда поворотил он в Масфу и осадил и взял ее, избил весь мужеский пол в ней, взял добычи ее и сожег ее огнем;
\vs 1Ma 5:36 отправившись оттуда, он взял Хасфон, Макед, Восор и прочие города Галаадские.
\rsbpar\vs 1Ma 5:37 После этих событий Тимофей собрал другое войско и расположился станом перед Рафоном по ту сторону потока.
\vs 1Ma 5:38 И послал Иуда осмотреть войско, и объявили ему и сказали: собрались к ним все окружающие нас язычники~--- сила весьма многочисленная,
\vs 1Ma 5:39 и они наняли в помощь себе Аравитян и расположились станом за потоком, будучи готовы идти против тебя войною. И пошел Иуда навстречу им.
\vs 1Ma 5:40 Тогда Тимофей сказал своим военачальникам, когда Иуда и войско его приближались к потоку воды: если он перейдет к нам прежде, то мы не в силах будем устоять против него, ибо он превозможет нас.
\vs 1Ma 5:41 Если же он убоится и расположится станом по ту сторону потока, то мы перейдем к нему и превозможем его.
\vs 1Ma 5:42 Как только подошел Иуда к потоку воды, то поставил при потоке народных писцов и приказал им, сказав: не оставляйте ни одного человека в стане, но пусть все идут на сражение.
\vs 1Ma 5:43 И переправился к ним первый и весь народ за ним. И сокрушены были пред лицем его все язычники, и бросили оружие свое, и убежали в капище, которое было в Карнаине.
\vs 1Ma 5:44 Тогда взяли они этот город и сожгли огнем капище со всеми находившимися в нем; и побежден был Карнаин и не мог более противостоять Иуде.
\vs 1Ma 5:45 И собрал Иуда всех Израильтян, находившихся в Галааде, от малого до большого, и жен их, и детей их, и имение, очень большое ополчение, чтобы идти в землю Иудейскую.
\vs 1Ma 5:46 И дошли они до Ефрона. Это был большой город, весьма укрепленный, на пути; невозможно было уклониться от него ни вправо, ни влево; надобно было пройти посреди него,
\vs 1Ma 5:47 а жители заперлись в нем и ворота завалили камнями.
\vs 1Ma 5:48 Иуда послал к ним с мирным предложением: мы пройдем по земле вашей, чтобы идти нам в землю нашу, и никто не обидит вас, только ногами нашими пройдем. Но они не захотели отворить ему.
\vs 1Ma 5:49 Тогда Иуда приказал объявить в ополчении, чтобы каждый ополчился на своем месте;
\vs 1Ma 5:50 и ополчились воины и осаждали город весь тот день и всю ночь, и сдался город в руки его.
\vs 1Ma 5:51 И побил он весь мужеский пол острием меча и до основания разрушил город, и взял добычи его, и прошел через город по убитым.
\vs 1Ma 5:52 И переправились через Иордан на великую равнину против Вефсана.
\vs 1Ma 5:53 И собирал Иуда отставших и ободрял народ в продолжение всего пути, доколе не пришли в землю Иудейскую.
\vs 1Ma 5:54 И взошли на гору Сион с весельем и радостью и принесли всесожжения, потому что никто не пал из них до самого возвращения в мире.
\rsbpar\vs 1Ma 5:55 В те дни, когда Иуда и Ионафан находились в Галааде, а Симон, брат его,~--- в Галилее перед Птолемаидою,
\vs 1Ma 5:56 услышали Иосиф, сын Захарии, и Азарий, военачальники, о славных воинских подвигах, совершенных ими,
\vs 1Ma 5:57 и сказали: сделаем и мы себе имя; пойдем воевать с язычниками, окружающими нас.
\vs 1Ma 5:58 Так объявили они бывшему при них войску и пошли на Иамнию.
\vs 1Ma 5:59 И вышел Горгий из города и воины его навстречу им на сражение.
\vs 1Ma 5:60 И, обратившись в бегство, Иосиф и Азария были преследуемы до пределов Иудеи; и пали в этот день из народа Израильского до двух тысяч мужей.
\vs 1Ma 5:61 И было великое замешательство в народе Израильском, потому что не послушались Иуды и братьев его, мечтая показать храбрость,
\vs 1Ma 5:62 тогда как они не были от семени тех мужей, руке которых предоставлено спасение Израиля.
\vs 1Ma 5:63 Но муж Иуда и братья его весьма прославились перед всем Израилем и перед всеми народами, где только слышно было имя их,~---
\vs 1Ma 5:64 и собирались к ним приветствующие.
\rsbpar\vs 1Ma 5:65 После того вышел Иуда и братья его и воевали против сынов Исава в земле, лежащей к югу, и поразил Хеврон и селения его, и разрушил укрепление его, и сожег башни его вокруг него,
\vs 1Ma 5:66 и поднялся, чтобы идти в землю иноплеменников, и прошел Самарию.
\vs 1Ma 5:67 В то время пали в сражении священники, желавшие прославиться храбростью и безрассудно вышедшие на войну.
\vs 1Ma 5:68 И обратился Иуда в Азот, землю иноплеменников, разрушил жертвенники их, сожег огнем резные изображения богов их, взял добычи городов и возвратился в землю Иудейскую.
\vs 1Ma 6:1 Между тем царь Антиох, проходя верхние области, услышал, что есть в Персии город Елимаис, славящийся богатством, серебром и золотом,
\vs 1Ma 6:2 и в нем~--- храм, весьма богатый, и есть там золотые покровы, брони и оружия, которые оставил там Александр, сын Филиппа, царь Македонский,~--- первый, воцарившийся над Еллинами.
\vs 1Ma 6:3 И он пришел и старался взять этот город и ограбить его, но не мог, потому что намерение его стало известно жителям города.
\vs 1Ma 6:4 Они поднялись против него войною, и он обратился в бегство и ушел оттуда с великою скорбью, чтобы отправиться в Вавилон.
\vs 1Ma 6:5 Тогда пришел некто к нему в Персию с известием, что ополчения, ходившие в землю Иуды, обращены в бегство,
\vs 1Ma 6:6 что Лисий ходил с сильным войском впереди всех, но был поражен \bibemph{Иудеями}, и они усилились и оружием, и войском, и многими добычами, которые взяли от пораженных ими войск,
\vs 1Ma 6:7 и что они разрушили мерзость, которую он воздвиг над жертвенником в Иерусалиме, а святилище по-прежнему обнесли высокими стенами, также и Вефсуру, город его.
\vs 1Ma 6:8 Когда царь услышал слова сии, сильно испугался и встревожился, упал на постель и впал в изнеможение от печали, что не сбылось так, как он желал.
\vs 1Ma 6:9 И много дней пробыл он там, ибо возобновлялась в нем сильная печаль; он думал, что умирает.
\vs 1Ma 6:10 И созвал он всех друзей своих и сказал им: удалился сон от глаз моих, и я изнемог сердцем от печали.
\vs 1Ma 6:11 И сказал я в сердце моем: до какой скорби дошел я и до какого великого смущения, в котором нахожусь теперь! А был я полезен и любим во владычестве моем.
\vs 1Ma 6:12 Теперь же я воспоминаю о тех злодеяниях, которые я совершил в Иерусалиме, и как взял все находившиеся в нем золотые и серебряные сосуды и посылал истреблять обитающих в Иудее напрасно.
\vs 1Ma 6:13 Теперь я позна\acc{ю}, что за это постигли меня эти беды,~--- и вот, я погибаю от великой печали в чужой земле.
\vs 1Ma 6:14 И призвал он Филиппа, одного из друзей своих, и поставил его правителем над всем царством своим;
\vs 1Ma 6:15 и дал ему венец и царскую одежду свою и перстень, чтобы он руководил Антиоха, сына его, и воспитывал его для царствования.
\vs 1Ma 6:16 И умер царь Антиох в сто сорок девятом году.
\rsbpar\vs 1Ma 6:17 Когда Лисий узнал, что царь умер, то поставил вместо него на царство сына его, Антиоха, которого воспитывал в юности его, и назвал его именем Евпатора.
\vs 1Ma 6:18 Между тем находившиеся в крепости теснили Израиля вокруг святилища и всегда старались делать ему зло, а язычникам служить опорою;
\vs 1Ma 6:19 тогда Иуда решил выгнать их и созвал весь народ, чтобы осадить их.
\vs 1Ma 6:20 Все собрались и осадили их в сто пятидесятом году, и устроил он против них стрелометательные орудия и машины.
\vs 1Ma 6:21 Но некоторые из осажденных вышли, и к ним пристали некоторые из нечестивых Израильтян;
\vs 1Ma 6:22 и пошли они к царю и сказали: доколе ты не сделаешь суда и не отмстишь за братьев наших?
\vs 1Ma 6:23 Мы согласились служить отцу твоему и ходить по заповедям его и следовать повелениям его;
\vs 1Ma 6:24 а сыны народа нашего осадили крепость и за то чуждаются нас, и кого из нас находят, умерщвляют, и имущества наши расхищают,
\vs 1Ma 6:25 и не на нас только простерли они руку, но и на все пределы наши.
\vs 1Ma 6:26 И вот, теперь осадили они крепость в Иерусалиме, чтобы овладеть ею, а святилище и Вефсуру укрепили.
\vs 1Ma 6:27 Если ты не поспешишь предупредить их, то они сделают больше этого, и тогда ты не в силах будешь удержать их.
\vs 1Ma 6:28 Услышав это, царь разгневался и собрал всех друзей своих и начальников войска своего и начальников конницы;
\vs 1Ma 6:29 пришли к нему и из других царств и с морских островов войска наемные,
\vs 1Ma 6:30 так что число войск его было: сто тысяч пеших, двадцать тысяч всадников и тридцать два слона, приученных к войне.
\vs 1Ma 6:31 И прошли они через Идумею и расположились станом против Вефсуры, и сражались много дней и устроили машины; но те сделали вылазку и сожгли их огнем и сразились мужественно.
\vs 1Ma 6:32 После сего Иуда отступил от крепости и расположился станом в Вефсахаре против стана царского.
\vs 1Ma 6:33 Царь же, встав рано утром, поспешно отправился с войском своим по дороге к Вефсахаре, и приготовились войска к сражению и затрубили трубами.
\vs 1Ma 6:34 Слонам показывали кровь винограда и тутовых ягод, чтобы возбудить их к битве,
\vs 1Ma 6:35 и разделили этих животных на отряды и приставили к каждому слону по тысяче мужей в железных кольчугах и с медными шлемами на головах, сверх того по пятисот отборных всадников назначено было к каждому слону.
\vs 1Ma 6:36 Они становились заблаговременно там, где был слон, и куда он шел, шли и они вместе, не отставая от него.
\vs 1Ma 6:37 Притом на них были крепкие деревянные башни, покрывавшие каждого слона, укрепленные на них помочами, и в каждой из них по тридцати по два сильных мужей, которые сражались на них, и при слоне Индиец его.
\vs 1Ma 6:38 Остальных же всадников расставили здесь и там~--- на двух сторонах ополчения, чтобы подавать знаки и подкреплять в тесных местах.
\vs 1Ma 6:39 Когда солнце блеснуло на золотых и медных щитах, то заблистали от них горы и светились, как огненные светильники.
\vs 1Ma 6:40 Одна часть царского войска протянута была по высоким горам, а другие~--- по низменным местам; и шли они твердо и стройно.
\vs 1Ma 6:41 И смутились все, слышавшие шум множества их и шествие такого полчища и стук оружий, ибо войско было весьма великое и сильное.
\vs 1Ma 6:42 И вступил Иуда и войско его в сражение~--- и пали из ополчения царского шестьсот мужей.
\vs 1Ma 6:43 Тогда Елеазар, сын Саварана, увидел, что один из слонов покрыт бронею царскою и превосходил всех, и казалось, что на нем был царь,~---
\vs 1Ma 6:44 и он предал себя, чтобы спасти народ свой и приобрести себе вечное имя;
\vs 1Ma 6:45 и смело побежал к нему в средину отряда, поражая направо и налево, и расступались от него и в ту, и в другую сторону;
\vs 1Ma 6:46 и подбежал он под того слона, лег под него и убил его, и пал на него слон на землю, и он умер там.
\vs 1Ma 6:47 Но, увидев силу царского ополчения и стремительность войск, Иудеи уклонились от них.
\vs 1Ma 6:48 Царские же войска пошли против них на Иерусалим: царь направил войска на Иудею и на гору Сион.
\vs 1Ma 6:49 И заключил он мир с бывшими в Вефсуре, которые вышли из города, ибо не было у них продовольствия, чтобы держаться в нем в осаде, потому что был субботний год на земле.
\vs 1Ma 6:50 И овладел царь Вефсурою и оставил в ней стражу, чтобы стеречь ее.
\vs 1Ma 6:51 Потом много дней осаждал святилище и поставил там стрелометательные орудия и машины, и огнеметательные, и камнеметательные, и копьеметательные, чтобы бросать стрелы и камни.
\vs 1Ma 6:52 Но и Иудеи устроили машины против их машин и сражались много дней;
\vs 1Ma 6:53 съестных же припасов недостало в хранилищах, потому что был седьмой год, и искавшие в Иудее безопасности от язычников издержали остатки запасов;
\vs 1Ma 6:54 и осталось при святилище немного мужей, ибо одолел их голод, и разошлись каждый в свое место.
\rsbpar\vs 1Ma 6:55 Услышал Лисий, что Филипп, которому царь Антиох еще при жизни поручил воспитывать сына своего, Антиоха, для царствования,
\vs 1Ma 6:56 возвратился из Персии и Мидии и с ним ходившие с царем войска, и что он домогается принять на себя дела царства.
\vs 1Ma 6:57 Почему поспешно пошел и сказал царю, начальникам войска и вельможам: мы каждый день терпим недостаток и продовольствия у нас мало, а место, осаждаемое нами, крепко, между тем лежит на нас попечение о царстве.
\vs 1Ma 6:58 Итак, подадим правую руку этим людям и заключим с ними мир и со всем народом их,
\vs 1Ma 6:59 и предоставим им поступать по законам их, как прежде; ибо за свои законы, которые мы отменили, они раздражились и сделали всё это.
\vs 1Ma 6:60 И угодно было это слово царю и начальникам,~--- и послал он к ним, чтобы заключить мир, что они и приняли;
\vs 1Ma 6:61 и клялся им царь и начальники. После сего они вышли из крепости.
\vs 1Ma 6:62 И взошел царь на гору Сион и, осмотрев укрепленные места, пренебрег клятвою, которою клялся, и велел разорить стены кругом.
\vs 1Ma 6:63 Потом поспешно отправился, и, возвратившись в Антиохию, он нашел, что Филипп владеет городом, вступил с ним в сражение и силою взял город.
\vs 1Ma 7:1 В сто пятьдесят первом году вышел из Рима Димитрий, сын Селевка, и с немногими людьми вошел в один приморский город и там воцарился.
\vs 1Ma 7:2 Когда же он входил в царственный дом отцов своих, войско схватило Антиоха и Лисия, чтобы привести их к нему.
\vs 1Ma 7:3 Это стало известно ему, и он сказал: не показывайте мне лиц их.
\vs 1Ma 7:4 Тогда воины убили их, и воссел Димитрий на престоле царства своего.
\vs 1Ma 7:5 И пришли к нему все мужи беззаконные и нечестивые из Израильтян, и Алким предводительствовал ими, домогаясь священства;
\vs 1Ma 7:6 и обвиняли они перед царем народ, говоря: погубил Иуда и братья его друзей твоих, и нас выгнали из земли нашей.
\vs 1Ma 7:7 Итак, пошли теперь мужа, кому ты доверяешь; пусть он пойдет и увидит все разорение, которое они причинили нам и стране царя, и пусть накажет их и всех, помогающих им.
\vs 1Ma 7:8 Царь избрал Вакхида из друзей царских, который управлял по ту сторону реки, был велик в царстве и верен царю,
\vs 1Ma 7:9 и послал его и нечестивого Алкима, предоставив ему священство, и повелел ему сделать отмщение сынам Израиля.
\vs 1Ma 7:10 Они отправились и пришли в землю Иудейскую с большим войском; и он послал к Иуде и братьям его послов с мирным, но коварным предложением.
\vs 1Ma 7:11 Но они не вняли словам их, ибо видели, что они пришли с большим войском.
\vs 1Ma 7:12 К Алкиму же и Вакхиду сошлось собрание книжников искать справедливости.
\vs 1Ma 7:13 Первые из сынов Израилевых были Асидеи; они искали у них мира,
\vs 1Ma 7:14 ибо говорили: священник от племени Аарона пришел вместе с войском и не обидит нас.
\vs 1Ma 7:15 И он говорил с ними мирно и клялся им, и сказал: мы не сделаем зла вам и друзьям вашим.
\vs 1Ma 7:16 И они поверили ему, а он, захватив из них шестьдесят мужей, умертвил их в один день, как сказано в Писании:
\vs 1Ma 7:17 <<тела святых Твоих и кровь их пролили вокруг Иерусалима, и некому было похоронить их>>.
\vs 1Ma 7:18 И напал от них страх и ужас на весь народ, и говорили: нет в них истины и правды, ибо они нарушили постановление и клятву, которою клялись.
\vs 1Ma 7:19 Тогда Вакхид отступил от Иерусалима и расположился станом при Визефе, и, послав, поймал многих из бежавших от него мужей и некоторых из народа, заколол и бросил их в глубокий колодезь.
\rsbpar\vs 1Ma 7:20 Потом, поручив страну Алкиму и оставив с ним войско на помощь ему, Вакхид отправился к царю.
\vs 1Ma 7:21 Алким же домогался первосвященства.
\vs 1Ma 7:22 И собрались к нему все возмущавшие народ свой, и овладели землею Иудейскою, и произвели великое поражение в Израиле.
\vs 1Ma 7:23 И увидел Иуда все зло, какое причинил Алким со своими сообщниками сынам Израилевым,~--- больше, нежели язычники;
\vs 1Ma 7:24 и, обойдя все пределы Иудеи, сделал отмщение отступникам,~--- и они перестали входить в эту страну.
\vs 1Ma 7:25 Когда же Алким увидел, что Иуда и находящиеся с ним усилились, и понял, что не может противостоять им, возвратился к царю и жестоко обвинял их.
\vs 1Ma 7:26 Тогда царь послал Никанора, одного из славных вождей своих, ненавистника и враждебного Израилю, и приказал ему истребить этот народ.
\vs 1Ma 7:27 Никанор, придя в Иерусалим с большим войском, послал к Иуде и братьям его коварно со словами мирными:
\vs 1Ma 7:28 да не будет войны между мною и вами; я войду с немногими людьми, чтобы видеть лица ваши в мире.
\vs 1Ma 7:29 И пришел он к Иуде, и приветствовали они друг друга мирно; а между тем воины были приготовлены схватить Иуду.
\vs 1Ma 7:30 Иуде сделалось известным, что он пришел к нему с коварством, поэтому он убоялся его и не хотел более видеть лица его.
\vs 1Ma 7:31 Когда Никанор узнал, что умысел его открылся, вышел против Иуды на сражение близ Хафарсаламы.
\vs 1Ma 7:32 И пало из бывших при Никаноре около пяти тысяч мужей, а прочие убежали в город Давидов.
\vs 1Ma 7:33 После того Никанор взошел на гору Сион; и вышли из святилища некоторые из священников и старейшин народа, чтобы мирно приветствовать его и показать ему всесожжение, приносимое за царя.
\vs 1Ma 7:34 Но он осмеял их, надругался над ними и осквернил их, и говорил высокомерно,
\vs 1Ma 7:35 и, поклявшись, с гневом сказал: если не предан будет ныне Иуда и войско его в мои руки, то, когда возвращусь благополучно, сожгу дом сей. И ушел с великим гневом.
\vs 1Ma 7:36 А священники вошли и стали пред лицем жертвенника и храма, заплакали и сказали:
\vs 1Ma 7:37 Ты, Господи, избрал дом сей, чтобы на нем нарицалось имя Твое и чтобы он был домом молитвы и моления для народа Твоего.
\vs 1Ma 7:38 Сделай отмщение человеку сему и войску его, и пусть падут они от меча; вспомни злохуления их и не дай им оставаться долее.
\vs 1Ma 7:39 И вышел Никанор из Иерусалима и расположился станом при Вефороне, и пристало к нему здесь войско Сирийское.
\vs 1Ma 7:40 А Иуда с тремя тысячами мужей расположился станом при Адасе; и помолился Иуда, и сказал:
\vs 1Ma 7:41 Господи! когда посланные царя Ассирийского произносили злохуления, то пришел Ангел Твой и поразил из них сто восемьдесят пять тысяч.
\vs 1Ma 7:42 Так сокруши ныне пред нами сие полчище, да познают прочие, что они произносили хулу на святыни Твои, и суди их по злобе их.
\vs 1Ma 7:43 И вступили войска в сражение в тринадцатый день месяца Адара, и разбито было войско Никанора, и он первый пал в сражении.
\vs 1Ma 7:44 Когда же воины его увидели, что Никанор пал, то, побросав оружие свое, обратились в бегство.
\vs 1Ma 7:45 И преследовали их Израильтяне целый день, от Адаса до самой Газиры, и трубили вслед их вестовыми трубами.
\vs 1Ma 7:46 И выходили из всех окрестных селений Иудейских и окружали их,~--- и они, оборачиваясь к преследовавшим их, все пали от меча, и ни одного не осталось из них.
\vs 1Ma 7:47 И взяли \bibemph{Иудеи} добычи их и награбленное ими, и отрубили голову Никанора и правую руку его, которую он простирал надменно, и принесли и повесили перед Иерусалимом.
\vs 1Ma 7:48 Народ весьма радовался и провел тот день, как день великого веселья;
\vs 1Ma 7:49 и установили ежегодно праздновать этот день тринадцатого числа Адара.
\vs 1Ma 7:50 И успокоилась земля Иудейская на некоторое время.
\vs 1Ma 8:1 Иуда услышал о славе Римлян, что они могущественны и сильны и благосклонно принимают всех, обращающихся к ним, и кто ни приходил к ним, со всеми заключали они дружбу.
\vs 1Ma 8:2 А что они могущественны и сильны,~--- рассказывали ему о войнах их, о мужественных подвигах, которые они показали над Галатами, как они покорили их и сделали данниками;
\vs 1Ma 8:3 также о том, что сделали они в стране Испанской, чтобы овладеть находящимися там серебряными и золотыми рудниками,
\vs 1Ma 8:4 и своим благоразумием и твердостью овладели всем краем, хотя тот край весьма далеко отстоял от них, равно о царях, которые приходили против них от конца земли, и они сокрушили их и поразили великим поражением, а прочие платят им ежегодно дань;
\vs 1Ma 8:5 они также сокрушили на войне и покорили себе Филиппа и Персея, царя Китийского, и других, восставших против них,
\vs 1Ma 8:6 и Антиоха, великого царя Азии, который вышел против них на войну со ста двадцатью слонами, и с конницею, и колесницами, и весьма многочисленным войском и был разбит ими;
\vs 1Ma 8:7 они взяли его живого и заставили платить им великую дань,~--- как его, так и следующих после него царей,~--- дать заложников и допустить раздел,
\vs 1Ma 8:8 а страну Индийскую и Мидию, и Лидию, и другие из лучших областей его, взяв от него, отдали царю Евмению;
\vs 1Ma 8:9 и о том, как Еллины вознамерились прийти и истребить их,
\vs 1Ma 8:10 но это намерение сделалось им известным, и они послали против них одного военачальника и воевали против них,~--- и много из них пало пораженных, и взяли в плен жен их и детей их и разграбили их, и овладели их землею, и разорили крепости их, и поработили их до сего дня;
\vs 1Ma 8:11 и другие царства и острова, которые когда-либо восставали против них, они разорили и поработили.
\vs 1Ma 8:12 А с друзьями своими и с доверявшимися им они сохраняли дружбу; и овладели царствами ближними и дальними, и все, слышавшие имя их, боялись их.
\vs 1Ma 8:13 Если захотят кому помочь и кого воцарить, те царствуют, и кого хотят, сменяют, и они весьма возвысились;
\vs 1Ma 8:14 но при всем том никто из них не возлагал на себя венца и не облекался в порфиру, чтобы величаться ею.
\vs 1Ma 8:15 Они составили у себя совет, и постоянно каждый день триста двадцать человек совещаются обо всем, что относится до народа и благоустроения его;
\vs 1Ma 8:16 и каждый год одному человеку вверяют они начальство над собою и господство над всею землею их, и все слушают одного, и не бывает ни зависти, ни ревности между ними.
\rsbpar\vs 1Ma 8:17 Тогда избрал Иуда Евполема, сына Иоаннова, сына Аккосова, и Иасона, сына Елеазарова, и послал их в Рим, чтобы заключить с ними дружбу и союз
\vs 1Ma 8:18 и чтобы они сняли с них иго, ибо они видят, что Еллинское царство хочет поработить Израиля.
\vs 1Ma 8:19 Итак, они отправились в Рим, хотя путь был очень долгий, и вошли в собрание совета и, приступив, сказали:
\vs 1Ma 8:20 Иуда Маккавей и братья его и весь народ Иудейский послали нас к вам, чтобы заключить с вами союз и мир и чтобы вы вписали нас в число соратников и друзей ваших.
\vs 1Ma 8:21 И угодно было это слово перед ними.
\vs 1Ma 8:22 И вот список того послания, которое написали они в ответ на медных досках и послали в Иерусалим, чтобы оно служило для них там памятником мира и союза:
\vs 1Ma 8:23 <<благо да будет Римлянам и народу Иудейскому на море и на суше навеки, и меч и враг да будут далеко от них!
\vs 1Ma 8:24 Если же настанет война прежде у Римлян или у всех союзников их во всем владении их,
\vs 1Ma 8:25 то народ Иудейский должен оказать им всем сердцем помощь в войне, как потребует того время;
\vs 1Ma 8:26 и воюющим они не будут ни давать, ни доставлять ни хлеба, ни оружия, ни денег, ни кораблей, ибо так угодно Римлянам; они должны исполнять обязанность свою, ничего не получая.
\vs 1Ma 8:27 Точно так же, если прежде случится война у народа Иудейского, Римляне от души будут помогать им в войне, как потребует того время,
\vs 1Ma 8:28 и помогающим в войне не будут давать ни хлеба, ни оружия, ни денег, ни кораблей: так угодно Риму; они должны исполнять свои обязанности~--- и без обмана>>.
\vs 1Ma 8:29 На таких условиях заключили Римляне союз с народом Иудейским.
\vs 1Ma 8:30 Если же после сих условий те и другие вздумают что-нибудь прибавить или убавить, пусть сделают это по их общему произволению, и то, что они прибавят или убавят, будет иметь силу.
\vs 1Ma 8:31 А о том зле, какое делает \bibemph{Иудеям} царь Димитрий, мы написали ему так: <<для чего ты наложил тяжкое твое иго на друзей наших и союзников~--- Иудеев?
\vs 1Ma 8:32 Если они еще обратятся к нам с жалобою на тебя, то мы окажем им справедливость и будем воевать против тебя на море и на суше>>.
\vs 1Ma 9:1 Когда Димитрий услышал, что Никанор и воины его пали в сражении, послал Вакхида и Алкима во второй раз в землю Иудейскую и правое крыло с ними.
\vs 1Ma 9:2 И отправились они по дороге в Галгалы и расположились станом при Месалофе, что в Арвилах, и, овладев им, погубили множество людей.
\vs 1Ma 9:3 В первом месяце сто пятьдесят второго года расположились они станом у Иерусалима,
\vs 1Ma 9:4 но снялись и пошли к Верее с двадцатью тысячами мужей и двумя тысячами конницы.
\vs 1Ma 9:5 А Иуда расположился станом при Елеасе, и три тысячи избранных мужей с ним.
\vs 1Ma 9:6 Но, увидев множество войска, как оно многочисленно, они весьма устрашились, и многие из стана его разбежались, и осталось из них не более восьмисот мужей.
\vs 1Ma 9:7 Когда увидел Иуда, что разбежалось ополчение его, а война тревожила его, он смутился сердцем, потому что не имел времени собрать их.
\vs 1Ma 9:8 Он опечалился и сказал оставшимся: встанем и пойдем на противников наших; может быть, мы в силах будем сражаться с ними.
\vs 1Ma 9:9 Но они отклоняли его и говорили: мы не в силах, но будем теперь спасать жизнь нашу, и потом возвратимся с братьями нашими, и тогда будем сражаться против них, а теперь нас мало.
\vs 1Ma 9:10 Но Иуда сказал: нет, да не будет этого со мною, чтобы бежать от них; а если пришел час наш, то умрем мужественно за братьев наших и не оставим нарекания на славу нашу.
\vs 1Ma 9:11 И двинулось войско из стана и стало против них; и разделилась конница на две части, а впереди войска шли пращники и стрельцы и все сильные передовые воины.
\vs 1Ma 9:12 Вакхид же находился на правом крыле, и приближались отряды с обеих сторон и трубили трубами.
\vs 1Ma 9:13 Затрубили трубами и бывшие с Иудою, и поколебалась земля от шума войск, и было упорное сражение от утра до вечера.
\vs 1Ma 9:14 Когда увидел Иуда, что Вакхид и крепчайшая часть его войска находится на правой стороне, то собрались к нему все храбрые сердцем,~---
\vs 1Ma 9:15 и разбито ими правое крыло, и они преследовали их до горы Азота.
\vs 1Ma 9:16 Когда находившиеся на левом крыле увидели, что правое крыло разбито, то обратились вслед за Иудою и бывшими с ним, с тыла.
\vs 1Ma 9:17 И сражение было жестокое, и много пало пораженных с той и другой стороны,
\vs 1Ma 9:18 пал и Иуда, а прочие обратились в бегство.
\vs 1Ma 9:19 И взяли Ионафан и Симон Иуду, брата своего, и похоронили его во гробе отцов его в Модине.
\vs 1Ma 9:20 И оплакивали его и рыдали о нем сильно все Израильтяне, и печалились много дней и говорили:
\vs 1Ma 9:21 как пал сильный, спасавший Израиля?
\vs 1Ma 9:22 Прочие же дела Иуды, и сражения, и мужественные подвиги, которые совершил он, и величие его не описаны, ибо их было весьма много.
\rsbpar\vs 1Ma 9:23 По смерти же Иуды во всех пределах Израильских явились люди беззаконные, и поднялись все делатели неправды.
\vs 1Ma 9:24 В те самые дни был очень сильный голод, и страна пристала к ним.
\vs 1Ma 9:25 И выбрал Вакхид нечестивых мужей и поставил их начальниками страны.
\vs 1Ma 9:26 Они разведывали и разыскивали друзей Иуды и приводили их к Вакхиду, а он мстил им и издевался над ними.
\vs 1Ma 9:27 И была великая скорбь в Израиле, какой не бывало с того дня, как не видно стало у них пророка.
\vs 1Ma 9:28 Тогда собрались все друзья Иуды и сказали Ионафану:
\vs 1Ma 9:29 с того времени, как скончался брат твой Иуда, нет подобного ему мужа, чтобы выйти против врагов и Вакхида и против ненавистников нашего народа.
\vs 1Ma 9:30 Итак, теперь мы тебя избрали~--- быть нам вместо него начальником и вождем, чтобы вести войну нашу.
\vs 1Ma 9:31 И принял Ионафан в то время предводительство и стал на место Иуды, брата своего.
\vs 1Ma 9:32 И узнал о том Вакхид и искал убить его.
\vs 1Ma 9:33 Об этом узнали Ионафан и Симон, брат его, и все бывшие с ним и убежали в пустыню Фекое и расположились станом при водах озера Асфар.
\vs 1Ma 9:34 Вакхид, узнав о том в день субботний, переправился сам и все войско его за Иордан.
\vs 1Ma 9:35 А Ионафан отправил брата своего~--- предводителя народа~--- и просил друзей своих, Наватеев, чтобы сложить у них большой запас свой.
\vs 1Ma 9:36 Но вышли из Мидавы сыны Иамври и схватили Иоанна и все, что он имел, и ушли.
\vs 1Ma 9:37 После сих происшествий сказали Ионафану и Симону, брату его, что сыны Иамври торжественно совершают знатный брак и провожают из Надавафа с великою пышностью невесту, дочь одного из знатных вельмож Хананейских.
\vs 1Ma 9:38 Тогда вспомнили они об Иоанне, брате своем, и вышли, и скрылись под кровом горы.
\vs 1Ma 9:39 Подняв глаза свои, они увидели: вот восклицания и большое приданое; навстречу вышел жених и друзья его и братья его с тимпанами и музыкою и со многими оружиями.
\vs 1Ma 9:40 Тогда бывшие с Ионафаном поднялись на них из засады и побили их, и много пало пораженных, а остальные убежали на гору; и взяли они всю добычу их.
\vs 1Ma 9:41 И обратилось брачное торжество в печаль, и звук музыки их~--- в плач.
\vs 1Ma 9:42 Так отмстили они за кровь брата своего и возвратились к болотистому месту у Иордана.
\vs 1Ma 9:43 И услышал об этом Вакхид~--- и в день субботний пришел к берегам Иордана с большим войском.
\vs 1Ma 9:44 Тогда сказал Ионафан бывшим с ним: встанем теперь и сразимся за жизнь нашу, ибо ныне~--- не то, что вчера и третьего дня.
\vs 1Ma 9:45 Вот, неприятель и спереди нас и сзади нас, вода Иордана с той и с другой стороны, и болото и лес, и нет места, куда уклониться.
\vs 1Ma 9:46 Итак, теперь воззовите на небо, чтобы избавиться вам от руки врагов ваших.
\vs 1Ma 9:47 И началось сражение. И простер Ионафан руку свою, чтобы поразить Вакхида, но тот уклонился от него назад.
\vs 1Ma 9:48 И бросился Ионафан и бывшие с ним в Иордан и переплыли на другой берег, а те не перешли за ними Иордана.
\vs 1Ma 9:49 И пало у Вакхида в тот день до тысячи мужей.
\vs 1Ma 9:50 И возвратился он в Иерусалим и построил в Иудее крепкие города: крепость в Иерихоне, и Еммаум и Вефорон, и Вефиль и Фамнафу в Фарафоне, и Тефон с высокими стенами, воротами и запорами,
\vs 1Ma 9:51 и поставил в них стражу, чтобы враждебно действовать против Израиля.
\vs 1Ma 9:52 Укрепил также город в Вефсуре и Газару и крепость и оставил в них войско со съестными запасами,
\vs 1Ma 9:53 и взял в заложники сыновей вождей страны и поместил их в Иерусалимской крепости под стражею.
\rsbpar\vs 1Ma 9:54 В сто пятьдесят третьем году, во втором месяце, Алким велел разорить стену внутреннего двора храма и разрушить дело пророков, и уже начал разрушение.
\vs 1Ma 9:55 Но в то самое время Алким поражен был ударом, и остановились предприятия его; уста его сомкнулись, он онемел и не мог более вымолвить ни одного слова и завещать о доме своем.
\vs 1Ma 9:56 И умер Алким в то же время в тяжких мучениях.
\vs 1Ma 9:57 Когда Вакхид узнал, что Алким умер, возвратился к царю; и земля Иудейская два года оставалась в покое.
\vs 1Ma 9:58 Тогда все беззаконники совещались и говорили: вот, Ионафан и находящиеся с ним живут безопасно в покое; приведем теперь Вакхида, и он схватит всех их в одну ночь.
\vs 1Ma 9:59 Пошли и предложили ему такой совет.
\vs 1Ma 9:60 Он решился идти с большим войском и послал тайно письма всем союзникам своим, которые находились в Иудее, чтобы они схватили Ионафана и находящихся с ним, но они не могли, потому что замысел их сделался известен им.
\vs 1Ma 9:61 И поймали они из мужей страны виновников этого злодейства до пятидесяти человек и убили их.
\vs 1Ma 9:62 После сего удалились Ионафан и Симон и бывшие с ними в Вефваси, что в пустыне, и возобновили разрушенное там и укрепили город.
\vs 1Ma 9:63 Узнав об этом, Вакхид собрал все войско свое, известив и тех, которые находились в Иудее,
\vs 1Ma 9:64 пришел и осадил Вефваси, и сражался против него много дней и устроил машины.
\vs 1Ma 9:65 Ионафан же оставил в городе Симона, брата своего, а сам вышел в страну, и вышел с небольшим числом,
\vs 1Ma 9:66 и поразил Одоааррина и братьев его и сыновей Фасирона в шатрах их и начал поражать и наступать с силою.
\vs 1Ma 9:67 Тогда и Симон и бывшие с ним выступили из города и сожгли машины,
\vs 1Ma 9:68 и сражались против Вакхида, и он был разбит ими; этим они сильно опечалили его, потому что замысел его и поход остался тщетным.
\vs 1Ma 9:69 Сильно разгневался он на мужей беззаконных, которые присоветовали ему идти в эту страну, и многих из них умертвил, и решился возвратиться в землю свою.
\vs 1Ma 9:70 Узнав об этом, Ионафан послал к нему старейшин, чтобы заключить с ним мир и чтобы он отдал пленных.
\vs 1Ma 9:71 Он принял это и сделал по словам его, и поклялся не причинять ему никакого зла во все дни жизни своей,
\vs 1Ma 9:72 и отдал ему пленных, которых прежде взял в плен в земле Иудейской, и возвратился в землю свою и не приходил более в пределы их.
\vs 1Ma 9:73 И унялся меч в Израиле, и поселился Ионафан в Махмасе; и начал Ионафан судить народ и истребил нечестивых из среды Израиля.
\vs 1Ma 10:1 В сто шестидесятом году выступил Александр, сын Антиоха Епифана, и овладел Птолемаидою: и приняли его, и он воцарился там.
\vs 1Ma 10:2 Когда услышал о том царь Димитрий, собрал весьма многочисленное войско и вышел против него на войну.
\vs 1Ma 10:3 И послал Димитрий письма Ионафану с мирным предложением, как бы желая возвеличить его,
\vs 1Ma 10:4 ибо говорил: предупредим заключить с ним мир, прежде нежели он заключит с Александром против нас:
\vs 1Ma 10:5 тогда он припомнит все зло, которое мы сделали против него и братьев его и народа его.
\vs 1Ma 10:6 И он дал ему власть набирать войско и приготовлять оружия, чтобы быть союзником его, и велел отдать ему заложников, которые находились в крепости.
\vs 1Ma 10:7 Ионафан пришел в Иерусалим и прочитал письма вслух всего народа и бывших в крепости;
\vs 1Ma 10:8 и убоялись все великим страхом, услышав, что царь дал ему власть набирать войско;
\vs 1Ma 10:9 а бывшие в крепости выдали Ионафану заложников, и он возвратил их родителям их.
\vs 1Ma 10:10 И жил Ионафан в Иерусалиме; и начал строить и возобновлять город,
\vs 1Ma 10:11 и сказал производившим работы, чтобы они строили стены и вокруг горы Сиона для твердости из четырехугольных камней,~--- и делали так.
\vs 1Ma 10:12 Тогда иноплеменные, бывшие в крепостях, построенных Вакхидом, бежали:
\vs 1Ma 10:13 каждый оставил свое место и ушел в свою землю.
\vs 1Ma 10:14 Только в Вефсуре остались некоторые из тех, которые оставили закон и заповеди, ибо это место служило для них убежищем.
\vs 1Ma 10:15 И услышал царь Александр о тех обещаниях, какие Димитрий послал Ионафану, и рассказали ему о войнах и храбрых подвигах, которые совершил Ионафан и братья его, и о трудностях, понесенных ими.
\vs 1Ma 10:16 Тогда он сказал: найдем ли мы еще такого мужа, как этот? Сделаем же его нашим другом и союзником.
\vs 1Ma 10:17 И написал и послал ему письмо в таких словах:
\vs 1Ma 10:18 <<Царь Александр брату Ионафану~--- радоваться.
\vs 1Ma 10:19 Услышали мы о тебе, что ты~--- муж, крепкий силою и достойный быть нашим другом.
\vs 1Ma 10:20 Итак, мы поставляем тебя ныне первосвященником народа твоего; и ты будешь именоваться другом царя (он послал ему порфиру и золотой венец) и будешь держать нашу сторону и хранить дружбу с нами>>.
\vs 1Ma 10:21 И облекся Ионафан в священную одежду в седьмом месяце сто шестидесятого года, в праздник кущей, и собрал войско и заготовил множество оружий.
\vs 1Ma 10:22 И услышал об этом Димитрий и огорчился, и сказал:
\vs 1Ma 10:23 что это мы сделали, что Александр предупредил нас заключить дружбу с Иудеями в подкрепление себе?
\vs 1Ma 10:24 Напишу и я им слова приветствия, восхваления и обещаний, чтобы были они в помощь мне.
\vs 1Ma 10:25 И послал им письмо в таких словах: <<Царь Димитрий народу Иудейскому~--- радоваться.
\vs 1Ma 10:26 Слышали мы и радовались, что вы сохраняете договоры наши, пребываете в дружбе с нами и не склоняетесь к врагам нашим.
\vs 1Ma 10:27 Продолжайте и ныне сохранять верность к нам, и мы воздадим вам добром за то, что вы делаете для нас:
\vs 1Ma 10:28 сделаем вам многие уступки и дадим вам дары.
\vs 1Ma 10:29 Ныне же разрешаю вас и освобождаю всех Иудеев от податей и пошлины с соли и с венцов;
\vs 1Ma 10:30 и за третью часть семян и половинную часть древесных плодов, принадлежащую мне, отныне и впредь я отменяю брать с земли Иудейской и с трех областей, присоединенных к ней от Самарии и Галилеи, от нынешнего дня и на вечные времена.
\vs 1Ma 10:31 И Иерусалим да будет священным и свободным и пределы его, десятины и доходы его.
\vs 1Ma 10:32 Предоставляю и власть над крепостью Иерусалимскою и даю право первосвященнику поставить в ней людей, каких он сам изберет, для охранения ее;
\vs 1Ma 10:33 и всякого человека из Иудеев, взятого в плен из земли Иудейской, во всем царстве моем отпускаю на свободу даром: пусть все будут свободны от повинностей за себя и за скот свой.
\vs 1Ma 10:34 Все праздники и субботы и новомесячия, и дни установленные~--- три дня пред праздником и три дня после праздника,~--- все эти дни пусть будут днями льготы и свободы всем Иудеям, находящимся в моем царстве.
\vs 1Ma 10:35 Никто не будет иметь права притеснять и отягощать кого-нибудь из них ни по какому делу.
\vs 1Ma 10:36 И пусть из Иудеев записываются в царские войска до тридцати тысяч человек,~--- и им будет даваться жалованье наравне со всеми войсками царскими.
\vs 1Ma 10:37 И из них да будут поставляемы начальствующими над большими крепостями царскими, из них же да будут поставляемы и над делами царства, требующими верности, и их приставники и начальники да будут из них же, и пусть они живут по своим законам, как повелел царь в земле Иудейской.
\vs 1Ma 10:38 И три области, присоединенные к Иудее от страны Самарийской, пусть останутся присоединенными к Иудее, чтобы считаться и быть им за одну и не подлежать другой власти, кроме власти первосвященника.
\vs 1Ma 10:39 Птолемаиду с округом ее я отдаю в дар святилищу в Иерусалиме на издержки, потребные для святилища;
\vs 1Ma 10:40 я же даю ежегодно пятнадцать тысяч сиклей серебра из царских сборов с подлежащих мест.
\vs 1Ma 10:41 И все остальное, чего не отдали заведующие сборами, как в прежние годы, отныне будут отдавать на работы храма.
\vs 1Ma 10:42 Сверх того пять тысяч сиклей серебра, которые брали от доходов святилища из ежегодного сбора, и те уступаются, как принадлежащие служащим священникам.
\vs 1Ma 10:43 И все, которые убегут в храм Иерусалимский и во все пределы его по причине повинностей царских и всех других, пусть будут свободны со всем, что принадлежит им в царстве моем.
\vs 1Ma 10:44 И на строение и возобновление святилища издержки будут выдаваемы из сборов царских.
\vs 1Ma 10:45 И на построение стен Иерусалима и укрепление их вокруг издержки будут выдаваемы из доходов царских, а также на построение стен в Иудее>>.
\vs 1Ma 10:46 Ионафан и народ, выслушав эти слова, не поверили им и не приняли их, ибо вспомнили о тех великих бедствиях, которые нанес Димитрий Израильтянам, жестоко притеснив их,
\vs 1Ma 10:47 и предпочли союз с Александром, ибо он первый сделал им мирные предложения,~--- и помогали ему в войнах во все дни.
\rsbpar\vs 1Ma 10:48 Царь Александр собрал большое войско и ополчился против Димитрия.
\vs 1Ma 10:49 И вступили два царя в сражение, и войско Димитрия обратилось в бегство; Александр преследовал его, и превозмог,
\vs 1Ma 10:50 и весьма настойчиво продолжал сражение до самого захождения солнца,~--- и пал Димитрий в этот день.
\rsbpar\vs 1Ma 10:51 После того Александр отправил послов к Птоломею, царю Египетскому, с такими словами:
\vs 1Ma 10:52 <<Я возвратился в землю царства моего и воссел на престоле отцов моих, принял верховную власть, сокрушил Димитрия и стал обладателем страны нашей.
\vs 1Ma 10:53 Я вступил с ним в сражение, и он разбит нами и войско его, и воссели мы на престоле царства его.
\vs 1Ma 10:54 Итак, заключим теперь дружбу между нами, и ты дай мне дочь твою в жену, и буду я тебе зятем и дам тебе и ей дары, достойные тебя>>.
\vs 1Ma 10:55 И отвечал царь Птоломей так: <<Счастлив день, в который ты возвратился в землю отцов твоих и воссел на престоле царства их.
\vs 1Ma 10:56 Ныне я исполню для тебя то, о чем ты писал, только ты выйди ко мне в Птолемаиду, чтобы нам видеть друг друга, и я породнюсь с тобою, как ты сказал>>.
\vs 1Ma 10:57 И отправился Птоломей из Египта сам и Клеопатра, дочь его, и прибыли в Птолемаиду в сто шестьдесят втором году.
\vs 1Ma 10:58 Царь Александр встретил его, и он выдал за него Клеопатру, дочь свою, и устроил брак ее в Птолемаиде, как прилично царям, с великою пышностью.
\vs 1Ma 10:59 Писал также царь Александр Ионафану, чтобы он вышел к нему навстречу.
\vs 1Ma 10:60 И отправился Ионафан в Птолемаиду с пышностью, и представлялся обоим царям и одарил их и приближенных их серебром и золотом и многими дарами, и приобрел благоволение их.
\vs 1Ma 10:61 И собрались против него мужи зловредные из среды Израиля, мужи беззаконные, чтобы оклеветать его; но царь не внял им.
\vs 1Ma 10:62 И повелел царь снять с Ионафана одежды его и облечь его в порфиру,~--- и сделали так.
\vs 1Ma 10:63 И посадил его царь с собою и сказал своим правителям: выйдите с ним на средину города и провозгласите, чтобы никто не смел клеветать на него ни в каком деле и никто не тревожил его никаким делом.
\vs 1Ma 10:64 Когда клеветавшие увидели славу его, как он был провозглашаем и как облечен в порфиру, все разбежались.
\vs 1Ma 10:65 Так прославил его царь и вписал его в число первых друзей, и назначил его военачальником и областным правителем.
\vs 1Ma 10:66 И возвратился Ионафан в Иерусалим с миром и веселием.
\rsbpar\vs 1Ma 10:67 Но в сто шестьдесят пятом году пришел из Крита Димитрий, сын Димитрия, в землю отцов своих.
\vs 1Ma 10:68 Услышав о том, царь Александр весьма огорчился и возвратился в Антиохию.
\vs 1Ma 10:69 И поставил Димитрий военачальником Аполлония, правителя Келе-Сирии,~--- и он собрал большое войско и расположился станом при Иамнии и послал к первосвященнику Ионафану сказать:
\vs 1Ma 10:70 ты только один превозносишься над нами, я же подвергся осмеянию и посрамлению через тебя. Зачем ты противостоишь нам в горах?
\vs 1Ma 10:71 Если ты надеешься на твои военные силы, то сойди к нам на равнину, и там мы померяемся, ибо со мною войско городов.
\vs 1Ma 10:72 Спроси и узнай, кто я и прочие помогающие нам, и скажут тебе: невозможно вам устоять пред лицем нашим, ибо дважды обращены были в бегство отцы твои в земле своей.
\vs 1Ma 10:73 И ныне ты не можешь устоять против такой конницы и такого войска на равнине, где нет ни камней, ни ущелий, ни места для убежища.
\vs 1Ma 10:74 Когда Ионафан выслушал эти слова Аполлония, то подвигся духом и, избрав десять тысяч мужей, вышел из Иерусалима, и брат его Симон сошелся с ним на помощь ему.
\vs 1Ma 10:75 И расположился станом при Иоппии; но не впустили его в город, ибо в Иоппии была стража Аполлония, и они начали воевать против нее.
\vs 1Ma 10:76 Тогда устрашенные жители отворили ему город, и Ионафан овладел Иоппиею.
\vs 1Ma 10:77 Услышав о сем, Аполлоний взял три тысячи конницы и большое войско и пошел в Азот, как бы делая переход, а между тем прошел на равнину, ибо имел множество конницы и надеялся на нее.
\vs 1Ma 10:78 Ионафан же преследовал его до Азота, и вступили войска в сражение.
\vs 1Ma 10:79 Между тем Аполлоний оставил тысячу всадников в скрытном месте позади них;
\vs 1Ma 10:80 но Ионафан узнал, что есть засада сзади него. И обступили войско его и бросали в народ стрелы с утра до вечера,
\vs 1Ma 10:81 народ же стоял, как приказал Ионафан; наконец всадники утомились.
\vs 1Ma 10:82 Тогда Симон подвел войско свое и напал на отряд, ибо всадники изнемогли,~--- и были разбиты им и обратились в бегство.
\vs 1Ma 10:83 И рассеялись всадники по равнине и убежали в Азот, и вошли в Бетдагон, капище их, чтобы спастись.
\vs 1Ma 10:84 Но Ионафан сжег Азот и окрестные города и взял добычу их, и капище Дагона с убежавшими в него сжег огнем.
\vs 1Ma 10:85 И было павших от меча с сожженными до восьми тысяч мужей.
\vs 1Ma 10:86 Отправившись оттуда, Ионафан расположился станом против Аскалона; но жители города вышли к нему навстречу с великою почестью.
\vs 1Ma 10:87 И возвратился Ионафан со всеми бывшими при нем в Иерусалим, имея при себе много добычи.
\vs 1Ma 10:88 Когда царь Александр услышал о сих событиях, то вновь почтил Ионафана
\vs 1Ma 10:89 и послал ему золотую пряжку, какая по обычаю давалась царским родственникам, и подарил ему Аккарон и всю область его в наследственное владение.
\vs 1Ma 11:1 Между тем царь Египетский, собрав многочисленное войско, как песок на берегу морском, и множество кораблей, домогался овладеть царством Александра хитростью и присоединить его к своему царству.
\vs 1Ma 11:2 Он пришел в Сирию с мирными речами, и жители отворяли ему города и выходили навстречу, ибо дано было от царя Александра повеление встречать его, потому что он был тесть его.
\vs 1Ma 11:3 Когда же Птоломей входил в города, то оставлял войско для стражи в каждом городе.
\vs 1Ma 11:4 Когда приблизился он к Азоту, то показали ему сожженное капище Дагона, и Азот и окрестные города разрушенные, и тела пораженные и сожженные во время сражения, ибо сложили их в груды по пути его,
\vs 1Ma 11:5 и рассказали царю о всем, что сделал Ионафан, жалуясь на него; но царь промолчал.
\vs 1Ma 11:6 Тогда вышел Ионафан навстречу царю в Иоппию с почетом, и приветствовали друг друга и ночевали там.
\vs 1Ma 11:7 И шел Ионафан с царем до реки, называемой Елевфера, и потом возвратился в Иерусалим.
\vs 1Ma 11:8 Царь же Птоломей овладел городами на морском берегу до Селевкии приморской и составлял злые замыслы против Александра.
\vs 1Ma 11:9 И послал послов к царю Димитрию, говоря: приди сюда, заключим между собою союз, и я дам тебе дочь мою, которую имеет Александр, и ты будешь царствовать в царстве отца твоего.
\vs 1Ma 11:10 Я раскаиваюсь, что отдал ему дочь мою, ибо он старался убить меня.
\vs 1Ma 11:11 Так клеветал он на него, потому что сам домогался царства его.
\vs 1Ma 11:12 И, отняв у него дочь свою, отдал ее Димитрию, и стал чужим для Александра, и обнаружилась вражда их.
\vs 1Ma 11:13 И вошел Птоломей в Антиохию и возложил на свою голову два венца~--- Азии и Египта.
\vs 1Ma 11:14 Царь Александр находился в то время в Киликии, потому что жители тех мест отпали от него.
\vs 1Ma 11:15 Услышав об этом, Александр пошел против него воевать; тогда Птоломей вывел войско и встретил его с крепкою силою, и обратил его в бегство.
\vs 1Ma 11:16 И убежал Александр в Аравию, чтобы укрыться там; царь же Птоломей возвысился.
\vs 1Ma 11:17 Завдиил, Аравитянин, снял голову с Александра и послал ее Птоломею.
\vs 1Ma 11:18 Царь же Птоломей на третий день умер, а оставшиеся в крепостях истреблены были жителями крепостей.
\vs 1Ma 11:19 И воцарился Димитрий в сто шестьдесят седьмом году.
\rsbpar\vs 1Ma 11:20 В те дни собрал Ионафан Иудеев, чтобы завоевать крепость Иерусалимскую, и устроил перед нею множество машин.
\vs 1Ma 11:21 Но некоторые ненавистники народа своего, отступники от закона, пошли к царю и донесли, что Ионафан облагает крепость.
\vs 1Ma 11:22 Когда он услышал об этом, разгневался и, поспешно собравшись, отправился в Птолемаиду, и написал Ионафану, чтобы он не облагал крепости, а как можно скорее шел к нему навстречу в Птолемаиду, чтобы переговорить с ним.
\vs 1Ma 11:23 Но Ионафан, выслушав это, приказал продолжать осаду и, избрав из старейшин Израильских и священников, решился подвергнуться опасности.
\vs 1Ma 11:24 Взяв серебра и золота, одежды и много других даров, он пошел к царю в Птолемаиду и приобрел благоволение его.
\vs 1Ma 11:25 И хотя некоторые отступники из того же народа клеветали на него,
\vs 1Ma 11:26 но царь поступил с ним так же, как поступали с ним предшественники его, и возвысил его пред всеми друзьями своими,
\vs 1Ma 11:27 и утвердил за ним первосвященство и другие почетные отличия, какие он имел прежде, и сделал его одним из первых друзей своих.
\vs 1Ma 11:28 И просил Ионафан царя освободить от податей Иудею и три области и Самарию и обещал ему триста талантов.
\vs 1Ma 11:29 Царь согласился и написал Ионафану обо всем этом письмо такого содержания:
\vs 1Ma 11:30 <<Царь Димитрий брату Ионафану и народу Иудейскому~--- радоваться.
\vs 1Ma 11:31 Список письма, которое мы писали о вас Ласфену, родственнику нашему, посылаем и к вам, чтобы вы знали.
\vs 1Ma 11:32 Царь Димитрий Ласфену-отцу~--- радоваться.
\vs 1Ma 11:33 Народу Иудейскому, друзьям нашим, верно исполняющим свои обязанности перед нами, мы рассудили оказать благодеяние за их доброе расположение к нам.
\vs 1Ma 11:34 Итак, мы утверждаем за ними как пределы Иудеи, так и три области: Аферему, Лидду и Рамафем, которые присоединены к Иудее от Самарии, и все, принадлежащее всем жрецам их в Иерусалиме, за те царские оброки, которые прежде ежегодно получал от них царь с произрастаний земли и с плодов древесных,
\vs 1Ma 11:35 и все прочее, принадлежащее нам отныне из десятин и даней, следующих нам, соленые озера и венечный сбор, нам принадлежащий, все вполне уступаем им.
\vs 1Ma 11:36 И ничего не будет отменено из сего отныне и навсегда.
\vs 1Ma 11:37 Итак, позаботьтесь сделать список с сего, и пусть будет отдан он Ионафану и положен на святой горе в известном месте>>.
\rsbpar\vs 1Ma 11:38 И увидел царь Димитрий, что преклонилась земля пред ним и ничто не противилось ему, и отпустил все войска свои, каждого в свое место, кроме войск чужеземных, которые он нанял с островов чужих народов, за что все войска отцов его ненавидели его.
\vs 1Ma 11:39 Трифон, один из прежних приверженцев Александра, видя, что все войска ропщут на Димитрия, отправился к Емалкую Аравитянину, который воспитывал Антиоха, малолетнего сына Александрова;
\vs 1Ma 11:40 и настаивал, чтобы он выдал его ему, дабы сделать его царем вместо него; и рассказал ему обо всем, что сделал Димитрий, и о неприязни, которую имеют к нему войска его, и пробыл там много дней.
\rsbpar\vs 1Ma 11:41 И послал Ионафан к царю Димитрию, чтобы он вывел оставленных им в Иерусалимской крепости и укреплениях, ибо они нападали на Израиля.
\vs 1Ma 11:42 Димитрий послал сказать Ионафану: не только это сделаю для тебя и для народа твоего, но и почту тебя и народ твой великою честью, как скоро буду иметь благоприятное время.
\vs 1Ma 11:43 Теперь же ты справедливо поступишь, если пришлешь мне людей на помощь в войне, ибо отложились от меня все войска мои.
\vs 1Ma 11:44 И послал к нему Ионафан в Антиохию три тысячи храбрых мужей, и пришли они к царю, и обрадовался царь прибытию их.
\vs 1Ma 11:45 Граждане же, собравшись на средину города до ста двадцати тысяч человек, хотели убить царя.
\vs 1Ma 11:46 Но царь убежал во дворец, а граждане заняли все улицы города и начали осаждать его.
\vs 1Ma 11:47 Тогда царь призвал на помощь Иудеев, и все они тотчас собрались к нему, и вдруг рассыпались по городу, и умертвили в тот день в городе до ста тысяч,
\vs 1Ma 11:48 и зажгли город, и взяли в тот день много добычи, и спасли царя.
\vs 1Ma 11:49 И увидели граждане, что Иудеи овладели городом, как хотели, и упали духом, и начали взывать к царю, умоляя и говоря:
\vs 1Ma 11:50 прости нас, и пусть Иудеи перестанут нападать на нас и на город.
\vs 1Ma 11:51 И сложили оружие и заключили мир. И прославились Иудеи перед царем и перед всеми в царстве его и возвратились в Иерусалим с большою добычею.
\vs 1Ma 11:52 И воссел царь Димитрий на престоле царства своего, и успокоилась земля пред ним.
\vs 1Ma 11:53 Но он солгал во всем, что обещал, и изменил Ионафану и не воздал за сделанное ему добро и сильно оскорбил его.
\vs 1Ma 11:54 После того возвратился Трифон и с ним Антиох, еще очень юный; он воцарился и возложил на себя венец.
\vs 1Ma 11:55 И собрались к нему все войска, которые распустил Димитрий, и начали воевать с ним, и он обратился в бегство, и был поражен.
\vs 1Ma 11:56 И взял Трифон слонов и овладел Антиохиею.
\vs 1Ma 11:57 И писал юный Антиох Ионафану, говоря: предоставляю тебе первосвященство и поставляю тебя над четырьмя областями, и ты будешь в числе друзей царских.
\vs 1Ma 11:58 И послал ему золотые сосуды и домашнюю утварь и дал ему право пить из золотых сосудов и носить порфиру и золотую пряжку,
\vs 1Ma 11:59 а Симона, брата его, поставил военачальником от области Тирской до пределов Египта.
\vs 1Ma 11:60 И выступил Ионафан в поход, и проходил по ту сторону реки \bibemph{(Иордана)} и по городам, и собрались к нему на помощь все Сирийские войска; и пришел он к Аскалону, и встретили его жители города с честью.
\vs 1Ma 11:61 Оттуда пошел он в Газу; но жители Газы заперлись; и осадил он город, и сжег огнем предместья его, и опустошил их.
\vs 1Ma 11:62 И упросили жители Газы Ионафана, и он примирился с ними, только взял в заложники сыновей начальников их и отослал их в Иерусалим, и прошел страну до Дамаска.
\rsbpar\vs 1Ma 11:63 И услышал Ионафан, что пришли в Кадис, в Галилее, военачальники Димитрия с многочисленным войском, чтобы удалить его от страны.
\vs 1Ma 11:64 Но он пошел навстречу им, брата же своего, Симона, оставил в стране.
\vs 1Ma 11:65 И расположил Симон стан свой при Вефсуре, и осаждал его многие дни, и запер его.
\vs 1Ma 11:66 И просили его о мире, и он согласился, но выгнал их оттуда, и овладел городом, и поставил в нем стражу.
\vs 1Ma 11:67 А Ионафан и войско его расположились станом при водах Геннисаретских и утром стали на равнине Насор.
\vs 1Ma 11:68 И вот, войско иноплеменников встретилось с ним на равнине, оставив против него засаду в горах, само же шло навстречу ему с противной стороны.
\vs 1Ma 11:69 И вышли бывшие в засаде из своих мест, и начали сражаться: тогда все бывшие с Ионафаном обратились в бегство,
\vs 1Ma 11:70 и ни одного из них не осталось, кроме Маттафии, сына Авессаломова, и Иуды, сына Халфиева, начальников воинских отрядов.
\vs 1Ma 11:71 И разодрал Ионафан одежды свои, и посыпал землю на голову свою, и молился.
\vs 1Ma 11:72 Потом возвратился сражаться с ними и поразил их, и они бежали.
\vs 1Ma 11:73 Увидев это, убежавшие от него возвратились к нему, и с ним преследовали их до Кадиса, до самого стана их, и там остановились.
\vs 1Ma 11:74 В тот день пало от иноплеменников до трех тысяч мужей; и возвратился Ионафан в Иерусалим.
\vs 1Ma 12:1 Ионафан, видя, что время благоприятствует ему, избрал мужей и послал в Рим установить и возобновить дружбу с Римлянами,
\vs 1Ma 12:2 и к Спартанцам и в другие места послал письма о том же.
\vs 1Ma 12:3 И пришли они в Рим, и вошли в совет, и сказали: <<Ионафан-первосвященник и народ Иудейский прислали нас, чтобы возобновить дружбу с вами и союз по-прежнему>>.
\vs 1Ma 12:4 И там дали им письма к местным начальникам, чтобы проводили их в землю Иудейскую с миром.
\vs 1Ma 12:5 Вот список письма, которое писал Ионафан Спартанцам:
\vs 1Ma 12:6 <<Первосвященник Ионафан и народные старейшины и священники и остальной народ Иудейский братьям Спартанцам~--- радоваться.
\vs 1Ma 12:7 Еще прежде от Дария [Арея], царствовавшего у вас, присланы были к первосвященнику Онии письма, что вы~--- братья наши, как показывает список.
\vs 1Ma 12:8 И принял Ония посланного мужа с честью, и получил письма, в которых ясно говорилось о союзе и дружбе.
\vs 1Ma 12:9 Мы же, хотя и не имеем надобности в них, имея утешением священные книги, которые в руках наших,
\vs 1Ma 12:10 но предприняли послать к вам для возобновления братства и дружбы, чтобы не отчуждаться от вас, ибо много прошло времени после того, как вы присылали к нам.
\vs 1Ma 12:11 Мы неопустительно во всякое время, как в праздники, так и в прочие установленные дни, воспоминаем о вас при жертвоприношениях наших и молитвах, как должно и прилично воспоминать братьев.
\vs 1Ma 12:12 Мы радуемся о вашей славе;
\vs 1Ma 12:13 нас же обстоят многие беды и частые войны; ибо воевали против нас окрестные цари.
\vs 1Ma 12:14 Но мы не хотели беспокоить вас и прочих союзников и друзей наших в этих войнах,
\vs 1Ma 12:15 ибо мы имеем помощь небесную, помогающую нам; мы избавились от врагов наших, и враги наши усмирены.
\vs 1Ma 12:16 Теперь мы избрали Нуминия, сына Антиохова, и Антипатра, сына Иасонова, и послали их к Римлянам возобновить дружбу с ними и прежний союз.
\vs 1Ma 12:17 Поручили им идти и к вам, приветствовать вас и вручить вам письма от нас о возобновлении и с вами нашего братства.
\vs 1Ma 12:18 И вы хорошо сделаете, ответив нам на них>>.
\rsbpar\vs 1Ma 12:19 Вот и список писем, которые прислал Дарий [Арей]:
\vs 1Ma 12:20 <<Царь Спартанский Онии первосвященнику~--- радоваться.
\vs 1Ma 12:21 Найдено в писании о Спартанцах и Иудеях, что они~--- братья и от рода Авраамова.
\vs 1Ma 12:22 Теперь, когда мы узнали об этом, вы хорошо сделаете, написав нам о благосостоянии вашем.
\vs 1Ma 12:23 Мы же уведомляем вас: скот ваш и имущество ваше~--- наши, а что у нас есть, то ваше. И мы повелели объявить вам о том>>.
\rsbpar\vs 1Ma 12:24 И услышал Ионафан, что возвратились военачальники Димитрия с б\acc{о}льшим войском, нежели прежде, чтобы воевать против него,
\vs 1Ma 12:25 и вышел из Иерусалима, и встретил их в стране Амафитской, и не дал им времени войти в страну его.
\vs 1Ma 12:26 И послал соглядатаев в стан их, которые, возвратившись, объявили ему, что они готовятся напасть на них в эту ночь.
\vs 1Ma 12:27 Посему, когда зашло солнце, Ионафан приказал своим бодрствовать, быть в вооружении и готовиться к сражению всю ночь, и поставил вокруг стана передовых сторожей.
\vs 1Ma 12:28 И услышали неприятели, что Ионафан со своими приготовился к сражению, и устрашились, и затрепетали сердцем своим, и, зажегши огни в стане своем, ушли.
\vs 1Ma 12:29 Ионафан же и бывшие с ним не знали о том до утра, ибо видели горящие огни.
\vs 1Ma 12:30 И погнался Ионафан за ними, но не настиг их, потому что они перешли реку Елевферу.
\vs 1Ma 12:31 Тогда Ионафан обратился на Арабов, называемых Заведеями, поразил их и взял добычу их.
\vs 1Ma 12:32 Потом, возвратившись, пришел в Дамаск и прошел по всей той стране.
\vs 1Ma 12:33 И Симон вышел, и прошел до Аскалона и ближайших крепостей, и обратился в Иоппию, и овладел ею
\vs 1Ma 12:34 ибо он услышал, что \bibemph{Иоппияне} хотят сдать крепость войскам Димитрия,~--- и поставил там стражу, чтобы охранять ее.
\vs 1Ma 12:35 И возвратился Ионафан, и созвал старейшин народа, и советовался с ними, чтобы построить крепости в Иудее,
\vs 1Ma 12:36 возвысить стены Иерусалима и воздвигнуть высокую стену между крепостью и городом, дабы отделить ее от города, так чтобы она была особо и не было бы в ней ни купли, ни продажи.
\vs 1Ma 12:37 Когда собрались устроить город и дошли до стены у потока с восточной стороны, то построили так называемую Хафенафу.
\vs 1Ma 12:38 А Симон построил Адиду в Сефиле и укрепил ворота и запоры.
\rsbpar\vs 1Ma 12:39 Между тем Трифон домогался сделаться царем Азии и возложить на себя венец и поднять руку на царя Антиоха,
\vs 1Ma 12:40 но опасался, как бы не воспрепятствовал ему Ионафан и не начал против него войны; поэтому искал случая, чтобы взять Ионафана и убить, и, поднявшись, пошел в Вефсан.
\vs 1Ma 12:41 И вышел Ионафан навстречу ему с сорока тысячами избранных мужей, готовых к битве, и пришел в Вефсан.
\vs 1Ma 12:42 Когда Трифон увидел, что Ионафан идет с многочисленным войском, то побоялся поднять на него руки.
\vs 1Ma 12:43 И принял его с честью, и представил его всем друзьям своим, дал ему подарки, приказал войскам своим повиноваться ему, как себе самому.
\vs 1Ma 12:44 Потом сказал Ионафану: для чего ты утруждаешь весь этот народ, когда не предстоит нам войны?
\vs 1Ma 12:45 Итак, отпусти их теперь в домы их, а для себя избери немногих мужей, которые были бы с тобою, и пойдем со мною в Птолемаиду, и я передам ее тебе и другие крепости и остальные войска и всех, заведующих сборами, и потом возвращусь; ибо для этого я и нахожусь здесь.
\vs 1Ma 12:46 И поверил ему Ионафан, и сделал так, как он сказал, и отпустил войска, и они отправились в землю Иудейскую;
\vs 1Ma 12:47 с собою же оставил три тысячи мужей, из которых две тысячи оставил в Галилее, тысяча же отправилась с ним.
\vs 1Ma 12:48 Но как скоро вошел Ионафан в Птолемаиду, Птолемаидяне заперли ворота, и схватили его, и всех вошедших с ним убили мечом.
\vs 1Ma 12:49 Тогда Трифон послал войско и конницу в Галилею и на великую равнину, чтобы истребить всех бывших с Ионафаном.
\vs 1Ma 12:50 Но они, услышав, что Ионафан схвачен и погиб и бывшие с ним, ободрили друг друга и вышли густым строем, готовые сразиться.
\vs 1Ma 12:51 И увидели преследующие, что дело идет о жизни, и возвратились назад.
\vs 1Ma 12:52 А они все благополучно пришли в землю Иудейскую и оплакивали Ионафана и бывших с ним, и были в большом страхе, и весь Израиль плакал горьким плачем.
\vs 1Ma 12:53 Тогда все окрестные народы искали истребить их, ибо говорили: теперь нет у них начальника и поборника; итак, будем теперь воевать против них и истребим из среды людей память их.
\vs 1Ma 13:1 Услышал Симон, что Трифон собрал большое войско, чтобы идти в землю Иудейскую и разорить ее.
\vs 1Ma 13:2 И, видя, что народ в страхе и трепете, взошел в Иерусалим и собрал народ.
\vs 1Ma 13:3 И, ободряя их, говорил им: сами вы знаете, сколько я и братья мои и дом отца моего сделали ради этих законов и святыни, знаете войны и угнетения, какие мы испытали.
\vs 1Ma 13:4 Потому и погибли все братья мои за Израиля, и остался я один.
\vs 1Ma 13:5 И ныне да не будет того, чтобы я стал щадить жизнь мою во все время угнетения, ибо я не лучше братьев моих.
\vs 1Ma 13:6 Но буду мстить за народ мой и за святилище, и за жен и за детей наших, ибо соединились все народы, чтобы истребить нас по неприязни.
\vs 1Ma 13:7 И воспламенился дух народа, как только услышал он такие слова;
\vs 1Ma 13:8 и отвечали громким голосом, и сказали: ты~--- наш вождь на место Иуды и Ионафана, брата твоего.
\vs 1Ma 13:9 Веди нашу войну, и, что ты ни скажешь нам, мы всё сделаем.
\vs 1Ma 13:10 Тогда собрал он всех мужей ратных, и поспешил окончить стены Иерусалима, и со всех сторон укрепил его.
\vs 1Ma 13:11 Потом послал Ионафана, сына Авессаломова, и с ним достаточное число войска в Иоппию, и он выгнал бывших в ней и остался там.
\rsbpar\vs 1Ma 13:12 Между тем Трифон поднялся из Птолемаиды с многочисленным войском, чтобы войти в землю Иудейскую; с ним был и Ионафан под стражею.
\vs 1Ma 13:13 Симон же расположил стан при Адиде напротив равнины.
\vs 1Ma 13:14 Когда Трифон узнал, что Симон заступил место Ионафана, брата своего, и намеревается вступить в сражение с ним, то послал к нему послов сказать:
\vs 1Ma 13:15 за серебро, которым брат твой Ионафан задолжал царской казне по надобностям, какие он имел, мы удержали его.
\vs 1Ma 13:16 Итак, пришли теперь сто талантов серебра и в заложники двух сыновей его, чтобы он, быв отпущен, не отложился от нас,~--- и мы отпустим его.
\vs 1Ma 13:17 Симон понимал, что они говорят с ним коварно, но послал серебро и детей, чтобы не навлечь большой ненависти от народа,
\vs 1Ma 13:18 который сказал бы: оттого, что я не послал ему серебра и детей, \bibemph{Ионафан} погиб.
\vs 1Ma 13:19 Итак, послал детей и сто талантов; но Трифон обманул и не отпустил Ионафана.
\vs 1Ma 13:20 После сего Трифон пошел, чтобы войти в страну и разорить ее, и пошел окольным путем на Адару. Но Симон и войско его следовали за ним повсюду, куда он ни шел.
\vs 1Ma 13:21 Бывшие же в крепости послали к Трифону послов, чтобы побудить его прийти к ним чрез пустыню и прислать им съестных припасов.
\vs 1Ma 13:22 И приготовил Трифон всю свою конницу, чтобы идти в ту же ночь, но был очень большой снег, и он не пошел по причине снега, а, поднявшись, отправился в Галаад.
\vs 1Ma 13:23 Когда же приблизился к Васкаме, умертвил Ионафана, и он погребен там.
\vs 1Ma 13:24 И возвратился Трифон и ушел в землю свою.
\vs 1Ma 13:25 Тогда Симон послал и взял кости Ионафана, брата своего, и похоронил их в Модине, городе отцов своих.
\vs 1Ma 13:26 И оплакивал его весь Израиль горьким плачем, и сокрушались о нем многие дни.
\vs 1Ma 13:27 И воздвиг Симон здание над гробом отца своего и братьев своих и вывел его высоко, для благовидности, из тесаного камня с передней и задней стороны,
\vs 1Ma 13:28 и поставил на нем семь пирамид, одну против другой, отцу и матери и четырем братьям;
\vs 1Ma 13:29 сделал на них искусные украшения, поставив вокруг высокие столбы, а на столбах полное вооружение~--- на вечную память, и подле оружий~--- изваянные корабли, так что они были видимы всеми, плавающими по морю.
\vs 1Ma 13:30 Этот надгробный памятник, который сделал он в Модине, стоит до сего дня.
\rsbpar\vs 1Ma 13:31 Трифон же с коварством отправился в путь с юным царем Антиохом и убил его,
\vs 1Ma 13:32 и воцарился вместо него, и возложил на себя венец Азии, и произвел великое поражение на земле.
\vs 1Ma 13:33 А Симон строил крепости в Иудее, укрепляя их высокими башнями и большими стенами, воротами и запорами, и складывал в крепостях съестные запасы.
\vs 1Ma 13:34 Потом избрал Симон мужей и послал к царю Димитрию просить, чтобы он сделал облегчение стране, ибо все деяния Трифона были грабительские.
\vs 1Ma 13:35 И послал ему царь Димитрий ответ на эти слова и написал такое письмо:
\vs 1Ma 13:36 <<Царь Димитрий Симону, первосвященнику и другу царей, и старейшинам и народу Иудейскому~--- радоваться.
\vs 1Ma 13:37 Золотой венец и пальмовую ветвь, посланную вами, мы получили и готовы заключить с вами полный мир и написать заведующим сборами, чтобы отпустить вам дани.
\vs 1Ma 13:38 И всё, что мы постановили о вас, да будет неизменно, и крепости, которые вы построили, пусть принадлежат вам.
\vs 1Ma 13:39 Прощаем вам также неумышленные проступки ваши до сего дня и венечный сбор, который платить вы обязаны, и если другое что взимаемо было в Иерусалиме, более не будет взиматься.
\vs 1Ma 13:40 И если найдутся из вас способные быть вписанными в число состоящих при нас, пусть записываются, и да будет между нами мир>>.
\rsbpar\vs 1Ma 13:41 В сто семидесятом году снято иго язычников с Израиля;
\vs 1Ma 13:42 и народ Израильский в переписке и договорах начал писать: <<Первого года при Симоне, великом первосвященнике, вожде и правителе Иудеев>>.
\vs 1Ma 13:43 В это время Симон сделал нападение на Газу, окружил ее войском, устроил осадные машины и придвинул их к городу, разбил одну башню и овладел ею.
\vs 1Ma 13:44 А бывшие на машине вскочили в город, и произошло в городе великое смятение.
\vs 1Ma 13:45 И взошли граждане с женами и детьми на стену, разодрав одежды свои, и громко взывали, умоляя Симона дать им помилование,
\vs 1Ma 13:46 и говорили: поступи с нами не по злым делам нашим, но по милости твоей.
\vs 1Ma 13:47 И умилосердился над ними Симон, и не сражался с ними, а только выгнал их из города, и очистил домы, в которых находились идолы, и так вошел в город с славословиями и благословениями.
\vs 1Ma 13:48 И выбросил из него все нечистое, и поселил там мужей, соблюдающих закон, и укрепил его, и устроил в нем для себя жилище.
\vs 1Ma 13:49 Бывшим же в Иерусалимской крепости не позволяли ни выходить, ни вступать в страну, ни покупать, ни продавать, и они терпели сильный голод, и многие из них погибли от голода.
\vs 1Ma 13:50 Тогда воззвали они к Симону о мире, и он дал им его, но выгнал их оттуда и очистил крепость от осквернения,
\vs 1Ma 13:51 и взошел в нее в двадцать третий день второго месяца сто семьдесят первого года с славословиями, пальмовыми ветвями, с гуслями, кимвалами и цитрами, с псалмами и песнями, ибо сокрушен великий враг Израиля.
\vs 1Ma 13:52 И установил каждогодно проводить этот день с весельем, и укрепил гору храма, находящуюся близ крепости, и поселился там сам и бывшие с ним.
\vs 1Ma 13:53 И увидел Симон, что сын его Иоанн возмужал, и поставил его начальником над всеми войсками, и поселился в Газаре.
\vs 1Ma 14:1 В сто семьдесят втором году царь Димитрий собрал войска свои и отправился в Мидию, чтобы получить помощь себе для войны против Трифона.
\vs 1Ma 14:2 Но Арсак, царь Персидский и Мидийский, услышав, что Димитрий пришел в пределы его, послал одного из военачальников своих взять его живого.
\vs 1Ma 14:3 Тот отправился и разбил войско Димитрия, взял его и привел к Арсаку, который заключил его в темницу.
\rsbpar\vs 1Ma 14:4 И покоилась земля Иудейская во все дни Симона; он старался о благе народа своего, и нравилась им власть и слава его во все дни.
\vs 1Ma 14:5 И ко всей своей славе, он взял еще Иоппию для пристани и открыл вход островам морским,
\vs 1Ma 14:6 и распространил пределы народа своего, и овладел тою страною.
\vs 1Ma 14:7 Он набрал множество пленных и господствовал над Газарою и Вефсурою и над крепостью, очистил ее от осквернения, и не было противящегося ему.
\vs 1Ma 14:8 \bibemph{Иудеи} спокойно возделывали землю свою, и земля давала произведения свои и дерева в полях~--- плод свой.
\vs 1Ma 14:9 Старцы, сидя на улицах, все совещались о пользах общественных, и юноши облекались в пышные и воинские одежды.
\vs 1Ma 14:10 Городам доставлял он съестные припасы и делал их местами укрепленными, так что славное имя его произносилось до конца земли.
\vs 1Ma 14:11 Он восстановил мир в стране, и радовался Израиль великою радостью.
\vs 1Ma 14:12 И сидел каждый под виноградом своим и под смоковницею своею, и никто не страшил их.
\vs 1Ma 14:13 И не осталось никого на земле, кто воевал бы против них, и цари смирились в те дни.
\vs 1Ma 14:14 Он подкреплял всех бедных в народе своем, требовал исполнения закона и истреблял всякого беззаконника и злодея,
\vs 1Ma 14:15 украсил святилище и умножил священную утварь.
\rsbpar\vs 1Ma 14:16 Когда дошел слух до Рима и до Спарты, что Ионафан умер, они весьма опечалились.
\vs 1Ma 14:17 Когда же услышали, что Симон, брат его, сделался вместо него первосвященником и господствует над страною и находящимися в ней городами,
\vs 1Ma 14:18 то написали к нему на медных досках, чтобы возобновить с ним дружбу и союз, заключенный ими с братьями его Иудою и Ионафаном.
\vs 1Ma 14:19 Они были прочитаны в Иерусалиме пред собранием.
\vs 1Ma 14:20 Вот список с писем, присланных Спартанцами: <<Спартанские начальники и город Симону первосвященнику, старейшинам и священникам и всему народу Иудейскому, братьям нашим~--- радоваться.
\vs 1Ma 14:21 Послы, присланные к народу нашему, рассказали нам о вашей славе и чести, и мы возрадовались прибытию их
\vs 1Ma 14:22 и записали сказанное ими в народном совете так: Нуминий, сын Антиоха, и Антипатр, сын Иасона, послы Иудейские, пришли к нам возобновить с нами дружбу.
\vs 1Ma 14:23 И угодно было народу принять этих мужей с честью и внести запись слов их в открытые народные книги, на память народу Спартанскому. А список с этого мы написали для первосвященника Симона>>.
\rsbpar\vs 1Ma 14:24 После того Симон послал Нуминия в Рим с большим золотым щитом, весом в тысячу мин, чтобы заключить с ними союз.
\vs 1Ma 14:25 Когда услышал об этом народ, то сказал: какую благодарность воздадим мы Симону и сыновьям его?
\vs 1Ma 14:26 Ибо он твердо стоял и братья его и дом отца его, и отразили врагов Израиля, и доставили ему свободу.
\vs 1Ma 14:27 И написали о том на медных досках и выставили их на столбах на горе Сион. Вот список написанного: <<В восемнадцатый день Елула сто семьдесят второго года~--- это был третий год при первосвященнике Симоне~---
\vs 1Ma 14:28 в Сарамели, в великом собрании священников и народа и князей народных и старейшин страны, объявлено нам:
\vs 1Ma 14:29 так как много раз бывали войны в этой стране, то Симон, сын Маттафии, сын сынов Иарива, и братья его, подвергая себя опасности, противостали врагам народа своего, чтобы сохранить святилище его и закон, и великою славою прославили народ свой.
\vs 1Ma 14:30 Ионафан собрал народ свой и сделался первосвященником его, но он приложился к народу своему.
\rsbpar\vs 1Ma 14:31 Когда же враги их вознамерились войти в страну их, чтобы разорить страну их и простереть руки на святилище их,
\vs 1Ma 14:32 тогда восстал Симон и воевал за народ свой и издержал много собственных денег, снабжая храбрых мужей народа своего оружием и давая им жалованье.
\vs 1Ma 14:33 Он укрепил города Иудеи и Вефсуру на границах Иудеи, где прежде находились оружия неприятелей, и поставил там стражу из Иудеев.
\vs 1Ma 14:34 Также укрепил Иоппию при море и Газару на пределах Азота, в которой прежде обитали враги, и поселил там Иудеев, снабдив эти \bibemph{места} всем, что нужно было к восстановлению их.
\vs 1Ma 14:35 И видел народ деяния Симона и славу, какую старался он доставить народу своему, и поставил его своим начальником и первосвященником за то, что все это сделал он, и за справедливость и верность, которую он хранил к племени своему, всячески стараясь возвысить народ свой.
\vs 1Ma 14:36 Во дни его руками его успешно изгнаны из страны язычники и занимавшие город Давидов в Иерусалиме, которые, устроив себе крепость, выходили из нее и оскверняли все вокруг святилища и много вредили святыне.
\vs 1Ma 14:37 Он поселил в ней Иудеев и укрепил ее для безопасности страны и города и возвысил стены Иерусалима.
\vs 1Ma 14:38 Посему и царь Димитрий утвердил за ним первосвященство,
\vs 1Ma 14:39 и причислил его к друзьям своим, и почтил его великою славою.
\vs 1Ma 14:40 Ибо он услышал, что Римляне назвали Иудеев друзьями и союзниками и братьями и с честью приняли послов Симона,
\vs 1Ma 14:41 что Иудеи и священники согласились, чтобы Симон был у них начальником и первосвященником навек, доколе восстанет Пророк верный,
\vs 1Ma 14:42 чтобы он был у них военачальником и имел попечение о святых и поставлял их над работами их, и над областью, и над оружиями, и над крепостями,
\vs 1Ma 14:43 чтобы имел попечение о святилище и все слушались его, чтобы все договоры в стране писались на его имя и чтобы он одевался в порфиру и носил золотые украшения.
\vs 1Ma 14:44 И никому из народа и священников да не будет позволено отменить что-либо из сего или противоречить словам его, или без него созывать собрание в стране и одеваться в порфиру и носить золотую пряжку.
\vs 1Ma 14:45 А кто сделает что-нибудь против сего или отменит что из сего, будет повинен>>.
\vs 1Ma 14:46 И согласился весь народ подчиниться Симону и поступать по словам сим.
\vs 1Ma 14:47 Симон принял и согласился быть первосвященником и военачальником и правителем Иудеев и священников и начальствовать над всеми.
\vs 1Ma 14:48 И решили начертать запись сию на медных досках и поставить их в ограде храма на видном месте,
\vs 1Ma 14:49 а списки с них положить в сокровищнице, чтобы имел их Симон и сыновья его.
\vs 1Ma 15:1 И прислал Антиох, сын царя Димитрия, письма с островов морских к Симону, великому священнику и правителю народа Иудейского, и всему народу.
\vs 1Ma 15:2 Они были такого содержания: <<Царь Антиох Симону, первосвященнику и правителю народа, и народу Иудейскому~--- радоваться.
\vs 1Ma 15:3 Так как люди зловредные овладели царством отцов наших, то я хочу возвратить царство, чтобы восстановить его, как оно было прежде. Я набрал множество войска и приготовил военные корабли;
\vs 1Ma 15:4 и хочу пройти по области, чтобы наказать тех, которые опустошили область нашу и разорили многие города в царстве.
\vs 1Ma 15:5 Оставляю теперь за тобою все дани, какие уступали тебе цари, бывшие прежде меня, и другие дары, какие они уступали тебе;
\vs 1Ma 15:6 дозволяю тебе чеканить свою монету в стране твоей.
\vs 1Ma 15:7 Иерусалим и святилище пусть будут свободны; и все оружия, которые ты заготовил, и крепости, построенные тобою, которыми ты владеешь, пусть остаются у тебя.
\vs 1Ma 15:8 И всякий долг царский и будущие царские долги отныне и навсегда пусть будут отпущены тебе.
\vs 1Ma 15:9 Когда же мы овладеем царством нашим, тогда почтим тебя и народ твой и храм великою честью, чтобы слава ваша стала известна по всей земле>>.
\rsbpar\vs 1Ma 15:10 В сто семьдесят четвертом году вступил Антиох в землю отцов своих, и собрались к нему все войска, так что оставшихся с Трифоном было немного.
\vs 1Ma 15:11 И преследовал его царь Антиох, и он убежал в Дору, которая при море;
\vs 1Ma 15:12 ибо он увидел, что обрушились на него беды и оставили его войска.
\vs 1Ma 15:13 И пришел Антиох к Доре и с ним сто двадцать тысяч воинов и восемь тысяч конницы
\vs 1Ma 15:14 и окружил город, а корабли подошли с моря, и теснил он город с суши и моря, и не давал никому ни выйти, ни войти.
\rsbpar\vs 1Ma 15:15 Тогда пришел из Рима Нуминий и сопровождавшие его с письмами к царям и странам, в которых было написано следующее:
\vs 1Ma 15:16 <<Левкий, консул Римский, царю Птоломею~--- радоваться.
\vs 1Ma 15:17 Пришли к нам Иудейские послы, друзья наши и союзники, посланные от первосвященника Симона и народа Иудейского, возобновить давнюю дружбу и союз,
\vs 1Ma 15:18 и принесли золотой щит в тысячу мин.
\vs 1Ma 15:19 Итак, мы заблагорассудили написать царям и странам, чтобы они не причиняли им зла, и не воевали против них и городов их и страны их, и не помогали воюющим против них.
\vs 1Ma 15:20 Мы рассудили принять от них щит.
\vs 1Ma 15:21 Итак, если какие зловредные люди убежали к вам из страны их, выдайте их первосвященнику Симону, чтобы он наказал их по закону их>>.
\vs 1Ma 15:22 То же самое написал он царю Димитрию и Атталу, Ариарафе и Арсаку,
\vs 1Ma 15:23 и во все области, и Сампсаме и Спартанцам, и в Делос и в Минд, и в Сикион, и в Карию, и в Самос, и в Памфилию, и в Ликию, и в Галикарнасс, и в Родос, и в Фасилиду, и в Кос, и в Сиду, и в Арад, и в Гортину, и в Книду, и в Кипр, и в Киринию.
\vs 1Ma 15:24 Список с этих писем написали Симону первосвященнику.
\rsbpar\vs 1Ma 15:25 Царь же Антиох обложил Дору вторично, нападая на нее со всех сторон и устраивая машины, и запер Трифона так, что невозможно было ему ни войти, ни выйти.
\vs 1Ma 15:26 И послал к нему Симон две тысячи избранных мужей в помощь ему, и серебро и золото, и довольно запасов;
\vs 1Ma 15:27 но он не захотел принять это и отверг все, в чем прежде условился с ним, и отчуждился от него.
\vs 1Ma 15:28 И послал к нему Афиновия, одного из друзей своих, чтобы переговорить с ним и сказать: <<Вы владеете Иоппиею и Газарою и крепостью Иерусалимскою~--- городами царства моего;
\vs 1Ma 15:29 вы опустошили пределы их и произвели великое поражение на земле, и овладели многими местами в царстве моем.
\vs 1Ma 15:30 Итак, отдайте теперь города, которые вы взяли, и дани с тех мест, которыми вы владеете вне пределов Иудейских.
\vs 1Ma 15:31 Если же не так, то дайте за них пятьсот талантов серебра, и за опустошение, которое произвели, и за дани с городов другие пятьсот талантов; а если не дадите, то мы придем и будем сражаться с вами>>.
\vs 1Ma 15:32 И пришел Афиновий, друг царя, в Иерусалим, и когда увидел славу Симона и сокровищницу с золотою и серебряною утварью и окружающее великолепие, то изумился и объявил ему слова царя.
\vs 1Ma 15:33 Симон сказал ему в ответ: мы ни чужой земли не брали, ни господствовали над чужим, но \bibemph{владеем} наследием отцов наших, которое враги наши в одно время неправедно присвоили себе.
\vs 1Ma 15:34 Мы же, улучив время, опять возвратили себе наследие отцов наших.
\vs 1Ma 15:35 Что касается до Иоппии и Газары, которых ты требуешь, то они сами причинили много зла народу в стране нашей; за них мы дадим сто талантов. На это Афиновий ничего не отвечал;
\vs 1Ma 15:36 но, с досадою возвратившись к царю, рассказал ему эти слова и о славе Симона, и о всем, что видел, и царь сильно разгневался.
\vs 1Ma 15:37 Трифон же, сев на корабль, убежал в Орфосиаду.
\vs 1Ma 15:38 Тогда царь, сделав военачальником приморской страны Кендевея, вручил ему пешие и конные войска
\vs 1Ma 15:39 и приказал ему идти войною против Иудеи, приказал ему также построить Кедрон и укрепить ворота, и как воевать с народом; сам же царь погнался за Трифоном.
\vs 1Ma 15:40 И пришел Кендевей в Иамнию, и начал вызывать на бой народ и вторгаться в Иудею и брать народ в плен и убивать;
\vs 1Ma 15:41 и построил Кедрон, и расположил там конницу и войско, чтобы они, выходя оттуда, обходили пути Иудеи, как приказал ему царь.
\vs 1Ma 16:1 И возвратился Иоанн из Газары и рассказал Симону, отцу своему, о том, что делал Кендевей.
\vs 1Ma 16:2 Тогда Симон призвал двух старших сыновей своих, Иуду и Иоанна, и сказал им: я и братья мои и дом отца моего воевали против врагов Израиля от юности до сего дня и много раз успешно спасали руками нашими Израиля.
\vs 1Ma 16:3 Но вот, я состарился, а вы по милости \bibemph{Божией} находитесь в летах зрелых: заступите место мое и брата моего, идите и сражайтесь за народ наш, и да будет с вами помощь небесная.
\vs 1Ma 16:4 И избрал из страны двадцать тысяч воинов и всадников, и пошли они против Кендевея, и ночевали в Модине.
\vs 1Ma 16:5 Встав же утром, вышли на равнину, и вот многочисленное войско навстречу им, пешие и конные, и между ними был поток.
\vs 1Ma 16:6 И двинулся против них сам и народ его, и, видя, что народ боится переходить поток, он перешел первый, и увидели это воины, и перешли за ним.
\vs 1Ma 16:7 И разделил он народ, поставив конных среди пеших; конница же неприятелей была весьма многочисленна.
\vs 1Ma 16:8 И затрубили священными трубами; и Кендевей обратился в бегство и войско его, и пало у них много раненых, остальные же бежали в крепость.
\vs 1Ma 16:9 Тогда был ранен Иуда, брат Иоанна; но Иоанн преследовал их, доколе не пришел в Кедрон, который он построил.
\vs 1Ma 16:10 И убежали они в башни, находящиеся в области Азота, но он сжег его огнем, и погибло из них до двух тысяч мужей; и возвратился он с миром в землю Иудейскую.
\rsbpar\vs 1Ma 16:11 Птоломей же, сын Авува, поставлен был военачальником на равнине Иерихонской и имел много серебра и золота;
\vs 1Ma 16:12 ибо он был зять первосвященника.
\vs 1Ma 16:13 И надмилось сердце его, и захотел он овладеть страною, и делал коварные замыслы против Симона и сыновей его, чтобы погубить их.
\vs 1Ma 16:14 Между тем Симон, посещая города страны и заботясь о потребностях их, пришел в Иерихон, сам и Маттафия и Иуда, сыновья его, в сто семьдесят седьмом году в одиннадцатом месяце~--- это месяц Сават.
\vs 1Ma 16:15 И с коварством принял их радушно сын Авувов в небольшую крепость, называемую Док, им устроенную, и сделал для них большой пир, и спрятал там людей.
\vs 1Ma 16:16 И когда опьянел Симон и сыновья его, тогда встал Птоломей и бывшие при нем, взяли оружия свои и вошли к Симону во время пира и убили его и двух сыновей его и некоторых из служителей его.
\vs 1Ma 16:17 Так совершил он великое вероломство и воздал за добро злом.
\vs 1Ma 16:18 Птоломей написал об этом и послал к царю, чтобы прислал ему войско на помощь, и он предаст ему страну их и города.
\vs 1Ma 16:19 И некоторых послал в Газару убить Иоанна, а тысяченачальникам послал письма, чтобы они пришли к нему, и он даст им серебра и золота и подарки;
\vs 1Ma 16:20 а других послал овладеть Иерусалимом и горою храма.
\vs 1Ma 16:21 Но некто, прибежав к Иоанну в Газару, известил его, что отец его и братья умерщвлены и что \bibemph{Птоломей} послал убить и его.
\vs 1Ma 16:22 Услышав об этом, \bibemph{Иоанн} весьма смутился и, схватив мужей, пришедших погубить его, убил их, ибо узнал, что они искали погубить его.
\rsbpar\vs 1Ma 16:23 Прочие же дела Иоанна и в\acc{о}йны его и мужественные подвиги его, славно совершенные, и сооружение стен, им воздвигнутых, и другие деяния его,
\vs 1Ma 16:24 вот, они описаны в книге дней первосвященства его, с того времени, как сделался он первосвященником после отца своего.

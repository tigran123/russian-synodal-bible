\bibbookdescr{Jub}{
  inline={Книга Юбилеев},
  toc={Книга Юбилеев},
  bookmark={Книга Юбилеев},
  header={Книга Юбилеев},
  abbr={Юбл}
}
\vs Jub 0:0
Вот слова деления дней по закону и свидетельству,
по событиям годов,
по их седминам, по их юбилеям, на все годы мира,
согласно с тем, что говорил он с Моисеем на горе Синай.

\vs Jub 1:1
Случилось в первый год по выходе сынов Израиля из Египта, в третий месяц, в
шестнадцатый день его, тогда сказал Бог Моисею, говоря: <<Взойди ко Мне на
гору, чтобы Я дал тебе две каменные скрижали закона и все заповеди, которые Я
написал, дабы ты возвестил их им (сынам Израиля)!>> И Моисей взошел на гору
Господню, и слава Господня обитала на горе Синай, и облако осеняло ее шесть
дней. И Он воззвал Моисея в седьмой день среди облака. И он видел славу Божию,
как пылающий огонь, на горе Синай, когда взошел, чтобы получить каменные
скрижали закона и заповедей, по слову Господа, как Он сказал ему:
<<Поднимись на вершину горы!>> И Моисей был на горе сорок дней и сорок
ночей, и Господь научил его относительно того, что было прежде и что случится в
будущем; Он изъяснил ему деление дней закона и свидетельства и сказал:
<<Внимай всем словам, которые Я тебе говорю, и запиши их в книгу, дабы их
роды (потомки) видели, как Я оставил их за все зло, какое сделали они,
уклонившись от завета, который Я утверждаю ныне между Мною и тобою на горе
Синай для будущих родов их. И будет это слово, когда придут все наказания,
свидетельствовать против них, и они познают, что Я справедливее, нежели они во
всей их правде и во всяком их деле, и узнают, что Я был с ними. И ты запиши
себе все слова, которые Я тебе возвещаю ныне,~--- ибо Я знаю их противление
и жестоковыйность,~--- прежде чем приведу их в землю, о которой клялся
Аврааму, Исааку и Иакову, когда сказал: <<Вашему семени Я дам землю,
текущую молоком и медом>>. И они будут есть, и насыщаться, и уклоняться к
чуждым богам, к тем, которые их не спасли от всей их тяготы. И будет это
свидетельство услышано во свидетельство им: ибо они будут забывать Мои
заповеди, все, что Я заповедаю им, и пойдут вослед язычников и за их нечистотою
и мерзостию, и будут служить их богам, и эти (боги) сделаются для них
претыканием в бедствие, и страдание, и сетию. И многие погибнут, и будут
пленены, и впадут в руки врага, так как они забудут Мои постановления, и Мои
заповеди, и Мои праздники, Мой завет, и Мои субботы, и Мою святыню, которую Я
освящу Себе между ними, и Мою скинию, и Мое святилище, которое Я освящу Себе в
стране, чтобы положить на нем Свое имя, дабы оно обитало там. И они будут
делать себе изображения из камня и из дерева, и будут преклоняться пред ними,
чтобы впадать в грехи, и будут приносить своих сыновей в жертву демонам и
предаваться всем делам заблуждения своего сердца. И Я буду посылать к
ним свидетелей, чтобы дать им свидетельство, но они не послушают их и будут
убивать Моих свидетелей; и также тех, которые следуют закону, они будут убивать
и преследовать, и отвергнут его (закон) совершенно, и начнут делать то, что
есть зло пред Моими очами. Тогда Я сокрою Свое лице от них, и предам их
язычникам в пленение, и в узы, и на истребление, и изгоню их из земли
(Ханаанской), и рассею между язычниками. И они забудут весь Мой закон, и
все Мои заповеди, и всю Мою правду, и не будут более хранить ни новолуния, ни
субботы, и никакого праздника и юбилейного года, и никакого установления. После
сего они опять обратятся ко Мне из среды язычников всем сердцем и всею
душою и всеми своими силами. И Я соберу их всех из среды язычников; и они опять
будут искать Меня, чтобы Я явил им Себя. Когда же они будут искать Меня всем
сердцем и всею душою, Я открою им великий мир с правдою и восставлю их как
растение праведности от всего Моего сердца и от всей души; и они будут
во благословение, а не в проклятие, и соделаются главою, а не хвостом.

И Я воссоздам Мое святилище между ними и буду обитать с ними, и буду их
Богом, и они будут Моим народом воистину и вправду; и Я не оставлю их, не
отрекусь от них, ибо Я Господь, Бог их>>.

И Моисей пал на свое лице, и молился, и говорил: <<Господи, Боже мой! не
оставляй Твоего народа и Твоего наследия, чтобы не ходить им в заблуждении
своего сердца, и не предавай их в руки врагов-язычников, чтобы они
владычествовали над ними; не допусти их до сего, чтобы им не потерять Тебя!
Простри, Господи, Свое милосердие над народом Своим, и дух правый соделай в
них, и не допусти духа Велиара владычествовать над ними, чтобы он
клеветал на них пред Тобою и совращал их со всех путей правды, дабы они
погибли пред лицем Твоим! Ибо они Твой народ и наследие, который Ты великою
силою освободил из рук египтян; соделай в них чистое сердце и святой дух и не
допусти их, чтобы они были доведены до падения чрез свои грехи, отныне и до
века!>>

И Бог сказал Моисею: <<Я знаю их противление, и их помышления, и их
жестоковыйность; они не покорятся, пока не познают своих грехов и грехов
отцов их. И после сего они обратятся ко Мне во всей праведности, и от всего
сердца, и от всей души, и Я обрежу крайнюю плоть их сердца и крайнюю плоть
сердца их семени, и соделаю в них дух святой, и очищу их, чтобы они более не
отвращались от Меня с того дня и до века. И их душа прилепится ко Мне и ко всем
Моим заповедям, и они будут исполнять Мои повеления, и Я буду их отцом, и они
будут Моим сыном, и все будут именоваться сынами Божиими и все сынами
Духа. И тогда откроется, что они сыны Мои и Я отец их в праведность и
правду, и что Я люблю их. Ты же запиши себе все эти слова, которые Я возвещаю
тебе на этой горе, первое и последнее, и грядущее, согласно со всем делением
времени под законом и свидетельством и по седминам юбилейных годов, до века,
пока Я не сойду и не буду жить с ними от века до века>>.

И Он сказал Ангелу лица: <<Запиши для Моисея события с первого
творения до того времени, когда Мое святилище будет устроено между ними,
навсегда и навечно, и Бог откроется для очей каждого, и всякий познает, что Я
Бог Израиля, и Отец всех детей Иакова, и Царь на горе Сионе, от века до века. И
Сион Иерусалим будет святым>>. И Ангел лица, который шел пред станом
израильтян, взял скрижали деления лет от творения, седмин и юбилеев, закона и
свидетельства, каждый год по его числу и юбилеи по годам, со дня нового
творения, когда были сотворены небо и земля и все их произведения, равно как и
небесные силы и все творение земли, до того времени, когда будет создано
святилище Господа в Иерусалиме на горе Сионе, и все светила будут обновлены к
освящению, и к миру, и к благословению для всех избранных Израиля, чтобы сие
пребывало так от того дня в продолжение всех дней земли.

\vs Jub 2:1
И Ангел лица сказал Моисею по слову Господа, говоря: <<Напиши все
повествование о творении, как Господь Бог совершил в шесть дней все Свои
произведения, которые Он сотворил, и в седьмой день соблюдал субботу, и освятил
ее на все века, и утвердил ее в знамение для всех Своих творений>>.

Ибо в первый день Он сотворил небеса, которые вверху, и землю, и воды, и
всех духов, которые Ему служат, и Ангелов лица, и Ангелов прославления, и
Ангелов духа огня, и Ангелов духа ветров, и Ангелов облачных духов мрака, и
града и инея, и Ангелов долин, и громов и молний, и Ангелов духов холода и
зноя, зимы и весны, осени и лета, и Ангелов всех духов Его творений на небе, и
на земле, и во всех долинах, и духов мрака и света, и утренней зари, и вечера,
которые Он приготовил по предвидению Своей премудрости. И тогда мы увидели Его
произведения, и прославили Его, и восхвалили Его за все произведения Его, ибо
семь великих произведений Он сотворил в первый день.

И во второй день Он сотворил твердь между водами; и разделились воды в тот
день: половина их поднялась вверх, и половина опустилась вниз под твердь,
которая в середине, на поверхность всей земли. И это единственное произведение,
которое Он сотворил во второй день.

И в третий день сотворил Он, как сказал водам: да стекут они с поверхности
всей земли в одно место и да явится суша. И Он сделал таким образом с водами,
как сказал им. И они стекли с поверхности земли в одно место, вне тверди, и
явилась суша. И в тот день Он создал для нее (воды) бездны морей по их
отдельным вместилищам, и все реки, и вместилища вод в горах и во всей земле, и
все озера, и всякую росу земную, и семя, которое сеется по роду своему, и все,
что употребляется в пищу, и плодовые и лесные деревья, и сад Едем для веселия.
Все эти четыре великие творения Он сотворил в третий день.

И в четвертый день Он сотворил солнце, и луну, и звезды, и поставил их на
тверди небесной, чтобы они светили на всю землю, и повелел им управлять днем и
ночью, и разделять между светом и между тьмою. И Бог сделал солнце великим
знамением на земле для дней, и суббот, и годов, и юбилеев, и для всех времен
года, и повелел ему разделять между светом и между тьмою, и
предназначил его для роста, чтобы росло все, что прозябает и
произрастает на земле. Эти три рода творения Он создал в четвертый
день.

И в пятый день Он сотворил больших морских животных в глубинах вод,~---
ибо они были созданы Его рукою прежде всего,~--- всякую плоть, и все, что
движется в водах, рыб, и все, что летает, птиц и весь их род. И солнце взошло
над ними для развития, и над всем, что существует на земле, и над всем, что
прозябает из земли, и над всеми плодовыми деревьями, и над всякою плотью. Все
эти три рода Он сотворил именно в пятый день.

И в шестой день Он создал всех зверей земных, и всякий скот, и все, что
движется на земле. И после всего этого Он сотворил человека, одного, мужа и
жену сотворил их, и поставил его владыкою над всем, что на земле и что в морях,
и над тем, что летает, и над зверями, и над скотом, и над всем, что движется на
земле, и над всею землею: надо всем этим Он сделал его господином. И эти четыре
рода творений Он сотворил в шестой день.

И было всего сотворено в шесть дней двадцать два рода. И Он закончил все
Свои произведения в шестой день~--- все, что на небе, и на земле, и в морях,
и в долинах, во свете и во тьме и всюду. И Он дал нам (Ангелам) великое
знамение~--- день субботний, чтобы мы в продолжение шести дней делали дела и
в седьмой день соблюдали субботу ото всех дел, все Ангелы лица и все Ангелы
прославления. Нам, этим двум великим родам, сказал Он, чтобы мы хранили с Ним
субботу на небе и на земле. И Он сказал нам: <<Вот Я выделю Себе народ из
среды всех народов, чтобы и они (он?) праздновали субботу; и Я освящу
его Себе в Свой народ, и благословлю его, как Я освятил день субботний и
посвятил их (субботы) Себе; так благословлю Я его; и они будут Моим народом, и
Я буду их Богом. И Я избрал семя Иакова между всеми из тех, которых Я увидел, и
написал Его у Себя перворожденным сыном, и освятил его для Себя навсегда и
навечно. И Я возвещу им о субботнем дне, чтобы они хранили в него субботу ото
всех дел своих>>. Так положил Он знамение в нем, дабы и они праздновали с
нами субботу в седьмой день, чтобы есть и пить, и прославлять Того, Кто
сотворил все, как и Он благословил сие и освятил Себя для Своего народа, чтобы
это было явлено пред всеми народами и чтобы они (потомки Иакова?) одинаково с
нами праздновали субботу. И Он установил, чтобы Его повеления возносились пред
Него, как благовоние, которое было бы приятно Ему, во все дни двадцати двух
глав людей от Адама до Иакова. И двадцать два рода произведений были сотворены
до сего седьмого дня. Этот благословлен и освящен, и тот (Иаков) также
благословлен и освящен. И этот вместе с тем служит к освящению и благословению.
И сему (Иакову и его потомкам) даровано было, чтобы они были всегда
благословенными и святыми в свидетельстве и законе, как прежде седьмой день Он
освятил и благословил быть седьмым днем (т.е. субботою). Он сотворил небо
и землю и все, что создано в шесть дней, и Господь установил святой праздник
для всех Своих тварей. Посему Он дал повеление относительно него всем Своим
творениям, что нарушители седьмого дня должны умереть: если кто
осквернит его, тот да умрет смертию. И ты с своей стороны скажи сынам
Израилевым, чтобы они соблюдали этот день, святили его, и никакого дела не
делали в него, и не оскверняли его; ибо он святее, нежели все другие
дни, и всякий, кто оскверняет его, должен умереть смертию; и всякий, кто делает
в него какое-либо дело, должен умереть смертию, навсегда и навечно чтобы сыны
Израиля соблюдали этот день в своих родах и не были истреблены на земле. Ибо
это святой день и благословенный день, и всякий человек, соблюдающий его и
празднующий в него субботу от всякого своего дела, будет свят и благословен
всегда, как мы (Ангелы). И ты возвести и изъясни сынам Израилевым закон этого
дня, чтобы они праздновали в него субботу и не забывали бы его в заблуждении
своего сердца, чтобы они не делали в него ничего из своих нужд, не приготовляли
в него чего-либо из пищи и питья, ни воды не черпали, ни какой-либо ноши не
вносили и не выносили в него чрез свои врата, если бы им не пришлось
приготовить себе чего-нибудь в течение шести дней в своих домах. И они не
должны ничего выносить и вносить в этот день из одного дома в другой, ибо он
святее и благословеннее всех юбилейных дней юбилейного года. В него мы
праздновали субботу, прежде чем кому-либо из смертных сделалось известным
празднование в него на земле субботы. И Творец всех вещей благословил его; но
Он освятил не всех людей и не все народы праздновать в него субботу, а
только Израиля; ему только предназначил Он есть и пить, и праздновать в него
субботу на земле. И благословил его Творец всех вещей, создавший этот день к
благословению, и к освящению, и к прославлению пред всеми другими днями.
Этот закон и свидетельство даны сынам Израилевым как вечный закон для
всех родов.

\vs Jub 3:1
И в шесть дней второй субботы (седмицы) мы по повелению Господа привели к
Адаму всех зверей, и всякий скот, и всех птиц, и все, что движется на земле, и
все, что движется в воде, по их родам и видам, именно~--- зверей в первый
день, скот во второй, птиц в третий, все, что движется по земле, в четвертый,
все, что движется в воде, в пятый день; и Адам дал им всем имена, и как он
назвал их, так и было им имя. И в продолжение этих пяти дней Адам видел все
это, самца и самку в каждом роде, что есть на земле, между тем как только он
был одинок, и он не мог найти себе никого подобного, кто был бы ему
помощником.

И Господь сказал мне: <<Нехорошо быть человеку одному: создадим ему
помощника, подобного ему>>. И Господь Бог наш навел на него усыпление,
чтобы он заснул. И Он взял для жены одно из ребер его, как вещество для жены, и
создал плоть вместо него; и Он создал жену и пробудил Адама от сна. И когда
Адам пробудился, поднялся в шестой день, и взял ее к себе, и узнал ее, и сказал
ей: <<Это кость от моей кости, и плоть от моей плоти; она назовется моею
женою; ибо от своего мужа она взята. Посему муж и жена да будут одно, и посему
он оставит отца своего и матерь свою и прилепится к жене своей, и будут они
одною плотью>>. И в первую седмицу был создан Адам и его жена, и во
вторую седмицу Он (Бог) поставил ее пред ним. И ради сего дана
заповедь~--- семь дней для мальчика, а для девочки дважды семь дней пребывать
женщине в ее нечистоте.

И после того как Адам пробыл сорок дней в стране, где он был сотворен, мы
привели его в сад Едем. Ради сего на небесных скрижалях рожденных предписано:
<<Если она родила дитя мужеского пола, то должна оставаться в своей
нечистоте семь дней, соответственно первой неделе, и тридцать три дня должна
оставаться в крови своего очищения, и не должна прикасаться ни к чему святому,
ни вступать в святилище, пока она не окончит этих дней, та, которая родила
мужеского пола. А родившая младенца женского пола должна две недели,
соответственно двум первым неделям, пребывать в своей нечистоте и шестьдесят
шесть дней в крови очищения; и будет для нее всего восемьдесят дней. И когда
жена (Ева) окончила восемьдесят дней, мы привели ее в сад Едем, ибо он свят во
всей земле, и каждое дерево, которое насаждено в нем, свято. Ради сего для
рождающей мальчика или девочку установлен закон этих дней, чтобы она не
прикасалась ни к чему святому, ни в святилище не входила, пока не окончатся эти
дни для мальчика или девочки. Это закон и свидетельство, написанное для
израильтян, чтобы они соблюдали это всегда.

И в начале первого юбилея Адам и жена его были в саду Едем семь лет,
возделывая и храня его. И мы дали ему занятие и научили его все видимое
употреблять в дело, и он трудился. Он же был наг, не зная сего и не стыдясь. И
он охранял сад от птиц, и зверей, и скота, и собирал плоды сада и ел, и
сберегал остаток для себя и своей жены, и делал запас.

И по истечении семи лет, которые он там провел, ровно семи лет, во второй
месяц в семнадцатый день его пришел змий и приблизился к жене. И сказал змий
жене: <<Разве Бог запретил вам все плоды деревьев, которые в раю, чтобы вы
не ели от них?>> И она сказала ему: <<От всех плодов деревьев, которые
в раю, сказал нам Бог, можно нам есть, но от плода дерева, которое в средине
рая, сказал нам Бог, мы не должны есть и прикасаться к нему, дабы нам не
умереть>>. И змий сказал жене: <<Вы не умрете смертию; напротив, Бог
знает, что в день, в который вы вкусите от него, откроются очи ваши, и
вы будете как боги и будете знать доброе и злое>>. И вот жена увидела
дерево, что оно было хорошо и приятно для глаза, и его плод хорош для пищи,
тотчас взяла его и ела. И она первая покрыла свою срамоту смоковничным листом;
и она дала его (плод) Адаму, и он ел, и его глаза открылись, и он увидел, что
был наг, и взял смоковничных листьев, и сшил их, и сделал себе препоясание, и
покрыл свою срамоту. И Господь проклял змия и разгневался на него навсегда. И
на жену также разгневался Он, так как она послушалась голоса змия и ела. И Он
сказал ей: <<Я умножу твои болезни и твое страдание; с болезнями ты будешь
рождать детей, и у своего мужа будешь находить свою защиту, и он будет
твоим господином>>. И Адаму Он сказал: <<Так как ты послушался гласа
жены своей и ел от того дерева, от которого Я запретил тебе есть, то земля
будет проклята из-за тебя; тернии и волчцы будут произрастать тебе, и свой хлеб
ты будешь есть в поте лица твоего, пока не возвратишься в землю, из которой ты
взят. Ибо ты на земле и в землю возвратишься>>. И Он сделал им кожаные
одежды и одел их ими, и изгнал их из рая Едем. И в тот день, когда Адам вышел
из рая Едем, он принес в приятное благоухание жертву благовонную: ладан, и
халван, и стакти, и Сенегал, утром с восходом солнца, в день, когда он покрыл
свою срамоту. И в тот день заключились уста всех зверей, и скота, и птиц, и
того, что ходит (ногами), и того, что движется, так что они не могли более
говорить, ибо до сего все они говорили между собою одними устами и одним
языком. И Он изгнал из сада Едем всякую плоть, которая была в саду Едем; и
рассеялась всякая плоть по своим породам и видам в места, которые для них были
созданы (удобны). Только Адаму Он повелел покрывать свою срамоту~--- ему
одному между всеми зверями и скотом. Ради сего Он на скрижалях повелел всем,
знающим правду закона, покрывать свою срамоту и не обнажаться, как обнажаются
язычники.

И в новолуние четвертого месяца Адам и его жена вышли из рая Едем, и жили в
земле Елдад, в той земле, где они были созданы. И Адам дал имя жене своей Ева.
И у них не было ни одного сына до первого юбилейного года. И после сего он
познал ее. Он же обрабатывал свою землю, как был научен в саду Едем.

\vs Jub 4:1
И в третью седмину во второй юбилей родила она Каина, и в четвертую родила
Авеля, и в пятую родила дочь свою Аван. И в первую седмину третьего юбилея Каин
убил Авеля, ибо Он (Бог) принял дар от руки его милостиво, а от руки Каина
жертву плодов не милостиво. И он убил его на поле, и его кровь вопиет от земли
к небу, восклицая, что он убит. И Бог наказал Каина за Авеля, которого он убил,
и сделал его проклятым на земле за кровь его брата, и проклял его на
земле, ради чего на небесных скрижалях написано так: <<Да будет проклят,
кто убивает своего ближнего по злобе, и все видящие это должны говорить: да
будет так! И человек, который видит и не объявит сего, да будет проклят, как
он!>> Ради сего мы являемся к Господу, Богу нашему, возвещать все грехи,
которые совершаются на небе и на земле, во свете и во тьме, и всюду.

И Адам и его жена скорбели об Авеле четыре седмины. И в четвертый год пятой
седмины он утешился, и опять познал жену свою, и она родила ему сына, и он
нарек ему имя~--- Сиф; ибо он сказал: <<Господь восставил нам другое
семя на земле вместо Авеля, ибо Каин убил его>>. В шестую седмину он родил
свою дочь Азуру. И Каин взял себе свою сестру Аван в жены, и она родила ему
Еноха в конце четвертого юбилея. И в первый год первой седмины пятого, юбилея
были построены дома на земле, и Каин построил город и назвал его по имени сына
своего Енох. И Адам познал свою жену Еву, и она родила еще девять сыновей. И в
пятую седмину сего юбилея Сиф взял себе в жены свою сестру Азуру, и она родила
ему в четвертый год Эноса. И он первый начал призывать имя Господне на земле. И
в седьмой юбилей в третью седмину Энос взял свою сестру Ноаму в жены, и она
родила ему сына в третий год пятой седмины, и он нарек ему имя Каинан. И в
восьмой юбилей в конце его Каинан взял свою сестру Муалелиту в жены, и
она родила ему сына в девятый юбилей, в первую седмину, в третий год той
седмины, и он нарек ему имя Малалел. И во вторую седмину десятого юбилея
Малалел взял себе в жены Дину, дочь Боракиэла, дочь сестры его отца,~---
себе в жены,~--- и она родила ему сына в третью седмину в шестой год, и он
нарек ему имя Иаред; ибо в его дни сошли на землю Ангелы Господни, которые
назывались стражами, чтобы научить сынов человеческих совершать на земле правду
и справедливость.

И в одиннадцатый юбилей Иаред взял себе жену, по имени Барака, дочь
Разузаила, дочь сестры его отца, в четвертую седмину сего юбилея. И она родила
ему сына в пятую седмину, в четвертый год юбилея, и он нарек ему имя
Енох. Он был первый из сынов человеческих, рожденных на земле, который научился
письму, и знанию, и мудрости; и он описал знамения неба по порядку их месяцев в
книге, чтобы сыны человеческие могли знать время годов в порядке их отдельных
месяцев. Он прежде всех записал свидетельство, и дал сынам человеческим
свидетельство о родах земли, и изъяснил им седмины юбилеев, и возвестил им дни
годов, и распределил в порядке месяцы, и изъяснил субботние годы, как мы ему
возвестили их. И что было, и что будет, он видел в своем сне, как произойдет
это с сынами детей человеческих в их поколениях до дня суда. Все видел и узнал
он, и записал во свидетельство, и положил сие, как свидетельство, на земле для
всех сынов детей человеческих и для их родов. И в двенадцатый юбилей в седьмую
седмину взял он себе жену именем Адни, дочь Даниала, дочь сестры его отца. И в
шестой год этой седмины она родила ему сына, и он нарек ему имя Мефусалаг. И
вот он был с Ангелами Божиими в продолжение шести лет, и они показали ему все,
что на земле и на небесах, господство солнца; и он записал все. И он дал
свидетельство стражам, которые согрешили с дочерьми человеческими. Ибо они
стали смешиваться, чтобы оскверняться с дочерьми человеческими. И Енох дал
свидетельство против всех них. И он был взят из среды сынов детей человеческих,
и мы привели его в рай Едем к славе и почести. И вот здесь он записывает суд и
вечное наказание, и всякое зло сынов детей человеческих. И ради него Он (Бог)
послал потоп на землю; ибо он был поставлен в знамение, и чтобы дать
свидетельство против всех сынов детей человеческих, чтобы объявлять все деяния
родов до дня суда. И он принес в жертву курение..., которое было приятно Богу,
на горе полудня; ибо четыре места Божий существуют на земле: рай Едем, и гора
востока, и эта гора, на которой ты теперь,~--- гора Синай, и гора Сион,
которая будет освящена в новом творении для освящения земли; чрез нее земля
освятится от всей своей вины и нечистоты навсегда и навечно.

И в четырнадцатый юбилей взял Мефусалаг Адину, дочь Азраела, дочь сестры его
отца, себе в жены, в третью седмину в первый год, и он родил сына и нарек ему
имя Ламех. И в пятнадцатый юбилей в третью седмину взял себе Ламех жену, по
имени Битанос, дочь Баракела, дочь сестры его отца, себе в жены; и в эту
седмину она родила ему сына, и он назвал его Ноем, говоря: <<Он утешит меня
о всех моих трудах и о земле, которую проклял Бог>>.

И в конце девятнадцатого юбилея в седьмую седмину в шестой год ее умер Адам,
и все сыны его погребли его в стране, где он был сотворен. И он был погребен
прежде всех в земле. И он жил на семьдесят лет меньше тысячи лет, ибо тысяча
лет как один день по небесному свидетельству. Ради сего о древе познания
написано: <<В день, когда вы вкусите от него, вы умрете>>. Посему он не
окончил годы этого дня, но умер в этот день.

В конце этого юбилея был убит Каин после него (Адама) в том же году. Его дом
упал на него, и он умер посреди своего дома, и погиб под его камнями. Ибо
камнем он убил Авеля, и камнем был убит по праведному суду. Сего ради на
небесных скрижалях предписано: <<Орудием, которым муж убил своего ближнего,
должен быть и он убит; как ранил он его, так должны они сделать и
ему>>.

И в двадцать пятый юбилей Ной взял себе жену, по имени Емзараг, дочь
Ракиела, дочь его сестры (?), себе в жены, в первый год в пятую седмину. И в
третий год ее она родила ему Сима, и в пятый год родила ему Хама, и в первый
год в шестую седмину родила ему Иафета.

\vs Jub 5:1
И случилось, когда сыны детей человеческих начали умножаться на поверхности
всей земли и у них родились дочери, Ангелы Господни увидели в один год этого
юбилея, что они были прекрасны на вид. И они взяли их себе в жены, выбрав их из
всех; и они родили им сыновей, которые сделались исполинами. И неправда
усилилась на земле, и всякая плоть извратила свой путь, от людей до скота, и до
зверей, и до птиц, и до всего, что ходит по земле. Все извратили свой путь и
свой порядок, и начали пожирать друг друга. И неправда усилилась на земле, и
все помышления разума сынов человеческих сделались столь злыми во всякое время.
И Господь воззрел на землю, и вот она извратилась, и всякая плоть извратила
свой порядок, и они совершали всякое зло пред Его очами~--- все, что было на
земле. И Он сказал, что Он уничтожит людей и всякую плоть, которую Он сотворил
на земле.

И только Ной обрел милость пред Его очами. И на Ангелов Своих, которых Он
посылал на землю, Он весьма разгневался, так что решил истребить их. И Он
сказал нам, чтобы мы связали их в пропастях земли. И вот они были связаны в них
и разобщены. И относительно детей их вышло повеление от Его лица, чтобы
поразить их мечом и умертвить их под небом. И Он сказал: <<Моему духу не
вечно пребывать на людях, ибо они плоть, и дней их пусть будет сто двадцать
лет!>> И Он послал Свой меч в среду их, чтобы они умертвили друг друга. И
они начали убивать друг друга, пока не пали все от меча и не были уничтожены с
земли на глазах своих отцов. После сего они (их отцы) были связаны в пропастях
земных до дня великого суда, когда придет наказание на всех, извративших свои
пути и свои дела пред Господом. И Он уничтожил все их пристанища, и ни один из
них не остался, которого Он не уничтожил бы за все их зло. И Он соделал для
всех Своих творений новое и праведное естество, чтобы они не согрешали вовек по
всему своему естеству, и каждый был бы праведен чрез свою отрасль. И наказание
всех их определено и записано на небесных скрижалях без неправды. И все,
преступившие путь, который им определен, чтобы ходить по нему, если не ходят по
нему, то наказание написано для каждого естества и для каждого рода. И ничто
не избежит его, что на небе и на земле, во свете и во тьме, в царстве
мертвых, и в пропасти, и в мрачном месте. Все наказания их определены, и
записаны, и начертаны для всех. Великого Он будет судить по его величию, и
малого~--- по его малости, и каждого отдельно~--- по его пути. И Он не
примет никаких даров, ибо говорит, что будет совершать суд над каждым отдельно.
И если бы кто-нибудь дал Ему все, что есть на земле, то Он не обратит лица
Своего и не примет этого от него: ибо Он Судия. И о сынах Израиля написано и
определено: если они обратятся к Нему в справедливости, то Он отпустит им
всякую вину и все грехи их простит. Написано и определено, что милосердие будет
оказано всем, которые обратятся от всякого своего злодеяния, однажды в год. Но
всем тем, которые свои пути и свое стремление извратили пред потопом, не дано
снисхождения, кроме только Ноя, ибо Господь призрел на лице его ради
сыновей, которых Он спас из-за него от потопа. Ибо сердце его было праведно во
всех путях его, как было повелено ему. И он ничего не преступил из того, что
было ему предписано.

И Господь сказал: <<Да будет истреблено все, что на суше, от всякого
скота до диких зверей и птиц и до всего, что движется на земле!>> И Он
повелел Ною сделать себе ковчег, чтобы спастись в нем от потопа. И Ной сделал
ковчег для всех тварей, как Он повелел ему, в (двадцать седьмой)
юбилей, в пятую седмину, в пятый год. И он вошел в него в шестой год ее, в
другой месяц, в новолуние другого месяца. До шестнадцатого дня его вошел в
ковчег он и все, что мы привели к нему. И Господь затворил его снаружи в
семнадцатый день вечером. И Бог открыл семь окон небесных, чтобы они
изливали воду с неба на землю в продолжение сорока дней и сорока ночей. И
источники бездны также изливали воду, так что весь мир наполнился водою. И
поднялись воды на земле: на пятнадцать локтей поднялась вода над всеми высокими
горами. И ковчег носился над землею и плавал на поверхности воды. И вода стояла
на поверхности земли пять месяцев, сто пятьдесят дней. И он (ковчег) пришел и
остановился на вершине Любара, одной из гор Арарата. И в четвертый месяц
замкнулись источники великой бездны и хляби небесные затворились. И в новолуние
седьмого месяца все отверстия пропастей земли открылись, и вода стала стекать в
преисподнюю бездну. И в новолуние десятого месяца показались вершины гор. И в
новолуние первого месяца обнаружилась земля, и вода стекла с земли в пятую
седмину в седьмой год ее. И в семнадцатый день второго месяца просохла земля. И
в двадцать седьмой день его он отворил ковчег и выпустил из него зверей,
птиц и что двигалось.

\vs Jub 6:1
И в новолуние третьего месяца вышел он из ковчега, и устроил жертвенник на
этой горе, и показался на земле. И он взял молодого козла и пролил кровь его в
искупление за всю вину земли, ибо все, что существовало на ней, было
истреблено, кроме тех, которые были в ковчеге с Ноем. И он положил тук его на
жертвенник, и взял тельца, и овна, и овцу, и козлов, и соли, и горлицу, и
молодого голубя, и принес всесожжение на жертвеннике, и примешал к сему
испеченные в масле жертвенные плоды, и возлил кровь и вино, и положил на все
фимиам, и вознес приятное благоухание, которое было приятно Господу. И Господь
обонял приятное благоухание и заключил с ним завет, что не придет более потоп,
который погубил бы землю, что во все дни земли сеяние и жатва не прекратятся,
мороз и жар, лето и зима, день и ночь не изменят своего порядка и не
прекратятся. <<И вы растите и плодитесь на земле, и размножайтесь на ней, и
будьте во благословение на ней! Ваш страх и трепет Я положу на все, что на
земле и в море. И вот Я всех диких зверей, и всякий скот, и все, что летает, и
все, что движется на земле, и рыб в водах, и все~--- дал вам в пищу, как
зелень травную дал Я вам все, чтобы вы ели. Только плоть, в которой живая душа,
вы не должны вкушать с кровию, ибо душа всякой плоти есть кровь, да не взыщется
кровь вашей души. От каждого человека, от каждого Я взыщу кровь человека; кто
проливает человеческую кровь, того кровь пусть прольется от руки человеческой,
ибо по образу Божию Он сотворил Адама. А вы раститесь и умножайтесь на
земле!>> И дети его поклялись, что они не будут есть крови, которая в
какой-либо плоти. И он заключил завет пред Господом Богом навечно, на все роды
земли, в этом месяце.

Ради сего Он говорил с тобою, чтобы и ты с сынами Израиля в этом месяце на
горе заключил завет с клятвою и окропил их кровию ради всех слов завета,
который Господь заключил с ними на все время. И это свидетельство предписано
им, дабы и вы соблюдали это во все дни, чтобы вам никогда не есть крови
зверей (...). И человек, который ест кровь дикого зверя, и скота, и птиц, пока
стоит земля, будет истреблен на земле~--- он и его семя. И Он повелел сынам
Израиля не есть крови, дабы они и их семя существовали пред Господом, Богом
нашим, всегда. И для сего закона нет конца времени; вечно они должны соблюдать
его вместе с потомками, чтобы непрерывно кровию за вас испрашивалось прощение
пред жертвенником; ежедневно, утром и вечером, должно испрашивать у Господа
прощение за них, чтобы они соблюдали это и не были истреблены.

И Он дал Ною и его сыновьям знамение, что не придет опять потоп на землю. Он
поставил Свою радугу в облаках в знамение вечного завета, что потоп более не
придет на землю для истребления ее, пока стоит земля. Посему определено и
написано на небесных скрижалях, чтобы они соблюдали праздник седмиц в этом
месяце однажды в год, чтобы возобновлять завет каждый год. И всего времени, в
течение которого праздновался этот праздник на небе, от дней творения до дней
Ноя было двадцать семо юбилеев и пять седмин. И Ной праздновал его в
продолжение семи юбилеев и одной седмины до дня своей смерти; а сыны Ноя
оскверняли его до дней Авраама и ели кровь. Только Авраам соблюдал его, и
сыновья его Исаак и Иаков соблюдали его до твоих дней. И в твои дни сыны
Израиля забыли его, пока я не обновил их при этой горе. И ты сделай также
повеление сынам Израиля, чтобы они соблюдали этот праздник во всех своих родах,
как закон для себя. Один день в году в этом месяце пусть празднуют они
праздник. Ибо это праздник седмиц, и это праздник первого творения; праздник
этот имеет двоякого рода значение и установлен для двух родов
сообразно тому, что об этом написано и начертано. Ибо я записал это в книге
первого закона, в той, которую я написал тебе, да празднуешь ты всякий раз по
одному дню в году. Я изъяснил тебе и жертвенные дары в него, дабы они хранились
в памяти, и сыны Израиля праздновали бы его в своих родах в этом месяце по
одному дню в год.

И новолуния первого, четвертого, седьмого и десятого месяцев суть дни
воспоминания и праздничные дни в четыре времени года. Они записаны и
установлены к ежегодному свидетельству. И Ной назначил их себе в праздники для
будущих родов, чтобы иметь в них праздник воспоминания. В новолуние первого
месяца было сказано ему, чтобы он сделал ковчег; и в этот день земля стала
сухою, и он отворил ковчег и увидел землю. В новолуние четвертого месяца
заключилось отверстие преисподней глубины земли. И в новолуние седьмого месяца
все отверстия и глубины бездны открылись и воды стали стекать в них. И в
новолуние десятого месяца показались вершины гор, и Ной возрадовался. Посему он
определил их себе в праздники воспоминания навек, и так они утверждены. И они
внесены на небесные скрижали: двенадцать (?) суббот имеет каждое из них, от
одного новолуния до другого (т.е. от первого до четвертого) идет
их воспоминание, от первого до второго, от второго до третьего, от третьего до
четвертого. И всех дней, которые предписаны, пятьдесят две субботы дней; этим
весь год исполняется. Так начертано и установлено на небесных скрижалях, и не
бывает пропуска, ежегодно, из года в год.

И ты скажи сынам Израилевым, чтобы они содержали годы по сему числу, триста
шестьдесят четыре дня: и это будет полный год, и определенное время дней и
праздники года не будут извращены; ибо все совершается в нем (в году) согласно
тому, что утверждено относительно сего, и они не должны опускать ни одного дня
и не должны нарушать ни одного праздника. Если же они преступят и не будут
поступать по его повелениям, то они враз все определенные времена извратят и
годы будут подвинуты с мест. И они будут преступать свой порядок; и все сыны
Израиля забудут путь годов, и не обретут более, и забудут новолуние и его время
и субботы, и заблудятся относительно всего порядка годов. Ибо я знаю это и
отныне возвещаю тебе сие, и это не по моему разумению, но так, как написано в
книге у меня, и на небесных скрижалях определено деление дней, ибо они не
должны забывать праздников завета, и не должны соблюдать праздников язычников,
и ходить по их заблуждениям и по их мыслям. И это будет с людьми, которые будут
наблюдать над луною, они именно извратят времена, и каждый год уйдет вперед на
десять дней. И из-за этого они извратят будущий год, и сделают мнимый день за
день свидетельства и нечистый день за день праздничный. И каждый будет
смешивать святой день с нечистым и нечистый со святым; ибо они будут
заблуждаться в месяцах, и субботах, и праздниках, и юбилейных годах. Посему я
повелеваю и подтверждаю тебе, чтобы ты засвидетельствовал им,~--- так как
после твоей смерти твои дети (?) извратят это,~--- что они должны считать
год только в триста шестьдесят четыре дня. Из-за сего они будут заблуждаться в
новолунии, и субботе, и в дне торжества и праздника и будут всегда есть плоть в
крови.

\vs Jub 7:1
И в седьмую седмину в первый год ее в этом
юбилее Ной насадил виноградные деревья на горе,
на которой остановился ковчег, называемой Лубар,
на одной из гор Арарата. И они принесли плод на
четвертом году. И он берег свои плоды, и снял их в
том году в седьмом месяце, и сделал из них вино, и
влил его в сосуд, и держал его даже до пятого года,
до первого дня, т.е. до новолуния первого месяца.
И он принес всесожжение для Господа, молодого
тельца, и овна, и семь однолетних агнцев, и
молодого козла, чтобы испросить прощение себе и
своим сыновьям. И он приготовил прежде всего
козла, и принес его кровь к (...) алтаря, который он
сделал, и весь тук его положил на алтарь, где он
приготовил всесожжение, и от тельца, и от овна, и
от агнцев он взял все мясо на жертвенник и
возложил на него все плодовые жертвы, какие
принадлежали к ней, смешанные с елеем. Тогда он
возлил прежде всего вино в огонь на жертвеннике,
и положил фимиам на жертвенник, и вознес доброе,
приятное благоухание, чтобы оно вознеслось пред
Господа, Бога его. И он возрадовался, и испил от
этого вина~--- он и его дети, исполненные радости. И
настал вечер; тогда он вошел в свой шатер, и лег
опьяненный и заснул, обнажился во время сна в
своем шатре. И Хам увидел своего отца Ноя нагого,
и вышел, и рассказал своим двум братьям. И Сим
взял свою одежду и поднялся вместе с Иафетом, и
они сняли свою одежду с своих плеч, обратив лицо
назад, и покрыли срамоту своего отца, обративши лицо
назад. И когда Ной пробудился от своего сна, то
узнал все, что сделал с ним его младший сын. И он
проклял его сына и сказал: <<Проклят Ханаан,
послушнейшим рабом да будет он своим братьям!>>
И он благословил Сима: <<Да будет прославлен
Господь Бог Сима, и Ханаан да будет его рабом! Да
распространит Господь Иафета, и да живет Господь
в жилище Сима, и Ханаан будет его рабом!>>

И Хам узнал, что его отец проклял его сына, и
отделился от своего отца, он и его сыновья с ним, в
Хуш, и Мистрем, и Фуд, и Ханаан. И он выстроил себе
город и назвал его по имени своей жены
Неелатамек. И Иафет увидел это, и позавидовал
своему брату, и также выстроил город, и назвал его
по имени своей жены Адотанелек. Но Сим жил со
своим отцом Ноем, и выстроил город близ города
своего отца при горе, и он также назвал его по
имени своей жены Седукательбаб. Вот три города
близ горы Лубар: Седукательбаб пред горою на ее
восточной стороне, и Неелтамаук на южной стороне,
Адатанезес (?) к западу. И вот сыновья Сима: Елам,
Асур, Арфаскад [...].

В двадцать восьмой юбилей Ной начал учить своих
внуков всем постановлениям и заповедям, которые
он знал, и закону; и дал свидетельство своим
сыновьям, чтобы они делали справедливость, и
покрывали срамоту своего тела, и прославляли
Того, Кто сотворил их, и почитали отца и матерь,
чтобы любили друг друга и ограждали свои души от
всякого любодеяния и нечистоты и от всякой
несправедливости. Ибо за эти три вины пришел на
землю потоп, именно~--- за любодеяние, которым
стражи вопреки предписаниям их закона блудили с
дочерьми человеческими и взяли себе жен из всех,
которые им понравились: они положили начало
нечистоте. И их сыны, Нефилимы и все другие стали
разногласить друг с другом, и один пожирал
другого: исполин убивал Нефила, и Нефил убивал
Елъйо, и Елъйо сынов человеческих, и один человек
другого. И каждый был [...], чтобы делать неправду и
проливать неповинную кровь; и земля наполнилась
нечестием. И за ними последовали все дикие звери,
и птицы, и что движется, и что ходит по земле; и
пролилось много крови на земле. И все помышление
и стремление людей было пустое и злое. И Господь
истребил все с поверхности земли; за лукавство их
дел и за кровь, которую они пролили на земле, Он
истребил все. И я, и вы, мои сыны, и все, что с нами
вошло в кочег, сохранилось целым. И вот я вижу
прежде всего ваши дела, как вы ходите не в
справедливости, но начали ходить по пути
развращения, и отделяться друг от друга, и быть
завистливыми Друг к другу, один к другому, и как
вы не единодушны, мои сыны, брат с своим братом.
Ибо я вижу, что демоны начали обольщать вас и
ваших сыновей. И теперь я страшусь за вас, чтобы
вы, когда я умру, не стали проливать на земле
кровь человеческую, а чтобы и вы не были
истреблены с поверхности земли. Ибо каждый, кто
проливает человеческую кровь, и каждый, кто ест
кровь какой-либо плоти, будет истреблен из среды
всех с лица земли, и ни одного человека не
останется на земле, который ест кровь и проливает
кровь на земле; и не останется у него семени и
потомства под небом; но они пойдут в царство
мертвых и сойдут в место осуждения; все они
погрузятся в мрак бездны через мучительную
смерть~--- каждый из вас, кто от всякой крови не
принесет за себя для очищения; т.е. как только вы
заколете зверя, или скот, или что летает на земле,
то делайте доброе дело за себя кровию, где только
она проливается на земле. И никто из вас не должен
есть плоть с кровию; удерживайте, чтобы не ели
кровь пред вами. Закапывайте кровь, ибо так было
заповедано мне; я свидетельствую о сем как вам,
так и вашим сыновьям, вместе со всякою плотию. И
не ешьте душу с плотию, да не взыщется ваша кровь,
которая есть ваша душа, от всякой плоти, которая
проливает ее на земле. Ибо земля будет нечиста от
крови со времени ее пролития на ней, но только через
кровь того, кто пролил ее, земля будет чистою в
продолжение всех своих родов. И теперь, мои
сыновья, послушайте меня, творите правду и
справедливость, чтобы вы были насаждены в
справедливости на всем лице земли, и да
вознесется ваша слава к Богу моему, Который спас
меня от потопа. И вот вы пойдете и выстроите себе
города, и разведете в них всякие растения,
которые на земле. И теперь от всех плодовых
деревьев в продолжение трех лет не должен
собираться плод ни от какого дерева, чтобы
есть его, и в четвертый год их плод должен быть
освящен, и начаток плодов [...] должен быть
принесен пред Господа, Всевышнего, Который
создал небо и землю и все, чтобы с лучшим начатком
плодов принести вино и елей на жертвенник
Господа, который Он изберет; и что останется,
слуги дома Божия должны съесть пред
жертвенником, который Он изберет. И в пятый год
сделайте обнародование, чтобы вы обнародовали
это в справедливости и праведности, и вы будете
праведными, и все ваши растения умножатся. Ибо
так заповедал Енох, отец вашего отца Мефусалага,
своему сыну, и Мефусалаг своему сыну Ламеху, и
Ламех заповедал мне все, что заповедали ему отцы
его. И я также заповедую вам это, мои сыны, как
Енох заповедал своему сыну в его первый юбилей,
когда он был еще жив, седьмой в своем роде, он
заповедовал и свидетельствовал это своему сыну и
сыновьям его сыновей до дня своей смерти.

\vs Jub 8:1
И в двадцать девятый юбилей в первую седмину в
первый год Арфаскад взял себе жену, по имени
Разуйю, дочь Сусаны, дочери Елама, себе в жены, и
она родила ему сына в третий год этой седмины, и
он нарек ему имя Каинам. И его сын возрос, и его
отец научил его писанию, и он пошел искать себе
место, где бы основать себе город. И он нашел
надписание, которое праотцы начертали на скале; и
он прочитал, что было на ней, и перевел это, и
нашел, что на ней было знание, которому научили
стражи, о колесницах солнца, и луны, и звезд, и обо
всех замениях неба. И он записал это, но ничего о
сем не рассказал, ибо он боялся рассказать о сем
Ною, чтобы он не разгневался на него за это.

И в тридцатый юбилей во вторую седмину в первый
год ее взял он себе жену, по имени Мелку, дочь
Абадая, сына Иафета. И в четвертый год она родила
ему сына, и он нарек ему имя Сала, ибо сказал: <<Я
отпущен>>. В четвертый год родился Сала, и он
возрос и взял себе жену по имени Муак, дочь
Кеседа, брата его отца, себе в жены. И в тридцать
первый юбилей в пятую седмину в первый год она
родила ему сына [...], и он нарек ему имя Ебор. И он
взял ему жену по имени Ацурад, дочь Неброда, и
именно в тридцать второй юбилей в седьмую
седмину в третий год. И в шестой год она родила
ему сына, и он нарек ему имя Фалек. Ибо во Дни,
когда он родился, дети Ноя начали делить землю
между собою; и ради этого он нарек ему имя Фалек. А
они делили между собою лукаво, и об этом было
сказано Ною.

И в начале тридцать третьего юбилея они
разделили землю на три части~--- Симу, Хаму и Иафету,
по их наследственным частям в первый год первой
седмины; в то время Ангел, один из нас, посланных к
ним, был при этом. И он (Ной) призвал своих сыновей,
и они приблизились к нему~--- они со своими
сыновьями~--- и он разделил землю по жребию, что
должны были получить три его сына, и они
распростерли руки, и взяли жребий из пазухи
своего отца Ноя.

И на жребий Сима вышла средина земли, которую он
должен был получить как наследие для своих
сыновей и потомков вовек, от средины горы Рафу,
где изливается вода из реки Тоны; и идет его
наследие к западу чрез средину той реки, и идет,
пока не подойдешь к водному бассейну, из которого
выходит эта река, и река эта вытекает и изливает
свою воду в море Миот, и идет эта река до великого
моря. И все, что к югу от него, принадлежит Симу; и
идет его наследие, пока не подойдешь к Карасо,
т.е. до залива перешейка, который смотрит к югу. И
идет его наследие к великому морю и выходит
прямо, пока не подойдешь к западу перешейка,
который смотрит к югу. Ибо это море называется
египетским морским заливом. И оттуда
направляется на юг к устью великого моря до
берегов воды, и идет к Аравии в Офру, и идет, пока
не достигнет воды потока Гигон, и на юг от воды
Гигон, вдоль берега этой реки, и идет на юг, пока
не подойдет к раю Едем на юг от него и на восток от
всей страны Едем [...]; и обращается на восток от
него, и идет, так что подходит к востоку горы,
которая называется Рафа, и спускается к берегу
устья реки Тины. Это наследие досталось по жребию
Симу и его детям, чтобы владеть им (наследием), и
его потомкам до века. И Ной возрадовался, что это
наследие досталось Симу и его детям, и он
размышлял обо всем, что он сказал своими устами в
своем пророчестве, когда говорил: <<Да будет
прославлен Господь, Бог Сима, и да вселится
Господь в жилищах Сима!>> И он знал, что рай Едем
есть святейшая из святынь и жилище Господа и что
гора Сион, центр пустыни, и гора Синай, центр пупа
земли, эти три, одна против другой, созданы были
святынями земли. И он прославил Бога богов,
который вложил речь Господа в уста его [...]. И он
познал, что блаженное и благословенное наследие
Симу и его детям будет уделом для вечных родов;
именно~--- вся страна Эритрейского моря, и вся
страна востока и Индия (и при Эритрейском море) и
горы ее, и вся страна Бала, и вся страна Либанос, и
острова Кафтор, и весь горный хребет Санер и Амар,
и горный хребет Ассур на севере, и вся страна
Елам, Ассур, и Бабель, и Сузан, и Мадай, и вся
страна Арарат, и вся страна по ту сторону горного
хребта Ассур к северу~--- благословенная и обширная
страна, и все, что в ней, очень хорошо.

И Хаму досталась вторая наследственная часть,
по ту сторону Гигона, к югу, направо от рая, и она
идет к югу. И направляется она к огненным горам
и к западу к морю Атил, и направляется на запад,
пока не подойдет к морю бассейна, того, в котором
погибает все, что бы ни стекало, и идет к северу к
пределу Гадит, и идет до берегов моря по ту
сторону великого моря, пока не подойдет к потоку
Гигон, [...], направо от рая Едем. И эта страна,
которая досталась Хаму как наследственная часть,
которой он должен владеть вовек,~--- ему и его
сыновьям в их родах вовек.

И Иафету вышла третья наследственная часть, по
ту сторону реки Тины, к северным странам истока
ее воды, и идет к северо-востоку вся область Лага
и все восточные страны ее; и идет на крайний
север, и простирается до гор Кильта к северу, и к
морю Маук, и идет на восток Гадира, до берегов
моря; и направляется, пока не подойдет к западу
Пары, и обращается назад к Аферагу, и
направляется к востоку, к воде моря Миот, и
направляется вдоль реки Тины, к востоку севера,
пока не подойдет к границе ее воды, к горе Рафы, и
обходит кругом к северу. Это страна, доставшаяся
Иафету и его сыновьям как его наследие, которым
он должен владеть вовек,~--- ему и его сыновьям в их
родах до века: пять великих островов и великая
страна на севере, только она холодная, а страна
Хама жаркая. Но земля Сима не имеет ни жары, ни
мороза, а в ней холод и тепло смешаны.

\vs Jub 9:1
И Хам разделил свою часть между своими
сыновьями. И вышла первая наследственная часть
для всех к востоку и западу Фуду, и запад ее
Ханаану, и к западу моря. И Сим также разделил
между сыновьями. И вышла первая наследственная
часть Еламу и его сыновьям, к востоку от реки
Тигр, пока не подойдешь к стране востока, вся
страна Индия и страна при Эритрейском море, и
воды Дудина, и все горы и Ила (Ела), и вся страна
Сузан, и все, что находится к стороне Фарнака, до
Эритрейского моря и до реки Тины. И Ассуру вышла
вторая наследственная часть: страна Ассур, и
Ниневе, и Синаар, и до границ Индии, и она идет
вверх к реке. И Арфаскаду вышла третья
наследственная часть: вся страна владения
Халдеев, к востоку от Евфрата, вблизи
Эритрейского моря, и все воды пустыни, пока не
придешь к морскому заливу, который смотрит к
Египту, вся страна Либаноса, и Сапера, и Амано, до
соседства с Евфратом. И Араму вышла четвертая
наследственная часть: вся страна Месопотамия,
между Тигром и Евфратом, на север от Халдеев, пока
не придешь к горному хребту Ассур, и все
отдельные страны, до великого моря, и
приближается к востоку к своему брату Ассуру.

И Иафет также разделил страну наследия между
своими сыновьями. И вышел первый жребий Гомеру к
востоку, от севера до реки Тины. И на севере
Магогу досталась вся внутренность севера, пока
не придешь к морю Миот. И Мадаю вышел удел, чтобы
он владел им, на запад от обоих его братьев, до
островов и до границ островов. И Ийоайону вышел
четвертый удел~--- весь остров и острова к Адлуду. И
Толбелу вышел пятый удел, между перешейком,
который подходит к Уда, уделу Луда, до другого
перешейка, внутрь в третий перешеек. И Месеку
вышел шестой удел, и все по ту сторону третьего
перешейка, пока не придешь к востоку Гадира. И
Терасу вышел седьмой удел: он имел великие
острова в средине моря, которые принадлежали к
наследию Хама, и острова Каматури. И детям
Арфаскада вышло блаженное наследие.

Так разделили дети Ноя уделы своим сыновьям
пред Ноем, своим отцом, и он велел им поклясться,
заклиная клятвою каждого, который пытался бы
получить удел, не доставшийся ему по жребию. И все
сказали: <<Да будет так!>> И да будет это так
для них и их сыновей до века, в их родах, до дня
суда, в который Господь Бог будет судить их мечом
и огнем за все лукавство и нечистоту их деяний,
так как они наполнили землю злодеянием,
нечестием, блудодеянием и грехом.

\vs Jub 10:1
И в третью седмину этого юбилея начали нечистые
демоны обольщать сыновей Ноя, чтобы ослеплять их
и губить. И дети Ноя пришли к своему отцу и
рассказали ему о демонах, которые соблазняют
сыновей их сыновей, ослепляют и умерщвляют их. И
он молился Господу Богу своему и сказал: <<Боже
духов всякой плоти, являющий Свое милосердие, и
спасший меня и моих детей от воды потопа, и не
допустивший меня погибнуть, как поступил Ты с
сынами погибели, ибо велика милость Твоя ко мне и
велико Твое милосердие к моей душе: яви милость
Твою на сынах Твоих, не допусти злых духов
господствовать над ними, дабы они не истребили их
от земли! Вот Ты благословил меня и моих сыновей,
чтобы мы возрастали, и умножались, и наполняли
землю. Ты знаешь, как Твои стражи, отцы этих духов,
поступили в мои дни. И этих духов, которые живы,
также заключи и свяжи в месте осуждения, чтобы
они не производили развращения между сынами
Твоего раба, Боже мой, ибо они злобны и созданы на
погибель! Не допусти их господствовать над
духами живущих, ибо Ты один знаешь суд их; и не
допусти их иметь власть над детьми
справедливости отныне и до века!>>

И Бог наш сказал нам, чтобы мы связали всех.
Тогда пришел высший из духов Мастема и сказал:
<<Господи, нельзя ли некоторым из них остаться у
меня, чтобы они слушались моего голоса и делали
все, что я скажу им? Ибо если ни одного из них не
останется у меня, то я не могу являть могущества
своей воли над сынами человеческими; ибо они
существуют для того, чтоб развращать и обольщать
по моему повелению под моим господством, так как
злоба людей велика>>. И он сказал: <<Десятая
часть их пусть останется у меня, и девять частей
пусть сойдут в место суда!>> И один из нас
сказал: <<Мы научим Ноя всем целебным
средствам>>; ибо он знал, что они ходят не в
справедливости, и будут вести борьбу не в
праведности. И мы сделали по Его повелению: всех
злых, лютых духов мы связали в месте
наказания, и десятую часть из них мы оставили,
чтобы они предстали пред Сатаною на земле. И
целебные средства от всех их (т.е. причиняемых
демонами) болезней вместе с их способами
обольщения мы сказали Ною, как излечивать себя
растениями земли. И Ной записал все, как мы
научили его, в книгу, о каждом роде лекарств. Так
злые духи были отделены в заключение от детей
Ноя.

И он дал все писания, которые написал, своему
старейшему сыну Симу, ибо он любил его больше из
всех своих сыновей. И Ной почил с своими отцами и
был погребен на горе Лубар в земле Арарат.
Девятьсот пятьдесят лет он окончил в своей жизни,
девятнадцать юбилеев, две седмины, пять лет. И его
жизнь на земле была знаменитее, чем жизнь остальных сынов человеческих,
ради его справедливости, в которой он усовершился, кроме только Еноха; ибо
история Еноха была предназначена во свидетельство для родов вечности, чтобы
показать все, что случится с родами родов до дня суда.

И в тридцать четвертый юбилей, в первый год
второй седмины, Фалек взял себе жену по имени
Ломна, дочь Синаара. И она родила ему сына в
четвертый год этой седмины, и он нарек ему имя
Рагев, ибо сказал: <<Вот сыны человеческие
сделались дурными через гнусный замысел, что они
построят себе город и башню в земле Синаар, ибо
они переселились от Арарата к востоку в
Синаар>>. Ибо в его дни они построили город и
башню, говоря: <<Мы поднимемся по ней на небо>>.
И они начали строить в четвертую седмину, и
обжигали огнем (кирпичи), и кирпичи служили им
вместо камня, и цементом, которым они укрепляли
промежутки, был асфальт из моря и из водных
источников в стране Синаар. И они строили это в
продолжение сорока трех лет. И Господь Бог наш
сказал нам: <<Вот, это один народ, и он начал
делать это! И ныне Я не отступлю от них! Вот, мы
сойдем и смешаем языки их, чтобы они не понимали
друг друга и рассеялись в страны и народы, и да не
осуществится никогда их замысел до дня суда!>> И
Господь сошел, и мы сошли с Ним, видеть город и
башню, которую строили сыны человеческие; и Он
расторг каждое слово их языка, и никто уже не
понимал слово другого. И вот они отказались
строить город и башню. Ради сего вся страна
Синаар была названа Бабель (Вавилон). Ибо так
расторг Бог все языки сынов человеческих; и
оттуда они рассеялись в свои города по их языкам
и народам. И Бог послал сильный ветер на их башню
и поверг ее на землю. И вот она стояла между
страной Ассур и Вавилоном в земле Синаар; и
нарекли ей имя развалины.

В первый год четвертой седмины тридцать пятого
юбилея они рассеялись в стране Синаар. И Хам с
своими сыновьями ушел в страну, которая стала его
собственностью и которая досталась ему при
разделе, в страну юга. А Ханаан увидел страну
Либаноса, до ручья Египетского, что она очень
хороша, и пошел не в страну своего наследия, на
запад от моря, но жил в стране Либанос, на востоке
и на западе от сынов народа Либаноса и вдоль моря.
И отец его Хам, и Куш, и Мицраим, его братья,
сказали ему: <<Ты поселился в стране, которая не
принадлежит тебе и которая по жребию не
досталась нам. Ты не должен так делать. Ибо если
ты сделаешь так, то погибнешь, так же как и твои
сыновья, в стране, как подвергшийся проклятию,
силой оружия; ибо вы силой оружия поселились, и
силой оружия падут твои сыновья, и ты будешь
истреблен вовек. Не живи в месте обитания Сима,
ибо оно досталось по жребию Симу и его детям.
Проклят ты и проклят будешь пред всеми сыновьями
Ноя проклятием, которым мы обязались в клятве
пред святым Судиею и пред нашим отцом Ноем>>. Но
он не послушал их и жил в стране Либанос, от
Гамафа до начала Египта, он и его сыновья до
нынешнего дня. И посему та страна была названа
Ханаан. Но Иафет и его сыновья пошли на запад и
жили в стране своего наследия. И Мадай увидел
страну моря, и она понравилась ему, и он выпросил
ее себе у Елама и Ассура и Арфаскада, брата его
жены, и жил в стране Мидакин (Мидийской стране)
вблизи брата своей жены до сего дня; и он назвал
свое место обитания и место обитания своих детей Медекин, по имени их отца
Мадая.

\vs Jub 11:1
И в тридцать пятый юбилей в третью седмину в
первый год ее Рагев взял жену по имени Ара, дочь
сына Кеседа. И она родила ему сына, и он нарек ему
имя Серуг в седьмой год этой седмины и этого
юбилея. И сыны Ноя начали вести борьбу друг с
другом; они стали друг друга брать и убивать,
проливать кровь человеческую на земле, и есть
кровь, и строить укрепленные города, и стены и
башни, и помимо того возноситься над народом, и
повсюду основывать царство, и вести войну~--- один
народ против другого, и народы против народов, и
город против города, и подвергать все порче, и
делать себе оружие, и учить своих детей войне. И
они начали покорять города, и продавать
невольников и невольниц.

И Ур, сын Кеседа, построил город Ару Халдейскую
и назвал его по имени себя и по имени отца своего.
И он делал им звезды и поклонялся каждому идолу,
которого он лил себе. И они начали делать
изваяния, и статуи, и нечистое, и духи нечистые
помогали в этом и обольщали их совершать грех и
нечистоту. И князь Мастема прилагал свою власть,
чтобы делали это, и побуждал чрез духов, которые
были отданы в его руки, совершать различного рода
злодеяния, и грехи и всякое развращение, чтобы
развращать, и губить, и проливать на земле кровь.
Посему ему было наречено имя Серух, ибо он
удалился, чтобы свободнее совершать грех и
злодеяние. И он сделался великим, и жил в Уре
Халдейском, вблизи родителей (своей матери), и
поклонялся идолам. И он взял себе жену в тридцать
шестой юбилей, в пятую седмину, в первый год, по
имени Мелка, дочь Кгебера, дочь (сестры) его отца.
И она родила ему Накгора в первый год этой
седмины; и он возрос и жил в Уре, в Уре Халдейском.
И его отец, мудрец Халдейский, научил его
предсказанию и гаданию по знамениям неба.

И в тридцать седьмой юбилей, в шестую седмину, в
первый год взял он себе жену по имени Ийосака,
дочь Нестега Халдейского, и она родила ему сына
Фарага в седьмой год этой седмины. И князь
Мастема послал воронов и птиц, чтобы они пожирали
семя, посеянное на земле, чтобы произвести порчу
на земле, чтобы они расхищали у сынов
человеческих их произведения. Ибо прежде чем они
запахивали семя, вороны подбирали его с
поверхности земли. Посему он нарек ему имя Фараг,
так как вороны и птицы обкрадывали их и пожирали
у них семя их. И годы стали делаться неурожайными
от птиц; и все древесные плоды они пожирали с
деревьев [...]. Только с великим трудом можно было в
их дни спасти кое-что от всех плодов земли.

И в тридцать девятый юбилей во вторую седмину, в
первый год, взял себе Фараг жену, по имени Една,
дочь Арема, сестрину дочь его отца, себе в жены. И
в седьмой год этой седмины она родила сына, и он
нарек ему имя Аврам, по имени отца его матери; ибо
он умер, прежде чем приобретен был ее и его сын. И
дитя начало замечать греховность земли, как она
была соблазнена к греху чрез изваяния и
нечистоту. И его отец научил его писать. И когда
он был двух седмин, то отдалился от своего отца,
чтобы не поклоняться вместе с ним идолам. И он
начал молиться Творцу всех вещей, чтобы Он спас
его от обольщения сынов человеческих и чтобы его
наследие, после того как он стал праведным, не
впало в греховность и нечестие.

И пришло время посева для тех, кто засевает
землю. И они вышли все вместе, чтобы стеречь свои
семена от воронов. И Аврам вышел с другими, будучи
дитею четырнадцати лет. И налетело облако (стая)
воронов, чтобы пожирать семена. Но Аврам побежал
к ним, прежде чем они сели на землю, и закричал на
них, прежде нежели они сели на землю, чтобы
пожирать семена, и сказал: <<Не смейте
спускаться, воротитесь в то место, откуда
прилетели!>> И они воротились. И они (?) сделали
так в тот день с семью стаями воронов. И из всех
воронов ни один не сел где-либо на пашню, где был
сам Аврам,~--- даже ни один. И все, бывшие около него
на той пашне, видели, как он закричал и сказал:
<<Воротитесь, вороны!>> И его имя сделалось
великим во всей стране Халдейской. К нему
приходили в этот год все, которые сеяли; и он
ходил с ними, пока не прошло время сеяния. И они
засеяли свою землю и собрали в том году хлеб, так
что ели и были сыты.

И в первый год пятой седмины Аврам научил тех,
которые делают воловью упряжь,~--- плотников, и они
сделали прибор над землею против деревянной дуги
плуга, чтобы класть на него семена и выбрасывать
их оттуда в семенную борозду, чтобы они
скрывались в земле. И они не боялись более
воронов и делали так у всех дуг плугов нечто над
землею. И они засеяли и обработали всю страну
вполне так, как им велел Аврам; и они не боялись
более воронов.

\vs Jub 12:1
И случилось в шестую седмину в седьмой год ее,
сказал Аврам отцу своему Фарагу, говоря: <<Отец,
отец мой!>> И он сказал: <<Вот я здесь, мой
сын!>> И он сказал: <<Что нам за помощь и
услаждение от всех идолов [...], что ты
поклоняешься им? Ибо в них совсем нет духа (души);
но они, которых вы почитаете, суть проклятие и
соблазн сердца. Почитайте Бога небесного,
Который низводит на землю дождь и росу, и все
совершает на земле, и все сотворил Своим словом, и
вся жизнь пред Его лицем! Зачем вы почитаете тех,
которые не имеют духа? ибо они нечто сделанное, и
на своих плечах вы носите их, и не имеете от них
никакой помощи, но они служат великим поношением
для тех, которые делают их к соблазну сердца и
почитают их. Не почитайте их!>> И отец его сказал
ему: <<И я знаю это, сын мой. Но что я сделаю с
моим родством, которое заставило меня служить им?
Если я скажу им истину, то они убьют меня, ибо их
душа прилепилась к ним, чтобы почитать и
прославлять их. Молчи, сын мой, чтобы они не убили
тебя!>> И он сказал эту речь своим двум братьям,
и они разгневались на него. Тогда он замолчал.

И в сороковой юбилей во вторую седмину в
седьмой год ее Аврам взял жену по имени Сора, дочь
его отца (?), и она сделалась его женою. Аран, брат
его, взял себе жену в [...] год третьей седмины,
и она родила ему сына в седьмой год этой седмины;
и он нарек ему имя Лот. И его брат Накгор также
взял себе жену.

И (в шестидесятый) год жизни Аврама, т.е. в
четвертый год четвертой седмины, встал Аврам
ночью, и сожег капище идолов и все, что было в нем,
так что люди ничего не знали об этом. И они встали
ночью и хотели спасти своих идолов из огня. И Аран
поспешил сюда, чтобы спасти их; тогда пламя
бросилось на него, и он сгорел в огне и умер в Уре.
Халдейском, прежде своего отца Фарага; и они
погребли его в Уре Халдейском.

И Фараг вышел из Ура Халдейского, он и дети его,
чтобы идти в страну Либаноса и в страну Ханаан; и
он жил в стране Харран . И Аврам жил со своим отцом
Фарагом в Харране две седмины.

И в шестую седмину в пятый год ее встал Аврам и
сидел в течение ночи, в новолуние седьмого
месяца, чтобы наблюдать звезды, от вечера до утра,
чтобы видеть, что будет с погодою в этот год. И он
был один, когда сидел и наблюдал. И пришло на его
мысль слово, и он сказал: <<Все знамения звезд и
знамения солнца и луны в руке Господа. Зачем мне
исследовать их? Когда Он хочет, то посылает дождь
рано и поздно, и когда хочет, то изливает потоки
(дождя), и все в Его руке>>. И он молился в эту
ночь и сказал: <<Боже мой, Боже мой! Ты всевышний
Бог, Ты один только Бог мой, и Ты все сотворил и
все есть дело рук Твоих; и Тебя, Твое божество
избрал я. Спаси меня от руки злых духов, которые
сильны над помышлениями человеческого сердца,
чтобы они не отвратили меня от Тебя, Боже мой! И
соделай, чтобы я и мое семя вовек не отвращались от
Тебя, отныне и до века!>> И он сказал:
<<Возвратиться ли мне в Ур Халдеев, которые
ищут моего лица, чтобы я возвратился к ним, или
оставаться мне здесь в этом месте? Укажи рабу
Твоему правый путь пред Тобою, чтобы исполнять
его, и чтобы я не ходил в обольщении моего сердца,
Боже мой!>> И когда он окончил речь и молитву,
вот тогда было послано чрез меня слово Господа к
нему, говоря: <<Поднимись из земли твоей и из
рода твоего и из дома отца твоего в землю, которую
Я тебе покажу! И Я произведу от тебя великий и
бесчисленный народ, и благословлю тебя, и сделаю
твое имя великим. И ты будешь благословлен на
земле, и в тебе благословятся все народы земли;
благословляющих тебя Я благословлю и
проклинающих тебя прокляну; и Я буду Богом тебе, и
твоим сыновьям, и сыновьям сынов твоих, и всему
твоему семени; и буду за тобою, Я Бог твой. Не
бойся, отныне до всех родов земли я Бог твой>>. И
Господь Бог сказал мне: <<Открой его уста, и его
уши, и его губы!>> И я начал говорить по-еврейски
на его коренном языке. И он взял книги своего
отца, которые были написаны по-еврейски, и списал
их. Тогда он начал изучать их, и я объяснял ему
все, чего он не понимал, и он изучал их в
продолжение шести дождливых месяцев.

И был седьмой год шестой седмины. Тогда говорил
он со своим отцом и возвестил ему, что он выйдет
из Харрана, чтобы идти в землю Ханаан, что он
осмотрит ее и возвратится к нему. И отец Фараг
сказал ему: <<Иди в мире, Бог мира да соделает
путь твой правый, и да будет Господь с тобою, и
хранит тебя от всех зол, и да даст тебе милость, и
благоволение, и милосердие пред теми, которые
увидят тебя, чтобы никакой человек не возымел
силы над тобою, чтобы предпринять что-либо против
тебя! Иди в мире! И если ты найдешь страну угодною
очам твоим, чтобы жить там, то возьми и меня с
собою; и возьми с собою Лота, сына Арана, брата
твоего, как своего сына! И Господь да будет с
тобою!>>

\vs Jub 13:1
И Аврам вышел из Харрана и взял с собою жену
свою Сору и Лота, сына брата своего Харрана, в
страну Ханаан. И он пошел [...] и прошел до Сикимона,
близ высокого дуба. И Господь сказал ему: <<Тебе
и твоему семени Я дам эту страну!>> И он устроил
там жертвенник и принес на нем Господу, Который
явился ему, всесожжение, И оттуда он поднялся к
горному хребту Бетель (Вефиль), который был от
него на запад и Ай (Гай) на восток, и разбил там
шатер свой. И он увидел, что земля была очень
обширна и хороша, и что в ней росло все:
виноградные лозы, смоквы, гранаты, дубы, и твердые
деревья, и теревинфы, и масличные деревья, и
кедры, и кипарисы, и ливанские деревья, и все
деревья полевые, и что она имела воду на горах. И
он благословил Господа, Который привел его из Ура
Халдейского на эту гору.

И случилось, в первый год, в седьмую седмину, в
новолуние первого месяца он устроил на этой горе
жертвенник и призвал имя Господа: <<Ты, Боже мой,
вечный Бог>>. И он принес на жертвеннике Божием
всесожжение, чтобы Он был с ним и не оставлял его
в течение его жизни. И он поднялся оттуда и пошел
(на юг), и достиг Хеврона; и Хеврон был тогда
построен. И он оставался там два года [...]. Тогда
пошел Аврам в Египет в третий год седмины, и жил
в Египте пять лет, прежде чем у него была похищена
жена. Санай же был тогда построен в Египте чрез
семь лет после Хеврона. И случилось, когда Фараон
похитил Сору, жену Аврама, Господь поразил
Фараона и весь дом его тяжкими бедствиями за
Сору, жену Аврама. И Аврам был очень обогащен
овцами, и рогатым скотом, и ослами, и конями, и
верблюдами, и рабами, и служанками, и серебром, и
золотом вполне; и Лот, сын его брата, также был
обогащен. И когда Фараон возвратил Сору, его жену,
он переселился из земли Египетской, и пришел в
одно место, на восток от Вефиля, и прославил
Господа Бога своего, который вывел его обратно с
миром.

И случилось, в (сорок первый) юбилей в
третий год первой седмины он возвратился в это
место, и принес там всесожжение, и призвал имя
Господне и сказал: <<Ты Господь, Бог всевышний,
Бог мой вовек!>> И в четвертый год седмины Лот
отделился от него. И Лот жил в Содоме, но жители
содомские были очень злы. И он (Аврам) опечалился
в сердце своем, что его племянник отделился от
него, ибо он не имел детей. В тот год этой седмины,
когда Лот был пленен, Господь говорил Авраму,
после того как Лот отделился от него, и сказал
ему: <<Возведи очи твои от места, где ты живешь, к
(северу), и к югу, и к утру, ибо всю страну,
которую ты видишь, Я дам тебе и дам семени твоему
вовек. И Я сделаю твое семя, как песок при море, (и
как человек не может сосчитать песок при море),
так нельзя исчислить и твоего семени. Встань и
пройди ее по длине и широте, и посмотри все, ибо Я
дам ее твоему семени>>.

И Аврам пошел в Хеврон и жил там. И в этот год
пришли Колодогомер, царь еламский, и Амалфал,
царь синаарский, и Ариох, царь селасарский, и
Тергал, языческий царь; и они поразили царя
Гоморры, и царь Содома бежал, и многие,
обратившиеся в Сиддин, солончатую страну, в
Содом, Адом и Севоим, пали. И они взяли в плен Лота,
племянника Аврама, со всем его имуществом, и
отвели в Дан. И пришел один, который спасся
бегством, и рассказал Авраму, что племянник его
взят в плен [...]. И его раб принес в умилостивление
за Аврама и его семя десятину начатков Господу. И
Господь сделал отсюда постановление навсегда,
чтобы давать ее (десятину) священникам, которые
служат пред Его лицем, дабы они пользовались ею
вовек. И это установление не на день, но Он
утвердил его на вечные роды, чтобы давать Господу
десятину от семян, и вина, и масла, и рогатого
скота, и овец. И Он дал ее Своим священникам, чтобы
они ели от нее с радостью и пили пред Ним.

И вышел к нему царь содомский, и пал пред ним
ниц, и сказал: <<Господин мой Аврам, отдай мне
людей, которых ты освободил; добыча же пусть
будет твоя!>> И Аврам сказал ему: <<Я воздвигаю
руки мои к всевышнему Богу: ни нитки, ни
башмачного ремня я не возьму из всего, что
принадлежит тебе, дабы ты не сказал: <<Я сделал
Аврама богатым>>, кроме того, что съели отроки. И
мужи, ходившие со мною, Аунан, и Ескол, и Мамре,
должны взять свою долю>>.

\vs Jub 14:1
И после сего события, в четвертый год этой
седмины, в новолуние третьего месяца, было слово
Господне к Авраму в сновидении, говорящее: <<Не
бойся, Аврам, Я твоя защита, и награда твоя будет
чрезмерна>>. И он сказал: <<Господи, Господи,
что Ты дашь мне? Вот я иду туда без детей, и сын
Месек мой раб тот Дамаск Елиезер, будет
наследником мне; а мне Ты не дал семени>>. И Он
сказал ему: <<Он не наследит тебе, но
происшедший от плоти твоей будет тебе
наследником>>. И Он вывел его и сказал ему:
<<Взгляни на небо и сосчитай звезды небесные:
можешь ли ты сосчитать их?>> И он взглянул на
небо и увидел звезды. И Он сказал ему: <<Так
будет твое семя>>. И он поверил Господу, и это
было вменено ему в праведность. И Он сказал: <<Я
Господь Бог твой, выведший тебя из Ура Халдейского,
чтобы дать тебе в вечное владение землю
Ханаанитов, и чтобы Я был твоим Богом и Богом
твоего семени>>. И он сказал: <<Господи,
Господи!>> И он сказал: <<Господи, по чему я
узнаю, что наследую ее?>> И Он сказал ему:
<<Принеси Мне трехлетнюю телицу, и трехлетнюю
козу, и трехлетнюю овцу, и трехлетнюю горлицу, и
голубя>>. И он взял все это в средине месяца. И он
жил при дубе Мамре, который близ Хеврона. Там
устроил он жертвенник, и заколол все, возлил
кровь их на жертвенник, и разделил их пополам, и
положил их друг против друга; но птиц он не
касался. И птицы спустились на куски, но Аврам
отгонял их, и не давал птицам прикасаться к ним. И
было, когда солнце зашло, бессилие напало на
Аврама, и вот сильный страх мрака напал на него. И
было сказано: <<Аврам, знай, что твое семя будет
странником в чужой земле и его будут порабощать и
угнетать в продолжение четырехсот лет. Но Я
произведу суд над народом, которому они будут
служить; после того они выйдут оттуда с большим
имуществом. И ты в мире отойдешь к своим отцам, и
будешь погребен в доброй старости. И в четвертом
роде оно (твое семя) возвратится сюда, ибо грех
Аморреев доселе еще не наполнился>>.

И он пробудился от своего сна и встал, и солнце
было зашедшим. Тогда появилось пламя, и вот~--- печь
дымилась, и огненное пламя прошло между кусками.
И в ту ночь Бог заключил завет с Аврамом, сказав:
<<Твоему семени Я отдам эту землю, от реки
Египетской до великой реки, реки Евфрат,~--- Кенеев,
Кенезеев, Ферезеев, Рафейн, [...], Евеев, Аморреев,
Канаанеев, Гергесеев>>. И Он отошел. И Аврам
принес куски, и птиц, и жертву плодовую, и жертву
возлияния, которые принадлежали к сему, и огонь
пожрал их.

И в эту ночь Он заключил завет с Аврамом,
согласно завету, который мы заключили в этом
месяце с Ноем. И Аврам возобновил его в праздник и
в постановление для себя, до века. И Аврам
возрадовался и рассказал все это происшествие
своей жене Соре. И он поверил, что у него будет
семя, но она не рождала. Тогда Сора посоветовала
своему мужу Авраму и сказала ему: <<Войди к моей
служанке Агари, египтянке; быть может, я
произведу тебе от нее семя>>. И Аврам послушался
голоса жены своей Соры и сказал ей: <<Сделай
это>>. Тогда Сора взяла египетскую служанку
Агарь и дала ее своему мужу Авраму, чтобы она была
его женою. И он вошел к ней, и она сделалась
беременной, и родила сына, и он нарек ему имя
Измаил в пятый год этой седмины. В том году был
восемьдесят шестой год жизни Аврама.

\vs Jub 15:1
И в пятый год четвертой седмины этого юбилея, в
третий месяц, в средине месяца, Аврам праздновал
праздник начатков жатвы хлеба и принес свежую
хлебную жертву; к жертвам начатков хлеба для
Господа (он присоединил) тельца, и овна, и овцу на
жертвенник вместе с благовонным курением. И
Господь явился ему и сказал Авраму: <<Я~--- Бог
Владыка, благоугождай предо Мною и будь
благочестив. И Я заключу завет между Мною и тобою
и сделаю тебя весьма великим>>. И Аврам пал на
свое лице. И Господь говорил с ним и сказал:
<<Вот завет Мой с тобою, и Я сделаю тебя отцом
многих народов, и ты не будешь более называться
Аврам отныне до века; но Авраам будет тебе имя, ибо
Я сделал тебя отцом многих народов, и сделаю тебя
весьма великим, и произведу от тебя народов и
царей. И я поставлю завет Мой между тобою и Мною, и
между твоим семенем после тебя, в их родах, в
вечное установление, чтобы Я был твоим Богом и
Богом твоего семени после тебя во всех родах. И
Я дам тебе и семени твоему после тебя землю~---
ибо ты пришлец в ней~--- землю Ханаанскую, чтобы ты
был господином над нею навсегда. И Я буду им
Богом>>. И Господь сказал Аврааму: <<И храни
Мой завет ты и твое семя после тебя, и обрезывайте
все ваши крайние плоти. И это будет знамением
Моего вечного установления между Мною и тобою и
для родов (потомков). В осьмой день вы должны
обрезывать все мужеское, в ваших родах,
рожденного дома и купленного вами за золото у
всех сыновей чужеземцев, что приобрели вы. Кто от
твоего семени, тот да будет обрезан, рожденный
дома и купленный за золото да будет обрезан. И Мой
завет на теле вашем пусть будет в вечное
установление.

И кто не обрезан, всякий мужеского пола между
вами, крайняя плоть которого не обрезана в
восьмой день, душа та да истребится из рода
вашего, ибо она нарушила завет Мой>>. И Господь
сказал Аврааму: <<Сора, жена твоя, не будет более
называться Сорою, но Сара~--- имя ее; и Я благословлю
ее, и дам тебе от нее сына; и Я благословлю его, и
произведу от него народ, и цари над народами
произойдут от него>>.

И Авраам пал на лице свое, и возрадовался, и
сказал в сердце своем: <<У меня ли, имеющего сто
лет, родится сын, и Сара девяноста лет родит ли
сына?>> И Авраам сказал Господу: <<Хотя бы
Измаил остался жив пред Тобою!>> И Господь
сказал: <<Да! но и Сара родит тебе сына, и ты
наречешь ему имя Исаак. И Я восстановлю завет Мой
с ними, завет Мой вечный, и с его семенем после
него. И о Измаиле Я услышал тебя, и вот я
благословлю и умножу его, и сделаю его весьма
многочисленным. И двенадцать царей произведет
он; и Я произведу от него великий народ; но завет
Мой Я поставлю с Исааком, которого родит тебе
Сара около сего времени на другой год>>. И после
того, как Он кончил говорить с ним, Господь
восшел.

И он (Авраам) взял своего сына Измаила и всех
своих рожденных дома (рабов) и купленных за
золото, весь мужеский пол, который был в его доме,
и обрезал плоть их члена. И в этот день был
обрезан Авраам, и люди его дома были обрезаны, и
также все, которых он купил за золото у сынов
иноплеменников, были обрезаны вместе с ним. И
этот закон~--- для всех родов вовек. И нельзя
изменять дней, ни пропускать одного из восьми
дней, ибо это вечное благословение, утвержденное
и записанное на небесных скрижалях. И каждый
рожденный, крайняя плоть которого не обрезана до
восьмого дня, не принадлежит к сынам завета,
который Господь заключил с Авраамом, но к сынам
погибели, и вот он не имеет знака на себе, что он
Господень; он предназначен к погибели, и
уничтожению, и истреблению от земли, ибо он
нарушил завет Господа нашего Бога. Ибо Он освятил
Израиля, чтобы он был со всеми Его Ангелами лица,
и со всеми Ангелами прославления, и со святыми
Его Ангелами. И ты повели также сынам Израиля,
чтобы они хранили знак сего завета в своих родах,
как вечное установление, чтобы не быть им
истребленными от земли. Ибо постановление cue
утверждено для завета, чтобы оно соблюдалось
навсегда между всеми сынами Израиля. Ибо Измаила,
и сыновей его и братьев, и Исава не приблизил
Господь и не избрал их; но сынов Авраама познал Он
и избрал Израиля, чтобы они были Его народом, и
освятил его, и собрал его из всех сынов
человеческих. Ибо много народов, и бесчисленны
люди, и все принадлежат Ему, и над всеми Он
поставил духов вместо Господа, чтобы они
отвращали их от Него. Над Израилем же Он никого не
поставил господом~--- ни Ангела, ни духа, но Он
единый их Владыка, и Он охраняет их и ведет тяжбы
их против Своих Ангелов, и Своих духов, и против
всего. И если они будут хранить все Его повеления,
то Он благословит их, и они будут Его сынами, и Он
будет их Отцом отныне до века. И теперь я
предсказываю тебе, что сыны Израиля будут
поступать вопреки этому установлению, и их сыны
не будут обрезываться согласно всему этому
закону. Ибо на плоти своего обрезания они не
будут совершать оного обрезания своих сыновей, и
они все, сыны Велиара, будут оставлять своих
сыновей необрезанными, как они родились. И гнев
Господа на детей Израиля будет велик, ибо они
оставили завет Его, и уклонились от Его слова, и
возбудили Его на гнев, и восхулили Его, и не
сделали сего знака по их закону, но оставили свою
плоть необрезанною подобно язычникам, чтобы
быть уничтоженными и истребленными с земли. И они
впредь не обретут прощения и помилования, чтобы
быть прощенными и помилованными во всех своих
грехах за сие отступление вовек.

\vs Jub 16:1
И в новолуние четвертого месяца явились мы
Аврааму при дубе Мамврийском и беседовали с ним.
И мы также возвестили ему, что у него родится сын
от жены его Сары. Тогда Сара рассмеялась, ибо она
слышала, что мы говорили эту речь Аврааму. И мы
заметили ей; но она испугалась и стала отрицать,
что она смеялась над нашими словами. И мы
сказали ей имя его сына, как определено и
написано было на небесных скрижалях, именно
Исаак. И когда мы возвратимся к ней в
определенное время, тогда она будет беременной
сыном.

И в этот месяц Господь совершил суд над Содомом,
и Гоморрою, и Севоимом, и всею страною Иорданскою,
и сожег их огнем и серой, и предал их погибели до
сего дня; согласно тому, как мы рассказывали тебе
о всех их делах, что они были гнусными и весьма
греховными и что они осквернялись, и
блудодействовали, и делали мерзость на земле~---
согласно сему Бог совершил суд; во гневе и ярости
за нечистоту Содома совершил Он суд над Содомом.
И мы спасли Лота, ибо Господь вспомнил об Аврааме
и вывел его (Лота) из разрушения. Но и он, и дочери
его совершили на земле грех, какого не было на
земле от Адама до того времени; ибо муж переспал с
своею дочерью. И вот, относительно всего его
семени определено и начертано на скрижалях,
чтобы уничтожить и истребить его, и совершить суд
над ним, как над Содомом, и не оставить ему семени
на земле ко дню осуждения.

И в этот месяц поднялся Авраам от Хеврона и
пошел и жил между Кадетом и Суром на горах
Герарона. И в средине пятого месяца он поднялся
оттуда и жил при клятвенном колодезе. И в средине
шестого месяца Господь посетил Сару, и сотворил
ей, как сказал, и она сделалась беременною. И она
родила ему сына в третий месяц, в средине месяца,
как сказал Бог Аврааму. В праздник начатков жатвы
родился Исаак, и Авраам обрезал своего сына в
восьмой день. Он первый был обрезан согласно
завету, как определено навечно.

И в шестой год четвертой седмины пришли мы к
Аврааму к клятвенному колодезю и явились ему, как
сказали Саре, что придем к ней. А она сделалась
беременною сыном, и мы возвратились в седьмой
месяц, и нашли Сару беременною пред нами, и
благословили Сару, и рассказали Саре все, что
было повелено нам относительно него (т.е.
Авраама), что он не умрет, пока не родит шесть
сыновей, и что он увидит их, прежде чем умрет, но
что в Исааке будет наречено имя его и семя, и что
все семя его сыновей будет язычниками и
причтется к язычникам; но только семя от сыновей
Исаака будет святым, и не причтется к язычникам;
ибо оно будет наследием Всевышнего, и все его
семя будет между теми, которые почитают Бога,
чтобы быть для Господа драгоценным украшением
пред всеми народами и быть царством и народом
святым. И мы прошли наш путь, и передали Саре все,
что мы сказали ему (Аврааму). И они оба друг с
другом были в великой радости. И он устроил там
жертвенник Господу, Который спас его и
возвеселил его в стране его странствования, и
праздновал торжество в этом месяце в течение
семи дней близ жертвенника, который он устроил
при клятвенном колодезе, и устроил кущи для себя
и своих рабов к этому празднику. И он праздновал этот
праздник в первый раз на земле; и в эти семь
дней он приносил каждодневно на жертвеннике
Господу всесожжение: семь волов, двух молодых
козлов, двух овнов, семь овец; одного козла в
жертву за грех, чтобы искупить ею себя и свое
семя; и в жертву благодарения семь овнов, семь
молодых козлов, семь овец, семь тельцов вместе с
плодовою жертвою и возлиянием, которые
относились к сему. Над всем их туком он воскурял
на жертвеннике избранное всесожжение в приятное
благовоние. Утром и вечером он воскурял ладан, и
халван, и стакти, и нард, и мирру, и Сенегал и кост;
все эти семь веществ он приносил
истолченными, смешанными между собою по равной
части и очищенными. И он праздновал этот праздник
в течение семи дней, радуясь в своем сердце и всею
душою,~--- он и все, бывшие в его доме; и ни одного
чужеземца не было с ним, и ни одного
незаконнорожденного. И он прославлял своего
Творца, Который создал его в его роде, ибо Он по
Своему благоволению создал его. Ибо он знал и
уразумел, что от него придет растение
праведности для будущих родов и что равным
образом от Него придет святое семя, от Него,
который все создал. И он прославил Его, и нарек
имя этому празднику~--- праздник Господень, и
радовался радостию, которая была приятна
Всевышнему Богу. И мы благословили его вовек и
все его семя после него на все роды земли, ибо он
праздновал тогда этот праздник по свидетельству
небесных скрижалей. Посему на небесных скрижалях
определено для Израиля, чтобы они праздновали
праздник кущей в течение семи дней с радостию, в
седьмой месяц, дабы это было приятно Господу, в
вечный закон для родов их, на все века и годы; и
нет для сего установления конца дней, но
навек определено относительно Израиля, чтобы они
праздновали его, и жили в кущах, и полагали венки
на свои головы. И как они берут от ручья покрытую
листьями ивовую ветвь, так брал и Авраам сережки
от пальмовых ветвей и хорошие древесные плоды, и
обходил каждый день с ветвями вокруг жертвенника
семь раз в день, и утром он восхвалял и благодарил
Бога своего за все с радостию.

\vs Jub 17:1
И в первый год пятой седмины этого юбилея Исаак
был отнят от груди, и Авраам сделал большой пир на
третий месяц, в день, когда сын его Исаак был
отнят от груди. И Измаил, сын египтянки Агари, был
пред лицем отца своего Авраама на своем месте. И
Авраам радовался и прославлял Бога, что он увидел
от себя сыновей и не умер без сыновей. И он
вспомнил слово, как Он говорил с ним в тот день,
когда Лот отделился от него. И он радовался, что
Бог дал ему семя на земле, чтобы получить в
наследие страну. И он прославил громким голосом
Творца всех вещей. И когда Сара увидела Измаила,
как он был весел и плясал и что даже Авраам
радовался при этом, то почувствовала зависть при
взгляде на Измаила и сказала Аврааму: <<Выгони
эту служанку и ее сына; сын этой служанки не
должен наследовать с моим сыном Исааком>>. И это
показалось неприятным Аврааму ради его служанки
и его сына, что он должен выгнать их от себя. И
Господь сказал Аврааму: <<Не нужно тебе
огорчаться из-за отрока и рабыни; все, что сказала
тебе Сара, послушайся ее слова и исполни его, ибо
в Исааке наречется тебе имя и семя. Сына же этой
рабыни Я сделаю великим народом, ибо он~--- твой
род>>. И Авраам собрался рано утром, взял хлеба и
мех с водою, и положил их на плечи Агари вместе с
отроком, и отослал их. И она пошла, блуждая в
пустыне Вирсавии. И не стало воды в мехе; и отрок
истомился от жажды, и не мог идти и упал. И мать
взяла его, и пошла и бросила под масличное дерево.
И она пошла дальше, и села против него, удалившись
на выстрел из лука, ибо сказала: <<Я не могу
смотреть на смерть моего сына>>. И вот она села и
плакала. Тогда Ангел Божий, один из святых, сказал
ей: <<Что ты плачешь, Агарь? встань, подними
отрока и возьми его своею рукою, ибо Господь
услышал твой голос>>. И когда она увидела
отрока, подняла свои глаза и увидела колодезь с
водою, и пошла туда, наполнила свой мех водою и
напоила свое дитя. И она встала и пошла к Фараону.
И отрок вырос и сделался стрелком из лука, и
Господь был с ним. И мать его взяла ему жену из
дочерей Египетских, и она родила ему сына. И он
нарек ему имя Навайвоф, ибо она сказала: <<Бог
был близ меня, когда я призывала его>>.

И случилось в седьмую седмину в первый год в
первый месяц этого юбилея, в двенадцатый день
сего месяца, были сказаны на небесах некоторые
слова об Аврааме, что он верен во всем, что
Господь говорит ему, и что он любит Его и верен во
всяком искушении. Тогда пришел начальный Мастема
и сказал пред Богом: <<Вот Авраам любит и
дорожит своим сыном Исааком больше всего; скажи
ему, чтобы он принес его во всесожжение на
жертвеннике, и Ты увидишь, исполнит ли он это
повеление, чтобы узнать Тебе, верен ли он во всем,
чем Ты его испытываешь>>. И Бог знал, что Авраам
верен во всех испытаниях, которые Он назначает
ему, ибо Он искушал его царством царей, и затем
женою его, когда она была похищена у него, и далее
Измаилом и Агарью, его служанкою, когда он
отослал их, и во всем, чем Он искушал его, он
оказался верным, и его душа не была мятежною, и не
медлил он исполнять сие, ибо был верен и любил
Бога.

\vs Jub 18:1
И Господь сказал Аврааму: <<Авраам!>> И он
сказал: <<Вот я!>> И Он сказал ему: <<Возьми
возлюбленного твоего сына Исаака, и пойди на
высокую гору, и принеси его в жертву на одной
из гор, которую Я тебе покажу>>. И он собрался
оттуда утром на рассвете, и оседлал свою ослицу, и
взял двух своих рабов с собою и своего сына
Исаака, и наколол дров для жертвы. И он шел к назначенному
месту три дня и увидел то место издали. И он
пришел к колодезю с водою и сказал своим рабам:
<<Останьтесь здесь с ослицею; я и отрок пойдем и,
когда помолимся, возвратимся к вам>>. И он взял
дрова для жертвы и возложил на плечи сыну своему
Исааку, и взял в руки огонь и нож, и они пошли оба
вместе к тому месту. И Исаак сказал своему отцу:
<<Отец!>> И он сказал: <<Вот я, сын мой>>.
И он сказал: <<Вот здесь нож и дрова, где же овца
для всесожжения, отец мой?>> И он сказал:
<<Господь усмотрит себе овцу для всесожжения,
сын мой>>. И он пошел к месту горы Божией, и
устроил жертвенник, и положил дрова на
жертвенник, и поднял сына своего Исаака, и
положил его на дрова на жертвенник, и простер
руку свою взять нож, чтобы заколоть сына своего
Исаака. И я (Ангел) стал пред ним (пред Богом?) и
пред высшим Мастемой. И Господь сказал: <<Скажи
ему, чтобы он не возлагал руки своей на отрока и
не делал ему никакого вреда, ибо Я знаю, что он
богобоязнен>>. И я воззвал к нему с неба и
сказал: <<Авраам, Авраам!>> И он убоялся и
сказал: <<Вот я>>. И Он сказал ему: <<Не
возлагай руки своей на отрока и не делай ему
никакого вреда, ибо теперь Я знаю, что ты
богобоязнен и сам не пожалел твоего
перворожденного сына предо Мною>>. И посрамился
высший Мастема. И Авраам возвел очи свои и увидел,
и вот там был овен, зацепившийся своими рогами. И
Авраам пошел, и взял овна, и принес его во
всесожжение вместо сына своего. И Авраам назвал
то место: <<Господь усмотрел сие>>, так что
говорят: <<Господь усмотрел сие>>, т.е.
гора Сион.

И Господь вторично воззвал Авраама по имени с
неба, как Он возвестил мне, чтобы я говорил с ним
во имя Господа. И Он сказал: <<Моею главою Я
поклялся, говорит Господь: так как ты сделал это,
и твоего перворожденного сына, которого любишь,
ты не пожалел предо Мною, то Я поистине
благословлю тебя и умножу семя твое, как звезды
небесные и как песок на берегу моря. Твое семя
получит в наследие города врагов своих, и
благословятся в семени твоем все народы земли за
то, что ты послушался гласа Моего и показал всем,
что ты верен Мне во всем, что Я возложил на тебя.
Иди в мире!>>

И Авраам пошел к своим рабам, и встали, и пошли
они вместе в Вирсавию, и Авраам жил при
клятвенном колодезе. И он соблюдал сей праздник
ежегодно в течение семи дней с радостию, и назвал
его праздником 1Ъсподним соответственно семи
дням, в продолжение которых он ходил и
возвратился в мире. И так утверждено сие и
записано на небесных скрижалях относительно
Израиля и его семени, чтобы они праздновали этот
праздник в течение семи дней с радостию.

\vs Jub 19:1
И в первый год первой седмины сорок второго
юбилея возвратился Авраам и жил против Хеврона,
т.е. Каръяфарбока. Во вторую седмину в третий год
этого юбилея окончились дни жизни Сары, и она
умерла в Хевроне. И пришел Авраам оплакать и
погребсти ее. И мы испытывали его, покорен ли дух
его и не произнесет ли он устами своими мятежного
слова, но он и здесь оказался покорным и не
возмущался, а с спокойным духом говорил с детьми
Киту (т.е. Хета), чтобы они дали ему место, на
котором он похоронил бы свою умершую. И Господь
наградил его благоволением пред всеми, которые
видели его, и он просил, полный смирения, детей
Хета, и они дали ему землю двойной пещеры против
Мамре, т.е. Хеврона, за сорок серебреников. Но они
просили его, говоря: <<Мы отдадим вам это
даром>>. Но он не взял у них даром, а отдал им
цену за место~--- хорошее серебро, и поклонился им
дважды. И после сего он похоронил свою умершую в
двойной пещере. И всех дней жизни Сары было сто
двадцать семь лет, т.е. два юбилея четыре седмины
и один год. Это годы жизни Сары. И это было десятое
испытание, которым был искушаем Авраам; и он
обнаружил верный и покорный дух. И он не сказал
никакого слова о том, что Бог обещал ему дать
страну ему и его семени после него, но он просил
там только о местах, чтобы похоронить свою
умершую. Так оказался он верным и покорным, и
записан был, как друг Господа, на небесных
скрижалях.

И в четвертый год ее (второй седмины) взял он
сыну своему Исааку жену по имени Ревекка, дочь
Вафуила, сына Нахорова, брата Авраама. И Авраам
взял себе третью жену по имени Кетура, из дочерей
своих домашних рабов; ибо Агарь умерла прежде
Сары; и она родила ему шесть сыновей: Ценбари, и
Якзана, и Мадая, и Ийясбока, и Зигийю.

Во второй год шестой седмины Ревекка родила
Исааку двух сыновей~--- Иакова и Исава. И Иаков был
благочестив, а Исав~--- муж грубый, земледелец и
волосатый; и Иаков жил в шатрах. И юноши подросли:
и Исав научился, так как он был земледельцем и
охотником, войне и всякому грубому затятию. И
Иакова любил Авраам, а Исава Исаак. И Авраам видел
занятие Исава и уразумел, что в Иакове будет
наречено ему имя и семя. И он призвал Ревекку и
дал ей повеление относительно Иакова, ибо он
видел, что она также любила гораздо более Иакова,
нежели Исава. И он сказал ей: <<Дочь моя! береги
сына моего Иакова, ибо он будет вместо меня на
земле в благословление между сынами
человеческими и всему своему семени имя его будет
во славу. Ибо я знаю, что Господь произведет от
него народ и он будет предпочтен пред всеми,
которые на лице земли. И вот, сын мой Исаак любит
Исава более, нежели Иакова, и я вижу, что ты
действительно любишь Иакова. Так сделай ему еще
больше добра, и да будет он твоим возлюбленным сыном,
ибо он будет мне в благословение на земле
отныне до всех родов века. Да укрепятся руки твои
и да возрадуешься ты о сыне твоем Иакове, ибо я
люблю его более всех моих сыновей; ибо он будет
благословен вовек, и семя его наполнит всю землю.
Ибо как не может человек сосчитать пыль земную,
так не может быть исчислено и семя его. И все
благословения, которыми Господь благословил
меня и мое семя, будут также уделом Иакову и его
семени во все дни. И в его семени будет
благословлено мое имя, и имя моих отцов Сима, и
Ноя, и Еноха, и Малалела, и Сифа, и Адама. Да
послужат они к тому, чтобы основать небо, и
утвердить землю, и обновить светила, которые на
тверди небесной>>.

И он призвал Иакова пред очи матери его Ревекки,
и поцеловал его, и благословил его, и сказал:
<<Возлюбленный сын мой Иаков, которого
возлюбила душа моя! Да благословит тебя Бог с
высоты тверди небесной, и да даст тебе все
благословения, которыми Он благословил Адама, и
Еноха, и Ноя, и Сима; и все, что Он говорил со мною,
и все, что он обещал дать только мне, да пошлет Он
на тебя и на твое семя до века, пока небо
существует над землею. И да не владычествуют над
тобою и над твоим семенем духи Мастемы, чтобы
отвращать тебя от Господа, Который есть Бог твой,
отныне до века! И да будет Господь твоим Богом и
твоим отцом, а ты Его первородным сыном и Его
народом во все дни! Иди, сын мой, в мире!>> И они
все (?) вместе вышли от Авраама. И Ревекка любила
Иакова всем сердцем и всею душою и гораздо
больше, чем Исава. И Исаак любил гораздо больше
Исава, нежели Иакова.

\vs Jub 20:1
И в сорок второй юбилей в первый год седьмой
седмины призвал Авраам Измаила и двенадцать его
сыновей, и Исаака и обоих его сыновей, и шесть
сыновей Кетуры и детей их, и заповедал им хранить
пути Господа, чтобы поступали по справедливости
и любили друг друга, чтобы поступали таким же
образом во всякой войне, чтобы против каждого,
кто будет против них, они выходили все вместе, и
совершали правду и справедливость на земле,
чтобы они своих сыновей обрезывали по завету,
который Он заключил с ними, и не уклонялись бы ни
направо, ни налево от всех путей, <<которые
Господь заповедал нам>>, и соблюдали бы себя от
всякой мерзости, и избегали бы всякой мерзости и
блуда. <<И если какая-либо женщина или девица
совершит прелюбодеяние между вами, то сожгите ее
огнем, и не блудите вслед за нею очами и
сердцем>>. И пусть они не берут себе жен из
дочерей Ханаанских, ибо семя Ханаана будет
истреблено на земле. И он говорил им о суде над
исполинами и суде над Содомом, как они были
наказаны за их порочность, и блудодеяние, и
нечистоту, и взаимное развращение. За
блудодеяние погибли они, но вы воздерживайтесь
от всякого любодеяния и мерзости, и от всякого
осквернения грехами и мерзостию их, чтобы вам не
сделать имя наше проклятием и всю жизнь вашу позором,
и не предать бы всех сыновей ваших погибели
от меча, и чтобы проклятие ваше не было как Содом
и остаток ваш как сыны Гоморры. Я свидетельствую
вам, сыны мои: любите Бога небес и покоряйтесь
всем Его заповедям, и не обращайтесь к идолам и
мерзостям их (язычников); и не делайте себе ни
литых идолов, ни изваяний, ибо они ничтожны, и не
имеют души, но они суть дело рук, и все, которые
полагаются на них, не получают помощи,~--- все,
которые положились на них. Не почитайте их и не
поклоняйтесь им, а почитайте Бога Всевышнего, и
поклоняйтесь всегда Ему, и надейтесь на Твое
лице, о Господи, (?) во всякое время, и совершайте
правду, и справедливость, и праведность пред Ним,
чтобы Он имел благоволение к вам, и являл вам Свое
милосердие, и ниспосылал дождь утром и вечером, и
благословлял всякий труд ваш и все, над чем вы
трудитесь на земле; и ваш посев, и твою (?) воду, и
семя твоей плоти, и семя твоей земли, и твои стада
и овец благословит Он, и ты будешь во
благословение на земле, и все народы земли будут
иметь благоволение к вам и благословлять сынов
ваших именем моим, дабы они были благословлены,
как я>>.

И он дал Измаилу и его сыновьям и сыновьям
Кетуры подарки, и отослал их от своего сына
Исаака. И Измаил с сыновьями своими и сыновья
Кетуры с сыновьями их пошли вместе, и жили от
Фармона (вероятно, Фаран), пока не придешь к
Вавилону, во всей области, которая лежит к
востоку против пустыни, И они соединились вместе,
и были названы арабами и измаильтянами.

\vs Jub 21:1
И в шестой год седьмой седмины этого юбилея
Авраам призвал сына своего Исаака и заповедал
ему, говоря: <<Я стар и не знаю, когда умру, ибо я
пресытился днями своими. И вот мне сто семьдесят
пять лет, и в продолжение всей жизни моей я
помышлял о Господе, и от всего сердца стремился
исполнять волю Бога моего и ходить право по всем
Его путям. Идолов ненавидела душа моя, дабы быть
внимательным к исполнению воли Того, Кто
сотворил меня; ибо Он~--- Бог живый, и свят, и верен, и
праведен во всем, и нет неправды в Нем, чтобы
взирать на лице и принимать дары; но Он есть Бог
правды, совершающий наказание над всеми, которые
преступают его заповеди и нарушают Его завет. И
ты также, сын мой, соблюдай Его заповеди, Его
установление и правду, и не ходите (?) вслед за
мерзостию язычников, и изваяниями и литыми
изображениями, и не ешьте крови ни<b> </b>зверей, ни
скота, ни различных птиц, которые летают на небе.
И если ты закалываешь, то закалывай в жертву мира,
которая приятна Богу: закалывай ее, и кровь ее
выливай к жертвеннику, с мукою и плодовыми
жертвами, смешанными с маслом, вместе с жертвою
возлияния. Принеси все эта на жертвеннике
всесожжения в приятное благоухание пред
Господом. Как при жертве благодарения, положи
куски тука на огонь жертвенника, именно~--- тук
чрева, и тук внутренностей, и обе почки и весь тук
на них, и тук на стегнах, и печень вместе с
прилежащими к ней почками, И ты принесешь все в
доброе благоухание, которое приятно Господу, с
первыми жертвами и возлияниями, которые к сему
относятся, в доброе благоухание, как хлеб
всесожжения для Господа. Мясо же сей жертвы ешь
в этот и в следующий дни, и не дай солнцу во второй
день зайти над ним, пока оно не съедено. И ничего
не должно оставлять на третий день, ибо это
неприятно и неугодно Господу и его нельзя уже
съедать. Все, которые будут есть его, понесут на
себе грех; ибо так нашел я написанным о сем в
книге моих праотцев, в словах Еноха и Ноя. На свою
плодовую жертву ты должен положить соли, и без
соли завета не должны быть оставляемы все твои
плодовые жертвы пред Господом.

И в отношении к жертвенным дровам ты должен
остерегаться, чтобы не принести какое-нибудь
другое жертвенное дерево, как только кипарис, и
ель, и миндаль, и сосна, и пихта, и кедр, и
можжевельник, и лимон, и маслина, и мирт, и лавр, и
кедр, называемый арбот, и бальзамовый кустарник.
Из этих пород деревьев полагай под всесожжение
на жертвеннике, после того как ты рассмотришь их
наружность, и не клади [...] разрушенного дерева; но
твердое и безукоризненное, лучшее и
новорастущее, и не старое, ибо запах у него исчез
и его нет уже в нем, как прежде. Кроме этих дров не
клади других, ибо они не имеют запаха. И да
вознесется от тебя воня благоухания их к небу.
Соблюдай сию заповедь и исполняй ее, сын мой,
чтобы поступать право во всяком своем деле.

И всякий раз будь чист своим телом и омывайся
водою, прежде чем приступишь принести жертву на
жертвеннике; омой руки и ноги, прежде чем
приблизиться к жертвеннику. И когда ты
приготовишь жертвоприношение, то опять омой руки
и ноги, чтобы не оказалось следов крови ни на вас,
ни на ваших одеждах. Будь очень осторожен, сын
мой, с кровью, будь очень осторожен. Закопай ее в
землю, и не ешьте крови, ибо она есть душа; совсем
не ешь крови.

И не бери выкупа за кровь какого-либо человека,
чтобы она не была пролита даром без наказания;
ибо эта кровь, которая проливается, делает землю
греховною, и она не может быть очищена от крови,
как только кровью того, кто пролил ее. И не
принимай выкупа и дара за человеческую кровь:
кровь за кровь; тогда она вас сделает угодными
Господу, всевышнему Богу, и он будет хранителем
блага, чтобы сохранять тебя от всякого зла и
спасать тебя от всякой смерти. Я вижу, сын мой, все
дела сынов человеческих, что они~--- грех и зло; и
всякое дело их~--- мерзость, и жестоковыйность, и
осквернение, и нет правды в нем. Берегись, не ходи
по их путям, и не следуй по стезям их, и не
совершай смертного греха пред всевышним Богом, а
не то отвратит Он лице Свое от тебя, и вменит тебе
вину твою, и истребит тебя в сей стране и твое
семя под небом, чтобы имя твое и семя твое исчезли
на всей земле. Удаляйся от всех дел их и всякой
мерзости их, и храни защиту Бога всевышнего, и
исполняй волю Его и поступай право во всем. Тогда
Он благословит тебя во всех твоих делах, и
произведет от тебя растение правды для всей
земли, на все роды земли. И будут знать имя мое и
имя твое под небом во все дни. Иди, сын мой, в мире;
да укрепит тебя всевышний Бог, Бог мой и Бог твой,
исполнять Его волю; да благословит Он все семя
твое и остаток твоего семени на вечные роды всеми
благословениями правды, дабы ты был благословен
на всей земле!>> И он вышел от него, исполненный
радости.

\vs Jub 22:1
И было в первую седмину сорок третьего юбилея
во второй год, т.е. в тот год, когда умер Авраам,
пришли Исаак и Измаил от клятвенного колодезя,
чтобы праздновать семидневный праздник, т.е.
праздник начатков жатвы, со своим отцом Авраамом.
И Авраам обрадовался, что пришли два его сына.
Именно, Исаак имел много имущества в Вирсавии, и
ходил туда, чтобы осмотреть свое имущество, и
возвратился теперь к своему отцу. И в эти дни
пришел Измаил, чтобы видеть своего отца; и они
пришли оба вместе. Тогда Исаак заколол жертву во
всесожжение, и принес ее на жертвеннике своего
отца, устроенном им в Хевроне, и принес жертву, и
сделал торжественный пир своему брату Измаилу. И
Ревекка приготовила новый хлеб из нового жита; и
она дала его Иакову, своему предпочтенному
сыну, чтобы он отнес своему отцу Аврааму первый
плод земли, дабы он ел и благословил Творца всех
вещей, прежде чем умрет. И Исаак также послал чрез
Иакова, предпочтенного, Аврааму от
благодарственной жертвы, чтобы он ел и пил. И он
ел и пил, и благословлял всевышнего Бога, Который
создал и небо и землю, и распростер всю землю, и
дал сынам человеческим пищу и питие. И он благословил
своего Творца: <<И ныне благодарю Тебя, Боже
мой, что Ты удостоил меня видеть сей день. Вот я
теперь ста семидесяти пяти лет, седой и
престарелый. И все мои дни~--- суд мира: меч
ненавистника не победил меня; [...] и во всем, что Ты
давал мне и моим детям во все дни жизни моей до
сего дня. Боже мой, да будет милость Твоя на рабе
Твоем и на семени его сыновей, чтобы оно было для
Тебя избранным народом и наследием пред всеми
народами земли, отныне до всех дней родов земли,
во все века>>.

И он подозвал Иакова и сказал ему: <<Сын мой
Иаков! да благословит тебя Бог всех вещей, и да
укрепит тебя~--- совершать правду и волю Его [...], и
изберет тебя и семя твое, чтобы вы были Ему
народом, как наследие Его, согласно Его воле! И
подойди сюда, сын мой Иаков, и поцелуй меня!>> И
он подошел и поцеловал его.

Тогда он сказал: <<Да будут благословлены
Иаков и все сыны его Господом, Всевышним, во все
века! Да даст тебе Господь семя правды от сынов
твоих, которое святило бы Его по всей земле!
Да послужат тебе и падут пред семенем твоим все
народы! Будь силен пред людьми! И так как ты
уподобишься во всем семени Сифа, то да будут пути
твои и пути сыновей твоих правыми, чтобы народ
твой был свят. Бог, Всевышний, да даст тебе все те
благословения, которыми Он благословил меня и
которыми благословил Ноя и Адама! Да покоятся они
на священном темени (главе) твоего потомства на
все роды и до всей вечности! И да сохранит тебя
Господь чистым от всякого мерзкого осквернения,
чтобы получить тебе прощение во всякой вине,
которую ты по неведению совершишь; и да укрепит
Он тебя и да благословит тебя, чтобы ты
наследовал всю землю. Да восстановит Он завет
Свой с тобою, чтобы ты был Ему народом наследия
Его во все века! И да будет Он тебе и семени твоему
Богом, в действительность и истину, во все дни
земли! Помни же, сын мой Иаков, слово мое и храни
заповедь Авраама, отца твоего! Не сообщайся с
народами, и не ешь с ними, и не поступай по делам
их, и не вступай в родство с ними, ибо (всякое) дело
их нечисто, и все пути их осквернены и суть
мерзость. Свои жертвы они закалают мертвым, и
почитают демонов, и едят на могилах; они лишены
мудрости, чтобы разуметь, и очи их ничего не
видят; как еще погрешать им, если они говорят
дереву. <<Ты бог мой>> и камню: <<Ты господь
мой и спаситель мой>>, тогда как они (дерево и
камень) не имеют разума? И ты, сын мой Иаков,~--- Бог,
Всевышний, да вспомоществует тебе, и Бог небесный
да благословит тебя и да удалит тебя от нечистоты
их и от всей греховности их! Берегись, сын мой
Иаков, чтобы не брать жены из всего семени
дочерей Ханаана; ибо семя его предназначено к
истреблению на земле; ибо за вину Хама и за
проступок Ханаана будет уничтожено и все семя
его, и весь остаток его, и что избегло гибели. И
все поклоняющиеся идолам и все упорствующие не
имеют надежды в земле живых, но они сойдут в
царство мертвых, и пойдут к месту осуждения, и не
оставят по себе памяти на земле. Как сыны Содома
были истреблены на земле, так будут истреблены
все, поклоняющиеся идолам. Не бойся, сын мой
Иаков, и не страшись! Бог, Всевышний, будет
охранять тебя от погибели, и от всякого пути
греховного Он спасет тебя. Здесь в этой стране построишь
мне дом, чтобы я положил имя мое на нем,
предназначено тебе и семени твоему вовек, и он
будет называться домом Авраама. Это
предназначено тебе и семени твоему вовек, ибо ты
построишь дом мой и имя мое поддержишь пред
Богом. Вовек будет пребывать семя твое и имя твое
во все роды земли>>. И он перестал изрекать
заповеди и благословения.

И они легли оба вместе на одно ложе, и Иаков
заснул при персях своего деда Авраама. И его душа
семь раз прижимала его к сердцу, и любовь его и
сердце его радовались о нем, и он благословил его
от всего сердца и сказал: <<Бог, Всевышний, Бог
всех вещей и Творец всего, изведший меня из Ура
Халдейского, чтобы дать мне эту страну, дабы я
владел ею вовек и воскресил святое семя, чтобы
оно было благословенно вовек! Благослови и сына
моего Иакова, о котором я радуюсь всем сердцем
моим и любовию моею! Твоя милость и Твоя благость
да пребудут на нем и на семени его всегда! Не
оставляй его и не покидай его отныне до века! И да
будут очи Твои открытыми на него и на его семя,
чтобы охранять его, и благослови его и освяти его
в народ наследия Твоего! Благослови его всеми
благословениями Твоими, отныне до всех дней
вечности; и восстанови завет Твой и милость Твою
с ним и семенем его, и всю волю Твою восстанови с
ним на все роды земли!>>

\vs Jub 23:1
И он положил два перста Иакова на свои очи, и
прославил Бога богов, и закрыл свои глаза. И он
простер ноги свои, и уснул вечным сном, и
приложился к отцам своим. И во все это время Иаков
лежал при его персях, не зная, что отец его Авраам
умер. И Иаков пробудился от своего сна, и вот~---
Авраам был холоден как лед; и он сказал: <<Отец,
отец!>>, но он не говорил. Тогда он узнал, что
Авраам умер, и он встал, и побежал, и известил о
сем мать свою Ревекку. И Ревекка пошла ночью к
Исааку, и известила его о сем. И они пошли вместе с
Иаковом, который имел светильник в руке своей; и
когда они вошли, то нашли лежащее тело Авраама. И
Исаак пал на лице отца своего, и плакал, и
благословлял его, и лобызал его; и вопль раздался
в доме Авраама. Тогда встал сын его Измаил, и
пришел к отцу своему Аврааму, и оплакивал отца
своего Авраама, он и весь дом Авраама; и они
подняли великий плач. И сыновья его Исаак и
Измаил погребли его в двойной пещере вместе с
женою его Сарой. И оплакивали его в продолжение
сорока дней все домочадцы его, Исаак и Измаил со
всеми сыновьями своими и сыновья Кетуры в местах
своего поселения. Потом окончился плач по
Аврааме.

И он жил три юбилея и четыре седмины, сто
семьдесят пять лет, и дни его окончились. Ибо дни
предков простирались до девятнадцати юбилеев; но
после дней потопа они начали уменьшаться и
становиться короче девятнадцати юбилеев. И они
(люди) стали скоро достигать старости и
пресыщаться жизнью вследствие многих несчастий
и вследствие неправды своих путей, исключая
Авраама; ибо Авраам был совершенным во всяком
своем деле с Богом и благоугоден, и (ходил) в
правде в продолжение своей жизни. И вот он не
окончил четырех юбилеев в своей жизни, так что
состарился ради неправды и пресытился жизнью. И
все роды, которые явятся отныне до дня великого
суда, будут скоро достигать старости, прежде чем
достигнут двух юбилеев. И так как и знание их
будет оставлять их по причине их престарелости,
то уменьшится все знание их. И в те дни, если кто
проживет полтора юбилея, то об нем будут
говорить: <<Он жил долго>>, но большая часть
его жизни пройдет в несчастии, и труде, и
страдании, и без мира; ибо наказание последует за
наказанием, мучение за мучением, нужда за нуждой,
зло за злом, болезнь за болезнью, и все таковые
злые наказания одно за другим: болезнь, и резь в
животе, и град, и лед, и снег [...], и несчастие, и
оцепенение, и неплодородие, и смерть, и меч, и
пленение, и всякие наказания и несчастия. Все это
придет на злой род, который наполнит беззаконием
землю чрез нечистоту блудодеяния и скверноты и
чрез мерзость своих деяний. И тогда будут
говорить: <<Жизнь предков продолжалась до
тысячи лет, и она была хороша; а дней нашей жизни,
если человек проживет долго, семьдесят лет, и
если они сильны, восемьдесят лет, и вся она
нехороша>>. И не будет мира во дни того злого
рода. И в том роде дети будут злословить своих
отцов и старцев за греховность и нечистоту, и за
речи их уст, и за великие нечестия, которые они
совершают, и за то, что они оставили завет,
который Господь заключил между ними и Собою, дабы
они соблюдали и хранили все Его заповеди, и
постановления, и весь закон Его, не уклоняясь ни
налево, ни направо; так что все они совершают
злое, и все уста их говорят беззаконное, и всякое
дело их нечистота и мерзость, и все пути их
осквернение, и нечистота, и погибель. Вот земля
погибнет ради всех дел их; и не будет более семени
от вина и елея, так как все дела их~--- полное
нечестие; и все вместе погибнут: дикие звери, и
скот, и птицы, и все морские рыбы~--- из-за сынов
человеческих. И они будут враждовать друг с
другом, эти с теми, юноши со старейшинами, и
старейшины с юношами, бедные с богатыми, и
униженные с великими, и нищий с князьями~--- именно,
относительно закона и завета; ибо они забыли Его
заповеди, и завет, и праздники, и новолуния, и
субботы, и юбилеи, и всякую правду. И они будут
восставать с мечами и луком, чтобы привести их
обратно на путь, но они не возвратятся, пока не
прольется много крови на земле; один будет против
другого, и те, которые останутся, не возвратятся
на путь правды от своего нечестия. Ибо все они
будут восставать, чтобы расхищать богатство, и
брать, что принадлежит другому, и приобретать
себе великое имя, но не для правды и истины; и
святое святых осквернят они мерзостью своего
осквернения. И придет великое осуждение ради дел
того рода от Господа, и Он предаст их мечу, и на
осуждение, и пленение, и расхищение, и пожрание. И
Он возбудит на них грешников~--- которые не знают
сострадания и милости и не взирают на лицо ни
старого, ни юного, ни другого кого-либо, но на злых
и могущественных людей,~--- чтобы они поступали
яростнее, чем все сыны человеческие, и совершали
насилие против Израиля и делали беззаконие
Иакову. И много крови прольется на земле. И не
будет никого собирающего и никого ближнего. В те
дни будут они издавать вопль, и взывать, и
умолять, чтобы их освободили от руки греховных
язычников, но не будет никого спасающего. И
головы детей будут белыми от седых волос, и
трехнедельное дитя будет казаться старым, как
столетний; и их существование будет приведено к
погибели чрез страдание и бедствие.

И в те дни дети начнут оставлять свои
(греховные) законы, и стремиться к заповедям, и
возвращаться на путь правды. И дни начнут
возрастать, и сыны человеческие будут достигать
большей старости, от рода до рода и от дня до дня,
так что время жизни их продлится тысячу лет. И не
будет старого и пресыщенного жизнью, но все они
будут как дети и отроки, и скончаются все дни их в
мире и радости, и будут они жить так, что тогда не
будет сатаны и какого-либо губителя; ибо все дни
их будут днями благословения и спасения. В то
время Господь исцелит своих слуг; и они
вознесутся и будут наслаждаться глубоким миром,
и опять преследовать своих врагов. И они увидят
это и будут благодарить и радоваться радостью до
века. И они увидят на своих врагах все наказания
их и все проклятие их; и хотя кости их будут
покоиться в земле, но для духа их будет многая
радость, и они познают, что это Господь,
совершающий суд и являющий милость на сотнях и
тысячах, и на всех, которые любят Его. И ты, Моисей,
запиши сие слово; ибо так начертано на
свидетельстве небесных скрижалей для вечных
родов.

\vs Jub 24:1
И было, после того как Авраам умер, Бог
благословил сына его Исаака. И он поднялся от
Хеврона и пошел дальше, и жил при кладезе видения,
в первый год третьей седмины этого юбилея, в
продолжение семи лет.

И в первый год четвертой седмины начался голод
в стране, сверх прежнего, который был во время
Авраама. И Иаков сварил чечевичное кушанье; тогда
пришел Исав с поля голодный. И он сказал брату
своему Иакову: <<Дай мне от этого кушанья
плод!>> И Иаков сказал ему: <<Передай мне твое
первородство, тогда я дам тебе хлеба и также плод
от этого кушанья>>. И Исав сказал в сердце своем:
<<Я должен умереть: что мне за польза быть
первородным?>> И он сказал Иакову: <<Я отдаю
тебе его>>. И Иаков сказал: <<Поклянись мне!>>
И он поклялся ему. И Иаков дал брату своему Исаву
хлеба и кушанья; и он ел, пока не насытился. Так
пренебрег Исав первородством; посему Исав
называется также Едомом ради плода кушанья,
которое Иаков дал ему за его первородство. И
Иаков сделался старшим; Исав же потерял свое
преимущество.

И был голод в стране; тогда Исаак пошел, чтобы
спуститься в Египет, во второй год этой седмины. И
он пошел к царю Филистимскому в Герарон к
Авимелеху. И Господь явился ему и сказал ему:
<<Не ходи в Египет, оставайся в стране, которую Я
указываю, и будь чужеземцем в оной стране; Я буду
с тобою и благословлю тебя. Ибо тебе и твоему
семени Я дам всю эту землю, и исполню клятву Свою,
которою Я поклялся отцу твоему Аврааму, и умножу
семя твое, как звезды небесные, и дам всю эту
землю твоему семени. И благословятся в твоем
семени все народы земли~--- за то, что отец твой
слушался гласа Моего, и соблюдал слово Мое, и Мои
заповеди, и законы, и установления, и завет. И
теперь и ты слушайся гласа Моего и Моих заповедей
и живи в этой стране!>>

И он жил в Герароне три седмины. И Авимелех
отдал приказание относительно него и
относительно всего имущества его и сказал:
<<Всякий, кто прикоснется к нему или к
чему-нибудь из его имения, умрет смертию>>. И
Исаак стал великим в Филистимской земле, и
приобрел много волов и овец, и верблюдов, и много
имущества. И они сеяли в земле филистимлян, и
получили прибыли во сто крат. И Исаак сделался
очень великим. И филистимляне стали завидовать
ему; и все колодези, которые отроки Авраама
выкопали при его жизни, филистимляне засыпали
после смерти Авраама и завалили их землею. И
Авимелех сказал Исааку: <<Дались от нас, ибо ты
стал великим для нас!>> И Исаак вышел оттуда в
первый год седьмой седмины и странствовал в
долинах Герарона. И они опять выкопали колодези,
которые отроки отца его Авраама выкопали, и
филистимляне после смерти отца его Авраама
засыпали. И он назвал их именем, которое нарек им
отец его Авраам. И отроки Исаака выкопали
колодези в долине и нашли источник воды. И
пастухи герарские спорили с пастухами Исаака,
говоря: <<Вода принадлежит нам>>. И Исаак
назвал место сего колодезя: <<противный>>,
<<ибо они враждовали с нами>>. И они выкопали
другой колодезь и спорили из-за него, и Исаак дал
ему имя~--- <<теснота>>. И он вышел оттуда, и они
выкопали другой колодезь, и о нем они не спорили;
тогда он дал ему имя~--- <<пространный>>. И Исаак
сказал: <<Теперь Господь распространил нас>>.
И он усилился в земле той и пришел к кладезю
клятвенному в первый год первой седмины в сорок
четвертый юбилей.

И Господь явился ему в ту ночь, в новолуние
первого месяца, и сказал ему: <<Я Бог Авраама,
отуа твоего; не бойся, ибо Я с тобою. И Я
благословлю тебя и умножу семя твое, как песок
морской, ради раба Моего Авраама>>. И он устроил
там жертвенник, где прежде устроил отец его
Авраам, и призвал имя Господа, и принес жертву
Богу отца своего Авраама. И они выкопали колодезь
и нашли источник воды. И отроки Исаака выкопали
еще колодезь и не нашли воды. И они пришли и
сказали Исааку, что не нашли воды. И Исаак сказал:
<<Я поклялся ныне филистимлянам, и это есть
причина (безводности кладезя)>>. И он дал имя
месту сему <<клятвенный колодезь>>. Ибо здесь
он поклялся Авимелеху и Акофу, другу его, и
Фиколу, надзирателю его. И Исаак познал в тот
день, что они ложно поклялись~--- хранить с ними мир.
И Исаак проклял в тот день филистимлян и сказал:
<<Да будут прокляты филистимляне в день гнева и
ярости всеми народами! Пусть сделает их Господь
посмешищем, и проклятием, и гневом, и яростью в
руках грешных язычников и истребит их рукою
Хеттеев. И что избегнет меча врагов и Хеттеев, то
да истребит народ праведных судом праведным под
небом. Ибо врагами и ненавистниками будут они для
сынов моих во дни их и в земле их. И никто из
них не останется и никто не спасется в день
гневного суда. Но погибнет, и истребится, и будет
уничтожено в стране сей все семя филистимлян, и
от них не останется более и потомства их на земле.
Если бы оно взошло даже на небо, то да
низвергнется оттуда; и если бы оно утвердилось на
земле, то да будет исторгнуто, и если бы укрылось
между народами, то да будет истреблено и отсюда, и
если бы оно взошло в царство мертвых, то и там да
будет велико его наказание, и пусть не будет ему
там мира, и если бы оно странствовало в плену, то
да будет умерщвлено теми, которые подстерегают
на пути его душу. Не оставляй ему Ты, Который да
будешь прославлен, имени и семени на всей земле, и
да сопровождает его вечное проклятие!>> И
относительно него написано и начертано на
небесных скрижалях, чтобы так было поступлено с
ним в день суда, дабы оно было истреблено на
земле.

\vs Jub 25:1
И во второй год этой седмины в этом юбилее
Ревекка призвала сына своего Иакова и беседовала
с ним, говоря: <<Сын мой, не бери себе жену из
дочерей Ханаана, как брат твой Исав, который взял
себе двух жен из семени Ханаана; и они поразили
мой дух всеми делами своими, нечистотою блуда и
брака, и нет правды в них, но злы дела их. И я
весьма люблю тебя, сын мой; моя нежность
благословляет тебя каждый час и стражу нощную. И
ныне послушай гласа моего, и исполни волю матери
твоей, и не бери себе жену из дочерей сей страны, а
только из дома отца твоего и из рода отца твоего.
Возьми себе жену из дома отца моего; и Бог
всевышний благословит тебя и сынов твоих сделает
праведным родом и семя твое святым>>. После сего
Иаков говорил с матерью своей Ревеккою и сказал
ей: <<Вот, мать моя, мне девять седмин, и я не знаю
жены; ни одна не прикасалась ко мне и не
обручалась мне, и я не думаю брать себе жену из
какого-либо семени дочерей Ханаана, ибо я помню
слова отца нашего Авраама и что он заповедал мне,
что я не должен брать жены из всего семени дома
Ханаана. Но я хочу взять себе жену из семени дома
отца моего и из рода моего. Я слышал прежде, что
брат твой Лаван имеет в потомстве дочерей. На них
обратил я свое сердце, чтобы из них взять жену. И
посему я остерегался в своем духе, чтобы не
согрешить и не развратиться на всех путях моих,
во все дни жизни моей. Ибо относительно брака и
блуда отец мой Авраам дал мне много заповедей. И
вопреки всему тому, что он заповедал мне, брат мой
теперь прекословит мне в продолжение двадцати
двух лет и говорит часто мне, говоря: <<Брат мой,
возьми в жены сестру моих двух жен!>> Но я не
хочу делать так, как делает брат мой. Я клянусь
пред тобою, что я в продолжение всей моей жизни не
возьму себе жену из семени всех дочерей Ханаана и
не поступлю худо, как поступил брат мой. Не бойся,
мать моя! Поверь мне, что я исполню волю твою, и
буду ходить в праведности, и не извращу моих
путей вовек!>>

После сего она возвела лицо свое на небо, и
распростерла персты своей руки, и открыла уста
свои, и прославила Бога всевышнего, сотворившего
небо и землю, и принесла Ему благодарение и хвалу,
и сказала: <<Да будет прославлен Господь, Бог
мой, и да прославится имя Его вовек, что Он дал мне
Иакова, невинного сына и святое семя; ибо он Твой,
и семя его всегда Твое во все роды вовек.
Благослови его, Владыка, и вложи благословение
правды в уста мои, чтобы я благословила его!>> В
тот самый час дух святой сошел в уста ее, и она
положила руки свои на главу Иакова и сказала:
<<Будь прославлен Ты, Господь правды и Бог
миров, и да прославляют Тебя люди всех родов! Да
дарует Он тебе, сын мой, путь правды, и да откроет
семени твоему правду, и умножит сынов твоих во
время жизни твоей, и восставит их по числу
месяцев года! И да умножатся сыны их, и будут
бесчисленны, как звезды небесные, и да будет
число их больше, чем песок морской! Да даст Он тебе
эту плодоносную землю, как сказал Он, что Он
даст ее Аврааму и семени его после него навсегда
и что они вечно будут владеть ею. И да увижу я в
тебе, сын мой, благословенного сына во время
жизни моей; и все семя твое да будет святым
семенем! И как покоил тебя дух твоей матери во
время жизни ее на лоне родившей тебя, так
благословляет тебя моя нежность, и перси мои
благословляют тебя, и уста мои, и язык мой
прославляют тебя. Умножайся, и возрастай, и
распространяйся на земле, и да будет семя твое
совершенным в небесной и земной радости во всю
вечность! И да ликует семя твое, и в великие дни
мира да будет ему уделом мир твоего имени! И да
пребывает семя твое во всю вечность; и Бог
всевышний да будет их Богом, и Бог всевышний да
обитает с ними, и да будет устроено между ними
святилище Его на все века! Благословляющий тебя
да будет благословен, и всякая плоть,
проклинающая тебя напрасно, да будет
проклята!>> И она поцеловала его и сказала ему:
<<Господь мира да возлюбит тебя, как сердце
матери твоей и нежность ее; да возрадуется Он о
тебе и да благословит тебя!>> И вот она
перестала благословлять его.

\vs Jub 26:1
И в седьмой год этой седмины Исаак призвал
старшего сына своего Исава и сказал ему: <<Сын
мой, я стар, и вот очи мои притупились для зрения,
и я не знаю, когда умру. И теперь возьми свои
охотничьи орудия, свой колчан и лук, и выйди в
поле, и добудь дичи для меня, и налови мне
что-нибудь, сын мой, и приготовь мне кушанье, как
любит душа моя, и принеси его мне, чтобы я ел, и
душа моя благословила тебя, прежде чем я умру>>.
И Ревекка услышала речь его, когда Исаак говорил
Исаву. И Исав вышел рано в поле, чтобы добыть дичи,
и наловить, и принести своему отцу.

И Ревекка позвала сына своего Иакова и сказала
ему: <<Вот я слышала, как отец твой Исаак говорил
с братом твоим Исавом, говоря: <<Налови мне
какой-нибудь дичи, и приготовь мне кушанье, и
принеси его мне, чтобы я благословил тебя пред
Господом, прежде чем я умру>>. И теперь выслушай
слово мое, сын мой, что я тебе велю! Ступай в свое
стадо и принеси мне двух хороших козлят, я
приготовлю из них кушанье, как он любит. И ты
отнесешь его отцу твоему поесть, дабы он
благословил тебя пред Господом, прежде чем умрет,
и ты будешь благословен!>> И Иаков сказал матери
своей Ревекке: <<Мать, я ничего не жалею, что
отец мой может есть н что ему угодно. Только я
боюсь, мать моя, как бы он не узнал моего голоса и
не ощупал меня; ты знаешь ведь, что я гладкий, а
брат мой Исав волосат; и как бы мне не явиться в
его очах преступником, и не сделать чего-либо
такого, чего он не повелел мне, и как бы он не
разгневался на меня, и я навлеку на себя
проклятие, а не благословение>>. И мать его
Ревекка сказала ему: <<Пусть на меня придет твое
проклятие, сын мой; скорее послушайся гласа
моего!>>

И Иаков послушался гласа матери своей Ревекки,
и пошел, и сходил за двумя хорошими тучными
козлятами, и принес их матери своей, и мать его
приготовила их, как он любил. И Ревекка взяла
одежды старшего сына своего Исава, самые дорогие,
какие были у нее в доме, и одела у себя в них
Иакова, и кожею козлят обложила его руки и
открытые части его тела. И она дала кушанье и
обед, который приготовила, в руки сыну своему
Иакову; и он вошел к своему отцу и сказал: <<Я,
сын твой, сделал, что ты сказал мне; встань и сядь,
и поешь того, что я наловил, отец, чтобы душа твоя
благословила меня>>. И Исаак сказал сыну своему:
<<Как это ты так скоро наловил дичи, сын мой?>>
И Иаков сказал: <<Пославший мне это, Бог твой,
был со мною>>. И Исаак сказал ему: <<Подойди
сюда, чтобы я тебя ощупал, сын мой, сын ли ты мой
Исав или нет>>. И Иаков подошел к отцу своему
Исааку, и он ощупал его и сказал: <<Голос~--- голос
Иакова, но руки Исава>>; и он не узнал его; ибо
было соизволение (послание) с неба, которое
отняло чувство его. И Исаак не узнал его, ибо руки
его были, как руки того, и волосаты, как руки
Исава, дабы он благословил его. И он сказал:
<<Сын ли ты мой?>> И он сказал: <<Я сын твой>>.
И он сказал: <<Подай мне поесть того, что наловил
ты, сын мой, дабы душа моя благословила тебя!>> И
он поднес ему, и он ел; и он подал ему вина, и он
пил. И отец его Исаак сказал: <<Подойди и поцелуй
меня, сын мой!>> И он подошел и поцеловал его; и
он ощутил запах одежды его. И он благословил его и
сказал: <<Вот запах от сына моего, как запах от
поля, которое благословил Господь. Да даст тебе
Господь и сделает жребием твоим много росы
небесной и плодородия земли, и много хлеба; и
масла да даст тебе Он в изобилии! И да послужат
тебе народы, и люди (племена) да поклонятся тебе!
Ты будешь господином над братьями своими, и сыны
матери твоей да поклонятся тебе! И все
благословения, которыми Господь благословил
меня и отца моего Авраама, да будут на тебе и
семени твоем до века! Проклинающий тебя да будет
проклят, и благословляющий тебя да будет
благословен!>>

И после того, как Исаак кончил благословлять
сына своего Иакова, и Иаков вышел от отца своего
Исаака и скрылся, пришел его брат Исав с охоты; и
он также приготовил кушанье, и принес его отцу
своему, и сказал отцу своему: <<Отец мой, встань
и поешь моей дичи, чтобы душа твоя благословила
меня!>> И отец его Исаак сказал ему: <<Кто
ты?>> И он сказал: <<Я первенец твой Исав, я
сделал, как ты повелел мне>>. И Исаак
вострепетал великим трепетом и сказал: <<Кто же
тот, который мне добыл дичи, и наловил, и принес,
чтобы я ел от всего, прежде чем ты пришел, и я
благословил его? Да будет благословен он и все
семя его вовек!>> И когда Исав услышал речь отца
своего Исаака, то поднял громкий вопль, горько
сетуя, и сказал отцу своему: <<Благослови и меня,
отец!>> И он сказал ему: <<Твой брат пришел
хитростью и восхитил благословения твои>>. И
Исав сказал: <<Теперь я знаю, почему он
называется Иаковом; дважды он теперь запнул меня:
в первый раз он взял мое первородство, а теперь он
взял у меня мое благословение>>. И он сказал:
<<Разве ты не оставил для меня благословения,
отец?>> И Исаак отвечал и сказал Исаву: <<Вот я
сделал его господином над тобою и над всеми его
братьями, и дал их ему в рабы; изобилием хлеба и
масла я сделал его сильным; что я теперь сделаю
тебе, сын мой?>> И Исав сказал отцу своему
Исааку: <<Разве у тебя только одно
благословение, отец? Благослови и меня, отец!>> И
Исав возвысил голос свой и заплакал. И Исаак
отвечал и сказал ему: <<Вот от тучности земли
будет благословение твое и от росы небесной
свыше; своим мечом будешь питаться ты и будешь
служить твоему брату. И будет, если ты сделаешься
великим и свергнешь ярмо его с выи твоей, то
совершишь смертный грех, и все твое семя будет
истреблено под небом>>. И Исав разгневался на
Иакова за благословение, которым отец его
благословил его, и сказал в сердце своем:
<<Теперь придут дни плача по отце моем, и я убью
брата моего Иакова>>.

\vs Jub 27:1
Тогда было открыто во сне Ревекке слово Исава,
старшего ее сына. И Ревекка послала и призвала
Иакова, старшего сына своего, и сказала ему:
<<Вот брат твой Исав замышляет мщение, чтобы
убить тебя. И ныне послушайся гласа моего, встань
и беги к брату моему Лавану, и оставайся у него
некоторое время, пока не переменится гнев брата
твоего, и он не оставит гнева своего против тебя,
и забудет все, что ты сделал ему, и тогда я пошлю и
вызову тебя оттуда>>. И Иаков сказал: <<Я не
боюсь; если он хочет убить меня, то я сам убью
его>>. И она сказала: <<Так я лишилась бы в один
день обоих моих сыновей>>. И Иаков сказал своей
матери Ревекке: <<Вот ты знаешь, что мой отец
стар, и я вижу, что очи его ослабели; и если я
покину его, то это будет злом пред очами его, что я
оставлю его и уйду от вас; и отец мой разгневается
и проклянет меня. Я не могу идти; только если он
меня отошлет, чтобы я шел, то я пойду>>. И Ревекка
сказала Иакову: <<Я войду и скажу ему это, чтобы
он отпустил тебя>>. И Ревекка вошла и сказала
Исааку: <<Мне стала противною моя жизнь из-за
обеих дочерей Хета, которых Исав взял себе в жены,
и если Иаков возьмет себе жену между дочерями этой
страны, которые такие же, как и те, то зачем
мне еще жить? ибо дочери земли Ханаанской злы>>.
И Исаак призвал своего сына Иакова, и благословил
его, и увещевал его, и сказал ему: <<Не бери себе
жену из всех дочерей Ханаана; соберись, иди в
Месопотамию, в дом Бефуела, отца твоей матери, и
возьми себе оттуда жену из дочерей Лавана, брата
твоей матери. И Бог небесный да благословит тебя,
и возрастит тебя, и умножит тебя, чтобы ты
сделался обществом народов. И да даст Он тебе
благословения отца моего Авраама, тебе и твоему
семени после тебя, дабы ты наследовал землю
твоего странствования и всю землю, которую
Господь дал Аврааму. Иди, сын мой, с миром!>> И
Исаак отпустил Иакова, и он пошел в Месопотамию к
Лавану, сыну Бефуела, сирийцу, брату Ревекки,
матери Иакова.

И было, когда Иаков собрался идти в Месопотамию,
Ревекка опечалилась о своем сыне и плакала. И
Исаак сказал Ревекке: <<Сестра моя! не плачь о
моем сыне Иакове, ибо с миром он пойдет и с миром
возвратится. Бог всевышний охранит его от
всякого зла, и будет с ним, и не оставит его во все
дни; ибо я знаю, что Господь даст успех в путях
его, повсюду, где он пойдет, пока не возвратится к
нам с миром, и мы увидим его в мире. Не бойся за
него, сестра моя, ибо путь его прямой, и он муж
благочестивый и верный, и посему не погибнет.
Не плачь!>> И Исаак утешал Ревекку о сыне ее
Иакове и благословил его.

И Иаков вышел от клятвенного колодезя, чтобы
идти в Харран, в первый год второй седмины сорок
четвертого юбилея, и пришел в Лозу на горе, т.е. в
Вефиль, в новолуние первого месяца, в эту седмину;
и дошел до некоторого места, когда был вечер.
И он уклонился несколько к западу от дороги в ту
ночь и заснул здесь, ибо солнце зашло. И он взял
один из камней того места и положил его под
дерево,~--- ибо он странствовал один~--- и заснул, и
видел сон в ту ночь. И вот лестница была
утверждена на земле, вершина которой досягала до
неба. И вот Ангелы Господни поднимались и
опускались по ней, и сам Господь стоял на ней. И
Господь говорил с Иаковом и сказал: <<Я Бог отца
твоего Авраама и Бог Исаака; землю, на которой ты
стоишь, Я дам тебе и твоему семени после тебя; и
твое семя будет как пыль земная, и ты
размножишься к западу и востоку, и югу, и северу, и
благословятся в тебе и твоем семени все страны
народов. И вот Я буду с тобою, и буду охранять тебя
повсюду, где ты будешь ходить, и возвращу тебя в
мире в эту землю. Ибо Я не оставлю тебя, пока не
исполню все, что Я сказал тебе>>. И Иаков спал
(пробудился) и сказал: <<Точно, это место~--- дом
Господа, и я не знал сего>>. И он убоялся и
сказал: <<Священно это место, на котором ничего
нет иного, как только дом Господень; и это врата
небесные>>. И утром рано встал Иаков и взял
камень, который он положил себе в изголовье, и
поставил его памятником в знамение на этом месте,
и возлил на него сверху елей, и нарек имя тому
месту Вефиль. Прежде же оно называлось Луз, как
страна. И Иаков дал Богу обет, говоря: <<Если
Господь будет со мною, и сохранит меня на том
пути, в который я иду, и даст мне хлеб в пищу и
одежду для одеяния, и я возвращусь в мире в
дом отца моего, то да будет Господь моим Богом, и
также камень этот, который я поставил в этом
месте памятником в знамение, да будет домом
Господним! И из всего, что Ты дашь мне, я дам
десятую часть Тебе, Боже мой!>>

\vs Jub 28:1
И он встал и пошел в Месопотамию, в землю Лавана,
брата Ревекки, лежащую к востоку. И он оставался у
него и служил ему за Рахиль, одну из дочерей его. И
в первый год третьей седмины сказал он ему:
<<Дай мне мою жену, за которую я служил тебе семь
лет>>. И Лаван сказал Иакову: <<Я отдам тебе
твою жену>>. И Лаван устроил пир, и взял Лию, свою
старшую дочь, и отдал ее Иакову в жены, и дал ей
свою рабу Залафу в служанки. И Иаков не заметил
этого, ибо он думал, что это Рахиль. И он вошел к
ней, и вот это была Лия. Тогда Иаков разгневался
на Лавана и сказал ему: <<Зачем ты сделал так? Не
служил ли я тебе за Рахиль, а не за Лию? Зачем ты
обидел меня? Возьми свою дочь и отпусти меня, ибо
ты нехорошо поступил со мною>>. А Иаков любил
Рахиль больше, чем Лию. Ибо глаза Лии были слабы,
но лицом она была очень красива. Рахиль же имела
прекрасные глаза, и лицом она была очень красива
и привлекательна. И Лаван сказал Иакову: <<В
нашей стране нет такого обычая, чтобы выдавать
младшую дочь прежде старшей, и несправедливо
делать это. Ибо так сие определено и написано в
небесных скрижалях, и неправеден тот, кто делает
это, ибо это нехорошее дело пред Господом. И ты
также с своей стороны скажи сынам Израиля, чтобы
они не делали этого, и не позволяли брать и
выдавать младшую, прежде чем выдадут старшую; ибо
это очень нехорошо>>. И Лаван сказал Иакову;
<<Пусть пройдут семь дней пиршества, тогда я дам
тебе Рахиль, чтобы ты служил мне другие семь лет,
чтобы ты пас моих овец, как ты служил в течение
первой седмины (семилетия). [Когда же семь дней
пиршества Лии прошли], Лаван дал Иакову Рахиль,
чтобы он служил ему другие семь лет. И Рахили он
дал в служанки Баллу, сестру Залафы. И он служил
еще семь лет за Рахиль. [...].

И Господь отверз утробу Лии, и она зачала, и
родила Иакову сына, и он дал ему имя Робел, в
четырнадцатый день девятого месяца в первый год
третьей седмины. Утроба же Рахили была заключена,
ибо Господь видел, что Лия была ненавидима, а
Рахиль любима. И Иаков опять вошел к Лии, и она
зачала, и родила Иакову второго сына, и он дал ему
имя Симеон в двадцать первый день десятого
месяца в третий год этой седмины. И Иаков опять
вошел к Лии, и она зачала, и родила ему третьего
сына, и он дал ему имя Левий в новолуние первого
месяца в шестой год этой седмины. И Иаков опять
вошел к Лии, и она зачала, и родила ему четвертого
сына, и он дал ему имя Иуда в пятнадцатый день
третьего месяца в первый год четвертой седмины. И
ради всего этого Рахиль позавидовала Лии, так как
сама она не рождала. И она сказала Иакову: <<Дай
мне сына!>> И Иаков сказал: <<Разве я задержал
плод тебе, плод утробы твоей, разве я покинул
тебя?>> И когда Рахиль увидела, что Лия родила
Иакову четверых детей~--- Робела, Симеона, Левия и
Иуду,~--- то Рахиль сказала ему: <<Войди к моей
служанке Балле, чтобы она зачала и родила мне
сына!>> И он вошел, и она зачала и родила ему
сына, и она нарекла ему имя Дан в девятый день
шестого месяца в шестой год третьей седмины. И
Иаков опять вошел к Балле, и она зачала и родила
Иакову второго сына, и Рахиль дала ему имя
Наффали в пятый день седьмого месяца во второй
год четвертой седмины. И когда Лия увидела, что
она стала неплодною и не рождала более, то
позавидовала, и дала точно так же свою служанку
Залафу Иакову в жены; и она зачала и родила сына, и
она дала ему имя Асер в двенадцатый день восьмого
месяца в третий год четвертой седмины. И опять он
вошел к ней, и она зачала и родила ему второго
сына; и Лия дала ему имя Исашар во второй день
одиннадцатого месяца в пятый год четвертой
седмины. И Иаков вошел к Лии, и она зачала и родила
Иакову сына, и он дал ему имя Заблон в четвертый
день пятого месяца в четвертый год четвертой
седмины; и она передала его няньке. И Иаков опять
вошел к ней, и она зачала и родила двоих детей,
сына и дочь, и дала имя ему Заблон и дочери Дина в
седьмой день седьмого месяца в шестой год
четвертой седмины. И Господь умилостивился над
Рахилью и отверз утробу ее, и она зачала и родила
сына, и дала ему имя Иосиф в новолуние четвертого
месяца в шестой год этой четвертой седмины.

И когда родился Иосиф, Иаков сказал Лавану:
<<Дай мне моих жен и детей, чтобы идти мне к отцу
моему Исааку и чтобы он (?) сделал мне дом; ибо я
кончил годы, которые должен был служить тебе за
двух твоих дочерей, и я хочу идти в дом моего
отца>>. И Лаван сказал Иакову: <<Останься у
меня за вознаграждение, и паси опять у меня мои
стада, и возьми себе вознаграждение>>. И они
договорились друг с другом, чтобы он дал ему в
вознаграждение из овец и коз всех, которые [...]. И
овцы опять родили других, подобных себе, и все
были со знаком Иакова, и ни одна со знаком Лавана.
И имущество Иакова очень увеличилось. И он
приобрел себе рогатого скота, и овец, и ослов, и
верблюдов, и рабов, и служанок. И Лаван вместе с
своими сыновьями стали завидовать Иакову. И
Лаван отнял у него овец и замышлял злое против
него.

\vs Jub 29:1
И случилось, когда Рахиль родила Иосифа, пошел
Лаван стричь своих овец на расстояние трех дней
пути. И Иаков увидел, что Лаван пошел стричь своих
овец, и призвал Баллу и Рахиль, и уговаривал их
идти с ним в землю Ханаанскую; он рассказал им
именно все, что он видел во сне, и все, что Он
говорил с ним, чтобы он возвратился в дом своего
отца. И они сказали: <<Мы пойдем в то место; куда пойдешь
ты, пойдем и мы с тобою>>. И Иаков прославил
Бога отца своего Исаака и Бога деда своего
Авраама, и собрался, и посадил на верблюдов своих
жен и детей, и взял все свое имущество, и
переправился через реку, и пришел в землю
Гилеадскую. Но Иаков скрыл свой замысел от Лавана
и ничего не сказал ему об этом. В седьмой год
четвертой седмины Иаков возвратился в Гилеад, в
двадцать первый день первого месяца. И Лаван
преследовал его и настиг Иакова на горе Гилеад в
тринадцатый день третьего месяца. Но Господь не
допустил, чтобы он причинил вред Иакову; ибо Он
явился ему во сне ночью. И Лаван говорил с
Иаковом. И в пятнадцатый день того месяца сделал
Иаков Лавану и всем, которые пришли с ним,
пиршество. И Иаков поклялся Лавану в тот день, и
Лаван Иакову, что они не перейдут гору Гилеад с
злым умыслом друг против друга. И он устроил там
большой каменный холм во свидетельство; посему
дано имя тому месту~--- <<каменный холм
свидетельства>>. [...]. Прежде же звали землю
Гилеад землею Рефаил, ибо она была страною
Рефаимов, и рождались там Рефаимы-исполины,
которые были высотою до десяти, девяти, восьми,
семи локтей, и жилища которых простирались от
земли Аммонитян до горы Гермон, и главные города
которых были Хоронаим, и Астарос, и Эдрао, и Мисур,
и Беон. И Господь истребил их за нечестие их дел,
ибо они были очень мерзкими. И они оставили ее
(страну) вместо себя Аморреям, злому и греховному
народу; и нет ныне никакого другого народа, который
совершил бы все их грехи; посему они не имеют
долгой жизни на земле.

И Иаков отпустил Лавана в Месопотамию, в
восточную страну; и Иаков с своей стороны
направился в землю Гилеадскую и перешел Иаббок в
девятый месяц в одиннадцатый день его. И в этот
день пришел к нему брат его Исав, и они прекратили
свою распрю. И он ушел от него в землю Сеир, а
Иаков жил в шатрах. И в первый год в пятую седмину
в этот юбилей перешел он Иордан, и жил по ту
сторону Иордана, и пас свои стада от моря [...] до
Бефазона, и Дафаама, и Акрабита. И он посылал отцу
своему Исааку от всего своего имения одеяние, и
пищи, и мяса, и питья, и молока, и масла, и сыра, и
плодов от всяких пальм долины; и также матери своей
Ревекке он посылал четыре раза в год, между
месячными периодами, между пашней и жатвой, между
весной и дождем, между зимой и летом. И он (Исаак)
жил в башне Авраама; ибо Исаак возратился от
клятвенного колодезя и пошел в башню своего отца
Авраама, и жил здесь без (далеко от) Исава, своего
сына. К тому времени, когда Иаков отправился в
Месопотамию, Исав взял себе Маалиф, дочь Измаила,
в жены, и собрал все стада своего отца и своих жен,
и поднялся, и жил в горе Сеир, и оставил отца
своего Исаака одного при клятвенном колодезе. И
Исаак поднялся от клятвенного колодезя, и жил в
башне Авраама, отца своего, в горе Хеврон. И сюда
посылал Иаков все, что он от времени до времени
посылал своему отцу и своей матери, чтобы
облегчить всякую их скорбь. И они
благословляли Иакова от всего сердца и от всей
души.

\vs Jub 30:1
И в первый год шестой седмины поднялся он в
Салем, который находится на востоке от Сихема, с
миром, в четвертый месяц. И там увезли они Дину,
дочь Иакова, в дом Сихема, сына Емора, Гевитянина,
владетеля страны; и он спал с нею и обесчестил ее.
И она была маленькая девушка двенадцати лет. И он
просил ее отца, чтобы она была отдана ему в жены, и
у ее братьев он просил ее себе. Но Иаков и его
сыновья разгневались на сихемских мужей, которые
обесчестили их сестру Дину. И они замыслили между
собою нечто злое, и перехитрили, и обманули их. И
Симеон и Левий пришли тайно в Сихем, и совершили
наказание над всеми сихемскими мужьями, и убили
всех мужей, которых нашли в нем, и не оставили в
нем ни одного. Всех предали они мучительной
смерти, так как они обесчестили сестру их Дину.

И вы не должны отныне более делать так~---
бесчестить дочерей Израиля! Ибо на небе было
определено против них наказание, чтобы истребить
мечом всех сихемских мужей, так как они причинили
дочери Израиля бесчестие. И Господь предал их в
руки сыновей Иакова, чтобы они истребили их мечом
и совершили над ними наказание. И пусть никогда
не случится более в Израиле что-либо подобное
тому, чтобы бесчестили израильскую девицу. И если
муж в Израиле отдаст свою дочь или сестру
какому-либо мужу от семени язычников, или отдал,
то да умрет он смертию, и его должно побить
камнями, ибо он совершил грех и бесчестие
Израилю. И жену должно сожечь огнем, ибо она
осквернила имя дома своего отца, и она должна
быть истреблена из Израиля. И да не обретется
мерзость и блуд во Израиле во все роды земли, ибо
Израиль свят Господу. И каждый человек, который
совершит мерзость, должен умереть смертию и быть
побитым камнями. Ибо так утверждено это и
написано на небесных скрижалях относительно
всего семени Израиля: кто совершит мерзость, тот
должен умереть, смертию и быть побитым камнями. И
для сего закона нет конца дней, и прекращения, и
послабления, но непременно должен быть истреблен
тот муж, который осквернил свою дочь, во всем
Израиле, ибо он от своего семени дал Молоху и
совершил вину, осквернив его (семя). И ты, Моисей,
скажи сынам Израиля и положи свидетельство
против них, чтобы они не отдавали дочерей своих
язычникам и не брали дочерей языческих; ибо это
преступно пред Господом. Посему я записал тебе во
всех словах закона все деяние Сихемлян, что они
сделали с Диной, и как сыновья Иакова совещались,
говоря: <<Мы не отдадим нашу дочь (?) мужам
необрезанным, ибо это~--- поношение для нас и для
Израиля, если отдать ее или если взять из дочерей
языческих; ибо это нечисто и преступно для
Израиля, и Израиль не был бы чистым>>. И за эту
нечистоту, что один имеет жену из дочерей
языческих или что один отдает из своих дочерей
мужу из разного рода язычников, придут мучение за
мучением, и проклятие за проклятием, и все
наказания, и мучения, и проклятия. И если ты это
сделаешь, а он (народ) будет закрывать свои глаза, чтобы
не видеть тех, которые совершают мерзость, и
делают нечистым храм Господень, и оскверняют
святое имя, то весь народ должен быть наказан за
всю эту мерзость и осквернение, и не должно быть
допускаемо никакого лицеприятия и никакого
снисхождения, и не должны быть принимаемы от его
рук плоды, и плодовые жертвы, и всесожжения, и тук,
и жертвы курения в добрую воню, чтобы он
(согрешивший) был угоден. Так да будет с каждым
мужем и женщиною во Израиле, которые оскверняют
храм Его. Ради сего я дал тебе повеление, говоря:
<<Засвидетельствуй Израилю, что было
засвидетельствовано: вот как поступлено с
Сихемлянами и их сыновьями, как они были преданы
в руки двоих сыновей Иакова и были преданы
мучительной смерти. И это послужило им к правде, и
семя Левиино было избрано во священники и левиты,
чтобы они служили пред Господом, как мы, во все
дни. И да будет благословен Левий с его сыновьями
вовек, ибо они возревновали, чтобы совершить
правду, и суд, и мщение в отношении ко всем,
которые восстают против Израиля. И таким образом
отмечаются мужу в свидетельстве небесных
скрижалей благословение и справедливость пред
Ним, Богом всех вещей; и мы также будем вспоминать
правду, которую он совершил в своей жизни, во все
времена до тысячи родов. Благословение
предначертано ему, и оно придет на него~--- на него и
его род после него; и он будет записан, как друг и
праведник, на небесных скрижалях. И все это
событие я записал тебе, и дал тебе повеление,
чтобы ты открыл его сынам Израиля, дабы они не
совершали вины и не преступали закона, и не
разрушали завета, заключенного с ними, дабы они
хранили его, и были записаны друзьями. Если же они
преступят и будут ходить по всем путям нечистоты,
то будут написаны на небесных скрижалях врагами,
чтобы быть изглаженными из книги живых и
записанными в книгу тех, которые будут
уничтожены, и вместе с теми, которые будут
истреблены в стране>>. В тот день, когда сыновья
Иакова умертвили Сихемлян, было начертано им это
в книге на небе, что они совершили
справедливость, и правду, и мщение в отношении к
грешникам, и было записано им в благословение.

И они увели сестру свою Дину из дома Сихема. И
они увели в добычу все, что было в Сихеме: овец их,
и рогатый скот, и ослов, и все их имущество, и все
стада их~--- и привели все это к своему отцу Иакову.
И он совещался с ними о разрушении города; ибо они
страшились жителей страны~--- Хананеев и Ферезеев.
Но пришел страх Божий на все города вокруг
Сихемлян, так что они не решились изгнать сыновей
Иакова, ибо напало на них смущение.

\vs Jub 31:1
И в новолуние [...] месяца говорил Иаков со всеми
своими домочадцами, говоря: <<Очиститесь и
смените ваши одежды; соберемтесь и пойдемте в
Вефиль, где я дал обет в тот день, когда бежал от
моего брата Исава, ибо Он был со мною и возвратил
меня в мире в эту страну. И удалите чуждых богов,
которые у вас, и бросьте чуждых богов, и что
имеете на шеях и в ушах, и идола, которого Рахиль
украла у своего отца Лавана!>> И она (?) отдала
все Иакову и [...]. И он разбил и уничтожил их, и
оставил их под теревинфом в стране Сихемлян.

И в новолуние седьмого месяца пошел он в Вефиль
и устроил жертвенник на том месте, где он спал, и
поставил памятник. И он послал к отцу своему
Исааку, чтобы он пришел к нему к его жертве, а
также и к своей матери Ревекке. И Исаак сказал:
<<Пусть придет сын мой Иаков, чтобы я увидел его,
прежде чем умру>>. И Иаков пошел к своему отцу
Исааку и к своей матери Ревекке в дом отца его
Авраама, и взял с собою двоих из своих сыновей~---
Левия и Иуду, и пришел к отцу своему Исааку и к
своей матери Ревекке. И Ревекка вышла из башни к
нему, чтобы поцеловать и обнять Иакова; ибо ожил
дух ее, как только она услышала: <<Вот пришел сын
твой Иаков>>. И она поцеловала его, и увидела
двоих сыновей его, и узнала их, и сказала ему:
<<Это твои сыновья, сын мой?>> И она обняла их, и
поцеловала их, и благословила их, говоря: <<Да
будет чрез вас честь семени Авраама, и да будете
вы во благословение на земле!>> И Иаков вошел к
отцу своему Исааку в покой его, где он спал, и двое
сыновей его с ним. И он взял руку отца своего, и
наклонился, и поцеловал его; и Исаак пал на шею
сыну своему Иакову и плакал на шее его. И мрак сошел
с очей Исаака, и он увидел обоих сыновей Иакова
- Левия и Иуду, и сказал: <<Это твои сыновья, сын
мой? ибо они похожи на тебя>>. И он сказал ему,
что они действительно его сыновья, и
<<действительно я видел, что они истинно мои
сыновья>>. И они подошли и обернулись к нему, и
он поцеловал их и обнял их обоих вместе. И дух
пророчества нисшел в уста его. И он взял Левия за
правую руку, и обратился к Левию, и начал прежде
благословлять его, и сказал: <<Да благословит
прежде всего тебя Господь миров~--- тебя и твоих
сыновей во весь век! И да прославит Господь тебя и
твое семя великой честью; и да благоволит Он,
чтобы из всякой плоти ты и твое семя приступали к
Нему для служения Ему в Его святилище; как Ангелы
лица и как святые, так да будет семя твоих сыновей
в честь, и достоинство, и освящение! И да соделает
Он их великими во все века; и владыками, и
князьями, и начальниками да будут они над всем
семенем сыновей Иакова; да изрекают они слово
Господне с искренностью, и Его правду да
исполняют они по всей справедливости, и да
повествуют они о моих путях Иакову и об
откровении Израиля; благословение Господне да
будет вложено в уста их, дабы все семя их
благословляло тебя, возлюбленный! И мать твоя
нарекла тебе имя Левий, и справедливо она назвала
тебя так: ты будешь стоять ближе всех к Господу и
будешь иметь долю у всех детей Иакова; его стол да
будет твоим, и ты и сыновья твои должны питаться
от него; во все роды да изобилует стол твой, и твое
пропитание да не умалится никогда во все века!
Все ненавидящие тебя за что-либо погибнут пред
тобою, и все твои враги да будут истреблены и да
погибнут! Благословляющие тебя да будут
благословлены, и все люди, проклинающие тебя, да
будут прокляты!>>

И Иуде сказал он: <<Да даст тебе Господь силу и
крепость низложить всех, ненавидящих тебя! Будь
господином, ты и один из сыновей твоих, над сынами
Иакова! Твое имя и имя сыновей твоих да пойдет и
распространится по всей земле и по городам! Тогда
устрашатся язычники пред лицем твоим, и все
народы будут поражены, и все люди будут поражены.
Да придет Иакову чрез тебя помощь Его, и да
обретет Израиль чрез тебя избавление! И когда ты
будешь восседать на престоле славы, да
возвеличится справедливость твоя! Мир да будет
всему семени сыновей возлюбленного!
Благословляющий тебя да будет благословлен, и
все ненавидящие тебя, и гнетущие, и проклинающие
тебя да истребятся и погибнут от земли, и да будут
прокляты!>> И он обратился, и поцеловал его
опять, и обнял его, и очень радовался, что увидел
сыновей Иакова, который был его истинным сыном.

И он (Иаков) отошел от его лона, и пал ниц, и
склонился пред ним, и так он благословил их. И он
оставался там у своего отца Исаака в ту ночь, и
они ели и пили, исполненные радости. И он поставил
обоих сыновей Иакова, одного направо, другого
налево от себя, и это было вменено ему в
праведность. И Иаков рассказал своему отцу все в
ту ночь, как являл Господь к нему великую милость
и давал ему на всех его путях счастие и охранял
его от всякого зла. И Исаак благословил Бога отца
своего Авраама, Который Своим милосердием и
справедливостью не отступил от сына раба Своего
Исаака. И утром Иаков открыл отцу своему Исааку
обет, какой он дал Господу, и видение, какое он
видел, и сказал, что он устроил жертвенник и что
все приготовлено к жертве, чтобы совершить ее
пред Господом, как он дал обет, и что он пришел,
чтобы посадить его на осла. И Исаак сказал Иакову:
<<Я не могу идти с тобою, ибо я стар, и не могу
перенести путешествия. Иди, сын мой, с миром; ибо
мне теперь сто шестьдесят пять лет, я не могу
путешествовать. Посади на осла твою мать,
чтобы она шла с тобою. Но я знаю, сын мой, что ты
пришел ради меня; и да будет благословен этот
день, в который ты увидел меня живым и я тебя, сын
мой! Будь счастлив и исполни обет, который ты дал,
и не откладывай своего обета, (ибо это радостный
обет). И теперь поспеши исполнить его, и да примет
его Сотворивший все, Которому ты дал обет!>> И он
сказал Ревекке: <<Иди ты с своим сыном
Иаковом!>> И Ревекка пошла с Иаковом и Девора с
ней; и они пришли в Вефиль.

И Иаков вспомнил о молитве, которою отец
благословил его и двоих его сыновей~--- Левия и
Иуду, и возрадовался, и прославил Бога отцов
своих~--- Авраама и Исаака,~--- и сказал: <<Теперь я
знаю, что у меня есть вечная надежда~--- у меня и
моих сыновей~--- пред Богом всех вещей; и это
определено относительно них обоих, и начертано
это для них в вечное свидетельство па небесных
скрижалях~--- так, как благословил Исаак>>.

\vs Jub 32:1
И он оставался в ту ночь в Вефиле. И Левий видел
во сне, что он и его сыновья поставлены и
определены навек ко священству для Бога
всевышнего. И он пробудился от сна и прославил
Бога. И Иаков собрался рано утром в четырнадцатый
день этого месяца и дал десятину от всего, что
прибыло с ним, от человека до скота, от золота до
сосудов и одежд, и дал десятину от всего. И в те
дни Рахиль была беременною Вениамином, своим
сыном; и Иаков исчислил, начиная с него, своих
сыновей; и жребий Господа пал на Левия. Тогда он
одел его в священнические одежды и наполнил руки
его. И в пятнадцатый день этого месяца он принес
на жертвеннике четырнадцать тельцов из рогатого
скота, и двадцать восемь овнов, и сорок девять
овец, и шестьдесят агнцев, и двадцать девять
молодых козлят, как всесожжение на жертвеннике и
как благоприятный дар в добрую воню пред
Господом. Это была дань его ради обета, который он
дал~--- отделить десятину~--- вместе с плодовыми
жертвами и жертвами возлияния, которые
относились сюда. И когда огонь пожрал их, он
воскурил над ними фимиам на огне; и в жертву
благодарения он принес двух тельцов, и
четырех овнов, и двух годовых ягнят, и десять
телят, и четырех овец и двух молодых козлят. Так
делал он, давая свою дань, в продолжение семи
дней. И он ел там со всеми своими сыновьями и
людьми в радости в течение семи дней, и прославил
и благодарил при сем Господа, Который спас его от
всякого зла и исполнил на нем Свое обетование. И
он отделил десятину от всего чистого скота и
совершил всесожжение. А нечистый скот он отдал
сыну своему Левию, и людей отдал он ему. И Левий
исполнил в Вефиле священнические обязанности
пред своим отцом Иаковом, будучи предпочтен
своим десяти братьям, и был там священником. И
Иаков отдал ему свой обет. И таким образом он
отдал вторую десятину Господу и посвятил ее, и
она стала посвященною ему. И посему определено
это как закон на небе~--- давать вторую десятину,
чтобы есть ее пред Господом в том месте, которое
избрано, чтобы имя Его обитало там, во все годы. И
для сего закона нет конца дней; навечно записано
то постановление, чтобы делать это ежегодно, именно~---
есть вторую десятину пред Господом в том
месте, которое избрано. И ничего не должно быть
оставляемо от нее на следующий год, но в том же
году должно быть съедаемо семя до следующего
года, от дней начатков года, семени, и вина, и
масла, опять до этих же дней. И все, что останется
от нее и сделается устаревшим, должно считать
оскверненным и сожечь огнем, ибо это нечистое. И
так они должны вместе есть ее во святом доме и не
давать ей залеживаться. И все десятины от
рогатого скота и овец суть святы Господу, и Его
священникам должны принадлежать они, чтобы они
ели пред Ним из года в год. Ибо так это определено
и начертано на небесных скрижалях относительно
десятины.

И в следующую ночь, в двадцать второй день этого
месяца, Иаков решил обстроить то место, и обнести
место стенами, и посвятить, и сделать его святым
навечно для себя и своих детей после себя до века.
И Господь явился ему ночью, и благословил его, и
сказал ему: <<Твое имя не должно быть только
Иаков, но должно быть наречено имя тебе
Израиль>>. И Он опять сказал ему: <<Я Господь
Бог твой, сотворивший небо и землю. Я возращу
тебя, и весьма умножу тебя, и цари произойдут от
тебя, и будут они господствовать всюду, где
только ступит нога сынов человеческих. И Я дам
твоему семени всю землю, которая под небом, и они
будут по своей воле господствовать над всеми
народами; и после этого они завладеют всею землею
и наследуют ее навеки>>. И Он окончил Свою
беседу с ним и поднялся от него. И Иаков видел, как
Он вознесся на небо; и он видел ночью в видении, и
вот Ангел сошел с неба с семью скрижалями в своих
руках, и он дал их Иакову, и он читал их и прочитал
все, что было написано на них, что случится с ним и
с его сыновьями во все века. И он показал ему все,
что было написано на скрижалях, и сказал ему:
<<Ты не должен строить на этом месте и делать
святыню навечно, и Он не хочет обитать здесь, ибо
не это Его место. Иди в дом Авраама, отца
твоего, и живи в доме отца твоего Исаака до дня
смерти твоего отца. Ибо в Египте ты умрешь в мире,
и будешь погребен в этой стране с честью в гробах
твоих отцов с Авраамом и Исааком. Не бойся! ибо
как ты видел и прочитал, так все и случится. И
запиши все, как ты видел это и прочитал>>. И
Иаков сказал: <<Как я упомню все так, как видел
это и прочитал?>> И он сказал ему: <<Я опять
приведу тебе все на память>>. И он поднялся от
него.

И он пробудился от сна своего, и вспомнил все,
что читал и видел, и записал всю речь, которую
читал и видел. И он остался там еще на один день и
принес в этот день жертву совершенно так же, как в
прежние дни, и назвал его~--- <<прибавление>>. Ибо
тот день прибавлен. И прежние дни он назвал
праздником. И так ему было открыто, что должно
случиться, и это написано на небесных скрижалях.
И ради того было ему это открыто, чтобы он хранил
его, и прибавлял его таким образом к семи
праздничным дням. И он назван был прибавлением,
как заканчивающий в мире праздничные дни по
числу дней года.

И в ночь на двадцать третий день этого месяца
умерла Девора, нянька Ревекки, и они похоронили
ее внизу города под дубом реки, и он нарек имя той
реке~--- <<река Деворы>> и дубу~--- <<дуб плача
Деворы>>. И Ревекка пошла и возвратилась в дом к
его отцу Исааку. И Иаков послал ему чрез нее
барана, и телят, и овец, чтобы она приготовила его
отцу кушанье, как он любил. И после отправления своей
матери он пошел дальше, пока не пришел в страну
Кебрафан, и жил там. И Рахиль родила ночью сына и
назвала его: <<мой сын болезни>>, ибо она имела
тяжелые роды. А отец его назвал его Вениамином в
десятый день восьмого месяца в первый год шестой
седмины этого юбилея. И Рахиль умерла там и была
погребена в стране Ефрафе, т.е. Вифлееме. И Иаков
устроил на могиле Рахили памятник при дороге, над
ее могилою.

\vs Jub 33:1
И Иаков пошел дальше и жил к северу в Магд-Ладре
Еф(рафа). И он пошел к своему отцу Исааку, он и его
жена Лия, в новолуние десятого месяца, и Робел
увидел Баллу, служанку Рахили, наложницу своего
отца, когда она купалась в воде в уединенном
месте, возымел любовь к ней, и спрятался ночью, и
вошел в жилище Баллы, и нашел ее одну ночью
лежащей на своей постели и спящей в своем жилище.
И он лег к ней на ложе, и открыл покрывало ее; и она
схватила его и вскрикнула. И когда она узнала его,
что это был Робел, застыдилась его, и отняла свою
руку от него, и убежала, и очень скорбела о
случившемся, но не сказала ничего ни одному
человеку. И когда Иаков пришел и отыскивал ее, она
сказала ему: <<Я не чиста для тебя, но обесчещена
для тебя, ибо Робел обесчестил меня, и лег со мною
ночью, когда я спала у себя, и я не узнала его, пока
он не открыл моего покрывала, и он спал со
мною>>. И Иаков сильно разгневался на Робела,
что он спал с Валлою и открыл покров своего
отца. И Иаков не приближался более к ней, так как
Робел обесчестил ее, и пред всеми людьми открыл
покров своего отца. Ибо поступок его был очень
нехорош; это постыдно пред Господом.

Посему написано и определено на небесных
скрижалях, что муж не должен лежать с женою
своего отца и открывать покров своего отца, ибо
это мерзость. Смертию должен умереть преступный
муж, который ляжет с женою своего отца, а также и
жена: ибо они мерзость совершили на земле. И пред
нашим Господом не должно быть ничего мерзкого в
народе, который Он избрал Себе в царское
достояние. И еще написано: да будет проклят, если
кто лежит с женою своего отца, за то, что он
открывает срамоту своего отца. И все святые
Господа пусть скажут: <<Аминь, аминь!>> И ты,
Моисей, скажи сынам Израиля, чтобы они соблюдали
сие слово, ибо за него угрожает наказание
смертию, и это мерзость, и нет за это прощения,
чтобы можно было искупить мужа, который совершит
сие зло, кроме наказания смертию, и умерщвления, и
побиения камнями, и истребления из народа нашего
Бога. Ни одного дня не должен жить на земле муж,
который совершит это во Израиле, ибо это
преступно и мерзко. И не должно говорить, что
Робел остался в живых и получил прощение, хотя он
лежал с наложницей своего отца, в то время как муж
ее, отец его Иаков, еще был жив. Ибо Он тогда же
вполне открыл всем постановление, и правду, и
закон, который существует вовек. Но во все дни
твои он должен иметь силу закона, с его дней и
есть вечный закон для вечных родов. И этот закон
не прекратится, и никакое прощение не будет
уделом такового, кроме того, что оба вместе будут
истреблены из народа; в тот день, когда они
совершили это, должно умертвить их. И ты, Моисей,
напиши это для Израиля, чтобы они соблюдали сие и
поступали по сему слову и не совершали смертного
греха, ибо Господь Бог наш есть судия
нелицеприятный и неподкупный. И скажи им это
постановление, чтобы они слушались, и
оберегались, и внимали сему, и не погибли бы, и не
были бы истреблены на земле. Ибо нечисты, и
мерзки, и преступны, и скверны все они,
совершающие сие на земле пред нашим Господом. И
нет большего греха на земле, как любодеяние,
которым они любодействуют; ибо Израиль есть
народ, святый Господу, и народ наследия для
своего Бога, и народ священства и царства, и
достояние Божие. И никто не должен
существовать, кто является столь нечистым среди
святого народа.

И в третий год этой шестой седмины вышел Иаков и
все его сыновья, и жили в доме Авраама у своего
отца Исаака и своей матери Ревекки. И вот имена
сыновей Иакова: его первенец Робел, Симеон, Левий,
Иуда, Исашар, Завулон~--- сыновья Лии; и сыновья
Рахили: Иосиф и Вениамин; и сыновья Баллы: Дан и
Наффали; и сыновья Залафы: Гад и Асер; и Дина, дочь
Лии, единственная дочь Иакова. И они пошли и
поклонились Исааку и Ревекке. И когда последние
увидели их, благословили Иакова и всех его детей.
И Исаак очень обрадовался, что увидел сыновей
Иакова, своего младшего сына, и благословил их.

\vs Jub 34:1
И в шестой год этой седмины сорок четвертого
юбилея Иаков отослал своих сыновей~--- пасти его
овец~--- и своих рабов с ними на поля Сихемские. И
собрались против них семь царей аморрейских,
чтобы умертвить их, укрывшись под деревьями, и
увести их скот. Но жены их, и Иаков, и Левий, и Иуда,
и Иосиф оставались дома у своего отца Исаака, ибо
дух его был прискорбен, и он не хотел отпустить
их; Вениамин, как юнейший, оставался с своим
отцом. И пришли цари Фафы и Арезы, и Сарагана, и
Село, и Гаиза, и царь Бефорона, и Маанизакира, и
все, живущие на тех горах и обитающие в лесах
страны Ханаанской. И известили Иакова: <<Цари
аморрейские окружили твоих сыновей и похитили их
стада>>. И отправился из своего дома он, и его
три сына, и все рабы его отца, и его рабы, и вышли
против них в числе восьмисот мужей, носивших
мечи; и они поразили их на поле Сихемском, и
преследовали бегущих, и убили Арезу, и Фафу, и
Сарагана, и Аманискино, и Гаганиса. И он собрал
свои стада, и был могущественнее их, и наложил на
них дань, чтобы они давали плоды своей страны. И
они построили Робел и Фамнафарес. И он
возвратился благополучно, и заключил с ним мир, и
они сделались его рабами, пока он не ушел с своими
сыновьями в Египет.

И в седьмой год этой седмины послал он Иосифа,
чтобы он осведомился о состоянии своих братьев,
из своего дома в Сихем. И он нашел их в стране
Дуфаим. И они подстерегали его, и сделали против
него умысел убить его. И когда они изменили свое
намерение, то продали его измаильским
странствующим купцам. И они отвели его в Египет и
продали его Питфаре, евнуху Фараона, главному
повару, жрецу Гелиопольскому. И сыновья Иакова
закололи козленка, и омочили одежду Иосифа в
его крови, и послали ее Иакову в десятый день
седьмого месяца. [...]. И они принесли ее ему, и он
занемог болезнию от печали о его смерти. И он
сказал: <<Дикий зверь пожрал Иосифа>>. И все
его домочадцы были при нем в этот день; и его
сыновья, и его дочери собрались утешать его; но он
оставался безутешным о своем сыне. И в тот день
услышала Балла, что Иосиф потерян, и умерла от
печали по нем, в то время как она была в Караффифе.
И дочь его Дина также умерла, после того как Иосиф
был потерян. Эта тройная скорбь пришла на Израиля
в один месяц. И они похоронили Баллу напротив
могилы Рахили, а также и дочь его Дину похоронили
там. И он скорбел об Иосифе год и не переставал
печалиться; ибо он сказал: <<Я сойду в могилу,
печалясь об Иосифе>>. Ради сего определено
между сынами Израиля, чтобы скорбеть в десятый
день седьмого месяца, в тот день, когда пришло
печальное известие об Иосифе к его отцу Иакову,
чтобы испрашивать в оный день прощение чрез
козла, в десятый день седьмого месяца, один раз в
год, в своих грехах; ибо они превратили любовь
своего отца к его сыну Иосифу в печаль о нем. И
этот день установлен, чтобы они в течение его
скорбели о своих грехах, и о всякой своей вине, и о
своем проступке, дабы очищаться в этот день
однажды в год.

И после того как Иосиф был потерян, сыновья
Иакова взяли себе жен. 1)Имя жены Робела~--- Ада;
2)жены Симеона~--- Адиба, хананеянка;
3)жены Левия~--- Мелха, из дочерей Аррама, из семени сыновей
Фарана; 4)жены Иуды~--- Бефазуел, хананеянка; 5)жены
Исашара~--- Гизека; 6)жены Дана~--- Эгла; 7)жены
Завулона~--- Нииман; 8)жены Наффалима~--- Разуу из
Месопотамии; 9)жены Гада~--- Михи; 10)жены Асера~---
Ийона; 11)жены Иосифа~--- Асанеф, египтянка; 12)жены
Витамина~--- Ийоска. И Симеон изменил намерение, и
взял вторую жену из Месопотамии, как и его братья.

\vs Jub 35:1
И в первый год первой седмины сорок пятого
юбилея призвала Ревекка сына своего Иакова и
дала ему повеление относительно его отца и брата,
чтобы он почитал их во все дни жизни своей. Он
сказал: <<Я буду поступать так, как ты повелела
мне, ибо это будет для меня честью, и
достоинством, и праведностью пред Господом, что я
почитаю их. Ты же знаешь от дня моего рождения до
сего дня каждое мое деяние и всякое мое
помышление, что я всегда благожелаю всем. Как же
мне не исполнять того, что ты заповедала мне,~--- именно
почитать моего отца и брата? Скажи мне, мать
моя, какое зло ты заметила во мне? Я же и далек от
него (от Исава), и между нами существует
доброе согласие>>. И она сказала ему: <<Сын мой,
в продолжение всей своей жизни я не видела в тебе
ничего предосудительного, а только
справедливое. Я говорю тебе, сын мой: в этом году я
кончу свою жизнь. Ибо я видела во сне день
моей смерти, что я не проживу более ста
пятидесяти лет. И вот я кончила все дни своей
жизни, которые надлежало мне прожить>>. И Иаков
усмехнулся над словами своей матери, что мать
сказала ему, будто она умрет, между тем как она
сидела против него в полной силе, не ослабевшая;
ибо она входила и выходила, и видела, и зубы ее
были здоровы, и никакая болезнь не коснулась ее в
течение всей ее жизни. И Иаков сказал ей: <<Я
буду счастлив, мать моя, если моя жизнь
сравняется по продолжительности с твоей жизнью и
если я так же сохранюсь в полной своей силе, как
ты. Ты не умрешь, и напрасно говоришь со мною о
своей смерти>>.

И она вошла к Исааку и сказала ему: <<Я имею к
тебе просьбу: заставь поклясться Исава, что он не
причинит обиды Иакову и никогда не будет
преследовать его. Ибо ты знаешь нрав Исава, что он
груб от юности, и нет в нем добродушия; ибо он
замышляет после твоей смерти убить его. И ты
знаешь все, что совершил он во все дни от того дня,
когда брат его Иаков пошел в Харран, до сего дня;
как он оставил нас всем своим сердцем и сделал
нам злое; как он присвоил себе твои стада и все
достояние твое похитил пред лицем твоим. И когда
мы умоляли и просили о нашем достоянии, он
действовал подобно человеку, как бы оказывающему
нам свою милость. И он гневается на тебя, ибо ты
благословил своего благочестивого и праведного
сына Иакова; ибо в нем нет ничего злого, но одно
только доброе. И с того времени, как он
возвратился из Харрана, до сего дня он не обидел
нас ни в малейшем; но мы все получаем от него вовремя
и всегда; и он радуется от всего сердца, если мы
что-нибудь принимаем от него, и благословляет
нас; и он не отделился от нас с того времени, как
пришел из Харрана, до сего дня. И он живет всегда с
нами в доме, почитая нас>>. И Исаак сказал ей:
<<Знаю и я, и вижу отношение Иакова к нам, что он
нас почитает во всем. Я раньше любил более Исава,
чем Иакова, ибо он родился прежде; но теперь я
люблю Иакова больше, чем Исава, так как он
оказался в своих делах весьма дурным и в нем нет
никакой справедливости. Ибо все пути его~---
несправедливость и насилие, и нет в нем
справедливости. Мое сердце также потрясено
теперь из-за всех его дел, и ему и семени его не
будет счастия, но они погибнут на земле и будут
истреблены под небом. Ибо он оставил Бога Авраама
и последовал за своими женами, за мерзостию и
соблазном их~--- он и его сыновья. И ты говоришь мне,
чтобы я заставил его поклясться, что он не убьет
Иакова; но если он и поклялся бы, то это будет
бесполезно, и он будет совершать не добро, а
только зло. И если он захочет умертвить своего
брата Иакова, то будет предан в руки Иакова, и не
избегнет рук его, но впадет в руки его. И ты не
бойся за Иакова: ибо хранитель Иакова~---
могущественный, и досточтимый, и
достопоклоняемый всеми>>. [...]

И Ревекка послала и призвала Исава; и он пришел
к ней. И она сказала ему: <<У меня есть просьба к
тебе, сын мой, и ты обещай, что исполнишь то, что я
скажу тебе, сын мой!>> И он сказал: <<Я сделаю
все, что ты скажешь мне, и не откажу в твоей
просьбе>>. И она сказала ему: <<Я прошу тебя,
чтобы ты, когда я умру, перенес меня и похоронил с
Сарой, матерью отца твоего, и чтобы вы любили друг
друга, ты и брат твой Иаков, и никто не
предпринимал бы никакого зла против своего
брата, а оказывал бы только взаимную любовь, дабы
вы были счастливы, сыновья мои, и были почитаемы
на земле, и никакой враг не восторжествовал бы
над вами, и вы были бы достойными милосердия пред
очами тех, которые любят вас>>. И он сказал: <<Я
все исполню, что ты сказала мне, и похороню тебя,
когда ты умрешь, вместе с Сарой, матерью отца
моего, так как ты любишь кости ее, чтобы они были с
твоими костями. И также брата моего Иакова я буду
любить больше, чем всякую плоть; у меня на всей
земле нет брата, кроме его одного; и нет ничего
великого (трудного) для меня в том, чтобы любить
его, ибо он брат мой, и мы вместе были посеяны в
твоем чреве и вместе вышли из твоих недр. И если
не любить мне своего брата, то кого же мне любить?
И я также прошу тебя, чтобы ты сделала увещание
Иакову относительно меня и моих детей, так как я
знаю, что он как царь будет господствовать надо
мною и над моими сыновьями. Ибо в тот день, когда
мой отец благословил его, он сделал его высшим, а
меня подчиненным. И я клянусь тебе, что я буду
любить его и ничего злого не замыслю против него
в продолжение всей моей жизни, а только
доброе>>. И он подтвердил клятвою все эти слова.
И она призвала Иакова пред очи Исава и дала Иакову
повеление согласно беседе, какую она вела с
Исавом, и он сказал: <<Я исполню твою волю,
ручаясь за то, что от меня и моих сыновей не
выйдет ничего злого против моего брата Исава, и
только лишь одну любовь встретит он>>. И они ели
и пили, она и ее сыновья, в эту ночь. И она умерла,
трех юбилеев одной седмины и одного года, в эту
ночь. И оба ее сына Исав и Иаков похоронили ее в
пещере около Сары, матери их отца.

\vs Jub 36:1
И в шестой год этой седмины призвал Исаак обоих
своих сыновей~--- Исава и Иакова, и они пришли к
нему, и он сказал им: <<Сыны мои, я иду по пути
моих отцов в вечное жилище, где отцы мои.
Похороните меня с моим отцом Авраамом в двойной
пещере на полях Эфрона Хеттеянина, которые
Авраам купил для могилы; там похороните меня! И я
заповедую вам, сыны мои, совершать на земле
справедливость и правду, чтобы Господь послал
вам все, что обещал сделать Аврааму и семени его.
И любите друг друга, как братья, сыны мои, так, как
каждый любит самого себя, и стараясь сделать
лучшее для другого, действуя единодушно на земле
и каждый любя другого, как самого себя. И
относительно идолов я заповедую вам, чтобы вы
отвергали их, и ненавидели, и не любили их; ибо они
исполнены соблазна для тех, которые почитают их,
и для тех, которые поклоняются им. Памятуйте, сыны
мои, о Господе, Боге Авраама, отца вашего, как и я
после него почитал Его и служил Ему воистину,
дабы Он умножил вас в радости и возрастил семя
ваше~--- умножил вас в числе, как звезды небесные, и
насадал бы на земле вас и всякое растение правды,
которое не будет истреблено во все роды века. И
ныне я заклинаю вас великою клятвою~--- ибо нет
большей клятвы, как клятва славным, и честнейшим,
и великим именем Того, Кто сотворил небо и землю и
все в совокупности,~--- чтобы вы страшились Его и
почитали и чтобы каждый любил своего брата нежно
и искренно, и не желал бы своему брату зла отныне
до века, во все дни вашей жизни, дабы вы были
счастливы во всех своих делах и не погибли. И если
кто из вас предпримет что-либо злое против своего
брата, то знайте отныне, что всякий, замышляющий
что-либо злое против своего брата, падет от его
руки и будет истреблен из страны живых, и семя его
также погибнет под небом. И в день проклятия и
власти Он сожжет пылающим и поедающим огнем и
его страну, и город, и все принадлежащее ему,
подобно тому как Он сожег Содом; и он будет
изглажен из книги наставления сынов
человеческих и не будет записан в книге жизни. Но
он погибнет и подпадет вечному осуждению, чтобы
их наказание беспрерывно возобновлялось чрез
ненависть, и проклятие, и гнев, и мучение, и злобу,
и муки, и болезнь, вовек. Я говорю и возвещаю вам,
сыны мои, суд, как он придет на мужа, который
захочет сделать что-нибудь дурное против своего
брата>>.

И он разделил все свое имущество между ними
обоими в тот день. И он дал преимущество тому, кто
был рожден прежде, и отдал ему башню, и все
кругом ее, и все, что Авраам приобрел у
клятвенного колодезя. И он сказал:
<<Преимущество это должен иметь тот, кто рожден
прежде>>. И Исав сказал: <<Я продал и передал
свое старшинство Иакову; пусть будет отдано это
Иакову! И я не буду более говорить ему об этом, ибо
так случилось это>>. И Исаак сказал: <<Да
покоится, сыны мои, благословение на вас и на
вашем семени в этот день, что вы остались
спокойными и не огорчили меня из-за старшинства,
что вы не допускаете ничего постыдного из-за
него! Господь, Всевышний, да благословит того
мужа, который совершает справедливость, его и
семя его вовек!>> И он перестал давать заповеди
и благословлять их. И они ели и пили вместе пред
ним, и он радовался, что между ними совершилось
примирение. И они вышли от него, и отдыхали в тот
день, и спали.

И Исаак почил в тот день на своем ложе, полный
радости, и почил вечным сном, и умер ста
восьмидесяти лет, окончив двадцать пять седмин и
пять лет. И оба сына его, Исав и Иаков, похоронили
его. И Исав пошел в страну Едом, на горе Сеир, и
оставался там. И Иаков жил на горе Хеврон в башне
страны странствования отца своего Авраама; и он
почитал Господа от всего сердца и по Его заповеди
[...].

И жена его Лия умерла в четвертый год второй
седмины сорок пятого юбилея; и он похоронил ее в
двойной пещере возле своей матери Ревекки,
налево от могилы Сары, матери отца его. И все ее и
его сыновья пришли оплакивать вместе с ним Лию,
жену его, и утешать его в скорби по ней. Ибо он
скорбел об ней, так как любил ее еще более после
того, как умерла сестра ее Рахиль. Ибо она была
благочестива и праведна во всех путях своих и
почитала Иакова. И в течение всего времени, как
она жила с ним, он не слышал из уст ее никакого
грубого слова; ибо она была кротка, и миролюбива,
и праведна, и досточтима. И он вспомнил ее дела,
какие она делала во время своей жизни, и очень
оплакивал ее, ибо он чрезвычайно любил ее от
всего сердца и от всей души.

\vs Jub 37:1
И когда Исаак, отец Иакова и Исава, умер,
услышали сыновья Исава, что Исаак отдал
первенство своему младшему сыну Иакову, и
разгневались чрезмерно, и препирались с своим
отцом, говоря: <<Почему, когда ты старший, а
Иаков~--- младший, твой отец отдал первенство
Иакову, и тебя поставил ниже?>> И он сказал им:
<<Потому что я свое первородство продал за
немногое~--- за чечевичное кушанье. И в тот день,
когда мой отец послал меня на охоту~--- наловить
чего-нибудь и принести к нему, чтобы он ел и
благословил меня, пришел он (Иаков) хитростью и
принес моему отцу есть и пить, и мой отец
благословил его, а меня отдал в его руки. И вот
отец наш заставил нас поклясться, меня и его, что
мы ничего злого не замыслим друг против друга, и
будем жить друг с другом в любви и мире, и не
извратим наших путей>>. И они сказали ему: <<Мы
не послушаемся тебя в том, чтобы поддерживать с
ним мир, ибо мы сильнее, нежели он, и мы преодолеем
его. Мы выйдем против него, и умертвим его, и
истребим его сыновей. И если ты не пойдешь с нами,
мы причиним зло и тебе. Послушай же нас теперь: в
Араме, и Филистее, и Моаве, и Аммоне мы наберем
себе отборных людей, которые способны к войне, и
пойдем против него, и сразимся с ним, и истребим
его в стране, прежде чем он приобретет силу>>. И
отец их сказал им: <<Не ходите, и не начинайте с
ним войны, дабы вам не пасть от него>>. И они
сказали ему: <<Неужели тебе от юности и до сего
дня только и делать, чтобы склонять свою выю под
его ярмо? Мы не послушаемся сих слов>>. И они
послали в Арам и к Адураму, другу своего отца, и
наняли себе у них тысячу способных к войне мужей
и отборных воинов. И пришли к ним от Моава и от
сынов Аммона нанятых тысяча отборных воинов, и от
филистимлян тысяча отборных воинов, и от Эдома и
хореев тысяча отборных ратников, и от хетитов
сильные, способные к войне мужи. И они сказали
своему отцу: <<Выходи, веди нас! а иначе мы убьем
тебя>>. И он разгневался и пришел в ярость, когда
увидел, как сыновья употребляли в отношении к
нему насилие, чтобы он был предводителем их и вел
их против своего брата Иакова.

После сего ему вспомнилось все то зло, которое
лежало сокрытым внутри его против его брата
Иакова, и он не вспомнил о клятве, которую он дал
своему отцу и своей матери, что он не предпримет
ничего злого против своего брата Иакова во всю
свою жизнь.

И в продолжение всего этого времени Иаков
ничего не знал о том, что они выступают против
него войною,~--- он сильно скорбел о своей жене Лии,~---
пока они не подошли к башне против него~--- четыре
тысячи способных к войне, сильных, воинственных,
отборных мужей. И жители Хеврона послали к нему
сказать: <<Вот брат твой пришел на тебя, чтобы
победить тебя, с четырьмя тысячами мужей,
препоясанных мечами и носящих щит и оружие>>.
Они любили Иакова более, чем Исава, поэтому и
сказали ему это; ибо Иаков был муж милостивый и
более любвеобильный, чем Исав. И Иаков не поверил
этому, пока они не приблизились к самой башне. И
он взошел на башню, и говорил с своим братом
Исавом, и сказал: <<Приносишь ли ты мне доброе
утешение? Пришел ли ты ко мне ради моей умершей
жены? Это ли клятва, которою ты дважды поклялся
твоим родителям пред их смертию? Ты нарушил
клятву, и тем, чем ты поклялся своему отцу, ты
осужден>>. Тогда Исав отвечал и сказал ему:
<<Никогда не клянутся между сынами
человеческими и между зверями земли истинною
клятвою до века; но в тот самый день они уже
замышляют злое друг против друга, и враг ищет
убить своего врага. И ты также ненавидишь меня и
моих сыновей до века, и с тобою нельзя сохранять братской
любви. Слушай эти слова мои, которые я скажу
тебе. Если бы я мог изменить кожу и щетину свиньи,
чтобы она (щетина) стала шерстью, и если бы на ее
голове выросли рога, подобно рогам овец, тогда я
поддерживал бы с тобою братскую любовь. И если
грудь у матери отделится~--- ибо ты отселе мне не
брат,~--- и если волки заключат мир с ягнятами, что
они не будут пожирать и похищать их, и если сердце
их склонится к тому, чтобы делать друг другу
добро, тогда я буду иметь в своем сердце мир с
тобою. И если лев сделается другом вола, и будет
запрягаться с ним в одно ярмо, и будет пахать с
ним, тогда я заключу мир с тобою. И если вороны
сделаются белыми, как рис, тогда я узнаю, что я
люблю тебя и храню мир с тобою. Ты должен быть
истреблен, и сыновья твои должны быть истреблены,
и да не будет с тобою мира!>> И Иаков увидел в тот
час, что он замыслил против него злое [...], чтобы
убить его, и что он пришел, стремясь как дикий
зверь, бросающийся на копье, которое пронзает и
убивает его самого, и он не отступает от него.
Тогда он сказал домочадцам и своим рабам, чтобы
они напали на него~--- на него и на всех его
соучастников.

\vs Jub 38:1
И после сего Иуда говорил со своим отцом
Иаковом и сказал ему: <<Отец! Натяни лук свой, и
пусти стрелу свою, и порази злодея, и убей врага.
Да будет у тебя сила на это, ибо мы не хотим
убивать твоего брата!>> [...]. И Иаков тотчас
натянул лук свой, и пустил он стрелу свою, и
поразил брата своего Исава, и убил его. И еще
пустил он стрелу свою и попал в арамеянина Адрона
в его левый грудной сосок, и обратил его в
бегство, и убил его. После сего сыновья Иакова
выступили со своими рабами и распределились на
четырех сторонах башни. Вперед вышел Иуда с
Наффали, и Гадом, и пятьюдесятью рабами на
северной стороне башни, и они умертвили все, что
было пред ними, и никто не спасся от них, даже ни
один. И Левий, и Дан, и Асер выступили на восточной
башне с пятьюдесятью мужами и убили ратников
моавитян и аммонитян. И Робел с Исашаром и
Завулоном выступили на южной стороне башни с
пятьюдесятью мужами и убили воинов филистимлян.
И Симеон, и Вениамин, и Енох, сын Робела, выступили
на западной стороне башни с пятьюдесятью мужами
и перебили из едомитян и хореев (триста) сильных
воинственных мужей; и семьсот убежали. И четыре
сына Исава бежали с ними, и оставили отца своего
убитого, как он пал на холме, который находится в
Адураме. И сыновья Иакова преследовали их до горы
Сеир; а Иаков похоронил своего брата на холме,
который находится в Адураме, и возвратился в свой
дом. И сыновья Иакова стеснили сыновей Исава на
горе Сеир, и согнули их выю, так что они стали
рабами сыновей Иакова. И они послали к своему
отцу спросить, заключить ли мир с ними или
умертвить их. И Иаков велел сказать своим
сыновьям, чтобы они заключили мир. И они
заключили мир с ними и наложили на них ярмо
рабства, чтобы они платили Иакову и его сыновьям
дань всякий год. И они, не переставая, платили
Иакову дань до того дня, когда он ушел в Египет
[...].

И вот цари, которые владычествовали над Едомом,
- прежде чем стал владычествовать царь над сынами
Израиля,~--- до сего дня в стране Едом. И был царем в
Едоме Балак, сын Беора, и имя его города было
Динаба. И Балак умер, и вместо него стал царем
Иобаб, сын Зары из Базуры. И вместо него стал
царем Адафа, сын Барада, который поразил
Мидианитян на поле Моав; и имя его города Авуф. И
Адафа умер, и вместо него стал царем Салман из
Амелека. И Салман умер, и вместо него стал царем
Суал из Робаофа при реке. И Суал умер, и вместо
него стал царем Беулуман, сын Акбура. И Беулуман,
сын Акбура, умер, и вместо него стал царем Адафа, и
имя жены его было Майя-Тобиф, дочь Матрифы, дочери
Мимифбид-Цаобы. Вот цари, которые управляли в
стране Едом.

\vs Jub 39:1
И Иаков жил в земле странствования отца своего,
в стране Ханаанской. Вот роды Иакова. Иосиф был
семнадцати лет, когда они отвели его в Египет, и
Питфаран, евнух Фараона, главный повар, купил его.
И он поставил Иосифа над всем своим домом. И
благословение Господа было над домом египтянина
ради Иосифа, и во всем, что он делал, Господь давал
ему успех. И египтянин предоставил Иосифу все,
что было у него, ибо видел, что Господь был с
ним, и во всем, что он делал, давал ему успех. Иосиф
же был красив и весьма миловиден лицом. И жена
господина его обратила на него свои взоры, и
увидела Иосифа, и почувствовала любовь к нему, и
просила его, чтобы он лег с нею. Но он не предал ей
свою душу, и вспомнил о Господе и о словах,
которые отец его Иаков читал в словах Авраама,
что никто не должен прелюбодействовать с женою,
имеющей мужа, и что для такового определено
наказание смертию на небесах пред Господом
всевышним, и что грех будет записан за ним в
книгах, которые до века всегда существуют пред
Господом. И Иосиф вспомнил эти слова, и не хотел
лечь с нею. И она просила его в течение года, но он
отказывал ей, и не хотел слушаться ее. Но она
обняла его и схватила его в доме, чтобы принудить
его лечь с нею, и заперла двери дома. Но он
вырвался из рук ее, и оставил в руках ее свою
одежду, и разломал запор, и выбежал от нее. И когда
та жена увидела, что он не хочет лечь с нею,
очернила его пред своим господином, говоря:
<<Твой еврейский раб, которого ты любишь, хотел
причинить мне насилие, чтобы лечь со мною; но
когда я возвысила голос свой, он убежал, и оставил
свою одежду в моих руках, как только я схватила
его, и разломал запор>>. И египтянин увидел
одежду Иосифа и также запор, который был
разломан; и послушался слов жены своей, и посадил
Иосифа в темницу в одно место, где сидели
заключенные, которых царь велел заключить в
темницу. И он оставался там в темнице. И Господь
дал Иосифу милость в глазах главного темничного
стража и милосердие в глазах его. Ибо он видел,
что Господь был с ним и во всем, что он делал,
давал ему успех. И он передал ему все, и главный
темничный страж не смотрел ни за чем; ибо все, что
делал Иосиф, совершал Господь. И он оставался там
два года.

И в те дни Фараон, царь египетский, разгневался
на двух своих евнухов, на главного кравчего и на
главного хлебника, и посадил их в темницу в доме
главного повара~--- в темницу, где был заключен
Иосиф. И главный темничный страж приказал Иосифу,
чтобы он служил им; и он служил им. И они оба
видели сон~--- кравчий и хлебник, и рассказали его
Иосифу. И как он истолковал его им, так с ними и
случилось. Главного кравчего Фараон опять
приставил к его должности, а главного хлебника
предал смерти~--- как он истолковал им. И главный
кравчий забыл Иосифа в темнице, хотя он возвестил
ему, что с ним случится; и он не думал о том, чтобы
объявить Фараону, как Иосиф сказал ему; но он
забыл о нем.

\vs Jub 40:1
И в те дни Фараон видел два сна в одну ночь о
голоде, который придет на всю землю. И он
пробудился от сна своего, и призвал всех
снотолкователей, которые были в Египте, и
волхвов, и рассказал им оба свои сна; но они не
могли ничего узнать. После этого главный кравчий
вспомнил об Иосифе и сказал о нем царю. И он велел
привести его из темницы и рассказал ему оба свои
сна. И он сказал Фараону: <<Два сна означают одно
и то же>>. И он сказал ему: <<В продолжение семи
лет будет изобилие во всем Египте, и после этого в
продолжение семи лет голод, подобного которому
не было на всей земле. Теперь назначь, Фараон, во
всей земле Египетской житницы, чтобы в них
собирали пищу в каждом городе в продолжение лет
изобилия, чтобы иметь пищу на семь лет голода, ибо
он будет весьма велик>>. И Господь дал Иосифу
милость и милосердие в очах Фараона. И Фараон
сказал своим слугам: <<Мы не найдем столь мудрого
и разумного мужа, как он, ибо дух Господа с ним>>.
И он поставил его вторым над всем своим царством,
и сделал его господином над всем Египтом, и велел
везти на своей второй колеснице, и одел его в
виссонную одежду, и возложил ему золотую цепь на
шею, и велел возвещать впереди него: <<Ел Ел
Ваабрир>>. И он надел (кольцо) на руку его, и
сделал его господином над всем своим домом, и
возвеличил его, и сказал ему: <<Только престолом
одним я буду больше тебя>>. И Иосиф был
господином над всею Египетскою страною. И любили
его все князья Фараона, и все слуги его, и все
исполнявшие царские дела, ибо он ходил в
праведности и без гордости и надменности и был
нелицеприятным и неподкупным, но по
справедливости судил все народы страны. И страна
была хорошо управляема Фараоном благодаря
Иосифу, ибо Господь был с ним и дал ему милость и
благоволение на весь его род в глазах всех,
которые его знали и о нем слышали. И царство
Фараона было благоустроено: ни злоумышленника,
ни злодея не было там. И царь нарек имя Иосифу
Сафанфи-фанс и дал Иосифу в жены дочь Патифарана,
дочь жреца Гелиопольского, главного повара. И в
тот день, когда Иосиф стоял пред Фараоном, ему
было (тридцать) лет, когда он стоял пред
Фараоном. И в тот самый год умер Исаак. И сбылось
так, как Иосиф сказал в толковании его сна: и
пришли семь лет изобилия на всю Египетскую
страну~--- на одну меру тысяча восемьсот мер. И
Иосиф собирал пищу в каждом городе, пока они
не наполнились хлебом, так что нельзя было уже
считать его и мерить по причине великого
изобилия.

\vs Jub 41:1
И в сорок пятый юбилей во вторую седмину во
второй год взял Иуда своему первенцу Еру жену из
дочерей Арама, по имени Фамарь. Но он ненавидел
ее, и не спал с нею, так как мать его была из
дочерей Ханаанских, и он хотел взять себе жену из
родства своей матери, но отец его Иуда не
позволил ему этого. И этот первенец его был
дурной, и Господь лишил его жизни. И Иуда сказал
сыну своему Онану: <<Войди к жене брата твоего, и
соверши с нею брак ужичества, и восстанови свое
семя брату твоему!>> И Онан знал, что это было бы
семя не его, а его брата, и вошел к жене своего
брата, и излил свое семя на землю. И это было злом
пред очами Господа, и Он лишил его жизни. И Иуда
сказал своей невестке Фамари: <<Оставайся в
доме отца твоего вдовою, пока сын мой Шела не
подрастет; тогда я отдам тебя ему в жены>>. И он
подрос. Но Бефзуел, жена Иуды, не допускала, чтобы
сын ее Шела женился на ней. И Бефзуел, жена Иуды,
умерла в пятый год этой седмины.

И в шестой год отправился Иуда стричь своих
овец в Фимнафу. И она сняла вдовьи одежды, и
надела покрывало, и нарядилась, и села при
воротах на дороге в Фимнафу. И когда Иуда вошел,
он встретил ее, и принял ее за блудницу, и сказал
ей: <<Я войду к тебе>>. И она сказала:
<<Войди!>> И он вошел. И она сказала: <<Дай мне
плату блудницы>>. И он сказал: <<Я ничего не
имею при себе, кроме кольца на пальце, и серег, и
трости, которая у меня в руке>>. И она сказала
ему: <<Дай их мне, пока ты не пришлешь мне плату
блудницы>>. И он сказал ей: <<Я пришлю тебе
козленка>>, и отдал их ей. И она зачала от него; и
Иуда пошел к своим овцам, а она в дом отца своего.
И Иуда послал чрез пастуха едолламского
козленка, но он не нашел ее. И он спрашивал людей той
местности, и сказал им: <<Где блудница,
которая была там?>> И они сказали: <<У нас
нет блудницы>>. И он возвратился и известил его,
что он не встретил ее, и сказал ему: <<Я
спрашивал людей того места, и они сказали мне:
<<Нет там блудницы>>>>. И он сказал:
<<Пойдемте, чтобы не быть осмеянными>>. И когда
прошло три месяца, она узнала, что зачала; и они
известили об этом Иуду, говоря: <<Вот твоя
невестка Фамарь сделалась беременной от
блуда>>. И Иуда пошел в дом отца ее, и сказал ее
родителям и братьям: <<Выведите ее, чтобы она
была сожжена, ибо она совершила нечто нечистое в
Израиле>>. И вот, когда они вывели ее, чтобы
сжечь, она послала своему свекру кольцо, и серьгу,
и трость, говоря: <<Узнай, кому принадлежит
это: ибо от того я зачала>>. И Иуда узнал, и
сказал: <<Фамарь правее меня>>. И они не сожгли
ее. И посему она не была отдана Шеле. И он уже не
приближался больше к ней. И после сего она родила
двоих сыновей~--- Фареса и Зару, в седьмой год этой
второй седмины. И тогда окончились семь лет
плодородия, о которых Иосиф сказал Фараону.

И Иуда сознал, что это было дурное дело, которое
он совершил, так как он преспал с своею невесткою,
и нашел это неправым пред своими очами, и сознал,
что он совершил вину и согрешил, так как открыл
покров своего сына. И он стал скорбеть и умолять
Господа о милосердии к своей вине. И мы сказали
ему в сновидении, что она прощена ему, ибо он
неотступно просил о милости, и скорбел, и вновь не
совершил сего. И он получил прощение, ибо он
обратился от своего греха и неразумия. Ибо велика
эта вина пред нашим Господом; всякого, кто делает
так, и всякого, кто преспит с своею тещею, должно
сожечь огнем, чтобы он сгорел в нем. Ибо мерзость
и осквернение лежит на них; огнем должно сожечь
их. И ты также скажи сынам Израиля, чтобы не было
между ними мерзости; огнем должно сожечь мужа,
который преспит с нею, и также жену, дабы Он
отвратил Свой гнев и Свое наказание от Израиля. И
Иуде мы сказали, что так как два его сына не
сочетались браком, то семя его восстановлено для
другого рода, и оно не будет истреблено; ибо он
пришел по своему неведению и желал наказания;
именно по закону Авраама, который он заповедал
своим детям, Иуда хотел сожечь ее огнем.

\vs Jub 42:1
И в первый год третьей седмины сорок пятого юбилея
настал в стране голод; и на земле не было дождя,
так что совсем ничего не падало, и земля
сделалась бесплодною. И только в стране
Египетской была пища, так как Иосиф собрал, чтобы
можно было давать им пищу. И Иосиф собирал в
течение семи лет плодородия семя в стране и
сберегал его. И египтяне пришли к Иосифу, чтобы он
дал им пищи; и он открыл житницы, где был хлеб от
первого года, и продавал его жителям страны за
золото.

И Иаков услышал, что в Египте была пища; (тогда
он послал своих сыновей в Египет приобрести
хлеба), но Вениамина не послал. И они пришли
вместе с сопровождавшими их; и Иосиф узнал их,
но они его не узнали. И он беседовал с ними, и
спрашивал их, и говорил им: <<Не соглядатаи ли
вы, и не пришли ли, чтобы разузнать след
страны?>> И он заключил их; потом он освободил
их, и оставил одного только Симеона, и его девять
братьев отпустил, и наполнил мешки их хлебом; а их
золото он положил им в их мешки, но так, что они не
знали. И он повелел им привести своего младшего
брата, ибо они сказали ему, что их отец и младший
брат живы. И они вышли из страны Египетской, и
пришли в землю Ханаанскую, и рассказали своему
отцу все, что с ними случилось, и как правитель
страны говорил с ними, и как он посадил Симеона в
заключение, пока они не привезут к нему
Вениамина. И Иаков сказал: <<Вы похитили у меня
моих детей; Иосифа нет более, и Симеона нет, и
Вениамина еще хотите взять? Ваши дурные действия
ложатся тяготою на мне>>. И он сказал: <<Сын мой
не пойдет с вами; он может заболеть во время
пути. Ибо мать их родила только двоих; один
из них потерян, и еще этого хотите у меня взять? С
ним может случиться в путешествии болезнь, и вы
доведете до смерти мою седую старость от горя>>.
Ибо он видел, что золото их принесено назад в их
мешках, и посему он боялся послать его с ними.

И усилился голод, и сделался великим в стране
Ханаанской и во всех странах, кроме только земли
Египетской. Ибо многие из египтян собирали себе
семена в пищу, после того как увидели, что Иосиф
собирает семена, и кладет их в житницы, и
сберегает на голодные годы. И жители Египта
прокармливались этим в первый год голода. И когда
Израиль увидел, что голод в стране очень
усилился, и не было более спасения, он сказал
своим сыновьям: <<Идите опять, и приобретите
себе пищи, чтобы нам не умереть>>. И они сказали:
<<Мы не пойдем; если наш младший брат не пойдет с
нами, то мы не пойдем>>. И (Иаков) увидел, что
если он не пошлет его с ними, то все они погибнут
от голода. И Робел сказал: <<Передай мне его в
мои руки, и если я его не приведу к тебе назад, то
умертви двух моих сыновей за его душу>>. Но он
сказал: <<Он не пойдет с тобою>>. И Иуда подошел
и сказал: <<Отпусти его со мною, и если я его не
приведу к тебе назад, то буду пред тобою
преступником во все дни моей жизни>>. И он
отпустил его с ними во второй год седмины в
новолуние, и они пришли в Египетскую страну
вместе со всеми другими, шедшими туда, с дарами в
своих руках, с стираксой (стакти), и миндальными
орехами, и фисташками, и чистым медом. И они
пришли и предстали пред Иосифа, и он увидел
Вениамина, своего брата, и узнал их, и сказал им:
<<Это ваш младший брат?>> И они сказали ему:
<<Это он>>. И он сказал: <<Да будет милость
Господня с тобою, сын мой!>> И он послал их в свой
дом, и выдал им также Симеона, и приготовил им
обед. И они передали ему дар, который они привезли
для него. И они ели пред ним, и он дал каждому из
них по части, но часть Вениамина была в семь раз
больше, чем часть остальных. И они ели, и пили, и
встали, и оставались у своих ослов. И Иосиф
придумал способ, посредством которого он мог бы
узнать их помышления, господствуют ли в них
человеческие помышления. И он сказал мужу,
который управлял его домом: <<Наполни им все их
мешки хлебом, положи им также назад их золото в их
хранилища, и мою серебряную чашу, из которой я
пью, положи в мешок младшего, и отпусти их>>.

\vs Jub 43:1
И он сделал, как сказал ему Иосиф, и наполнил
мешки их пищею, и золото их положил также в их
мешки, и чашу в мешок Вениамина. И рано утром они
отправились. И когда они выехали оттуда, Иосиф
сказал мужу: <<Гонись за ними, беги и обличи их,
говоря: <<Вы отплатили злом за добро, и похитили
серебряную чашу, из которой пьет господин мой>>.
И приведи назад ко мне их младшего брата, и
приведи его немедленно, прежде чем я займусь
своими делами>>. И он побежал за ними и сказал им
по его словам. И они сказали ему: <<Да будет это
далеко от рабов твоих; они не сделают ничего
подобного, и не украдут никакого имущества из
дома твоего господина. И даже золото, которое мы в
первый раз нашли в наших мешках, мы, рабы твои,
принесли назад из земли Ханаанской. Украдем ли мы
какое-нибудь имущество? Вот мы здесь и мешки наши:
ищи, и тот из нас, в мешке которого ты найдешь
чашу, пусть будет наказан смертию, и мы с своими
ослами будем в подчинении у твоего господина>>.
И он сказал им: <<Нет; мужа, у которого я найду,
его одного только возьму я в рабы; а вы идите с
миром>>. И когда он искал в их сосудах, он начал
со старшего и кончил младшим, и она была найдена в
мешке Вениамина, младшего. И они пришли в ужас, и
разорвали свои одежды, и навьючили своих ослов, и
возвратились назад в город. И они пришли в дом
Иосифа, и пали все пред ним на свое лице на землю.
И Иосиф сказал им: <<Вы сделали это>>. И они
сказали: <<Что нам сказать и как оправдаться,
когда наш господин нашел вину за своими рабами?
Вот мы рабы господина нашего вместе с нашими
ослами>>. И Иосиф сказал им: <<Я страшусь
Господа, и вы пойдете домой; но ваш брат будет
принадлежать мне, ибо вы сделали злое. Вы не
знаете, что муж, как я, пьющий из этой чаши,
дорожит своею чашею? И вы похитили ее у меня>>. И
Иуда сказал: <<Да будет позволено мне, господин
мой, сказать слово в уши господина моего. Двоих
братьев мать моя родила нашему отцу, рабу твоему:
один ушел и погиб, так что его уже не нашли; и
только тот один остался от своей матери, и раб
твой, отец наш, любит его, и душа его привязалась к
этой душе. И будет, что если мы возвратимся к рабу
твоему, отцу нашему, и младшего не будет с нами, то
он умрет, и мы погубим нашего отца, и он умрет от
печали. Лучше я буду рабом твоим вместо дитяти,
рабом моего господина; но юноше позволь идти с
его братьями, ибо я поручился за него пред рабом
твоим, отцом нашим; и если ты не отдашь его, то раб
твой будет всегда виновным пред нашим отцом>>.

И Иосиф увидел, что все они были единодушными и
благожелательными друг к другу; и он не мог более
удерживаться, и сказал им, что он~--- Иосиф, и
разговаривал с ними по-еврейски, и пал им на шею, и
плакал; и они не узнали его. Теперь и они начали
плакать. И он сказал им: <<Не плачьте из-за меня.
Поспешите и приведите ко мне отца моего, чтобы я
увидел моего отца, прежде чем умру [...]. Ибо вот это
второй год голода, и еще предстоят пять лет, когда
не будет жатвы, и плода с деревьев, и никаких
растений. Поспешите с вашими домочадцами, чтобы
вам не погибнуть от голода и не быть в
беспокойстве за себя и за свое имущество. Ибо
Господь послал меня вам как вашего питателя,
чтобы остались в живых многие. И расскажите отцу
моему, что я жив еще. Вы сами видите, что Господь
поставил меня отцом Фараону и господином в доме
его и над всею страною Египетскою. И расскажите
отцу моему о всей моей славе и о всем богатстве и
славе, которые дал мне Господь>>. И он дал им по
повелению Фараона колесницы и пищу на дорогу и
дал им цветные одежды и серебро; и отцу их также
Фараон послал одежд, и серебра, и десять ослов,
которые везли хлеб. И он отпустил их, и они пошли и
рассказали своему отцу, что он жив, и что он всем
народам земли отпускает хлеб, и что он поставлен
господином над всею Египетскою землею. И отец их
не поверил этому, ибо он был поражен в своей душе.
И после сего он увидел колесницы, которые прислал
Иосиф; тогда опять ожил вновь дух его. И он сказал:
<<Довольно для меня, что Иосиф жив; я пойду и
увижу его, прежде чем умру>>.

\vs Jub 44:1
И Израиль пошел из своего жилища Хеврона в
новолуние третьего месяца, и зашел к клятвенному
колодезю, и принес жертву Богу отца своего Исаака
в седьмой день этого месяца. И Иаков вспомнил сон,
который он видел в Вефиле, и убоялся идти в
Египет. И, подумав, он хотел известить Иосифа,
чтобы он пришел к нему, и что он сам не пойдет; он
оставался там семь дней, ожидая, не увидит ли
он, быть может, видение о том, оставаться ли ему
или идти. И он совершил праздник жатвы~--- начатков
хлеба~--- со старым хлебом, ибо во всей стране
Ханаанской не было и пригоршни семян, но был
голод для всех зверей, и скота, и птиц, и людей. И в
шестнадцатый день явился ему Господь и сказал:
<<Иаков, Иаков!>> И он сказал: <<Вот я
здесь>>. И Он сказал ему: <<Я Бог отцов твоих,
Авраама и Исаака; не бойся и иди в Египет! Ибо Я
сделаю тебя там великим народом; Я пойду с тобою,
и приведу (возвращу) тебя в эту землю, чтобы ты был
погребен здесь. И Иосиф закроет своими руками
твои глаза. Не бойся, иди в Египет!>>

И они собрались, его дети и дети его детей, и
посадили своего отца и положили свое имущество
на колесницы. И Израиль пошел от клятвенного
колодезя в шестнадцатый день этого третьего
месяца и отправился в страну Египет. И Израиль
послал сына своего Иуду вперед себя к Иосифу,
чтобы осмотреть страну Гесем. Ибо сюда~--- так
сказал Иосиф братьям~--- они должны были прийти,
чтобы жить здесь, дабы быть им вблизи его. И это
хорошая страна в земле Египте; и она была
недалеко от него.

Вот имена сыновей Израиля, которые пошли с
своим отцом Иаковом в Египет. Иаков, отец их.
Робел, перворожденный Израиля. И вот имена его
сыновей: Енох, Фалус, Есером, Карами~--- пятеро.
Симеон и его сыновья; и вот имена его сыновей:
Иямуел, Иямин, Аод, Ияким, Саар, Саул, сын
Сефенсеянки~--- семеро. Левий и его сыновья; вот
имена сыновей его: Гедеон, Кааф и Мерари~--- четверо.
Иуда и его сыновья; и вот имена его сыновей: Селом,
Фарес, Зара~--- [четверо]. Исашар и его сыновья; и
вот имена его сыновей: Фола, Фуа, Иясоб, и Сам~---
пятеро. Заблон и его сыновья; и вот имена его
сыновей: Саор, и Елом, и Иялиел~--- четверо. И вот
сыновья Иакова, которых родила Иакову Лия в
Месопотамии, шесть сыновей и одна сестра их Дина.

И всех душ детей Лии и их детей, которые пошли со
своим отцом Иаковом в Египет, было двадцать
девять; с отцом их Иаковом было тридцать. И дети
Залафы, служанки Лии, жены Иакова, которых она
родила Иакову, суть Гад и Асер. И вот имена их
детей, которые пошли с ними в Египет. Дети Гада:
Сафион, Агафи, Суни, Асон, Араби, Аради~--- восьмеро.
И дети Асера: Иямна, Иесуа, Баръя и Сара, их сестра.
Всего четырнадцать душ. И всех детей Лии было
сорок четыре. И дети Рахили, жены Иакова,~--- Иосиф и
Вениамин. И у Иосифа родились в Египте, прежде чем
отец его пришел в Египет, сыновья, которых родила
ему Ассенеф, дочь Питфары Гелиопольского,~---
Манассе и Ефрем~--- трое. Дети Вениамина: Лаубаел,
Асбел, Гуав, Нееман, Абродио, Раифес, Ианини, Афим,
Яам, Гаам~--- одиннадцать. И всех детей Рахили было
четырнадцать. И дети Баллы, служанки Рахили, жены
Иакова, которых она родила Иакову,~--- Дан и
Неффалим. И вот имена их детей, которые пошли с
ними в Египет. Дети Дана: Куси, Самой, Асуд, Иясек,
Саломон~--- шестеро. И они умерли в Египте в тот год,
в который пришли, и у Дана остался только Куси. И
вот имена детей Неффалима: Иязиел, Гахан, Асаар,
Якум, Ау~--- шестеро. И умер Ау, родившийся после
первого голодного года. И всех детей Рахили
вместе было двадцать шесть. И всех душ Иакова,
пришедших в Египет, было семьдесят душ. Вот его
дети и дети его детей~--- всего семьдесят. И пятеро
из них умерли в Египте при Иосифе, не имея детей. И
в стране Ханаанской умерли у Иуды два его сына~---
Ер и Онан, не имея детей. И сыновья Израиля
похоронили тех, которые умерли, и они входят в
число семидесяти человек.

\vs Jub 45:1
И Израиль пришел в Египетскую землю, в страну
Гесем, в новолуние четвертого месяца во второй
год третьей седмины сорок пятого юбилея. Иосиф
вышел навстречу своему отцу Иакову, в страну
Гесем, и пал отцу на шею, и плакал. И Израиль
сказал Иосифу: <<Теперь я умру спокойно, так
как увидел тебя. И ныне да будет прославлен
Господь, Бог Израилев, Бог Авраама, и Бог Исаака,
Который не отвратил Своего милосердия и
благоволения от раба Своего Иакова! Довольно для
меня, что я увидел лицо твое, пока я жив. Да,
истинно видение, которое я видел в Вефиле. Да
будет прославлен Господь, Бог мой, во весь век!>>
И Иосиф и братья его ели пред очами своего отца
хлеб, и пили вино; и Иаков был исполнен великой
радости, что видел Иосифа, как он с братьями
своими пред его глазами ел и пил. И он прославил
Творца всех вещей, Который сохранил его, и
сохранил ему двенадцать его сыновей. И Иосиф дал
своему отцу и своим братьям в дар страну Гесем,
чтобы они жили в ней и в Рамизифино и во всей ее
области, чтобы они владели ею пред глазами
Фараона. И Израиль жил с своими сыновьями в
стране Гесем, лучшей части земли Египетской.
Израиль же был ста тридцати лет, когда он пришел в
Египет. И Иосиф снабжал своего отца, и своих
братьев, и их домочадцев съестными припасами,
насколько они нуждались в них, в продолжение семи
лет голода. И земля Египетская страдала от
голода. И Иосиф подчинил всю страну Египет
Фараону за хлеб, и также людей и скот; все
приобрел Фараон.

И кончились неурожайные годы, и Иосиф дал
народам, жившим в стране, семян и съестных
продуктов, чтобы они посеяли их в восьмой год; ибо
река наводнила всю страну Египет. Именно в семь
лет неурожая она орошала только отдельные места
около берега реки; теперь же она переполнилась. И
египтяне засеяли страну, и она принесла в том
году много хлеба, и это был первый год четвертой
седмины сорок литого юбилея. И Иосиф взял из
хлеба, который они засевали, пятую часть для царя,
и четыре (части) оставил им в пищу и для посева. И
Иосиф сделал это законом для Египетской земли до
сего дня.

И Израиль жил в стране Египте семнадцать лет, и
всей его жизни было три юбилея, сто сорок семь
лет. И он умер в четвертый год пятой седмины сорок
пятого юбилея. И Израиль благословил своих
сыновей пред своею смертью, и сказал им все, что
случится с ними в последние дни; все возвестил он
им, н благословил их. И он дал Иосифу две части в
стране. И он почил с своими отцами, и был погребен
в двойной пещере в земле Ханаанской, рядом с
своим отцом Авраамом, в могиле, которую он
выкопал для себя, в двойной пещере, в стране
Хеврон. И он отдал все свои книги и книги своих
отцов сыну своему Левию, чтобы он хранил их и
возобновлял их для своих детей до сего дня.

\vs Jub 46:1
И было, после того как Иаков умер, умножились
сыны Израиля в стране Египетской и сделались
многочисленными; и они были все единодушными в
своих мыслях, так что брат любил своего брата, и
каждый помогал своему брату; и они умножились
чрезмерно. И было всей жизни Иосифа десять
седмин, которые он прожил после своего отца. И
Иосиф не имел зложелателя, и не случилось с ним
чего-либо худого во все время его жизни, которую
он прожил после отца своего Иакова. Ибо все
Египтяне почитали сынов Израиля в продолжение
всего времени, пока жил Иосиф. И Иосиф умер ста
десяти лет; семнадцать лет он пробыл в стране
Ханаанской, и десять лет был слугою, и три гада
пробыл в темнице, и восемьдесят лет у царя
управлял всею страною Египетскою, И он умер, и все
его братья, и весь тот род.

И он завещал сынам Израиля перед смертью, чтобы
они взяли с собою его кости, когда они выйдут из
Египта. И он взял с них клятву относительно
костей своих; ибо он знал, что Египтяне не отнесут
его тело и не похоронят его в свое время в
стране Ханаанской, так как Ханаанский царь
Мемкерон, владевший страною Ассур, сражался в
долине с царем Египетским, и убил там его, и
преследовал Египтян до ворот Эромона. Но он не
мог вступить в Египет, ибо восстал другой
новый царь над Египтом для управления, и был
могущественнее его. И он возвратился в страну
Ханаанскую, а ворота Египта были заперты, и никто
не приходил в Египет.

И Иосиф умер в сорок шестой юбилей в шестую
седмину во второй год, и они похоронили его в
стране Египетской. И все братья его умерли после
него. И царь Египетский выступил, чтобы сразиться
с царем Ханаанским, в сорок седьмой юбилей во
вторую седмину во второй год. И сыны Израиля
вынесли кости всех сыновей Иакова, кроме Иосифа,
и похоронили их на поле, в двойной пещере на горе.
И большинство возвратилось в Египет; и только
немногие из них остались на горе Хеврон, и твой
отец [Амрам] остался с ними. И царь Ханаанский
победил царя Египетского, и запер ворота Египта.

И он (царь Египетский) замыслил недоброе дело
против сынов Израиля~--- притеснять их, и сказал
египтянам: <<Вот народ сынов Израиля возрос и
сделался многочисленнее нас; употребим же против
них хитрость, прежде чем они слишком размножатся,
и будем притеснять их рабскою работою, прежде чем
нас постигнет поражение и они победят нас в
битве. А не то они вступят в союз с врагами и
выйдут из нашей страны; ибо их сердце и лицо
обращено к стране Ханаанской>>. И он поставил
над ними смотрителей за работами, чтобы они
притесняли их рабскою работою. И они должны были
строить крепкие города для Фараона~--- Питофо и
Рамзе, и должны были строить всякие стены и
оплоты, которые обрушились в городах египетских,
и они сильно притесняли их. Но чем хуже поступали
они с ними, тем больше умножались и увеличивались
они. И египтяне считали сынов Израиля нечистыми.

\vs Jub 47:1
И в седьмую седмину в седьмой год сорок
седьмого юбилея пришел отец твой из страны
Ханаанской, и ты родился в четвертую седмину, в
шестой год, в сорок восьмой юбилей, когда были дни
преследования сынов Израиля. И царь Фараон
Египетский дал повеление относительно них, чтобы
детей их~--- всякое дитя мужеского пола, которое
родится,~--- бросали в реку. И они бросали их в
течение семи месяцев до того месяца, когда ты был
рожден. И твоя мать скрывала тебя три месяца, и на
нее донесли. Тогда она сделала для тебя корзину, и
осмолила ее смолою и асфальтом, и положила ее в
траву на берегу реки, и клала тебя в нее в течение
семи дней. И мать твоя приходила ночью и кормила
тебя грудью; и днем тебя стерегла от птиц сестра
твоя Мария.

И в те дни пришла дочь Фараона Фармуф
искупаться в реке. И она услышала твой голос,
когда ты плакал, и сказала своей служанке, чтобы
она принесла тебя. И она принесла тебя к ней. И она
вынула тебя из корзинки, и сжалилась над тобою. А
сестра твоя сказала ей: <<Не пойти ли мне, и не
призвать ли к тебе одну из евреек, чтобы она
воспитала это дитя для тебя и кормила грудью?>>
И она пошла, и призвала твою мать Ийокабиф, и она
дала ей плату, чтобы она ходила за тобою. И после
сего ты возрос, и тебя привели в дом Фараона, и ты
сделался отроком. И твой отец (Амбран) научил тебя
писанию. И после того как ты окончил три седмины,
он привел тебя в царский дворец, и ты был при
дворе три седмины до того времени, когда ты вышел
из царского дворца и увидел египтянина, который
бил твоего друга из сынов Израиля. И ты убил его и
скрыл его в песке. И в следующий день ты встретил
двоих из сынов Израиля, которые ссорились, и
сказал обидчику: <<Зачем ты бьешь своего
брата?>> И он разгневался, и озлобился, и сказал:
<<Кто поставил тебя начальником и судьею над
нами, разве ты хочешь убить меня, как ты убил
египтянина?>> И ты испугался и убежал
вследствие этих слов.

\vs Jub 48:1
И в шестой год третьей седмины сорок девятого
юбилея ты ушел и оставался (вне Египта) шесть
седмин и один год. И ты возвратился в Египет во
вторую седмину во второй год в пятидесятый
юбилей. И ты знаешь, что Он говорил с тобою у горы
Синай, и что высший Мастема хотел сделать с тобою
на пути, когда ты возвращался в Египет, в праздник
кущей. Не хотел ли он всеми силами умертвить тебя
и спасти египтян от рук твоих, когда увидел, как
ты был послан совершить над египтянами суд и
мщение? И я спас тебя от руки его и совершил
знамения и чудеса, которые ты был послан
совершить в Египте пред Фараоном, и всем его
домом, и рабами его, и его народом. И Господь
совершил мщение над ними, тяжкое мщение за
Израиля, и поражал, и умерщвлял их чрез кровь, и
чрез жаб, и мошек, и песьих мух, и злокачественные
воспалительные нарывы, и их скот Он поразил смертию,
и градом~--- чрез это Он истребил все, что росло у
них,~--- и саранчой, которая поела остаток,
оставшийся от града, и тьмою; и их первенцев из
людей и скота Он истребил. И всем их идолам
отметил Господь и сожег их огнем. И все это
послано было чрез твою руку, чтобы ты совершил
это, [...]. И ты говорил с царем египетским, и пред
всеми его служителями, и пред его народом; и все
случилось по твоему слову; десять великих и
страшных наказаний пришли на страну Египетскую,
чтобы чрез них отметить за Израиля. И все это
совершил Господь за Израиля и согласно завету,
который Он заключил с Авраамом, чтобы отметить
им, ибо они жестоко притесняли их. И высший
Мастема восстал против тебя, и хотел предать тебя
в руки Фараона, и содействовал египетским
волхвам, и помогал им, чтобы и они сделали это
пред твоими глазами. Хотя мы и допустили их
произвести зло, но, однако, не позволили им
врачебных средств, чтобы они воспользовались ими
своими руками. И Господь поразил их (волхвов)
злокачественными нарывами, чтобы они не могли
противостоять ему; ибо мы погубили их, чтобы они
не могли совершить ни одного знамения. Но
несмотря на все знамения и чудеса, высший Мастема
не смутился, ибо он приложил все силы и воззвал к
египтянам, чтобы они преследовали тебя всеми
силами Египта, с своими колесницами и конями, и со
всем множеством народов Египта. И я встал между
тобою и ими, между египтянами и израильтянами, и
спас израильтян от руки их, от руки египтян. И
Господь провел их чрез море, как по сухой земле; и
всех людей, которые выступали для преследования
Израиля, Господь Бог наш ввергнул в море, в
глубину бездны, вместо детей Израиля, за то, что
египтяне бросали их в реку сотня за сотней; за это
совершено над ними мщение, и тысяча сильных мужей
[...] была истреблена за одного погибшего младенца
из детей твоего народа, брошенного ими в реку. В
четырнадцатый, и в пятнадцатый, шестнадцатый,
семнадцатый и восемнадцатый дни высший Мастема
был связан и заключен позади сынов Израиля, чтобы
он не мог обвинять их (пред египтянами). А в
девятнадцатый день мы освободили его, чтобы он
помогал египтянам и чтобы они преследовали сынов
Израиля. И он очерствил сердце их, и ожесточил их,
и стал могущественным над ними по воле Господа,
Бога нашего, чтобы поразить египтян и ввергнуть
их в море. И в пятнадцатый день мы связали его,
чтобы он не обвинял сынов Израиля, в тот день,
когда они требовали у египтян утварь и одежды,
утварь серебряную, золотую и медную, чтобы
обобрать египтян за то, что когда они служили им,
они сильно притесняли их; и мы не допустили, чтобы
сыны Израиля вышли из Египта с пустыми руками.

\vs Jub 49:1
Вспомни заповедь, которую дал тебе Господь
относительно пасхи, чтобы ты соблюдал ее в свое
время, в четырнадцатый день первого месяца, чтобы
ты заколол его (агнца), прежде чем наступит вечер,
и чтобы ели его ночью, в вечер пятнадцатого дня, с
солнечного захода. Ибо день этот есть первый
праздник и первый день пасхи. И вы ели пасху в
Египте, в то время как все силы Мастемы были
освобождены, чтобы умерщвлять всякого первенца в
стране Египетской, от первенца фараонова до
первенца пленной рабыни на мельнице и до екота. И
вот знамение, которое дал им Бог. В каждый дом, у
которого дверной косяк был обрызган кровью
агнца, в этот дом они не должны были входить для
избиения находящихся в нем, так что все, бывшие в
этом доме, спаслись, потому что на двери был знак
крови. И силы Господин сделали все, что только
Господь повелел им, и прошли мимо всех сынов
Израиля. И на них не простерлось бедствие, чтобы
погубить из них чью-либо душу, ни на скот, ни на
человека, ни даже на собаку. В Египте же бедствие
было очень велико, и не было дома, в котором не
было бы умершего, и плача, и сетования. И весь
Израиль спокойно вкушал пасхальное мясо, и пил
вино, и хвалил, и благодарил, и прославлял
Господа, Бога отцов своих, и приготовлялся к
исходу из-под ига. рабства и из злого Египта. И ты
помни этот день во все дни твоей жизни, раз в год,
в свой (определенный) день, согласно со всем
законом относительно сего, и не смешивай этого
дня с другими и этого месяца с другим. Ибо это~---
вечное установление, и оно начертано на небесных
скрижалях для сынов Израиля, чтобы они каждый год
соблюдали праздник, один раз в год, во все свои
роды; и нет предела времени сему, но он (праздник)
утвержден навек. И муж, если он чист и не придет
совершить его в назначенный день, чтобы принести
дар, угодный Господу, в день Его праздника и чтобы
есть и пить пред Господом в день Его праздника,
тот муж должен быть истреблен, если он чист и
находится недалеко, ибо не принес дар Господень в
назначенное время. И грех примет на себя тот муж.
Сыны Израиля, грядущие, должны праздновать пасху
в назначенное для нее время, в четырнадцатый день
первого месяца вечером, в третью часть дня до
третьей части ночи. Ибо две части дня назначены
для света и третья~--- для вечера. Вот то, что
повелел Господь, чтобы ты совершал это в исходе
вечера. И не должно совершаться это утром в
какой-либо час света, но в вечернее время. И они
должны вкушать его в вечернее время до третьей
части ночи, и что останется от всего мяса после
третьей части ночи, они опять должны сожечь
огнем. И они не должны варить его в воде, и не
должны его есть сырым, но тщательно испекши на
огне и изжарив на огне. Его голову, со
внутренностями и ногами его, они должны изжарить
на огне и не раздроблять ему костей. Ради сего
Господь повелел сынам Израиля, чтобы они
праздновали пасху в назначенный для нее день и не
раздробляли у него (агнца) костей; ибо это
праздничный день и назначенный для празднования
день, и нельзя уклоняться от него на день или на
месяц, но в свой праздничный день он должен
праздноваться. И ты скажи сынам Израиля, чтобы
они соблюдали пасху в ее день, ежегодно, один раз
в год, в определенный день, чтобы это было
воспоминанием, которое будет приятно для
Господа, и чтобы не случилось с ними в том году
никакого бедствия и они не были бы умерщвлены и
поражены. Если они будут праздновать пасху в свое
время, соблюдая все, как заповедано, то они не
должны вкушать ее вне святилища Господня; пред
всем народом общества Израилева должны
соблюдать ее в свое время все люди, которые
явились в день ее, чтобы вкушать пред Господом в
святилище вашего Бога, кто имеет двадцать лет и
выше. Ибо так написано и определено, чтобы ели ее
в доме святилища Господня. И когда сыны Израиля
придут в страну, которою они будут владеть, в
страну Ханаанскую, и устроят скинию Господню в
сей стране, в одном из своих отрядов (колен), так
что святилище Господа будет устроено в стране, то
они должны приходить и праздновать пасху среди
скинии Господней, и закалать ее пред Господом из
года в год. И во дни, когда будет устроен дом во
имя Господне в стране их наследия, они должны
ходить туда и закалать пасху вечером, когда
зайдет солнце, в третью часть дня, и должны
окропить кровью порог алтаря, и тук положить на
огонь, который на жертвеннике, мясо же его,
изжаренное на огне, есть в преддверии дома
святилища во имя Господне. И они не должны
совершать пасху в своих городах и в своих местах,
а только пред скиниею Господнею, или пред Его
домом, так как имя Его живет в нем, дабы им не
согрешить пред Господом. И ты, Моисей, скажи сынам
Израиля, чтобы они соблюдали постановление о
пасхе, как повелено тебе, что вы должны соблюдать
ее ежегодно в день ее и также праздник
опресноков, чтобы они ели пресное в продолжение
семи дней, соблюдая праздник Его и принося для
Него ежедневно дар, в те семь пасхальных дней,
пред Господом, на жертвеннике вашего Бога. Ибо
этот праздник вы праздновали с боязливою
робостию, когда вы вышли из Египта, пока не
перешли чрез море в пустыню Сур; ибо на берегу
моря вы окончили его.

\vs Jub 50:1
И потом после сего закона я возвестил тебе о
субботних днях в пустыне Синая, которая
находится между Еломом и Синаем. И также о
субботах земли я сказал тебе на горе Синай и о
юбилейных годах вместе с субботними годами.

Но год его мы не сказали тебе, пока ты не придешь
в страну, которою вы будете владеть. Тогда и
страна должна праздновать свои субботы, когда
они будут жить в ней, и они узнают год юбилея.
Посему я определил тебе седмины и юбилейные годы:
сорок девять юбилейных годов от дней Адама до
сего дня и одна седмина и два года. И еще
предлежат тебе сорок лет, чтобы узнать заповеди
Господа, пока они не переправятся чрез Иордан к
западу. И юбилеи прекратятся, когда Израиль
очистится от всякого блуда, и вины, и нечистоты, и
осквернения, и греха, и злодеяния, и спокойно
будет жить во всей стране, и против него не
восстанет более ни сатана, ни какой-либо
ненавистник, и земля будет с тех пор чистою
всегда.

И вот я записал тебе также повеление
относительно суббот, и все установления законов
относительно них: шесть дней делай дела, и в
седьмой день суббота для Господа Бога вашего. Вы
не должны делать в нее никакого дела, вы, и ваши
сыновья, и ваши рабы, и служанки, и весь ваш скот, и
чужеземец, который у тебя. И человек, который
делает какое-либо дело, должен умереть. Всякий,
кто оскверняет этот день, кто спит с своею женою,
и кто говорит о том, что он хочет предпринять в
нее (в субботу) путешествие или о разного рода
купле и продаже, и кто черпает воду, не приготовив
ее себе в шестой день, и кто поднимает ношу, чтобы
перенести ее из своего шатра или из своего дома,
тот должен умереть. Вы не должны делать никакого
дела в субботу, которого вы не приготовили себе в
шестой день, чтобы есть, и пить, и покоиться, и
соблюдать субботу от всякого дела в этот день, и
прославлять Господа Бога вашего, Который дал ее
вам в праздник. И днем святым, и даем святого
царства для всего Израиля должен быть этот день в
вашей жизни непрестанно. Ибо велика честь,
которой Господь удостоил Израиля, чтобы они ели,
и пили, и насыщались в этот праздничный день, и
отдыхали от всякого дела, которое относится к
человеческим делам, кроме воскурения фимиама и
принесения даров и жертв пред Господом в субботы.
Только это дело пусть совершается в субботы, во
дни дома святилища Господа Бога вашего, чтобы
приносить в умилостивление за Израиля
непрестанно и ежедневно дары в воспоминание,
которое приятно и которое делает их угодными
пред Господом, каждый день года, как повелено
тебе. Но каждый человек, который совершает дело, и
предпринимает путешествие, и ухаживает за своим
скотом, будь это дома или в другом месте, и кто
зажигает огонь, или едет верхом на каком-нибудь
животном, или путешествует на корабле по морю, и
каждый, кто убивает и умерщвляет кого-либо, и кто
закалывает животное или птицу, и кто ловит зверя,
или птицу, или рыбу, и кто постится, и кто ведет
войну в субботний день; всякий, кто делает
что-нибудь из этого в субботний день, тот должен
умереть, чтобы дети Израиля хранили субботу по
заповедям о субботах земли, как это списано с
небесных скрижалей, которые Он дал мне в мои руки,
дабы я написал тебе законы времени и время по
делению его дней.

\chhdr{Отрывки из Книги Юбилеев, сохранившиеся у греческих церковных писателей}
\chhdr{1. Св. Епифаний Кипрский}
Но в Книге Юбилеев, называемой также и Малым
Бытием, можно найти, что эта книга содержит имена
жен Каиновой и Сифовой, чтобы всяким образом были
посрамлены эти слагатели басен для жизни (т.е.
Сифияне). Когда Адам родил сынов и дочерей, было
необходимостью в то время, чтобы его сыновья
вступили в брак с собственными сестрами; ибо это
не было беззаконным, потому что иного рода не
было. Да и сам Адам, можно сказать, был в
супружестве почти с собственной дочерью,
образованною из его тела и созданною Богом для
супружества с ним, и это не было беззаконным. Так
и сыновья его вступили в брак~--- Каин с старшею
сестрою, так называемой Савою, а Сиф, третий сын,
рожденный после Авеля, с сестрою своею,
называемой Азурою. У Адама родились, как
описывает Малое Бытие, и другие девять сыновей,
после тех трех, так что у него было две дочери, а
детей мужеского пола двенадцать: один убитый, а
одиннадцать оставшихся в живых. Ты имеешь
указание на это в Бытии мира и первой книге
Моисеевой, где говорится так: <<и поживе Адам
лет девятьсот тридесять, и роди сыны и дщери, и
умре>>. (Ср. Кн. Юбил., IV).

\chhdr{2. Иоанн Зонара}
Действительно я знаю записанное в Малом Бытии,
что в первый день и небесные силы прежде прочего
были созданы Творцом вселенной; но так как это
Малое Бытие не отнесено к книгам еврейской
мудрости, написанным божественными отцами, то я
ничего, что в ней написано, не считаю достаточно
твердым и не соглашаюсь с этим учением (ср. Кн. Юбил., II).

\chhdr{3. Георгий Синкелл}
В первосозданные сутки, по-еврейскому, в первый
день первого месяца Нисана, как указано прежде,
по-римскому, в двадцать пятый день месяца марта и,
по-египетскому, в двадцать девятый день Фаменофа,
в день божественный, именно в первую неделю, Бог
сотворил небо и землю, мрак и воды, дух и свет, и
сутки, всего семь творений. Во вторые сутки
явилась твердь~--- одно творение. В третьи сутки
было четыре творения~--- появление земли и осушение
ее, рай, многоразличные деревья, травы и семена. В
четвертый день Бог сотворил солнце, и луну, и
звезды. В пятый день Бог сотворил пресмыкающихся
и всех плавающих, великих морских животных и рыб,
и все, что в водах, а также пернатых~--- всего три
творения. В шестой день Бог сотворил
четвероногих и пресмыкающихся на земле, зверей и
человека~--- четыре творения. Вместе всех творений~---
двадцать два, соответственно двадцати двум
еврейским буквам, затем двадцати двум еврейским
книгам и наконец двадцати двум генерациям от
Адама до Иакова, как говорится в Малом Бытии,
которое называют иные откровением Моисея. Эта же
книга говорит, что небесные силы были сотворены в
первый день (Кн.Юбил.,II).

Необходимость побудила меня сообщить нечто и
из того, что и другими историками, записавшими
иудейские древности и христианские
повествования, заимствуется о сем из Малого
Бытия и так называемой Жизни Адама~--- хотя она и не
считается божественною,~--- чтобы исследующие это
не впали в нелепейшие вымыслы. Итак, в известной
под именем Жизни Адама указывается число дней,
когда было наименование животных, и образование
жены, и вход Адама в рай, и заповедь Божия к нему о
пище с дерева, и вход Евы после сего в рай, также
обстоятельства преступления заповеди и
последствия преступления, как далее следует.

В первый день недели, который был третьим от
сотворения Адама, восьмой первого месяца Нисана,
первый месяца апреля и, по-египетскому, шестой
месяца Фармуфи, Адам по некоему божественному
благоизволению наименовал диких зверей; во
второй день второй недели он дал имена скотам; в
третий день второй недели он наименовал
пернатых; в четвертый день второй недели он
наименовал пресмыкающихся; в пятый день второй
недели он наименовал плавающих. В шестой день
второй недели, который, по-римскому, был шестой
день апреля, а по-египетскому, одиннадцатый
месяца Фармуфи, Бог, взявши некую часть ребра
Адамова, образовал жену. В сорок шестой день от
сотворения мира, в четвертый день седьмой недели,
четырнадцатого Пахона, девятого мая, когда
Солнце было в знаке Тельца и Луна против
созвездия Скорпиона, в восход Плеяд, Бог ввел
Адама в рай в сороковой день после его
сотворения. В пятидесятый день от сотворения
мира, в сорок четвертый от сотворения Адама, день
божественный, восемнадцатого Пахона,
тринадцатого мая, через три дня после входа его в
рай, когда Солнце было в знаке Тельца и Луна в
знаке Козерога, Бог заповедал Адаму не вкушать от
древа познания.

В девяносто третий день творения, во второй
день четырнадцатой недели, во время летнего
поворота Солнца, когда и Солнце и Луна были в
созвездии Рака, в двадцать пятый день месяца
июня, первого Епифи, введена была Богом в рай
помощница Адама Ева, в восьмидесятый день по
сотворении ее. Взяв ее, Адам дал ей имя~--- Ева, что
значит жизнь. Посему Бог повелел чрез Моисея в
книге Левит, именно, ради дней пребывания их вне
рая по сотворении, чтобы она (женщина) оставалась
нечистою при рождении мальчика сорок, а при
рождении девочки восемьдесят дней; посему и Адам
в сороковой день по сотворении введен был в рай,
ради чего и новорожденных в сороковой день
приносят в храм по закону. При рождении же
девочки она должна быть нечистою восемьдесят
дней, ради того, что она (Ева) вошла в рай в
восьмидесятый день, и ради женской нечистоты в
отношении к мужу; даже и находящаяся в месячном
очищении не входит в храм до семи дней по
божественному закону.

Это я ради любознания в сокращении заимствовал
из так называемой Жизни Адама (Кн.Юбил.,III).

Из Малого Бытия:

В седьмой год он согрешил, и в осьмой они были
изгнаны из рая, как говорит (Малое Бытие), чрез
сорок пять дней после падения, в восход Плеяд.

Пробыл же Адам в раю седмину трехсот
шестидесяти пяти дней; и изгнан был с женою Евою
за преступление заповеди в десятый день месяца
мая.

Звери, и четвероногие и пресмыкающиеся, говорит
Иосиф и Малое Бытие, до падения говорили одним
языкам с первосотворенными; посему, говорит, змей
беседовал с Евою человеческим голосом, что,
кажется, невозможно.

В восьмой год (говорит) Адам познал Еву, жену
свою.

В восьмидесятый род родился у них первородньм
сын Каин.

В семьдесят седьмой год, говорят, родился
праведный Авель.

В восемьдесят пятый год родилась у них дочь, и
они дали ей имя Асуам.

В девяносто седьмой год Каин принес жертву.

В девяносто девятый год Авель принес жертву
Богу, имея от роду двадцать два года, в полнолуние
седьмого еврейского месяца, то есть в праздник
кущей.

Достойно примечания, что Писание называет
жертву Каина принесением плодов, а жертву Авеля
дарами, обозначая сим настроение каждого.

В тот же девяносто девятый год Каин убил Авеля,
и первозданные оплакивали его четыре седмины, то
есть двадцать восемь лет.

В сто двадцать седьмой год Адам и Ева
прекратили свой плач. В сто тридцать пятый год
Каин взял собственную сестру Асавнан, которой
было пятьдесят лет; а сам он был шестидесяти пяти
лет (Кн.Юбил.,III,IV).

В двести тридцать четвертом году он родил дочь,
которой дал имя Азуран (Кн.Юбил.,IV).

В четыреста двадцать пятом году Сиф взял в жены
собственную сестру Азуран; Сиф же был девяносто
одного года (Кн.Юбил.,IV).

В том же девятьсот тридцатом году умер и Каин от
обрушившегося на него дома; ибо и сам он камнями
убил Авеля (Кн.Юбил.,IV).

В этом 2251 году, как говорят, Ной насадил
виноградник на горе Лувар в Армении (Кн.Юбил.,VII).

Ангел, говоривший с Моисеем, сказал ему: я
научил Авраама еврейскому языку, каким он был от
начала творения, чтобы он говорил на нем, как на
природном, о чем говорится в Малом Бытии (XII).

В 3373 году от сотворения мира, когда Аврааму был
61 год, сожег Авраам идолов отца своего, и вместе с
ними сожжен был Арран, хотевший тушить огонь
ночью. И вышел Фарра с Авраамом, чтобы идти в
землю Ханаанскую, и, переменив намерение, жил в
Харране, предаваясь идолопоклонству до своей
смерти (Кн.Юбил.,XII).

В сто пятьдесят третьем году жизни Исаака,
Иаков возвратился к нему из Месопотамии. И Исаак,
возведя очи и увидя сыновей Иакова, благословил
Левия, как первосвященника, и Иуду, как царя и
начальника. Ревекка побудила Исаака, уже бывшего
в старости, чтобы он внушил Исаву и Иакову любить
друг друга. И он, увещевая их, предсказал, что если
Исав восстанет на Иакова, то впадет в руки его. И
вот после смерти Исаака, Исав, возмущаемый своими
сыновьями, собрав людей, вышел войною против
Иакова и его сыновей. Иаков, заперев ворота башни,
увещевал Исава вспомнить родительские
завещания. Когда же он не склонился на увещания, а
напротив, стал оскорблять и поносить его, Иаков,
побуждаемый Иудой, натянул лук и поверг Исава,
поразив его в правый сосок груди. После его
смерти сыновья Иакова, открыв ворота, перебили
весьма многих. Это говорится в Малом Бытии
(XXXVII,XXXVIII).

\chhdr{4. Михаил Глика}
Не потому, что он (змий) имел прежде ноги, как
говорит Иосиф и так называемое Малое Бытие,
теперь Бог объявляет, что он будет ходить на
чреве; но, как объясняет Златоустый Иоанн, прежде
он благодаря прямому положению имел такую
смелость, что приблизился к самому уху Евы и
разговаривал с ней, а теперь осужден, и
совершенно справедливо, ползать по земле
(Кн.Юбил.,III).
Малое Бытие говорит, что Адам неосмотрительно
взял от древа и ел и не обратил полного внимания
на слова Евы, потому что изнемог от труда и
голода. Но об этом, возлюбленный, лучше умолчать,
ибо, как сказано выше, бывает нечто достойное и
молчания; разве только и ты хочешь говорить, что
Адам взял жену, чтобы не обратиться на других
животных. Змий стал пресмыкающимся из скота, и
имел руки и ноги; но это было отнято ради того, что
он дерзновенно вошел в рай и посему первый взял
от древа и ел. Адам отгонял птиц и пресмыкающихся,
собирал плод в раю и ел его с своею женою. Вот
это-то, чтобы не сказать, и еще гораздо большее из
подобного, содержит Малое Бытие. Но оставь это;
ибо иначе относящимся к Священному Писанию (это)
покажется, напротив, смешным и забавным (Кн.Юбил.,III).

\chhdr{5. Георгий Кедрин}
В Малом Бытии говорится, что Мастифат,
начальник демонов, приблизясь к Богу, сказал Ему:
если Авраам любит Тебя, пусть принесет Тебе в
жертву сына своего (Кн.Юбил.,XVII).
Ревекка, приготовив кушанье, отдала его Иакову
и ввела его вместе с другими дарами для Исаака к
Аврааму; взяв его на свое лоно и многообразно
благословив его, Авраам, почивши, умер, на
пятнадцатом году жизни Иакова (Кн.Юбил.,IXX).
В Малом Бытии говорится, что израильские дети
были бросаемы в реку только в течение десяти
месяцев, пока Моисей не был поднят царицею.
Посему на египтян были посланы десять казней в
течение десяти месяцев, и наконец они были
ввергнуты в море, по образу того, как они погубили
в реке еврейских детей,~--- за одного израильского
мальчика тысяча погубленных сильных мужей из
египтян. Самого же Моисея дочь Фараона усыновила
в царском достоинстве, но, конечно, не освободила
израильтян от порученной им работы
(ср.Кн.Юбил.,XLVII,XLVIII).
Моисей первый написал законы для иудеев.
Оставив занятия, соответственные Египту, Моисей
в пустыне изучал мудрость, получая откровения от
архангела Гавриила о происхождении мира и
первого человека, о бывшем после него, о потопе, о
смешении и многообразии языков, о событиях из
жизни первого человека, о происшествиях до его
времени, о законе, который он должен был дать
народу иудейскому, также о положении звезд, о
стихиях, арифметике, геометрии и всякой мудрости,
как говорится в Малом Бытии (ср.Кн.Юбил.,I).

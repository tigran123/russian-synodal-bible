\documentclass{book}
\usepackage[nofnpara,keepfnmark,vssup,nomarnvs,chapbookm,lettrine,nogeometry]{bible}
\usepackage{xltxtra,polyglossia,fancybox,multicol}
%\multicoltolerance=3900
\tolerance=9990
\let\rsbpar\par
\setdefaultlanguage{russian}

\DeclareTextAccent{\'}{EU1}{180}

% US Letter, two columns
%\tunemarkuptag{pgletter}

% 6" (9x12cm) Kindle ebook reader
\tunemarkuptag{pghanlin}

\tunemarkup{pghanlin}{\RequirePackage[dvips,headheight=14pt,headsep=0pt,vmargin={0.2in},hmargin={0.1in},marginparsep=0pt,twoside=false,papersize={90mm,120mm}]{geometry}}
\tunemarkup{pgletter}{\RequirePackage[dvips,headheight=20pt,headsep=8pt,vmargin={0.6in,0.5in},hmargin={0.5in},twoside=true,papersize={8.25in,10.75in}]{geometry}}

\makeatletter
% fix page dimentions in xelatex - can be dropped when geometry.sty
% will become xetex-aware and/or geometry.cfg in texlive will be fixed
\@ifundefined{XeTeXversion}{}{\def\Gm@checkdrivers{\Gm@setdriver{pdftex}}}
\makeatother

%\tunemarkup{pghanlin}{\setmainfont[Mapping=tex-text,BoldFont=Garamond Premier Pro Bold]{Garamond Premier Pro}}
%\newfontfamily\cyrillicfont[Mapping=tex-text]{Garamond Premier Pro}
\tunemarkup{pgletter}{\setmainfont[Mapping=tex-text]{Minion Pro}}

%\tunemarkup{pghanlin}{\setmainfont[Mapping=tex-text,Script=Cyrillic]{Octava}}
%\newfontfamily\cyrillicfont[Mapping=tex-text,Script=Cyrillic]{Octava}

\tunemarkup{pghanlin}{\setmainfont[Mapping=tex-text,Script=Cyrillic]{Academy}}
\newfontfamily\cyrillicfont[Mapping=tex-text,Script=Cyrillic]{Academy}

\newfontfamily\engfont[Mapping=tex-text]{Linux Libertine O}
\newfontfamily\armfont[Mapping=tex-text]{ArTarumianGrqiNor_U}
\newcommand{\latintext}{\normalfont}

\bibmetadata
{Russian Bible with Apocrypha}
{Russian Synodal Bible (Bibles.org.uk 6th Edition)}
{russian,synodal,1876,bible}
{20050408174820}
{Russian}

\def\mytoday{{\ddmmyyyydate\today}}
\columnseprule=0.0pt

\tunemarkup{pgletter}{%
\definecolor{ubdarkblue}{rgb}{0.0, 0.0, 0.5}
\renewcommand{\bibtocheadfont}{\fontsize{14}{14}\selectfont}
\renewcommand{\bibtocfont}{\fontsize{12}{14}\selectfont}
\renewcommand{\bibheadfont}{\Large\addfontfeature{Letters=SmallCaps}}
\renewcommand{\bibheadchapsize}{\bfseries\large}
\renewcommand{\bibheadversesize}{\bfseries\normalsize}
\renewcommand{\bibheadpagesize}{\bfseries\large}
}

\tunemarkup{pghanlin}{%
\definecolor{ubdarkblue}{rgb}{0.0, 0.0, 0.0}
\renewcommand{\bibtocheadfont}{\fontsize{11}{11}\scshape}
\renewcommand{\bibtocfont}{\fontsize{8}{10}\selectfont}
\renewcommand{\bibheadfont}{\normalsize\scshape\addfontfeature{Letters=SmallCaps,LetterSpace=2.0}}
\renewcommand{\bibheadchapsize}{\bfseries\normalsize}
\renewcommand{\bibheadversesize}{\bfseries\small}
\renewcommand{\bibheadpagesize}{\bfseries\small}
%\renewcommand{\LARGE}{\fontsize{26}{18}\selectfont}
%\renewcommand{\Huge}{\fontsize{30}{24}\selectfont}
}

\renewcommand{\bibdropcapscolor}{ubdarkblue}

\renewcommand{\bibmarnchfont}{\normalsize\bfseries\upshape}
\renewcommand{\bibbookend}{\fontsize{15}{17}\selectfont}
\newcommand{\bibmainfontsize}{15/17}
\renewcommand{\footnotesize}{\fontsize{13}{15}\selectfont}
\renewcommand{\bibpage}{\tiny Стр.}
\renewcommand{\bibpsalmname}{Псалом}
\renewcommand{\bibchapname}{Глава}
%\renewcommand{\bibpsalmname}{Psalm}
%\renewcommand{\bibchapname}{Chapter}
%\renewcommand{\bibcontname}{Оглавление}
\renewcommand{\bibcontname}{Contents}
\renewcommand{\bibtitlefont}{\large\scshape}

%\renewcommand{\LARGE}{\large}
%\renewcommand{\Huge}{\Large}

\setlength{\fboxsep}{1.5pt}
\makeatletter
\def\bibchap{\if@firstvs@bk\else\noindent{\doublebox{\normalfont\Large\bfseries\thechnum}}\nobreakspace\fi}
%%\def\bibchap{\if@firstvs@bk\else\printvssup{\curr@vs}\fi}
\protected\def\print@vssup#1{\@textsuperscript{\fontsize{6}{6}\selectfont#1}\kern0.1em\relax}
\makeatother

% to start a book on a new page, as an exception
%\newcommand{\newbookpage}{\newpage\thispagestyle{empty}}
\newcommand{\newbookpage}{}

\input select-book

\begin{hyphenrules}{russian}
\lccode`\==`\- % for setting exceptions for compound words with hyphen
\hyphenation{
не-у-же-ли
Шу-шан=-Эдуф
ос-мотр
не-ис-тов-ство
Иу-дей-ские
Аа-ро-на
Ер-аст
Ев-вул
ос-корб-ля-ли
Иу-дею
бе-зум-ным
уб-ранст-ва
оз-ло-бил-ся
не-о-би-та-е-мой
бла-го-де-тельст-во-вал
пра-во-те
Фас-ги
не-прав-ды
Иу-дою
Ио-а-са
Ио-а-фа-ма
Ио-а-ким
оп-рес-но-ков
Иу-дей-ско-го
ус-трой
Ецион=-Га-ве-ре
Ио-си-фа
Не-бес-ное
Не-во
Ус-лышь
Ио-аки-мо-ва
Ио-сии
Ок-руг-ле-ние
ус-во-ить
не-о-би-та-е-мая
не-пра-вед-ные
оск-вер-ни-ли
тетра-драх-мой
}
\end{hyphenrules}

\begin{document}
\pagenumbering{arabic}
\newcommand{\tux}{{\engfont ^^^^e000}}
\newcommand{\wheel}{\raisebox{0.3mm}{\fontsize{11}{12}\armfont ^^^^00a1}}
\newcommand{\titlesep}{\begin{center}\usefont{U}{webo}{xl}{n}\huge ahb\end{center}}
\newcommand{\copyrsepline}{\begin{center}\usefont{U}{webo}{xl}{n}\fontsize{5}{6}\selectfont IJLKIJLKIJLKIJLKIJ\end{center}}
\pagestyle{empty}

\begin{titlepage}
\centering
%\selectlanguage{russian}
\vspace*{\stretch{1}}
{
\addfontfeature{LetterSpace=2.0,WordSpace=2}
{\fontsize{30}{36}\bfseries БИБЛИЯ}\\*[6ex]
\fontsize{14}{20}\selectfont
КНИГИ СВЯЩЕННОГО ПИСАНИЯ\\*[1ex]
ВЕТХОГО И НОВОГО ЗАВЕТА\\*[2ex]
\fontsize{11}{20}\selectfont
СИНОДАЛЬНЫЙ ПЕРЕВОД\\
{\itshape (с книгами апокрифальными)}\\
}
\vspace*{\stretch{2}}
\titlesep
\vspace*{\stretch{1}}

\wheel~{\Large\bfseries\upshape Bibles.org.uk}~\wheel\\
\vspace*{\stretch{0.3}}
\end{titlepage}

\tunemarkup{pgletter}{\onecolumn}
%\selectlanguage{russian}
\vspace*{\stretch{1}}
\begin{center}
% uncomment the next group for the gift edition.
%{
%\newfontfamily\coventry{CoventryCyrillic}
%\coventry\fontsize{20}{26}\selectfont
%\vspace*{\stretch{1}}
%XXXXX\\
%от редактора.\\[1cm]
%Тигран Айвазян,\\
%Лондон, 19 июля 2011 г.\\
%}
\vspace*{\stretch{1}}
{\Large\itshape\bfseries Седьмое издание}
\vspace*{\stretch{4}}
\end{center}

\begin{center}
\fontsize{9}{12}\selectfont
%\selectlanguage{russian}
Все права защищены. \copyright\ 2002--2016 Bibles.org.uk\\
Издание подготовили Тигран Айвазян и Владимир Волович на основе материалов, представленных на сайте Издательства Московской Патриархии.\\
\tux\ Книга набрана в \XeLaTeX\ в системе Linux\\
\vspace*{1mm}

\copyrsepline

\vspace*{1mm}
All rights reserved. Copyright \copyright\ 2002--2016 Bibles.org.uk.\\
This Edition was prepared by Tigran Aivazian and Vladimir Volovich, based on the sources provided by The Publishing House of the Moscow Patriarchate.\\
Text set in \textbf{\itshape Academy} at \bibmainfontsize pt\\
\tux\ PDF typeset with \XeLaTeX\ under Linux (\mytoday)\\
\end{center}
\vspace*{\stretch{0.3}}
\tunemarkup{pgletter}{\twocolumn}
\newpage
%\selectlanguage{russian}
\tunemarkup{pgletter}{\onecolumn\bibtableofcontents{twocol}\newpage\null\twocolumn}
%\tunemarkup{pghanlin}{\bibtableofcontents{twocol}\newpage}
\pagestyle{fancy}
\thispagestyle{empty}
\bibmark{book}{ВВЕДЕНИЕ}
\bibpdfbookmark{Введение}{intro}
\begin{center}
\Large\bfseries ВВЕДЕНИЕ\\
\end{center}
\fontsize{12}{15}\selectfont

%\begin{multicols}{2}
В 1804 году было основано \bibemph{Британское и Иностранное Библейское Общество} (BFBS),
полу-автономным филиалом которого стало \bibemph{Российское Библейское Общество} (РБО),
основанное 6 декабря 1812 года.
В своей работе РБО опиралось на поддержку царя Александра~I, а председателем Общества
был избран князь Александр Голицын (1773--1844),
который тогда был обер-прокурором Святейшего Правительствующего Синода
Русской Православной Церкви, а позже Министром Религии и Народного Образования ---
так называемого <<сугубого министерства>>.
Общество было открыто под именем Санкт-Петербургского, а в сентябре 1814 года
переименовано на Российское.

О русском библейском переводе впервые открыто заговорили в 1816 году.
Князь Голицын, как председатель РБО, получил Высочайшее изустное повеление,
<<дабы предложить Святейшему Синоду искреннее и точное желание Его Величества
доставить и россиянам способ читать Слово Божие на природном своем российском
языке, яко вразумительнейшем для них славянского наречия, на коем книги
Священного Писания у нас издаются>>.
Предполагалось при этом, что новый перевод будет издаваться со славянским
текстом совокупно, как еще раньше уже было выпущено послание к Римлянам,
с дозволения Синода (имелась в виду книга архиепископа Мефодия Смирнова,
перевод и толкование; первое издание в 1794 г., третье в 1815 г.).

Голицын в оправдание предложенного перевода на современный русский язык
ссылался на то, что греческой патриаршей грамотой одобрено народу
чтение священного писания Нового Завета на новейшем греческом наречии
вместо древнего (сама грамота патр. Кирилла была припечатана в отчете
РБО за 1814 год).

Синод не принял на себя руководства библейским переводом и не взял за
него ответственности на себя.
Перевод был отдан в ведение Комиссии духовных училищ, которой надлежало
избрать надежных переводчиков в местной Духовной Академии. 

Перевод был поставлен под охрану Высочайшего имени.
Замысел сей принадлежал самому Государю, или был ему приписан:
<<Не токмо одобряет все споспешествующее сему спасительному делу, но и
одушевляет деятельность Общества внушениями собственного сердца.
Он сам снимает печать невразумительного наречия, заграждавшую доныне от
многих из Россиян евангелие Иисусово, и открывает сию книгу для самых
младенцев народа, от которых не ея назначение, но единственно мрак
времен закрыл оную.>>
Невразумительное наречие закрывало Библию не столько
от народа, сколько именно от высшего круга, от самого императора, прежде
всего, он сам привык читать Новый Завет по-французски (в известном
переводе Де-Саси), и не изменил этой привычке и с изданием российского
перевода.

Ведение перевода от Комиссии духовных училищ было поручено Филарету,
тогда архимандриту и ректору Санкт-Пе\-тер\-бургс\-кой Академии, и он имел
избрать переводчиков по своему усмотрению.
Считалось, что перевод производится при Академии.
Филарет сам взял на себя Евангелие от Иоанна.
От Матфея переводил Павский, от Марка архим. Поликарп (Гайтанников),
тогда ректор Санкт-Петербургской семинарии, а вскоре и Московской
Академии, и от Луки архим. Моисей (Антипов-Платонов), ректор Киевской
семинарии, а потом и Академии, бывший перед тем бакалавром в
Санкт-Петербурге, впоследствии Экзарх Грузии.
Работа отдельных сотрудников пересматривалась и сверялась в особом
комитете из членов Библейского общества.
В нем участвовали: митр. Михаил (Десницкий), впоследствии митрополит
Санкт-Петербургский;
Серафим (Глаголевский), тоже будущий митрополит;
Филарет;
Лабзин;
В.~М.~Попов, директор департамента в двойном министерстве и секретарь
Библейского общества --- человек крайних мистических взглядов, переводчик
Линдля и Госнера, член кружка Татариновой, окончивший жизнь свою в
Зилантовом монастыре в Казани, как заточенный, кроткий изувер, как
его остроумно называет Вигель.

Правила для перевода были составлены Филаретом, это сразу чувствуется
уже в их стиле.
Переводить надлежало с греческого, как первоначального, преимущественно
перед славянским, с тем, чтобы в переводе удерживать или употреблять
слова славянские, если они ближе русских подходят к греческим, не
производя в речи темноты или нестройности, или если соответственные
русские не принадлежат к чистому книжному языку.
В переводе всего важнее точность, затем ясность, наконец, чистота.
Очень характерны некоторые стилистические директивы.
Величие Священного Писания состоит в силе, а не в блеске слов; из сего
следует, что не должно слишком привязываться к славянским словам и выражениям,
ради мнимой их важности.
Еще важнее другое замечание.
Тщательно наблюдать должно дух речи, дабы разговор перелагать слогом
разговорным, повествование повествовательным, и так далее.

Эти положения литературным архаистам показались дурной стилистической
ересью, и это был один из решающих моментов взволнованного восстания
или интриги против русской Библии в 20-х годах.

В этот период были переведены на русский язык и в сотрудничестве с BFBS
опубликованы: Евангелие (1819), Новый Завет (1820) и Псалтирь (1822).
В то же время началась работа над Пятикнижием.
Филарет в своих Записках на книгу Бытия (первое издание уже в 1816 г.)
всюду дает библейский текст в русском переводе, с еврейского.
К переводческим работам были привлечены и вновь открытые Академии:
Московская и Киевская, также и некоторые семинарии.
Сразу же встал трудный и сложный вопрос о соотношении еврейского и
греческого текстов, о достоинстве и достоинствах перевода Семидесяти, о
значении Массоретских чтений, и эти вопросы обострялись тем, что
всякое отступление от Семидесяти означало практически и расхождение
со славянской Библией, остававшейся в богослужебном употреблении,
а потому нуждалось в нарочитых оправданиях и оговорках.
Для начала вопрос был решен просто.
В основу был положен еврейский (Массоретский) текст, как подлинный,
а в объяснение расхождений со славянской Библией было составлено
особое предисловие, убедительное и для незнающих древних языков.
Составил его Филарет, и подписано оно было митр. Михаилом, митр.
Серафимом, тогда еще Московским, и самим Филаретом, тогда
архиепископом Ярославским.

Окончательная корректура Пятикнижия была поручена Герасиму Павскому.
Печатание было закончено в 1825 году, но по изменившимся
обстоятельствам  издание не только не было выпущено в свет, но было
арестовано и вскоре сожжено.
Само библейское дело было остановлено и Библейское общество закрыто
и запрещено 12 апреля 1826 года, в основном благодаря интриграм архимандрита
Фотия, адмирала Шишкова и Аракчеева.

В 1840-х годах профессор Павский впервые перевел на русский язык весь
Ветхий Завет непосредственно с еврейских оригиналов, за что был отдан
под суд и результаты его стараний были уничтожены.

С приходом к власти царя Александра~II работа РБО была возобновлена
под руководством Митрополита Московкого Филарета.
В декабре 1857 года библейское дело получило официальное движение.
Синодальное определение состоялось 20 марта 1858 года, а Высочайшее
повеление о возобновлении русского перевода было опубликовано в мае.

Перевод был возобновлен с Нового Завета, к участию в работах снова были
привлечены все академии, а редактирование поручено петербургскому
профессору Е.~И.~Ловягину.
Высшее наблюдение и последний просмотр были доверены Филарету.
Несмотря на свой преклонный возраст, он очень деятельно участвовал в
работе, со вниманием перечитывая и проверяя весь материал.

В 1860 году было издано русское Четвероевангелие, а в 1862 и полный
Новый Завет.

Перевод Ветхого Завета потребовал больше времени. Уже с самого начала
60-х годов в различных духовных журналах стали появляться частные опыты
перевода отдельных книг.
И, прежде всего, были опубликованы эти так незадолго перед тем запретные
переводы Павского (в журнале Дух Христианина за 1862 и 1863 годы) и
арх. Макария (в Православном Обозрении с 1860-го по 1867-ой, особым
приложением).
Это был очень живой и яркий симптом сдвига и поворота.
Было признано полезным и нужным предать гласности эти опыты, чтобы через
свободное обсуждение в печати подготовить окончательное издание.
С этой целью было предложено и профессорам академии заняться переводами
отдельных книг, с тем чтобы эти новые опыты были в свое время использованы
Синодальной комиссией.
Нечто подобное предлагал в свое время о. Макарий Глухарев,
издавать при Петербургской Академии особый журнал: Опыты в переводе с
еврейского и греческого, и рассылать по академиям и семинариям, с
примечаниями и сносками, потом этот материал пригодится.

В академических изданиях, в Христианском Чтении и в Трудах Киевской
духовной академии в эти годы появляется перевод многих книг.
В Киеве особенно потрудился проф. М.~С.~Гуляев, а в Петербурге проф.
М.~А.~Голубев в сотрудничестве с П.~И.~Савваитовым, Д.~А.~Хвольсоном и др.
Появились и отдельные издания.
Издавал свои библейские переводы с греческого Порфирий Успенский, тогда епископ
Чигиринский.
Это был полный разрыв с режимом предыдущего царствования.

Но встречались и трудности.
Не сразу удалось решить вопрос о принципах перевода.
Было заявлено мнение, что и Ветхий Завет переводить нужно с греческого,
к этому мнению удалось склонить и митр. Григория.
Филарет Московский настоял, чтобы перевод делался по сличению обоих
текстов, и расхождение в важнейших местах было отмечаемо под чертой.
Сперва предложено было начать с Псалмов; над исправлением перевода
Псалмов Филарет работал в свои последние годы.
Но затем он сам предложил издавать в порядке обычного текста,
т.~к. Пятикнижие легче Псалмов по языку.
Синодальный перевод начал выходить с 1868 года отдельными томами, а
всё издание закончилось в 1875 со включением и книг неканонических.

Особенно резким противником еврейского текста был епископ Феофан
Говоров, тогда уже Вышенский затворник.
Новый русский перевод Ветхого Завета он называл Синодальным сочинением,
совсем как Афанасий, и мечтал, что эту Библию новомодную доведет до
сожжения на Исаакиевской площади.
Употребление еврейского текста, никогда не бывшего в церковном
употреблении, означало в его понимании прямое отступничество.
Еврейская библия к нам нейдет, потому что никогда не было ее в Церкви и
в церковном употреблении.
Поэтому принимать ее значит отступать от того, что всегда было в Церкви,
т.~е. сдвигаться с коренного основания православия.
Феофан вполне признавал нужду в русском переводе, он возражал только
против еврейского образца.
И синодальный перевод считал поэтому соблазнительным и вредным.
Церковь Божия не знала другого Слова Божия, кроме 70-ти толковников, и
когда говорила, что Писание богодухновенно, разумела Писание именно в
этом переводе.
Об этом он очень резко писал в Душеполезном Чтении (1875 и 1876),
ему отвечал в Православном Обозрении проф. П. И. Горский-Платонов с
неменьшей резкостью.
Но Феофан не ограничивался критикой.
Он предлагал заняться изданием общедоступных толкований Библии по
славянскому тексту (и особенно книг учительных и пророческих),
чтобы приучить именно к этому тексту, т.~е. к Семидесяти.
Выйдет, что, несмотря на существование Библии в переводе с
еврейского, знать ее и понимать и читать все будут по
Семидесяти, по причине сего толкования.
Проект этот не был осуществлен, сам Феофан издал только
толкование на Псалом Сто Осмьнадцатый (сто восемнадцатый).
Возникла у него и мысль сесть за перевод всей Библии с греческого,
с замечаниями в оправдание греческого текста и в осуждение
еврейского.
Это намерение осталось тоже без исполнения.
Уже только много позже некоторые книги Ветхого Завета были переведены с
греческого казанским профессором П.~А. Юнгеровым (пророки, Псалтирь,
Притчи, Бытие, книги неканонические).

В процессе работы над переводом Ветхого Завета снова и снова
открывалось, что соотношение Массоретской редакции и Семидесяти слишком
сложно, чтобы можно было ставить вопрос о выборе между ними в общем
виде.
Можно спрашивать только о предпочтительном или надежном чтении
отдельных отрывков или стихов, и приходится выбирать иногда еврейскую
истину, иногда же греческое чтение.
Филологически лучшим будет именно сводный текст.
Богословскому заключению о догматическом достоинстве определенного
текста, во всяком случае, должно предшествовать подробное исследование
отдельных книг.
Примером такой работы в те годы была диссертация И.~С.~Якимова о книге
пророка Иеремии (1874).
Следует упомянуть и работы Д.~А.~Хвольсона и И.~А.~Олесницкого.  

Обнаруживалась и другая трудность.
Оказывалось, что и Славянскую Библию не приходится в целом приравнивать
к Семидесяти, что и сам славянский текст есть уже сводный, в известном
смысле и пределах.
В этом и была принципиальная важность описания библейских рукописей Горским
и Невоструевым в Московской Синодальной библиотеке.
Начинается историческое изучение Славянской Библии.
И уже нельзя так упрощенно спрашивать о выборе между славянским и русским.  

Оживает интерес и к вопросам библейской критики.
Большинство русских исследователей придерживались умеренных
взглядов, но и у них влияние западной критической
литературы сказывалось очень заметно.
Достаточно назвать работы архим. Филарета Филаретова (ректора
Киевской академии, впоследствии епископа Рижского, 1824-1882).
В его диссертации о Происхождении книги Иова (1872) он не только
принимал позднюю послепленную датировку книги, но и разбирал ее скорее,
как памятник литературы, нежели как книгу священного канона.
К тому же всё исследование было проведено по еврейскому тексту,
безо всякого внимания к славянским чтениям.
Митр. Арсений Киевский нашел сам тон диссертации
несоответствующим богодухновенному характеру библейской книги, и
публичная защита диссертации была запрещена Святейшим Синодом.
А в следующем (1873) году в Трудах Киевской Академии были
напечатаны устаревшие лекции по введению в священные книги
Ветхого Завета, читанные самим митр. Арсением в Петербургской
академии еще в 1823--1825 годах.
Впрочем, в кратком предисловии от редакции было оговорено, что
читатель сам сможет судить, насколько вперед подвинулись у нас
библиологическая наука с того времени до настоящего.

Переводы, выполненные в 1810--25~гг. и отредактированные
в 1860--70~гг., составили книгу именуемую \bibemph{Русской Синодальной Библией}.
Однако не все отнеслись благосклонно к появлению Библии на русском языке,
предпочитая старо-славянский перевод используемый и по сей день в церковном служении.
Даже Святейший Синод благословил Библию 1876 года \bibemph{исключительно}
для приватного употребления, для чтения дома, но не для церковного служения.

Впоследствии, текст Русской Синодальной Библии был существенно изменен
с целью распространения <<протестантизма>>.
А именно, слова и целые фразы соответствующие текстам греческой
Септуагинты и латинской Вульгаты были удалены, хотя и не полностью и
с многочисленными ошибками, чтобы поддержать \bibemph{миф} о том, что,
якобы, Бог чудодейственным образом <<сохранил>> Свое
Слово в одном единственном варианте и естественно выбор такого
<<идеального>> варианта пал на Массоретский Текст.
В данном издании мы не делаем идола из Слова Божиего и, посему,
приводим текст Синодального издания в его изначальной форме (за исключением
использования архаичного правописания и букв), включая неканонические
книги в том порядке и виде, в каком они были приведены в Библии 1876 года.

Тексты Книг Священного Писания Ветхого и Нового Завета и приложения,
использованные в данном издании, взяты с сайта Издательства Московской
Патриархии,
и соответствуют Си\-но\-даль\-но\-му переводу издания Московской Патриархии
кроме 70-и стихов находящихся между 35 и 36 стихами 7-й главы
3-й Книги Ездры, взятых нами из ``Толковой Библии'' А.~П.~Лопухина (Петербург, 1904)
и имеющихся также в Брюссельской Библии (Брюссель, 1973).
Электронные тексты были переведены в формат типографской системы,
используемой для всех Библий, издаваемых Bibles.org.uk, основанной
на \XeLaTeX\ в системе Linux.
Выражаем благодарность Самуэлю Ким за найденные опечатки.
Мы будем очень признательны, если найденные Вами в этом издании
опечатки, будут отправлены по электронной почте по адресу
{\makeatletter aivazian.tigran@gmail.com\makeatother}.

Оформление текста Библии в данном издании имеет следующие особенности:
\begin{itemize}
\item Для облегчения ссылок и чтения нумерация стихов выведена на поля.
\item Слова, напечатанные \bibemph{курсивом}, приведены для ясности
      и отсутствуют в оригиналах.
\item В тексте Ветхого Завета в квадратные скобки заключены слова,
      заимствованные из греческого перевода 70-ти толковников (III в.~до Р.~Х.)
      --- Септуагинты.
\end{itemize}

Несмотря на <<до-Пушкинский>> язык Русской Синодальной Библии, она
продолжает успешно служить миллионам людей на планете как самый
достоверный и читаемый перевод Священного Писания на русский язык.

Да благословит Господь Бог ваше изучение Его Слова, дабы подчинить
Сыну своему Иисусу Христу Господу нашему всякую мысль вашего сердца
и всякое слово, исходящее из уст ваших. Аминь.
%\end{multicols}

\begingroup
\vfill
\noindent
\itshape
\parbox{4cm}{
Владимир Волович,\\
Воронеж, Россия.
}
\hfill
\parbox{4cm}{
Тигран Айвазян,\\
Лондон, Англия.
}
\vfill
\endgroup
\newpage
%\begin{center}
\Large\bfseries
БИБЛЕЙСКИЙ КАЛЕНДАРЬ
\end{center}

\makeatletter
\def\bibstrut{%
  \vrule
    \@height\dimexpr\f@size pt*13/10\relax
    \@depth.6\baselineskip
    \@width\z@
}
\makeatother

\newsavebox{\nazvanija}
\newsavebox{\mesjatsi}
\sbox{\nazvanija}{\parbox{9cm}{\centering Названия месяцев {\bfseries ДРЕВНИЕ} и \bibemph{АССИРО-ВАВИЛОНСКИЕ},\\ число дней в месяце и особые дни}}
\sbox{\mesjatsi}{\parbox{2cm}{\centering Соответствует месяцам современного календаря}}

\hspace*{-5mm}
\begin{tabular}{|>{\raggedleft}p{1cm}|>{\raggedleft}p{1cm}|p{9cm}|c|}
\hline
\multicolumn{2}{|c|}{\bibstrut Счет месяцев} & & \\
\cline{1-2}
\multicolumn{1}{|p{1cm}|}{\footnotesize в священном году} & 
\multicolumn{1}{p{1cm}|}{\footnotesize в гражданском году} &
\usebox{\nazvanija} & \usebox{\mesjatsi} \tabularnewline
\hline
\bibstrut I & \bibstrut 7 &
\bibstrut {\bfseries АВИВ}, \bibemph{НИСАН}. 30 дней.
14. Пасха (\bibref[Исх. гл. 12]{Exo 12:1}; \bibref{Lev 23:5}; \bibref{Num 28:16}).
16. Принесение первого снопа жатвы ячменя (\bibref[Лев 23:10--14]{Lev 23:10}).&
\bibstrut март --- апрель\tabularnewline
\hline
\bibstrut II & \bibstrut 8 &
\bibstrut {\bfseries ЗИФ}, \bibemph{ИЯР}. 29 дней.
14. Вторая пасха --- для тех, кто не мог совершить первую (\bibref[Чис 9:10--12]{Num 9:10}).&
\bibstrut апрель --- май\tabularnewline
\hline
\bibstrut III & \bibstrut 9 &
\bibstrut \bibemph{СИВАН}. 30 дней.
6. Пятидесятница (\bibref{Lev 23:16}) или праздник седмиц (\bibref[Втор 16:9--10]{Deu 16:9}).
Принесение начатков жатвы пшеницы (\bibref[Лев 23:15--21]{Lev 23:15}) и начатков всех плодов земли
(\bibref{Num 28:26}; \bibref[Втор 26:2,10]{Deu 26:2}).&
\bibstrut май --- июнь\tabularnewline
\hline
\bibstrut IV & \bibstrut 10 &
\bibstrut \bibemph{ФАММУЗ}. 29 дней.
17. Пост. Взятие Иерусалима (\bibref{Zec 8:19}).&
\bibstrut июнь --- июль\tabularnewline
\hline
\bibstrut V & \bibstrut 11 &
\bibstrut \bibemph{АВ}. 30 дней.
9. Пост. Разрушение храма иерусалимского (\bibref{Zec 8:19}).&
\bibstrut июль --- август\tabularnewline
\hline
\bibstrut VI & \bibstrut 12 &
\bibstrut \bibemph{ЭЛУЛ}. 29 дней.&
\bibstrut август --- сентябрь\tabularnewline
\hline
\bibstrut VII & \bibstrut 1 &
\bibstrut {\bfseries АФАНИМ}, \bibemph{ТИШРИ}. 30 дней.
1. Праздник труб (\bibref{Num 29:1}). Новый год.
10. День очищения (\bibref{Lev 16:29}; \bibref[25:9]{Lev 25:9}).
15-22. Праздник кущей (\bibref[Лев 23:34--36]{Lev 23:34}; \bibref[Чис 29:12--35]{Num 29:12}).&
\bibstrut сентябрь --- октябрь\tabularnewline
\hline
\bibstrut VIII & \bibstrut 2 &
\bibstrut {\bfseries БУЛ}, \bibemph{МАРХЕШВАН}. 29 дней.&
\bibstrut октябрь --- ноябрь\tabularnewline
\hline
\bibstrut IX & \bibstrut 3 &
\bibstrut \bibemph{КИСЛЕВ}. 30 дней.
25. Праздник обновления (\bibref[1~Мак 4:52--59]{1Ma 4:52}; \bibref[Ин 10:22]{Joh 10:22}).&
\bibstrut ноябрь --- декабрь\tabularnewline
\hline
\bibstrut X & \bibstrut 4 &
\bibstrut \bibemph{ТЕБЕФ}. 29 дней.&
\bibstrut декабрь --- январь\tabularnewline
\hline
\bibstrut XI & \bibstrut 5 &
\bibstrut \bibemph{ШЕВАТ}. 30 дней.&
\bibstrut январь --- февраль\tabularnewline
\hline
\bibstrut XII & \bibstrut 6 &
\bibstrut \bibemph{АДАР}. 29 дней.
11. Пост Есфири (\bibref{Est 4:16}).
14--15. Праздник Пурим (\bibref[Есф 9:17--32]{Est 9:17}).&
\bibstrut февраль --- март\tabularnewline
\hline
\end{tabular}

\vspace*{6mm}

%\begin{multicols}{2}
В Библии священный год со времени исхода из Египта начинается с весны, с
месяца авив, что значит месяц зрелого колоса
(\bibref{Exo 13:4}; \bibref[12:2]{Exo 12:2}).
Это был месяц весеннего равноденствия и время созревания ячменя
(\bibref[Лев 23:10--14]{Lev 23:10}).
Позже он стал называться нисаном.
В 14-й день этого месяца, который приходится в полнолуние, праздновали
Пасху (\bibref[Исх. гл. 12]{Exo 12:1}).
Другие месяцы названий не имели, о них говорили: второй месяц, десятый месяц
и т.~д.
Лишь в рассказе о постройке храма Соломона при участии финикийцев три месяца
названы особо:
зиф (месяц цветения) --- \bibref{1Ki 6:1},
афаним (месяц бурных ветров) --- \bibref{1Ki 8:2},
и бул (месяц произрастания) --- \bibref{1Ki 6:38}; это финикийские названия.
После плена вавилонского появились ассиро-вавилонские названия месяцев:
нисан (\bibref{Neh 2:1}),
ияр, сиван (\bibref{Est 8:9}), фаммуз или таммуз, ав, элул (\bibref{Neh 6:15}),
тишри, мархешван, кислев или хаслев (\bibref{Neh 1:1}; \bibref{Zec 7:1}),
тебеф (\bibref{Est 2:16}), шеват (\bibref{Zec 1:7})
и адар (\bibref{Est 3:7}).
Те из них, которые не встречаются в Библии, известны по сочинениям Иосифа Флавия
(I век по Р.~Х.) и другим древним источникам.

Гражданский год начинался и кончался осенью
(ср. \bibref{Exo 23:16}; \bibref[34:22]{Exo 34:22}), после уборки урожая
(в месяце тишри).
В Библии встречается счет месяцев и по священному и по гражданскому году.

Начало месяца определялось по появлению видимого серпа новой луны; этот
день, новомесячие, был праздничным (\bibref{Num 10:10}; \bibref[28:11]{Num 28:11}).
От одного новолуния до другого проходит 29 1/2 суток, поэтому месяцы имели
продолжительность в 29 и 30 дней попеременно.
12 лунных месяцев составляют год в 354 дня, что на 11 дней меньше солнечного
года.
За три года разница между луннным и солнечным годом составит целый месяц,
поэтому примерно раз в три года добавлялся 13-й месяц и получался год
продолжительностью в 384 дня.
Это делалось для того, чтобы авив оставался весенним месяцем.

В прилагаемой таблице указаны названия месяцев в гражданском и священном 
году (т.~е. первый месяц, второй и т.~д.), а также древние (ханаанские и
финикийские) и послепленные (ассиро-вавилонские) названия в том виде, в
каком они приведены в русской Библии. Обозначено количество дней в
месяце, перечислены библейские праздники и посты и показано
приблизительное соответствие библейских месяцев современным.

Дни недели, кроме субботы (шаб\'ат), особых наименований не имели, если не
считать существовавшего в эллинистическую эпоху греческого названия дня
перед субботой --- параскев\'и, что значит <<приготовление>>
(к дню покоя --- субботе).
Неделя завершалась субботой, поэтому <<день первый>> (после субботы, см.
\bibref[Мф 28:1]{Mat 28:1}) соответствует нашему воскресенью,
<<день второй>> --- понедельнику и т.~д.

День (в смысле суток) начинался с захода солнца, т.~е. с позднего
вечера. В древности как ночь, так и день делились на три части: ночь на
первую, вторую и третью стражи (\bibref{Jdg 7:19}), а
день --- на утро, полдень и вечер (см. \bibref{Psa 54:18}).
Позднее, со времен римского владычества, ночь делилась на четыре стражи
(\bibref[Лк 12:38]{Luk 12:38}; \bibref[Мф 14:25]{Mat 14:25}) и вошло в употребление понятие <<час>> ---
двенадцатая часть дня или ночи (\bibref[Мф 20:1--8]{Mat 20:1}; \bibref[Деян 23:23]{Act 23:23}).

%\end{multicols}

\thispagestyle{empty}
\begin{center}
\normalsize\bfseries
О КНИГАХ КАНОНИЧЕСКИХ И НЕКАНОНИЧЕСКИХ
\end{center}

%\begin{multicols}{2}
Христианская Библия состоит из двух частей: Ветхого Завета и Нового Завета,
Книги Ветхого Завета писались на протяжении более тысячи лет до Рождества
Христова на древнееврейском языке, книги Нового Завета написаны на греческом
языке в I в. по Р.~Х.

В Ветхом Завете есть книги канонические и неканонические.
Основное различие между ними в том, что книги канонические более древние,
написаны в XV--V вв.~до Р.~Х., а книги неканонические, т.~е. не вошедшие в
канон, в собрание священных книг, написаны позже, в IV--I вв.~до Р.~Х.
Ветхозаветный канон создавался постепенно.
Первым собирателем священных книг воедино считают Ездру (V в.~до Р.~Х.).
В III в.~до Р.~Х. -- I в. по Р.~Х. ветхозаветный канон приобрел тот вид,
который существует в современной еврейской, так называемой массоретской,
Библии (она содержит лишь Ветхий Завет; массореты, хранители предания,
закончили работу над ней в VIII в. по Р.~Х.).
В ней 39 книг, которые разделены на три отдела: \bibemph{закон},
\bibemph{пророки} и \bibemph{писания} (этими словами в древности называли
Ветхий Завет,-- см. \bibref{Mat 7:12}; \bibref{Luk 24:44}).
\bibemph{Закон} (по-еврейски тор\'а) содержит Пятикнижие Моисея: Бытие, Исход, Левит,
Числа и Второзаконие.
\bibemph{Пророки} (неби\'им) делятся на первых или старших, которым принадлежат книги
Иисуса Навина, Судей, две книги Самуила (в нашей Библии это 1 и 2 Царств) и
две книги Царей (наши 3 и 4 Царств; в христианской Церкви книги старших
пророков, а также Руфь, Есфирь, Ездры, Неемии и Паралипоменон принято
считать историческими книгами), и на последних или младших, которые в
свою очередь подразделяются на великих пророков и малых.
Книги трёх великих пророков: Исайя, Иеремия, Иезекииль;
двенадцати малых: Осия.  Иоиль, Амос, Авдий, Иона, Михей, Наум, Аввакум,
Софония, Аггей, Захария и Малахия.
\bibemph{Писания} (кетуб\'им) составляют: Псалмы, Притчи, Иов, Песнь Песней, Руфь,
Плач Иеремии, Екклезиаст, Есфирь, Даниил. Ездра, Неемия и Летописи
($=$ 1 и 2 Паралипоменон).

После возвращения евреев из плена вавилонского, т.~е. после V в.~до Р.~Х.,
было составлено и написано еще несколько книг на еврейском и греческом языках.
В канон еврейских священных книг их уже не включили, но они вошли, как
полезные и назидательные, в Септуагинту, т.~е. греческий перевод Библии.  

Этот перевод был сделан в III--II вв.~до Р.~Х. для александрийских
евреев-эллинистов и иудеев рассеяния, т.~е. живущих вне Палестины, которые
уже забывали родной язык и говорили по-гречески
(см. предисловие к Книге Иисуса сына Сирахова).
Древнее предание говорит о 70-ти (или 72-х) толковниках, т.~е. переводчиках,
которые перевели священные книги с еврейского языка на греческий, поэтому и
перевод этот называется <<переводом семидесяти>> или по-гречески <<Септуагинта>>.
Он в некоторых деталях отличается от масоретского текста, так как массореты и
переводчики на греческий пользовались разными списками (рукописями) древнего текста.

Библейские книги на латинском языке были известны уже в конце II века по Р.~Х.
Блаж. Иероним перевел их заново в конце IV -- начале V в., и этот перевод, известный
под названием <<Вульгата>>, получил широкое распространение в Католической Церкви.

Перевод книг Священного Писания на славянский язык начат был святыми равноапостольными Кириллом
и Мефодием в IX в.
Современная славянская Библия используемая в церковном служении представляет собой
перепечатку Елизаветинского издания 1751--1756~гг., в котором текст Ветхого Завета
был выверен по греческой Библии.

На русский язык Библия переведена в середине XIX века.
Канонические книги переводились с еврейского массоретского текста с
дополнениями и вариантами из Септуагинты, а неканонические --- с греческого,
за исключением Третьей книги Ездры, переведенной с латинского, так как этой
книги нет ни в еврейской, ни в греческой Библии.
Русская православная Библия, как и славянская, содержит все 39 канонических и
11 неканонических книг Ветхого Завета.

Что касается деления книг на главы, то оно было введено уже в XIII веке
двумя западными исследователями Библии --- кардиналом Стефаном Лангтоном и
доминиканцем Гуго де Сен-Шира.

Деление глав на стихи ввел в середине XVI века парижский типограф
Робертус Стефанус. \bibemph{(Новая толковая Библия, 1990, Ленинград)}
%\end{multicols}
\newpage
%\makeatletter

\define@key{bibevnt}{n}{\gdef\bibevnt@n{#1}}
\define@key{bibevnt}{event}{\gdef\bibevnt@event{#1}}
\define@key{bibevnt}{Mat}{\gdef\bibevnt@Mat{#1}}
\define@key{bibevnt}{Mar}{\gdef\bibevnt@Mar{#1}}
\define@key{bibevnt}{Luk}{\gdef\bibevnt@Luk{#1}}
\define@key{bibevnt}{Joh}{\gdef\bibevnt@Joh{#1}}

\def\DescribeEvent#1{%
  \global\let\bibevnt@n\@empty
  \global\let\bibevnt@event\@empty
  \global\let\bibevnt@Mat\@empty
  \global\let\bibevnt@Mar\@empty
  \global\let\bibevnt@Luk\@empty
  \global\let\bibevnt@Joh\@empty
  \setkeys{bibevnt}{#1}%
  \noindent\hangindent=\evntw\hangafter=1
  \hb@xt@\evntw{\hfill\footnotesize\bibevnt@n\kern3pt}%
  \footnotesize\bibevnt@event
  &\footnotesize\bibevnt@Mat
  &\footnotesize\bibevnt@Mar
  &\footnotesize\bibevnt@Luk
  &\footnotesize\bibevnt@Joh
  \tabularnewline
}

\def\EventHeaderCap#1{\multicolumn{5}{c}{%
  \fontsize{10}{12}\bfseries\scshape\bibstrut#1}\tabularnewline}
\def\EventHeader#1{\multicolumn{5}{c}{%
  \footnotesize\bfseries\bibstrut#1}\tabularnewline}

\newlength\maxrangew
\setbox0=\hbox{\footnotesize 88:88-88:88,}
\maxrangew=\wd0
\advance\maxrangew by 2pt

\newlength\evntw
\setbox0=\hbox{\footnotesize 888}
\evntw=\wd0
\advance\evntw by 3pt

\def\bibstrut{%
  \vrule
    \@height\dimexpr\f@size pt*11/10\relax
    \@depth.3\baselineskip
    \@width\z@
}

\makeatother

\bibmark{book}{последовательность евангельских событий}
\bibpdfbookmark{Последовательность Евангельских событий}{ntevents}

\begin{landscape}

\begin{center}
\Large\bfseries
ПОСЛЕДОВАТЕЛЬНОСТЬ ЕВАНГЕЛЬСКИХ СОБЫТИЙ ПО ЧЕТЫРЕМ ЕВАНГЕЛИСТАМ
\end{center}

\begin{longtable}{|>{\raggedright}p{0.5\linewidth}|p{\maxrangew}|p{\maxrangew}|p{\maxrangew}|p{\maxrangew}|}
\hline
\multicolumn{1}{|c|}{\multirow{2}*{Евангельское повествование}}&\multicolumn{4}{c|}{\bibstrut Главы и стихи евангелистов}\tabularnewline
\cline{2-5}
&\multicolumn{1}{c|}{\bibstrut Матфея}&\multicolumn{1}{c|}{Марка}&\multicolumn{1}{c|}{Луки}&\multicolumn{1}{c|}{Иоанна}\tabularnewline
\hline
\endhead
\hline
\endfoot
\DescribeEvent{
  n={1},
  event={Пролог},
  Mat={\bibref[1:1]{Mat 1:1}},
  Mar={\bibref[1:1--3]{Mar 1:1}},
  Luk={\bibref[1:1--4]{Luk 1:1}},
  Joh={\bibref[1:1--18]{Joh 1:1}}
}
\DescribeEvent{
  n={2},
  event={Родословие Иисуса Христа},
  Mat={\bibref[1:1--17]{Mat 1:1}},
  Mar={},
  Luk={\bibref[3:23--38]{Luk 3:23}},
  Joh={}
}
\DescribeEvent{
  n={3},
  event={Благовестие Захарии о рождении Иоанна Предтечи},
  Mat={},
  Mar={},
  Luk={\bibref[1:5--25]{Luk 1:5}},
  Joh={}
}
\DescribeEvent{
  n={4},
  event={Благовещение Марии},
  Mat={},
  Mar={},
  Luk={\bibref[1:26--38]{Luk 1:26}},
  Joh={}
}
\DescribeEvent{
  n={},
  event={Мария в доме Елисаветы},
  Mat={},
  Mar={},
  Luk={\bibref[1:39--56]{Luk 1:39}},
  Joh={}
}
\DescribeEvent{
  n={5},
  event={Рождение Иоанна Предтечи},
  Mat={},
  Mar={},
  Luk={\bibref[1:57--80]{Luk 1:57}},
  Joh={}
}
\DescribeEvent{
  n={6},
  event={Откровение Иосифу Праведному о Боговоплощении},
  Mat={\bibref[1:18--25]{Mat 1:18}},
  Mar={},
  Luk={},
  Joh={}
}
\DescribeEvent{
  n={7},
  event={Рождество Иисуса Христа},
  Mat={\bibref[2:1]{Mat 2:1}},
  Mar={},
  Luk={\bibref[2:1--7]{Luk 2:1}},
  Joh={}
}
\DescribeEvent{
  n={8},
  event={Поклонение пастырей},
  Mat={},
  Mar={},
  Luk={\bibref[2:8--20]{Luk 2:8}},
  Joh={}
}
\DescribeEvent{
  n={9},
  event={Обрезание и наречение имени Иисус},
  Mat={\bibref[1:25]{Mat 1:25}},
  Mar={},
  Luk={\bibref[2:21]{Luk 2:21}},
  Joh={}
}
\DescribeEvent{
  n={10},
  event={Сретение Господа в храме},
  Mat={},
  Mar={},
  Luk={\bibref[2:22--33]{Luk 2:22}},
  Joh={}
}
\DescribeEvent{
  n={11},
  event={Поклонение волхвов},
  Mat={\bibref[2:1--12]{Mat 2:1}},
  Mar={},
  Luk={},
  Joh={}
}
\DescribeEvent{
  n={12},
  event={Бегство в Египет},
  Mat={\bibref[2:13--15]{Mat 2:13}},
  Mar={},
  Luk={},
  Joh={}
}
\DescribeEvent{
  n={13},
  event={Избиение младенцев в Вифлееме},
  Mat={\bibref[2:16--18]{Mat 2:16}},
  Mar={},
  Luk={},
  Joh={}
}
\DescribeEvent{
  n={14},
  event={Возвращение из Египта, поселение в Назарете; отрок Иисус в храме},
  Mat={\bibref[2:19--23]{Mat 2:19}},
  Mar={},
  Luk={\bibref[2:39--52]{Luk 2:39}},
  Joh={}
}
\DescribeEvent{
  n={15},
  event={Проповедь Иоанна Предтечи в пустыне; свидетельство об Иисусе Христе},
  Mat={\bibref[3:1--12]{Mat 3:1}},
  Mar={\bibref[1:1--8]{Mar 1:1}},
  Luk={\bibref[3:1--18]{Luk 3:1}},
  Joh={\bibref[1:19--28]{Joh 1:19}}
}
\DescribeEvent{
  n={16},
  event={Крещение Господне},
  Mat={\bibref[13:13--17]{Mat 13:13}},
  Mar={\bibref[1:9--11]{Mar 1:9}},
  Luk={\bibref[3:21--22]{Luk 3:21}},
  Joh={}
}
\DescribeEvent{
  n={17},
  event={Искушение Иисуса Христа в пустыне},
  Mat={\bibref[4:1--11]{Mat 4:1}},
  Mar={\bibref[1:12--13]{Mar 1:12}},
  Luk={\bibref[4:1--13]{Luk 4:1}},
  Joh={}
}
\hline
\EventHeader{Дела Господа Иисуса Христа в Иудее после искушения Его в пустыне до первой Пасхи}
\hline
\DescribeEvent{
  n={18},
  event={Последующие свидетельства Иоанна Предтечи об Иисусе Христе},
  Mat={},
  Mar={},
  Luk={},
  Joh={\bibref[1:19--36]{Joh 1:19}}
}
\DescribeEvent{
  n={},
  event={Первые ученики Господа},
  Mat={},
  Mar={},
  Luk={},
  Joh={\bibref[1:37--51]{Joh 1:37}}
}
\hline
\EventHeader{Дела Господа Иисуса Христа в Галилее по возвращении из Иудеи}
\hline
\DescribeEvent{
  n={19},
  event={Чудо в Кане галилейской},
  Mat={},
  Mar={},
  Luk={},
  Joh={\bibref[2:1--11]{Joh 2:1}}
}
\DescribeEvent{
  n={20},
  event={Посещение Капернаума},
  Mat={},
  Mar={},
  Luk={},
  Joh={\bibref[2:12]{Joh 2:12}}
}
\hline
\EventHeaderCap{Служение Господа Иисуса Христа от первой пасхи до второй}
\EventHeader{События в Иудее}
\hline
\DescribeEvent{
  n={21},
  event={Изгнание торгующих из храма. Свидетельство Иисуса Христа о Своем Богосыновстве},
  Mat={},
  Mar={},
  Luk={},
  Joh={\bibref[2:13--25]{Joh 2:13}}
}
\DescribeEvent{
  n={22},
  event={Беседа Иисуса Христа с Никодимом},
  Mat={},
  Mar={},
  Luk={},
  Joh={\bibref[3:1--21]{Joh 3:1}}
}
\DescribeEvent{
  n={23},
  event={Иисус с учениками в Иудее},
  Mat={},
  Mar={},
  Luk={},
  Joh={\bibref[3:22]{Joh 3:22}}
}
\DescribeEvent{
  n={},
  event={Последнее свидетельство Иоанна Крестителя об Иисусе Христе пред учениками},
  Mat={},
  Mar={},
  Luk={},
  Joh={\bibref[3:23--36]{Joh 3:23}}
}
\DescribeEvent{
  n={24},
  event={Заключение Иоанна Предтечи в темницу},
  Mat={\bibref[14:3--5]{Mat 14:3}},
  Mar={\bibref[6:17--20]{Mar 6:17}},
  Luk={\bibref[3:19--20]{Luk 3:19}},
  Joh={}
}
\hline
\EventHeader{События по пути из Иудеи в Галилею}
\hline
\DescribeEvent{
  n={25},
  event={Возвращение в Галилею},
  Mat={\bibref[4:12--17]{Mat 4:12}},
  Mar={},
  Luk={},
  Joh={\bibref[4:1--3]{Joh 4:1}}
}
\DescribeEvent{
  n={},
  event={Беседа с самарянкой},
  Mat={},
  Mar={},
  Luk={},
  Joh={\bibref[4:4--42]{Joh 4:4}}
}
\hline
\EventHeader{Служение Иисуса Христа в Галилее}
\hline
\DescribeEvent{
  n={26},
  event={Приход Иисуса Христа в Кану галилейскую; исцеление сына капернаумского царедворца},
  Mat={},
  Mar={},
  Luk={},
  Joh={\bibref[4:43--54]{Joh 4:43}}
}
\DescribeEvent{
  n={27},
  event={Начало Евангельской проповеди},
  Mat={},
  Mar={\bibref[1:14--15]{Mar 1:14}},
  Luk={\bibref[4:14--15]{Luk 4:14}},
  Joh={}
}
\DescribeEvent{
  n={28},
  event={Проповедь Иисуса Христа в назаретской синагоге},
  Mat={},
  Mar={},
  Luk={\bibref[4:16--30]{Luk 4:16}},
  Joh={}
}
\DescribeEvent{
  n={29},
  event={Поселение и проповедь в Капернауме},
  Mat={\bibref[4:13--16]{Mat 4:13}},
  Mar={\bibref[1:21]{Mar 1:21}},
  Luk={\bibref[4:31--32]{Luk 4:31}},
  Joh={}
}
\DescribeEvent{
  n={30},
  event={Призвание к апостольству Петра, Андрея, Иакова и Иоанна},
  Mat={\bibref[4:18--22]{Mat 4:18}},
  Mar={\bibref[1:16--20]{Mar 1:16}},
  Luk={\bibref[5:1--11]{Luk 5:1}},
  Joh={}
}
\DescribeEvent{
  n={31},
  event={Исцеление бесноватого в капернаумской синагоге},
  Mat={},
  Mar={\bibref[1:21--28]{Mar 1:21}},
  Luk={\bibref[4:31--37]{Luk 4:31}},
  Joh={}
}
\DescribeEvent{
  n={32},
  event={Исцеление тещи Симона},
  Mat={},
  Mar={\bibref[1:29--31]{Mar 1:29}},
  Luk={\bibref[4:38--39]{Luk 4:38}},
  Joh={}
}
\DescribeEvent{
  n={33},
  event={Исцеление многих больных},
  Mat={\bibref[8:16--17]{Mat 8:16}},
  Mar={\bibref[1:32--34]{Mar 1:32}},
  Luk={\bibref[4:40--41]{Luk 4:40}},
  Joh={}
}
\DescribeEvent{
  n={34},
  event={Благовестие в Галилее},
  Mat={\bibref[4:23--25]{Mat 4:23}},
  Mar={\bibref[1:35--39]{Mar 1:35}},
  Luk={\bibref[4:42--44]{Luk 4:42}},
  Joh={}
}
\DescribeEvent{
  n={35},
  event={Исцеление прокаженного},
  Mat={\bibref[8:2--4]{Mat 8:2}},
  Mar={\bibref[1:40--45]{Mar 1:40}},
  Luk={\bibref[5:12--16]{Luk 5:12}},
  Joh={}
}
\DescribeEvent{
  n={36},
  event={Исцеление расслабленного в Капернауме},
  Mat={\bibref[9:1--8]{Mat 9:1}},
  Mar={\bibref[2:1--12]{Mar 2:1}},
  Luk={\bibref[5:17--26]{Luk 5:17}},
  Joh={}
}
\DescribeEvent{
  n={37},
  event={Призвание Левия-Матфея к апостольству},
  Mat={\bibref[9:9--13]{Mat 9:9}},
  Mar={\bibref[2:13--17]{Mar 2:13}},
  Luk={\bibref[5:27--32]{Luk 5:27}},
  Joh={}
}
\DescribeEvent{
  n={38},
  event={Ответ ученикам Иоанна о посте},
  Mat={\bibref[9:14--17]{Mat 9:14}},
  Mar={\bibref[2:18--22]{Mar 2:18}},
  Luk={\bibref[5:33--39]{Luk 5:33}},
  Joh={}
}
\hline
\EventHeaderCap{Служение Господа Иисуса Христа от второй пасхи до третьей}
\EventHeader{События в Иудее}
\hline
\DescribeEvent{
  n={39},
  event={Иисус Христос в Иерусалиме на второй Пасхе. Исцеление расслабленного при Овчей купели},
  Mat={},
  Mar={},
  Luk={},
  Joh={\bibref[5:1--17]{Joh 5:1}}
}
\DescribeEvent{
  n={40},
  event={Откровение Иисуса Христа о Своем Богосыновстве},
  Mat={},
  Mar={},
  Luk={},
  Joh={\bibref[5:17--47]{Joh 5:17}}
}
\hline
\EventHeader{Проповедь и чудеса Иисуса Христа в Галилее}
\hline
\DescribeEvent{
  n={41},
  event={О значении субботы; срывание колосьев},
  Mat={\bibref[12:1--8]{Mat 12:1}},
  Mar={\bibref[2:23--28]{Mar 2:23}},
  Luk={\bibref[6:1--5]{Luk 6:1}},
  Joh={}
}
\DescribeEvent{
  n={42},
  event={Исцеление сухорукого},
  Mat={\bibref[12:9--13]{Mat 12:9}},
  Mar={\bibref[3:1--5]{Mar 3:1}},
  Luk={\bibref[6:6--11]{Luk 6:6}},
  Joh={}
}
\DescribeEvent{
  n={43},
  event={Злоба фарисеев и стремление народа к Иисусу},
  Mat={\bibref[12:14--21]{Mat 12:14}},
  Mar={\bibref[3:6--12]{Mar 3:6}},
  Luk={\bibref[6:11]{Luk 6:11}, \bibref[17--19]{Luk 6:17}},
  Joh={}
}
\DescribeEvent{
  n={44},
  event={Избрание 12 апостолов},
  Mat={\bibref[10:1--4]{Mat 10:1}},
  Mar={\bibref[3:13--19]{Mar 3:13}},
  Luk={\bibref[6:12--16]{Luk 6:12}},
  Joh={}
}
\DescribeEvent{
  n={45},
  event={Нагорная проповедь},
  Mat={\bibref[5:1--7:29]{Mat 5:1}},
  Mar={},
  Luk={\bibref[6:17--49]{Luk 6:17}},
  Joh={}
}
\DescribeEvent{
  n={46},
  event={Исцеление слуги капернаумского сотника},
  Mat={\bibref[8:5--13]{Mat 8:5}},
  Mar={},
  Luk={\bibref[7:1--10]{Luk 7:1}},
  Joh={}
}
\DescribeEvent{
  n={47},
  event={Воскрешение сына вдовы в Наине},
  Mat={},
  Mar={},
  Luk={\bibref[7:11--17]{Luk 7:11}},
  Joh={}
}
\DescribeEvent{
  n={48},
  event={Свидетельство Иисуса о Себе и об Иоанне Крестителе пред Иоанновыми учениками},
  Mat={\bibref[11:1--19]{Mat 11:1}},
  Mar={},
  Luk={\bibref[7:18--35]{Luk 7:18}},
  Joh={}
}
\DescribeEvent{
  n={49},
  event={Призыв труждающихся и обремененных},
  Mat={\bibref[11:27--30]{Mat 11:27}},
  Mar={},
  Luk={},
  Joh={}
}
\DescribeEvent{
  n={50},
  event={Прощение грешницы в доме фарисея Симона},
  Mat={},
  Mar={},
  Luk={\bibref[7:36--50]{Luk 7:36}},
  Joh={}
}
\DescribeEvent{
  n={51},
  event={Исцеление бесноватого глухонемого слепца},
  Mat={\bibref[12:22--23]{Mat 12:22}},
  Mar={},
  Luk={\bibref[11:14]{Luk 11:14}},
  Joh={}
}
\DescribeEvent{
  n={52},
  event={Изобличение хулы на Духа Святого},
  Mat={\bibref[12:24--37]{Mat 12:24}},
  Mar={\bibref[3:20--30]{Mar 3:20}},
  Luk={\bibref[11:15--26]{Luk 11:15}},
  Joh={}
}
\DescribeEvent{
  n={53},
  event={О требовании знамения},
  Mat={\bibref[12:38--45]{Mat 12:38}},
  Mar={},
  Luk={\bibref[11:16]{Luk 11:16}, \bibref[29--32]{Luk 11:29}},
  Joh={}
}
\DescribeEvent{
  n={54},
  event={О внутреннем свете},
  Mat={},
  Mar={},
  Luk={\bibref[11:33--36]{Luk 11:33}},
  Joh={}
}
\DescribeEvent{
  n={55},
  event={Похвала слушающим слово Божие},
  Mat={\bibref[12:46--50]{Mat 12:46}},
  Mar={\bibref[3:31--35]{Mar 3:31}},
  Luk={\bibref[8:19--21]{Luk 8:19}, \bibref[11:27--28]{Luk 11:27}},
  Joh={}
}
\DescribeEvent{
  n={56},
  event={Изобличение внешней праведности},
  Mat={},
  Mar={},
  Luk={\bibref[11:37--54]{Luk 11:37}},
  Joh={}
}
\DescribeEvent{
  n={57},
  event={Учение при море притчами о Царствии Божием},
  Mat={\bibref[13:1--58]{Mat 13:1}},
  Mar={\bibref[4:1--34]{Mar 4:1}},
  Luk={\bibref[12:1--59]{Luk 12:1}, \bibref[8:4--18]{Luk 8:4}, \bibref[13:18--21]{Luk 13:18}},
  Joh={}
}
\DescribeEvent{
  n={58},
  event={О последовании Иисусу},
  Mat={\bibref[8:18--22]{Mat 8:18}},
  Mar={},
  Luk={},
  Joh={}
}
\DescribeEvent{
  n={59},
  event={Укрощение бури на пути через Геннисаретское озеро в Гадаринскую страну},
  Mat={\bibref[8:23--27]{Mat 8:23}},
  Mar={\bibref[4:35--41]{Mar 4:35}},
  Luk={\bibref[8:22--25]{Luk 8:22}},
  Joh={}
}
\DescribeEvent{
  n={60},
  event={Исцеление бесноватых в Гадаринской стране},
  Mat={\bibref[8:28--31]{Mat 8:28}},
  Mar={\bibref[5:1--20]{Mar 5:1}},
  Luk={\bibref[8:26--39]{Luk 8:26}},
  Joh={}
}
\DescribeEvent{
  n={61},
  event={Воскрешение дочери Иаира, исцеление кровоточивой больной},
  Mat={},
  Mar={\bibref[5:22--43]{Mar 5:22}},
  Luk={\bibref[8:40--56]{Luk 8:40}},
  Joh={}
}
\DescribeEvent{
  n={62},
  event={Исцеление двух слепцов},
  Mat={\bibref[9:27--31]{Mat 9:27}},
  Mar={},
  Luk={},
  Joh={}
}
\DescribeEvent{
  n={63},
  event={Исцеление бесноватого немого},
  Mat={\bibref[9:32--34]{Mat 9:32}},
  Mar={},
  Luk={},
  Joh={}
}
\DescribeEvent{
  n={64},
  event={Проповедь в назаретской синагоге},
  Mat={\bibref[13:54--58]{Mat 13:54}},
  Mar={\bibref[6:1--6]{Mar 6:1}},
  Luk={},
  Joh={}
}
\DescribeEvent{
  n={65},
  event={Проповедь в окрестных городах и селениях},
  Mat={\bibref[9:35--38]{Mat 9:35}},
  Mar={\bibref[6:6]{Mar 6:6}},
  Luk={},
  Joh={}
}
\DescribeEvent{
  n={66},
  event={Наставления 12 при послании их на проповедь},
  Mat={\bibref[10:1--42]{Mat 10:1}},
  Mar={\bibref[6:7--13]{Mar 6:7}},
  Luk={\bibref[9:1--6]{Luk 9:1}},
  Joh={}
}
\DescribeEvent{
  n={67},
  event={Смерть Иоанна Крестителя},
  Mat={\bibref[14:6--12]{Mat 14:6}},
  Mar={\bibref[6:17--29]{Mar 6:17}},
  Luk={},
  Joh={}
}
\DescribeEvent{
  n={68},
  event={Молва об Иисусе Христе, смятение Ирода},
  Mat={\bibref[14:1--2]{Mat 14:1}},
  Mar={\bibref[6:14--16]{Mar 6:14}},
  Luk={\bibref[9:7--9]{Luk 9:7}},
  Joh={}
}
\DescribeEvent{
  n={69},
  event={Возвращение 12 с проповеди},
  Mat={},
  Mar={\bibref[6:30]{Mar 6:30}},
  Luk={\bibref[9:10]{Luk 9:10}},
  Joh={}
}
\DescribeEvent{
  n={70},
  event={Насыщение 5000 народа пятью хлебами и двумя рыбами},
  Mat={\bibref[14:13--21]{Mat 14:13}},
  Mar={\bibref[6:31--44]{Mar 6:31}},
  Luk={\bibref[9:11--17]{Luk 9:11}},
  Joh={\bibref[6:1--14]{Joh 6:1}}
}
\DescribeEvent{
  n={71},
  event={Шествие Иисуса Христа к ученикам по воде},
  Mat={\bibref[14:22--34]{Mat 14:22}},
  Mar={\bibref[6:45--53]{Mar 6:45}},
  Luk={},
  Joh={\bibref[6:15--21]{Joh 6:15}}
}
\DescribeEvent{
  n={72},
  event={Исцеление больных в Геннисаретской стране},
  Mat={\bibref[14:35--36]{Mat 14:35}},
  Mar={\bibref[6:54--56]{Mar 6:54}},
  Luk={},
  Joh={}
}
\DescribeEvent{
  n={73},
  event={Беседа Иисуса Христа о небесном хлебе},
  Mat={},
  Mar={},
  Luk={},
  Joh={\bibref[6:22--71]{Joh 6:22}}
}
\hline
\EventHeaderCap{События от третьей пасхи до четвертой --- пасхи страданий}
\hline
\DescribeEvent{
  n={74},
  event={Обличение иудеев в лицемерном исполнении заповедей},
  Mat={},
  Mar={\bibref[7:1--23]{Mar 7:1}},
  Luk={},
  Joh={}
}
\DescribeEvent{
  n={75},
  event={Путешествие Иисуса Христа в пределы Тира и Сидона, исцеление дочери хананеянки},
  Mat={\bibref[15:21--28]{Mat 15:21}},
  Mar={\bibref[7:24--30]{Mar 7:24}},
  Luk={},
  Joh={}
}
\DescribeEvent{
  n={76},
  event={Исцеление глухонемого},
  Mat={},
  Mar={\bibref[7:31--37]{Mar 7:31}},
  Luk={},
  Joh={}
}
\DescribeEvent{
  n={77},
  event={Исцеление множества народа у Геннисаретского озера},
  Mat={\bibref[15:29--31]{Mat 15:29}},
  Mar={},
  Luk={},
  Joh={}
}
\DescribeEvent{
  n={78},
  event={Насыщение 4000 семью хлебами и несколькими рыбами},
  Mat={\bibref[15:32--38]{Mat 15:32}},
  Mar={\bibref[8:1--9]{Mar 8:1}},
  Luk={},
  Joh={}
}
\DescribeEvent{
  n={79},
  event={Прибытие в страну Магдалинскую. Ответ фарисеям просившим знамения с неба},
  Mat={\bibref[15:39--16:4]{Mat 15:39}},
  Mar={\bibref[8:10--13]{Mar 8:10}},
  Luk={},
  Joh={}
}
\DescribeEvent{
  n={80},
  event={Предостережение ученикам от закваски фарисейской},
  Mat={\bibref[15:1--12]{Mat 15:1}},
  Mar={\bibref[8:14--21]{Mar 8:14}},
  Luk={},
  Joh={}
}
\DescribeEvent{
  n={81},
  event={Исцеление слепого в Вифсаиде},
  Mat={},
  Mar={\bibref[8:22--26]{Mar 8:22}},
  Luk={},
  Joh={}
}
\DescribeEvent{
  n={82},
  event={Исповедание Петра у Кесарии Филипповой},
  Mat={\bibref[16:13--20]{Mat 16:13}},
  Mar={\bibref[8:27--30]{Mar 8:27}},
  Luk={\bibref[9:18--21]{Luk 9:18}},
  Joh={}
}
\DescribeEvent{
  n={83},
  event={Первое предсказание Иисуса Христа о крестных страданиях; наставление о несении креста},
  Mat={\bibref[16:21--28]{Mat 16:21}},
  Mar={\bibref[8:31--9:1]{Mar 8:31}},
  Luk={\bibref[9:22--27]{Luk 9:22}},
  Joh={}
}
\DescribeEvent{
  n={84},
  event={Преображение Господне},
  Mat={\bibref[17:1--13]{Mat 17:1}},
  Mar={\bibref[9:1--9]{Mar 9:1}},
  Luk={\bibref[9:28--36]{Luk 9:28}},
  Joh={}
}
\DescribeEvent{
  n={85},
  event={Второе предсказание Иисуса Христа о крестных страданиях},
  Mat={},
  Mar={\bibref[9:10--13]{Mar 9:10}},
  Luk={},
  Joh={}
}
\DescribeEvent{
  n={86},
  event={Исцеление бесноватого лунатика},
  Mat={\bibref[17:14--21]{Mat 17:14}},
  Mar={\bibref[9:14--29]{Mar 9:14}},
  Luk={\bibref[9:37--43]{Luk 9:37}},
  Joh={}
}
\DescribeEvent{
  n={87},
  event={Третье предсказание Иисуса Христа о крестных страданиях},
  Mat={\bibref[17:22--23]{Mat 17:22}},
  Mar={\bibref[9:30--32]{Mar 9:30}},
  Luk={\bibref[9:43--45]{Luk 9:43}},
  Joh={}
}
\DescribeEvent{
  n={88},
  event={Последнее пребывание Господа Иисуса Христа в Капернауме; ответ Иисуса Христа о подати на храм},
  Mat={\bibref[17:24--27]{Mat 17:24}},
  Mar={},
  Luk={},
  Joh={}
}
\DescribeEvent{
  n={89},
  event={Наставление о смирении},
  Mat={\bibref[18:1--6]{Mat 18:1}},
  Mar={\bibref[9:33--37]{Mar 9:33}},
  Luk={\bibref[9:46--50]{Luk 9:46}},
  Joh={}
}
\DescribeEvent{
  n={90},
  event={Речь о спасении погибающих, притча о пропавшей овце},
  Mat={\bibref[18:7--17]{Mat 18:7}},
  Mar={},
  Luk={},
  Joh={}
}
\DescribeEvent{
  n={91},
  event={Учение о прощении грехов ближнего},
  Mat={\bibref[18:18--22]{Mat 18:18}, \bibref[18:23--35]{Mat 18:23}},
  Mar={},
  Luk={},
  Joh={}
}
\hline
\EventHeader{Господь Иисус Христос в Галилее пред праздником Кущей}
\hline
\DescribeEvent{
  n={92},
  event={Отказ Иисуса Христа идти на праздник в Иерусалим},
  Mat={},
  Mar={},
  Luk={},
  Joh={\bibref[7:2--9]{Joh 7:2}}
}
\DescribeEvent{
  n={93},
  event={Путешествие Иисуса Христа в Иерусалим},
  Mat={\bibref[19:1]{Mat 19:1}},
  Mar={\bibref[10:1]{Mar 10:1}},
  Luk={\bibref[9:51]{Luk 9:51}},
  Joh={}
}
\DescribeEvent{
  n={94},
  event={Неприязненное отношение к Иисусу Христу самарийцев},
  Mat={},
  Mar={},
  Luk={\bibref[9:52--56]{Luk 9:52}},
  Joh={}
}
\DescribeEvent{
  n={95},
  event={Ответ Господа Иисуса Христа пожелавшим следовать за Ним},
  Mat={},
  Mar={},
  Luk={\bibref[9:57--62]{Luk 9:57}},
  Joh={}
}
\DescribeEvent{
  n={96},
  event={Послание 70 на проповедь; укор городам не принявшим проповеди},
  Mat={\bibref[11:20--24]{Mat 11:20}},
  Mar={},
  Luk={\bibref[10:1--24]{Luk 10:1}},
  Joh={}
}
\DescribeEvent{
  n={97},
  event={Призыв труждающихся и обремененных},
  Mat={\bibref[11:27--30]{Mat 11:27}},
  Mar={},
  Luk={},
  Joh={}
}
\DescribeEvent{
  n={98},
  event={Вопрос законника о вечной жизни и о ближнем и притча о милосердном самарянине},
  Mat={},
  Mar={},
  Luk={\bibref[10:25--37]{Luk 10:25}},
  Joh={}
}
\DescribeEvent{
  n={99},
  event={Иисус Христос в доме Марфы и Марии},
  Mat={},
  Mar={},
  Luk={\bibref[10:38--42]{Luk 10:38}},
  Joh={}
}
\DescribeEvent{
  n={100},
  event={Учение Иисуса Христа о молитве},
  Mat={},
  Mar={},
  Luk={\bibref[11:1--13]{Luk 11:1}},
  Joh={}
}
\DescribeEvent{
  n={101},
  event={Иисус на обеде у фарисея: обличение законников},
  Mat={},
  Mar={},
  Luk={\bibref[11:37--54]{Luk 11:37}},
  Joh={}
}
\DescribeEvent{
  n={102},
  event={Наставления о правилах жизни Христовым последователям},
  Mat={},
  Mar={},
  Luk={\bibref[12:1--59]{Luk 12:1}},
  Joh={}
}
\hline
\EventHeader{Иисус Христос в Иерусалиме на празднике Кущей}
\hline
\DescribeEvent{
  n={103},
  event={Спор в народе о Христе},
  Mat={},
  Mar={},
  Luk={},
  Joh={\bibref[7:10--53]{Joh 7:10}}
}
\DescribeEvent{
  n={104},
  event={О женщине взятой в прелюбодеянии},
  Mat={},
  Mar={},
  Luk={},
  Joh={\bibref[8:1--11]{Joh 8:1}}
}
\DescribeEvent{
  n={105},
  event={Обличение иудеев при сокровищнице храма},
  Mat={},
  Mar={},
  Luk={},
  Joh={\bibref[8:12--59]{Joh 8:12}}
}
\DescribeEvent{
  n={106},
  event={Исцеление слепорожденного},
  Mat={},
  Mar={},
  Luk={},
  Joh={\bibref[9:1--41]{Joh 9:1}}
}
\DescribeEvent{
  n={107},
  event={<<Пастырь добрый>>},
  Mat={},
  Mar={},
  Luk={},
  Joh={\bibref[10:1--21]{Joh 10:1}}
}
\DescribeEvent{
  n={108},
  event={Известие об избитых Пилатом галилеянах, наставление о покаянии},
  Mat={},
  Mar={},
  Luk={\bibref[13:1--5]{Luk 13:1}},
  Joh={}
}
\DescribeEvent{
  n={109},
  event={Притча о бесплодной смоковнице},
  Mat={},
  Mar={},
  Luk={\bibref[13:6--9]{Luk 13:6}},
  Joh={}
}
\DescribeEvent{
  n={110},
  event={Исцеление в синагоге в субботу женщины согбенной 18 лет},
  Mat={},
  Mar={},
  Luk={\bibref[13:10--17]{Luk 13:10}},
  Joh={}
}
\DescribeEvent{
  n={111},
  event={Учение Иисуса Христа о Царствии Божием в притчах},
  Mat={},
  Mar={},
  Luk={\bibref[13:18--21]{Luk 13:18}},
  Joh={}
}
\DescribeEvent{
  n={112},
  event={Беседа Иисуса Христа в праздник Обновления в притворе Соломоновом},
  Mat={},
  Mar={},
  Luk={},
  Joh={\bibref[10:22--39]{Joh 10:22}}
}
\DescribeEvent{
  n={113},
  event={Удаление Господа Иисуса Христа из Иерусалима в заиорданскую страну Перею},
  Mat={},
  Mar={},
  Luk={},
  Joh={\bibref[10:40--42]{Joh 10:40}}
}
\hline
\EventHeader{Учение Господа Иисуса Христа на обратном пути из заиорданской страны в Иерусалим}
\hline
\DescribeEvent{
  n={114},
  event={Речь Иисуса Христа о числе спасающихся},
  Mat={},
  Mar={},
  Luk={\bibref[18:22--30]{Luk 18:22}},
  Joh={}
}
\DescribeEvent{
  n={115},
  event={Предсказание о страданиях в Иерусалиме},
  Mat={},
  Mar={},
  Luk={\bibref[13:31--35]{Luk 13:31}},
  Joh={}
}
\DescribeEvent{
  n={116},
  event={Иисус Христос на обеде у фарисея; исцеление больного водянкой; притча о званых на вечерю},
  Mat={},
  Mar={},
  Luk={\bibref[14:1--24]{Luk 14:1}},
  Joh={}
}
\DescribeEvent{
  n={117},
  event={Самоотвержение последователей Христа},
  Mat={},
  Mar={},
  Luk={\bibref[14:25--35]{Luk 14:25}},
  Joh={}
}
\DescribeEvent{
  n={118},
  event={Притчи о погибшей овце, потерянной драхме и блудном сыне},
  Mat={},
  Mar={},
  Luk={\bibref[15:1--32]{Luk 15:1}},
  Joh={}
}
\DescribeEvent{
  n={119},
  event={О неверном домоправителе},
  Mat={},
  Mar={},
  Luk={\bibref[16:1--13]{Luk 16:1}},
  Joh={}
}
\DescribeEvent{
  n={120},
  event={Обличение фарисеев; притча о богаче и Лазаре},
  Mat={},
  Mar={},
  Luk={\bibref[16:14--31]{Luk 16:14}},
  Joh={}
}
\DescribeEvent{
  n={121},
  event={Наставление ученикам о исполнении долга},
  Mat={},
  Mar={},
  Luk={\bibref[17:1--10]{Luk 17:1}},
  Joh={}
}
\DescribeEvent{
  n={122},
  event={Исцеление десяти прокаженных},
  Mat={},
  Mar={},
  Luk={\bibref[17:11--19]{Luk 17:11}},
  Joh={}
}
\DescribeEvent{
  n={123},
  event={Ответ о втором пришествии},
  Mat={},
  Mar={},
  Luk={\bibref[17:20--37]{Luk 17:20}},
  Joh={}
}
\DescribeEvent{
  n={124},
  event={Притча о несправедливом судье},
  Mat={},
  Mar={},
  Luk={\bibref[18:1--8]{Luk 18:1}},
  Joh={}
}
\DescribeEvent{
  n={125},
  event={Притча о мытаре и фарисее},
  Mat={},
  Mar={},
  Luk={\bibref[18:9--14]{Luk 18:9}},
  Joh={}
}
\DescribeEvent{
  n={126},
  event={Учение о браке},
  Mat={\bibref[19:1--12]{Mat 19:1}},
  Mar={\bibref[10:1--12]{Mar 10:1}},
  Luk={\bibref[16:18]{Luk 16:18}},
  Joh={}
}
\DescribeEvent{
  n={127},
  event={Благословение детей},
  Mat={\bibref[19:13--15]{Mat 19:13}},
  Mar={\bibref[10:13--16]{Mar 10:13}},
  Luk={\bibref[18:15--17]{Luk 18:15}},
  Joh={}
}
\DescribeEvent{
  n={128},
  event={Ответ богатому юноше},
  Mat={\bibref[19:16--26]{Mat 19:16}},
  Mar={\bibref[10:17--27]{Mar 10:17}},
  Luk={\bibref[18:18--27]{Luk 18:18}},
  Joh={}
}
\DescribeEvent{
  n={129},
  event={Вопрос Петра о награде последователям Иисуса Христа},
  Mat={\bibref[19:27--30]{Mat 19:27}},
  Mar={\bibref[10:28--31]{Mar 10:28}},
  Luk={\bibref[18:28--30]{Luk 18:28}},
  Joh={}
}
\DescribeEvent{
  n={130},
  event={Притча о нанятых в виноградник работниках},
  Mat={\bibref[20:1--16]{Mat 20:1}},
  Mar={},
  Luk={},
  Joh={}
}
\DescribeEvent{
  n={131},
  event={Воскрешение Лазаря},
  Mat={},
  Mar={},
  Luk={},
  Joh={\bibref[11:1--46]{Joh 11:1}}
}
\DescribeEvent{
  n={132},
  event={Заговор начальников иудейских против Иисуса Христа},
  Mat={},
  Mar={},
  Luk={},
  Joh={\bibref[11:47--57]{Joh 11:47}}
}
\DescribeEvent{
  n={133},
  event={Предсказание Иисуса Христа о крестных страданиях на пути к Иерусалиму},
  Mat={\bibref[20:17--19]{Mat 20:17}},
  Mar={\bibref[10:32--34]{Mar 10:32}},
  Luk={\bibref[18:31--34]{Luk 18:31}},
  Joh={}
}
\DescribeEvent{
  n={134},
  event={Просьба сынов Зеведеевых о месте в Царствии Иисуса Христа},
  Mat={\bibref[20:20--28]{Mat 20:20}},
  Mar={\bibref[10:35--45]{Mar 10:35}},
  Luk={},
  Joh={}
}
\DescribeEvent{
  n={135},
  event={Иисус Христос в Иерихоне; исцеление слепцов},
  Mat={\bibref[20:29--34]{Mat 20:29}},
  Mar={\bibref[10:46--52]{Mar 10:46}},
  Luk={\bibref[18:35--43]{Luk 18:35}},
  Joh={}
}
\DescribeEvent{
  n={136},
  event={Обращение Закхея},
  Mat={},
  Mar={},
  Luk={\bibref[19:1--10]{Luk 19:1}},
  Joh={}
}
\DescribeEvent{
  n={137},
  event={Притча об ушедшем на войну и об умноживших таланты},
  Mat={\bibref[25:13--30]{Mat 25:13}},
  Mar={},
  Luk={\bibref[19:11--28]{Luk 19:11}},
  Joh={}
}
\DescribeEvent{
  n={138},
  event={Иисус Христос в Вифании},
  Mat={\bibref[26:6--13]{Mat 26:6}},
  Mar={\bibref[14:3--9]{Mar 14:3}},
  Luk={},
  Joh={\bibref[11:55--12:11]{Joh 11:55}}
}
\DescribeEvent{
  n={139},
  event={Вход Господень в Иерусалим; исцеление больных в Иерусалиме},
  Mat={\bibref[21:1--11]{Mat 21:1}, \bibref[14--17]{Mat 21:14}},
  Mar={\bibref[11:1--11]{Mar 11:1}},
  Luk={\bibref[19:29--44]{Luk 19:29}},
  Joh={\bibref[12:12--19]{Joh 12:12}}
}
\hline
\EventHeader{Великий понедельник}
\hline
\DescribeEvent{
  n={140},
  event={Бесплодная смоковница},
  Mat={\bibref[21:18--22]{Mat 21:18}},
  Mar={\bibref[11:12--14]{Mar 11:12}, \bibref[20--26]{Mar 11:20}},
  Luk={},
  Joh={}
}
\DescribeEvent{
  n={141},
  event={Изгнание торгующих из храма},
  Mat={\bibref[21:12--13]{Mat 21:12}},
  Mar={\bibref[11:15--19]{Mar 11:15}},
  Luk={\bibref[19:45--46]{Luk 19:45}},
  Joh={}
}
\hline
\EventHeader{Великий вторник}
\hline
\DescribeEvent{
  n={142},
  event={Обличение начальников иудейских и поучения Господа в храме},
  Mat={\bibref[21:23--23:39]{Mat 21:23}},
  Mar={\bibref[11:27--12:40]{Mar 11:27}},
  Luk={\bibref[19:47--20:47]{Luk 19:47}},
  Joh={}
}
\DescribeEvent{
  n={143},
  event={О жертве бедной вдовицы},
  Mat={},
  Mar={\bibref[12:41--44]{Mar 12:41}},
  Luk={\bibref[21:1--4]{Luk 21:1}},
  Joh={}
}
\DescribeEvent{
  n={144},
  event={Речь Иисуса Христа пред эллинами и иудеями},
  Mat={},
  Mar={},
  Luk={},
  Joh={\bibref[12:20--50]{Joh 12:20}}
}
\DescribeEvent{
  n={145},
  event={Пророчества и притчи Иисуса Христа о Иерусалиме и о втором пришествии},
  Mat={\bibref[24:1--25:46]{Mat 24:1}},
  Mar={\bibref[13:1--37]{Mar 13:1}},
  Luk={\bibref[21:5--38]{Luk 21:5}},
  Joh={}
}
\hline
\EventHeader{Великая среда}
\hline
\DescribeEvent{
  n={146},
  event={Иисус Христос в Вифании},
  Mat={\bibref[26:1--2]{Mat 26:1}, \bibref[6--13]{Mat 26:6}},
  Mar={\bibref[14:3--9]{Mar 14:3}},
  Luk={},
  Joh={}
}
\DescribeEvent{
  n={147},
  event={Заговор иудеев, предательство Иуды},
  Mat={\bibref[26:3--5]{Mat 26:3}, \bibref[14--16]{Mat 26:14}},
  Mar={\bibref[14:1--2]{Mar 14:1}, \bibref[10--11]{Mar 14:10}},
  Luk={\bibref[22:1--6]{Luk 22:1}},
  Joh={}
}
\hline
\EventHeader{Великий четверг}
\hline
\DescribeEvent{
  n={148},
  event={Пасхальная вечеря},
  Mat={\bibref[26:17--35]{Mat 26:17}},
  Mar={\bibref[14:12--31]{Mar 14:12}},
  Luk={\bibref[22:7--38]{Luk 22:7}},
  Joh={\bibref[13:1--17]{Joh 13:1}, \bibref[26]{Joh 13:26}}
}
\DescribeEvent{
  n={149},
  event={Моление о чаше},
  Mat={\bibref[26:36--46]{Mat 26:36}},
  Mar={\bibref[14:32--42]{Mar 14:32}},
  Luk={\bibref[22:39--46]{Luk 22:39}},
  Joh={\bibref[18:1]{Joh 18:1}}
}
\hline
\EventHeader{Великая пятница}
\hline
\DescribeEvent{
  n={150},
  event={Взятие Иисуса Христа под стражу},
  Mat={\bibref[26:47--56]{Mat 26:47}},
  Mar={\bibref[14:43--52]{Mar 14:43}},
  Luk={\bibref[22:47--53]{Luk 22:47}},
  Joh={\bibref[18:2--12]{Joh 18:2}}
}
\DescribeEvent{
  n={151},
  event={Допрос Иисуса Христа у Анны},
  Mat={},
  Mar={},
  Luk={},
  Joh={\bibref[18:13]{Joh 18:13}, \bibref[19--24]{Joh 18:19}}
}
\DescribeEvent{
  n={152},
  event={Суд Синедриона над Иисусом Христом у Каиафы},
  Mat={\bibref[26:57--68]{Mat 26:57}},
  Mar={\bibref[14:53--65]{Mar 14:53}},
  Luk={\bibref[22:54]{Luk 22:54}, \bibref[63--65]{Luk 22:63}},
  Joh={\bibref[18:14]{Joh 18:14}}
}
\DescribeEvent{
  n={153},
  event={Отречение Петра},
  Mat={\bibref[26:58]{Mat 26:58}, \bibref[69--75]{Mat 26:69}},
  Mar={\bibref[14:54]{Mar 14:54}, \bibref[66--72]{Mar 14:66}},
  Luk={\bibref[22:54--62]{Luk 22:54}},
  Joh={\bibref[18:15--18]{Joh 18:15}, \bibref[25--27]{Joh 18:25}}
}
\DescribeEvent{
  n={154},
  event={Приговор Синедриона},
  Mat={\bibref[27:1]{Mat 27:1}},
  Mar={\bibref[15:1]{Mar 15:1}},
  Luk={\bibref[22:66--71]{Luk 22:66}},
  Joh={}
}
\DescribeEvent{
  n={155},
  event={Конец Иуды},
  Mat={\bibref[27:3--10]{Mat 27:3}},
  Mar={},
  Luk={},
  Joh={}
}
\DescribeEvent{
  n={156},
  event={Иисус Христос у Пилата},
  Mat={\bibref[27:2--31]{Mat 27:2}},
  Mar={\bibref[15:1--20]{Mar 15:1}},
  Luk={\bibref[23:1--25]{Luk 23:1}},
  Joh={\bibref[18:28--19:16]{Joh 18:28}}
}
\DescribeEvent{
  n={157},
  event={Крестный путь и Голгофа},
  Mat={\bibref[27:31--56]{Mat 27:31}},
  Mar={\bibref[15:20--41]{Mar 15:20}},
  Luk={\bibref[23:26--49]{Luk 23:26}},
  Joh={\bibref[19:16--37]{Joh 19:16}}
}
\DescribeEvent{
  n={158},
  event={Погребение Иисуса Христа},
  Mat={\bibref[27:57--66]{Mat 27:57}},
  Mar={\bibref[15:42--47]{Mar 15:42}},
  Luk={\bibref[23:50--56]{Luk 23:50}},
  Joh={\bibref[19:38--42]{Joh 19:38}}
}
\hline
\EventHeaderCap{Воскресение Иисуса Христа}
\EventHeader{Утро Воскресения}
\hline
\DescribeEvent{
  n={159},
  event={Поздно в субботу Мария Магдалина с другой Марией идут смотреть гроб},
  Mat={\bibref[28:1]{Mat 28:1}},
  Mar={},
  Luk={},
  Joh={}
}
\DescribeEvent{
  n={160},
  event={Мария Магдалина и другие женщины покупают ароматы чтобы утром помазать Иисуса},
  Mat={},
  Mar={\bibref[16:1]{Mar 16:1}},
  Luk={},
  Joh={}
}
\DescribeEvent{
  n={161},
  event={Землетрясение, ангел отваливает камень от пещеры},
  Mat={\bibref[28:2--4]{Mat 28:2}},
  Mar={},
  Luk={},
  Joh={}
}
\DescribeEvent{
  n={162},
  event={Мария Магдалина спешит ко гробу. Увидев, что он пуст, она возвращается к Петру и Иоанну},
  Mat={},
  Mar={},
  Luk={},
  Joh={\bibref[20:1--3]{Joh 20:1}}
}
\DescribeEvent{
  n={163},
  event={Приход ко гробу до восхода солнца группы галилейских жен-мироносиц; явление им ангелов},
  Mat={},
  Mar={},
  Luk={\bibref[24:1--9]{Luk 24:1}},
  Joh={}
}
\DescribeEvent{
  n={164},
  event={Петр и Иоанн с Марией Магдалиной прибегают к пустому гробу},
  Mat={},
  Mar={},
  Luk={},
  Joh={\bibref[20:4--10]{Joh 20:4}}
}
\DescribeEvent{
  n={165},
  event={Явление Воскресшего Христа Марии Магдалине},
  Mat={},
  Mar={},
  Luk={},
  Joh={\bibref[20:11--18]{Joh 20:11}}
}
\DescribeEvent{
  n={166},
  event={Приход на гроб при восходе солнца другой группы жен-мироносиц; явление им ангела},
  Mat={\bibref[28:5--8]{Mat 28:5}},
  Mar={\bibref[16:1--8]{Mar 16:1}},
  Luk={},
  Joh={}
}
\DescribeEvent{
  n={167},
  event={Уход мироносиц от гроба; явление им Воскресшего Христа},
  Mat={\bibref[28:9--10]{Mat 28:9}},
  Mar={},
  Luk={},
  Joh={}
}
\DescribeEvent{
  n={168},
  event={Извещение учеников о Воскресении Господа группой галилейских жен-мироносиц},
  Mat={},
  Mar={\bibref[16:8]{Mar 16:8}},
  Luk={\bibref[24:9--12]{Luk 24:9}},
  Joh={}
}
\DescribeEvent{
  n={169},
  event={Извещение учеников о Воскресении Господа Марией Магдалиной},
  Mat={},
  Mar={\bibref[16:9--11]{Mar 16:9}},
  Luk={},
  Joh={}
}
\DescribeEvent{
  n={170},
  event={Явление Иисуса Христа Эммаусским путникам},
  Mat={},
  Mar={\bibref[16:12--13]{Mar 16:12}},
  Luk={\bibref[24:13--35]{Luk 24:13}},
  Joh={}
}
\DescribeEvent{
  n={171},
  event={Явление Иисуса Христа в первый день недели ученикам без Фомы},
  Mat={},
  Mar={\bibref[16:14]{Mar 16:14}},
  Luk={\bibref[24:36--49]{Luk 24:36}},
  Joh={\bibref[20:19--25]{Joh 20:19}}
}
\DescribeEvent{
  n={172},
  event={Явление Иисуса Христа по прошествии восьми дней},
  Mat={},
  Mar={},
  Luk={},
  Joh={\bibref[20:26--29]{Joh 20:26}}
}
\DescribeEvent{
  n={173},
  event={Явление Иисуса Христа при море Тивериадском},
  Mat={},
  Mar={},
  Luk={},
  Joh={\bibref[21:1--25]{Joh 21:1}}
}
\DescribeEvent{
  n={174},
  event={Явление Иисуса Христа на горе},
  Mat={\bibref[28:16--20]{Mat 28:16}},
  Mar={\bibref[16:15--18]{Mar 16:15}},
  Luk={},
  Joh={}
}
\DescribeEvent{
  n={175},
  event={Вознесение Господа},
  Mat={},
  Mar={\bibref[16:19--20]{Mar 16:19}},
  Luk={\bibref[24:50--53]{Luk 24:50}},
  Joh={}
}
\end{longtable}
\end{landscape}
\newpage
%\bibpdfbookmark{Денежные Единицы}{ntmoney}
\bibmark{book}{ДЕНЕЖНЫЕ ЕДИНИЦЫ}
\thispagestyle{empty}
\pagestyle{fancy}

\begin{center}
\normalsize\bfseries
ДЕНЕЖНЫЕ ЕДИНИЦЫ В НОВОМ ЗАВЕТЕ
\end{center}

%\begin{multicols}{2}
В новозаветное время в Палестине в основном имели хождение монеты
греческие и римские. В книгах Нового Завета упоминается десять видов
монет: одна монета иудейской чеканки, пять --- греческой и четыре ---
римской.

Ходячей \bibemph{иудейской монетой} был \textbf{сребреник}, остаток
маккавейской чеканки. Он равнялся сиклю и считался национальной
монетой, употреблявшейся предпочтительно пред всеми другими при
храме. За тридцать таких сребреников Иуда предал Христа (\bibref{Mat
26:15}; \bibref[27:3--6,9]{Mat 27:3}). По тогдашним ценам это была
достаточная сумма, чтобы купить небольшой участок земли даже в
окрестностях Иерусалима.

\bibemph{Греческие монеты}. Основная денежная единица \textbf{драхма}
(\bibref[\bk{Luk}~15:8,~9]{Luk 15:8}) --- серебряная монета, равная
римскому динарию. Одна драхма составляла 6000-ю часть аттического таланта,
100-ю часть мины и разделялась на 6 оволов (оболов). В зависимости от
места чеканки драхма имела разный вес: аттическая драхма ---
4,37~\bibemph{г} серебра (т.~е.\ в два раза меньше нашего серебряного
полтинника чеканки 1922 и 1924 года), эгинская --- 6,3~\bibemph{г}. В
разное время вес монеты и ее цена тоже колебались, поэтому вообще
сравнивать покупную способность древних денег с современными можно
только приблизительно. Серебряная монета достоинством в две драхмы
называлась \textbf{дидрахма} (\bibref{Mat 17:24}); внешне дидрахма
могла походить на серебряный полтинник. Дидрахма приравнивалась к
полусиклю, так что принималась вместо последнего в уплату храмовой
подати. Четыре драхмы составляли \textbf{статир} (\bibref{Mat 17:27}) ---
серебряную монету, называвшуюся также \textbf{тетрадрахмой} (он мог
быть вроде серебряного рубля чеканки 1924 года). Статир приравнивался
к полному священному сиклю или сребренику. Такой статир был найден
ап. Петром в пойманной им рыбе и отдан в уплату храмовой подати за
Иисуса Христа и за себя. Сто драхм или 25 статиров составляли
\textbf{мину} (\bibref{Luk 19:13}). Высшей денежной единицей был
\textbf{талант}, золотой или серебряный (\bibref{Mat 18:24};
\bibref[25:15]{Mat 25:15}; \bibref{Rev 16:21}). Золотой талант был
равен десяти серебряным. Аттический талант равнялся 60 минам или 6000
драхм, а коринфский талант --- 100 минам. Последний более подходит к
ценности собственно еврейского (ветхозаветного) серебряного таланта,
но к I~веку по Р.~Х.\ вес и стоимость таланта понизились.


\bibemph{Римские монеты}. \textbf{Динарий} (denarius) --- серебряная
монета, часто упоминаемая в евангелиях (\bibref{Mat 18:28};
\bibref[20:2]{Mat 20:2}; \bibref{Mar 6:37}; \bibref[12:15]{Mar 12:15};
\bibref{Luk 7:41}; \bibref[20:24]{Luk 20:24}; \bibref{Joh 6:7};
\bibref[12:5]{Joh 12:5}, также \bibref{Rev 6:6}). По весу и ценности
динарий приравнивался к греческой драхме или 1/4 сикля, но во время
земной жизни Спасителя он имел меньшую ценность. На лицевой стороне
монеты изображался царствующий император
(\bibref[\bk{Mat}~22:19--21]{Mat 22:19}).  Динарий составлял
ежедневную плату римскому воину, как драхма --- ежедневную плату
афинским воинам. Он же составлял обычную поденную плату рабочим
(\bibref{Mat 20:2}). Динарию же равнялась поголовная подать, которую
иудеи обязаны были платить римлянам (\bibref{Mat 22:19}). Динарий
разделялся на десять, а позднее --- на шестнадцать \textbf{ассариев}
или \textbf{асов} (\bibref{Mat 10:29}; \bibref{Luk 12:6}). Это была
медная монета. Четвертую часть ассария составлял \textbf{кодрант}
(quadrans) (\bibref{Mat 5:26}; \bibref{Mar 12:42}). На этих монетах
тоже изображался император. Половину кодранта составляла минута
(minutum) или \textbf{лепта} --- в русском переводе <<полушка>>
(\bibref{Luk 12:59}; \bibref{Mar 12:42}) --- самая мелкая медная
монета. Две такие монеты и положила в сокровищницу храма бедная
вдовица (\bibref[\bk{Mar}~12:41--44]{Mar 12:41}).
%\end{multicols}

\pagestyle{fancy}
\bibpart{Книги Ветхого Завета\\[5pt]\normalfont\protect\small(Знаком * отмечены книги неканонические)}{Ветхий Завет}{OT}
\thispagestyle{empty}
\include{tex/Gen}
\include{tex/Exo}
\bibbookdescr{Lev}{
  inline={\LARGE Третья книга Моисеева\\\Huge Левит},
  toc={Левит},
  bookmark={Левит},
  header={Левит},
  %headerleft={},
  %headerright={},
  abbr={Лев}
}
\vs Lev 1:1 И воззвал Господь к Моисею и сказал ему из скинии собрания, говоря:
\vs Lev 1:2 объяви сынам Израилевым и скажи им: когда кто из вас хочет принести жертву Господу, то, если из скота, приносите жертву вашу из скота крупного и мелкого.
\vs Lev 1:3 Если жертва его есть всесожжение из крупного скота, пусть принесет ее мужеского пола, без порока; пусть приведет ее к дверям скинии собрания, чтобы приобрести ему благоволение пред Господом;
\vs Lev 1:4 и возложит руку свою на голову \bibemph{жертвы} всесожжения~--- и приобретет он благоволение, во очищение грехов его;
\vs Lev 1:5 и заколет тельца пред Господом; сыны же Аароновы, священники, принесут кровь и покропят кровью со всех сторон на жертвенник, который у входа скинии собрания;
\vs Lev 1:6 и снимет кожу с \bibemph{жертвы} всесожжения и рассечет ее на части;
\vs Lev 1:7 сыны же Аароновы, священники, положат на жертвенник огонь и на огне разложат дрова;
\vs Lev 1:8 и разложат сыны Аароновы, священники, части, голову и тук на дровах, которые на огне, на жертвеннике;
\vs Lev 1:9 а внутренности \bibemph{жертвы} и ноги ее вымоет он водою, и сожжет священник все на жертвеннике: \bibemph{это} всесожжение, жертва, благоухание, приятное Господу.
\rsbpar\vs Lev 1:10 Если жертва всесожжения его [Господу] из мелкого скота, из овец, или из коз, пусть принесет ее мужеского пола, без порока, [и пусть возложит руку на голову ее,]
\vs Lev 1:11 и заколет ее пред Господом на северной стороне жертвенника, и сыны Аароновы, священники, покропят кровью ее на жертвенник со всех сторон;
\vs Lev 1:12 и рассекут ее на части, \bibemph{отделив} голову ее и тук ее, и разложит их священник на дровах, которые на огне, на жертвеннике,
\vs Lev 1:13 а внутренности и ноги вымоет водою, и принесет священник всё и сожжет на жертвеннике: \bibemph{это} всесожжение, жертва, благоухание, приятное Господу.
\rsbpar\vs Lev 1:14 Если же из птиц приносит он Господу всесожжение, пусть принесет жертву свою из горлиц, или из молодых голубей;
\vs Lev 1:15 священник принесет ее к жертвеннику, и свернет ей голову, и сожжет на жертвеннике, а кровь выцедит к стене жертвенника;
\vs Lev 1:16 зоб ее с перьями ее отнимет и бросит его подле жертвенника на восточную сторону, где пепел;
\vs Lev 1:17 и надломит ее в крыльях ее, не отделяя их, и сожжет ее священник на жертвеннике, на дровах, которые на огне: это всесожжение, жертва, благоухание, приятное Господу.
\vs Lev 2:1 Если какая душа хочет принести Господу жертву приношения хлебного, пусть принесет пшеничной муки, и вольет на нее елея, и положит на нее ливана,
\vs Lev 2:2 и принесет ее к сынам Аароновым, священникам, и возьмет полную горсть муки с елеем и со всем ливаном, и сожжет сие священник в память на жертвеннике; \bibemph{это} жертва, благоухание, приятное Господу;
\vs Lev 2:3 а остатки от приношения хлебного Аарону и сынам его: \bibemph{это} великая святыня из жертв Господних.
\rsbpar\vs Lev 2:4 Если же приносишь жертву приношения хлебного из печеного в печи, \bibemph{то приноси} пшеничные хлебы пресные, смешанные с елеем, и лепешки пресные, помазанные елеем.
\vs Lev 2:5 Если жертва твоя приношение хлебное со сковороды, то это должна быть пшеничная мука, смешанная с елеем, пресная;
\vs Lev 2:6 разломи ее на куски и влей на нее елея: это приношение хлебное [Господу].
\vs Lev 2:7 Если жертва твоя приношение хлебное из горшка, то должно сделать оное из пшеничной муки с елеем,
\vs Lev 2:8 и принеси приношение, которое из сего составлено, Господу; представь оное священнику, а он принесет его к жертвеннику;
\vs Lev 2:9 и возьмет священник из сей жертвы часть в память и сожжет на жертвеннике: \bibemph{это} жертва, благоухание, приятное Господу;
\vs Lev 2:10 а остатки приношения хлебного Аарону и сынам его: \bibemph{это} великая святыня из жертв Господних.
\rsbpar\vs Lev 2:11 Никакого приношения хлебного, которое прин\acc{о}сите Господу, не делайте квасного, ибо ни квасного, ни меду не должны вы сожигать в жертву Господу;
\vs Lev 2:12 как приношение начатков принос\acc{и}те их Господу, а на жертвенник не должно возносить их в приятное благоухание.
\vs Lev 2:13 Всякое приношение твое хлебное сол\acc{и} солью, и не оставляй жертвы твоей без соли завета Бога твоего: при всяком приношении твоем приноси [Господу Богу твоему] соль.
\rsbpar\vs Lev 2:14 Если приносишь Господу приношение хлебное из первых плодов, приноси в дар от первых плодов твоих из колосьев, высушенных на огне, растолченные зерна,
\vs Lev 2:15 и влей на них елея, и положи на них ливана: это приношение хлебное;
\vs Lev 2:16 и сожжет священник в память часть зерен и елея со всем ливаном: \bibemph{это} жертва Господу.
\vs Lev 3:1 Если жертва его жертва мирная, и если он приносит из крупного скота, мужеского или женского пола, пусть принесет ее Господу, не имеющую порока,
\vs Lev 3:2 и возложит руку свою на голову жертвы своей, и заколет ее у дверей скинии собрания; сыны же Аароновы, священники, покропят кровью на жертвенник со всех сторон;
\vs Lev 3:3 и принесет он из мирной жертвы в жертву Господу тук, покрывающий внутренности, и весь тук, который на внутренностях,
\vs Lev 3:4 и обе почки и тук, который на них, который на стегнах, и сальник, который на печени; с почками он отделит это;
\vs Lev 3:5 и сыны Аароновы сожгут это на жертвеннике вместе со всесожжением, которое на дровах, на огне: \bibemph{это} жертва, благоухание, приятное Господу.
\rsbpar\vs Lev 3:6 А если из мелкого скота приносит он мирную жертву Господу, мужеского или женского пола, пусть принесет ее, не имеющую порока.
\vs Lev 3:7 Если из овец приносит он жертву свою, пусть представит ее пред Господа,
\vs Lev 3:8 и возложит руку свою на голову жертвы своей, и заколет ее пред скиниею собрания, и сыны Аароновы покропят кровью ее на жертвенник со всех сторон;
\vs Lev 3:9 и пусть принесет из мирной жертвы в жертву Господу тук ее, весь курдюк, отрезав его по самую хребтовую кость, и тук, покрывающий внутренности, и весь тук, который на внутренностях,
\vs Lev 3:10 и обе почки и тук, который на них, который на стегнах, и сальник, который на печени; с почками он отделит это;
\vs Lev 3:11 священник сожжет это на жертвеннике; \bibemph{это} пища огня~--- жертва Господу.
\rsbpar\vs Lev 3:12 А если он приносит жертву из коз, пусть представит ее пред Господа,
\vs Lev 3:13 и возложит руку свою на голову ее, и заколет ее перед скиниею собрания, и покропят сыны Аароновы кровью ее на жертвенник со всех сторон;
\vs Lev 3:14 и принесет из нее в приношение, в жертву Господу тук, покрывающий внутренности, и весь тук, который на внутренностях,
\vs Lev 3:15 и обе почки и тук, который на них, который на стегнах, и сальник, который на печени; с почками он отделит это;
\vs Lev 3:16 и сожжет их священник на жертвеннике: \bibemph{это} пища огня~--- приятное благоухание [Господу]; весь тук Господу.
\vs Lev 3:17 Это постановление вечное в роды ваши, во всех жилищах ваших; никакого тука и никакой крови не ешьте.
\vs Lev 4:1 И сказал Господь Моисею, говоря:
\vs Lev 4:2 скажи сынам Израилевым: если какая душа согрешит по ошибке против каких-либо заповедей Господних и сделает что-нибудь, чего не должно делать;
\vs Lev 4:3 если священник помазанный согрешит и сделает виновным народ,~--- то за грех свой, которым согрешил, пусть представит из крупного скота тельца, без порока, Господу в жертву о грехе,
\vs Lev 4:4 и приведет тельца к дверям скинии собрания пред Господа, и возложит руки свои на голову тельца, и заколет тельца пред Господом;
\vs Lev 4:5 и возьмет священник помазанный, [посвященный совершенным посвящением,] крови тельца и внесет ее в скинию собрания,
\vs Lev 4:6 и омочит священник перст свой в кровь и покропит кровью семь раз пред Господом пред завесою святилища;
\vs Lev 4:7 и возложит священник крови [тельца] пред Господом на роги жертвенника благовонных курений, который в скинии собрания, а остальную кровь тельца выльет к подножию жертвенника всесожжений, который у входа скинии собрания;
\vs Lev 4:8 и вынет из тельца за грех весь тук его, тук, покрывающий внутренности, и весь тук, который на внутренностях,
\vs Lev 4:9 и обе почки и тук, который на них, который на стегнах, и сальник на печени; с почками отделит он это,
\vs Lev 4:10 как отделяется из тельца жертвы мирной; и сожжет их священник на жертвеннике всесожжения;
\vs Lev 4:11 а кожу тельца и все мясо его с головою и с ногами его, и внутренности его и нечистоту его,
\vs Lev 4:12 всего тельца пусть вынесет вне стана на чистое место, где высыпается пепел, и сожжет его огнем на дровах; где высыпается пепел, там пусть сожжен будет.
\rsbpar\vs Lev 4:13 Если же все общество Израилево согрешит по ошибке и скрыто будет дело от глаз собрания, и сделает что-нибудь против заповедей Господних, чего не надлежало делать, и будет виновно,
\vs Lev 4:14 то, когда узнан будет грех, которым они согрешили, пусть от всего общества представят они из крупного скота тельца в жертву за грех и приведут его пред скинию собрания;
\vs Lev 4:15 и возложат старейшины общества руки свои на голову тельца пред Господом и заколют тельца пред Господом.
\vs Lev 4:16 И внесет священник помазанный крови тельца в скинию собрания,
\vs Lev 4:17 и омочит священник перст свой в кровь [тельца] и покропит семь раз пред Господом пред завесою [святилища],
\vs Lev 4:18 и возложит крови на роги жертвенника [благовонных курений], который пред лицем Господним в скинии собрания, а остальную кровь выльет к подножию жертвенника всесожжений, который у входа скинии собрания;
\vs Lev 4:19 и весь тук его вынет из него и сожжет на жертвеннике;
\vs Lev 4:20 и сделает с тельцом то, что делается с тельцом за грех; так должен сделать с ним, и так очистит их священник, и прощено будет им;
\vs Lev 4:21 и вынесет тельца вне стана, и сожжет его так, как сожег прежнего тельца. Это жертва за грех общества.
\rsbpar\vs Lev 4:22 А если согрешит начальник, и сделает по ошибке что-нибудь против заповедей Господа, Бога своего, чего не надлежало делать, и будет виновен,
\vs Lev 4:23 то, когда узнан будет им грех, которым он согрешил, пусть приведет он в жертву козла без порока,
\vs Lev 4:24 и возложит руку свою на голову козла, и заколет его на месте, где заколаются всесожжения пред Господом: это жертва за грех;
\vs Lev 4:25 и возьмет священник перстом своим крови от жертвы за грех и возложит на роги жертвенника всесожжения, а остальную кровь его выльет к подножию жертвенника всесожжения;
\vs Lev 4:26 и весь тук его сожжет на жертвеннике, подобно как тук жертвы мирной, и так очистит его священник от греха его, и прощено будет ему.
\rsbpar\vs Lev 4:27 Если же кто из народа земли согрешит по ошибке и сделает что-нибудь против заповедей Господних, чего не надлежало делать, и виновен будет,
\vs Lev 4:28 то, когда узнан будет им грех, которым он согрешил, пусть приведет он в жертву козу без порока за грех свой, которым он согрешил,
\vs Lev 4:29 и возложит руку свою на голову жертвы за грех, и заколют [козу] в жертву за грех на месте, [где заколают] жертву всесожжения;
\vs Lev 4:30 и возьмет священник крови ее перстом своим, и возложит на роги жертвенника всесожжения, а остальную кровь ее выльет к подножию жертвенника;
\vs Lev 4:31 и весь тук ее отделит, подобно как отделяется тук из жертвы мирной, и сожжет \bibemph{его} священник на жертвеннике в приятное благоухание Господу; и так очистит его священник, и прощено будет ему.
\vs Lev 4:32 А если из стада овец захочет он принести жертву за грех, пусть принесет женского пола, без порока,
\vs Lev 4:33 и возложит руку свою на голову жертвы за грех, и заколет ее в жертву за грех на том месте, где заколают жертву всесожжения;
\vs Lev 4:34 и возьмет священник перстом своим крови от сей жертвы за грех и возложит на роги жертвенника всесожжения, а остальную кровь ее выльет к подножию жертвенника;
\vs Lev 4:35 и весь тук ее отделит, как отделяется тук овцы из жертвы мирной, и сожжет сие священник на жертвеннике в жертву Господу; и так очистит его священник от греха, которым он согрешил, и прощено будет ему.
\vs Lev 5:1 Если кто согрешит тем, что слышал голос проклятия и был свидетелем, или видел, или знал, но не объявил, то он понесет на себе грех.
\vs Lev 5:2 Или если прикоснется к чему-нибудь нечистому, или к трупу зверя нечистого, или к трупу скота нечистого, или к трупу гада нечистого, но не знал того, то он нечист и виновен.
\vs Lev 5:3 Или если прикоснется к нечистоте человеческой, какая бы то ни была нечистота, от которой оскверняются, и он не знал того, но после узнает, то он виновен.
\vs Lev 5:4 Или если кто безрассудно устами своими поклянется сделать что-нибудь худое или доброе, какое бы то ни было дело, в котором люди безрассудно клянутся, и он не знал того, но после узнает, то он виновен в том.
\vs Lev 5:5 Если он виновен в чем-нибудь из сих, и исповедается, в чем он согрешил,
\vs Lev 5:6 то пусть принесет Господу за грех свой, которым он согрешил, жертву повинности из мелкого скота, овцу или козу, за грех, и очистит его священник от греха его.
\vs Lev 5:7 Если же он не в состоянии принести овцы, то в повинность за грех свой пусть принесет Господу двух горлиц или двух молодых голубей, одного в жертву за грех, а другого во всесожжение;
\vs Lev 5:8 пусть принесет их к священнику, и [священник] представит прежде ту \bibemph{из сих птиц}, которая за грех, и надломит голову ее от шеи ее, но не отделит;
\vs Lev 5:9 и покропит кровью сей жертвы за грех на стену жертвенника, а остальную кровь выцедит к подножию жертвенника: это жертва за грех;
\vs Lev 5:10 а другую употребит во всесожжение по установлению; и так очистит его священник от греха его, которым он согрешил, и прощено будет ему.
\vs Lev 5:11 Если же он не в состоянии принести двух горлиц или двух молодых голубей, пусть принесет за то, что согрешил, десятую часть ефы пшеничной муки в жертву за грех; пусть не льет на нее елея, и ливана пусть не кладет на нее, ибо это жертва за грех;
\vs Lev 5:12 и принесет ее к священнику, а священник возьмет из нее полную горсть в память и сожжет на жертвеннике в жертву Господу: это жертва за грех;
\vs Lev 5:13 и так очистит его священник от греха его, которым он согрешил в котором-нибудь из оных \bibemph{случаев}, и прощено будет ему; [остаток] же принадлежит священнику, как приношение хлебное.
\rsbpar\vs Lev 5:14 И сказал Господь Моисею, говоря:
\vs Lev 5:15 если кто сделает преступление и по ошибке согрешит против посвященного Господу, пусть за вину свою принесет Господу из стада овец овна без порока, по твоей оценке, серебряными сиклями по сиклю священному, в жертву повинности;
\vs Lev 5:16 за ту святыню, против которой он согрешил, пусть воздаст и прибавит к тому пятую долю, и отдаст сие священнику, и священник очистит его овном жертвы повинности, и прощено будет ему.
\vs Lev 5:17 Если кто согрешит и сделает что-нибудь против заповедей Господних, чего не надлежало делать, и по неведению сделается виновным и понесет на себе грех,
\vs Lev 5:18 пусть принесет к священнику в жертву повинности овна без порока, по оценке твоей, и загладит священник проступок его, в чем он преступил по неведению, и прощено будет ему.
\vs Lev 5:19 Это жертва повинности, \bibemph{которою} он провинился пред Господом.
\vs Lev 6:1 И сказал Господь Моисею, говоря:
\vs Lev 6:2 если кто согрешит и сделает преступление пред Господом и запрется пред ближним своим в том, что ему поручено, или у него положено, или им похищено, или обманет ближнего своего,
\vs Lev 6:3 или найдет потерянное и запрется в том, и поклянется ложно в чем-нибудь, что люди делают и тем грешат,~---
\vs Lev 6:4 то, согрешив и сделавшись виновным, он должен возвратить похищенное, что похитил, или отнятое, что отнял, или порученное, что ему поручено, или потерянное, что он нашел;
\vs Lev 6:5 или если он в чем поклялся ложно, то должен отдать сполна, и приложить к тому пятую долю и отдать тому, кому принадлежит, в день приношения жертвы повинности;
\vs Lev 6:6 и за вину свою пусть принесет Господу к священнику в жертву повинности из стада овец овна без порока, по оценке твоей;
\vs Lev 6:7 и очистит его священник пред Господом, и прощено будет ему, что бы он ни сделал, все, в чем он сделался виновным.
\rsbpar\vs Lev 6:8 И сказал Господь Моисею, говоря:
\vs Lev 6:9 заповедай Аарону и сынам его: вот закон всесожжения: всесожжение пусть остается на месте сожигания на жертвеннике всю ночь до утра, и огонь жертвенника пусть горит на нем [и не угасает];
\vs Lev 6:10 и пусть священник оденется в льняную одежду свою, и наденет на тело свое льняное нижнее платье, и снимет пепел от всесожжения, которое сжег огонь на жертвеннике, и положит его подле жертвенника;
\vs Lev 6:11 и пусть снимет с себя одежды свои, и наденет другие одежды, и вынесет пепел вне стана на чистое место;
\vs Lev 6:12 а огонь на жертвеннике пусть горит [и] не угасает; и пусть священник зажигает на нем дрова каждое утро, и раскладывает на нем всесожжение, и сожигает на нем тук мирной жертвы;
\vs Lev 6:13 огонь непрестанно пусть горит на жертвеннике \bibemph{и} не угасает.
\rsbpar\vs Lev 6:14 Вот закон о приношении хлебном: [священники] сыны Аароновы должны приносить его пред Господа к жертвеннику;
\vs Lev 6:15 и пусть возьмет [священник] горстью своею из приношения хлебного и пшеничной муки и елея и весь ливан, который на жертве, и сожжет на жертвеннике: \bibemph{это} приятное благоухание, в память пред Господом;
\vs Lev 6:16 а остальное из него пусть едят Аарон и сыны его; пресным должно есть его на святом месте, на дворе скинии собрания пусть едят его;
\vs Lev 6:17 не должно печь его квасным. Сие даю Я им в долю из жертв Моих. Это великая святыня, подобно как жертва за грех и жертва повинности.
\vs Lev 6:18 Все потомки Аароновы мужеского пола могут есть ее. Это вечный участок в роды ваши из жертв Господних. Все, прикасающееся к ним, освятится.
\rsbpar\vs Lev 6:19 И сказал Господь Моисею, говоря:
\vs Lev 6:20 вот приношение от Аарона и сынов его, которое принесут они Господу в день помазания его: десятая часть ефы пшеничной муки в жертву постоянную, половина сего для утра и половина для вечера;
\vs Lev 6:21 на сковороде в елее она должна быть приготовлена; напитанную \bibemph{елеем} приноси ее в кусках, как разламывается в куски приношение хлебное; приноси ее в приятное благоухание Господу;
\vs Lev 6:22 и священник, помазанный на место его из сынов его, должен совершать сие: это вечный устав Господа. Вся она должна быть сожжена;
\vs Lev 6:23 и всякое хлебное приношение от священника все да будет сожигаемо, а не съедаемо.
\rsbpar\vs Lev 6:24 И сказал Господь Моисею, говоря:
\vs Lev 6:25 скажи Аарону и сынам его: вот закон о жертве за грех: жертва за грех должна быть заколаема пред Господом на том месте, где заколается всесожжение; это великая святыня;
\vs Lev 6:26 священник, совершающий жертву за грех, должен есть ее; она должна быть съедаема на святом месте, на дворе скинии собрания;
\vs Lev 6:27 все, что прикоснется к мясу ее, освятится; и если кровью ее обрызгана будет одежда, то обрызганное омой на святом месте;
\vs Lev 6:28 глиняный сосуд, в котором она варилась, должно разбить; если же она варилась в медном сосуде, то должно его вычистить и вымыть водою;
\vs Lev 6:29 весь мужеский пол священнического рода может есть ее: это великая святыня [у Господа];
\vs Lev 6:30 а всякая жертва за грех, от которой кровь вносится в скинию собрания для очищения во святилище, не должна быть съедаема; ее должно сожигать на огне.
\vs Lev 7:1 Вот закон о жертве повинности: это великая святыня;
\vs Lev 7:2 жертву повинности должно заколать на том месте, где заколается всесожжение, и кровью ее кропить на жертвенник со всех сторон;
\vs Lev 7:3 \bibemph{приносящий} должен представить из нее весь тук, курдюк и тук, покрывающий внутренности,
\vs Lev 7:4 и обе почки и тук, который на них, который на стегнах, и сальник, который на печени; с почками пусть он отделит сие;
\vs Lev 7:5 и сожжет сие священник на жертвеннике в жертву Господу: это жертва повинности.
\vs Lev 7:6 Весь мужеский пол священнического рода может есть ее; на святом месте должно есть ее: это великая святыня.
\vs Lev 7:7 Как о жертве за грех, так и о жертве повинности закон один: она принадлежит священнику, который очищает посредством ее.
\vs Lev 7:8 И когда священник приносит чью-нибудь жертву всесожжения, кожа от \bibemph{жертвы} всесожжения, которое он приносит, принадлежит священнику;
\vs Lev 7:9 и всякое приношение хлебное, которое печено в печи, и всякое приготовленное в горшке или на сковороде, принадлежит священнику, приносящему его;
\vs Lev 7:10 и всякое приношение хлебное, смешанное с елеем и сухое, принадлежит всем сынам Аароновым, как одному, так и другому.
\rsbpar\vs Lev 7:11 Вот закон о жертве мирной, которую приносят Господу:
\vs Lev 7:12 если кто в благодарность приносит ее, то при жертве благодарности он должен принести пресные хлебы, смешанные с елеем, и пресные лепешки, помазанные елеем, и пшеничную муку, напитанную \bibemph{елеем}, хлебы, смешанные с елеем;
\vs Lev 7:13 кроме лепешек пусть он приносит в приношение свое квасный хлеб, при мирной жертве благодарной;
\vs Lev 7:14 одно что-нибудь из всего приношения своего пусть принесет он в возношение Господу: это принадлежит священнику, кропящему кровью мирной жертвы;
\vs Lev 7:15 мясо мирной жертвы благодарности должно съесть в день приношения ее, не должно оставлять от него до утра.
\vs Lev 7:16 Если же кто приносит жертву по обету, или от усердия, то жертву его должно есть в день приношения, и на другой день оставшееся от нее есть можно,
\vs Lev 7:17 а оставшееся от жертвенного мяса к третьему дню должно сжечь на огне;
\vs Lev 7:18 если же будут есть мясо мирной жертвы на третий день, то она не будет благоприятна; кто ее принесет, тому ни во что не вменится: это осквернение, и кто будет есть ее, тот понесет на себе грех;
\vs Lev 7:19 мяса сего, если оно прикоснется к чему-либо нечистому, не должно есть, но должно сжечь его на огне; а мясо чистое может есть всякий чистый;
\vs Lev 7:20 если же какая душа, имея на себе нечистоту, будет есть мясо мирной жертвы Господней, то истребится душа та из народа своего;
\vs Lev 7:21 и если какая душа, прикоснувшись к чему-нибудь нечистому, к нечистоте человеческой, или к нечистому скоту, или какому-нибудь нечистому гаду, будет есть мясо мирной жертвы Господней, то истребится душа та из народа своего.
\rsbpar\vs Lev 7:22 И сказал Господь Моисею, говоря:
\vs Lev 7:23 скажи сынам Израилевым: никакого тука ни из вола, ни из овцы, ни из козла не ешьте.
\vs Lev 7:24 Тук из мертвого и тук из растерзанного зверем можно употреблять на всякое дело; а есть не ешьте его;
\vs Lev 7:25 ибо, кто будет есть тук из скота, который приносится в жертву Господу, истребится душа та из народа своего;
\vs Lev 7:26 и никакой крови не ешьте во всех жилищах ваших ни из птиц, ни из скота;
\vs Lev 7:27 а кто будет есть какую-нибудь кровь, истребится душа та из народа своего.
\rsbpar\vs Lev 7:28 И сказал Господь Моисею, говоря:
\vs Lev 7:29 скажи сынам Израилевым: кто представляет мирную жертву свою Господу, тот из мирной жертвы часть должен принести в приношение Господу;
\vs Lev 7:30 своими руками должен он принести в жертву Господу: тук с грудью должен он принести [и сальник на печени], потрясая грудь пред лицем Господним;
\vs Lev 7:31 тук сожжет священник на жертвеннике, а грудь принадлежит Аарону и сынам его;
\vs Lev 7:32 и правое плечо, как возношение, из мирных жертв ваших отдавайте священнику:
\vs Lev 7:33 кто из сынов Аароновых приносит кровь из мирной жертвы и тук, тому и правое плечо на долю;
\vs Lev 7:34 ибо Я беру от сынов Израилевых из мирных жертв их грудь потрясания и плечо возношения, и отдаю их Аарону священнику и сынам его в вечный участок от сынов Израилевых.
\vs Lev 7:35 Вот участок Аарону и участок сынам его из жертв Господних со дня, когда они предстанут пред Господа для священнодействия,
\vs Lev 7:36 который повелел Господь давать им со дня помазания их от сынов Израилевых. \bibemph{Это} вечное постановление в роды их.~---
\vs Lev 7:37 Вот закон о всесожжении, о приношении хлебном, о жертве за грех, о жертве повинности, о жертве посвящения и о жертве мирной,
\vs Lev 7:38 который дал Господь Моисею на горе Синае, когда повелел сынам Израилевым, в пустыне Синайской, приносить Господу приношения их.
\vs Lev 8:1 И сказал Господь Моисею, говоря:
\vs Lev 8:2 возьми Аарона и сынов его с ним, и одежды и елей помазания, и тельца для жертвы за грех и двух овнов, и корзину опресноков,
\vs Lev 8:3 и собери все общество ко входу скинии собрания.
\rsbpar\vs Lev 8:4 Моисей сделал так, как повелел ему Господь, и собралось общество ко входу скинии собрания.
\vs Lev 8:5 И сказал Моисей к обществу: вот что повелел Господь сделать.
\vs Lev 8:6 И привел Моисей Аарона и сынов его и омыл их водою;
\vs Lev 8:7 и возложил на него хитон, и опоясал его поясом, и надел на него верхнюю ризу, и возложил на него ефод, и опоясал его поясом ефода и прикрепил им ефод на нем,
\vs Lev 8:8 и возложил на него наперсник, и на наперсник положил урим и туммим,
\vs Lev 8:9 и возложил на голову его кидар, а на кидар с передней стороны его возложил полированную дощечку, диадиму святыни, как повелел Господь Моисею.
\vs Lev 8:10 И взял Моисей елей помазания, и помазал скинию и все, что в ней, и освятил это;
\vs Lev 8:11 и покропил им на жертвенник семь раз, и помазал жертвенник и все принадлежности его и умывальницу и подножие ее, чтобы освятить их;
\vs Lev 8:12 и возлил [Моисей] елей помазания на голову Аарона и помазал его, чтоб освятить его.
\vs Lev 8:13 И привел Моисей сынов Аароновых, и одел их в хитоны, и опоясал их поясом, и возложил на них кидары, как повелел Господь Моисею.
\rsbpar\vs Lev 8:14 И привел [Моисей] тельца для жертвы за грех, и Аарон и сыны его возложили руки свои на голову тельца за грех;
\vs Lev 8:15 и заколол \bibemph{его} [Моисей] и взял крови, и перстом своим возложил на роги жертвенника со всех сторон, и очистил жертвенник, а \bibemph{остальную} кровь вылил к подножию жертвенника, и освятил его, чтобы сделать его чистым.
\vs Lev 8:16 И взял [Моисей] весь тук, который на внутренностях, и сальник на печени, и обе почки и тук их, и сжег Моисей на жертвеннике;
\vs Lev 8:17 а тельца и кожу его, и мясо его, и нечистоту его сжег на огне вне стана, как повелел Господь Моисею.
\vs Lev 8:18 И привел [Моисей] овна для всесожжения, и возложили Аарон и сыны его руки свои на голову овна;
\vs Lev 8:19 и заколол \bibemph{его} Моисей и покропил кровью на жертвенник со всех сторон;
\vs Lev 8:20 и рассек овна на части, и сжег Моисей голову и части и тук,
\vs Lev 8:21 а внутренности и ноги вымыл водою, и сжег Моисей всего овна на жертвеннике: это всесожжение в приятное благоухание, это жертва Господу, как повелел Господь Моисею.
\vs Lev 8:22 И привел [Моисей] другого овна, овна посвящения, и возложили Аарон и сыны его руки свои на голову овна;
\vs Lev 8:23 и заколол \bibemph{его} Моисей, и взял крови его, и возложил на край правого уха Ааронова и на большой палец правой руки его и на большой палец правой ноги его.
\vs Lev 8:24 И привел Моисей сынов Аароновых, и возложил крови на край правого уха их и на большой палец правой руки их и на большой палец правой ноги их, и покропил Моисей кровью на жертвенник со всех сторон.
\vs Lev 8:25 И взял [Моисей] тук и курдюк и весь тук, который на внутренностях, и сальник на печени, и обе почки и тук их и правое плечо;
\vs Lev 8:26 и из корзины с опресноками, которая пред Господом, взял один опреснок и один хлеб с елеем и одну лепешку, и возложил на тук и на правое плечо;
\vs Lev 8:27 и положил все это на руки Аарону и на руки сынам его, и принес это, потрясая пред лицем Господним;
\vs Lev 8:28 и взял это Моисей с рук их и сжег на жертвеннике со всесожжением: это жертва посвящения в приятное благоухание, это жертва Господу.
\vs Lev 8:29 И взял Моисей грудь и принес ее, потрясая пред лицем Господним: это была доля Моисеева от овна посвящения, как повелел Господь Моисею.
\vs Lev 8:30 И взял Моисей елея помазания и крови, которая на жертвеннике, и покропил Аарона и одежды его, и сынов его и одежды сынов его с ним; и так освятил Аарона и одежды его, и сынов его и одежды сынов его с ним.
\rsbpar\vs Lev 8:31 И сказал Моисей Аарону и сынам его: сварите мясо у входа скинии собрания и там ешьте его с хлебом, который в корзине посвящения, как мне повелено и сказано: Аарон и сыны его должны есть его;
\vs Lev 8:32 а остатки мяса и хлеба сожгите на огне.
\vs Lev 8:33 Семь дней не отходите от дверей скинии собрания, пока не исполнятся дни посвящения вашего, ибо семь дней должно совершаться посвящение ваше;
\vs Lev 8:34 как сегодня было сделано, так повелел Господь делать для очищения вас;
\vs Lev 8:35 у входа скинии собрания будьте день и ночь в продолжение семи дней и будьте на страже у Господа, чтобы не умереть, ибо так мне повелено [от Господа Бога].
\vs Lev 8:36 И исполнил Аарон и сыны его все, что повелел Господь чрез Моисея.
\vs Lev 9:1 В восьмой день призвал Моисей Аарона и сынов его и старейшин Израилевых
\vs Lev 9:2 и сказал Аарону: возьми себе из волов тельца в жертву за грех и овна во всесожжение, обоих без порока, и представь пред лице Господне;
\vs Lev 9:3 и сынам Израилевым скажи: возьмите козла в жертву за грех, [и овна,] и тельца, и агнца, однолетних, без порока, во всесожжение,
\vs Lev 9:4 и вола и овна в жертву мирную, чтобы совершить жертвоприношение пред лицем Господним, и приношение хлебное, смешанное с елеем, ибо сегодня Господь явится вам.
\vs Lev 9:5 И принесли то, что приказал Моисей, пред скинию собрания, и пришло все общество и стало пред лицем Господним.
\vs Lev 9:6 И сказал Моисей: вот что повелел Господь сделать, и явится вам слава Господня.
\vs Lev 9:7 И сказал Моисей Аарону: приступи к жертвеннику и соверши жертву твою о грехе и всесожжение твое, и очисти себя и народ, и сделай приношение от народа, и очисти их, как повелел Господь.
\rsbpar\vs Lev 9:8 И приступил Аарон к жертвеннику и заколол тельца, который за него, в жертву за грех:
\vs Lev 9:9 сыны Аарона поднесли ему кровь, и он омочил перст свой в крови и возложил на роги жертвенника, а \bibemph{остальную} кровь вылил к подножию жертвенника,
\vs Lev 9:10 а тук и почки и сальник на печени от жертвы за грех сжег на жертвеннике, как повелел Господь Моисею;
\vs Lev 9:11 мясо же и кожу сжег на огне вне стана.
\vs Lev 9:12 И заколол всесожжение, и сыны Аарона поднесли ему кровь; он покропил ею на жертвенник со всех сторон;
\vs Lev 9:13 и принесли ему всесожжение в кусках и голову, и он сжег на жертвеннике,
\vs Lev 9:14 а внутренности и ноги омыл и сжег со всесожжением на жертвеннике.
\vs Lev 9:15 И принес приношение от народа, и взял от народа козла за грех, и заколол его, и принес его в жертву за грех, как и прежнего.
\vs Lev 9:16 И принес всесожжение и совершил его по уставу.
\vs Lev 9:17 И принес приношение хлебное, и наполнил им руки свои, и сжег на жертвеннике сверх утреннего всесожжения.
\vs Lev 9:18 И заколол вола и овна, которые от народа, в жертву мирную; и сыны Аарона поднесли ему кровь, и он покропил ею на жертвенник со всех сторон;
\vs Lev 9:19 \bibemph{поднесли} и тук из вола, и из овна курдюк, и [тук] покрывающий [внутренности], почки и сальник на печени,
\vs Lev 9:20 и положили тук на грудь, и он сжег тук на жертвеннике;
\vs Lev 9:21 грудь же и правое плечо принес Аарон, потрясая пред лицем Господним, как повелел Моисей.
\vs Lev 9:22 И поднял Аарон руки свои, \bibemph{обратившись} к народу, и благословил его, и сошел, совершив жертву за грех, всесожжение и жертву мирную.
\vs Lev 9:23 И вошли Моисей и Аарон в скинию собрания, и вышли, и благословили народ. И явилась слава Господня всему народу:
\vs Lev 9:24 и вышел огонь от Господа и сжег на жертвеннике всесожжение и тук; и видел весь народ, и воскликнул от радости, и пал на лице свое.
\vs Lev 10:1 Надав и Авиуд, сыны Аароновы, взяли каждый свою кадильницу, и положили в них огня, и вложили в него курений, и принесли пред Господа огонь чуждый, которого Он не велел им;
\vs Lev 10:2 и вышел огонь от Господа и сжег их, и умерли они пред лицем Господним.
\vs Lev 10:3 И сказал Моисей Аарону: вот о чем говорил Господь, когда сказал: в приближающихся ко Мне освящусь и пред всем народом прославлюсь. Аарон молчал.
\vs Lev 10:4 И позвал Моисей Мисаила и Елцафана, сынов Узиила, дяди Ааронова, и сказал им: пойдите, вынесите братьев ваших из святилища за стан.
\vs Lev 10:5 И пошли и вынесли их в хитонах их за стан, как сказал Моисей.
\vs Lev 10:6 Аарону же и Елеазару и Ифамару, сынам его, Моисей сказал: голов ваших не обнажайте и одежд ваших не раздирайте, чтобы вам не умереть и не навести гнева на все общество; но братья ваши, весь дом Израилев, могут плакать о сожженных, которых сожег Господь,
\vs Lev 10:7 и из дверей скинии собрания не выходите, чтобы не умереть вам, ибо на вас елей помазания Господня. И сделали по слову Моисея.
\rsbpar\vs Lev 10:8 И сказал Господь Аарону, говоря:
\vs Lev 10:9 вина и крепких напитков не пей ты и сыны твои с тобою, когда входите в скинию собрания, [или приступаете к жертвеннику,] чтобы не умереть. \bibemph{Это} вечное постановление в роды ваши,
\vs Lev 10:10 чтобы вы могли отличать священное от несвященного и нечистое от чистого,
\vs Lev 10:11 и научать сынов Израилевых всем уставам, которые изрек им Господь чрез Моисея.
\rsbpar\vs Lev 10:12 И сказал Моисей Аарону и Елеазару и Ифамару, оставшимся сынам его: возьмите приношение хлебное, оставшееся от жертв Господних, и ешьте его пресное у жертвенника, ибо это великая святыня;
\vs Lev 10:13 и ешьте его на святом месте, ибо это участок твой и участок сынов твоих из жертв Господних: так мне повелено [от Господа];
\vs Lev 10:14 и грудь потрясания и плечо возношения ешьте на чистом месте, ты и сыновья твои и дочери твои с тобою, ибо это дано в участок тебе и в участок сынам твоим из мирных жертв сынов Израилевых;
\vs Lev 10:15 плечо возношения и грудь потрясания должны они приносить с жертвами тука, потрясая пред лицем Господним, и да будет это вечным участком тебе и сыновьям твоим [и дочерям твоим] с тобою, как повелел Господь [Моисею].
\vs Lev 10:16 И козла жертвы за грех искал Моисей, и вот, он сожжен. И разгневался [Моисей] на Елеазара и Ифамара, оставшихся сынов Аароновых, и сказал:
\vs Lev 10:17 почему вы не ели жертвы за грех на святом месте? ибо она святыня великая, и она дана вам, чтобы снимать грехи с общества и очищать их пред Господом;
\vs Lev 10:18 вот, кровь ее не внесена внутрь святилища, а вы должны были есть ее на святом месте, как повелено мне.
\vs Lev 10:19 Аарон сказал Моисею: вот, сегодня принесли они жертву свою за грех и всесожжение свое пред Господом, и это случилось со мною; если я сегодня съем жертву за грех, будет ли это угодно Господу?
\vs Lev 10:20 И услышал Моисей и одобрил.
\vs Lev 11:1 И сказал Господь Моисею и Аарону, говоря им:
\vs Lev 11:2 скажите сынам Израилевым: вот животные, которые можно вам есть из всего скота на земле:
\vs Lev 11:3 всякий скот, у которого раздвоены копыта и на копытах глубокий разрез, и который жует жвачку, ешьте;
\vs Lev 11:4 только сих не ешьте из жующих жвачку и имеющих раздвоенные копыта: верблюда, потому что он жует жвачку, но копыта у него не раздвоены, нечист он для вас;
\vs Lev 11:5 и тушканчика, потому что он жует жвачку, но копыта у него не раздвоены, нечист он для вас,
\vs Lev 11:6 и зайца, потому что он жует жвачку, но копыта у него не раздвоены, нечист он для вас;
\vs Lev 11:7 и свиньи, потому что копыта у нее раздвоены и на копытах разрез глубокий, но она не жует жвачки, нечиста она для вас;
\vs Lev 11:8 мяса их не ешьте и к трупам их не прикасайтесь; нечисты они для вас.
\vs Lev 11:9 Из всех \bibemph{животных}, которые в воде, ешьте сих: у которых есть перья и чешуя в воде, в морях ли, или реках, тех ешьте;
\vs Lev 11:10 а все те, у которых нет перьев и чешуи, в морях ли, или реках, из всех плавающих в водах и из всего живущего в водах, скверны для вас;
\vs Lev 11:11 они должны быть скверны для вас: мяса их не ешьте и трупов их гнушайтесь;
\vs Lev 11:12 все \bibemph{животные}, у которых нет перьев и чешуи в воде, скверны для вас.
\vs Lev 11:13 Из птиц же гнушайтесь сих [не должно их есть, скверны они]: орла, грифа и морского орла,
\vs Lev 11:14 коршуна и сокола с породою его,
\vs Lev 11:15 всякого ворона с породою его,
\vs Lev 11:16 страуса, совы, чайки и ястреба с породою его,
\vs Lev 11:17 филина, рыболова и ибиса,
\vs Lev 11:18 лебедя, пеликана и сипа,
\vs Lev 11:19 цапли, зуя с породою его, удода и нетопыря.
\vs Lev 11:20 Все \bibemph{животные} пресмыкающиеся, крылатые, ходящие на четырех \bibemph{ногах}, скверны для вас;
\vs Lev 11:21 из всех пресмыкающихся, крылатых, ходящих на четырех \bibemph{ногах}, тех только ешьте, у которых есть голени выше ног, чтобы скакать ими по земле;
\vs Lev 11:22 сих ешьте из них: саранчу с ее породою, солам с ее породою, харгол с ее породою и хагаб с ее породою.
\vs Lev 11:23 Всякое \bibemph{другое} пресмыкающееся, крылатое, у которого четыре ноги, скверно для вас;
\vs Lev 11:24 от них вы будете нечисты: всякий, кто прикоснется к трупу их, нечист будет до вечера;
\vs Lev 11:25 и всякий, кто возьмет труп их, должен омыть одежду свою и нечист будет до вечера.
\vs Lev 11:26 Всякий скот, у которого копыта раздвоены, но нет глубокого разреза, и который не жует жвачки, нечист для вас: всякий, кто прикоснется к нему, будет нечист [до вечера].
\vs Lev 11:27 Из всех зверей четвероногих те, которые ходят на лапах, нечисты для вас: всякий, кто прикоснется к трупу их, нечист будет до вечера;
\vs Lev 11:28 кто возьмет труп их, тот должен омыть одежды свои и нечист будет до вечера: нечисты они для вас.
\rsbpar\vs Lev 11:29 Вот что нечисто для вас из животных, пресмыкающихся по земле: крот, мышь, ящерица с ее породою,
\vs Lev 11:30 анака, хамелеон, летаа, хомет и тиншемет,~---
\vs Lev 11:31 сии нечисты для вас из всех пресмыкающихся: всякий, кто прикоснется к ним мертвым, нечист будет до вечера.
\vs Lev 11:32 И всё, на что упадет которое-нибудь из них мертвое, всякий деревянный сосуд, или одежда, или кожа, или мешок, и всякая вещь, которая употребляется на дело, будут нечисты: в воду должно положить их, и нечисты будут до вечера, потом будут чисты;
\vs Lev 11:33 если же которое-нибудь из них упадет в какой-нибудь глиняный сосуд, то находящееся в нем будет нечисто, и самый [сосуд] разбейте.
\vs Lev 11:34 Всякая пища, которую едят, на которой была вода \bibemph{из такого сосуда}, нечиста будет [для вас], и всякое питье, которое пьют, во всяком \bibemph{таком} сосуде нечисто будет.
\vs Lev 11:35 Всё, на что упадет что-нибудь от трупа их, нечисто будет: печь и очаг должно разломать, они нечисты; и они должны быть нечисты для вас;
\vs Lev 11:36 только источник и колодезь, вмещающий воду, остаются чистыми; а кто прикоснется к трупу их, тот нечист.
\vs Lev 11:37 И если что-нибудь от трупа их упадет на какое-либо семя, которое сеют, то оно чисто;
\vs Lev 11:38 если же тогда, как вода налита на семя, упадет на него что-нибудь от трупа их, то оно нечисто для вас.
\vs Lev 11:39 И когда умрет какой-либо скот, который употребляется вами в пищу, то прикоснувшийся к трупу его нечист будет до вечера;
\vs Lev 11:40 и тот, кто будет есть мертвечину его, должен омыть одежды свои и нечист будет до вечера; и тот, кто понесет труп его, должен омыть одежды свои и нечист будет до вечера.
\vs Lev 11:41 Всякое животное, пресмыкающееся по земле, скверно для вас, не должно есть \bibemph{его};
\vs Lev 11:42 всего ползающего на чреве и всего ходящего на четырех ногах, и многоножных из животных пресмыкающихся по земле, не ешьте, ибо они скверны;
\vs Lev 11:43 не оскверняйте душ ваших каким-либо животным пресмыкающимся и не делайте себя чрез них нечистыми, чтоб быть чрез них нечистыми,
\vs Lev 11:44 ибо Я~--- Господь Бог ваш: освящайтесь и будьте святы, ибо Я [Господь, Бог ваш] свят; и не оскверняйте душ ваших каким-либо животным, ползающим по земле,
\vs Lev 11:45 ибо Я~--- Господь, выведший вас из земли Египетской, чтобы быть вашим Богом. Итак будьте святы, потому что Я свят.
\rsbpar\vs Lev 11:46 Вот закон о скоте, о птицах, о всех животных, живущих в водах, и о всех животных, пресмыкающихся по земле,
\vs Lev 11:47 чтобы отличать нечистое от чистого, и животных, которых можно есть, от животных, которых есть не должно.
\vs Lev 12:1 И сказал Господь Моисею, говоря:
\vs Lev 12:2 скажи сынам Израилевым: если женщина зачнет и родит \bibemph{младенца} мужеского пола, то она нечиста будет семь дней; как во дни страдания ее очищением, она будет нечиста;
\vs Lev 12:3 в восьмой же день обрежется у него крайняя плоть его;
\vs Lev 12:4 и тридцать три дня должна она сидеть, очищаясь от кровей своих; ни к чему священному не должна прикасаться и к святилищу не должна приходить, пока не исполнятся дни очищения ее.
\vs Lev 12:5 Если же она родит \bibemph{младенца} женского пола, то во время очищения своего она будет нечиста две недели, и шестьдесят шесть дней должна сидеть, очищаясь от кровей своих.
\vs Lev 12:6 По окончании дней очищения своего за сына или за дочь она должна принести однолетнего агнца во всесожжение и молодого голубя или горлицу в жертву за грех, ко входу скинии собрания к священнику;
\vs Lev 12:7 он принесет это пред Господа и очистит ее, и она будет чиста от течения кровей ее. Вот закон о родившей \bibemph{младенца} мужеского или женского пола.
\vs Lev 12:8 Если же она не в состоянии принести агнца, то пусть возьмет двух горлиц или двух молодых голубей, одного во всесожжение, а другого в жертву за грех, и очистит ее священник, и она будет чиста.
\vs Lev 13:1 И сказал Господь Моисею и Аарону, говоря:
\vs Lev 13:2 когда у кого появится на коже тела его опухоль, или лишаи, или пятно, и на коже тела его сделается как бы язва проказы, то должно привести его к Аарону священнику, или к одному из сынов его, священников;
\vs Lev 13:3 священник осмотрит язву на коже тела, и если волосы на язве изменились в белые, и язва оказывается углубленною в кожу тела его, то это язва проказы; священник, осмотрев его, объявит его нечистым.
\vs Lev 13:4 А если на коже тела его пятно белое, но оно не окажется углубленным в кожу, и волосы на нем не изменились в белые, то священник \bibemph{имеющего} язву должен заключить на семь дней;
\vs Lev 13:5 в седьмой день священник осмотрит его, и если язва остается в своем виде и не распространяется язва по коже, то священник должен заключить его на другие семь дней;
\vs Lev 13:6 в седьмой день опять священник осмотрит его, и если язва менее приметна и не распространилась язва по коже, то священник должен объявить его чистым: это лишаи, и пусть он омоет одежды свои, и будет чист.
\vs Lev 13:7 Если же лишаи станут распространяться по коже, после того как он являлся к священнику для очищения, то он вторично должен явиться к священнику;
\vs Lev 13:8 священник, увидев, что лишаи распространяются по коже, объявит его нечистым: это проказа.
\rsbpar\vs Lev 13:9 Если будет на ком язва проказы, то должно привести его к священнику;
\vs Lev 13:10 священник осмотрит, и если опухоль на коже бела, и волос изменился в белый, и на опухоли живое мясо,
\vs Lev 13:11 то это застарелая проказа на коже тела его; и священник объявит его нечистым и заключит его, ибо он нечист.
\vs Lev 13:12 Если же проказа расцветет на коже, и покроет проказа всю кожу больного от головы его до ног, сколько могут видеть глаза священника,
\vs Lev 13:13 и увидит священник, что проказа покрыла все тело его, то он объявит больного чистым, потому что все превратилось в белое: он чист.
\vs Lev 13:14 Когда же окажется на нем живое мясо, то он нечист;
\vs Lev 13:15 священник, увидев живое мясо, объявит его нечистым; живое мясо нечисто: это проказа.
\vs Lev 13:16 Если же живое мясо изменится и обратится в белое, пусть он придет к священнику;
\vs Lev 13:17 священник осмотрит его, и если язва обратилась в белое, священник объявит больного чистым; он чист.
\rsbpar\vs Lev 13:18 Если у кого на коже тела был нарыв и зажил,
\vs Lev 13:19 и на месте нарыва появилась белая опухоль, или пятно белое или красноватое, то он должен явиться к священнику;
\vs Lev 13:20 священник осмотрит его, и если оно окажется ниже кожи, и волос его изменился в белый, то священник объявит его нечистым: это язва проказы, она расцвела на нарыве;
\vs Lev 13:21 если же священник увидит, что волос на ней не бел, и она не ниже кожи, и притом мало приметна, то священник заключит его на семь дней;
\vs Lev 13:22 если она станет очень распространяться по коже, то священник объявит его нечистым: это язва;
\vs Lev 13:23 если же пятно остается на своем месте и не распространяется, то это воспаление нарыва, и священник объявит его чистым.
\vs Lev 13:24 Или если у кого на коже тела будет ожог, и на зажившем ожоге окажется красноватое или белое пятно,
\vs Lev 13:25 и священник увидит, что волос на пятне изменился в белый, и оно окажется углубленным в коже, то это проказа, она расцвела на ожоге; и священник объявит его нечистым: это язва проказы;
\vs Lev 13:26 если же священник увидит, что волос на пятне не бел, и оно не ниже кожи, и притом мало приметно, то священник заключит его на семь дней;
\vs Lev 13:27 в седьмой день священник осмотрит его, и если оно очень распространяется по коже, то священник объявит его нечистым: это язва проказы;
\vs Lev 13:28 если же пятно остается на своем месте и не распространяется по коже, и притом мало приметно, то это опухоль от ожога; священник объявит его чистым, ибо это воспаление от ожога.
\rsbpar\vs Lev 13:29 Если у мужчины или у женщины будет язва на голове или на бороде,
\vs Lev 13:30 и осмотрит священник язву, и она окажется углубленною в коже, и волос на ней желтоватый тонкий, то священник объявит их нечистыми: это паршивость, это проказа на голове или на бороде;
\vs Lev 13:31 если же священник осмотрит язву паршивости и она не окажется углубленною в коже, и волос на ней не черный, то священник \bibemph{имеющего} язву паршивости заключит на семь дней;
\vs Lev 13:32 в седьмой день священник осмотрит язву, и если паршивость не распространяется, и нет на ней желтоватого волоса, и паршивость не окажется углубленною в коже,
\vs Lev 13:33 то \bibemph{больного} должно остричь, но паршивого места не остригать, и священник должен паршивого вторично заключить на семь дней;
\vs Lev 13:34 в седьмой день священник осмотрит паршивость, и если паршивость не распространяется по коже и не окажется углубленною в коже, то священник объявит его чистым; пусть он омоет одежды свои, и будет чист.
\vs Lev 13:35 Если же после очищения его будет очень распространяться паршивость по коже,
\vs Lev 13:36 и священник увидит, что паршивость распространяется по коже, то священник пусть не ищет желтоватого волоса: он нечист.
\vs Lev 13:37 Если же паршивость остается в своем виде, и показывается на ней волос черный, то паршивость прошла, он чист; священник объявит его чистым.
\vs Lev 13:38 Если у мужчины или у женщины на коже тела их будут пятна, пятна белые,
\vs Lev 13:39 и священник увидит, что на коже тела их пятна бледно-белые, то это лишай, расцветший на коже: он чист.
\rsbpar\vs Lev 13:40 Если у кого на голове вылезли \bibemph{волосы}, то это плешивый: он чист;
\vs Lev 13:41 а если на передней стороне головы вылезли \bibemph{волосы}, то это лысый: он чист.
\vs Lev 13:42 Если же на плеши или на лысине будет белое или красноватое пятно, то на плеши его или на лысине его расцвела проказа;
\vs Lev 13:43 священник осмотрит его, и если увидит, что опухоль язвы бела \bibemph{или} красновата на плеши его или на лысине его, видом похожа на проказу кожи тела,
\vs Lev 13:44 то он прокаженный, нечист он; священник должен объявить его нечистым, у него на голове язва.
\vs Lev 13:45 У прокаженного, на котором эта язва, должна быть разодрана одежда, и голова его должна быть не покрыта, и до уст он должен быть закрыт и кричать: нечист! нечист!
\vs Lev 13:46 Во все дни, доколе на нем язва, он должен быть нечист, нечист он; он должен жить отдельно, вне стана жилище его.
\rsbpar\vs Lev 13:47 Если язва проказы будет на одежде, на одежде шерстяной, или на одежде льняной,
\vs Lev 13:48 или на основе, или на утоке из льна или шерсти, или на коже, или на каком-нибудь изделии кожаном,
\vs Lev 13:49 и пятно будет зеленоватое или красноватое на одежде, или на коже, или на основе, или на утоке, или на какой-нибудь кожаной вещи,~--- то это язва проказы: должно показать ее священнику;
\vs Lev 13:50 священник осмотрит язву и заключит зараженное язвою на семь дней;
\vs Lev 13:51 в седьмой день осмотрит священник зараженное, и если язва распространилась по одежде, или по основе, или по утоку, или по коже, или по какому-либо изделию, сделанному из кожи, то это проказа едкая, язва нечистая;
\vs Lev 13:52 он должен сжечь одежду, или основу, или уток шерстяной или льняной, или какую бы то ни было кожаную вещь, на которой будет язва, ибо это проказа едкая: должно сжечь на огне.
\vs Lev 13:53 Если же священник увидит, что язва не распространилась по одежде, или по основе, или по утоку, или по какой бы то ни было кожаной вещи,
\vs Lev 13:54 то священник прикажет омыть то, на чем язва, и вторично заключит на семь дней;
\vs Lev 13:55 если по омытии зараженной \bibemph{вещи} священник увидит, что язва не изменила вида своего и не распространилась язва, то она нечиста, сожги ее на огне; это выеденная ямина на лицевой стороне или на изнанке;
\vs Lev 13:56 если же священник увидит, что язва по омытии ее сделалась менее приметна, то священник пусть оторвет ее от одежды, или от кожи, или от основы, или от утока.
\vs Lev 13:57 Если же она опять покажется на одежде, или на основе, или на утоке, или на какой-нибудь кожаной вещи, то это расцветающая язва: сожги на огне то, на чем язва.
\vs Lev 13:58 Если же одежду, или основу, или уток, или какую-нибудь кожаную вещь вымоешь, и сойдет с них язва, то должно вымыть их вторично, и они будут чисты.
\vs Lev 13:59 Вот закон о язве проказы на одежде шерстяной или льняной, или на основе и на утоке, или на какой-нибудь кожаной вещи, как объявлять ее чистою или нечистою.
\vs Lev 14:1 И сказал Господь Моисею, говоря:
\vs Lev 14:2 вот закон о прокаженном, когда надобно его очистить: приведут его к священнику;
\vs Lev 14:3 священник выйдет вон из стана, и если священник увидит, что прокаженный исцелился от болезни прокажения,
\vs Lev 14:4 то священник прикажет взять для очищаемого двух птиц живых чистых, кедрового дерева, червленую нить и иссопа,
\vs Lev 14:5 и прикажет священник заколоть одну птицу над глиняным сосудом, над живою водою;
\vs Lev 14:6 а сам он возьмет живую птицу, кедровое дерево, червленую нить и иссоп, и омочит их и живую птицу в крови птицы заколотой над живою водою,
\vs Lev 14:7 и покропит на очищаемого от проказы семь раз, и объявит его чистым, и пустит живую птицу в поле.
\vs Lev 14:8 Очищаемый омоет одежды свои, острижет все волосы свои, омоется водою, и будет чист; потом войдет в стан и пробудет семь дней вне шатра своего;
\vs Lev 14:9 в седьмой день обреет все волосы свои, голову свою, бороду свою, брови глаз своих, все волосы свои обреет, и омоет одежды свои, и омоет тело свое водою, и будет чист;
\vs Lev 14:10 в восьмой день возьмет он двух овнов [однолетних] без порока, и одну овцу однолетнюю без порока, и три десятых части ефы пшеничной муки, смешанной с елеем, в приношение хлебное, и один лог елея;
\vs Lev 14:11 священник очищающий поставит очищаемого человека с ними пред Господом у входа скинии собрания;
\vs Lev 14:12 и возьмет священник одного овна, и представит его в жертву повинности, и лог елея, и принесет это, потрясая пред Господом;
\vs Lev 14:13 и заколет овна на том месте, где заколают жертву за грех и всесожжение, на месте святом, ибо сия жертва повинности, подобно жертве за грех, принадлежит священнику: это великая святыня;
\vs Lev 14:14 и возьмет священник крови жертвы повинности, и возложит священник на край правого уха очищаемого и на большой палец правой руки его и на большой палец правой ноги его;
\vs Lev 14:15 и возьмет священник из лога елея и польет на левую свою ладонь;
\vs Lev 14:16 и омочит священник правый перст свой в елей, который на левой ладони его, и покропит елеем с перста своего семь раз пред лицем Господа;
\vs Lev 14:17 оставшийся же елей, который на ладони его, возложит священник на край правого уха очищаемого, на большой палец правой руки его и на большой палец правой ноги его, на \bibemph{места, где} кровь жертвы повинности;
\vs Lev 14:18 а остальной елей, который на ладони священника, возложит он на голову очищаемого, и очистит его священник пред лицем Господа.
\vs Lev 14:19 И совершит священник жертву за грех, и очистит очищаемого от нечистоты его; после того заколет \bibemph{жертву} всесожжения;
\vs Lev 14:20 и возложит священник всесожжение и приношение хлебное на жертвенник; и очистит его священник, и он будет чист.
\vs Lev 14:21 Если же он беден и не имеет достатка, то пусть возьмет одного овна в жертву повинности для потрясания, чтоб очистить себя, и одну десятую часть \bibemph{ефы} пшеничной муки, смешанной с елеем, в приношение хлебное, и лог елея,
\vs Lev 14:22 и двух горлиц или двух молодых голубей, что достанет рука его, одну \bibemph{из птиц} в жертву за грех, а другую во всесожжение;
\vs Lev 14:23 и принесет их в восьмой день очищения своего к священнику ко входу скинии собрания, пред лице Господа;
\vs Lev 14:24 священник возьмет овна жертвы повинности и лог елея, и принесет это священник, потрясая пред Господом;
\vs Lev 14:25 и заколет овна в жертву повинности, и возьмет священник крови жертвы повинности, и возложит на край правого уха очищаемого и на большой палец правой руки его и на большой палец правой ноги его;
\vs Lev 14:26 и нальет священник елея на левую свою ладонь,
\vs Lev 14:27 и елеем, который на левой ладони его, покропит священник с правого перста своего семь раз пред лицем Господним;
\vs Lev 14:28 и возложит священник елея, который на ладони его, на край правого уха очищаемого, на большой палец правой руки его и на большой палец правой ноги его, на места, \bibemph{где} кровь жертвы повинности;
\vs Lev 14:29 а остальной елей, который на ладони священника, возложит он на голову очищаемого, чтоб очистить его пред лицем Господа;
\vs Lev 14:30 и принесет одну из горлиц или одного из молодых голубей, что достанет рука \bibemph{очищаемого},
\vs Lev 14:31 из того, что достанет рука его, одну \bibemph{птицу} в жертву за грех, а другую во всесожжение, вместе с приношением хлебным; и очистит священник очищаемого пред лицем Господа.
\vs Lev 14:32 Вот закон о прокаженном, который во время очищения своего не имеет достатка.
\rsbpar\vs Lev 14:33 И сказал Господь Моисею и Аарону, говоря:
\vs Lev 14:34 когда войдете в землю Ханаанскую, которую Я даю вам во владение, и Я наведу язву проказы на домы в земле владения вашего,
\vs Lev 14:35 тогда тот, чей дом, должен пойти и сказать священнику: у меня на доме показалась как бы язва.
\vs Lev 14:36 Священник прикажет опорожнить дом, прежде нежели войдет священник осматривать язву, чтобы не сделалось нечистым все, что в доме; после сего придет священник осматривать дом.
\vs Lev 14:37 Если он, осмотрев язву, увидит, что язва на стенах дома состоит из зеленоватых или красноватых ямин, которые окажутся углубленными в стене,
\vs Lev 14:38 то священник выйдет из дома к дверям дома и запрет дом на семь дней.
\vs Lev 14:39 В седьмой день опять придет священник, и если увидит, что язва распространилась по стенам дома,
\vs Lev 14:40 то священник прикажет выломать камни, на которых язва, и бросить их вне города на место нечистое;
\vs Lev 14:41 а дом внутри пусть весь оскоблят, и обмазку, которую отскоблят, высыпят вне города на место нечистое;
\vs Lev 14:42 и возьмут другие камни, и вставят вместо тех камней, и возьмут другую обмазку, и обмажут дом.
\vs Lev 14:43 Если язва опять появится и будет цвести на доме после того, как выломали камни и оскоблили дом и обмазали,
\vs Lev 14:44 то священник придет и осмотрит, и если язва на доме распространилась, то это едкая проказа на доме, нечист он;
\vs Lev 14:45 должно разломать сей дом, и камни его и дерево его и всю обмазку дома вынести вне города на место нечистое;
\vs Lev 14:46 кто входит в дом во все время, когда он заперт, тот нечист до вечера;
\vs Lev 14:47 и кто спит в доме том, тот должен вымыть одежды свои [и нечист будет до вечера]; и кто ест в доме том, тот должен вымыть одежды свои [и нечист будет до вечера].
\vs Lev 14:48 Если же священник придет и увидит, что язва на доме не распространилась после того, как обмазали дом, то священник объявит дом чистым, потому что язва прошла.
\vs Lev 14:49 И чтобы очистить дом, возьмет он две птицы, кедрового дерева, червленую нить и иссопа,
\vs Lev 14:50 и заколет одну птицу над глиняным сосудом, над живою водою;
\vs Lev 14:51 и возьмет кедровое дерево и иссоп, и червленую нить и живую птицу, и омочит их в крови птицы заколотой и в живой воде, и покропит дом семь раз;
\vs Lev 14:52 и очистит дом кровью птицы и живою водою, и живою птицею и кедровым деревом, и иссопом и червленою нитью;
\vs Lev 14:53 и пустит живую птицу вне города в поле и очистит дом, и будет чист.
\rsbpar\vs Lev 14:54 Вот закон о всякой язве проказы и о паршивости,
\vs Lev 14:55 и о проказе на одежде и на доме, и об опухоли, и о лишаях, и о пятнах,~---
\vs Lev 14:56 чтобы указать, когда это нечисто и когда чисто. Вот закон о проказе.
\vs Lev 15:1 И сказал Господь Моисею и Аарону, говоря:
\vs Lev 15:2 объявите сынам Израилевым и скажите им: если у кого будет истечение из тела его, то от истечения своего он нечист.
\vs Lev 15:3 И вот [закон] о нечистоте его от истечения его: когда течет из тела его истечение его, и когда задерживается в теле его истечение его, это нечистота его;
\vs Lev 15:4 всякая постель, на которой ляжет имеющий истечение, нечиста, и всякая вещь, на которую сядет [имеющий истечение семени], нечиста;
\vs Lev 15:5 и кто прикоснется к постели его, тот должен вымыть одежды свои и омыться водою и нечист будет до вечера;
\vs Lev 15:6 кто сядет на какую-либо вещь, на которой сидел имеющий истечение, тот должен вымыть одежды свои и омыться водою и нечист будет до вечера;
\vs Lev 15:7 и кто прикоснется к телу имеющего истечение, тот должен вымыть одежды свои и омыться водою и нечист будет до вечера;
\vs Lev 15:8 если имеющий истечение плюнет на чистого, то сей должен вымыть одежды свои и омыться водою, и нечист будет до вечера;
\vs Lev 15:9 и всякая повозка, в которой ехал имеющий истечение, нечиста [будет до вечера];
\vs Lev 15:10 и всякий, кто прикоснется к чему-нибудь, что было под ним, нечист будет до вечера; и кто понесет это, должен вымыть одежды свои и омыться водою, и нечист будет до вечера;
\vs Lev 15:11 и всякий, к кому прикоснется имеющий истечение, не омыв рук своих водою, должен вымыть одежды свои и омыться водою, и нечист будет до вечера;
\vs Lev 15:12 глиняный сосуд, к которому прикоснется имеющий истечение, должно разбить, а всякий деревянный сосуд должно вымыть водою [и будет чист].
\vs Lev 15:13 А когда имеющий истечение освободится от истечения своего, тогда должен он отсчитать себе семь дней для очищения своего и вымыть одежды свои, и омыть тело свое живою водою, и будет чист;
\vs Lev 15:14 и в восьмой день возьмет он себе двух горлиц или двух молодых голубей, и придет пред лице Господне ко входу скинии собрания, и отдаст их священнику;
\vs Lev 15:15 и принесет священник из сих \bibemph{птиц} одну в жертву за грех, а другую во всесожжение, и очистит его священник пред Господом от истечения его.
\vs Lev 15:16 Если у кого случится излияние семени, то он должен омыть водою все тело свое, и нечист будет до вечера;
\vs Lev 15:17 и всякая одежда и всякая кожа, на которую попадет семя, должна быть вымыта водою, и нечиста будет до вечера;
\vs Lev 15:18 если мужчина ляжет с женщиной и \bibemph{будет} у него излияние семени, то они должны омыться водою, и нечисты будут до вечера.
\vs Lev 15:19 Если женщина имеет истечение крови, текущей из тела ее, то она должна сидеть семь дней во время очищения своего, и всякий, кто прикоснется к ней, нечист будет до вечера;
\vs Lev 15:20 и всё, на чем она ляжет в продолжение очищения своего, нечисто; и всё, на чем сядет, нечисто;
\vs Lev 15:21 и всякий, кто прикоснется к постели ее, должен вымыть одежды свои и омыться водою и нечист будет до вечера;
\vs Lev 15:22 и всякий, кто прикоснется к какой-нибудь вещи, на которой она сидела, должен вымыть одежды свои и омыться водою, и нечист будет до вечера;
\vs Lev 15:23 и если кто прикоснется к чему-нибудь на постели или на той вещи, на которой она сидела, нечист будет до вечера;
\vs Lev 15:24 если переспит с нею муж, то нечистота ее будет на нем; он нечист будет семь дней, и всякая постель, на которой он ляжет, будет нечиста.
\vs Lev 15:25 Если у женщины течет кровь многие дни не во время очищения ее, или если она имеет истечение долее \bibemph{обыкновенного} очищения ее, то во все время истечения нечистоты ее, подобно как в продолжение очищения своего, она нечиста;
\vs Lev 15:26 всякая постель, на которой она ляжет во все время истечения своего, будет \bibemph{нечиста}, подобно как постель в продолжение очищения ее; и всякая вещь, на которую она сядет, будет нечиста, как нечисто это во время очищения ее;
\vs Lev 15:27 и всякий, кто прикоснется к ним, будет нечист, и должен вымыть одежды свои и омыться водою, и нечист будет до вечера.
\vs Lev 15:28 А когда она освободится от истечения своего, тогда должна отсчитать себе семь дней, и потом будет чиста;
\vs Lev 15:29 в восьмой день возьмет она себе двух горлиц или двух молодых голубей и принесет их к священнику ко входу скинии собрания;
\vs Lev 15:30 и принесет священник одну \bibemph{из птиц} в жертву за грех, а другую во всесожжение, и очистит ее священник пред Господом от истечения нечистоты ее.
\rsbpar\vs Lev 15:31 Так предохраняйте сынов Израилевых от нечистоты их, чтоб они не умерли в нечистоте своей, оскверняя жилище Мое, которое среди них:
\vs Lev 15:32 вот закон об имеющем истечение и о том, у кого случится излияние семени, делающее его нечистым,
\vs Lev 15:33 и о страдающей очищением своим, и о имеющих истечение, мужчине или женщине, и о муже, который переспит с нечистою.
\vs Lev 16:1 И говорил Господь Моисею по смерти двух сынов Аароновых, когда они, приступив [с чуждым огнем] пред лице Господне, умерли,
\vs Lev 16:2 и сказал Господь Моисею: скажи Аарону, брату твоему, чтоб он не во всякое время входил во святилище за завесу пред крышку [очистилище], что на ковчеге [откровения], дабы ему не умереть, ибо над крышкою Я буду являться в облаке.
\vs Lev 16:3 Вот с чем должен входить Аарон во святилище: с тельцом в жертву за грех и с овном во всесожжение;
\vs Lev 16:4 священный льняной хитон должен надевать он, нижнее платье льняное да будет на теле его, и льняным поясом пусть опоясывается, и льняной кидар надевает: это священные одежды; и пусть омывает он тело свое водою и надевает их;
\vs Lev 16:5 и от общества сынов Израилевых пусть возьмет [из стада коз] двух козлов в жертву за грех и одного овна во всесожжение.
\vs Lev 16:6 И принесет Аарон тельца в жертву за грех за себя и очистит себя и дом свой.
\vs Lev 16:7 И возьмет двух козлов и поставит их пред лицем Господним у входа скинии собрания;
\vs Lev 16:8 и бросит Аарон об обоих козлах жребии: один жребий для Господа, а другой жребий для отпущения;
\vs Lev 16:9 и приведет Аарон козла, на которого вышел жребий для Господа, и принесет его в жертву за грех,
\vs Lev 16:10 а козла, на которого вышел жребий для отпущения, поставит живого пред Господом, чтобы совершить над ним очищение и отослать его в пустыню для отпущения [и чтоб он понес на себе их беззакония в землю непроходимую].
\vs Lev 16:11 И приведет Аарон тельца в жертву за грех за себя, и очистит себя и дом свой, и заколет тельца в жертву за грех за себя;
\vs Lev 16:12 и возьмет горящих угольев полную кадильницу с жертвенника, который пред лицем Господним, и благовонного мелко истолченного курения полные горсти, и внесет за завесу;
\vs Lev 16:13 и положит курение на огонь пред лицем Господним, и облако курения покроет крышку, которая над \bibemph{ковчегом} откровения, дабы ему не умереть;
\vs Lev 16:14 и возьмет крови тельца и покропит перстом своим на крышку спереди и пред крышкою, семь раз покропит кровью с перста своего.
\vs Lev 16:15 И заколет козла в жертву за грех за народ, и внесет кровь его за завесу, и сделает с кровью его то же, что делал с кровью тельца, и покропит ею на крышку и пред крышкою,~---
\vs Lev 16:16 и очистит святилище от нечистот сынов Израилевых и от преступлений их, во всех грехах их. Так должен поступить он и со скиниею собрания, находящеюся у них, среди нечистот их.
\vs Lev 16:17 Ни один человек не должен быть в скинии собрания, когда входит он для очищения святилища, до самого выхода его. И так очистит он себя, дом свой и все общество Израилево.
\vs Lev 16:18 И выйдет он к жертвеннику, который пред лицем Господним, и очистит его, и возьмет крови тельца и крови козла, и возложит на роги жертвенника со всех сторон,
\vs Lev 16:19 и покропит на него кровью с перста своего семь раз, и очистит его, и освятит его от нечистот сынов Израилевых.
\vs Lev 16:20 И совершив очищение святилища, скинии собрания и жертвенника [и очистив священников], приведет он живого козла,
\vs Lev 16:21 и возложит Аарон обе руки свои на голову живого козла, и исповедает над ним все беззакония сынов Израилевых и все преступления их и все грехи их, и возложит их на голову козла, и отошлет с нарочным человеком в пустыню:
\vs Lev 16:22 и понесет козел на себе все беззакония их в землю непроходимую, и пустит он козла в пустыню.
\vs Lev 16:23 И войдет Аарон в скинию собрания, и снимет льняные одежды, которые надевал, входя во святилище, и оставит их там,
\vs Lev 16:24 и омоет тело свое водою на святом месте, и наденет одежды свои, и выйдет и совершит всесожжение за себя и всесожжение за народ, и очистит себя, [дом свой] и народ [и священников];
\vs Lev 16:25 а тук жертвы за грех воскурит на жертвеннике.
\vs Lev 16:26 И тот, кто отводил козла для отпущения, должен вымыть одежды свои, омыть тело свое водою, и потом может войти в стан.
\vs Lev 16:27 А тельца за грех и козла за грех, которых кровь внесена была для очищения святилища, пусть вынесут вон из стана и сожгут на огне кожи их и мясо их и нечистоту их;
\vs Lev 16:28 кто сожжет их, тот должен вымыть одежды свои и омыть тело свое водою, и после того может войти в стан.
\rsbpar\vs Lev 16:29 И да будет сие для вас вечным постановлением: в седьмой месяц, в десятый [день] месяца смиряйте души ваши и никакого дела не делайте, ни туземец, ни пришлец, поселившийся между вами,
\vs Lev 16:30 ибо в сей день очищают вас, чтобы сделать вас чистыми от всех грехов ваших, чтобы вы были чисты пред лицем Господним;
\vs Lev 16:31 это суббота покоя для вас, смиряйте души ваши: это постановление вечное.
\vs Lev 16:32 Очищать же должен священник, который помазан и который посвящен, чтобы священнодействовать ему вместо отца своего: и наденет он льняные одежды, одежды священные,
\vs Lev 16:33 и очистит Святое Святых и скинию собрания, и жертвенник очистит, и священников и весь народ общества очистит.
\vs Lev 16:34 И да будет сие для вас вечным постановлением: очищать сынов Израилевых от всех грехов их однажды в году. И сделал он так, как повелел Господь Моисею.
\vs Lev 17:1 И сказал Господь Моисею, говоря:
\vs Lev 17:2 объяви Аарону и сынам его и всем сынам Израилевым и скажи им: вот что повелевает Господь:
\vs Lev 17:3 если кто из дома Израилева [или из пришельцев, присоединившихся к вам] заколет тельца или овцу или козу в стане, или если кто заколет вне стана
\vs Lev 17:4 и не приведет ко входу скинии собрания, [чтобы принести во всесожжение или в жертву о спасении, угодную Господу, в приятное благоухание, и если кто заколет вне \bibemph{стана} и ко входу скинии собрания не принесет,] чтобы представить в жертву Господу пред жилищем Господним, то человеку тому вменена будет кровь: он пролил кровь, и истребится человек тот из народа своего;
\vs Lev 17:5 \bibemph{это} для того, чтобы приводили сыны Израилевы жертвы свои, которые они заколают на поле, чтобы приводили их пред Господа ко входу скинии собрания, к священнику, и заколали их Господу в жертвы мирные;
\vs Lev 17:6 и покропит священник кровью на жертвенник Господень у входа скинии собрания и воскурит тук в приятное благоухание Господу,
\vs Lev 17:7 чтоб они впредь не приносили жертв своих идолам, за которыми блудно ходят они. Сие да будет для них постановлением вечным в роды их.
\rsbpar\vs Lev 17:8 \bibemph{Еще} скажи им: если кто из дома Израилева и из пришельцев, которые живут между вами, приносит всесожжение или жертву
\vs Lev 17:9 и не приведет ко входу скинии собрания, чтобы совершить ее Господу, то истребится человек тот из народа своего.
\vs Lev 17:10 Если кто из дома Израилева и из пришельцев, которые живут между вами, будет есть какую-нибудь кровь, то обращу лице Мое на душу того, кто будет есть кровь, и истреблю ее из народа ее,
\vs Lev 17:11 потому что душа тела в крови, и Я назначил ее вам для жертвенника, чтобы очищать души ваши, ибо кровь сия душу очищает;
\vs Lev 17:12 потому Я и сказал сынам Израилевым: ни одна душа из вас не должна есть крови, и пришлец, живущий между вами, не должен есть крови.
\vs Lev 17:13 Если кто из сынов Израилевых и из пришельцев, живущих между вами, на ловле поймает зверя или птицу, которую можно есть, то он должен дать вытечь крови ее и покрыть ее землею,
\vs Lev 17:14 ибо душа всякого тела \bibemph{есть} кровь его, она душа его; потому Я сказал сынам Израилевым: не ешьте крови ни из какого тела, потому что душа всякого тела есть кровь его: всякий, кто будет есть ее, истребится.
\vs Lev 17:15 И всякий, кто будет есть мертвечину или растерзанное зверем, туземец или пришлец, должен вымыть одежды свои и омыться водою, и нечист будет до вечера, а \bibemph{потом} будет чист;
\vs Lev 17:16 если же не вымоет [одежд своих] и не омоет тела своего, то понесет на себе беззаконие свое.
\vs Lev 18:1 И сказал Господь Моисею, говоря:
\vs Lev 18:2 объяви сынам Израилевым и скажи им: Я Господь, Бог ваш.
\vs Lev 18:3 По делам земли Египетской, в которой вы жили, не поступайте, и по делам земли Ханаанской, в которую Я веду вас, не поступайте, и по установлениям их не ходите:
\vs Lev 18:4 Мои законы исполняйте и Мои постановления соблюдайте, поступая по ним. Я Господь, Бог ваш.
\vs Lev 18:5 Соблюдайте постановления Мои и законы Мои, которые исполняя, человек будет жив. Я Господь [Бог ваш].
\rsbpar\vs Lev 18:6 Никто ни к какой родственнице по плоти не должен приближаться с тем, чтобы открыть наготу. Я Господь.
\vs Lev 18:7 Наготы отца твоего и наготы матери твоей не открывай: она мать твоя, не открывай наготы ее.
\vs Lev 18:8 Наготы жены отца твоего не открывай: это нагота отца твоего.
\vs Lev 18:9 Наготы сестры твоей, дочери отца твоего или дочери матери твоей, родившейся в доме или вне дома, не открывай наготы их.
\vs Lev 18:10 Наготы дочери сына твоего или дочери дочери твоей, не открывай наготы их, ибо они твоя нагота.
\vs Lev 18:11 Наготы дочери жены отца твоего, родившейся от отца твоего, она сестра твоя [по отцу], не открывай наготы ее.
\vs Lev 18:12 Наготы сестры отца твоего не открывай, она единокровная отцу твоему.
\vs Lev 18:13 Наготы сестры матери твоей не открывай, ибо она единокровная матери твоей.
\vs Lev 18:14 Наготы брата отца твоего не открывай и к жене его не приближайся: она тетка твоя.
\vs Lev 18:15 Наготы невестки твоей не открывай: она жена сына твоего, не открывай наготы ее.
\vs Lev 18:16 Наготы жены брата твоего не открывай, это нагота брата твоего.
\vs Lev 18:17 Наготы жены и дочери ее не открывай; дочери сына ее и дочери дочери ее не бери, чтоб открыть наготу их, они единокровные ее; это беззаконие.
\vs Lev 18:18 Не бери жены вместе с сестрою ее, чтобы сделать ее соперницею, чтоб открыть наготу ее при ней, при жизни ее.
\vs Lev 18:19 И к жене во время очищения нечистот ее не приближайся, чтоб открыть наготу ее.
\vs Lev 18:20 И с женою ближнего твоего не ложись, чтобы излить семя и оскверниться с нею.
\vs Lev 18:21 Из детей твоих не отдавай на служение Молоху и не бесчести имени Бога твоего. Я Господь.
\vs Lev 18:22 Не ложись с мужчиною, как с женщиною: это мерзость.
\vs Lev 18:23 И ни с каким скотом не ложись, чтоб излить [семя] и оскверниться от него; и женщина не должна становиться пред скотом для совокупления с ним: это гнусно.
\rsbpar\vs Lev 18:24 Не оскверняйте себя ничем этим, ибо всем этим осквернили себя народы, которых Я прогоняю от вас:
\vs Lev 18:25 и осквернилась земля, и Я воззрел на беззаконие ее, и свергнула с себя земля живущих на ней.
\vs Lev 18:26 А вы соблюдайте постановления Мои и законы Мои и не делайте всех этих мерзостей, ни туземец, ни пришлец, живущий между вами,
\vs Lev 18:27 ибо все эти мерзости делали люди сей земли, что пред вами, и осквернилась земля;
\vs Lev 18:28 чтоб и вас не свергнула с себя земля, когда вы станете осквернять ее, как она свергнула народы, бывшие прежде вас;
\vs Lev 18:29 ибо если кто будет делать все эти мерзости, то души делающих это истреблены будут из народа своего.
\vs Lev 18:30 Итак соблюдайте повеления Мои, чтобы не поступать по гнусным обычаям, по которым поступали прежде вас, и чтобы не оскверняться ими. Я Господь, Бог ваш.
\vs Lev 19:1 И сказал Господь Моисею, говоря:
\vs Lev 19:2 объяви всему обществу сынов Израилевых и скажи им: святы будьте, ибо свят Я Господь, Бог ваш.
\vs Lev 19:3 Бойтесь каждый матери своей и отца своего и субботы Мои храните. Я Господь, Бог ваш.
\vs Lev 19:4 Не обращайтесь к идолам и богов литых не делайте себе. Я Господь, Бог ваш.
\vs Lev 19:5 Когда будете приносить Господу жертву мирную, то приносите ее, чтобы приобрести себе благоволение:
\vs Lev 19:6 в день жертвоприношения вашего и на другой день должно есть ее, а оставшееся к третьему дню должно сжечь на огне;
\vs Lev 19:7 если же кто станет есть ее на третий день, это гнусно, это не будет благоприятно;
\vs Lev 19:8 кто станет есть ее, тот понесет на себе грех, ибо он осквернил святыню Господню, и истребится душа та из народа своего.
\vs Lev 19:9 Когда будете жать жатву на земле вашей, не дожинай до края поля твоего, и оставшегося от жатвы твоей не подбирай,
\vs Lev 19:10 и виноградника твоего не обирай дочиста, и поп\acc{а}давших ягод в винограднике не подбирай; оставь это бедному и пришельцу. Я Господь, Бог ваш.
\vs Lev 19:11 Не крадите, не лгите и не обманывайте друг друга.
\vs Lev 19:12 Не клянитесь именем Моим во лжи, и не бесчести имени Бога твоего. Я Господь [Бог ваш].
\vs Lev 19:13 Не обижай ближнего твоего и не грабительствуй. Плата наемнику не должна оставаться у тебя до утра.
\vs Lev 19:14 Не злословь глухого и пред слепым не клади ничего, чтобы преткнуться ему; бойся [Господа] Бога твоего. Я Господь [Бог ваш].
\vs Lev 19:15 Не делайте неправды на суде; не будь лицеприятен к нищему и не угождай лицу великого; по правде суди ближнего твоего.
\vs Lev 19:16 Не ходи переносчиком в народе твоем и не восставай на жизнь ближнего твоего. Я Господь [Бог ваш].
\vs Lev 19:17 Не враждуй на брата твоего в сердце твоем; обличи ближнего твоего, и не понесешь за него греха.
\vs Lev 19:18 Не мсти и не имей злобы на сынов народа твоего, но люби ближнего твоего, как самого себя. Я Господь [Бог ваш].
\vs Lev 19:19 Уставы Мои соблюдайте; скота твоего не своди с иною породою; поля твоего не засевай двумя родами \bibemph{семян}; в одежду из разнородных нитей, из шерсти и льна, не одевайся.
\vs Lev 19:20 Если кто переспит с женщиною, а она раба, обрученная мужу, но еще не выкупленная, или свобода еще не дана ей, то должно наказать их, но не смертью, потому что она несвободная:
\vs Lev 19:21 пусть приведет он Господу ко входу скинии собрания жертву повинности, овна в жертву повинности своей;
\vs Lev 19:22 и очистит его священник овном повинности пред Господом от греха, которым он согрешил, и прощен будет ему грех, которым он согрешил.
\vs Lev 19:23 Когда придете в землю, [которую Господь Бог даст вам,] и посадите какое-либо плодовое дерево, то плоды его почитайте за необрезанные: три года должно почитать их за необрезанные, не должно есть их;
\vs Lev 19:24 а в четвертый год все плоды его должны быть посвящены для празднеств Господних;
\vs Lev 19:25 в пятый же год вы можете есть плоды его и собирать себе все произведения его. Я Господь, Бог ваш.
\vs Lev 19:26 Не ешьте с кровью; не ворожите и не гадайте.
\vs Lev 19:27 Не стригите головы вашей кругом, и не порти края бороды твоей.
\vs Lev 19:28 Ради умершего не делайте нарезов на теле вашем и не накалывайте на себе письмен. Я Господь [Бог ваш].
\vs Lev 19:29 Не оскверняй дочери твоей, допуская ее до блуда, чтобы не блудодействовала земля и не наполнилась земля развратом.
\vs Lev 19:30 Субботы Мои храните и святилище Мое чтите. Я Господь.
\vs Lev 19:31 Не обращайтесь к вызывающим мертвых, и к волшебникам не ходите, и не доводите себя до осквернения от них. Я Господь, Бог ваш.
\vs Lev 19:32 Пред лицем седого вставай и почитай лице старца, и бойся [Господа] Бога твоего. Я Господь [Бог ваш].
\vs Lev 19:33 Когда поселится пришлец в земле вашей, не притесняйте его:
\vs Lev 19:34 пришлец, поселившийся у вас, да будет для вас то же, что туземец ваш; люби его, как себя; ибо и вы были пришельцами в земле Египетской. Я Господь, Бог ваш.
\vs Lev 19:35 Не делайте неправды в суде, в мере, в весе и в измерении:
\vs Lev 19:36 да будут у вас весы верные, гири верные, ефа верная и гин верный. Я Господь, Бог ваш, Который вывел вас из земли Египетской.
\vs Lev 19:37 Соблюдайте все уставы Мои и все законы Мои и исполняйте их. Я Господь [Бог ваш].
\vs Lev 20:1 И сказал Господь Моисею, говоря:
\vs Lev 20:2 скажи сие сынам Израилевым: кто из сынов Израилевых и из пришельцев, живущих между Израильтянами, даст из детей своих Молоху, тот да будет предан смерти: народ земли да побьет его камнями;
\vs Lev 20:3 и Я обращу лице Мое на человека того и истреблю его из народа его за то, что он дал из детей своих Молоху, чтоб осквернить святилище Мое и обесчестить святое имя Мое;
\vs Lev 20:4 и если народ земли не обратит очей своих на человека того, когда он даст из детей своих Молоху, и не умертвит его,
\vs Lev 20:5 то Я обращу лице Мое на человека того и на род его и истреблю его из народа его, и всех блудящих по следам его, чтобы блудно ходить вслед Молоха.
\vs Lev 20:6 И если какая душа обратится к вызывающим мертвых и к волшебникам, чтобы блудно ходить вслед их, то Я обращу лице Мое на ту душу и истреблю ее из народа ее.
\vs Lev 20:7 Освящайте себя и будьте святы, ибо Я Господь, Бог ваш, [свят].
\vs Lev 20:8 Соблюдайте постановления Мои и исполняйте их, ибо Я Господь, освящающий вас.
\vs Lev 20:9 Кто будет злословить отца своего или мать свою, тот да будет предан смерти; отца своего и мать свою он злословил: кровь его на нем.
\vs Lev 20:10 Если кто будет прелюбодействовать с женой замужнею, если кто будет прелюбодействовать с женою ближнего своего,~--- да будут преданы смерти и прелюбодей и прелюбодейка.
\vs Lev 20:11 Кто ляжет с женою отца своего, тот открыл наготу отца своего: оба они да будут преданы смерти, кровь их на них.
\vs Lev 20:12 Если кто ляжет с невесткою своею, то оба они да будут преданы смерти: мерзость сделали они, кровь их на них.
\vs Lev 20:13 Если кто ляжет с мужчиною, как с женщиною, то оба они сделали мерзость: да будут преданы смерти, кровь их на них.
\vs Lev 20:14 Если кто возьмет себе жену и мать ее: это беззаконие; на огне должно сжечь его и их, чтобы не было беззакония между вами.
\vs Lev 20:15 Кто смесится со скотиною, того предать смерти, и скотину убейте.
\vs Lev 20:16 Если женщина пойдет к какой-нибудь скотине, чтобы совокупиться с нею, то убей женщину и скотину: да будут они преданы смерти, кровь их на них.
\vs Lev 20:17 Если кто возьмет сестру свою, дочь отца своего или дочь матери своей, и увидит наготу ее, и она увидит наготу его: это срам, да будут они истреблены пред глазами сынов народа своего; он открыл наготу сестры своей: грех свой понесет он.
\vs Lev 20:18 Если кто ляжет с женою во время болезни \bibemph{кровоочищения} и откроет наготу ее, то он обнажил истечения ее, и она открыла течение кровей своих: оба они да будут истреблены из народа своего.
\vs Lev 20:19 Наготы сестры матери твоей и сестры отца твоего не открывай, ибо таковой обнажает плоть свою: грех свой понесут они.
\vs Lev 20:20 Кто ляжет с теткою своею, тот открыл наготу дяди своего; грех свой понесут они, бездетными умрут.
\vs Lev 20:21 Если кто возьмет жену брата своего: это гнусно; он открыл наготу брата своего, бездетны будут они.
\rsbpar\vs Lev 20:22 Соблюдайте все уставы Мои и все законы Мои и исполняйте их,~--- и не свергнет вас с себя земля, в которую Я веду вас жить.
\vs Lev 20:23 Не поступайте по обычаям народа, который Я прогоняю от вас; ибо они всё это делали, и Я вознегодовал на них,
\vs Lev 20:24 и сказал Я вам: вы владейте землею их, и вам отдаю в наследие землю, в которой течет молоко и мед. Я Господь, Бог ваш, Который отделил вас от всех народов.
\vs Lev 20:25 Отличайте скот чистый от нечистого и птицу чистую от нечистой и не оскверняйте душ ваших скотом и птицею и всем пресмыкающимся по земле, что отличил Я, как нечистое.
\vs Lev 20:26 Будьте предо Мною святы, ибо Я свят Господь [Бог ваш], и Я отделил вас от народов, чтобы вы были Мои.
\vs Lev 20:27 Мужчина ли или женщина, если будут они вызывать мертвых или волхвовать, да будут преданы смерти: камнями должно побить их, кровь их на них.
\vs Lev 21:1 И сказал Господь Моисею: объяви священникам, сынам Аароновым, и скажи им: да не оскверняют себя \bibemph{прикосновением} к умершему из народа своего;
\vs Lev 21:2 только к ближнему родственнику своему, к матери своей и к отцу своему, к сыну своему и дочери своей, к брату своему
\vs Lev 21:3 и к сестре своей, девице, живущей при нем и не бывшей замужем, можно ему \bibemph{прикасаться}, не оскверняя себя;
\vs Lev 21:4 и \bibemph{прикосновением} к кому бы то ни было в народе своем не должен он осквернять себя, чтобы не сделаться нечистым.
\vs Lev 21:5 Они не должны брить головы своей и подстригать края бороды своей и делать нарезы на теле своем.
\vs Lev 21:6 Они должны быть святы Богу своему и не должны бесчестить имени Бога своего, ибо они приносят жертвы Господу, хлеб Богу своему, и потому должны быть святы.
\vs Lev 21:7 Они не должны брать за себя блудницу и опороченную, не должны брать и жену, отверженную мужем своим, ибо они святы [Господу] Богу своему.
\vs Lev 21:8 Святи его, ибо он приносит хлеб [Господу] Богу твоему: да будет он у тебя свят, ибо свят Я Господь, освящающий вас.
\vs Lev 21:9 Если дочь священника осквернит себя блудодеянием, то она бесчестит отца своего; огнем должно сжечь ее.
\rsbpar\vs Lev 21:10 Великий же священник из братьев своих, на голову которого возлит елей помазания, и который освящен, чтобы облачаться в \bibemph{священные} одежды, не должен обнажать головы своей и раздирать одежд своих;
\vs Lev 21:11 и ни к какому умершему не должен он приступать: даже \bibemph{прикосновением к умершему} отцу своему и матери своей он не должен осквернять себя.
\vs Lev 21:12 И от святилища он не должен отходить и бесчестить святилище Бога своего, ибо освящение елеем помазания Бога его на нем. Я Господь.
\vs Lev 21:13 В жену он должен брать девицу [из народа своего]:
\vs Lev 21:14 вдову, или отверженную, или опороченную, [или] блудницу, не должен он брать, но девицу из народа своего должен он брать в жену;
\vs Lev 21:15 он не должен порочить семени своего в народе своем, ибо Я Господь [Бог], освящающий его.
\rsbpar\vs Lev 21:16 И сказал Господь Моисею, говоря:
\vs Lev 21:17 скажи Аарону: никто из семени твоего во \bibemph{все} роды их, у которого \bibemph{на теле} будет недостаток, не должен приступать, чтобы приносить хлеб Богу своему;
\vs Lev 21:18 никто, у кого на теле есть недостаток, не должен приступать, ни слепой, ни хромой, ни уродливый,
\vs Lev 21:19 ни такой, у которого переломлена нога или переломлена рука,
\vs Lev 21:20 ни горбатый, ни с сухим членом, ни с бельмом на глазу, ни коростовый, ни паршивый, ни с поврежденными ятрами;
\vs Lev 21:21 ни один человек из семени Аарона священника, у которого \bibemph{на теле} есть недостаток, не должен приступать, чтобы приносить жертвы Господу; недостаток \bibemph{на нем}, поэтому не должен он приступать, чтобы приносить хлеб Богу своему;
\vs Lev 21:22 хлеб Бога своего из великих святынь и из святынь он может есть;
\vs Lev 21:23 но к завесе не должен он приходить и к жертвеннику не должен приступать, потому что недостаток на нем: не должен он бесчестить святилища Моего, ибо Я Господь, освящающий их.
\vs Lev 21:24 И объявил \bibemph{это} Моисей Аарону и сынам его и всем сынам Израилевым.
\vs Lev 22:1 И сказал Господь Моисею, говоря:
\vs Lev 22:2 скажи Аарону и сынам его, чтоб они осторожно поступали со святынями сынов Израилевых и не бесчестили святаго имени Моего в том, что они посвящают Мне. Я Господь.
\vs Lev 22:3 Скажи им: если кто из всего потомства вашего в роды ваши, имея на себе нечистоту, приступит к святыням, которые посвящают сыны Израилевы Господу, то истребится душа та от лица Моего. Я Господь [Бог ваш].
\vs Lev 22:4 Кто из семени Ааронова прокажен, или имеет истечение, тот не должен есть святынь, пока не очистится; и кто прикоснется к чему-нибудь нечистому от мертвого, или у кого случится излияние семени,
\vs Lev 22:5 или кто прикоснется к какому-нибудь гаду, от которого он сделается нечист, или к человеку, от которого он сделается нечист какою бы то ни было нечистотою,~---
\vs Lev 22:6 тот, прикоснувшийся к сему, нечист будет до вечера и не должен есть святынь, прежде нежели омоет тело свое водою;
\vs Lev 22:7 но когда зайдет солнце и он очистится, тогда может он есть святыни, ибо это его пища.
\vs Lev 22:8 Мертвечины и звероядины он не должен есть, чтобы не оскверниться этим. Я Господь.
\vs Lev 22:9 Да соблюдают они повеления Мои, чтобы не понести на себе греха и не умереть в нем, когда нарушат сие. Я Господь [Бог], освящающий их.
\vs Lev 22:10 Никто посторонний не должен есть святыни; поселившийся у священника и наемник не должен есть святыни;
\vs Lev 22:11 если же священник купит себе человека за серебро, то сей может есть оную; также и домочадцы его могут есть хлеб его.
\vs Lev 22:12 Если дочь священника выйдет в замужество за постороннего, то она не должна есть приносимых святынь;
\vs Lev 22:13 когда же дочь священника будет вдова, или разведенная, и детей нет у нее, и возвратится в дом отца своего, как \bibemph{была} в юности своей, тогда она может есть хлеб отца своего; а посторонний никто не должен есть его.
\vs Lev 22:14 Кто по ошибке съест \bibemph{что-нибудь} из святыни, тот должен отдать священнику святыню и приложить к ней пятую ее долю.
\vs Lev 22:15 \bibemph{Священники} сами не должны порочить святыни сынов Израилевых, которые они приносят Господу,
\vs Lev 22:16 и не должны навлекать на себя вину в преступлении, когда будут есть святыни свои, ибо Я Господь, освящающий их.
\rsbpar\vs Lev 22:17 И сказал Господь Моисею, говоря:
\vs Lev 22:18 объяви Аарону и сынам его и всем сынам Израилевым и скажи им: если кто из дома Израилева, или из пришельцев, [поселившихся] между Израильтянами, по обету ли какому, или по усердию приносит жертву свою, которую приносят Господу во всесожжение,
\vs Lev 22:19 то, чтобы сим приобрести благоволение \bibemph{от Бога, жертва должна быть} без порока, мужеского пола, из крупного скота, из овец и из коз;
\vs Lev 22:20 никакого \bibemph{животного}, на котором есть порок, не приносите [Господу], ибо это не приобретет вам благоволения.
\vs Lev 22:21 И если кто приносит мирную жертву Господу, исполняя обет, или по усердию, [или в праздники ваши,] из крупного скота или из мелкого, то \bibemph{жертва должна быть} без порока, чтоб быть угодною \bibemph{Богу}: никакого порока не должно быть на ней;
\vs Lev 22:22 \bibemph{животного} слепого, или поврежденного, или уродливого, или больного, или коростового, или паршивого, таких не приносите Господу и в жертву не давайте их на жертвенник Господень;
\vs Lev 22:23 тельца и агнца с членами, несоразмерно длинными или короткими, в жертву усердия принести можешь; а если по обету, то это не угодно будет \bibemph{Богу};
\vs Lev 22:24 \bibemph{животного}, у которого ятра раздавлены, разбиты, оторваны или вырезаны, не приносите Господу и в земле вашей не делайте \bibemph{сего};
\vs Lev 22:25 и из рук иноземцев не приносите всех таковых \bibemph{животных} в дар Богу вашему, потому что на них повреждение, порок на них: не приобретут они вам благоволения.
\rsbpar\vs Lev 22:26 И сказал Господь Моисею, говоря:
\vs Lev 22:27 когда родится теленок, или ягненок, или козленок, то семь дней он должен пробыть при матери своей, а от восьмого дня и далее будет благоугоден для приношения в жертву Господу;
\vs Lev 22:28 но ни коровы, ни овцы не заколайте в один день с порождением ее.
\vs Lev 22:29 Если прин\acc{о}сите Господу жертву благодарения, то принос\acc{и}те ее так, чтоб она приобрела вам благоволение;
\vs Lev 22:30 в тот же день должно съесть ее, не оставляйте от нее до утра. Я Господь.
\rsbpar\vs Lev 22:31 И соблюдайте заповеди Мои и исполняйте их. Я Господь.
\vs Lev 22:32 Не бесчестите святого имени Моего, чтоб Я был святим среди сынов Израилевых. Я Господь, освящающий вас,
\vs Lev 22:33 Который вывел вас из земли Египетской, чтоб быть вашим Богом. Я Господь.
\vs Lev 23:1 И сказал Господь Моисею, говоря:
\vs Lev 23:2 объяви сынам Израилевым и скажи им о праздниках Господних, в которые должно созывать священные собрания. Вот праздники Мои:
\vs Lev 23:3 шесть дней можно делать дела, а в седьмой день суббота покоя, священное собрание; никакого дела не делайте; это суббота Господня во всех жилищах ваших.
\rsbpar\vs Lev 23:4 Вот праздники Господни, священные собрания, которые вы должны созывать в свое время:
\vs Lev 23:5 в первый месяц, в четырнадцатый [день] месяца вечером Пасха Господня;
\vs Lev 23:6 и в пятнадцатый день того же месяца праздник опресноков Господу; семь дней ешьте опресноки;
\vs Lev 23:7 в первый день да будет у вас священное собрание; никакой работы не работайте;
\vs Lev 23:8 и в течение семи дней приносите жертвы Господу; в седьмой день также священное собрание; никакой работы не работайте.
\rsbpar\vs Lev 23:9 И сказал Господь Моисею, говоря:
\vs Lev 23:10 объяви сынам Израилевым и скажи им: когда придете в землю, которую Я даю вам, и будете жать на ней жатву, то принесите первый сноп жатвы вашей к священнику;
\vs Lev 23:11 он вознесет этот сноп пред Господом, чтобы вам приобрести благоволение; на другой день праздника вознесет его священник;
\vs Lev 23:12 и в день возношения снопа принесите во всесожжение Господу агнца однолетнего, без порока,
\vs Lev 23:13 и с ним хлебного приношения две десятых части \bibemph{ефы} пшеничной муки, смешанной с елеем, в жертву Господу, в приятное благоухание, и возлияния к нему четверть гина вина;
\vs Lev 23:14 никакого \bibemph{нового} хлеба, ни сушеных зерен, ни зерен сырых не ешьте до того дня, в который принесете приношения Богу вашему: это вечное постановление в роды ваши во всех жилищах ваших.
\vs Lev 23:15 Отсчитайте себе от первого дня после праздника, от того дня, в который приносите сноп потрясания, семь полных недель,
\vs Lev 23:16 до первого дня после седьмой недели отсчитайте пятьдесят дней, \bibemph{и тогда} принесите новое хлебное приношение Господу:
\vs Lev 23:17 от жилищ ваших приносите два хлеба возношения, которые должны состоять из двух десятых частей \bibemph{ефы} пшеничной муки и должны быть испечены кислые, \bibemph{как} первый плод Господу;
\vs Lev 23:18 вместе с хлебами представьте семь агнцев без порока, однолетних, и из крупного скота одного тельца и двух овнов [без порока]; да будет это во всесожжение Господу, и хлебное приношение и возлияние к ним, в жертву, в приятное благоухание Господу.
\vs Lev 23:19 Приготовьте также из \bibemph{стада} коз одного козла в жертву за грех и двух однолетних агнцев в жертву мирную [вместе с хлебом первого плода];
\vs Lev 23:20 священник должен принести это, потрясая пред Господом, вместе с потрясаемыми хлебами первого плода и с двумя агнцами, и это будет святынею Господу; священнику, [который приносит, это принадлежит];
\vs Lev 23:21 и созывайте \bibemph{народ} в сей день, священное собрание да будет у вас, никакой работы не работайте: это постановление вечное во всех жилищах ваших в роды ваши.
\vs Lev 23:22 Когда будете жать жатву на земле вашей, не дожинай до края поля твоего, когда жнешь, и оставшегося от жатвы твоей не подбирай; бедному и пришельцу оставь это. Я Господь, Бог ваш.
\rsbpar\vs Lev 23:23 И сказал Господь Моисею, говоря:
\vs Lev 23:24 скажи сынам Израилевым: в седьмой месяц, в первый [день] месяца да будет у вас покой, праздник труб, священное собрание [да будет у вас];
\vs Lev 23:25 никакой работы не работайте и приносите жертву Господу.
\rsbpar\vs Lev 23:26 И сказал Господь Моисею, говоря:
\vs Lev 23:27 также в девятый [день] седьмого месяца сего, день очищения, да будет у вас священное собрание; смиряйте души ваши и приносите жертву Господу;
\vs Lev 23:28 никакого дела не делайте в день сей, ибо это день очищения, дабы очистить вас пред лицем Господа, Бога вашего;
\vs Lev 23:29 а всякая душа, которая не смирит себя в этот день, истребится из народа своего;
\vs Lev 23:30 и если какая душа будет делать какое-нибудь дело в день сей, Я истреблю ту душу из народа ее;
\vs Lev 23:31 никакого дела не делайте: это постановление вечное в роды ваши, во всех жилищах ваших;
\vs Lev 23:32 это для вас суббота покоя, и смиряйте души ваши, с вечера девятого [дня] месяца; от вечера до вечера [десятого дня месяца] празднуйте субботу вашу.
\rsbpar\vs Lev 23:33 И сказал Господь Моисею, говоря:
\vs Lev 23:34 скажи сынам Израилевым: с пятнадцатого дня того же седьмого месяца праздник кущей, семь дней Господу;
\vs Lev 23:35 в первый день священное собрание, никакой работы не работайте;
\vs Lev 23:36 в \bibemph{течение} семи дней приносите жертву Господу; в восьмой день священное собрание да будет у вас, и приносите жертву Господу: это отдание праздника, никакой работы не работайте.
\vs Lev 23:37 Вот праздники Господни, в которые должно созывать священные собрания, чтобы приносить в жертву Господу всесожжение, хлебное приношение, заколаемые жертвы и возлияния, каждое в свой день,
\vs Lev 23:38 кроме суббот Господних и кроме даров ваших, и кроме всех обетов ваших и кроме всего \bibemph{приносимого} по усердию вашему, что вы даете Господу.
\vs Lev 23:39 А в пятнадцатый день седьмого месяца, когда вы собираете произведения земли, празднуйте праздник Господень семь дней: в первый день покой и в восьмой день покой;
\vs Lev 23:40 в первый день возьмите себе ветви красивых дерев, ветви пальмовые и ветви дерев широколиственных и верб речных, и веселитесь пред Господом Богом вашим семь дней;
\vs Lev 23:41 и празднуйте этот праздник Господень семь дней в году: это постановление вечное в роды ваши; в седьмой месяц празднуйте его;
\vs Lev 23:42 в кущах живите семь дней; всякий туземец Израильтянин должен жить в кущах,
\vs Lev 23:43 чтобы знали роды ваши, что в кущах поселил Я сынов Израилевых, когда вывел их из земли Египетской. Я Господь, Бог ваш.
\vs Lev 23:44 И объявил Моисей сынам Израилевым о праздниках Господних.
\vs Lev 24:1 И сказал Господь Моисею, говоря:
\vs Lev 24:2 прикажи сынам Израилевым, чтоб они принесли тебе елея чистого, выбитого, для освещения, чтобы непрестанно горел светильник;
\vs Lev 24:3 вне завесы \bibemph{ковчега} откровения в скинии собрания Аарон [и сыны его] должны ставить оный пред Господом от вечера до утра всегда: это вечное постановление в роды ваши;
\vs Lev 24:4 на подсвечнике чистом должны они ставить светильник пред Господом всегда.
\vs Lev 24:5 И возьми пшеничной муки и испеки из нее двенадцать хлебов; в каждом хлебе должны быть две десятых \bibemph{ефы};
\vs Lev 24:6 и положи их в два ряда, по шести в ряд, на чистом столе пред Господом;
\vs Lev 24:7 и положи на [каждый] ряд чистого ливана [и соли], и будет это при хлебе, в память, в жертву Господу;
\vs Lev 24:8 в каждый день субботы постоянно должно полагать их пред Господом от сынов Израилевых: это завет вечный;
\vs Lev 24:9 они будут принадлежать Аарону и сынам его, которые будут есть их на святом месте, ибо это великая святыня для них из жертв Господних: \bibemph{это} постановление вечное.
\rsbpar\vs Lev 24:10 И вышел сын одной Израильтянки, родившейся от Египтянина, к сынам Израилевым, и поссорился в стане сын Израильтянки с Израильтянином;
\vs Lev 24:11 хулил сын Израильтянки имя [Господне] и злословил. И привели его к Моисею [имя же матери его Саломиф, дочь Давриина, из племени Данова];
\vs Lev 24:12 и посадили его под стражу, доколе не будет объявлена им воля Господня.
\rsbpar\vs Lev 24:13 И сказал Господь Моисею, говоря:
\vs Lev 24:14 выведи злословившего вон из стана, и все слышавшие пусть положат руки свои на голову его, и все общество побьет его камнями;
\vs Lev 24:15 и сынам Израилевым скажи: кто будет злословить Бога своего, тот понесет грех свой;
\vs Lev 24:16 и хулитель имени Господня должен умереть, камнями побьет его все общество: пришлец ли, туземец ли станет хулить имя [Господне], предан будет смерти.
\vs Lev 24:17 Кто убьет какого-либо человека, тот предан будет смерти.
\vs Lev 24:18 Кто убьет скотину, должен заплатить за нее, скотину за скотину.
\vs Lev 24:19 Кто сделает повреждение на теле ближнего своего, тому должно сделать то же, что он сделал:
\vs Lev 24:20 перелом за перелом, око за око, зуб за зуб; как он сделал повреждение на \bibemph{теле} человека, так и ему должно сделать.
\vs Lev 24:21 Кто убьет скотину, должен заплатить за нее; а кто убьет человека, того должно предать смерти.
\vs Lev 24:22 Один суд должен быть у вас, как для пришельца, так и для туземца; ибо Я Господь, Бог ваш.
\vs Lev 24:23 И сказал Моисей сынам Израилевым; и вывели злословившего вон из стана, и побили его камнями, и сделали сыны Израилевы, как повелел Господь Моисею.
\vs Lev 25:1 И сказал Господь Моисею на горе Синае, говоря:
\vs Lev 25:2 объяви сынам Израилевым и скажи им: когда придете в землю, которую Я даю вам, тогда земля должна покоиться в субботу Господню;
\vs Lev 25:3 шесть лет засевай поле твое и шесть лет обрезывай виноградник твой, и собирай произведения их,
\vs Lev 25:4 а в седьмой год да будет суббота покоя земли, суббота Господня: поля твоего не засевай и виноградника твоего не обрезывай;
\vs Lev 25:5 что само вырастет на жатве твоей, не сжинай, и гроздов с необрезанных лоз твоих не снимай; да будет это год покоя земли;
\vs Lev 25:6 и будет это в продолжение субботы земли \bibemph{всем} вам в пищу, тебе и рабу твоему, и рабе твоей, и наемнику твоему, и поселенцу твоему, поселившемуся у тебя;
\vs Lev 25:7 и скоту твоему и зверям, которые на земле твоей, да будут все произведения ее в пищу.
\rsbpar\vs Lev 25:8 И насчитай себе семь субботних лет, семь раз по семи лет, чтоб было у тебя в семи субботних годах сорок девять лет;
\vs Lev 25:9 и воструби трубою в седьмой месяц, в десятый [день] месяца, в день очищения вострубите трубою по всей земле вашей;
\vs Lev 25:10 и освятите пятидесятый год и объявите свободу на земле всем жителям ее: да будет это у вас юбилей; и возвратитесь каждый во владение свое, и каждый возвратитесь в свое племя.
\vs Lev 25:11 Пятидесятый год да будет у вас юбилей: не сейте и не жните, что само вырастет на земле, и не снимайте ягод с необрезанных \bibemph{лоз} ее,
\vs Lev 25:12 ибо это юбилей: священным да будет он для вас; с поля ешьте произведения ее.
\vs Lev 25:13 В юбилейный год возвратитесь каждый во владение свое.
\vs Lev 25:14 Если будешь продавать что ближнему твоему, или будешь покупать что у ближнего твоего, не обижайте друг друга;
\vs Lev 25:15 по расчислению лет после юбилея ты должен покупать у ближнего твоего, и по расчислению лет дохода он должен продавать тебе;
\vs Lev 25:16 если много \bibemph{остается} лет, умножь цену; а если мало лет \bibemph{остается}, уменьши цену, ибо известное число \bibemph{лет} жатв он продает тебе.
\vs Lev 25:17 Не обижайте один другого; бойся Бога твоего, ибо Я Господь, Бог ваш.
\vs Lev 25:18 Исполняйте постановления Мои, и храните законы Мои и исполняйте их, и будете жить спокойно на земле;
\vs Lev 25:19 и будет земля давать плод свой, и будете есть досыта, и будете жить спокойно на ней.
\vs Lev 25:20 Если скажете: что же нам есть в седьмой год, когда мы не будем ни сеять, ни собирать произведений наших?
\vs Lev 25:21 Я пошлю благословение Мое на вас в шестой год, и он принесет произведений на три года;
\vs Lev 25:22 и будете сеять в восьмой год, но есть будете произведения старые до девятого года; доколе не поспеют произведения его, будете есть старое.
\rsbpar\vs Lev 25:23 Землю не должно продавать навсегда, ибо Моя земля: вы пришельцы и поселенцы у Меня;
\vs Lev 25:24 по всей земле владения вашего дозволяйте выкуп земли.
\vs Lev 25:25 Если брат твой обеднеет и продаст от владения своего, то придет близкий его родственник и выкупит проданное братом его;
\vs Lev 25:26 если же некому за него выкупить, но сам он будет иметь достаток и найдет, сколько нужно на выкуп,
\vs Lev 25:27 то пусть он расчислит годы продажи своей и возвратит остальное тому, кому он продал, и вступит опять во владение свое;
\vs Lev 25:28 если же не найдет рука его, сколько нужно возвратить ему, то проданное им останется в руках покупщика до юбилейного года, а в юбилейный год отойдет оно, и он опять вступит во владение свое.
\vs Lev 25:29 Если кто продаст жилой дом в городе, \bibemph{огражденном} стеною, то выкупить его можно до истечения года от продажи его: в течение года выкупить его можно;
\vs Lev 25:30 если же не будет он выкуплен до истечения целого года, то дом, который в городе, имеющем стену, останется навсегда у купившего его в роды его, и в юбилей не отойдет \bibemph{от него}.
\vs Lev 25:31 А домы в селениях, вокруг которых нет стены, должно считать наравне с полем земли: выкупать их [всегда] можно, и в юбилей они отходят.
\vs Lev 25:32 А города левитов, домы в городах владения их, левитам всегда можно выкупать;
\vs Lev 25:33 а кто из левитов не выкупит, то проданный дом в городе владения их в юбилей отойдет, потому что домы в городах левитских составляют их владение среди сынов Израилевых;
\vs Lev 25:34 и полей вокруг городов их продавать нельзя, потому что это вечное владение их.
\vs Lev 25:35 Если брат твой обеднеет и придет в упадок у тебя, то поддержи его, пришлец ли он, или поселенец, чтоб он жил с тобою;
\vs Lev 25:36 не бери от него роста и прибыли и бойся Бога твоего; [Я Господь,] чтоб жил брат твой с тобою;
\vs Lev 25:37 серебра твоего не отдавай ему в рост и хлеба твоего не отдавай ему для \bibemph{получения} прибыли.
\vs Lev 25:38 Я Господь, Бог ваш, Который вывел вас из земли Египетской, чтобы дать вам землю Ханаанскую, чтоб быть вашим Богом.
\vs Lev 25:39 Когда обеднеет у тебя брат твой и продан будет тебе, то не налагай на него работы рабской:
\vs Lev 25:40 он должен быть у тебя как наемник, как поселенец; до юбилейного года пусть работает у тебя,
\vs Lev 25:41 а \bibemph{тогда} пусть отойдет он от тебя, сам и дети его с ним, и возвратится в племя свое, и вступит опять во владение отцов своих,
\vs Lev 25:42 потому что они~--- Мои рабы, которых Я вывел из земли Египетской: не должно продавать их, как продают рабов;
\vs Lev 25:43 не господствуй над ним с жестокостью и бойся Бога твоего.
\vs Lev 25:44 А чтобы раб твой и рабыня твоя были у тебя, то покупайте себе раба и рабыню у народов, которые вокруг вас;
\vs Lev 25:45 также и из детей поселенцев, поселившихся у вас, можете покупать, и из племени их, которое у вас, которое у них родилось в земле вашей, и они могут быть вашей собственностью;
\vs Lev 25:46 можете передавать их в наследство и сынам вашим по себе, как имение; вечно владейте ими, как рабами. А над братьями вашими, сынами Израилевыми, друг над другом, не господствуйте с жестокостью.
\vs Lev 25:47 Если пришлец или поселенец твой будет иметь достаток, а брат твой пред ним обеднеет и продастся пришельцу, поселившемуся у тебя, или кому-нибудь из племени пришельца,
\vs Lev 25:48 то после продажи можно выкупить его; кто-нибудь из братьев его должен выкупить его,
\vs Lev 25:49 или дядя его, или сын дяди его должен выкупить его, или кто-нибудь из родства его, из племени его, должен выкупить его; или если будет иметь достаток, сам выкупится.
\vs Lev 25:50 И он должен рассчитаться с купившим его, \bibemph{начиная} от того года, когда он продал себя, до года юбилейного, и серебро, за которое он продал себя, должно отдать ему по числу лет; как временный наемник он должен быть у него;
\vs Lev 25:51 и если еще много \bibemph{остается} лет, то по мере их он должен отдать в выкуп за себя серебро, за которое он куплен;
\vs Lev 25:52 если же мало остается лет до юбилейного года, то он должен сосчитать и по мере лет отдать за себя выкуп.
\vs Lev 25:53 Он должен быть у него, как наемник, во все годы; он не должен господствовать над ним с жестокостью в глазах твоих.
\vs Lev 25:54 Если же он не выкупится таким образом, то в юбилейный год отойдет сам и дети его с ним,
\vs Lev 25:55 потому что сыны Израилевы Мои рабы; они Мои рабы, которых Я вывел из земли Египетской. Я Господь, Бог ваш.
\vs Lev 26:1 Не делайте себе кумиров и изваяний, и столбов не ставьте у себя, и камней с изображениями не кладите в земле вашей, чтобы кланяться пред ними, ибо Я Господь Бог ваш.
\vs Lev 26:2 Субботы Мои соблюдайте и святилище Мое чтите: Я Господь.
\vs Lev 26:3 Если вы будете поступать по уставам Моим и заповеди Мои будете хранить и исполнять их,
\vs Lev 26:4 то Я дам вам дожди в свое время, и земля даст произрастения свои, и дерева полевые дадут плод свой;
\vs Lev 26:5 и молотьба \bibemph{хлеба} будет достигать у вас собирания винограда, собирание винограда будет достигать посева, и будете есть хлеб свой досыта, и будете жить на земле [вашей] безопасно;
\vs Lev 26:6 пошлю мир на землю [вашу], ляжете, и никто вас не обеспокоит, сгоню лютых зверей с земли [вашей], и меч не пройдет по земле вашей;
\vs Lev 26:7 и будете прогонять врагов ваших, и падут они пред вами от меча;
\vs Lev 26:8 пятеро из вас прогонят сто, и сто из вас прогонят тьму, и падут враги ваши пред вами от меча;
\vs Lev 26:9 призрю на вас [и благословлю вас], и плодородными сделаю вас, и размножу вас, и буду тверд в завете Моем с вами;
\vs Lev 26:10 и будете есть старое прошлогоднее, и выбросите старое ради нового;
\vs Lev 26:11 и поставлю жилище Мое среди вас, и душа Моя не возгнушается вами;
\vs Lev 26:12 и буду ходить среди вас и буду вашим Богом, а вы будете Моим народом.
\vs Lev 26:13 Я Господь Бог ваш, Который вывел вас из земли Египетской, чтоб вы не были там рабами, и сокрушил узы ярма вашего, и повел вас с поднятою головою.
\vs Lev 26:14 Если же не послушаете Меня и не будете исполнять всех заповедей сих,
\vs Lev 26:15 и если презрите Мои постановления, и если душа ваша возгнушается Моими законами, так что вы не будете исполнять всех заповедей Моих, нарушив завет Мой,~---
\vs Lev 26:16 то и Я поступлю с вами так: пошлю на вас ужас, чахлость и горячку, от которых истомятся глаза и измучится душа, и будете сеять семена ваши напрасно, и враги ваши съедят их;
\vs Lev 26:17 обращу лице Мое на вас, и падете пред врагами вашими, и будут господствовать над вами неприятели ваши, и побежите, когда никто не гонится за вами.
\vs Lev 26:18 Если и при всем том не послушаете Меня, то Я всемеро увеличу наказание за грехи ваши,
\vs Lev 26:19 и сломлю гордое упорство ваше, и небо ваше сделаю, как железо, и землю вашу, как медь;
\vs Lev 26:20 и напрасно будет истощаться сила ваша, и земля ваша не даст произрастений своих, и дерева земли [вашей] не дадут плодов своих.
\vs Lev 26:21 Если же [после сего] пойдете против Меня и не захотите слушать Меня, то Я прибавлю вам ударов всемеро за грехи ваши:
\vs Lev 26:22 пошлю на вас зверей полевых, которые лишат вас детей, истребят скот ваш и вас уменьшат, так что опустеют дороги ваши.
\vs Lev 26:23 Если и после сего не исправитесь и пойдете против Меня,
\vs Lev 26:24 то и Я [в ярости] пойду против вас и поражу вас всемеро за грехи ваши,
\vs Lev 26:25 и наведу на вас мстительный меч в отмщение за завет; если же вы укроетесь в города ваши, то пошлю на вас язву, и преданы будете в руки врага;
\vs Lev 26:26 хлеб, подкрепляющий \bibemph{человека}, истреблю у вас; десять женщин будут печь хлеб ваш в одной печи и будут отдавать хлеб ваш весом; вы будете есть и не будете сыты.
\vs Lev 26:27 Если же и после сего не послушаете Меня и пойдете против Меня,
\vs Lev 26:28 то и Я в ярости пойду против вас и накажу вас всемеро за грехи ваши,
\vs Lev 26:29 и будете есть плоть сынов ваших, и плоть дочерей ваших будете есть;
\vs Lev 26:30 разорю высоты ваши и разрушу столбы ваши, и повергну трупы ваши на обломки идолов ваших, и возгнушается душа Моя вами;
\vs Lev 26:31 города ваши сделаю пустынею, и опустошу святилища ваши, и не буду обонять приятного благоухания [жертв] ваших;
\vs Lev 26:32 опустошу землю [вашу], так что изумятся о ней враги ваши, поселившиеся на ней;
\vs Lev 26:33 а вас рассею между народами и обнажу вслед вас меч, и будет земля ваша пуста и города ваши разрушены.
\vs Lev 26:34 Тогда удовлетворит себя земля за субботы свои во все дни запустения [своего]; когда вы будете в земле врагов ваших, тогда будет покоиться земля и удовлетворит себя за субботы свои;
\vs Lev 26:35 во все дни запустения [своего] будет она покоиться, сколько не покоилась в субботы ваши, когда вы жили на ней.
\vs Lev 26:36 Оставшимся из вас пошлю в сердца робость в земле врагов их, и шум колеблющегося листа погонит их, и побегут, как от меча, и падут, когда никто не преследует,
\vs Lev 26:37 и споткнутся друг на друга, как от меча, между тем как никто не преследует, и не будет у вас силы противостоять врагам вашим;
\vs Lev 26:38 и погибнете между народами, и пожрет вас земля врагов ваших;
\vs Lev 26:39 а оставшиеся из вас исчахнут за свои беззакония в землях врагов ваших и за беззакония отцов своих исчахнут;
\vs Lev 26:40 тогда признаются они в беззаконии своем и в беззаконии отцов своих, как они совершали преступления против Меня и шли против Меня,
\vs Lev 26:41 \bibemph{за что} и Я [в ярости] шел против них и ввел их в землю врагов их; тогда покорится необрезанное сердце их, и тогда потерпят они за беззакония свои.
\vs Lev 26:42 И Я вспомню завет Мой с Иаковом и завет Мой с Исааком, и завет Мой с Авраамом вспомню, и землю вспомню;
\vs Lev 26:43 тогда как земля оставлена будет ими и будет удовлетворять себя за субботы свои, опустев от них, и они будут терпеть за свое беззаконие, за то, что презирали законы Мои и душа их гнушалась постановлениями Моими,
\vs Lev 26:44 и тогда как они будут в земле врагов их,~--- Я не презрю их и не возгнушаюсь ими до того, чтоб истребить их, чтоб разрушить завет Мой с ними, ибо Я Господь, Бог их;
\vs Lev 26:45 вспомню для них завет с предками, которых вывел Я из земли Египетской пред глазами народов, чтоб быть их Богом. Я Господь.
\rsbpar\vs Lev 26:46 Вот постановления и определения и законы, которые постановил Господь между Собою и между сынами Израилевыми на горе Синае, чрез Моисея.
\vs Lev 27:1 И сказал Господь Моисею, говоря:
\vs Lev 27:2 объяви сынам Израилевым и скажи им: если кто дает обет посвятить душу Господу по оценке твоей,
\vs Lev 27:3 то оценка твоя мужчине от двадцати лет до шестидесяти должна быть пятьдесят сиклей серебряных, по сиклю священному;
\vs Lev 27:4 если же это женщина, то оценка твоя должна быть тридцать сиклей;
\vs Lev 27:5 от пяти лет до двадцати оценка твоя мужчине должна быть двадцать сиклей, а женщине десять сиклей;
\vs Lev 27:6 а от месяца до пяти лет оценка твоя мужчине должна быть пять сиклей серебра, а женщине оценка твоя три сикля серебра;
\vs Lev 27:7 от шестидесяти лет и выше мужчине оценка твоя должна быть пятнадцать сиклей серебра, а женщине десять сиклей.
\vs Lev 27:8 Если же он беден и не в силах \bibemph{отдать} по оценке твоей, то пусть представят его священнику, и священник пусть оценит его: соразмерно с состоянием давшего обет пусть оценит его священник.
\vs Lev 27:9 Если же то будет скот, который приносят в жертву Господу, то все, что дано Господу, должно быть свято:
\vs Lev 27:10 не должно выменивать его и заменять хорошее худым, или худое хорошим; если же станет кто заменять скотину скотиною, то и она и замен ее будет святынею.
\vs Lev 27:11 Если же то будет какая-нибудь скотина нечистая, которую не приносят в жертву Господу, то должно представить скотину священнику,
\vs Lev 27:12 и священник оценит ее, хороша ли она, или худа, и как оценит священник, так и должно быть;
\vs Lev 27:13 если же кто хочет выкупить ее, то пусть прибавит пятую долю к оценке твоей.
\vs Lev 27:14 Если кто посвящает дом свой в святыню Господу, то священник должен оценить его, хорош ли он, или худ, и как оценит его священник, так и состоится;
\vs Lev 27:15 если же посвятивший захочет выкупить дом свой, то пусть прибавит пятую часть серебра оценки твоей, и \bibemph{тогда} будет его.
\vs Lev 27:16 Если поле из своего владения посвятит кто Господу, то оценка твоя должна быть по мере посева: за посев хомера ячменя пятьдесят сиклей серебра;
\vs Lev 27:17 если от юбилейного года посвящает кто поле свое,~--- должно состояться по оценке твоей;
\vs Lev 27:18 если же после юбилея посвящает кто поле свое, то священник должен рассчитать серебро по мере лет, оставшихся до юбилейного года, и должно убавить из оценки твоей;
\vs Lev 27:19 если же захочет выкупить поле посвятивший его, то пусть он прибавит пятую часть серебра оценки твоей, и оно останется за ним;
\vs Lev 27:20 если же он не выкупит поля, и будет продано поле другому человеку, то уже нельзя выкупить:
\vs Lev 27:21 поле то, когда оно в юбилей отойдет, будет святынею Господу, как бы поле заклятое; священнику достанется оно во владение.
\vs Lev 27:22 А если кто посвятит Господу поле купленное, которое не из полей его владения,
\vs Lev 27:23 то священник должен рассчитать ему количество оценки до юбилейного года, и должен он отдать по расчету в тот же день, \bibemph{как} святыню Господню;
\vs Lev 27:24 поле же в юбилейный год перейдет опять к тому, у кого куплено, кому принадлежит владение той земли.
\vs Lev 27:25 Всякая оценка твоя должна быть по сиклю священному, двадцать гер должно быть в сикле.
\vs Lev 27:26 Только первенцев из скота, которые по первенству принадлежат Господу, не должен никто посвящать: вол ли то, или мелкий скот,~--- Господни они.
\vs Lev 27:27 Если же скот нечистый, то должно выкупить по оценке твоей и приложить к тому пятую часть; если не выкупят, то должно продать по оценке твоей.
\vs Lev 27:28 Только все заклятое, что под заклятием отдает человек Господу из своей собственности,~--- человека ли, скотину ли, поле ли своего владения,~--- не продается и не выкупается: все заклятое есть великая святыня Господня;
\vs Lev 27:29 все заклятое, что заклято от людей, не выкупается: оно должно быть предано смерти.
\vs Lev 27:30 И всякая десятина на земле из семян земли и из плодов дерева принадлежит Господу: это святыня Господня;
\vs Lev 27:31 если же кто захочет выкупить десятину свою, то пусть приложит к \bibemph{цене} ее пятую долю.
\vs Lev 27:32 И всякую десятину из крупного и мелкого скота, из всего, что проходит под жезлом десятое, должно посвящать Господу;
\vs Lev 27:33 не должно разбирать, хорошее ли то, или худое, и не должно заменять его; если же кто заменит его, то и само оно и замен его будет святынею и не может быть выкуплено.
\rsbpar\vs Lev 27:34 Вот заповеди, которые заповедал Господь Моисею для сынов Израилевых на горе Синае.

\include{tex/Num}
\bibbookdescr{Deu}{
  inline={\LARGE Пятая книга Моисеева\\\Huge Второзаконие},
  toc={Второзаконие},
  bookmark={Второзаконие},
  header={Второзаконие},
  %headerleft={},
  %headerright={},
  abbr={Втор}
}
\vs Deu 1:1 Сии суть слова, которые говорил Моисей всем Израильтянам за Иорданом в пустыне на равнине против Суфа, между Фараном и Тофелом, и Лаваном, и Асирофом, и Дизагавом,
\vs Deu 1:2 в расстоянии одиннадцати дней пути от Хорива, по дороге от горы Сеир к Кадес-Варни.
\vs Deu 1:3 Сорокового года, одиннадцатого месяца, в первый [день] месяца говорил Моисей [всем] сынам Израилевым все, что заповедал ему Господь о них.
\vs Deu 1:4 По убиении им Сигона, царя Аморрейского, который жил в Есевоне, и Ога, царя Васанского, который жил в Аштерофе в Едреи,
\vs Deu 1:5 за Иорданом, в земле Моавитской, начал Моисей изъяснять закон сей и сказал:
\vs Deu 1:6 Господь, Бог наш, говорил нам в Хориве и сказал: <<полно вам жить на горе сей!
\vs Deu 1:7 обратитесь, отправьтесь в путь и пойдите на гору Аморреев и ко всем соседям их, на равнину, на гору, на низкие места и на южный край и к берегам моря, в землю Ханаанскую и к Ливану, даже до реки великой, реки Евфрата;
\vs Deu 1:8 вот, Я даю вам землю сию, пойдите, возьмите в наследие землю, которую Господь с клятвою обещал дать отцам вашим, Аврааму, Исааку и Иакову, им и потомству их>>.
\vs Deu 1:9 И я сказал вам в то время: не могу один водить вас;
\vs Deu 1:10 Господь, Бог ваш, размножил вас, и вот, вы ныне многочисленны, как звезды небесные;
\vs Deu 1:11 Господь, Бог отцов ваших, да умножит вас в тысячу крат против того, сколько вас \bibemph{теперь}, и да благословит вас, как Он говорил вам:
\vs Deu 1:12 как же мне одному носить тягости ваши, бремена ваши и распри ваши?
\vs Deu 1:13 изберите себе по коленам вашим мужей мудрых, разумных и испытанных, и я поставлю их начальниками вашими.
\vs Deu 1:14 Вы отвечали мне и сказали: хорошее дело велишь ты сделать.
\vs Deu 1:15 И взял я главных из колен ваших, мужей мудрых, [разумных] и испытанных, и сделал их начальниками над вами, тысяченачальниками, стоначальниками, пятидесятиначальниками, десятиначальниками и надзирателями по коленам вашим.
\vs Deu 1:16 И дал я повеление судьям вашим в то время, говоря: выслушивайте братьев ваших и судите справедливо, как брата с братом, так и пришельца его;
\vs Deu 1:17 не различайте лиц на суде, как малого, так и великого выслушивайте: не бойтесь лица человеческого, ибо суд~--- дело Божие; а дело, которое для вас трудно, доводите до меня, и я выслушаю его.
\vs Deu 1:18 И дал я вам в то время повеления обо всем, что надлежит вам делать.
\vs Deu 1:19 И отправились мы от Хорива, и шли по всей этой великой и страшной пустыне, которую вы видели, по пути к горе Аморрейской, как повелел Господь, Бог наш, и пришли в Кадес-Варни.
\vs Deu 1:20 И сказал я вам: вы пришли к горе Аморрейской, которую Господь, Бог наш, дает нам;
\vs Deu 1:21 вот, Господь, Бог твой, отдает тебе землю сию, иди, возьми ее во владение, как говорил тебе Господь, Бог отцов твоих, не бойся и не ужасайся.
\vs Deu 1:22 Но вы все подошли ко мне и сказали: пошлем пред собою людей, чтоб они исследовали нам землю и принесли нам известие о дороге, по которой идти нам, и о городах, в которые идти нам.
\vs Deu 1:23 Слово это мне понравилось, и я взял из вас двенадцать человек, по одному человеку от [каждого] колена.
\vs Deu 1:24 Они пошли, взошли на гору и дошли до долины Есхол, и обозрели ее;
\vs Deu 1:25 и взяли в руки свои плодов земли и доставили нам, и принесли нам известие и сказали: хороша земля, которую Господь, Бог наш, дает нам.
\vs Deu 1:26 Но вы не захотели идти и воспротивились повелению Господа, Бога вашего,
\vs Deu 1:27 и роптали в шатрах ваших и говорили: Господь, по ненависти к нам, вывел нас из земли Египетской, чтоб отдать нас в руки Аморреев \bibemph{и} истребить нас;
\vs Deu 1:28 куда мы пойдем? братья наши расслабили сердце наше, говоря: народ тот более, [многочисленнее] и выше нас, города \bibemph{там} большие и с укреплениями до небес, да и сынов Енаковых видели мы там.
\vs Deu 1:29 И я сказал вам: не страшитесь и не бойтесь их;
\vs Deu 1:30 Господь, Бог ваш, идет перед вами; Он будет сражаться за вас, как Он сделал с вами в Египте, пред глазами вашими,
\vs Deu 1:31 и в пустыне сей, где, как ты видел, Господь, Бог твой, носил тебя, как человек носит сына своего, на всем пути, которым вы проходили, до пришествия вашего на сие место.
\vs Deu 1:32 Но и при этом вы не верили Господу, Богу вашему,
\vs Deu 1:33 Который шел перед вами путем~--- искать вам места, где остановиться вам, ночью в огне, чтобы указывать вам дорогу, по которой идти, а днем в облаке.
\vs Deu 1:34 И Господь [Бог] услышал слова ваши, и разгневался, и поклялся, говоря:
\vs Deu 1:35 никто из людей сих, из сего злого рода, не увидит доброй земли, которую Я клялся дать отцам вашим;
\vs Deu 1:36 только Халев, сын Иефонниин, увидит ее; ему дам Я землю, по которой он проходил, и сынам его, за то, что он повиновался Господу.
\vs Deu 1:37 И на меня прогневался Господь за вас, говоря: и ты не войдешь туда;
\vs Deu 1:38 Иисус, сын Навин, который при тебе, он войдет туда; его утверди, ибо он введет Израиля во владение ею;
\vs Deu 1:39 дети ваши, о которых вы говорили, что они достанутся в добычу \bibemph{врагам}, и сыновья ваши, которые не знают ныне ни добра ни зла, они войдут туда, им дам ее, и они овладеют ею;
\vs Deu 1:40 а вы обратитесь и отправьтесь в пустыню по дороге к Чермному морю.
\vs Deu 1:41 И вы отвечали тогда и сказали мне: согрешили мы пред Господом, [Богом нашим,] пойдем и сразимся, как повелел нам Господь, Бог наш. И препоясались вы, каждый ратным оружием своим, и безрассудно решились взойти на гору.
\vs Deu 1:42 Но Господь сказал мне: скажи им: не всходите и не сражайтесь, потому что нет Меня среди вас, чтобы не поразили вас враги ваши.
\vs Deu 1:43 И я говорил вам, но вы не послушали и воспротивились повелению Господню и по упорству своему взошли на гору.
\vs Deu 1:44 И выступил против вас Аморрей, живший на горе той, и преследовали вас так, как делают пчелы, и поражали вас на Сеире до самой Хормы.
\vs Deu 1:45 И возвратились вы и плакали пред Господом: но Господь не услышал вопля вашего и не внял вам.
\vs Deu 1:46 И пробыли вы в Кадесе много времени, сколько времени вы \bibemph{там} были.
\vs Deu 2:1 И обратились мы и отправились в пустыню к Чермному морю, как говорил мне Господь, и много времени ходили вокруг горы Сеира.
\vs Deu 2:2 И сказал мне Господь, говоря:
\vs Deu 2:3 полно вам ходить вокруг этой горы, обратитесь к северу;
\vs Deu 2:4 и народу дай повеление и скажи: вы будете проходить пределы братьев ваших, сынов Исавовых, живущих на Сеире, и они убоятся вас; но остерегайтесь
\vs Deu 2:5 начинать с ними войну, ибо Я не дам вам земли их ни на стопу ноги, потому что гору Сеир Я дал во владение Исаву;
\vs Deu 2:6 пищу покупайте у них за серебро и ешьте; и воду покупайте у них за серебро и пейте;
\vs Deu 2:7 ибо Господь, Бог твой, благословил тебя во всяком деле рук твоих, покровительствовал \bibemph{тебе} во время путешествия твоего по великой [и страшной] пустыне сей; вот, сорок лет Господь, Бог твой, с тобою; ты ни в чем не терпел недостатка.
\vs Deu 2:8 И шли мы мимо братьев наших, сынов Исавовых, живущих на Сеире, путем равнины, от Елафа и Ецион-Гавера, и поворотили, и шли к пустыне Моава.
\vs Deu 2:9 И сказал мне Господь: не вступай во вражду с Моавом и не начинай с ними войны; ибо Я не дам тебе ничего от земли его во владение, потому что Ар отдал Я во владение сынам Лотовым;
\vs Deu 2:10 прежде жили там Эмимы, народ великий, многочисленный и высокий, как \bibemph{сыны} Енаковы,
\vs Deu 2:11 и они считались между Рефаимами, как \bibemph{сыны} Енаковы; Моавитяне же называют их Эмимами;
\vs Deu 2:12 а на Сеире жили прежде Хорреи; но сыны Исавовы прогнали их и истребили их от лица своего и поселились вместо их~--- так, как поступил Израиль с землею наследия своего, которую дал им Господь;
\vs Deu 2:13 итак встаньте и пройдите долину Заред. И прошли мы долину Заред.
\vs Deu 2:14 С тех пор, как мы пошли в Кадес-Варни и как прошли долину Заред, \bibemph{минуло} тридцать восемь лет, и у нас перевелся из среды стана весь род ходящих на войну, как клялся им Господь [Бог];
\vs Deu 2:15 да и рука Господня была на них, чтоб истреблять их из среды стана, пока не вымерли.
\vs Deu 2:16 Когда же перевелись все ходящие на войну и вымерли из среды народа,
\vs Deu 2:17 тогда сказал мне Господь, говоря:
\vs Deu 2:18 ты проходишь ныне мимо пределов Моава, мимо Ара,
\vs Deu 2:19 и приблизился к Аммонитянам; не вступай с ними во вражду, и не начинай с ними войны, ибо Я не дам тебе ничего от земли сынов Аммоновых во владение, потому что Я отдал ее во владение сынам Лотовым;
\vs Deu 2:20 и она считалась землею Рефаимов; прежде жили на ней Рефаимы; Аммонитяне же называют их Замзумимами;
\vs Deu 2:21 народ великий, многочисленный и высокий, как \bibemph{сыны} Енаковы, и истребил их Господь пред лицем их, и изгнали они их и поселились на месте их,
\vs Deu 2:22 как Он сделал для сынов Исавовых, живущих на Сеире, истребив пред лицем их Хорреев, и они изгнали их, и поселились на месте их, и \bibemph{живут} до сего дня;
\vs Deu 2:23 и Аввеев, живших в селениях до самой Газы, Кафторимы, исшедшие из Кафтора, истребили и поселились на месте их.
\vs Deu 2:24 Встаньте, отправьтесь и перейдите поток Арнон; вот, Я предаю в руку твою Сигона, царя Есевонского, Аморреянина, и землю его; начинай овладевать ею, и веди с ним войну;
\vs Deu 2:25 с сего дня Я начну распространять страх и ужас пред тобою на народы под всем небом; те, которые услышат о тебе, вострепещут и ужаснутся тебя.
\vs Deu 2:26 И послал я послов из пустыни Кедемоф к Сигону, царю Есевонскому, с словами мирными, чтобы сказать:
\vs Deu 2:27 позволь пройти мне землею твоею; я пойду дорогою, не сойду ни направо, ни налево;
\vs Deu 2:28 пищу продавай мне за серебро, и я буду есть, и воду для питья давай мне за серебро, и я буду пить, только ногами моими пройду~---
\vs Deu 2:29 так, как сделали мне сыны Исава, живущие на Сеире, и Моавитяне, живущие в Аре, доколе не перейду чрез Иордан в землю, которую Господь, Бог наш, дает нам.
\vs Deu 2:30 Но Сигон, царь Есевонский, не согласился позволить пройти нам через свою \bibemph{землю}, потому что Господь, Бог твой, ожесточил дух его и сердце его сделал упорным, чтобы предать его в руку твою, как \bibemph{это видно} ныне.
\vs Deu 2:31 И сказал мне Господь: вот, Я начинаю предавать тебе Сигона [царя Есевонского, Аморреянина,] и землю его; начинай овладевать землею его.
\vs Deu 2:32 И Сигон [царь Есевонский] со всем народом своим выступил против нас на сражение к Иааце;
\vs Deu 2:33 и предал его Господь, Бог наш, [в руки наши,] и мы поразили его и сынов его и весь народ его,
\vs Deu 2:34 и взяли в то время все города его, и предали заклятию все города, мужчин и женщин и детей, не оставили никого в живых;
\vs Deu 2:35 только взяли мы себе в добычу скот их и захваченное во взятых нами городах.
\vs Deu 2:36 От Ароера, который на берегу потока Арнона, и от города, который на долине, до [горы] Галаада не было города, который был бы неприступен для нас: всё предал Господь, Бог наш, [в руки наши].
\vs Deu 2:37 Только к земле Аммонитян ты не подходил, ни к \bibemph{местам} [лежащим] близ потока Иавока, ни к городам [которые] на горе, ни ко всему, к чему не повелел [нам] Господь, Бог наш.
\vs Deu 3:1 И обратились мы оттуда, и шли к Васану, и выступил против нас на войну Ог, царь Васанский, со всем народом своим, при Едреи.
\vs Deu 3:2 И сказал мне Господь: не бойся его, ибо Я отдам в руку твою его, и весь народ его, и всю землю его, и ты поступишь с ним так, как поступил с Сигоном, царем Аморрейским, который жил в Есевоне.
\vs Deu 3:3 И предал Господь, Бог наш, в руки наши и Ога, царя Васанского, и весь народ его; и мы поразили его, так что никого не осталось у него в живых;
\vs Deu 3:4 и взяли мы в то время все города его; не было города, которого мы не взяли бы у них: шестьдесят городов, всю область Аргов, царство Ога Васанского;
\vs Deu 3:5 все эти города укреплены были высокими стенами, воротами и запорами, кроме городов неукрепленных, весьма многих;
\vs Deu 3:6 и предали мы их заклятию, как поступили с Сигоном, царем Есевонским, предав заклятию всякий город с мужчинами, женщинами и детьми;
\vs Deu 3:7 но весь скот и захваченное в городах взяли себе в добычу.
\vs Deu 3:8 И взяли мы в то время из руки двух царей Аморрейских землю сию, которая по эту сторону Иордана, от потока Арнона до горы Ермона,~---
\vs Deu 3:9 Сидоняне Ермон называют Сирионом, а Аморреи называют его Сениром,~---
\vs Deu 3:10 все города на равнине, весь Галаад и весь Васан до Салхи и Едреи, город\acc{а} царства Ога Васанского;
\vs Deu 3:11 ибо только Ог, царь Васанский, оставался из Рефаимов. Вот, одр его, одр железный, и теперь в Равве, у сынов Аммоновых: длина его девять локтей, а ширина его четыре локтя, локтей мужеских.
\vs Deu 3:12 Землю сию взяли мы в то время начиная от Ароера, который у потока Арнона; и половину горы Галаада с городами ее отдал я \bibemph{колену} Рувимову и Гадову;
\vs Deu 3:13 а остаток Галаада и весь Васан, царство Ога, отдал я половине колена Манассиина, всю область Аргов со всем Васаном. [Она называется землею Рефаимов.]
\vs Deu 3:14 Иаир, сын Манассиин, взял всю область Аргов, до пределов Гесурских и Маахских, и назвал Васан, по имени своему, селениями Иаировыми, что и доныне;
\vs Deu 3:15 Махиру дал я Галаад;
\vs Deu 3:16 а \bibemph{колену} Рувимову и Гадову дал от Галаада до потока Арнона, \bibemph{землю} между потоком и пределом, до потока Иавока, предела сынов Аммоновых,
\vs Deu 3:17 также равнину и Иордан, \bibemph{который есть} и предел, от Киннерефа до моря равнины, моря Соленого, при подошве \bibemph{горы} Фасги к востоку.
\vs Deu 3:18 И дал я вам в то время повеление, говоря: Господь, Бог ваш, дал вам землю сию во владение; все способные к войне, вооружившись, идите впереди братьев ваших, сынов Израилевых;
\vs Deu 3:19 только жены ваши и дети ваши и скот ваш [\bibemph{ибо} я знаю, что скота у вас много,] пусть останутся в городах ваших, которые я дал вам,
\vs Deu 3:20 доколе Господь [Бог] не даст покоя братьям вашим, как вам, и доколе и они не получат во владение землю, которую Господь, Бог ваш, дает им за Иорданом; тогда возвратитесь каждый в свое владение, которое я дал вам.
\vs Deu 3:21 И Иисусу заповедал я в то время, говоря: глаза твои видели все, что сделал Господь, Бог ваш, с двумя царями сими; то же сделает Господь со всеми царствами, которые ты будешь проходить;
\vs Deu 3:22 не бойтесь их, ибо Господь, Бог ваш, Сам сражается за вас.
\vs Deu 3:23 И молился я Господу в то время, говоря:
\vs Deu 3:24 Владыко Господи, Ты начал показывать рабу Твоему величие Твое [и силу Твою,] и крепкую руку Твою [и высокую мышцу]; ибо какой бог есть на небе, или на земле, который мог бы делать такие дела, как Твои, и с могуществом таким, как Твое?
\vs Deu 3:25 дай мне перейти и увидеть ту добрую землю, которая за Иорданом, и ту прекрасную гору и Ливан.
\vs Deu 3:26 Но Господь гневался на меня за вас и не послушал меня, и сказал мне Господь: полно тебе, впредь не говори Мне более об этом;
\vs Deu 3:27 взойди на вершину Фасги и взгляни глазами твоими к морю и к северу, и к югу и к востоку, и посмотри глазами твоими, потому что ты не перейдешь за Иордан сей;
\vs Deu 3:28 и дай наставление Иисусу, и укрепи его, и утверди его; ибо он будет предшествовать народу сему и он разделит им на уделы [всю] землю, на которую ты посмотришь.
\vs Deu 3:29 И остановились мы на долине, напротив Беф-Фегора.
\vs Deu 4:1 Итак, Израиль, слушай постановления и законы, которые я [сегодня] научаю вас исполнять, дабы вы были живы [и размножились], и пошли и наследовали ту землю, которую Господь, Бог отцов ваших, дает вам [в наследие];
\vs Deu 4:2 не прибавляйте к тому, что я заповедую вам, и не убавляйте от того; соблюдайте заповеди Господа, Бога вашего, которые я вам [сегодня] заповедую.
\vs Deu 4:3 Глаза ваши видели [все], что сделал Господь [Бог наш] с Ваал-Фегором: всякого человека, последовавшего Ваал-Фегору, истребил Господь, Бог твой, из среды тебя;
\vs Deu 4:4 а вы, прилепившиеся к Господу, Богу вашему, живы все доныне.
\vs Deu 4:5 Вот, я научил вас постановлениям и законам, как повелел мне Господь, Бог мой, дабы вы так поступали в той земле, в которую вы вступаете, чтоб овладеть ею;
\vs Deu 4:6 итак храните и исполняйте их, ибо в этом мудрость ваша и разум ваш пред глазами народов, которые, услышав о всех сих постановлениях, скажут: только этот великий народ есть народ мудрый и разумный.
\vs Deu 4:7 Ибо есть ли какой великий народ, к которому боги \bibemph{его} были бы столь близки, как близок к нам Господь, Бог наш, когда ни призовем Его?
\vs Deu 4:8 и есть ли какой великий народ, у которого были бы такие справедливые постановления и законы, как весь закон сей, который я предлагаю вам сегодня?
\vs Deu 4:9 Только берегись и тщательно храни душу твою, чтобы тебе не забыть тех дел, которые видели глаза твои, и чтобы они не выходили из сердца твоего во все дни жизни твоей; и поведай о них сынам твоим и сынам сынов твоих,~---
\vs Deu 4:10 о том дне, когда ты стоял пред Господом, Богом твоим, при Хориве, [в день собрания,] и когда сказал Господь мне: собери ко Мне народ, и Я возвещу им слова Мои, из которых они научатся бояться Меня во все дни жизни своей на земле и научат сыновей своих.
\vs Deu 4:11 Вы приблизились и стали под горою, а гора горела огнем до самых небес, \bibemph{и была} тьма, облако и мрак.
\vs Deu 4:12 И говорил Господь к вам [на горе] из среды огня; глас слов [Его] вы слышали, но образа не видели, а только глас;
\vs Deu 4:13 и объявил Он вам завет Свой, который повелел вам исполнять, десятословие, и написал его на двух каменных скрижалях;
\vs Deu 4:14 и повелел мне Господь в то время научить вас постановлениям и законам, дабы вы исполняли их в той земле, в которую вы входите, чтоб овладеть ею.
\vs Deu 4:15 Твердо держите в душах ваших, что вы не видели никакого образа в тот день, когда говорил к вам Господь на [горе] Хориве из среды огня,
\vs Deu 4:16 дабы вы не развратились и не сделали себе изваяний, изображений какого-либо кумира, представляющих мужчину или женщину,
\vs Deu 4:17 изображения какого-либо скота, который на земле, изображения какой-либо птицы крылатой, которая летает под небесами,
\vs Deu 4:18 изображения какого-либо [гада,] ползающего по земле, изображения какой-либо рыбы, которая в водах ниже земли;
\vs Deu 4:19 и дабы ты, взглянув на небо и увидев солнце, луну и звезды [и] все воинство небесное, не прельстился и не поклонился им и не служил им, так как Господь, Бог твой, уделил их всем народам под всем небом.
\vs Deu 4:20 А вас взял Господь [Бог] и вывел вас из печи железной, из Египта, дабы вы были народом Его удела, как это ныне \bibemph{видно}.
\vs Deu 4:21 И Господь [Бог] прогневался на меня за вас, и клялся, что я не перейду за Иордан и не войду в ту добрую землю, которую Господь, Бог твой, дает тебе в удел;
\vs Deu 4:22 я умру в сей земле, не перейдя за Иордан, а вы перейдете и овладеете тою доброю землею.
\vs Deu 4:23 Берегитесь, чтобы не забыть вам завета Господа, Бога вашего, который Он поставил с вами, и чтобы не делать себе кумиров, изображающих что-либо, как повелел тебе Господь, Бог твой;
\vs Deu 4:24 ибо Господь, Бог твой, есть огнь поядающий, Бог ревнитель.
\vs Deu 4:25 Если же родятся у тебя сыны и сыны у сынов [твоих], и, долго жив на земле, вы развратитесь и сделаете изваяние, изображающее что-либо, и сделаете зло сие пред очами Господа, Бога вашего, и раздражите Его,
\vs Deu 4:26 то свидетельствуюсь вам сегодня небом и землею, что скоро потеряете землю, для наследования которой вы переходите за Иордан; не пробудете много времени на ней, но погибнете;
\vs Deu 4:27 и рассеет вас Господь по [всем] народам, и останетесь в малом числе между народами, к которым отведет вас Господь;
\vs Deu 4:28 и будете там служить [другим] богам, сделанным руками человеческими из дерева и камня, которые не видят и не слышат, и не едят и не обоняют.
\vs Deu 4:29 Но когда ты взыщешь там Господа, Бога твоего, то найдешь [Его], если будешь искать Его всем сердцем твоим и всею душею твоею.
\vs Deu 4:30 Когда ты будешь в скорби, и когда все это постигнет тебя в последствие времени, то обратишься к Господу, Богу твоему, и послушаешь гласа Его.
\vs Deu 4:31 Господь, Бог твой, есть Бог [благий и] милосердый; Он не оставит тебя и не погубит тебя, и не забудет завета с отцами твоими, который Он клятвою утвердил им.
\vs Deu 4:32 Ибо спроси у времен прежних, бывших прежде тебя, с того дня, в который сотворил Бог человека на земле, и от края неба до края неба: бывало ли что-нибудь такое, как сие великое дело, или слыхано ли подобное сему?
\vs Deu 4:33 слышал ли [какой] народ глас Бога [живаго], говорящего из среды огня, и остался жив, как слышал ты?
\vs Deu 4:34 или покушался ли \bibemph{какой} бог пойти, взять себе народ из среды \bibemph{другого} народа казнями, знамениями и чудесами, и войною, и рукою крепкою, и мышцею высокою, и великими ужасами, как сделал для вас Господь, Бог ваш, в Египте пред глазами твоими?
\vs Deu 4:35 Тебе дано видеть \bibemph{это}, чтобы ты знал, что \bibemph{только} Господь [Бог твой] есть Бог, [и] нет еще кроме Его;
\vs Deu 4:36 с неба дал Он слышать тебе глас Свой, дабы научить тебя, и на земле показал тебе великий огнь Свой, и ты слышал слова Его из среды огня;
\vs Deu 4:37 и так как Он возлюбил отцов твоих и избрал [вас,] потомство их после них, то и вывел тебя Сам великою силою Своею из Египта,
\vs Deu 4:38 чтобы прогнать от лица твоего народы, которые больше и сильнее тебя, \bibemph{и} ввести тебя \bibemph{и} дать тебе землю их в удел, как это ныне \bibemph{видно}.
\vs Deu 4:39 Итак знай ныне и положи на сердце твое, что Господь [Бог твой] есть Бог на небе вверху и на земле внизу, [и] нет еще [кроме Его];
\vs Deu 4:40 и храни постановления Его и заповеди Его, которые я заповедую тебе ныне, чтобы хорошо было тебе и сынам твоим после тебя, и чтобы ты много времени пробыл на той земле, которую Господь, Бог твой, дает тебе навсегда.
\rsbpar\vs Deu 4:41 Тогда отделил Моисей три города по эту сторону Иордана на восток солнца,
\vs Deu 4:42 чтоб убегал туда убийца, который убьет ближнего своего без намерения, не быв врагом ему ни вчера, ни третьего дня, \bibemph{и} чтоб, убежав в один из этих городов, остался жив:
\vs Deu 4:43 Бецер в пустыне, на равнине в \bibemph{колене} Рувимовом, и Рамоф в Галааде в \bibemph{колене} Гадовом, и Голан в Васане в \bibemph{колене} Манассиином.
\rsbpar\vs Deu 4:44 Вот закон, который предложил Моисей сынам Израилевым;
\vs Deu 4:45 вот повеления, постановления и уставы, которые изрек Моисей сынам Израилевым [в пустыне], по исшествии их из Египта,
\vs Deu 4:46 за Иорданом, на долине против Беф-Фегора, в земле Сигона, царя Аморрейского, жившего в Есевоне, которого поразил Моисей с сынами Израилевыми, по исшествии их из Египта.
\vs Deu 4:47 И овладели они землею его и землею Ога, царя Васанского, двух царей Аморрейских, которая за Иорданом к востоку солнца,
\vs Deu 4:48 \bibemph{начиная} от Ароера, который \bibemph{лежит} на берегу потока Арнона, до горы Сиона, она же Ермон,
\vs Deu 4:49 и всею равниною по эту сторону Иордана к востоку, до самого моря равнины при подошве Фасги.
\vs Deu 5:1 И созвал Моисей весь Израиль и сказал им: слушай, Израиль, постановления и законы, которые я изреку сегодня в уши ваши, и выучите их и старайтесь исполнять их.
\vs Deu 5:2 Господь, Бог наш, поставил с нами завет на Хориве;
\vs Deu 5:3 не с отцами нашими поставил Господь завет сей, но с нами, \bibemph{которые} здесь сегодня все живы.
\vs Deu 5:4 Лицем к лицу говорил Господь с вами на горе из среды огня;
\vs Deu 5:5 я же стоял между Господом и между вами в то время, дабы пересказывать вам слово Господа, ибо вы боялись огня и не восходили на гору. Он \bibemph{тогда} сказал:
\rsbpar\vs Deu 5:6 Я Господь, Бог твой, Который вывел тебя из земли Египетской, из дома рабства;
\vs Deu 5:7 да не будет у тебя других богов перед лицем Моим.
\rsbpar\vs Deu 5:8 Не делай себе кумира и никакого изображения того, что на небе вверху и что на земле внизу, и что в водах ниже земли,
\vs Deu 5:9 не поклоняйся им и не служи им; ибо Я Господь, Бог твой, Бог ревнитель, за вину отцов наказывающий детей до третьего и четвертого рода, ненавидящих Меня,
\vs Deu 5:10 и творящий милость до тысячи \bibemph{родов} любящим Меня и соблюдающим заповеди Мои.
\rsbpar\vs Deu 5:11 Не произноси имени Господа, Бога твоего, напрасно; ибо не оставит Господь [Бог твой] без наказания того, кто употребляет имя Его напрасно.
\rsbpar\vs Deu 5:12 Наблюдай день субботний, чтобы свято хранить его, как заповедал тебе Господь, Бог твой;
\vs Deu 5:13 шесть дней работай и делай всякие дела твои,
\vs Deu 5:14 а день седьмой~--- суббота Господу, Богу твоему. Не делай [в оный] никакого дела, ни ты, ни сын твой, ни дочь твоя, ни раб твой, ни раба твоя, ни вол твой, ни осел твой, ни всякий скот твой, ни пришелец твой, который у тебя, чтобы отдохнул раб твой, и раба твоя [и осел твой,] как и ты;
\vs Deu 5:15 и помни, что [ты] был рабом в земле Египетской, но Господь, Бог твой, вывел тебя оттуда рукою крепкою и мышцею высокою, потому и повелел тебе Господь, Бог твой, соблюдать день субботний [и свято хранить его].
\rsbpar\vs Deu 5:16 Почитай отца твоего и матерь твою, как повелел тебе Господь, Бог твой, чтобы продлились дни твои, и чтобы хорошо тебе было на той земле, которую Господь, Бог твой, дает тебе.
\rsbpar\vs Deu 5:17 Не убивай.
\rsbpar\vs Deu 5:18 Не прелюбодействуй.
\rsbpar\vs Deu 5:19 Не кради.
\rsbpar\vs Deu 5:20 Не произноси ложного свидетельства на ближнего твоего.
\rsbpar\vs Deu 5:21 Не желай жены ближнего твоего и не желай дома ближнего твоего, ни поля его, ни раба его, ни рабы его, ни вола его, ни осла его, [ни всякого скота его,] ни всего, что есть у ближнего твоего.
\rsbpar\vs Deu 5:22 Слова сии изрек Господь ко всему собранию вашему на горе из среды огня, облака и мрака [и бури] громогласно, и более не говорил, и написал их на двух каменных скрижалях, и дал их мне.
\vs Deu 5:23 И когда вы услышали глас из среды мрака, и гора горела огнем, то вы подошли ко мне, все начальники колен ваших и старейшины ваши,
\vs Deu 5:24 и сказали: вот, показал нам Господь, Бог наш, славу Свою и величие Свое, и глас Его слышали мы из среды огня; сегодня видели мы, что Бог говорит с человеком, и сей остается жив;
\vs Deu 5:25 но теперь для чего нам умирать? ибо великий огонь сей пожрет нас; если мы еще услышим глас Господа, Бога нашего, то умрем,
\vs Deu 5:26 ибо есть ли какая плоть, которая слышала бы глас Бога живаго, говорящего из среды огня, как мы, и осталась жива?
\vs Deu 5:27 приступи ты и слушай все, что скажет [тебе] Господь, Бог наш, и ты пересказывай нам все, что будет говорить тебе Господь, Бог наш, и мы будем слушать и исполнять.
\vs Deu 5:28 И Господь услышал слова ваши, как вы разговаривали со мною, и сказал мне Господь: слышал Я слова народа сего, которые они говорили тебе; все, что ни говорили они, хорошо;
\vs Deu 5:29 о, если бы сердце их было у них таково, чтобы бояться Меня и соблюдать все заповеди Мои во все дни, дабы хорошо было им и сынам их вовек!
\vs Deu 5:30 пойди, скажи им: <<возвратитесь в шатры свои>>;
\vs Deu 5:31 а ты здесь останься со Мною, и Я изреку тебе все заповеди и постановления и законы, которым ты должен научить их, чтобы они [так] поступали на той земле, которую Я даю им во владение.
\vs Deu 5:32 Смотрите, поступайте так, как повелел вам Господь, Бог ваш; не уклоняйтесь ни направо, ни налево;
\vs Deu 5:33 ходите по тому пути, по которому повелел вам Господь, Бог ваш, дабы вы были живы, и хорошо было вам, и прожили много времени на той земле, которую получите во владение.
\vs Deu 6:1 Вот заповеди, постановления и законы, которым повелел Господь, Бог ваш, научить вас, чтобы вы поступали [так] в той земле, в которую вы идете, чтоб овладеть ею;
\vs Deu 6:2 дабы ты боялся Господа, Бога твоего, и все постановления Его и заповеди Его, которые [сегодня] заповедую тебе, соблюдал ты и сыны твои и сыны сынов твоих во все дни жизни твоей, дабы продлились дни твои.
\vs Deu 6:3 Итак слушай, Израиль, и старайся исполнить это, чтобы тебе хорошо было, и чтобы вы весьма размножились, как Господь, Бог отцов твоих, говорил тебе, [что Он даст тебе] землю, где течет молоко и мед. [Сии суть постановления и законы, которые заповедал Господь Бог сынам Израилевым в пустыне, по исшествии их из земли Египетской.]
\rsbpar\vs Deu 6:4 Слушай, Израиль: Господь, Бог наш, Господь един есть;
\vs Deu 6:5 и люби Господа, Бога твоего, всем сердцем твоим, и всею душею твоею и всеми силами твоими.
\vs Deu 6:6 И да будут слова сии, которые Я заповедую тебе сегодня, в сердце твоем [и в душе твоей];
\vs Deu 6:7 и внушай их детям твоим и говори о них, сидя в доме твоем и идя дорогою, и ложась и вставая;
\vs Deu 6:8 и навяжи их в знак на руку твою, и да будут они повязкою над глазами твоими,
\vs Deu 6:9 и напиши их на косяках дома твоего и на воротах твоих.
\vs Deu 6:10 Когда же введет тебя Господь, Бог твой, в ту землю, которую Он клялся отцам твоим, Аврааму, Исааку и Иакову, дать тебе с большими и хорошими городами, которых ты не строил,
\vs Deu 6:11 и с домами, наполненными всяким добром, которых ты не наполнял, и с колодезями, высеченными \bibemph{из камня}, которых ты не высекал, с виноградниками и маслинами, которых ты не садил, и будешь есть и насыщаться,
\vs Deu 6:12 тогда берегись, чтобы [не обольстилось сердце твое и] не забыл ты Господа, Который вывел тебя из земли Египетской, из дома рабства.
\rsbpar\vs Deu 6:13 Господа, Бога твоего, бойся, и Ему [одному] служи, [и к Нему прилепись,] и Его именем клянись.
\vs Deu 6:14 Не последуйте иным богам, богам тех народов, которые будут вокруг вас;
\vs Deu 6:15 ибо Господь, Бог твой, Который среди тебя, есть Бог ревнитель; чтобы не воспламенился гнев Господа, Бога твоего, на тебя, и не истребил Он тебя с лица земли.
\rsbpar\vs Deu 6:16 Не искушайте Господа, Бога вашего, как вы искушали Его в Массе.
\rsbpar\vs Deu 6:17 Твердо храните заповеди Господа, Бога вашего, и уставы Его и постановления, которые Он заповедал тебе;
\vs Deu 6:18 и делай справедливое и доброе пред очами Господа [Бога твоего], дабы хорошо тебе было, и дабы ты вошел и овладел доброю землею, которую Господь с клятвою обещал отцам твоим,
\vs Deu 6:19 и чтобы Он прогнал всех врагов твоих от лица твоего, как говорил Господь.
\vs Deu 6:20 Если спросит у тебя сын твой в последующее время, говоря: <<что \bibemph{значат} сии уставы, постановления и законы, которые заповедал вам Господь, Бог ваш?>>
\vs Deu 6:21 то скажи сыну твоему: <<рабами были мы у фараона в Египте, но Господь [Бог] вывел нас из Египта рукою крепкою [и мышцею высокою],
\vs Deu 6:22 и явил Господь [Бог] знамения и чудеса великие и казни над Египтом, над фараоном и над всем домом его [и над войском его] пред глазами нашими;
\vs Deu 6:23 а нас вывел оттуда [Господь, Бог наш,] чтобы ввести нас и дать нам землю, которую [Господь, Бог наш,] клялся отцам нашим [дать нам];
\vs Deu 6:24 и заповедал нам Господь исполнять все постановления сии, чтобы мы боялись Господа, Бога нашего, дабы хорошо было нам во все дни, дабы сохранить нашу жизнь, как и теперь;
\vs Deu 6:25 и в сем будет наша праведность, если мы будем стараться исполнять все сии заповеди [закона] пред лицем Господа, Бога нашего, как Он заповедал нам>>.
\vs Deu 7:1 Когда введет тебя Господь, Бог твой, в землю, в которую ты идешь, чтоб овладеть ею, и изгонит от лица твоего многочисленные народы, Хеттеев, Гергесеев, Аморреев, Хананеев, Ферезеев, Евеев и Иевусеев, семь народов, которые многочисленнее и сильнее тебя,
\vs Deu 7:2 и предаст их тебе Господь, Бог твой, и поразишь их, тогда предай их заклятию, не вступай с ними в союз и не щади их;
\vs Deu 7:3 и не вступай с ними в родство: дочери твоей не отдавай за сына его, и дочери его не бери за сына твоего;
\vs Deu 7:4 ибо они отвратят сынов твоих от Меня, чтобы служить иным богам, и \bibemph{тогда} воспламенится на вас гнев Господа, и Он скоро истребит тебя.
\vs Deu 7:5 Но поступите с ними так: жертвенники их разрушьте, столбы их сокрушите, и рощи их вырубите, и истуканов [богов] их сожгите огнем;
\vs Deu 7:6 ибо ты народ святый у Господа, Бога твоего: тебя избрал Господь, Бог твой, чтобы ты был собственным Его народом из всех народов, которые на земле.
\vs Deu 7:7 Не потому, чтобы вы были многочисленнее всех народов, принял вас Господь и избрал вас,~--- ибо вы малочисленнее всех народов,~---
\vs Deu 7:8 но потому, что любит вас Господь, и для того, чтобы сохранить клятву, которою Он клялся отцам вашим, вывел вас Господь рукою крепкою [и мышцею высокою] и освободил тебя из дома рабства, из руки фараона, царя Египетского.
\vs Deu 7:9 Итак знай, что Господь, Бог твой, есть Бог, Бог верный, Который хранит завет [Свой] и милость к любящим Его и сохраняющим заповеди Его до тысячи родов,
\vs Deu 7:10 и воздает ненавидящим Его в лице их, погубляя их; Он не замедлит, ненавидящему Его самому лично воздаст.
\rsbpar\vs Deu 7:11 Итак, соблюдай заповеди и постановления и законы, которые сегодня заповедую тебе исполнять.
\vs Deu 7:12 И если вы будете слушать законы сии и хранить и исполнять их, то Господь, Бог твой, будет хранить завет и милость к тебе, как Он клялся отцам твоим,
\vs Deu 7:13 и возлюбит тебя, и благословит тебя, и размножит тебя, и благословит плод чрева твоего и плод земли твоей, и хлеб твой, и вино твое, и елей твой, рождаемое от крупного скота твоего и от стада овец твоих, на той земле, которую Он клялся отцам твоим дать тебе;
\vs Deu 7:14 благословен ты будешь больше всех народов; не будет ни бесплодного, ни бесплодной, ни у тебя, ни в скоте твоем;
\vs Deu 7:15 и отдалит от тебя Господь [Бог твой] всякую немощь, и никаких лютых болезней Египетских, [которые ты видел и] которые ты знаешь, не наведет на тебя, но наведет их на всех, ненавидящих тебя;
\vs Deu 7:16 и истребишь все народы, которые Господь, Бог твой, дает тебе: да не пощадит их глаз твой; и не служи богам их, ибо это сеть для тебя.
\vs Deu 7:17 Если скажешь в сердце твоем: <<народы сии многочисленнее меня; как я могу изгнать их?>>
\vs Deu 7:18 Не бойся их, вспомни то, что сделал Господь, Бог твой, с фараоном и всем Египтом,
\vs Deu 7:19 те великие испытания, которые видели глаза твои, [великие] знамения, чудеса, и руку крепкую и мышцу высокую, с какими вывел тебя Господь, Бог твой; то же сделает Господь, Бог твой, со всеми народами, которых ты боишься;
\vs Deu 7:20 и шершней нашлет Господь, Бог твой, на них, доколе не погибнут оставшиеся и скрывшиеся от лица твоего;
\vs Deu 7:21 не страшись их, ибо Господь, Бог твой, среди тебя, Бог великий и страшный.
\vs Deu 7:22 И будет Господь, Бог твой, изгонять пред тобою народы сии мало-помалу; не можешь ты истребить их скоро, чтобы [земля не сделалась пуста и] не умножились против тебя полевые звери;
\vs Deu 7:23 но предаст их тебе Господь, Бог твой, и приведет их в великое смятение, так что они погибнут;
\vs Deu 7:24 и предаст царей их в руки твои, и ты истребишь имя их из поднебесной: не устоит никто против тебя, доколе не искоренишь их.
\vs Deu 7:25 Кумиры богов их сожгите огнем; не пожелай взять себе серебра или золота, которое на них, дабы это не было для тебя сетью, ибо это мерзость для Господа, Бога твоего;
\vs Deu 7:26 и не вноси мерзости в дом твой, дабы не подпасть заклятию, как она; отвращайся сего и гнушайся сего, ибо это заклятое.
\vs Deu 8:1 Все заповеди, которые я заповедую вам сегодня, старайтесь исполнять, дабы вы были живы и размножились, и пошли и завладели [доброю] землею, которую с клятвою обещал Господь [Бог] отцам вашим.
\vs Deu 8:2 И помни весь путь, которым вел тебя Господь, Бог твой, по пустыне, вот уже сорок лет, чтобы смирить тебя, чтобы испытать тебя и узнать, что в сердце твоем, будешь ли хранить заповеди Его, или нет;
\vs Deu 8:3 Он смирял тебя, томил тебя голодом и питал тебя манною, которой не знал ты и не знали отцы твои, дабы показать тебе, что не одним хлебом живет человек, но всяким [словом], исходящим из уст Господа, живет человек;
\vs Deu 8:4 одежда твоя не ветшала на тебе, и нога твоя не пухла, вот уже сорок лет.
\vs Deu 8:5 И знай в сердце твоем, что Господь, Бог твой, учит тебя, как человек учит сына своего.
\vs Deu 8:6 Итак храни заповеди Господа, Бога твоего, ходя путями Его и боясь Его.
\vs Deu 8:7 Ибо Господь, Бог твой, ведет тебя в землю добрую, в землю, \bibemph{где} потоки вод, источники и озера выходят из долин и гор,
\vs Deu 8:8 в землю, [где] пшеница, ячмень, виноградные лозы, смоковницы и гранатовые деревья, в землю, \bibemph{где} масличные деревья и мед,
\vs Deu 8:9 в землю, в которой без скудости будешь есть хлеб твой и ни в чем не будешь иметь недостатка, в землю, в которой камни~--- железо, и из гор которой будешь высекать медь.
\vs Deu 8:10 И когда будешь есть и насыщаться, тогда благословляй Господа, Бога твоего, за добрую землю, которую Он дал тебе.
\vs Deu 8:11 Берегись, чтобы ты не забыл Господа, Бога твоего, не соблюдая заповедей Его, и законов Его, и постановлений Его, которые сегодня заповедую тебе.
\vs Deu 8:12 Когда будешь есть и насыщаться, и построишь хорошие домы и будешь жить [в них],
\vs Deu 8:13 и когда будет у тебя много крупного и мелкого скота, и будет много серебра и золота, и всего у тебя будет много,~---
\vs Deu 8:14 то смотри, чтобы не надмилось сердце твое и не забыл ты Господа, Бога твоего, Который вывел тебя из земли Египетской, из дома рабства;
\vs Deu 8:15 Который провел тебя по пустыне великой и страшной, \bibemph{где} змеи, василиски, скорпионы и места сухие, на которых нет воды; Который источил для тебя [источник] воды из скалы гранитной,
\vs Deu 8:16 питал тебя в пустыне манною, которой [не знал ты и] не знали отцы твои, дабы смирить тебя и испытать тебя, чтобы впоследствии сделать тебе добро,
\vs Deu 8:17 и чтобы ты не сказал в сердце твоем: <<моя сила и крепость руки моей приобрели мне богатство сие>>,
\vs Deu 8:18 но чтобы помнил Господа, Бога твоего, ибо Он дает тебе силу приобретать богатство, дабы исполнить, как ныне, завет Свой, который Он клятвою утвердил отцам твоим.
\vs Deu 8:19 Если же ты забудешь Господа, Бога твоего, и пойдешь вслед богов других, и будешь служить им и поклоняться им, то свидетельствуюсь вам сегодня [небом и землею], что вы погибнете;
\vs Deu 8:20 как народы, которые Господь [Бог] истребляет от лица вашего, так погибнете \bibemph{и вы} за то, что не послушаете гласа Господа, Бога вашего.
\vs Deu 9:1 Слушай, Израиль: ты теперь идешь за Иордан, чтобы пойти овладеть народами, которые больше и сильнее тебя, городами большими, с укреплениями до небес,
\vs Deu 9:2 народом [великим,] многочисленным и великорослым, сынами Енаковыми, о которых ты знаешь и слышал: <<кто устоит против сынов Енаковых?>>
\vs Deu 9:3 Знай же ныне, что Господь, Бог твой, идет пред тобою, \bibemph{как} огнь поядающий; Он будет истреблять их и низлагать их пред тобою, и ты изгонишь их, и погубишь их скоро, как говорил тебе Господь.
\vs Deu 9:4 Когда будет изгонять их Господь, Бог твой, от лица твоего, не говори в сердце твоем, что за праведность мою привел меня Господь овладеть сею [доброю] землею, и что за нечестие народов сих Господь изгоняет их от лица твоего;
\vs Deu 9:5 не за праведность твою и не за правоту сердца твоего идешь ты наследовать землю их, но за нечестие [и беззакония] народов сих Господь, Бог твой, изгоняет их от лица твоего, и дабы исполнить слово, которым клялся Господь отцам твоим Аврааму, Исааку и Иакову;
\vs Deu 9:6 посему знай [ныне], что не за праведность твою Господь, Бог твой, дает тебе овладеть сею доброю землею, ибо ты народ жестоковыйный.
\vs Deu 9:7 Помни, не забудь, сколько ты раздражал Господа, Бога твоего, в пустыне: с самого того дня, как вышел ты из земли Египетской, и до самого прихода вашего на место сие вы противились Господу.
\vs Deu 9:8 И при Хориве вы раздражали Господа, и прогневался на вас Господь, так что \bibemph{хотел} истребить вас,
\vs Deu 9:9 когда я взошел на гору, чтобы принять скрижали каменные, скрижали завета, который поставил Господь с вами, и пробыл на горе сорок дней и сорок ночей, хлеба не ел и воды не пил,
\vs Deu 9:10 и дал мне Господь две скрижали каменные, написанные перстом Божиим, а на них [написаны были] все слова, которые изрек вам Господь на горе из среды огня в день собрания.
\vs Deu 9:11 По окончании же сорока дней и сорока ночей дал мне Господь две скрижали каменные, скрижали завета,
\vs Deu 9:12 и сказал мне Господь: встань, пойди скорее отсюда, ибо развратился народ твой, который ты вывел из Египта; скоро уклонились они от пути, который Я заповедал им; они сделали себе литой истукан.
\vs Deu 9:13 И сказал мне Господь: [Я говорил тебе один и другой раз:] вижу Я народ сей, вот он народ жестоковыйный;
\vs Deu 9:14 не удерживай Меня, и Я истреблю их, и изглажу имя их из поднебесной, а от тебя произведу народ, \bibemph{который будет} [больше,] сильнее и многочисленнее их.
\vs Deu 9:15 Я обратился и пошел с горы, гора же горела огнем; две скрижали завета \bibemph{были} в обеих руках моих;
\vs Deu 9:16 и видел я, что вы согрешили против Господа, Бога вашего, сделали себе литого тельца, скоро уклонились от пути, которого [держаться] заповедал вам Господь;
\vs Deu 9:17 и взял я обе скрижали, и бросил их из обеих рук своих, и разбил их пред глазами вашими.
\vs Deu 9:18 И [вторично] повергшись пред Господом, молился я, как прежде, сорок дней и сорок ночей, хлеба не ел и воды не пил, за все грехи ваши, которыми вы согрешили, сделав зло в очах Господа [Бога вашего] и раздражив Его;
\vs Deu 9:19 ибо я страшился гнева и ярости, которыми Господь прогневался на вас \bibemph{и хотел} погубить вас. И послушал меня Господь и на сей раз.
\vs Deu 9:20 И на Аарона весьма прогневался Господь \bibemph{и хотел} погубить его; но я молился и за Аарона в то время.
\vs Deu 9:21 Грех же ваш, который вы сделали,~--- тельца я взял, сожег его в огне, разбил его и всего истер до того, что он стал мелок, как прах, и я бросил прах сей в поток, текущий с горы.
\vs Deu 9:22 И в Тавере, в Массе и в Киброт-Гаттааве вы раздражили Господа [Бога вашего].
\vs Deu 9:23 И когда посылал вас Господь из Кадес-Варни, говоря: пойдите, овладейте землею, которую Я даю вам,~--- то вы воспротивились повелению Господа Бога вашего, и не поверили Ему, и не послушали гласа Его.
\vs Deu 9:24 Вы были непокорны Господу с того самого дня, как я стал знать вас.
\vs Deu 9:25 И повергшись пред Господом, умолял я сорок дней и сорок ночей, в которые я молился, ибо Господь хотел погубить вас;
\vs Deu 9:26 и молился я Господу и сказал: Владыка Господи, [Царь богов,] не погубляй народа Твоего и удела Твоего, который Ты избавил величием [крепости] Твоей, который вывел Ты из Египта рукою сильною [и мышцею Твоею высокою];
\vs Deu 9:27 вспомни рабов Твоих, Авраама, Исаака и Иакова, [которым Ты клялся Собою]; не смотри на ожесточение народа сего и на нечестие его и на грехи его,
\vs Deu 9:28 дабы [живущие] в той земле, откуда Ты вывел нас, не сказали: <<Господь не мог ввести их в землю, которую обещал им, и, ненавидя их, вывел Он их, чтоб умертвить их в пустыне>>.
\vs Deu 9:29 А они Твой народ и Твой удел, который Ты вывел [из земли Египетской] силою Твоею великою и мышцею Твоею высокою.
\vs Deu 10:1 В то время сказал мне Господь: вытеши себе две скрижали каменные, подобные первым, и взойди ко Мне на гору, и сделай себе деревянный ковчег;
\vs Deu 10:2 и Я напишу на скрижалях те слова, которые были на прежних скрижалях, которые ты разбил; и положи их в ковчег.
\vs Deu 10:3 И сделал я ковчег из дерева ситтим, и вытесал две каменные скрижали, как прежние, и пошел на гору; и две сии скрижали \bibemph{были} в руках моих.
\vs Deu 10:4 И написал Он на скрижалях, как написано было прежде, те десять слов, которые изрек вам Господь на горе из среды огня в день собрания, и отдал их Господь мне.
\vs Deu 10:5 И обратился я, и сошел с горы, и положил скрижали в ковчег, который я сделал, чтоб они там были, как повелел мне Господь.
\vs Deu 10:6 И сыны Израилевы отправились из Беероф-Бене-Яакана в Мозер; там умер Аарон и погребен там, и стал священником вместо него сын его Елеазар.
\vs Deu 10:7 Оттуда отправились в Гудгод, из Гудгода в Иотвафу, в землю, где потоки вод.
\vs Deu 10:8 В то время отделил Господь колено Левиино, чтобы носить ковчег завета Господня, предстоять пред Господом, служить Ему [и молиться] и благословлять именем Его, \bibemph{как это продолжается} до сего дня;
\vs Deu 10:9 потому нет левиту части и удела с братьями его: Сам Господь есть удел его, как говорил ему Господь, Бог твой.
\vs Deu 10:10 И пробыл я на горе, как и в прежнее время, сорок дней и сорок ночей; и послушал меня Господь и на сей раз, [и] не восхотел Господь погубить тебя;
\vs Deu 10:11 и сказал мне Господь: встань, пойди в путь пред народом [сим]; пусть они пойдут и овладеют землею, которую Я клялся отцам их дать им.
\vs Deu 10:12 Итак, Израиль, чего требует от тебя Господь, Бог твой? Того только, чтобы ты боялся Господа, Бога твоего, ходил всеми путями Его, и любил Его, и служил Господу, Богу твоему, от всего сердца твоего и от всей души твоей,
\vs Deu 10:13 чтобы соблюдал заповеди Господа [Бога твоего] и постановления Его, которые сегодня заповедую тебе, дабы тебе было хорошо.
\vs Deu 10:14 Вот у Господа, Бога твоего, небо и небеса небес, земля и все, что на ней;
\vs Deu 10:15 но только отцов твоих принял Господь и возлюбил их, и избрал вас, семя их после них, из всех народов, как ныне \bibemph{видишь}.
\vs Deu 10:16 Итак обрежьте крайнюю плоть сердца вашего и не будьте впредь жестоковыйны;
\vs Deu 10:17 ибо Господь, Бог ваш, есть Бог богов и Владыка владык, Бог великий, сильный и страшный, Который не смотрит на лица и не берет даров,
\vs Deu 10:18 Который дает суд сироте и вдове, и любит пришельца, и дает ему хлеб и одежду.
\vs Deu 10:19 Люб\acc{и}те и вы пришельца, ибо \bibemph{сами} были пришельцами в земле Египетской.
\vs Deu 10:20 Господа, Бога твоего, бойся [и] Ему [одному] служи, и к Нему прилепись и Его именем клянись:
\vs Deu 10:21 Он хвала твоя и Он Бог твой, Который сделал с тобою те великие и страшные \bibemph{дела}, какие видели глаза твои;
\vs Deu 10:22 в семидесяти [пяти] душах пришли отцы твои в Египет, а ныне Господь, Бог твой, сделал тебя многочисленным, как звезды небесные.
\vs Deu 11:1 Итак люби Господа, Бога твоего, и соблюдай, что повелено Им соблюдать, и постановления Его и законы Его и заповеди Его во все дни.
\vs Deu 11:2 И вспомните ныне,~--- ибо \bibemph{я говорю} не с сынами вашими, которые не знают и не видели наказания Господа Бога вашего,~--- Его величие [и] Его крепкую руку и высокую мышцу Его,
\vs Deu 11:3 знамения Его и дела Его, которые Он сделал среди Египта с фараоном, царем Египетским, и со всею землею его,
\vs Deu 11:4 и что Он сделал с войском Египетским, с конями его и колесницами его, которых Он потопил в водах Чермного моря, когда они гнались за вами,~--- и погубил их Господь [Бог] даже до сего дня;
\vs Deu 11:5 и что Он делал для вас в пустыне, доколе вы не дошли до места сего,
\vs Deu 11:6 и что Он сделал с Дафаном и Авироном, сынами Елиава, сына Рувимова, когда земля разверзла уста свои и среди всего Израиля поглотила их и семейства их, и шатры их, и все имущество их, которое было у них;
\vs Deu 11:7 ибо глаза ваши видели все великие дела Господа, которые Он сделал.
\vs Deu 11:8 Итак соблюдайте все заповеди [Его], которые я заповедую вам сегодня, дабы вы [были живы,] укрепились и пошли и овладели землею, в которую вы переходите [за Иордан], чтоб овладеть ею;
\vs Deu 11:9 и дабы вы жили много времени на той земле, которую клялся Господь отцам вашим дать им и семени их, на земле, в которой течет молоко и мед.
\vs Deu 11:10 Ибо земля, в которую ты идешь, чтоб овладеть ею, не такова, как земля Египетская, из которой вышли вы, где ты, посеяв семя твое, поливал [ее] при помощи ног твоих, как масличный сад;
\vs Deu 11:11 но земля, в которую вы переходите, чтоб овладеть ею, есть земля с горами и долинами, и от дождя небесного напояется водою,~---
\vs Deu 11:12 земля, о которой Господь, Бог твой, печется: очи Господа, Бога твоего, непрестанно на ней, от начала года и до конца года.
\vs Deu 11:13 Если вы будете слушать заповеди Мои, которые заповедую вам сегодня, любить Господа, Бога вашего, и служить Ему от всего сердца вашего и от всей души вашей,
\vs Deu 11:14 то дам земле вашей дождь в свое время, ранний и поздний; и ты соберешь хлеб твой и вино твое и елей твой;
\vs Deu 11:15 и дам траву на поле твоем для скота твоего, и будешь есть и насыщаться.
\vs Deu 11:16 Берегитесь, чтобы не обольстилось сердце ваше, и вы не уклонились и не стали служить иным богам и не поклонились им;
\vs Deu 11:17 и тогда воспламенится гнев Господа на вас, и заключит Он небо, и не будет дождя, и земля не принесет произведений своих, и вы скоро погибнете с доброй земли, которую Господь дает вам.
\vs Deu 11:18 Итак положите сии слова Мои в сердце ваше и в душу вашу, и навяжите их в знак на руку свою, и да будут они повязкою над глазами вашими;
\vs Deu 11:19 и учите им сыновей своих, говоря о них, когда ты сидишь в доме твоем, и когда идешь дорогою, и когда ложишься, и когда встаешь;
\vs Deu 11:20 и напиши их на косяках дома твоего и на воротах твоих,
\vs Deu 11:21 дабы столько же много было дней ваших и дней детей ваших на той земле, которую Господь клялся дать отцам вашим, сколько дней небо будет над землею.
\vs Deu 11:22 Ибо если вы будете соблюдать все заповеди сии, которые заповедую вам исполнять, будете любить Господа, Бога вашего, ходить всеми путями Его и прилепляться к Нему,
\vs Deu 11:23 то изгонит Господь все народы сии от лица вашего, и вы овладеете народами, которые больше и сильнее вас;
\vs Deu 11:24 всякое место, на которое ступит нога ваша, будет ваше; от пустыни и Ливана, от реки, реки Евфрата, даже до моря западного будут пределы ваши;
\vs Deu 11:25 никто не устоит против вас: Господь, Бог ваш, наведет страх и трепет пред вами на всякую землю, на которую вы ступите, как Он говорил вам.
\vs Deu 11:26 Вот, я предлагаю вам сегодня благословение и проклятие:
\vs Deu 11:27 благословение, если послушаете заповедей Господа, Бога вашего, которые я заповедую вам сегодня,
\vs Deu 11:28 а проклятие, если не послушаете заповедей Господа, Бога вашего, и уклонитесь от пути, который заповедую вам сегодня, и пойдете вслед богов иных, которых вы не знаете.
\vs Deu 11:29 Когда введет тебя Господь, Бог твой, в ту землю, в которую ты идешь, чтоб овладеть ею, тогда произнеси благословение на горе Гаризим, а проклятие на горе Гевал:
\vs Deu 11:30 вот они за Иорданом, по дороге к захождению солнца, в земле Хананеев, живущих на равнине, против Галгала, близ дубравы Мор\acc{е}.
\vs Deu 11:31 Ибо вы переходите Иордан, чтобы пойти овладеть землею, которую Господь, Бог ваш, дает вам [в удел навсегда], и овладеете ею и будете жить на ней.
\vs Deu 11:32 Итак старайтесь соблюдать все постановления и законы [Его], которые предлагаю я вам сегодня.
\vs Deu 12:1 Вот постановления и законы, которые вы должны стараться исполнять в земле, которую Господь, Бог отцов твоих, дает тебе во владение, во все дни, которые вы будете жить на той земле.
\vs Deu 12:2 Истребите все места, где народы, которыми вы овладеете, служили богам своим, на высоких горах и на холмах, и под всяким ветвистым деревом;
\vs Deu 12:3 и разрушьте жертвенники их, и сокрушите столбы их, и сожгите огнем рощи их, и разбейте истуканы богов их, и истребите имя их от места того.
\vs Deu 12:4 Не то должны вы делать для Господа, Бога вашего;
\vs Deu 12:5 но к месту, какое изберет Господь, Бог ваш, из всех колен ваших, чтобы пребывать имени Его там, обращайтесь и туда приходите,
\vs Deu 12:6 и туда приносите всесожжения ваши, и жертвы ваши, и десятины ваши, и возношение рук ваших, и обеты ваши, и добровольные приношения ваши, [и мирные жертвы ваши,] и первенцев крупного скота вашего и мелкого скота вашего;
\vs Deu 12:7 и ешьте там пред Господом, Богом вашим, и веселитесь вы и семейства ваши о всем, что делалось руками вашими, чем благословил тебя Господь, Бог твой.
\vs Deu 12:8 Там вы не должны делать всего, как мы теперь здесь делаем, каждый, что ему кажется правильным;
\vs Deu 12:9 ибо вы ныне еще не вступили в место покоя и в удел, который Господь, Бог твой, дает тебе.
\vs Deu 12:10 Но когда перейдете Иордан и поселитесь на земле, которую Господь, Бог ваш, дает вам в удел, и когда Он успокоит вас от всех врагов ваших, окружающих \bibemph{вас}, и будете жить безопасно,
\vs Deu 12:11 тогда, какое место изберет Господь, Бог ваш, чтобы пребывать имени Его там, туда приносите всё, что я заповедую вам [сегодня]: всесожжения ваши и жертвы ваши, десятины ваши и возношение рук ваших, и все, избранное по обетам вашим, что вы обещали Господу [Богу вашему];
\vs Deu 12:12 и веселитесь пред Господом, Богом вашим, вы и сыны ваши, и дочери ваши, и рабы ваши, и рабыни ваши, и левит, который посреди жилищ ваших, ибо нет ему части и удела с вами.
\rsbpar\vs Deu 12:13 Берегись приносить всесожжения твои на всяком месте, которое ты увидишь;
\vs Deu 12:14 но на том только месте, которое изберет Господь [Бог твой] в одном из колен твоих, приноси всесожжения твои и делай все, что заповедую тебе [сегодня].
\vs Deu 12:15 Впрочем, когда только пожелает душа твоя, можешь заколать и есть, по благословению Господа, Бога твоего, мясо, которое Он дал тебе, во всех жилищах твоих: нечистый и чистый могут есть сие, как серну и как оленя;
\vs Deu 12:16 только крови не ешьте: на землю выливайте ее, как воду.
\vs Deu 12:17 Нельзя тебе есть в жилищах твоих десятины хлеба твоего, и вина твоего, и елея твоего, и первенцев крупного скота твоего и мелкого скота твоего, и всех обетов твоих, которые ты обещал, и добровольных приношений твоих, и возношения рук твоих;
\vs Deu 12:18 но ешь сие [только] пред Господом, Богом твоим, на том месте, которое изберет Господь, Бог твой,~--- ты и сын твой, и дочь твоя, и раб твой, и раба твоя, и левит, [и пришелец,] который в жилищах твоих, и веселись пред Господом, Богом твоим, о всем, что делалось руками твоими.
\vs Deu 12:19 Смотри, не оставляй левита во все дни, [которые будешь жить] на земле твоей.
\rsbpar\vs Deu 12:20 Когда распространит Господь, Бог твой, пределы твои, как Он говорил тебе, и ты скажешь: <<поем я мяса>>, потому что душа твоя пожелает есть мяса,~--- тогда, по желанию души твоей, ешь мясо.
\vs Deu 12:21 Если далеко будет от тебя то место, которое изберет Господь, Бог твой, чтобы пребывать имени Его там, то заколай из крупного и мелкого скота твоего, который дал тебе Господь [Бог твой], как я повелел тебе, и ешь в жилищах твоих, по желанию души твоей;
\vs Deu 12:22 но ешь их так, как едят серну и оленя; нечистый как и чистый [у тебя] могут есть сие;
\vs Deu 12:23 только строго наблюдай, чтобы не есть крови, потому что кровь есть душа: не ешь души вместе с мясом;
\vs Deu 12:24 не ешь ее: выливай ее на землю, как воду;
\vs Deu 12:25 не ешь ее, дабы хорошо было тебе и детям твоим после тебя [во веки], если будешь делать [доброе и] справедливое пред очами Господа [Бога твоего].
\vs Deu 12:26 Только святыни твои, какие будут у тебя, и обеты твои приноси, и приходи на то место, которое изберет Господь [Бог твой, чтобы призываемо было там имя Его];
\vs Deu 12:27 и совершай всесожжения твои, мясо и кровь, на жертвеннике Господа, Бога твоего; но кровь \bibemph{других} жертв твоих должна быть проливаема у жертвенника Господа, Бога твоего, а мясо ешь.
\vs Deu 12:28 Слушай и исполняй все слова сии, которые заповедую тебе, дабы хорошо было тебе и детям твоим после тебя во век, если будешь делать доброе и угодное пред очами Господа, Бога твоего.
\rsbpar\vs Deu 12:29 Когда Господь, Бог твой, истребит от лица твоего народы, к которым ты идешь, чтобы взять их во владение, и ты, взяв их, поселишься в земле их;
\vs Deu 12:30 тогда берегись, чтобы ты не попал в сеть, последуя им, по истреблении их от лица твоего, и не искал богов их, говоря: <<как служили народы сии богам своим, так буду и я делать>>;
\vs Deu 12:31 не делай так Господу, Богу твоему, ибо все, чего гнушается Господь, что ненавидит Он, они делают богам своим: они и сыновей своих и дочерей своих сожигают на огне богам своим.
\vs Deu 12:32 Все, что я заповедую вам, старайтесь исполнить; не прибавляй к тому и не убавляй от того.
\vs Deu 13:1 Если восстанет среди тебя пророк, или сновидец, и представит тебе знамение или чудо,
\vs Deu 13:2 и сбудется то знамение или чудо, о котором он говорил тебе, и скажет притом: <<пойдем вслед богов иных, которых ты не знаешь, и будем служить им>>,~---
\vs Deu 13:3 то не слушай слов пророка сего, или сновидца сего; ибо \bibemph{чрез сие} искушает вас Господь, Бог ваш, чтобы узнать, любите ли вы Господа, Бога вашего, от всего сердца вашего и от всей души вашей;
\vs Deu 13:4 Господу, Богу вашему, последуйте и Его бойтесь, заповеди Его соблюдайте и гласа Его слушайте, и Ему служите, и к Нему прилепляйтесь;
\vs Deu 13:5 а пророка того или сновидца того должно предать смерти за то, что он уговаривал вас отступить от Господа, Бога вашего, выведшего вас из земли Египетской и избавившего тебя из дома рабства, желая совратить тебя с пути, по которому заповедал тебе идти Господь, Бог твой; и \bibemph{так} истреби зло из среды себя.
\vs Deu 13:6 Если будет уговаривать тебя тайно брат твой, [сын отца твоего или] сын матери твоей, или сын твой, или дочь твоя, или жена на лоне твоем, или друг твой, который для тебя, как душа твоя, говоря: <<пойдем и будем служить богам иным, которых не знал ты и отцы твои>>,
\vs Deu 13:7 богам тех народов, которые вокруг тебя, близких к тебе или отдаленных от тебя, от одного края земли до другого,~---
\vs Deu 13:8 то не соглашайся с ним и не слушай его; и да не пощадит его глаз твой, не жалей его и не прикрывай его,
\vs Deu 13:9 но убей его; твоя рука прежде \bibemph{всех} должна быть на нем, чтоб убить его, а потом руки всего народа;
\vs Deu 13:10 побей его камнями до смерти, ибо он покушался отвратить тебя от Господа, Бога твоего, Который вывел тебя из земли Египетской, из дома рабства;
\vs Deu 13:11 весь Израиль услышит сие и убоится, и не станут впредь делать среди тебя такого зла.
\vs Deu 13:12 Если услышишь о каком-либо из городов твоих, которые Господь, Бог твой, дает тебе для жительства,
\vs Deu 13:13 что появились в нем нечестивые люди из среды тебя и соблазнили жителей города их, говоря: <<пойдем и будем служить богам иным, которых вы не знали>>,~---
\vs Deu 13:14 то ты разыщи, исследуй и хорошо расспроси; и если это точная правда, что случилась мерзость сия среди тебя,
\vs Deu 13:15 порази жителей того города острием меча, предай заклятию его и все, что в нем, и скот его \bibemph{порази} острием меча;
\vs Deu 13:16 всю же добычу его собери на средину площади его и сожги огнем город и всю добычу его во всесожжение Господу, Богу твоему, и да будет он вечно в развалинах, не должно никогда вновь созидать его;
\vs Deu 13:17 ничто из заклятого да не прилипнет к руке твоей, дабы укротил Господь ярость гнева Своего, и дал тебе милость и помиловал тебя, и размножил тебя, [как Он говорил тебе,] как клялся отцам твоим,
\vs Deu 13:18 если будешь слушать гласа Господа, Бога твоего, соблюдая все заповеди Его, которые ныне заповедую тебе, делая [доброе и] угодное пред очами Господа, Бога твоего.
\vs Deu 14:1 Вы сыны Господа Бога вашего; не делайте нарезов \bibemph{на теле вашем} и не выстригайте волос над глазами вашими по умершем;
\vs Deu 14:2 ибо ты народ святой у Господа Бога твоего, и тебя избрал Господь, чтобы ты был собственным Его народом из всех народов, которые на земле.
\rsbpar\vs Deu 14:3 Не ешь никакой мерзости.
\vs Deu 14:4 Вот скот, который вам можно есть: волы, овцы, козы,
\vs Deu 14:5 олень и серна, и буйвол, и лань, и зубр, и орикс, и камелопард.
\vs Deu 14:6 Всякий скот, у которого раздвоены копыта и на обоих копытах глубокий разрез, и который скот жует жвачку, тот ешьте;
\vs Deu 14:7 только сих не ешьте из жующих жвачку и имеющих раздвоенные копыта с глубоким разрезом: верблюда, зайца и тушканчика, потому что, хотя они жуют жвачку, но копыта у них не раздвоены: нечисты они для вас;
\vs Deu 14:8 и свиньи, потому что копыта у нее раздвоены, но не жует жвачки: нечиста она для вас; не ешьте мяса их, и к трупам их не прикасайтесь.
\vs Deu 14:9 Из всех \bibemph{животных}, которые в воде, ешьте всех, у которых есть перья и чешуя;
\vs Deu 14:10 а всех тех, у которых нет перьев и чешуи, не ешьте: нечисто это для вас.
\vs Deu 14:11 Всякую птицу чистую ешьте;
\vs Deu 14:12 но сих не должно вам есть из них: орла, грифа и морского орла,
\vs Deu 14:13 и коршуна, и сокола, и кречета с породою их,
\vs Deu 14:14 и всякого ворона с породою его,
\vs Deu 14:15 и страуса, и совы, и чайки, и ястреба с породою его,
\vs Deu 14:16 и филина, и ибиса, и лебедя,
\vs Deu 14:17 и пеликана, и сипа, и рыболова,
\vs Deu 14:18 и цапли, и зуя с породою его, и удода, и нетопыря.
\vs Deu 14:19 Все крылатые пресмыкающиеся нечисты для вас, не ешьте [их].
\vs Deu 14:20 Всякую птицу чистую ешьте.
\vs Deu 14:21 Не ешьте никакой мертвечины; иноземцу, который \bibemph{случится} в жилищах твоих, отдай ее, он пусть ест ее, или продай ему, ибо ты народ святой у Господа Бога твоего. Не вари козленка в молоке матери его.
\rsbpar\vs Deu 14:22 Отделяй десятину от всего произведения семян твоих, которое приходит с поля [твоего] каждогодно,
\vs Deu 14:23 и ешь пред Господом, Богом твоим, на том месте, которое изберет Он, чтобы пребывать имени Его там; [приноси] десятину хлеба твоего, вина твоего и елея твоего, и первенцев крупного скота твоего и мелкого скота твоего, дабы ты научился бояться Господа, Бога твоего, во все дни.
\vs Deu 14:24 Если же длинна будет для тебя дорога, так что ты не можешь нести сего, потому что далеко от тебя то место, которое изберет Господь, Бог твой, чтоб положить там имя Свое, и Господь, Бог твой, благословил тебя,
\vs Deu 14:25 то променяй это на серебро и возьми серебро в руку твою и приходи на место, которое изберет Господь, Бог твой;
\vs Deu 14:26 и покупай на серебро сие всего, чего пожелает душа твоя, волов, овец, вина, сикера и всего, чего потребует от тебя душа твоя; и ешь там пред Господом, Богом твоим, и веселись ты и семейство твое.
\vs Deu 14:27 И левита, который в жилищах твоих, не оставь, ибо нет ему части и удела с тобою.
\vs Deu 14:28 По прошествии же трех лет отделяй все десятины произведений твоих в тот год и клади [сие] в жилищах твоих;
\vs Deu 14:29 и пусть придет левит, ибо ему нет части и удела с тобою, и пришелец, и сирота, и вдова, которые \bibemph{находятся} в жилищах твоих, и пусть едят и насыщаются, дабы благословил тебя Господь, Бог твой, во всяком деле рук твоих, которое ты будешь делать.
\vs Deu 15:1 В седьмой год делай прощение.
\vs Deu 15:2 Прощение же состоит в том, чтобы всякий заимодавец, который дал взаймы ближнему своему, простил \bibemph{долг} и не взыскивал с ближнего своего или с брата своего, ибо провозглашено прощение ради Господа [Бога твоего];
\vs Deu 15:3 с иноземца взыскивай, а что будет твое у брата твоего, прости.
\vs Deu 15:4 Разве только не будет у тебя нищего: ибо благословит тебя Господь на той земле, которую Господь, Бог твой, дает тебе в удел, чтобы ты взял ее в наследство,
\vs Deu 15:5 если только будешь слушать гласа Господа, Бога твоего, и стараться исполнять все заповеди сии, которые я сегодня заповедую тебе;
\vs Deu 15:6 ибо Господь, Бог твой, благословит тебя, как Он говорил тебе, и ты будешь давать взаймы многим народам, а сам не будешь брать взаймы; и господствовать будешь над многими народами, а они над тобою не будут господствовать.
\vs Deu 15:7 Если же будет у тебя нищий кто-либо из братьев твоих, в одном из жилищ твоих, на земле твоей, которую Господь, Бог твой, дает тебе, то не ожесточи сердца твоего и не сожми руки твоей пред нищим братом твоим,
\vs Deu 15:8 но открой ему руку твою и дай ему взаймы, смотря по его нужде, в чем он нуждается;
\vs Deu 15:9 берегись, чтобы не вошла в сердце твое беззаконная мысль: <<приближается седьмой год, год прощения>>, и чтоб \bibemph{оттого} глаз твой не сделался немилостив к нищему брату твоему, и ты не отказал ему; ибо он возопиет на тебя к Господу, и будет на тебе [великий] грех;
\vs Deu 15:10 дай ему [и взаймы дай ему, сколько он просит и сколько ему нужно], и когда будешь давать ему, не должно скорбеть сердце твое, ибо за то благословит тебя Господь, Бог твой, во всех делах твоих и во всем, что будет делаться твоими руками;
\vs Deu 15:11 ибо нищие всегда будут среди земли [твоей]; потому я и повелеваю тебе: отверзай руку твою брату твоему, бедному твоему и нищему твоему на земле твоей.
\vs Deu 15:12 Если продастся тебе брат твой, Еврей, или Евреянка, то шесть лет должен он быть рабом тебе, а в седьмой год отпусти его от себя на свободу;
\vs Deu 15:13 когда же будешь отпускать его от себя на свободу, не отпусти его с пустыми \bibemph{руками},
\vs Deu 15:14 но снабди его от стад твоих, от гумна твоего и от точила твоего: дай ему, чем благословил тебя Господь, Бог твой:
\vs Deu 15:15 помни, что [и] ты был рабом в земле Египетской и избавил тебя Господь, Бог твой, потому я сегодня и заповедую тебе сие.
\vs Deu 15:16 Если же он скажет тебе: <<не пойду я от тебя, потому что я люблю тебя и дом твой>>, потому что хорошо ему у тебя,
\vs Deu 15:17 то возьми шило и проколи ухо его к двери; и будет он рабом твоим на век. Так поступай и с рабою твоею.
\vs Deu 15:18 Не считай этого для себя тяжким, что ты должен отпустить его от себя на свободу, ибо он в шесть лет заработал тебе вдвое против платы наемника; и благословит тебя Господь, Бог твой, во всем, что ни будешь делать.
\rsbpar\vs Deu 15:19 Все первородное мужеского пола, что родится от крупного скота твоего и от мелкого скота твоего, посвящай Господу, Богу твоему: не работай на первородном воле твоем и не стриги первородного из мелкого скота твоего;
\vs Deu 15:20 пред Господом, Богом твоим, каждогодно съедай это ты и семейство твое, на месте, которое изберет Господь [Бог твой];
\vs Deu 15:21 если же будет на нем порок, хромота или слепота [или] другой какой-нибудь порок, то не приноси его в жертву Господу, Богу твоему,
\vs Deu 15:22 но в жилищах твоих ешь его; нечистый, как и чистый, [могут есть,] как серну и как оленя;
\vs Deu 15:23 только крови его не ешь: на землю выливай ее, как воду.
\vs Deu 16:1 Наблюдай месяц Авив, и совершай Пасху Господу, Богу твоему, потому что в месяце Авиве вывел тебя Господь, Бог твой, из Египта ночью.
\vs Deu 16:2 И заколай Пасху Господу, Богу твоему, из мелкого и крупного скота на месте, которое изберет Господь, чтобы пребывало там имя Его.
\vs Deu 16:3 Не ешь с нею квасного; семь дней ешь с нею опресноки, хлебы бедствия, ибо ты с поспешностью вышел из земли Египетской, дабы ты помнил день исшествия своего из земли Египетской во все дни жизни твоей;
\vs Deu 16:4 не должно находиться у тебя ничто квасное во всем уделе твоем в продолжение семи дней, и из мяса, которое ты принес в жертву вечером в первый день, ничто не должно оставаться до утра.
\vs Deu 16:5 Не можешь ты заколать Пасху в котором-нибудь из жилищ твоих, которые Господь, Бог твой, даст тебе;
\vs Deu 16:6 но только на том месте, которое изберет Господь, Бог твой, чтобы пребывало там имя Его, заколай Пасху вечером при захождении солнца, в то самое время, в которое ты вышел из Египта;
\vs Deu 16:7 и испеки и съешь на том месте, которое изберет Господь, Бог твой, а на другой день можешь возвратиться и войти в шатры твои.
\vs Deu 16:8 Шесть дней ешь пресные хлебы, а в седьмой день отдание праздника Господу, Богу твоему; не занимайся работою.
\rsbpar\vs Deu 16:9 Семь седмиц отсчитай себе; начинай считать семь седмиц с того времени, как появится серп на жатве;
\vs Deu 16:10 тогда совершай праздник седмиц Господу, Богу твоему, по усердию руки твоей, сколько ты дашь, смотря по тому, чем благословит тебя Господь, Бог твой;
\vs Deu 16:11 и веселись пред Господом, Богом твоим, ты, и сын твой, и дочь твоя, и раб твой, и раба твоя, и левит, который в жилищах твоих, и пришелец, и сирота, и вдова, которые среди тебя, на месте, которое изберет Господь, Бог твой, чтобы пребывало там имя Его;
\vs Deu 16:12 помни, что ты был рабом в Египте, и соблюдай и исполняй постановления сии.
\rsbpar\vs Deu 16:13 Праздник кущей совершай у себя семь дней, когда уберешь с гумна твоего и из точила твоего;
\vs Deu 16:14 и веселись в праздник твой ты и сын твой, и дочь твоя, и раб твой, и раба твоя, и левит, и пришелец, и сирота, и вдова, которые в жилищах твоих;
\vs Deu 16:15 семь дней празднуй Господу, Богу твоему, на месте, которое изберет Господь, Бог твой, [чтобы призываемо было там имя Его]; ибо благословит тебя Господь, Бог твой, во всех произведениях твоих и во всяком деле рук твоих, и ты будешь только веселиться.
\rsbpar\vs Deu 16:16 Три раза в году весь мужеский пол должен являться пред лице Господа, Бога твоего, на место, которое изберет Он: в праздник опресноков, в праздник седмиц и в праздник кущей; и \bibemph{никто} не должен являться пред лице Господа с пустыми \bibemph{руками},
\vs Deu 16:17 но каждый с даром в руке своей, смотря по благословению Господа, Бога твоего, какое Он дал тебе.
\rsbpar\vs Deu 16:18 Во всех жилищах твоих, которые Господь, Бог твой, даст тебе, поставь себе судей и надзирателей по коленам твоим, чтоб они судили народ судом праведным;
\vs Deu 16:19 не извращай закона, не смотри на лица и не бери даров, ибо дары ослепляют глаза мудрых и превращают дело правых;
\vs Deu 16:20 правды, правды ищи, дабы ты был жив и овладел землею, которую Господь, Бог твой, дает тебе.
\vs Deu 16:21 Не сади себе рощи из каких-либо дерев при жертвеннике Господа, Бога твоего, который ты сделаешь себе,
\vs Deu 16:22 и не ставь себе столба, что ненавидит Господь Бог твой.
\vs Deu 17:1 Не приноси в жертву Господу, Богу твоему, вола, или овцы, на которой будет порок, \bibemph{или} что-нибудь худое, ибо это мерзость для Господа, Бога твоего.
\vs Deu 17:2 Если найдется среди тебя в каком-либо из жилищ твоих, которые Господь, Бог твой, дает тебе, мужчина или женщина, кто сделает зло пред очами Господа, Бога твоего, преступив завет Его,
\vs Deu 17:3 и пойдет и станет служить иным богам, и поклонится им, или солнцу, или луне, или всему воинству небесному, чего я не повелел,
\vs Deu 17:4 и тебе возвещено будет, и ты услышишь, то ты хорошо разыщи; и если это точная правда, если сделана мерзость сия в Израиле,
\vs Deu 17:5 то выведи мужчину того, или женщину ту, которые сделали зло сие, к воротам твоим и побей их камнями до смерти.
\vs Deu 17:6 По словам двух свидетелей, или трех свидетелей, должен умереть осуждаемый на смерть: не должно предавать смерти по словам одного свидетеля;
\vs Deu 17:7 рука свидетелей должна быть на нем прежде \bibemph{всех}, чтоб убить его, потом рука всего народа; и \bibemph{так} истреби зло из среды себя.
\vs Deu 17:8 Если по какому делу затруднительным будет для тебя рассудить между кровью и кровью, между судом и судом, между побоями и побоями, \bibemph{и будут} несогласные мнения в воротах твоих, то встань и пойди на место, которое изберет Господь, Бог твой, [чтобы призываемо было там имя Его,]
\vs Deu 17:9 и приди к священникам левитам и к судье, который будет в те дни, и спроси их, и они скажут тебе, как рассудить;
\vs Deu 17:10 и поступи по слову, какое они скажут тебе, на том месте, которое изберет Господь [Бог твой, чтобы призываемо было там имя Его,] и постарайся исполнить все, чему они научат тебя;
\vs Deu 17:11 по закону, которому научат они тебя, и по определению, какое они скажут тебе, поступи, и не уклоняйся ни направо, ни налево от того, что они скажут тебе.
\vs Deu 17:12 А кто поступит так дерзко, что не послушает священника, стоящего там на служении пред Господом, Богом твоим, или судьи, [который будет в те дни], тот должен умереть,~--- и \bibemph{так} истреби зло от Израиля;
\vs Deu 17:13 и весь народ услышит и убоится, и не будут впредь поступать дерзко.
\rsbpar\vs Deu 17:14 Когда ты придешь в землю, которую Господь, Бог твой, дает тебе, и овладеешь ею, и поселишься на ней, и скажешь: <<поставлю я над собою царя, подобно прочим народам, которые вокруг меня>>,
\vs Deu 17:15 то поставь над собою царя, которого изберет Господь, Бог твой; из среды братьев твоих поставь над собою царя; не можешь поставить над собою [царем] иноземца, который не брат тебе.
\vs Deu 17:16 Только чтоб он не умножал себе коней и не возвращал народа в Египет для умножения себе коней, ибо Господь сказал вам: <<не возвращайтесь более путем сим>>;
\vs Deu 17:17 и чтобы не умножал себе жен, дабы не развратилось сердце его, и чтобы серебра и золота не умножал себе чрезмерно.
\vs Deu 17:18 Но когда он сядет на престоле царства своего, должен списать для себя список закона сего с книги, \bibemph{находящейся} у священников левитов,
\vs Deu 17:19 и пусть он будет у него, и пусть он читает его во все дни жизни своей, дабы научался бояться Господа, Бога своего, и старался исполнять все слова закона сего и постановления сии;
\vs Deu 17:20 чтобы не надмевалось сердце его пред братьями его, и чтобы не уклонялся он от закона ни направо, ни налево, дабы долгие дни пребыл на царстве своем он и сыновья его посреди Израиля.
\vs Deu 18:1 Священникам левитам, всему колену Левиину, не будет части и удела с Израилем: они должны питаться жертвами Господа и Его частью;
\vs Deu 18:2 удела же не будет ему между братьями его: Сам Господь удел его, как говорил Он ему.
\vs Deu 18:3 Вот что должно быть положено священникам от народа, от приносящих в жертву волов или овец: должно отдавать священнику плечо, челюсти и желудок;
\vs Deu 18:4 также начатки от хлеба твоего, вина твоего и елея твоего, и начатки от шерсти овец твоих отдавай ему,
\vs Deu 18:5 ибо его избрал Господь Бог твой из всех колен твоих, чтобы он предстоял [пред Господом, Богом твоим], служил [и благословлял] во имя Господа, сам и сыны его во все дни.
\vs Deu 18:6 И если левит придет из одного из жилищ твоих, из всей \bibemph{земли} [сынов] Израилевых, где он жил, и придет по желанию души своей на место, которое изберет Господь,
\vs Deu 18:7 и будет служить во имя Господа Бога своего, как и все братья его левиты, предстоящие там пред Господом,~---
\vs Deu 18:8 то пусть они пользуются одинаковою частью, сверх полученного от продажи отцовского \bibemph{имущества}.
\rsbpar\vs Deu 18:9 Когда ты войдешь в землю, которую дает тебе Господь Бог твой, тогда не научись делать мерзости, какие делали народы сии:
\vs Deu 18:10 не должен находиться у тебя проводящий сына своего или дочь свою чрез огонь, прорицатель, гадатель, ворожея, чародей,
\vs Deu 18:11 обаятель, вызывающий духов, волшебник и вопрошающий мертвых;
\vs Deu 18:12 ибо мерзок пред Господом всякий, делающий это, и за сии-то мерзости Господь Бог твой изгоняет их от лица твоего;
\vs Deu 18:13 будь непорочен пред Господом Богом твоим;
\vs Deu 18:14 ибо народы сии, которых ты изгоняешь, слушают гадателей и прорицателей, а тебе не то дал Господь Бог твой.
\vs Deu 18:15 Пророка из среды тебя, из братьев твоих, как меня, воздвигнет тебе Господь Бог твой,~--- Его слушайте,~---
\vs Deu 18:16 так как ты просил у Господа Бога твоего при Хориве в день собрания, говоря: да не услышу впредь гласа Господа Бога моего и огня сего великого да не увижу более, дабы мне не умереть.
\vs Deu 18:17 И сказал мне Господь: хорошо то, что они говорили [тебе];
\vs Deu 18:18 Я воздвигну им Пророка из среды братьев их, такого как ты, и вложу слова Мои в уста Его, и Он будет говорить им все, что Я повелю Ему;
\vs Deu 18:19 а кто не послушает слов Моих, которые [Пророк тот] будет говорить Моим именем, с того Я взыщу;
\vs Deu 18:20 но пророка, который дерзнет говорить Моим именем то, чего Я не повелел ему говорить, и который будет говорить именем богов иных, такого пророка предайте смерти.
\vs Deu 18:21 И если скажешь в сердце твоем: <<как мы узнаем слово, которое не Господь говорил?>>
\vs Deu 18:22 Если пророк скажет именем Господа, но слово то не сбудется и не исполнится, то не Господь говорил сие слово, но говорил сие пророк по дерзости своей,~--- не бойся его.
\vs Deu 19:1 Когда Господь Бог твой истребит народы, которых землю дает тебе Господь Бог твой и ты вступишь в наследие после них, и поселишься в городах их и домах их,
\vs Deu 19:2 тогда отдели себе три города среди земли твоей, которую Господь Бог твой дает тебе во владение;
\vs Deu 19:3 устрой себе дорогу и раздели на три части всю землю твою, которую Господь Бог твой дает тебе в удел; они будут служить убежищем всякому убийце.
\vs Deu 19:4 И вот какой убийца может убегать туда и остаться жив: кто убьет ближнего своего без намерения, не быв врагом ему вчера и третьего дня;
\vs Deu 19:5 кто пойдет с ближним своим в лес рубить дрова, и размахнется рука его с топором, чтобы срубить дерево, и соскочит железо с топорища и попадет в ближнего, и он умрет,~--- такой пусть убежит в один из городов тех, чтоб остаться живым,
\vs Deu 19:6 дабы мститель за кровь в горячности сердца своего не погнался за убийцею и не настиг его, если далек будет путь, и не убил его, между тем как он не \bibemph{подлежит} осуждению на смерть, ибо не был врагом ему вчера и третьего дня;
\vs Deu 19:7 посему я и дал тебе повеление, говоря: отдели себе три города.
\vs Deu 19:8 Когда же Господь Бог твой распространит пределы твои, как Он клялся отцам твоим, и даст тебе всю землю, которую Он обещал дать отцам твоим,
\vs Deu 19:9 если ты будешь стараться исполнять все сии заповеди, которые я заповедую тебе сегодня, любить Господа Бога твоего и ходить путями Его во все дни,~--- тогда к сим трем городам прибавь еще три города,
\vs Deu 19:10 дабы не проливалась кровь невинного среди земли твоей, которую Господь Бог твой дает тебе в удел, и чтобы не было на тебе [вины] крови.
\vs Deu 19:11 Но если кто [у тебя] будет врагом ближнему своему и будет подстерегать его, и восстанет на него и убьет его до смерти, и убежит в один из городов тех,
\vs Deu 19:12 то старейшины города его должны послать, чтобы взять его оттуда и предать его в руки мстителя за кровь, чтоб он умер;
\vs Deu 19:13 да не пощадит его глаз твой; смой с Израиля кровь невинного, и будет тебе хорошо.
\rsbpar\vs Deu 19:14 Не нарушай межи ближнего твоего, которую положили предки в уделе твоем, доставшемся тебе в земле, которую Господь Бог твой дает тебе во владение.
\rsbpar\vs Deu 19:15 Недостаточно одного свидетеля против кого-либо в какой-нибудь вине и в каком-нибудь преступлении и в каком-нибудь грехе, которым он согрешит: при словах двух свидетелей, или при словах трех свидетелей состоится [всякое] дело.
\vs Deu 19:16 Если выступит против кого свидетель несправедливый, обвиняя его в преступлении,
\vs Deu 19:17 то пусть предстанут оба сии человека, у которых тяжба, пред Господа, пред священников и пред судей, которые будут в те дни;
\vs Deu 19:18 судьи должны хорошо исследовать, и если свидетель тот свидетель ложный, ложно донес на брата своего,
\vs Deu 19:19 то сделайте ему то, что он умышлял сделать брату своему; и \bibemph{так} истреби зло из среды себя;
\vs Deu 19:20 и прочие услышат, и убоятся, и не станут впредь делать такое зло среди тебя;
\vs Deu 19:21 да не пощадит [его] глаз твой: душу за душу, глаз за глаз, зуб за зуб, руку за руку, ногу за ногу. [Какой кто сделает вред ближнему своему, тем должно отплатить ему.]
\vs Deu 20:1 Когда ты выйдешь на войну против врага твоего и увидишь коней и колесницы [и] народа более, нежели у тебя, то не бойся их, ибо с тобою Господь Бог твой, Который вывел тебя из земли Египетской.
\vs Deu 20:2 Когда же приступаете к сражению, тогда пусть подойдет священник, и говорит народу,
\vs Deu 20:3 и скажет ему: слушай, Израиль! вы сегодня вступаете в сражение с врагами вашими, да не ослабеет сердце ваше, не бойтесь, не смущайтесь и не ужасайтесь их,
\vs Deu 20:4 ибо Господь Бог ваш идет с вами, чтобы сразиться за вас с врагами вашими [и] спасти вас.
\vs Deu 20:5 Надзиратели же пусть объявят народу, говоря: кто построил новый дом и не обновил его, тот пусть идет и возвратится в дом свой, дабы не умер на сражении, и другой не обновил его;
\vs Deu 20:6 и кто насадил виноградник и не пользовался им, тот пусть идет и возвратится в дом свой, дабы не умер на сражении, и другой не воспользовался им;
\vs Deu 20:7 и кто обручился с женою и не взял ее, тот пусть идет и возвратится в дом свой, дабы не умер на сражении, и другой не взял ее.
\vs Deu 20:8 И еще объявят надзиратели народу, и скажут: кто боязлив и малодушен, тот пусть идет и возвратится в дом свой, дабы он не сделал робкими сердца братьев его, как его сердце.
\vs Deu 20:9 Когда надзиратели скажут все это народу, тогда должно поставить военных начальников в вожди народу.
\rsbpar\vs Deu 20:10 Когда подойдешь к городу, чтобы завоевать его, предложи ему мир;
\vs Deu 20:11 если он согласится на мир с тобою и отворит тебе \bibemph{ворота}, то весь народ, который найдется в нем, будет платить тебе дань и служить тебе;
\vs Deu 20:12 если же он не согласится на мир с тобою и будет вести с тобою войну, то осади его,
\vs Deu 20:13 и \bibemph{когда} Господь Бог твой предаст его в руки твои, порази в нем весь мужеский пол острием меча;
\vs Deu 20:14 только жен и детей и скот и все, что в городе, всю добычу его возьми себе и пользуйся добычею врагов твоих, которых предал тебе Господь Бог твой;
\vs Deu 20:15 так поступай со всеми городами, которые от тебя весьма далеко, которые не из \bibemph{числа} городов народов сих.
\vs Deu 20:16 А в городах сих народов, которых Господь Бог твой дает тебе во владение, не оставляй в живых ни одной души,
\vs Deu 20:17 но предай их заклятию: Хеттеев и Аморреев, и Хананеев, и Ферезеев, и Евеев, и Иевусеев, [и Гергесеев,] как повелел тебе Господь Бог твой,
\vs Deu 20:18 дабы они не научили вас делать такие же мерзости, какие они делали для богов своих, и дабы вы не грешили пред Господом Богом вашим.
\vs Deu 20:19 Если долгое время будешь держать в осаде [какой-нибудь] город, чтобы завоевать его и взять его, то не порти дерев его, от которых можно питаться, и не опустошай окрестностей, ибо дерево на поле не человек, чтобы могло уйти от тебя в укрепление;
\vs Deu 20:20 только те дерева, о которых ты знаешь, что они ничего не приносят в пищу, можешь портить и рубить, и строить укрепление против города, который ведет с тобою войну, доколе не покоришь его.
\vs Deu 21:1 Если в земле, которую Господь, Бог твой, дает тебе во владение, найден будет убитый, лежащий на поле, и неизвестно, кто убил его,
\vs Deu 21:2 то пусть выйдут старейшины твои и судьи твои и измерят \bibemph{расстояние} до городов, которые вокруг убитого;
\vs Deu 21:3 и старейшины города того, который будет ближайшим к убитому, пусть возьмут телицу, на которой не работали, [и] которая не носила ярма,
\vs Deu 21:4 и пусть старейшины того города отведут сию телицу в дикую долину, которая не разработана и не засеяна, и заколют там телицу в долине;
\vs Deu 21:5 и придут священники, сыны Левиины [ибо их избрал Господь Бог твой служить Ему и благословлять именем Господа, и по слову их должно \bibemph{решить} всякое спорное дело и всякий причиненный вред,]
\vs Deu 21:6 и все старейшины города того, ближайшие к убитому, пусть омоют руки свои над [головою] телицы, зарезанной в долине,
\vs Deu 21:7 и объявят и скажут: руки наши не пролили крови сей, и глаза наши не видели;
\vs Deu 21:8 очисти народ Твой, Израиля, который Ты, Господи, освободил [из земли Египетской], и не вмени народу Твоему, Израилю, невинной крови. И они очистятся от крови.
\vs Deu 21:9 \bibemph{Так} должен ты смывать у себя кровь невинного, если хочешь делать [доброе и] справедливое пред очами Господа [Бога твоего].
\rsbpar\vs Deu 21:10 Когда выйдешь на войну против врагов твоих, и Господь Бог твой предаст их в руки твои, и возьмешь их в плен,
\vs Deu 21:11 и увидишь между пленными женщину, красивую видом, и полюбишь ее, и захочешь взять ее себе в жену,
\vs Deu 21:12 то приведи ее в дом свой, и пусть она острижет голову свою и обрежет ногти свои,
\vs Deu 21:13 и снимет с себя пленническую одежду свою, и живет в доме твоем, и оплакивает отца своего и матерь свою в продолжение месяца; и после того ты можешь войти к ней и сделаться ее мужем, и она будет твоею женою;
\vs Deu 21:14 если же она \bibemph{после} не понравится тебе, то отпусти ее, \bibemph{куда} она захочет, но не продавай ее за серебро и не обращай ее в рабство, потому что ты смирил ее.
\rsbpar\vs Deu 21:15 Если у кого будут две жены~--- одна любимая, а другая нелюбимая, и как любимая, \bibemph{так} и нелюбимая родят ему сыновей, и первенцем будет сын нелюбимой,~---
\vs Deu 21:16 то, при разделе сыновьям своим имения своего, он не может сыну жены любимой дать первенство пред первородным сыном нелюбимой;
\vs Deu 21:17 но первенцем должен признать сына нелюбимой [и] дать ему двойную часть из всего, что у него найдется, ибо он есть начаток силы его, ему \bibemph{принадлежит} право первородства.
\rsbpar\vs Deu 21:18 Если у кого будет сын буйный и непокорный, не повинующийся голосу отца своего и голосу матери своей, и они наказывали его, но он не слушает их,~---
\vs Deu 21:19 то отец его и мать его пусть возьмут его и приведут его к старейшинам города своего и к воротам своего местопребывания
\vs Deu 21:20 и скажут старейшинам города своего: <<сей сын наш буен и непокорен, не слушает слов наших, мот и пьяница>>;
\vs Deu 21:21 тогда все жители города его пусть побьют его камнями до смерти; и \bibemph{так} истреби зло из среды себя, и все Израильтяне услышат и убоятся.
\rsbpar\vs Deu 21:22 Если в ком найдется преступление, достойное смерти, и он будет умерщвлен, и ты повесишь его на дереве,
\vs Deu 21:23 то тело его не должно ночевать на дереве, но погреби его в тот же день, ибо проклят пред Богом [всякий] повешенный [на дереве], и не оскверняй земли твоей, которую Господь Бог твой дает тебе в удел.
\vs Deu 22:1 Когда увидишь вола брата твоего или овцу его заблудившихся, не оставляй их, но возврати их брату твоему;
\vs Deu 22:2 если же не близко будет к тебе брат твой, или ты не знаешь его, то прибери их в дом свой, и пусть они будут у тебя, доколе брат твой не будет искать их, и тогда возврати ему их;
\vs Deu 22:3 так поступай и с ослом его, так поступай с одеждой его, так поступай со всякою потерянною \bibemph{вещью} брата твоего, которая будет им потеряна и которую ты найдешь; нельзя тебе уклоняться \bibemph{от сего}.
\vs Deu 22:4 Когда увидишь осла брата твоего или вола его упадших на пути, не оставляй их, но подними их с ним вместе.
\rsbpar\vs Deu 22:5 На женщине не должно быть мужской одежды, и мужчина не должен одеваться в женское платье, ибо мерзок пред Господом Богом твоим всякий делающий сие.
\rsbpar\vs Deu 22:6 Если попадется тебе на дороге птичье гнездо на каком-либо дереве или на земле, с птенцами или яйцами, и мать сидит на птенцах или на яйцах, то не бери матери вместе с детьми:
\vs Deu 22:7 мать пусти, а детей возьми себе, чтобы тебе было хорошо, и чтобы продлились дни твои.
\rsbpar\vs Deu 22:8 Если будешь строить новый дом, то сделай перила около кровли твоей, чтобы не навести тебе крови на дом твой, когда кто-нибудь упадет с него.
\rsbpar\vs Deu 22:9 Не засевай виноградника своего двумя родами семян, чтобы не сделать тебе заклятым сбора семян, которые ты посеешь вместе с плодами виноградника [своего].
\vs Deu 22:10 Не паши на воле и осле вместе.
\vs Deu 22:11 Не надевай одежды, сделанной из разных веществ, из шерсти и льна вместе.
\vs Deu 22:12 Сделай себе кисточки на четырех углах покрывала твоего, которым ты покрываешься.
\rsbpar\vs Deu 22:13 Если кто возьмет жену, и войдет к ней, и возненавидит ее,
\vs Deu 22:14 и будет возводить на нее порочные дела, и пустит о ней худую молву, и скажет: <<я взял сию жену, и вошел к ней, и не нашел у нее девства>>,
\vs Deu 22:15 то отец отроковицы и мать ее пусть возьмут и вынесут \bibemph{признаки} девства отроковицы к старейшинам города, к воротам;
\vs Deu 22:16 и отец отроковицы скажет старейшинам: дочь мою я отдал в жену сему человеку, и [ныне] он возненавидел ее,
\vs Deu 22:17 и вот, он взводит [на нее] порочные дела, говоря: <<я не нашел у дочери твоей девства>>; но вот признаки девства дочери моей. И расстелют одежду пред старейшинами города.
\vs Deu 22:18 Тогда старейшины того города пусть возьмут мужа и накажут его,
\vs Deu 22:19 и наложат на него сто \bibemph{сиклей} серебра пени и отдадут отцу отроковицы за то, что он пустил худую молву о девице Израильской; она же пусть останется его женою, и он не может развестись с нею во всю жизнь свою.
\vs Deu 22:20 Если же сказанное будет истинно, и не найдется девства у отроковицы,
\vs Deu 22:21 то отроковицу пусть приведут к дверям дома отца ее, и жители города ее побьют ее камнями до смерти, ибо она сделала срамное дело среди Израиля, блудодействовав в доме отца своего; и \bibemph{так} истреби зло из среды себя.
\vs Deu 22:22 Если найден будет кто лежащий с женою замужнею, то должно предать смерти обоих: и мужчину, лежавшего с женщиною, и женщину; и \bibemph{так} истреби зло от Израиля.
\vs Deu 22:23 Если будет молодая девица обручена мужу, и кто-нибудь встретится с нею в городе и ляжет с нею,
\vs Deu 22:24 то обоих их приведите к воротам того города, и побейте их камнями до смерти: отроковицу за то, что она не кричала в городе, а мужчину за то, что он опорочил жену ближнего своего; и \bibemph{так} истреби зло из среды себя.
\vs Deu 22:25 Если же кто в поле встретится с отроковицею обрученною и, схватив ее, ляжет с нею, то должно предать смерти только мужчину, лежавшего с нею,
\vs Deu 22:26 а отроковице ничего не делай; на отроковице нет преступления смертного: ибо это то же, как если бы кто восстал на ближнего своего и убил его;
\vs Deu 22:27 ибо он встретился с нею в поле, и \bibemph{хотя} отроковица обрученная кричала, но некому было спасти ее.
\vs Deu 22:28 Если кто-нибудь встретится с девицею необрученною, и схватит ее и ляжет с нею, и застанут их,
\vs Deu 22:29 то лежавший с нею должен дать отцу отроковицы пятьдесят [сиклей] серебра, а она пусть будет его женою, потому что он опорочил ее; во всю жизнь свою он не может развестись с нею.
\rsbpar\vs Deu 22:30 Никто не должен брать жены отца своего и открывать край \bibemph{одежды} отца своего.
\vs Deu 23:1 У кого раздавлены ятра или отрезан детородный член, тот не может войти в общество Господне.
\vs Deu 23:2 Сын блудницы не может войти в общество Господне, и десятое поколение его не может войти в общество Господне.
\vs Deu 23:3 Аммонитянин и Моавитянин не может войти в общество Господне, и десятое поколение их не может войти в общество Господне во веки,
\vs Deu 23:4 потому что они не встретили вас с хлебом и водою на пути, когда вы шли из Египта, и потому что они наняли против тебя Валаама, сына Веорова, из Пефора Месопотамского, чтобы проклясть тебя;
\vs Deu 23:5 но Господь, Бог твой, не восхотел слушать Валаама и обратил Господь Бог твой проклятие его в благословение тебе, ибо Господь Бог твой любит тебя.
\vs Deu 23:6 Не желай им мира и благополучия во все дни твои, во веки.
\vs Deu 23:7 Не гнушайся Идумеянином, ибо он брат твой; не гнушайся Египтянином, ибо ты был пришельцем в земле его;
\vs Deu 23:8 дети, которые у них родятся, в третьем поколении могут войти в общество Господне.
\rsbpar\vs Deu 23:9 Когда пойдешь в поход против врагов твоих, берегись всего худого.
\vs Deu 23:10 Если у тебя будет кто нечист от случившегося [ему] ночью, то он должен выйти вон из стана и не входить в стан,
\vs Deu 23:11 а при наступлении вечера должен омыть [тело свое] водою, и по захождении солнца может войти в стан.
\vs Deu 23:12 Место должно быть у тебя вне стана, куда бы тебе выходить;
\vs Deu 23:13 кроме оружия твоего должна быть у тебя лопатка; и когда будешь садиться вне \bibemph{стана}, выкопай ею [яму] и опять зарой [ею] испражнение твое;
\vs Deu 23:14 ибо Господь Бог твой ходит среди стана твоего, чтобы избавлять тебя и предавать врагов твоих [в руки твои], а \bibemph{посему} стан твой должен быть свят, чтобы Он не увидел у тебя чего срамного и не отступил от тебя.
\rsbpar\vs Deu 23:15 Не выдавай раба господину его, когда он прибежит к тебе от господина своего;
\vs Deu 23:16 пусть он у тебя живет, среди вас [пусть он живет] на месте, которое он изберет в каком-нибудь из жилищ твоих, где ему понравится; не притесняй его.
\rsbpar\vs Deu 23:17 Не должно быть блудницы из дочерей Израилевых и не должно быть блудника из сынов Израилевых.
\vs Deu 23:18 Не вноси платы блудницы и цены пса в дом Господа Бога твоего ни по какому обету, ибо то и другое есть мерзость пред Господом Богом твоим.
\rsbpar\vs Deu 23:19 Не отдавай в рост брату твоему ни серебра, ни хлеба, ни чего-либо другого, что \bibemph{можно} отдавать в рост;
\vs Deu 23:20 иноземцу отдавай в рост, а брату твоему не отдавай в рост, чтобы Господь Бог твой благословил тебя во всем, что делается руками твоими, на земле, в которую ты идешь, чтобы овладеть ею.
\rsbpar\vs Deu 23:21 Если дашь обет Господу Богу твоему, немедленно исполни его, ибо Господь Бог твой взыщет его с тебя, и на тебе будет грех;
\vs Deu 23:22 если же ты не дал обета, то не будет на тебе греха.
\vs Deu 23:23 Что вышло из уст твоих, соблюдай и исполняй так, как обещал ты Господу Богу твоему добровольное приношение, о котором сказал ты устами своими.
\rsbpar\vs Deu 23:24 Когда войдешь в виноградник ближнего твоего, можешь есть ягоды досыта, сколько \bibemph{хочет} душа твоя, а в сосуд твой не клади.
\vs Deu 23:25 Когда придешь на жатву ближнего твоего, срывай колосья руками твоими, но серпа не заноси на жатву ближнего твоего.
\vs Deu 24:1 Если кто возьмет жену и сделается ее мужем, и она не найдет благоволения в глазах его, потому что он находит в ней что-нибудь противное, и напишет ей разводное письмо, и даст ей в руки, и отпустит ее из дома своего,
\vs Deu 24:2 и она выйдет из дома его, пойдет, и выйдет за другого мужа,
\vs Deu 24:3 но и сей последний муж возненавидит ее и напишет ей разводное письмо, и даст ей в руки, и отпустит ее из дома своего, или умрет сей последний муж ее, взявший ее себе в жену,~---
\vs Deu 24:4 то не может первый ее муж, отпустивший ее, опять взять ее себе в жену, после того как она осквернена, ибо сие есть мерзость пред Господом [Богом твоим], и не порочь земли, которую Господь Бог твой дает тебе в удел.
\rsbpar\vs Deu 24:5 Если кто взял жену недавно, то пусть не идет на войну, и ничего не должно возлагать на него; пусть он остается свободен в доме своем в продолжение одного года и увеселяет жену свою, которую взял.
\rsbpar\vs Deu 24:6 Никто не должен брать в залог верхнего и нижнего жернова, ибо таковой берет в залог душу.
\rsbpar\vs Deu 24:7 Если найдут кого, что он украл кого-нибудь из братьев своих, из сынов Израилевых, и поработил его, и продал его, то такого вора должно предать смерти; и \bibemph{так} истреби зло из среды себя.
\rsbpar\vs Deu 24:8 Смотри, в язве проказы тщательно соблюдай и исполняй весь [закон], которому научат вас священники левиты; тщательно исполняйте, что я повелел им;
\vs Deu 24:9 помни, что Господь Бог твой сделал Мариами на пути, когда вы шли из Египта.
\rsbpar\vs Deu 24:10 Если ты ближнему твоему дашь что-нибудь взаймы, то не ходи к нему в дом, чтобы взять у него залог,
\vs Deu 24:11 постой на улице, а тот, которому ты дал взаймы, вынесет тебе залог свой на улицу;
\vs Deu 24:12 если же он будет человек бедный, то ты не ложись спать, имея [у себя] залог его:
\vs Deu 24:13 возврати ему залог при захождении солнца, чтоб он лег спать в одежде своей и благословил тебя,~--- и тебе поставится \bibemph{сие} в праведность пред Господом Богом твоим.
\vs Deu 24:14 Не обижай наемника, бедного и нищего, из братьев твоих или из пришельцев твоих, которые в земле твоей, в жилищах твоих;
\vs Deu 24:15 в тот же день отдай плату его, чтобы солнце не зашло прежде того, ибо он беден, и ждет ее душа его; чтоб он не возопил на тебя к Господу, и не было на тебе греха.
\rsbpar\vs Deu 24:16 Отцы не должны быть наказываемы смертью за детей, и дети не должны быть наказываемы смертью за отцов; каждый должен быть наказываем смертью за свое преступление.
\rsbpar\vs Deu 24:17 Не суди превратно пришельца, сироту [и вдову], и у вдовы не бери одежды в залог;
\vs Deu 24:18 помни, что и ты был рабом в Египте, и Господь [Бог твой] освободил тебя оттуда: посему я и повелеваю тебе делать сие.
\rsbpar\vs Deu 24:19 Когда будешь жать на поле твоем, и забудешь сноп на поле, то не возвращайся взять его; пусть он остается пришельцу, [нищему,] сироте и вдове, чтобы Господь Бог твой благословил тебя во всех делах рук твоих.
\vs Deu 24:20 Когда будешь обивать маслину твою, то не пересматривай за собою ветвей: пусть остается пришельцу, сироте и вдове. [И помни, что ты был рабом в земле Египетской: посему я и повелеваю тебе делать сие.]
\vs Deu 24:21 Когда будешь снимать плоды в винограднике твоем, не собирай остатков за собою: пусть остается пришельцу, сироте и вдове;
\vs Deu 24:22 и помни, что ты был рабом в земле Египетской: посему я и повелеваю тебе делать сие.
\vs Deu 25:1 Если будет тяжба между людьми, то пусть приведут их в суд и рассудят их, правого пусть оправдают, а виновного осудят;
\vs Deu 25:2 и если виновный достоин будет побоев, то судья пусть прикажет положить его и бить при себе, смотря по вине его, по счету;
\vs Deu 25:3 сорок ударов можно дать ему, а не более, чтобы от многих ударов брат твой не был обезображен пред глазами твоими.
\rsbpar\vs Deu 25:4 Не заграждай рта волу, когда он молотит.
\rsbpar\vs Deu 25:5 Если братья живут вместе и один из них умрет, не имея у себя сына, то жена умершего не должна выходить на сторону за человека чужого, но деверь ее должен войти к ней и взять ее себе в жену, и жить с нею,~---
\vs Deu 25:6 и первенец, которого она родит, останется с именем брата его умершего, чтоб имя его не изгладилось в Израиле.
\vs Deu 25:7 Если же он не захочет взять невестку свою, то невестка его пойдет к воротам, к старейшинам, и скажет: <<деверь мой отказывается восставить имя брата своего в Израиле, не хочет жениться на мне>>;
\vs Deu 25:8 тогда старейшины города его должны призвать его и уговаривать его, и если он станет и скажет: <<не хочу взять ее>>,
\vs Deu 25:9 \bibemph{тогда} невестка его пусть пойдет к нему в глазах старейшин, и снимет сапог его с ноги его, и плюнет в лице его, и скажет: <<так поступают с человеком, который не созидает дома брату своему [у Израиля]>>;
\vs Deu 25:10 и нарекут ему имя в Израиле: дом разутого.
\rsbpar\vs Deu 25:11 Когда дерутся между собою мужчины, и жена одного [из них] подойдет, чтобы отнять мужа своего из рук бьющего его, и протянув руку свою, схватит его за срамный уд,
\vs Deu 25:12 то отсеки руку ее: да не пощадит [ее] глаз твой.
\rsbpar\vs Deu 25:13 В кисе твоей не должны быть двоякие гири, б\acc{о}льшие и меньшие;
\vs Deu 25:14 в доме твоем не должна быть двоякая ефа, б\acc{о}льшая и меньшая;
\vs Deu 25:15 гиря у тебя должна быть точная и правильная, и ефа у тебя должна быть точная и правильная, чтобы продлились дни твои на земле, которую Господь Бог твой дает тебе [в удел];
\vs Deu 25:16 ибо мерзок пред Господом Богом твоим всякий делающий неправду.
\vs Deu 25:17 Помни, как поступил с тобою Амалик на пути, когда вы шли из Египта:
\vs Deu 25:18 как он встретил тебя на пути, и побил сзади тебя всех ослабевших, когда ты устал и утомился, и не побоялся он Бога;
\vs Deu 25:19 итак, когда Господь Бог твой успокоит тебя от всех врагов твоих со всех сторон, на земле, которую Господь Бог твой дает тебе в удел, чтоб овладеть ею, изгладь память Амалика из поднебесной; не забудь.
\vs Deu 26:1 Когда ты придешь в землю, которую Господь Бог твой дает тебе в удел, и овладеешь ею, и поселишься в ней;
\vs Deu 26:2 то возьми начатков всех плодов земли, которые ты получишь от земли твоей, которую Господь Бог твой дает тебе, и положи в корзину, и пойди на то место, которое Господь Бог твой изберет, чтобы пребывало там имя Его;
\vs Deu 26:3 и приди к священнику, который будет в те дни, и скажи ему: сегодня исповедую пред Господом Богом твоим, что я вошел в ту землю, которую Господь клялся отцам нашим дать нам.
\vs Deu 26:4 Священник возьмет корзину из руки твоей и поставит ее пред жертвенником Господа Бога твоего.
\vs Deu 26:5 Ты же отвечай и скажи пред Господом Богом твоим: отец мой был странствующий Арамеянин, и пошел в Египет и поселился там с немногими людьми, и произошел там от него народ великий, сильный и многочисленный;
\vs Deu 26:6 но Египтяне худо поступали с нами, и притесняли нас, и налагали на нас тяжкие работы;
\vs Deu 26:7 и возопили мы к Господу Богу отцов наших, и услышал Господь вопль наш и увидел бедствие наше, труды наши и угнетение наше;
\vs Deu 26:8 и вывел нас Господь из Египта [Сам крепостию Своею великою и] рукою сильною и мышцею простертою, великим ужасом, знамениями и чудесами,
\vs Deu 26:9 и привел нас на место сие, и дал нам землю сию, землю, в которой течет молоко и мед;
\vs Deu 26:10 итак вот, я принес начатки плодов от земли, которую Ты, Господи, дал мне, [от земли, где течет молоко и мед]. И поставь это пред Господом Богом твоим, и поклонись пред Господом Богом твоим,
\vs Deu 26:11 и веселись о всех благах, которые Господь Бог твой дал тебе и дому твоему, ты и левит и пришелец, который будет у тебя.
\vs Deu 26:12 Когда ты отделишь все десятины произведений [земли] твоей в третий год, год десятин, и отдашь левиту, пришельцу, сироте и вдове, чтоб они ели в жилищах твоих и насыщались,
\vs Deu 26:13 тогда скажи пред Господом Богом твоим: я отобрал от дома [моего] святыню и отдал ее левиту, пришельцу, сироте и вдове, по всем повелениям Твоим, которые Ты заповедал мне: я не преступил заповедей Твоих и не забыл;
\vs Deu 26:14 я не ел от нее в печали моей, и не отделял ее в нечистоте, и не давал из нее для мертвого; я повиновался гласу Господа Бога моего, исполнил все, что Ты заповедал мне;
\vs Deu 26:15 призри от святого жилища Твоего, с небес, и благослови народ Твой, Израиля, и землю, которую Ты дал нам~--- так как Ты клялся отцам нашим [дать нам] землю, в которой течет молоко и мед.
\vs Deu 26:16 В день сей Господь Бог твой завещевает тебе исполнять [все] постановления сии и законы: соблюдай и исполняй их от всего сердца твоего и от всей души твоей.
\vs Deu 26:17 Господу сказал ты ныне, что Он будет твоим Богом, и что ты будешь ходить путями Его и хранить постановления Его и заповеди Его и законы Его, и слушать гласа Его;
\vs Deu 26:18 и Господь обещал тебе ныне, что ты будешь собственным Его народом, как Он говорил тебе, если ты будешь хранить все заповеди Его,
\vs Deu 26:19 и что Он поставит тебя выше всех народов, которых Он сотворил, в чести, славе и великолепии, и что ты будешь святым народом у Господа Бога твоего, как Он говорил.
\vs Deu 27:1 И заповедал Моисей и старейшины [сынов] Израилевых народу, говоря: исполняйте все заповеди, которые заповедую вам ныне.
\vs Deu 27:2 И когда перейдете за Иордан, в землю, которую Господь Бог твой дает тебе, тогда поставь себе большие камни и обмажь их известью;
\vs Deu 27:3 и напиши на [камнях] сих все слова закона сего, когда перейдешь [Иордан], чтобы вступить в землю, которую Господь Бог твой дает тебе, в землю, где течет молоко и мед, как говорил тебе Господь Бог отцов твоих.
\vs Deu 27:4 Когда перейдете Иордан, поставьте камни те, как я повелеваю вам сегодня, на горе Гевал, и обмажьте их известью;
\vs Deu 27:5 и устрой там жертвенник Господу Богу твоему, жертвенник из камней, не поднимая на них железа;
\vs Deu 27:6 из камней цельных устрой жертвенник Господа Бога твоего, и возноси на нем всесожжения Господу Богу твоему,
\vs Deu 27:7 и приноси жертвы мирные, и ешь [и насыщайся] там, и веселись пред Господом Богом твоим;
\vs Deu 27:8 и напиши на камнях [сих] все слова закона сего очень явственно.
\rsbpar\vs Deu 27:9 И сказал Моисей и священники левиты всему Израилю, говоря: внимай и слушай, Израиль: в день сей ты сделался народом Господа Бога твоего;
\vs Deu 27:10 итак слушай гласа Господа Бога твоего и исполняй [все] заповеди Его и постановления Его, которые заповедую тебе сегодня.
\vs Deu 27:11 И заповедал Моисей народу в день тот, говоря:
\vs Deu 27:12 сии должны стать на горе Гаризим, чтобы благословлять народ, когда перейдете Иордан: Симеон, Левий, Иуда, Иссахар, Иосиф и Вениамин;
\vs Deu 27:13 а сии должны стать на горе Гевал, чтобы \bibemph{произносить} проклятие: Рувим, Гад, Асир, Завулон, Дан и Неффалим.
\vs Deu 27:14 Левиты возгласят и скажут всем Израильтянам громким голосом:
\vs Deu 27:15 проклят, кто сделает изваянный или литой кумир, мерзость пред Господом, произведение рук художника, и поставит его в тайном месте! Весь народ возгласит и скажет: аминь.
\vs Deu 27:16 Проклят злословящий отца своего или матерь свою! И весь народ скажет: аминь.
\vs Deu 27:17 Проклят нарушающий межи ближнего своего! И весь народ скажет: аминь.
\vs Deu 27:18 Проклят, кто слепого сбивает с пути! И весь народ скажет: аминь.
\vs Deu 27:19 Проклят, кто превратно судит пришельца, сироту и вдову! И весь народ скажет: аминь.
\vs Deu 27:20 Проклят, кто ляжет с женою отца своего, ибо он открыл край \bibemph{одежды} отца своего! И весь народ скажет: аминь.
\vs Deu 27:21 Проклят, кто ляжет с каким-либо скотом! И весь народ скажет: аминь.
\vs Deu 27:22 Проклят, кто ляжет с сестрою своею, с дочерью отца своего, или дочерью матери своей! И весь народ скажет: аминь.
\vs Deu 27:23 Проклят, кто ляжет с тещею своею! И весь народ скажет: аминь. [Проклят, кто ляжет с сестрою жены своей! И весь народ скажет: аминь.]
\rsbpar\vs Deu 27:24 Проклят, кто тайно убивает ближнего своего! И весь народ скажет: аминь.
\vs Deu 27:25 Проклят, кто берет подкуп, чтоб убить душу \bibemph{и пролить} кровь невинную! И весь народ скажет: аминь.
\vs Deu 27:26 Проклят [всякий человек], кто не исполнит [всех] слов закона сего и не будет поступать по ним! И весь народ скажет: аминь.
\vs Deu 28:1 Если ты, когда перейдете [за Иордан в землю, которую Господь Бог ваш дает вам], будешь слушать гласа Господа Бога твоего, тщательно исполнять все заповеди Его, которые заповедую тебе сегодня, то Господь Бог твой поставит тебя выше всех народов земли;
\vs Deu 28:2 и придут на тебя все благословения сии и исполнятся на тебе, если будешь слушать гласа Господа, Бога твоего.
\vs Deu 28:3 Благословен ты в городе и благословен на поле.
\vs Deu 28:4 Благословен плод чрева твоего, и плод земли твоей, и плод скота твоего, и плод твоих волов, и плод овец твоих.
\vs Deu 28:5 Благословенны житницы твои и кладовые твои.
\vs Deu 28:6 Благословен ты при входе твоем и благословен ты при выходе твоем.
\vs Deu 28:7 Поразит пред тобою Господь врагов твоих, восстающих на тебя; одним путем они выступят против тебя, а семью путями побегут от тебя.
\vs Deu 28:8 Пошлет Господь тебе благословение в житницах твоих и во всяком деле рук твоих; и благословит тебя на земле, которую Господь Бог твой дает тебе.
\vs Deu 28:9 Поставит тебя Господь [Бог твой] народом святым Своим, как Он клялся тебе [и отцам твоим], если ты будешь соблюдать заповеди Господа Бога твоего и будешь ходить путями Его;
\vs Deu 28:10 и увидят все народы земли, что имя Господа [Бога твоего] нарицается на тебе, и убоятся тебя.
\vs Deu 28:11 И даст тебе Господь [Бог твой] изобилие во всех благах, в плоде чрева твоего, и в плоде скота твоего, и в плоде полей твоих на земле, которую Господь клялся отцам твоим дать тебе.
\vs Deu 28:12 Откроет тебе Господь добрую сокровищницу Свою, небо, чтоб оно давало дождь земле твоей во время свое, и чтобы благословлять все дела рук твоих: и будешь давать взаймы многим народам, а сам не будешь брать взаймы [и будешь господствовать над многими народами, а они над тобою не будут господствовать].
\vs Deu 28:13 Сделает тебя Господь [Бог твой] главою, а не хвостом, и будешь только на высоте, а не будешь внизу, если будешь повиноваться заповедям Господа Бога твоего, которые заповедую тебе сегодня хранить и исполнять,
\vs Deu 28:14 и не отступишь от всех слов, которые заповедую вам сегодня, ни направо ни налево, чтобы пойти вслед иных богов \bibemph{и} служить им.
\vs Deu 28:15 Если же не будешь слушать гласа Господа Бога твоего и не будешь стараться исполнять все заповеди Его и постановления Его, которые я заповедую тебе сегодня, то придут на тебя все проклятия сии и постигнут тебя.
\vs Deu 28:16 Проклят ты [будешь] в городе и проклят ты [будешь] на поле.
\vs Deu 28:17 Прокляты [будут] житницы твои и кладовые твои.
\vs Deu 28:18 Проклят [будет] плод чрева твоего и плод земли твоей, плод твоих волов и плод овец твоих.
\vs Deu 28:19 Проклят ты [будешь] при входе твоем и проклят при выходе твоем.
\vs Deu 28:20 Пошлет Господь на тебя проклятие, смятение и несчастье во всяком деле рук твоих, какое ни станешь ты делать, доколе не будешь истреблен,~--- и ты скоро погибнешь за злые дела твои, за то, что ты оставил Меня.
\vs Deu 28:21 Пошлет Господь на тебя моровую язву, доколе не истребит Он тебя с земли, в которую ты идешь, чтобы владеть ею.
\vs Deu 28:22 Поразит тебя Господь чахлостью, горячкою, лихорадкою, воспалением, засухою, палящим ветром и ржавчиною, и они будут преследовать тебя, доколе не погибнешь.
\vs Deu 28:23 И небеса твои, которые над головою твоею, сделаются медью, и земля под тобою железом;
\vs Deu 28:24 вместо дождя Господь даст земле твоей пыль, и прах с неба будет падать, падать на тебя, [доколе не погубит тебя и] доколе не будешь истреблен.
\vs Deu 28:25 Предаст тебя Господь на поражение врагам твоим; одним путем выступишь против них, а семью путями побежишь от них; и будешь рассеян по всем царствам земли.
\vs Deu 28:26 И будут трупы твои пищею всем птицам небесным и зверям, и не будет отгоняющего их.
\vs Deu 28:27 Поразит тебя Господь проказою Египетскою, почечуем, коростою и чесоткою, от которых ты не возможешь исцелиться;
\vs Deu 28:28 поразит тебя Господь сумасшествием, слепотою и оцепенением сердца.
\vs Deu 28:29 И ты будешь ощупью ходить в полдень, как слепой ощупью ходит впотьмах, и не будешь иметь успеха в путях твоих, и будут теснить и обижать тебя всякий день, и никто не защитит тебя.
\vs Deu 28:30 С женою обручишься, и другой будет спать с нею; дом построишь, и не будешь жить в нем; виноградник насадишь, и не будешь пользоваться им.
\vs Deu 28:31 Вола твоего заколют в глазах твоих, и не будешь есть его; осла твоего уведут от тебя и не возвратят тебе; овцы твои отданы будут врагам твоим, и никто не защитит тебя.
\vs Deu 28:32 Сыновья твои и дочери твои будут отданы другому народу; глаза твои будут видеть и всякий день истаевать о них, и не будет силы в руках твоих.
\vs Deu 28:33 Плоды земли твоей и все труды твои будет есть народ, которого ты не знал; и ты будешь только притесняем и мучим во все дни.
\vs Deu 28:34 И сойдешь с ума от того, что будут видеть глаза твои.
\vs Deu 28:35 Поразит тебя Господь злою проказою на коленях и голенях, от которой ты не возможешь исцелиться, от подошвы ноги твоей до самого темени [головы] твоей.
\vs Deu 28:36 Отведет Господь тебя и царя твоего, которого ты поставишь над собою, к народу, которого не знал ни ты, ни отцы твои, и там будешь служить иным богам, деревянным и каменным;
\vs Deu 28:37 и будешь ужасом, притчею и посмешищем у всех народов, к которым отведет тебя Господь [Бог].
\vs Deu 28:38 Семян много вынесешь в поле, а соберешь мало, потому что поест их саранча.
\vs Deu 28:39 Виноградники будешь садить и возделывать, а вина не будешь пить, и не соберешь \bibemph{плодов} [их], потому что поест их червь.
\vs Deu 28:40 Маслины будут у тебя во всех пределах твоих, но елеем не помажешься, потому что осыплется маслина твоя.
\vs Deu 28:41 Сынов и дочерей родишь, но их не будет у тебя, потому что пойдут в плен.
\vs Deu 28:42 Все дерева твои и плоды земли твоей погубит ржавчина.
\vs Deu 28:43 Пришелец, который среди тебя, будет возвышаться над тобою выше и выше, а ты опускаться будешь ниже и ниже;
\vs Deu 28:44 он будет давать тебе взаймы, а ты не будешь давать ему взаймы; он будет главою, а ты будешь хвостом.
\vs Deu 28:45 И придут на тебя все проклятия сии, и будут преследовать тебя и постигнут тебя, доколе не будешь истреблен, за то, что ты не слушал гласа Господа Бога твоего и не соблюдал заповедей Его и постановлений Его, которые Он заповедал тебе:
\vs Deu 28:46 они будут знамением и указанием на тебе и на семени твоем вовек.
\vs Deu 28:47 За то, что ты не служил Господу Богу твоему с веселием и радостью сердца, при изобилии всего,
\vs Deu 28:48 будешь служить врагу твоему, которого пошлет на тебя Господь [Бог твой], в голоде, и жажде, и наготе и во всяком недостатке; он возложит на шею твою железное ярмо, так что измучит тебя.
\vs Deu 28:49 Пошлет на тебя Господь народ издалека, от края земли: как орел налетит народ, которого языка ты не разумеешь,
\vs Deu 28:50 народ наглый, который не уважит старца и не пощадит юноши;
\vs Deu 28:51 и будет он есть плод скота твоего и плод земли твоей, доколе не разорит тебя, так что не оставит тебе ни хлеба, ни вина, ни елея, ни плода волов твоих, ни плода овец твоих, доколе не погубит тебя;
\vs Deu 28:52 и будет теснить тебя во всех жилищах твоих, доколе во всей земле твоей не разрушит высоких и крепких стен твоих, на которые ты надеешься; и будет теснить тебя во всех жилищах твоих, во всей земле твоей, которую Господь Бог твой дал тебе.
\vs Deu 28:53 И ты будешь есть плод чрева твоего, плоть сынов твоих и дочерей твоих, которых Господь Бог твой дал тебе, в осаде и в стеснении, в котором стеснит тебя враг твой.
\vs Deu 28:54 Муж, изнеженный и живший между вами в великой роскоши, безжалостным оком будет смотреть на брата своего, на жену недра своего и на остальных детей своих, которые останутся у него,
\vs Deu 28:55 и не даст ни одному из них плоти детей своих, которых он будет есть, потому что у него не останется ничего в осаде и в стеснении, в котором стеснит тебя враг твой во всех жилищах твоих.
\vs Deu 28:56 [Женщина] жившая у тебя в неге и роскоши, которая никогда ноги своей не ставила на землю по причине роскоши и изнеженности, будет безжалостным оком смотреть на мужа недра своего и на сына своего и на дочь свою
\vs Deu 28:57 и \bibemph{не даст} им последа, выходящего из среды ног ее, и детей, которых она родит; потому что она, при недостатке во всем, тайно будет есть их, в осаде и стеснении, в котором стеснит тебя враг твой в жилищах твоих.
\vs Deu 28:58 Если не будешь стараться исполнять все слова закона сего, написанные в книге сей, и не будешь бояться сего славного и страшного имени Господа Бога твоего,
\vs Deu 28:59 то Господь поразит тебя и потомство твое необычайными язвами, язвами великими и постоянными, и болезнями злыми и постоянными;
\vs Deu 28:60 и наведет на тебя все [злые] язвы Египетские, которых ты боялся, и они прилипнут к тебе;
\vs Deu 28:61 и всякую болезнь и всякую язву, не написанную [и всякую написанную] в книге закона сего, Господь наведет на тебя, доколе не будешь истреблен;
\vs Deu 28:62 и останется вас немного, тогда как множеством вы подобны были звездам небесным, ибо ты не слушал гласа Господа Бога твоего.
\vs Deu 28:63 И как радовался Господь, делая вам добро и умножая вас, так будет радоваться Господь, погубляя вас и истребляя вас, и извержены будете из земли, в которую ты идешь, чтобы владеть ею.
\vs Deu 28:64 И рассеет тебя Господь [Бог твой] по всем народам, от края земли до края земли, и будешь там служить иным богам, которых не знал ни ты, ни отцы твои, дереву и камням.
\vs Deu 28:65 Но и между этими народами не успокоишься, и не будет места покоя для ноги твоей, и Господь даст тебе там трепещущее сердце, истаевание очей и изнывание души;
\vs Deu 28:66 жизнь твоя будет висеть пред тобою, и будешь трепетать ночью и днем, и не будешь уверен в жизни твоей;
\vs Deu 28:67 от трепета сердца твоего, которым ты будешь объят, и от того, что ты будешь видеть глазами твоими, утром ты скажешь: <<о, если бы пришел вечер!>>, а вечером скажешь: <<о, если бы наступило утро!>>
\vs Deu 28:68 и возвратит тебя Господь в Египет на кораблях тем путем, о котором я сказал тебе: <<ты более не увидишь его>>; и там будете продаваться врагам вашим в рабов и в рабынь, и не будет покупающего.
\vs Deu 29:1 Вот слова завета, который Господь повелел Моисею поставить с сынами Израилевыми в земле Моавитской, кроме завета, который Господь поставил с ними на Хориве.
\rsbpar\vs Deu 29:2 И созвал Моисей всех [сынов] Израилевых и сказал им: вы видели всё, что сделал Господь пред глазами вашими в земле Египетской с фараоном и всеми рабами его и всею землею его;
\vs Deu 29:3 те великие казни, которые видели глаза твои, и те великие знамения и чудеса, [руку крепкую и мышцу простертую];
\vs Deu 29:4 но до сего дня не дал вам Господь [Бог] сердца, чтобы разуметь, очей, чтобы видеть, и ушей, чтобы слышать.
\vs Deu 29:5 Сорок лет водил вас по пустыне, и одежды ваши на вас не обветшали, и обувь твоя не обветшала на ноге твоей;
\vs Deu 29:6 хлеба вы не ели и вина и сикера не пили, дабы вы знали, что Я Господь Бог ваш.
\vs Deu 29:7 И когда пришли вы на место сие, выступил против нас Сигон, царь Есевонский, и Ог, царь Васанский, чтобы сразиться \bibemph{с нами}, и мы поразили их;
\vs Deu 29:8 и взяли землю их и отдали ее в удел \bibemph{колену} Рувимову и Гадову и половине колена Манассиина.
\vs Deu 29:9 Соблюдайте же [все] слова завета сего и исполняйте их, чтобы вам иметь успех во всем, что ни будете делать.
\vs Deu 29:10 Все вы сегодня стоите пред лицем Господа Бога вашего, начальники колен ваших, старейшины ваши, [судьи ваши,] надзиратели ваши, все Израильтяне,
\vs Deu 29:11 дети ваши, жены ваши и пришельцы твои, находящиеся в стане твоем, от секущего дрова твои до черпающего воду твою,
\vs Deu 29:12 чтобы вступить тебе в завет Господа Бога твоего и в клятвенный договор с Ним, который Господь Бог твой сегодня поставляет с тобою,
\vs Deu 29:13 дабы соделать тебя сегодня Его народом, и Ему быть тебе Богом, как Он говорил тебе и как клялся отцам твоим Аврааму, Исааку и Иакову.
\vs Deu 29:14 Не с вами только одними я поставляю сей завет и сей клятвенный договор,
\vs Deu 29:15 но как с теми, которые сегодня здесь с нами стоят пред лицем Господа Бога нашего, так и с теми, которых нет здесь с нами сегодня.
\vs Deu 29:16 Ибо вы знаете, как мы жили в земле Египетской и как мы проходили посреди народов, чрез которые вы прошли,
\vs Deu 29:17 и видели мерзости их и кумиры их, деревянные и каменные, серебряные и золотые, которые у них.
\vs Deu 29:18 Да не будет между вами мужчины или женщины, или рода или колена, которых сердце уклонилось бы ныне от Господа Бога нашего, чтобы ходить служить богам тех народов; да не будет между вами корня, произращающего яд и полынь,
\vs Deu 29:19 такого человека, который, услышав слова проклятия сего, похвалялся бы в сердце своем, говоря: <<я буду счастлив, несмотря на то, что буду ходить по произволу сердца моего>>; и пропадет таким образом сытый с голодным;
\vs Deu 29:20 не простит Господь такому, но тотчас возгорится гнев Господа и ярость Его на такого человека, и падет на него все проклятие [завета сего], написанное в сей книге [закона], и изгладит Господь имя его из поднебесной;
\vs Deu 29:21 и отделит его Господь на погибель от всех колен Израилевых, сообразно со всеми проклятиями завета, написанными в сей книге закона.
\vs Deu 29:22 И скажет последующий род, дети ваши, которые будут после вас, и чужеземец, который придет из земли дальней, увидев поражение земли сей и болезни, которыми изнурит ее Господь:
\vs Deu 29:23 сера и соль, пожарище~--- вся земля; не засевается и не произращает она, и не выходит на ней никакой травы, как по истреблении Содома, Гоморры, Адмы и Севоима, которые ниспроверг Господь во гневе Своем и в ярости Своей.
\vs Deu 29:24 И скажут все народы: за что Господь так поступил с сею землею? какая великая ярость гнева Его!
\vs Deu 29:25 И скажут: за то, что они оставили завет Господа Бога отцов своих, который Он поставил с ними, когда вывел их из земли Египетской,
\vs Deu 29:26 и пошли и стали служить иным богам и поклоняться им, богам, которых они не знали и \bibemph{которых} Он не назначал им:
\vs Deu 29:27 \bibemph{за то} возгорелся гнев Господа на землю сию, и навел Он на нее все проклятия [завета], написанные в сей книге [закона],
\vs Deu 29:28 и извергнул их Господь из земли их в гневе, ярости и великом негодовании, и поверг их на другую землю, как ныне \bibemph{видим}.
\vs Deu 29:29 Сокрытое \bibemph{принадлежит} Господу Богу нашему, а открытое~--- нам и сынам нашим до века, чтобы мы исполняли все слова закона сего.
\vs Deu 30:1 Когда придут на тебя все слова сии~--- благословение и проклятие, которые изложил я тебе, и примешь \bibemph{их} к сердцу своему среди всех народов, в которых рассеет тебя Господь Бог твой,
\vs Deu 30:2 и обратишься к Господу Богу твоему и послушаешь гласа Его, как я заповедую тебе сегодня, ты и сыны твои от всего сердца твоего и от всей души твоей,~---
\vs Deu 30:3 тогда Господь Бог твой возвратит пленных твоих и умилосердится над тобою, и опять соберет тебя от всех народов, между которыми рассеет тебя Господь Бог твой.
\vs Deu 30:4 Хотя бы ты был рассеян [от края неба] до края неба, и оттуда соберет тебя Господь Бог твой, и оттуда возьмет тебя,
\vs Deu 30:5 и [оттуда] приведет тебя Господь Бог твой в землю, которою владели отцы твои, и получишь ее во владение, и облагодетельствует тебя и размножит тебя более отцов твоих;
\vs Deu 30:6 и обрежет Господь Бог твой сердце твое и сердце потомства твоего, чтобы ты любил Господа Бога твоего от всего сердца твоего и от всей души твоей, дабы жить тебе;
\vs Deu 30:7 тогда Господь Бог твой все проклятия сии обратит на врагов твоих и ненавидящих тебя, которые гнали тебя,
\vs Deu 30:8 а ты обратишься и будешь слушать гласа Господа [Бога твоего] и исполнять все заповеди Его, которые заповедую тебе сегодня;
\vs Deu 30:9 с избытком даст тебе Господь Бог твой успех во всяком деле рук твоих, в плоде чрева твоего, в плоде скота твоего, в плоде земли твоей; ибо снова радоваться будет Господь [Бог твой] о тебе, благодетельствуя \bibemph{тебе}, как Он радовался об отцах твоих,
\vs Deu 30:10 если будешь слушать гласа Господа Бога твоего, соблюдая [и исполняя все] заповеди Его и постановления Его [и законы Его], написанные в сей книге закона, и если обратишься к Господу Богу твоему всем сердцем твоим и всею душею твоею.
\vs Deu 30:11 Ибо заповедь сия, которую я заповедую тебе сегодня, не недоступна для тебя и не далека;
\vs Deu 30:12 она не на небе, чтобы можно \bibemph{было} говорить: <<кто взошел бы для нас на небо и принес бы ее нам, и дал бы нам услышать ее, и мы исполнили бы ее?>>
\vs Deu 30:13 и не за морем она, чтобы можно \bibemph{было} говорить: <<кто сходил бы для нас за море и принес бы ее нам, и дал бы нам услышать ее, и мы исполнили бы ее?>>
\vs Deu 30:14 но весьма близко к тебе слово сие: \bibemph{оно} в устах твоих и в сердце твоем, чтобы исполнять его.
\vs Deu 30:15 Вот, я сегодня предложил тебе жизнь и добро, смерть и зло.
\vs Deu 30:16 [Если будешь слушать заповеди Господа Бога твоего,] которые заповедую тебе сегодня, любить Господа Бога твоего, ходить по [всем] путям Его и исполнять заповеди Его и постановления Его и законы Его, то будешь жить и размножишься, и благословит тебя Господь Бог твой на земле, в которую ты идешь, чтоб овладеть ею;
\vs Deu 30:17 если же отвратится сердце твое, и не будешь слушать, и заблудишь, и станешь поклоняться иным богам и будешь служить им,
\vs Deu 30:18 то я возвещаю вам сегодня, что вы погибнете и не пробудете долго на земле, [которую Господь Бог дает тебе,] для овладения которою ты переходишь Иордан.
\vs Deu 30:19 Во свидетели пред вами призываю сегодня небо и землю: жизнь и смерть предложил я тебе, благословение и проклятие. Избери жизнь, дабы жил ты и потомство твое,
\vs Deu 30:20 любил Господа Бога твоего, слушал глас Его и прилеплялся к Нему; ибо в этом жизнь твоя и долгота дней твоих, чтобы пребывать тебе на земле, которую Господь [Бог] с клятвою обещал отцам твоим Аврааму, Исааку и Иакову дать им.
\vs Deu 31:1 И пошел Моисей, и говорил слова сии всем [сынам] Израиля,
\vs Deu 31:2 и сказал им: теперь мне сто двадцать лет, я не могу уже выходить и входить, и Господь сказал мне: <<ты не перейдешь Иордан сей>>;
\vs Deu 31:3 Господь Бог твой Сам пойдет пред тобою; Он истребит народы сии от лица твоего, и ты овладеешь ими; Иисус пойдет пред тобою, как говорил Господь;
\vs Deu 31:4 и поступит Господь с ними так же, как Он поступил с Сигоном и Огом, царями Аморрейскими, [которые были по эту сторону Иордана,] и с землею их, которых он истребил;
\vs Deu 31:5 и предаст их Господь вам, и вы поступите с ними по всем заповедям, какие заповедал я вам;
\vs Deu 31:6 будьте тверды и мужественны, не бойтесь, [не ужасайтесь] и не страшитесь их, ибо Господь Бог твой Сам пойдет с тобою [и] не отступит от тебя и не оставит тебя.
\rsbpar\vs Deu 31:7 И призвал Моисей Иисуса и пред очами всех Израильтян сказал ему: будь тверд и мужествен, ибо ты войдешь с народом сим в землю, которую Господь клялся отцам его дать ему, и ты разделишь ее на уделы ему;
\vs Deu 31:8 Господь Сам пойдет пред тобою, Сам будет с тобою, не отступит от тебя и не оставит тебя, не бойся и не ужасайся.
\vs Deu 31:9 И написал Моисей закон сей, и отдал его священникам, сынам Левииным, носящим ковчег завета Господня, и всем старейшинам [сынов] Израилевых.
\vs Deu 31:10 И завещал им Моисей и сказал: по прошествии семи лет, в год отпущения, в праздник кущей,
\vs Deu 31:11 когда весь Израиль придет явиться пред лице Господа Бога твоего на место, которое изберет [Господь], читай сей закон пред всем Израилем вслух его;
\vs Deu 31:12 собери народ, мужей и жен, и детей, и пришельцев твоих, которые будут в жилищах твоих, чтоб они слушали и учились, и чтобы боялись Господа Бога вашего, и старались исполнять все слова закона сего;
\vs Deu 31:13 и сыны их, которые не знают \bibemph{сего}, услышат и научатся бояться Господа Бога вашего во все дни, доколе вы будете жить на земле, в которую вы переходите за Иордан, чтоб овладеть ею.
\rsbpar\vs Deu 31:14 И сказал Господь Моисею: вот, дни твои приблизились к смерти; призови Иисуса и станьте у [входа] скинии собрания, и Я дам ему наставления. И пришел Моисей и Иисус, и стали у [входа] скинии собрания.
\vs Deu 31:15 И явился Господь в скинии в столпе облачном, и стал столп облачный у входа скинии [собрания].
\vs Deu 31:16 И сказал Господь Моисею: вот, ты почиешь с отцами твоими, и станет народ сей блудно ходить вслед чужих богов той земли, в которую он вступает, и оставит Меня, и нарушит завет Мой, который Я поставил с ним;
\vs Deu 31:17 и возгорится гнев Мой на него в тот день, и Я оставлю их и сокрою лице Мое от них, и он истреблен будет, и постигнут его многие бедствия и скорби, и скажет он в тот день: <<не потому ли постигли меня сии бедствия, что нет [Господа] Бога моего среди меня?>>
\vs Deu 31:18 и Я сокрою лице Мое [от него] в тот день за все беззакония его, которые он сделает, обратившись к иным богам.
\vs Deu 31:19 Итак напишите себе [слова] песни сей, и научи ей сынов Израилевых, и вложи ее в уста их, чтобы песнь сия была Мне свидетельством на сынов Израилевых;
\vs Deu 31:20 ибо Я введу их в землю [добрую], как Я клялся отцам их, где течет молоко и мед, и они будут есть и насыщаться, и утучнеют, и обратятся к иным богам, и будут служить им, а Меня отвергнут и нарушат завет Мой, [который Я завещал им];
\vs Deu 31:21 и когда постигнут их многие бедствия и скорби, тогда песнь сия будет против них свидетельством, ибо она не выйдет [из уст их и] из уст потомства их. Я знаю мысли их, которые они имеют ныне, прежде нежели Я ввел их в [добрую] землю, о которой Я клялся [отцам их].
\vs Deu 31:22 И написал Моисей песнь сию в тот день и научил ей сынов Израилевых.
\vs Deu 31:23 И заповедал Господь Иисусу, сыну Навину, и сказал [ему]: будь тверд и мужествен, ибо ты введешь сынов Израилевых в землю, о которой Я клялся им, и Я буду с тобою.
\rsbpar\vs Deu 31:24 Когда Моисей вписал в книгу все слова закона сего до конца,
\vs Deu 31:25 тогда Моисей повелел левитам, носящим ковчег завета Господня, сказав:
\vs Deu 31:26 возьмите сию книгу закона и положите ее одесную ковчега завета Господа Бога вашего, и она там будет свидетельством против тебя;
\vs Deu 31:27 ибо я знаю упорство твое и жестоковыйность твою: вот и теперь, когда я живу с вами ныне, вы упорны пред Господом; не тем ли более по смерти моей?
\vs Deu 31:28 соберите ко мне всех старейшин колен ваших [и судей ваших] и надзирателей ваших, и я скажу вслух их слова сии и призову во свидетельство на них небо и землю;
\vs Deu 31:29 ибо я знаю, что по смерти моей вы развратитесь и уклонитесь от пути, который я завещал вам, и в последствие времени постигнут вас бедствия за то, что вы будете делать зло пред очами Господа [Бога], раздражая Его делами рук своих.
\vs Deu 31:30 И изрек Моисей вслух всего собрания Израильтян слова песни сей до конца:
\vs Deu 32:1 Внимай, небо, я буду говорить; и слушай, земля, слова уст моих.
\vs Deu 32:2 Польется как дождь учение мое, как роса речь моя, как мелкий дождь на зелень, как ливень на траву.
\vs Deu 32:3 Имя Господа прославляю; воздайте славу Богу нашему.
\vs Deu 32:4 Он твердыня; совершенны дела Его, и все пути Его праведны; Бог верен, и нет неправды [в Нем]; Он праведен и истинен;
\vs Deu 32:5 но они развратились пред Ним, они не дети Его по своим порокам, род строптивый и развращенный.
\vs Deu 32:6 Сие ли воздаете вы Господу, народ глупый и несмысленный? не Он ли Отец твой, \bibemph{Который} усвоил тебя, создал тебя и устроил тебя?
\vs Deu 32:7 Вспомни дни древние, помысли о летах прежних родов; спроси отца твоего, и он возвестит тебе, старцев твоих, и они скажут тебе.
\vs Deu 32:8 Когда Всевышний давал уделы народам и расселял сынов человеческих, тогда поставил пределы народов по числу сынов Израилевых\fns{В греческом переводе: по числу Ангелов Божиих.};
\vs Deu 32:9 ибо часть Господа народ Его, Иаков наследственный удел Его.
\vs Deu 32:10 Он нашел его в пустыне, в степи печальной и дикой, ограждал его, смотрел за ним, хранил его, как зеницу ока Своего;
\vs Deu 32:11 как орел вызывает гнездо свое, носится над птенцами своими, распростирает крылья свои, берет их и носит их на перьях своих,
\vs Deu 32:12 так Господь один водил его, и не было с Ним чужого бога.
\vs Deu 32:13 Он вознес его на высоту земли и кормил произведениями полей, и питал его медом из камня и елеем из твердой скалы,
\vs Deu 32:14 маслом коровьим и молоком овечьим, и туком агнцев и овнов Васанских и козлов, и тучною пшеницею, и ты пил вино, кровь виноградных ягод.
\vs Deu 32:15 И [ел Иаков, и] утучнел Израиль, и стал упрям; утучнел, отолстел и разжирел; и оставил он Бога, создавшего его, и презрел твердыню спасения своего.
\vs Deu 32:16 \bibemph{Богами} чуждыми они раздражили Его и мерзостями [своими] разгневали Его:
\vs Deu 32:17 приносили жертвы бесам, а не Богу, богам, которых они не знали, новым, \bibemph{которые} пришли от соседей и о которых не помышляли отцы ваши.
\vs Deu 32:18 А Заступника, родившего тебя, ты забыл, и не помнил Бога, создавшего тебя.
\vs Deu 32:19 Господь увидел [и вознегодовал], и в негодовании пренебрег сынов Своих и дочерей Своих,
\vs Deu 32:20 и сказал: сокрою лице Мое от них [и] увижу, какой будет конец их; ибо они род развращенный; дети, в которых нет верности;
\vs Deu 32:21 они раздражили Меня не богом, суетными своими огорчили Меня: и Я раздражу их не народом, народом бессмысленным огорчу их;
\vs Deu 32:22 ибо огонь возгорелся во гневе Моем, жжет до ада преисподнего, и поядает землю и произведения ее, и попаляет основания гор;
\vs Deu 32:23 соберу на них бедствия и истощу на них стрелы Мои:
\vs Deu 32:24 \bibemph{будут} истощены голодом, истреблены горячкою и лютою заразою; и пошлю на них зубы зверей и яд ползающих по земле;
\vs Deu 32:25 извне будет губить их меч, а в домах ужас~--- и юношу, и девицу, и грудного младенца, и покрытого сединою старца.
\vs Deu 32:26 Я сказал бы: рассею их и изглажу из среды людей память о них;
\vs Deu 32:27 но отложил это ради озлобления врагов, чтобы враги его не возомнили и не сказали: наша рука высока, и не Господь сделал все сие.
\vs Deu 32:28 Ибо они народ, потерявший рассудок, и нет в них смысла.
\vs Deu 32:29 О, если бы они рассудили, подумали о сем, уразумели, что с ними будет!
\vs Deu 32:30 Как бы мог один преследовать тысячу и двое прогонять тьму, если бы Заступник их не предал их, и Господь не отдал их!
\vs Deu 32:31 Ибо заступник их не таков, как наш Заступник; сами враги наши судьи в том.
\vs Deu 32:32 Ибо виноград их от виноградной лозы Содомской и с полей Гоморрских; ягоды их ягоды ядовитые, грозды их горькие;
\vs Deu 32:33 вино их яд драконов и гибельная отрава аспидов.
\vs Deu 32:34 Не сокрыто ли это у Меня? не запечатано ли в хранилищах Моих?
\vs Deu 32:35 У Меня отмщение и воздаяние, когда поколеблется нога их; ибо близок день погибели их, скоро наступит уготованное для них.
\vs Deu 32:36 Но Господь будет судить народ Свой и над рабами Своими умилосердится, когда Он увидит, что рука их ослабела, и не стало ни заключенных, ни оставшихся \bibemph{вне}.
\vs Deu 32:37 Тогда скажет [Господь]: где боги их, твердыня, на которую они надеялись,
\vs Deu 32:38 которые ели тук жертв их [и] пили вино возлияний их? пусть они восстанут и помогут вам, пусть будут для вас покровом!
\vs Deu 32:39 Видите ныне, [видите,] что это Я, Я~--- и нет Бога, кроме Меня: Я умерщвляю и оживляю, Я поражаю и Я исцеляю, и никто не избавит от руки Моей.
\vs Deu 32:40 Я подъемлю к небесам руку Мою и [клянусь десницею Моею и] говорю: живу Я вовек!
\vs Deu 32:41 Когда изострю сверкающий меч Мой, и рука Моя приимет суд, то отмщу врагам Моим и ненавидящим Меня воздам;
\vs Deu 32:42 упою стрелы Мои кровью, и меч Мой насытится плотью, кровью убитых и пленных, головами начальников врага.
\vs Deu 32:43 [Веселитесь, небеса, вместе с Ним, и поклонитесь Ему, все Ангелы Божии.] Веселитесь, язычники, с народом Его [и да укрепятся все сыны Божии]! ибо Он отмстит за кровь рабов Своих, и воздаст мщение врагам Своим, [и ненавидящим Его воздаст,] и очистит [Господь] землю Свою \bibemph{и} народ Свой!
\rsbpar\vs Deu 32:44 [И написал Моисей песнь сию в тот день, и научил ей сынов Израилевых.] И пришел Моисей [к народу] и изрек все слова песни сей вслух народа, он и Иисус, сын Навин.
\vs Deu 32:45 Когда Моисей изрек все слова сии всему Израилю,
\vs Deu 32:46 тогда сказал им: положите на сердце ваше все слова, которые я объявил вам сегодня, и завещевайте их детям своим, чтобы они старались исполнять все слова закона сего;
\vs Deu 32:47 ибо это не пустое для вас, но это жизнь ваша, и чрез это вы долгое время пробудете на той земле, в которую вы идете чрез Иордан, чтоб овладеть ею.
\rsbpar\vs Deu 32:48 И говорил Господь Моисею в тот же самый день и сказал:
\vs Deu 32:49 взойди на сию гору Аварим, на гору Нево, которая в земле Моавитской, против Иерихона, и посмотри на землю Ханаанскую, которую я даю во владение сынам Израилевым;
\vs Deu 32:50 и умри на горе, на которую ты взойдешь, и приложись к народу твоему, как умер Аарон, брат твой, на горе Ор, и приложился к народу своему,
\vs Deu 32:51 за то, что вы согрешили против Меня среди сынов Израилевых при водах Меривы в Кадесе, в пустыне Син, за то, что не явили святости Моей среди сынов Израилевых;
\vs Deu 32:52 пред \bibemph{собою} ты увидишь землю, а не войдешь туда, в землю, которую Я даю сынам Израилевым.
\vs Deu 33:1 Вот благословение, которым Моисей, человек Божий, благословил сынов Израилевых пред смертью своею.
\vs Deu 33:2 Он сказал: Господь пришел от Синая, открылся им от Сеира, воссиял от горы Фарана и шел со тьмами святых; одесную Его огнь закона.
\vs Deu 33:3 Истинно Он любит народ [Свой]; все святые его в руке Твоей, и они припали к стопам Твоим, чтобы внимать словам Твоим.
\vs Deu 33:4 Закон дал нам Моисей, наследие обществу Иакова.
\vs Deu 33:5 И он был царь Израиля, когда собирались главы народа вместе с коленами Израилевыми.
\vs Deu 33:6 Да живет Рувим и да не умирает, и [Симеон] да \bibemph{не} будет малочислен!
\vs Deu 33:7 Но об Иуде сказал сие: услыши, Господи, глас Иуды и приведи его к народу его; руками своими да защитит он себя, и Ты будь помощником против врагов его.
\vs Deu 33:8 И о Левии сказал: туммим Твой и урим Твой на святом муже Твоем, которого Ты искусил в Массе, с которым Ты препирался при водах Меривы,
\vs Deu 33:9 который говорит об отце своем и матери своей: <<я на них не смотрю>>, и братьев своих не признает, и сыновей своих не знает; ибо они, \bibemph{левиты}, слова Твои хранят и завет Твой соблюдают,
\vs Deu 33:10 учат законам Твоим Иакова и заповедям Твоим Израиля, возлагают курение пред лице Твое и всесожжения на жертвенник Твой;
\vs Deu 33:11 благослови, Господи, силу его и о деле рук его благоволи, порази чресла восстающих на него и ненавидящих его, чтобы они не могли стоять.
\vs Deu 33:12 О Вениамине сказал: возлюбленный Господом обитает у Него безопасно, [Бог] покровительствует ему всякий день, и он покоится между раменами Его.
\vs Deu 33:13 Об Иосифе сказал: да благословит Господь землю его вожделенными дарами неба, росою и \bibemph{дарами} бездны, лежащей внизу,
\vs Deu 33:14 вожделенными плодами от солнца и вожделенными произведениями луны,
\vs Deu 33:15 превосходнейшими произведениями гор древних и вожделенными дарами холмов вечных,
\vs Deu 33:16 и вожделенными дарами земли и того, что наполняет ее; благословение Явившегося в терновом кусте да приидет на главу Иосифа и на темя наилучшего из братьев своих;
\vs Deu 33:17 крепость его как первородного тельца, и роги его, как роги буйвола; ими избодет он народы все до пределов земли: это тьмы Ефремовы, это тысячи Манассиины.
\vs Deu 33:18 О Завулоне сказал: веселись, Завулон, в путях твоих, и Иссахар, в шатрах твоих;
\vs Deu 33:19 созывают они народ на гору, там заколают законные жертвы, ибо они питаются богатством моря и сокровищами, сокрытыми в песке.
\vs Deu 33:20 О Гаде сказал: благословен распространивший Гада; он покоится как лев и сокрушает и мышцу и голову;
\vs Deu 33:21 он избрал себе начаток \bibemph{земли}, там почтен уделом от законодателя, и пришел с главами народа, и исполнил правду Господа и суды с Израилем.
\vs Deu 33:22 О Дане сказал: Дан молодой лев, который выбегает из Васана.
\vs Deu 33:23 О Неффалиме сказал: Неффалим насыщен благоволением и исполнен благословения Господа; море и юг во владении \bibemph{его}.
\vs Deu 33:24 Об Асире сказал: благословен между сынами Асир, он будет любим братьями своими, и окунет в елей ногу свою;
\vs Deu 33:25 железо и медь~--- запоры твои; как дни твои, \bibemph{будет умножаться} богатство твое.
\vs Deu 33:26 Нет подобного Богу Израилеву, Который по небесам принесся на помощь тебе и во славе Своей на облаках;
\vs Deu 33:27 прибежище [твое] Бог древний, и [ты] под мышцами вечными; Он прогонит врагов от лица твоего и скажет: истребляй!
\vs Deu 33:28 Израиль живет безопасно, один; око Иакова \bibemph{видит пред собою} землю обильную хлебом и вином, и небеса его каплют росу.
\vs Deu 33:29 Блажен ты, Израиль! кто подобен тебе, народ, хранимый Господом, Который есть щит, охраняющий тебя, и меч славы твоей? Враги твои раболепствуют тебе, и ты попираешь выи их.
\vs Deu 34:1 И взошел Моисей с равнин Моавитских на гору Нево, на вершину Фасги, что против Иерихона, и показал ему Господь всю землю Галаад до самого Дана,
\vs Deu 34:2 и всю [землю] Неффалимову, и [всю] землю Ефремову и Манассиину, и всю землю Иудину, даже до самого западного моря,
\vs Deu 34:3 и полуденную страну и равнину долины Иерихона, город Пальм, до Сигора.
\vs Deu 34:4 И сказал ему Господь: вот земля, о которой Я клялся Аврааму, Исааку и Иакову, говоря: <<семени твоему дам ее>>; Я дал тебе увидеть ее глазами твоими, но в нее ты не войдешь.
\rsbpar\vs Deu 34:5 И умер там Моисей, раб Господень, в земле Моавитской, по слову Господню;
\vs Deu 34:6 и погребен на долине в земле Моавитской против Беф-Фегора, и никто не знает \bibemph{места} погребения его даже до сего дня.
\vs Deu 34:7 Моисею было сто двадцать лет, когда он умер; но зрение его не притупилось, и крепость в нем не истощилась.
\vs Deu 34:8 И оплакивали Моисея сыны Израилевы на равнинах Моавитских [у Иордана близ Иерихона] тридцать дней. И прошли дни плача и сетования о Моисее.
\vs Deu 34:9 И Иисус, сын Навин, исполнился духа премудрости, потому что Моисей возложил на него руки свои, и повиновались ему сыны Израилевы и делали так, как повелел Господь Моисею.
\vs Deu 34:10 И не было более у Израиля пророка такого, как Моисей, которого Господь знал лицем к лицу,
\vs Deu 34:11 по всем знамениям и чудесам, которые послал его Господь сделать в земле Египетской над фараоном и над всеми рабами его и над всею землею его,
\vs Deu 34:12 и по руке сильной и по великим чудесам, которые Моисей совершил пред глазами всего Израиля.

\include{tex/Jos}
\include{tex/Jdg}
\bibbookdescr{Rut}{
  inline={\LARGE Книга\\\Huge Руфь},
  toc={Руфь},
  bookmark={Руфь},
  header={Руфь},
  %headerleft={},
  %headerright={},
  abbr={Руфь}
}
\vs Rut 1:1 В те дни, когда управляли судьи, случился голод на земле. И пошел один человек из Вифлеема Иудейского со своею женою и двумя сыновьями своими жить на полях Моавитских.
\vs Rut 1:2 Имя человека того Елимелех, имя жены его Ноеминь, а имена двух сынов его Махлон и Хилеон; \bibemph{они были} Ефрафяне из Вифлеема Иудейского. И пришли они на поля Моавитские и остались там.
\vs Rut 1:3 И умер Елимелех, муж Ноемини, и осталась она с двумя сыновьями своими.
\vs Rut 1:4 Они взяли себе жен из Моавитянок, имя одной Орфа, а имя другой Руфь, и жили там около десяти лет.
\vs Rut 1:5 Но потом и оба [сына ее], Махлон и Хилеон, умерли, и осталась та женщина после обоих своих сыновей и после мужа своего.
\vs Rut 1:6 И встала она со снохами своими и пошла обратно с полей Моавитских, ибо услышала на полях Моавитских, что Бог посетил народ Свой и дал им хлеб.
\vs Rut 1:7 И вышла она из того места, в котором жила, и обе снохи ее с нею. Когда они шли по дороге, возвращаясь в землю Иудейскую,
\vs Rut 1:8 Ноеминь сказала двум снохам своим: пойдите, возвратитесь каждая в дом матери своей; да сотворит Господь с вами милость, как вы поступали с умершими и со мною!
\vs Rut 1:9 да даст вам Господь, чтобы вы нашли пристанище каждая в доме своего мужа! И поцеловала их. Но они подняли вопль и плакали
\vs Rut 1:10 и сказали: нет, мы с тобою возвратимся к народу твоему.
\vs Rut 1:11 Ноеминь же сказала: возвратитесь, дочери мои; зачем вам идти со мною? Разве еще есть у меня сыновья в моем чреве, которые были бы вам мужьями?
\vs Rut 1:12 Возвратитесь, дочери мои, пойдите, ибо я уже стара, чтоб быть замужем. Да если б я и сказала: <<есть мне еще надежда>>, и даже если бы я сию же ночь была с мужем и потом родила сыновей,~---
\vs Rut 1:13 то можно ли вам ждать, пока они выросли бы? можно ли вам медлить и не выходить замуж? Нет, дочери мои, я весьма сокрушаюсь о вас, ибо рука Господня постигла меня.
\vs Rut 1:14 Они подняли вопль и опять стали плакать. И Орфа простилась со свекровью своею [и возвратилась к народу своему], а Руфь осталась с нею.
\vs Rut 1:15 [Ноеминь] сказала [Руфи]: вот, невестка твоя возвратилась к народу своему и к своим богам; возвратись и ты вслед за невесткою твоею.
\vs Rut 1:16 Но Руфь сказала: не принуждай меня оставить тебя и возвратиться от тебя; но куда ты пойдешь, туда и я пойду, и где ты жить будешь, там и я буду жить; народ твой будет моим народом, и твой Бог~--- моим Богом;
\vs Rut 1:17 и где ты умрешь, там и я умру и погребена буду; пусть то и то сделает мне Господь, и еще больше сделает; смерть одна разлучит меня с тобою.
\vs Rut 1:18 [Ноеминь,] видя, что она твердо решилась идти с нею, перестала уговаривать ее.
\vs Rut 1:19 И шли обе они, доколе не пришли в Вифлеем. Когда пришли они в Вифлеем, весь город пришел в движение от них, и говорили: это Ноеминь?
\vs Rut 1:20 Она сказала им: не называйте меня Ноеминью\fns{Приятная.}, а называйте меня Марою\fns{Горькая.}, потому что Вседержитель послал мне великую горесть;
\vs Rut 1:21 я вышла отсюда с достатком, а возвратил меня Господь с пустыми руками; зачем называть меня Ноеминью, когда Господь заставил меня страдать, и Вседержитель послал мне несчастье?
\vs Rut 1:22 И возвратилась Ноеминь, и с нею сноха ее Руфь Моавитянка, пришедшая с полей Моавитских, и пришли они в Вифлеем в начале жатвы ячменя.
\vs Rut 2:1 У Ноемини был родственник по мужу ее, человек весьма знатный, из племени Елимелехова, имя ему Вооз.
\vs Rut 2:2 И сказала Руфь Моавитянка Ноемини: пойду я на поле и буду подбирать колосья по следам того, у кого найду благоволение. Она сказала ей: пойди, дочь моя.
\vs Rut 2:3 Она пошла, и пришла, и подбирала в поле \bibemph{колосья} позади жнецов. И случилось, что та часть поля принадлежала Воозу, который из племени Елимелехова.
\vs Rut 2:4 И вот, Вооз пришел из Вифлеема и сказал жнецам: Господь с вами! Они сказали ему: да благословит тебя Господь!
\vs Rut 2:5 И сказал Вооз слуге своему, приставленному к жнецам: чья это молодая женщина?
\vs Rut 2:6 Слуга, приставленный к жнецам, отвечал и сказал: эта молодая женщина~--- Моавитянка, пришедшая с Ноеминью с полей Моавитских;
\vs Rut 2:7 она сказала: <<буду я подбирать и собирать между снопами позади жнецов>>; и пришла, и находится \bibemph{здесь} с самого утра доселе; мало бывает она дома.
\vs Rut 2:8 И сказал Вооз Руфи: послушай, дочь моя, не ходи подбирать на другом поле и не переходи отсюда, но будь здесь с моими служанками;
\vs Rut 2:9 пусть в глазах твоих будет то поле, где они жнут, и ходи за ними; вот, я приказал слугам моим не трогать тебя; когда захочешь пить, иди к сосудам и пей, откуда черпают слуги мои.
\vs Rut 2:10 Она пала на лице свое и поклонилась до земли и сказала ему: чем снискала я в глазах твоих милость, что ты принимаешь меня, хотя я и чужеземка?
\vs Rut 2:11 Вооз отвечал и сказал ей: мне сказано все, что сделала ты для свекрови своей по смерти мужа твоего, что ты оставила твоего отца и твою мать и твою родину и пришла к народу, которого ты не знала вчера и третьего дня;
\vs Rut 2:12 да воздаст Господь за это дело твое, и да будет тебе полная награда от Господа Бога Израилева, к Которому ты пришла, чтоб успокоиться под Его крылами!
\vs Rut 2:13 Она сказала: да буду я в милости пред очами твоими, господин мой! Ты утешил меня и говорил по сердцу рабы твоей, между тем как я не ст\acc{о}ю ни одной из рабынь твоих.
\vs Rut 2:14 И сказал ей Вооз: время обеда; приди сюда и ешь хлеб и обмакивай кусок твой в уксус. И села она возле жнецов. Он подал ей хлеба; она ела, наелась, и еще осталось.
\vs Rut 2:15 И встала, чтобы подбирать. Вооз дал приказ слугам своим, сказав: пусть подбирает она и между снопами, и не обижайте ее;
\vs Rut 2:16 да и от снопов откидывайте ей и оставляйте, пусть она подбирает [и ест], и не браните ее.
\vs Rut 2:17 Так подбирала она на поле до вечера и вымолотила собранное, и вышло около ефы ячменя.
\vs Rut 2:18 Взяв это, она пошла в город, и свекровь ее увидела, что она набрала. И вынула [Руфь из пазухи своей] и дала ей то, что оставила, наевшись сама.
\vs Rut 2:19 И сказала ей свекровь ее: где ты собирала сегодня и где работала? да будет благословен принявший тебя! [Руфь] объявила свекрови своей, у кого она работала, и сказала: человеку тому, у которого я сегодня работала, имя Вооз.
\vs Rut 2:20 И сказала Ноеминь снохе своей: благословен он от Господа за то, что не лишил милости своей ни живых, ни мертвых! И сказала ей Ноеминь: человек этот близок к нам; он из наших родственников.
\vs Rut 2:21 Руфь Моавитянка сказала [свекрови своей]: он даже сказал мне: будь с моими служанками, доколе не докончат они жатвы моей.
\vs Rut 2:22 И сказала Ноеминь снохе своей Руфи: хорошо, дочь моя, что ты будешь ходить со служанками его, и не будут оскорблять тебя на другом поле.
\vs Rut 2:23 Так была она со служанками Воозовыми и подбирала [колосья], доколе не кончилась жатва ячменя и жатва пшеницы, и жила у свекрови своей.
\vs Rut 3:1 И сказала ей Ноеминь, свекровь ее: дочь моя, не поискать ли тебе пристанища, чтобы тебе хорошо было?
\vs Rut 3:2 Вот, Вооз, со служанками которого ты была, родственник наш; вот, он в эту ночь веет на гумне ячмень;
\vs Rut 3:3 умойся, помажься, надень на себя [нарядные] одежды твои и пойди на гумно, но не показывайся ему, доколе не кончит есть и пить;
\vs Rut 3:4 когда же он ляжет спать, узнай место, где он ляжет; тогда придешь и откроешь у ног его и ляжешь; он скажет тебе, что тебе делать.
\vs Rut 3:5 [Руфь] сказала ей: сделаю все, что ты сказала мне.
\vs Rut 3:6 И пошла на гумно и сделала все так, как приказывала ей свекровь ее.
\vs Rut 3:7 Вооз наелся и напился, и развеселил сердце свое, и пошел \bibemph{и лег} спать подле скирда. И она пришла тихонько, открыла у ног его и легла.
\vs Rut 3:8 В полночь он содрогнулся, приподнялся, и вот, у ног его лежит женщина.
\vs Rut 3:9 И сказал [ей Вооз]: кто ты? Она сказала: я Руфь, раба твоя, простри крыло твое на рабу твою, ибо ты родственник.
\vs Rut 3:10 [Вооз] сказал: благословенна ты от Господа [Бога], дочь моя! это последнее твое доброе дело сделала ты еще лучше прежнего, что ты не пошла искать молодых людей, ни бедных, ни богатых;
\vs Rut 3:11 итак, дочь моя, не бойся, я сделаю тебе все, что ты сказала; ибо у всех ворот народа моего знают, что ты женщина добродетельная;
\vs Rut 3:12 хотя и правда, что я родственник, но есть еще родственник ближе меня;
\vs Rut 3:13 переночуй эту ночь; завтра же, если он примет тебя, то хорошо, пусть примет; а если он не захочет принять тебя, то я приму; жив Господь! Спи до утра.
\vs Rut 3:14 И спала она у ног его до утра и встала прежде, нежели могли они распознать друг друга. И сказал Вооз: пусть не знают, что женщина приходила на гумно.
\vs Rut 3:15 И сказал ей: подай верхнюю одежду, которая на тебе, подержи ее. Она держала, и он отмерил [ей] шесть мер ячменя, и положил на нее, и пошел в город.
\vs Rut 3:16 А [Руфь] пришла к свекрови своей. Та сказала [ей]: что, дочь моя? Она пересказала ей все, что сделал ей человек тот.
\vs Rut 3:17 И сказала [ей]: эти шесть мер ячменя он дал мне и сказал мне: не ходи к свекрови своей с пустыми руками.
\vs Rut 3:18 Та сказала: подожди, дочь моя, доколе не узнаешь, чем кончится дело; ибо человек тот не останется в покое, не кончив сегодня дела.
\vs Rut 4:1 Вооз вышел к воротам и сидел там. И вот, идет мимо родственник, о котором говорил Вооз. И сказал ему [Вооз]: зайди сюда и сядь здесь. Тот зашел и сел.
\vs Rut 4:2 [Вооз] взял десять человек из старейшин города и сказал: сядьте здесь. И они сели.
\vs Rut 4:3 И сказал [Вооз] родственнику: Ноеминь, возвратившаяся с полей Моавитских, продает часть поля, принадлежащую брату нашему Елимелеху;
\vs Rut 4:4 я решился довести до ушей твоих и сказать: купи при сидящих здесь и при старейшинах народа моего; если хочешь выкупить, выкуп\acc{а}й; а если не хочешь выкупить, скажи мне, и я буду знать; ибо кроме тебя некому выкупить; а по тебе я. Тот сказал: я выкуп\acc{а}ю.
\vs Rut 4:5 Вооз сказал: когда ты купишь поле у Ноемини, то должен купить и у Руфи Моавитянки, жены умершего, и должен взять ее в замужество, чтобы восстановить имя умершего в уделе его.
\vs Rut 4:6 И сказал тот родственник: не могу я взять ее себе, чтобы не расстроить своего удела; прими ее ты, ибо я не могу принять.
\vs Rut 4:7 Прежде такой был \bibemph{обычай} у Израиля при выкупе и при мене для подтверждения какого-либо дела: один снимал сапог свой и давал другому, [который принимал право родственника,] и это было свидетельством у Израиля.
\vs Rut 4:8 И сказал тот родственник Воозу: купи себе. И снял сапог свой [и дал ему].
\vs Rut 4:9 И сказал Вооз старейшинам и всему народу: вы теперь свидетели тому, что я покупаю у Ноемини все Елимелехово и все Хилеоново и Махлоново;
\vs Rut 4:10 также и Руфь Моавитянку, жену Махлонову, беру себе в жену, чтоб оставить имя умершего в уделе его, и чтобы не исчезло имя умершего между братьями его и у ворот местопребывания его: вы сегодня свидетели тому.
\vs Rut 4:11 И сказал весь народ, который при воротах, и старейшины: мы свидетели; да соделает Господь жену, входящую в дом твой, как Рахиль и как Лию, которые обе устроили дом Израилев; приобретай богатство в Ефрафе, и да славится имя твое в Вифлееме;
\vs Rut 4:12 и да будет дом твой, как дом Фареса, которого родила Фамарь Иуде, от того семени, которое даст тебе Господь от этой молодой женщины.
\vs Rut 4:13 И взял Вооз Руфь, и она сделалась его женою. И вошел он к ней, и Господь дал ей беременность, и она родила сына.
\vs Rut 4:14 И говорили женщины Ноемини: благословен Господь, что Он не оставил тебя ныне без наследника! И да будет славно имя его в Израиле!
\vs Rut 4:15 Он будет тебе отрадою и питателем в старости твоей, ибо его родила сноха твоя, которая любит тебя, которая для тебя лучше семи сыновей.
\vs Rut 4:16 И взяла Ноеминь дитя сие, и носила его в объятиях своих, и была ему нянькою.
\vs Rut 4:17 Соседки нарекли ему имя и говорили: <<у Ноемини родился сын>>, и нарекли ему имя: Овид. Он отец Иессея, отца Давидова.
\rsbpar\vs Rut 4:18 И вот род Фаресов: Фарес родил Есрома;
\vs Rut 4:19 Есром родил Арама; Арам родил Аминадава;
\vs Rut 4:20 Аминадав родил Наассона; Наассон родил Салмона;
\vs Rut 4:21 Салмон родил Вооза; Вооз родил Овида;
\vs Rut 4:22 Овид родил Иессея; Иессей родил Давида.
\newbookpage
\include{tex/1Sa}
\include{tex/2Sa}
\bibbookdescr{1Ki}{
  inline={\LARGE Третья книга\\\Huge Царств\fns{У Евреев: <<Первая царей>>.}},
  toc={3-я Царств},
  bookmark={3-я Царств},
  header={3-я Царств},
  %headerleft={},
  %headerright={},
  abbr={3~Цар}
}
\vs 1Ki 1:1 Когда царь Давид состарился, вошел в \bibemph{преклонные} лета, то покрывали его одеждами, но не мог он согреться.
\vs 1Ki 1:2 И сказали ему слуги его: пусть поищут для господина нашего царя молодую девицу, чтоб она предстояла царю и ходила за ним и лежала с ним,~--- и будет тепло господину нашему, царю.
\vs 1Ki 1:3 И искали красивой девицы во всех пределах Израильских, и нашли Ависагу Сунамитянку, и привели ее к царю.
\vs 1Ki 1:4 Девица была очень красива, и ходила она за царем и прислуживала ему; но царь не познал ее.
\rsbpar\vs 1Ki 1:5 Адония, сын Аггифы, возгордившись говорил: я буду царем. И завел себе колесницы и всадников и пятьдесят человек скороходов.
\vs 1Ki 1:6 Отец же никогда не стеснял его вопросом: для чего ты это делаешь? Он же был очень красив и родился ему после Авессалома.
\vs 1Ki 1:7 И советовался он с Иоавом, сыном Саруиным, и с Авиафаром священником, и они помогали Адонии.
\vs 1Ki 1:8 Но священник Садок и Ванея, сын Иодаев, и пророк Нафан, и Семей, и Рисий, и сильные Давидовы не были на стороне Адонии.
\vs 1Ki 1:9 И заколол Адония овец и волов и тельцов у камня Зохелет, что у источника Рогель, и пригласил всех братьев своих, сыновей царя, со всеми Иудеянами, служившими у царя.
\vs 1Ki 1:10 Пророка же Нафана и Ванею, и тех сильных, и Соломона, брата своего, не пригласил.
\vs 1Ki 1:11 Тогда Нафан сказал Вирсавии, матери Соломона, говоря: слышала ли ты, что Адония, сын Аггифин, сделался царем, а господин наш Давид не знает \bibemph{о том}?
\vs 1Ki 1:12 Теперь, вот, я советую тебе: спасай жизнь твою и жизнь сына твоего Соломона.
\vs 1Ki 1:13 Иди и войди к царю Давиду и скажи ему: не клялся ли ты, господин мой царь, рабе твоей, говоря: <<сын твой Соломон будет царем после меня и он сядет на престоле моем>>? Почему же воцарился Адония?
\vs 1Ki 1:14 И вот, когда ты еще будешь говорить там с царем, войду и я вслед за тобою и дополню слова твои.
\vs 1Ki 1:15 Вирсавия пошла к царю в спальню; царь был очень стар, и Ависага Сунамитянка прислуживала царю;
\vs 1Ki 1:16 и наклонилась Вирсавия и поклонилась царю; и сказал царь: что тебе?
\vs 1Ki 1:17 Она сказала ему: господин мой царь! ты клялся рабе твоей Господом Богом твоим: <<сын твой Соломон будет царствовать после меня и он сядет на престоле моем>>.
\vs 1Ki 1:18 А теперь, вот, Адония \bibemph{воцарился}, и ты, господин мой царь, не знаешь \bibemph{о том}.
\vs 1Ki 1:19 И заколол он множество волов, тельцов и овец, и пригласил всех сыновей царских и священника Авиафара, и военачальника Иоава; Соломона же, раба твоего, не пригласил.
\vs 1Ki 1:20 Но ты, господин мой,~--- царь, и глаза всех Израильтян \bibemph{устремлены} на тебя, чтобы ты объявил им, кто сядет на престоле господина моего царя после него;
\vs 1Ki 1:21 иначе, когда господин мой царь почиет с отцами своими, падет обвинение на меня и на сына моего Соломона.
\vs 1Ki 1:22 Когда она еще говорила с царем, пришел и пророк Нафан.
\vs 1Ki 1:23 И сказали царю, говоря: вот Нафан пророк. И вошел он к царю и поклонился царю лицем до земли.
\vs 1Ki 1:24 И сказал Нафан: господин мой царь! сказал ли ты: <<Адония будет царствовать после меня и он сядет на престоле моем>>?
\vs 1Ki 1:25 Потому что он ныне сошел и заколол множество волов, тельцов и овец, и пригласил всех сыновей царских и военачальников и священника Авиафара, и вот, они едят и пьют у него и говорят: да живет царь Адония!
\vs 1Ki 1:26 А меня, раба твоего, и священника Садока, и Ванею, сына Иодаева, и Соломона, раба твоего, не пригласил.
\vs 1Ki 1:27 Не сталось ли это по \bibemph{воле} господина моего царя, и для чего ты не открыл рабу твоему, кто сядет на престоле господина моего царя после него?
\vs 1Ki 1:28 И отвечал царь Давид и сказал: позовите ко мне Вирсавию. И вошла она и стала пред царем.
\vs 1Ki 1:29 И клялся царь и сказал: жив Господь, избавлявший душу мою от всякой беды!
\vs 1Ki 1:30 Как я клялся тебе Господом Богом Израилевым, говоря, что Соломон, сын твой, будет царствовать после меня и он сядет на престоле моем вместо меня, так я и сделаю это сегодня.
\vs 1Ki 1:31 И наклонилась Вирсавия лицем до земли, и поклонилась царю, и сказала: да живет господин мой царь Давид во веки!
\rsbpar\vs 1Ki 1:32 И сказал царь Давид: позовите ко мне священника Садока и пророка Нафана и Ванею, сына Иодаева. И вошли они к царю.
\vs 1Ki 1:33 И сказал им царь: возьмите с собою слуг господина вашего и посадите Соломона, сына моего, на мула моего, и сведите его к Гиону,
\vs 1Ki 1:34 и да помажет его там Садок священник и Нафан пророк в царя над Израилем, и затрубите трубою и возгласите: да живет царь Соломон!
\vs 1Ki 1:35 Потом проводите его назад, и он придет и сядет на престоле моем; он будет царствовать вместо меня; ему завещал я быть вождем Израиля и Иуды.
\vs 1Ki 1:36 И отвечал Ванея, сын Иодаев, царю и сказал: аминь,~--- да скажет так Господь Бог господина моего царя!
\vs 1Ki 1:37 Как был Господь Бог с господином моим царем, так да будет Он с Соломоном и да возвеличит престол его более престола господина моего царя Давида!
\vs 1Ki 1:38 И пошли Садок священник и Нафан пророк и Ванея, сын Иодая, и Хелефеи и Фелефеи, и посадили Соломона на мула царя Давида, и повели его к Гиону.
\vs 1Ki 1:39 И взял Садок священник рог с елеем из скинии и помазал Соломона. И затрубили трубою, и весь народ восклицал: да живет царь Соломон!
\vs 1Ki 1:40 И весь народ провожал Соломона, и играл народ на свирелях, и весьма радовался, так что земля расседалась от криков его.
\vs 1Ki 1:41 И услышал Адония и все приглашенные им, как только перестали есть; а Иоав, услышав звук трубы, сказал: отчего этот шум волнующегося города?
\vs 1Ki 1:42 Еще он говорил, как пришел Ионафан, сын священника Авиафара. И сказал Адония: войди; ты~--- честный человек и несешь добрую весть.
\vs 1Ki 1:43 И отвечал Ионафан и сказал Адонии: да, господин наш царь Давид поставил Соломона царем;
\vs 1Ki 1:44 и послал царь с ним Садока священника и Нафана пророка, и Ванею, сына Иодая, и Хелефеев и Фелефеев, и они посадили его на мула царского;
\vs 1Ki 1:45 и помазали его Садок священник и Нафан пророк в царя в Гионе, и оттуда отправились с радостью, и пришел в движение город. Вот отчего шум, который вы слышите.
\vs 1Ki 1:46 И Соломон уже сел на царском престоле.
\vs 1Ki 1:47 И слуги царя приходили поздравить господина нашего царя Давида, говоря: Бог твой да прославит имя Соломона более твоего имени и да возвеличит престол его более твоего престола. И поклонился царь на ложе своем,
\vs 1Ki 1:48 и сказал царь так: <<благословен Господь Бог Израилев, Который сегодня дал [от семени моего] сидящего на престоле моем, и очи мои видят это!>>
\vs 1Ki 1:49 \bibemph{Тогда} испугались и встали все приглашенные, которые были у Адонии, и пошли каждый своею дорогою.
\vs 1Ki 1:50 Адония же, боясь Соломона, встал и пошел и ухватился за роги жертвенника.
\vs 1Ki 1:51 И донесли Соломону, говоря: вот, Адония боится царя Соломона, и вот, он держится за роги жертвенника, говоря: пусть поклянется мне теперь царь Соломон, что он не умертвит раба своего мечом.
\vs 1Ki 1:52 И сказал Соломон: если он будет человеком честным, то ни один волос его не упадет на землю; если же найдется в нем лукавство, то умрет.
\vs 1Ki 1:53 И послал царь Соломон, и привели его от жертвенника. И он пришел и поклонился царю Соломону; и сказал ему Соломон: иди в дом свой.
\vs 1Ki 2:1 Приблизилось время умереть Давиду, и завещал он сыну своему Соломону, говоря:
\vs 1Ki 2:2 вот, я отхожу в путь всей земли, ты же будь тверд и будь мужествен
\vs 1Ki 2:3 и храни завет Господа Бога твоего, ходя путями Его и соблюдая уставы Его и заповеди Его, и определения Его и постановления Его, как написано в законе Моисеевом, чтобы быть тебе благоразумным во всем, что ни будешь делать, и везде, куда ни обратишься;
\vs 1Ki 2:4 чтобы Господь исполнил слово Свое, которое Он сказал обо мне, говоря: <<если сыны твои будут наблюдать за путями своими, чтобы ходить предо Мною в истине от всего сердца своего и от всей души своей, то не прекратится муж от тебя на престоле Израилевом>>.
\vs 1Ki 2:5 Еще: ты знаешь, что сделал мне Иоав, сын Саруин, как поступил он с двумя вождями войска Израильского, с Авениром, сыном Нировым, и Амессаем, сыном Иеферовым, как он умертвил их и пролил кровь бранную во время мира, обагрив кровью бранною пояс на чреслах своих и обувь на ногах своих:
\vs 1Ki 2:6 поступи по мудрости твоей, чтобы не отпустить седины его мирно в преисподнюю.
\vs 1Ki 2:7 А сынам Верзеллия Галаадитянина окажи милость, чтоб они были между питающимися твоим столом, ибо они пришли ко мне, когда я бежал от Авессалома, брата твоего.
\vs 1Ki 2:8 Вот еще у тебя Семей, сын Геры Вениамитянина из Бахурима; он злословил меня тяжким злословием, когда я шел в Маханаим; но он вышел навстречу мне у Иордана, и я поклялся ему Господом, говоря: <<я не умерщвлю тебя мечом>>.
\vs 1Ki 2:9 Ты же не оставь его безнаказанным; ибо ты человек мудрый и знаешь, что тебе сделать с ним, чтобы низвести седину его в крови в преисподнюю.
\vs 1Ki 2:10 И почил Давид с отцами своими и погребен был в городе Давидовом.
\vs 1Ki 2:11 Времени царствования Давида над Израилем было сорок лет: в Хевроне царствовал он семь лет и тридцать три года царствовал в Иерусалиме.
\rsbpar\vs 1Ki 2:12 И сел Соломон на престоле Давида, отца своего, и царствование его было очень твердо.
\vs 1Ki 2:13 И пришел Адония, сын Аггифы, к Вирсавии, матери Соломона, [и поклонился ей]. Она сказала: с миром ли приход твой? И сказал он: с миром.
\vs 1Ki 2:14 И сказал он: у меня есть слово к тебе. Она сказала: говори.
\vs 1Ki 2:15 И сказал он: ты знаешь, что царство принадлежало мне, и весь Израиль обращал на меня взоры свои, как на будущего царя; но царство отошло от меня и досталось брату моему, ибо от Господа это было ему;
\vs 1Ki 2:16 теперь я прошу тебя об одном, не откажи мне. Она сказала ему: говори.
\vs 1Ki 2:17 И сказал он: прошу тебя, поговори царю Соломону, ибо он не откажет тебе, чтоб он дал мне Ависагу Сунамитянку в жену.
\vs 1Ki 2:18 И сказала Вирсавия: хорошо, я поговорю о тебе царю.
\vs 1Ki 2:19 И вошла Вирсавия к царю Соломону говорить ему об Адонии. Царь встал перед нею, и поклонился ей, и сел на престоле своем. Поставили престол и для матери царя, и она села по правую руку его
\vs 1Ki 2:20 и сказала: я имею к тебе одну небольшую просьбу, не откажи мне. И сказал ей царь: проси, мать моя; я не откажу тебе.
\vs 1Ki 2:21 И сказала она: дай Ависагу Сунамитянку Адонии, брату твоему, в жену.
\vs 1Ki 2:22 И отвечал царь Соломон и сказал матери своей: а зачем ты просишь Ависагу Сунамитянку для Адонии? проси ему \bibemph{также} и царства; ибо он мой старший брат, и ему священник Авиафар и Иоав, сын Саруин, [военачальник, друг].
\vs 1Ki 2:23 И поклялся царь Соломон Господом, говоря: то и то пусть сделает со мною Бог и еще больше сделает, если не на свою душу сказал Адония такое слово;
\vs 1Ki 2:24 ныне же,~--- жив Господь, укрепивший меня и посадивший меня на престоле Давида, отца моего, и устроивший мне дом, как говорил Он,~--- ныне же Адония должен умереть.
\vs 1Ki 2:25 И послал царь Соломон Ванею, сына Иодаева, который поразил его, и он умер.
\vs 1Ki 2:26 А священнику Авиафару царь сказал: ступай в Анафоф на твое поле; ты достоин смерти, но в настоящее время я не умерщвлю тебя, ибо ты носил ковчег Владыки Господа пред Давидом, отцом моим, и терпел все, что терпел отец мой.
\vs 1Ki 2:27 И удалил Соломон Авиафара от священства Господня, и исполнилось слово Господа, которое сказал Он о доме Илия в Силоме.
\vs 1Ki 2:28 Слух \bibemph{об этом} дошел до Иоава,~--- так как Иоав склонялся на сторону Адонии, а на сторону Соломона не склонялся,~--- и убежал Иоав в скинию Господню и ухватился за роги жертвенника.
\vs 1Ki 2:29 И донесли царю Соломону, что Иоав убежал в скинию Господню и что он у жертвенника. И послал Соломон Ванею, сына Иодаева, говоря: пойди, умертви его [и похорони его].
\vs 1Ki 2:30 И пришел Ванея в скинию Господню и сказал ему: так сказал царь: выходи. И сказал тот: нет, я хочу умереть здесь. Ванея передал это царю, говоря: так сказал Иоав, и так отвечал мне.
\vs 1Ki 2:31 Царь сказал ему: сделай, как он сказал, и умертви его и похорони его, и сними невинную кровь, пролитую Иоавом, с меня и с дома отца моего;
\vs 1Ki 2:32 да обратит Господь кровь его на голову его за то, что он убил двух мужей невинных и лучших его: поразил мечом, без ведома отца моего Давида, Авенира, сына Нирова, военачальника Израильского, и Амессая, сына Иеферова, военачальника Иудейского;
\vs 1Ki 2:33 да обратится кровь их на голову Иоава и на голову потомства его на веки, а Давиду и потомству его, и дому его и престолу его да будет мир на веки от Господа!
\vs 1Ki 2:34 И пошел Ванея, сын Иодаев, и поразил Иоава, и умертвил его, и он был похоронен в доме своем в пустыне.
\vs 1Ki 2:35 И поставил царь Соломон Ванею, сына Иодаева, вместо его над войском; [управление же царством было в Иерусалиме,] а Садока священника поставил царь [первосвященником] вместо Авиафара.\rsbpar [И дал Господь Соломону разум и мудрость весьма великую и обширный ум, как песок при море. И Соломон имел разум выше разума всех сынов востока и всех мудрых Египтян. И взял за себя дочь фараона и ввел ее в город Давидов, доколе не построил дома своего и, во-первых, дома Господня и стены вокруг Иерусалима; в семь лет окончил он строение. И было у Соломона семьдесят тысяч человек, носящих тяжести, и восемьдесят тысяч каменосеков в горах. И сделал Соломон море и подпоры, и большие бани и столбы, и источник на дворе и медное море, и построил замок и укрепления его, и разделил город Давидов. Тогда дочь фараона перешла из города Давидова в дом свой, который он построил ей; тогда построил Соломон стену вокруг города. И приносил Соломон три раза в год всесожжения и мирные жертвы на жертвеннике, который он устроил Господу, и курение совершал на нем пред Господом, и окончил строение дома. Главных приставников над работами Соломоновыми было три тысячи шестьсот, которые управляли народом, производившим работы. И построил он Ассур, и Магдон, и Газер, и Вефорон вышний и Валалаф; но эти города он построил после построения дома Господня и стены вокруг Иерусалима. И еще при жизни Давид завещал Соломону, говоря: вот у тебя Семей, сын Геры, сына Иеминиина из Бахурима; он злословил меня тяжким злословием, как я шел в Маханаим; но он вышел навстречу мне у Иордана, и я поклялся ему Господом, говоря: <<я не умерщвлю тебя мечом>>. Ты же не оставь его безнаказанным, ибо ты человек мудрый и знаешь, что тебе сделать с ним, чтобы низвести седину его в крови в преисподнюю.]
\vs 1Ki 2:36 И послав царь призвал Семея и сказал ему: построй себе дом в Иерусалиме и живи здесь, и никуда не выходи отсюда;
\vs 1Ki 2:37 и знай, что в тот день, в который ты выйдешь и перейдешь поток Кедрон, непременно умрешь; кровь твоя будет на голове твоей.
\vs 1Ki 2:38 И сказал Семей царю: хорошо; как приказал господин мой царь, так сделает раб твой. И жил Семей в Иерусалиме долгое время.
\vs 1Ki 2:39 Но через три года случилось, что у Семея двое рабов убежали к Анхусу, сыну Маахи, царю Гефскому. И сказали Семею, говоря: вот, рабы твои в Гефе.
\vs 1Ki 2:40 И встал Семей, и оседлал осла своего, и отправился в Геф к Анхусу искать рабов своих. И возвратился Семей и привел рабов своих из Гефа.
\vs 1Ki 2:41 И донесли Соломону, что Семей ходил из Иерусалима в Геф и возвратился.
\vs 1Ki 2:42 И послав призвал царь Семея и сказал ему: не клялся ли я тебе Господом и не объявлял ли тебе, говоря: <<знай, что в тот день, в который ты выйдешь и пойдешь куда-нибудь, непременно умрешь>>? и ты сказал мне: <<хорошо>>;
\vs 1Ki 2:43 зачем же ты не соблюл приказания, которое я дал тебе пред Господом с клятвою?
\vs 1Ki 2:44 И сказал царь Семею: ты знаешь и знает сердце твое все зло, какое ты сделал отцу моему Давиду; да обратит же Господь злобу твою на голову твою!
\vs 1Ki 2:45 а царь Соломон да будет благословен, и престол Давида да будет непоколебим пред Господом во веки!
\vs 1Ki 2:46 и повелел царь Ванее, сыну Иодаеву, и он пошел и поразил Семея, и тот умер.
\vs 1Ki 3:1 [Когда утвердилось царство в руках Соломона,] Соломон породнился с фараоном, царем Египетским, и взял за себя дочь фараона и ввел ее в город Давидов, доколе не построил дома своего и дома Господня и стены вокруг Иерусалима.
\vs 1Ki 3:2 Народ еще приносил жертвы на высотах, ибо не был построен дом имени Господа до того времени.
\vs 1Ki 3:3 И возлюбил Соломон Господа, ходя по уставу Давида, отца своего; но и он приносил жертвы и курения на высотах.
\vs 1Ki 3:4 И пошел царь в Гаваон, чтобы принести там жертву, ибо там был главный жертвенник. Тысячу всесожжений вознес Соломон на том жертвеннике.
\rsbpar\vs 1Ki 3:5 В Гаваоне явился Господь Соломону во сне ночью, и сказал Бог: проси, что дать тебе.
\vs 1Ki 3:6 И сказал Соломон: Ты сделал рабу Твоему Давиду, отцу моему, великую милость; и за то, что он ходил пред Тобою в истине и правде и с искренним сердцем пред Тобою, Ты сохранил ему эту великую милость и даровал ему сына, который сидел бы на престоле его, как это и есть ныне;
\vs 1Ki 3:7 и ныне, Господи Боже мой, Ты поставил раба Твоего царем вместо Давида, отца моего; но я отрок малый, не знаю ни моего выхода, ни входа;
\vs 1Ki 3:8 и раб Твой~--- среди народа Твоего, который избрал Ты, народа столь многочисленного, что по множеству его нельзя ни исчислить его, ни обозреть;
\vs 1Ki 3:9 даруй же рабу Твоему сердце разумное, чтобы судить народ Твой и различать, что добро и что зло; ибо кто может управлять этим многочисленным народом Твоим?
\vs 1Ki 3:10 И благоугодно было Господу, что Соломон просил этого.
\vs 1Ki 3:11 И сказал ему Бог: за то, что ты просил этого и не просил себе долгой жизни, не просил себе богатства, не просил себе душ врагов твоих, но просил себе разума, чтоб уметь судить,~---
\vs 1Ki 3:12 вот, Я сделаю по слову твоему: вот, Я даю тебе сердце мудрое и разумное, так что подобного тебе не было прежде тебя, и после тебя не восстанет подобный тебе;
\vs 1Ki 3:13 и то, чего ты не просил, Я даю тебе, и богатство и славу, так что не будет подобного тебе между царями во все дни твои;
\vs 1Ki 3:14 и если будешь ходить путем Моим, сохраняя уставы Мои и заповеди Мои, как ходил отец твой Давид, Я продолжу и дни твои.
\vs 1Ki 3:15 И пробудился Соломон, и вот, \bibemph{это было} сновидение. И пошел он в Иерусалим и стал [пред жертвенником] пред ковчегом завета Господня, и принес всесожжения и совершил \bibemph{жертвы} мирные, и сделал большой пир для всех слуг своих.
\rsbpar\vs 1Ki 3:16 Тогда пришли две женщины блудницы к царю и стали пред ним.
\vs 1Ki 3:17 И сказала одна женщина: о, господин мой! я и эта женщина живем в одном доме; и я родила при ней в этом доме;
\vs 1Ki 3:18 на третий день после того, как я родила, родила и эта женщина; и были мы вместе, и в доме никого постороннего с нами не было; только мы две были в доме;
\vs 1Ki 3:19 и умер сын этой женщины ночью, ибо она заспала его;
\vs 1Ki 3:20 и встала она ночью, и взяла сына моего от меня, когда я, раба твоя, спала, и положила его к своей груди, а своего мертвого сына положила к моей груди;
\vs 1Ki 3:21 утром я встала, чтобы покормить сына моего, и вот, он был мертвый; а когда я всмотрелась в него утром, то это был не мой сын, которого я родила.
\vs 1Ki 3:22 И сказала другая женщина: нет, мой сын живой, а твой сын мертвый. А та говорила ей: нет, твой сын мертвый, а мой живой. И говорили они так пред царем.
\vs 1Ki 3:23 И сказал царь: эта говорит: мой сын живой, а твой сын мертвый; а та говорит: нет, твой сын мертвый, а мой сын живой.
\vs 1Ki 3:24 И сказал царь: подайте мне меч. И принесли меч к царю.
\vs 1Ki 3:25 И сказал царь: рассеките живое дитя надвое и отдайте половину одной и половину другой.
\vs 1Ki 3:26 И отвечала та женщина, которой сын был живой, царю, ибо взволновалась вся внутренность ее от жалости к сыну своему: о, господин мой! отдайте ей этого ребенка живого и не умерщвляйте его. А другая говорила: пусть же не будет ни мне, ни тебе, рубите.
\vs 1Ki 3:27 И отвечал царь и сказал: отдайте этой живое дитя, и не умерщвляйте его: она~--- его мать.
\vs 1Ki 3:28 И услышал весь Израиль о суде, как рассудил царь; и стали бояться царя, ибо увидели, что мудрость Божия в нем, чтобы производить суд.
\vs 1Ki 4:1 И был царь Соломон царем над всем Израилем.
\vs 1Ki 4:2 И вот начальники, которые \bibemph{были} у него: Азария, сын Садока священника;
\vs 1Ki 4:3 Елихореф и Ахия, сыновья Сивы, писцы; Иосафат, сын Ахилуда, дееписатель;
\vs 1Ki 4:4 Ванея, сын Иодая, военачальник; Садок и Авиафар~--- священники;
\vs 1Ki 4:5 Азария, сын Нафана, начальник над приставниками, и Завуф, сын Нафана священника~--- друг царя;
\vs 1Ki 4:6 Ахисар~--- начальник над домом \bibemph{царским}, и Адонирам, сын Авды,~--- над податями.
\rsbpar\vs 1Ki 4:7 И было у Соломона двенадцать приставников над всем Израилем, и они доставляли продовольствие царю и дому его; каждый должен был доставлять продовольствие на один месяц в году.
\vs 1Ki 4:8 Вот имена их: Бен-Хур~--- на горе Ефремовой;
\vs 1Ki 4:9 Бен-Декер~--- в Макаце и в Шаалбиме, в Вефсамисе и в Елоне и в Беф-Ханане;
\vs 1Ki 4:10 Бен-Хесед~--- в Арюбофе; ему же принадлежал Соко и вся земля Хефер;
\vs 1Ki 4:11 Бен-Авинадав~--- \bibemph{над} всем Нафаф-Дором; Тафафь, дочь Соломона, была его женою;
\vs 1Ki 4:12 Ваана, сын Ахилуда, в Фаанахе и Мегиддо и во всем Беф-Сане, что близ Цартана ниже Иезрееля, от Беф-Сана до Абел-Мехола, и даже за Иокмеам;
\vs 1Ki 4:13 Бен-Гевер~--- в Рамофе Галаадском; у него были селения Иаира, сына Манассиина, что в Галааде; у него также область Аргов, что в Васане, шестьдесят больших городов со стенами и медными затворами;
\vs 1Ki 4:14 Ахинадав, сын Гиддо, в Маханаиме;
\vs 1Ki 4:15 Ахимаас~--- в \bibemph{земле} Неффалимовой; он взял себе в жену Васемафу, дочь Соломона;
\vs 1Ki 4:16 Ваана, сын Хушая, в \bibemph{земле} Асировой и в Баалофе;
\vs 1Ki 4:17 Иосафат, сын Паруаха, в \bibemph{земле} Иссахаровой;
\vs 1Ki 4:18 Шимей, сын Елы, в \bibemph{земле} Вениаминовой;
\vs 1Ki 4:19 Гевер, сын Урия, в земле Галаадской, в земле Сигона, царя Аморрейского, и Ога, царя Васанского. Он был приставник в этой земле.
\vs 1Ki 4:20 Иуда и Израиль, многочисленные как песок у моря, ели, пили и веселились.
\vs 1Ki 4:21 Соломон владел всеми царствами от реки \bibemph{Евфрата} до земли Филистимской и до пределов Египта. Они приносили дары и служили Соломону во все дни жизни его.
\vs 1Ki 4:22 Продовольствие Соломона на каждый день составляли: тридцать к\acc{о}ров муки пшеничной и шестьдесят к\acc{о}ров прочей муки,
\vs 1Ki 4:23 десять волов откормленных и двадцать волов с пастбища, и сто овец, кроме оленей, и серн, и сайгаков, и откормленных птиц;
\vs 1Ki 4:24 ибо он владычествовал над всею землею по эту сторону реки, от Типсаха до Газы, над всеми царями по эту сторону реки, и был у него мир со всеми окрестными странами.
\vs 1Ki 4:25 И жили Иуда и Израиль спокойно, каждый под виноградником своим и под смоковницею своею, от Дана до Вирсавии, во все дни Соломона.
\vs 1Ki 4:26 И было у Соломона сорок тысяч стойл для коней колесничных и двенадцать тысяч для конницы.
\vs 1Ki 4:27 И те приставники доставляли царю Соломону все принадлежащее к столу царя, каждый в свой месяц, и не допускали недостатка ни в чем.
\vs 1Ki 4:28 И ячмень и солому для коней и для мулов доставляли каждый в свою очередь на место, где находился царь.
\rsbpar\vs 1Ki 4:29 И дал Бог Соломону мудрость и весьма великий разум, и обширный ум, как песок на берегу моря.
\vs 1Ki 4:30 И была мудрость Соломона выше мудрости всех сынов востока и всей мудрости Египтян.
\vs 1Ki 4:31 Он был мудрее всех людей, мудрее и Ефана Езрахитянина, и Емана, и Халкола, и Дарды, сыновей Махола, и имя его было в славе у всех окрестных народов.
\vs 1Ki 4:32 И изрек он три тысячи притчей, и песней его было тысяча и пять;
\vs 1Ki 4:33 и говорил он о деревах, от кедра, что в Ливане, до иссопа, вырастающего из стены; говорил и о животных, и о птицах, и о пресмыкающихся, и о рыбах.
\vs 1Ki 4:34 И приходили от всех народов послушать мудрости Соломона, от всех царей земных, которые слышали о мудрости его.
\vs 1Ki 5:1 И послал Хирам, царь Тирский, слуг своих к Соломону, когда услышал, что его помазали в царя на место отца его; ибо Хирам был другом Давида во всю жизнь.
\vs 1Ki 5:2 И послал также и Соломон к Хираму сказать:
\vs 1Ki 5:3 ты знаешь, что Давид, отец мой, не мог построить дом имени Господа Бога своего по причине войн с окрестными народами, доколе Господь не покорил их под стопы ног его;
\vs 1Ki 5:4 ныне же Господь Бог мой даровал мне покой отовсюду: нет противника и нет более препон;
\vs 1Ki 5:5 и вот, я намерен построить дом имени Господа Бога моего, как сказал Господь отцу моему Давиду, говоря: <<сын твой, которого Я посажу вместо тебя на престоле твоем, он построит дом имени Моему>>;
\vs 1Ki 5:6 итак прикажи нарубить для меня кедров с Ливана; и вот, рабы мои будут вместе с твоими рабами, и я буду давать тебе плату за рабов твоих, какую ты назначишь; ибо ты знаешь, что у нас нет людей, которые умели бы рубить дерева так, как Сидоняне.
\vs 1Ki 5:7 Когда услышал Хирам слова Соломона, очень обрадовался и сказал: благословен ныне Господь, Который дал Давиду сына мудрого \bibemph{для управления} этим многочисленным народом!
\vs 1Ki 5:8 И послал Хирам к Соломону сказать: я выслушал то, за чем ты посылал ко мне, и исполню все желание твое о деревах кедровых и деревах кипарисовых;
\vs 1Ki 5:9 рабы мои свезут их с Ливана к морю, и я плотами доставлю их морем к месту, которое ты назначишь мне, и там сложу их, и ты возьмешь; но и ты исполни мое желание, чтобы доставлять хлеб для моего дома.
\vs 1Ki 5:10 И давал Хирам Соломону дерева кедровые и дерева кипарисовые, вполне по его желанию.
\vs 1Ki 5:11 А Соломон давал Хираму двадцать тысяч к\acc{о}ров пшеницы для продовольствия дома его и двадцать к\acc{о}ров оливкового выбитого масла: столько давал Соломон Хираму каждый год.
\rsbpar\vs 1Ki 5:12 Господь дал мудрость Соломону, как обещал ему. И был мир между Хирамом и Соломоном, и они заключили между собою союз.
\vs 1Ki 5:13 И обложил царь Соломон повинностью весь Израиль; повинность же состояла в тридцати тысячах человек.
\vs 1Ki 5:14 И посылал их на Ливан, по десяти тысяч на месяц, попеременно; месяц они были на Ливане, а два месяца в доме своем. Адонирам же начальствовал над ними.
\vs 1Ki 5:15 Еще у Соломона было семьдесят тысяч носящих тяжести и восемьдесят тысяч каменосеков в горах,
\vs 1Ki 5:16 кроме трех тысяч трехсот начальников, поставленных Соломоном над работою для надзора за народом, который производил работу.
\vs 1Ki 5:17 И повелел царь привозить камни большие, камни дорогие, для основания дома, камни обделанные.
\vs 1Ki 5:18 Обтесывали же их работники Соломоновы и работники Хирамовы и Гивлитяне, и приготовляли дерева и камни для строения дома [три года].
\vs 1Ki 6:1 В четыреста восьмидесятом году по исшествии сынов Израилевых из земли Египетской, в четвертый год царствования Соломонова над Израилем, в месяц Зиф, который есть второй месяц, начал он строить храм Господу.
\vs 1Ki 6:2 Храм, который построил царь Соломон Господу, длиною был в шестьдесят локтей, шириною в двадцать и вышиною в тридцать локтей,
\vs 1Ki 6:3 и притвор пред храмом в двадцать локтей длины, соответственно ширине храма, и в десять локтей ширины пред храмом.
\vs 1Ki 6:4 И сделал он в доме окна решетчатые, глухие с откосами.
\vs 1Ki 6:5 И сделал пристройку вокруг стен храма, вокруг храма и давира\fns{Отделение для Святаго Святых.}; и сделал боковые комнаты кругом.
\vs 1Ki 6:6 Нижний \bibemph{ярус} пристройки шириною был в пять локтей, средний шириною в шесть локтей, а третий шириною в семь локтей; ибо вокруг храма извне сделаны были уступы, дабы пристройка не прикасалась к стенам храма.
\vs 1Ki 6:7 Когда строился храм, на строение употребляемы были обтесанные камни; ни молота, ни тесла, ни всякого другого железного орудия не было слышно в храме при строении его.
\vs 1Ki 6:8 Вход в средний ярус был с правой стороны храма. По круглым лестницам всходили в средний \bibemph{ярус}, а от среднего в третий.
\vs 1Ki 6:9 И построил он храм, и кончил его, и обшил храм кедровыми досками.
\vs 1Ki 6:10 И пристроил ко всему храму боковые комнаты вышиною в пять локтей; они прикреплены были к храму посредством кедровых бревен.
\rsbpar\vs 1Ki 6:11 И было слово Господа к Соломону, и сказано ему:
\vs 1Ki 6:12 вот, ты строишь храм; если ты будешь ходить по уставам Моим, и поступать по определениям Моим и соблюдать все заповеди Мои, поступая по ним, то Я исполню на тебе слово Мое, которое Я сказал Давиду, отцу твоему,
\vs 1Ki 6:13 и буду жить среди сынов Израилевых, и не оставлю народа Моего Израиля.
\rsbpar\vs 1Ki 6:14 И построил Соломон храм и кончил его.
\vs 1Ki 6:15 И обложил стены храма внутри кедровыми досками; от пола храма до потолка внутри обложил деревом и покрыл пол храма кипарисовыми досками.
\vs 1Ki 6:16 И устроил в задней стороне храма, в двадцати локтях от края, стену, и обложил стены и потолок кедровыми досками, и устроил давир для Святаго Святых.
\vs 1Ki 6:17 Сорока локтей \bibemph{был} храм, то есть передняя часть храма.
\vs 1Ki 6:18 На кедрах внутри храма были вырезаны \bibemph{подобия} огурцов и распускающихся цветов; все было покрыто кедром, камня не видно было.
\vs 1Ki 6:19 Давир же внутри храма он приготовил для того, чтобы поставить там ковчег завета Господня.
\vs 1Ki 6:20 И давир был длиною в двадцать локтей, шириною в двадцать локтей и вышиною в двадцать локтей; он обложил его чистым золотом; обложил также и кедровый жертвенник.
\vs 1Ki 6:21 И обложил Соломон храм внутри чистым золотом, и протянул золотые цепи пред давиром, и обложил его золотом.
\vs 1Ki 6:22 Весь храм он обложил золотом, весь храм до конца, и весь жертвенник, который пред давиром, обложил золотом.
\vs 1Ki 6:23 И сделал в давире двух херувимов из масличного дерева, вышиною в десять локтей.
\vs 1Ki 6:24 Одно крыло херувима было в пять локтей и другое крыло херувима в пять локтей; десять локтей было от одного конца крыльев его до другого конца крыльев его.
\vs 1Ki 6:25 В десять локтей \bibemph{был} и другой херувим; одинаковой меры и одинакового вида \bibemph{были} оба херувима.
\vs 1Ki 6:26 Высота одного херувима \bibemph{была} десять локтей, также и другого херувима.
\vs 1Ki 6:27 И поставил он херувимов среди внутренней части храма. Крылья же херувимов были распростерты, и касалось крыло одного \bibemph{одной} стены, а крыло другого херувима касалось другой стены; другие же крылья их среди храма сходились крыло с крылом.
\vs 1Ki 6:28 И обложил он херувимов золотом.
\vs 1Ki 6:29 И на всех стенах храма кругом сделал резные изображения херувимов и пальмовых дерев и распускающихся цветов, внутри и вне.
\vs 1Ki 6:30 И пол в храме обложил золотом во внутренней и передней части.
\vs 1Ki 6:31 Для входа в давир сделал двери из масличного дерева, с пятиугольными косяками.
\vs 1Ki 6:32 На двух половинах дверей из масличного дерева он сделал резных херувимов и пальмы и распускающиеся цветы и обложил золотом; покрыл золотом и херувимов и пальмы.
\vs 1Ki 6:33 И у входа в храм сделал косяки из масличного дерева четырехугольные,
\vs 1Ki 6:34 и две двери из кипарисового дерева; обе половинки одной двери были подвижные, и обе половинки другой двери были подвижные.
\vs 1Ki 6:35 И вырезал \bibemph{на них} херувимов и пальмы и распускающиеся цветы и обложил золотом по резьбе.
\vs 1Ki 6:36 И построил внутренний двор из трех рядов обтесанного камня и из ряда кедровых брусьев.
\vs 1Ki 6:37 В четвертый год, в месяц Зиф, [в месяц второй,] положил он основание храму Господа,
\vs 1Ki 6:38 а на одиннадцатом году, в месяце Буле,~--- это месяц восьмой,~--- он окончил храм со всеми принадлежностями его и по всем предначертаниям его; строил его семь лет.
\vs 1Ki 7:1 А свой дом Соломон строил тринадцать лет и окончил весь дом свой.
\vs 1Ki 7:2 И построил он дом из дерева Ливанского, длиною во сто локтей, шириною в пятьдесят локтей, а вышиною в тридцать локтей, на четырех рядах кедровых столбов; и кедровые бревна \bibemph{положены были} на столбах.
\vs 1Ki 7:3 И настлан был помост из кедра над бревнами на сорока пяти столбах, по пятнадцати в ряд.
\vs 1Ki 7:4 Оконных косяков \bibemph{было} три ряда; и три ряда \bibemph{окон}, окно против окна.
\vs 1Ki 7:5 И все двери и дверные косяки были четырехугольные, и окно против окна, в три ряда.
\vs 1Ki 7:6 И притвор из столбов сделал он длиною в пятьдесят локтей, шириною в тридцать локтей, и пред ними крыльцо, и столбы, и порог пред ними.
\vs 1Ki 7:7 Еще притвор с престолом, с которого он судил, притвор для судилища сделал он и покрыл все полы кедром.
\vs 1Ki 7:8 В доме, где он жил, другой двор позади притвора был такого же устройства. И в доме дочери фараоновой, которую взял за себя Соломон, он сделал такой же притвор.
\vs 1Ki 7:9 Все это сделано было из дорогих камней, обтесанных по размеру, обрезанных пилою, с внутренней и наружной стороны, от основания до выступов, и с наружной стороны до большого двора.
\vs 1Ki 7:10 И в основание положены были камни дорогие, камни большие, камни в десять локтей и камни в восемь локтей,
\vs 1Ki 7:11 и сверху дорогие камни, обтесанные по размеру, и кедр.
\vs 1Ki 7:12 Большой двор огорожен был кругом тремя рядами тесаных камней и одним рядом кедровых бревен; также и внутренний двор храма Господа и притвор храма.
\rsbpar\vs 1Ki 7:13 И послал царь Соломон и взял из Тира Хирама,
\vs 1Ki 7:14 сына одной вдовы, из колена Неффалимова. Отец его Тирянин был медник; он владел способностью, искусством и уменьем выделывать всякие вещи из меди. И пришел он к царю Соломону и производил у него всякие работы:
\vs 1Ki 7:15 и сделал он два медных столба, каждый в восемнадцать локтей вышиною, и снурок в двенадцать локтей обнимал \bibemph{окружность} того и другого столба;
\vs 1Ki 7:16 и два венца, вылитых из меди, он сделал, чтобы положить наверху столбов: пять локтей вышины в одном венце и пять локтей вышины в другом венце;
\vs 1Ki 7:17 сетки плетеной работы и снурки в виде цепочек для венцов, которые были на верху столбов: семь на одном венце и семь на другом венце.
\vs 1Ki 7:18 Так сделал он столбы и два ряда гранатовых яблок вокруг сетки, чтобы покрыть венцы, которые на верху столбов; то же сделал и для другого венца.
\vs 1Ki 7:19 А в притворе венцы на верху столбов сделаны \bibemph{наподобие} лилии, в четыре локтя,
\vs 1Ki 7:20 и венцы на обоих столбах вверху, прямо над выпуклостью, которая подле сетки; и на другом венце, рядами кругом, двести гранатовых яблок.
\vs 1Ki 7:21 И поставил столбы к притвору храма; поставил столб на правой стороне и дал ему имя Иахин, и поставил столб на левой стороне и дал ему имя Воаз.
\vs 1Ki 7:22 И над столбами поставил \bibemph{венцы}, сделанные \bibemph{наподобие} лилии; так окончена работа над столбами.
\vs 1Ki 7:23 И сделал литое \bibemph{из меди} море,~--- от края его до края его десять локтей,~--- совсем круглое, вышиною в пять локтей, и снурок в тридцать локтей обнимал его кругом.
\vs 1Ki 7:24 \bibemph{Подобия} огурцов под краями его окружали его по десяти на локоть, окружали море со всех сторон в два ряда; \bibemph{подобия} огурцов были вылиты с ним одним литьем.
\vs 1Ki 7:25 Оно стояло на двенадцати волах: три глядели к северу, три глядели к западу, три глядели к югу и три глядели к востоку; море лежало на них, и зады их \bibemph{обращены были} внутрь под него.
\vs 1Ki 7:26 Толщиною оно было в ладонь, и края его, сделанные подобно краям чаши, \bibemph{походили} на распустившуюся лилию. Оно вмещало две тысячи батов.
\vs 1Ki 7:27 И сделал он десять медных подстав; длина каждой подставы~--- четыре локтя, ширина~--- четыре локтя и три локтя~--- вышина.
\vs 1Ki 7:28 И вот устройство подстав: у них стенки, стенки между наугольными пластинками;
\vs 1Ki 7:29 на стенках, которые между наугольниками, \bibemph{изображены} были львы, волы и херувимы; также и на наугольниках, а выше и ниже львов и волов~--- развесистые венки;
\vs 1Ki 7:30 у каждой подставы по четыре медных колеса и оси медные. На четырех углах выступы наподобие плеч, выступы литые внизу, под чашею, подле каждого венка.
\vs 1Ki 7:31 Отверстие от внутреннего венка до верха в один локоть; отверстие его круглое, подобно подножию столбов, в полтора локтя, и при отверстии его изваяния; но боковые стенки четырехугольные, не круглые.
\vs 1Ki 7:32 Под стенками было четыре колеса, и оси колес в подставах; вышина каждого колеса~--- полтора локтя.
\vs 1Ki 7:33 Устройство колес такое же, как устройство колес в колеснице; оси их, и ободья их, и спицы их, и ступицы их, все было литое.
\vs 1Ki 7:34 Четыре выступа на четырех углах каждой подставы; из подставы \bibemph{выходили} выступы ее.
\vs 1Ki 7:35 И на верху подставы круглое возвышение на пол-локтя вышины; и на верху подставы рукоятки ее и стенки ее из одной с нею массы.
\vs 1Ki 7:36 И изваял он на дощечках ее рукоятки и на стенках ее херувимов, львов и пальмы, сколько где позволяло место, и вокруг развесистые венки.
\vs 1Ki 7:37 Так сделал он десять подстав: у всех их одно литье, одна мера, один вид.
\vs 1Ki 7:38 И сделал десять медных умывальниц: каждая умывальница вмещала сорок батов, каждая умывальница была в четыре локтя, каждая умывальница стояла на одной из десяти подстав.
\vs 1Ki 7:39 И расставил подставы~--- пять на правой стороне храма и пять на левой стороне храма, а море поставил на правой стороне храма, на восточно-южной стороне.
\vs 1Ki 7:40 И сделал Хирам умывальницы и лопатки и чаши. И кончил Хирам всю работу, которую производил у царя Соломона для храма Господня:
\vs 1Ki 7:41 два столба и две опояски венцов, которые на верху столбов, и две сетки для покрытия двух опоясок венцов, которые на верху столбов;
\vs 1Ki 7:42 и четыреста гранатовых яблок на двух сетках; два ряда гранатовых яблок для каждой сетки, для покрытия двух опоясок венцов, которые на столбах;
\vs 1Ki 7:43 и десять подстав и десять умывальниц на подставах;
\vs 1Ki 7:44 одно море и двенадцать волов под морем;
\vs 1Ki 7:45 и тазы, и лопатки, и чаши. Все вещи, которые сделал Хирам царю Соломону для храма Господа, \bibemph{были} из полированной меди.
\vs 1Ki 7:46 Царь выливал их в глинистой земле, в окрестности Иордана, между Сокхофом и Цартаном.
\vs 1Ki 7:47 И поставил Соломон все сии вещи \bibemph{на место}. По причине чрезвычайного их множества, вес меди не определен.
\rsbpar\vs 1Ki 7:48 И сделал Соломон все вещи, которые в храме Господа: золотой жертвенник и золотой стол, на котором хлебы предложения;
\vs 1Ki 7:49 и светильники~--- пять по правую сторону и пять по левую сторону, пред задним отделением храма, из чистого золота, и цветы, и лампадки, и щипцы из золота;
\vs 1Ki 7:50 и блюда, и ножи, и чаши, и лотки, и кадильницы из чистого золота, и петли у дверей внутреннего храма во Святом Святых и у дверей в храме из золота же.
\vs 1Ki 7:51 Так совершена вся работа, которую производил царь Соломон для храма Господа. И принес Соломон посвященное Давидом, отцом его; серебро и золото и вещи отдал в сокровищницы храма Господня.
\vs 1Ki 8:1 Тогда созвал Соломон старейшин Израилевых и всех начальников колен, глав поколений сынов Израилевых, к царю Соломону в Иерусалим, чтобы перенести ковчег завета Господня из города Давидова, то есть Сиона.
\vs 1Ki 8:2 И собрались к царю Соломону на праздник все Израильтяне в месяце Афаниме, который есть седьмой месяц.
\vs 1Ki 8:3 И пришли все старейшины Израилевы; и подняли священники ковчег,
\vs 1Ki 8:4 и понесли ковчег Господень и скинию собрания и все священные вещи, которые были в скинии; и несли их священники и левиты.
\vs 1Ki 8:5 А царь Соломон и с ним все общество Израилево, собравшееся к нему, шли пред ковчегом, принося жертвы из мелкого и крупного скота, которых невозможно исчислить и определить, по множеству их.
\vs 1Ki 8:6 И внесли священники ковчег завета Господня на место его, в давир храма, во Святое Святых, под крылья херувимов.
\vs 1Ki 8:7 Ибо херувимы простирали крылья над местом ковчега, и покрывали херувимы сверху ковчег и шесты его.
\vs 1Ki 8:8 И выдвинулись шесты так, что головки шестов видны были из святилища пред давиром, но не выказывались наружу; они там и до сего дня.
\vs 1Ki 8:9 В ковчеге ничего не было, кроме двух каменных скрижалей, которые положил туда Моисей на Хориве, когда Господь заключил завет с сынами Израилевыми, по исшествии их из земли Египетской.
\vs 1Ki 8:10 Когда священники вышли из святилища, облако наполнило дом Господень;
\vs 1Ki 8:11 и не могли священники стоять на служении, по причине облака, ибо слава Господня наполнила храм Господень.
\vs 1Ki 8:12 Тогда сказал Соломон: Господь сказал, что Он благоволит обитать во мгле;
\vs 1Ki 8:13 я построил храм в жилище Тебе, место, чтобы пребывать Тебе во веки.
\vs 1Ki 8:14 И обратился царь лицем своим, и благословил все собрание Израильтян; все собрание Израильтян стояло,~---
\vs 1Ki 8:15 и сказал: благословен Господь Бог Израилев, Который сказал Своими устами Давиду, отцу моему, и ныне исполнил рукою Своею! Он говорил:
\vs 1Ki 8:16 <<с того дня, как Я вывел народ Мой Израиля из Египта, Я не избрал города ни в одном из колен Израилевых, чтобы построен был дом, в котором пребывало бы имя Мое; [но избрал Иерусалим для пребывания в нем имени Моего] и избрал Давида, чтобы быть ему над народом Моим Израилем>>.
\vs 1Ki 8:17 У Давида, отца моего, было на сердце построить храм имени Господа Бога Израилева;
\vs 1Ki 8:18 но Господь сказал Давиду, отцу моему: <<у тебя есть на сердце построить храм имени Моему; хорошо, что это у тебя лежит на сердце;
\vs 1Ki 8:19 однако не ты построишь храм, а сын твой, исшедший из чресл твоих, он построит храм имени Моему>>.
\vs 1Ki 8:20 И исполнил Господь слово Свое, которое изрек. Я вступил на место отца моего Давида и сел на престоле Израилевом, как сказал Господь, и построил храм имени Господа Бога Израилева;
\vs 1Ki 8:21 и приготовил там место для ковчега, в котором завет Господа, заключенный Им с отцами нашими, когда Он вывел их из земли Египетской.
\rsbpar\vs 1Ki 8:22 И стал Соломон пред жертвенником Господним впереди всего собрания Израильтян, и воздвиг руки свои к небу,
\vs 1Ki 8:23 и сказал: Господи Боже Израилев! нет подобного Тебе Бога на небесах вверху и на земле внизу; Ты хранишь завет и милость к рабам Твоим, ходящим пред Тобою всем сердцем своим.
\vs 1Ki 8:24 Ты исполнил рабу Твоему Давиду, отцу моему, что говорил ему; что изрек Ты устами Твоими, то в сей день совершил рукою Твоею.
\vs 1Ki 8:25 И ныне, Господи Боже Израилев, исполни рабу Твоему Давиду, отцу моему, то, что говорил Ты ему, сказав: <<не прекратится у тебя пред лицем Моим сидящий на престоле Израилевом, если только сыновья твои будут держаться пути своего, ходя предо Мною так, как ты ходил предо Мною>>.
\vs 1Ki 8:26 И ныне, Боже Израилев, да будет верно слово Твое, которое Ты изрек рабу Твоему Давиду, отцу моему!
\vs 1Ki 8:27 Поистине, Богу ли жить на земле? Небо и небо небес не вмещают Тебя, тем менее сей храм, который я построил [имени Твоему];
\vs 1Ki 8:28 но призри на молитву раба Твоего и на прошение его, Господи Боже мой; услышь воззвание и молитву, которою раб Твой умоляет Тебя ныне.
\vs 1Ki 8:29 Да будут очи Твои отверсты на храм сей день и ночь, на сие место, о котором Ты сказал: <<Мое имя будет там>>; услышь молитву, которою будет молиться раб Твой на месте сем.
\vs 1Ki 8:30 Услышь моление раба Твоего и народа Твоего Израиля, когда они будут молиться на месте сем; услышь на месте обитания Твоего, на небесах, услышь и помилуй.
\vs 1Ki 8:31 Когда кто согрешит против ближнего своего, и потребует от него клятвы, чтобы он поклялся, и для клятвы придут пред жертвенник Твой в храм сей,
\vs 1Ki 8:32 тогда Ты услышь с неба и произведи суд над рабами Твоими, обвини виновного, возложив поступок его на голову его, и оправдай правого, воздав ему по правде его.
\vs 1Ki 8:33 Когда народ Твой Израиль будет поражен неприятелем за то, что согрешил пред Тобою, и когда они обратятся к Тебе, и исповедают имя Твое, и будут просить и умолять Тебя в сем храме,
\vs 1Ki 8:34 тогда Ты услышь с неба и прости грех народа Твоего Израиля, и возврати их в землю, которую Ты дал отцам их.
\vs 1Ki 8:35 Когда заключится небо и не будет дождя за то, что они согрешат пред Тобою, и когда помолятся на месте сем и исповедают имя Твое и обратятся от греха своего, ибо Ты смирил их,
\vs 1Ki 8:36 тогда услышь с неба и прости грех рабов Твоих и народа Твоего Израиля, указав им добрый путь, по которому идти, и пошли дождь на землю Твою, которую Ты дал народу Твоему в наследие.
\vs 1Ki 8:37 Будет ли на земле голод, будет ли моровая язва, будет ли палящий ветер, ржавчина, саранча, червь, неприятель ли будет теснить его в земле его, \bibemph{будет ли} какое бедствие, какая болезнь,~---
\vs 1Ki 8:38 при всякой молитве, при всяком прошении, какое будет от какого-либо человека во всем народе Твоем Израиле, когда они почувствуют бедствие в сердце своем и прострут руки свои к храму сему,
\vs 1Ki 8:39 Ты услышь с неба, с места обитания Твоего, и помилуй; соделай и воздай каждому по путям его, как Ты усмотришь сердце его, ибо Ты один знаешь сердце всех сынов человеческих:
\vs 1Ki 8:40 чтобы они боялись Тебя во все дни, доколе живут на земле, которую Ты дал отцам нашим.
\vs 1Ki 8:41 Если и иноплеменник, который не от Твоего народа Израиля, придет из земли далекой ради имени Твоего,~---
\vs 1Ki 8:42 ибо и они услышат о Твоем имени великом и о Твоей руке сильной и о Твоей мышце простертой,~--- и придет он и помолится у храма сего,
\vs 1Ki 8:43 услышь с неба, с места обитания Твоего, и сделай все, о чем будет взывать к Тебе иноплеменник, чтобы все народы земли знали имя Твое, чтобы боялись Тебя, как народ Твой Израиль, чтобы знали, что именем Твоим называется храм сей, который я построил.
\vs 1Ki 8:44 Когда выйдет народ Твой на войну против врага своего путем, которым Ты пошлешь его, и будет молиться Господу, обратившись к городу, который Ты избрал, и к храму, который я построил имени Твоему,
\vs 1Ki 8:45 тогда услышь с неба молитву их и прошение их и сделай, что потребно для них.
\vs 1Ki 8:46 Когда они согрешат пред Тобою,~--- ибо нет человека, который не грешил бы,~--- и Ты прогневаешься на них и предашь их врагам, и пленившие их отведут их в неприятельскую землю, далекую или близкую;
\vs 1Ki 8:47 и когда они в земле, в которой будут находиться в плену, войдут в себя и обратятся и будут молиться Тебе в земле пленивших их, говоря: <<мы согрешили, сделали беззаконие, мы виновны>>;
\vs 1Ki 8:48 и когда обратятся к Тебе всем сердцем своим и всею душею своею в земле врагов, которые пленили их, и будут молиться Тебе, обратившись к земле своей, которую Ты дал отцам их, к городу, который Ты избрал, и к храму, который я построил имени Твоему,
\vs 1Ki 8:49 тогда услышь с неба, с места обитания Твоего, молитву и прошение их и сделай, что потребно для них;
\vs 1Ki 8:50 и прости народу Твоему, в чем он согрешил пред Тобою, и все проступки его, которые он сделал пред Тобою, и возбуди сострадание к ним в пленивших их, чтобы они были милостивы к ним:
\vs 1Ki 8:51 ибо они Твой народ и Твой удел, который Ты вывел из Египта, из железной печи.
\vs 1Ki 8:52 Да будут [уши Твои и] очи Твои отверсты на молитву раба Твоего и на молитву народа Твоего Израиля, чтобы слышать их всегда, когда они будут призывать Тебя,
\vs 1Ki 8:53 ибо Ты отделил их Себе в удел из всех народов земли, как Ты изрек чрез Моисея, раба Твоего, когда вывел отцов наших из Египта, Владыка Господи!
\rsbpar\vs 1Ki 8:54 Когда Соломон произнес все сие моление и прошение к Господу, тогда встал с колен от жертвенника Господня, \bibemph{руки же} его были распростерты к небу.
\vs 1Ki 8:55 И стоя благословил все собрание Израильтян, громким голосом говоря:
\vs 1Ki 8:56 благословен Господь [Бог], Который дал покой народу Своему Израилю, как говорил! не осталось неисполненным ни одного слова из всех благих слов Его, которые Он изрек чрез раба Своего Моисея;
\vs 1Ki 8:57 да будет с нами Господь Бог наш, как был Он с отцами нашими, да не оставит нас, да не покинет нас,
\vs 1Ki 8:58 наклоняя к Себе сердце наше, чтобы мы ходили по всем путям Его и соблюдали заповеди Его и уставы Его и законы Его, которые Он заповедал отцам нашим;
\vs 1Ki 8:59 и да будут слова сии, которыми я молился [ныне] пред Господом, близки к Господу Богу нашему день и ночь, дабы Он делал, что потребно для раба Своего, и что потребно для народа Своего Израиля, изо дня в день,
\vs 1Ki 8:60 чтобы все народы познали, что Господь есть Бог и нет кроме Его;
\vs 1Ki 8:61 да будет сердце ваше вполне предано Господу Богу нашему, чтобы ходить по уставам Его и соблюдать заповеди Его, как ныне.
\rsbpar\vs 1Ki 8:62 И царь и все Израильтяне с ним принесли жертву Господу.
\vs 1Ki 8:63 И принес Соломон в мирную жертву, которую принес он Господу, двадцать две тысячи крупного скота и сто двадцать тысяч мелкого скота. Так освятили храм Господу царь и все сыны Израилевы.
\vs 1Ki 8:64 В тот же день освятил царь среднюю часть двора, который пред храмом Господним, совершив там всесожжение и хлебное приношение и \bibemph{вознеся} тук мирных жертв, потому что медный жертвенник, который пред Господом, был мал для помещения всесожжения и хлебного приношения и тука мирных жертв.
\vs 1Ki 8:65 И сделал Соломон в это время праздник, и весь Израиль с ним,~--- большое собрание, \bibemph{сошедшееся} от входа в Емаф до реки Египетской пред Господом Богом нашим; [и ели, и пили, и молились пред Господом Богом нашим у построенного храма]~--- семь дней и еще семь дней, четырнадцать дней.
\vs 1Ki 8:66 В восьмой день Соломон отпустил народ. И благословили царя и пошли в шатры свои, радуясь и веселясь в сердце о всем добром, что сделал Господь рабу Своему Давиду и народу Своему Израилю.
\vs 1Ki 9:1 После того, как Соломон кончил строение храма Господня и дома царского и все, что Соломон желал сделать,
\vs 1Ki 9:2 явился Соломону Господь во второй раз, как явился ему в Гаваоне.
\vs 1Ki 9:3 И сказал ему Господь: Я услышал молитву твою и прошение твое, о чем ты просил Меня; [сделал все по молитве твоей]. Я освятил сей храм, который ты построил, чтобы пребывать имени Моему там вовек; и будут очи Мои и сердце Мое там во все дни.
\vs 1Ki 9:4 И если ты будешь ходить пред лицем Моим, как ходил отец твой Давид, в чистоте сердца и в правоте, исполняя все, что Я заповедал тебе, и если будешь хранить уставы Мои и законы Мои,
\vs 1Ki 9:5 то Я поставлю царский престол твой над Израилем вовек, как Я сказал отцу твоему Давиду, говоря: <<не прекратится у тебя сидящий на престоле Израилевом>>.
\vs 1Ki 9:6 Если же вы и сыновья ваши отступите от Меня и не будете соблюдать заповедей Моих и уставов Моих, которые Я дал вам, и пойдете и станете служить иным богам и поклоняться им,
\vs 1Ki 9:7 то Я истреблю Израиля с лица земли, которую Я дал ему, и храм, который Я освятил имени Моему, отвергну от лица Моего, и будет Израиль притчею и посмешищем у всех народов.
\vs 1Ki 9:8 И о храме сем высоком всякий, проходящий мимо его, ужаснется и свистнет, и скажет: <<за что Господь поступил так с сею землею и с сим храмом?>>
\vs 1Ki 9:9 И скажут: <<за то, что они оставили Господа Бога своего, Который вывел отцов их из земли Египетской, и приняли других богов, и поклонялись им и служили им,~--- за это навел на них Господь все сие бедствие>>.
\vs 1Ki 9:10 По окончании двадцати лет, в которые Соломон построил два дома,~--- дом Господень и дом царский,~---
\vs 1Ki 9:11 на что Хирам, царь Тирский, доставлял Соломону дерева кедровые и дерева кипарисовые и золото, по его желанию,~--- царь Соломон дал Хираму двадцать городов в земле Галилейской.
\vs 1Ki 9:12 И вышел Хирам из Тира посмотреть города, которые дал ему Соломон, и они не понравились ему.
\vs 1Ki 9:13 И сказал он: что это за города, которые ты, брат мой, дал мне? И назвал их землею Кавул, \bibemph{как называются они} до сего дня.
\vs 1Ki 9:14 И послал Хирам царю сто двадцать талантов золота.
\vs 1Ki 9:15 Вот распоряжение о подати, которую наложил царь Соломон, чтобы построить храм Господень и дом свой, и Милло, и стену Иерусалимскую, Гацор, и Мегиддо, и Газер.
\rsbpar\vs 1Ki 9:16 Фараон, царь Египетский, пришел и взял Газер, и сжег его огнем, и Хананеев, живших в городе, побил, и отдал его в приданое дочери своей, жене Соломоновой.
\vs 1Ki 9:17 И построил Соломон Газер и нижний Бефорон,
\vs 1Ki 9:18 и Ваалаф и Фадмор в пустыне,
\vs 1Ki 9:19 и все города для запасов, которые были у Соломона, и города для колесниц, и города для конницы и все то, что Соломон хотел построить в Иерусалиме и на Ливане и во всей земле своего владения.
\vs 1Ki 9:20 Весь народ, оставшийся от Аморреев, Хеттеев, Ферезеев, [Хананеев,] Евеев, Иевусеев и [Гергесеев], которые были не из сынов Израилевых,
\vs 1Ki 9:21 детей их, оставшихся после них на земле, которых сыны Израилевы не могли истребить, Соломон сделал оброчными работниками до сего дня.
\vs 1Ki 9:22 Сынов же Израилевых Соломон не делал работниками, но они были его воинами, его слугами, его вельможами, его военачальниками и вождями его колесниц и его всадников.
\vs 1Ki 9:23 Вот главные приставники над работами Соломоновыми: управлявших народом, который производил работы, было пятьсот пятьдесят.
\vs 1Ki 9:24 Дочь фараонова перешла из города Давидова в свой дом, который построил для нее Соломон; потом построил он Милло.
\vs 1Ki 9:25 И приносил Соломон три раза в год всесожжения и мирные жертвы на жертвеннике, который он построил Господу, и курение на нем совершал пред Господом. И окончил он \bibemph{строение} дома.
\vs 1Ki 9:26 Царь Соломон также сделал корабль в Ецион-Гавере, что при Елафе, на берегу Чермного моря, в земле Идумейской.
\vs 1Ki 9:27 И послал Хирам на корабле своих подданных корабельщиков, знающих море, с подданными Соломоновыми;
\vs 1Ki 9:28 и отправились они в Офир, и взяли оттуда золота четыреста двадцать талантов, и привезли царю Соломону.
\vs 1Ki 10:1 Царица Савская, услышав о славе Соломона во имя Господа, пришла испытать его загадками.
\vs 1Ki 10:2 И пришла она в Иерусалим с весьма большим богатством: верблюды навьючены \bibemph{были} благовониями и великим множеством золота и драгоценными камнями; и пришла к Соломону и беседовала с ним обо всем, что было у нее на сердце.
\vs 1Ki 10:3 И объяснил ей Соломон все слова ее, и не было ничего незнакомого царю, чего бы он не изъяснил ей.
\vs 1Ki 10:4 И увидела царица Савская всю мудрость Соломона и дом, который он построил,
\vs 1Ki 10:5 и пищу за столом его, и жилище рабов его, и стройность слуг его, и одежду их, и виночерпиев его, и всесожжения его, которые он приносил в храме Господнем. И не могла она более удержаться
\vs 1Ki 10:6 и сказала царю: верно то, что я слышала в земле своей о делах твоих и о мудрости твоей;
\vs 1Ki 10:7 но я не верила словам, доколе не пришла, и не увидели глаза мои: и вот, мне и в половину не сказано; мудрости и богатства у тебя больше, нежели как я слышала.
\vs 1Ki 10:8 Блаженны люди твои и блаженны сии слуги твои, которые всегда предстоят пред тобою и слышат мудрость твою!
\vs 1Ki 10:9 Да будет благословен Господь Бог твой, Который благоволил посадить тебя на престол Израилев! Господь, по вечной любви Своей к Израилю, поставил тебя царем, творить суд и правду.
\vs 1Ki 10:10 И подарила она царю сто двадцать талантов золота и великое множество благовоний и драгоценные камни; никогда еще не приходило такого множества благовоний, какое подарила царица Савская царю Соломону.
\vs 1Ki 10:11 И корабль Хирамов, который привозил золото из Офира, привез из Офира великое множество красного дерева и драгоценных камней.
\vs 1Ki 10:12 И сделал царь из сего красного дерева перила для храма Господня и для дома царского, и гусли и псалтири для певцов; никогда не приходило столько красного дерева и не видано было до сего дня.
\vs 1Ki 10:13 И царь Соломон дал царице Савской все, чего она желала и чего просила, сверх того, что подарил ей царь Соломон своими руками. И отправилась она обратно в свою землю, она и все слуги ее.
\rsbpar\vs 1Ki 10:14 В золоте, которое приходило Соломону в каждый год, весу было шестьсот шестьдесят шесть талантов золотых,
\vs 1Ki 10:15 сверх того, что \bibemph{получаемо было} от разносчиков товара и от торговли купцов, и от всех царей Аравийских и от областных начальников.
\vs 1Ki 10:16 И сделал царь Соломон двести больших щитов из кованого золота, по шестисот \bibemph{сиклей} пошло на каждый щит;
\vs 1Ki 10:17 и триста меньших щитов из кованого золота, по три мины золота пошло на каждый щит; и поставил их царь в доме из Ливанского дерева.
\vs 1Ki 10:18 И сделал царь большой престол из слоновой кости и обложил его чистым золотом;
\vs 1Ki 10:19 к престолу было шесть ступеней; верх сзади у престола был круглый, и были с обеих сторон у места сиденья локотники, и два льва стояли у локотников;
\vs 1Ki 10:20 и еще двенадцать львов стояли там на шести ступенях по обе стороны. Подобного сему не бывало ни в одном царстве.
\vs 1Ki 10:21 И все сосуды для питья у царя Соломона \bibemph{были} золотые, и все сосуды в доме из Ливанского дерева были из чистого золота; из серебра ничего не было, потому что серебро во дни Соломоновы считалось ни за что;
\vs 1Ki 10:22 ибо у царя был на море Фарсисский корабль с кораблем Хирамовым; в три года раз приходил Фарсисский корабль, привозивший золото и серебро, и слоновую кость, и обезьян, и павлинов.
\rsbpar\vs 1Ki 10:23 Царь Соломон превосходил всех царей земли богатством и мудростью.
\vs 1Ki 10:24 И все [цари] на земле искали видеть Соломона, чтобы послушать мудрости его, которую вложил Бог в сердце его.
\vs 1Ki 10:25 И они подносили ему, каждый от себя, в дар: сосуды серебряные и сосуды золотые, и одежды, и оружие, и благовония, коней и мулов, каждый год.
\vs 1Ki 10:26 И набрал Соломон колесниц и всадников; у него было тысяча четыреста колесниц\fns{В греческом переводе: сорок тысяч коней колесничных.} и двенадцать тысяч всадников; и разместил он их по колесничным городам и при царе в Иерусалиме. [И господствовал он над всеми морями от реки до земли Филистимской и до пределов Египта.]
\vs 1Ki 10:27 И сделал царь серебро в Иерусалиме равноценным с простыми камнями, а кедры, по их множеству, сделал равноценными с сикоморами, \bibemph{растущими} на низких местах.
\vs 1Ki 10:28 Коней же царю Соломону приводили из Египта и из Кувы; царские купцы покупали их из Кувы за деньги.
\vs 1Ki 10:29 Колесница из Египта получаема и доставляема была за шестьсот \bibemph{сиклей} серебра, а конь за сто пятьдесят. Таким же образом они руками своими доставляли \bibemph{все это} царям Хеттейским и царям Арамейским.
\vs 1Ki 11:1 И полюбил царь Соломон многих чужестранных женщин, кроме дочери фараоновой, Моавитянок, Аммонитянок, Идумеянок, Сидонянок, Хеттеянок,
\vs 1Ki 11:2 из тех народов, о которых Господь сказал сынам Израилевым: <<не входите к ним, и они пусть не входят к вам, чтобы они не склонили сердца вашего к своим богам>>; к ним прилепился Соломон любовью.
\vs 1Ki 11:3 И было у него семьсот жен и триста наложниц; и развратили жены его сердце его.
\vs 1Ki 11:4 Во время старости Соломона жены его склонили сердце его к иным богам, и сердце его не было вполне предано Господу Богу своему, как сердце Давида, отца его.
\vs 1Ki 11:5 И стал Соломон служить Астарте, божеству Сидонскому, и Милхому, мерзости Аммонитской.
\vs 1Ki 11:6 И делал Соломон неугодное пред очами Господа и не вполне последовал Господу, как Давид, отец его.
\vs 1Ki 11:7 Тогда построил Соломон капище Хамосу, мерзости Моавитской, на горе, которая пред Иерусалимом, и Молоху, мерзости Аммонитской.
\vs 1Ki 11:8 Так сделал он для всех своих чужестранных жен, которые кадили и приносили жертвы своим богам.
\vs 1Ki 11:9 И разгневался Господь на Соломона за то, что он уклонил сердце свое от Господа Бога Израилева, Который два раза являлся ему
\vs 1Ki 11:10 и заповедал ему, чтобы он не следовал иным богам; но он не исполнил того, что заповедал ему Господь [Бог].
\vs 1Ki 11:11 И сказал Господь Соломону: за то, что так у тебя делается, и ты не сохранил завета Моего и уставов Моих, которые Я заповедал тебе, Я отторгну от тебя царство и отдам его рабу твоему;
\vs 1Ki 11:12 но во дни твои Я не сделаю сего ради Давида, отца твоего; из руки сына твоего исторгну его;
\vs 1Ki 11:13 и не все царство исторгну; одно колено дам сыну твоему ради Давида, раба Моего, и ради Иерусалима, который Я избрал.
\rsbpar\vs 1Ki 11:14 И воздвиг Господь противника на Соломона, Адера Идумеянина, из царского Идумейского рода.
\vs 1Ki 11:15 Когда Давид был в Идумее, и военачальник Иоав пришел для погребения убитых и избил весь мужеский пол в Идумее,~---
\vs 1Ki 11:16 ибо шесть месяцев прожил там Иоав и все Израильтяне, доколе не истребили всего мужеского пола в Идумее,~---
\vs 1Ki 11:17 тогда сей Адер убежал в Египет и с ним несколько Идумеян, служивших при отце его; Адер \bibemph{был тогда} малым ребенком.
\vs 1Ki 11:18 Отправившись из Мадиама, они пришли в Фаран и взяли с собою людей из Фарана и пришли в Египет к фараону, царю Египетскому. [Адер вошел к фараону, и] он дал ему дом, и назначил ему содержание, и дал ему землю.
\vs 1Ki 11:19 Адер снискал у фараона большую милость, так что он дал ему в жену сестру своей жены, сестру царицы Тахпенесы.
\vs 1Ki 11:20 И родила ему сестра Тахпенесы сына Генувата. Тахпенеса воспитывала его в доме фараоновом; и жил Генуват в доме фараоновом вместе с сыновьями фараоновыми.
\vs 1Ki 11:21 Когда Адер услышал, что Давид почил с отцами своими и что военачальник Иоав умер, то сказал фараону: отпусти меня, я пойду в свою землю.
\vs 1Ki 11:22 И сказал ему фараон: разве ты нуждаешься в чем у меня, что хочешь идти в свою землю? Он отвечал: нет, но отпусти меня. [И возвратился Адер в землю свою.]
\vs 1Ki 11:23 И воздвиг Бог против Соломона еще противника, Разона, сына Елиады, который убежал от государя своего Адраазара, царя Сувского,
\vs 1Ki 11:24 и, собрав около себя людей, сделался начальником шайки, после того, как Давид разбил \bibemph{Адраазара}; и пошли они в Дамаск, и водворились там, и владычествовали в Дамаске.
\vs 1Ki 11:25 И был он противником Израиля во все дни Соломона. Кроме зла, \bibemph{причиненного} Адером, он всегда вредил Израилю и сделался царем Сирии.
\vs 1Ki 11:26 И Иеровоам, сын Наватов, Ефремлянин из Цареды,~--- имя матери его вдовы: Церуа,~--- раб Соломонов, поднял руку на царя.
\vs 1Ki 11:27 И вот обстоятельство, по которому он поднял руку на царя: Соломон строил Милло, починивал повреждения в городе Давида, отца своего.
\vs 1Ki 11:28 Иеровоам был человек мужественный. Соломон, заметив, что этот молодой человек умеет делать дело, поставил его смотрителем над оброчными из дома Иосифова.
\vs 1Ki 11:29 В то время случилось Иеровоаму выйти из Иерусалима; и встретил его на дороге пророк Ахия Силомлянин, и на нем была новая одежда. На поле их было только двое.
\vs 1Ki 11:30 И взял Ахия новую одежду, которая была на нем, и разодрал ее на двенадцать частей,
\vs 1Ki 11:31 и сказал Иеровоаму: возьми себе десять частей, ибо так говорит Господь Бог Израилев: вот, Я исторгаю царство из руки Соломоновой и даю тебе десять колен,
\vs 1Ki 11:32 а одно колено\fns{В греческом переводе: два колена.} останется за ним ради раба Моего Давида и ради города Иерусалима, который Я избрал из всех колен Израилевых.
\vs 1Ki 11:33 Это за то, что они оставили Меня и стали поклоняться Астарте, божеству Сидонскому, и Хамосу, богу Моавитскому, и Милхому, богу Аммонитскому, и не пошли путями Моими, чтобы делать угодное пред очами Моими и \bibemph{соблюдать} уставы Мои и заповеди Мои, подобно Давиду, отцу его.
\vs 1Ki 11:34 Я не беру всего царства из руки его, но Я оставляю его владыкою на все дни жизни его ради Давида, раба Моего, которого Я избрал, который соблюдал заповеди Мои и уставы Мои;
\vs 1Ki 11:35 но возьму царство из руки сына его и дам тебе из него десять колен;
\vs 1Ki 11:36 а сыну его дам одно колено, дабы оставался светильник Давида, раба Моего, во все дни пред лицем Моим, в городе Иерусалиме, который Я избрал Себе для пребывания там имени Моего.
\vs 1Ki 11:37 Тебя Я избираю, и ты будешь владычествовать над всем, чего пожелает душа твоя, и будешь царем над Израилем;
\vs 1Ki 11:38 и если будешь соблюдать все, что Я заповедую тебе, и будешь ходить путями Моими и делать угодное пред очами Моими, соблюдая уставы Мои и заповеди Мои, как делал раб Мой Давид, то Я буду с тобою и устрою тебе дом твердый, как Я устроил Давиду, и отдам тебе Израиля;
\vs 1Ki 11:39 и смирю Я род Давидов за сие, но не на все дни.
\vs 1Ki 11:40 Соломон же хотел умертвить Иеровоама; но Иеровоам встал и убежал в Египет к Сусакиму, царю Египетскому, и жил в Египте до смерти Соломоновой.
\rsbpar\vs 1Ki 11:41 Прочие события Соломоновы и все, что он делал, и мудрость его описаны в книге дел Соломоновых.
\vs 1Ki 11:42 Времени царствования Соломонова в Иерусалиме над всем Израилем \bibemph{было} сорок лет.
\vs 1Ki 11:43 И почил Соломон с отцами своими и погребен был в городе Давида, отца своего, и воцарился вместо него сын его Ровоам.
\vs 1Ki 12:1 И пошел Ровоам в Сихем, ибо в Сихем пришли все Израильтяне, чтобы воцарить его.
\vs 1Ki 12:2 И услышал о том Иеровоам, сын Наватов, когда находился еще в Египте, куда убежал от царя Соломона, и возвратился Иеровоам из Египта;
\vs 1Ki 12:3 и послали за ним и призвали его. Тогда Иеровоам и все собрание Израильтян пришли и говорили [царю] Ровоаму и сказали:
\vs 1Ki 12:4 отец твой наложил на нас тяжкое иго, ты же облегчи нам жестокую работу отца твоего и тяжкое иго, которое он наложил на нас, и тогда мы будем служить тебе.
\vs 1Ki 12:5 И сказал он им: пойдите и чрез три дня опять придите ко мне. И пошел народ.
\vs 1Ki 12:6 Царь Ровоам советовался со старцами, которые предстояли пред Соломоном, отцом его, при жизни его, и говорил: как посоветуете вы мне отвечать сему народу?
\vs 1Ki 12:7 Они говорили ему и сказали: если ты на сей день будешь слугою народу сему и услужишь ему, и удовлетворишь им и будешь говорить им ласково, то они будут твоими рабами на все дни.
\vs 1Ki 12:8 Но он пренебрег совет старцев, что они советовали ему, и советовался с молодыми людьми, которые выросли вместе с ним и которые предстояли пред ним,
\vs 1Ki 12:9 и сказал им: что вы посоветуете мне отвечать народу сему, который говорил мне и сказал: <<облегчи иго, которое наложил на нас отец твой>>?
\vs 1Ki 12:10 И говорили ему молодые люди, которые выросли вместе с ним, и сказали: так скажи народу сему, который говорил тебе и сказал: <<отец твой наложил на нас тяжкое иго, ты же облегчи нас>>; так скажи им: <<мой мизинец толще чресл отца моего;
\vs 1Ki 12:11 итак, если отец мой обременял вас тяжким игом, то я увеличу иго ваше; отец мой наказывал вас бичами, а я буду наказывать вас скорпионами>>.
\vs 1Ki 12:12 Иеровоам и весь народ пришли к Ровоаму на третий день, как приказал царь, сказав: придите ко мне на третий день.
\vs 1Ki 12:13 И отвечал царь народу сурово и пренебрег совет старцев, что они советовали ему;
\vs 1Ki 12:14 и говорил он по совету молодых людей и сказал: отец мой наложил на вас тяжкое иго, а я увеличу иго ваше; отец мой наказывал вас бичами, а я буду наказывать вас скорпионами.
\vs 1Ki 12:15 И не послушал царь народа, ибо так суждено было Господом, чтобы исполнилось слово Его, которое изрек Господь чрез Ахию Силомлянина Иеровоаму, сыну Наватову.
\vs 1Ki 12:16 И увидели все Израильтяне, что царь не послушал их. И отвечал народ царю и сказал: какая нам часть в Давиде? нет нам доли в сыне Иессеевом; по шатрам своим, Израиль! теперь знай свой дом, Давид! И разошелся Израиль по шатрам своим.
\vs 1Ki 12:17 Только над сынами Израилевыми, жившими в городах Иудиных, царствовал Ровоам.
\vs 1Ki 12:18 И послал царь Ровоам Адонирама, начальника над податью; но все Израильтяне забросали его каменьями, и он умер; царь же Ровоам поспешно взошел на колесницу, чтоб убежать в Иерусалим.
\vs 1Ki 12:19 И отложился Израиль от дома Давидова до сего дня.
\vs 1Ki 12:20 Когда услышали все Израильтяне, что Иеровоам возвратился [из Египта], то послали и призвали его в собрание, и воцарили его над всеми Израильтянами. За домом Давидовым не осталось никого, кроме колена Иудина [и Вениаминова].
\vs 1Ki 12:21 Ровоам, прибыв в Иерусалим, собрал из всего дома Иудина и из колена Вениаминова сто восемьдесят тысяч отборных воинов, дабы воевать с домом Израилевым для того, чтобы возвратить царство Ровоаму, сыну Соломонову.
\rsbpar\vs 1Ki 12:22 И было слово Божие к Самею, человеку Божию, и сказано:
\vs 1Ki 12:23 скажи Ровоаму, сыну Соломонову, царю Иудейскому, и всему дому Иудину и Вениаминову и прочему народу:
\vs 1Ki 12:24 так говорит Господь: не ходите и не начинайте войны с братьями вашими, сынами Израилевыми; возвратитесь каждый в дом свой, ибо от Меня это было. И послушались они слова Господня и пошли назад по слову Господню.
\rsbpar\vs 1Ki 12:25 И обстроил Иеровоам Сихем на горе Ефремовой и поселился в нем; оттуда пошел и построил Пенуил.
\vs 1Ki 12:26 И говорил Иеровоам в сердце своем: царство может опять перейти к дому Давидову;
\vs 1Ki 12:27 если народ сей будет ходить в Иерусалим для жертвоприношения в доме Господнем, то сердце народа сего обратится к государю своему, к Ровоаму, царю Иудейскому, и убьют они меня и возвратятся к Ровоаму, царю Иудейскому.
\vs 1Ki 12:28 И посоветовавшись царь сделал двух золотых тельцов и сказал [народу]: не нужно вам ходить в Иерусалим; вот боги твои, Израиль, которые вывели тебя из земли Египетской.
\vs 1Ki 12:29 И поставил одного в Вефиле, а другого в Дане.
\vs 1Ki 12:30 И повело это ко греху, ибо народ стал ходить к одному \bibemph{из них}, даже в Дан, [и оставили храм Господень].
\vs 1Ki 12:31 И построил он капище на высоте и поставил из народа священников, которые не были из сынов Левииных.
\vs 1Ki 12:32 И установил Иеровоам праздник в восьмой месяц, в пятнадцатый день месяца, подобный тому празднику, какой был в Иудее, и приносил жертвы на жертвеннике; то же сделал он в Вефиле, чтобы приносить жертву тельцам, которых сделал. И поставил в Вефиле священников высот, которые устроил,
\vs 1Ki 12:33 и принес жертвы на жертвеннике, который он сделал в Вефиле, в пятнадцатый день восьмого месяца, месяца, который он произвольно назначил; и установил праздник для сынов Израилевых, и подошел к жертвеннику, чтобы совершить курение.
\vs 1Ki 13:1 И вот, человек Божий пришел из Иудеи по слову Господню в Вефиль, в то время, как Иеровоам стоял у жертвенника, чтобы совершить курение.
\vs 1Ki 13:2 И произнес к жертвеннику слово Господне и сказал: жертвенник, жертвенник! так говорит Господь: вот, родится сын дому Давидову, имя ему Иосия, и принесет на тебе в жертву священников высот, совершающих на тебе курение, и человеческие кости сожжет на тебе.
\vs 1Ki 13:3 И дал в тот день знамение, сказав: вот знамение того, что это изрек Господь: вот, этот жертвенник распадется, и пепел, который на нем, рассыплется.
\vs 1Ki 13:4 Когда царь услышал слово человека Божия, произнесенное к жертвеннику в Вефиле, то простер Иеровоам руку свою от жертвенника, говоря: возьмите его. И одеревенела рука его, которую он простер на него, и не мог он поворотить ее к себе.
\vs 1Ki 13:5 И жертвенник распался, и пепел с жертвенника рассыпался, по знамению, которое дал человек Божий словом Господним.
\vs 1Ki 13:6 И сказал царь [Иеровоам] человеку Божию: умилостиви лице Господа Бога твоего и помолись обо мне, чтобы рука моя могла поворотиться ко мне. И умилостивил человек Божий лице Господа, и рука царя поворотилась к нему и стала, как прежде.
\vs 1Ki 13:7 И сказал царь человеку Божию: зайди со мною в дом и подкрепи себя пищею, и я дам тебе подарок.
\vs 1Ki 13:8 Но человек Божий сказал царю: хотя бы ты давал мне полдома твоего, я не пойду с тобою и не буду есть хлеба и не буду пить воды в этом месте,
\vs 1Ki 13:9 ибо так заповедано мне словом Господним: <<не ешь там хлеба и не пей воды и не возвращайся тою дорогою, которою ты шел>>.
\vs 1Ki 13:10 И пошел он другою дорогою и не пошел обратно тою дорогою, которою пришел в Вефиль.
\rsbpar\vs 1Ki 13:11 В Вефиле жил один пророк-старец. Сын его пришел и рассказал ему все, что сделал сегодня человек Божий в Вефиле; и слова, какие он говорил царю, пересказали \bibemph{сыновья} отцу своему.
\vs 1Ki 13:12 И спросил их отец их: какою дорогою он пошел? И показали сыновья его, какою дорогою пошел человек Божий, приходивший из Иудеи.
\vs 1Ki 13:13 И сказал он сыновьям своим: оседлайте мне осла. И оседлали ему осла, и он сел на него.
\vs 1Ki 13:14 И поехал за человеком Божиим, и нашел его сидящего под дубом, и сказал ему: ты ли человек Божий, пришедший из Иудеи? И сказал тот: я.
\vs 1Ki 13:15 И сказал ему: зайди ко мне в дом и поешь хлеба.
\vs 1Ki 13:16 Тот сказал: я не могу возвратиться с тобою и пойти к тебе; не буду есть хлеба и не буду пить у тебя воды в сем месте,
\vs 1Ki 13:17 ибо словом Господним сказано мне: <<не ешь хлеба и не пей там воды и не возвращайся тою дорогою, которою ты шел>>.
\vs 1Ki 13:18 И сказал он ему: и я пророк такой же, как ты, и Ангел говорил мне словом Господним, и сказал: <<вороти его к себе в дом; пусть поест он хлеба и напьется воды>>.~--- Он солгал ему.
\vs 1Ki 13:19 И тот воротился с ним, и поел хлеба в его доме, и напился воды.
\vs 1Ki 13:20 Когда они еще сидели за столом, слово Господне было к пророку, воротившему его.
\vs 1Ki 13:21 И произнес он к человеку Божию, пришедшему из Иудеи, и сказал: так говорит Господь: за то, что ты не повиновался устам Господа и не соблюл повеления, которое заповедал тебе Господь Бог твой,
\vs 1Ki 13:22 но воротился, ел хлеб и пил воду в том месте, о котором Он сказал тебе: <<не ешь хлеба и не пей воды>>, тело твое не войдет в гробницу отцов твоих.
\vs 1Ki 13:23 После того, как тот поел хлеба и напился, он оседлал осла для пророка, которого он воротил.
\vs 1Ki 13:24 И отправился тот. И встретил его на дороге лев и умертвил его. И лежало тело его, брошенное на дороге; осел же стоял подле него, и лев стоял подле тела.
\vs 1Ki 13:25 И вот, проходившие мимо люди увидели тело, брошенное на дороге, и льва, стоящего подле тела, и пошли и рассказали в городе, в котором жил пророк-старец.
\vs 1Ki 13:26 Пророк, воротивший его с дороги, услышав \bibemph{это}, сказал: это тот человек Божий, который не повиновался устам Господа; Господь предал его льву, который изломал его и умертвил его, по слову Господа, которое Он изрек ему.
\vs 1Ki 13:27 И сказал сыновьям своим: оседлайте мне осла. И оседлали они.
\vs 1Ki 13:28 Он отправился и нашел тело его, брошенное на дороге; осел же и лев стояли подле тела; лев не съел тела и не изломал осла.
\vs 1Ki 13:29 И поднял пророк тело человека Божия, и положил его на осла, и повез его обратно. И пошел пророк-старец в город \bibemph{свой}, чтобы оплакать и похоронить его.
\vs 1Ki 13:30 И положил тело его в своей гробнице и плакал по нем: увы, брат мой!
\vs 1Ki 13:31 После погребения его он сказал сыновьям своим: когда я умру, похороните меня в гробнице, в которой погребен человек Божий; подле костей его положите кости мои;
\vs 1Ki 13:32 ибо сбудется слово, которое он по повелению Господню произнес о жертвеннике в Вефиле и о всех капищах на высотах, в городах Самарийских.
\vs 1Ki 13:33 И после сего события Иеровоам не сошел со своей худой дороги, но продолжал ставить из народа священников высот; кто хотел, того он посвящал, и тот становился священником высот.
\vs 1Ki 13:34 Это вело дом Иеровоамов ко греху и к погибели и к истреблению его с лица земли.
\vs 1Ki 14:1 В то время заболел Авия, сын Иеровоамов.
\vs 1Ki 14:2 И сказал Иеровоам жене своей: встань и переоденься, чтобы не узнали, что ты жена Иеровоамова, и пойди в Силом. Там есть пророк Ахия; он предсказал мне, что я буду царем сего народа.
\vs 1Ki 14:3 И возьми с собою [для человека Божия] десять хлебов, и лепешек, и кувшин меду, и пойди к нему: он скажет тебе, что будет с отроком.
\vs 1Ki 14:4 Жена Иеровоама так и сделала: встала, пошла в Силом и пришла в дом Ахии. Ахия уже не мог видеть, ибо глаза его сделались неподвижны от старости.
\vs 1Ki 14:5 И сказал Господь Ахии: вот, идет жена Иеровоамова спросить тебя о сыне своем, ибо он болен; так и так говори ей; она придет переодетая.
\vs 1Ki 14:6 Ахия, услышав шорох от ног ее, когда она вошла в дверь, сказал: войди, жена Иеровоамова; для чего было тебе переодеваться? Я грозный посланник к тебе.
\vs 1Ki 14:7 Пойди, скажи Иеровоаму: так говорит Господь Бог Израилев: Я возвысил тебя из среды простого народа и поставил вождем народа Моего Израиля,
\vs 1Ki 14:8 и отторг царство от дома Давидова и дал его тебе; а ты не таков, как раб Мой Давид, который соблюдал заповеди Мои и который последовал Мне всем сердцем своим, делая только угодное пред очами Моими;
\vs 1Ki 14:9 ты поступал хуже всех, которые были прежде тебя, и пошел, и сделал себе иных богов и истуканов, чтобы раздражить Меня, Меня же отбросил назад;
\vs 1Ki 14:10 за это Я наведу беды на дом Иеровоамов и истреблю у Иеровоама \bibemph{до} мочащегося к стене, заключенного и оставшегося в Израиле, и вымету дом Иеровоамов, как выметают сор, дочиста;
\vs 1Ki 14:11 кто умрет у Иеровоама в городе, того съедят псы, а кто умрет на поле, того склюют птицы небесные; так Господь сказал.
\vs 1Ki 14:12 Встань и иди в дом твой; и как скоро нога твоя ступит в город, умрет дитя;
\vs 1Ki 14:13 и оплачут его все Израильтяне и похоронят его, ибо он один у Иеровоама войдет в гробницу, так как в нем, из дома Иеровоамова, нашлось нечто доброе пред Господом Богом Израилевым.
\vs 1Ki 14:14 И восставит Себе Господь над Израилем царя, который истребит дом Иеровоамов в тот день; и что? даже теперь.
\vs 1Ki 14:15 И поразит Господь Израиля, и \bibemph{будет он}, как тростник, колеблемый в воде, и извергнет Израильтян из этой доброй земли, которую дал отцам их, и развеет их за реку, за то, что они сделали у себя идолов, раздражая Господа;
\vs 1Ki 14:16 и предаст [Господь] Израиля за грехи Иеровоама, которые он сам сделал и которыми ввел в грех Израиля.
\vs 1Ki 14:17 И встала жена Иеровоамова, и пошла, и пришла в Фирцу; и лишь только переступила чрез порог дома, дитя умерло.
\vs 1Ki 14:18 И похоронили его, и оплакали его все Израильтяне, по слову Господа, которое Он изрек чрез раба Своего Ахию пророка.
\rsbpar\vs 1Ki 14:19 Прочие дела Иеровоама, как он воевал и как царствовал, описаны в летописи царей Израильских.
\vs 1Ki 14:20 Времени царствования Иеровоамова было двадцать два года; и почил он с отцами своими, и воцарился Нават, сын его, вместо него.
\rsbpar\vs 1Ki 14:21 Ровоам, сын Соломонов, царствовал в Иудее. Сорок один год было Ровоаму, когда он воцарился, и семнадцать лет царствовал в Иерусалиме, в городе, который избрал Господь из всех колен Израилевых, чтобы пребывало там имя Его. Имя матери его Наама Аммонитянка.
\vs 1Ki 14:22 И делал Иуда неугодное пред очами Господа, и раздражали Его более всего того, что сделали отцы их своими грехами, какими они грешили.
\vs 1Ki 14:23 И устроили они у себя высоты и статуи и капища на всяком высоком холме и под всяким тенистым деревом.
\vs 1Ki 14:24 И блудники были также в этой земле и делали все мерзости тех народов, которых Господь прогнал от лица сынов Израилевых.
\vs 1Ki 14:25 На пятом году царствования Ровоамова, Сусаким, царь Египетский, вышел против Иерусалима
\vs 1Ki 14:26 и взял сокровища дома Господня и сокровища дома царского [и золотые щиты, которые взял Давид от рабов Адраазара, царя Сувского, и внес в Иерусалим]. Всё взял; взял и все золотые щиты, которые сделал Соломон.
\vs 1Ki 14:27 И сделал царь Ровоам вместо них медные щиты и отдал их на руки начальникам телохранителей, которые охраняли вход в дом царя.
\vs 1Ki 14:28 Когда царь выходил в дом Господень, телохранители несли их, и потом опять относили их в палату телохранителей.
\vs 1Ki 14:29 Прочее о Ровоаме и обо всем, что он делал, описано в летописи царей Иудейских.
\vs 1Ki 14:30 Между Ровоамом и Иеровоамом была война во все дни \bibemph{жизни их}.
\vs 1Ki 14:31 И почил Ровоам с отцами своими и погребен с отцами своими в городе Давидовом. Имя матери его Наама Аммонитянка. И воцарился Авия, сын его, вместо него.
\vs 1Ki 15:1 В восемнадцатый год царствования Иеровоама, сына Наватова, Авия воцарился над Иудеями.
\vs 1Ki 15:2 Три года он царствовал в Иерусалиме; имя матери его Мааха, дочь Авессалома.
\vs 1Ki 15:3 Он ходил во всех грехах отца своего, которые тот делал прежде него, и сердце его не было предано Господу Богу его, как сердце Давида, отца его.
\vs 1Ki 15:4 Но ради Давида Господь Бог его дал ему светильник в Иерусалиме, восставив по нем сына его и утвердив Иерусалим,
\vs 1Ki 15:5 потому что Давид делал угодное пред очами Господа и не отступал от всего того, что Он заповедал ему, во все дни жизни своей, кроме поступка с Уриею Хеттеянином.
\vs 1Ki 15:6 И война была между Ровоамом и Иеровоамом во все дни жизни их.
\rsbpar\vs 1Ki 15:7 Прочие дела Авии, всё, что он сделал, описано в летописи царей Иудейских. И была война между Авиею и Иеровоамом.
\vs 1Ki 15:8 И почил Авия с отцами своими, и похоронили его в городе Давидовом. И воцарился Аса, сын его, вместо него.
\rsbpar\vs 1Ki 15:9 В двадцатый год \bibemph{царствования} Иеровоама, царя Израильского, воцарился Аса над Иудеями
\vs 1Ki 15:10 и сорок один год царствовал в Иерусалиме; имя матери его Ан\acc{а}, дочь Авессалома.
\vs 1Ki 15:11 Аса делал угодное пред очами Господа, как Давид, отец его.
\vs 1Ki 15:12 Он изгнал блудников из земли и отверг всех идолов, которых сделали отцы его,
\vs 1Ki 15:13 и даже мать свою Ан\acc{у} лишил звания царицы за то, что она сделала истукан Астарты; и изрубил Аса истукан ее и сжег у потока Кедрона.
\vs 1Ki 15:14 Высоты же не были уничтожены. Но сердце Асы было предано Господу во все дни его.
\vs 1Ki 15:15 И внес он в дом Господень вещи, посвященные отцом его, и вещи, посвященные им: серебро и золото и сосуды.
\vs 1Ki 15:16 И война была между Асою и Ваасою, царем Израильским, во все дни их.
\vs 1Ki 15:17 И вышел Вааса, царь Израильский, против Иудеи и начал строить Раму, чтобы никто не выходил и не уходил к Асе, царю Иудейскому.
\vs 1Ki 15:18 И взял Аса все серебро и золото, остававшееся в сокровищницах дома Господня и в сокровищницах дома царского, и дал его в руки слуг своих, и послал их царь Аса к Венададу, сыну Тавримона, сына Хезионова, царю Сирийскому, жившему в Дамаске, и сказал:
\vs 1Ki 15:19 союз да будет между мною и между тобою, \bibemph{как был} между отцом моим и между отцом твоим; вот, я посылаю тебе в дар серебро и золото; расторгни союз твой с Ваасою, царем Израильским, чтобы он отошел от меня.
\vs 1Ki 15:20 И послушался Венадад царя Асы, и послал военачальников своих против городов Израильских, и поразил Аин и Дан и Авел-Беф-Мааху и весь Киннероф, по всей земле Неффалима.
\vs 1Ki 15:21 Услышав \bibemph{о сем}, Вааса перестал строить Раму и возвратился в Фирцу.
\vs 1Ki 15:22 Царь же Аса созвал всех Иудеев, никого не исключая, и вынесли они из Рамы камни и дерева, которые Вааса употреблял для строения. И выстроил из них царь Аса Гиву Вениаминову и Мицпу.
\rsbpar\vs 1Ki 15:23 Все прочие дела Асы и все подвиги его, и всё, что он сделал, и города, которые он построил, описаны в летописи царей Иудейских, кроме того, что в старости своей он был болен ногами.
\vs 1Ki 15:24 И почил Аса с отцами своими и погребен с отцами своими в городе Давида, отца своего. И воцарился Иосафат, сын его, вместо него.
\rsbpar\vs 1Ki 15:25 Нават же, сын Иеровоамов, воцарился над Израилем во второй год Асы, царя Иудейского, и царствовал над Израилем два года.
\vs 1Ki 15:26 И делал он неугодное пред очами Господа, ходил путем отца своего и во грехах его, которыми тот ввел Израиля в грех.
\vs 1Ki 15:27 И сделал против него заговор Вааса, сын Ахии, из дома Иссахарова, и убил его Вааса при Гавафоне Филистимском, когда Нават и все Израильтяне осаждали Гавафон;
\vs 1Ki 15:28 и умертвил его Вааса в третий год Асы, царя Иудейского, и воцарился вместо него.
\vs 1Ki 15:29 Когда он воцарился, то избил весь дом Иеровоамов, не оставил ни души у Иеровоама, доколе не истребил его, по слову Господа, которое Он изрек чрез раба Своего Ахию Силомлянина,
\vs 1Ki 15:30 за грехи Иеровоама, которые он сам делал и которыми ввел в грех Израиля, за оскорбление, которым он прогневал Господа Бога Израилева.
\rsbpar\vs 1Ki 15:31 Прочие дела Навата, всё, что он сделал, описано в летописи царей Израильских.
\vs 1Ki 15:32 И война была между Асою и Ваасою, царем Израильским, во все дни их.
\rsbpar\vs 1Ki 15:33 В третий год Асы, царя Иудейского, воцарился Вааса, сын Ахии, над всеми Израильтянами в Фирце \bibemph{и царствовал} двадцать четыре года.
\vs 1Ki 15:34 И делал неугодное пред очами Господними и ходил путем Иеровоама и во грехах его, которыми тот ввел в грех Израиля.
\vs 1Ki 16:1 И было слово Господне к Иую, сыну Ананиеву, о Ваасе:
\vs 1Ki 16:2 за то, что Я поднял тебя из праха и сделал тебя вождем народа Моего Израиля, ты же пошел путем Иеровоама и ввел в грех народ Мой Израильтян, чтобы он прогневлял Меня грехами своими,
\vs 1Ki 16:3 вот, Я отвергну дом Ваасы и дом потомства его и сделаю с домом твоим то же, что с домом Иеровоама, сына Наватова;
\vs 1Ki 16:4 кто умрет у Ваасы в городе, того съедят псы; а кто умрет у него на поле, того склюют птицы небесные.
\rsbpar\vs 1Ki 16:5 Прочие дела Ваасы, всё, чт\acc{о} он сделал, и подвиги его описаны в летописи царей Израильских.
\vs 1Ki 16:6 И почил Вааса с отцами своими, и погребен в Фирце. И воцарился Ила, сын его, вместо него.
\vs 1Ki 16:7 Но чрез Иуя, сына Ананиева, уже было \bibemph{сказано} слово Господне о Ваасе и о доме его и о всем зле, какое он делал пред очами Господа, раздражая Его делами рук своих, подражая дому Иеровоамову, за чт\acc{о} он истреблен был.
\rsbpar\vs 1Ki 16:8 В двадцать шестой год Асы, царя Иудейского, воцарился Ила, сын Ваасы, над Израилем в Фирце, \bibemph{и царствовал} два года.
\vs 1Ki 16:9 И составил против него заговор раб его Замврий, начальствовавший над половиною колесниц. Когда он в Фирце напился допьяна в доме Арсы, начальствующего над дворцом в Фирце,
\vs 1Ki 16:10 тогда вошел Замврий, поразил его и умертвил его, в двадцать седьмой год Асы, царя Иудейского, и воцарился вместо него.
\vs 1Ki 16:11 Когда он воцарился и сел на престоле его, то истребил весь дом Ваасы, не оставив ему мочащегося к стене, ни родственников его, ни друзей его.
\vs 1Ki 16:12 И истребил Замврий весь дом Ваасы, по слову Господа, которое Он изрек о Ваасе чрез Иуя пророка,
\vs 1Ki 16:13 за все грехи Ваасы и за грехи Илы, сына его, которые они сами делали и которыми вводили Израиля в грех, раздражая Господа Бога Израилева своими идолами.
\rsbpar\vs 1Ki 16:14 Прочие дела Илы, все, что он сделал, описано в летописи царей Израильских.
\rsbpar\vs 1Ki 16:15 В двадцать седьмой год Асы, царя Иудейского, воцарился Замврий и царствовал семь дней в Фирце, когда народ осаждал Гавафон Филистимский.
\vs 1Ki 16:16 Когда народ осаждавший услышал, что Замврий сделал заговор и умертвил царя, то все Израильтяне воцарили Амврия, военачальника, над Израилем в тот же день, в стане.
\vs 1Ki 16:17 И отступили Амврий и все Израильтяне с ним от Гавафона и осадили Фирцу.
\vs 1Ki 16:18 Когда увидел Замврий, что город взят, вошел во внутреннюю комнату царского дома и зажег за собою царский дом огнем и погиб
\vs 1Ki 16:19 за свои грехи, в чем он согрешил, делая неугодное пред очами Господними, ходя путем Иеровоама и во грехах его, которые тот сделал, чтобы ввести Израиля в грех.
\rsbpar\vs 1Ki 16:20 Прочие дела Замврия и заговор его, который он составил, описаны в летописи царей Израильских.
\rsbpar\vs 1Ki 16:21 Тогда разделился народ Израильский надвое: половина народа стояла за Фамния, сына Гонафова, чтобы воцарить его, а половина за Амврия.
\vs 1Ki 16:22 И одержал верх народ, который за Амврия, над народом, который за Фамния, сына Гонафова, и умер Фамний, и воцарился Амврий.
\rsbpar\vs 1Ki 16:23 В тридцать первый год Асы, царя Иудейского, воцарился Амврий над Израилем \bibemph{и царствовал} двенадцать лет. В Фирце он царствовал шесть лет.
\vs 1Ki 16:24 И купил Амврий гору Семерон у Семира за два таланта серебра, и застроил гору, и назвал построенный им город Самариею, по имени Семира, владельца горы.
\vs 1Ki 16:25 И делал Амврий неугодное пред очами Господа и поступал хуже всех бывших перед ним.
\vs 1Ki 16:26 Он во всем ходил путем Иеровоама, сына Наватова, и во грехах его, которыми тот ввел в грех Израильтян, чтобы прогневлять Господа Бога Израилева идолами своими.
\rsbpar\vs 1Ki 16:27 Прочие дела Амврия, которые он сделал, и мужество, которое он показал, описаны в летописи царей Израильских.
\vs 1Ki 16:28 И почил Амврий с отцами своими и погребен в Самарии. И воцарился Ахав, сын его, вместо него.
\rsbpar\vs 1Ki 16:29 Ахав, сын Амвриев, воцарился над Израилем в тридцать восьмой год Асы, царя Иудейского, и царствовал Ахав, сын Амврия, над Израилем в Самарии двадцать два года.
\vs 1Ki 16:30 И делал Ахав, сын Амврия, неугодное пред очами Господа более всех бывших прежде него.
\vs 1Ki 16:31 Мало было для него впадать в грехи Иеровоама, сына Наватова; он взял себе в жену Иезавель, дочь Ефваала царя Сидонского, и стал служить Ваалу и поклоняться ему.
\vs 1Ki 16:32 И поставил он Ваалу жертвенник в капище Ваала, который построил в Самарии.
\vs 1Ki 16:33 И сделал Ахав дубраву, и более всех царей Израильских, которые были прежде него, Ахав делал то, что раздражает Господа Бога Израилева, [и погубил душу свою].
\vs 1Ki 16:34 В его дни Ахиил Вефилянин построил Иерихон: на первенце своем Авираме он положил основание его и на младшем своем \bibemph{сыне} Сегубе поставил ворота его, по слову Господа, которое Он изрек чрез Иисуса, сына Навина.
\vs 1Ki 17:1 И сказал Илия [пророк], Фесвитянин, из жителей Галаадских, Ахаву: жив Господь Бог Израилев, пред Которым я стою! в сии годы не будет ни росы, ни дождя, разве только по моему слову.
\vs 1Ki 17:2 И было к нему слово Господне:
\vs 1Ki 17:3 пойди отсюда и обратись на восток и скройся у потока Хорафа, что против Иордана;
\vs 1Ki 17:4 из этого потока ты будешь пить, а в\acc{о}ронам Я повелел кормить тебя там.
\vs 1Ki 17:5 И пошел он и сделал по слову Господню; пошел и остался у потока Хорафа, что против Иордана.
\vs 1Ki 17:6 И в\acc{о}роны приносили ему хлеб и мясо поутру, и хлеб и мясо по вечеру, а из потока он пил.
\vs 1Ki 17:7 По прошествии некоторого времени этот поток высох, ибо не было дождя на землю.
\vs 1Ki 17:8 И было к нему слово Господне:
\vs 1Ki 17:9 встань и пойди в Сарепту Сидонскую, и оставайся там; Я повелел там женщине вдове кормить тебя.
\vs 1Ki 17:10 И встал он и пошел в Сарепту; и когда пришел к воротам города, вот, там женщина вдова собирает дрова. И подозвал он ее и сказал: дай мне немного воды в сосуде напиться.
\vs 1Ki 17:11 И пошла она, чтобы взять; а он закричал вслед ей и сказал: возьми для меня и кусок хлеба в руки свои.
\vs 1Ki 17:12 Она сказала: жив Господь Бог твой! у меня ничего нет печеного, а только есть горсть муки в кадке и немного масла в кувшине; и вот, я наберу полена два дров, и пойду, и приготовлю это для себя и для сына моего; съедим это и умрем.
\vs 1Ki 17:13 И сказал ей Илия: не бойся, пойди, сделай, чт\acc{о} ты сказала; но прежде из этого сделай небольшой опреснок для меня и принеси мне; а для себя и для своего сына сделаешь после;
\vs 1Ki 17:14 ибо так говорит Господь Бог Израилев: мука в кадке не истощится, и масло в кувшине не убудет до того дня, когда Господь даст дождь на землю.
\vs 1Ki 17:15 И пошла она и сделала так, как сказал Илия; и кормилась она, и он, и дом ее несколько времени.
\vs 1Ki 17:16 Мука в кадке не истощалась, и масло в кувшине не убывало, по слову Господа, которое Он изрек чрез Илию.
\vs 1Ki 17:17 После этого заболел сын этой женщины, хозяйки дома, и болезнь его была так сильна, что не осталось в нем дыхания.
\vs 1Ki 17:18 И сказала она Илии: что мне и тебе, человек Божий? ты пришел ко мне напомнить грехи мои и умертвить сына моего.
\vs 1Ki 17:19 И сказал он ей: дай мне сына твоего. И взял его с рук ее, и понес его в горницу, где он жил, и положил его на свою постель,
\vs 1Ki 17:20 и воззвал к Господу и сказал: Господи Боже мой! неужели Ты и вдове, у которой я пребываю, сделаешь зло, умертвив сына ее?
\vs 1Ki 17:21 И простершись над отроком трижды, он воззвал к Господу и сказал: Господи Боже мой! да возвратится душа отрока сего в него!
\vs 1Ki 17:22 И услышал Господь голос Илии, и возвратилась душа отрока сего в него, и он ожил.
\vs 1Ki 17:23 И взял Илия отрока, и свел его из горницы в дом, и отдал его матери его, и сказал Илия: смотри, сын твой жив.
\vs 1Ki 17:24 И сказала та женщина Илии: теперь-то я узнала, что ты человек Божий, и что слово Господне в устах твоих истинно.
\vs 1Ki 18:1 По прошествии многих дней было слово Господне к Илии в третий год: пойди и покажись Ахаву, и Я дам дождь на землю.
\vs 1Ki 18:2 И пошел Илия, чтобы показаться Ахаву. Голод же сильный был в Самарии.
\vs 1Ki 18:3 И призвал Ахав Авдия, начальствовавшего над дворцом. Авдий же был человек весьма богобоязненный,
\vs 1Ki 18:4 и когда Иезавель истребляла пророков Господних, Авдий взял сто пророков, и скрывал их, по пятидесяти человек, в пещерах, и питал их хлебом и водою.
\vs 1Ki 18:5 И сказал Ахав Авдию: пойди по земле ко всем источникам водным и ко всем потокам на земле, не найдем ли где травы, чтобы нам прокормить коней и лошаков и не лишиться скота.
\vs 1Ki 18:6 И разделили они между собою землю, чтобы обойти ее: Ахав особо пошел одною дорогою, и Авдий особо пошел другою дорогою.
\rsbpar\vs 1Ki 18:7 Когда Авдий шел дорогою, вот, навстречу ему идет Илия. Он узнал его и пал на лице свое и сказал: ты ли это, господин мой Илия?
\vs 1Ki 18:8 Тот сказал ему: я; пойди, скажи господину твоему: <<Илия здесь>>.
\vs 1Ki 18:9 Он сказал: чем я провинился, что ты предаешь раба твоего в руки Ахава, чтоб умертвить меня?
\vs 1Ki 18:10 Жив Господь Бог твой! нет ни одного народа и царства, куда бы не посылал государь мой искать тебя; и когда ему говорили, \bibemph{что тебя} нет, он брал клятву с того царства и народа, что не могли отыскать тебя;
\vs 1Ki 18:11 а ты теперь говоришь: <<пойди, скажи господину твоему: Илия здесь>>.
\vs 1Ki 18:12 Когда я пойду от тебя, тогда Дух Господень унесет тебя, не знаю, куда; и если я пойду уведомить Ахава, и он не найдет тебя, то он убьет меня; а раб твой богобоязнен от юности своей.
\vs 1Ki 18:13 Разве не сказано господину моему, чт\acc{о} я сделал, когда Иезавель убивала пророков Господних, как я скрывал сто человек пророков Господних, по пятидесяти человек, в пещерах и питал их хлебом и водою?
\vs 1Ki 18:14 А ты теперь говоришь: <<пойди, скажи господину твоему: Илия здесь>>; он убьет меня.
\vs 1Ki 18:15 И сказал Илия: жив Господь Саваоф, пред Которым я стою! сегодня я покажусь ему.
\vs 1Ki 18:16 И пошел Авдий навстречу Ахаву и донес ему. И пошел Ахав навстречу Илии.
\vs 1Ki 18:17 Когда Ахав увидел Илию, то сказал Ахав ему: ты ли это, смущающий Израиля?
\vs 1Ki 18:18 И сказал Илия: не я смущаю Израиля, а ты и дом отца твоего, тем, что вы презрели повеления Господни и идете вслед Ваалам;
\vs 1Ki 18:19 теперь пошли и собери ко мне всего Израиля на гору Кармил, и четыреста пятьдесят пророков Вааловых, и четыреста пророков дубравных, питающихся от стола Иезавели.
\vs 1Ki 18:20 И послал Ахав ко всем сынам Израилевым и собрал всех пророков на гору Кармил.
\vs 1Ki 18:21 И подошел Илия ко всему народу и сказал: долго ли вам хромать на оба колена? если Господь есть Бог, то последуйте Ему; а если Ваал, то ему последуйте. И не отвечал народ ему ни слова.
\vs 1Ki 18:22 И сказал Илия народу: я один остался пророк Господень, а пророков Вааловых четыреста пятьдесят человек [и четыреста пророков дубравных];
\vs 1Ki 18:23 пусть дадут нам двух тельцов, и пусть они выберут себе одного тельца, и рассекут его, и положат на дрова, но огня пусть не подкладывают; а я приготовлю другого тельца и положу на дрова, а огня не подложу;
\vs 1Ki 18:24 и призовите вы имя бога вашего, а я призову имя Господа Бога моего. Тот Бог, Который даст ответ посредством огня, есть Бог. И отвечал весь народ и сказал: хорошо, [пусть будет так].
\vs 1Ki 18:25 И сказал Илия пророкам Вааловым: выберите себе одного тельца и приготовьте вы прежде, ибо вас много; и призовите имя бога вашего, но огня не подкладывайте.
\vs 1Ki 18:26 И взяли они тельца, который дан был им, и приготовили, и призывали имя Ваала от утра до полудня, говоря: Ваале, услышь нас! Но не было ни голоса, ни ответа. И скакали они у жертвенника, который сделали.
\vs 1Ki 18:27 В полдень Илия стал смеяться над ними и говорил: кричите громким голосом, ибо он бог; может быть, он задумался, или занят чем-либо, или в дороге, а может быть, и спит, так он проснется!
\vs 1Ki 18:28 И стали они кричать громким голосом, и кололи себя по своему обыкновению ножами и копьями, так что кровь лилась по ним.
\vs 1Ki 18:29 Прошел полдень, а они всё еще бесновались до самого времени вечернего жертвоприношения; но не было ни голоса, ни ответа, ни слуха. [И сказал Илия Фесвитянин пророкам Вааловым: теперь отойдите, чтоб и я совершил мое жертвоприношение. Они отошли и умолкли.]
\vs 1Ki 18:30 Тогда Илия сказал всему народу: подойдите ко мне. И подошел весь народ к нему. Он восстановил разрушенный жертвенник Господень.
\vs 1Ki 18:31 И взял Илия двенадцать камней, по числу колен сынов Иакова, которому Господь сказал так: Израиль будет имя твое.
\vs 1Ki 18:32 И построил из сих камней жертвенник во имя Господа, и сделал вокруг жертвенника ров, вместимостью в две саты зерен,
\vs 1Ki 18:33 и положил дрова [на жертвенник], и рассек тельца, и возложил его на дрова,
\vs 1Ki 18:34 и сказал: наполните четыре ведра воды и выливайте на всесожигаемую жертву и на дрова. [И сделали так.] Потом сказал: повторите. И они повторили. И сказал: сделайте \bibemph{то же} в третий раз. И сделали в третий раз,
\vs 1Ki 18:35 и вода полилась вокруг жертвенника, и ров наполнился водою.
\vs 1Ki 18:36 Во время приношения вечерней жертвы подошел Илия пророк [и воззвал на небо] и сказал: Господи, Боже Авраамов, Исааков и Израилев! [Услышь меня, Господи, услышь меня ныне в огне!] Да познают в сей день [люди сии], что Ты один Бог в Израиле, и что я раб Твой и сделал всё по слову Твоему.
\vs 1Ki 18:37 Услышь меня, Господи, услышь меня! Да познает народ сей, что Ты, Господи, Бог, и Ты обратишь сердце их [к Тебе].
\vs 1Ki 18:38 И ниспал огонь Господень и пожрал всесожжение, и дрова, и камни, и прах, и поглотил воду, которая во рве.
\vs 1Ki 18:39 Увидев \bibemph{это}, весь народ пал на лице свое и сказал: Господь есть Бог, Господь есть Бог!
\vs 1Ki 18:40 И сказал им Илия: схватите пророков Вааловых, чтобы ни один из них не укрылся. И схватили их, и отвел их Илия к потоку Киссону и заколол их там.
\vs 1Ki 18:41 И сказал Илия Ахаву: пойди, ешь и пей, ибо слышен шум дождя.
\vs 1Ki 18:42 И пошел Ахав есть и пить, а Илия взошел на верх Кармила и наклонился к земле, и положил лице свое между коленами своими,
\vs 1Ki 18:43 и сказал отроку своему: пойди, посмотри к морю. Тот пошел и посмотрел, и сказал: ничего нет. Он сказал: продолжай \bibemph{это} до семи раз.
\vs 1Ki 18:44 В седьмой раз тот сказал: вот, небольшое облако поднимается от моря, величиною в ладонь человеческую. Он сказал: пойди, скажи Ахаву: <<запрягай [колесницу твою] и поезжай, чтобы не застал тебя дождь>>.
\vs 1Ki 18:45 Между тем небо сделалось мрачно от туч и от ветра, и пошел большой дождь. Ахав же сел в колесницу, [заплакал] и поехал в Изреель.
\vs 1Ki 18:46 И была на Илии рука Господня. Он опоясал чресла свои и бежал пред Ахавом до самого Изрееля.
\vs 1Ki 19:1 И пересказал Ахав Иезавели всё, что сделал Илия, и то, что он убил всех пророков мечом.
\vs 1Ki 19:2 И послала Иезавель посланца к Илии сказать: [если ты Илия, а я Иезавель, то] пусть то и то сделают мне боги, и еще больше сделают, если я завтра к этому времени не сделаю с твоею душею того, что \bibemph{сделано} с душею каждого из них.
\vs 1Ki 19:3 Увидев это, он встал и пошел, чтобы спасти жизнь свою, и пришел в Вирсавию, которая в Иудее, и оставил отрока своего там.
\vs 1Ki 19:4 А сам отошел в пустыню на день пути и, придя, сел под можжевеловым кустом, и просил смерти себе и сказал: довольно уже, Господи; возьми душу мою, ибо я не лучше отцов моих.
\vs 1Ki 19:5 И лег и заснул под можжевеловым кустом. И вот, Ангел коснулся его и сказал ему: встань, ешь [и пей].
\vs 1Ki 19:6 И взглянул Илия, и вот, у изголовья его печеная лепешка и кувшин воды. Он поел и напился и опять заснул.
\vs 1Ki 19:7 И возвратился Ангел Господень во второй раз, коснулся его и сказал: встань, ешь [и пей], ибо дальняя дорога пред тобою.
\vs 1Ki 19:8 И встал он, поел и напился, и, подкрепившись тою пищею, шел сорок дней и сорок ночей до горы Божией Хорива.
\vs 1Ki 19:9 И вошел он там в пещеру и ночевал в ней. И вот, было к нему слово Господне, и сказал ему \bibemph{Господь}: что ты здесь, Илия?
\vs 1Ki 19:10 Он сказал: возревновал я о Господе Боге Саваофе, ибо сыны Израилевы оставили завет Твой, разрушили Твои жертвенники и пророков Твоих убили мечом; остался я один, но и моей души ищут, чтобы отнять ее.
\vs 1Ki 19:11 И сказал: выйди и стань на горе пред лицем Господним, и вот, Господь пройдет, и большой и сильный ветер, раздирающий горы и сокрушающий скалы пред Господом, но не в ветре Господь; после ветра землетрясение, но не в землетрясении Господь;
\vs 1Ki 19:12 после землетрясения огонь, но не в огне Господь; после огня веяние тихого ветра, [и там Господь].
\vs 1Ki 19:13 Услышав \bibemph{сие}, Илия закрыл лице свое милотью своею, и вышел, и стал у входа в пещеру. И был к нему голос и сказал ему: что ты здесь, Илия?
\vs 1Ki 19:14 Он сказал: возревновал я о Господе Боге Саваофе, ибо сыны Израилевы оставили завет Твой, разрушили жертвенники Твои и пророков Твоих убили мечом; остался я один, но и моей души ищут, чтоб отнять ее.
\vs 1Ki 19:15 И сказал ему Господь: пойди обратно своею дорогою чрез пустыню в Дамаск, и когда придешь, то помажь Азаила в царя над Сириею,
\vs 1Ki 19:16 а Ииуя, сына Намессиина, помажь в царя над Израилем; Елисея же, сына Сафатова, из Авел-Мехолы, помажь в пророка вместо себя;
\vs 1Ki 19:17 кто убежит от меча Азаилова, того умертвит Ииуй; а кто спасется от меча Ииуева, того умертвит Елисей.
\vs 1Ki 19:18 Впрочем, Я оставил между Израильтянами семь тысяч [мужей]; всех сих колени не преклонялись пред Ваалом, и всех сих уста не лобызали его.
\vs 1Ki 19:19 И пошел он оттуда, и нашел Елисея, сына Сафатова, когда он орал; двенадцать пар [волов] было у него, и сам он был при двенадцатой. Илия, проходя мимо него, бросил на него милоть свою.
\vs 1Ki 19:20 И оставил [Елисей] волов, и побежал за Илиею, и сказал: позволь мне поцеловать отца моего и мать мою, и я пойду за тобою. Он сказал ему: пойди и приходи назад, ибо что сделал я тебе?
\vs 1Ki 19:21 Он, отойдя от него, взял пару волов и заколол их и, зажегши плуг волов, изжарил мясо их, и раздал людям, и они ели. А сам встал и пошел за Илиею, и стал служить ему.
\vs 1Ki 20:1 Венадад, царь Сирийский, собрал все свое войско, и с ним были тридцать два царя, и кони и колесницы, и пошел, осадил Самарию и воевал против нее.
\vs 1Ki 20:2 И послал послов к Ахаву, царю Израильскому, в город,
\vs 1Ki 20:3 и сказал ему: так говорит Венадад: серебро твое и золото твое~--- мои, и жены твои и лучшие сыновья твои~--- мои.
\vs 1Ki 20:4 И отвечал царь Израильский и сказал: да будет по слову твоему, господин мой царь: я и все мое~--- твое.
\vs 1Ki 20:5 И опять пришли послы и сказали: так говорит Венадад: я послал к тебе сказать: <<серебро твое, и золото твое, и жён твоих, и сыновей твоих отдай мне>>;
\vs 1Ki 20:6 поэтому я завтра, к этому времени, пришлю к тебе рабов моих, чтобы они осмотрели твой дом и домы служащих при тебе, и все дорогое для глаз твоих взяли в свои руки и унесли.
\vs 1Ki 20:7 И созвал царь Израильский всех старейшин земли и сказал: замечайте и смотрите, он замышляет зло; когда он присылал ко мне за жёнами моими, и сыновьями моими, и серебром моим, и золотом моим, я ему не отказал.
\vs 1Ki 20:8 И сказали ему все старейшины и весь народ: не слушай и не соглашайся.
\vs 1Ki 20:9 И сказал он послам Венадада: скажите господину моему царю: все, о чем ты присылал в первый раз к рабу твоему, я готов сделать, а этого не могу сделать. И пошли послы и отнесли ему ответ.
\vs 1Ki 20:10 И прислал к нему Венадад сказать: пусть то и то сделают мне боги, и еще больше сделают, если праха Самарийского достанет по горсти для всех людей, идущих за мною.
\vs 1Ki 20:11 И отвечал царь Израильский и сказал: скажите: пусть не хвалится подпоясывающийся, как распоясывающийся.
\vs 1Ki 20:12 Услышав это слово, Венадад, который пил вместе с царями в палатках, сказал рабам своим: осаждайте город. И они осадили город.
\rsbpar\vs 1Ki 20:13 И вот, один пророк подошел к Ахаву, царю Израильскому, и сказал: так говорит Господь: видишь ли все это большое полчище? вот, Я сегодня предам его в руку твою, чтобы ты знал, что Я Господь.
\vs 1Ki 20:14 И сказал Ахав: чрез кого? Он сказал: так говорит Господь: чрез слуг областных начальников. И сказал [Ахав]: кто начнет сражение? Он сказал: ты.
\vs 1Ki 20:15 [Ахав] счел слуг областных начальников, и нашлось их двести тридцать два; после них счел весь народ, всех сынов Израилевых, семь тысяч.
\vs 1Ki 20:16 И они выступили в полдень. Венадад же напился допьяна в палатках вместе с царями, с тридцатью двумя царями, помогавшими ему.
\vs 1Ki 20:17 И выступили прежде слуги областных начальников. И послал Венадад, и донесли ему, что люди вышли из Самарии.
\vs 1Ki 20:18 Он сказал: если за миром вышли они, то схватите их живыми, и если на войну вышли, также схватите их живыми.
\vs 1Ki 20:19 Вышли из города слуги областных начальников, и войско за ними.
\vs 1Ki 20:20 И поражал каждый противника своего; и побежали Сирияне, а Израильтяне погнались за ними. Венадад же, царь Сирийский, спасся на коне с всадниками.
\vs 1Ki 20:21 И вышел царь Израильский, и взял коней и колесниц, и произвел большое поражение у Сириян.
\rsbpar\vs 1Ki 20:22 И подошел пророк к царю Израильскому и сказал ему: пойди, укрепись, и знай и смотри, что тебе делать, ибо по прошествии года царь Сирийский опять пойдет против тебя.
\vs 1Ki 20:23 Слуги царя Сирийского сказали ему: Бог их есть Бог гор, [а не Бог долин,] поэтому они одолели нас; если же мы сразимся с ними на равнине, то верно одолеем их.
\vs 1Ki 20:24 Итак вот что сделай: удали царей, каждого с места его, и вместо них поставь областеначальников;
\vs 1Ki 20:25 и набери себе войска столько, сколько пало у тебя, и коней, сколько было коней, и колесниц, сколько было колесниц; и сразимся с ними на равнине, и тогда верно одолеем их. И послушался он голоса их и сделал так.
\rsbpar\vs 1Ki 20:26 По прошествии года Венадад собрал Сириян и выступил к Афеку, чтобы сразиться с Израилем.
\vs 1Ki 20:27 Собраны были и сыны Израилевы и, взяв продовольствие, пошли навстречу им. И расположились сыны Израилевы станом пред ними, как бы два небольшие стада коз, а Сирияне наполнили землю.
\vs 1Ki 20:28 И подошел человек Божий, и сказал царю Израильскому: так говорит Господь: за то, что Сирияне говорят: <<Господь есть Бог гор, а не Бог долин>>, Я все это большое полчище предам в руку твою, чтобы вы знали, что Я~--- Господь.
\vs 1Ki 20:29 И стояли станом одни против других семь дней. В седьмой день началась битва, и сыны Израилевы поразили сто тысяч пеших Сириян в один день.
\vs 1Ki 20:30 Остальные убежали в город Афек; \bibemph{там} упала стена на остальных двадцать семь тысяч человек. А Венадад ушел в город и бегал из одной внутренней комнаты в другую.
\vs 1Ki 20:31 И сказали ему слуги его: мы слышали, что цари дома Израилева цари милостивые; позволь нам возложить вретища на чресла свои и веревки на головы свои и пойти к царю Израильскому; может быть, он пощадит жизнь твою.
\vs 1Ki 20:32 И опоясали они вретищами чресла свои и возложили веревки на головы свои, и пришли к царю Израильскому и сказали: раб твой Венадад говорит: <<пощади жизнь мою>>. Тот сказал: разве он жив? он брат мой.
\vs 1Ki 20:33 Люди сии приняли это за \bibemph{хороший} знак и поспешно подхватили слово из уст его и сказали: брат твой Венадад. И сказал он: пойдите, приведите его. И вышел к нему Венадад, и он посадил его \bibemph{с собою} на колесницу.
\vs 1Ki 20:34 И сказал ему \bibemph{Венадад}: города, которые взял мой отец у твоего отца, я возвращу, и площади ты можешь иметь для себя в Дамаске, как отец мой имел в Самарии. \bibemph{Ахав сказал}: после договора я отпущу тебя. И, заключив с ним договор, отпустил его.
\rsbpar\vs 1Ki 20:35 Тогда один человек из сынов пророческих сказал другому, по слову Господа: бей меня. Но этот человек не согласился бить его.
\vs 1Ki 20:36 И сказал ему: за то, что ты не слушаешь гласа Господня, убьет тебя лев, когда пойдешь от меня. Он пошел от него, и лев, встретив его, убил его.
\vs 1Ki 20:37 И нашел он другого человека, и сказал: бей меня. Этот человек бил его до того, что изранил побоями.
\vs 1Ki 20:38 И пошел пророк и предстал пред царя на дороге, прикрыв покрывалом глаза свои.
\vs 1Ki 20:39 Когда царь проезжал мимо, он закричал царю и сказал: раб твой ходил на сражение, и вот, один человек, отошедший в сторону, подвел ко мне человека и сказал: <<стереги этого человека; если его не станет, то твоя душа будет за его душу, или ты должен будешь отвесить талант серебра>>.
\vs 1Ki 20:40 Когда раб твой занялся теми и другими делами, его не стало.~--- И сказал ему царь Израильский: таков тебе и приговор, ты сам решил.
\vs 1Ki 20:41 Он тотчас снял покрывало с глаз своих, и узнал его царь, что он из пророков.
\vs 1Ki 20:42 И сказал ему: так говорит Господь: за то, что ты выпустил из рук твоих человека, заклятого Мною, душа твоя будет вместо его души, народ твой вместо его народа.
\vs 1Ki 20:43 И отправился царь Израильский домой встревоженный и огорченный, и прибыл в Самарию.
\vs 1Ki 21:1 И было после сих происшествий: у Навуфея Изреелитянина в Изреели был виноградник подле дворца Ахава, царя Самарийского.
\vs 1Ki 21:2 И сказал Ахав Навуфею, говоря: отдай мне свой виноградник; из него будет у меня овощной сад, ибо он близко к моему дому; а вместо него я дам тебе виноградник лучше этого, или, если угодно тебе, дам тебе серебра, сколько он стоит.
\vs 1Ki 21:3 Но Навуфей сказал Ахаву: сохрани меня Господь, чтоб я отдал тебе наследство отцов моих!
\vs 1Ki 21:4 И пришел Ахав домой встревоженный и огорченный тем словом, которое сказал ему Навуфей Изреелитянин, говоря: не отдам тебе наследства отцов моих. И [в смущенном духе] лег на постель свою, и отворотил лице свое, и хлеба не ел.
\vs 1Ki 21:5 И вошла к нему жена его Иезавель и сказала ему: отчего встревожен дух твой, что ты и хлеба не ешь?
\vs 1Ki 21:6 Он сказал ей: когда я стал говорить Навуфею Изреелитянину и сказал ему: <<отдай мне виноградник твой за серебро, или, если хочешь, я дам тебе \bibemph{другой} виноградник вместо него>>, тогда он сказал: <<не отдам тебе виноградника моего, [наследства отцов моих]>>.
\vs 1Ki 21:7 И сказала ему Иезавель, жена его: что за царство было бы в Израиле, если бы ты так поступал? встань, ешь хлеб и будь спокоен; я доставлю тебе виноградник Навуфея Изреелитянина.
\vs 1Ki 21:8 И написала она от имени Ахава письма, и запечатала их его печатью, и послала эти письма к старейшинам и знатным в его городе, живущим с Навуфеем.
\vs 1Ki 21:9 В письмах она писала так: объявите пост и посадите Навуфея на первое место в народе;
\vs 1Ki 21:10 и против него посадите двух негодных людей, которые свидетельствовали бы на него и сказали: <<ты хулил Бога и царя>>; и потом выведите его, и побейте его камнями, чтоб он умер.
\vs 1Ki 21:11 И сделали мужи города его, старейшины и знатные, жившие в городе его, как приказала им Иезавель, так, как писано в письмах, которые она послала к ним.
\vs 1Ki 21:12 Объявили пост и посадили Навуфея во главе народа;
\vs 1Ki 21:13 и выступили два негодных человека и сели против него, и свидетельствовали на него эти недобрые люди пред народом, и говорили: Навуфей хулил Бога и царя. И вывели его за город, и побили его камнями, и он умер.
\vs 1Ki 21:14 И послали к Иезавели сказать: Навуфей побит камнями и умер.
\vs 1Ki 21:15 Услышав, что Навуфей побит камнями и умер, Иезавель сказала Ахаву: встань, возьми во владение виноградник Навуфея Изреелитянина, который не хотел отдать тебе за серебро; ибо Навуфея нет в живых, он умер.
\vs 1Ki 21:16 Когда услышал Ахав, что Навуфей [Изреелитянин] был убит, [разодрал одежды свои и надел на себя вретище, а потом] встал Ахав, чтобы пойти в виноградник Навуфея Изреелитянина и взять его во владение.
\rsbpar\vs 1Ki 21:17 И было слово Господне к Илии Фесвитянину:
\vs 1Ki 21:18 встань, пойди навстречу Ахаву, царю Израильскому, который в Самарии, вот, он теперь в винограднике Навуфея, куда пришел, чтобы взять \bibemph{его} во владение;
\vs 1Ki 21:19 и скажи ему: <<так говорит Господь: ты убил, и еще вступаешь в наследство?>> и скажи ему: <<так говорит Господь: на том месте, где псы лизали кровь Навуфея, псы будут лизать и твою кровь>>.
\vs 1Ki 21:20 И сказал Ахав Илии: нашел ты меня, враг мой! Он сказал: нашел, ибо ты предался тому, чтобы делать неугодное пред очами Господа [и раздражать Его].
\vs 1Ki 21:21 [Так говорит Господь:] вот, Я наведу на тебя беды и вымету за тобою и истреблю у Ахава мочащегося к стене и заключенного и оставшегося в Израиле.
\vs 1Ki 21:22 И поступлю с домом твоим так, как поступил Я с домом Иеровоама, сына Наватова, и с домом Ваасы, сына Ахиина, за оскорбление, которым ты раздражил \bibemph{Меня} и ввел Израиля в грех.
\vs 1Ki 21:23 Также и о Иезавели сказал Господь: псы съедят Иезавель за стеною Изрееля.
\vs 1Ki 21:24 Кто умрет у Ахава в городе, того съедят псы, а кто умрет на поле, того расклюют птицы небесные;
\vs 1Ki 21:25 не было еще такого, как Ахав, который предался бы тому, чтобы делать неугодное пред очами Господа, к чему подущала его жена его Иезавель;
\vs 1Ki 21:26 он поступал весьма гнусно, последуя идолам, как делали Аморреи, которых Господь прогнал от лица сынов Израилевых.
\vs 1Ki 21:27 Выслушав все слова сии, Ахав [умилился пред Господом, ходил и плакал,] разодрал одежды свои, и возложил на тело свое вретище, и постился, и спал во вретище, и ходил печально.
\vs 1Ki 21:28 И было слово Господне к Илии Фесвитянину [об Ахаве], и сказал Господь:
\vs 1Ki 21:29 видишь, как смирился предо Мною Ахав? За то, что он смирился предо Мною, Я не наведу бед в его дни; во дни сына его наведу беды на дом его.
\vs 1Ki 22:1 Прожили три года, и не было войны между Сириею и Израилем.
\vs 1Ki 22:2 На третий год Иосафат, царь Иудейский, пошел к царю Израильскому.
\vs 1Ki 22:3 И сказал царь Израильский слугам своим: знаете ли, что Рамоф Галаадский наш? А мы так долго молчим, и не берем его из руки царя Сирийского.
\vs 1Ki 22:4 И сказал он Иосафату: пойдешь ли ты со мною на войну против Рамофа Галаадского? И сказал Иосафат царю Израильскому: как ты, так и я; как твой народ, так и мой народ; как твои кони, так и мои кони.
\vs 1Ki 22:5 И сказал Иосафат царю Израильскому: спроси сегодня, что скажет Господь.
\vs 1Ki 22:6 И собрал царь Израильский пророков, около четырехсот человек и сказал им: идти ли мне войною на Рамоф Галаадский, или нет? Они сказали: иди, Господь предаст \bibemph{его} в руки царя.
\vs 1Ki 22:7 И сказал Иосафат: нет ли здесь еще пророка Господня, чтобы нам вопросить чрез него Господа?
\vs 1Ki 22:8 И сказал царь Израильский Иосафату: есть еще один человек, чрез которого можно вопросить Господа, но я не люблю его, ибо он не пророчествует о мне доброго, а только худое,~--- это Михей, сын Иемвлая. И сказал Иосафат: не говори, царь, так.
\vs 1Ki 22:9 И позвал царь Израильский одного евнуха и сказал: сходи поскорее за Михеем, сыном Иемвлая.
\vs 1Ki 22:10 Царь Израильский и Иосафат, царь Иудейский, сидели каждый на седалище своем, одетые в \bibemph{царские} одежды, на площади у ворот Самарии, и все пророки пророчествовали пред ними.
\vs 1Ki 22:11 И сделал себе Седекия, сын Хенааны, железные рога, и сказал: так говорит Господь: сими избодешь Сириян до истребления их.
\vs 1Ki 22:12 И все пророки пророчествовали то же, говоря: иди на Рамоф Галаадский, будет успех, Господь предаст \bibemph{его} в руку царя.
\vs 1Ki 22:13 Посланный, который пошел позвать Михея, говорил ему: вот, речи пророков единогласно \bibemph{предвещают} царю доброе; пусть бы и твое слово было согласно с словом каждого из них; изреки и ты доброе.
\vs 1Ki 22:14 И сказал Михей: жив Господь! я изреку то, что скажет мне Господь.
\vs 1Ki 22:15 И пришел он к царю. Царь сказал ему: Михей! идти ли нам войною на Рамоф Галаадский, или нет? И сказал тот ему: иди, будет успех, Господь предаст \bibemph{его} в руку царя.
\vs 1Ki 22:16 И сказал ему царь: еще и еще заклинаю тебя, чтобы ты не говорил мне ничего, кроме истины во имя Господа.
\vs 1Ki 22:17 И сказал он: я вижу всех Израильтян, рассеянных по горам, как овец, у которых нет пастыря. И сказал Господь: нет у них начальника, пусть возвращаются с миром каждый в свой дом.
\vs 1Ki 22:18 И сказал царь Израильский Иосафату: не говорил ли я тебе, что он не пророчествует о мне доброго, а только худое?
\vs 1Ki 22:19 И сказал [Михей]: [не так; не я, а] выслушай слово Господне: я видел Господа, сидящего на престоле Своем, и все воинство небесное стояло при Нем, по правую и по левую руку Его;
\vs 1Ki 22:20 и сказал Господь: кто склонил бы Ахава, чтобы он пошел и пал в Рамофе Галаадском? И один говорил так, другой говорил иначе;
\vs 1Ki 22:21 и выступил один дух, стал пред лицем Господа и сказал: я склоню его. И сказал ему Господь: чем?
\vs 1Ki 22:22 Он сказал: я выйду и сделаюсь духом лживым в устах всех пророков его. \bibemph{Господь} сказал: ты склонишь его и выполнишь это; пойди и сделай так.
\vs 1Ki 22:23 И вот, теперь попустил Господь духа лживого в уста всех сих пророков твоих; но Господь изрек о тебе недоброе.
\vs 1Ki 22:24 И подошел Седекия, сын Хенааны, и, ударив Михея по щеке, сказал: как, неужели от меня отошел Дух Господень, чтобы говорить в тебе?
\vs 1Ki 22:25 И сказал Михей: вот, ты увидишь \bibemph{это} в тот день, когда будешь бегать из одной комнаты в другую, чтоб укрыться,
\vs 1Ki 22:26 и сказал царь Израильский: возьмите Михея и отведите его к Амону градоначальнику и к Иоасу, сыну царя,
\vs 1Ki 22:27 и скажите: так говорит царь: посадите этого в темницу и кормите его скудно хлебом и скудно водою, доколе я не возвращусь в мире.
\vs 1Ki 22:28 И сказал Михей: если возвратишься в мире, то не Господь говорил чрез меня. И сказал: слушай, весь народ!
\rsbpar\vs 1Ki 22:29 И пошел царь Израильский и Иосафат, царь Иудейский, к Рамофу Галаадскому.
\vs 1Ki 22:30 И сказал царь Израильский Иосафату: я переоденусь и вступлю в сражение, а ты надень твои \bibemph{царские} одежды. И переоделся царь Израильский и вступил в сражение.
\vs 1Ki 22:31 Сирийский царь повелел начальникам колесниц, которых у него было тридцать два, сказав: не сражайтесь ни с малым, ни с великим, а только с одним царем Израильским.
\vs 1Ki 22:32 Начальники колесниц, увидев Иосафата, подумали: <<верно это царь Израильский>>, и поворотили на него, чтобы сразиться \bibemph{с ним}. И закричал Иосафат.
\vs 1Ki 22:33 Начальники колесниц, видя, что это не Израильский царь, поворотили от него.
\vs 1Ki 22:34 А один человек случайно натянул лук и ранил царя Израильского сквозь швы лат. И сказал он своему вознице: повороти назад и вывези меня из войска, ибо я ранен.
\vs 1Ki 22:35 Но сражение в тот день усилилось, и царь стоял на колеснице против Сириян, и вечером умер, и кровь из раны лилась в колесницу.
\vs 1Ki 22:36 И провозглашено было по всему стану при захождении солнца: каждый иди в свой город, каждый в свою землю!
\vs 1Ki 22:37 И умер царь, и привезен был в Самарию, и похоронили царя в Самарии.
\vs 1Ki 22:38 И обмыли колесницу на пруде Самарийском, и псы лизали кровь его, и омывали блудницы, по слову Господа, которое Он изрек.
\rsbpar\vs 1Ki 22:39 Прочие дела Ахава, все, что он делал, и дом из слоновой кости, который он построил, и все города, которые он строил, описаны в летописи царей Израильских.
\vs 1Ki 22:40 И почил Ахав с отцами своими, и воцарился Охозия, сын его, вместо него.
\rsbpar\vs 1Ki 22:41 Иосафат, сын Асы, воцарился над Иудеею в четвертый год Ахава, царя Израильского.
\vs 1Ki 22:42 Тридцати пяти лет был Иосафат, когда воцарился, и двадцать пять лет царствовал в Иерусалиме. Имя матери его Азува, дочь Салаиля.
\vs 1Ki 22:43 Он ходил во всем путем отца своего Асы, не сходил с него, делая угодное пред очами Господними. Только высоты не были отменены; народ еще совершал жертвы и курения на высотах.
\vs 1Ki 22:44 Иосафат заключил мир с царем Израильским.
\rsbpar\vs 1Ki 22:45 Прочие дела Иосафата и подвиги его, какие он совершил, и как он воевал, описаны в летописи царей Иудейских.
\vs 1Ki 22:46 И остатки блудников, которые остались во дни отца его Асы, он истребил с земли.
\vs 1Ki 22:47 В Идумее тогда не было царя; \bibemph{был} наместник царский.
\vs 1Ki 22:48 [Царь] Иосафат сделал корабли на море, чтобы ходить в Офир за золотом; но они не дошли, ибо разбились в Ецион-Гавере.
\vs 1Ki 22:49 Тогда сказал Охозия, сын Ахава, Иосафату: пусть мои слуги пойдут с твоими слугами на кораблях. Но Иосафат не согласился.
\vs 1Ki 22:50 И почил Иосафат с отцами своими и был погребен с отцами своими в городе Давида, отца своего. И воцарился Иорам, сын его, вместо него.
\rsbpar\vs 1Ki 22:51 Охозия, сын Ахава, воцарился над Израилем в Самарии, в семнадцатый год Иосафата, царя Иудейского, и царствовал над Израилем [в Самарии] два года,
\vs 1Ki 22:52 и делал неугодное пред очами Господа, и ходил путем отца своего и путем матери своей и путем Иеровоама, сына Наватова, который ввел Израиля в грех:
\vs 1Ki 22:53 он служил Ваалу и поклонялся ему и прогневал Господа Бога Израилева всем тем, что делал отец его.

\include{tex/2Ki}\newbookpage
\include{tex/1Ch}
\include{tex/2Ch}\newbookpage
\include{tex/Ezr}
\include{tex/Neh}
\include{tex/2Ez}\newbookpage
\bibbookdescr{Tob}{
  inline={\LARGE Книга\\\Huge Товита\fns{Переведена с греческого.}},
  toc={Товит*},
  bookmark={Товит},
  header={Товит},
  %headerleft={},
  %headerright={},
  abbr={Тов}
}
\vs Tob 1:1 Книга сказаний Товита, сына Товиилова, Ананиилова, Адуилова, Гаваилова, из племени Асиилова, из колена Неффалимова,
\vs Tob 1:2 который во дни Ассирийского царя Енемессара взят был в плен из Фисвы, находящейся по правую \bibemph{сторону} Кидия Неффалимова, в Галилее, выше Асира. Я, Товит, во все дни жизни моей ходил путями истины и правды
\vs Tob 1:3 и делал много благодеяний братьям моим и народу моему, пришедшим вместе со мною в страну Ассирийскую, в Ниневию.
\vs Tob 1:4 Когда я жил в стране моей, в земле Израиля, будучи еще юношею, тогда все колено Неффалима, отца моего, находилось в отпадении от дома Иерусалима, избранного от всех колен Израиля, чтобы всем им приносить \bibemph{там} жертвы, где освящен храм селения Всевышнего и утвержден во все роды навек.
\vs Tob 1:5 Как все отложившиеся колена приносили жертвы Ваалу, юнице, так и дом Неффалима, отца моего.
\vs Tob 1:6 Я же один часто ходил в Иерусалим на праздники, как предписано всему Израилю установлением вечным, с начатками и десятинами произведений \bibemph{земли} и начатками шерсти овец,
\vs Tob 1:7 и отдавал это священникам, сынам Аароновым, для жертвенника: десятину всех произведений давал сынам Левииным, служащим в Иерусалиме; другую десятину продавал, и каждый год ходил и издерживал ее в Иерусалиме;
\vs Tob 1:8 а третью давал, кому следовало, как заповедала мне Деввора, мать отца моего, когда я после отца моего остался сиротою.
\vs Tob 1:9 Достигнув мужеского возраста, я взял жену Анну из отеческого нашего рода и родил от нее Товию.
\vs Tob 1:10 Когда я отведен был в плен в Ниневию, все братья мои и одноплеменники мои ели от снедей языческих,
\vs Tob 1:11 а я соблюдал душу мою и не ел,
\vs Tob 1:12 ибо я помнил Бога всею душею моею.
\vs Tob 1:13 И даровал мне Всевышний милость и благоволение у Енемессара, и я был у него поставщиком;
\vs Tob 1:14 и ходил в Мидию, и отдал \bibemph{на сохранение} Гаваилу, брату Гаврия, в Рагах Мидийских, десять талантов серебра.
\vs Tob 1:15 Когда же умер Енемессар, вместо него воцарился сын его Сеннахирим, которого пути не были постоянны, и я уже не мог ходить в Мидию.
\vs Tob 1:16 Во дни Енемессара я делал много благодеяний братьям моим:
\vs Tob 1:17 алчущим давал хлеб мой, нагим одежды мои и, если кого из племени моего видел умершим и выброшенным за стену Ниневии, погребал его.
\vs Tob 1:18 Тайно погребал я и тех, которых убивал царь Сеннахирим, когда, обращенный в бегство, возвратился из Иудеи. А он многих умертвил в ярости своей. И отыскивал царь трупы, но их не находили.
\vs Tob 1:19 Один из Ниневитян пошел и донес царю, что я погребаю их; тогда я скрылся. Узнав же, что меня ищут убить, от страха убежал \bibemph{из города}.
\vs Tob 1:20 И было расхищено все имущество мое, и не осталось у меня ничего, кроме Анны, жены моей, и Товии, сына моего.
\vs Tob 1:21 Но не прошло пятидесяти дней, как два сына его убили его и убежали в горы Араратские. И воцарился вместо него сын его Сахердан, который поставил Ахиахара Анаила, сына брата моего, над всею счетною частью царства своего и над всем домоправлением.
\vs Tob 1:22 И ходатайствовал Ахиахар за меня, и я возвратился в Ниневию. Ахиахар же был и виночерпий и хранитель перстня, и домоправитель и казначей; и Сахердан поставил его вторым по себе; он был сын брата моего.
\vs Tob 2:1 Когда я возвратился в дом свой, и отданы мне были Анна, жена моя, и Товия, сын мой, в праздник пятидесятницы, в святую седмицу седмиц, приготовлен у меня был хороший обед, и я возлег есть.
\vs Tob 2:2 Увидев много снедей, я сказал сыну моему: пойди и приведи, кого найдешь, бедного из братьев наших, который помнит Господа, а я подожду тебя.
\vs Tob 2:3 И пришел он и сказал: отец \bibemph{мой}, один из племени нашего удавленный брошен на площади.
\vs Tob 2:4 Тогда я, прежде нежели стал есть, поспешно выйдя, убрал его в одно жилье до захождения солнца.
\vs Tob 2:5 Возвратившись, совершил омовение и ел хлеб мой в скорби.
\vs Tob 2:6 И вспомнил я пророчество Амоса, как он сказал: праздники ваши обратятся в скорбь, и все увеселения ваши~--- в плач.
\vs Tob 2:7 И я плакал. Когда же зашло солнце, я пошел и, выкопав \bibemph{могилу}, похоронил его.
\vs Tob 2:8 Соседи насмехались \bibemph{надо мною} и говорили: еще не боится он быть убитым за это дело; бегал уже, и вот опять погребает мертвых.
\vs Tob 2:9 В эту самую ночь, возвратившись после погребения и будучи нечистым, я лег спать за стеною двора, и лице мое не было покрыто.
\vs Tob 2:10 И не заметил я, что на стене были воробьи. Когда глаза мои были открыты, воробьи испустили теплое на глаза мои, и сделались на глазах моих бельма. И ходил я к врачам, но они не помогли мне. Ахиахар доставлял мне пропитание, доколе не отправился в Елимаиду.
\vs Tob 2:11 А потом жена моя Анна в женских отделениях пряла шерсть
\vs Tob 2:12 и посылала богатым людям, которые давали ей плату и однажды в придачу дали козленка.
\vs Tob 2:13 Когда принесли его ко мне, он начал блеять; и я спросил \bibemph{жену}: откуда этот козленок? не краденый ли? отдай его, кому он принадлежит! ибо непозволительно есть краденое.
\vs Tob 2:14 Она отвечала: это подарили мне сверх платы. Но я не верил ей и настаивал, чтобы отдала его, кому он принадлежит, и разгневался на нее. А она в ответ сказала мне: где же милостыни твои и праведные дела? вот как все они обнаружились на тебе!
\vs Tob 3:1 Опечалившись, я заплакал и молился со скорбью, говоря:
\vs Tob 3:2 праведен Ты, Господи, и все дела Твои и все пути Твои~--- милость и истина, и судом истинным и правым судишь Ты вовек!
\vs Tob 3:3 Воспомяни меня и призри на меня: не наказывай меня за грехи мои и заблуждения мои и отцов моих, которыми они согрешили пред Тобою!
\vs Tob 3:4 Ибо они не послушали заповедей Твоих, и Ты предал нас на расхищение и пленение и смерть, и в притчу поношения пред всеми народами, между которыми мы рассеяны.
\vs Tob 3:5 И, поистине, многи и праведны суды Твои~--- делать со мною по грехам моим и грехам отцов моих, потому что не исполняли заповедей Твоих и не поступали по правде пред Тобою.
\vs Tob 3:6 Итак, твори со мною, что Тебе благоугодно; повели взять дух мой, чтобы я разрешился и обратился в землю, ибо мне лучше умереть, нежели жить, так как я слышу лживые упреки, и глубока скорбь во мне! Повели освободить меня от этой тяготы в обитель вечную и не отврати лица Твоего от меня.
\vs Tob 3:7 В тот самый день случилось и Сарре, дочери Рагуиловой, в Екбатанах Мидийских терпеть укоризны от служанок отца своего
\vs Tob 3:8 за то, что она была отдаваема семи мужьям, но Асмодей, злой дух, умерщвлял их прежде, нежели они были с нею, как с женою. Они говорили ей: разве тебе не совестно, что ты задушила мужей твоих? Уже семерых ты имела, но не назвалась именем ни одного из них.
\vs Tob 3:9 Что нас бить за них? Они умерли: иди и ты за ними, чтобы нам не видеть твоего сына или дочери вовек!
\vs Tob 3:10 Услышав это, она весьма опечалилась, так что решилась было лишить себя жизни, но подумала: я одна у отца моего; если сделаю это, бесчестие ему будет, и я сведу старость его с печалью в преисподнюю.
\vs Tob 3:11 И стала она молиться у окна и говорила: благословен Ты, Господи Боже мой, и благословенно имя Твое святое и славное вовеки: да благословляют Тебя все творения Твои вовек!
\vs Tob 3:12 И ныне к Тебе, Господи, обращаю очи мои и лице мое;
\vs Tob 3:13 молю, возьми меня от земли сей и не дай мне слышать еще укоризны!
\vs Tob 3:14 Ты знаешь, Господи, что я чиста от всякого греха с мужем
\vs Tob 3:15 и не обесчестила имени моего, ни имени отца моего в земле плена моего; я единородная у отца моего, и нет у него сына, который мог бы наследовать ему, ни брата близкого, ни сына братнего, которому я могла бы сберечь себя в жену: уже семеро погибли у меня. Для чего же мне жить? А если не угодно Тебе умертвить меня, то благоволи призреть на меня и помиловать меня, чтобы мне не слышать более укоризны!
\vs Tob 3:16 И услышана была молитва обоих пред славою великого Бога, и послан был Рафаил исцелить обоих:
\vs Tob 3:17 снять бельма у Товита и Сарру, дочь Рагуилову, дать в жену Товии, сыну Товитову, связав Асмодея, злого духа; ибо Товии предназначено наследовать ее.~--- И в одно и то же время Товит, по возвращении, вошел в дом свой, а Сарра, дочь Рагуилова, сошла с горницы своей.
\vs Tob 4:1 В тот день вспомнил Товит о серебре, которое отдал на сохранение Гаваилу в Рагах Мидийских,
\vs Tob 4:2 и сказал сам себе: я просил смерти; что же не позову сына моего Товии, чтобы объявить ему об этом, пока я не умер?
\vs Tob 4:3 И, призвав его, сказал: сын \bibemph{мой}! когда я умру, похорони меня и не покидай матери своей; почитай ее во все дни жизни твоей, делай угодное ей и не причиняй ей огорчения.
\vs Tob 4:4 Помни, сын мой, что она много имела скорбей из-за тебя \bibemph{еще} во время чревоношения. Когда она умрет, похорони ее подле меня в одном гробе.
\vs Tob 4:5 Во все дни помни, сын \bibemph{мой}, Господа Бога нашего и не желай грешить и преступать заповеди Его. Во все дни жизни твоей делай правду и не ходи путями беззакония,
\vs Tob 4:6 ибо, если ты будешь поступать по истине, в делах твоих будет успех, как у всех поступающих по правде.
\vs Tob 4:7 Из имения твоего подавай милостыню, и да не жалеет глаз твой, когда будешь творить милостыню. Ни от какого нищего не отвращай лица твоего, тогда и от тебя не отвратится лице Божие.
\vs Tob 4:8 Когда у тебя будет много, твори из того милостыню, и когда у тебя будет мало, не бойся творить милостыню и понемногу;
\vs Tob 4:9 ты запасешь себе богатое сокровище на день нужды,
\vs Tob 4:10 ибо милостыня избавляет от смерти и не попускает сойти во тьму.
\vs Tob 4:11 Милостыня есть богатый дар для всех, кто творит ее пред Всевышним.
\vs Tob 4:12 Берегись, сын \bibemph{мой}, всякого \bibemph{вида} распутства. Возьми себе жену из племени отцов твоих, но не бери жены иноземной, которая не из колена отца твоего, ибо мы сыны пророков. Издревле отцы наши~--- Ной, Авраам, Исаак и Иаков. Помни, сын \bibemph{мой}, что все они брали жен из \bibemph{среды} братьев своих и были благословенны в детях своих, и потомство их наследует землю.
\vs Tob 4:13 Итак, сын \bibemph{мой}, люби братьев твоих и не превозносись сердцем пред братьями твоими и пред сынами и дочерями народа твоего, чтобы не от них взять тебе жену, потому что от гордости~--- погибель и великое неустройство, а от непотребства~--- оскудение и разорение: непотребство есть мать голода.
\vs Tob 4:14 Плата наемника, который будет работать у тебя, да не переночует у тебя, а отдавай ее тотчас: и тебе воздастся, если будешь служить Богу. Будь осторожен, сын \bibemph{мой}, во всех поступках твоих и будь благоразумен во всем поведении твоем.
\vs Tob 4:15 Что ненавистно тебе самому, того не делай никому. Вина до опьянения не пей, и пьянство да не ходит с тобою в пути твоем.
\vs Tob 4:16 Давай алчущему от хлеба твоего и нагим от одежд твоих; от всего, в чем у тебя избыток, твори милостыни, и да не жалеет глаз твой, когда будешь творить милостыню.
\vs Tob 4:17 Раздавай хлебы твои при гробе праведных, но не давай грешникам.
\vs Tob 4:18 У всякого благоразумного проси совета, и не пренебрегай советом полезным.
\vs Tob 4:19 Благословляй Господа Бога во всякое время и проси у Него, чтобы пути твои были правы и все дела и намерения твои благоуспешны, ибо ни один народ не властен в \bibemph{успехе} начинаний, но Сам Господь ниспосылает все благое и, кого хочет, уничижает по Своей воле. Помни же, сын \bibemph{мой}, заповеди мои, и да не изгладятся они из сердца твоего!
\vs Tob 4:20 Теперь я открою тебе, что я отдал десять талантов серебра на сохранение Гаваилу, сыну Гавриеву, в Рагах Мидийских.
\vs Tob 4:21 Не бойся, сын \bibemph{мой}, что мы обнищали: у тебя много, если ты будешь бояться Господа и, удаляясь от всякого греха, делать угодное пред Ним.
\vs Tob 5:1 И сказал Товия в ответ ему: отец \bibemph{мой}, я исполню все, что ты завещаешь мне;
\vs Tob 5:2 но как я могу получить серебро, не зная того \bibemph{человека}?
\vs Tob 5:3 Тогда \bibemph{отец} дал ему расписку и сказал: найди себе человека, который сопутствовал бы тебе; я дам ему плату, пока еще жив, и ступайте за серебром.
\rsbpar\vs Tob 5:4 И пошел он искать человека и встретил Рафаила. Это был Ангел, но он не знал
\vs Tob 5:5 и сказал ему: можешь ли ты идти со мною в Раги Мидийские и знаешь ли эти места?
\vs Tob 5:6 Ангел отвечал: могу идти с тобою и дорогу знаю; я уже останавливался у Гаваила, брата нашего.
\vs Tob 5:7 И сказал ему Товия: подожди меня, я скажу отцу моему.
\vs Tob 5:8 Тот сказал: ступай, только не медли.
\vs Tob 5:9 Он, придя, сказал отцу: вот я нашел себе спутника. \bibemph{Отец} сказал: пригласи его ко мне; я узнаю, из какого он колена и надежный ли спутник тебе.
\vs Tob 5:10 И позвал его, и он вошел, и приветствовали друг друга.
\vs Tob 5:11 Товит спросил: скажи мне, брат, из какого ты колена и из какого рода?
\vs Tob 5:12 Он отвечал: колена и рода ты ищешь или наемника, который пошел бы с сыном твоим? И сказал ему Товит: брат, мне хочется знать род твой и имя твое.
\vs Tob 5:13 Он сказал: я Азария, \bibemph{из рода} Анании великого, из братьев твоих.
\vs Tob 5:14 Тогда \bibemph{Товит} сказал ему: брат, иди благополучно, и не гневайся на меня за то, что я спросил о колене и роде твоем. Ты доводишься брат мне, из честного и доброго рода. Я знал Ананию и Ионафана, сыновей Семея великого; мы вместе ходили в Иерусалим на поклонение, с первородными и десятинами \bibemph{земных} произведений, ибо не увлекались заблуждением братьев наших: ты, брат, от хорошего корня!
\vs Tob 5:15 Но скажи мне: какую плату я должен буду дать тебе? Я дам тебе драхму на день и все необходимое для тебя и для сына моего,
\vs Tob 5:16 и еще прибавлю тебе сверх этой платы, если благополучно возвратитесь.
\vs Tob 5:17 Так и условились. Тогда он сказал Товии: будь готов в путь, и отправляйтесь благополучно. И приготовил сын его нужное для пути. И сказал ему отец: иди с этим человеком; живущий же на небесах Бог да благоустроит путь ваш, и Ангел Его да сопутствует вам!~--- И отправились оба, и собака юноши с ними.
\rsbpar\vs Tob 5:18 Анна, мать его, заплакала и сказала Товиту: зачем отпустил ты сына нашего? Не он ли был опорою рук наших, когда входил и выходил пред нами?
\vs Tob 5:19 Не предпочитай серебра серебру; пусть оно будет как сор \bibemph{в сравнении} с сыном нашим!
\vs Tob 5:20 Ибо, сколько Господом определено нам жить, на это у нас довольно есть.
\vs Tob 5:21 Товит сказал ей: не печалься, сестра; он придет здоровым, и глаза твои увидят его,
\vs Tob 5:22 ибо ему будет сопутствовать добрый Ангел; путь его будет благоуспешен, и он возвратится здоровым.
\vs Tob 6:1 И перестала она плакать.
\vs Tob 6:2 А путники вечером пришли к реке Тигру и остановились там на ночь.
\vs Tob 6:3 Юноша пошел помыться, но из реки показалась рыба и хотела поглотить юношу.
\vs Tob 6:4 Тогда Ангел сказал ему: возьми эту рыбу. И юноша схватил рыбу и вытащил на землю.
\vs Tob 6:5 И сказал ему Ангел: разрежь рыбу, возьми сердце, печень и желчь, и сбереги \bibemph{их}.
\vs Tob 6:6 Юноша так и сделал, как сказал ему Ангел; рыбу же испекли и съели; и пошли дальше и дошли до Екбатан.
\vs Tob 6:7 И сказал юноша Ангелу: брат Азария, к чему эта печень и сердце и желчь из рыбы?
\vs Tob 6:8 Он отвечал: если кого мучит демон или злой дух, то сердцем и печенью должно курить пред \bibemph{таким} мужчиною или женщиною, и более уже не будет мучиться;
\vs Tob 6:9 а желчью помазать человека, который имеет бельма на глазах, и он исцелится.
\vs Tob 6:10 Когда же приближались к Раге,
\vs Tob 6:11 Ангел сказал юноше: брат, ныне мы переночуем у Рагуила, твоего родственника, у которого есть дочь, по имени Сарра.
\vs Tob 6:12 Я поговорю о ней, чтобы дали ее тебе в жену, ибо тебе предназначено наследство ее, так как ты один из рода ее; а девица прекрасная и умная.
\vs Tob 6:13 Так послушайся меня; я поговорю с ее отцом и, когда мы возвратимся из Раг, совершим брак. Я знаю Рагуила: он никак не даст ее мужу чужому вопреки закону Моисееву; иначе повинен будет смерти, так как наследство следует получить тебе, а не другому кому.
\vs Tob 6:14 Тогда юноша сказал Ангелу: брат Азария, я слышал, что эту девицу отдавали семи мужам, но все они погибли в брачной комнате;
\vs Tob 6:15 а я один у отца и боюсь, как бы, войдя \bibemph{к ней}, не умереть подобно прежним; ее любит демон, который никому не вредит, кроме приближающихся к ней. И потому я боюсь, как бы мне не умереть и не свести жизнь отца моего и матери моей печалью обо мне во гроб их; а другого сына, который похоронил бы их, нет у них.
\vs Tob 6:16 Ангел сказал ему: разве ты забыл слова, которые заповедал тебе отец твой, чтобы ты взял жену из рода твоего? Послушай же меня, брат: ей следует быть твоею женою, а о демоне не беспокойся; в эту же ночь отдадут тебе ее в жену.
\vs Tob 6:17 Только, когда ты войдешь в брачную комнату, возьми курильницу, вложи в нее с\acc{е}рдца и печени рыбы и покури;
\vs Tob 6:18 и демон ощутит запах и удалится, и не возвратится никогда. Когда же тебе надобно будет приблизиться к ней, встаньте оба, воззовите к милосердому Богу, и Он спасет и помилует вас. Не бойся; ибо она предназначена тебе от века, и ты спасешь ее, и она пойдет с тобою, и я знаю, что у тебя будут от нее дети. Выслушав это, Товия полюбил ее, и душа его крепко прилепилась к ней. И пришли они в Екбатаны.
\vs Tob 7:1 И подошли к дому Рагуила. Сарра встретила и приветствовала их, и они ее, и ввела их в дом.
\vs Tob 7:2 И сказал Рагуил Едне, жене своей: как похож этот юноша на Товита, сына брата моего!
\vs Tob 7:3 И спросил их Рагуил: откуда вы, братья? Они отвечали ему: мы из сынов Неффалима, плененных в Ниневию.
\vs Tob 7:4 Еще спросил их: знаете ли брата нашего Товита? Они отвечали: знаем. Потом спросил: здравствует ли он? Они отвечали: жив и здоров.
\vs Tob 7:5 А Товия сказал: это мой отец.
\vs Tob 7:6 И бросился к нему Рагуил и целовал его и плакал.
\vs Tob 7:7 И благословил его и сказал: ты сын честного и доброго человека. Но, услышав, что Товит потерял зрение, опечалился и плакал;
\vs Tob 7:8 плакали и Една, жена его, и Сарра, дочь его. И приняли их весьма радушно,
\vs Tob 7:9 и закололи овна, и предложили обильные снеди. Товия же сказал Рафаилу: брат Азария, переговори, о чем ты говорил на пути; пусть устроится это дело!
\vs Tob 7:10 И он передал эту речь Рагуилу, а Рагуил сказал Товии: ешь, пей и веселись, ибо тебе надлежит взять мою дочь. Впрочем, скажу тебе правду:
\vs Tob 7:11 я отдавал свою дочь семи мужам, и, когда они входили к ней, в ту же ночь умирали. Но ты ныне будь весел! И сказал Товия: я ничего не буду здесь есть до тех пор, пока не сговоритесь и не условитесь со мною. Рагуил сказал: возьми ее теперь же по праву; ты брат ее, и она твоя. Милосердый Бог да устроит вас наилучшим образом!
\vs Tob 7:12 И призвал Сарру, дочь свою, и, взяв руку ее, отдал ее Товии в жену и сказал: вот, по закону Моисееву, возьми ее и веди к отцу твоему. И благословил их.
\vs Tob 7:13 И призвал Едну, жену свою, и, взяв свиток, написал договор и запечатал.
\vs Tob 7:14 И начали есть.
\vs Tob 7:15 И призвал Рагуил Едну, жену свою, и сказал ей: приготовь, сестра, другую спальню и введи ее.
\vs Tob 7:16 И сделала, как он сказал; и ввела ее туда, и заплакала, и приняла взаимно слезы дочери своей, и сказала ей:
\vs Tob 7:17 успокойся, дочь; Господь неба и земли даст тебе радость вместо печали твоей. Успокойся, дочь \bibemph{моя}!
\vs Tob 8:1 Когда окончили ужин, ввели к ней Товию.
\vs Tob 8:2 Он же, идя, вспомнил слова Рафаила, и взял курильницу, и положил сердце и печень рыбы, и курил.
\vs Tob 8:3 Демон, ощутив этот запах, убежал в верхние страны Египта, и связал его Ангел.
\rsbpar\vs Tob 8:4 Когда они остались в комнате вдвоем, Товия встал с постели и сказал: встань, сестра, и помолимся, чтобы Господь помиловал нас.
\vs Tob 8:5 И начал Товия говорить: благословен Ты, Боже отцов наших, и благословенно имя Твое святое и славное вовеки! Да благословляют Тебя небеса и все творения Твои!
\vs Tob 8:6 Ты сотворил Адама и дал ему помощницею Еву, подпорою~--- жену его. От них произошел род человеческий. Ты сказал: нехорошо быть человеку одному, сотворим помощника, подобного ему.
\vs Tob 8:7 И ныне, Господи, я беру сию сестру мою не для удовлетворения похоти, но поистине \bibemph{как жену}: благоволи же помиловать меня и \bibemph{дай} мне состариться с нею!
\vs Tob 8:8 И она сказала с ним: аминь.
\vs Tob 8:9 И оба спокойно спали в эту ночь. Между тем Рагуил, встав, пошел и выкопал могилу,
\vs Tob 8:10 говоря: не умер ли и этот?
\vs Tob 8:11 И пришел Рагуил в дом свой
\vs Tob 8:12 и сказал Едне, жене своей: пошли одну из служанок посмотреть, жив ли он; если нет, похороним его, и никто не будет знать.
\vs Tob 8:13 Служанка, отворив дверь, вошла и увидела, что оба они спят.
\vs Tob 8:14 И, выйдя, объявила им, что он жив.
\rsbpar\vs Tob 8:15 И благословил Рагуил Бога, говоря: благословен Ты, Боже, всяким благословением чистым и святым! Да благословляют Тебя святые Твои, и все создания Твои, и все Ангелы Твои, и все избранные Твои, да благословляют Тебя вовеки!
\vs Tob 8:16 Благословен Ты, что возвеселил меня, и не случилось со мною так, как я думал, но сотворил с нами по великой Твоей милости!
\vs Tob 8:17 Благословен Ты, что помиловал двух единородных! Доверши, Владыка, милость над ними: дай им окончить жизнь во здравии, с весельем и милостью!
\vs Tob 8:18 И приказал рабам своим зарыть могилу.
\vs Tob 8:19 И сделал для них брачный пир на четырнадцать дней.
\vs Tob 8:20 И сказал ему Рагуил с клятвою прежде исполнения дней брачного пира: не уходи, доколе не исполнятся эти четырнадцать дней брачного пира;
\vs Tob 8:21 а тогда, взяв половину имения, благополучно отправляйся к отцу твоему: остальное же \bibemph{получишь}, когда умру я и жена моя.
\vs Tob 9:1 И позвал Товия Рафаила и сказал ему:
\vs Tob 9:2 брат Азария, возьми с собою раба и двух верблюдов и сходи в Раги Мидийские к Гаваилу; принеси мне серебро и самого его приведи ко мне на брак;
\vs Tob 9:3 ибо Рагуил обязал меня клятвою, чтоб я не уходил;
\vs Tob 9:4 между тем отец мой считает дни и, если я много замедлю, он будет очень скорбеть.
\vs Tob 9:5 И пошел Рафаил и остановился у Гаваила и отдал ему расписку; а тот принес мешки за печатями и передал ему.
\vs Tob 9:6 И на утро рано встали они вместе и пришли на брак. И благословил Товия жену свою.
\vs Tob 10:1 Товит, отец его, считал каждый день. И когда исполнились дни путешествия, а он не приходил,
\vs Tob 10:2 Товит сказал: не задержали ли их? или не умер ли Гаваил, и некому отдать им серебра?
\vs Tob 10:3 И очень печалился.
\vs Tob 10:4 Жена же его сказала ему: погиб сын наш, потому и не приходит. И начала плакать по нем и говорила:
\vs Tob 10:5 ничто не занимает меня, сын мой, потому что я отпустила тебя, свет очей моих!
\vs Tob 10:6 Товит говорит ей: молчи, не тревожься, он здоров.
\vs Tob 10:7 А она сказала ему: молчи ты, не обманывай меня; погибло детище мое.~--- И ежедневно ходила за город на дорогу, по которой они отправились; днем не ела хлеба, а по ночам не переставала плакать о сыне своем Товии, пока не окончились четырнадцать дней брачного пира, которые Рагуил заклял его провести там. Тогда Товия сказал Рагуилу: отпусти меня, потому что отец мой и мать моя не надеются уже видеть меня.
\vs Tob 10:8 Тесть же сказал ему: побудь у меня; я пошлю к отцу твоему, и известят его о тебе.
\vs Tob 10:9 А Товия говорит: нет, отпусти меня к отцу моему.
\vs Tob 10:10 И встал Рагуил и отдал ему Сарру, жену его, и половину имения, рабов и скота и серебро,
\vs Tob 10:11 и, благословив их, отпустил и сказал: дети! да благопоспешит вам Бог Небесный, прежде нежели я умру.
\vs Tob 10:12 Потом сказал дочери своей: почитай твоего свекра и свекровь; теперь они~--- родители твои; желаю слышать добрый слух о тебе. И поцеловал ее. И Една сказала Товии: возлюбленный брат, да восставит тебя Господь Небесный и дарует мне видеть детей от Сарры, дочери моей, дабы я возрадовалась пред Господом. И вот, отдаю тебе дочь мою на сохранение; не огорчай ее.
\rsbpar\vs Tob 10:13 После того отправился Товия, благословляя Бога, что Он благоустроил путь его, и благословлял Рагуила и Едну, жену его. И продолжал путь, и приблизились они к Ниневии.
\vs Tob 11:1 И сказал Рафаил Товии: ты знаешь, брат, \bibemph{в} каком \bibemph{положении} ты оставил отца твоего;
\vs Tob 11:2 пойдем вперед, прежде жены твоей, и приготовим помещение;
\vs Tob 11:3 а ты возьми в руку и желчь рыбью. И пошли; за ними побежала и собака.
\rsbpar\vs Tob 11:4 Между тем Анна сидела, высматривая на дороге сына своего,
\vs Tob 11:5 и, заметив, что он идет, сказала отцу его: вот, идет сын твой и человек, отправившийся с ним.
\vs Tob 11:6 Рафаил сказал: я знаю, Товия, что у отца твоего откроются глаза;
\vs Tob 11:7 ты только помажь желчью глаза его, и он, ощутив едкость, оботрет \bibemph{их}, и спадут бельма, и он увидит тебя.
\vs Tob 11:8 Анна, подбежав, бросилась на шею к сыну своему и сказала ему: увидела я тебя, дитя \bibemph{мое},~--- теперь мне хотя умереть. И оба заплакали.
\vs Tob 11:9 А Товит пошел к дверям и споткнулся, но сын его поспешил к нему, и поддержал отца своего,
\vs Tob 11:10 и приложил желчь к глазам отца своего, и сказал: ободрись, отец \bibemph{мой}!
\vs Tob 11:11 Глаза его заело, и он отер их,
\vs Tob 11:12 и снялись с краев глаз его бельма. Увидев сына своего, он пал на шею к нему
\vs Tob 11:13 и заплакал и сказал: благословен Ты, Боже, и благословенно имя Твое вовеки, и благословенны все святые Ангелы Твои!
\vs Tob 11:14 Потому что Ты наказал и помиловал меня. Вот, я вижу Товию, сына моего.~--- И вошел сын его радостно и рассказал отцу своему о чудных \bibemph{делах}, бывших с ним в Мидии.
\vs Tob 11:15 И вышел Товит навстречу невестке своей к воротам Ниневии, радуясь и благословляя Бога. Видевшие, что он идет, удивлялись, как он прозрел.
\vs Tob 11:16 И Товит исповедал пред ними, что Бог помиловал его. Когда подошел Товит к Сарре, невестке своей, благословил ее и сказал: здравствуй, дочь \bibemph{моя}! Благословен Бог, Который привел тебя к нам, и \bibemph{благословенны} отец твой и мать твоя! Обрадовались и все братья его в Ниневии.
\vs Tob 11:17 И пришел Ахиахар и Насвас, племянник его,
\vs Tob 11:18 и весело праздновали брак Товии семь дней.
\vs Tob 12:1 И призвал Товит сына своего Товию и сказал ему: приготовь, сын \bibemph{мой}, плату человеку, который ходил с тобою; ему надобно еще прибавить.
\vs Tob 12:2 Он отвечал: отец \bibemph{мой}, я не буду в убытке, если отдам ему половину всего, что принес;
\vs Tob 12:3 потому что он привел меня к тебе здоровым и жену мою уврачевал, и серебро мое принес, и тебя также исцелил.
\vs Tob 12:4 Старец сказал: так и следует ему.
\vs Tob 12:5 И призвал Ангела и сказал ему: возьми половину всего, что вы принесли, и иди с миром.
\rsbpar\vs Tob 12:6 Тогда, отозвав обоих особо, \bibemph{Ангел} сказал им: благословляйте Бога, прославляйте Его, признавайте величие Его и исповедуйте пред всеми живущими, что Он сделал для вас. Доброе \bibemph{дело}~--- благословлять Бога, превозносить имя Его и благоговейно проповедовать о делах Божиих; и вы не ленитесь прославлять Его.
\vs Tob 12:7 Тайну цареву прилично хранить, а о делах Божиих объявлять похвально. Делайте добро, и зло не постигнет вас.
\vs Tob 12:8 Доброе \bibemph{дело}~--- молитва с постом и милостынею и справедливостью. Лучше малое со справедливостью, нежели многое с неправдою; лучше творить милостыню, нежели собирать золото,
\vs Tob 12:9 ибо милостыня от смерти избавляет и может очищать всякий грех. Творящие милостыни и \bibemph{дела} правды будут долгоденствовать.
\vs Tob 12:10 Грешники же суть враги своей жизни.
\vs Tob 12:11 Не скрою от вас ничего; я сказал уже: тайну цареву прилично хранить, а о делах Божиих объявлять похвально.
\vs Tob 12:12 Когда молился ты и невестка твоя Сарра, я возносил память молитвы вашей пред Святаго, и когда ты хоронил мертвых, я также был с тобою.
\vs Tob 12:13 И когда ты не обленился встать и оставить обед свой, чтобы пойти и убрать мертвого, твоя благотворительность не утаилась от меня, но я был с тобою.
\vs Tob 12:14 И ныне Бог послал меня уврачевать тебя и невестку твою Сарру.
\vs Tob 12:15 Я~--- Рафаил, один из семи святых Ангелов, которые возносят молитвы святых и восходят пред славу Святаго.
\rsbpar\vs Tob 12:16 Тогда оба смутились и пали лицем \bibemph{на землю}, потому что были в страхе.
\vs Tob 12:17 Но он сказал им: не бойтесь, мир будет вам. Благословляйте Бога вовек.
\vs Tob 12:18 Ибо я пришел не по своему произволению, а по воле Бога нашего; потому и благословляйте Его вовек.
\vs Tob 12:19 Все дни я был видим вами, но я не ел и не пил,~--- только взорам вашим представлялось \bibemph{это}.
\vs Tob 12:20 Итак, прославляйте теперь Бога, потому что я восхожу к Пославшему меня, и напишите все совершившееся в книгу.
\vs Tob 12:21 И встали они и более уже не видели его.
\vs Tob 12:22 И стали рассказывать о великих и чудных делах Божиих, и как явился им Ангел Господень.
\vs Tob 13:1 В радости Товит написал молитву \bibemph{в сих} словах: благословен Бог, вечно живущий, и \bibemph{благословенно} царство Его!
\vs Tob 13:2 Ибо Он наказует и милует, низводит до ада и возводит, и нет никого, кто избежал бы от руки Его.
\vs Tob 13:3 Сыны Израилевы! прославляйте Его пред язычниками, ибо Он рассеял нас между ними.
\vs Tob 13:4 Там возвещайте величие Его, превозносите Его пред всем живущим, ибо Он Господь наш и Бог, Отец наш во все веки:
\vs Tob 13:5 накажет нас за неправды наши, и опять помилует и соберет нас из всех народов, где бы вы ни были рассеяны между ними.
\vs Tob 13:6 Если вы будете обращаться к Нему всем сердцем вашим и всею душею вашею, чтобы поступать пред Ним по истине, тогда Он обратится к вам и не скроет от вас лица Своего. Увидите, чт\acc{о} Он сделает с вами. Прославляйте Его всеми \bibemph{глаголами} уст ваших и благословляйте Господа правды и превозносите Царя веков. В земле плена моего я прославляю Его и проповедую силу и величие Его народу грешников. Обратитесь, грешники, и делайте правду пред Ним. Кто знает, может быть, Он возблаговолит о вас и окажет вам милость?
\vs Tob 13:7 Превозношу я Бога моего, и душа моя~--- Небесного Царя, и радуется о величии Его.
\vs Tob 13:8 Пусть все возвещают о Нем и прославляют Его в Иерусалиме.
\vs Tob 13:9 Иерусалим, город святый! Он накажет \bibemph{тебя} за дела сынов твоих и опять помилует сынов праведных.
\vs Tob 13:10 Славь Господа усердно и благословляй Царя веков, чтобы снова сооружена была скиния Его в тебе с радостью, чтобы Он возвеселил среди тебя пленных и возлюбил в тебе несчастных во все роды века.
\vs Tob 13:11 Многие народы издалека придут к имени Господа Бога с дарами в руках, с дарами Царю Небесному; роды родов восхвалят тебя с восклицаниями радостными.
\vs Tob 13:12 Прокляты все ненавидящие тебя, благословенны будут вовек все любящие тебя!
\vs Tob 13:13 Радуйся и веселись о сынах праведных, ибо они соберутся и будут благословлять Господа праведных.
\vs Tob 13:14 О, блаженны любящие тебя! они возрадуются о мире твоем. Блаженны скорбевшие о всех бедствиях твоих, ибо они возрадуются о тебе, когда увидят всю славу твою, и будут веселиться вечно.
\vs Tob 13:15 Да благословляет душа моя Бога, Царя великого,
\vs Tob 13:16 ибо Иерусалим отстроен будет из сапфира и смарагда и из дорогих камней; стены твои, башни и укрепления~--- из чистого золота;
\vs Tob 13:17 и площади Иерусалимские выстланы будут бериллом, анфраксом и камнем из Офира.
\vs Tob 13:18 На всех улицах его будет раздаваться: аллилуия,~--- и будут славословить, говоря: благословен Бог, Который превознес \bibemph{Иерусалим}, на все веки!
\vs Tob 14:1 И окончил Товит славословие.
\rsbpar\vs Tob 14:2 Он был восьмидесяти восьми лет, когда потерял зрение, и чрез восемь лет прозрел. И творил милостыни, и продолжал быть благоговейным пред Господом Богом и прославлять Его.
\vs Tob 14:3 Наконец он очень состарился, и призвал сына своего и шесть сыновей его, и сказал ему: сын \bibemph{мой}, возьми сыновей твоих; вот я состарился и уже на исходе жизни моей.
\vs Tob 14:4 Отправься в Мидию, сын \bibemph{мой}, ибо я уверен, что Ниневия будет разорена, как говорил пророк Иона; а в Мидии будет спокойнее до времени. Братья наши, находящиеся в \bibemph{отечественной} земле, будут рассеяны из сей доброй земли; Иерусалим будет пустынею, и дом Божий в нем будет сожжен и до времени останется пуст.
\vs Tob 14:5 Но опять Бог помилует их и возвратит их в землю; и воздвигнут дом \bibemph{Божий}, не такой, как прежний, доколе не исполнятся времена века. И после того возвратятся из плена и построят Иерусалим великолепно, и дом Божий восстановлен будет в нем на все роды века,~--- здание величественное, как говорили о нем пророки.
\vs Tob 14:6 И все народы обратятся и будут истинно благоговеть пред Господом Богом и ниспровергнут идолов своих;
\vs Tob 14:7 и все народы будут благословлять Господа. И Его народ будет прославлять Бога, и Господь вознесет народ Свой; и все, истинно и праведно любящие Господа Бога, будут радоваться, оказывая милость братьям нашим.
\vs Tob 14:8 Итак, сын \bibemph{мой}, выйди из Ниневии, ибо непременно исполнится то, что говорил пророк Иона.
\vs Tob 14:9 Ты же соблюдай закон и повеления и будь любомилостив и справедлив, чтобы хорошо было тебе.
\vs Tob 14:10 Похорони меня прилично, и мать твою со мною, и потом не оставайтесь в Ниневии.~--- Сын \bibemph{мой}, смотри, чт\acc{о} сделал Аман с Ахиахаром, который воспитал его: как он из света привел его в тьму, и как воздано ему. Ахиахар спасен, а тот получил достойное возмездие~--- сошел во тьму. Манассия творил милостыню, и спасен от смертной сети, которую расставили ему; Аман же пал в сеть и погиб.
\vs Tob 14:11 Итак, дети, знайте, чт\acc{о} делает милостыня и как спасает справедливость.~--- Когда он это сказал, душа его оставила его на ложе; было же ему сто пятьдесят восемь лет, и сын с честью похоронил его.
\rsbpar\vs Tob 14:12 Когда умерла Анна, он похоронил и ее с отцом своим. После того Товия с женою своею и детьми своими отправился в Екбатаны к Рагуилу, тестю своему,
\vs Tob 14:13 и достиг честной старости, и похоронил прилично тестя и тещу своих, и получил в наследство имение их и Товита, отца своего.
\vs Tob 14:14 И умер ста двадцати семи лет в Екбатанах Мидийских.
\vs Tob 14:15 Но прежде нежели умер, он слышал о погибели Ниневии, которую пленил Навуходоносор и Асуир, и возрадовался пред смертью о Ниневии.

\bibbookdescr{Jdt}{
  inline={\LARGE Книга\\\Huge Иудифи\fns{Переведена с греческого.}},
  toc={Иудифь*},
  bookmark={Иудифь},
  header={Иудифь},
  %headerleft={},
  %headerright={},
  abbr={Иудифь}
}
\vs Jdt 1:1 В двенадцатый год царствования Навуходоносора, царствовавшего над Ассириянами в великом городе Ниневии,~--- во дни Арфаксада, который царствовал над Мидянами в Екбатанах
\vs Jdt 1:2 и построил вокруг Екбатан стены из тесаных камней, шириною в три локтя, а длиною в шесть локтей; и сделал высоту стены в семьдесят, а ширину в пятьдесят локтей,
\vs Jdt 1:3 и поставил над воротами башни во сто локтей, имевшие в основании до шестидесяти локтей ширины;
\vs Jdt 1:4 а ворота, построенные им для выхода сильных войск его и для строев пехоты его, поднимались в высоту на семьдесят локтей, а в ширину имели сорок локтей:
\vs Jdt 1:5 в те дни царь Навуходоносор предпринял войну против царя Арфаксада на великой равнине, которая в пределах Рагава.
\vs Jdt 1:6 К нему собрались все живущие в нагорной стране, и все живущие при Евфрате, Тигре и Идасписе, и с равнины Ариох, царь Елимейский, и сошлись очень многие народы в ополчение сынов Хелеуда.
\vs Jdt 1:7 И послал Навуходоносор, царь Ассирийский, ко всем живущим в Персии и ко всем живущим на западе, к живущим в Киликии и Дамаске, Ливане и Антиливане, и ко всем живущим на передней стороне приморья,
\vs Jdt 1:8 и между народами Кармила и Галаада и в верхней Галилее и на великой равнине Ездрилон,
\vs Jdt 1:9 и ко всем живущим в Самарии и городах ее, и за Иорданом до Иерусалима, и Ветани и Хела, и Кадиса и реки Египетской, и Тафны и Рамессы, и во всей земле Гесемской
\vs Jdt 1:10 до входа в верхний Танис и Мемфис, и ко всем живущим в Египте до входа в пределы Ефиопии.
\vs Jdt 1:11 Но все обитавшие во всей этой земле презрели слово Ассирийского царя Навуходоносора и не собрались к нему на войну, потому что они не боялись его, но он был для них как один из них: они отослали от себя его послов ни с чем, в бесчестии.
\vs Jdt 1:12 Навуходоносор весьма разгневался на всю эту землю и поклялся престолом и царством своим отмстить всем пределам Киликии, Дамаска и Сирии, и мечом своим умертвить всех, живущих в земле Моава, и сынов Аммона и всю Иудею, и всех, обитающих в Египте до входа в пределы двух морей.
\rsbpar\vs Jdt 1:13 И в семнадцатый год он ополчился со своим войском против царя Арфаксада и одолел его в сражении и обратил в бегство все войско Арфаксада, всю конницу его и все колесницы его,
\vs Jdt 1:14 и овладел городами его, дошел до Екбатан, занял укрепления, опустошил улицы \bibemph{города} и красоту его обратил в позор.
\vs Jdt 1:15 А Арфаксада схватил на горах Рагава и, пронзив его копьем своим, в тот же день погубил его.
\vs Jdt 1:16 Потом пошел назад со своими в Ниневию,~--- он и все союзники его~--- весьма многое множество ратных мужей; там он отдыхал, и пировал с войском своим сто двадцать дней.
\vs Jdt 2:1 В восемнадцатом году, в двадцать второй день первого месяца, последовало в доме Навуходоносора, царя Ассирийского, повеление~--- совершить, как он сказал, отмщение всей земле.
\vs Jdt 2:2 Созвав всех служителей и всех сановников своих, он открыл им тайну своего намерения и своими устами определил всякое зло той земле.
\vs Jdt 2:3 И они решили погубить всех, кто не повиновался слову уст его.
\vs Jdt 2:4 По окончании своего совещания, Навуходоносор, царь Ассирийский, призвал главного вождя войска своего, Олоферна, который был вторым по нем, и сказал ему:
\vs Jdt 2:5 так говорит великий царь, господин всей земли: вот, ты пойдешь от лица моего и возьмешь с собою мужей, уверенных в своей силе,~--- пеших сто двадцать тысяч и множество коней с двенадцатью тысячами всадников,~---
\vs Jdt 2:6 и выйдешь против всей земли на западе за то, что не повиновались слову уст моих.
\vs Jdt 2:7 И объявишь им, чтобы они приготовляли землю и воду, потому что я с гневом выйду на них, покрою все лице земли \bibemph{их} ногами войска моего и предам ему их на расхищение.
\vs Jdt 2:8 Долы и потоки наполнятся их ранеными, и река, запруженная трупами их, переполнится;
\vs Jdt 2:9 а пленных их я рассею по концам всей земли.
\vs Jdt 2:10 Ты же отправившись завладей для меня всеми пределами их: которые сами сдадутся тебе, тех ты сохрани до дня обличения их;
\vs Jdt 2:11 а непокорных да не пощадит глаз твой: предавай их смерти и разграблению по всей земле твоей.
\vs Jdt 2:12 Ибо жив я,~--- и крепко царство мое: что сказал, то сделаю моею рукою.
\vs Jdt 2:13 Не преступи же ни в чем слов господина твоего, но непременно исполни, как я приказал тебе, и не медли исполнением.
\rsbpar\vs Jdt 2:14 Олоферн, выйдя от лица господина своего, пригласил к себе всех сановников, полководцев и начальников войска Ассирийского,
\vs Jdt 2:15 отсчитал для сражения отборных мужей, как повелел ему господин его, сто двадцать тысяч, и конных стрелков двенадцать тысяч,
\vs Jdt 2:16 и привел их в такой порядок, каким строится войско, идущее на сражение.
\vs Jdt 2:17 Он взял весьма много верблюдов, ослов и мулов для обоза их, а овец, волов и коз для продовольствия их~--- без числа,
\vs Jdt 2:18 и много пищи для всех, и очень много золота и серебра из царского дома.
\vs Jdt 2:19 И выступил в поход со всем войском своим, чтобы предварить царя Навуходоносора и покрыть все лице земли на западе колесницами, конницею и отборною пехотою своею.
\vs Jdt 2:20 И с ним вышли союзники в таком множестве, как саранча и как песок земной, потому что от множества не было и счета им.
\vs Jdt 2:21 Пройдя путь трех дней от Ниневии до передней стороны равнины Вектелеф, они поворотили от Вектелефа, близ горы, лежащей по левую сторону верхней Киликии.
\vs Jdt 2:22 Оттуда, взяв все войско свое, пеших и конных и колесницы свои, он отправился в нагорную страну;
\vs Jdt 2:23 разбил Фудян и Лудян и разграбил всех сынов Рассиса и сынов Исмаила, живших в пустыне на юг к земле Хеллеонской.
\vs Jdt 2:24 \bibemph{Потом}, переправившись чрез Евфрат, он прошел Месопотамию и разрушил все высокие города при потоке Авроне до входа в море.
\vs Jdt 2:25 Заняв пределы Киликии, он избил всех, противоставших ему, и, пройдя до пределов Иафета, лежащих к югу на передней стороне Аравии,
\vs Jdt 2:26 обошел кругом всех сынов Мадиама, выжег жилища их и разграбил стада их.
\vs Jdt 2:27 Потом спустился на равнину Дамаска, во время жатвы пшеницы, выжег все нивы их, отдал на истребление стада овец и волов, разграбил города их, опустошил их поля и избил всех юношей их острием меча.
\vs Jdt 2:28 Страх и ужас напал на жителей приморской страны, обитавших в Сидоне и Тире, на жителей Сура и Окины и на всех жителей Иемнаана,~--- и все обитатели Азота и Аскалона сильно испугались его.
\vs Jdt 3:1 И послали к нему вестников с таким мирным предложением:
\vs Jdt 3:2 вот мы, рабы великого царя Навуходоносора, повергаемся перед тобою: делай с нами, что тебе угодно.
\vs Jdt 3:3 Вот перед тобою: и селения наши, и все места наши, и все нивы с пшеницею, и стада овец и волов, и все строения наших жилищ: употребляй их, как пожелаешь.
\vs Jdt 3:4 Вот и города наши и обитающие в них~--- рабы твои: иди и поступай с ними, как будет глазам твоим угодно.
\rsbpar\vs Jdt 3:5 И пришли к Олоферну мужи и передали ему эти слова.
\vs Jdt 3:6 \bibemph{Тогда} он пришел в приморскую страну с войском своим, окружил высокие города стражею и взял из них отборных мужей в соратники себе.
\vs Jdt 3:7 А они и вся окрестность их приняли его с венками, ликами и тимпанами.
\vs Jdt 3:8 Он же разорил все высоты их и вырубил рощи их: ему приказано было истребить всех богов той земли, чтобы все народы служили одному Навуходоносору, и все языки и все племена их призывали его, как Бога.
\vs Jdt 3:9 Придя к Ездрилону близ Дотеи, лежащей против великой теснины Иудейской,
\vs Jdt 3:10 он расположился лагерем между Гаваем и городом Скифов и оставался там целый месяц, чтобы собрать весь обоз своего войска.
\vs Jdt 4:1 Сыны Израиля, жившие в Иудее, услышав обо всем, что сделал с народами Олоферн, военачальник Ассирийского царя Навуходоносора, и как разграбил он все святилища их и отдал их на уничтожение,
\vs Jdt 4:2 очень, очень испугались его и трепетали за Иерусалим и храм Господа Бога своего;
\vs Jdt 4:3 потому что недавно возвратились они из плена, недавно весь народ Иудейский собрался, и освящены от осквернения сосуды, жертвенник и дом \bibemph{Господень}.
\vs Jdt 4:4 Они послали во все пределы Самарии и Конии, и Ветерона и Вельмена, и Иерихона, и в Хову и Эсору, и в равнину Салимскую,
\vs Jdt 4:5 заняли все вершины высоких гор, оградили стенами находящиеся на них селения и отложили запасы хлеба на случай войны, так как нивы их недавно были сжаты,
\vs Jdt 4:6 а великий священник Иоаким, бывший в те дни в Иерусалиме, написал жителям Ветилуи и Ветомесфема, лежащего против Ездрилона, на передней стороне равнины, близкой к Дофаиму,
\vs Jdt 4:7 чтобы они заняли восходы в нагорную страну, потому что чрез них был вход в Иудею, и легко было им воспрепятствовать приходящим, так как тесен был проход даже для двух человек.
\rsbpar\vs Jdt 4:8 Сыны Израиля поступили так, как велел им великий священник Иоаким и старейшины всего народа Израильского, пребывавшие в Иерусалиме.
\vs Jdt 4:9 И с великим усердием возопили к Богу все мужи Израиля и смирили души свои с великим усердием:
\vs Jdt 4:10 они и жены их, и дети их, и скот их; и всякий пришлец, и наемник, и купленный за серебро наложили вретища на чресла свои.
\vs Jdt 4:11 И всякий муж Израильский и \bibemph{всякая} жена, и дети, и жители Иерусалима пали пред храмом, посыпали пеплом свои головы, разостлали пред Господом свои вретища,
\vs Jdt 4:12 облекли жертвенник во вретище и прилежно и единодушно взывали к Богу Израилеву, чтобы Он, на радость язычникам, не предал детей их на расхищение, жен их в добычу, городов наследия их на разорение, святынь их на осквернение и поругание.
\vs Jdt 4:13 И Господь услышал голос их и призрел на скорбь их; и во всей Иудее и Иерусалиме народ много дней постился пред святилищем Господа Вседержителя.
\vs Jdt 4:14 А Иоаким, великий священник, и все предстоящие пред Господом священники, служители Его, препоясав вретищем чресла свои, приносили непрестанные всесожжения, обеты и доброхотные дары народа.
\vs Jdt 4:15 На кидарах их был пепел, и они от всей силы взывали к Господу, чтобы Он посетил милостью весь дом Израиля.
\vs Jdt 5:1 Между тем Олоферну, военачальнику войска Ассирийского, дано было знать, что сыны Израиля приготовились к войне: заложили входы в нагорную страну и укрепили стенами всякую вершину высокой горы, а на равнинах устроили преграды.
\vs Jdt 5:2 Он весьма разгневался и, призвав всех начальников Моава и вождей Аммона и всех правителей приморской страны, сказал им:
\vs Jdt 5:3 скажите мне, сыны Ханаана, что это за народ, живущий в нагорной стране, какие обитаемые ими города, много ли у них войска, в чем их крепость и сила, кто поставлен над ними царем, предводителем войска их,
\vs Jdt 5:4 и почему они больше всех, живущих на западе, упорствуют выйти мне навстречу?
\vs Jdt 5:5 Ахиор, предводитель всех сынов Аммона, сказал ему: выслушай, господин мой, слово из уст раба твоего; я скажу тебе истину об этом народе, живущем близ тебя в этой нагорной стране, и не выйдет лжи из уст раба твоего.
\vs Jdt 5:6 Этот народ происходит от Халдеев.
\vs Jdt 5:7 Прежде они поселились в Месопотамии, потому что не хотели служить богам отцов своих, которые были в земле Халдейской,
\vs Jdt 5:8 и уклонились от пути предков своих и начали поклоняться Богу неба, Богу, Которого они познали; и \bibemph{Халдеи} выгнали их от лица богов своих,~--- и они бежали в Месопотамию и долго там обитали.
\vs Jdt 5:9 Но Бог их сказал, чтобы они вышли из места переселения и шли в землю Ханаанскую; они поселились там и весьма обогатились золотом, серебром и множеством скота.
\vs Jdt 5:10 \bibemph{Отсюда} перешли они в Египет, так как голод накрыл лице земли Ханаанской, и там оставались, пока находили пропитание, и умножились там до того, что не было и числа роду их.
\vs Jdt 5:11 И восстал на них царь Египетский, употребил против них хитрость, обременяя их трудом и деланьем кирпича, и сделал их рабами.
\vs Jdt 5:12 Тогда они воззвали к Богу своему,~--- и Он поразил всю землю Египетскую неисцельными язвами,~--- и Египтяне прогнали их от себя.
\vs Jdt 5:13 Бог иссушил перед ними Чермное море
\vs Jdt 5:14 и вел их путем Сины и Кадис-Варни; они выгнали всех обитавших в этой пустыне;
\vs Jdt 5:15 поселились в земле Аморреев, своею силою истребили всех Есевонитян, перешли Иордан, наследовали всю нагорную страну
\vs Jdt 5:16 и, прогнав от себя Хананея, Ферезея, Иевусея, Сихема и всех Гергесеян, жили в ней много дней.
\vs Jdt 5:17 И доколе не согрешили пред Богом своим, счастье было с ними, потому что с ними Бог, ненавидящий неправду.
\vs Jdt 5:18 Но когда уклонились от пути, который Он завещал им, то во многих войнах они потерпели весьма сильные поражения, отведены в плен, в чужую землю, храм Бога их разрушен, и города их взяты неприятелями.
\vs Jdt 5:19 Ныне же, обратившись к Богу своему, они возвратились из рассеяния, в котором были, овладели Иерусалимом, в котором святилище их, и поселились в нагорной стране, так как она была пуста.
\vs Jdt 5:20 И теперь, повелитель-господин, если есть заблуждение в этом народе, и они грешат пред Богом своим, и мы заметим, что в них есть это преткновение, то мы пойдем и победим их.
\vs Jdt 5:21 А если нет в этом народе беззакония, то пусть удалится господин мой, чтобы Господь не защитил их, и Бог их \bibemph{не был} за них,~--- и тогда мы для всей земли будем предметом поношения.
\rsbpar\vs Jdt 5:22 Когда Ахиор окончил эту речь, весь народ, стоявший вокруг шатра, возроптал, а вельможи Олоферна и все, населявшие приморье и землю Моава, заговорили: тотчас надобно убить его;
\vs Jdt 5:23 потому что мы не побоимся сынов Израиля: это~--- народ, у которого нет ни войска, ни силы для крепкого ополчения.
\vs Jdt 5:24 Итак, пойдем, повелитель Олоферн,~--- и они сделаются добычею всего войска твоего.
\vs Jdt 6:1 Когда утих шум вокруг собрания, Олоферн, военачальник войска Ассирийского, сказал Ахиору пред всем народом иноплеменных и всем сынам Моава:
\vs Jdt 6:2 кто ты такой, Ахиор, с наемниками Ефрема, что напророчил нам сегодня и сказал, чтобы мы не воевали с родом Израильским, потому что Бог их защищает? Кто же Бог, как не Навуходоносор? Он пошлет свою силу и сотрет их с лица земли,~--- и Бог их не избавит их.
\vs Jdt 6:3 Но мы, рабы его, поразим их, как одного человека, и не устоять им против силы наших коней.
\vs Jdt 6:4 Мы растопчем их; горы их упьются их кровью, равнины их наполнятся их трупами, и не станет стопа ног их против нашего лица, но гибелью погибнут они, говорит царь Навуходоносор, господин всей земли. Ибо он сказал,~--- и не напрасны будут слова повелений его.
\vs Jdt 6:5 А ты, Ахиор, наемник Аммона, высказавший слова эти в день неправды твоей, от сего дня не увидишь больше лица моего, доколе я не отомщу этому народу, \bibemph{пришедшему} из Египта.
\vs Jdt 6:6 Когда же я возвращусь, меч войска моего и толпа слуг моих пройдет по ребрам твоим,~--- и ты падешь между ранеными их.
\vs Jdt 6:7 Рабы мои отведут тебя в нагорную страну и оставят в одном из городов на высотах,
\vs Jdt 6:8 и ты не умрешь там, доколе не будешь с ними истреблен.
\vs Jdt 6:9 Если же ты надеешься в сердце твоем, что они не будут взяты, то да не спадает лице твое. Я сказал, и ни одно из слов моих не пропадет.
\vs Jdt 6:10 И приказал Олоферн рабам своим, предстоявшим в шатре его, взять Ахиора, отвести его в Ветилую и предать в руки сынов Израиля.
\vs Jdt 6:11 Рабы его схватили и вывели его за стан на поле, а со среды равнины поднялись в нагорную страну и пришли к источникам, бывшим под Ветилуею.
\vs Jdt 6:12 Когда увидели их жители города на вершине горы, то взялись за оружия свои и, выйдя за город на вершину горы, все мужи-пращники охраняли восход свой и бросали в них каменьями.
\vs Jdt 6:13 А они, подойдя под гору, связали Ахиора и, оставив его брошенным при подошве горы, ушли к своему господину.
\vs Jdt 6:14 Сыны же Израиля, вышедшие из своего города, остановились над ним и, развязав его, привели в Ветилую, и представили его начальникам своего города,
\vs Jdt 6:15 которыми были в те дни Озия, сын Михи из колена Симеонова, Хаврий, сын Гофониила, и Хармий, сын Мелхиила.
\vs Jdt 6:16 Они созвали всех старейшин города, и сбежались в собрание все юноши их и жены, и поставили Ахиора среди всего народа своего, и Озия спросил его о случившемся.
\vs Jdt 6:17 Он в ответ пересказал им слова собрания Олофернова и все слова, которые он высказал среди начальников сынов Ассура, и все высокомерные речи Олоферна о доме Израиля.
\vs Jdt 6:18 \bibemph{Тогда} народ пал, поклонился Богу и воззвал:
\vs Jdt 6:19 Господи, Боже Небесный! воззри на их гордыню и помилуй смирение рода нашего, и призри на лице освященных Тебе в этот день.
\vs Jdt 6:20 И утешили Ахиора и расхвалили его.
\vs Jdt 6:21 Потом Озия взял его из собрания в свой дом и сделал пир для старейшин,~--- и целую ночь ту они призывали Бога Израилева на помощь.
\vs Jdt 7:1 На другой день Олоферн приказал всему войску своему и всему народу своему, пришедшему к нему на помощь, подступить к Ветилуе, занять высоты нагорной страны и начать войну против сынов Израилевых.
\vs Jdt 7:2 И в тот же день поднялись все сильные мужи их: войско их \bibemph{состояло} из ста семидесяти тысяч ратников, воинов пеших, и из двенадцати тысяч конных, кроме обоза и пеших людей, бывших при них,~--- а и этих было многое множество.
\vs Jdt 7:3 Остановившись на долине близ Ветилуи при источнике, они протянулись в ширину от Дофаима до Велфема, а в длину от Ветилуи до Киамона, лежащего против Ездрилона.
\vs Jdt 7:4 Сыны же Израиля, увидев множество их, очень смутились, и каждый говорил ближнему своему: теперь они опустошат всю землю, и ни высокие горы, ни долины, ни холмы не выдержат их тяжести.
\vs Jdt 7:5 И, взяв каждый свое боевое оружие и зажегши огни на башнях своих, они всю эту ночь провели на страже.
\rsbpar\vs Jdt 7:6 На другой день Олоферн вывел всю свою конницу пред лице сынов Израилевых, бывших в Ветилуе,
\vs Jdt 7:7 осмотрел восходы города их, обошел и занял источники вод их и, оцепив их ратными мужами, возвратился к своему народу.
\vs Jdt 7:8 Но пришли к нему все начальники сынов Исава, и все вожди народа Моавитского, и все военачальники приморья и сказали:
\vs Jdt 7:9 выслушай, господин наш, слово, чтобы не было потери в войске твоем.
\vs Jdt 7:10 Этот народ сынов Израиля надеется не на копья свои, но на высоты гор своих, на которых живут, потому что неудобно восходить на вершины их гор.
\vs Jdt 7:11 Итак, господин, не воюй с ним так, как бывает обыкновенная война,~--- и ни один муж не падет из народа твоего.
\vs Jdt 7:12 Ты останься в своем лагере, чтобы сберечь каждого мужа в войске твоем, а рабы твои пусть овладеют источником воды, который вытекает из подошвы горы;
\vs Jdt 7:13 потому что оттуда берут воду все жители Ветилуи,~--- и погубит их жажда, и они сдадут свой город; а мы с нашим народом взойдем на ближние вершины гор и расположимся на них для стражи, чтобы ни один человек не вышел из города.
\vs Jdt 7:14 И будут томиться они голодом, и жены их и дети их, и прежде, нежели коснется их меч, падут на улицах обиталища своего;
\vs Jdt 7:15 и ты воздашь им злом за то, что они возмутились и не встретили тебя с миром.
\vs Jdt 7:16 Понравились эти слова их Олоферну и всем слугам его, и он решил поступить так, как они сказали.
\vs Jdt 7:17 И двинулся полк сынов Аммона и с ними пять тысяч сынов Ассура и, расположившись в долине, овладели водами и источниками вод сынов Израиля.
\vs Jdt 7:18 А сыны Исава и сыны Аммона взошли и заняли нагорную область против Дофаима, и отправили \bibemph{часть} их на юг и на восток против Екревиля, что близ Хуса, стоящего при потоке Мохмур; остальное же Ассирийское войско расположилось на равнине и покрыло все лице земли: шатры и обозы их с множеством народа растянулись на весьма большом пространстве.
\rsbpar\vs Jdt 7:19 Сыны Израиля воззвали к Господу Богу своему, потому что они пришли в уныние, так как все враги их окружили их, и им нельзя было бежать от них.
\vs Jdt 7:20 Вокруг них стояло все войско Ассирийское,~--- пешие, колесницы и конница их,~--- тридцать четыре дня; у всех жителей Ветилуи истощились все сосуды с водою,
\vs Jdt 7:21 опустели водоемы, и ни в один день они не могли пить воды досыта, потому что давали им пить мерою.
\vs Jdt 7:22 И уныли дети их и жены их и юноши, и в изнеможении от жажды падали на улицах города и в проходах ворот, и уже не было в них крепости.
\vs Jdt 7:23 \bibemph{Тогда} весь народ собрался к Озии и к начальникам города,~--- юноши, жены и дети,~--- и с громким воплем говорили всем старейшинам:
\vs Jdt 7:24 суди Бог между нами и вами; вы сделали нам великую неправду, потому что не предложили мира сынам Ассура;
\vs Jdt 7:25 и теперь нет нам помощника: Бог предал нас в их руки, чтобы погубить нас жаждою и великою погибелью.
\vs Jdt 7:26 Пригласите же их теперь и отдайте весь город на разграбление народу Олоферна и всему войску его,
\vs Jdt 7:27 ибо лучше для нас достаться им на расхищение: хотя мы будем рабами их, зато жива будет душа наша, и глаза наши не увидят смерти младенцев наших и жен и детей наших, расстающихся с душами своими.
\vs Jdt 7:28 Призываем пред вами во свидетели небо и землю, Бога нашего и Господа отцов наших, Который наказывает нас за грехи наши и за грехи отцов наших, да соделает по словам сим в нынешний день.
\vs Jdt 7:29 И подняли они единодушно великий плач среди собрания и громко взывали к Господу Богу.
\vs Jdt 7:30 Озия сказал им: не унывайте, братья! потерпим еще пять дней, в которые Господь Бог наш обратит милость Свою на нас, ибо Он не оставит нас вконец.
\vs Jdt 7:31 Если же они пройдут, и помощь к нам не придет,~--- я сделаю по вашим словам.
\vs Jdt 7:32 И отпустил народ в свой стан, и они пошли на стены и башни своего города, а жен и детей отослал по домам их; и в великой скорби оставались они в городе.
\vs Jdt 8:1 В эти дни услышала Иудифь, дочь Мерарии, сына Окса, сына Иосифа, сына Озиила, сына Елкия, сына Анании, сына Гедеона, сына Рафаина, сына Акифона, сына Илия, сына Елиава, сына Нафанаила, сына Саламиила, сына Саласадая, сына Иеиля.
\vs Jdt 8:2 Муж ее Манассия, из одного с нею колена и племени, умер во время жатвы ячменя;
\vs Jdt 8:3 потому что, когда он стоял в поле близ вязавших снопы, зной пал на его голову,~--- и он слег в постель и умер в своем городе Ветилуе; его похоронили с отцами его на поле между Дофаимом и Валамоном.
\vs Jdt 8:4 И вдовствовала Иудифь в своем доме три года и четыре месяца.
\vs Jdt 8:5 Она сделала для себя на кровле дома своего шатер, возложила на чресла свои вретище, и были на ней одежды вдовства ее.
\vs Jdt 8:6 Она постилась все дни вдовства своего, кроме дней пред субботами и суббот, дней пред новомесячиями и новомесячий, и праздников и торжеств дома Израилева.
\vs Jdt 8:7 Она была красива видом и весьма привлекательна взором; муж ее Манассия оставил ей золото и серебро, слуг и служанок, скот и поля, чем она и владела.
\vs Jdt 8:8 И никто не укорял ее злым словом, потому что она была очень богобоязненна.
\vs Jdt 8:9 Услышала она о дурных речах народа против начальника, потому что они малодушествовали по причине оскудения воды, услышала Иудифь и о всех словах, которые сказал им Озия, как он поклялся им чрез пять дней сдать город Ассириянам,
\vs Jdt 8:10 и послала она служанку свою, распоряжавшуюся всем ее имуществом, пригласить Озию, Хаврина и Хармина, старейшин ее города.
\rsbpar\vs Jdt 8:11 Они пришли,~--- и она сказала им: выслушайте меня, начальники жителей Ветилуи! неправо слово ваше, которое вы сегодня сказали перед народом, и положили клятву, которую изрекли между Богом и вами, и сказали, что сдадите город нашим врагам, если на этих \bibemph{днях} Господь не поможет нам.
\vs Jdt 8:12 Кто же вы, искушавшие сегодня Бога и ставшие вместо Бога посреди сынов человеческих?
\vs Jdt 8:13 Вот, вы теперь испытуете Господа Вседержителя, но никогда ничего не узнаете;
\vs Jdt 8:14 потому что вам не постигнуть глубины сердца у человека и не понять слов мысли его: как же испытаете вы Бога, сотворившего все это, и познаете ум Его, и поймете мысль Его? Нет, братья, не прогневляйте Господа, Бога нашего!
\vs Jdt 8:15 Ибо если Он не захочет помочь нам в эти пять дней, то Он имеет власть защитить нас в какие угодно Ему дни, или поразить нас пред лицем врагов наших.
\vs Jdt 8:16 Не отдавайте же в залог советов Господа Бога нашего: Богу нельзя грозить, как человеку, нельзя и указывать Ему, как сыну человеческому.
\vs Jdt 8:17 Посему, ожидая от Него спасения, будем призывать Его к себе на помощь, и Он услышит голос наш, если это Ему будет угодно.
\vs Jdt 8:18 Ибо не было в родах наших, и нет в настоящее время ни колена, ни племени, ни народа, ни города у нас, которые кланялись бы богам рукотворенным, как было в прежние дни,
\vs Jdt 8:19 за что отцы наши преданы были мечу и расхищению и пали великим падением пред нашими врагами.
\vs Jdt 8:20 Но мы не знаем другого Бога, кроме Его, а потому и надеемся, что Он не презрит нас и никого из нашего рода.
\vs Jdt 8:21 Ибо с пленением нас падет и вся Иудея, и святыни наши будут разграблены, и Он взыщет осквернение их от уст наших,
\vs Jdt 8:22 и убиение братьев наших и пленение земли и опустошение наследия нашего обратит на нашу голову среди народов, которым мы будем порабощены, и будем в соблазн и поношение у тех, которые овладеют нами;
\vs Jdt 8:23 потому что рабство не послужит нам в честь, но Господь, Бог наш, вменит его в бесчестие.
\vs Jdt 8:24 Итак, братья, покажем братьям нашим, что от нас зависит жизнь их, и на нас утверждаются и святыни, и дом \bibemph{Господень}, и жертвенник.
\vs Jdt 8:25 За все это возблагодарим Господа, Бога нашего, Который испытует нас, как и отцов наших.
\vs Jdt 8:26 Вспомните, чт\acc{о} Он сделал с Авраамом, чем искушал Исаака, чт\acc{о} было с Иаковом в Сирской Месопотамии, когда он пас овец Лавана, брата матери своей:
\vs Jdt 8:27 как их искушал Он не для истязания сердца их, так и нам не мстит, а только для вразумления наказывает Господь приближающихся к Нему.
\vs Jdt 8:28 Озия сказал ей: все, что ты сказала, сказала от доброго сердца, и никто не будет противиться словам твоим,
\vs Jdt 8:29 ибо не с настоящего только дня известна мудрость твоя, но от начала дней твоих весь народ знает разум твой и доброе расположение твоего сердца.
\vs Jdt 8:30 Но народ истомился от жажды и принудил нас поступить так, как мы сказали им, и обязал нас клятвою, которой мы не нарушим.
\vs Jdt 8:31 Помолись же о нас, ибо ты жена благочестивая, и Господь пошлет дождь для наполнения водохранилищ наших, и мы больше не будем изнемогать \bibemph{от жажды}.
\vs Jdt 8:32 Иудифь сказала им: послушайте меня,~--- и я совершу дело, которое пронесется сынами рода нашего в роды родов.
\vs Jdt 8:33 Станьте в эту ночь у ворот,~--- а я выйду с моею служанкою, и в продолжение дней, после которых вы решили отдать город нашим врагам, Господь посетит Израиля моею рукою.
\vs Jdt 8:34 Только не расспрашивайте о моем предприятии, потому что я не скажу вам, доколе не совершится то, что я намерена сделать.
\vs Jdt 8:35 И сказал ей Озия и начальники: ступай с миром, и Господь Бог пред тобою на отмщение врагам нашим!
\vs Jdt 8:36 И вышли из шатра ее и пошли к полкам своим.
\vs Jdt 9:1 А Иудифь пала на лице, посыпала голову свою пеплом и сбросила с себя вретище, в которое была одета; и только что воскурили в Иерусалиме, в доме Господнем, вечерний фимиам, Иудифь громким голосом воззвала к Господу и сказала:
\vs Jdt 9:2 Господи Боже отца моего Симеона, которому Ты дал в руку меч на отмщение иноплеменным, которые открыли ложесна девы для оскорбления, обнажили бедро для позора и осквернили ложесна для посрамления! Ты сказал: да не будет сего, а они сделали.
\vs Jdt 9:3 И за то Ты предал князей их на убиение, постель их, которая видела обольщение их, обагрил кровью и поразил рабов подле владетелей и владетелей на тронах их,
\vs Jdt 9:4 и отдал жен их в расхищение, дочерей их в плен и всю добычу в раздел сынам, возлюбленным Тобою, которые возревновали Твоею ревностью, возгнушались осквернением крови их, и призвали Тебя на помощь. Боже, Боже мой, услышь меня вдову!
\vs Jdt 9:5 Ты сотворил прежде сего бывшее, и сие и последующее за сим, и содержал в уме настоящее и грядущее, и, что помыслил Ты, то и совершилось;
\vs Jdt 9:6 что определил, то и явилось и сказало: вот я. Ибо все пути Твои готовы, и суд Твой \bibemph{Тобою} предвиден.
\vs Jdt 9:7 Вот, Ассирияне умножились в силе своей, гордятся конем и всадником, тщеславятся мышцею пеших, надеются и на щит и на копье и на лук и на пращу, а не знают того, что Ты~--- Господь, сокрушающий брани.
\vs Jdt 9:8 Господь~--- имя Тебе; сокруши же их крепость силою Твоею, и уничтожь их силу гневом Твоим, ибо они замыслили осквернить святилище Твое, поругаться над мирным селением имени славы Твоей и железом сокрушить рог Твоего жертвенника.
\vs Jdt 9:9 Воззри на превозношение их, пошли гнев Твой на главы их, дай вдовьей руке моей крепость на то, что задумала я.
\vs Jdt 9:10 Устами хитрости моей порази раба перед вождем, и вождя~--- перед рабом его, \bibemph{и} сокруши гордыню их рукою женскою;
\vs Jdt 9:11 ибо не во множестве сила Твоя и не в могучих могущество Твое; но Ты~--- Бог смиренных, Ты~--- помощник умаленных, заступник немощных, покровитель упавших духом, спаситель безнадежных.
\vs Jdt 9:12 Так, так, Боже отца моего и Боже наследия Израилева, Владыка неба и земли, Творец вод, Царь всякого создания Твоего! Услышь молитву мою,
\vs Jdt 9:13 сделай слово мое и хитрость мою раною и язвою для тех, которые задумали жестокое против завета Твоего, святаго дома Твоего, высоты Сиона и дома наследия сынов Твоих.
\vs Jdt 9:14 Вразуми весь народ Твой и всякое племя, чтобы видели они, что Ты~--- Бог, Бог всякой крепости и силы, и нет другого защитника рода Израилева, кроме Тебя.
\vs Jdt 10:1 Когда она перестала взывать к Богу Израилеву и окончила все эти слова
\vs Jdt 10:2 то поднялась на ноги, позвала служанку свою и вошла в дом, в котором она проводила субботние дни и праздники свои.
\vs Jdt 10:3 Здесь она сняла с себя вретище, которое надевала, сняла и одежды вдовства своего, омыла тело водою и намастилась драгоценным миром, причесала волосы и надела на голову повязку, оделась в одежды веселия своего, в которые она наряжалась во дни жизни мужа своего Манассии;
\vs Jdt 10:4 обула ноги свои в сандалии, и возложила на себя цепочки, запястья, кольца, серьги и все свои наряды, и разукрасила себя, чтобы прельстить глаза мужчин, которые увидят ее.
\vs Jdt 10:5 И дала служанке своей мех вина и сосуд масла, наполнила мешок мукою и сушеными плодами и чистыми хлебами и, обвернув все эти припасы свои, возложила их на нее.
\rsbpar\vs Jdt 10:6 Выйдя к воротам города Ветилуи, они нашли стоявшими при них Озию и старейшин города, Хаврина и Хармина.
\vs Jdt 10:7 Когда они увидели ее и перемену в ее лице и одежде, очень много дивились красоте ее и сказали ей:
\vs Jdt 10:8 Бог, Бог отцов наших, да даст тебе благодать и да совершит твои намерения на радость сынов Израиля и на возвеличение Иерусалима. Она поклонилась Богу
\vs Jdt 10:9 и сказала им: велите отворить для меня ворота города; я выйду для исполнения дела, о котором вы говорили со мною. И велели юношам отворить для нее, как она сказала.
\vs Jdt 10:10 Они исполнили это. И вышла Иудифь и служанка ее с нею; а мужи городские смотрели вслед за нею, пока она сходила с горы, пока проходила долиной и пока не скрылась от их глаз.
\vs Jdt 10:11 Они шли прямо долиною, и встретила \bibemph{Иудифь} передовая стража Ассириян,
\vs Jdt 10:12 и взяли ее и спросили: чья ты, откуда идешь и куда отправляешься? Она сказала: я дочь Евреев и бегу от них, потому что они будут преданы вам на истребление.
\vs Jdt 10:13 Я иду к Олоферну, вождю вашего войска, возвестить слова истины и указать ему путь, которым он пойдет и овладеет всею нагорною страною, так что не погибнет из мужей его ни один человек и ни одна живая душа.
\vs Jdt 10:14 Когда эти люди слушали слова ее и всматривались в лице ее,~--- она показалась им чудом по красоте, и они сказали ей:
\vs Jdt 10:15 ты спасла душу твою, поспешив прийти к господину нашему; ступай же к шатру его, а наши проводят тебя, пока не передадут тебя ему на руки.
\vs Jdt 10:16 Когда ты станешь перед ним,~--- не бойся сердцем твоим, но выскажи слова твои, и он тебя облагодетельствует.
\vs Jdt 10:17 И, выбрав из среды своей сто человек, приставили их к ней и к служанке ее, и они повели их к шатру Олоферна.
\vs Jdt 10:18 Во всем стане произошло движение, потому что весть о приходе ее разнеслась по шатрам: сбежавшиеся окружили ее, так как она стояла вне шатра Олоферна, пока не возвестили ему о ней;
\vs Jdt 10:19 и дивились красоте ее, а из-за нее дивились и сынам Израиля, и говорили каждый ближнему своему: кто пренебрежет таким народом, который имеет таких жен у себя! Неблагоразумно оставить из них ни одного мужа, потому что оставшиеся будут в состоянии перехитрить всю землю.
\vs Jdt 10:20 \bibemph{Между тем} спавшие при Олоферне и все служители его вышли и ввели ее в шатер.
\vs Jdt 10:21 Олоферн отдыхал на своей постели за занавесом, украшенным пурпуром, золотом, изумрудом и драгоценными камнями.
\vs Jdt 10:22 \bibemph{Когда} ему доложили о ней, он вышел в переднее отделение шатра, и перед ним несли серебряные лампады.
\vs Jdt 10:23 Когда Иудифь представилась ему и служителям его, все удивились красоте лица ее. Она, пав на лице, поклонилась ему, и служители его подняли ее.
\vs Jdt 11:1 Олоферн сказал ей: ободрись, жена; не бойся сердцем твоим, потому что я не сделал зла никому, кто добровольно решился служить Навуходоносору, царю всей земли.
\vs Jdt 11:2 И теперь, если бы народ твой, живущий в нагорной стране, не пренебрег мною, я не поднял бы на них копья моего; но они сами это сделали для себя.
\vs Jdt 11:3 Скажи же мне: почему ты бежала от них и пришла к нам? Ты найдешь себе \bibemph{здесь} спасение; не бойся: ты будешь жива в эту ночь и после,
\vs Jdt 11:4 потому что тебя никто не обидит, напротив, всякий будет благодетельствовать тебе, как бывает с рабами господина моего, царя Навуходоносора.
\vs Jdt 11:5 Иудифь сказала ему: выслушай слова рабы твоей; пусть раба говорит пред лицем твоим: я не скажу лжи господину моему в эту ночь.
\vs Jdt 11:6 И если ты последуешь словам рабы твоей, то Бог чрез тебя совершит дело, и господин мой не ошибется в своих предприятиях.
\vs Jdt 11:7 Да живет Навуходоносор, царь всей земли, и да живет держава его, пославшего тебя для исправления всякой души, потому что не только люди чрез тебя будут служить ему, но и звери полевые, и скот, и птицы небесные чрез твою силу будут жить под властью Навуходоносора и всего дома его.
\vs Jdt 11:8 Ибо мы слышали о твоей мудрости и хитрости ума твоего, и всей земле известно, что ты один добр во всем царстве, силен в знании и дивен в воинских подвигах.
\vs Jdt 11:9 А что говорил Ахиор в собрании твоем, мы слышали слова его, потому что мужи Ветилуи оставили его в живых, и он рассказал им все, о чем говорил тебе.
\vs Jdt 11:10 Посему, владыка-господин, не оставляй без внимания сл\acc{о}ва его, но сложи его в сердце твоем, потому что оно истинно: род наш не наказывается, меч не имеет силы над нами, если они не грешат пред Богом своим.
\vs Jdt 11:11 Итак, чтобы господин мой не был отражен и безуспешен и чтобы их постигла смерть,~--- овладел ими грех, которым они прогневляют Бога своего, делая то, чего не следует;
\vs Jdt 11:12 потому что у них оказался недостаток в пище и вся вода истощилась,~--- и \bibemph{вот}, они решились броситься на скот свой и думают питаться всем, что Бог строго запретил в законе Своем употреблять в пищу.
\vs Jdt 11:13 Даже начатки пшеницы и десятины вина и масла, которые, по освящении, хранятся для священников, предстоящих пред лицем Бога нашего в Иерусалиме, они решились употребить, тогда как и руками касаться их не следовало никому из народа.
\vs Jdt 11:14 Они послали в Иерусалим, так как и тамошние жители делали это, принести к ним разрешение на то от собрания старейшин.
\vs Jdt 11:15 И как скоро им дано будет известие, и они сделают это, то в тот же день будут преданы тебе на погубление.
\vs Jdt 11:16 Вот почему я, раба твоя, узнав обо всем этом, бежала от них, и Бог послал меня сделать вместе с тобою такие дела, которым изумится вся земля, где только услышат о них,
\vs Jdt 11:17 ибо раба твоя благочестива и день и ночь служит Богу Небесному. Теперь, господин мой, я останусь у тебя; только пусть раба твоя по ночам выходит на долину молиться Богу,~--- и Он откроет мне, когда они сделают свое преступление.
\vs Jdt 11:18 Я приду и объявлю тебе, и ты выходи \bibemph{тогда} со всем твоим войском,~--- и никто из них не противостанет тебе.
\vs Jdt 11:19 Я поведу тебя чрез Иудею, доколе не дойдем до Иерусалима; поставлю среди его седалище твое, и ты погонишь их, как овец, не имеющих пастуха,~--- и пес не пошевелит против тебя языком своим. Это сказано мне по откровению и объявлено мне, и я послана возвестить тебе.
\vs Jdt 11:20 Понравились слова ее Олоферну и всем слугам его. Они дивились мудрости ее и говорили:
\vs Jdt 11:21 от края до края земли нет такой жены по красоте лица и по разумным речам.
\vs Jdt 11:22 Олоферн сказал ей: хорошо Бог сделал, что вперед этого народа послал тебя, чтобы в руках наших была сила, а среди презревших господина моего~--- гибель.
\vs Jdt 11:23 Прекрасна ты лицем, и добры речи твои. Если ты сделаешь, как сказала, то твой Бог будет моим Богом; ты будешь жить в доме царя Навуходоносора и будешь именита во всей земле.
\vs Jdt 12:1 И приказал ввести ее \bibemph{туда}, где хранились серебряные сосуды его, и велел ей пользоваться пищею от стола его и пить вино его.
\vs Jdt 12:2 Но Иудифь сказала: не буду есть этого, чтобы не было соблазна, но пусть подают мне то, что принесено со мною.
\vs Jdt 12:3 Олоферн сказал ей: а когда истощится то, что с тобою, откуда мы возьмем, чтобы подавать тебе подобное этому? Ибо среди нас нет никого из рода твоего.
\vs Jdt 12:4 Иудифь отвечала ему: да живет душа твоя, господин мой; раба твоя не издержит того, что со мною, прежде, нежели Господь совершит моею рукою то, что Он определил.
\vs Jdt 12:5 И ввели ее слуги Олоферна в шатер, и спала она до полночи; а пред утреннею стражею встала
\vs Jdt 12:6 и послала сказать Олоферну: да даст господин мой повеление, чтобы рабе твоей дозволили выходить на молитву.
\vs Jdt 12:7 Олоферн приказал своим телохранителям не препятствовать ей. И пробыла она в лагере три дня, а по ночам выходила в долину Ветилуи, омывалась при источнике воды у лагеря.
\vs Jdt 12:8 И, выходя, молилась Господу, Богу Израилеву, чтоб Он направил путь ее к избавлению сынов Его народа.
\vs Jdt 12:9 По возвращении она пребывала в шатре чистою, а к вечеру приносили ей пищу.
\vs Jdt 12:10 В четвертый день Олоферн сделал пир для одних слуг своих и не пригласил к услужению никого из приставленных к службам.
\vs Jdt 12:11 И сказал евнуху Вагою, управлявшему всем, что у него было: ступай и убеди Еврейскую женщину, которая у тебя, прийти к нам и есть и пить с нами:
\vs Jdt 12:12 стыдно нам оставить такую жену, не побеседовав с нею; она осмеет нас, если мы не пригласим ее.
\vs Jdt 12:13 Вагой, выйдя от Олоферна, пришел к ней и сказал: не откажись, прекрасная молодая женщина, прийти к господину моему, чтобы принять честь пред лицем его и пить с нами вино в веселие и быть в этот день как одною из дочерей сынов Ассура, которые предстоят в доме Навуходоносора.
\vs Jdt 12:14 Иудифь сказала ему: кто я, чтобы прекословить господину моему? поспешу исполнить все, что будет угодно господину моему, и это будет служить мне утешением до дня смерти моей.
\vs Jdt 12:15 Она встала и нарядилась в одежду и во все женское украшение; а служанка ее пришла и разостлала для нее по земле пред Олоферном ковры, которые она получила от Вагоя для всегдашнего употребления, чтобы есть, возлежа на них.
\vs Jdt 12:16 Затем Иудифь пришла и возлегла. Подвиглось сердце Олоферна к ней, и душа его взволновалась: он сильно желал сойтись с нею и искал случая обольстить ее с того самого дня, как увидел ее.
\vs Jdt 12:17 И сказал ей Олоферн: пей же и веселись с нами.
\vs Jdt 12:18 А Иудифь сказала: буду пить, господин, потому что сегодня жизнь моя возвеличилась во мне больше, нежели во все дни от рождения моего.
\vs Jdt 12:19 И она брала, ела и пила пред ним, что приготовила служанка ее.
\vs Jdt 12:20 А Олоферн любовался на нее и пил вина весьма много, сколько не пил никогда, ни в один день от рождения.
\vs Jdt 13:1 Когда поздно стало, рабы его поспешили удалиться, а Вагой, отпустив предстоявших пред лицем его господина, затворил шатер снаружи, и они пошли к постелям своим, так как все были утомлены продолжительностью пира.
\vs Jdt 13:2 В шатре осталась одна Иудифь с Олоферном, упавшим на ложе свое, потому что был переполнен вином.
\vs Jdt 13:3 Иудифь велела служанке своей стать вне спальни ее и ожидать ее выхода, как было каждый день, сказав, что она выйдет на молитву. То же самое сказала она и Вагою.
\vs Jdt 13:4 \bibemph{Когда} все от нее ушли и никого в спальне не осталось, ни малого, ни большого, Иудифь, став у постели \bibemph{Олоферна}, сказала в сердце своем: Господи, Боже всякой силы! призри в час сей на дела рук моих к возвышению Иерусалима,
\vs Jdt 13:5 ибо теперь время защитить наследие Твое и исполнить мое намерение, поразить врагов, восставших на нас.
\vs Jdt 13:6 \bibemph{Потом}, подойдя к столбику постели, стоявшему в головах у Олоферна, она сняла с него меч его
\vs Jdt 13:7 и, приблизившись к постели, схватила волосы головы его и сказала: Господи, Боже Израиля! укрепи меня в этот день.
\vs Jdt 13:8 И изо всей силы дважды ударила по шее \bibemph{Олоферна} и сняла с него голову
\vs Jdt 13:9 и, сбросив с постели тело его, взяла со столбов занавес. Спустя немного она вышла и отдала служанке своей голову Олоферна,
\vs Jdt 13:10 а эта положила ее в мешок со съестными припасами, и обе вместе вышли, по обычаю своему, на молитву. Пройдя стан, они обошли кругом ущелье, поднялись на гору Ветилуи и пошли к воротам ее.
\vs Jdt 13:11 Иудифь издали кричала сторожившим при воротах: отворите, отворите ворота! с нами Бог, Бог наш, чтобы даровать еще силу Израилю и победу над врагами, как даровал Он и сегодня.
\vs Jdt 13:12 Как только услышали городские мужи голос ее, поспешили прийти к городским воротам и созвали старейшин города.
\vs Jdt 13:13 И сбежались все, от малого до большого, так как приход ее был для них сверх ожидания, и, отворив ворота, приняли их, и, зажегши для освещения огонь, окружили их.
\vs Jdt 13:14 Она же сказала им громким голосом: хвалите Господа, хвалите, хвалите Господа, что Он не удалил милости Своей от дома Израилева, но в эту ночь сокрушил врагов наших моею рукою.
\vs Jdt 13:15 И, вынув голову из мешка, показала ее и сказала им: вот голова Олоферна, вождя Ассирийского войска, и вот занавес его, за которым он лежал от опьянения,~--- и Господь поразил его рукою женщины.
\vs Jdt 13:16 Жив Господь, сохранивший меня в пути, которым я шла! ибо лице мое прельстило \bibemph{Олоферна} на погибель его, но он не сделал со мною скверного и постыдного греха.
\vs Jdt 13:17 Весь народ чрезвычайно изумился; пали, поклонились Богу и единодушно сказали: благословен Ты, Боже наш, уничиживший сегодня врагов народа Твоего!
\vs Jdt 13:18 А Озия сказал ей: благословенна ты, дочь, Всевышним Богом более всех жен на земле, и благословен Господь Бог, создавший небеса и землю и наставивший тебя на поражение головы начальника наших врагов;
\vs Jdt 13:19 ибо надежда твоя не отступит от сердца людей, помнящих силу Божию, до века.
\vs Jdt 13:20 Да вменит тебе это Бог в вечную славу и да наградит тебя благами за то, что ты жизни твоей не пощадила при унижении рода нашего, но выступила вперед, когда мы падали, ты, право ходившая пред Богом нашим. И весь народ сказал: да будет, да будет!
\vs Jdt 14:1 Иудифь сказала им: послушайте же меня, братья, возьмите эту голову и повесьте на зубцах вашей стены.
\vs Jdt 14:2 Когда же настанет утро и солнце взойдет над землею, возьмите каждый боевое свое оружие, идите все сильные за город и дайте им вождя, как будто намереваясь сойти на равнину против передовой стражи сынов Ассура, но не сходите.
\vs Jdt 14:3 Тогда они, взяв все свое оружие, пойдут в свой стан, разбудят вождей войска Ассирийского, и сбегутся к шатру \bibemph{Олоферна}, но не найдут его; оттого нападет на них страх, и они побегут от вас.
\vs Jdt 14:4 А вы и все живущие во всяком пределе Израильском, преследуя их, поражайте их на пути.
\vs Jdt 14:5 Но прежде, чем сделаете это, пригласите ко мне Ахиора Аммонитянина: пусть увидит и узнает он того, кто уничижал дом Израиля и прислал его к нам будто на смерть.
\rsbpar\vs Jdt 14:6 И призвали Ахиора из дома Озии. Когда он пришел и увидел голову Олоферна в руке одного мужа среди собрания народа, то пал на лице свое и ослабел духом.
\vs Jdt 14:7 Когда же подняли его, он припал к ногам Иудифи, поклонился ей и сказал: благословенна ты во всяком селении Иуды и во всяком народе, которые, услышав об имени твоем, изумятся.
\vs Jdt 14:8 Расскажи же мне теперь, что ты делала в эти дни? И Иудифь среди народа рассказала ему все, что она сделала с того дня, как вышла, до того дня, в который говорила с ними.
\vs Jdt 14:9 Когда она перестала говорить, народ громко воскликнул, и радостный крик его раздался в городе.
\vs Jdt 14:10 Ахиор же, видя все, что сделал Бог Израилев, искренно уверовал в Бога, обрезал крайнюю плоть свою и присоединился к дому Израилеву, даже до сего дня.
\rsbpar\vs Jdt 14:11 Когда настало утро, повесили голову Олоферна на стену; каждый муж взял свое оружие, и вышли отрядами на всходы горы.
\vs Jdt 14:12 Сыны Ассура, увидев их, послали к своим начальникам, а они пошли к вождям, к тысяченачальникам и ко всякому предводителю своему.
\vs Jdt 14:13 Придя к шатру Олоферна, они сказали управлявшему всем имением его: разбуди нашего господина, потому что эти рабы осмелились выйти на сражение с нами, чтобы быть совершенно истребленными.
\vs Jdt 14:14 Вагой вошел и постучался в дверь шатра, ибо думал, что он спит с Иудифью.
\vs Jdt 14:15 Когда же никто не отзывался ему, то, отворив, вошел в спальню и нашел, что \bibemph{Олоферн} мертвый лежит у порога и голова его снята с него.
\vs Jdt 14:16 И он громко воскликнул с плачем, стоном и крепким воплем, и разорвал свои одежды.
\vs Jdt 14:17 Потом вошел в шатер, в котором пребывала Иудифь, и не нашел ее. Тогда он выскочил к народу и закричал:
\vs Jdt 14:18 рабы поступили вероломно; одна Еврейская жена опозорила дом царя Навуходоносора, ибо вот Олоферн на полу и головы нет на нем.
\vs Jdt 14:19 Когда услышали эти слова начальники войска Ассирийского, то разорвали одежды свои, и душа их сильно смутилась, и раздался у них крик и весьма великий вопль среди стана.
\vs Jdt 15:1 Когда бывшие в шатрах услышали о том, что случилось, то смутились,
\vs Jdt 15:2 и напал на них страх и трепет, и ни один из них не остался в глазах ближнего, но все они бросившись бежали по всем дорогам равнины и нагорной страны.
\vs Jdt 15:3 И расположившиеся лагерем в нагорной стране около Ветилуи также обратились в бегство. Тогда сыны Израиля, каждый из них воинственный муж, погнались за ними.
\vs Jdt 15:4 Озия послал в Ветомасфем, Виваю, Ховаю и Холу и во все пределы Израильские, чтобы известить о совершившемся и чтобы все погнались за неприятелями для истребления их.
\vs Jdt 15:5 Как скоро услышали об этом сыны Израиля, все дружно напали на них и поражали их до Ховы; равно и пришедшие из Иерусалима и из всей нагорной страны, так как им возвещено было о том, чт\acc{о} случилось в стане врагов их, и из Галаада и Галилеи, со всех сторон наносили им большое поражение, доколе они не прошли за Дамаск и за пределы его.
\vs Jdt 15:6 Прочие жители Ветилуи напали на стан Ассирийский, разграбили его и весьма обогатились.
\vs Jdt 15:7 А сыны Израиля, возвратившиеся от поражения, овладели остальным; и села и деревни в нагорной стране и на равнине получили большую добычу, потому что ее было весьма многое множество.
\vs Jdt 15:8 Великий священник Иоаким и старейшины сынов Израилевых, жившие в Иерусалиме, пришли посмотреть, какое благо сотворил Господь для Израиля, и видеть Иудифь и приветствовать ее.
\vs Jdt 15:9 Как только они вошли к ней, то все единодушно благословили ее и сказали ей: ты величие Израиля, ты великая радость Израиля, ты великая слава нашего рода.
\vs Jdt 15:10 Все это ты сделала твоею рукою; ты сделала добро Израилю, и да благоволит к нему Бог; будь \bibemph{же} благословенна от Господа Вседержителя на вечное время. И весь народ сказал: да будет!
\rsbpar\vs Jdt 15:11 Народ расхищал лагерь в продолжение тридцати дней, и Иудифи отдали шатер Олоферна и все серебряные сосуды и постели и чаши и всю утварь его. Она взяла, возложила на мула своего, запрягла колесницы свои и сложила это на них.
\vs Jdt 15:12 И сбежались все жены Израильские видеть ее, и благословляли ее и составили из себя для нее хор; а она взяла в свои руки обвитые виноградными листьями жезлы и дала женщинам, бывшим с нею,
\vs Jdt 15:13 и возложили на себя масличные венки~--- она и бывшие с нею. Она шла впереди всего народа в хоре и вела за собою всех жен; за нею следовали все мужи Израильские, вооруженные, с венками и с торжественными песнями в своих устах.
\vs Jdt 15:14 Иудифь начала пред всем Израилем благодарственную песнь, и весь народ подпевал эту песнь.
\vs Jdt 16:1 И сказала Иудифь: начните Богу моему на тимпанах, пойте Господу моему на кимвалах, стройно воспевайте Ему новую песнь, возносите и призывайте имя Его;
\vs Jdt 16:2 потому что Он есть Бог Господь, сокрушающий брани, потому что Он ополчился за меня среди народа и исторг меня из руки моих преследователей.
\vs Jdt 16:3 Пришел Ассур с гор севера, пришел с мириадами войска своего, и множество их запрудило воду в источниках, и конница их покрыла холмы.
\vs Jdt 16:4 Он сказал, что пределы мои сожжет, юношей моих мечом истребит, грудных младенцев бросит о землю, малых детей моих отдаст на расхищение, дев моих пленит.
\vs Jdt 16:5 Но Господь Вседержитель низложил их рукою жены.
\vs Jdt 16:6 Не от юношей пал сильный их, не сыны титанов поразили его, и не рослые исполины налегли на него, но Иудифь, дочь Мерарии, красотою лица своего погубила его;
\vs Jdt 16:7 потому что она для возвышения бедствовавших в Израиле сняла с себя одежды вдовства своего, помазала лице свое благовонною мастью,
\vs Jdt 16:8 украсила волосы свои головным убором, надела для прельщения его льняную одежду.
\vs Jdt 16:9 Ее сандалии восхитили взор его, и красота ее пленила душу его; меч прошел по шее его.
\vs Jdt 16:10 Персы ужаснулись отваги ее, и М\acc{и}дяне растерялись от смелости ее.
\vs Jdt 16:11 Тогда воскликнули смиренные мои,~--- и они испугались; немощные мои,~--- и они пришли в смущение; возвысили голос свой,~--- и они обратились в бегство.
\vs Jdt 16:12 Сыновья молодых жен кололи их и, как детям беглых рабов, наносили им раны; они погибли от ополчения Господа моего.
\vs Jdt 16:13 Воспою Господу моему песнь новую. Велик Ты, Господи, и славен, дивен силою и непобедим!
\vs Jdt 16:14 Да работает Тебе всякое создание Твое: ибо Ты сказал,~--- и совершилось; Ты послал Духа Твоего,~--- и устроилось,~--- и нет \bibemph{никого}, кто противостал бы гласу Твоему.
\vs Jdt 16:15 Горы с водами подвигнутся с оснований, и камни, как воск, растают от лица Твоего, но к боящимся Тебя Ты благомилостив.
\vs Jdt 16:16 Мала всякая жертва для вон\acc{и} благоухания, и всякий тук ничтожен для всесожжения Тебе, но боящийся Господа всегда велик.
\vs Jdt 16:17 Горе народам, восстающим на род мой: Господь Вседержитель отмстит им в день суда, пошлет огонь и червей на их тела,~--- и они будут чувствовать \bibemph{боль} и плакать вечно.
\rsbpar\vs Jdt 16:18 Когда пришли в Иерусалим, они поклонились Богу, и, когда народ очистился, вознесли всесожжения свои и доброхотные \bibemph{жертвы} свои и дары свои.
\vs Jdt 16:19 Иудифь же принесла все сосуды Олоферна, которые отдал ей народ, и занавес, который она взяла из спальни его, отдала в жертву Господу.
\vs Jdt 16:20 Народ веселился в Иерусалиме пред святилищем три месяца, и Иудифь пребывала с ними.
\vs Jdt 16:21 Но после сих дней каждый возвратился в удел свой, а Иудифь отправилась в Ветилую, \bibemph{где} оставалась в имении своем, и была в свое время славною во всей земле.
\vs Jdt 16:22 Многие желали ее, но мужчина не познал ее во все дни ее жизни с того дня, как муж ее Манассия умер и приложился к народу своему.
\vs Jdt 16:23 Она приобрела великую славу и состарилась в доме мужа своего, \bibemph{прожив} до ста пяти лет, и отпустила служанку свою на свободу. Она умерла в Ветилуе, и похоронили ее в пещере мужа ее Манассии.
\vs Jdt 16:24 Дом Израиля оплакивал ее семь дней. Имение же свое прежде смерти своей она разделила между родственниками Манассии, мужа своего, и между близкими из рода своего.
\vs Jdt 16:25 И никто более не устрашал сынов Израиля во дни Иудифи и много дней по смерти ее.
\newbookpage
\bibbookdescr{Est}{
  inline={\LARGE Книга\\\Huge Есфирь},
  toc={Есфирь},
  bookmark={Есфирь},
  header={Есфирь},
  %headerleft={},
  %headerright={},
  abbr={Есф}
}
\vs Est 0:0 [Во второй год царствования Артаксеркса великого, в первый день месяца Нисана, сон видел Мардохей, сын Иаиров, Семеев, Кисеев, из колена Вениаминова, Иудеянин, живший в городе Сузах, человек великий, служивший при царском дворце. Он был из пленников, которых Навуходоносор, царь Вавилонский, взял в плен из Иерусалима с Иехониею, царем Иудейским. Сон же его такой: вот ужасный шум, гром и землетрясение и смятение на земле; и вот, вышли два больших змея, готовые драться друг с другом; и велик был вой их, и по вою их все народы приготовились к войне, чтобы поразить народ праведных; и вот~--- день тьмы и мрака, скорбь и стеснение, страдание и смятение великое на земле; и смутился весь народ праведных, опасаясь бед себе, и приготовились они погибнуть и стали взывать к Господу; от вопля их произошла, как бы от малого источника, великая река с множеством воды; и воссиял свет и солнце, и вознеслись смиренные и истребили тщеславных.~--- Мардохей, пробудившись после этого сновидения, \bibemph{изображавшего}, чт\acc{о} Бог хотел совершить, содержал этот сон в сердце и желал уразуметь его во всех частях его, до ночи. И пребывал Мардохей во дворце вместе с Гавафою и Фаррою, двумя царскими евнухами, оберегавшими дворец, и услышал разговоры их и разведал замыслы их и узнал, что они готовятся наложить руки на царя Артаксеркса, и донес о них царю; а царь пытал этих двух евнухов, и, когда они сознались, были казнены. Царь записал это событие на память, и Мардохей записал об этом событии. И приказал царь Мардохею служить во дворце и дал ему подарки за это. При царе же был \bibemph{тогда} знатен Аман, сын Амадафов, Вугеянин, и старался он причинить зло Мардохею и народу его за двух евнухов царских.]
\rsbpar\vs Est 1:1 И было [после сего] во дни Артаксеркса,~--- этот Артаксеркс царствовал над ста двадцатью семью областями от Индии и до Ефиопии,~---
\vs Est 1:2 в то время, как царь Артаксеркс сел на царский престол свой, что в Сузах, городе престольном,
\vs Est 1:3 в третий год своего царствования он сделал пир для всех князей своих и для служащих при нем, для главных начальников войска Персидского и Мидийского и для правителей областей своих,
\vs Est 1:4 показывая великое богатство царства своего и отличный блеск величия своего \bibemph{в течение} многих дней, ста восьмидесяти дней.
\vs Est 1:5 По окончании сих дней, сделал царь для народа своего, находившегося в престольном городе Сузах, от большого до малого, пир семидневный на садовом дворе дома царского.
\vs Est 1:6 Белые, бумажные и яхонтового цвета шерстяные ткани, прикрепленные виссонными и пурпуровыми шнурами, \bibemph{висели} на серебряных кольцах и мраморных столбах.
\vs Est 1:7 Золотые и серебряные ложа \bibemph{были} на помосте, устланном камнями зеленого цвета и мрамором, и перламутром, и камнями черного цвета.
\vs Est 1:8 Напитки подаваемы \bibemph{были} в золотых сосудах и сосудах разнообразных, ценою в тридцать тысяч талантов; и вина царского было множество, по богатству царя. Питье \bibemph{шло} чинно, никто не принуждал, потому что царь дал такое приказание всем управляющим в доме его, чтобы делали по воле каждого.
\vs Est 1:9 И царица Астинь сделала также пир для женщин в царском доме царя Артаксеркса.
\vs Est 1:10 В седьмой день, когда развеселилось сердце царя от вина, он сказал Мегуману, Бизфе, Харбоне, Бигфе и Авагфе, Зефару и Каркасу~--- семи евнухам, служившим пред лицем царя Артаксеркса,
\vs Est 1:11 чтобы они привели царицу Астинь пред лице царя в венце царском для того, чтобы показать народам и князьям красоту ее; потому что она была очень красива.
\vs Est 1:12 Но царица Астинь не захотела прийти по приказанию царя, \bibemph{объявленному} чрез евнухов.
\vs Est 1:13 И разгневался царь сильно, и ярость его загорелась в нем. И сказал царь мудрецам, знающим \bibemph{прежние} времена,~--- ибо дела царя \bibemph{делались} пред всеми знающими закон и прав\acc{а},~---
\vs Est 1:14 приближенными же к нему \bibemph{тогда были}: Каршена, Шефар, Адмафа, Фарсис, Мерес, Марсена, Мемухан~--- семь князей Персидских и Мидийских, которые могли видеть лице царя \bibemph{и} сидели первыми в царстве:
\vs Est 1:15 как поступить по закону с царицею Астинь за то, что она не сделала по слову царя Артаксеркса, \bibemph{объявленному} чрез евнухов?
\vs Est 1:16 И сказал Мемухан пред лицем царя и князей: не пред царем одним виновна царица Астинь, а пред всеми князьями и пред всеми народами, которые по всем областям царя Артаксеркса;
\vs Est 1:17 потому что поступок царицы дойдет до всех жен, и они будут пренебрегать мужьями своими и говорить: царь Артаксеркс велел привести царицу Астинь пред лице свое, а она не пошла.
\vs Est 1:18 Теперь княгини Персидские и Мидийские, которые услышат о поступке царицы, будут \bibemph{то же} говорить всем князьям царя; и пренебрежения и огорчения будет довольно.
\vs Est 1:19 Если благоугодно царю, пусть выйдет от него царское постановление и впишется в законы Персидские и Мидийские и не отменяется, о том, что Астинь не будет входить пред лице царя Артаксеркса, а царское достоинство ее царь передаст другой, которая лучше ее.
\vs Est 1:20 Когда услышат о сем постановлении царя, которое разойдется по всему царству его, как оно ни велико, тогда все жены будут почитать мужей своих, от большого до малого.
\vs Est 1:21 И угодно было слово сие в глазах царя и князей; и сделал царь по слову Мемухана.
\vs Est 1:22 И послал во все области царя письма, писанные в каждую область письменами ее и к каждому народу на языке его, чтобы всякий муж был господином в доме своем, и чтобы это было объявлено каждому на природном языке его.
\vs Est 2:1 После сего, когда утих гнев царя Артаксеркса, он вспомнил об Астинь и о том, что она сделала, и что было определено о ней.
\vs Est 2:2 И сказали отроки царя, служившие при нем: пусть бы поискали царю молодых красивых девиц,
\vs Est 2:3 и пусть бы назначил царь наблюдателей во все области своего царства, которые собрали бы всех молодых девиц, красивых видом, в престольный город Сузы, в дом жен под надзор Гегая, царского евнуха, стража жен, и пусть бы выдавали им притиранья [и прочее, что нужно];
\vs Est 2:4 и девица, которая понравится глазам царя, пусть будет царицею вместо Астинь. И угодно было слово это в глазах царя, и он так и сделал.
\rsbpar\vs Est 2:5 Был в Сузах, городе престольном, один Иудеянин, имя его Мардохей, сын Иаира, сын Семея, сын Киса, из колена Вениаминова.
\vs Est 2:6 Он был переселен из Иерусалима вместе с пленниками, выведенными с Иехониею, царем Иудейским, которых переселил Навуходоносор, царь Вавилонский.
\vs Est 2:7 И был он воспитателем Гадассы,~--- она же Есфирь,~--- дочери дяди его, так как не было у нее ни отца, ни матери. Девица эта была красива станом и пригожа лицем. И по смерти отца ее и матери ее, Мардохей взял ее к себе вместо дочери.
\rsbpar\vs Est 2:8 Когда объявлено было повеление царя и указ его, и когда собраны были многие девицы в престольный город Сузы под надзор Гегая, тогда взята была и Есфирь в царский дом под надзор Гегая, стража жен.
\vs Est 2:9 И понравилась эта девица глазам его и приобрела у него благоволение, и он поспешил выдать ей притиранья и \bibemph{все, назначенное на} часть ее, и приставить к ней семь девиц, достойных быть при ней, из дома царского, и переместил ее и девиц ее в лучшее отделение женского дома.
\vs Est 2:10 Не сказывала Есфирь ни о народе своем, ни о родстве своем, потому что Мардохей дал ей приказание, чтобы она не сказывала.
\vs Est 2:11 И всякий день Мардохей приходил ко двору женского дома, чтобы наведываться о здоровье Есфири и о том, что делается с нею.
\rsbpar\vs Est 2:12 Когда наступало время каждой девице входить к царю Артаксерксу, после того, как в течение двенадцати месяцев выполнено было над нею все, определенное женщинам,~--- ибо столько времени продолжались дни притиранья их: шесть месяцев мирровым маслом и шесть месяцев ароматами и другими притираньями женскими,~---
\vs Est 2:13 тогда девица входила к царю. Чего бы она ни потребовала, ей давали всё для выхода из женского дома в дом царя.
\vs Est 2:14 Вечером она входила и утром возвращалась в другой дом женский под надзор Шаазгаза, царского евнуха, стража наложниц; и уже не входила к царю, разве только царь пожелал бы ее, и она призывалась бы по имени.
\rsbpar\vs Est 2:15 Когда настало время Есфири, дочери Аминадава, дяди Мардохея, который взял ее к себе вместо дочери,~--- идти к царю, тогда она не просила ничего, кроме того, о чем сказал ей Гегай, евнух царский, страж жен. И приобрела Есфирь расположение \bibemph{к себе} в глазах всех, видевших ее.
\vs Est 2:16 И взята была Есфирь к царю Артаксерксу, в царский дом его, в десятом месяце, то есть в месяце Тебефе, в седьмой год его царствования.
\vs Est 2:17 И полюбил царь Есфирь более всех жен, и она приобрела его благоволение и благорасположение более всех девиц; и он возложил царский венец на голову ее и сделал ее царицею на место Астинь.
\vs Est 2:18 И сделал царь большой пир для всех князей своих и для служащих при нем,~--- пир ради Есфири, и сделал льготу областям и раздал дары с царственною щедростью.
\vs Est 2:19 И когда во второй раз собраны были девицы, и Мардохей сидел у ворот царских,
\vs Est 2:20 Есфирь все еще не сказывала о родстве своем и о народе своем, как приказал ей Мардохей; а слово Мардохея Есфирь выполняла \bibemph{и теперь} так же, как тогда, когда была у него на воспитании.
\vs Est 2:21 В это время, как Мардохей сидел у ворот царских, два царских евнуха, Гавафа и Фарра, оберегавшие порог, озлобились [за то, что предпочтен был Мардохей], и замышляли наложить руку на царя Артаксеркса.
\vs Est 2:22 Узнав о том, Мардохей сообщил царице Есфири, а Есфирь сказала царю от имени Мардохея.
\vs Est 2:23 Дело было исследовано и найдено \bibemph{верным}, и их обоих повесили на дереве. И было вписано о благодеянии Мардохея в книгу дневных записей у царя.
\vs Est 3:1 После сего возвеличил царь Артаксеркс Амана, сына Амадафа, Вугеянина, и вознес его, и поставил седалище его выше всех князей, которые у него;
\vs Est 3:2 и все служащие при царе, которые \bibemph{были} у царских ворот, кланялись и падали ниц пред Аманом, ибо так приказал царь. А Мардохей не кланялся и не падал ниц.
\vs Est 3:3 И говорили служащие при царе, которые у царских ворот, Мардохею: зачем ты преступаешь повеление царское?
\vs Est 3:4 И как они говорили ему каждый день, а он не слушал их, то они донесли Аману, чтобы посмотреть, устоит ли в слове \bibemph{своем} Мардохей, ибо он сообщил им, что он Иудеянин.
\rsbpar\vs Est 3:5 И когда увидел Аман, что Мардохей не кланяется и не падает ниц пред ним, то исполнился гнева Аман.
\vs Est 3:6 И показалось ему ничтожным наложить руку на одного Мардохея; но так как сказали ему, из какого народа Мардохей, то задумал Аман истребить всех Иудеев, которые \bibemph{были} во всем царстве Артаксеркса, \bibemph{как} народ Мардохеев.
\vs Est 3:7 [И сделал совет] в первый месяц, который есть месяц Нисан, в двенадцатый год царя Артаксеркса, и бросали пур, то есть жребий, пред лицем Амана изо дня в день и из месяца в месяц, [чтобы в один день погубить народ Мардохеев, и пал жребий] на двенадцатый \bibemph{месяц}, то есть на месяц Адар.
\vs Est 3:8 И сказал Аман царю Артаксерксу: есть один народ, разбросанный и рассеянный между народами по всем областям царства твоего; и законы их отличны от \bibemph{законов} всех народов, и законов царя они не выполняют; и царю не следует \bibemph{так} оставлять их.
\vs Est 3:9 Если царю благоугодно, то пусть будет предписано истребить их, и десять тысяч талантов серебра я отвешу в руки приставников, чтобы внести в казну царскую.
\vs Est 3:10 Тогда снял царь перстень свой с руки своей и отдал его Аману, сыну Амадафа, Вугеянину, чтобы скрепить указ против Иудеев.
\vs Est 3:11 И сказал царь Аману: отдаю тебе \bibemph{это} серебро и народ; поступи с ним, как тебе угодно.
\rsbpar\vs Est 3:12 И призваны были писцы царские в первый месяц, в тринадцатый день его, и написано было, как приказал Аман, к сатрапам царским и к начальствующим над каждою областью [от области Индийской до Ефиопии, над ста двадцатью семью областями], и к князьям у каждого народа, в каждую область письменами ее и к каждому народу на языке его: \bibemph{все} было написано от имени царя Артаксеркса и скреплено царским перстнем.
\vs Est 3:13 И посланы были письма через гонцов во все области царя, чтобы убить, погубить и истребить всех Иудеев, малого и старого, детей и женщин в один день, в тринадцатый день двенадцатого месяца, то есть месяца Адара, и имение их разграбить. [Вот список с этого письма: великий царь Артаксеркс начальствующим от Индии до Ефиопии над ста двадцатью семью областями и подчиненным им наместникам. Царствуя над многими народами и властвуя над всею вселенною, я хотел, не превозносясь гордостью власти, но управляя всегда кротко и тихо, сделать жизнь подданных постоянно безмятежною и, соблюдая царство свое мирным и удобопроходимым до пределов \bibemph{его}, восстановить желаемый для всех людей мир. Когда же я спросил советников, каким бы образом привести это в исполнение, то отличающийся у нас мудростью и \bibemph{пользующийся} неизменным благоволением, и доказавший твердую верность, и получивший вторую честь по царе, Аман объяснил нам, что во всех племенах вселенной замешался один враждебный народ, по законам \bibemph{своим} противный всякому народу, постоянно пренебрегающий царскими повелениями, дабы не благоустроялось безукоризненно совершаемое нами соуправление. Итак, узнав, что один только этот народ всегда противится всякому человеку, ведет образ жизни, чуждый законам, и, противясь нашим действиям, совершает величайшие злодеяния, чтобы царство \bibemph{наше} не достигло благосостояния, мы повелели указанных вам в грамотах Амана, поставленного над делами и второго отца нашего, всех с женами и детьми всецело истребить вражескими мечами, без всякого сожаления и пощады, в тринадцатый день двенадцатого месяца Адара настоящего года, чтобы эти и прежде и теперь враждебные \bibemph{люди}, быв в один день насильно низвергнуты в преисподнюю, не препятствовали нам в последующее время проводить жизнь мирно и безмятежно до конца.]
\vs Est 3:14 Список с указа отдать в каждую область \bibemph{как} закон, объявляемый для всех народов, чтобы они были готовы к тому дню.
\vs Est 3:15 Гонцы отправились быстро с царским повелением. Объявлен был указ и в Сузах, престольном городе; и царь и Аман сидели и пили, а город Сузы \bibemph{был} в смятении.
\vs Est 4:1 Когда Мардохей узнал все, что делалось, разодрал одежды свои и возложил на себя вретище и пепел, и вышел на средину города и взывал с воплем великим и горьким: [истребляется народ ни в чем не повинный!]
\vs Est 4:2 И дошел до царских ворот [и остановился,] так как нельзя было входить в царские ворота во вретище [и с пеплом].
\vs Est 4:3 Равно и во всякой области и месте, куда \bibemph{только} доходило повеление царя и указ его, было большое сетование у Иудеев, и пост, и плач, и вопль; вретище и пепел служили постелью для многих.
\rsbpar\vs Est 4:4 И пришли служанки Есфири и евнухи ее и рассказали ей, и сильно встревожилась царица. И послала одежды, чтобы Мардохей надел их и снял с себя вретище свое. Но он не принял.
\vs Est 4:5 Тогда позвала Есфирь Гафаха, одного из евнухов царя, которого он приставил к ней, и послала его к Мардохею узнать: что это и отчего это?
\vs Est 4:6 И пошел Гафах к Мардохею на городскую площадь, которая пред царскими воротами.
\vs Est 4:7 И рассказал ему Мардохей обо всем, что с ним случилось, и об определенном числе серебра, которое обещал Аман отвесить в казну царскую за Иудеев, чтобы истребить их;
\vs Est 4:8 и вручил ему список с указа, обнародованного в Сузах, об истреблении их, чтобы показать Есфири и дать ей знать \bibemph{обо всем}; притом наказывал ей, чтобы она пошла к царю и молила его о помиловании и просила его за народ свой, [вспомнив дни смирения своего, когда она воспитывалась под рукою моею, потому что Аман, второй по царе, осудил нас на смерть, и чтобы призвала Господа и сказала о нас царю, да избавит нас от смерти].
\vs Est 4:9 И пришел Гафах и пересказал Есфири слова Мардохея.
\vs Est 4:10 И сказала Есфирь Гафаху и послала его \bibemph{сказать} Мардохею:
\vs Est 4:11 все служащие при царе и народы в областях царских знают, что всякому, и мужчине и женщине, кто войдет к царю во внутренний двор, не быв позван, один суд~--- смерть; только тот, к кому прострет царь свой золотой скипетр, останется жив. А я не звана к царю вот уже тридцать дней.
\vs Est 4:12 И пересказали Мардохею слова Есфири.
\vs Est 4:13 И сказал Мардохей в ответ Есфири: не думай, что ты \bibemph{одна} спасешься в доме царском из всех Иудеев.
\vs Est 4:14 Если ты промолчишь в это время, то свобода и избавление придет для Иудеев из другого места, а ты и дом отца твоего погибнете. И кто знает, не для такого ли времени ты и достигла достоинства царского?
\vs Est 4:15 И сказала Есфирь в ответ Мардохею:
\vs Est 4:16 пойди, собери всех Иудеев, находящихся в Сузах, и поститесь ради меня, и не ешьте и не пейте три дня, ни днем, ни ночью, и я с служанками моими буду также поститься и потом пойду к царю, хотя это против закона, и если погибнуть~--- погибну.
\rsbpar\vs Est 4:17 И пошел Мардохей и сделал, как приказала ему Есфирь. [И молился он Господу, воспоминая все дела Господни, и говорил: Господи, Господи, Царю, Вседержителю! Все в Твоей власти, и нет противящегося Тебе, когда Ты захочешь спасти Израиля; Ты сотворил небо и землю и все дивное в поднебесной; Ты~--- Господь всех, и нет \bibemph{такого}, кто воспротивился бы Тебе, Господу. Ты знаешь всё; Ты знаешь, Господи, что не для обиды и не по гордости и не по тщеславию я делал это, что не поклонялся тщеславному Аману, ибо я охотно стал бы лобызать следы ног его для спасения Израиля; но я делал это для того, чтобы не воздать славы человеку выше славы Божией и не поклоняться никому, кроме Тебя, Господа моего, и я не стану делать этого по гордости. И ныне, Господи Боже, Царю, Боже Авраамов, пощади народ Твой; ибо замышляют нам погибель и хотят истребить изначальное наследие Твое; не презри достояния Твоего, которое Ты избавил для Себя из земли Египетской; услышь молитву мою и умилосердись над наследием Твоим и обрати сетование наше в веселие, дабы мы, живя, воспевали имя Твое, Господи, и не погуби уст, прославляющих Тебя, Господи. И все Израильтяне взывали \bibemph{всеми} силами своими, потому что смерть их \bibemph{была} пред глазами их. И царица Есфирь прибегла к Господу, объятая смертною горестью, и, сняв одежды славы своей, облеклась в одежды скорби и сетования, и, вместо многоценных мастей, пеплом и прахом посыпала голову свою, и весьма изнурила тело свое, и всякое место, украшаемое в веселии ее, покрыла распущенными волосами своими, и молилась Господу Богу Израилеву, говоря: Господи мой! Ты один Царь наш; помоги мне, одинокой и не имеющей помощника, кроме Тебя; ибо беда моя близ меня. Я слышала, Господи, от отца моего, в родном колене моем, что Ты, Господи, избрал себе Израиля из всех народов и отцов наших из всех предков их в наследие вечное, и сделал для них то, о чем говорил им. И ныне мы согрешили пред Тобою, и предал Ты нас в руки врагов наших за то, что мы славили богов их: праведен Ты, Господи! А ныне они не удовольствовались горьким рабством нашим, но положили руки свои в руки идолов своих, чтобы ниспровергнуть заповедь уст Твоих, и истребить наследие Твое, и заградить уста воспевающих Тебя, и погасить славу \bibemph{храма} Твоего и жертвенника Твоего, и отверзть уста народов на прославление тщетных \bibemph{богов}, и царю плотскому величаться вовек. Не предай, Господи, скипетра Твоего \bibemph{богам} несуществующим, и пусть не радуются падению нашему, но обрати замысел их на них самих: наветника же против нас предай позору. Помяни, Господи, яви Себя нам во время скорби нашей и дай мне мужество. Царь богов и Владыка всякого начальства! даруй устам моим слово благоприятное пред этим львом и исполни сердце его ненавистью к преследующему нас, на погибель ему и единомышленникам его; нас же избавь рукою Твоею и помоги мне, одинокой и не имеющей помощника, кроме Тебя, Господи. Ты имеешь ведение всего и знаешь, что я ненавижу славу беззаконных и гнушаюсь ложа необрезанных и всякого иноплеменника; Ты знаешь необходимость мою, что я гнушаюсь знака гордости моей, который бывает на голове моей во дни появления моего, гнушаюсь его, как одежды, оскверненной кровью, и не ношу его во дни уединения моего. И не вкушала раба Твоя от трапезы Амана и не дорожила пиром царским, и не пила вина идоложертвенного, и не веселилась раба Твоя со дня перемены \bibemph{судьбы} моей доныне, кроме как о Тебе, Господи Боже Авраамов. Боже, имеющий силу над всеми! услышь голос безнадежных, и спаси нас от руки злоумышляющих, и избавь меня от страха моего.]
\vs Est 5:1 На третий день Есфирь [перестав молиться, сняла одежды сетования и] оделась по-царски, [и сделавшись великолепною, призывая всевидца Бога и Спасителя, взяла двух служанок, и на одну опиралась, как бы предавшись неге, а другая следовала \bibemph{за нею}, поддерживая одеяние ее. Она была прекрасна во цвете красоты своей, и лице ее радостно, как бы исполненное любви, но сердце ее было стеснено от страха]. И стала она на внутреннем дворе царского дома, перед домом царя; царь же сидел \bibemph{тогда} на царском престоле своем, в царском доме, прямо против входа в дом, [облеченный во все одеяние величия своего, весь в золоте и драгоценных камнях, и был весьма страшен]. Когда царь увидел царицу Есфирь, стоящую на дворе, она нашла милость в глазах его. [Обратив лице свое, пламеневшее славою, он взглянул с сильным гневом; и царица упала \bibemph{духом} и изменилась в лице своем от ослабления и склонилась на голову служанки, которая сопровождала ее. И изменил Бог дух царя на кротость, и поспешно встал он с престола своего и принял ее в объятия свои, пока она не пришла в себя. Потом он утешил ее ласковыми словами, сказав ей: что \bibemph{тебе}, Есфирь? Я~--- брат твой; ободрись, не умрешь, ибо наше владычество общее; подойди.]
\vs Est 5:2 И простер царь к Есфири золотой скипетр, который был в руке его, и подошла Есфирь и коснулась конца скипетра, [и положил \bibemph{царь} скипетр на шею ее и поцеловал ее и сказал: говори мне. И сказала она: я видела в тебе, господин, как бы Ангела Божия, и смутилось сердце мое от страха пред славою твоею, ибо дивен ты, господин, и лице твое исполнено благодати.~--- Но во время беседы она упала от ослабления; и царь смутился, и все слуги его утешали ее].
\vs Est 5:3 И сказал ей царь: что тебе, царица Есфирь, и какая просьба твоя? Даже до полуцарства будет дано тебе.
\vs Est 5:4 И сказала Есфирь: [ныне у меня день праздничный;] если царю благоугодно, пусть придет царь с Аманом сегодня на пир, который я приготовила ему.
\vs Est 5:5 И сказал царь: сходите скорее за Аманом, чтобы сделать по слову Есфири. И пришел царь с Аманом на пир, который приготовила Есфирь.
\vs Est 5:6 И сказал царь Есфири при питье вина: какое желание твое? оно будет удовлетворено; и какая просьба твоя? \bibemph{хотя бы} до полуцарства, она будет исполнена.
\vs Est 5:7 И отвечала Есфирь, и сказала: \bibemph{вот} мое желание и моя просьба:
\vs Est 5:8 если я нашла благоволение в очах царя, и если царю благоугодно удовлетворить желание мое и исполнить просьбу мою, то пусть царь с Аманом придет [еще завтра] на пир, который я приготовлю для них, и завтра я исполню слово царя.
\vs Est 5:9 И вышел Аман в тот день веселый и благодушный. Но когда увидел Аман Мардохея у ворот царских, и тот не встал и с места не тронулся пред ним, тогда исполнился Аман гневом на Мардохея.
\vs Est 5:10 Однако же скрепился Аман. А когда пришел в дом свой, то послал позвать друзей своих и Зерешь, жену свою.
\vs Est 5:11 И рассказывал им Аман о великом богатстве своем и о множестве сыновей своих и обо всем том, как возвеличил его царь и как вознес его над князьями и слугами царскими.
\vs Est 5:12 И сказал Аман: да и царица Есфирь никого не позвала с царем на пир, который она приготовила, кроме меня; так и на завтра я зван к ней с царем.
\vs Est 5:13 Но всего этого не довольно для меня, доколе я вижу Мардохея Иудеянина сидящим у ворот царских.
\vs Est 5:14 И сказала ему Зерешь, жена его, и все друзья его: пусть приготовят дерево вышиною в пятьдесят локтей, и утром скажи царю, чтобы повесили Мардохея на нем, и тогда весело иди на пир с царем. И понравилось это слово Аману, и он приготовил дерево.
\vs Est 6:1 В ту ночь Господь отнял сон от царя, и он велел [слуге] принести памятную книгу дневных записей; и читали их пред царем,
\vs Est 6:2 и найдено записанным \bibemph{там}, как донес Мардохей на Гавафу и Фарру, двух евнухов царских, оберегавших порог, которые замышляли наложить руку на царя Артаксеркса.
\vs Est 6:3 И сказал царь: какая дана почесть и отличие Мардохею за это? И сказали отроки царя, служившие при нем: ничего не сделано ему.
\vs Est 6:4 [Когда царь расспрашивал о благодеянии Мардохея, пришел на двор Аман,] и сказал царь: кто на дворе? Аман же пришел \bibemph{тогда} на внешний двор царского дома поговорить с царем, чтобы повесили Мардохея на дереве, которое он приготовил для него.
\vs Est 6:5 И сказали отроки царю: вот, Аман стоит на дворе. И сказал царь: пусть войдет.
\vs Est 6:6 И вошел Аман. И сказал ему царь: что сделать бы тому человеку, которого царь хочет отличить почестью? Аман подумал в сердце своем: кому \bibemph{другому} захочет царь оказать почесть, кроме меня?
\vs Est 6:7 И сказал Аман царю: тому человеку, которого царь хочет отличить почестью,
\vs Est 6:8 пусть принесут одеяние царское, в которое одевается царь, и \bibemph{приведут} коня, на котором ездит царь, возложат царский венец на голову его,
\vs Est 6:9 и пусть подадут одеяние и коня в руки одному из первых князей царских,~--- и облекут того человека, которого царь хочет отличить почестью, и выведут его на коне на городскую площадь, и провозгласят пред ним: так делается тому человеку, которого царь хочет отличить почестью!
\vs Est 6:10 И сказал царь Аману: [хорошо ты сказал;] тотчас же возьми одеяние и коня, как ты сказал, и сделай это Мардохею Иудеянину, сидящему у царских ворот; ничего не опусти из всего, что ты говорил.
\vs Est 6:11 И взял Аман одеяние и коня и облек Мардохея, и вывел его на коне на городскую площадь и провозгласил пред ним: так делается тому человеку, которого царь хочет отличить почестью!
\vs Est 6:12 И возвратился Мардохей к царским воротам. Аман же поспешил в дом свой, печальный и закрыв голову.
\vs Est 6:13 И пересказал Аман Зереши, жене своей, и всем друзьям своим все, что случилось с ним. И сказали ему мудрецы его и Зерешь, жена его: если из племени Иудеев Мардохей, из-за которого ты начал падать, то не пересилишь его, а наверно падешь пред ним, [ибо с ним Бог живый].
\vs Est 6:14 Они еще разговаривали с ним, \bibemph{как} пришли евнухи царя и стали торопить Амана идти на пир, который приготовила Есфирь.
\vs Est 7:1 И пришел царь с Аманом пировать у Есфири царицы.
\vs Est 7:2 И сказал царь Есфири также и в \bibemph{этот} второй день во время пира: какое желание твое, царица Есфирь? оно будет удовлетворено; и какая просьба твоя? \bibemph{хотя бы} до полуцарства, она будет исполнена.
\vs Est 7:3 И отвечала царица Есфирь и сказала: если я нашла благоволение в очах твоих, царь, и если царю благоугодно, то да будут дарованы мне жизнь моя, по желанию моему, и народ мой, по просьбе моей!
\vs Est 7:4 Ибо проданы мы, я и народ мой, на истребление, убиение и погибель. Если бы мы проданы были в рабы и рабыни, я молчала бы, хотя враг не вознаградил бы ущерба царя.
\vs Est 7:5 И отвечал царь Артаксеркс и сказал царице Есфири: кто это такой, и где тот, который отважился в сердце своем сделать так?
\vs Est 7:6 И сказала Есфирь: враг и неприятель~--- этот злобный Аман! И Аман затрепетал пред царем и царицею.
\vs Est 7:7 И царь встал во гневе своем с пира \bibemph{и пошел} в сад при дворце; Аман же остался умолять о жизни своей царицу Есфирь, ибо видел, что определена ему злая участь от царя.
\vs Est 7:8 Когда царь возвратился из сада при дворце в дом пира, Аман был припавшим к ложу, на котором находилась Есфирь. И сказал царь: даже и насиловать царицу \bibemph{хочет} в доме у меня! Слово вышло из уст царя,~--- и накрыли лице Аману.
\vs Est 7:9 И сказал Харбона, один из евнухов при царе: вот и дерево, которое приготовил Аман для Мардохея, говорившего доброе для царя, стоит у дома Амана, вышиною в пятьдесят локтей. И сказал царь: повесьте его на нем.
\vs Est 7:10 И повесили Амана на дереве, которое он приготовил для Мардохея. И гнев царя утих.
\vs Est 8:1 В тот день царь Артаксеркс отдал царице Есфири дом Амана, врага Иудеев; а Мардохей вошел пред лице царя, ибо Есфирь объявила, чт\acc{о} он для нее.
\vs Est 8:2 И снял царь перстень свой, который он отнял у Амана, и отдал его Мардохею; Есфирь же поставила Мардохея смотрителем над домом Амана.
\vs Est 8:3 И продолжала Есфирь говорить пред царем и пала к ногам его, и плакала и умоляла его отвратить злобу Амана Вугеянина и замысел его, который он замыслил против Иудеев.
\vs Est 8:4 И простер царь к Есфири золотой скипетр; и поднялась Есфирь, и стала пред лицем царя,
\vs Est 8:5 и сказала: если царю благоугодно, и если я нашла благоволение пред лицем его, и справедливо дело сие пред лицем царя, и нравлюсь я очам его, то пусть было бы написано, чтобы возвращены были письма по замыслу Амана, сына Амадафа, Вугеянина, писанные им об истреблении Иудеев во всех областях царя;
\vs Est 8:6 ибо, как я могу видеть бедствие, которое постигнет народ мой, и как я могу видеть погибель родных моих?
\vs Est 8:7 И сказал царь Артаксеркс царице Есфири и Мардохею Иудеянину: вот, я дом Амана отдал Есфири, и его самого повесили на дереве за то, что он налагал руку свою на Иудеев;
\vs Est 8:8 напишите и вы о Иудеях, что вам угодно, от имени царя и скрепите царским перстнем, ибо письма, написанного от имени царя и скрепленного перстнем царским, нельзя изменить.
\rsbpar\vs Est 8:9 И позваны были тогда царские писцы в третий месяц, то есть в месяц Сиван, в двадцать третий день его, и написано было все так, как приказал Мардохей, к Иудеям, и к сатрапам, и областеначальникам, и правителям областей от Индии до Ефиопии, ста двадцати семи областей, в каждую область письменами ее и к каждому народу на языке его, и к Иудеям письменами их и на языке их.
\vs Est 8:10 И написал он от имени царя Артаксеркса, и скрепил царским перстнем, и послал письма чрез гонцов на конях, на дромадерах и мулах царских,
\vs Est 8:11 о том, что царь позволяет Иудеям, находящимся во всяком городе, собраться и стать на защиту жизни своей, истребить, убить и погубить всех сильных в народе и в области, которые во вражде с ними, детей и жен, и имение их разграбить,
\vs Est 8:12 в один день по всем областям царя Артаксеркса, в тринадцатый день двенадцатого месяца, то есть месяца Адара. [Список с этого указа следующий: великий царь Артаксеркс начальствующим от Индии до Ефиопии над ста двадцатью семью областями и властителям, доброжелательствующим нам, радоваться. Многие, по чрезвычайной доброте благодетелей щедро награждаемые почестями, чрезмерно возгордились и не только подданным нашим ищут причинить зло, но, не могши насытить гордость, покушаются строить козни самим благодетелям своим, не только теряют чувство человеческой признательности, но, кичась надменностью безумных, преступно думают избежать суда всё и всегда видящего Бога. Но часто и многие, будучи облечены властью, чтоб устроять дела доверивших им друзей, своим убеждением делают их виновниками \bibemph{пролития} невинной крови и подвергают неисправимым бедствиям, хитросплетением коварной лжи обманывая непорочное благомыслие державных. \bibemph{Это} можно видеть не столько из древних историй, как мы сказали, сколько из дел, преступно совершаемых пред вами злобою недостойно властвующих. Посему нужно озаботиться на последующее время, чтобы нам устроить царство безмятежным для всех людей в мире, не допуская изменений, но представляющиеся дела обсуждая с надлежащей предусмотрительностью. Так Аман Амадафов, Македонянин, поистине чуждый персидской крови и весьма далекий от нашей благости, быв принят у нас гостем, удостоился благосклонности, которую мы имеем ко всякому народу, настолько, что был провозглашен нашим отцом и почитаем всеми, представляя второе лицо при царском престоле; но, не умерив гордости, замышлял лишить нас власти и души, а нашего спасителя и всегдашнего благодетеля Мардохея и непорочную общницу царства Есфирь, со всем народом их, домогался разнообразными коварными мерами погубить. Таким образом он думал сделать нас безлюдными, а державу Персидскую передать Македонянам. Мы же находим Иудеев, осужденных этим злодеем на истребление, не зловредными, а живущими по справедливейшим законам, сынами Вышнего, величайшего живаго Бога, даровавшего нам и предкам нашим царство в самом лучшем состоянии. Посему вы хорошо сделаете, не приводя в исполнение грамот, посланных Аманом Амадафовым; ибо он, совершивший это, при воротах Сузских повешен со всем домом, \bibemph{по воле} владычествующего всем Бога, воздавшего ему скоро достойный суд. Список же с этого указа выставив на всяком месте открыто, оставьте Иудеев пользоваться своими законами и содействуйте им, чтобы восстававшим на них во время скорби они могли отмстить в тринадцатый день двенадцатого месяца Адара, в самый тот день. Ибо владычествующий над всем Бог, вместо погибели избранного рода, устроил им такую радость. И вы, в числе именитых праздников ваших, проводите сей знаменитый день со всем весельем, дабы и ныне и после памятно было спасение для нас и для благорасположенных \bibemph{к нам} Персов и погубление строивших нам козни. Всякий город или область вообще, которая не исполнит сего, нещадно опустошится мечом и огнем и сделается не только необитаемою для людей, но и для зверей и птиц навсегда отвратительною.]
\vs Est 8:13 Список с сего указа отдать в каждую область, \bibemph{как} закон, объявляемый для всех народов, чтоб Иудеи готовы были к тому дню мстить врагам своим.
\vs Est 8:14 Гонцы, поехавшие верхом на быстрых конях царских, погнали скоро и поспешно, с царским повелением. Объявлен был указ и в Сузах, престольном городе.
\vs Est 8:15 И Мардохей вышел от царя в царском одеянии яхонтового и белого цвета и в большом золотом венце, и в мантии виссонной и пурпуровой. И город Сузы возвеселился и возрадовался.
\vs Est 8:16 А у Иудеев было \bibemph{тогда} освещение и радость, и веселье, и торжество.
\vs Est 8:17 И во всякой области и во всяком городе, во \bibemph{всяком} месте, куда \bibemph{только} доходило повеление царя и указ его, была радость у Иудеев и веселье, пиршество и праздничный день. И многие из народов страны сделались Иудеями, потому что напал на них страх пред Иудеями.
\vs Est 9:1 В двенадцатый месяц, то есть в месяц Адар, в тринадцатый день его, в который пришло время исполниться повелению царя и указу его, в тот день, когда надеялись неприятели Иудеев взять власть над ними, а вышло наоборот, что сами Иудеи взяли власть над врагами своими,~---
\vs Est 9:2 собрались Иудеи в городах своих по всем областям царя Артаксеркса, чтобы наложить руку на зложелателей своих; и никто не мог устоять пред лицем их, потому что страх пред ними напал на все народы.
\vs Est 9:3 И все князья в областях и сатрапы, и областеначальники, и исполнители дел царских поддерживали Иудеев, потому что напал на них страх пред Мардохеем.
\vs Est 9:4 Ибо велик был Мардохей в доме у царя, и слава о нем ходила по всем областям, так как сей человек, Мардохей, поднимался выше и выше.
\rsbpar\vs Est 9:5 И избивали Иудеи всех врагов своих, побивая мечом, умерщвляя и истребляя, и поступали с неприятелями своими по своей воле.
\vs Est 9:6 В Сузах, городе престольном, умертвили Иудеи и погубили пятьсот человек;
\vs Est 9:7 и Паршандафу и Далфона и Асфафу,
\vs Est 9:8 и Порафу и Адалью и Аридафу,
\vs Est 9:9 и Пармашфу и Арисая и Аридая и Ваиезафу,~---
\vs Est 9:10 десятерых сыновей Амана, сына Амадафа, врага Иудеев, умертвили они, а на грабеж не простерли руки своей.
\vs Est 9:11 В тот же день донесли царю о числе умерщвленных в Сузах, престольном городе.
\vs Est 9:12 И сказал царь царице Есфири: в Сузах, городе престольном, умертвили Иудеи и погубили пятьсот человек и десятерых сыновей Амана; что же сделали они в прочих областях царя? Какое желание твое? и оно будет удовлетворено. И какая еще просьба твоя? она будет исполнена.
\vs Est 9:13 И сказала Есфирь: если царю благоугодно, то пусть бы позволено было Иудеям, которые в Сузах, делать то же и завтра, что сегодня, и десятерых сыновей Амановых пусть бы повесили на дереве.
\vs Est 9:14 И приказал царь сделать так; и дан \bibemph{на это} указ в Сузах, и десятерых сыновей Амановых повесили.
\rsbpar\vs Est 9:15 И собрались Иудеи, которые в Сузах, также и в четырнадцатый день месяца Адара и умертвили в Сузах триста человек, а на грабеж не простерли руки своей.
\vs Est 9:16 И прочие Иудеи, находившиеся в царских областях, собрались, чтобы стать на защиту жизни своей и быть покойными от врагов своих, и умертвили из неприятелей своих семьдесят пять тысяч, а на грабеж не простерли руки своей.
\vs Est 9:17 \bibemph{Это было} в тринадцатый день месяца Адара; а в четырнадцатый день сего же месяца они успокоились и сделали его днем пиршества и веселья.
\vs Est 9:18 Иудеи же, которые в Сузах, собирались в тринадцатый день его и в четырнадцатый день его, а в пятнадцатый день его успокоились и сделали его днем пиршества и веселья.
\vs Est 9:19 Поэтому Иудеи сельские, живущие в селениях открытых, проводят четырнадцатый день месяца Адара в веселье и пиршестве, как день праздничный, посылая подарки друг ко другу; [живущие же в митрополиях и пятнадцатый день Адара проводят в добром веселье, посылая подарки ближним].
\rsbpar\vs Est 9:20 И описал Мардохей эти происшествия и послал письма ко всем Иудеям, которые в областях царя Артаксеркса, к близким и к дальним,
\vs Est 9:21 \bibemph{о том}, чтобы они установили каждогодно празднование у себя четырнадцатого дня месяца Адара и пятнадцатого дня его,
\vs Est 9:22 как таких дней, в которые Иудеи сделались покойны от врагов своих, и \bibemph{как} такого месяца, в который превратилась у них печаль в радость, и сетование~--- в день праздничный,~--- чтобы сделали их днями пиршества и веселья, посылая подарки друг другу и подаяния бедным.
\vs Est 9:23 И приняли Иудеи то, что уже сами начали делать, и о чем Мардохей написал к ним,
\vs Est 9:24 как Аман, сын Амадафа, Вугеянин, враг всех Иудеев, думал погубить Иудеев и бросал пур, \bibemph{жребий}, об истреблении и погублении их,
\vs Est 9:25 и как Есфирь дошла до царя, и как царь приказал новым письмом, чтобы злой замысл Амана, который он задумал на Иудеев, обратился на голову его, и чтобы повесили его и сыновей его на дереве.
\vs Est 9:26 Потому и назвали эти дни Пурим, от имени: пур [\bibemph{жребий}, ибо на языке их жребии называются пурим]. Поэтому, согласно со всеми словами сего письма и с тем, что сами видели и до чего доходило у них,
\vs Est 9:27 постановили Иудеи и приняли на себя и на детей своих и на всех, присоединяющихся к ним, неотменно, чтобы праздновать эти два дня, по предписанному о них и в свое для них время, каждый год;
\vs Est 9:28 и чтобы дни эти были памятны и празднуемы во все роды в каждом племени, в каждой области и в каждом городе; и чтобы дни эти Пурим не отменялись у Иудеев, и память о них не исчезла у детей их.
\rsbpar\vs Est 9:29 Написала также царица Есфирь, дочь Абихаила, и Мардохей Иудеянин, со всею настойчивостью, чтобы исполняли это новое письмо о Пуриме;
\vs Est 9:30 и послали письма ко всем Иудеям в сто двадцать семь областей царства Артаксерксова со словами мира и правды,
\vs Est 9:31 чтобы они твердо наблюдали эти дни Пурим в свое время, какое уставил о них Мардохей Иудеянин и царица Есфирь, и как они сами назначали их для себя и для детей своих в дни пощения и воплей.
\vs Est 9:32 Так повеление Есфири подтвердило это слово о Пуриме, и оно вписано в книгу.
\vs Est 10:1 Потом наложил царь Артаксеркс подать на землю и на острова морские.
\vs Est 10:2 Впрочем, все дела силы его и могущества его и обстоятельное показание о величии Мардохея, которым возвеличил его царь, записаны в книге дневных записей царей Мидийских и Персидских,
\vs Est 10:3 \bibemph{равно как и то}, что Мардохей Иудеянин \bibemph{был} вторым по царе Артаксерксе и великим у Иудеев и любимым у множества братьев своих, \bibemph{ибо} искал добра народу своему и говорил во благо всего племени своего. [И сказал Мардохей: от Бога было это, ибо я вспомнил сон, который я видел о сих событиях; не осталось в нем ничего неисполнившимся. Малый источник сделался рекою, и был свет и солнце и множество воды: эта река есть Есфирь, которую взял себе в жену царь и сделал царицею. А два змея~--- это я и Аман; народы~--- это собравшиеся истребить имя Иудеев; а народ мой~--- это Израильтяне, воззвавшие к Богу и спасенные. И спас Господь народ Свой, и избавил нас Господь от всех сих зол, и совершил Бог знамения и чудеса великие, какие не бывали между язычниками. Так устроил Бог два жребия: один для народа Божия, а другой для всех язычников, и вышли эти два жребия в час и время и в день суда пред Богом и всеми язычниками. И вспомнил Господь о народе Своем и оправдал наследие Свое. И будут праздноваться эти дни месяца Адара, в четырнадцатый и пятнадцатый день этого месяца, с торжеством и радостью и весельем пред Богом, в роды вечные, в народе Его Израиле. В четвертый год царствования Птоломея и Клеопатры Досифей, который, говорят, был священником и левитом, и Птоломей, сын его, принесли \bibemph{в Александрию} это послание о Пуриме, которое, говорят, истолковал Лисимах, \bibemph{сын} Птоломея, бывший в Иерусалиме.]

\bibbookdescr{Job}{
  inline={\LARGE Книга\\\Huge Иова},
  toc={Иов},
  bookmark={Иов},
  header={Иов},
  %headerleft={},
  %headerright={},
  abbr={Иов}
}
\vs Job 1:1 Был человек в земле Уц, имя его Иов; и был человек этот непорочен, справедлив и богобоязнен и удалялся от зла.
\vs Job 1:2 И родились у него семь сыновей и три дочери.
\vs Job 1:3 Имения у него было: семь тысяч мелкого скота, три тысячи верблюдов, пятьсот пар волов и пятьсот ослиц и весьма много прислуги; и был человек этот знаменитее всех сынов Востока.
\vs Job 1:4 Сыновья его сходились, делая пиры каждый в своем доме в свой день, и посылали и приглашали трех сестер своих есть и пить с ними.
\vs Job 1:5 Когда круг пиршественных дней совершался, Иов посылал \bibemph{за ними} и освящал их и, вставая рано утром, возносил всесожжения по числу всех их [и одного тельца за грех о душах их]. Ибо говорил Иов: может быть, сыновья мои согрешили и похулили Бога в сердце своем. Так делал Иов во все \bibemph{такие} дни.
\rsbpar\vs Job 1:6 И был день, когда пришли сыны Божии предстать пред Господа; между ними пришел и сатана.
\vs Job 1:7 И сказал Господь сатане: откуда ты пришел? И отвечал сатана Господу и сказал: я ходил по земле и обошел ее.
\vs Job 1:8 И сказал Господь сатане: обратил ли ты внимание твое на раба Моего Иова? ибо нет такого, как он, на земле: человек непорочный, справедливый, богобоязненный и удаляющийся от зла.
\vs Job 1:9 И отвечал сатана Господу и сказал: разве даром богобоязнен Иов?
\vs Job 1:10 Не Ты ли кругом оградил его и дом его и все, что у него? Дело рук его Ты благословил, и стада его распространяются по земле;
\vs Job 1:11 но простри руку Твою и коснись всего, что у него,~--- благословит ли он Тебя?
\vs Job 1:12 И сказал Господь сатане: вот, все, что у него, в руке твоей; только на него не простирай руки твоей. И отошел сатана от лица Господня.
\rsbpar\vs Job 1:13 И был день, когда сыновья его и дочери его ели и вино пили в доме первородного брата своего.
\vs Job 1:14 И \bibemph{вот}, приходит вестник к Иову и говорит:
\vs Job 1:15 волы орали, и ослицы паслись подле них, как напали Савеяне и взяли их, а отроков поразили острием меча; и спасся только я один, чтобы возвестить тебе.
\vs Job 1:16 Еще он говорил, как приходит другой и сказывает: огонь Божий упал с неба и опалил овец и отроков и пожрал их; и спасся только я один, чтобы возвестить тебе.
\vs Job 1:17 Еще он говорил, как приходит другой и сказывает: Халдеи расположились тремя отрядами и бросились на верблюдов и взяли их, а отроков поразили острием меча; и спасся только я один, чтобы возвестить тебе.
\vs Job 1:18 Еще этот говорил, приходит другой и сказывает: сыновья твои и дочери твои ели и вино пили в доме первородного брата своего;
\vs Job 1:19 и вот, большой ветер пришел от пустыни и охватил четыре угла дома, и дом упал на отроков, и они умерли; и спасся только я один, чтобы возвестить тебе.
\rsbpar\vs Job 1:20 Тогда Иов встал и разодрал верхнюю одежду свою, остриг голову свою и пал на землю и поклонился
\vs Job 1:21 и сказал: наг я вышел из чрева матери моей, наг и возвращусь. Господь дал, Господь и взял; [как угодно было Господу, так и сделалось;] да будет имя Господне благословенно!
\vs Job 1:22 Во всем этом не согрешил Иов и не произнес ничего неразумного о Боге.
\vs Job 2:1 Был день, когда пришли сыны Божии предстать пред Господа; между ними пришел и сатана предстать пред Господа.
\vs Job 2:2 И сказал Господь сатане: откуда ты пришел? И отвечал сатана Господу и сказал: я ходил по земле и обошел ее.
\vs Job 2:3 И сказал Господь сатане: обратил ли ты внимание твое на раба Моего Иова? ибо нет такого, как он, на земле: человек непорочный, справедливый, богобоязненный и удаляющийся от зла, и доселе тверд в своей непорочности; а ты возбуждал Меня против него, чтобы погубить его безвинно.
\vs Job 2:4 И отвечал сатана Господу и сказал: кожу за кожу, а за жизнь свою отдаст человек все, что есть у него;
\vs Job 2:5 но простри руку Твою и коснись кости его и плоти его,~--- благословит ли он Тебя?
\vs Job 2:6 И сказал Господь сатане: вот, он в руке твоей, только душу его сбереги.
\rsbpar\vs Job 2:7 И отошел сатана от лица Господня и поразил Иова проказою лютою от подошвы ноги его по самое темя его.
\vs Job 2:8 И взял он себе черепицу, чтобы скоблить себя ею, и сел в пепел [вне селения].
\vs Job 2:9 И сказала ему жена его: ты все еще тверд в непорочности твоей! похули Бога и умри.\fns{Этот стих по переводу 70-ти: По многом времени сказала ему жена его: доколе ты будешь терпеть? Вот, подожду еще немного в надежде спасения моего. Ибо погибли с земли память твоя, сыновья и дочери, болезни чрева моего и труды, которыми напрасно трудилась. Сам ты сидишь в смраде червей, проводя ночь без покрова, а я скитаюсь и служу, перехожу с места на место, из дома в дом, ожидая, когда зайдет солнце, чтобы успокоиться от трудов моих и болезней, которые ныне удручают меня. Но скажи некое слово к Богу и умри.}
\vs Job 2:10 Но он сказал ей: ты говоришь как одна из безумных: неужели доброе мы будем принимать от Бога, а злого не будем принимать? Во всем этом не согрешил Иов устами своими.
\vs Job 2:11 И услышали трое друзей Иова о всех этих несчастьях, постигших его, и пошли каждый из своего места: Елифаз Феманитянин, Вилдад Савхеянин и Софар Наамитянин, и сошлись, чтобы идти вместе сетовать с ним и утешать его.
\vs Job 2:12 И подняв глаза свои издали, они не узнали его; и возвысили голос свой и зарыдали; и разодрал каждый верхнюю одежду свою, и бросали пыль над головами своими к небу.
\vs Job 2:13 И сидели с ним на земле семь дней и семь ночей; и никто не говорил ему ни слова, ибо видели, что страдание его весьма велико.
\vs Job 3:1 После того открыл Иов уста свои и проклял день свой.
\vs Job 3:2 И начал Иов и сказал:
\vs Job 3:3 погибни день, в который я родился, и ночь, в которую сказано: зачался человек!
\vs Job 3:4 День тот да будет тьмою; да не взыщет его Бог свыше, и да не воссияет над ним свет!
\vs Job 3:5 Да омрачит его тьма и тень смертная, да обложит его туча, да страшатся его, как палящего зноя!
\vs Job 3:6 Ночь та,~--- да обладает ею мрак, да не сочтется она в днях года, да не войдет в число месяцев!
\vs Job 3:7 О! ночь та~--- да будет она безлюдна; да не войдет в нее веселье!
\vs Job 3:8 Да проклянут ее проклинающие день, способные разбудить левиафана!
\vs Job 3:9 Да померкнут звезды рассвета ее: пусть ждет она света, и он не приходит, и да не увидит она ресниц денницы
\vs Job 3:10 за то, что не затворила дверей чрева \bibemph{матери} моей и не сокрыла горести от очей моих!
\vs Job 3:11 Для чего не умер я, выходя из утробы, и не скончался, когда вышел из чрева?
\vs Job 3:12 Зачем приняли меня колени? зачем было мне сосать сосцы?
\vs Job 3:13 Теперь бы лежал я и почивал; спал бы, и мне было бы покойно
\vs Job 3:14 с царями и советниками земли, которые застраивали для себя пустыни,
\vs Job 3:15 или с князьями, у которых было золото, и которые наполняли домы свои серебром;
\vs Job 3:16 или, как выкидыш сокрытый, я не существовал бы, как младенцы, не увидевшие света.
\vs Job 3:17 Там беззаконные перестают наводить страх, и там отдыхают истощившиеся в силах.
\vs Job 3:18 Там узники вместе наслаждаются покоем и не слышат криков приставника.
\vs Job 3:19 Малый и великий там равны, и раб свободен от господина своего.
\vs Job 3:20 На что дан страдальцу свет, и жизнь огорченным душею,
\vs Job 3:21 которые ждут смерти, и нет ее, которые вырыли бы ее охотнее, нежели клад,
\vs Job 3:22 обрадовались бы до восторга, восхитились бы, что нашли гроб?
\vs Job 3:23 \bibemph{На что дан свет} человеку, которого путь закрыт, и которого Бог окружил мраком?
\vs Job 3:24 Вздохи мои предупреждают хлеб мой, и стоны мои льются, как вода,
\vs Job 3:25 ибо ужасное, чего я ужасался, то и постигло меня; и чего я боялся, то и пришло ко мне.
\vs Job 3:26 Нет мне мира, нет покоя, нет отрады: постигло несчастье.
\vs Job 4:1 И отвечал Елифаз Феманитянин и сказал:
\vs Job 4:2 \bibemph{если} попытаемся мы \bibemph{сказать} к тебе слово,~--- не тяжело ли будет тебе? Впрочем кто может возбранить слову!
\vs Job 4:3 Вот, ты наставлял многих и опустившиеся руки поддерживал,
\vs Job 4:4 падающего восставляли слова твои, и гнущиеся колени ты укреплял.
\vs Job 4:5 А теперь дошло до тебя, и ты изнемог; коснулось тебя, и ты упал духом.
\vs Job 4:6 Богобоязненность твоя не должна ли быть твоею надеждою, и непорочность путей твоих~--- упованием твоим?
\vs Job 4:7 Вспомни же, погибал ли кто невинный, и где праведные бывали искореняемы?
\vs Job 4:8 Как я видал, то оравшие нечестие и сеявшие зло пожинают его;
\vs Job 4:9 от дуновения Божия погибают и от духа гнева Его исчезают.
\vs Job 4:10 Рев льва и голос рыкающего \bibemph{умолкает}, и зубы скимнов сокрушаются;
\vs Job 4:11 могучий лев погибает без добычи, и дети львицы рассеиваются.
\vs Job 4:12 И вот, ко мне тайно принеслось слово, и ухо мое приняло нечто от него.
\vs Job 4:13 Среди размышлений о ночных видениях, когда сон находит на людей,
\vs Job 4:14 объял меня ужас и трепет и потряс все кости мои.
\vs Job 4:15 И дух прошел надо мною; дыбом стали волосы на мне.
\vs Job 4:16 Он стал,~--- но я не распознал вида его,~--- только облик был пред глазами моими; тихое веяние,~--- и я слышу голос:
\vs Job 4:17 человек праведнее ли Бога? и муж чище ли Творца своего?
\vs Job 4:18 Вот, Он и слугам Своим не доверяет и в Ангелах Своих усматривает недостатки:
\vs Job 4:19 тем более~--- в обитающих в храминах из брения, которых основание прах, которые истребляются скорее моли.
\vs Job 4:20 Между утром и вечером они распадаются; не увидишь, как они вовсе исчезнут.
\vs Job 4:21 Не погибают ли с ними и достоинства их? Они умирают, не достигнув мудрости.
\vs Job 5:1 Взывай, если есть отвечающий тебе. И к кому из святых обратишься ты?
\vs Job 5:2 Так, глупца убивает гневливость, и несмысленного губит раздражительность.
\vs Job 5:3 Видел я, как глупец укореняется, и тотчас проклял дом его.
\vs Job 5:4 Дети его далеки от счастья, их будут бить у ворот, и не будет заступника.
\vs Job 5:5 Жатву его съест голодный и из-за терна возьмет ее, и жаждущие поглотят имущество его.
\vs Job 5:6 Так, не из праха выходит горе, и не из земли вырастает беда;
\vs Job 5:7 но человек рождается на страдание, \bibemph{как} искры, чтобы устремляться вверх.
\vs Job 5:8 Но я к Богу обратился бы, предал бы дело мое Богу,
\vs Job 5:9 Который творит дела великие и неисследимые, чудные без числа,
\vs Job 5:10 дает дождь на лице земли и посылает воды на лице полей;
\vs Job 5:11 униженных поставляет на высоту, и сетующие возносятся во спасение.
\vs Job 5:12 Он разрушает замыслы коварных, и руки их не довершают предприятия.
\vs Job 5:13 Он уловляет мудрецов их же лукавством, и совет хитрых становится тщетным:
\vs Job 5:14 днем они встречают тьму и в полдень ходят ощупью, как ночью.
\vs Job 5:15 Он спасает бедного от меча, от уст их и от руки сильного.
\vs Job 5:16 И есть несчастному надежда, и неправда затворяет уста свои.
\vs Job 5:17 Блажен человек, которого вразумляет Бог, и потому наказания Вседержителева не отвергай,
\vs Job 5:18 ибо Он причиняет раны и Сам обвязывает их; Он поражает, и Его же руки врачуют.
\vs Job 5:19 В шести бедах спасет тебя, и в седьмой не коснется тебя зло.
\vs Job 5:20 Во время голода избавит тебя от смерти, и на войне~--- от руки меча.
\vs Job 5:21 От бича языка укроешь себя и не убоишься опустошения, когда оно придет.
\vs Job 5:22 Опустошению и голоду посмеешься и зверей земли не убоишься,
\vs Job 5:23 ибо с камнями полевыми у тебя союз, и звери полевые в мире с тобою.
\vs Job 5:24 И узн\acc{а}ешь, что шатер твой в безопасности, и будешь смотреть за домом твоим, и не согрешишь.
\vs Job 5:25 И увидишь, что семя твое многочисленно, и отрасли твои, как трава на земле.
\vs Job 5:26 Войдешь во гроб в зрелости, как укладываются снопы пшеницы в свое время.
\vs Job 5:27 Вот, что мы дознали; так оно и есть; выслушай это и заметь для себя.
\vs Job 6:1 И отвечал Иов и сказал:
\vs Job 6:2 о, если бы верно взвешены были вопли мои, и вместе с ними положили на весы страдание мое!
\vs Job 6:3 Оно верно перетянуло бы песок морей! Оттого слова мои неистовы.
\vs Job 6:4 Ибо стрелы Вседержителя во мне; яд их пьет дух мой; ужасы Божии ополчились против меня.
\vs Job 6:5 Ревет ли дикий осел на траве? мычит ли бык у месива своего?
\vs Job 6:6 Едят ли безвкусное без соли, и есть ли вкус в яичном белке?
\vs Job 6:7 До чего не хотела коснуться душа моя, то составляет отвратительную пищу мою.
\vs Job 6:8 О, когда бы сбылось желание мое и чаяние мое исполнил Бог!
\vs Job 6:9 О, если бы благоволил Бог сокрушить меня, простер руку Свою и сразил меня!
\vs Job 6:10 Это было бы еще отрадою мне, и я крепился бы в моей беспощадной болезни, ибо я не отвергся изречений Святаго.
\vs Job 6:11 Что за сила у меня, чтобы надеяться мне? и какой конец, чтобы длить мне жизнь мою?
\vs Job 6:12 Твердость ли камней твердость моя? и медь ли плоть моя?
\vs Job 6:13 Есть ли во мне помощь для меня, и есть ли для меня какая опора?
\vs Job 6:14 К страждущему должно быть сожаление от друга его, если только он не оставил страха к Вседержителю.
\vs Job 6:15 Но братья мои неверны, как поток, как быстро текущие ручьи,
\vs Job 6:16 которые черны от льда и в которых скрывается снег.
\vs Job 6:17 Когда становится тепло, они умаляются, а во время жары исчезают с мест своих.
\vs Job 6:18 Уклоняют они направление путей своих, заходят в пустыню и теряются;
\vs Job 6:19 смотрят на них дороги Фемайские, надеются на них пути Савейские,
\vs Job 6:20 но остаются пристыженными в своей надежде; приходят туда и от стыда краснеют.
\vs Job 6:21 Так и вы теперь ничто: увидели страшное и испугались.
\vs Job 6:22 Говорил ли я: дайте мне, или от достатка вашего заплатите за меня;
\vs Job 6:23 и избавьте меня от руки врага, и от руки мучителей выкупите меня?
\vs Job 6:24 Науч\acc{и}те меня, и я замолчу; укажите, в чем я погрешил.
\vs Job 6:25 Как сильны слова правды! Но что доказывают обличения ваши?
\vs Job 6:26 Вы придумываете речи для обличения? На ветер пускаете слова ваши.
\vs Job 6:27 Вы нападаете на сироту и роете яму другу вашему.
\vs Job 6:28 Но прошу вас, взгляните на меня; буду ли я говорить ложь пред лицем вашим?
\vs Job 6:29 Пересмотрите, есть ли неправда? пересмотрите,~--- правда моя.
\vs Job 6:30 Есть ли на языке моем неправда? Неужели гортань моя не может различить горечи?
\vs Job 7:1 Не определено ли человеку время на земле, и дни его не то же ли, что дни наемника?
\vs Job 7:2 Как раб жаждет тени, и как наемник ждет окончания работы своей,
\vs Job 7:3 так я получил в удел месяцы суетные, и ночи горестные отчислены мне.
\vs Job 7:4 Когда ложусь, то говорю: <<когда-то встану?>>, а вечер длится, и я ворочаюсь досыта до самого рассвета.
\vs Job 7:5 Тело мое одето червями и пыльными струпами; кожа моя лопается и гноится.
\vs Job 7:6 Дни мои бегут скорее челнока и кончаются без надежды.
\vs Job 7:7 Вспомни, что жизнь моя дуновение, что око мое не возвратится видеть доброе.
\vs Job 7:8 Не увидит меня око видевшего меня; очи Твои на меня,~--- и нет меня.
\vs Job 7:9 Редеет облако и уходит; так нисшедший в преисподнюю не выйдет,
\vs Job 7:10 не возвратится более в дом свой, и место его не будет уже знать его.
\vs Job 7:11 Не буду же я удерживать уст моих; буду говорить в стеснении духа моего; буду жаловаться в горести души моей.
\vs Job 7:12 Разве я море или морское чудовище, что Ты поставил надо мною стражу?
\vs Job 7:13 Когда подумаю: утешит меня постель моя, унесет горесть мою ложе мое,
\vs Job 7:14 Ты страшишь меня снами и видениями пугаешь меня;
\vs Job 7:15 и душа моя желает лучше прекращения дыхания, лучше смерти, нежели \bibemph{сбережения} костей моих.
\vs Job 7:16 Опротивела мне жизнь. Не вечно жить мне. Отступи от меня, ибо дни мои суета.
\vs Job 7:17 Что такое человек, что Ты столько ценишь его и обращаешь на него внимание Твое,
\vs Job 7:18 посещаешь его каждое утро, каждое мгновение испытываешь его?
\vs Job 7:19 Доколе же Ты не оставишь, доколе не отойдешь от меня, доколе не дашь мне проглотить слюну мою?
\vs Job 7:20 Если я согрешил, то что я сделаю Тебе, страж человеков! Зачем Ты поставил меня противником Себе, так что я стал самому себе в тягость?
\vs Job 7:21 И зачем бы не простить мне греха и не снять с меня беззакония моего? ибо, вот, я лягу в прахе; завтра поищешь меня, и меня нет.
\vs Job 8:1 И отвечал Вилдад Савхеянин и сказал:
\vs Job 8:2 долго ли ты будешь говорить так?~--- слов\acc{а} уст твоих бурный ветер!
\vs Job 8:3 Неужели Бог извращает суд, и Вседержитель превращает правду?
\vs Job 8:4 Если сыновья твои согрешили пред Ним, то Он и предал их в руку беззакония их.
\vs Job 8:5 Если же ты взыщешь Бога и помолишься Вседержителю,
\vs Job 8:6 и если ты чист и прав, то Он ныне же встанет над тобою и умиротворит жилище правды твоей.
\vs Job 8:7 И если вначале у тебя было мало, то впоследствии будет весьма много.
\vs Job 8:8 Ибо спроси у прежних родов и вникни в наблюдения отцов их;
\vs Job 8:9 а мы~--- вчерашние и ничего не знаем, потому что наши дни на земле тень.
\vs Job 8:10 Вот, они научат тебя, скажут тебе и от сердца своего произнесут слова:
\vs Job 8:11 поднимается ли тростник без влаги? растет ли камыш без воды?
\vs Job 8:12 Еще он в свежести своей и не срезан, а прежде всякой травы засыхает.
\vs Job 8:13 Таковы пути всех забывающих Бога, и надежда лицемера погибнет;
\vs Job 8:14 упование его подсечено, и уверенность его~--- дом паука.
\vs Job 8:15 Обопрется о дом свой и не устоит; ухватится за него и не удержится.
\vs Job 8:16 Зеленеет он пред солнцем, за сад простираются ветви его;
\vs Job 8:17 в кучу \bibemph{камней} вплетаются корни его, между камнями врезываются.
\vs Job 8:18 Но когда вырвут его с места его, оно откажется от него: <<я не видало тебя!>>
\vs Job 8:19 Вот радость пути его! а из земли вырастают другие.
\vs Job 8:20 Видишь, Бог не отвергает непорочного и не поддерживает рук\acc{и} злодеев.
\vs Job 8:21 Он еще наполнит смехом уста твои и губы твои радостным восклицанием.
\vs Job 8:22 Ненавидящие тебя облекутся в стыд, и шатра нечестивых не станет.
\vs Job 9:1 И отвечал Иов и сказал:
\vs Job 9:2 правда! знаю, что так; но как оправдается человек пред Богом?
\vs Job 9:3 Если захочет вступить в прение с Ним, то не ответит Ему ни на одно из тысячи.
\vs Job 9:4 Премудр сердцем и могущ силою; кто восставал против Него и оставался в покое?
\vs Job 9:5 Он передвигает горы, и не узна\acc{ю}т их: Он превращает их в гневе Своем;
\vs Job 9:6 сдвигает землю с места ее, и столбы ее дрожат;
\vs Job 9:7 скажет солнцу,~--- и не взойдет, и на звезды налагает печать.
\vs Job 9:8 Он один распростирает небеса и ходит по высотам моря;
\vs Job 9:9 сотворил Ас, Кесиль и Хима\fns{Созвездия, соответствующие нынешним названиям: Медведицы, Ориона и Плеяд.} и тайники юга;
\vs Job 9:10 делает великое, неисследимое и чудное без числа!
\vs Job 9:11 Вот, Он пройдет предо мною, и не увижу Его; пронесется, и не замечу Его.
\vs Job 9:12 Возьмет, и кто возбранит Ему? кто скажет Ему: что Ты делаешь?
\vs Job 9:13 Бог не отвратит гнева Своего; пред Ним падут поборники гордыни.
\vs Job 9:14 Тем более могу ли я отвечать Ему и приискивать себе слова пред Ним?
\vs Job 9:15 Хотя бы я и прав был, но не буду отвечать, а буду умолять Судию моего.
\vs Job 9:16 Если бы я воззвал, и Он ответил мне,~--- я не поверил бы, что голос мой услышал Тот,
\vs Job 9:17 Кто в вихре разит меня и умножает безвинно мои раны,
\vs Job 9:18 не дает мне перевести духа, но пресыщает меня горестями.
\vs Job 9:19 Если \bibemph{действовать} силою, то Он могуществен; если судом, кто сведет меня с Ним?
\vs Job 9:20 Если я буду оправдываться, то мои же уста обвинят меня; \bibemph{если} я невинен, то Он призн\acc{а}ет меня виновным.
\vs Job 9:21 Невинен я; не хочу знать души моей, презираю жизнь мою.
\vs Job 9:22 Все одно; поэтому я сказал, что Он губит и непорочного и виновного.
\vs Job 9:23 Если этого поражает Он бичом вдруг, то пытке невинных посмевается.
\vs Job 9:24 Земля отдана в руки нечестивых; лица судей ее Он закрывает. Если не Он, то кто же?
\vs Job 9:25 Дни мои быстрее гонца,~--- бегут, не видят добра,
\vs Job 9:26 несутся, как легкие ладьи, как орел стремится на добычу.
\vs Job 9:27 Если сказать мне: забуду я жалобы мои, отложу мрачный вид свой и ободрюсь;
\vs Job 9:28 то трепещу всех страданий моих, зная, что Ты не объявишь меня невинным.
\vs Job 9:29 Если же я виновен, то для чего напрасно томлюсь?
\vs Job 9:30 Хотя бы я омылся и снежною водою и совершенно очистил руки мои,
\vs Job 9:31 то и тогда Ты погрузишь меня в грязь, и возгнушаются мною одежды мои.
\vs Job 9:32 Ибо Он не человек, как я, чтоб я мог отвечать Ему и идти вместе с Ним на суд!
\vs Job 9:33 Нет между нами посредника, который положил бы руку свою на обоих нас.
\vs Job 9:34 Да отстранит Он от меня жезл Свой, и страх Его да не ужасает меня,~---
\vs Job 9:35 и тогда я буду говорить и не убоюсь Его, ибо я не таков сам в себе.
\vs Job 10:1 Опротивела душе моей жизнь моя; предамся печали моей; буду говорить в горести души моей.
\vs Job 10:2 Скажу Богу: не обвиняй меня; объяви мне, за что Ты со мною борешься?
\vs Job 10:3 Хорошо ли для Тебя, что Ты угнетаешь, что презираешь дело рук Твоих, а на совет нечестивых посылаешь свет?
\vs Job 10:4 Разве у Тебя плотские очи, и Ты смотришь, как смотрит человек?
\vs Job 10:5 Разве дни Твои, как дни человека, или лета Твои, как дни мужа,
\vs Job 10:6 что Ты ищешь порока во мне и допытываешься греха во мне,
\vs Job 10:7 хотя знаешь, что я не беззаконник, и что некому избавить меня от руки Твоей?
\vs Job 10:8 Твои руки трудились надо мною и образовали всего меня кругом,~--- и Ты губишь меня?
\vs Job 10:9 Вспомни, что Ты, как глину, обделал меня, и в прах обращаешь меня?
\vs Job 10:10 Не Ты ли вылил меня, как молоко, и, как творог, сгустил меня,
\vs Job 10:11 кожею и плотью одел меня, костями и жилами скрепил меня,
\vs Job 10:12 жизнь и милость даровал мне, и попечение Твое хранило дух мой?
\vs Job 10:13 Но и то скрывал Ты в сердце Своем,~--- знаю, что это было у Тебя,~---
\vs Job 10:14 что если я согрешу, Ты заметишь и не оставишь греха моего без наказания.
\vs Job 10:15 Если я виновен, горе мне! если и прав, то не осмелюсь поднять головы моей. Я пресыщен унижением; взгляни на бедствие мое:
\vs Job 10:16 оно увеличивается. Ты гонишься за мною, как лев, и снова нападаешь на меня и чудным являешься во мне.
\vs Job 10:17 Выводишь новых свидетелей Твоих против меня; усиливаешь гнев Твой на меня; и беды, одни за другими, ополчаются против меня.
\vs Job 10:18 И зачем Ты вывел меня из чрева? пусть бы я умер, когда еще ничей глаз не видел меня;
\vs Job 10:19 пусть бы я, как небывший, из чрева перенесен был во гроб!
\vs Job 10:20 Не малы ли дни мои? Оставь, отступи от меня, чтобы я немного ободрился,
\vs Job 10:21 прежде нежели отойду,~--- и уже не возвращусь,~--- в страну тьмы и сени смертной,
\vs Job 10:22 в страну мрака, каков есть мрак тени смертной, где нет устройства, \bibemph{где} темно, как самая тьма.
\vs Job 11:1 И отвечал Софар Наамитянин и сказал:
\vs Job 11:2 разве на множество слов нельзя дать ответа, и разве человек многоречивый прав?
\vs Job 11:3 Пустословие твое заставит ли молчать мужей, чтобы ты глумился, и некому было постыдить тебя?
\vs Job 11:4 Ты сказал: суждение мое верно, и чист я в очах Твоих.
\vs Job 11:5 Но если бы Бог возглаголал и отверз уста Свои к тебе
\vs Job 11:6 и открыл тебе тайны премудрости, что тебе вдвое больше следовало бы понести! Итак знай, что Бог для тебя некоторые из беззаконий твоих предал забвению.
\vs Job 11:7 Можешь ли ты исследованием найти Бога? Можешь ли совершенно постигнуть Вседержителя?
\vs Job 11:8 Он превыше небес,~--- что можешь сделать? глубже преисподней,~--- что можешь узнать?
\vs Job 11:9 Длиннее земли мера Его и шире моря.
\vs Job 11:10 Если Он пройдет и заключит кого в оковы и представит на суд, то кто отклонит Его?
\vs Job 11:11 Ибо Он знает людей лживых и видит беззаконие, и оставит ли его без внимания?
\vs Job 11:12 Но пустой человек мудрствует, хотя человек рождается подобно дикому осленку.
\vs Job 11:13 Если ты управишь сердце твое и прострешь к Нему руки твои,
\vs Job 11:14 и если есть порок в руке твоей, а ты удалишь его и не дашь беззаконию обитать в шатрах твоих,
\vs Job 11:15 то поднимешь незапятнанное лице твое и будешь тверд и не будешь бояться.
\vs Job 11:16 Тогда забудешь горе: как о воде протекшей, будешь вспоминать о нем.
\vs Job 11:17 И яснее полдня пойдет жизнь твоя; просветлеешь, как утро.
\vs Job 11:18 И будешь спокоен, ибо есть надежда; ты огражден, и можешь спать безопасно.
\vs Job 11:19 Будешь лежать, и не будет устрашающего, и многие будут заискивать у тебя.
\vs Job 11:20 А глаза беззаконных истают, и убежище пропадет у них, и надежда их исчезнет.
\vs Job 12:1 И отвечал Иов и сказал:
\vs Job 12:2 подлинно, \bibemph{только} вы люди, и с вами умрет мудрость!
\vs Job 12:3 И у меня \bibemph{есть} сердце, как у вас; не ниже я вас; и кто не знает того же?
\vs Job 12:4 Посмешищем стал я для друга своего, я, который взывал к Богу, и которому Он отвечал, посмешищем~--- \bibemph{человек} праведный, непорочный.
\vs Job 12:5 Так презрен по мыслям сидящего в покое факел, приготовленный для спотыкающихся ногами.
\vs Job 12:6 Покойны шатры у грабителей и безопасны у раздражающих Бога, которые как бы Бога носят в руках своих.
\vs Job 12:7 И подлинно: спроси у скота, и научит тебя, у птицы небесной, и возвестит тебе;
\vs Job 12:8 или побеседуй с землею, и наставит тебя, и скажут тебе рыбы морские.
\vs Job 12:9 Кто во всем этом не узнает, что рука Господа сотворила сие?
\vs Job 12:10 В Его руке душа всего живущего и дух всякой человеческой плоти.
\vs Job 12:11 Не ухо ли разбирает слова, и не язык ли распознает вкус пищи?
\vs Job 12:12 В старцах~--- мудрость, и в долголетних~--- разум.
\vs Job 12:13 У Него премудрость и сила; Его совет и разум.
\vs Job 12:14 Что Он разрушит, то не построится; кого Он заключит, тот не высвободится.
\vs Job 12:15 Остановит воды, и все высохнет; пустит их, и превратят землю.
\vs Job 12:16 У Него могущество и премудрость, пред Ним заблуждающийся и вводящий в заблуждение.
\vs Job 12:17 Он приводит советников в необдуманность и судей делает глупыми.
\vs Job 12:18 Он лишает перевязей царей и поясом обвязывает чресла их;
\vs Job 12:19 князей лишает достоинства и низвергает храбрых;
\vs Job 12:20 отнимает язык у велеречивых и старцев лишает смысла;
\vs Job 12:21 покрывает стыдом знаменитых и силу могучих ослабляет;
\vs Job 12:22 открывает глубокое из среды тьмы и выводит на свет тень смертную;
\vs Job 12:23 умножает народы и истребляет их; рассевает народы и собирает их;
\vs Job 12:24 отнимает ум у глав народа земли и оставляет их блуждать в пустыне, где нет пути:
\vs Job 12:25 ощупью ходят они во тьме без света и шатаются, как пьяные.
\vs Job 13:1 Вот, все \bibemph{это} видело око мое, слышало ухо мое и заметило для себя.
\vs Job 13:2 Сколько знаете вы, знаю и я: не ниже я вас.
\vs Job 13:3 Но я к Вседержителю хотел бы говорить и желал бы состязаться с Богом.
\vs Job 13:4 А вы сплетчики лжи; все вы бесполезные врачи.
\vs Job 13:5 О, если бы вы только молчали! это было бы \bibemph{вменено} вам в мудрость.
\vs Job 13:6 Выслушайте же рассуждения мои и вникните в возражение уст моих.
\vs Job 13:7 Надлежало ли вам ради Бога говорить неправду и для Него говорить ложь?
\vs Job 13:8 Надлежало ли вам быть лицеприятными к Нему и за Бога так препираться?
\vs Job 13:9 Хорошо ли будет, когда Он испытает вас? Обманете ли Его, как обманывают человека?
\vs Job 13:10 Строго накажет Он вас, хотя вы и скрытно лицемерите.
\vs Job 13:11 Неужели величие Его не устрашает вас, и страх Его не нападает на вас?
\vs Job 13:12 Напоминания ваши подобны пеплу; оплоты ваши~--- оплоты глиняные.
\vs Job 13:13 Замолчите предо мною, и я буду говорить, что бы ни постигло меня.
\vs Job 13:14 Для чего мне терзать тело мое зубами моими и душу мою полагать в руку мою?
\vs Job 13:15 Вот, Он убивает меня, но я буду надеяться; я желал бы только отстоять пути мои пред лицем Его!
\vs Job 13:16 И это уже в оправдание мне, потому что лицемер не пойдет пред лице Его!
\vs Job 13:17 Выслушайте внимательно слово мое и объяснение мое ушами вашими.
\vs Job 13:18 Вот, я завел судебное дело: знаю, что буду прав.
\vs Job 13:19 Кто в состоянии оспорить меня? Ибо я скоро умолкну и испущу дух.
\vs Job 13:20 Двух только \bibemph{вещей} не делай со мною, и тогда я не буду укрываться от лица Твоего:
\vs Job 13:21 удали от меня руку Твою, и ужас Твой да не потрясает меня.
\vs Job 13:22 Тогда зови, и я буду отвечать, или буду говорить я, а Ты отвечай мне.
\vs Job 13:23 Сколько у меня пороков и грехов? покажи мне беззаконие мое и грех мой.
\vs Job 13:24 Для чего скрываешь лице Твое и считаешь меня врагом Тебе?
\vs Job 13:25 Не сорванный ли листок Ты сокрушаешь и не сухую ли соломинку преследуешь?
\vs Job 13:26 Ибо Ты пишешь на меня горькое и вменяешь мне грехи юности моей,
\vs Job 13:27 и ставишь в колоду ноги мои и подстерегаешь все стези мои,~--- гонишься по следам ног моих.
\vs Job 13:28 А он, как гниль, распадается, как одежда, изъеденная молью.
\vs Job 14:1 Человек, рожденный женою, краткодневен и пресыщен печалями:
\vs Job 14:2 как цветок, он выходит и опадает; убегает, как тень, и не останавливается.
\vs Job 14:3 И на него-то Ты отверзаешь очи Твои, и меня ведешь на суд с Тобою?
\vs Job 14:4 Кто родится чистым от нечистого? Ни один.
\vs Job 14:5 Если дни ему определены, и число месяцев его у Тебя, если Ты положил ему предел, которого он не перейдет,
\vs Job 14:6 то уклонись от него: пусть он отдохнет, доколе не окончит, как наемник, дня своего.
\vs Job 14:7 Для дерева есть надежда, что оно, если и будет срублено, снова оживет, и отрасли от него \bibemph{выходить} не перестанут:
\vs Job 14:8 если и устарел в земле корень его, и пень его замер в пыли,
\vs Job 14:9 но, лишь почуяло воду, оно дает отпрыски и пускает ветви, как бы вновь посаженное.
\vs Job 14:10 А человек умирает и распадается; отошел, и где он?
\vs Job 14:11 Уходят воды из озера, и река иссякает и высыхает:
\vs Job 14:12 так человек ляжет и не встанет; до скончания неба он не пробудится и не воспрянет от сна своего.
\vs Job 14:13 О, если бы Ты в преисподней сокрыл меня и укрывал меня, пока пройдет гнев Твой, положил мне срок и потом вспомнил обо мне!
\vs Job 14:14 Когда умрет человек, то будет ли он опять жить? Во все дни определенного мне времени я ожидал бы, пока придет мне смена.
\vs Job 14:15 Воззвал бы Ты, и я дал бы Тебе ответ, и Ты явил бы благоволение творению рук Твоих;
\vs Job 14:16 ибо тогда Ты исчислял бы шаги мои и не подстерегал бы греха моего;
\vs Job 14:17 в свитке было бы запечатано беззаконие мое, и Ты закрыл бы вину мою.
\vs Job 14:18 Но гора падая разрушается, и скала сходит с места своего;
\vs Job 14:19 вода стирает камни; разлив ее смывает земную пыль: так и надежду человека Ты уничтожаешь.
\vs Job 14:20 Теснишь его до конца, и он уходит; изменяешь ему лице и отсылаешь его.
\vs Job 14:21 В чести ли дети его~--- он не знает, унижены ли~--- он не замечает;
\vs Job 14:22 но плоть его на нем болит, и душа его в нем страдает.
\vs Job 15:1 И отвечал Елифаз Феманитянин и сказал:
\vs Job 15:2 станет ли мудрый отвечать знанием пустым и наполнять чрево свое ветром палящим,
\vs Job 15:3 оправдываться словами бесполезными и речью, не имеющею никакой силы?
\vs Job 15:4 Да ты отложил и страх и за малость считаешь речь к Богу.
\vs Job 15:5 Нечестие твое настроило так уста твои, и ты избрал язык лукавых.
\vs Job 15:6 Тебя обвиняют уста твои, а не я, и твой язык говорит против тебя.
\vs Job 15:7 Разве ты первым человеком родился и прежде холмов создан?
\vs Job 15:8 Разве совет Божий ты слышал и привлек к себе премудрость?
\vs Job 15:9 Что знаешь ты, чего бы не знали мы? что разумеешь ты, чего не было бы и у нас?
\vs Job 15:10 И седовласый и старец есть между нами, днями превышающий отца твоего.
\vs Job 15:11 Разве малость для тебя утешения Божии? И это неизвестно тебе?
\vs Job 15:12 К чему порывает тебя сердце твое, и к чему так гордо смотришь?
\vs Job 15:13 Что устремляешь против Бога дух твой и устами твоими произносишь такие речи?
\vs Job 15:14 Что такое человек, чтоб быть ему чистым, и чтобы рожденному женщиною быть праведным?
\vs Job 15:15 Вот, Он и святым Своим не доверяет, и небеса нечисты в очах Его:
\vs Job 15:16 тем больше нечист и растлен человек, пьющий беззаконие, как воду.
\vs Job 15:17 Я буду говорить тебе, слушай меня; я расскажу тебе, что видел,
\vs Job 15:18 что слышали мудрые и не скрыли слышанного от отцов своих,
\vs Job 15:19 которым одним отдана была земля, и среди которых чужой не ходил.
\vs Job 15:20 Нечестивый мучит себя во все дни свои, и число лет закрыто от притеснителя;
\vs Job 15:21 звук ужасов в ушах его; среди мира идет на него губитель.
\vs Job 15:22 Он не надеется спастись от тьмы; видит пред собою меч.
\vs Job 15:23 Он скитается за куском хлеба повсюду; знает, что уже готов, в руках у него день тьмы.
\vs Job 15:24 Устрашает его нужда и теснота; одолевает его, как царь, приготовившийся к битве,
\vs Job 15:25 за то, что он простирал против Бога руку свою и противился Вседержителю,
\vs Job 15:26 устремлялся против Него с \bibemph{гордою} выею, под толстыми щитами своими;
\vs Job 15:27 потому что он покрыл лице свое жиром своим и обложил туком лядвеи свои.
\vs Job 15:28 И он селится в городах разоренных, в домах, в которых не живут, которые обречены на развалины.
\vs Job 15:29 Не пребудет он богатым, и не уцелеет имущество его, и не распрострется по земле приобретение его.
\vs Job 15:30 Не уйдет от тьмы; отрасли его иссушит пламя и дуновением уст своих увлечет его.
\vs Job 15:31 Пусть не доверяет суете заблудший, ибо суета будет и воздаянием ему.
\vs Job 15:32 Не в свой день он скончается, и ветви его не будут зеленеть.
\vs Job 15:33 Сбросит он, как виноградная лоза, недозрелую ягоду свою и, как маслина, стряхнет цвет свой.
\vs Job 15:34 Так опустеет дом нечестивого, и огонь пожрет шатры мздоимства.
\vs Job 15:35 Он зачал зло и родил ложь, и утроба его приготовляет обман.
\vs Job 16:1 И отвечал Иов и сказал:
\vs Job 16:2 слышал я много такого; жалкие утешители все вы!
\vs Job 16:3 Будет ли конец ветреным словам? и что побудило тебя так отвечать?
\vs Job 16:4 И я мог бы так же говорить, как вы, если бы душа ваша была на месте души моей; ополчался бы на вас словами и кивал бы на вас головою моею;
\vs Job 16:5 подкреплял бы вас языком моим и движением губ утешал бы.
\vs Job 16:6 Говорю ли я, не утоляется скорбь моя; перестаю ли, что отходит от меня?
\vs Job 16:7 Но ныне Он изнурил меня. Ты разрушил всю семью мою.
\vs Job 16:8 Ты покрыл меня морщинами во свидетельство против меня; восстает на меня изможденность моя, в лицо укоряет меня.
\vs Job 16:9 Гнев Его терзает и враждует против меня, скрежещет на меня зубами своими; неприятель мой острит на меня глаза свои.
\vs Job 16:10 Разинули на меня пасть свою; ругаясь бьют меня по щекам: все сговорились против меня.
\vs Job 16:11 Предал меня Бог беззаконнику и в руки нечестивым бросил меня.
\vs Job 16:12 Я был спокоен, но Он потряс меня; взял меня за шею и избил меня и поставил меня целью для Себя.
\vs Job 16:13 Окружили меня стрельцы Его; Он рассекает внутренности мои и не щадит, пролил на землю желчь мою,
\vs Job 16:14 пробивает во мне пролом за проломом, бежит на меня, как ратоборец.
\vs Job 16:15 Вретище сшил я на кожу мою и в прах положил голову мою.
\vs Job 16:16 Лицо мое побагровело от плача, и на веждах моих тень смерти,
\vs Job 16:17 при всем том, что нет хищения в руках моих, и молитва моя чиста.
\vs Job 16:18 Земля! не закрой моей крови, и да не будет места воплю моему.
\vs Job 16:19 И ныне вот на небесах Свидетель мой, и Заступник мой в вышних!
\vs Job 16:20 Многоречивые друзья мои! К Богу слезит око мое.
\vs Job 16:21 О, если бы человек мог иметь состязание с Богом, как сын человеческий с ближним своим!
\vs Job 16:22 Ибо летам моим приходит конец, и я отхожу в путь невозвратный.
\vs Job 17:1 Дыхание мое ослабело; дни мои угасают; гробы предо мною.
\vs Job 17:2 Если бы не насмешки их, то и среди споров их око мое пребывало бы спокойно.
\vs Job 17:3 Заступись, поручись \bibemph{Сам} за меня пред Собою! иначе кто поручится за меня?
\vs Job 17:4 Ибо Ты закрыл сердце их от разумения, и потому не дашь восторжествовать \bibemph{им}.
\vs Job 17:5 Кто обрекает друзей своих в добычу, у детей того глаза истают.
\vs Job 17:6 Он поставил меня притчею для народа и посмешищем для него.
\vs Job 17:7 Помутилось от горести око мое, и все члены мои, как тень.
\vs Job 17:8 Изумятся о сем праведные, и невинный вознегодует на лицемера.
\vs Job 17:9 Но праведник будет крепко держаться пути своего, и чистый руками будет больше и больше утверждаться.
\vs Job 17:10 Выступайте, все вы, и подойдите; не найду я мудрого между вами.
\vs Job 17:11 Дни мои прошли; думы мои~--- достояние сердца моего~--- разбиты.
\vs Job 17:12 А они ночь \bibemph{хотят} превратить в день, свет приблизить к лицу тьмы.
\vs Job 17:13 Если бы я и ожидать стал, то преисподняя~--- дом мой; во тьме постелю я постель мою;
\vs Job 17:14 гробу скажу: ты отец мой, червю: ты мать моя и сестра моя.
\vs Job 17:15 Где же после этого надежда моя? и ожидаемое мною кто увидит?
\vs Job 17:16 В преисподнюю сойдет она и будет покоиться со мною в прахе.
\vs Job 18:1 И отвечал Вилдад Савхеянин и сказал:
\vs Job 18:2 когда же положите вы конец таким речам? обдумайте, и потом будем говорить.
\vs Job 18:3 Зачем считаться нам за животных и быть униженными в собственных глазах ваших?
\vs Job 18:4 \bibemph{О ты}, раздирающий душу твою в гневе твоем! Неужели для тебя опустеть земле, и скале сдвинуться с места своего?
\vs Job 18:5 Да, свет у беззаконного потухнет, и не останется искры от огня его.
\vs Job 18:6 Померкнет свет в шатре его, и светильник его угаснет над ним.
\vs Job 18:7 Сократятся шаги могущества его, и низложит его собственный замысл его,
\vs Job 18:8 ибо он попадет в сеть своими ногами и по тенетам ходить будет.
\vs Job 18:9 Петля зацепит за ногу его, и грабитель уловит его.
\vs Job 18:10 Скрытно разложены по земле силки для него и западни на дороге.
\vs Job 18:11 Со всех сторон будут страшить его ужасы и заставят его бросаться туда и сюда.
\vs Job 18:12 Истощится от голода сила его, и гибель готова, сбоку у него.
\vs Job 18:13 Съест члены тела его, съест члены его первенец смерти.
\vs Job 18:14 Изгнана будет из шатра его надежда его, и это низведет его к царю ужасов.
\vs Job 18:15 Поселятся в шатре его, потому что он уже не его; жилище его посыпано будет серою.
\vs Job 18:16 Снизу подсохнут корни его, и сверху увянут ветви его.
\vs Job 18:17 Память о нем исчезнет с земли, и имени его не будет на площади.
\vs Job 18:18 Изгонят его из света во тьму и сотрут его с лица земли.
\vs Job 18:19 Ни сына его, ни внука не будет в народе его, и никого не останется в жилищах его.
\vs Job 18:20 О дне его ужаснутся потомки, и современники будут объяты трепетом.
\vs Job 18:21 Таковы жилища беззаконного, и таково место того, кто не знает Бога.
\vs Job 19:1 И отвечал Иов и сказал:
\vs Job 19:2 доколе будете мучить душу мою и терзать меня речами?
\vs Job 19:3 Вот, уже раз десять вы срамили меня и не стыдитесь теснить меня.
\vs Job 19:4 Если я и действительно погрешил, то погрешность моя при мне остается.
\vs Job 19:5 Если же вы хотите повеличаться надо мною и упрекнуть меня позором моим,
\vs Job 19:6 то знайте, что Бог ниспроверг меня и обложил меня Своею сетью.
\vs Job 19:7 Вот, я кричу: обида! и никто не слушает; вопию, и нет суда.
\vs Job 19:8 Он преградил мне дорогу, и не могу пройти, и на стези мои положил тьму.
\vs Job 19:9 Совлек с меня славу мою и снял венец с головы моей.
\vs Job 19:10 Кругом разорил меня, и я отхожу; и, как дерево, Он исторг надежду мою.
\vs Job 19:11 Воспылал на меня гневом Своим и считает меня между врагами Своими.
\vs Job 19:12 Полки Его пришли вместе и направили путь свой ко мне и расположились вокруг шатра моего.
\vs Job 19:13 Братьев моих Он удалил от меня, и знающие меня чуждаются меня.
\vs Job 19:14 Покинули меня близкие мои, и знакомые мои забыли меня.
\vs Job 19:15 Пришлые в доме моем и служанки мои чужим считают меня; посторонним стал я в глазах их.
\vs Job 19:16 Зову слугу моего, и он не откликается; устами моими я должен умолять его.
\vs Job 19:17 Дыхание мое опротивело жене моей, и я должен умолять ее ради детей чрева моего.
\vs Job 19:18 Даже малые дети презирают меня: поднимаюсь, и они издеваются надо мною.
\vs Job 19:19 Гнушаются мною все наперсники мои, и те, которых я любил, обратились против меня.
\vs Job 19:20 Кости мои прилипли к коже моей и плоти моей, и я остался только с кожею около зубов моих.
\vs Job 19:21 Помилуйте меня, помилуйте меня вы, друзья мои, ибо рука Божия коснулась меня.
\vs Job 19:22 Зачем и вы преследуете меня, как Бог, и плотью моею не можете насытиться?
\vs Job 19:23 О, если бы записаны были слова мои! Если бы начертаны были они в книге
\vs Job 19:24 резцом железным с оловом,~--- на вечное время на камне вырезаны были!
\vs Job 19:25 А я знаю, Искупитель мой жив, и Он в последний день восставит из праха распадающуюся кожу мою сию,
\vs Job 19:26 и я во плоти моей узрю Бога.
\vs Job 19:27 Я узрю Его сам; мои глаза, не глаза другого, увидят Его. Истаевает сердце мое в груди моей!
\vs Job 19:28 Вам надлежало бы сказать: зачем мы преследуем его? Как будто корень зла найден во мне.
\vs Job 19:29 Убойтесь меча, ибо меч есть отмститель неправды, и знайте, что есть суд.
\vs Job 20:1 И отвечал Софар Наамитянин и сказал:
\vs Job 20:2 размышления мои побуждают меня отвечать, и я поспешаю выразить их.
\vs Job 20:3 Упрек, позорный для меня, выслушал я, и дух разумения моего ответит за меня.
\vs Job 20:4 Разве не знаешь ты, что от века,~--- с того времени, как поставлен человек на земле,~---
\vs Job 20:5 веселье беззаконных кратковременно, и радость лицемера мгновенна?
\vs Job 20:6 Хотя бы возросло до небес величие его, и голова его касалась облаков,~---
\vs Job 20:7 как помет его, на веки пропадает он; видевшие его скажут: где он?
\vs Job 20:8 Как сон, улетит, и не найдут его; и, как ночное видение, исчезнет.
\vs Job 20:9 Глаз, видевший его, больше не увидит его, и уже не усмотрит его место его.
\vs Job 20:10 Сыновья его будут заискивать у нищих, и руки его возвратят похищенное им.
\vs Job 20:11 Кости его наполнены грехами юности его, и с ним лягут они в прах.
\vs Job 20:12 Если сладко во рту его зло, и он таит его под языком своим,
\vs Job 20:13 бережет и не бросает его, а держит его в устах своих,
\vs Job 20:14 то эта пища его в утробе его превратится в желчь аспидов внутри его.
\vs Job 20:15 Имение, которое он глотал, изблюет: Бог исторгнет его из чрева его.
\vs Job 20:16 Змеиный яд он сосет; умертвит его язык ехидны.
\vs Job 20:17 Не видать ему ручьев, рек, текущих медом и молоком!
\vs Job 20:18 Нажитое трудом возвратит, не проглотит; по мере имения его будет и расплата его, а он не порадуется.
\vs Job 20:19 Ибо он угнетал, отсылал бедных; захватывал домы, которых не строил;
\vs Job 20:20 не знал сытости во чреве своем и в жадности своей не щадил ничего.
\vs Job 20:21 Ничего не спаслось от обжорства его, зато не устоит счастье его.
\vs Job 20:22 В полноте изобилия будет тесно ему; всякая рука обиженного поднимется на него.
\vs Job 20:23 Когда будет чем наполнить утробу его, Он пошлет на него ярость гнева Своего и одождит на него болезни в плоти его.
\vs Job 20:24 Убежит ли он от оружия железного,~--- пронзит его лук медный;
\vs Job 20:25 станет вынимать \bibemph{стрелу},~--- и она выйдет из тела, выйдет, сверкая сквозь желчь его; ужасы смерти найдут на него!
\vs Job 20:26 Все мрачное сокрыто внутри его; будет пожирать его огонь, никем не раздуваемый; зло постигнет и оставшееся в шатре его.
\vs Job 20:27 Небо откроет беззаконие его, и земля восстанет против него.
\vs Job 20:28 Исчезнет стяжание дома его; все расплывется в день гнева Его.
\vs Job 20:29 Вот удел человеку беззаконному от Бога и наследие, определенное ему Вседержителем!
\vs Job 21:1 И отвечал Иов и сказал:
\vs Job 21:2 выслушайте внимательно речь мою, и это будет мне утешением от вас.
\vs Job 21:3 Потерпите меня, и я буду говорить; а после того, как поговорю, насмехайся.
\vs Job 21:4 Разве к человеку речь моя? как же мне и не малодушествовать?
\vs Job 21:5 Посмотрите на меня и ужаснитесь, и положите перст на уста.
\vs Job 21:6 Лишь только я вспомню,~--- содрогаюсь, и трепет объемлет тело мое.
\vs Job 21:7 Почему беззаконные живут, достигают старости, да и силами крепки?
\vs Job 21:8 Дети их с ними перед лицем их, и внуки их перед глазами их.
\vs Job 21:9 Домы их безопасны от страха, и нет жезла Божия на них.
\vs Job 21:10 Вол их оплодотворяет и не извергает, корова их зачинает и не выкидывает.
\vs Job 21:11 Как стадо, выпускают они малюток своих, и дети их прыгают.
\vs Job 21:12 Восклицают под \bibemph{голос} тимпана и цитры и веселятся при \bibemph{звуках} свирели;
\vs Job 21:13 проводят дни свои в счастьи и мгновенно нисходят в преисподнюю.
\vs Job 21:14 А между тем они говорят Богу: отойди от нас, не хотим мы знать путей Твоих!
\vs Job 21:15 Что Вседержитель, чтобы нам служить Ему? и что пользы прибегать к Нему?
\vs Job 21:16 Видишь, счастье их не от их рук.~--- Совет нечестивых будь далек от меня!
\vs Job 21:17 Часто ли угасает светильник у беззаконных, и находит на них беда, и Он дает им в удел страдания во гневе Своем?
\vs Job 21:18 Они должны быть, как соломинка пред ветром и как плева, уносимая вихрем.
\vs Job 21:19 \bibemph{Скажешь}: Бог бережет для детей его несчастье его.~--- Пусть воздаст Он ему самому, чтобы он это знал.
\vs Job 21:20 Пусть его глаза увидят несчастье его, и пусть он сам пьет от гнева Вседержителева.
\vs Job 21:21 Ибо какая ему забота до дома своего после него, когда число месяцев его кончится?
\vs Job 21:22 Но Бога ли учить мудрости, когда Он судит и горних?
\vs Job 21:23 Один умирает в самой полноте сил своих, совершенно спокойный и мирный;
\vs Job 21:24 внутренности его полны жира, и кости его напоены мозгом.
\vs Job 21:25 А другой умирает с душею огорченною, не вкусив добра.
\vs Job 21:26 И они вместе будут лежать во прахе, и червь покроет их.
\vs Job 21:27 Знаю я ваши мысли и ухищрения, какие вы против меня сплетаете.
\vs Job 21:28 Вы скажете: где дом князя, и где шатер, в котором жили беззаконные?
\vs Job 21:29 Разве вы не спрашивали у путешественников и незнакомы с их наблюдениями,
\vs Job 21:30 что в день погибели пощажен бывает злодей, в день гнева отводится в сторону?
\vs Job 21:31 Кто представит ему пред лице путь его, и кто воздаст ему за то, что он делал?
\vs Job 21:32 Его провожают ко гробам и на его могиле ставят стражу.
\vs Job 21:33 Сладки для него глыбы долины, и за ним идет толпа людей, а идущим перед ним нет числа.
\vs Job 21:34 Как же вы хотите утешать меня пустым? В ваших ответах остается \bibemph{одна} ложь.
\vs Job 22:1 И отвечал Елифаз Феманитянин и сказал:
\vs Job 22:2 разве может человек доставлять пользу Богу? Разумный доставляет пользу себе самому.
\vs Job 22:3 Что за удовольствие Вседержителю, что ты праведен? И будет ли Ему выгода от того, что ты содержишь пути твои в непорочности?
\vs Job 22:4 Неужели Он, боясь тебя, вступит с тобою в состязание, пойдет судиться с тобою?
\vs Job 22:5 Верно, злоба твоя велика, и беззакониям твоим нет конца.
\vs Job 22:6 Верно, ты брал залоги от братьев твоих ни за что и с полунагих снимал одежду.
\vs Job 22:7 Утомленному жаждою не подавал воды напиться и голодному отказывал в хлебе;
\vs Job 22:8 а человеку сильному ты \bibemph{давал} землю, и сановитый селился на ней.
\vs Job 22:9 Вдов ты отсылал ни с чем и сирот оставлял с пустыми руками.
\vs Job 22:10 За то вокруг тебя петли, и возмутил тебя неожиданный ужас,
\vs Job 22:11 или тьма, в которой ты ничего не видишь, и множество вод покрыло тебя.
\vs Job 22:12 Не превыше ли небес Бог? посмотри вверх на звезды, как они высоко!
\vs Job 22:13 И ты говоришь: что знает Бог? может ли Он судить сквозь мрак?
\vs Job 22:14 Облака~--- завеса Его, так что Он не видит, а ходит \bibemph{только} по небесному кругу.
\vs Job 22:15 Неужели ты держишься пути древних, по которому шли люди беззаконные,
\vs Job 22:16 которые преждевременно были истреблены, когда вода разлилась под основание их?
\vs Job 22:17 Они говорили Богу: отойди от нас! и что сделает им Вседержитель?
\vs Job 22:18 А Он наполнял домы их добром. Но совет нечестивых будь далек от меня!
\vs Job 22:19 Видели праведники и радовались, и непорочный смеялся им:
\vs Job 22:20 враг наш истреблен, а оставшееся после них пожрал огонь.
\vs Job 22:21 Сблизься же с Ним~--- и будешь спокоен; чрез это придет к тебе добро.
\vs Job 22:22 Прими из уст Его закон и положи слова Его в сердце твое.
\vs Job 22:23 Если ты обратишься к Вседержителю, то вновь устроишься, удалишь беззаконие от шатра твоего
\vs Job 22:24 и будешь вменять в прах блестящий металл, и в камни потоков~--- \bibemph{золото} Офирское.
\vs Job 22:25 И будет Вседержитель твоим золотом и блестящим серебром у тебя,
\vs Job 22:26 ибо тогда будешь радоваться о Вседержителе и поднимешь к Богу лице твое.
\vs Job 22:27 Помолишься Ему, и Он услышит тебя, и ты исполнишь обеты твои.
\vs Job 22:28 Положишь намерение, и оно состоится у тебя, и над путями твоими будет сиять свет.
\vs Job 22:29 Когда кто уничижен будет, ты скажешь: возвышение! и Он спасет поникшего лицем,
\vs Job 22:30 избавит и небезвинного, и он спасется чистотою рук твоих.
\vs Job 23:1 И отвечал Иов и сказал:
\vs Job 23:2 еще и ныне горька речь моя: страдания мои тяжелее стонов моих.
\vs Job 23:3 О, если бы я знал, где найти Его, и мог подойти к престолу Его!
\vs Job 23:4 Я изложил бы пред Ним дело мое и уста мои наполнил бы оправданиями;
\vs Job 23:5 узнал бы слова, какими Он ответит мне, и понял бы, что Он скажет мне.
\vs Job 23:6 Неужели Он в полном могуществе стал бы состязаться со мною? О, нет! Пусть Он только обратил бы внимание на меня.
\vs Job 23:7 Тогда праведник мог бы состязаться с Ним,~--- и я навсегда получил бы свободу от Судии моего.
\vs Job 23:8 Но вот, я иду вперед~--- и нет Его, назад~--- и не нахожу Его;
\vs Job 23:9 делает ли Он что на левой стороне, я не вижу; скрывается ли на правой, не усматриваю.
\vs Job 23:10 Но Он знает путь мой; пусть испытает меня,~--- выйду, как золото.
\vs Job 23:11 Нога моя твердо держится стези Его; пути Его я хранил и не уклонялся.
\vs Job 23:12 От заповеди уст Его не отступал; глаголы уст Его хранил больше, нежели мои правила.
\vs Job 23:13 Но Он тверд; и кто отклонит Его? Он делает, чего хочет душа Его.
\vs Job 23:14 Так, Он выполнит положенное мне, и подобного этому много у Него.
\vs Job 23:15 Поэтому я трепещу пред лицем Его; размышляю~--- и страшусь Его.
\vs Job 23:16 Бог расслабил сердце мое, и Вседержитель устрашил меня.
\vs Job 23:17 Зачем я не уничтожен прежде этой тьмы, и Он не сокрыл мрака от лица моего!
\vs Job 24:1 Почему не сокрыты от Вседержителя времена, и знающие Его не видят дней Его?
\vs Job 24:2 Межи передвигают, угоняют стада и пасут \bibemph{у себя}.
\vs Job 24:3 У сирот уводят осла, у вдовы берут в залог вола;
\vs Job 24:4 бедных сталкивают с дороги, все уничиженные земли принуждены скрываться.
\vs Job 24:5 Вот они, \bibemph{как} дикие ослы в пустыне, выходят на дело свое, вставая рано на добычу; степь \bibemph{дает} хлеб для них и для детей их;
\vs Job 24:6 жнут они на поле не своем и собирают виноград у нечестивца;
\vs Job 24:7 нагие ночуют без покрова и без одеяния на стуже;
\vs Job 24:8 мокнут от горных дождей и, не имея убежища, жмутся к скале;
\vs Job 24:9 отторгают от сосцов сироту и с нищего берут залог;
\vs Job 24:10 заставляют ходить нагими, без одеяния, и голодных кормят колосьями;
\vs Job 24:11 между стенами выжимают масло оливковое, топчут в точилах и жаждут.
\vs Job 24:12 В городе люди стонут, и душа убиваемых вопит, и Бог не воспрещает того.
\vs Job 24:13 Есть из них враги света, не знают путей его и не ходят по стезям его.
\vs Job 24:14 С рассветом встает убийца, умерщвляет бедного и нищего, а ночью бывает вором.
\vs Job 24:15 И око прелюбодея ждет сумерков, говоря: ничей глаз не увидит меня,~--- и закрывает лице.
\vs Job 24:16 В темноте подкапываются под домы, которые днем они заметили для себя; не знают света.
\vs Job 24:17 Ибо для них утро~--- смертная тень, так как они знакомы с ужасами смертной тени.
\vs Job 24:18 Легок такой на поверхности воды, проклята часть его на земле, и не смотрит он на дорогу садов виноградных.
\vs Job 24:19 Засуха и жара поглощают снежную воду: так преисподняя~--- грешников.
\vs Job 24:20 Пусть забудет его утроба \bibemph{матери}; пусть лакомится им червь; пусть не остается о нем память; как дерево, пусть сломится беззаконник,
\vs Job 24:21 который угнетает бездетную, не рождавшую, и вдове не делает добра.
\vs Job 24:22 Он и сильных увлекает своею силою; он встает, и никто не уверен за жизнь свою.
\vs Job 24:23 А Он дает ему \bibemph{все} для безопасности, и он \bibemph{на то} опирается, и очи Его видят пути их.
\vs Job 24:24 Поднялись высоко,~--- и вот, нет их; падают и умирают, как и все, и, как верхушки колосьев, срезываются.
\vs Job 24:25 Если это не так,~--- кто обличит меня во лжи и в ничто обратит речь мою?
\vs Job 25:1 И отвечал Вилдад Савхеянин и сказал:
\vs Job 25:2 держава и страх у Него; Он творит мир на высотах Своих!
\vs Job 25:3 Есть ли счет воинствам Его? и над кем не восходит свет Его?
\vs Job 25:4 И как человеку быть правым пред Богом, и как быть чистым рожденному женщиною?
\vs Job 25:5 Вот даже луна, и та несветла, и звезды нечисты пред очами Его.
\vs Job 25:6 Тем менее человек, \bibemph{который} есть червь, и сын человеческий, \bibemph{который} есть моль.
\vs Job 26:1 И отвечал Иов и сказал:
\vs Job 26:2 как ты помог бессильному, поддержал мышцу немощного!
\vs Job 26:3 Какой совет подал ты немудрому и как во всей полноте объяснил дело!
\vs Job 26:4 Кому ты говорил эти слова, и чей дух исходил из тебя?
\vs Job 26:5 Рефаимы трепещут под водами, и живущие в них.
\vs Job 26:6 Преисподняя обнажена пред Ним, и нет покрывала Аваддону.
\vs Job 26:7 Он распростер север над пустотою, повесил землю ни на чем.
\vs Job 26:8 Он заключает воды в облаках Своих, и облако не расседается под ними.
\vs Job 26:9 Он поставил престол Свой, распростер над ним облако Свое.
\vs Job 26:10 Черту провел над поверхностью воды, до границ света со тьмою.
\vs Job 26:11 Столпы небес дрожат и ужасаются от грозы Его.
\vs Job 26:12 Силою Своею волнует море и разумом Своим сражает его дерзость.
\vs Job 26:13 От духа Его~--- великолепие неба; рука Его образовала быстрого скорпиона.
\vs Job 26:14 Вот, это части путей Его; и как мало мы слышали о Нем! А гром могущества Его кто может уразуметь?
\vs Job 27:1 И продолжал Иов возвышенную речь свою и сказал:
\vs Job 27:2 жив Бог, лишивший \bibemph{меня} суда, и Вседержитель, огорчивший душу мою,
\vs Job 27:3 что, доколе еще дыхание мое во мне и дух Божий в ноздрях моих,
\vs Job 27:4 не скажут уста мои неправды, и язык мой не произнесет лжи!
\vs Job 27:5 Далек я от того, чтобы признать вас справедливыми; доколе не умру, не уступлю непорочности моей.
\vs Job 27:6 Крепко держал я правду мою и не опущу ее; не укорит меня сердце мое во все дни мои.
\vs Job 27:7 Враг мой будет, как нечестивец, и восстающий на меня, как беззаконник.
\vs Job 27:8 Ибо какая надежда лицемеру, когда возьмет, когда исторгнет Бог душу его?
\vs Job 27:9 Услышит ли Бог вопль его, когда придет на него беда?
\vs Job 27:10 Будет ли он утешаться Вседержителем и призывать Бога во всякое время?
\vs Job 27:11 Возвещу вам, чт\acc{о} в руке Божией; чт\acc{о} у Вседержителя, не скрою.
\vs Job 27:12 Вот, все вы и сами видели; и для чего вы столько пустословите?
\vs Job 27:13 Вот доля человеку беззаконному от Бога, и наследие, какое получают от Вседержителя притеснители.
\vs Job 27:14 Если умножаются сыновья его, то под меч; и потомки его не насытятся хлебом.
\vs Job 27:15 Оставшихся по нем смерть низведет во гроб, и вдовы их не будут плакать.
\vs Job 27:16 Если он наберет кучи серебра, как праха, и наготовит одежд, как брение,
\vs Job 27:17 то он наготовит, а одеваться будет праведник, и серебро получит себе на долю беспорочный.
\vs Job 27:18 Он строит, как моль, дом свой и, как сторож, делает себе шалаш;
\vs Job 27:19 ложится спать богачом и таким не встанет; открывает глаза свои, и он уже не тот.
\vs Job 27:20 Как в\acc{о}ды, постигнут его ужасы; в ночи похитит его буря.
\vs Job 27:21 Поднимет его восточный ветер и понесет, и он быстро побежит от него.
\vs Job 27:22 Устремится на него и не пощадит, как бы он ни силился убежать от руки его.
\vs Job 27:23 Всплеснут о нем руками и посвищут над ним с места его!
\vs Job 28:1 Так! у серебра есть источная жила, и у золота место, \bibemph{где его} плавят.
\vs Job 28:2 Железо получается из земли; из камня выплавляется медь.
\vs Job 28:3 \bibemph{Человек} полагает предел тьме и тщательно разыскивает камень во мраке и тени смертной.
\vs Job 28:4 Вырывают рудокопный колодезь в местах, забытых ногою, спускаются вглубь, висят \bibemph{и} зыблются вдали от людей.
\vs Job 28:5 Земля, на которой вырастает хлеб, внутри изрыта как бы огнем.
\vs Job 28:6 Камни ее~--- место сапфира, и в ней песчинки золота.
\vs Job 28:7 Стези \bibemph{туда} не знает хищная птица, и не видал ее глаз коршуна;
\vs Job 28:8 не попирали ее скимны, и не ходил по ней шакал.
\vs Job 28:9 На гранит налагает он руку свою, с корнем опрокидывает горы;
\vs Job 28:10 в скалах просекает каналы, и все драгоценное видит глаз его;
\vs Job 28:11 останавливает течение потоков и сокровенное выносит на свет.
\vs Job 28:12 Но где премудрость обретается? и где место разума?
\vs Job 28:13 Не знает человек цены ее, и она не обретается на земле живых.
\vs Job 28:14 Бездна говорит: не во мне она; и море говорит: не у меня.
\vs Job 28:15 Не дается она за золото и не приобретается она за вес серебра;
\vs Job 28:16 не оценивается она золотом Офирским, ни драгоценным ониксом, ни сапфиром;
\vs Job 28:17 не равняется с нею золото и кристалл, и не выменяешь ее на сосуды из чистого золота.
\vs Job 28:18 А о кораллах и жемчуге и упоминать нечего, и приобретение премудрости выше рубинов.
\vs Job 28:19 Не равняется с нею топаз Ефиопский; чистым золотом не оценивается она.
\vs Job 28:20 Откуда же исходит премудрость? и где место разума?
\vs Job 28:21 Сокрыта она от очей всего живущего и от птиц небесных утаена.
\vs Job 28:22 Аваддон и смерть говорят: ушами нашими слышали мы слух о ней.
\vs Job 28:23 Бог знает путь ее, и Он ведает место ее.
\vs Job 28:24 Ибо Он прозирает до концов земли и видит под всем небом.
\vs Job 28:25 Когда Он ветру полагал вес и располагал воду по мере,
\vs Job 28:26 когда назначал устав дождю и путь для молнии громоносной,
\vs Job 28:27 тогда Он видел ее и явил ее, приготовил ее и еще испытал ее
\vs Job 28:28 и сказал человеку: вот, страх Господень есть истинная премудрость, и удаление от зла~--- разум.
\vs Job 29:1 И продолжал Иов возвышенную речь свою и сказал:
\vs Job 29:2 о, если бы я был, как в прежние месяцы, как в те дни, когда Бог хранил меня,
\vs Job 29:3 когда светильник Его светил над головою моею, и я при свете Его ходил среди тьмы;
\vs Job 29:4 как был я во дни молодости моей, когда милость Божия \bibemph{была} над шатром моим,
\vs Job 29:5 когда еще Вседержитель \bibemph{был} со мною, и дети мои вокруг меня,
\vs Job 29:6 когда пути мои обливались молоком, и скала источала для меня ручьи елея!
\vs Job 29:7 когда я выходил к воротам города и на площади ставил седалище свое,~---
\vs Job 29:8 юноши, увидев меня, прятались, а старцы вставали и стояли;
\vs Job 29:9 князья удерживались от речи и персты полагали на уста свои;
\vs Job 29:10 голос знатных умолкал, и язык их прилипал к гортани их.
\vs Job 29:11 Ухо, слышавшее меня, ублажало меня; око видевшее восхваляло меня,
\vs Job 29:12 потому что я спасал страдальца вопиющего и сироту беспомощного.
\vs Job 29:13 Благословение погибавшего приходило на меня, и сердцу вдовы доставлял я радость.
\vs Job 29:14 Я облекался в правду, и суд мой одевал меня, как мантия и увясло.
\vs Job 29:15 Я был глазами слепому и ногами хромому;
\vs Job 29:16 отцом был я для нищих и тяжбу, которой я не знал, разбирал внимательно.
\vs Job 29:17 Сокрушал я беззаконному челюсти и из зубов его исторгал похищенное.
\vs Job 29:18 И говорил я: в гнезде моем скончаюсь, и дни \bibemph{мои} будут многи, как песок;
\vs Job 29:19 корень мой открыт для воды, и роса ночует на ветвях моих;
\vs Job 29:20 слава моя не стареет, лук мой крепок в руке моей.
\vs Job 29:21 Внимали мне и ожидали, и безмолвствовали при совете моем.
\vs Job 29:22 После слов моих уже не рассуждали; речь моя капала на них.
\vs Job 29:23 Ждали меня, как дождя, и, \bibemph{как} дождю позднему, открывали уста свои.
\vs Job 29:24 Бывало, улыбнусь им~--- они не верят; и света лица моего они не помрачали.
\vs Job 29:25 Я назначал пути им и сидел во главе и жил как царь в кругу воинов, как утешитель плачущих.
\vs Job 30:1 А ныне смеются надо мною младшие меня летами, те, которых отцов я не согласился бы поместить с псами стад моих.
\vs Job 30:2 И сила рук их к чему мне? Над ними уже прошло время.
\vs Job 30:3 Бедностью и голодом истощенные, они убегают в степь безводную, мрачную и опустевшую;
\vs Job 30:4 щиплют зелень подле кустов, и ягоды можжевельника~--- хлеб их.
\vs Job 30:5 Из общества изгоняют их, кричат на них, как на воров,
\vs Job 30:6 чтобы жили они в рытвинах потоков, в ущельях земли и утесов.
\vs Job 30:7 Ревут между кустами, жмутся под терном.
\vs Job 30:8 Люди отверженные, люди без имени, отребье земли!
\vs Job 30:9 Их-то сделался я ныне песнью и пищею разговора их.
\vs Job 30:10 Они гнушаются мною, удаляются от меня и не удерживаются плевать пред лицем моим.
\vs Job 30:11 Так как Он развязал повод мой и поразил меня, то они сбросили с себя узду пред лицем моим.
\vs Job 30:12 С правого боку встает это исчадие, сбивает меня с ног, направляет гибельные свои пути ко мне.
\vs Job 30:13 А мою стезю испортили: всё успели сделать к моей погибели, не имея помощника.
\vs Job 30:14 Они пришли ко мне, как сквозь широкий пролом; с шумом бросились на меня.
\vs Job 30:15 Ужасы устремились на меня; как ветер, развеялось величие мое, и счастье мое унеслось, как облако.
\vs Job 30:16 И ныне изливается душа моя во мне: дни скорби объяли меня.
\vs Job 30:17 Ночью ноют во мне кости мои, и жилы мои не имеют покоя.
\vs Job 30:18 С великим трудом снимается с меня одежда моя; края хитона моего жмут меня.
\vs Job 30:19 Он бросил меня в грязь, и я стал, как прах и пепел.
\vs Job 30:20 Я взываю к Тебе, и Ты не внимаешь мне,~--- стою, а Ты \bibemph{только} смотришь на меня.
\vs Job 30:21 Ты сделался жестоким ко мне, крепкою рукою враждуешь против меня.
\vs Job 30:22 Ты поднял меня и заставил меня носиться по ветру и сокрушаешь меня.
\vs Job 30:23 Так, я знаю, что Ты приведешь меня к смерти и в дом собрания всех живущих.
\vs Job 30:24 Верно, Он не прострет руки Своей на дом костей: будут ли они кричать при своем разрушении?
\vs Job 30:25 Не плакал ли я о том, кто был в горе? не скорбела ли душа моя о бедных?
\vs Job 30:26 Когда я чаял добра, пришло зло; когда ожидал света, пришла тьма.
\vs Job 30:27 Мои внутренности кипят и не перестают; встретили меня дни печали.
\vs Job 30:28 Я хожу почернелый, но не от солнца; встаю в собрании и кричу.
\vs Job 30:29 Я стал братом шакалам и другом страусам.
\vs Job 30:30 Моя кожа почернела на мне, и кости мои обгорели от жара.
\vs Job 30:31 И цитра моя сделалась унылою, и свирель моя~--- голосом плачевным.
\vs Job 31:1 Завет положил я с глазами моими, чтобы не помышлять мне о девице.
\vs Job 31:2 Какая же участь \bibemph{мне} от Бога свыше? И какое наследие от Вседержителя с небес?
\vs Job 31:3 Не для нечестивого ли гибель, и не для делающего ли зло напасть?
\vs Job 31:4 Не видел ли Он путей моих, и не считал ли всех моих шагов?
\vs Job 31:5 Если я ходил в суете, и если нога моя спешила на лукавство,~---
\vs Job 31:6 пусть взвесят меня на весах правды, и Бог узнает мою непорочность.
\vs Job 31:7 Если стопы мои уклонялись от пути и сердце мое следовало за глазами моими, и если что-либо \bibemph{нечистое} пристало к рукам моим,
\vs Job 31:8 то пусть я сею, а другой ест, и пусть отрасли мои искоренены будут.
\vs Job 31:9 Если сердце мое прельщалось женщиною и я строил ковы у дверей моего ближнего,~---
\vs Job 31:10 пусть моя жена мелет на другого, и пусть другие издеваются над нею,
\vs Job 31:11 потому что это~--- преступление, это~--- беззаконие, подлежащее суду;
\vs Job 31:12 это~--- огонь, поядающий до истребления, который искоренил бы все добро мое.
\vs Job 31:13 Если я пренебрегал правами слуги и служанки моей, когда они имели спор со мною,
\vs Job 31:14 то что стал бы я делать, когда бы Бог восстал? И когда бы Он взглянул на меня, что мог бы я отвечать Ему?
\vs Job 31:15 Не Он ли, Который создал меня во чреве, создал и его и равно образовал нас в утробе?
\vs Job 31:16 Отказывал ли я нуждающимся в их просьбе и томил ли глаза вдовы?
\vs Job 31:17 Один ли я съедал кусок мой, и не ел ли от него и сирота?
\vs Job 31:18 Ибо с детства он рос со мною, как с отцом, и от чрева матери моей я руководил \bibemph{вдову}.
\vs Job 31:19 Если я видел кого погибавшим без одежды и бедного без покрова,~---
\vs Job 31:20 не благословляли ли меня чресла его, и не был ли он согрет шерстью овец моих?
\vs Job 31:21 Если я поднимал руку мою на сироту, когда видел помощь себе у ворот,
\vs Job 31:22 то пусть плечо мое отпадет от спины, и рука моя пусть отломится от локтя,
\vs Job 31:23 ибо страшно для меня наказание от Бога: пред величием Его не устоял бы я.
\vs Job 31:24 Полагал ли я в золоте опору мою и говорил ли сокровищу: ты~--- надежда моя?
\vs Job 31:25 Радовался ли я, что богатство мое было велико, и что рука моя приобрела много?
\vs Job 31:26 Смотря на солнце, как оно сияет, и на луну, как она величественно шествует,
\vs Job 31:27 прельстился ли я в тайне сердца моего, и целовали ли уста мои руку мою?
\vs Job 31:28 Это также было бы преступление, подлежащее суду, потому что я отрекся бы \bibemph{тогда} от Бога Всевышнего.
\vs Job 31:29 Радовался ли я погибели врага моего и торжествовал ли, когда несчастье постигало его?
\vs Job 31:30 Не позволял я устам моим грешить проклятием души его.
\vs Job 31:31 Не говорили ли люди шатра моего: о, если бы мы от мяс его не насытились?
\vs Job 31:32 Странник не ночевал на улице; двери мои я отворял прохожему.
\vs Job 31:33 Если бы я скрывал проступки мои, как человек, утаивая в груди моей пороки мои,
\vs Job 31:34 то я боялся бы большого общества, и презрение одноплеменников страшило бы меня, и я молчал бы и не выходил бы за двери.
\vs Job 31:35 О, если бы кто выслушал меня! Вот мое желание, чтобы Вседержитель отвечал мне, и чтобы защитник мой составил запись.
\vs Job 31:36 Я носил бы ее на плечах моих и возлагал бы ее, как венец;
\vs Job 31:37 объявил бы ему число шагов моих, сблизился бы с ним, как с князем.
\vs Job 31:38 Если вопияла на меня земля моя и жаловались на меня борозды ее;
\vs Job 31:39 если я ел плоды ее без платы и отягощал жизнь земледельцев,
\vs Job 31:40 то пусть вместо пшеницы вырастает волчец и вместо ячменя куколь. Слова Иова кончились.
\vs Job 32:1 Когда те три мужа перестали отвечать Иову, потому что он был прав в глазах своих,
\vs Job 32:2 тогда воспылал гнев Елиуя, сына Варахиилова, Вузитянина из племени Рамова: воспылал гнев его на Иова за то, что он оправдывал себя больше, нежели Бога,
\vs Job 32:3 а на трех друзей его воспылал гнев его за то, что они не нашли, что отвечать, а между тем обвиняли Иова.
\vs Job 32:4 Елиуй ждал, пока Иов говорил, потому что они летами были старше его.
\vs Job 32:5 Когда же Елиуй увидел, что нет ответа в устах тех трех мужей, тогда воспылал гнев его.
\vs Job 32:6 И отвечал Елиуй, сын Варахиилов, Вузитянин, и сказал: я молод летами, а вы~--- старцы; поэтому я робел и боялся объявлять вам мое мнение.
\vs Job 32:7 Я говорил сам себе: пусть говорят дни, и многолетие поучает мудрости.
\vs Job 32:8 Но дух в человеке и дыхание Вседержителя дает ему разумение.
\vs Job 32:9 Не многолетние \bibemph{только} мудры, и не старики разумеют правду.
\vs Job 32:10 Поэтому я говорю: выслушайте меня, объявлю вам мое мнение и я.
\vs Job 32:11 Вот, я ожидал слов ваших,~--- вслушивался в суждения ваши, доколе вы придумывали, чт\acc{о} сказать.
\vs Job 32:12 Я пристально смотрел на вас, и вот никто из вас не обличает Иова и не отвечает на слова его.
\vs Job 32:13 Не скажите: мы нашли мудрость: Бог опровергнет его, а не человек.
\vs Job 32:14 Если бы он обращал слова свои ко мне, то я не вашими речами отвечал бы ему.
\vs Job 32:15 Испугались, не отвечают более; перестали говорить.
\vs Job 32:16 И как я ждал, а они не говорят, остановились и не отвечают более,
\vs Job 32:17 то и я отвечу с моей стороны, объявлю мое мнение и я,
\vs Job 32:18 ибо я полон речами, и дух во мне теснит меня.
\vs Job 32:19 Вот, утроба моя, как вино неоткрытое: она готова прорваться, подобно новым мехам.
\vs Job 32:20 Поговорю, и будет легче мне; открою уста мои и отвечу.
\vs Job 32:21 На лице человека смотреть не буду и никакому человеку льстить не стану,
\vs Job 32:22 потому что я не умею льстить: сейчас убей меня, Творец мой.
\vs Job 33:1 Итак слушай, Иов, речи мои и внимай всем словам моим.
\vs Job 33:2 Вот, я открываю уста мои, язык мой говорит в гортани моей.
\vs Job 33:3 Слова мои от искренности моего сердца, и уста мои произнесут знание чистое.
\vs Job 33:4 Дух Божий создал меня, и дыхание Вседержителя дало мне жизнь.
\vs Job 33:5 Если можешь, отвечай мне и стань передо мною.
\vs Job 33:6 Вот я, по желанию твоему, вместо Бога. Я образован также из брения;
\vs Job 33:7 поэтому страх передо мною не может смутить тебя, и рука моя не будет тяжела для тебя.
\vs Job 33:8 Ты говорил в уши мои, и я слышал звук слов:
\vs Job 33:9 чист я, без порока, невинен я, и нет во мне неправды;
\vs Job 33:10 а Он нашел обвинение против меня и считает меня Своим противником;
\vs Job 33:11 поставил ноги мои в колоду, наблюдает за всеми путями моими.
\vs Job 33:12 Вот в этом ты неправ, отвечаю тебе, потому что Бог выше человека.
\vs Job 33:13 Для чего тебе состязаться с Ним? Он не дает отчета ни в каких делах Своих.
\vs Job 33:14 Бог говорит однажды и, если того не заметят, в другой раз:
\vs Job 33:15 во сне, в ночном видении, когда сон находит на людей, во время дремоты на ложе.
\vs Job 33:16 Тогда Он открывает у человека ухо и запечатлевает Свое наставление,
\vs Job 33:17 чтобы отвести человека от какого-либо предприятия и удалить от него гордость,
\vs Job 33:18 чтобы отвести душу его от пропасти и жизнь его от поражения мечом.
\vs Job 33:19 Или он вразумляется болезнью на ложе своем и жестокою болью во всех костях своих,~---
\vs Job 33:20 и жизнь его отвращается от хлеба и душа его от любимой пищи.
\vs Job 33:21 Плоть на нем пропадает, так что ее не видно, и показываются кости его, которых не было видно.
\vs Job 33:22 И душа его приближается к могиле и жизнь его~--- к смерти.
\vs Job 33:23 Если есть у него Ангел-наставник, один из тысячи, чтобы показать человеку прямой \bibemph{путь} его,~---
\vs Job 33:24 \bibemph{Бог} умилосердится над ним и скажет: освободи его от могилы; Я нашел умилостивление.
\vs Job 33:25 Тогда тело его сделается свеж\acc{е}е, нежели в молодости; он возвратится к дням юности своей.
\vs Job 33:26 Будет молиться Богу, и Он~--- милостив к нему; с радостью взирает на лице его и возвращает человеку праведность его.
\vs Job 33:27 Он будет смотреть на людей и говорить: грешил я и превращал правду, и не воздано мне;
\vs Job 33:28 Он освободил душу мою от могилы, и жизнь моя видит свет.
\vs Job 33:29 Вот, все это делает Бог два-три раза с человеком,
\vs Job 33:30 чтобы отвести душу его от могилы и просветить его светом живых.
\vs Job 33:31 Внимай, Иов, слушай меня, молчи, и я буду говорить.
\vs Job 33:32 Если имеешь, что сказать, отвечай; говори, потому что я желал бы твоего оправдания;
\vs Job 33:33 если же нет, то слушай меня: молчи, и я научу тебя мудрости.
\vs Job 34:1 И продолжал Елиуй и сказал:
\vs Job 34:2 выслушайте, мудрые, речь мою, и приклоните ко мне ухо, рассудительные!
\vs Job 34:3 Ибо ухо разбирает слова, как гортань различает вкус в пище.
\vs Job 34:4 Установим между собою рассуждение и распознаем, что хорошо.
\vs Job 34:5 Вот, Иов сказал: я прав, но Бог лишил меня суда.
\vs Job 34:6 Должен ли я лгать на правду мою? Моя рана неисцелима без вины.
\vs Job 34:7 Есть ли такой человек, как Иов, который пьет глумление, как воду,
\vs Job 34:8 вступает в сообщество с делающими беззаконие и ходит с людьми нечестивыми?
\vs Job 34:9 Потому что он сказал: нет пользы для человека в благоугождении Богу.
\vs Job 34:10 Итак послушайте меня, мужи мудрые! Не может быть у Бога неправда или у Вседержителя неправосудие,
\vs Job 34:11 ибо Он по делам человека поступает с ним и по путям мужа воздает ему.
\vs Job 34:12 Истинно, Бог не делает неправды и Вседержитель не извращает суда.
\vs Job 34:13 Кто кроме Его промышляет о земле? И кто управляет всею вселенною?
\vs Job 34:14 Если бы Он обратил сердце Свое к Себе и взял к Себе дух ее и дыхание ее,~---
\vs Job 34:15 вдруг погибла бы всякая плоть, и человек возвратился бы в прах.
\vs Job 34:16 Итак, если ты имеешь разум, то слушай это и внимай словам моим.
\vs Job 34:17 Ненавидящий правду может ли владычествовать? И можешь ли ты обвинить Всеправедного?
\vs Job 34:18 Можно ли сказать царю: ты~--- нечестивец, и князьям: вы~--- беззаконники?
\vs Job 34:19 Но Он не смотрит и на лица князей и не предпочитает богатого бедному, потому что все они дело рук Его.
\vs Job 34:20 Внезапно они умирают; среди ночи народ возмутится, и они исчезают; и сильных изгоняют не силою.
\vs Job 34:21 Ибо очи Его над путями человека, и Он видит все шаги его.
\vs Job 34:22 Нет тьмы, ни тени смертной, где могли бы укрыться делающие беззаконие.
\vs Job 34:23 Потому Он уже не требует от человека, чтобы шел на суд с Богом.
\vs Job 34:24 Он сокрушает сильных без исследования и поставляет других на их места;
\vs Job 34:25 потому что Он делает известными дела их и низлагает их ночью, и они истребляются.
\vs Job 34:26 Он поражает их, как беззаконных людей, пред глазами других,
\vs Job 34:27 за то, что они отвратились от Него и не уразумели всех путей Его,
\vs Job 34:28 так что дошел до Него вопль бедных, и Он услышал стенание угнетенных.
\vs Job 34:29 Дарует ли Он тишину, кто может возмутить? скрывает ли Он лице Свое, кто может увидеть Его? Будет ли это для народа, или для одного человека,
\vs Job 34:30 чтобы не царствовал лицемер к соблазну народа.
\vs Job 34:31 К Богу должно говорить: я потерпел, больше не буду грешить.
\vs Job 34:32 А чего я не знаю, Ты научи меня; и если я сделал беззаконие, больше не буду.
\vs Job 34:33 По твоему ли \bibemph{рассуждению} Он должен воздавать? И как ты отвергаешь, то тебе следует избирать, а не мне; говори, что знаешь.
\vs Job 34:34 Люди разумные скажут мне, и муж мудрый, слушающий меня:
\vs Job 34:35 Иов не умно говорит, и слова его не со смыслом.
\vs Job 34:36 Я желал бы, чтобы Иов вполне был испытан, по ответам его, свойственным людям нечестивым.
\vs Job 34:37 Иначе он ко греху своему прибавит отступление, будет рукоплескать между нами и еще больше наговорит против Бога.
\vs Job 35:1 И продолжал Елиуй и сказал:
\vs Job 35:2 считаешь ли ты справедливым, что сказал: я правее Бога?
\vs Job 35:3 Ты сказал: что пользы мне? и какую прибыль я имел бы пред тем, как если бы я и грешил?
\vs Job 35:4 Я отвечу тебе и твоим друзьям с тобою:
\vs Job 35:5 взгляни на небо и смотри; воззри на облака, они выше тебя.
\vs Job 35:6 Если ты грешишь, что делаешь ты Ему? и если преступления твои умножаются, что причиняешь ты Ему?
\vs Job 35:7 Если ты праведен, что даешь Ему? или что получает Он от руки твоей?
\vs Job 35:8 Нечестие твое относится к человеку, как ты, и праведность твоя к сыну человеческому.
\vs Job 35:9 От множества притеснителей стонут притесняемые, и от руки сильных вопиют.
\vs Job 35:10 Но никто не говорит: где Бог, Творец мой, Который дает песни в ночи,
\vs Job 35:11 Который научает нас более, нежели скотов земных, и вразумляет нас более, нежели птиц небесных?
\vs Job 35:12 Там они вопиют, и Он не отвечает им, по причине гордости злых людей.
\vs Job 35:13 Но неправда, что Бог не слышит и Вседержитель не взирает на это.
\vs Job 35:14 Хотя ты сказал, что ты не видишь Его, но суд пред Ним, и~--- жди его.
\vs Job 35:15 Но ныне, потому что гнев Его не посетил его и он не познал его во всей строгости,
\vs Job 35:16 Иов и открыл легкомысленно уста свои и безрассудно расточает слова.
\vs Job 36:1 И продолжал Елиуй и сказал:
\vs Job 36:2 подожди меня немного, и я покажу тебе, что я имею еще что сказать за Бога.
\vs Job 36:3 Начну мои рассуждения издалека и воздам Создателю моему справедливость,
\vs Job 36:4 потому что слова мои точно не ложь: пред тобою~--- совершенный в познаниях.
\vs Job 36:5 Вот, Бог могуществен и не презирает сильного крепостью сердца;
\vs Job 36:6 Он не поддерживает нечестивых и воздает должное угнетенным;
\vs Job 36:7 Он не отвращает очей Своих от праведников, но с царями навсегда посаждает их на престоле, и они возвышаются.
\vs Job 36:8 Если же они окованы цепями и содержатся в узах бедствия,
\vs Job 36:9 то Он указывает им на дела их и на беззакония их, потому что умножились,
\vs Job 36:10 и открывает их ухо для вразумления и говорит им, чтоб они отстали от нечестия.
\vs Job 36:11 Если послушают и будут служить Ему, то проведут дни свои в благополучии и лета свои в радости;
\vs Job 36:12 если же не послушают, то погибнут от стрелы и умрут в неразумии.
\vs Job 36:13 Но лицемеры питают в сердце гнев и не взывают к Нему, когда Он заключает их в узы;
\vs Job 36:14 поэтому душа их умирает в молодости и жизнь их с блудниками.
\vs Job 36:15 Он спасает бедного от беды его и в угнетении открывает ухо его.
\vs Job 36:16 И тебя вывел бы Он из тесноты на простор, где нет стеснения, и поставляемое на стол твой было бы наполнено туком;
\vs Job 36:17 но ты преисполнен суждениями нечестивых: суждение и осуждение~--- близки.
\vs Job 36:18 Да не поразит тебя гнев \bibemph{Божий} наказанием! Большой выкуп не спасет тебя.
\vs Job 36:19 Даст ли Он какую цену твоему богатству? Нет,~--- ни золоту и никакому сокровищу.
\vs Job 36:20 Не желай той ночи, когда народы истребляются на своем месте.
\vs Job 36:21 Берегись, не склоняйся к нечестию, которое ты предпочел страданию.
\vs Job 36:22 Бог высок могуществом Своим, и кто такой, как Он, наставник?
\vs Job 36:23 Кто укажет Ему путь Его; кто может сказать: Ты поступаешь несправедливо?
\vs Job 36:24 Помни о том, чтобы превозносить дела его, которые люди видят.
\vs Job 36:25 Все люди могут видеть их; человек может усматривать их издали.
\vs Job 36:26 Вот, Бог велик, и мы не можем познать Его; число лет Его неисследимо.
\vs Job 36:27 Он собирает капли воды; они во множестве изливаются дождем:
\vs Job 36:28 из облаков каплют и изливаются обильно на людей.
\vs Job 36:29 Кто может также постигнуть протяжение облаков, треск шатра Его?
\vs Job 36:30 Вот, Он распространяет над ним свет Свой и покрывает дно моря.
\vs Job 36:31 Оттуда Он судит народы, дает пищу в изобилии.
\vs Job 36:32 Он сокрывает в дланях Своих молнию и повелевает ей, кого разить.
\vs Job 36:33 Треск ее дает знать о ней; скот также чувствует происходящее.
\vs Job 37:1 И от сего трепещет сердце мое и подвиглось с места своего.
\vs Job 37:2 Слушайте, слушайте голос Его и гром, исходящий из уст Его.
\vs Job 37:3 Под всем небом раскат его, и блистание его~--- до краев земли.
\vs Job 37:4 За ним гремит глас; гремит Он гласом величества Своего и не останавливает его, когда голос Его услышан.
\vs Job 37:5 Дивно гремит Бог гласом Своим, делает дела великие, для нас непостижимые.
\vs Job 37:6 Ибо снегу Он говорит: будь на земле; равно мелкий дождь и большой дождь в Его власти.
\vs Job 37:7 Он полагает печать на руку каждого человека, чтобы все люди знали дело Его.
\vs Job 37:8 Тогда зверь уходит в убежище и остается в своих логовищах.
\vs Job 37:9 От юга приходит буря, от севера~--- стужа.
\vs Job 37:10 От дуновения Божия происходит лед, и поверхность воды сжимается.
\vs Job 37:11 Также влагою Он наполняет тучи, и облака сыплют свет Его,
\vs Job 37:12 и они направляются по намерениям Его, чтоб исполнить то, что Он повелит им на лице обитаемой земли.
\vs Job 37:13 Он повелевает им идти или для наказания, или в благоволение, или для помилования.
\vs Job 37:14 Внимай сему, Иов; стой и разумевай чудные дела Божии.
\vs Job 37:15 Знаешь ли, как Бог располагает ими и повелевает свету блистать из облака Своего?
\vs Job 37:16 Разумеешь ли равновесие облаков, чудное дело Совершеннейшего в знании?
\vs Job 37:17 Как нагревается твоя одежда, когда Он успокаивает землю от юга?
\vs Job 37:18 Ты ли с Ним распростер небеса, твердые, как литое зеркало?
\vs Job 37:19 Научи нас, что сказать Ему? Мы в этой тьме ничего не можем сообразить.
\vs Job 37:20 Будет ли возвещено Ему, что я говорю? Сказал ли кто, что сказанное доносится Ему?
\vs Job 37:21 Теперь не видно яркого света в облаках, но пронесется ветер и расчистит их.
\vs Job 37:22 Светлая погода приходит от севера, и окрест Бога страшное великолепие.
\vs Job 37:23 Вседержитель! мы не постигаем Его. Он велик силою, судом и полнотою правосудия. Он \bibemph{никого} не угнетает.
\vs Job 37:24 Посему да благоговеют пред Ним люди, и да трепещут пред Ним все мудрые сердцем!
\vs Job 38:1 [Когда Елиуй перестал говорить,] Господь отвечал Иову из бури и сказал:
\vs Job 38:2 кто сей, омрачающий Провидение словами без смысла?
\vs Job 38:3 Препояшь ныне чресла твои, как муж: Я буду спрашивать тебя, и ты объясняй Мне:
\vs Job 38:4 где был ты, когда Я полагал основания земли? Скажи, если знаешь.
\vs Job 38:5 Кто положил меру ей, если знаешь? или кто протягивал по ней вервь?
\vs Job 38:6 На чем утверждены основания ее, или кто положил краеугольный камень ее,
\vs Job 38:7 при общем ликовании утренних звезд, когда все сыны Божии восклицали от радости?
\vs Job 38:8 Кто затворил море воротами, когда оно исторглось, вышло как бы из чрева,
\vs Job 38:9 когда Я облака сделал одеждою его и мглу пеленами его,
\vs Job 38:10 и утвердил ему Мое определение, и поставил запоры и ворота,
\vs Job 38:11 и сказал: доселе дойдешь и не перейдешь, и здесь предел надменным волнам твоим?
\vs Job 38:12 Давал ли ты когда в жизни своей приказания утру и указывал ли заре место ее,
\vs Job 38:13 чтобы она охватила края земли и стряхнула с нее нечестивых,
\vs Job 38:14 чтобы \bibemph{земля} изменилась, как глина под печатью, и стала, как разноцветная одежда,
\vs Job 38:15 и чтобы отнялся у нечестивых свет их и дерзкая рука их сокрушилась?
\vs Job 38:16 Нисходил ли ты во глубину моря и входил ли в исследование бездны?
\vs Job 38:17 Отворялись ли для тебя врата смерти, и видел ли ты врата тени смертной?
\vs Job 38:18 Обозрел ли ты широту земли? Объясни, если знаешь все это.
\vs Job 38:19 Где путь к жилищу света, и где место тьмы?
\vs Job 38:20 Ты, конечно, доходил до границ ее и знаешь стези к дому ее.
\vs Job 38:21 Ты знаешь это, потому что ты был уже тогда рожден, и число дней твоих очень велико.
\vs Job 38:22 Входил ли ты в хранилища снега и видел ли сокровищницы града,
\vs Job 38:23 которые берегу Я на время смутное, на день битвы и войны?
\vs Job 38:24 По какому пути разливается свет и разносится восточный ветер по земле?
\vs Job 38:25 Кто проводит протоки для излияния воды и путь для громоносной молнии,
\vs Job 38:26 чтобы шел дождь на землю безлюдную, на пустыню, где нет человека,
\vs Job 38:27 чтобы насыщать пустыню и степь и возбуждать травные зародыши к возрастанию?
\vs Job 38:28 Есть ли у дождя отец? или кто рождает капли росы?
\vs Job 38:29 Из чьего чрева выходит лед, и иней небесный,~--- кто рождает его?
\vs Job 38:30 Воды, как камень, крепнут, и поверхность бездны замерзает.
\vs Job 38:31 Можешь ли ты связать узел Хима и разрешить узы Кесиль?
\vs Job 38:32 Можешь ли выводить созвездия в свое время и вести Ас с ее детьми?
\vs Job 38:33 Знаешь ли ты уставы неба, можешь ли установить господство его на земле?
\vs Job 38:34 Можешь ли возвысить голос твой к облакам, чтобы вода в обилии покрыла тебя?
\vs Job 38:35 Можешь ли посылать молнии, и пойдут ли они и скажут ли тебе: вот мы?
\vs Job 38:36 Кто вложил мудрость в сердце, или кто дал смысл разуму?
\vs Job 38:37 Кто может расчислить облака своею мудростью и удержать сосуды неба,
\vs Job 38:38 когда пыль обращается в грязь и глыбы слипаются?
\vs Job 38:39 Ты ли ловишь добычу львице и насыщаешь молодых львов,
\vs Job 38:40 когда они лежат в берлогах или покоятся под тенью в засаде?
\vs Job 38:41 Кто приготовляет в\acc{о}рону корм его, когда птенцы его кричат к Богу, бродя без пищи?
\vs Job 39:1 Знаешь ли ты время, когда рождаются дикие козы на скалах, и замечал ли роды ланей?
\vs Job 39:2 можешь ли расчислить месяцы беременности их? и знаешь ли время родов их?
\vs Job 39:3 Они изгибаются, рождая детей своих, выбрасывая свои ноши;
\vs Job 39:4 дети их приходят в силу, растут на поле, уходят и не возвращаются к ним.
\vs Job 39:5 Кто пустил дикого осла на свободу, и кто разрешил узы онагру,
\vs Job 39:6 которому степь Я назначил домом и солончаки~--- жилищем?
\vs Job 39:7 Он посмевается городскому многолюдству и не слышит криков погонщика,
\vs Job 39:8 по горам ищет себе пищи и гоняется за всякою зеленью.
\vs Job 39:9 Захочет ли единорог служить тебе и переночует ли у яслей твоих?
\vs Job 39:10 Можешь ли веревкою привязать единорога к борозде, и станет ли он боронить за тобою поле?
\vs Job 39:11 Понадеешься ли на него, потому что у него сила велика, и предоставишь ли ему работу твою?
\vs Job 39:12 Поверишь ли ему, что он семена твои возвратит и сложит на гумно твое?
\vs Job 39:13 Ты ли дал красивые крылья павлину и перья и пух страусу?
\vs Job 39:14 Он оставляет яйца свои на земле, и на песке согревает их,
\vs Job 39:15 и забывает, что нога может раздавить их и полевой зверь может растоптать их;
\vs Job 39:16 он жесток к детям своим, как бы не своим, и не опасается, что труд его будет напрасен;
\vs Job 39:17 потому что Бог не дал ему мудрости и не уделил ему смысла;
\vs Job 39:18 а когда поднимется на высоту, посмевается коню и всаднику его.
\vs Job 39:19 Ты ли дал коню силу и облек шею его гривою?
\vs Job 39:20 Можешь ли ты испугать его, как саранчу? Храпение ноздрей его~--- ужас;
\vs Job 39:21 роет ногою землю и восхищается силою; идет навстречу оружию;
\vs Job 39:22 он смеется над опасностью и не робеет и не отворачивается от меча;
\vs Job 39:23 колчан звучит над ним, сверкает копье и дротик;
\vs Job 39:24 в порыве и ярости он глотает землю и не может стоять при звуке трубы;
\vs Job 39:25 при трубном звуке он издает голос: гу! гу! и издалека чует битву, громкие голоса вождей и крик.
\vs Job 39:26 Твоею ли мудростью летает ястреб и направляет крылья свои на полдень?
\vs Job 39:27 По твоему ли слову возносится орел и устрояет на высоте гнездо свое?
\vs Job 39:28 Он живет на скале и ночует на зубце утесов и на местах неприступных;
\vs Job 39:29 оттуда высматривает себе пищу: глаза его смотрят далеко;
\vs Job 39:30 птенцы его пьют кровь, и где труп, там и он.
\vs Job 39:31 И продолжал Господь и сказал Иову:
\vs Job 39:32 будет ли состязающийся со Вседержителем еще учить? Обличающий Бога пусть отвечает Ему.
\vs Job 39:33 И отвечал Иов Господу и сказал:
\vs Job 39:34 вот, я ничтожен; что буду я отвечать Тебе? Руку мою полагаю на уста мои.
\vs Job 39:35 Однажды я говорил,~--- теперь отвечать не буду, даже дважды, но более не буду.
\vs Job 40:1 И отвечал Господь Иову из бури и сказал:
\vs Job 40:2 препояшь, как муж, чресла твои: Я буду спрашивать тебя, а ты объясняй Мне.
\vs Job 40:3 Ты хочешь ниспровергнуть суд Мой, обвинить Меня, чтобы оправдать себя?
\vs Job 40:4 Такая ли у тебя мышца, как у Бога? И можешь ли возгреметь голосом, как Он?
\vs Job 40:5 Укрась же себя величием и славою, облекись в блеск и великолепие;
\vs Job 40:6 излей ярость гнева твоего, посмотри на все гордое и смири его;
\vs Job 40:7 взгляни на всех высокомерных и унизь их, и сокруши нечестивых на местах их;
\vs Job 40:8 зарой всех их в землю и лица их покрой тьмою.
\vs Job 40:9 Тогда и Я признаю, что десница твоя может спасать тебя.
\vs Job 40:10 Вот бегемот, которого Я создал, как и тебя; он ест траву, как вол;
\vs Job 40:11 вот, его сила в чреслах его и крепость его в мускулах чрева его;
\vs Job 40:12 поворачивает хвостом своим, как кедром; жилы же на бедрах его переплетены;
\vs Job 40:13 ноги у него, как медные трубы; кости у него, как железные прутья;
\vs Job 40:14 это~--- верх путей Божиих; только Сотворивший его может приблизить к нему меч Свой;
\vs Job 40:15 горы приносят ему пищу, и там все звери полевые играют;
\vs Job 40:16 он ложится под тенистыми деревьями, под кровом тростника и в болотах;
\vs Job 40:17 тенистые дерева покрывают его своею тенью; ивы при ручьях окружают его;
\vs Job 40:18 вот, он пьет из реки и не торопится; остается спокоен, хотя бы Иордан устремился ко рту его.
\vs Job 40:19 Возьмет ли кто его в глазах его и проколет ли ему нос багром?
\vs Job 40:20 Можешь ли ты удою вытащить левиафана и веревкою схватить за язык его?
\vs Job 40:21 вденешь ли кольцо в ноздри его? проколешь ли иглою челюсть его?
\vs Job 40:22 будет ли он много умолять тебя и будет ли говорить с тобою кротко?
\vs Job 40:23 сделает ли он договор с тобою, и возьмешь ли его навсегда себе в рабы?
\vs Job 40:24 станешь ли забавляться им, как птичкою, и свяжешь ли его для девочек твоих?
\vs Job 40:25 будут ли продавать его товарищи ловли, разделят ли его между Хананейскими купцами?
\vs Job 40:26 можешь ли пронзить кожу его копьем и голову его рыбачьею острогою?
\vs Job 40:27 Клади на него руку твою, и помни о борьбе: вперед не будешь.
\vs Job 41:1 Надежда тщетна: не упадешь ли от одного взгляда его?
\vs Job 41:2 Нет столь отважного, который осмелился бы потревожить его; кто же может устоять перед Моим лицем?
\vs Job 41:3 Кто предварил Меня, чтобы Мне воздавать ему? под всем небом все Мое.
\vs Job 41:4 Не умолчу о членах его, о силе и красивой соразмерности их.
\vs Job 41:5 Кто может открыть верх одежды его, кто подойдет к двойным челюстям его?
\vs Job 41:6 Кто может отворить двери лица его? круг зубов его~--- ужас;
\vs Job 41:7 крепкие щиты его~--- великолепие; они скреплены как бы твердою печатью;
\vs Job 41:8 один к другому прикасается близко, так что и воздух не проходит между ними;
\vs Job 41:9 один с другим лежат плотно, сцепились и не раздвигаются.
\vs Job 41:10 От его чихания показывается свет; глаза у него как ресницы зари;
\vs Job 41:11 из пасти его выходят пламенники, выскакивают огненные искры;
\vs Job 41:12 из ноздрей его выходит дым, как из кипящего горшка или котла.
\vs Job 41:13 Дыхание его раскаляет угли, и из пасти его выходит пламя.
\vs Job 41:14 На шее его обитает сила, и перед ним бежит ужас.
\vs Job 41:15 Мясистые части тела его сплочены между собою твердо, не дрогнут.
\vs Job 41:16 Сердце его твердо, как камень, и жестко, как нижний жернов.
\vs Job 41:17 Когда он поднимается, силачи в страхе, совсем теряются от ужаса.
\vs Job 41:18 Меч, коснувшийся его, не устоит, ни копье, ни дротик, ни латы.
\vs Job 41:19 Железо он считает за солому, медь~--- за гнилое дерево.
\vs Job 41:20 Дочь лука не обратит его в бегство; пращные камни обращаются для него в плеву.
\vs Job 41:21 Булава считается у него за соломину; свисту дротика он смеется.
\vs Job 41:22 Под ним острые камни, и он на острых камнях лежит в грязи.
\vs Job 41:23 Он кипятит пучину, как котел, и море претворяет в кипящую мазь;
\vs Job 41:24 оставляет за собою светящуюся стезю; бездна кажется сединою.
\vs Job 41:25 Нет на земле подобного ему; он сотворен бесстрашным;
\vs Job 41:26 на все высокое смотрит смело; он царь над всеми сынами гордости.
\vs Job 42:1 И отвечал Иов Господу и сказал:
\vs Job 42:2 знаю, что Ты все можешь, и что намерение Твое не может быть остановлено.
\vs Job 42:3 Кто сей, омрачающий Провидение, ничего не разумея?~--- Так, я говорил о том, чего не разумел, о делах чудных для меня, которых я не знал.
\vs Job 42:4 Выслушай, \bibemph{взывал я}, и я буду говорить, и что буду спрашивать у Тебя, объясни мне.
\vs Job 42:5 Я слышал о Тебе слухом уха; теперь же мои глаза видят Тебя;
\vs Job 42:6 поэтому я отрекаюсь и раскаиваюсь в прахе и пепле.
\rsbpar\vs Job 42:7 И было после того, как Господь сказал слова те Иову, сказал Господь Елифазу Феманитянину: горит гнев Мой на тебя и на двух друзей твоих за то, что вы говорили о Мне не так верно, как раб Мой Иов.
\vs Job 42:8 Итак возьмите себе семь тельцов и семь овнов и пойдите к рабу Моему Иову и принесите за себя жертву; и раб Мой Иов помолится за вас, ибо только лице его Я приму, дабы не отвергнуть вас за то, что вы говорили о Мне не так верно, как раб Мой Иов.
\vs Job 42:9 И пошли Елифаз Феманитянин и Вилдад Савхеянин и Софар Наамитянин, и сделали так, как Господь повелел им,~--- и Господь принял лице Иова.
\rsbpar\vs Job 42:10 И возвратил Господь потерю Иова, когда он помолился за друзей своих; и дал Господь Иову вдвое больше того, что он имел прежде.
\vs Job 42:11 Тогда пришли к нему все братья его и все сестры его и все прежние знакомые его, и ели с ним хлеб в доме его, и тужили с ним, и утешали его за все зло, которое Господь навел на него, и дали ему каждый по кесите и по золотому кольцу.
\rsbpar\vs Job 42:12 И благословил Бог последние дни Иова более, нежели прежние: у него было четырнадцать тысяч мелкого скота, шесть тысяч верблюдов, тысяча пар волов и тысяча ослиц.
\vs Job 42:13 И было у него семь сыновей и три дочери.
\vs Job 42:14 И нарек он имя первой Емима, имя второй~--- Кассия, а имя третьей~--- Керенгаппух.
\vs Job 42:15 И не было на всей земле таких прекрасных женщин, как дочери Иова, и дал им отец их наследство между братьями их.
\vs Job 42:16 После того Иов жил сто сорок лет, и видел сыновей своих и сыновей сыновних до четвертого рода;
\vs Job 42:17 и умер Иов в старости, насыщенный днями.\fns{В Славянской Библии к книге Иова имеется следующее добавление: <<Написано, что он опять восстанет с теми, коих воскресит Господь. О нем толкуется в Сирской книге, что жил он в земле Авситидийской на пределах Идумеи и Аравии: прежде же было имя ему Иовав. Взяв жену Аравитянку, родил сына, которому имя Еннон. Происходил он от отца Зарефа, сынов Исавовых сын, матери же Воссоры, так что был он пятым от Авраама. И сии цари, царствовавшие в Едоме, какою страною и он обладал: первый Валак, сын Веора, и имя городу его Деннава; после же Валака Иовав, называемый Иовом; после сего Ассом, игемон из Феманитской страны; после него Адад, сын Варада, поразивший Мадиама на поле Моава,~--- и имя городу его Гефем. Пришедшие же к нему друзья, Елифаз (сын Софана) от сынов Исавовых, царь Феманский, Валдад (сын Амнона Ховарского) савхейский властитель, Софар Минейский царь. (Феман сын Елифаза, игемон Идумеи. О нем говорится в книге Сирской, что жил в земле Авситидийской, около берегов Евфрата; прежде имя его было Иовав, отец же его был Зареф, от востока солнца.)>>.}

\bibbookdescr{Psa}{
  inline={Псалтирь\fns{У Евреев: <<Книга Хвалений>>.}},
  toc={Псалтирь},
  bookmark={Псалтирь},
  header={Псалтирь},
  %headerleft={},
  %headerright={},
  abbr={Пс}
}
\vs Psa 1:0 Псалом Давида.
\rsbpar\vs Psa 1:1 Блажен муж, который не ходит на совет нечестивых и не стоит на пути грешных и не сидит в собрании развратителей,
\vs Psa 1:2 но в законе Господа воля его, и о законе Его размышляет он день и ночь!
\vs Psa 1:3 И будет он как дерево, посаженное при потоках вод, которое приносит плод свой во время свое, и лист которого не вянет; и во всем, что он ни делает, успеет.
\vs Psa 1:4 Не так~--- нечестивые, [не так]: но они~--- как прах, возметаемый ветром [с лица земли].
\vs Psa 1:5 Потому не устоят\fns{В славянском переводе: Сего ради не воскреснут\dots} нечестивые на суде, и грешники~--- в собрании праведных.
\vs Psa 1:6 Ибо знает Господь путь праведных, а путь нечестивых погибнет.
\vs Psa 2:0 Псалом Давида.
\rsbpar\vs Psa 2:1 Зачем мятутся народы, и племена замышляют тщетное?
\vs Psa 2:2 Восстают цари земли, и князья совещаются вместе против Господа и против Помазанника Его.
\vs Psa 2:3 <<Расторгнем узы их, и свергнем с себя оковы их>>.
\vs Psa 2:4 Живущий на небесах посмеется, Господь поругается им.
\vs Psa 2:5 Тогда скажет им во гневе Своем и яростью Своею приведет их в смятение:
\vs Psa 2:6 <<Я помазал Царя Моего над Сионом, святою горою Моею\fns{6-й стих по переводу 70-ти: Я поставлен от Него Царем над Сионом, святою горою Его.};
\vs Psa 2:7 возвещу определение: Господь сказал Мне: Ты Сын Мой; Я ныне родил Тебя;
\vs Psa 2:8 проси у Меня, и дам народы в наследие Тебе и пределы земли во владение Тебе;
\vs Psa 2:9 Ты поразишь их жезлом железным; сокрушишь их, как сосуд горшечника>>.
\vs Psa 2:10 Итак вразумитесь, цари; научитесь, судьи земли!
\vs Psa 2:11 Служите Господу со страхом и радуйтесь [пред Ним] с трепетом.
\vs Psa 2:12 Почтите Сына, чтобы Он не прогневался, и чтобы вам не погибнуть в пути \bibemph{вашем}, ибо гнев Его возгорится вскоре. Блаженны все, уповающие на Него.
\vs Psa 3:1 Псалом Давида, когда он бежал от Авессалома, сына своего.
\rsbpar\vs Psa 3:2 Господи! как умножились враги мои! Многие восстают на меня;
\vs Psa 3:3 многие говорят душе моей: <<нет ему спасения в Боге>>.
\vs Psa 3:4 Но Ты, Господи, щит предо мною, слава моя, и Ты возносишь голову мою.
\vs Psa 3:5 Гласом моим взываю к Господу, и Он слышит меня со святой горы Своей.
\vs Psa 3:6 Ложусь я, сплю и встаю, ибо Господь защищает меня.
\vs Psa 3:7 Не убоюсь тем народа, которые со всех сторон ополчились на меня.
\vs Psa 3:8 Восстань, Господи! спаси меня, Боже мой! ибо Ты поражаешь в ланиту всех врагов моих; сокрушаешь зубы нечестивых.
\vs Psa 3:9 От Господа спасение. Над народом Твоим благословение Твое.
\vs Psa 4:1 Начальнику хора. На струнных \bibemph{орудиях}. Псалом Давида.
\rsbpar\vs Psa 4:2 Когда я взываю, услышь меня, Боже правды моей! В тесноте Ты давал мне простор. Помилуй меня и услышь молитву мою.
\vs Psa 4:3 Сыны мужей! доколе слава моя будет в поругании? доколе будете любить суету и искать лжи?
\vs Psa 4:4 Знайте, что Господь отделил для Себя святаго Своего; Господь слышит, когда я призываю Его.
\vs Psa 4:5 Гневаясь, не согрешайте: размыслите в сердцах ваших на ложах ваших, и утишитесь;
\vs Psa 4:6 приносите жертвы правды и уповайте на Господа.
\vs Psa 4:7 Многие говорят: <<кто покажет нам благо?>> Яви нам свет лица Твоего, Господи!
\vs Psa 4:8 Ты исполнил сердце мое веселием с того времени, как у них хлеб и вино [и елей] умножились.
\vs Psa 4:9 Спокойно ложусь я и сплю, ибо Ты, Господи, един даешь мне жить в безопасности.
\vs Psa 5:1 Начальнику хора. На духовых \bibemph{орудиях}. Псалом Давида.
\rsbpar\vs Psa 5:2 Услышь, Господи, слова мои, уразумей помышления мои.
\vs Psa 5:3 Внемли гласу вопля моего, Царь мой и Бог мой! ибо я к Тебе молюсь.
\vs Psa 5:4 Господи! рано услышь голос мой,~--- рано предстану пред Тобою, и буду ожидать,
\vs Psa 5:5 ибо Ты Бог, не любящий беззакония; у Тебя не водворится злой;
\vs Psa 5:6 нечестивые не пребудут пред очами Твоими: Ты ненавидишь всех, делающих беззаконие.
\vs Psa 5:7 Ты погубишь говорящих ложь; кровожадного и коварного гнушается Господь.
\vs Psa 5:8 А я, по множеству милости Твоей, войду в дом Твой, поклонюсь святому храму Твоему в страхе Твоем.
\vs Psa 5:9 Господи! путеводи меня в правде Твоей, ради врагов моих; уровняй предо мною путь Твой.
\vs Psa 5:10 Ибо нет в устах их истины: сердце их~--- пагуба, гортань их~--- открытый гроб, языком своим льстят.
\vs Psa 5:11 Осуди их, Боже, да падут они от замыслов своих; по множеству нечестия их, отвергни их, ибо они возмутились против Тебя.
\vs Psa 5:12 И возрадуются все уповающие на Тебя, вечно будут ликовать, и Ты будешь покровительствовать им; и будут хвалиться Тобою любящие имя Твое.
\vs Psa 5:13 Ибо Ты благословляешь праведника, Господи; благоволением, как щитом, венчаешь его.
\vs Psa 6:1 Начальнику хора. На восьмиструнном. Псалом Давида.
\rsbpar\vs Psa 6:2 Господи! не в ярости Твоей обличай меня и не во гневе Твоем наказывай меня.
\vs Psa 6:3 Помилуй меня, Господи, ибо я немощен; исцели меня, Господи, ибо кости мои потрясены;
\vs Psa 6:4 и душа моя сильно потрясена; Ты же, Господи, доколе?
\vs Psa 6:5 Обратись, Господи, избавь душу мою, спаси меня ради милости Твоей,
\vs Psa 6:6 ибо в смерти нет памятования о Тебе: во гробе кто будет славить Тебя?
\vs Psa 6:7 Утомлен я воздыханиями моими: каждую ночь омываю ложе мое, слезами моими омочаю постель мою.
\vs Psa 6:8 Иссохло от печали око мое, обветшало от всех врагов моих.
\vs Psa 6:9 Удалитесь от меня все, делающие беззаконие, ибо услышал Господь голос плача моего,
\vs Psa 6:10 услышал Господь моление мое; Господь примет молитву мою.
\vs Psa 6:11 Да будут постыжены и жестоко поражены все враги мои; да возвратятся и постыдятся мгновенно.
\vs Psa 7:1 Плачевная песнь, которую Давид воспел Господу по делу Хуса, из племени Вениаминова.
\rsbpar\vs Psa 7:2 Господи, Боже мой! на Тебя я уповаю; спаси меня от всех гонителей моих и избавь меня;
\vs Psa 7:3 да не исторгнет он, подобно льву, души моей, терзая, когда нет избавляющего [и спасающего].
\vs Psa 7:4 Господи, Боже мой! если я что сделал, если есть неправда в руках моих,
\vs Psa 7:5 если я платил злом тому, кто был со мною в мире,~--- я, который спасал даже того, кто без причины стал моим врагом,~---
\vs Psa 7:6 то пусть враг преследует душу мою и настигнет, пусть втопчет в землю жизнь мою, и славу мою повергнет в прах.
\vs Psa 7:7 Восстань, Господи, во гневе Твоем; подвигнись против неистовства врагов моих, пробудись для меня на суд, который Ты заповедал,~---
\vs Psa 7:8 сонм людей станет вокруг Тебя; над ним поднимись на высоту.
\vs Psa 7:9 Господь судит народы. Суди меня, Господи, по правде моей и по непорочности моей во мне.
\vs Psa 7:10 Да прекратится злоба нечестивых, а праведника подкрепи, ибо Ты испытуешь сердца и утробы, праведный Боже!
\vs Psa 7:11 Щит мой в Боге, спасающем правых сердцем.
\vs Psa 7:12 Бог~--- судия праведный, [крепкий и долготерпеливый,] и Бог, всякий день строго взыскивающий,
\vs Psa 7:13 если \bibemph{кто} не обращается. Он изощряет Свой меч, напрягает лук Свой и направляет его,
\vs Psa 7:14 приготовляет для него сосуды смерти, стрелы Свои делает палящими.
\vs Psa 7:15 Вот, \bibemph{нечестивый} зачал неправду, был чреват злобою и родил себе ложь;
\vs Psa 7:16 рыл ров, и выкопал его, и упал в яму, которую приготовил:
\vs Psa 7:17 злоба его обратится на его голову, и злодейство его упадет на его темя.
\vs Psa 7:18 Славлю Господа по правде Его и пою имени Господа Всевышнего.
\vs Psa 8:1 Начальнику хора. На Гефском \bibemph{орудии}. Псалом Давида.
\rsbpar\vs Psa 8:2 Господи, Боже наш! как величественно имя Твое по всей земле! Слава Твоя простирается превыше небес!
\vs Psa 8:3 Из уст младенцев и грудных детей Ты устроил хвалу, ради врагов Твоих, дабы сделать безмолвным врага и мстителя.
\vs Psa 8:4 Когда взираю я на небеса Твои~--- дело Твоих перстов, на луну и звезды, которые Ты поставил,
\vs Psa 8:5 то чт\acc{о} \bibemph{есть} человек, что Ты помнишь его, и сын человеческий, что Ты посещаешь его?
\vs Psa 8:6 Не много Ты умалил его пред Ангелами: славою и честью увенчал его;
\vs Psa 8:7 поставил его владыкою над делами рук Твоих; всё положил под ноги его:
\vs Psa 8:8 овец и волов всех, и также полевых зверей,
\vs Psa 8:9 птиц небесных и рыб морских, все, преходящее морскими стезями.
\vs Psa 8:10 Господи, Боже наш! Как величественно имя Твое по всей земле!
\vs Psa 9:1 Начальнику хора. По смерти Лабена. Псалом Давида.
\rsbpar\vs Psa 9:2 Буду славить [Тебя], Господи, всем сердцем моим, возвещать все чудеса Твои.
\vs Psa 9:3 Буду радоваться и торжествовать о Тебе, петь имени Твоему, Всевышний.
\vs Psa 9:4 Когда враги мои обращены назад, то преткнутся и погибнут пред лицем Твоим,
\vs Psa 9:5 ибо Ты производил мой суд и мою тяжбу; Ты воссел на престоле, Судия праведный.
\vs Psa 9:6 Ты вознегодовал на народы, погубил нечестивого, имя их изгладил на веки и веки.
\vs Psa 9:7 У врага совсем не стало оружия, и город\acc{а} Ты разрушил; погибла память их с ними.
\vs Psa 9:8 Но Господь пребывает вовек; Он приготовил для суда престол Свой,
\vs Psa 9:9 и Он будет судить вселенную по правде, совершит суд над народами по правоте.
\vs Psa 9:10 И будет Господь прибежищем угнетенному, прибежищем во времена скорби;
\vs Psa 9:11 и будут уповать на Тебя знающие имя Твое, потому что Ты не оставляешь ищущих Тебя, Господи.
\vs Psa 9:12 Пойте Господу, живущему на Сионе, возвещайте между народами дела Его,
\vs Psa 9:13 ибо Он взыскивает за кровь; помнит их, не забывает вопля угнетенных.
\vs Psa 9:14 Помилуй меня, Господи; воззри на страдание мое от ненавидящих меня,~--- Ты, Который возносишь меня от врат смерти,
\vs Psa 9:15 чтобы я возвещал все хвалы Твои во вратах дщери Сионовой: буду радоваться о спасении Твоем.
\vs Psa 9:16 Обрушились народы в яму, которую выкопали; в сети, которую скрыли они, запуталась нога их.
\vs Psa 9:17 Познан был Господь по суду, который Он совершил; нечестивый уловлен делами рук своих.
\vs Psa 9:18 Да обратятся нечестивые в ад,~--- все народы, забывающие Бога.
\vs Psa 9:19 Ибо не навсегда забыт будет нищий, и надежда бедных не до конца погибнет.
\vs Psa 9:20 Восстань, Господи, да не преобладает человек, да судятся народы пред лицем Твоим.
\vs Psa 9:21 Наведи, Господи, страх на них; да знают народы, что человеки они.
\vs Psa 9:22 Для чего, Господи, стоишь вдали, скрываешь Себя во время скорби?
\vs Psa 9:23 По гордости своей нечестивый преследует бедного: да уловятся они ухищрениями, которые сами вымышляют.
\vs Psa 9:24 Ибо нечестивый хвалится похотью души своей; корыстолюбец ублажает себя.
\vs Psa 9:25 В надмении своем нечестивый пренебрегает Господа: <<не взыщет>>; во всех помыслах его: <<нет Бога!>>
\vs Psa 9:26 Во всякое время пути его гибельны; суды Твои далеки для него; на всех врагов своих он смотрит с пренебрежением;
\vs Psa 9:27 говорит в сердце своем: <<не поколеблюсь; в род и род не приключится \bibemph{мне} зла>>;
\vs Psa 9:28 уста его полны проклятия, коварства и лжи; под языком~--- его мучение и пагуба;
\vs Psa 9:29 сидит в засаде за двором, в потаенных местах убивает невинного; глаза его подсматривают за бедным;
\vs Psa 9:30 подстерегает в потаенном месте, как лев в логовище; подстерегает в засаде, чтобы схватить бедного; хватает бедного, увлекая в сети свои;
\vs Psa 9:31 сгибается, прилегает,~--- и бедные падают в сильные когти его;
\vs Psa 9:32 говорит в сердце своем: <<забыл Бог, закрыл лице Свое, не увидит никогда>>.
\vs Psa 9:33 Восстань, Господи, Боже [мой], вознеси руку Твою, не забудь угнетенных [Твоих до конца].
\vs Psa 9:34 Зачем нечестивый пренебрегает Бога, говоря в сердце своем: <<Ты не взыщешь>>?
\vs Psa 9:35 Ты видишь, ибо Ты взираешь на обиды и притеснения, чтобы воздать Твоею рукою. Тебе предает себя бедный; сироте Ты помощник.
\vs Psa 9:36 Сокруши мышцу нечестивому и злому, так чтобы искать и не найти его нечестия.
\vs Psa 9:37 Господь~--- царь на веки, навсегда; исчезнут язычники с земли Его.
\vs Psa 9:38 Господи! Ты слышишь желания смиренных; укрепи сердце их; открой ухо Твое,
\vs Psa 9:39 чтобы дать суд сироте и угнетенному, да не устрашает более человек на земле.
\vs Psa 10:0 Начальнику хора. Псалом Давида.
\rsbpar\vs Psa 10:1 На Господа уповаю; как же вы говорите душе моей: <<улетай на гору вашу, \bibemph{как} птица>>?
\vs Psa 10:2 Ибо вот, нечестивые натянули лук, стрелу свою приложили к тетиве, чтобы во тьме стрелять в правых сердцем.
\vs Psa 10:3 Когда разрушены основания, что сделает праведник?
\vs Psa 10:4 Господь во святом храме Своем, Господь,~--- престол Его на небесах, очи Его зрят [на нищего]; вежды Его испытывают сынов человеческих.
\vs Psa 10:5 Господь испытывает праведного, а нечестивого и любящего насилие ненавидит душа Его.
\vs Psa 10:6 Дождем прольет Он на нечестивых горящие угли, огонь и серу; и палящий ветер~--- их доля из чаши;
\vs Psa 10:7 ибо Господь праведен, любит правду; лице Его видит праведника.
\vs Psa 11:1 Начальнику хора. На восьмиструнном. Псалом Давида.
\rsbpar\vs Psa 11:2 Спаси [меня], Господи, ибо не стало праведного, ибо нет верных между сынами человеческими.
\vs Psa 11:3 Ложь говорит каждый своему ближнему; уста льстивы, говорят от сердца притворного.
\vs Psa 11:4 Истребит Господь все уста льстивые, язык велеречивый,
\vs Psa 11:5 \bibemph{тех}, которые говорят: <<языком нашим пересилим, уста наши с нами; кто нам господин>>?
\vs Psa 11:6 Ради страдания нищих и воздыхания бедных ныне восстану, говорит Господь, поставлю в безопасности того, кого уловить хотят.
\vs Psa 11:7 Слова Господни~--- слова чистые, серебро, очищенное от земли в горниле, семь раз переплавленное.
\vs Psa 11:8 Ты, Господи, сохранишь их, соблюдешь от рода сего вовек.
\vs Psa 11:9 Повсюду ходят нечестивые, когда ничтожные из сынов человеческих возвысились.
\vs Psa 12:1 Начальнику хора. Псалом Давида.
\rsbpar\vs Psa 12:2 Доколе, Господи, будешь забывать меня вконец, доколе будешь скрывать лице Твое от меня?
\vs Psa 12:3 Доколе мне слагать советы в душе моей, скорбь в сердце моем день [и ночь]? Доколе врагу моему возноситься надо мною?
\vs Psa 12:4 Призри, услышь меня, Господи Боже мой! Просвети очи мои, да не усну я \bibemph{сном} смертным;
\vs Psa 12:5 да не скажет враг мой: <<я одолел его>>. Да не возрадуются гонители мои, если я поколеблюсь.
\vs Psa 12:6 Я же уповаю на милость Твою; сердце мое возрадуется о спасении Твоем; воспою Господу, облагодетельствовавшему меня, [и буду петь имени Господа Всевышнего].
\vs Psa 13:0 Начальнику хора. Псалом Давида.
\rsbpar\vs Psa 13:1 Сказал безумец в сердце своем: <<нет Бога>>. Они развратились, совершили гнусные дела; нет делающего добро.
\vs Psa 13:2 Господь с небес призрел на сынов человеческих, чтобы видеть, есть ли разумеющий, ищущий Бога.
\vs Psa 13:3 Все уклонились, сделались равно непотребными; нет делающего добро, нет ни одного.
\vs Psa 13:4 Неужели не вразумятся все, делающие беззаконие, съедающие народ мой, \bibemph{как} едят хлеб, и не призывающие Господа?
\vs Psa 13:5 Там убоятся они страха, [где нет страха,] ибо Бог в роде праведных.
\vs Psa 13:6 Вы посмеялись над мыслью нищего, что Господь упование его.
\vs Psa 13:7 <<Кто даст с Сиона спасение Израилю!>> Когда Господь возвратит пленение народа Своего, тогда возрадуется Иаков и возвеселится Израиль.
\vs Psa 14:0 Псалом Давида.
\rsbpar\vs Psa 14:1 Господи! кто может пребывать в жилище Твоем? кто может обитать на святой горе Твоей?
\vs Psa 14:2 Тот, кто ходит непорочно и делает правду, и говорит истину в сердце своем;
\vs Psa 14:3 кто не клевещет языком своим, не делает искреннему своему зла и не принимает поношения на ближнего своего;
\vs Psa 14:4 тот, в глазах которого презрен отверженный, но который боящихся Господа славит; кто клянется, \bibemph{хотя бы} злому, и не изменяет;
\vs Psa 14:5 кто серебра своего не отдает в рост и не принимает даров против невинного. Поступающий так не поколеблется вовек.
\vs Psa 15:0 Песнь Давида.
\rsbpar\vs Psa 15:1 Храни меня, Боже, ибо я на Тебя уповаю.
\vs Psa 15:2 Я сказал Господу: Ты~--- Господь мой; блага мои Тебе не нужны.
\vs Psa 15:3 К святым, которые на земле, и к дивным \bibemph{Твоим}~--- к ним все желание мое.
\vs Psa 15:4 Пусть умножаются скорби у тех, которые текут к \bibemph{богу} чужому; я не возлию кровавых возлияний их и не помяну имен их устами моими.
\vs Psa 15:5 Господь есть часть наследия моего и чаши моей. Ты держишь жребий мой.
\vs Psa 15:6 Межи мои прошли по прекрасным \bibemph{местам}, и наследие мое приятно для меня.
\vs Psa 15:7 Благословлю Господа, вразумившего меня; даже и ночью учит меня внутренность моя.
\vs Psa 15:8 Всегда видел я пред собою Господа, ибо Он одесную меня; не поколеблюсь.
\vs Psa 15:9 Оттого возрадовалось сердце мое и возвеселился язык мой; даже и плоть моя успокоится в уповании,
\vs Psa 15:10 ибо Ты не оставишь души моей в аде и не дашь святому Твоему увидеть тление,
\vs Psa 15:11 Ты укажешь мне путь жизни: полнота радостей пред лицем Твоим, блаженство в деснице Твоей вовек.
\vs Psa 16:0 Молитва Давида.
\rsbpar\vs Psa 16:1 Услышь, Господи, правду [мою], внемли воплю моему, прими мольбу из уст нелживых.
\vs Psa 16:2 От Твоего лица суд мне да изыдет; да воззрят очи Твои на правоту.
\vs Psa 16:3 Ты испытал сердце мое, посетил меня ночью, искусил меня и ничего не нашел; от мыслей моих не отступают уста мои.
\vs Psa 16:4 В делах человеческих, по слову уст Твоих, я охранял себя от путей притеснителя.
\vs Psa 16:5 Утверди шаги мои на путях Твоих, да не колеблются стопы мои.
\vs Psa 16:6 К Тебе взываю я, ибо Ты услышишь меня, Боже; приклони ухо Твое ко мне, услышь слова мои.
\vs Psa 16:7 Яви дивную милость Твою, Спаситель уповающих [на Тебя] от противящихся деснице Твоей.
\vs Psa 16:8 Храни меня, как зеницу ока; в тени крыл Твоих укрой меня
\vs Psa 16:9 от лица нечестивых, нападающих на меня,~--- от врагов души моей, окружающих меня:
\vs Psa 16:10 они заключились в туке своем, надменно говорят устами своими.
\vs Psa 16:11 На всяком шагу нашем ныне окружают нас; они устремили глаза свои, чтобы низложить \bibemph{меня} на землю;
\vs Psa 16:12 они подобны льву, жаждущему добычи, подобны скимну, сидящему в местах скрытных.
\vs Psa 16:13 Восстань, Господи, предупреди их, низложи их. Избавь душу мою от нечестивого мечом Твоим,
\vs Psa 16:14 от людей~--- рукою Твоею, Господи, от людей мира, которых удел в \bibemph{этой} жизни, которых чрево Ты наполняешь из сокровищниц Твоих; сыновья их сыты и оставят остаток детям своим.
\vs Psa 16:15 А я в правде буду взирать на лице Твое; пробудившись, буду насыщаться образом Твоим.
\vs Psa 17:1 Начальнику хора. Раба Господня Давида, который произнес слова песни сей к Господу, когда Господь избавил его от рук всех врагов его и от руки Саула. И он сказал:
\rsbpar\vs Psa 17:2 Возлюблю тебя, Господи, крепость моя!
\vs Psa 17:3 Господь~--- твердыня моя и прибежище мое, Избавитель мой, Бог мой,~--- скала моя; на Него я уповаю; щит мой, рог спасения моего и убежище мое.
\vs Psa 17:4 Призову достопоклоняемого Господа и от врагов моих спасусь.
\vs Psa 17:5 Объяли меня муки смертные, и потоки беззакония устрашили меня;
\vs Psa 17:6 цепи ада облегли меня, и сети смерти опутали меня.
\vs Psa 17:7 В тесноте моей я призвал Господа и к Богу моему воззвал. И Он услышал от [святаго] чертога Своего голос мой, и вопль мой дошел до слуха Его.
\vs Psa 17:8 Потряслась и всколебалась земля, дрогнули и подвиглись основания гор, ибо разгневался [Бог];
\vs Psa 17:9 поднялся дым от гнева Его и из уст Его огонь поядающий; горячие угли \bibemph{сыпались} от Него.
\vs Psa 17:10 Наклонил Он небеса и сошел,~--- и мрак под ногами Его.
\vs Psa 17:11 И воссел на Херувимов и полетел, и понесся на крыльях ветра.
\vs Psa 17:12 И мрак сделал покровом Своим, сению вокруг Себя мрак вод, облаков воздушных.
\vs Psa 17:13 От блистания пред Ним бежали облака Его, град и угли огненные.
\vs Psa 17:14 Возгремел на небесах Господь, и Всевышний дал глас Свой, град и угли огненные.
\vs Psa 17:15 Пустил стрелы Свои и рассеял их, множество молний, и рассыпал их.
\vs Psa 17:16 И явились источники вод, и открылись основания вселенной от грозного \bibemph{гласа} Твоего, Господи, от дуновения духа гнева Твоего.
\vs Psa 17:17 Он простер \bibemph{руку} с высоты и взял меня, и извлек меня из вод многих;
\vs Psa 17:18 избавил меня от врага моего сильного и от ненавидящих меня, которые были сильнее меня.
\vs Psa 17:19 Они восстали на меня в день бедствия моего, но Господь был мне опорою.
\vs Psa 17:20 Он вывел меня на пространное место и избавил меня, ибо Он благоволит ко мне.
\vs Psa 17:21 Воздал мне Господь по правде моей, по чистоте рук моих вознаградил меня,
\vs Psa 17:22 ибо я хранил пути Господни и не был нечестивым пред Богом моим;
\vs Psa 17:23 ибо все заповеди Его предо мною, и от уставов Его я не отступал.
\vs Psa 17:24 Я был непорочен пред Ним и остерегался, чтобы не согрешить мне;
\vs Psa 17:25 и воздал мне Господь по правде моей, по чистоте рук моих пред очами Его.
\vs Psa 17:26 С милостивым Ты поступаешь милостиво, с мужем искренним~--- искренно,
\vs Psa 17:27 с чистым~--- чисто, а с лукавым~--- по лукавству его,
\vs Psa 17:28 ибо Ты людей угнетенных спасаешь, а очи надменные унижаешь.
\vs Psa 17:29 Ты возжигаешь светильник мой, Господи; Бог мой просвещает тьму мою.
\vs Psa 17:30 С Тобою я поражаю войско, с Богом моим восхожу на стену.
\vs Psa 17:31 Бог!~--- Непорочен путь Его, чисто слово Господа; щит Он для всех, уповающих на Него.
\vs Psa 17:32 Ибо кто Бог, кроме Господа, и кто защита, кроме Бога нашего?
\vs Psa 17:33 Бог препоясывает меня силою и устрояет мне верный путь;
\vs Psa 17:34 делает ноги мои, как оленьи, и на высотах моих поставляет меня;
\vs Psa 17:35 научает руки мои брани, и мышцы мои сокрушают медный лук.
\vs Psa 17:36 Ты дал мне щит спасения Твоего, и десница Твоя поддерживает меня, и милость Твоя возвеличивает меня.
\vs Psa 17:37 Ты расширяешь шаг мой подо мною, и не колеблются ноги мои.
\vs Psa 17:38 Я преследую врагов моих и настигаю их, и не возвращаюсь, доколе не истреблю их;
\vs Psa 17:39 поражаю их, и они не могут встать, падают под ноги мои,
\vs Psa 17:40 ибо Ты препоясал меня силою для войны и низложил под ноги мои восставших на меня;
\vs Psa 17:41 Ты обратил ко мне тыл врагов моих, и я истребляю ненавидящих меня:
\vs Psa 17:42 они вопиют, но нет спасающего; ко Господу,~--- но Он не внемлет им;
\vs Psa 17:43 я рассеваю их, как прах пред лицем ветра, как уличную грязь попираю их.
\vs Psa 17:44 Ты избавил меня от мятежа народа, поставил меня главою иноплеменников; народ, которого я не знал, служит мне;
\vs Psa 17:45 по одному слуху о мне повинуются мне; иноплеменники ласкательствуют предо мною;
\vs Psa 17:46 иноплеменники бледнеют и трепещут в укреплениях своих.
\vs Psa 17:47 Жив Господь и благословен защитник мой! Да будет превознесен Бог спасения моего,
\vs Psa 17:48 Бог, мстящий за меня и покоряющий мне народы,
\vs Psa 17:49 и избавляющий меня от врагов моих! Ты вознес меня над восстающими против меня и от человека жестокого избавил меня.
\vs Psa 17:50 За то буду славить Тебя, Господи, между иноплеменниками и буду петь имени Твоему,
\vs Psa 17:51 величественно спасающий царя и творящий милость помазаннику Твоему Давиду и потомству его во веки.
\vs Psa 18:1 Начальнику хора. Псалом Давида.
\rsbpar\vs Psa 18:2 Небеса проповедуют славу Божию, и о делах рук Его вещает твердь.
\vs Psa 18:3 День дню передает речь, и ночь ночи открывает знание.
\vs Psa 18:4 Нет языка, и нет наречия, где не слышался бы голос их.
\vs Psa 18:5 По всей земле проходит звук их, и до пределов вселенной слов\acc{а} их. Он поставил в них жилище солнцу,
\vs Psa 18:6 и оно выходит, как жених из брачного чертога своего, радуется, как исполин, пробежать поприще:
\vs Psa 18:7 от края небес исход его, и шествие его до края их, и ничто не укрыто от теплоты его.
\vs Psa 18:8 Закон Господа совершен, укрепляет душу; откровение Господа верно, умудряет простых.
\vs Psa 18:9 Повеления Господа праведны, веселят сердце; заповедь Господа светла, просвещает очи.
\vs Psa 18:10 Страх Господень чист, пребывает вовек. Суды Господни истина, все праведны;
\vs Psa 18:11 они вожделеннее золота и даже множества золота чистого, слаще меда и капель сота;
\vs Psa 18:12 и раб Твой охраняется ими, в соблюдении их великая награда.
\vs Psa 18:13 Кто усмотрит погрешности свои? От тайных \bibemph{моих} очисти меня
\vs Psa 18:14 и от умышленных удержи раба Твоего, чтобы не возобладали мною. Тогда я буду непорочен и чист от великого развращения.
\vs Psa 18:15 Да будут слова уст моих и помышление сердца моего благоугодны пред Тобою, Господи, твердыня моя и Избавитель мой!
\vs Psa 19:1 Начальнику хора. Псалом Давида.
\rsbpar\vs Psa 19:2 Да услышит тебя Господь в день печали, да защитит тебя имя Бога Иаковлева.
\vs Psa 19:3 Да пошлет тебе помощь из Святилища и с Сиона да подкрепит тебя.
\vs Psa 19:4 Да воспомянет все жертвоприношения твои и всесожжение твое да соделает тучным.
\vs Psa 19:5 Да даст тебе [Господь] по сердцу твоему и все намерения твои да исполнит.
\vs Psa 19:6 Мы возрадуемся о спасении твоем и во имя Бога нашего поднимем знамя. Да исполнит Господь все прошения твои.
\vs Psa 19:7 Ныне познал я, что Господь спасает помазанника Своего, отвечает ему со святых небес Своих могуществом спасающей десницы Своей.
\vs Psa 19:8 Иные колесницами, иные конями, а мы именем Господа Бога нашего хвалимся:
\vs Psa 19:9 они поколебались и пали, а мы встали и стоим прямо.
\vs Psa 19:10 Господи! спаси царя и услышь нас, когда будем взывать [к Тебе].
\vs Psa 20:1 Начальнику хора. Псалом Давида.
\rsbpar\vs Psa 20:2 Господи! силою Твоею веселится царь и о спасении Твоем безмерно радуется.
\vs Psa 20:3 Ты дал ему, чего желало сердце его, и прошения уст его не отринул,
\vs Psa 20:4 ибо Ты встретил его благословениями благости, возложил на голову его венец из чистого золота.
\vs Psa 20:5 Он просил у Тебя жизни; Ты дал ему долгоденствие на век и век.
\vs Psa 20:6 Велика слава его в спасении Твоем; Ты возложил на него честь и величие.
\vs Psa 20:7 Ты положил на него благословения на веки, возвеселил его радостью лица Твоего,
\vs Psa 20:8 ибо царь уповает на Господа, и по благости Всевышнего не поколеблется.
\vs Psa 20:9 Рука Твоя найдет всех врагов Твоих, десница Твоя найдет [всех] ненавидящих Тебя.
\vs Psa 20:10 Во время гнева Твоего Ты сделаешь их, как печь огненную; во гневе Своем Господь погубит их, и пожрет их огонь.
\vs Psa 20:11 Ты истребишь плод их с земли и семя их~--- из среды сынов человеческих,
\vs Psa 20:12 ибо они предприняли против Тебя злое, составили замыслы, но не могли [выполнить их].
\vs Psa 20:13 Ты поставишь их целью, из луков Твоих пустишь стрелы в лице их.
\vs Psa 20:14 Вознесись, Господи, силою Твоею: мы будем воспевать и прославлять Твое могущество.
\vs Psa 21:1 Начальнику хора. При появлении зари. Псалом Давида.
\rsbpar\vs Psa 21:2 Боже мой! Боже мой! [внемли мне] для чего Ты оставил меня? Далеки от спасения моего слова вопля моего.
\vs Psa 21:3 Боже мой! я вопию днем,~--- и Ты не внемлешь мне, ночью,~--- и нет мне успокоения.
\vs Psa 21:4 Но Ты, Святый, живешь среди славословий Израиля.
\vs Psa 21:5 На Тебя уповали отцы наши; уповали, и Ты избавлял их;
\vs Psa 21:6 к Тебе взывали они, и были спасаемы; на Тебя уповали, и не оставались в стыде.
\vs Psa 21:7 Я же червь, а не человек, поношение у людей и презрение в народе.
\vs Psa 21:8 Все, видящие меня, ругаются надо мною, говорят устами, кивая головою:
\vs Psa 21:9 <<он уповал на Господа; пусть избавит его, пусть спасет, если он угоден Ему>>.
\vs Psa 21:10 Но Ты извел меня из чрева, вложил в меня упование у грудей матери моей.
\vs Psa 21:11 На Тебя оставлен я от утробы; от чрева матери моей Ты~--- Бог мой.
\vs Psa 21:12 Не удаляйся от меня, ибо скорбь близка, а помощника нет.
\vs Psa 21:13 Множество тельцов обступили меня; тучные Васанские окружили меня,
\vs Psa 21:14 раскрыли на меня пасть свою, \bibemph{как} лев, алчущий добычи и рыкающий.
\vs Psa 21:15 Я пролился, как вода; все кости мои рассыпались; сердце мое сделалось, как воск, растаяло посреди внутренности моей.
\vs Psa 21:16 Сила моя иссохла, как черепок; язык мой прильпнул к гортани моей, и Ты свел меня к персти смертной.
\vs Psa 21:17 Ибо псы окружили меня, скопище злых обступило меня, пронзили руки мои и ноги мои.
\vs Psa 21:18 Можно было бы перечесть все кости мои; а они смотрят и делают из меня зрелище;
\vs Psa 21:19 делят ризы мои между собою и об одежде моей бросают жребий.
\vs Psa 21:20 Но Ты, Господи, не удаляйся от меня; сила моя! поспеши на помощь мне;
\vs Psa 21:21 избавь от меча душу мою и от псов одинокую мою;
\vs Psa 21:22 спаси меня от пасти льва и от рогов единорогов, услышав, \bibemph{избавь} меня.
\vs Psa 21:23 Буду возвещать имя Твое братьям моим, посреди собрания восхвалять Тебя.
\vs Psa 21:24 Боящиеся Господа! восхвалите Его. Все семя Иакова! прославь Его. Да благоговеет пред Ним все семя Израиля,
\vs Psa 21:25 ибо Он не презрел и не пренебрег скорби страждущего, не скрыл от него лица Своего, но услышал его, когда сей воззвал к Нему.
\vs Psa 21:26 О Тебе хвала моя в собрании великом; воздам обеты мои пред боящимися Его.
\vs Psa 21:27 Да едят бедные и насыщаются; да восхвалят Господа ищущие Его; да живут сердца ваши во веки!
\vs Psa 21:28 Вспомнят, и обратятся к Господу все концы земли, и поклонятся пред Тобою все племена язычников,
\vs Psa 21:29 ибо Господне есть царство, и Он~--- Владыка над народами.
\vs Psa 21:30 Будут есть и поклоняться все тучные земли; преклонятся пред Ним все нисходящие в персть и не могущие сохранить жизни своей.
\vs Psa 21:31 Потомство [мое] будет служить Ему, и будет называться Господним вовек:
\vs Psa 21:32 придут и будут возвещать правду Его людям, которые родятся, чт\acc{о} сотворил Господь.
\vs Psa 22:0 Псалом Давида.
\rsbpar\vs Psa 22:1 Господь~--- Пастырь мой; я ни в чем не буду нуждаться:
\vs Psa 22:2 Он покоит меня на злачных пажитях и водит меня к водам тихим,
\vs Psa 22:3 подкрепляет душу мою, направляет меня на стези правды ради имени Своего.
\vs Psa 22:4 Если я пойду и долиною смертной тени, не убоюсь зла, потому что Ты со мной; Твой жезл и Твой посох~--- они успокаивают меня.
\vs Psa 22:5 Ты приготовил предо мною трапезу в виду врагов моих; умастил елеем голову мою; чаша моя преисполнена.
\vs Psa 22:6 Так, благость и милость [Твоя] да сопровождают меня во все дни жизни моей, и я пребуду в доме Господнем многие дни.
\vs Psa 23:0 Псалом Давида. [В первый день недели.]
\rsbpar\vs Psa 23:1 Господня земля и что наполняет ее, вселенная и все живущее в ней,
\vs Psa 23:2 ибо Он основал ее на морях и на реках утвердил ее.
\vs Psa 23:3 Кто взойдет на гору Господню, или кто станет на святом месте Его?
\vs Psa 23:4 Тот, у которого руки неповинны и сердце чисто, кто не клялся душею своею напрасно и не божился ложно [ближнему своему],~---
\vs Psa 23:5 \bibemph{тот} получит благословение от Господа и милость от Бога, Спасителя своего.
\vs Psa 23:6 Таков род ищущих Его, ищущих лица Твоего, Боже Иакова!
\vs Psa 23:7 Поднимите, врата, верхи ваши, и поднимитесь, двери вечные, и войдет Царь славы!
\vs Psa 23:8 Кто сей Царь славы?~--- Господь крепкий и сильный, Господь, сильный в брани.
\vs Psa 23:9 Поднимите, врата, верхи ваши, и поднимитесь, двери вечные, и войдет Царь славы!
\vs Psa 23:10 Кто сей Царь славы?~--- Господь сил, Он~--- Царь славы.
\vs Psa 24:0 Псалом Давида.
\rsbpar\vs Psa 24:1 К Тебе, Господи, возношу душу мою.
\vs Psa 24:2 Боже мой! на Тебя уповаю, да не постыжусь [вовек], да не восторжествуют надо мною враги мои,
\vs Psa 24:3 да не постыдятся и все надеющиеся на Тебя: да постыдятся беззаконнующие втуне.
\vs Psa 24:4 Укажи мне, Господи, пути Твои и научи меня стезям Твоим.
\vs Psa 24:5 Направь меня на истину Твою и научи меня, ибо Ты Бог спасения моего; на Тебя надеюсь всякий день.
\vs Psa 24:6 Вспомни щедроты Твои, Господи, и милости Твои, ибо они от века.
\vs Psa 24:7 Грехов юности моей и преступлений моих не вспоминай; по милости Твоей вспомни меня Ты, ради благости Твоей, Господи!
\vs Psa 24:8 Благ и праведен Господь, посему наставляет грешников на путь,
\vs Psa 24:9 направляет кротких к правде, и научает кротких путям Своим.
\vs Psa 24:10 Все пути Господни~--- милость и истина к хранящим завет Его и откровения Его.
\vs Psa 24:11 Ради имени Твоего, Господи, прости согрешение мое, ибо велико оно.
\vs Psa 24:12 Кто есть человек, боящийся Господа? Ему укажет Он путь, который избрать.
\vs Psa 24:13 Душа его пребудет во благе, и семя его наследует землю.
\vs Psa 24:14 Тайна Господня~--- боящимся Его, и завет Свой Он открывает им.
\vs Psa 24:15 Очи мои всегда к Господу, ибо Он извлекает из сети ноги мои.
\vs Psa 24:16 Призри на меня и помилуй меня, ибо я одинок и угнетен.
\vs Psa 24:17 Скорби сердца моего умножились; выведи меня из бед моих,
\vs Psa 24:18 призри на страдание мое и на изнеможение мое и прости все грехи мои.
\vs Psa 24:19 Посмотри на врагов моих, как много их, и \bibemph{какою} лютою ненавистью они ненавидят меня.
\vs Psa 24:20 Сохрани душу мою и избавь меня, да не постыжусь, что я на Тебя уповаю.
\vs Psa 24:21 Непорочность и правота да охраняют меня, ибо я на Тебя надеюсь.
\vs Psa 24:22 Избавь, Боже, Израиля от всех скорбей его.
\vs Psa 25:0 Псалом Давида.
\rsbpar\vs Psa 25:1 Рассуди меня, Господи, ибо я ходил в непорочности моей, и, уповая на Господа, не поколеблюсь.
\vs Psa 25:2 Искуси меня, Господи, и испытай меня; расплавь внутренности мои и сердце мое,
\vs Psa 25:3 ибо милость Твоя пред моими очами, и я ходил в истине Твоей,
\vs Psa 25:4 не сидел я с людьми лживыми, и с коварными не пойду;
\vs Psa 25:5 возненавидел я сборище злонамеренных, и с нечестивыми не сяду;
\vs Psa 25:6 буду омывать в невинности руки мои и обходить жертвенник Твой, Господи,
\vs Psa 25:7 чтобы возвещать гласом хвалы и поведать все чудеса Твои.
\vs Psa 25:8 Господи! возлюбил я обитель дома Твоего и место жилища славы Твоей.
\vs Psa 25:9 Не погуби души моей с грешниками и жизни моей с кровожадными,
\vs Psa 25:10 у которых в руках злодейство, и которых правая рука полна мздоимства.
\vs Psa 25:11 А я хожу в моей непорочности; избавь меня, [Господи,] и помилуй меня.
\vs Psa 25:12 Моя нога стоит на прямом \bibemph{пути}; в собраниях благословлю Господа.
\vs Psa 26:0 Псалом Давида. [Прежде помазания.]
\rsbpar\vs Psa 26:1 Господь~--- свет мой и спасение мое: кого мне бояться? Господь крепость жизни моей: кого мне страшиться?
\vs Psa 26:2 Если будут наступать на меня злодеи, противники и враги мои, чтобы пожрать плоть мою, то они сами преткнутся и падут.
\vs Psa 26:3 Если ополчится против меня полк, не убоится сердце мое; если восстанет на меня война, и тогда буду надеяться.
\vs Psa 26:4 Одного просил я у Господа, того только ищу, чтобы пребывать мне в доме Господнем во все дни жизни моей, созерцать красоту Господню и посещать [святый] храм Его,
\vs Psa 26:5 ибо Он укрыл бы меня в скинии Своей в день бедствия, скрыл бы меня в потаенном месте селения Своего, вознес бы меня на скалу.
\vs Psa 26:6 Тогда вознеслась бы голова моя над врагами, окружающими меня; и я принес бы в Его скинии жертвы славословия, стал бы петь и воспевать пред Господом.
\vs Psa 26:7 Услышь, Господи, голос мой, которым я взываю, помилуй меня и внемли мне.
\vs Psa 26:8 Сердце мое говорит от Тебя: <<ищите лица Моего>>; и я буду искать лица Твоего, Господи.
\vs Psa 26:9 Не скрой от меня лица Твоего; не отринь во гневе раба Твоего. Ты был помощником моим; не отвергни меня и не оставь меня, Боже, Спаситель мой!
\vs Psa 26:10 ибо отец мой и мать моя оставили меня, но Господь примет меня.
\vs Psa 26:11 Научи меня, Господи, пути Твоему и наставь меня на стезю правды, ради врагов моих;
\vs Psa 26:12 не предавай меня на произвол врагам моим, ибо восстали на меня свидетели лживые и дышат злобою.
\vs Psa 26:13 Но я верую, что увижу благость Господа на земле живых.
\vs Psa 26:14 Надейся на Господа, мужайся, и да укрепляется сердце твое, и надейся на Господа.
\vs Psa 27:0 Псалом Давида.
\rsbpar\vs Psa 27:1 К тебе, Господи, взываю: твердыня моя! не будь безмолвен для меня, чтобы при безмолвии Твоем я не уподобился нисходящим в могилу.
\vs Psa 27:2 Услышь голос молений моих, когда я взываю к Тебе, когда поднимаю руки мои к святому храму Твоему.
\vs Psa 27:3 Не погуби меня с нечестивыми и с делающими неправду, которые с ближними своими говорят о мире, а в сердце у них зло.
\vs Psa 27:4 Воздай им по делам их, по злым поступкам их; по делам рук их воздай им, отдай им заслуженное ими.
\vs Psa 27:5 За то, что они невнимательны к действиям Господа и к делу рук Его, Он разрушит их и не созиждет их.
\vs Psa 27:6 Благословен Господь, ибо Он услышал голос молений моих.
\vs Psa 27:7 Господь~--- крепость моя и щит мой; на Него уповало сердце мое, и Он помог мне, и возрадовалось сердце мое; и я прославлю Его песнью моею.
\vs Psa 27:8 Господь~--- крепость народа Своего и спасительная защита помазанника Своего.
\vs Psa 27:9 Спаси народ Твой и благослови наследие Твое; паси их и возвышай их во веки!
\vs Psa 28:0 Псалом Давида. [При окончании праздника кущей.]
\rsbpar\vs Psa 28:1 Воздайте Господу, сыны Божии, воздайте Господу славу и честь,
\vs Psa 28:2 воздайте Господу славу имени Его; поклонитесь Господу в благолепном святилище \bibemph{Его}.
\vs Psa 28:3 Глас Господень над водами; Бог славы возгремел, Господь над водами многими.
\vs Psa 28:4 Глас Господа силен, глас Господа величествен.
\vs Psa 28:5 Глас Господа сокрушает кедры; Господь сокрушает кедры Ливанские
\vs Psa 28:6 и заставляет их скакать подобно тельцу, Ливан и Сирион, подобно молодому единорогу.
\vs Psa 28:7 Глас Господа высекает пламень огня.
\vs Psa 28:8 Глас Господа потрясает пустыню; потрясает Господь пустыню Кадес.
\vs Psa 28:9 Глас Господа разрешает от бремени ланей и обнажает леса; и во храме Его все возвещает о \bibemph{Его} славе.
\vs Psa 28:10 Господь восседал над потопом, и будет восседать Господь царем вовек.
\vs Psa 28:11 Господь даст силу народу Своему, Господь благословит народ Свой миром.
\vs Psa 29:1 Псалом Давида; песнь при обновлении дома.
\rsbpar\vs Psa 29:2 Превознесу Тебя, Господи, что Ты поднял меня и не дал моим врагам восторжествовать надо мною.
\vs Psa 29:3 Господи, Боже мой! я воззвал к Тебе, и Ты исцелил меня.
\vs Psa 29:4 Господи! Ты вывел из ада душу мою и оживил меня, чтобы я не сошел в могилу.
\vs Psa 29:5 Пойте Господу, святые Его, славьте память святыни Его,
\vs Psa 29:6 ибо на мгновение гнев Его, на \bibemph{всю} жизнь благоволение Его: вечером водворяется плач, а на утро радость.
\vs Psa 29:7 И я говорил в благоденствии моем: <<не поколеблюсь вовек>>.
\vs Psa 29:8 По благоволению Твоему, Господи, Ты укрепил гору мою; но Ты сокрыл лице Твое, \bibemph{и} я смутился.
\vs Psa 29:9 \bibemph{Тогда} к Тебе, Господи, взывал я, и Господа [моего] умолял:
\vs Psa 29:10 <<что пользы в крови моей, когда я сойду в могилу? будет ли прах славить Тебя? будет ли возвещать истину Твою?
\vs Psa 29:11 услышь, Господи, и помилуй меня; Господи! будь мне помощником>>.
\vs Psa 29:12 И Ты обратил сетование мое в ликование, снял с меня вретище и препоясал меня веселием,
\vs Psa 29:13 да славит Тебя душа моя и да не умолкает. Господи, Боже мой! буду славить Тебя вечно.
\vs Psa 30:1 Начальнику хора. Псалом Давида. [Во время смятения.]
\rsbpar\vs Psa 30:2 На Тебя, Господи, уповаю, да не постыжусь вовек; по правде Твоей избавь меня;
\vs Psa 30:3 приклони ко мне ухо Твое, поспеши избавить меня. Будь мне каменною твердынею, домом прибежища, чтобы спасти меня,
\vs Psa 30:4 ибо Ты каменная гора моя и ограда моя; ради имени Твоего води меня и управляй мною.
\vs Psa 30:5 Выведи меня из сети, которую тайно поставили мне, ибо Ты крепость моя.
\vs Psa 30:6 В Твою руку предаю дух мой; Ты избавлял меня, Господи, Боже истины.
\vs Psa 30:7 Ненавижу почитателей суетных идолов, но на Господа уповаю.
\vs Psa 30:8 Буду радоваться и веселиться о милости Твоей, потому что Ты призрел на бедствие мое, узнал горесть души моей
\vs Psa 30:9 и не предал меня в руки врага; поставил ноги мои на пространном месте.
\vs Psa 30:10 Помилуй меня, Господи, ибо тесно мне; иссохло от горести око мое, душа моя и утроба моя.
\vs Psa 30:11 Истощилась в печали жизнь моя и лета мои в стенаниях; изнемогла от грехов моих сила моя, и кости мои иссохли.
\vs Psa 30:12 От всех врагов моих я сделался поношением даже у соседей моих и страшилищем для знакомых моих; видящие меня на улице бегут от меня.
\vs Psa 30:13 Я забыт в сердцах, как мертвый; я~--- как сосуд разбитый,
\vs Psa 30:14 ибо слышу злоречие многих; отвсюду ужас, когда они сговариваются против меня, умышляют исторгнуть душу мою.
\vs Psa 30:15 А я на Тебя, Господи, уповаю; я говорю: Ты~--- мой Бог.
\vs Psa 30:16 В Твоей руке дни мои; избавь меня от руки врагов моих и от гонителей моих.
\vs Psa 30:17 Яви светлое лице Твое рабу Твоему; спаси меня милостью Твоею.
\vs Psa 30:18 Господи! да не постыжусь, что я к Тебе взываю; нечестивые же да посрамятся, да умолкнут в аде.
\vs Psa 30:19 Да онемеют уста лживые, которые против праведника говорят злое с гордостью и презреньем.
\vs Psa 30:20 Как много у Тебя благ, которые Ты хранишь для боящихся Тебя и которые приготовил уповающим на Тебя пред сынами человеческими!
\vs Psa 30:21 Ты укрываешь их под покровом лица Твоего от мятежей людских, скрываешь их под сенью от пререкания языков.
\vs Psa 30:22 Благословен Господь, что явил мне дивную милость Свою в укрепленном городе!
\vs Psa 30:23 В смятении моем я думал: <<отвержен я от очей Твоих>>; но Ты услышал голос молитвы моей, когда я воззвал к Тебе.
\vs Psa 30:24 Любите Господа, все праведные Его; Господь хранит верных и поступающим надменно воздает с избытком.
\vs Psa 30:25 Мужайтесь, и да укрепляется сердце ваше, все надеющиеся на Господа!
\vs Psa 31:0 Псалом Давида. Учение.
\rsbpar\vs Psa 31:1 Блажен, кому отпущены беззакония, и чьи грехи покрыты!
\vs Psa 31:2 Блажен человек, которому Господь не вменит греха, и в чьем духе нет лукавства!
\vs Psa 31:3 Когда я молчал, обветшали кости мои от вседневного стенания моего,
\vs Psa 31:4 ибо день и ночь тяготела надо мною рука Твоя; свежесть моя исчезла, как в летнюю засуху.
\vs Psa 31:5 Но я открыл Тебе грех мой и не скрыл беззакония моего; я сказал: <<исповедаю Господу преступления мои>>, и Ты снял с меня вину греха моего.
\vs Psa 31:6 За то помолится Тебе каждый праведник во время благопотребное, и тогда разлитие многих вод не достигнет его.
\vs Psa 31:7 Ты покров мой: Ты охраняешь меня от скорби, окружаешь меня радостями избавления.
\vs Psa 31:8 <<Вразумлю тебя, наставлю тебя на путь, по которому тебе идти; буду руководить тебя, око Мое над тобою>>.
\vs Psa 31:9 <<Не будьте как конь, как лошак несмысленный, которых челюсти нужно обуздывать уздою и удилами, чтобы они покорялись тебе>>.
\vs Psa 31:10 Много скорбей нечестивому, а уповающего на Господа окружает милость.
\vs Psa 31:11 Веселитесь о Господе и радуйтесь, праведные; торжествуйте, все правые сердцем.
\vs Psa 32:0 [Псалом Давида.]
\rsbpar\vs Psa 32:1 Радуйтесь, праведные, о Господе: правым прилично славословить.
\vs Psa 32:2 Славьте Господа на гуслях, пойте Ему на десятиструнной псалтири;
\vs Psa 32:3 пойте Ему новую песнь; пойте Ему стройно, с восклицанием,
\vs Psa 32:4 ибо слово Господне право и все дела Его верны.
\vs Psa 32:5 Он любит правду и суд; милости Господней полна земля.
\vs Psa 32:6 Словом Господа сотворены небеса, и духом уст Его~--- все воинство их:
\vs Psa 32:7 Он собрал, будто груды, морские воды, положил бездны в хранилищах.
\vs Psa 32:8 Да боится Господа вся земля; да трепещут пред Ним все живущие во вселенной,
\vs Psa 32:9 ибо Он сказал,~--- и сделалось; Он повелел,~--- и явилось.
\vs Psa 32:10 Господь разрушает советы язычников, уничтожает замыслы народов, [уничтожает советы князей].
\vs Psa 32:11 Совет же Господень стоит вовек; помышления сердца Его~--- в род и род.
\vs Psa 32:12 Блажен народ, у которого Господь есть Бог,~--- племя, которое Он избрал в наследие Себе.
\vs Psa 32:13 С небес призирает Господь, видит всех сынов человеческих;
\vs Psa 32:14 с престола, на котором восседает, Он призирает на всех, живущих на земле:
\vs Psa 32:15 Он создал сердца всех их и вникает во все дела их.
\vs Psa 32:16 Не спасется царь множеством воинства; исполина не защитит великая сила.
\vs Psa 32:17 Ненадежен конь для спасения, не избавит великою силою своею.
\vs Psa 32:18 Вот, око Господне над боящимися Его и уповающими на милость Его,
\vs Psa 32:19 что Он душу их спасет от смерти и во время голода пропитает их.
\vs Psa 32:20 Душа наша уповает на Господа: Он~--- помощь наша и защита наша;
\vs Psa 32:21 о Нем веселится сердце наше, ибо на святое имя Его мы уповали.
\vs Psa 32:22 Да будет милость Твоя, Господи, над нами, как мы уповаем на Тебя.
\vs Psa 33:1 Псалом Давида, когда он притворился безумным пред Авимелехом и был изгнан от него и удалился.
\rsbpar\vs Psa 33:2 Благословлю Господа во всякое время; хвала Ему непрестанно в устах моих.
\vs Psa 33:3 Господом будет хвалиться душа моя; услышат кроткие и возвеселятся.
\vs Psa 33:4 Величайте Господа со мною, и превознесем имя Его вместе.
\vs Psa 33:5 Я взыскал Господа, и Он услышал меня, и от всех опасностей моих избавил меня.
\vs Psa 33:6 Кто обращал взор к Нему, те просвещались, и лица их не постыдятся.
\vs Psa 33:7 Сей нищий воззвал,~--- и Господь услышал и спас его от всех бед его.
\vs Psa 33:8 Ангел Господень ополчается вокруг боящихся Его и избавляет их.
\vs Psa 33:9 Вкусите, и увидите, как благ Господь! Блажен человек, который уповает на Него!
\vs Psa 33:10 Бойтесь Господа, [все] святые Его, ибо нет скудости у боящихся Его.
\vs Psa 33:11 Скимны бедствуют и терпят голод, а ищущие Господа не терпят нужды ни в каком благе.
\vs Psa 33:12 Придите, дети, послушайте меня: страху Господню научу вас.
\vs Psa 33:13 Хочет ли человек жить и любит ли долгоденствие, чтобы видеть благо?
\vs Psa 33:14 Удерживай язык свой от зла и уста свои от коварных слов.
\vs Psa 33:15 Уклоняйся от зла и делай добро; ищи мира и следуй за ним.
\vs Psa 33:16 Очи Господни \bibemph{обращены} на праведников, и уши Его~--- к воплю их.
\vs Psa 33:17 Но лице Господне против делающих зло, чтобы истребить с земли память о них.
\vs Psa 33:18 Взывают [праведные], и Господь слышит, и от всех скорбей их избавляет их.
\vs Psa 33:19 Близок Господь к сокрушенным сердцем и смиренных духом спасет.
\vs Psa 33:20 Много скорбей у праведного, и от всех их избавит его Господь.
\vs Psa 33:21 Он хранит все кости его; ни одна из них не сокрушится.
\vs Psa 33:22 Убьет грешника зло, и ненавидящие праведного погибнут.
\vs Psa 33:23 Избавит Господь душу рабов Своих, и никто из уповающих на Него не погибнет.
\vs Psa 34:0 Псалом Давида.
\rsbpar\vs Psa 34:1 Вступись, Господи, в тяжбу с тяжущимися со мною, побори борющихся со мною;
\vs Psa 34:2 возьми щит и латы и восстань на помощь мне;
\vs Psa 34:3 обнажи меч и прегради \bibemph{путь} преследующим меня; скажи душе моей: <<Я~--- спасение твое!>>
\vs Psa 34:4 Да постыдятся и посрамятся ищущие души моей; да обратятся назад и покроются бесчестием умышляющие мне зло;
\vs Psa 34:5 да будут они, как прах пред лицем ветра, и Ангел Господень да прогоняет \bibemph{их};
\vs Psa 34:6 да будет путь их темен и скользок, и Ангел Господень да преследует их,
\vs Psa 34:7 ибо они без вины скрыли для меня яму~--- сеть свою, без вины выкопали \bibemph{ее} для души моей.
\vs Psa 34:8 Да придет на него гибель неожиданная, и сеть его, которую он скрыл \bibemph{для меня}, да уловит его самого; да впадет в нее на погибель.
\vs Psa 34:9 А моя душа будет радоваться о Господе, будет веселиться о спасении от Него.
\vs Psa 34:10 Все кости мои скажут: <<Господи! кто подобен Тебе, избавляющему слабого от сильного, бедного и нищего от грабителя его?>>
\vs Psa 34:11 Восстали на меня свидетели неправедные: чего я не знаю, о том допрашивают меня;
\vs Psa 34:12 воздают мне злом за добро, сиротством душе моей.
\vs Psa 34:13 Я во время болезни их одевался во вретище, изнурял постом душу мою, и молитва моя возвращалась в недро мое.
\vs Psa 34:14 Я поступал, как бы это был друг мой, брат мой; я ходил скорбный, с поникшею головою, как бы оплакивающий мать.
\vs Psa 34:15 А когда я претыкался, они радовались и собирались; собирались ругатели против меня, не знаю за что, поносили и не переставали;
\vs Psa 34:16 с лицемерными насмешниками скрежетали на меня зубами своими.
\vs Psa 34:17 Господи! долго ли будешь смотреть \bibemph{на это}? Отведи душу мою от злодейств их, от львов~--- одинокую мою.
\vs Psa 34:18 Я прославлю Тебя в собрании великом, среди народа многочисленного восхвалю Тебя,
\vs Psa 34:19 чтобы не торжествовали надо мною враждующие против меня неправедно, и не перемигивались глазами ненавидящие меня безвинно;
\vs Psa 34:20 ибо не о мире говорят они, но против мирных земли составляют лукавые замыслы;
\vs Psa 34:21 расширяют на меня уста свои; говорят: <<хорошо! хорошо! видел глаз наш>>.
\vs Psa 34:22 Ты видел, Господи, не умолчи; Господи! не удаляйся от меня.
\vs Psa 34:23 Подвигнись, пробудись для суда моего, для тяжбы моей, Боже мой и Господи мой!
\vs Psa 34:24 Суди меня по правде Твоей, Господи, Боже мой, и да не торжествуют они надо мною;
\vs Psa 34:25 да не говорят в сердце своем: <<хорошо! [хорошо!] по душе нашей!>> Да не говорят: <<мы поглотили его>>.
\vs Psa 34:26 Да постыдятся и посрамятся все, радующиеся моему несчастью; да облекутся в стыд и позор величающиеся надо мною.
\vs Psa 34:27 Да радуются и веселятся желающие правоты моей и говорят непрестанно: <<да возвеличится Господь, желающий мира рабу Своему!>>
\vs Psa 34:28 И язык мой будет проповедовать правду Твою и хвалу Твою всякий день.
\vs Psa 35:1 Начальнику хора. Раба Господня Давида.
\rsbpar\vs Psa 35:2 Нечестие беззаконного говорит в сердце моем: нет страха Божия пред глазами его,
\vs Psa 35:3 ибо он льстит себе в глазах своих, будто отыскивает беззаконие свое, чтобы возненавидеть его;
\vs Psa 35:4 слова уст его~--- неправда и лукавство; не хочет он вразумиться, чтобы делать добро;
\vs Psa 35:5 на ложе своем замышляет беззаконие, становится на путь недобрый, не гнушается злом.
\vs Psa 35:6 Господи! милость Твоя до небес, истина Твоя до облаков!
\vs Psa 35:7 Правда Твоя, как горы Божии, и судьбы Твои~--- бездна великая! Человеков и скотов хранишь Ты, Господи!
\vs Psa 35:8 Как драгоценна милость Твоя, Боже! Сыны человеческие в тени крыл Твоих покойны:
\vs Psa 35:9 насыщаются от тука дома Твоего, и из потока сладостей Твоих Ты напояешь их,
\vs Psa 35:10 ибо у Тебя источник жизни; во свете Твоем мы видим свет.
\vs Psa 35:11 Продли милость Твою к знающим Тебя и правду Твою к правым сердцем,
\vs Psa 35:12 да не наступит на меня нога гордыни, и рука грешника да не изгонит меня:
\vs Psa 35:13 там пали делающие беззаконие, низринуты и не могут встать.
\vs Psa 36:0 Псалом Давида.
\rsbpar\vs Psa 36:1 Не ревнуй злодеям, не завидуй делающим беззаконие,
\vs Psa 36:2 ибо они, как трава, скоро будут подкошены и, как зеленеющий злак, увянут.
\vs Psa 36:3 Уповай на Господа и делай добро; живи на земле и храни истину.
\vs Psa 36:4 Утешайся Господом, и Он исполнит желания сердца твоего.
\vs Psa 36:5 Предай Господу путь твой и уповай на Него, и Он совершит,
\vs Psa 36:6 и выведет, как свет, правду твою и справедливость твою, как полдень.
\vs Psa 36:7 Покорись Господу и надейся на Него. Не ревнуй успевающему в пути своем, человеку лукавствующему.
\vs Psa 36:8 Перестань гневаться и оставь ярость; не ревнуй до того, чтобы делать зло,
\vs Psa 36:9 ибо делающие зло истребятся, уповающие же на Господа наследуют землю.
\vs Psa 36:10 Еще немного, и не станет нечестивого; посмотришь на его место, и нет его.
\vs Psa 36:11 А кроткие наследуют землю и насладятся множеством мира.
\vs Psa 36:12 Нечестивый злоумышляет против праведника и скрежещет на него зубами своими:
\vs Psa 36:13 Господь же посмевается над ним, ибо видит, что приходит день его.
\vs Psa 36:14 Нечестивые обнажают меч и натягивают лук свой, чтобы низложить бедного и нищего, чтобы пронзить \bibemph{идущих} прямым путем:
\vs Psa 36:15 меч их войдет в их же сердце, и луки их сокрушатся.
\vs Psa 36:16 Малое у праведника~--- лучше богатства многих нечестивых,
\vs Psa 36:17 ибо мышцы нечестивых сокрушатся, а праведников подкрепляет Господь.
\vs Psa 36:18 Господь знает дни непорочных, и достояние их пребудет вовек:
\vs Psa 36:19 не будут они постыжены во время лютое и во дни голода будут сыты;
\vs Psa 36:20 а нечестивые погибнут, и враги Господни, как тук агнцев, исчезнут, в дыме исчезнут.
\vs Psa 36:21 Нечестивый берет взаймы и не отдает, а праведник милует и дает,
\vs Psa 36:22 ибо благословенные Им наследуют землю, а проклятые Им истребятся.
\vs Psa 36:23 Господом утверждаются стопы \bibemph{такого} человека, и Он благоволит к пути его:
\vs Psa 36:24 когда он будет падать, не упадет, ибо Господь поддерживает его за руку.
\vs Psa 36:25 Я был молод и состарился, и не видал праведника оставленным и потомков его просящими хлеба:
\vs Psa 36:26 он всякий день милует и взаймы дает, и потомство его в благословение будет.
\vs Psa 36:27 Уклоняйся от зла, и делай добро, и будешь жить вовек:
\vs Psa 36:28 ибо Господь любит правду и не оставляет святых Своих; вовек сохранятся они; [а беззаконные будут извержены] и потомство нечестивых истребится.
\vs Psa 36:29 Праведники наследуют землю и будут жить на ней вовек.
\vs Psa 36:30 Уста праведника изрекают премудрость, и язык его произносит правду.
\vs Psa 36:31 Закон Бога его в сердце у него; не поколеблются стопы его.
\vs Psa 36:32 Нечестивый подсматривает за праведником и ищет умертвить его;
\vs Psa 36:33 но Господь не отдаст его в руки его и не допустит обвинить его, когда он будет судим.
\vs Psa 36:34 Уповай на Господа и держись пути Его: и Он вознесет тебя, чтобы ты наследовал землю; и когда будут истребляемы нечестивые, ты увидишь.
\vs Psa 36:35 Видел я нечестивца грозного, расширявшегося, подобно укоренившемуся многоветвистому дереву;
\vs Psa 36:36 но он прошел, и вот нет его; ищу его и не нахожу.
\vs Psa 36:37 Наблюдай за непорочным и смотри на праведного, ибо будущность \bibemph{такого} человека есть мир;
\vs Psa 36:38 а беззаконники все истребятся; будущность нечестивых погибнет.
\vs Psa 36:39 От Господа спасение праведникам, Он~--- защита их во время скорби;
\vs Psa 36:40 и поможет им Господь и избавит их; избавит их от нечестивых и спасет их, ибо они на Него уповают.
\vs Psa 37:1 Псалом Давида. В воспоминание [о субботе].
\rsbpar\vs Psa 37:2 Господи! не в ярости Твоей обличай меня и не во гневе Твоем наказывай меня,
\vs Psa 37:3 ибо стрелы Твои вонзились в меня, и рука Твоя тяготеет на мне.
\vs Psa 37:4 Нет целого места в плоти моей от гнева Твоего; нет мира в костях моих от грехов моих,
\vs Psa 37:5 ибо беззакония мои превысили голову мою, как тяжелое бремя отяготели на мне,
\vs Psa 37:6 смердят, гноятся раны мои от безумия моего.
\vs Psa 37:7 Я согбен и совсем поник, весь день сетуя хожу,
\vs Psa 37:8 ибо чресла мои полны воспалениями, и нет целого места в плоти моей.
\vs Psa 37:9 Я изнемог и сокрушен чрезмерно; кричу от терзания сердца моего.
\vs Psa 37:10 Господи! пред Тобою все желания мои, и воздыхание мое не сокрыто от Тебя.
\vs Psa 37:11 Сердце мое трепещет; оставила меня сила моя, и свет очей моих,~--- и того нет у меня.
\vs Psa 37:12 Друзья мои и искренние отступили от язвы моей, и ближние мои стоят вдали.
\vs Psa 37:13 Ищущие же души моей ставят сети, и желающие мне зла говорят о погибели \bibemph{моей} и замышляют всякий день козни;
\vs Psa 37:14 а я, как глухой, не слышу, и как немой, который не открывает уст своих;
\vs Psa 37:15 и стал я, как человек, который не слышит и не имеет в устах своих ответа,
\vs Psa 37:16 ибо на Тебя, Господи, уповаю я; Ты услышишь, Господи, Боже мой.
\vs Psa 37:17 И я сказал: да не восторжествуют надо мною [враги мои]; когда колеблется нога моя, они величаются надо мною.
\vs Psa 37:18 Я близок к падению, и скорбь моя всегда предо мною.
\vs Psa 37:19 Беззаконие мое я сознаю, сокрушаюсь о грехе моем.
\vs Psa 37:20 А враги мои живут и укрепляются, и умножаются ненавидящие меня безвинно;
\vs Psa 37:21 и воздающие мне злом за добро враждуют против меня за то, что я следую добру.
\vs Psa 37:22 Не оставь меня, Господи, Боже мой! Не удаляйся от меня;
\vs Psa 37:23 поспеши на помощь мне, Господи, Спаситель мой!
\vs Psa 38:1 Начальнику хора, Идифуму. Псалом Давида.
\rsbpar\vs Psa 38:2 Я сказал: буду я наблюдать за путями моими, чтобы не согрешать мне языком моим; буду обуздывать уста мои, доколе нечестивый предо мною.
\vs Psa 38:3 Я был нем и безгласен, и молчал \bibemph{даже} о добром; и скорбь моя подвиглась.
\vs Psa 38:4 Воспламенилось сердце мое во мне; в мыслях моих возгорелся огонь; я стал говорить языком моим:
\vs Psa 38:5 скажи мне, Господи, кончину мою и число дней моих, какое оно, дабы я знал, какой век мой.
\vs Psa 38:6 Вот, Ты дал мне дни, \bibemph{как} пяди, и век мой как ничто пред Тобою. Подлинно, совершенная суета~--- всякий человек живущий.
\vs Psa 38:7 Подлинно, человек ходит подобно призраку; напрасно он суетится, собирает и не знает, кому достанется то.
\vs Psa 38:8 И ныне чего ожидать мне, Господи? надежда моя~--- на Тебя.
\vs Psa 38:9 От всех беззаконий моих избавь меня, не предавай меня на поругание безумному.
\vs Psa 38:10 Я стал нем, не открываю уст моих; потому что Ты соделал это.
\vs Psa 38:11 Отклони от меня удары Твои; я исчезаю от поражающей руки Твоей.
\vs Psa 38:12 Если Ты обличениями будешь наказывать человека за преступления, то рассыплется, как от моли, краса его. Так, суетен всякий человек!
\vs Psa 38:13 Услышь, Господи, молитву мою и внемли воплю моему; не будь безмолвен к слезам моим, ибо странник я у Тебя \bibemph{и} пришлец, как и все отцы мои.
\vs Psa 38:14 Отступи от меня, чтобы я мог подкрепиться, прежде нежели отойду и не будет меня.
\vs Psa 39:1 Начальнику хора. Псалом Давида.
\rsbpar\vs Psa 39:2 Твердо уповал я на Господа, и Он приклонился ко мне и услышал вопль мой;
\vs Psa 39:3 извлек меня из страшного рва, из тинистого болота, и поставил на камне ноги мои и утвердил стопы мои;
\vs Psa 39:4 и вложил в уста мои новую песнь~--- хвалу Богу нашему. Увидят многие и убоятся и будут уповать на Господа.
\vs Psa 39:5 Блажен человек, который на Господа возлагает надежду свою и не обращается к гордым и к уклоняющимся ко лжи.
\vs Psa 39:6 Много соделал Ты, Господи, Боже мой: о чудесах и помышлениях Твоих о нас~--- кто уподобится Тебе!~--- хотел бы я проповедовать и говорить, но они превышают число.
\vs Psa 39:7 Жертвы и приношения Ты не восхотел; Ты открыл мне уши\fns{Открыл мне уши~--- по переводу 70-ти: уготовил мне тело.}; всесожжения и жертвы за грех Ты не потребовал.
\vs Psa 39:8 Тогда я сказал: вот, иду; в свитке книжном написано о мне:
\vs Psa 39:9 я желаю исполнить волю Твою, Боже мой, и закон Твой у меня в сердце.
\vs Psa 39:10 Я возвещал правду Твою в собрании великом; я не возбранял устам моим: Ты, Господи, знаешь.
\vs Psa 39:11 Правды Твоей не скрывал в сердце моем, возвещал верность Твою и спасение Твое, не утаивал милости Твоей и истины Твоей пред собранием великим.
\vs Psa 39:12 Не удерживай, Господи, щедрот Твоих от меня; милость Твоя и истина Твоя да охраняют меня непрестанно,
\vs Psa 39:13 ибо окружили меня беды неисчислимые; постигли меня беззакония мои, так что видеть не могу: их более, нежели волос на голове моей; сердце мое оставило меня.
\vs Psa 39:14 Благоволи, Господи, избавить меня; Господи! поспеши на помощь мне.
\vs Psa 39:15 Да постыдятся и посрамятся все, ищущие погибели душе моей! Да будут обращены назад и преданы посмеянию желающие мне зла!
\vs Psa 39:16 Да смятутся от посрамления своего говорящие мне: <<хорошо! хорошо!>>
\vs Psa 39:17 Да радуются и веселятся Тобою все ищущие Тебя, и любящие спасение Твое да говорят непрестанно: <<велик Господь!>>
\vs Psa 39:18 Я же беден и нищ, но Господь печется о мне. Ты~--- помощь моя и избавитель мой, Боже мой! не замедли.
\vs Psa 40:1 Начальнику хора. Псалом Давида.
\rsbpar\vs Psa 40:2 Блажен, кто помышляет о бедном [и нищем]! В день бедствия избавит его Господь.
\vs Psa 40:3 Господь сохранит его и сбережет ему жизнь; блажен будет он на земле. И Ты не отдашь его на волю врагов его.
\vs Psa 40:4 Господь укрепит его на одре болезни его. Ты изменишь все ложе его в болезни его.
\vs Psa 40:5 Я сказал: Господи! помилуй меня, исцели душу мою, ибо согрешил я пред Тобою.
\vs Psa 40:6 Враги мои говорят обо мне злое: <<когда он умрет и погибнет имя его?>>
\vs Psa 40:7 И если приходит кто видеть меня, говорит ложь; сердце его слагает в себе неправду, и он, выйдя вон, толкует.
\vs Psa 40:8 Все ненавидящие меня шепчут между собою против меня, замышляют на меня зло:
\vs Psa 40:9 <<слово велиала пришло на него; он слег; не встать ему более>>.
\vs Psa 40:10 Даже человек мирный со мною, на которого я полагался, который ел хлеб мой, поднял на меня пяту.
\vs Psa 40:11 Ты же, Господи, помилуй меня и восставь меня, и я воздам им.
\vs Psa 40:12 Из того узнаю, что Ты благоволишь ко мне, если враг мой не восторжествует надо мною,
\vs Psa 40:13 а меня сохранишь в целости моей и поставишь пред лицем Твоим на веки.
\vs Psa 40:14 Благословен Господь Бог Израилев от века и до века! Аминь, аминь!
\vs Psa 41:1 Начальнику хора. Учение. Сынов Кореевых.
\rsbpar\vs Psa 41:2 Как лань желает к потокам воды, так желает душа моя к Тебе, Боже!
\vs Psa 41:3 Жаждет душа моя к Богу крепкому, живому: когда приду и явлюсь пред лице Божие!
\vs Psa 41:4 Слезы мои были для меня хлебом день и ночь, когда говорили мне всякий день: <<где Бог твой?>>
\vs Psa 41:5 Вспоминая об этом, изливаю душу мою, потому что я ходил в многолюдстве, вступал с ними в дом Божий со гласом радости и славословия празднующего сонма.
\vs Psa 41:6 Что унываешь ты, душа моя, и что смущаешься? Уповай на Бога, ибо я буду еще славить Его, Спасителя моего и Бога моего.
\vs Psa 41:7 Унывает во мне душа моя; посему я воспоминаю о Тебе с земли Иорданской, с Ермона, с горы Цоар.
\vs Psa 41:8 Бездна бездну призывает голосом водопадов Твоих; все воды Твои и волны Твои прошли надо мною.
\vs Psa 41:9 Днем явит Господь милость Свою, и ночью песнь Ему у меня, молитва к Богу жизни моей.
\vs Psa 41:10 Скажу Богу, заступнику моему: для чего Ты забыл меня? Для чего я сетуя хожу от оскорблений врага?
\vs Psa 41:11 Как бы поражая кости мои, ругаются надо мною враги мои, когда говорят мне всякий день: <<где Бог твой?>>
\vs Psa 41:12 Что унываешь ты, душа моя, и что смущаешься? Уповай на Бога, ибо я буду еще славить Его, Спасителя моего и Бога моего.
\vs Psa 42:1 Суди меня, Боже, и вступись в тяжбу мою с народом недобрым. От человека лукавого и несправедливого избавь меня,
\vs Psa 42:2 ибо Ты Бог крепости моей. Для чего Ты отринул меня? для чего я сетуя хожу от оскорблений врага?
\vs Psa 42:3 Пошли свет Твой и истину Твою; да ведут они меня и приведут на святую гору Твою и в обители Твои.
\vs Psa 42:4 И подойду я к жертвеннику Божию, к Богу радости и веселия моего, и на гуслях буду славить Тебя, Боже, Боже мой!
\vs Psa 42:5 Что унываешь ты, душа моя, и что смущаешься? Уповай на Бога; ибо я буду еще славить Его, Спасителя моего и Бога моего.
\vs Psa 43:1 Начальнику хора. Учение. Сынов Кореевых.
\rsbpar\vs Psa 43:2 Боже, мы слышали ушами своими, отцы наши рассказывали нам о деле, какое Ты соделал во дни их, во дни древние:
\vs Psa 43:3 Ты рукою Твоею истребил народы, а их насадил; поразил племена и изгнал их;
\vs Psa 43:4 ибо они не мечом своим приобрели землю, и не их мышца спасла их, но Твоя десница и Твоя мышца и свет лица Твоего, ибо Ты благоволил к ним.
\vs Psa 43:5 Боже, Царь мой! Ты~--- тот же; даруй спасение Иакову.
\vs Psa 43:6 С Тобою избодаем рогами врагов наших; во имя Твое попрем ногами восстающих на нас:
\vs Psa 43:7 ибо не на лук мой уповаю, и не меч мой спасет меня;
\vs Psa 43:8 но Ты спасешь нас от врагов наших, и посрамишь ненавидящих нас.
\vs Psa 43:9 О Боге похвалимся всякий день, и имя Твое будем прославлять вовек.
\vs Psa 43:10 Но ныне Ты отринул и посрамил нас, и не выходишь с войсками нашими;
\vs Psa 43:11 обратил нас в бегство от врага, и ненавидящие нас грабят нас;
\vs Psa 43:12 Ты отдал нас, как овец, на съедение и рассеял нас между народами;
\vs Psa 43:13 без выгоды Ты продал народ Твой и не возвысил цены его;
\vs Psa 43:14 отдал нас на поношение соседям нашим, на посмеяние и поругание живущим вокруг нас;
\vs Psa 43:15 Ты сделал нас притчею между народами, покиванием головы между иноплеменниками.
\vs Psa 43:16 Всякий день посрамление мое предо мною, и стыд покрывает лице мое
\vs Psa 43:17 от голоса поносителя и клеветника, от взоров врага и мстителя:
\vs Psa 43:18 все это пришло на нас, но мы не забыли Тебя и не нарушили завета Твоего.
\vs Psa 43:19 Не отступило назад сердце наше, и стопы наши не уклонились от пути Твоего,
\vs Psa 43:20 когда Ты сокрушил нас в земле драконов и покрыл нас тенью смертною.
\vs Psa 43:21 Если бы мы забыли имя Бога нашего и простерли руки наши к богу чужому,
\vs Psa 43:22 то не взыскал ли бы сего Бог? Ибо Он знает тайны сердца.
\vs Psa 43:23 Но за Тебя умерщвляют нас всякий день, считают нас за овец, \bibemph{обреченных} на заклание.
\vs Psa 43:24 Восстань, что спишь, Господи! пробудись, не отринь навсегда.
\vs Psa 43:25 Для чего скрываешь лице Твое, забываешь скорбь нашу и угнетение наше?
\vs Psa 43:26 ибо душа наша унижена до праха, утроба наша прильнула к земле.
\vs Psa 43:27 Восстань на помощь нам и избавь нас ради милости Твоей.
\vs Psa 44:1 Начальнику хора. На \bibemph{музыкальном орудии} Шошан. Учение. Сынов Кореевых. Песнь любви.
\rsbpar\vs Psa 44:2 Излилось из сердца моего слово благое; я говорю: песнь моя о Царе; язык мой~--- трость скорописца.
\vs Psa 44:3 Ты прекраснее сынов человеческих; благодать излилась из уст Твоих; посему благословил Тебя Бог на веки.
\vs Psa 44:4 Препояшь Себя по бедру мечом Твоим, Сильный, славою Твоею и красотою Твоею,
\vs Psa 44:5 и в сем украшении Твоем поспеши, воссядь на колесницу ради истины и кротости и правды, и десница Твоя покажет Тебе дивные дела.
\vs Psa 44:6 Остры стрелы Твои, [Сильный],~--- народы падут пред Тобою,~--- они~--- в сердце врагов Царя.
\vs Psa 44:7 Престол Твой, Боже, вовек; жезл правоты~--- жезл царства Твоего.
\vs Psa 44:8 Ты возлюбил правду и возненавидел беззаконие, посему помазал Тебя, Боже, Бог Твой елеем радости более соучастников Твоих.
\vs Psa 44:9 Все одежды Твои, как смирна и алой и касия; из чертогов слоновой кости увеселяют Тебя.
\vs Psa 44:10 Дочери царей между почетными у Тебя; стала царица одесную Тебя в Офирском золоте.
\vs Psa 44:11 Слыши, дщерь, и смотри, и приклони ухо твое, и забудь народ твой и дом отца твоего.
\vs Psa 44:12 И возжелает Царь красоты твоей; ибо Он Господь твой, и ты поклонись Ему.
\vs Psa 44:13 И дочь Тира с дарами, и богатейшие из народа будут умолять лице Твое.
\vs Psa 44:14 Вся слава дщери Царя внутри; одежда ее шита золотом;
\vs Psa 44:15 в испещренной одежде ведется она к Царю; за нею ведутся к Тебе девы, подруги ее,
\vs Psa 44:16 приводятся с весельем и ликованьем, входят в чертог Царя.
\vs Psa 44:17 Вместо отцов Твоих, будут сыновья Твои; Ты поставишь их князьями по всей земле.
\vs Psa 44:18 Сделаю имя Твое памятным в род и род; посему народы будут славить Тебя во веки и веки.
\vs Psa 45:1 Начальнику хора. Сынов Кореевых. На \bibemph{музыкальном орудии} Аламоф. Песнь.
\rsbpar\vs Psa 45:2 Бог нам прибежище и сила, скорый помощник в бедах,
\vs Psa 45:3 посему не убоимся, хотя бы поколебалась земля, и горы двинулись в сердце морей.
\vs Psa 45:4 Пусть шумят, вздымаются воды их, трясутся горы от волнения их.
\vs Psa 45:5 Речные потоки веселят град Божий, святое жилище Всевышнего.
\vs Psa 45:6 Бог посреди его; он не поколеблется: Бог поможет ему с раннего утра.
\vs Psa 45:7 Восшумели народы; двинулись царства: [Всевышний] дал глас Свой, и растаяла земля.
\vs Psa 45:8 Господь сил с нами, Бог Иакова заступник наш.
\vs Psa 45:9 Придите и видите дела Господа,~--- какие произвел Он опустошения на земле:
\vs Psa 45:10 прекращая брани до края земли, сокрушил лук и переломил копье, колесницы сжег огнем.
\vs Psa 45:11 Остановитесь и познайте, что Я~--- Бог: буду превознесен в народах, превознесен на земле.
\vs Psa 45:12 Господь сил с нами, заступник наш Бог Иакова.
\vs Psa 46:1 Начальнику хора. Сынов Кореевых. Псалом.
\rsbpar\vs Psa 46:2 Восплещите руками все народы, воскликните Богу гласом радости;
\vs Psa 46:3 ибо Господь Всевышний страшен,~--- великий Царь над всею землею;
\vs Psa 46:4 покорил нам народы и племена под ноги наши;
\vs Psa 46:5 избрал нам наследие наше, красу Иакова, которого возлюбил.
\vs Psa 46:6 Восшел Бог при восклицаниях, Господь при звуке трубном.
\vs Psa 46:7 Пойте Богу нашему, пойте; пойте Царю нашему, пойте,
\vs Psa 46:8 ибо Бог~--- Царь всей земли; пойте все разумно.
\vs Psa 46:9 Бог воцарился над народами, Бог воссел на святом престоле Своем;
\vs Psa 46:10 князья народов собрались к народу Бога Авраамова, ибо щиты земли~--- Божии; Он превознесен \bibemph{над ними}.
\vs Psa 47:1 Песнь. Псалом. Сынов Кореевых.
\rsbpar\vs Psa 47:2 Велик Господь и всехвален во граде Бога нашего, на святой горе Его.
\vs Psa 47:3 Прекрасная возвышенность, радость всей земли гора Сион; на северной стороне \bibemph{ее} город великого Царя.
\vs Psa 47:4 Бог в жилищах его ведом, как заступник:
\vs Psa 47:5 ибо вот, сошлись цари и прошли все мимо;
\vs Psa 47:6 увидели и изумились, смутились и обратились в бегство;
\vs Psa 47:7 страх объял их там и мука, как у женщин в родах;
\vs Psa 47:8 восточным ветром Ты сокрушил Фарсийские корабли.
\vs Psa 47:9 Как слышали мы, так и увидели во граде Господа сил, во граде Бога нашего: Бог утвердит его на веки.
\vs Psa 47:10 Мы размышляли, Боже, о благости Твоей посреди храма Твоего.
\vs Psa 47:11 Как имя Твое, Боже, так и хвала Твоя до концов земли; десница Твоя полна правды.
\vs Psa 47:12 Да веселится гора Сион, [и] да радуются дщери Иудейские ради судов Твоих, [Господи].
\vs Psa 47:13 Пойдите вокруг Сиона и обойдите его, пересчитайте башни его;
\vs Psa 47:14 обратите сердце ваше к укреплениям его, рассмотрите домы его, чтобы пересказать грядущему роду,
\vs Psa 47:15 ибо сей Бог есть Бог наш на веки и веки: Он будет вождем нашим до самой смерти.
\vs Psa 48:1 Начальнику хора. Сынов Кореевых. Псалом.
\rsbpar\vs Psa 48:2 Слушайте сие, все народы; внимайте сему, все живущие во вселенной,~---
\vs Psa 48:3 и простые и знатные, богатый, равно как бедный.
\vs Psa 48:4 Уста мои изрекут премудрость, и размышления сердца моего~--- знание.
\vs Psa 48:5 Приклоню ухо мое к притче, на гуслях открою загадку мою:
\vs Psa 48:6 <<для чего бояться мне во дни бедствия, \bibemph{когда} беззаконие путей моих окружит меня?>>
\vs Psa 48:7 Надеющиеся на силы свои и хвалящиеся множеством богатства своего!
\vs Psa 48:8 человек никак не искупит брата своего и не даст Богу выкупа за него:
\vs Psa 48:9 дорог\acc{а} цена искупления души их, и не будет того вовек,
\vs Psa 48:10 чтобы остался \bibemph{кто} жить навсегда и не увидел могилы.
\vs Psa 48:11 Каждый видит, что и мудрые умирают, равно как и невежды и бессмысленные погибают и оставляют имущество свое другим.
\vs Psa 48:12 В мыслях у них, что домы их вечны, и что жилища их в род и род, и земли свои они называют своими именами.
\vs Psa 48:13 Но человек в чести не пребудет; он уподобится животным, которые погибают.
\vs Psa 48:14 Этот путь их есть безумие их, хотя последующие за ними одобряют мнение их.
\vs Psa 48:15 Как овец, заключат их в преисподнюю; смерть будет пасти их, и наутро праведники будут владычествовать над ними; сила их истощится; могила~--- жилище их.
\vs Psa 48:16 Но Бог избавит душу мою от власти преисподней, когда примет меня.
\vs Psa 48:17 Не бойся, когда богатеет человек, когда слава дома его умножается:
\vs Psa 48:18 ибо умирая не возьмет ничего; не пойдет за ним слава его;
\vs Psa 48:19 хотя при жизни он ублажает душу свою, и прославляют тебя, что ты удовлетворяешь себе,
\vs Psa 48:20 но он пойдет к роду отцов своих, которые никогда не увидят света.
\vs Psa 48:21 Человек, который в чести и неразумен, подобен животным, которые погибают.
\vs Psa 49:0 Псалом Асафа.
\rsbpar\vs Psa 49:1 Бог богов, Господь возглаголал и призывает землю, от восхода солнца до запада.
\vs Psa 49:2 С Сиона, который есть верх красоты, является Бог,
\vs Psa 49:3 грядет Бог наш, и не в безмолвии: пред Ним огонь поядающий, и вокруг Его сильная буря.
\vs Psa 49:4 Он призывает свыше небо и землю, судить народ Свой:
\vs Psa 49:5 <<соберите ко Мне святых Моих, вступивших в завет со Мною при жертве>>.
\vs Psa 49:6 И небеса провозгласят правду Его, ибо судия сей есть Бог.
\vs Psa 49:7 <<Слушай, народ Мой, Я буду говорить; Израиль! Я буду свидетельствовать против тебя: Я Бог, твой Бог.
\vs Psa 49:8 Не за жертвы твои Я буду укорять тебя; всесожжения твои всегда предо Мною;
\vs Psa 49:9 не приму тельца из дома твоего, ни козлов из дворов твоих,
\vs Psa 49:10 ибо Мои все звери в лесу, и скот на тысяче гор,
\vs Psa 49:11 знаю всех птиц на горах, и животные на полях предо Мною.
\vs Psa 49:12 Если бы Я взалкал, то не сказал бы тебе, ибо Моя вселенная и все, что наполняет ее.
\vs Psa 49:13 Ем ли Я мясо волов и пью ли кровь козлов?
\vs Psa 49:14 Принеси в жертву Богу хвалу и воздай Всевышнему обеты твои,
\vs Psa 49:15 и призови Меня в день скорби; Я избавлю тебя, и ты прославишь Меня>>.
\vs Psa 49:16 Грешнику же говорит Бог: <<что ты проповедуешь уставы Мои и берешь завет Мой в уста твои,
\vs Psa 49:17 а сам ненавидишь наставление Мое и слова Мои бросаешь за себя?
\vs Psa 49:18 когда видишь вора, сходишься с ним, и с прелюбодеями сообщаешься;
\vs Psa 49:19 уста твои открываешь на злословие, и язык твой сплетает коварство;
\vs Psa 49:20 сидишь и говоришь на брата твоего, на сына матери твоей клевещешь;
\vs Psa 49:21 ты это делал, и Я молчал; ты подумал, что Я такой же, как ты. Изобличу тебя и представлю пред глаза твои [грехи твои].
\vs Psa 49:22 Уразумейте это, забывающие Бога, дабы Я не восхитил,~--- и не будет избавляющего.
\vs Psa 49:23 Кто приносит в жертву хвалу, тот чтит Меня, и кто наблюдает за путем своим, тому явлю Я спасение Божие>>.
\vs Psa 50:1 Начальнику хора. Псалом Давида,
\vs Psa 50:2 когда приходил к нему пророк Нафан, после того, как Давид вошел к Вирсавии.
\rsbpar\vs Psa 50:3 Помилуй меня, Боже, по великой милости Твоей, и по множеству щедрот Твоих изгладь беззакония мои.
\vs Psa 50:4 Многократно омой меня от беззакония моего, и от греха моего очисти меня,
\vs Psa 50:5 ибо беззакония мои я сознаю, и грех мой всегда предо мною.
\vs Psa 50:6 Тебе, Тебе единому согрешил я и лукавое пред очами Твоими сделал, так что Ты праведен в приговоре Твоем и чист в суде Твоем.
\vs Psa 50:7 Вот, я в беззаконии зачат, и во грехе родила меня мать моя.
\vs Psa 50:8 Вот, Ты возлюбил истину в сердце и внутрь меня явил мне мудрость [Твою].
\vs Psa 50:9 Окропи меня иссопом, и буду чист; омой меня, и буду белее снега.
\vs Psa 50:10 Дай мне услышать радость и веселие, и возрадуются кости, Тобою сокрушенные.
\vs Psa 50:11 Отврати лице Твое от грехов моих и изгладь все беззакония мои.
\vs Psa 50:12 Сердце чистое сотвори во мне, Боже, и дух правый обнови внутри меня.
\vs Psa 50:13 Не отвергни меня от лица Твоего и Духа Твоего Святаго не отними от меня.
\vs Psa 50:14 Возврати мне радость спасения Твоего и Духом владычественным утверди меня.
\vs Psa 50:15 Научу беззаконных путям Твоим, и нечестивые к Тебе обратятся.
\vs Psa 50:16 Избавь меня от кровей, Боже, Боже спасения моего, и язык мой восхвалит правду Твою.
\vs Psa 50:17 Господи! отверзи уста мои, и уста мои возвестят хвалу Твою:
\vs Psa 50:18 ибо жертвы Ты не желаешь,~--- я дал бы ее; к всесожжению не благоволишь.
\vs Psa 50:19 Жертва Богу~--- дух сокрушенный; сердца сокрушенного и смиренного Ты не презришь, Боже.
\vs Psa 50:20 Облагодетельствуй, [Господи,] по благоволению Твоему Сион; воздвигни стены Иерусалима:
\vs Psa 50:21 тогда благоугодны будут Тебе жертвы правды, возношение и всесожжение; тогда возложат на алтарь Твой тельцов.
\vs Psa 51:1 Начальнику хора. Учение Давида,
\vs Psa 51:2 после того, как приходил Доик Идумеянин и донес Саулу и сказал ему, что Давид пришел в дом Ахимелеха.
\rsbpar\vs Psa 51:3 Что хвалишься злодейством, сильный? милость Божия всегда \bibemph{со мною};
\vs Psa 51:4 гибель вымышляет язык твой; как изощренная бритва, он \bibemph{у тебя}, коварный!
\vs Psa 51:5 ты любишь больше зло, нежели добро, больше ложь, нежели говорить правду;
\vs Psa 51:6 ты любишь всякие гибельные речи, язык коварный:
\vs Psa 51:7 за то Бог сокрушит тебя вконец, изринет тебя и исторгнет тебя из жилища [твоего] и корень твой из земли живых.
\vs Psa 51:8 Увидят праведники и убоятся, посмеются над ним [и скажут]:
\vs Psa 51:9 <<вот человек, который не в Боге полагал крепость свою, а надеялся на множество богатства своего, укреплялся в злодействе своем>>.
\vs Psa 51:10 А я, как зеленеющая маслина, в доме Божием, и уповаю на милость Божию во веки веков,
\vs Psa 51:11 вечно буду славить Тебя за то, что Ты соделал, и уповать на имя Твое, ибо оно благо пред святыми Твоими.
\vs Psa 52:1 Начальнику хора. На духовом \bibemph{орудии}. Учение Давида.
\rsbpar\vs Psa 52:2 Сказал безумец в сердце своем: <<нет Бога>>. Развратились они и совершили гнусные преступления; нет делающего добро.
\vs Psa 52:3 Бог с небес призрел на сынов человеческих, чтобы видеть, есть ли разумеющий, ищущий Бога.
\vs Psa 52:4 Все уклонились, сделались равно непотребными; нет делающего добро, нет ни одного.
\vs Psa 52:5 Неужели не вразумятся делающие беззаконие, съедающие народ мой, \bibemph{как} едят хлеб, и не призывающие Бога?
\vs Psa 52:6 Там убоятся они страха, где нет страха, ибо рассыплет Бог кости ополчающихся против тебя. Ты постыдишь их, потому что Бог отверг их.
\vs Psa 52:7 Кто даст с Сиона спасение Израилю! Когда Бог возвратит пленение народа Своего, тогда возрадуется Иаков и возвеселится Израиль.
\vs Psa 53:1 Начальнику хора. На струнных \bibemph{орудиях}. Учение Давида,
\vs Psa 53:2 когда пришли Зифеи и сказали Саулу: <<не у нас ли скрывается Давид?>>
\rsbpar\vs Psa 53:3 Боже! именем Твоим спаси меня, и силою Твоею суди меня.
\vs Psa 53:4 Боже! услышь молитву мою, внемли словам уст моих,
\vs Psa 53:5 ибо чужие восстали на меня, и сильные ищут души моей; они не имеют Бога пред собою.
\vs Psa 53:6 Вот, Бог помощник мой; Господь подкрепляет душу мою.
\vs Psa 53:7 Он воздаст за зло врагам моим; истиною Твоею истреби их.
\vs Psa 53:8 Я усердно принесу Тебе жертву, прославлю имя Твое, Господи, ибо оно благо,
\vs Psa 53:9 ибо Ты избавил меня от всех бед, и на врагов моих смотрело око мое.
\vs Psa 54:1 Начальнику хора. На струнных \bibemph{орудиях}. Учение Давида.
\rsbpar\vs Psa 54:2 Услышь, Боже, молитву мою и не скрывайся от моления моего;
\vs Psa 54:3 внемли мне и услышь меня; я стенаю в горести моей, и смущаюсь
\vs Psa 54:4 от голоса врага, от притеснения нечестивого, ибо они возводят на меня беззаконие и в гневе враждуют против меня.
\vs Psa 54:5 Сердце мое трепещет во мне, и смертные ужасы напали на меня;
\vs Psa 54:6 страх и трепет нашел на меня, и ужас объял меня.
\vs Psa 54:7 И я сказал: <<кто дал бы мне крылья, как у голубя? я улетел бы и успокоился бы;
\vs Psa 54:8 далеко удалился бы я, и оставался бы в пустыне;
\vs Psa 54:9 поспешил бы укрыться от вихря, от бури>>.
\vs Psa 54:10 Расстрой, Господи, и раздели языки их, ибо я вижу насилие и распри в городе;
\vs Psa 54:11 днем и ночью ходят они кругом по стенам его; злодеяния и бедствие посреди его;
\vs Psa 54:12 посреди его пагуба; обман и коварство не сходят с улиц его:
\vs Psa 54:13 ибо не враг поносит меня,~--- это я перенес бы; не ненавистник мой величается надо мною,~--- от него я укрылся бы;
\vs Psa 54:14 но ты, который был для меня то же, что я, друг мой и близкий мой,
\vs Psa 54:15 с которым мы разделяли искренние беседы и ходили вместе в дом Божий.
\vs Psa 54:16 Да найдет на них смерть; да сойдут они живыми в ад, ибо злодейство в жилищах их, посреди их.
\vs Psa 54:17 Я же воззову к Богу, и Господь спасет меня.
\vs Psa 54:18 Вечером и утром и в полдень буду умолять и вопиять, и Он услышит голос мой,
\vs Psa 54:19 избавит в мире душу мою от восстающих на меня, ибо их много у меня;
\vs Psa 54:20 услышит Бог, и смирит их от века Живущий, потому что нет в них перемены; они не боятся Бога,
\vs Psa 54:21 простерли руки свои на тех, которые с ними в мире, нарушили союз свой;
\vs Psa 54:22 уста их мягче масла, а в сердце их вражда; слова их нежнее елея, но они суть обнаженные мечи.
\vs Psa 54:23 Возложи на Господа заботы твои, и Он поддержит тебя. Никогда не даст Он поколебаться праведнику.
\vs Psa 54:24 Ты, Боже, низведешь их в ров погибели; кровожадные и коварные не доживут и до половины дней своих. А я на Тебя, [Господи,] уповаю.
\vs Psa 55:1 Начальнику хора. О голубице, безмолвствующей в удалении. Писание Давида, когда Филистимляне захватили его в Гефе.
\rsbpar\vs Psa 55:2 Помилуй меня, Боже! ибо человек хочет поглотить меня; нападая всякий день, теснит меня.
\vs Psa 55:3 Враги мои всякий день ищут поглотить меня, ибо много восстающих на меня, о, Всевышний!
\vs Psa 55:4 Когда я в страхе, на Тебя я уповаю.
\vs Psa 55:5 В Боге восхвалю я слово Его; на Бога уповаю, не боюсь; что сделает мне плоть?
\vs Psa 55:6 Всякий день извращают слова мои; все помышления их обо мне~--- на зло:
\vs Psa 55:7 собираются, притаиваются, наблюдают за моими пятами, чтобы уловить душу мою.
\vs Psa 55:8 Неужели они избегнут воздаяния за неправду \bibemph{свою}? Во гневе низложи, Боже, народы.
\vs Psa 55:9 У Тебя исчислены мои скитания; положи слезы мои в сосуд у Тебя,~--- не в книге ли они Твоей?
\vs Psa 55:10 Враги мои обращаются назад, когда я взываю к Тебе, из этого я узна\acc{ю}, что Бог за меня.
\vs Psa 55:11 В Боге восхвалю я слово \bibemph{Его}, в Господе восхвалю слово \bibemph{Его}.
\vs Psa 55:12 На Бога уповаю, не боюсь; что сделает мне человек?
\vs Psa 55:13 На мне, Боже, обеты Тебе; Тебе воздам хвалы,
\vs Psa 55:14 ибо Ты избавил душу мою от смерти, [очи мои от слез,] да и ноги мои от преткновения, чтобы я ходил пред лицем Божиим во свете живых.
\vs Psa 56:1 Начальнику хора. Не погуби. Писание Давида, когда он убежал от Саула в пещеру.
\rsbpar\vs Psa 56:2 Помилуй меня, Боже, помилуй меня, ибо на Тебя уповает душа моя, и в тени крыл Твоих я укроюсь, доколе не пройдут беды.
\vs Psa 56:3 Воззову к Богу Всевышнему, Богу, благодетельствующему мне;
\vs Psa 56:4 Он пошлет с небес и спасет меня; посрамит ищущего поглотить меня; пошлет Бог милость Свою и истину Свою.
\vs Psa 56:5 Душа моя среди львов; я лежу среди дышущих пламенем, среди сынов человеческих, у которых зубы~--- копья и стрелы, и у которых язык~--- острый меч.
\vs Psa 56:6 Будь превознесен выше небес, Боже, и над всею землею да будет слава Твоя!
\vs Psa 56:7 Приготовили сеть ногам моим; душа моя поникла; выкопали предо мною яму, и \bibemph{сами} упали в нее.
\vs Psa 56:8 Готово сердце мое, Боже, готово сердце мое: буду петь и славить.
\vs Psa 56:9 Воспрянь, слава моя, воспрянь, псалтирь и гусли! Я встану рано.
\vs Psa 56:10 Буду славить Тебя, Господи, между народами; буду воспевать Тебя среди племен,
\vs Psa 56:11 ибо до небес велика милость Твоя и до облаков истина Твоя.
\vs Psa 56:12 Будь превознесен выше небес, Боже, и над всею землею да будет слава Твоя!
\vs Psa 57:1 Начальнику хора. Не погуби. Писание Давида.
\rsbpar\vs Psa 57:2 Подлинно ли правду говорите вы, судьи, и справедливо судите, сыны человеческие?
\vs Psa 57:3 Беззаконие составляете в сердце, кладете на весы злодеяния рук ваших на земле.
\vs Psa 57:4 С самого рождения отступили нечестивые, от утробы \bibemph{матери} заблуждаются, говоря ложь.
\vs Psa 57:5 Яд у них~--- как яд змеи, как глухого аспида, который затыкает уши свои
\vs Psa 57:6 и не слышит голоса заклинателя, самого искусного в заклинаниях.
\vs Psa 57:7 Боже! сокруши зубы их в устах их; разбей, Господи, челюсти львов!
\vs Psa 57:8 Да исчезнут, как вода протекающая; когда напрягут стрелы, пусть они будут как переломленные.
\vs Psa 57:9 Да исчезнут, как распускающаяся улитка; да не видят солнца, как выкидыш женщины.
\vs Psa 57:10 Прежде нежели котлы ваши ощутят горящий терн, и свежее и обгоревшее да разнесет вихрь.
\vs Psa 57:11 Возрадуется праведник, когда увидит отмщение; омоет стопы свои в крови нечестивого.
\vs Psa 57:12 И скажет человек: <<подлинно есть плод праведнику! итак есть Бог, судящий на земле!>>
\vs Psa 58:1 Начальнику хора. Не погуби. Писание Давида, когда Саул послал стеречь дом его, чтобы умертвить его.
\rsbpar\vs Psa 58:2 Избавь меня от врагов моих, Боже мой! защити меня от восстающих на меня;
\vs Psa 58:3 избавь меня от делающих беззаконие; спаси от кровожадных,
\vs Psa 58:4 ибо вот, они подстерегают душу мою; собираются на меня сильные не за преступление мое и не за грех мой, Господи;
\vs Psa 58:5 без вины \bibemph{моей} сбегаются и вооружаются; подвигнись на помощь мне и воззри.
\vs Psa 58:6 Ты, Господи, Боже сил, Боже Израилев, восстань посетить все народы, не пощади ни одного из нечестивых беззаконников:
\vs Psa 58:7 вечером возвращаются они, воют, как псы, и ходят вокруг города;
\vs Psa 58:8 вот они изрыгают хулу языком своим; в устах их мечи: <<ибо>>, \bibemph{думают они}, <<кто слышит?>>
\vs Psa 58:9 Но Ты, Господи, посмеешься над ними; Ты посрамишь все народы.
\vs Psa 58:10 Сила~--- у них, но я к Тебе прибегаю, ибо Бог~--- заступник мой.
\vs Psa 58:11 Бог мой, милующий меня, предварит меня; Бог даст мне смотреть на врагов моих.
\vs Psa 58:12 Не умерщвляй их, чтобы не забыл народ мой; расточи их силою Твоею и низложи их, Господи, защитник наш.
\vs Psa 58:13 Слово языка их есть грех уст их, да уловятся они в гордости своей за клятву и ложь, которую произносят.
\vs Psa 58:14 Расточи их во гневе, расточи, чтобы их не было; и да познают, что Бог владычествует над Иаковом до пределов земли.
\vs Psa 58:15 Пусть возвращаются вечером, воют, как псы, и ходят вокруг города;
\vs Psa 58:16 пусть бродят, чтобы найти пищу, и несытые проводят ночи.
\vs Psa 58:17 А я буду воспевать силу Твою и с раннего утра провозглашать милость Твою, ибо Ты был мне защитою и убежищем в день бедствия моего.
\vs Psa 58:18 Сила моя! Тебя буду воспевать я, ибо Бог~--- заступник мой, Бог мой, милующий меня.
\vs Psa 59:1 Начальнику хора. На \bibemph{музыкальном орудии} Шушан-Эдуф. Писание Давида для изучения,
\vs Psa 59:2 когда он воевал с Сириею Месопотамскою и с Сириею Цованскою, и когда Иоав, возвращаясь, поразил двенадцать тысяч Идумеев в долине Соляной.
\rsbpar\vs Psa 59:3 Боже! Ты отринул нас, Ты сокрушил нас, Ты прогневался: обратись к нам.
\vs Psa 59:4 Ты потряс землю, разбил ее: исцели повреждения ее, ибо она колеблется.
\vs Psa 59:5 Ты дал испытать народу твоему жестокое, напоил нас вином изумления.
\vs Psa 59:6 Даруй боящимся Тебя знамя, чтобы они подняли его ради истины,
\vs Psa 59:7 чтобы избавились возлюбленные Твои; спаси десницею Твоею и услышь меня.
\vs Psa 59:8 Бог сказал во святилище Своем: <<восторжествую, разделю Сихем и долину Сокхоф размерю:
\vs Psa 59:9 Мой Галаад, Мой Манассия, Ефрем крепость главы Моей, Иуда скипетр Мой,
\vs Psa 59:10 Моав умывальная чаша Моя; на Едома простру сапог Мой. Восклицай Мне, земля Филистимская!>>
\vs Psa 59:11 Кто введет меня в укрепленный город? Кто доведет меня до Едома?
\vs Psa 59:12 Не Ты ли, Боже, \bibemph{Который} отринул нас, и не выходишь, Боже, с войсками нашими?
\vs Psa 59:13 Подай нам помощь в тесноте, ибо защита человеческая суетна.
\vs Psa 59:14 С Богом мы окажем силу, Он низложит врагов наших.
\vs Psa 60:1 Начальнику хора. На струнном \bibemph{орудии}. Псалом Давида.
\rsbpar\vs Psa 60:2 Услышь, Боже, вопль мой, внемли молитве моей!
\vs Psa 60:3 От конца земли взываю к Тебе в унынии сердца моего; возведи меня на скалу, для меня недосягаемую,
\vs Psa 60:4 ибо Ты прибежище мое, Ты крепкая защита от врага.
\vs Psa 60:5 Да живу я вечно в жилище Твоем и покоюсь под кровом крыл Твоих,
\vs Psa 60:6 ибо Ты, Боже, услышал обеты мои и дал \bibemph{мне} наследие боящихся имени Твоего.
\vs Psa 60:7 Приложи дни ко дням царя, лета его \bibemph{продли} в род и род,
\vs Psa 60:8 да пребудет он вечно пред Богом; заповедуй милости и истине охранять его.
\vs Psa 60:9 И я буду петь имени Твоему вовек, исполняя обеты мои всякий день.
\vs Psa 61:1 Начальнику хора Идифумова. Псалом Давида.
\rsbpar\vs Psa 61:2 Только в Боге успокаивается душа моя: от Него спасение мое.
\vs Psa 61:3 Только Он~--- твердыня моя, спасение мое, убежище мое: не поколеблюсь более.
\vs Psa 61:4 Доколе вы будете налегать на человека? Вы будете низринуты, все вы, как наклонившаяся стена, как ограда пошатнувшаяся.
\vs Psa 61:5 Они задумали свергнуть его с высоты, прибегли ко лжи; устами благословляют, а в сердце своем клянут.
\vs Psa 61:6 Только в Боге успокаивайся, душа моя! ибо на Него надежда моя.
\vs Psa 61:7 Только Он~--- твердыня моя и спасение мое, убежище мое: не поколеблюсь.
\vs Psa 61:8 В Боге спасение мое и слава моя; крепость силы моей и упование мое в Боге.
\vs Psa 61:9 Народ! надейтесь на Него во всякое время; изливайте пред Ним сердце ваше: Бог нам прибежище.
\vs Psa 61:10 Сыны человеческие~--- только суета; сыны мужей~--- ложь; если положить их на весы, все они вместе легче пустоты.
\vs Psa 61:11 Не надейтесь на грабительство и не тщеславьтесь хищением; когда богатство умножается, не прилагайте \bibemph{к нему} сердца.
\vs Psa 61:12 Однажды сказал Бог, и дважды слышал я это, что сила у Бога,
\vs Psa 61:13 и у Тебя, Господи, милость, ибо Ты воздаешь каждому по делам его.
\vs Psa 62:1 Псалом Давида, когда он был в пустыне Иудейской.
\rsbpar\vs Psa 62:2 Боже! Ты Бог мой, Тебя от ранней зари ищу я; Тебя жаждет душа моя, по Тебе томится плоть моя в земле пустой, иссохшей и безводной,
\vs Psa 62:3 чтобы видеть силу Твою и славу Твою, как я видел Тебя во святилище:
\vs Psa 62:4 ибо милость Твоя лучше, нежели жизнь. Уста мои восхвалят Тебя.
\vs Psa 62:5 Так благословлю Тебя в жизни моей; во имя Твое вознесу руки мои.
\vs Psa 62:6 Как туком и елеем насыщается душа моя, и радостным гласом восхваляют Тебя уста мои,
\vs Psa 62:7 когда я вспоминаю о Тебе на постели моей, размышляю о Тебе в \bibemph{ночные} стражи,
\vs Psa 62:8 ибо Ты помощь моя, и в тени крыл Твоих я возрадуюсь;
\vs Psa 62:9 к Тебе прилепилась душа моя; десница Твоя поддерживает меня.
\vs Psa 62:10 А те, которые ищут погибели душе моей, сойдут в преисподнюю земли;
\vs Psa 62:11 сразят их силою меча; достанутся они в добычу лисицам.
\vs Psa 62:12 Царь же возвеселится о Боге, восхвален будет всякий, клянущийся Им, ибо заградятся уста говорящих неправду.
\vs Psa 63:1 Начальнику хора. Псалом Давида.
\rsbpar\vs Psa 63:2 Услышь, Боже, голос мой в молитве моей, сохрани жизнь мою от страха врага;
\vs Psa 63:3 укрой меня от замысла коварных, от мятежа злодеев,
\vs Psa 63:4 которые изострили язык свой, как меч; напрягли лук свой~--- язвительное слово,
\vs Psa 63:5 чтобы втайне стрелять в непорочного; они внезапно стреляют в него и не боятся.
\vs Psa 63:6 Они утвердились в злом намерении, совещались скрыть сеть, говорили: кто их увидит?
\vs Psa 63:7 Изыскивают неправду, делают расследование за расследованием даже до внутренней жизни человека и до глубины сердца.
\vs Psa 63:8 Но поразит их Бог стрелою: внезапно будут они уязвлены;
\vs Psa 63:9 языком своим они поразят самих себя; все, видящие их, удалятся \bibemph{от них}.
\vs Psa 63:10 И убоятся все человеки, и возвестят дело Божие, и уразумеют, что это Его дело.
\vs Psa 63:11 А праведник возвеселится о Господе и будет уповать на Него; и похвалятся все правые сердцем.
\vs Psa 64:1 Начальнику хора. Псалом Давида для пения.
\rsbpar\vs Psa 64:2 Тебе, Боже, принадлежит хвала на Сионе, и Тебе воздастся обет [в Иерусалиме].
\vs Psa 64:3 Ты слышишь молитву; к Тебе прибегает всякая плоть.
\vs Psa 64:4 Дела беззаконий превозмогают меня; Ты очистишь преступления наши.
\vs Psa 64:5 Блажен, кого Ты избрал и приблизил, чтобы он жил во дворах Твоих. Насытимся благами дома Твоего, святаго храма Твоего.
\vs Psa 64:6 Страшный в правосудии, услышь нас, Боже, Спаситель наш, упование всех концов земли и находящихся в море далеко,
\vs Psa 64:7 поставивший горы силою Своею, препоясанный могуществом,
\vs Psa 64:8 укрощающий шум морей, шум волн их и мятеж народов!
\vs Psa 64:9 И убоятся знамений Твоих живущие на пределах \bibemph{земли}. Утро и вечер возбудишь к славе \bibemph{Твоей}.
\vs Psa 64:10 Ты посещаешь землю и утоляешь жажду ее, обильно обогащаешь ее: поток Божий полон воды; Ты приготовляешь хлеб, ибо так устроил ее;
\vs Psa 64:11 напояешь борозды ее, уравниваешь глыбы ее, размягчаешь ее каплями дождя, благословляешь произрастания ее;
\vs Psa 64:12 венчаешь лето благости Твоей, и стези Твои источают тук,
\vs Psa 64:13 источают на пустынные пажити, и холмы препоясываются радостью;
\vs Psa 64:14 луга одеваются стадами, и долины покрываются хлебом, восклицают и поют.
\vs Psa 65:0 Начальнику хора. Песнь.
\rsbpar\vs Psa 65:1 Воскликните Богу, вся земля.
\vs Psa 65:2 Пойте славу имени Его, воздайте славу, хвалу Ему.
\vs Psa 65:3 Скажите Богу: как страшен Ты в делах Твоих! По множеству силы Твоей, покорятся Тебе враги Твои.
\vs Psa 65:4 Вся земля да поклонится Тебе и поет Тебе, да поет имени Твоему, [Вышний]!
\vs Psa 65:5 Придите и воззрите на дела Бога, страшного в делах над сынами человеческими.
\vs Psa 65:6 Он превратил море в сушу; через реку перешли стопами, там веселились мы о Нем.
\vs Psa 65:7 Могуществом Своим владычествует Он вечно; очи Его зрят на народы, да не возносятся мятежники.
\vs Psa 65:8 Благословите, народы, Бога нашего и провозгласите хвалу Ему.
\vs Psa 65:9 Он сохранил душе нашей жизнь и ноге нашей не дал поколебаться.
\vs Psa 65:10 Ты испытал нас, Боже, переплавил нас, как переплавляют серебро.
\vs Psa 65:11 Ты ввел нас в сеть, положил оковы на чресла наши,
\vs Psa 65:12 посадил человека на главу нашу. Мы вошли в огонь и в воду, и Ты вывел нас на свободу.
\vs Psa 65:13 Войду в дом Твой со всесожжениями, воздам Тебе обеты мои,
\vs Psa 65:14 которые произнесли уста мои и изрек язык мой в скорби моей.
\vs Psa 65:15 Всесожжения тучные вознесу Тебе с воскурением тука овнов, принесу в жертву волов и козлов.
\vs Psa 65:16 Придите, послушайте, все боящиеся Бога, и я возвещу \bibemph{вам}, что сотворил Он для души моей.
\vs Psa 65:17 Я воззвал к Нему устами моими и превознес Его языком моим.
\vs Psa 65:18 Если бы я видел беззаконие в сердце моем, то не услышал бы меня Господь.
\vs Psa 65:19 Но Бог услышал, внял гласу моления моего.
\vs Psa 65:20 Благословен Бог, Который не отверг молитвы моей и не отвратил от меня милости Своей.
\vs Psa 66:1 Начальнику хора. На струнных \bibemph{орудиях}. Псалом. Песнь.
\rsbpar\vs Psa 66:2 Боже! будь милостив к нам и благослови нас, освети нас лицем Твоим,
\vs Psa 66:3 дабы познали на земле путь Твой, во всех народах спасение Твое.
\vs Psa 66:4 Да восхвалят Тебя народы, Боже; да восхвалят Тебя народы все.
\vs Psa 66:5 Да веселятся и радуются племена, ибо Ты судишь народы праведно и управляешь на земле племенами.
\vs Psa 66:6 Да восхвалят Тебя народы, Боже, да восхвалят Тебя народы все.
\vs Psa 66:7 Земля дала плод свой; да благословит нас Бог, Бог наш.
\vs Psa 66:8 Да благословит нас Бог, и да убоятся Его все пределы земли.
\vs Psa 67:1 Начальнику хора. Псалом Давида. Песнь.
\rsbpar\vs Psa 67:2 Да восстанет Бог\fns{В славянском переводе: Да воскреснет Бог\dots}, и расточатся враги Его, и да бегут от лица Его ненавидящие Его.
\vs Psa 67:3 Как рассеивается дым, Ты рассей их; как тает воск от огня, так нечестивые да погибнут от лица Божия.
\vs Psa 67:4 А праведники да возвеселятся, да возрадуются пред Богом и восторжествуют в радости.
\vs Psa 67:5 Пойте Богу нашему, пойте имени Его, превозносите Шествующего на небесах; имя Ему: Господь, и радуйтесь пред лицем Его.
\vs Psa 67:6 Отец сирот и судья вдов Бог во святом Своем жилище.
\vs Psa 67:7 Бог одиноких вводит в дом, освобождает узников от оков, а непокорные остаются в знойной пустыне.
\vs Psa 67:8 Боже! когда Ты выходил пред народом Твоим, когда Ты шествовал пустынею,
\vs Psa 67:9 земля тряслась, даже небеса таяли от лица Божия, и этот Синай~--- от лица Бога, Бога Израилева.
\vs Psa 67:10 Обильный дождь проливал Ты, Боже, на наследие Твое, и когда оно изнемогало от труда, Ты подкреплял его.
\vs Psa 67:11 Народ Твой обитал там; по благости Твоей, Боже, Ты готовил \bibemph{необходимое} для бедного.
\vs Psa 67:12 Господь даст слово: провозвестниц великое множество.
\vs Psa 67:13 Цари воинств бегут, бегут, а сидящая дома делит добычу.
\vs Psa 67:14 Расположившись в уделах [своих], вы стали, как голубица, которой крылья покрыты серебром, а перья чистым золотом:
\vs Psa 67:15 когда Всемогущий рассеял царей на сей \bibemph{земле}, она забелела, как снег на Селмоне.
\vs Psa 67:16 Гора Божия~--- гора Васанская! гора высокая~--- гора Васанская!
\vs Psa 67:17 что вы завистливо смотрите, горы высокие, на гору, на которой Бог благоволит обитать и будет Господь обитать вечно?
\vs Psa 67:18 Колесниц Божиих тьмы, тысячи тысяч; среди их Господь на Синае, во святилище.
\vs Psa 67:19 Ты восшел на высоту, пленил плен, принял дары для человеков, так чтоб и из противящихся могли обитать у Господа Бога.
\vs Psa 67:20 Благословен Господь всякий день. Бог возлагает на нас бремя, но Он же и спасает нас.
\vs Psa 67:21 Бог для нас~--- Бог во спасение; во власти Господа Вседержителя врата смерти.
\vs Psa 67:22 Но Бог сокрушит голову врагов Своих, волосатое темя закоснелого в своих беззакониях.
\vs Psa 67:23 Господь сказал: <<от Васана возвращу, выведу из глубины морской,
\vs Psa 67:24 чтобы ты погрузил ногу твою, как и псы твои язык свой, в крови врагов>>.
\vs Psa 67:25 Видели шествие Твое, Боже, шествие Бога моего, Царя моего во святыне:
\vs Psa 67:26 впереди шли поющие, позади играющие на орудиях, в средине девы с тимпанами:
\vs Psa 67:27 <<в собраниях благословите \bibemph{Бога Господа}, вы~--- от семени Израилева!>>
\vs Psa 67:28 Там Вениамин младший~--- князь их; князья Иудины~--- владыки их, князья Завулоновы, князья Неффалимовы.
\vs Psa 67:29 Бог твой предназначил тебе силу. Утверди, Боже, то, что Ты соделал для нас!
\vs Psa 67:30 Ради храма Твоего в Иерусалиме цари принесут Тебе дары.
\vs Psa 67:31 Укроти зверя в тростнике, стадо волов среди тельцов народов, хвалящихся слитками серебра; рассыпь народы, желающие браней.
\vs Psa 67:32 Придут вельможи из Египта; Ефиопия прострет руки свои к Богу.
\vs Psa 67:33 Царства земные! пойте Богу, воспевайте Господа,
\vs Psa 67:34 шествующего на небесах небес от века. Вот, Он дает гласу Своему глас силы.
\vs Psa 67:35 Воздайте славу Богу! величие Его~--- над Израилем, и могущество Его~--- на облаках.
\vs Psa 67:36 Страшен Ты, Боже, во святилище Твоем. Бог Израилев~--- Он дает силу и крепость народу [Своему]. Благословен Бог!
\vs Psa 68:1 Начальнику хора. На Шошанниме. Псалом Давида.
\rsbpar\vs Psa 68:2 Спаси меня, Боже, ибо воды дошли до души [моей].
\vs Psa 68:3 Я погряз в глубоком болоте, и не на чем стать; вошел во глубину вод, и быстрое течение их увлекает меня.
\vs Psa 68:4 Я изнемог от вопля, засохла гортань моя, истомились глаза мои от ожидания Бога [моего].
\vs Psa 68:5 Ненавидящих меня без вины больше, нежели волос на голове моей; враги мои, преследующие меня несправедливо, усилились; чего я не отнимал, то должен отдать.
\vs Psa 68:6 Боже! Ты знаешь безумие мое, и грехи мои не сокрыты от Тебя.
\vs Psa 68:7 Да не постыдятся во мне все, надеющиеся на Тебя, Господи, Боже сил. Да не посрамятся во мне ищущие Тебя, Боже Израилев,
\vs Psa 68:8 ибо ради Тебя несу я поношение, и бесчестием покрывают лице мое.
\vs Psa 68:9 Чужим стал я для братьев моих и посторонним для сынов матери моей,
\vs Psa 68:10 ибо ревность по доме Твоем снедает меня, и злословия злословящих Тебя падают на меня;
\vs Psa 68:11 и пл\acc{а}чу, постясь душею моею, и это ставят в поношение мне;
\vs Psa 68:12 и возлагаю на себя вместо одежды вретище,~--- и делаюсь для них притчею;
\vs Psa 68:13 о мне толкуют сидящие у ворот, и поют в песнях пьющие вино.
\vs Psa 68:14 А я с молитвою моею к Тебе, Господи; во время благоугодное, Боже, по великой благости Твоей услышь меня в истине спасения Твоего;
\vs Psa 68:15 извлеки меня из тины, чтобы не погрязнуть мне; да избавлюсь от ненавидящих меня и от глубоких вод;
\vs Psa 68:16 да не увлечет меня стремление вод, да не поглотит меня пучина, да не затворит надо мною пропасть зева своего.
\vs Psa 68:17 Услышь меня, Господи, ибо блага милость Твоя; по множеству щедрот Твоих призри на меня;
\vs Psa 68:18 не скрывай лица Твоего от раба Твоего, ибо я скорблю; скоро услышь меня;
\vs Psa 68:19 приблизься к душе моей, избавь ее; ради врагов моих спаси меня.
\vs Psa 68:20 Ты знаешь поношение мое, стыд мой и посрамление мое: враги мои все пред Тобою.
\vs Psa 68:21 Поношение сокрушило сердце мое, и я изнемог, ждал сострадания, но нет его,~--- утешителей, но не нахожу.
\vs Psa 68:22 И дали мне в пищу желчь, и в жажде моей напоили меня уксусом.
\vs Psa 68:23 Да будет трапеза их сетью им, и мирное пиршество их~--- западнею;
\vs Psa 68:24 да помрачатся глаза их, чтоб им не видеть, и чресла их расслабь навсегда;
\vs Psa 68:25 излей на них ярость Твою, и пламень гнева Твоего да обымет их;
\vs Psa 68:26 жилище их да будет пусто, и в шатрах их да не будет живущих,
\vs Psa 68:27 ибо, кого Ты поразил, они \bibemph{еще} преследуют, и страдания уязвленных Тобою умножают.
\vs Psa 68:28 Приложи беззаконие к беззаконию их, и да не войдут они в правду Твою;
\vs Psa 68:29 да изгладятся они из книги живых и с праведниками да не напишутся.
\vs Psa 68:30 А я беден и страдаю; помощь Твоя, Боже, да восставит меня.
\vs Psa 68:31 Я буду славить имя Бога [моего] в песни, буду превозносить Его в славословии,
\vs Psa 68:32 и будет это благоугоднее Господу, нежели вол, нежели телец с рогами и с копытами.
\vs Psa 68:33 Увидят \bibemph{это} страждущие и возрадуются. И оживет сердце ваше, ищущие Бога,
\vs Psa 68:34 ибо Господь внемлет нищим и не пренебрегает узников Своих.
\vs Psa 68:35 Да восхвалят Его небеса и земля, моря и все движущееся в них;
\vs Psa 68:36 ибо спасет Бог Сион, создаст города Иудины, и поселятся там и наследуют его,
\vs Psa 68:37 и потомство рабов Его утвердится в нем, и любящие имя Его будут поселяться на нем.
\vs Psa 69:1 Начальнику хора. Псалом Давида. В воспоминание.
\rsbpar\vs Psa 69:2 Поспеши, Боже, избавить меня, \bibemph{поспеши}, Господи, на помощь мне.
\vs Psa 69:3 Да постыдятся и посрамятся ищущие души моей! Да будут обращены назад и преданы посмеянию желающие мне зла!
\vs Psa 69:4 Да будут обращены назад за поношение меня говорящие [мне]: <<хорошо! хорошо!>>
\vs Psa 69:5 Да возрадуются и возвеселятся о Тебе все, ищущие Тебя, и любящие спасение Твое да говорят непрестанно: <<велик Бог!>>
\vs Psa 69:6 Я же беден и нищ; Боже, поспеши ко мне! Ты помощь моя и Избавитель мой; Господи! не замедли.
\vs Psa 70:1 На Тебя, Господи, уповаю, да не постыжусь вовек.
\vs Psa 70:2 По правде Твоей избавь меня и освободи меня; приклони ухо Твое ко мне и спаси меня.
\vs Psa 70:3 Будь мне твердым прибежищем, куда я всегда мог бы укрываться; Ты заповедал спасти меня, ибо твердыня моя и крепость моя~--- Ты.
\vs Psa 70:4 Боже мой! избавь меня из руки нечестивого, из руки беззаконника и притеснителя,
\vs Psa 70:5 ибо Ты~--- надежда моя, Господи Боже, упование мое от юности моей.
\vs Psa 70:6 На Тебе утверждался я от утробы; Ты извел меня из чрева матери моей; Тебе хвала моя не престанет.
\vs Psa 70:7 Для многих я был как бы дивом, но Ты твердая моя надежда.
\vs Psa 70:8 Да наполнятся уста мои хвалою, [чтобы мне воспевать славу Твою,] всякий день великолепие Твое.
\vs Psa 70:9 Не отвергни меня во время старости; когда будет оскудевать сила моя, не оставь меня,
\vs Psa 70:10 ибо враги мои говорят против меня, и подстерегающие душу мою советуются между собою,
\vs Psa 70:11 говоря: <<Бог оставил его; преследуйте и схватите его, ибо нет избавляющего>>.
\vs Psa 70:12 Боже! не удаляйся от меня; Боже мой! поспеши на помощь мне.
\vs Psa 70:13 Да постыдятся и исчезнут враждующие против души моей, да покроются стыдом и бесчестием ищущие мне зла!
\vs Psa 70:14 А я всегда буду уповать [на Тебя] и умножать всякую хвалу Тебе.
\vs Psa 70:15 Уста мои будут возвещать правду Твою, всякий день благодеяния Твои; ибо я не знаю им числа.
\vs Psa 70:16 Войду в \bibemph{размышление} о силах Господа Бога; воспомяну правду Твою~--- единственно Твою.
\vs Psa 70:17 Боже! Ты наставлял меня от юности моей, и доныне я возвещаю чудеса Твои.
\vs Psa 70:18 И до старости, и до седины не оставь меня, Боже, доколе не возвещу силы Твоей роду сему и всем грядущим могущества Твоего.
\vs Psa 70:19 Правда Твоя, Боже, до превыспренних; великие дела соделал Ты; Боже, кто подобен Тебе?
\vs Psa 70:20 Ты посылал на меня многие и лютые беды, но и опять оживлял меня и из бездн земли опять выводил меня.
\vs Psa 70:21 Ты возвышал меня и утешал меня, [и из бездн земли выводил меня].
\vs Psa 70:22 И я буду славить Тебя на псалтири, Твою истину, Боже мой; буду воспевать Тебя на гуслях, Святый Израилев!
\vs Psa 70:23 Радуются уста мои, когда я пою Тебе, и душа моя, которую Ты избавил;
\vs Psa 70:24 и язык мой всякий день будет возвещать правду Твою, ибо постыжены и посрамлены ищущие мне зла.
\vs Psa 71:0 О Соломоне. [Псалом Давида.]
\rsbpar\vs Psa 71:1 Боже! даруй царю Твой суд и сыну царя Твою правду,
\vs Psa 71:2 да судит праведно людей Твоих и нищих Твоих на суде;
\vs Psa 71:3 да принесут горы мир людям и холмы правду;
\vs Psa 71:4 да судит нищих народа, да спасет сынов убогого и смирит притеснителя,~---
\vs Psa 71:5 и будут бояться Тебя, доколе пребудут солнце и луна, в роды родов.
\vs Psa 71:6 Он сойдет, как дождь на скошенный луг, как капли, орошающие землю;
\vs Psa 71:7 во дни его процветет праведник, и будет обилие мира, доколе не престанет луна;
\vs Psa 71:8 он будет обладать от моря до моря и от реки\fns{Евфрат.} до концов земли;
\vs Psa 71:9 падут пред ним жители пустынь, и враги его будут лизать прах;
\vs Psa 71:10 цари Фарсиса и островов поднесут ему дань; цари Аравии и Савы принесут дары;
\vs Psa 71:11 и поклонятся ему все цари; все народы будут служить ему;
\vs Psa 71:12 ибо он избавит нищего, вопиющего и угнетенного, у которого нет помощника.
\vs Psa 71:13 Будет милосерд к нищему и убогому, и души убогих спасет;
\vs Psa 71:14 от коварства и насилия избавит души их, и драгоценна будет кровь их пред очами его;
\vs Psa 71:15 и будет жить, и будут давать ему от золота Аравии, и будут молиться о нем непрестанно, всякий день благословлять его;
\vs Psa 71:16 будет обилие хлеба на земле, наверху гор; плоды его будут волноваться, как \bibemph{лес} на Ливане, и в городах размножатся люди, как трава на земле;
\vs Psa 71:17 будет имя его [благословенно] вовек; доколе пребывает солнце, будет передаваться имя его\fns{В славянском переводе: Прежде солнца пребывает имя его.}; и благословятся в нем [все племена земные], все народы ублажат его.
\vs Psa 71:18 Благословен Господь Бог, Бог Израилев, един творящий чудеса,
\vs Psa 71:19 и благословенно имя славы Его вовек, и наполнится славою Его вся земля. Аминь и аминь.
\vs Psa 71:20 Кончились молитвы Давида, сына Иесеева.
\vs Psa 72:0 Псалом Асафа.
\rsbpar\vs Psa 72:1 Как благ Бог к Израилю, к чистым сердцем!
\vs Psa 72:2 А я~--- едва не пошатнулись ноги мои, едва не поскользнулись стопы мои,~---
\vs Psa 72:3 я позавидовал безумным, видя благоденствие нечестивых,
\vs Psa 72:4 ибо им нет страданий до смерти их, и крепки силы их;
\vs Psa 72:5 на работе человеческой нет их, и с \bibemph{прочими} людьми не подвергаются ударам.
\vs Psa 72:6 Оттого гордость, как ожерелье, обложила их, и дерзость, \bibemph{как} наряд, одевает их;
\vs Psa 72:7 выкатились от жира глаза их, бродят помыслы в сердце;
\vs Psa 72:8 над всем издеваются, злобно разглашают клевету, говорят свысока;
\vs Psa 72:9 поднимают к небесам уста свои, и язык их расхаживает по земле.
\vs Psa 72:10 Потому туда же обращается народ Его, и пьют воду полною чашею,
\vs Psa 72:11 и говорят: <<как узнает Бог? и есть ли ведение у Вышнего?>>
\vs Psa 72:12 И вот, эти нечестивые благоденствуют в веке сем, умножают богатство.
\vs Psa 72:13 [И я сказал:] так не напрасно ли я очищал сердце мое и омывал в невинности руки мои,
\vs Psa 72:14 и подвергал себя ранам всякий день и обличениям всякое утро?
\vs Psa 72:15 \bibemph{Но} если бы я сказал: <<буду рассуждать так>>,~--- то я виновен был бы пред родом сынов Твоих.
\vs Psa 72:16 И думал я, как бы уразуметь это, но это трудно было в глазах моих,
\vs Psa 72:17 доколе не вошел я во святилище Божие и не уразумел конца их.
\vs Psa 72:18 Так! на скользких путях поставил Ты их и низвергаешь их в пропасти.
\vs Psa 72:19 Как нечаянно пришли они в разорение, исчезли, погибли от ужасов!
\vs Psa 72:20 Как сновидение по пробуждении, так Ты, Господи, пробудив \bibemph{их}, уничтожишь мечты их.
\vs Psa 72:21 Когда кипело сердце мое, и терзалась внутренность моя,
\vs Psa 72:22 тогда я был невежда и не разумел; как скот был я пред Тобою.
\vs Psa 72:23 Но я всегда с Тобою: Ты держишь меня за правую руку;
\vs Psa 72:24 Ты руководишь меня советом Твоим и потом примешь меня в славу.
\vs Psa 72:25 Кто мне на небе? и с Тобою ничего не хочу на земле.
\vs Psa 72:26 Изнемогает плоть моя и сердце мое: Бог твердыня сердца моего и часть моя вовек.
\vs Psa 72:27 Ибо вот, удаляющие себя от Тебя гибнут; Ты истребляешь всякого отступающего от Тебя.
\vs Psa 72:28 А мне благо приближаться к Богу! На Господа Бога я возложил упование мое, чтобы возвещать все дела Твои [во вратах дщери Сионовой].
\vs Psa 73:0 Учение Асафа.
\rsbpar\vs Psa 73:1 Для чего, Боже, отринул нас навсегда? возгорелся гнев Твой на овец пажити Твоей?
\vs Psa 73:2 Вспомни сонм Твой, \bibemph{который} Ты стяжал издревле, искупил в жезл достояния Твоего,~--- эту гору Сион, на которой Ты вселился.
\vs Psa 73:3 Подвигни стопы Твои к вековым развалинам: все разрушил враг во святилище.
\vs Psa 73:4 Рыкают враги Твои среди собраний Твоих; поставили знаки свои вместо знамений \bibemph{наших};
\vs Psa 73:5 показывали себя подобными поднимающему вверх секиру на сплетшиеся ветви дерева;
\vs Psa 73:6 и ныне все резьбы в нем в один раз разрушили секирами и бердышами;
\vs Psa 73:7 предали огню святилище Твое; совсем осквернили жилище имени Твоего;
\vs Psa 73:8 сказали в сердце своем: <<разорим их совсем>>,~--- и сожгли все места собраний Божиих на земле.
\vs Psa 73:9 Знамений наших мы не видим, нет уже пророка, и нет с нами, кто знал бы, доколе \bibemph{это будет}.
\vs Psa 73:10 Доколе, Боже, будет поносить враг? вечно ли будет хулить противник имя Твое?
\vs Psa 73:11 Для чего отклоняешь руку Твою и десницу Твою? Из среды недра Твоего порази \bibemph{их}.
\vs Psa 73:12 Боже, Царь мой от века, устрояющий спасение посреди земли!
\vs Psa 73:13 Ты расторг силою Твоею море, Ты сокрушил головы змиев в воде;
\vs Psa 73:14 Ты сокрушил голову левиафана, отдал его в пищу людям пустыни, [Ефиопским];
\vs Psa 73:15 Ты иссек источник и поток, Ты иссушил сильные реки.
\vs Psa 73:16 Твой день и Твоя ночь: Ты уготовал светила и солнце;
\vs Psa 73:17 Ты установил все пределы земли, лето и зиму Ты учредил.
\vs Psa 73:18 Вспомни же: враг поносит Господа, и люди безумные хулят имя Твое.
\vs Psa 73:19 Не предай зверям душу горлицы Твоей; собрания убогих Твоих не забудь навсегда.
\vs Psa 73:20 Призри на завет Твой; ибо наполнились все мрачные места земли жилищами насилия.
\vs Psa 73:21 Да не возвратится угнетенный посрамленным; нищий и убогий да восхвалят имя Твое.
\vs Psa 73:22 Восстань, Боже, защити дело Твое, вспомни вседневное поношение Твое от безумного;
\vs Psa 73:23 не забудь крика врагов Твоих; шум восстающих против Тебя непрестанно поднимается.
\vs Psa 74:1 Начальнику хора. Не погуби. Псалом Асафа. Песнь.
\rsbpar\vs Psa 74:2 Славим Тебя, Боже, славим, ибо близко имя Твое; возвещают чудеса Твои.
\vs Psa 74:3 <<Когда изберу время, Я произведу суд по правде.
\vs Psa 74:4 Колеблется земля и все живущие на ней: Я утвержу столпы ее>>.
\vs Psa 74:5 Говорю безумствующим: <<не безумствуйте>>, и нечестивым: <<не поднимайте р\acc{о}га,
\vs Psa 74:6 не поднимайте высоко р\acc{о}га вашего, [не] говорите [на Бога] жестоковыйно>>,
\vs Psa 74:7 ибо не от востока и не от запада и не от пустыни возвышение,
\vs Psa 74:8 но Бог есть судия: одного унижает, а другого возносит;
\vs Psa 74:9 ибо чаша в руке Господа, вино кипит в ней, полное смешения, и Он наливает из нее. Даже дрожжи ее будут выжимать и пить все нечестивые земли.
\vs Psa 74:10 А я буду возвещать вечно, буду воспевать Бога Иаковлева,
\vs Psa 74:11 все роги нечестивых сломлю, и вознесутся роги праведника.
\vs Psa 75:1 Начальнику хора. На струнных \bibemph{орудиях}. Псалом Асафа. Песнь.
\rsbpar\vs Psa 75:2 Ведом в Иудее Бог; у Израиля велико имя Его.
\vs Psa 75:3 И было в Салиме жилище Его и пребывание Его на Сионе.
\vs Psa 75:4 Там сокрушил Он стрелы лука, щит и меч и брань.
\vs Psa 75:5 Ты славен, могущественнее гор хищнических.
\vs Psa 75:6 Крепкие сердцем стали добычею, уснули сном своим, и не нашли все мужи силы рук своих.
\vs Psa 75:7 От прещения Твоего, Боже Иакова, вздремали и колесница и конь.
\vs Psa 75:8 Ты страшен, и кто устоит пред лицем Твоим во время гнева Твоего?
\vs Psa 75:9 С небес Ты возвестил суд; земля убоялась и утихла,
\vs Psa 75:10 когда восстал Бог на суд, чтобы спасти всех угнетенных земли.
\vs Psa 75:11 И гнев человеческий обратится во славу Тебе: остаток гнева Ты укротишь.
\vs Psa 75:12 Делайте и воздавайте обеты Господу, Богу вашему; все, которые вокруг Него, да принесут дары Страшному:
\vs Psa 75:13 Он укрощает дух князей, Он страшен для царей земных.
\vs Psa 76:1 Начальнику хора Идифумова. Псалом Асафа.
\rsbpar\vs Psa 76:2 Глас мой к Богу, и я буду взывать; глас мой к Богу, и Он услышит меня.
\vs Psa 76:3 В день скорби моей ищу Господа; рука моя простерта ночью и не опускается; душа моя отказывается от утешения.
\vs Psa 76:4 Вспоминаю о Боге и трепещу; помышляю, и изнемогает дух мой.
\vs Psa 76:5 Ты не даешь мне сомкнуть очей моих; я потрясен и не могу говорить.
\vs Psa 76:6 Размышляю о днях древних, о летах веков \bibemph{минувших};
\vs Psa 76:7 припоминаю песни мои в ночи, беседую с сердцем моим, и дух мой испытывает:
\vs Psa 76:8 неужели навсегда отринул Господь, и не будет более благоволить?
\vs Psa 76:9 неужели навсегда престала милость Его, и пресеклось слово Его в род и род?
\vs Psa 76:10 неужели Бог забыл миловать? Неужели во гневе затворил щедроты Свои?
\vs Psa 76:11 И сказал я: <<вот мое горе~--- изменение десницы Всевышнего>>.
\vs Psa 76:12 Буду вспоминать о делах Господа; буду вспоминать о чудесах Твоих древних;
\vs Psa 76:13 буду вникать во все дела Твои, размышлять о великих Твоих деяниях.
\vs Psa 76:14 Боже! свят путь Твой. Кто Бог так великий, как Бог [наш]!
\vs Psa 76:15 Ты~--- Бог, творящий чудеса; Ты явил могущество Свое среди народов;
\vs Psa 76:16 Ты избавил мышцею народ Твой, сынов Иакова и Иосифа.
\vs Psa 76:17 Видели Тебя, Боже, воды, видели Тебя воды и убоялись, и вострепетали бездны.
\vs Psa 76:18 Облака изливали воды, тучи издавали гром, и стрелы Твои летали.
\vs Psa 76:19 Глас грома Твоего в круге небесном; молнии освещали вселенную; земля содрогалась и тряслась.
\vs Psa 76:20 Путь Твой в море, и стезя Твоя в водах великих, и следы Твои неведомы.
\vs Psa 76:21 Как стадо, вел Ты народ Твой рукою Моисея и Аарона.
\vs Psa 77:0 Учение Асафа.
\rsbpar\vs Psa 77:1 Внимай, народ мой, закону моему, приклоните ухо ваше к словам уст моих.
\vs Psa 77:2 Открою уста мои в притче и произнесу гадания из древности.
\vs Psa 77:3 Что слышали мы и узнали, и отцы наши рассказали нам,
\vs Psa 77:4 не скроем от детей их, возвещая роду грядущему славу Господа, и силу Его, и чудеса Его, которые Он сотворил.
\vs Psa 77:5 Он постановил устав в Иакове и положил закон в Израиле, который заповедал отцам нашим возвещать детям их,
\vs Psa 77:6 чтобы знал грядущий род, дети, которые родятся, и чтобы они в свое время возвещали своим детям,~---
\vs Psa 77:7 возлагать надежду свою на Бога и не забывать дел Божиих, и хранить заповеди Его,
\vs Psa 77:8 и не быть подобными отцам их, роду упорному и мятежному, неустроенному сердцем и неверному Богу духом своим.
\vs Psa 77:9 Сыны Ефремовы, вооруженные, стреляющие из луков, обратились назад в день брани:
\vs Psa 77:10 они не сохранили завета Божия и отреклись ходить в законе Его;
\vs Psa 77:11 забыли дела Его и чудеса, которые Он явил им.
\vs Psa 77:12 Он пред глазами отцов их сотворил чудеса в земле Египетской, на поле Цоан:
\vs Psa 77:13 разделил море, и провел их чрез него, и поставил воды стеною;
\vs Psa 77:14 и днем вел их облаком, а во всю ночь светом огня;
\vs Psa 77:15 рассек камень в пустыне и напоил их, как из великой бездны;
\vs Psa 77:16 из скалы извел потоки, и воды потекли, как реки.
\vs Psa 77:17 Но они продолжали грешить пред Ним и раздражать Всевышнего в пустыне:
\vs Psa 77:18 искушали Бога в сердце своем, требуя пищи по душе своей,
\vs Psa 77:19 и говорили против Бога и сказали: <<может ли Бог приготовить трапезу в пустыне?>>
\vs Psa 77:20 Вот, Он ударил в камень, и потекли воды, и полились ручьи. <<Может ли Он дать и хлеб, может ли приготовлять мясо народу Своему?>>
\vs Psa 77:21 Господь услышал и воспламенился гневом, и огонь возгорелся на Иакова, и гнев подвигнулся на Израиля
\vs Psa 77:22 за то, что не веровали в Бога и не уповали на спасение Его.
\vs Psa 77:23 Он повелел облакам свыше и отверз двери неба,
\vs Psa 77:24 и одождил на них манну в пищу, и хлеб небесный дал им.
\vs Psa 77:25 Хлеб ангельский ел человек; послал Он им пищу до сытости.
\vs Psa 77:26 Он возбудил на небе восточный ветер и навел южный силою Своею
\vs Psa 77:27 и, как пыль, одождил на них мясо и, как песок морской, птиц пернатых:
\vs Psa 77:28 поверг их среди стана их, около жилищ их,~---
\vs Psa 77:29 и они ели и пресытились; и желаемое ими дал им.
\vs Psa 77:30 Но еще не прошла прихоть их, еще пища была в устах их,
\vs Psa 77:31 гнев Божий пришел на них, убил тучных их и юношей Израилевых низложил.
\vs Psa 77:32 При всем этом они продолжали грешить и не верили чудесам Его.
\vs Psa 77:33 И погубил дни их в суете и лета их в смятении.
\vs Psa 77:34 Когда Он убивал их, они искали Его и обращались, и с раннего утра прибегали к Богу,
\vs Psa 77:35 и вспоминали, что Бог~--- их прибежище, и Бог Всевышний~--- Избавитель их,
\vs Psa 77:36 и льстили Ему устами своими и языком своим лгали пред Ним;
\vs Psa 77:37 сердце же их было неправо пред Ним, и они не были верны завету Его.
\vs Psa 77:38 Но Он, Милостивый, прощал грех и не истреблял их, многократно отвращал гнев Свой и не возбуждал всей ярости Своей:
\vs Psa 77:39 Он помнил, что они плоть, дыхание, которое уходит и не возвращается.
\vs Psa 77:40 Сколько раз они раздражали Его в пустыне и прогневляли Его в \bibemph{стране} необитаемой!
\vs Psa 77:41 и снова искушали Бога и оскорбляли Святаго Израилева,
\vs Psa 77:42 не помнили рук\acc{и} Его, дня, когда Он избавил их от угнетения,
\vs Psa 77:43 когда сотворил в Египте знамения Свои и чудеса Свои на поле Цоан;
\vs Psa 77:44 и превратил реки их и потоки их в кровь, чтобы они не могли пить;
\vs Psa 77:45 послал на них насекомых, чтобы жалили их, и жаб, чтобы губили их;
\vs Psa 77:46 земные произрастения их отдал гусенице и труд их~--- саранче;
\vs Psa 77:47 виноград их побил градом и сикоморы их~--- льдом;
\vs Psa 77:48 скот их предал граду и стада их~--- молниям;
\vs Psa 77:49 послал на них пламень гнева Своего, и негодование, и ярость и бедствие, посольство злых ангелов;
\vs Psa 77:50 уравнял стезю гневу Своему, не охранял души их от смерти, и скот их предал моровой язве;
\vs Psa 77:51 поразил всякого первенца в Египте, начатки сил в шатрах Хамовых;
\vs Psa 77:52 и повел народ Свой, как овец, и вел их, как стадо, пустынею;
\vs Psa 77:53 вел их безопасно, и они не страшились, а врагов их покрыло море;
\vs Psa 77:54 и привел их в область святую Свою, на гору сию, которую стяжала десница Его;
\vs Psa 77:55 прогнал от лица их народы и землю их разделил в наследие им, и колена Израилевы поселил в шатрах их.
\vs Psa 77:56 Но они еще искушали и огорчали Бога Всевышнего, и уставов Его не сохраняли;
\vs Psa 77:57 отступали и изменяли, как отцы их, обращались назад, как неверный лук;
\vs Psa 77:58 огорчали Его высотами своими и истуканами своими возбуждали ревность Его.
\vs Psa 77:59 Услышал Бог и воспламенился гневом и сильно вознегодовал на Израиля;
\vs Psa 77:60 отринул жилище в Силоме, скинию, в которой обитал Он между человеками;
\vs Psa 77:61 и отдал в плен крепость Свою и славу Свою в руки врага,
\vs Psa 77:62 и предал мечу народ Свой и прогневался на наследие Свое.
\vs Psa 77:63 Юношей его поедал огонь, и девицам его не пели брачных песен;
\vs Psa 77:64 священники его падали от меча, и вдовы его не плакали.
\vs Psa 77:65 Но, как бы от сна, воспрянул Господь, как бы исполин, побежденный вином,
\vs Psa 77:66 и поразил врагов его в тыл, вечному сраму предал их;
\vs Psa 77:67 и отверг шатер Иосифов и колена Ефремова не избрал,
\vs Psa 77:68 а избрал колено Иудино, гору Сион, которую возлюбил.
\vs Psa 77:69 И устроил, как небо, святилище Свое и, как землю, утвердил его навек,
\vs Psa 77:70 и избрал Давида, раба Своего, и взял его от дворов овчих
\vs Psa 77:71 и от доящих привел его пасти народ Свой, Иакова, и наследие Свое, Израиля.
\vs Psa 77:72 И он пас их в чистоте сердца своего и руками мудрыми водил их.
\vs Psa 78:0 Псалом Асафа.
\rsbpar\vs Psa 78:1 Боже! язычники пришли в наследие Твое, осквернили святый храм Твой, Иерусалим превратили в развалины;
\vs Psa 78:2 трупы рабов Твоих отдали на съедение птицам небесным, тела святых Твоих~--- зверям земным;
\vs Psa 78:3 пролили кровь их, как воду, вокруг Иерусалима, и некому было похоронить их.
\vs Psa 78:4 Мы сделались посмешищем у соседей наших, поруганием и посрамлением у окружающих нас.
\vs Psa 78:5 Доколе, Господи, будешь гневаться непрестанно, будет пылать ревность Твоя, как огонь?
\vs Psa 78:6 Пролей гнев Твой на народы, которые не знают Тебя, и на царства, которые имени Твоего не призывают,
\vs Psa 78:7 ибо они пожрали Иакова и жилище его опустошили.
\vs Psa 78:8 Не помяни нам грехов \bibemph{наших} предков; скоро да предварят нас щедроты Твои, ибо мы весьма истощены.
\vs Psa 78:9 Помоги нам, Боже, Спаситель наш, ради славы имени Твоего; избавь нас и прости нам грехи наши ради имени Твоего.
\vs Psa 78:10 Для чего язычникам говорить: <<где Бог их?>> Да сделается известным между язычниками пред глазами нашими отмщение за пролитую кровь рабов Твоих.
\vs Psa 78:11 Да придет пред лице Твое стенание узника; могуществом мышцы Твоей сохрани обреченных на смерть.
\vs Psa 78:12 Семикратно возврати соседям нашим в недро их поношение, которым они Тебя, Господи, поносили.
\vs Psa 78:13 А мы, народ Твой и Твоей пажити овцы, вечно будем славить Тебя и в род и род возвещать хвалу Тебе.
\vs Psa 79:1 Начальнику хора. На музыкальном \bibemph{орудии} Шошанним-Эдуф. Псалом Асафа.
\rsbpar\vs Psa 79:2 Пастырь Израиля! внемли; водящий, как овец, Иосифа, восседающий на Херувимах, яви Себя.
\vs Psa 79:3 Пред Ефремом и Вениамином и Манассиею воздвигни силу Твою, и приди спасти нас.
\vs Psa 79:4 Боже! восстанови нас; да воссияет лице Твое, и спасемся!
\vs Psa 79:5 Господи, Боже сил! доколе будешь гневен к молитвам народа Твоего?
\vs Psa 79:6 Ты напитал их хлебом слезным, и напоил их слезами в большой мере,
\vs Psa 79:7 положил нас в пререкание соседям нашим, и враги наши издеваются \bibemph{над нами}.
\vs Psa 79:8 Боже сил! восстанови нас; да воссияет лице Твое, и спасемся!
\vs Psa 79:9 Из Египта перенес Ты виноградную лозу, выгнал народы и посадил ее;
\vs Psa 79:10 очистил для нее место, и утвердил корни ее, и она наполнила землю.
\vs Psa 79:11 Горы покрылись тенью ее, и ветви ее как кедры Божии;
\vs Psa 79:12 она пустила ветви свои до моря и отрасли свои до реки.
\vs Psa 79:13 Для чего разрушил Ты ограды ее, так что обрывают ее все, проходящие по пути?
\vs Psa 79:14 Лесной вепрь подрывает ее, и полевой зверь объедает ее.
\vs Psa 79:15 Боже сил! обратись же, призри с неба, и воззри, и посети виноград сей;
\vs Psa 79:16 охрани то, что насадила десница Твоя, и отрасли, которые Ты укрепил Себе.
\vs Psa 79:17 Он пожжен огнем, обсечен; от прещения лица Твоего погибнут.
\vs Psa 79:18 Да будет рука Твоя над мужем десницы Твоей, над сыном человеческим, которого Ты укрепил Себе,
\vs Psa 79:19 и мы не отступим от Тебя; оживи нас, и мы будем призывать имя Твое.
\vs Psa 79:20 Господи, Боже сил! восстанови нас; да воссияет лице Твое, и спасемся!
\vs Psa 80:1 Начальнику хора. На Гефском орудии. Псалом Асафа.
\rsbpar\vs Psa 80:2 Радостно пойте Богу, твердыне нашей; восклицайте Богу Иакова;
\vs Psa 80:3 возьмите псалом, дайте тимпан, сладкозвучные гусли с псалтирью;
\vs Psa 80:4 трубите в новомесячие трубою, в определенное время, в день праздника нашего;
\vs Psa 80:5 ибо это закон для Израиля, устав от Бога Иаковлева.
\vs Psa 80:6 Он установил это во свидетельство для Иосифа, когда он вышел из земли Египетской, где услышал звуки языка, которого не знал:
\vs Psa 80:7 <<Я снял с рамен его тяжести, и руки его освободились от корзин.
\vs Psa 80:8 В бедствии ты призвал Меня, и Я избавил тебя; из среды грома Я услышал тебя, при водах Меривы испытал тебя.
\vs Psa 80:9 Слушай, народ Мой, и Я буду свидетельствовать тебе: Израиль! о, если бы ты послушал Меня!
\vs Psa 80:10 Да не будет у тебя иного бога, и не поклоняйся богу чужеземному.
\vs Psa 80:11 Я Господь, Бог твой, изведший тебя из земли Египетской; открой уста твои, и Я наполню их>>.
\vs Psa 80:12 Но народ Мой не слушал гласа Моего, и Израиль не покорялся Мне;
\vs Psa 80:13 потому Я оставил их упорству сердца их, пусть ходят по своим помыслам.
\vs Psa 80:14 О, если бы народ Мой слушал Меня и Израиль ходил Моими путями!
\vs Psa 80:15 Я скоро смирил бы врагов их и обратил бы руку Мою на притеснителей их:
\vs Psa 80:16 ненавидящие Господа раболепствовали бы им, а их благоденствие продолжалось бы навсегда;
\vs Psa 80:17 Я питал бы их туком пшеницы и насыщал бы их медом из скалы.
\vs Psa 81:0 Псалом Асафа.
\rsbpar\vs Psa 81:1 Бог стал в сонме богов; среди богов произнес суд:
\vs Psa 81:2 доколе будете вы судить неправедно и оказывать лицеприятие нечестивым?
\vs Psa 81:3 Давайте суд бедному и сироте; угнетенному и нищему оказывайте справедливость;
\vs Psa 81:4 избавляйте бедного и нищего; исторгайте \bibemph{его} из руки нечестивых.
\vs Psa 81:5 Не знают, не разумеют, во тьме ходят; все основания земли колеблются.
\vs Psa 81:6 Я сказал: вы~--- боги, и сыны Всевышнего~--- все вы;
\vs Psa 81:7 но вы умрете, как человеки, и падете, как всякий из князей.
\vs Psa 81:8 Восстань\fns{В славянском переводе: Воскресни\dots}, Боже, суди землю, ибо Ты наследуешь все народы.
\vs Psa 82:1 Песнь. Псалом Асафа.
\rsbpar\vs Psa 82:2 Боже! Не премолчи, не безмолвствуй и не оставайся в покое, Боже,
\vs Psa 82:3 ибо вот, враги Твои шумят, и ненавидящие Тебя подняли голову;
\vs Psa 82:4 против народа Твоего составили коварный умысел и совещаются против хранимых Тобою;
\vs Psa 82:5 сказали: <<пойдем и истребим их из народов, чтобы не вспоминалось более имя Израиля>>.
\vs Psa 82:6 Сговорились единодушно, заключили против Тебя союз:
\vs Psa 82:7 селения Едомовы и Измаильтяне, Моав и Агаряне,
\vs Psa 82:8 Гевал и Аммон и Амалик, Филистимляне с жителями Тира.
\vs Psa 82:9 И Ассур пристал к ним: они стали мышцею для сынов Лотовых.
\vs Psa 82:10 Сделай им то же, что Мадиаму, что Сисаре, что Иавину у потока Киссона,
\vs Psa 82:11 которые истреблены в Аендоре, сделались навозом для земли.
\vs Psa 82:12 Поступи с ними, с князьями их, как с Оривом и Зивом и со всеми вождями их, как с Зевеем и Салманом,
\vs Psa 82:13 которые говорили: <<возьмем себе во владение селения Божии>>.
\vs Psa 82:14 Боже мой! Да будут они, как пыль в вихре, как солома перед ветром.
\vs Psa 82:15 Как огонь сжигает лес, и как пламя опаляет горы,
\vs Psa 82:16 так погони их бурею Твоею и вихрем Твоим приведи их в смятение;
\vs Psa 82:17 исполни лица их бесчестием, чтобы они взыскали имя Твое, Господи!
\vs Psa 82:18 Да постыдятся и смятутся на веки, да посрамятся и погибнут,
\vs Psa 82:19 и да познают, что Ты, Которого одного имя Господь, Всевышний над всею землею.
\vs Psa 83:1 Начальнику хора. На Гефском \bibemph{орудии}. Кореевых сынов. Псалом.
\rsbpar\vs Psa 83:2 Как вожделенны жилища Твои, Господи сил!
\vs Psa 83:3 Истомилась душа моя, желая во дворы Господни; сердце мое и плоть моя восторгаются к Богу живому.
\vs Psa 83:4 И птичка находит себе жилье, и ласточка гнездо себе, где положить птенцов своих, у алтарей Твоих, Господи сил, Царь мой и Бог мой!
\vs Psa 83:5 Блаженны живущие в доме Твоем: они непрестанно будут восхвалять Тебя.
\vs Psa 83:6 Блажен человек, которого сила в Тебе и у которого в сердце стези направлены \bibemph{к Тебе}.
\vs Psa 83:7 Проходя долиною плача, они открывают в ней источники, и дождь покрывает ее благословением;
\vs Psa 83:8 приходят от силы в силу, являются пред Богом на Сионе.
\vs Psa 83:9 Господи, Боже сил! Услышь молитву мою, внемли, Боже Иаковлев!
\vs Psa 83:10 Боже, защитник наш! Приникни и призри на лице помазанника Твоего.
\vs Psa 83:11 Ибо один день во дворах Твоих лучше тысячи. Желаю лучше быть у порога в доме Божием, нежели жить в шатрах нечестия.
\vs Psa 83:12 Ибо Господь Бог есть солнце и щит, Господь дает благодать и славу; ходящих в непорочности Он не лишает благ.
\vs Psa 83:13 Господи сил! Блажен человек, уповающий на Тебя!
\vs Psa 84:1 Начальнику хора. Кореевых сынов. Псалом.
\rsbpar\vs Psa 84:2 Господи! Ты умилосердился к земле Твоей, возвратил плен Иакова;
\vs Psa 84:3 простил беззаконие народа Твоего, покрыл все грехи его,
\vs Psa 84:4 отъял всю ярость Твою, отвратил лютость гнева Твоего.
\vs Psa 84:5 Восстанови нас, Боже спасения нашего, и прекрати негодование Твое на нас.
\vs Psa 84:6 Неужели вечно будешь гневаться на нас, прострешь гнев Твой от рода в род?
\vs Psa 84:7 Неужели снова не оживишь нас, чтобы народ Твой возрадовался о Тебе?
\vs Psa 84:8 Яви нам, Господи, милость Твою, и спасение Твое даруй нам.
\vs Psa 84:9 Послушаю, что скажет Господь Бог. Он скажет мир народу Своему и избранным Своим, но да не впадут они снова в безрассудство.
\vs Psa 84:10 Так, близко к боящимся Его спасение Его, чтобы обитала слава в земле нашей!
\vs Psa 84:11 Милость и истина сретятся, правда и мир облобызаются;
\vs Psa 84:12 истина возникнет из земли, и правда приникнет с небес;
\vs Psa 84:13 и Господь даст благо, и земля наша даст плод свой;
\vs Psa 84:14 правда пойдет пред Ним и поставит на путь стопы свои.
\vs Psa 85:0 Молитва Давида.
\rsbpar\vs Psa 85:1 Приклони, Господи, ухо Твое и услышь меня, ибо я беден и нищ.
\vs Psa 85:2 Сохрани душу мою, ибо я благоговею пред Тобою; спаси, Боже мой, раба Твоего, уповающего на Тебя.
\vs Psa 85:3 Помилуй меня, Господи, ибо к Тебе взываю каждый день.
\vs Psa 85:4 Возвесели душу раба Твоего, ибо к Тебе, Господи, возношу душу мою,
\vs Psa 85:5 ибо Ты, Господи, благ и милосерд и многомилостив ко всем, призывающим Тебя.
\vs Psa 85:6 Услышь, Господи, молитву мою и внемли гласу моления моего.
\vs Psa 85:7 В день скорби моей взываю к Тебе, потому что Ты услышишь меня.
\vs Psa 85:8 Нет между богами, как Ты, Господи, и нет дел, как Твои.
\vs Psa 85:9 Все народы, Тобою сотворенные, приидут и поклонятся пред Тобою, Господи, и прославят имя Твое,
\vs Psa 85:10 ибо Ты велик и творишь чудеса,~--- Ты, Боже, един Ты.
\vs Psa 85:11 Наставь меня, Господи, на путь Твой, и буду ходить в истине Твоей; утверди сердце мое в страхе имени Твоего.
\vs Psa 85:12 Буду восхвалять Тебя, Господи, Боже мой, всем сердцем моим и славить имя Твое вечно,
\vs Psa 85:13 ибо велика милость Твоя ко мне: Ты избавил душу мою от ада преисподнего.
\vs Psa 85:14 Боже! гордые восстали на меня, и скопище мятежников ищет души моей: не представляют они Тебя пред собою.
\vs Psa 85:15 Но Ты, Господи, Боже щедрый и благосердный, долготерпеливый и многомилостивый и истинный,
\vs Psa 85:16 призри на меня и помилуй меня; даруй крепость Твою рабу Твоему, и спаси сына рабы Твоей;
\vs Psa 85:17 покажи на мне знамение во благо, да видят ненавидящие меня и устыдятся, потому что Ты, Господи, помог мне и утешил меня.
\vs Psa 86:1 Сынов Кореевых. Псалом. Песнь.
\rsbpar\vs Psa 86:2 Основание его\fns{Иерусалима.} на горах святых. Господь любит врата Сиона более всех селений Иакова.
\vs Psa 86:3 Славное возвещается о тебе, град Божий!
\vs Psa 86:4 Упомяну знающим меня о Рааве\fns{О Египте.} и Вавилоне; вот Филистимляне и Тир с Ефиопиею,~--- \bibemph{скажут}: <<такой-то родился там>>.
\vs Psa 86:5 О Сионе же будут говорить: <<такой-то и такой-то муж родился в нем, и Сам Всевышний укрепил его>>.
\vs Psa 86:6 Господь в переписи народов напишет: <<такой-то родился там>>.
\vs Psa 86:7 И поющие и играющие,~--- все источники мои в тебе.
\vs Psa 87:1 Песнь. Псалом, Сынов Кореевых. Начальнику хора на Махалаф, для пения. Учение Емана Езрахита.
\rsbpar\vs Psa 87:2 Господи, Боже спасения моего! днем вопию и ночью пред Тобою:
\vs Psa 87:3 да внидет пред лице Твое молитва моя; приклони ухо Твое к молению моему,
\vs Psa 87:4 ибо душа моя насытилась бедствиями, и жизнь моя приблизилась к преисподней.
\vs Psa 87:5 Я сравнялся с нисходящими в могилу; я стал, как человек без силы,
\vs Psa 87:6 между мертвыми брошенный,~--- как убитые, лежащие во гробе, о которых Ты уже не вспоминаешь и которые от руки Твоей отринуты.
\vs Psa 87:7 Ты положил меня в ров преисподний, во мрак, в бездну.
\vs Psa 87:8 Отяготела на мне ярость Твоя, и всеми волнами Твоими Ты поразил [меня].
\vs Psa 87:9 Ты удалил от меня знакомых моих, сделал меня отвратительным для них; я заключен, и не могу выйти.
\vs Psa 87:10 Око мое истомилось от горести: весь день я взывал к Тебе, Господи, простирал к Тебе руки мои.
\vs Psa 87:11 Разве над мертвыми Ты сотворишь чудо? Разве мертвые встанут и будут славить Тебя?
\vs Psa 87:12 или во гробе будет возвещаема милость Твоя, и истина Твоя~--- в месте тления?
\vs Psa 87:13 разве во мраке позн\acc{а}ют чудеса Твои, и в земле забвения~--- правду Твою?
\vs Psa 87:14 Но я к Тебе, Господи, взываю, и рано утром молитва моя предваряет Тебя.
\vs Psa 87:15 Для чего, Господи, отреваешь душу мою, скрываешь лице Твое от меня?
\vs Psa 87:16 Я несчастен и истаеваю с юности; несу ужасы Твои и изнемогаю.
\vs Psa 87:17 Надо мною прошла ярость Твоя, устрашения Твои сокрушили меня,
\vs Psa 87:18 всякий день окружают меня, как вода: облегают меня все вместе.
\vs Psa 87:19 Ты удалил от меня друга и искреннего; знакомых моих не видно.
\vs Psa 88:1 Учение Ефама Езрахита.
\rsbpar\vs Psa 88:2 Милости [Твои], Господи, буду петь вечно, в род и род возвещать истину Твою устами моими.
\vs Psa 88:3 Ибо говорю: навек основана милость, на небесах утвердил Ты истину Твою, \bibemph{когда сказал}:
\vs Psa 88:4 <<Я поставил завет с избранным Моим, клялся Давиду, рабу Моему:
\vs Psa 88:5 навек утвержу семя твое, в род и род устрою престол твой>>.
\vs Psa 88:6 И небеса прославят чудные дела Твои, Господи, и истину Твою в собрании святых.
\vs Psa 88:7 Ибо кто на небесах сравнится с Господом? кто между сынами Божиими уподобится Господу?
\vs Psa 88:8 Страшен Бог в великом сонме святых, страшен Он для всех окружающих Его.
\vs Psa 88:9 Господи, Боже сил! кто силен, как Ты, Господи? И истина Твоя окрест Тебя.
\vs Psa 88:10 Ты владычествуешь над яростью моря: когда воздымаются волны его, Ты укрощаешь их.
\vs Psa 88:11 Ты низложил Раава, как пораженного; крепкою мышцею Твоею рассеял врагов Твоих.
\vs Psa 88:12 Твои небеса и Твоя земля; вселенную и что наполняет ее, Ты основал.
\vs Psa 88:13 Север и юг Ты сотворил; Фавор и Ермон о имени Твоем радуются.
\vs Psa 88:14 Крепка мышца Твоя, сильна рука Твоя, высока десница Твоя!
\vs Psa 88:15 Правосудие и правота~--- основание престола Твоего; милость и истина предходят пред лицем Твоим.
\vs Psa 88:16 Блажен народ, знающий трубный зов! Они ходят во свете лица Твоего, Господи,
\vs Psa 88:17 о имени Твоем радуются весь день и правдою Твоею возносятся,
\vs Psa 88:18 ибо Ты украшение силы их, и благоволением Твоим возвышается рог наш.
\vs Psa 88:19 От Господа~--- щит наш, и от Святаго Израилева~--- царь наш.
\vs Psa 88:20 Некогда говорил Ты в видении святому Твоему, и сказал: <<Я оказал помощь мужественному, вознес избранного из народа.
\vs Psa 88:21 Я обрел Давида, раба Моего, святым елеем Моим помазал его.
\vs Psa 88:22 Рука Моя пребудет с ним, и мышца Моя укрепит его.
\vs Psa 88:23 Враг не превозможет его, и сын беззакония не притеснит его.
\vs Psa 88:24 Сокрушу пред ним врагов его и поражу ненавидящих его.
\vs Psa 88:25 И истина Моя и милость Моя с ним, и Моим именем возвысится рог его.
\vs Psa 88:26 И положу на море руку его, и на реки~--- десницу его.
\vs Psa 88:27 Он будет звать Меня: Ты отец мой, Бог мой и твердыня спасения моего.
\vs Psa 88:28 И Я сделаю его первенцем, превыше царей земли,
\vs Psa 88:29 вовек сохраню ему милость Мою, и завет Мой с ним будет верен.
\vs Psa 88:30 И продолжу вовек семя его, и престол его~--- как дни неба.
\vs Psa 88:31 Если сыновья его оставят закон Мой и не будут ходить по заповедям Моим;
\vs Psa 88:32 если нарушат уставы Мои и повелений Моих не сохранят:
\vs Psa 88:33 посещу жезлом беззаконие их, и ударами~--- неправду их;
\vs Psa 88:34 милости же Моей не отниму от него, и не изменю истины Моей.
\vs Psa 88:35 Не нарушу завета Моего, и не переменю того, что вышло из уст Моих.
\vs Psa 88:36 Однажды Я поклялся святостью Моею: солгу ли Давиду?
\vs Psa 88:37 Семя его пребудет вечно, и престол его, как солнце, предо Мною,
\vs Psa 88:38 вовек будет тверд, как луна, и верный свидетель на небесах>>.
\vs Psa 88:39 Но \bibemph{ныне} Ты отринул и презрел, прогневался на помазанника Твоего;
\vs Psa 88:40 пренебрег завет с рабом Твоим, поверг на землю венец его;
\vs Psa 88:41 разрушил все ограды его, превратил в развалины крепости его.
\vs Psa 88:42 Расхищают его все проходящие путем; он сделался посмешищем у соседей своих.
\vs Psa 88:43 Ты возвысил десницу противников его, обрадовал всех врагов его;
\vs Psa 88:44 Ты обратил назад острие меча его и не укрепил его на брани;
\vs Psa 88:45 отнял у него блеск и престол его поверг на землю;
\vs Psa 88:46 сократил дни юности его и покрыл его стыдом.
\vs Psa 88:47 Доколе, Господи, будешь скрываться непрестанно, будет пылать ярость Твоя, как огонь?
\vs Psa 88:48 Вспомни, какой мой век: на какую суету сотворил Ты всех сынов человеческих?
\vs Psa 88:49 Кто из людей жил~--- и не видел смерти, избавил душу свою от руки преисподней?
\vs Psa 88:50 Где прежние милости Твои, Господи? Ты клялся Давиду истиною Твоею.
\vs Psa 88:51 Вспомни, Господи, поругание рабов Твоих, которое я ношу в недре моем от всех сильных народов;
\vs Psa 88:52 как поносят враги Твои, Господи, как бесславят следы помазанника Твоего.
\vs Psa 88:53 Благословен Господь вовек! Аминь, аминь.
\vs Psa 89:1 Молитва Моисея, человека Божия.
\rsbpar\vs Psa 89:2 Господи! Ты нам прибежище в род и род.
\vs Psa 89:3 Прежде нежели родились горы, и Ты образовал землю и вселенную, и от века и до века Ты~--- Бог.
\vs Psa 89:4 Ты возвращаешь человека в тление и говоришь: <<возвратитесь, сыны человеческие!>>
\vs Psa 89:5 Ибо пред очами Твоими тысяча лет, как день вчерашний, когда он прошел, и \bibemph{как} стража в ночи.
\vs Psa 89:6 Ты \bibemph{как} наводнением уносишь их; они~--- \bibemph{как} сон, как трава, которая утром вырастает, утром цветет и зеленеет, вечером подсекается и засыхает;
\vs Psa 89:7 ибо мы исчезаем от гнева Твоего и от ярости Твоей мы в смятении.
\vs Psa 89:8 Ты положил беззакония наши пред Тобою и тайное наше пред светом лица Твоего.
\vs Psa 89:9 Все дни наши прошли во гневе Твоем; мы теряем лета наши, как звук.
\vs Psa 89:10 Дней лет наших~--- семьдесят лет, а при большей крепости~--- восемьдесят лет; и самая лучшая пора их~--- труд и болезнь, ибо проходят быстро, и мы летим.
\vs Psa 89:11 Кто знает силу гнева Твоего, и ярость Твою по мере страха Твоего?
\vs Psa 89:12 Научи нас так счислять дни наши, чтобы нам приобрести сердце мудрое.
\vs Psa 89:13 Обратись, Господи! Доколе? Умилосердись над рабами Твоими.
\vs Psa 89:14 Рано насыти нас милостью Твоею, и мы будем радоваться и веселиться во все дни наши.
\vs Psa 89:15 Возвесели нас за дни, \bibemph{в которые} Ты поражал нас, за лета, \bibemph{в которые} мы видели бедствие.
\vs Psa 89:16 Да явится на рабах Твоих дело Твое и на сынах их слава Твоя;
\vs Psa 89:17 и да будет благоволение Господа Бога нашего на нас, и в деле рук наших споспешествуй нам, в деле рук наших споспешествуй.
\vs Psa 90:0 [Хвалебная песнь Давида.]
\rsbpar\vs Psa 90:1 Живущий под кровом Всевышнего под сенью Всемогущего покоится,
\vs Psa 90:2 говорит Господу: <<прибежище мое и защита моя, Бог мой, на Которого я уповаю!>>
\vs Psa 90:3 Он избавит тебя от сети ловца, от гибельной язвы,
\vs Psa 90:4 перьями Своими осенит тебя, и под крыльями Его будешь безопасен; щит и ограждение~--- истина Его.
\vs Psa 90:5 Не убоишься ужасов в ночи, стрелы, летящей днем,
\vs Psa 90:6 язвы, ходящей во мраке, заразы, опустошающей в полдень.
\vs Psa 90:7 Падут подле тебя тысяча и десять тысяч одесную тебя; но к тебе не приблизится:
\vs Psa 90:8 только смотреть будешь очами твоими и видеть возмездие нечестивым.
\vs Psa 90:9 Ибо ты \bibemph{сказал}: <<Господь~--- упование мое>>; Всевышнего избрал ты прибежищем твоим;
\vs Psa 90:10 не приключится тебе зло, и язва не приблизится к жилищу твоему;
\vs Psa 90:11 ибо Ангелам Своим заповедает о тебе~--- охранять тебя на всех путях твоих:
\vs Psa 90:12 на руках понесут тебя, да не преткнешься о камень ногою твоею;
\vs Psa 90:13 на аспида и василиска наступишь; попирать будешь льва и дракона.
\vs Psa 90:14 <<За то, что он возлюбил Меня, избавлю его; защищу его, потому что он познал имя Мое.
\vs Psa 90:15 Воззовет ко Мне, и услышу его; с ним Я в скорби; избавлю его и прославлю его,
\vs Psa 90:16 долготою дней насыщу его, и явлю ему спасение Мое>>.
\vs Psa 91:1 Псалом. Песнь на день субботний.
\rsbpar\vs Psa 91:2 Благо есть славить Господа и петь имени Твоему, Всевышний,
\vs Psa 91:3 возвещать утром милость Твою и истину Твою в ночи,
\vs Psa 91:4 на десятиструнном и псалтири, с песнью на гуслях.
\vs Psa 91:5 Ибо Ты возвеселил меня, Господи, творением Твоим: я восхищаюсь делами рук Твоих.
\vs Psa 91:6 Как велики дела Твои, Господи! дивно глубоки помышления Твои!
\vs Psa 91:7 Человек несмысленный не знает, и невежда не разумеет того.
\vs Psa 91:8 Тогда как нечестивые возникают, как трава, и делающие беззаконие цветут, чтобы исчезнуть на веки,~---
\vs Psa 91:9 Ты, Господи, высок во веки!
\vs Psa 91:10 Ибо вот, враги Твои, Господи,~--- вот, враги Твои гибнут, и рассыпаются все делающие беззаконие;
\vs Psa 91:11 а мой рог Ты возносишь, как рог единорога, и я умащен свежим елеем;
\vs Psa 91:12 и око мое смотрит на врагов моих, и уши мои слышат о восстающих на меня злодеях.
\vs Psa 91:13 Праведник цветет, как пальма, возвышается подобно кедру на Ливане.
\vs Psa 91:14 Насажденные в доме Господнем, они цветут во дворах Бога нашего;
\vs Psa 91:15 они и в старости плодовиты, сочны и свежи,
\vs Psa 91:16 чтобы возвещать, что праведен Господь, твердыня моя, и нет неправды в Нем.
\vs Psa 92:0 [Хвалебная песнь Давида. В день предсубботний, когда населена земля.]
\rsbpar\vs Psa 92:1 Господь царствует; Он облечен величием, облечен Господь могуществом [и] препоясан: потому вселенная тверда, не подвигнется.
\vs Psa 92:2 Престол Твой утвержден искони: Ты~--- от века.
\vs Psa 92:3 Возвышают реки, Господи, возвышают реки голос свой, возвышают реки волны свои.
\vs Psa 92:4 Но паче шума вод многих, сильных волн морских, силен в вышних Господь.
\vs Psa 92:5 Откровения Твои несомненно верны. Дому Твоему, Господи, принадлежит святость на долгие дни.
\vs Psa 93:0 [Псалом Давида в четвертый день недели.]
\rsbpar\vs Psa 93:1 Боже отмщений, Господи, Боже отмщений, яви Себя!
\vs Psa 93:2 Восстань, Судия земли, воздай возмездие гордым.
\vs Psa 93:3 Доколе, Господи, нечестивые, доколе нечестивые торжествовать будут?
\vs Psa 93:4 Они изрыгают дерзкие речи; величаются все делающие беззаконие;
\vs Psa 93:5 попирают народ Твой, Господи, угнетают наследие Твое;
\vs Psa 93:6 вдову и пришельца убивают, и сирот умерщвляют
\vs Psa 93:7 и говорят: <<не увидит Господь, и не узнает Бог Иаковлев>>.
\vs Psa 93:8 Образумьтесь, бессмысленные люди! когда вы будете умны, невежды?
\vs Psa 93:9 Насадивший ухо не услышит ли? и образовавший глаз не увидит ли?
\vs Psa 93:10 Вразумляющий народы неужели не обличит,~--- Тот, Кто учит человека разумению?
\vs Psa 93:11 Господь знает мысли человеческие, что они суетны.
\vs Psa 93:12 Блажен человек, которого вразумляешь Ты, Господи, и наставляешь законом Твоим,
\vs Psa 93:13 чтобы дать ему покой в бедственные дни, доколе нечестивому выроется яма!
\vs Psa 93:14 Ибо не отринет Господь народа Своего и не оставит наследия Своего.
\vs Psa 93:15 Ибо суд возвратится к правде, и за ним \bibemph{последуют} все правые сердцем.
\vs Psa 93:16 Кто восстанет за меня против злодеев? кто станет за меня против делающих беззаконие?
\vs Psa 93:17 Если бы не Господь был мне помощником, вскоре вселилась бы душа моя в \bibemph{страну} молчания.
\vs Psa 93:18 Когда я говорил: <<колеблется нога моя>>,~--- милость Твоя, Господи, поддерживала меня.
\vs Psa 93:19 При умножении скорбей моих в сердце моем, утешения Твои услаждают душу мою.
\vs Psa 93:20 Станет ли близ Тебя седалище губителей, умышляющих насилие вопреки закону?
\vs Psa 93:21 Толпою устремляются они на душу праведника и осуждают кровь неповинную.
\vs Psa 93:22 Но Господь~--- защита моя, и Бог мой~--- твердыня убежища моего,
\vs Psa 93:23 и обратит на них беззаконие их, и злодейством их истребит их, истребит их Господь Бог наш.
\vs Psa 94:0 [Хвалебная песнь Давида.]
\rsbpar\vs Psa 94:1 Приидите, воспоем Господу, воскликнем [Богу], твердыне спасения нашего;
\vs Psa 94:2 предстанем лицу Его со славословием, в песнях воскликнем Ему,
\vs Psa 94:3 ибо Господь есть Бог великий и Царь великий над всеми богами.
\vs Psa 94:4 В Его руке глубины земли, и вершины гор~--- Его же;
\vs Psa 94:5 Его~--- море, и Он создал его, и сушу образовали руки Его.
\vs Psa 94:6 Приидите, поклонимся и припадем, преклоним колени пред лицем Господа, Творца нашего;
\vs Psa 94:7 ибо Он есть Бог наш, и мы~--- народ паствы Его и овцы руки Его. О, если бы вы ныне послушали гласа Его:
\vs Psa 94:8 <<не ожесточите сердца вашего, как в Мериве, как в день искушения в пустыне,
\vs Psa 94:9 где искушали Меня отцы ваши, испытывали Меня, и видели дело Мое.
\vs Psa 94:10 Сорок лет Я был раздражаем родом сим, и сказал: это народ, заблуждающийся сердцем; они не познали путей Моих,
\vs Psa 94:11 и потому Я поклялся во гневе Моем, что они не войдут в покой Мой>>.
\vs Psa 95:0 [Хвалебная песнь Давида. На построение дома.]
\rsbpar\vs Psa 95:1 Воспойте Господу песнь новую; воспойте Господу, вся земля;
\vs Psa 95:2 пойте Господу, благословляйте имя Его, благовествуйте со дня на день спасение Его;
\vs Psa 95:3 возвещайте в народах славу Его, во всех племенах чудеса Его;
\vs Psa 95:4 ибо велик Господь и достохвален, страшен Он паче всех богов.
\vs Psa 95:5 Ибо все боги народов~--- идолы, а Господь небеса сотворил.
\vs Psa 95:6 Слава и величие пред лицем Его, сила и великолепие во святилище Его.
\vs Psa 95:7 Воздайте Господу, племена народов, воздайте Господу славу и честь;
\vs Psa 95:8 воздайте Господу славу имени Его, несите дары и идите во дворы Его;
\vs Psa 95:9 поклонитесь Господу во благолепии святыни. Трепещи пред лицем Его, вся земля!
\vs Psa 95:10 Скажите народам: Господь царствует! потому тверда вселенная, не поколеблется. Он будет судить народы по правде.
\vs Psa 95:11 Да веселятся небеса и да торжествует земля; да шумит море и что наполняет его;
\vs Psa 95:12 да радуется поле и все, что на нем, и да ликуют все дерева дубравные
\vs Psa 95:13 пред лицем Господа; ибо идет, ибо идет судить землю. Он будет судить вселенную по правде, и народы~--- по истине Своей.
\vs Psa 96:0 [Псалом Давида, когда устроялась земля его.]
\rsbpar\vs Psa 96:1 Господь царствует: да радуется земля; да веселятся многочисленные острова.
\vs Psa 96:2 Облако и мрак окрест Его; правда и суд~--- основание престола Его.
\vs Psa 96:3 Пред Ним идет огонь и вокруг попаляет врагов Его.
\vs Psa 96:4 Молнии Его освещают вселенную; земля видит и трепещет.
\vs Psa 96:5 Горы, как воск, тают от лица Господа, от лица Господа всей земли.
\vs Psa 96:6 Небеса возвещают правду Его, и все народы видят славу Его.
\vs Psa 96:7 Да постыдятся все служащие истуканам, хвалящиеся идолами. Поклонитесь пред Ним, все боги\fns{По переводу 70-ти: все Ангелы Его.}.
\vs Psa 96:8 Слышит Сион и радуется, и веселятся дщери Иудины ради судов Твоих, Господи,
\vs Psa 96:9 ибо Ты, Господи, высок над всею землею, превознесен над всеми богами.
\vs Psa 96:10 Любящие Господа, ненавидьте зло! Он хранит души святых Своих; из руки нечестивых избавляет их.
\vs Psa 96:11 Свет сияет на праведника, и на правых сердцем~--- веселие.
\vs Psa 96:12 Радуйтесь, праведные, о Господе и славьте память святыни Его.
\vs Psa 97:0 Псалом [Давида].
\rsbpar\vs Psa 97:1 Воспойте Господу новую песнь, ибо Он сотворил чудеса. Его десница и святая мышца Его доставили Ему победу.
\vs Psa 97:2 Явил Господь спасение Свое, открыл пред очами народов правду Свою.
\vs Psa 97:3 Вспомнил Он милость Свою [к Иакову] и верность Свою к дому Израилеву. Все концы земли увидели спасение Бога нашего.
\vs Psa 97:4 Восклицайте Господу, вся земля; торжествуйте, веселитесь и пойте;
\vs Psa 97:5 пойте Господу с гуслями, с гуслями и с гласом псалмопения;
\vs Psa 97:6 при звуке труб и рога торжествуйте пред Царем Господом.
\vs Psa 97:7 Да шумит море и что наполняет его, вселенная и живущие в ней;
\vs Psa 97:8 да рукоплещут реки, да ликуют вместе горы
\vs Psa 97:9 пред лицем Господа, ибо Он идет судить землю. Он будет судить вселенную праведно и народы~--- верно.
\vs Psa 98:0 [Псалом Давида.]
\rsbpar\vs Psa 98:1 Господь царствует: да трепещут народы! Он восседает на Херувимах: да трясется земля!
\vs Psa 98:2 Господь на Сионе велик, и высок Он над всеми народами.
\vs Psa 98:3 Да славят великое и страшное имя Твое: свято оно!
\vs Psa 98:4 И могущество царя любит суд. Ты утвердил справедливость; суд и правду Ты совершил в Иакове.
\vs Psa 98:5 Превозносите Господа, Бога нашего, и поклоняйтесь подножию Его: свято оно!
\vs Psa 98:6 Моисей и Аарон между священниками и Самуил между призывающими имя Его взывали к Господу, и Он внимал им.
\vs Psa 98:7 В столпе облачном говорил Он к ним; они хранили Его заповеди и устав, который Он дал им.
\vs Psa 98:8 Господи, Боже наш! Ты внимал им; Ты был для них Богом прощающим и наказывающим за дела их.
\vs Psa 98:9 Превозносите Господа, Бога нашего, и поклоняйтесь на святой горе Его, ибо свят Господь, Бог наш.
\vs Psa 99:0 Псалом [Давида] хвалебный.
\rsbpar\vs Psa 99:1 Воскликните Господу, вся земля!
\vs Psa 99:2 Слу\-ж\acc{и}\-те Господу с веселием; идите пред лице Его с восклицанием!
\vs Psa 99:3 Познайте, что Господь есть Бог, что Он сотворил нас, и мы~--- Его, Его народ и овцы паствы Его.
\vs Psa 99:4 Входите во врата Его со славословием, во дворы Его~--- с хвалою. Славьте Его, благословляйте имя Его,
\vs Psa 99:5 ибо благ Господь: милость Его вовек, и истина Его в род и род.
\vs Psa 100:0 Псалом Давида.
\rsbpar\vs Psa 100:1 Милость и суд буду петь; Тебе, Господи, буду петь.
\vs Psa 100:2 Буду размышлять о пути непорочном: <<когда ты придешь ко мне?>> Буду ходить в непорочности моего сердца посреди дома моего.
\vs Psa 100:3 Не положу пред очами моими вещи непотребной; дело преступное я ненавижу: не прилепится оно ко мне.
\vs Psa 100:4 Сердце развращенное будет удалено от меня; злого я не буду знать.
\vs Psa 100:5 Тайно клевещущего на ближнего своего изгоню; гордого очами и надменного сердцем не потерплю.
\vs Psa 100:6 Глаза мои на верных земли, чтобы они пребывали при мне; кто ходит путем непорочности, тот будет служить мне.
\vs Psa 100:7 Не будет жить в доме моем поступающий коварно; говорящий ложь не останется пред глазами моими.
\vs Psa 100:8 С раннего утра буду истреблять всех нечестивцев земли, дабы искоренить из града Господня всех делающих беззаконие.
\vs Psa 101:1 Молитва страждущего, когда он унывает и изливает пред Господом печаль свою.
\rsbpar\vs Psa 101:2 Господи! услышь молитву мою, и вопль мой да придет к Тебе.
\vs Psa 101:3 Не скрывай лица Твоего от меня; в день скорби моей приклони ко мне ухо Твое; в день, [когда] воззову [к Тебе], скоро услышь меня;
\vs Psa 101:4 ибо исчезли, как дым, дни мои, и кости мои обожжены, как головня;
\vs Psa 101:5 сердце мое поражено, и иссохло, как трава, так что я забываю есть хлеб мой;
\vs Psa 101:6 от голоса стенания моего кости мои прильпнули к плоти моей.
\vs Psa 101:7 Я уподобился пеликану в пустыне; я стал как филин на развалинах;
\vs Psa 101:8 не сплю и сижу, как одинокая птица на кровле.
\vs Psa 101:9 Всякий день поносят меня враги мои, и злобствующие на меня клянут мною.
\vs Psa 101:10 Я ем пепел, как хлеб, и питье мое растворяю слезами,
\vs Psa 101:11 от гнева Твоего и негодования Твоего, ибо Ты вознес меня и низверг меня.
\vs Psa 101:12 Дни мои~--- как уклоняющаяся тень, и я иссох, как трава.
\vs Psa 101:13 Ты же, Господи, вовек пребываешь, и память о Тебе в род и род.
\vs Psa 101:14 Ты восстанешь, умилосердишься над Сионом, ибо время помиловать его,~--- ибо пришло время;
\vs Psa 101:15 ибо рабы Твои возлюбили и камни его, и о прахе его жалеют.
\vs Psa 101:16 И убоятся народы имени Господня, и все цари земные~--- славы Твоей.
\vs Psa 101:17 Ибо созиждет Господь Сион и явится во славе Своей;
\vs Psa 101:18 призрит на молитву беспомощных и не презрит моления их.
\vs Psa 101:19 Напишется о сем для рода последующего, и поколение грядущее восхвалит Господа,
\vs Psa 101:20 ибо Он приникнул со святой высоты Своей, с небес призрел Господь на землю,
\vs Psa 101:21 чтобы услышать стон узников, разрешить сынов смерти,
\vs Psa 101:22 дабы возвещали на Сионе имя Господне и хвалу Его~--- в Иерусалиме,
\vs Psa 101:23 когда соберутся народы вместе и царства для служения Господу.
\vs Psa 101:24 Изнурил Он на пути силы мои, сократил дни мои.
\vs Psa 101:25 Я сказал: Боже мой! не восхити меня в половине дней моих. Твои лета в роды родов.
\vs Psa 101:26 В начале Ты, [Господи,] основал землю, и небеса~--- дело Твоих рук;
\vs Psa 101:27 они погибнут, а Ты пребудешь; и все они, как риза, обветшают, и, как одежду, Ты переменишь их, и изменятся;
\vs Psa 101:28 но Ты~--- тот же, и лета Твои не кончатся.
\vs Psa 101:29 Сыны рабов Твоих будут жить, и семя их утвердится пред лицем Твоим.
\vs Psa 102:0 Псалом Давида.
\rsbpar\vs Psa 102:1 Благослови, душа моя, Господа, и вся внутренность моя~--- святое имя Его.
\vs Psa 102:2 Благослови, душа моя, Господа и не забывай всех благодеяний Его.
\vs Psa 102:3 Он прощает все беззакония твои, исцеляет все недуги твои;
\vs Psa 102:4 избавляет от могилы жизнь твою, венчает тебя милостью и щедротами;
\vs Psa 102:5 насыщает благами желание твое: обновляется, подобно орлу, юность твоя.
\vs Psa 102:6 Господь творит правду и суд всем обиженным.
\vs Psa 102:7 Он показал пути Свои Моисею, сынам Израилевым~--- дела Свои.
\vs Psa 102:8 Щедр и милостив Господь, долготерпелив и многомилостив:
\vs Psa 102:9 не до конца гневается, и не вовек негодует.
\vs Psa 102:10 Не по беззакониям нашим сотворил нам, и не по грехам нашим воздал нам:
\vs Psa 102:11 ибо как высоко небо над землею, так велика милость [Господа] к боящимся Его;
\vs Psa 102:12 как далеко восток от запада, так удалил Он от нас беззакония наши;
\vs Psa 102:13 как отец милует сынов, так милует Господь боящихся Его.
\vs Psa 102:14 Ибо Он знает состав наш, помнит, что мы~--- персть.
\vs Psa 102:15 Дни человека~--- как трава; как цвет полевой, так он цветет.
\vs Psa 102:16 Пройдет над ним ветер, и нет его, и место его уже не узнает его.
\vs Psa 102:17 Милость же Господня от века и до века к боящимся Его,
\vs Psa 102:18 и правда Его на сынах сынов, хранящих завет Его и помнящих заповеди Его, чтобы исполнять их.
\vs Psa 102:19 Господь на небесах поставил престол Свой, и царство Его всем обладает.
\vs Psa 102:20 Благословите Господа, [все] Ангелы Его, крепкие силою, исполняющие слово Его, повинуясь гласу слова Его;
\vs Psa 102:21 благословите Господа, все воинства Его, служители Его, исполняющие волю Его;
\vs Psa 102:22 благословите Господа, все дела Его, во всех местах владычества Его. Благослови, душа моя, Господа!
\vs Psa 103:0 [Псалом Давида о сотворении мира.]
\rsbpar\vs Psa 103:1 Благослови, душа моя, Господа! Господи, Боже мой! Ты дивно велик, Ты облечен славою и величием;
\vs Psa 103:2 Ты одеваешься светом, как ризою, простираешь небеса, как шатер;
\vs Psa 103:3 устрояешь над водами горние чертоги Твои, делаешь облака Твоею колесницею, шествуешь на крыльях ветра.
\vs Psa 103:4 Ты творишь ангелами Твоими духов, служителями Твоими~--- огонь пылающий.
\vs Psa 103:5 Ты поставил землю на твердых основах: не поколеблется она во веки и веки.
\vs Psa 103:6 Бездною, как одеянием, покрыл Ты ее, на горах стоят воды.
\vs Psa 103:7 От прещения Твоего бегут они, от гласа грома Твоего быстро уходят;
\vs Psa 103:8 восходят на горы, нисходят в долины, на место, которое Ты назначил для них.
\vs Psa 103:9 Ты положил предел, которого не перейдут, и не возвратятся покрыть землю.
\vs Psa 103:10 Ты послал источники в долины: между горами текут [воды],
\vs Psa 103:11 поят всех полевых зверей; дикие ослы утоляют жажду свою.
\vs Psa 103:12 При них обитают птицы небесные, из среды ветвей издают голос.
\vs Psa 103:13 Ты напояешь горы с высот Твоих, плодами дел Твоих насыщается земля.
\vs Psa 103:14 Ты произращаешь траву для скота, и зелень на пользу человека, чтобы произвести из земли пищу,
\vs Psa 103:15 и вино, которое веселит сердце человека, и елей, от которого блистает лице его, и хлеб, который укрепляет сердце человека.
\vs Psa 103:16 Насыщаются древа Господа, кедры Ливанские, которые Он насадил;
\vs Psa 103:17 на них гнездятся птицы: ели~--- жилище аисту,
\vs Psa 103:18 высокие горы~--- сернам; каменные утесы~--- убежище зайцам.
\vs Psa 103:19 Он сотворил луну для \bibemph{указания} времен, солнце знает свой запад.
\vs Psa 103:20 Ты простираешь тьму и бывает ночь: во время нее бродят все лесные звери;
\vs Psa 103:21 львы рыкают о добыче и просят у Бога пищу себе.
\vs Psa 103:22 Восходит солнце, [и] они собираются и ложатся в свои логовища;
\vs Psa 103:23 выходит человек на дело свое и на работу свою до вечера.
\vs Psa 103:24 Как многочисленны дела Твои, Господи! Все соделал Ты премудро; земля полна произведений Твоих.
\vs Psa 103:25 Это~--- море великое и пространное: там пресмыкающиеся, которым нет числа, животные малые с большими;
\vs Psa 103:26 там плавают корабли, там этот левиафан, которого Ты сотворил играть в нем.
\vs Psa 103:27 Все они от Тебя ожидают, чтобы Ты дал им пищу их в свое время.
\vs Psa 103:28 Даешь им~--- принимают, отверзаешь руку Твою~--- насыщаются благом;
\vs Psa 103:29 скроешь лице Твое~--- мятутся, отнимешь дух их~--- умирают и в персть свою возвращаются;
\vs Psa 103:30 пошлешь дух Твой~--- созидаются, и Ты обновляешь лице земли.
\vs Psa 103:31 Да будет Господу слава во веки; да веселится Господь о делах Своих!
\vs Psa 103:32 Призирает на землю, и она трясется; прикасается к горам, и дымятся.
\vs Psa 103:33 Буду петь Господу во \bibemph{всю} жизнь мою, буду петь Богу моему, доколе есмь.
\vs Psa 103:34 Да будет благоприятна Ему песнь моя; буду веселиться о Господе.
\vs Psa 103:35 Да исчезнут грешники с земли, и беззаконных да не будет более. Благослови, душа моя, Господа! Аллилуия!
\vs Psa 104:1 Славьте Господа; призывайте имя Его; возвещайте в народах дела Его;
\vs Psa 104:2 воспойте Ему и пойте Ему; поведайте о всех чудесах Его.
\vs Psa 104:3 Хвалитесь именем Его святым; да веселится сердце ищущих Господа.
\vs Psa 104:4 Ищите Господа и силы Его, ищите лица Его всегда.
\vs Psa 104:5 Воспоминайте чудеса Его, которые сотворил, знамения Его и суды уст Его,
\vs Psa 104:6 вы, семя Авраамово, рабы Его, сыны Иакова, избранные Его.
\vs Psa 104:7 Он Господь Бог наш: по всей земле суды Его.
\vs Psa 104:8 Вечно помнит завет Свой, слово, [которое] заповедал в тысячу родов,
\vs Psa 104:9 которое завещал Аврааму, и клятву Свою Исааку,
\vs Psa 104:10 и поставил то Иакову в закон и Израилю в завет вечный,
\vs Psa 104:11 говоря: <<тебе дам землю Ханаанскую в удел наследия вашего>>.
\vs Psa 104:12 Когда их было еще мало числом, очень мало, и они были пришельцами в ней
\vs Psa 104:13 и переходили от народа к народу, из царства к иному племени,
\vs Psa 104:14 никому не позволял обижать их и возбранял о них царям:
\vs Psa 104:15 <<не прикасайтесь к помазанным Моим, и пророкам Моим не делайте зла>>.
\vs Psa 104:16 И призвал голод на землю; всякий стебель хлебный истребил.
\vs Psa 104:17 Послал пред ними человека: в рабы продан был Иосиф.
\vs Psa 104:18 Стеснили оковами ноги его; в железо вошла душа его,
\vs Psa 104:19 доколе исполнилось слово Его: слово Господне испытало его.
\vs Psa 104:20 Послал царь, и разрешил его владетель народов и освободил его;
\vs Psa 104:21 поставил его господином над домом своим и правителем над всем владением своим,
\vs Psa 104:22 чтобы он наставлял вельмож его по своей душе и старейшин его учил мудрости.
\vs Psa 104:23 Тогда пришел Израиль в Египет, и переселился Иаков в землю Хамову.
\vs Psa 104:24 И весьма размножил \bibemph{Бог} народ Свой и сделал его сильнее врагов его.
\vs Psa 104:25 Возбудил в сердце их ненависть против народа Его и ухищрение против рабов Его.
\vs Psa 104:26 Послал Моисея, раба Своего, Аарона, которого избрал.
\vs Psa 104:27 Они показали между ними слова знамений Его и чудеса [Его] в земле Хамовой.
\vs Psa 104:28 Послал тьму и сделал мрак, и не воспротивились слову Его.
\vs Psa 104:29 Преложил воду их в кровь, и уморил рыбу их.
\vs Psa 104:30 Земля их произвела множество жаб \bibemph{даже} в спальне царей их.
\vs Psa 104:31 Он сказал, и пришли разные насекомые, скнипы во все пределы их.
\vs Psa 104:32 Вместо дождя послал на них град, палящий огонь на землю их,
\vs Psa 104:33 и побил виноград их и смоковницы их, и сокрушил дерева в пределах их.
\vs Psa 104:34 Сказал, и пришла саранча и гусеницы без числа;
\vs Psa 104:35 и съели всю траву на земле их, и съели плоды на полях их.
\vs Psa 104:36 И поразил всякого первенца в земле их, начатки всей силы их.
\vs Psa 104:37 И вывел \bibemph{Израильтян} с серебром и золотом, и не было в коленах их болящего.
\vs Psa 104:38 Обрадовался Египет исшествию их; ибо страх от них напал на него.
\vs Psa 104:39 Простер облако в покров [им] и огонь, чтобы светить [им] ночью.
\vs Psa 104:40 Просили, и Он послал перепелов, и хлебом небесным насыщал их.
\vs Psa 104:41 Разверз камень, и потекли воды, потекли рекою по местам сухим,
\vs Psa 104:42 ибо вспомнил Он святое слово Свое к Аврааму, рабу Своему,
\vs Psa 104:43 и вывел народ Свой в радости, избранных Своих в веселии,
\vs Psa 104:44 и дал им земли народов, и они наследовали труд иноплеменных,
\vs Psa 104:45 чтобы соблюдали уставы Его и хранили законы Его. Аллилуия!
\vs Psa 105:0 Аллилуия.
\rsbpar\vs Psa 105:1 Славьте Господа, ибо Он благ, ибо вовек милость Его.
\vs Psa 105:2 Кто изречет могущество Господа, возвестит все хвалы Его?
\vs Psa 105:3 Блаженны хранящие суд и творящие правду во всякое время!
\vs Psa 105:4 Вспомни о мне, Господи, в благоволении к народу Твоему; посети меня спасением Твоим,
\vs Psa 105:5 дабы мне видеть благоденствие избранных Твоих, веселиться веселием народа Твоего, хвалиться с наследием Твоим.
\vs Psa 105:6 Согрешили мы с отцами нашими, совершили беззаконие, соделали неправду.
\vs Psa 105:7 Отцы наши в Египте не уразумели чудес Твоих, не помнили множества милостей Твоих, и возмутились у моря, у Чермного моря.
\vs Psa 105:8 Но Он спас их ради имени Своего, дабы показать могущество Свое.
\vs Psa 105:9 Грозно рек морю Чермному, и оно иссохло; и провел их по безднам, как по суше;
\vs Psa 105:10 и спас их от руки ненавидящего и избавил их от руки врага.
\vs Psa 105:11 Воды покрыли врагов их, ни одного из них не осталось.
\vs Psa 105:12 И поверили они словам Его, [и] воспели хвалу Ему.
\vs Psa 105:13 \bibemph{Но} скоро забыли дела Его, не дождались Его изволения;
\vs Psa 105:14 увлеклись похотением в пустыне, и искусили Бога в необитаемой.
\vs Psa 105:15 И Он исполнил прошение их, \bibemph{но} послал язву на души их.
\vs Psa 105:16 И позавидовали в стане Моисею \bibemph{и} Аарону, святому Господню.
\vs Psa 105:17 Разверзлась земля, и поглотила Дафана и покрыла скопище Авирона.
\vs Psa 105:18 И возгорелся огонь в скопище их, пламень попалил нечестивых.
\vs Psa 105:19 Сделали тельца у Хорива и поклонились истукану;
\vs Psa 105:20 и променяли славу свою на изображение вола, ядущего траву.
\vs Psa 105:21 Забыли Бога, Спасителя своего, совершившего великое в Египте,
\vs Psa 105:22 дивное в земле Хамовой, страшное у Чермного моря.
\vs Psa 105:23 И хотел истребить их, если бы Моисей, избранный Его, не стал пред Ним в расселине, чтобы отвратить ярость Его, да не погубит [их].
\vs Psa 105:24 И презрели они землю желанную, не верили слову Его;
\vs Psa 105:25 и роптали в шатрах своих, не слушались гласа Господня.
\vs Psa 105:26 И поднял Он руку Свою на них, чтобы низложить их в пустыне,
\vs Psa 105:27 низложить племя их в народах и рассеять их по землям.
\vs Psa 105:28 Они прилепились к Ваалфегору и ели жертвы бездушным,
\vs Psa 105:29 и раздражали \bibemph{Бога} делами своими, и вторглась к ним язва.
\vs Psa 105:30 И восстал Финеес и произвел суд,~--- и остановилась язва.
\vs Psa 105:31 И \bibemph{это} вменено ему в праведность в роды и роды во веки.
\vs Psa 105:32 И прогневали \bibemph{Бога} у вод Меривы, и Моисей потерпел за них,
\vs Psa 105:33 ибо они огорчили дух его, и он погрешил устами своими.
\vs Psa 105:34 Не истребили народов, о которых сказал им Господь,
\vs Psa 105:35 но смешались с язычниками и научились делам их;
\vs Psa 105:36 служили истуканам их, \bibemph{которые} были для них сетью,
\vs Psa 105:37 и приносили сыновей своих и дочерей своих в жертву бесам;
\vs Psa 105:38 проливали кровь невинную, кровь сыновей своих и дочерей своих, которых приносили в жертву идолам Ханаанским,~--- и осквернилась земля кровью;
\vs Psa 105:39 оскверняли себя делами своими, блудодействовали поступками своими.
\vs Psa 105:40 И воспылал гнев Господа на народ Его, и возгнушался Он наследием Своим
\vs Psa 105:41 и предал их в руки язычников, и ненавидящие их стали обладать ими.
\vs Psa 105:42 Враги их утесняли их, и они смирялись под рукою их.
\vs Psa 105:43 Много раз Он избавлял их; они же раздражали [Его] упорством своим, и были уничижаемы за беззаконие свое.
\vs Psa 105:44 Но Он призирал на скорбь их, когда слышал вопль их,
\vs Psa 105:45 и вспоминал завет Свой с ними и раскаивался по множеству милости Своей;
\vs Psa 105:46 и возбуждал к ним сострадание во всех, пленявших их.
\vs Psa 105:47 Спаси нас, Господи, Боже наш, и собери нас от народов, дабы славить святое имя Твое, хвалиться Твоею славою.
\vs Psa 105:48 Благословен Господь, Бог Израилев, от века и до века! И да скажет весь народ: аминь! Аллилуия!
\vs Psa 106:0 [Аллилуия.]
\rsbpar\vs Psa 106:1 Славьте Господа, ибо Он благ, ибо вовек милость Его!
\vs Psa 106:2 Так да скажут избавленные Господом, которых избавил Он от руки врага,
\vs Psa 106:3 и собрал от стран, от востока и запада, от севера и моря.
\vs Psa 106:4 Они блуждали в пустыне по безлюдному пути и не находили населенного города;
\vs Psa 106:5 терпели голод и жажду, душа их истаевала в них.
\vs Psa 106:6 Но воззвали к Господу в скорби своей, и Он избавил их от бедствий их,
\vs Psa 106:7 и повел их прямым путем, чтобы они шли к населенному городу.
\vs Psa 106:8 Да славят Господа за милость Его и за чудные дела Его для сынов человеческих:
\vs Psa 106:9 ибо Он насытил душу жаждущую и душу алчущую исполнил благами.
\vs Psa 106:10 Они сидели во тьме и тени смертной, окованные скорбью и железом;
\vs Psa 106:11 ибо не покорялись словам Божиим и небрегли о воле Всевышнего.
\vs Psa 106:12 Он смирил сердце их работами; они преткнулись, и не было помогающего.
\vs Psa 106:13 Но воззвали к Господу в скорби своей, и Он спас их от бедствий их;
\vs Psa 106:14 вывел их из тьмы и тени смертной, и расторгнул узы их.
\vs Psa 106:15 Да славят Господа за милость Его и за чудные дела Его для сынов человеческих:
\vs Psa 106:16 ибо Он сокрушил врата медные и вереи железные сломил.
\vs Psa 106:17 Безрассудные страдали за беззаконные пути свои и за неправды свои;
\vs Psa 106:18 от всякой пищи отвращалась душа их, и они приближались ко вратам смерти.
\vs Psa 106:19 Но воззвали к Господу в скорби своей, и Он спас их от бедствий их;
\vs Psa 106:20 послал слово Свое и исцелил их, и избавил их от могил их.
\vs Psa 106:21 Да славят Господа за милость Его и за чудные дела Его для сынов человеческих!
\vs Psa 106:22 Да приносят Ему жертву хвалы и да возвещают о делах Его с пением!
\vs Psa 106:23 Отправляющиеся на кораблях в море, производящие дела на больших водах,
\vs Psa 106:24 видят дела Господа и чудеса Его в пучине:
\vs Psa 106:25 Он речет,~--- и восстает бурный ветер и высоко поднимает волны его:
\vs Psa 106:26 восходят до небес, нисходят до бездны; душа их истаевает в бедствии;
\vs Psa 106:27 они кружатся и шатаются, как пьяные, и вся мудрость их исчезает.
\vs Psa 106:28 Но воззвали к Господу в скорби своей, и Он вывел их из бедствия их.
\vs Psa 106:29 Он превращает бурю в тишину, и волны умолкают.
\vs Psa 106:30 И веселятся, что они утихли, и Он приводит их к желаемой пристани.
\vs Psa 106:31 Да славят Господа за милость Его и за чудные дела Его для сынов человеческих!
\vs Psa 106:32 Да превозносят Его в собрании народном и да славят Его в сонме старейшин!
\vs Psa 106:33 Он превращает реки в пустыню и источники вод~--- в сушу,
\vs Psa 106:34 землю плодородную~--- в солончатую, за нечестие живущих на ней.
\vs Psa 106:35 Он превращает пустыню в озеро, и землю иссохшую~--- в источники вод;
\vs Psa 106:36 и поселяет там алчущих, и они строят город для обитания;
\vs Psa 106:37 засевают поля, насаждают виноградники, которые приносят им обильные плоды.
\vs Psa 106:38 Он благословляет их, и они весьма размножаются, и скота их не умаляет.
\vs Psa 106:39 Уменьшились они и упали от угнетения, бедствия и скорби,~---
\vs Psa 106:40 Он изливает бесчестие на князей и оставляет их блуждать в пустыне, где нет путей.
\vs Psa 106:41 Бедного же извлекает из бедствия и умножает род его, как стада овец.
\vs Psa 106:42 Праведники видят сие и радуются, а всякое нечестие заграждает уста свои.
\vs Psa 106:43 Кто мудр, тот заметит сие и уразумеет милость Господа.
\vs Psa 107:1 Песнь. Псалом Давида.
\rsbpar\vs Psa 107:2 Готово сердце мое, Боже, [готово сердце мое]; буду петь и воспевать во славе моей.
\vs Psa 107:3 Воспрянь, псалтирь и гусли! Я встану рано.
\vs Psa 107:4 Буду славить Тебя, Господи, между народами; буду воспевать Тебя среди племен,
\vs Psa 107:5 ибо превыше небес милость Твоя и до облаков истина Твоя.
\vs Psa 107:6 Будь превознесен выше небес, Боже; над всею землею \bibemph{да будет} слава Твоя,
\vs Psa 107:7 дабы избавились возлюбленные Твои: спаси десницею Твоею и услышь меня.
\vs Psa 107:8 Бог сказал во святилище Своем: <<восторжествую, разделю Сихем и долину Сокхоф размерю;
\vs Psa 107:9 Мой Галаад, Мой Манассия, Ефрем~--- крепость главы Моей, Иуда~--- скипетр Мой,
\vs Psa 107:10 Моав~--- умывальная чаша Моя, на Едома простру сапог Мой, над землею Филистимскою восклицать буду>>.
\vs Psa 107:11 Кто введет меня в укрепленный город? Кто доведет меня до Едома?
\vs Psa 107:12 Не Ты ли, Боже, \bibemph{Который} отринул нас и не выходишь, Боже, с войсками нашими?
\vs Psa 107:13 Подай нам помощь в тесноте, ибо защита человеческая суетна.
\vs Psa 107:14 С Богом мы окажем силу: Он низложит врагов наших.
\vs Psa 108:0 Начальнику хора. Псалом Давида.
\rsbpar\vs Psa 108:1 Боже хвалы моей! не премолчи,
\vs Psa 108:2 ибо отверзлись на меня уста нечестивые и уста коварные; говорят со мною языком лживым;
\vs Psa 108:3 отвсюду окружают меня словами ненависти, вооружаются против меня без причины;
\vs Psa 108:4 за любовь мою они враждуют на меня, а я молюсь;
\vs Psa 108:5 воздают мне за добро злом, за любовь мою~--- ненавистью.
\vs Psa 108:6 Поставь над ним нечестивого, и диавол да станет одесную его.
\vs Psa 108:7 Когда будет судиться, да выйдет виновным, и молитва его да будет в грех;
\vs Psa 108:8 да будут дни его кратки, и достоинство его да возьмет другой;
\vs Psa 108:9 дети его да будут сиротами, и жена его~--- вдовою;
\vs Psa 108:10 да скитаются дети его и нищенствуют, и просят \bibemph{хлеба} из развалин своих;
\vs Psa 108:11 да захватит заимодавец все, что есть у него, и чужие да расхитят труд его;
\vs Psa 108:12 да не будет сострадающего ему, да не будет милующего сирот его;
\vs Psa 108:13 да будет потомство его на погибель, и да изгладится имя их в следующем роде;
\vs Psa 108:14 да будет воспомянуто пред Господом беззаконие отцов его, и грех матери его да не изгладится;
\vs Psa 108:15 да будут они всегда в очах Господа, и да истребит Он память их на земле,
\vs Psa 108:16 за то, что он не думал оказывать милость, но преследовал человека бедного и нищего и сокрушенного сердцем, чтобы умертвить его;
\vs Psa 108:17 возлюбил проклятие,~--- оно и придет на него; не восхотел благословения,~--- оно и удалится от него;
\vs Psa 108:18 да облечется проклятием, как ризою, и да войдет оно, как вода, во внутренность его и, как елей, в кости его;
\vs Psa 108:19 да будет оно ему, как одежда, в которую он одевается, и как пояс, которым всегда опоясывается.
\vs Psa 108:20 Таково воздаяние от Господа врагам моим и говорящим злое на душу мою!
\vs Psa 108:21 Со мною же, Господи, Господи, твори ради имени Твоего, ибо блага милость Твоя; спаси меня,
\vs Psa 108:22 ибо я беден и нищ, и сердце мое уязвлено во мне.
\vs Psa 108:23 Я исчезаю, как уклоняющаяся тень; гонят меня, как саранчу.
\vs Psa 108:24 Колени мои изнемогли от поста, и тело мое лишилось тука.
\vs Psa 108:25 Я стал для них посмешищем: увидев меня, кивают головами [своими].
\vs Psa 108:26 Помоги мне, Господи, Боже мой, спаси меня по милости Твоей,
\vs Psa 108:27 да познают, что это~--- Твоя рука, и что Ты, Господи, соделал это.
\vs Psa 108:28 Они проклинают, а Ты благослови; они восстают, но да будут постыжены; раб же Твой да возрадуется.
\vs Psa 108:29 Да облекутся противники мои бесчестьем и, как одеждою, покроются стыдом своим.
\vs Psa 108:30 И я громко буду устами моими славить Господа и среди множества прославлять Его,
\vs Psa 108:31 ибо Он стоит одесную бедного, чтобы спасти его от судящих душу его.
\vs Psa 109:0 Псалом Давида.
\rsbpar\vs Psa 109:1 Сказал Господь Господу моему: седи одесную Меня, доколе положу врагов Твоих в подножие ног Твоих.
\vs Psa 109:2 Жезл силы Твоей пошлет Господь с Сиона: господствуй среди врагов Твоих.
\vs Psa 109:3 В день силы Твоей народ Твой готов во благолепии святыни; из чрева прежде денницы подобно росе рождение Твое\fns{По переводу 70-ти: из чрева прежде денницы Я родил Тебя.}.
\vs Psa 109:4 Клялся Господь и не раскается: Ты священник вовек по чину Мелхиседека.
\vs Psa 109:5 Господь одесную Тебя. Он в день гнева Своего поразит царей;
\vs Psa 109:6 совершит суд над народами, наполнит \bibemph{землю} трупами, сокрушит голову в земле обширной.
\vs Psa 109:7 Из потока на пути будет пить, и потому вознесет главу.
\vs Psa 110:0 Аллилуия.
\rsbpar\vs Psa 110:1 Славлю [Тебя], Господи, всем сердцем [моим] в совете праведных и в собрании.
\vs Psa 110:2 Велики дела Господни, вожделенны для всех, любящих оные.
\vs Psa 110:3 Дело Его~--- слава и красота, и правда Его пребывает вовек.
\vs Psa 110:4 Памятными соделал Он чудеса Свои; милостив и щедр Господь.
\vs Psa 110:5 Пищу дает боящимся Его; вечно помнит завет Свой.
\vs Psa 110:6 Силу дел Своих явил Он народу Своему, чтобы дать ему наследие язычников.
\vs Psa 110:7 Дела рук Его~--- истина и суд; все заповеди Его верны,
\vs Psa 110:8 тверды на веки и веки, основаны на истине и правоте.
\vs Psa 110:9 Избавление послал Он народу Своему; заповедал на веки завет Свой. Свято и страшно имя Его!
\vs Psa 110:10 Начало мудрости~--- страх Господень; разум верный у всех, исполняющих \bibemph{заповеди Его}. Хвала Ему пребудет вовек.
\vs Psa 111:0 Аллилуия.
\rsbpar\vs Psa 111:1 Блажен муж, боящийся Господа и крепко любящий заповеди Его.
\vs Psa 111:2 Сильно будет на земле семя его; род правых благословится.
\vs Psa 111:3 Обилие и богатство в доме его, и правда его пребывает вовек.
\vs Psa 111:4 Во тьме восходит свет правым; благ он и милосерд и праведен.
\vs Psa 111:5 Добрый человек милует и взаймы дает; он даст твердость словам своим на суде.
\vs Psa 111:6 Он вовек не поколеблется; в вечной памяти будет праведник.
\vs Psa 111:7 Не убоится худой молвы: сердце его твердо, уповая на Господа.
\vs Psa 111:8 Утверждено сердце его: он не убоится, когда посмотрит на врагов своих.
\vs Psa 111:9 Он расточил, раздал нищим; правда его пребывает во веки; рог его вознесется во славе.
\vs Psa 111:10 Нечестивый увидит \bibemph{это} и будет досадовать, заскрежещет зубами своими и истает. Желание нечестивых погибнет.
\vs Psa 112:0 Аллилуия.
\rsbpar\vs Psa 112:1 Хвалите, рабы Господни, хвалите имя Господне.
\vs Psa 112:2 Да будет имя Господне благословенно отныне и вовек.
\vs Psa 112:3 От восхода солнца до запада \bibemph{да будет} прославляемо имя Господне.
\vs Psa 112:4 Высок над всеми народами Господь; над небесами слава Его.
\vs Psa 112:5 Кто, как Господь, Бог наш, Который, обитая на высоте,
\vs Psa 112:6 приклоняется, чтобы призирать на небо и на землю;
\vs Psa 112:7 из праха поднимает бедного, из брения возвышает нищего,
\vs Psa 112:8 чтобы посадить его с князьями, с князьями народа его;
\vs Psa 112:9 неплодную вселяет в дом матерью, радующеюся о детях? Аллилуия!
\vs Psa 113:0 [Аллилуия.]
\rsbpar\vs Psa 113:1 Когда вышел Израиль из Египта, дом Иакова~--- из народа иноплеменного,
\vs Psa 113:2 Иуда сделался святынею Его, Израиль~--- владением Его.
\vs Psa 113:3 Море увидело и побежало; Иордан обратился назад.
\vs Psa 113:4 Горы прыгали, как овны, и холмы, как агнцы.
\vs Psa 113:5 Что с тобою, море, что ты побежало, и [с тобою], Иордан, что ты обратился назад?
\vs Psa 113:6 Что вы прыгаете, горы, как овны, и вы, холмы, как агнцы?
\vs Psa 113:7 Пред лицем Господа трепещи, земля, пред лицем Бога Иаковлева,
\vs Psa 113:8 превращающего скалу в озеро воды и камень в источник вод.
\vs Psa 113:9 Не нам, Господи, не нам, но имени Твоему дай славу, ради милости Твоей, ради истины Твоей.
\vs Psa 113:10 Для чего язычникам говорить: <<где же Бог их>>?
\vs Psa 113:11 Бог наш на небесах [и на земле]; творит все, что хочет.
\vs Psa 113:12 А их идолы~--- серебро и золото, дело рук человеческих.
\vs Psa 113:13 Есть у них уста, но не говорят; есть у них глаза, но не видят;
\vs Psa 113:14 есть у них уши, но не слышат; есть у них ноздри, но не обоняют;
\vs Psa 113:15 есть у них руки, но не осязают; есть у них ноги, но не ходят; и они не издают голоса гортанью своею.
\vs Psa 113:16 Подобны им да будут делающие их и все, надеющиеся на них.
\vs Psa 113:17 [Дом] Израилев! уповай на Господа: Он наша помощь и щит.
\vs Psa 113:18 Дом Ааронов! уповай на Господа: Он наша помощь и щит.
\vs Psa 113:19 Боящиеся Господа! уповайте на Господа: Он наша помощь и щит.
\vs Psa 113:20 Господь помнит нас, благословляет [нас], благословляет дом Израилев, благословляет дом Ааронов;
\vs Psa 113:21 благословляет боящихся Господа, малых с великими.
\vs Psa 113:22 Да приложит вам Господь более и более, вам и детям вашим.
\vs Psa 113:23 Благословенны вы Господом, сотворившим небо и землю.
\vs Psa 113:24 Небо~--- небо Господу, а землю Он дал сынам человеческим.
\vs Psa 113:25 Ни мертвые восхвалят Господа, ни все нисходящие в могилу;
\vs Psa 113:26 но мы [живые] будем благословлять Господа отныне и вовек. Аллилуия.
\vs Psa 114:0 [Аллилуия.]
\rsbpar\vs Psa 114:1 Я радуюсь, что Господь услышал голос мой, моление мое;
\vs Psa 114:2 приклонил ко мне ухо Свое, и потому буду призывать Его во \bibemph{все} дни мои.
\vs Psa 114:3 Объяли меня болезни смертные, муки адские постигли меня; я встретил тесноту и скорбь.
\vs Psa 114:4 Тогда призвал я имя Господне: Господи! избавь душу мою.
\vs Psa 114:5 Милостив Господь и праведен, и милосерд Бог наш.
\vs Psa 114:6 Хранит Господь простодушных: я изнемог, и Он помог мне.
\vs Psa 114:7 Возвратись, душа моя, в покой твой, ибо Господь облагодетельствовал тебя.
\vs Psa 114:8 Ты избавил душу мою от смерти, очи мои от слез и ноги мои от преткновения. Буду ходить пред лицем Господним на земле живых.
\vs Psa 115:0 [Аллилуия.]
\rsbpar\vs Psa 115:1 Я веровал, и потому говорил: я сильно сокрушен.
\vs Psa 115:2 Я сказал в опрометчивости моей: всякий человек ложь.
\vs Psa 115:3 Что воздам Господу за все благодеяния Его ко мне?
\vs Psa 115:4 Чашу спасения прииму и имя Господне призову.
\vs Psa 115:5 Обеты мои воздам Господу пред всем народом Его.
\vs Psa 115:6 Дорог\acc{а} в очах Господних смерть святых Его!
\vs Psa 115:7 О, Господи! я раб Твой, я раб Твой и сын рабы Твоей; Ты разрешил узы мои.
\vs Psa 115:8 Тебе принесу жертву хвалы, и имя Господне призову.
\vs Psa 115:9 Обеты мои воздам Господу пред всем народом Его,
\vs Psa 115:10 во дворах дома Господня, посреди тебя, Иерусалим! Аллилуия.
\vs Psa 116:0 [Аллилуия.]
\rsbpar\vs Psa 116:1 Хвалите Господа, все народы, прославляйте Его, все племена;
\vs Psa 116:2 ибо велика милость Его к нам, и истина Господня [пребывает] вовек. Аллилуия.
\vs Psa 117:0 [Аллилуия.]
\rsbpar\vs Psa 117:1 Славьте Господа, ибо Он благ, ибо вовек милость Его.
\vs Psa 117:2 Да скажет ныне [дом] Израилев: [Он благ,] ибо вовек милость Его.
\vs Psa 117:3 Да скажет ныне дом Ааронов: [Он благ,] ибо вовек милость Его.
\vs Psa 117:4 Да скажут ныне боящиеся Господа: [Он благ,] ибо вовек милость Его.
\vs Psa 117:5 Из тесноты воззвал я к Господу,~--- и услышал меня, и на пространное место \bibemph{вывел меня} Господь.
\vs Psa 117:6 Господь за меня~--- не устрашусь: что сделает мне человек?
\vs Psa 117:7 Господь мне помощник: буду смотреть на врагов моих.
\vs Psa 117:8 Лучше уповать на Господа, нежели надеяться на человека.
\vs Psa 117:9 Лучше уповать на Господа, нежели надеяться на князей.
\vs Psa 117:10 Все народы окружили меня, но именем Господним я низложил их;
\vs Psa 117:11 обступили меня, окружили меня, но именем Господним я низложил их;
\vs Psa 117:12 окружили меня, как пчелы [сот], и угасли, как огонь в терне: именем Господним я низложил их.
\vs Psa 117:13 Сильно толкнули меня, чтобы я упал, но Господь поддержал меня.
\vs Psa 117:14 Господь~--- сила моя и песнь; Он соделался моим спасением.
\vs Psa 117:15 Глас радости и спасения в жилищах праведников: десница Господня творит силу!
\vs Psa 117:16 Десница Господня высока, десница Господня творит силу!
\vs Psa 117:17 Не умру, но буду жить и возвещать дела Господни.
\vs Psa 117:18 Строго наказал меня Господь, но смерти не предал меня.
\vs Psa 117:19 Отворите мне врата правды; войду в них, прославлю Господа.
\vs Psa 117:20 Вот врата Господа; праведные войдут в них.
\vs Psa 117:21 Славлю Тебя, что Ты услышал меня и соделался моим спасением.
\vs Psa 117:22 Камень, который отвергли строители, соделался главою угла:
\vs Psa 117:23 это~--- от Господа, и есть дивно в очах наших.
\vs Psa 117:24 Сей день сотворил Господь: возрадуемся и возвеселимся в оный!
\vs Psa 117:25 О, Господи, спаси же! О, Господи, споспешествуй же!
\vs Psa 117:26 Благословен грядущий во имя Господне! Благословляем вас из дома Господня.
\vs Psa 117:27 Бог~--- Господь, и осиял нас; вяжите вервями жертву, \bibemph{ведите} к рогам жертвенника.
\vs Psa 117:28 Ты Бог мой: буду славить Тебя; Ты Бог мой: буду превозносить Тебя, [буду славить Тебя, ибо Ты услышал меня и соделался моим спасением].
\vs Psa 117:29 Славьте Господа, ибо Он благ, ибо вовек милость Его.
\vs Psa 118:0 [Аллилуия.]
\rsbpar\vs Psa 118:1 Блаженны непорочные в пути, ходящие в законе Господнем.
\vs Psa 118:2 Блаженны хранящие откровения Его, всем сердцем ищущие Его.
\vs Psa 118:3 Они не делают беззакония, ходят путями Его.
\vs Psa 118:4 Ты заповедал повеления Твои хранить твердо.
\vs Psa 118:5 О, если бы направлялись пути мои к соблюдению уставов Твоих!
\vs Psa 118:6 Тогда я не постыдился бы, взирая на все заповеди Твои:
\vs Psa 118:7 я славил бы Тебя в правоте сердца, поучаясь судам правды Твоей.
\vs Psa 118:8 Буду хранить уставы Твои; не оставляй меня совсем.
\vs Psa 118:9 Как юноше содержать в чистоте путь свой?~--- Хранением себя по слову Твоему.
\vs Psa 118:10 Всем сердцем моим ищу Тебя; не дай мне уклониться от заповедей Твоих.
\vs Psa 118:11 В сердце моем сокрыл я слово Твое, чтобы не грешить пред Тобою.
\vs Psa 118:12 Благословен Ты, Господи! научи меня уставам Твоим.
\vs Psa 118:13 Устами моими возвещал я все суды уст Твоих.
\vs Psa 118:14 На пути откровений Твоих я радуюсь, как во всяком богатстве.
\vs Psa 118:15 О заповедях Твоих размышляю, и взираю на пути Твои.
\vs Psa 118:16 Уставами Твоими утешаюсь, не забываю слова Твоего.
\vs Psa 118:17 Яви милость рабу Твоему, и буду жить и хранить слово Твое.
\vs Psa 118:18 Открой очи мои, и увижу чудеса закона Твоего.
\vs Psa 118:19 Странник я на земле; не скрывай от меня заповедей Твоих.
\vs Psa 118:20 Истомилась душа моя желанием судов Твоих во всякое время.
\vs Psa 118:21 Ты укротил гордых, проклятых, уклоняющихся от заповедей Твоих.
\vs Psa 118:22 Сними с меня поношение и посрамление, ибо я храню откровения Твои.
\vs Psa 118:23 Князья сидят и сговариваются против меня, а раб Твой размышляет об уставах Твоих.
\vs Psa 118:24 Откровения Твои~--- утешение мое, [и уставы Твои]~--- советники мои.
\vs Psa 118:25 Душа моя повержена в прах; оживи меня по слову Твоему.
\vs Psa 118:26 Объявил я пути мои, и Ты услышал меня; научи меня уставам Твоим.
\vs Psa 118:27 Дай мне уразуметь путь повелений Твоих, и буду размышлять о чудесах Твоих.
\vs Psa 118:28 Душа моя истаевает от скорби: укрепи меня по слову Твоему.
\vs Psa 118:29 Удали от меня путь лжи, и закон Твой даруй мне.
\vs Psa 118:30 Я избрал путь истины, поставил пред собою суды Твои.
\vs Psa 118:31 Я прилепился к откровениям Твоим, Господи; не постыди меня.
\vs Psa 118:32 Потеку путем заповедей Твоих, когда Ты расширишь сердце мое.
\vs Psa 118:33 Укажи мне, Господи, путь уставов Твоих, и я буду держаться его до конца.
\vs Psa 118:34 Вразуми меня, и буду соблюдать закон Твой и хранить его всем сердцем.
\vs Psa 118:35 Поставь меня на стезю заповедей Твоих, ибо я возжелал ее.
\vs Psa 118:36 Приклони сердце мое к откровениям Твоим, а не к корысти.
\vs Psa 118:37 Отврати очи мои, чтобы не видеть суеты; животвори меня на пути Твоем.
\vs Psa 118:38 Утверди слово Твое рабу Твоему, ради благоговения пред Тобою.
\vs Psa 118:39 Отврати поношение мое, которого я страшусь, ибо суды Твои благи.
\vs Psa 118:40 Вот, я возжелал повелений Твоих; животвори меня правдою Твоею.
\vs Psa 118:41 Да придут ко мне милости Твои, Господи, спасение Твое по слову Твоему,~---
\vs Psa 118:42 и я дам ответ поносящему меня, ибо уповаю на слово Твое.
\vs Psa 118:43 Не отнимай совсем от уст моих слова истины, ибо я уповаю на суды Твои
\vs Psa 118:44 и буду хранить закон Твой всегда, во веки и веки;
\vs Psa 118:45 буду ходить свободно, ибо я взыскал повелений Твоих;
\vs Psa 118:46 буду говорить об откровениях Твоих пред царями и не постыжусь;
\vs Psa 118:47 буду утешаться заповедями Твоими, которые возлюбил;
\vs Psa 118:48 руки мои буду простирать к заповедям Твоим, которые возлюбил, и размышлять об уставах Твоих.
\vs Psa 118:49 Вспомни слово [Твое] к рабу Твоему, на которое Ты повелел мне уповать:
\vs Psa 118:50 это~--- утешение в бедствии моем, что слово Твое оживляет меня.
\vs Psa 118:51 Гордые крайне ругались надо мною, но я не уклонился от закона Твоего.
\vs Psa 118:52 Вспоминал суды Твои, Господи, от века, и утешался.
\vs Psa 118:53 Ужас овладевает мною при виде нечестивых, оставляющих закон Твой.
\vs Psa 118:54 Уставы Твои были песнями моими на месте странствований моих.
\vs Psa 118:55 Ночью вспоминал я имя Твое, Господи, и хранил закон Твой.
\vs Psa 118:56 Он стал моим, ибо повеления Твои храню.
\vs Psa 118:57 Удел мой, Господи, сказал я, соблюдать слова Твои.
\vs Psa 118:58 Молился я Тебе всем сердцем: помилуй меня по слову Твоему.
\vs Psa 118:59 Размышлял о путях моих и обращал стопы мои к откровениям Твоим.
\vs Psa 118:60 Спешил и не медлил соблюдать заповеди Твои.
\vs Psa 118:61 Сети нечестивых окружили меня, но я не забывал закона Твоего.
\vs Psa 118:62 В полночь вставал славословить Тебя за праведные суды Твои.
\vs Psa 118:63 Общник я всем боящимся Тебя и хранящим повеления Твои.
\vs Psa 118:64 Милости Твоей, Господи, полна земля; научи меня уставам Твоим.
\vs Psa 118:65 Благо сотворил Ты рабу Твоему, Господи, по слову Твоему.
\vs Psa 118:66 Доброму разумению и ведению научи меня, ибо заповедям Твоим я верую.
\vs Psa 118:67 Прежде страдания моего я заблуждался; а ныне слово Твое храню.
\vs Psa 118:68 Благ и благодетелен Ты, [Господи]; научи меня уставам Твоим.
\vs Psa 118:69 Гордые сплетают на меня ложь; я же всем сердцем буду хранить повеления Твои.
\vs Psa 118:70 Ожирело сердце их, как тук; я же законом Твоим утешаюсь.
\vs Psa 118:71 Благо мне, что я пострадал, дабы научиться уставам Твоим.
\vs Psa 118:72 Закон уст Твоих для меня лучше тысяч золота и серебра.
\rsbpar\vs Psa 118:73 Руки Твои сотворили меня и устроили меня; вразуми меня, и научусь заповедям Твоим.
\vs Psa 118:74 Боящиеся Тебя увидят меня~--- и возрадуются, что я уповаю на слово Твое.
\vs Psa 118:75 Знаю, Господи, что суды Твои праведны и по справедливости Ты наказал меня.
\vs Psa 118:76 Да будет же милость Твоя утешением моим, по слову Твоему к рабу Твоему.
\vs Psa 118:77 Да придет ко мне милосердие Твое, и я буду жить; ибо закон Твой~--- утешение мое.
\vs Psa 118:78 Да будут постыжены гордые, ибо безвинно угнетают меня; я размышляю о повелениях Твоих.
\vs Psa 118:79 Да обратятся ко мне боящиеся Тебя и знающие откровения Твои.
\vs Psa 118:80 Да будет сердце мое непорочно в уставах Твоих, чтобы я не посрамился.
\vs Psa 118:81 Истаевает душа моя о спасении Твоем; уповаю на слово Твое.
\vs Psa 118:82 Истаевают очи мои о слове Твоем; я говорю: когда Ты утешишь меня?
\vs Psa 118:83 Я стал, как мех в дыму, \bibemph{но} уставов Твоих не забыл.
\vs Psa 118:84 Сколько дней раба Твоего? Когда произведешь суд над гонителями моими?
\vs Psa 118:85 Яму вырыли мне гордые, вопреки закону Твоему.
\vs Psa 118:86 Все заповеди Твои~--- истина; несправедливо преследуют меня: помоги мне;
\vs Psa 118:87 едва не погубили меня на земле, но я не оставил повелений Твоих.
\vs Psa 118:88 По милости Твоей оживляй меня, и буду хранить откровения уст Твоих.
\vs Psa 118:89 На веки, Господи, слово Твое утверждено на небесах;
\vs Psa 118:90 истина Твоя в род и род. Ты поставил землю, и она стоит.
\vs Psa 118:91 По определениям Твоим все стоит доныне, ибо все служит Тебе.
\vs Psa 118:92 Если бы не закон Твой был утешением моим, погиб бы я в бедствии моем.
\vs Psa 118:93 Вовек не забуду повелений Твоих, ибо ими Ты оживляешь меня.
\vs Psa 118:94 Твой я, спаси меня; ибо я взыскал повелений Твоих.
\vs Psa 118:95 Нечестивые подстерегают меня, чтобы погубить; \bibemph{а} я углубляюсь в откровения Твои.
\vs Psa 118:96 Я видел предел всякого совершенства, \bibemph{но} Твоя заповедь безмерно обширна.
\vs Psa 118:97 Как люблю я закон Твой! весь день размышляю о нем.
\vs Psa 118:98 Заповедью Твоею Ты соделал меня мудрее врагов моих, ибо она всегда со мною.
\vs Psa 118:99 Я стал разумнее всех учителей моих, ибо размышляю об откровениях Твоих.
\vs Psa 118:100 Я сведущ более старцев, ибо повеления Твои храню.
\vs Psa 118:101 От всякого злого пути удерживаю ноги мои, чтобы хранить слово Твое;
\vs Psa 118:102 от судов Твоих не уклоняюсь, ибо Ты научаешь меня.
\vs Psa 118:103 Как сладки гортани моей слова Твои! лучше меда устам моим.
\vs Psa 118:104 Повелениями Твоими я вразумлен; потому ненавижу всякий путь лжи.
\vs Psa 118:105 Слово Твое~--- светильник ноге моей и свет стезе моей.
\vs Psa 118:106 Я клялся хранить праведные суды Твои, и исполню.
\vs Psa 118:107 Сильно угнетен я, Господи; оживи меня по слову Твоему.
\vs Psa 118:108 Благоволи же, Господи, принять добровольную жертву уст моих, и судам Твоим научи меня.
\vs Psa 118:109 Душа моя непрестанно в руке моей, но закона Твоего не забываю.
\vs Psa 118:110 Нечестивые поставили для меня сеть, но я не уклонился от повелений Твоих.
\vs Psa 118:111 Откровения Твои я принял, как наследие на веки, ибо они веселие сердца моего.
\vs Psa 118:112 Я приклонил сердце мое к исполнению уставов Твоих навек, до конца.
\vs Psa 118:113 Вымыслы \bibemph{человеческие} ненавижу, а закон Твой люблю.
\vs Psa 118:114 Ты покров мой и щит мой; на слово Твое уповаю.
\vs Psa 118:115 Удалитесь от меня, беззаконные, и буду хранить заповеди Бога моего.
\vs Psa 118:116 Укрепи меня по слову Твоему, и буду жить; не посрами меня в надежде моей;
\vs Psa 118:117 поддержи меня, и спасусь; и в уставы Твои буду вникать непрестанно.
\vs Psa 118:118 Всех, отступающих от уставов Твоих, Ты низлагаешь, ибо ухищрения их~--- ложь.
\vs Psa 118:119 \bibemph{Как} изгарь, отметаешь Ты всех нечестивых земли; потому я возлюбил откровения Твои.
\vs Psa 118:120 Трепещет от страха Твоего плоть моя, и судов Твоих я боюсь.
\vs Psa 118:121 Я совершал суд и правду; не предай меня гонителям моим.
\vs Psa 118:122 Заступи раба Твоего ко благу \bibemph{его}, чтобы не угнетали меня гордые.
\vs Psa 118:123 Истаевают очи мои, ожидая спасения Твоего и слова правды Твоей.
\vs Psa 118:124 Сотвори с рабом Твоим по милости Твоей, и уставам Твоим научи меня.
\vs Psa 118:125 Я раб Твой: вразуми меня, и познаю откровения Твои.
\vs Psa 118:126 Время Господу действовать: закон Твой разорили.
\vs Psa 118:127 А я люблю заповеди Твои более золота, и золота чистого.
\vs Psa 118:128 Все повеления Твои~--- все призна\acc{ю} справедливыми; всякий путь лжи ненавижу.
\vs Psa 118:129 Дивны откровения Твои; потому хранит их душа моя.
\vs Psa 118:130 Откровение слов Твоих просвещает, вразумляет простых.
\vs Psa 118:131 Открываю уста мои и вздыхаю, ибо заповедей Твоих жажду.
\rsbpar\vs Psa 118:132 Призри на меня и помилуй меня, как поступаешь с любящими имя Твое.
\vs Psa 118:133 Утверди стопы мои в слове Твоем и не дай овладеть мною никакому беззаконию;
\vs Psa 118:134 избавь меня от угнетения человеческого, и буду хранить повеления Твои;
\vs Psa 118:135 осияй раба Твоего светом лица Твоего и научи меня уставам Твоим.
\vs Psa 118:136 Из глаз моих текут потоки вод оттого, что не хранят закона Твоего.
\vs Psa 118:137 Праведен Ты, Господи, и справедливы суды Твои.
\vs Psa 118:138 Откровения Твои, которые Ты заповедал,~--- правда и совершенная истина.
\vs Psa 118:139 Ревность моя снедает меня, потому что мои враги забыли слова Твои.
\vs Psa 118:140 Слово Твое весьма чисто, и раб Твой возлюбил его.
\vs Psa 118:141 Мал я и презрен, \bibemph{но} повелений Твоих не забываю.
\vs Psa 118:142 Правда Твоя~--- правда вечная, и закон Твой~--- истина.
\vs Psa 118:143 Скорбь и горесть постигли меня; заповеди Твои~--- утешение мое.
\vs Psa 118:144 Правда откровений Твоих вечна: вразуми меня, и буду жить.
\vs Psa 118:145 Взываю всем сердцем [моим]: услышь меня, Господи,~--- и сохраню уставы Твои.
\vs Psa 118:146 Призываю Тебя: спаси меня, и буду хранить откровения Твои.
\vs Psa 118:147 Предваряю рассвет и взываю; на слово Твое уповаю.
\vs Psa 118:148 Очи мои предваряют \bibemph{утреннюю} стражу, чтобы мне углубляться в слово Твое.
\vs Psa 118:149 Услышь голос мой по милости Твоей, Господи; по суду Твоему оживи меня.
\vs Psa 118:150 Приблизились замышляющие лукавство; далеки они от закона Твоего.
\vs Psa 118:151 Близок Ты, Господи, и все заповеди Твои~--- истина.
\vs Psa 118:152 Издавна узнал я об откровениях Твоих, что Ты утвердил их на веки.
\vs Psa 118:153 Воззри на бедствие мое и избавь меня, ибо я не забываю закона Твоего.
\vs Psa 118:154 Вступись в дело мое и защити меня; по слову Твоему оживи меня.
\vs Psa 118:155 Далеко от нечестивых спасение, ибо они уставов Твоих не ищут.
\vs Psa 118:156 Много щедрот Твоих, Господи; по суду Твоему оживи меня.
\vs Psa 118:157 Много у меня гонителей и врагов, \bibemph{но} от откровений Твоих я не удаляюсь.
\vs Psa 118:158 Вижу отступников, и сокрушаюсь, ибо они не хранят слова Твоего.
\vs Psa 118:159 Зри, как я люблю повеления Твои; по милости Твоей, Господи, оживи меня.
\vs Psa 118:160 Основание слова Твоего истинно, и вечен всякий суд правды Твоей.
\vs Psa 118:161 Князья гонят меня безвинно, но сердце мое боится слова Твоего.
\vs Psa 118:162 Радуюсь я слову Твоему, как получивший великую прибыль.
\vs Psa 118:163 Ненавижу ложь и гнушаюсь ею; закон же Твой люблю.
\vs Psa 118:164 Семикратно в день прославляю Тебя за суды правды Твоей.
\vs Psa 118:165 Велик мир у любящих закон Твой, и нет им преткновения.
\vs Psa 118:166 Уповаю на спасение Твое, Господи, и заповеди Твои исполняю.
\vs Psa 118:167 Душа моя хранит откровения Твои, и я люблю их крепко.
\vs Psa 118:168 Храню повеления Твои и откровения Твои, ибо все пути мои пред Тобою.
\vs Psa 118:169 Да приблизится вопль мой пред лице Твое, Господи; по слову Твоему вразуми меня.
\vs Psa 118:170 Да придет моление мое пред лице Твое; по слову Твоему избавь меня.
\vs Psa 118:171 Уста мои произнесут хвалу, когда Ты научишь меня уставам Твоим.
\vs Psa 118:172 Язык мой возгласит слово Твое, ибо все заповеди Твои праведны.
\vs Psa 118:173 Да будет рука Твоя в помощь мне, ибо я повеления Твои избрал.
\vs Psa 118:174 Жажду спасения Твоего, Господи, и закон Твой~--- утешение мое.
\vs Psa 118:175 Да живет душа моя и славит Тебя, и суды Твои да помогут мне.
\vs Psa 118:176 Я заблудился, как овца потерянная: взыщи раба Твоего, ибо я заповедей Твоих не забыл.
\vs Psa 119:0 Песнь восхождения.
\rsbpar\vs Psa 119:1 К Господу воззвал я в скорби моей, и Он услышал меня.
\vs Psa 119:2 Господи! избавь душу мою от уст лживых, от языка лукавого.
\vs Psa 119:3 Что даст тебе и что прибавит тебе язык лукавый?
\vs Psa 119:4 Изощренные стрелы сильного, с горящими углями дроковыми.
\vs Psa 119:5 Горе мне, что я пребываю у Мосоха, живу у шатров Кидарских.
\vs Psa 119:6 Долго жила душа моя с ненавидящими мир.
\vs Psa 119:7 Я мирен: но только заговорю, они~--- к войне.
\vs Psa 120:0 Песнь восхождения.
\rsbpar\vs Psa 120:1 Возвожу очи мои к горам, откуда придет помощь моя.
\vs Psa 120:2 Помощь моя от Господа, сотворившего небо и землю.
\vs Psa 120:3 Не даст Он поколебаться ноге твоей, не воздремлет хранящий тебя;
\vs Psa 120:4 не дремлет и не спит хранящий Израиля.
\vs Psa 120:5 Господь~--- хранитель твой; Господь~--- сень твоя с правой руки твоей.
\vs Psa 120:6 Днем солнце не поразит тебя, ни луна ночью.
\vs Psa 120:7 Господь сохранит тебя от всякого зла; сохранит душу твою [Господь].
\vs Psa 120:8 Господь будет охранять выхождение твое и вхождение твое отныне и вовек.
\vs Psa 121:0 Песнь восхождения. Давида.
\rsbpar\vs Psa 121:1 Возрадовался я, когда сказали мне: <<пойдем в дом Господень>>.
\vs Psa 121:2 Вот, стоят ноги наши во вратах твоих, Иерусалим,~---
\vs Psa 121:3 Иерусалим, устроенный как город, слитый в одно,
\vs Psa 121:4 куда восходят колена, колена Господни, по закону Израилеву, славить имя Господне.
\vs Psa 121:5 Там стоят престолы суда, престолы дома Давидова.
\vs Psa 121:6 Просите мира Иерусалиму: да благоденствуют любящие тебя!
\vs Psa 121:7 Да будет мир в стенах твоих, благоденствие~--- в чертогах твоих!
\vs Psa 121:8 Ради братьев моих и ближних моих говорю я: <<мир тебе!>>
\vs Psa 121:9 Ради дома Господа, Бога нашего, желаю блага тебе.
\vs Psa 122:0 Песнь восхождения.
\rsbpar\vs Psa 122:1 К Тебе возвожу очи мои, Живущий на небесах!
\vs Psa 122:2 Вот, как очи рабов \bibemph{обращены} на руку господ их, как очи рабы~--- на руку госпожи ее, так очи наши~--- к Господу, Богу нашему, доколе Он помилует нас.
\vs Psa 122:3 Помилуй нас, Господи, помилуй нас, ибо довольно мы насыщены презрением;
\vs Psa 122:4 довольно насыщена душа наша поношением от надменных и уничижением от гордых.
\vs Psa 123:0 Песнь восхождения. Давида.
\rsbpar\vs Psa 123:1 Если бы не Господь был с нами,~--- да скажет Израиль,~---
\vs Psa 123:2 если бы не Господь был с нами, когда восстали на нас люди,
\vs Psa 123:3 то живых они поглотили бы нас, когда возгорелась ярость их на нас;
\vs Psa 123:4 воды потопили бы нас, поток прошел бы над душею нашею;
\vs Psa 123:5 прошли бы над душею нашею воды бурные.
\vs Psa 123:6 Благословен Господь, Который не дал нас в добычу зубам их!
\vs Psa 123:7 Душа наша избавилась, как птица, из сети ловящих: сеть расторгнута, и мы избавились.
\vs Psa 123:8 Помощь наша~--- в имени Господа, сотворившего небо и землю.
\vs Psa 124:0 Песнь восхождения.
\rsbpar\vs Psa 124:1 Надеющийся на Господа, как гора Сион, не подвигнется: пребывает вовек.
\vs Psa 124:2 Горы окрест Иерусалима, а Господь окрест народа Своего отныне и вовек.
\vs Psa 124:3 Ибо не оставит [Господь] жезла нечестивых над жребием праведных, дабы праведные не простерли рук своих к беззаконию.
\vs Psa 124:4 Благотвори, Господи, добрым и правым в сердцах своих;
\vs Psa 124:5 а совращающихся на кривые пути свои да оставит Господь ходить с делающими беззаконие. Мир на Израиля!
\vs Psa 125:0 Песнь восхождения.
\rsbpar\vs Psa 125:1 Когда возвращал Господь плен Сиона, мы были как бы видящие во сне:
\vs Psa 125:2 тогда уста наши были полны веселья, и язык наш~--- пения; тогда между народами говорили: <<великое сотворил Господь над ними!>>
\vs Psa 125:3 Великое сотворил Господь над нами: мы радовались.
\vs Psa 125:4 Возврати, Господи, пленников наших, как потоки на полдень.
\vs Psa 125:5 Сеявшие со слезами будут пожинать с радостью.
\vs Psa 125:6 С плачем несущий семена возвратится с радостью, неся снопы свои.
\vs Psa 126:0 Песнь восхождения. Соломона.
\rsbpar\vs Psa 126:1 Если Господь не созиждет дома, напрасно трудятся строящие его; если Господь не охранит города, напрасно бодрствует страж.
\vs Psa 126:2 Напрасно вы рано встаете, поздно просиживаете, едите хлеб печали, тогда как возлюбленному Своему Он дает сон.
\vs Psa 126:3 Вот наследие от Господа: дети; награда от Него~--- плод чрева.
\vs Psa 126:4 Что стрелы в руке сильного, то сыновья молодые.
\vs Psa 126:5 Блажен человек, который наполнил ими колчан свой! Не останутся они в стыде, когда будут говорить с врагами в воротах.
\vs Psa 127:0 Песнь восхождения.
\rsbpar\vs Psa 127:1 Блажен всякий боящийся Господа, ходящий путями Его!
\vs Psa 127:2 Ты будешь есть от трудов рук твоих: блажен ты, и благо тебе!
\vs Psa 127:3 Жена твоя, как плодовитая лоза, в доме твоем; сыновья твои, как масличные ветви, вокруг трапезы твоей:
\vs Psa 127:4 так благословится человек, боящийся Господа!
\vs Psa 127:5 Благословит тебя Господь с Сиона, и увидишь благоденствие Иерусалима во все дни жизни твоей;
\vs Psa 127:6 увидишь сыновей у сыновей твоих. Мир на Израиля!
\vs Psa 128:0 Песнь восхождения.
\rsbpar\vs Psa 128:1 Много теснили меня от юности моей, да скажет Израиль:
\vs Psa 128:2 много теснили меня от юности моей, но не одолели меня.
\vs Psa 128:3 На хребте моем орали оратаи, проводили длинные борозды свои.
\vs Psa 128:4 Но Господь праведен: Он рассек узы нечестивых.
\vs Psa 128:5 Да постыдятся и обратятся назад все ненавидящие Сион!
\vs Psa 128:6 Да будут, как трава на кровлях, которая прежде, нежели будет исторгнута, засыхает,
\vs Psa 128:7 которою жнец не наполнит руки своей, и вяжущий снопы~--- горсти своей;
\vs Psa 128:8 и проходящие мимо не скажут: <<благословение Господне на вас; благословляем вас именем Господним!>>
\vs Psa 129:0 Песнь восхождения.
\rsbpar\vs Psa 129:1 Из глубины взываю к Тебе, Господи.
\vs Psa 129:2 Господи! услышь голос мой. Да будут уши Твои внимательны к голосу молений моих.
\vs Psa 129:3 Если Ты, Господи, будешь замечать беззакония,~--- Господи! кто устоит?
\vs Psa 129:4 Но у Тебя прощение, да благоговеют пред Тобою.
\vs Psa 129:5 Надеюсь на Господа, надеется душа моя; на слово Его уповаю.
\vs Psa 129:6 Душа моя ожидает Господа более, нежели стражи~--- утра, более, нежели стражи~--- утра.
\vs Psa 129:7 Да уповает Израиль на Господа, ибо у Господа милость и многое у Него избавление,
\vs Psa 129:8 и Он избавит Израиля от всех беззаконий его.
\vs Psa 130:0 Песнь восхождения. Давида.
\rsbpar\vs Psa 130:1 Господи! не надмевалось сердце мое и не возносились очи мои, и я не входил в великое и для меня недосягаемое.
\vs Psa 130:2 Не смирял ли я и не успокаивал ли души моей, как дитяти, отнятого от груди матери? душа моя была во мне, как дитя, отнятое от груди.
\vs Psa 130:3 Да уповает Израиль на Господа отныне и вовек.
\vs Psa 131:0 Песнь восхождения.
\rsbpar\vs Psa 131:1 Вспомни, Господи, Давида и все сокрушение его:
\vs Psa 131:2 как он клялся Господу, давал обет Сильному Иакова:
\vs Psa 131:3 <<не войду в шатер дома моего, не взойду на ложе мое;
\vs Psa 131:4 не дам сна очам моим и веждам моим~--- дремания,
\vs Psa 131:5 доколе не найду места Господу, жилища~--- Сильному Иакова>>.
\vs Psa 131:6 Вот, мы слышали о нем в Ефрафе, нашли его на полях Иарима.
\vs Psa 131:7 Пойдем к жилищу Его, поклонимся подножию ног Его.
\vs Psa 131:8 Стань, Господи, на \bibemph{место} покоя Твоего,~--- Ты и ковчег могущества Твоего.
\vs Psa 131:9 Священники Твои облекутся правдою, и святые Твои возрадуются.
\vs Psa 131:10 Ради Давида, раба Твоего, не отврати лица помазанника Твоего.
\vs Psa 131:11 Клялся Господь Давиду в истине, и не отречется ее: <<от плода чрева твоего посажу на престоле твоем.
\vs Psa 131:12 Если сыновья твои будут сохранять завет Мой и откровения Мои, которым Я научу их, то и их сыновья во веки будут сидеть на престоле твоем>>.
\vs Psa 131:13 Ибо избрал Господь Сион, возжелал [его] в жилище Себе.
\vs Psa 131:14 <<Это покой Мой на веки: здесь вселюсь, ибо Я возжелал его.
\vs Psa 131:15 Пищу его благословляя благословлю, нищих его насыщу хлебом;
\vs Psa 131:16 священников его облеку во спасение, и святые его радостью возрадуются.
\vs Psa 131:17 Там возращу рог Давиду, поставлю светильник помазаннику Моему.
\vs Psa 131:18 Врагов его облеку стыдом, а на нем будет сиять венец его>>.
\vs Psa 132:0 Песнь восхождения. Давида.
\rsbpar\vs Psa 132:1 Как хорошо и как приятно жить братьям вместе!
\vs Psa 132:2 \bibemph{Это}~--- как драгоценный елей на голове, стекающий на бороду, бороду Ааронову, стекающий на края одежды его;
\vs Psa 132:3 как роса Ермонская, сходящая на горы Сионские, ибо там заповедал Господь благословение и жизнь на веки.
\vs Psa 133:0 Песнь восхождения.
\rsbpar\vs Psa 133:1 Благословите ныне Господа, все рабы Господни, стоящие в доме Господнем, [во дворах дома Бога нашего,] во время ночи.
\vs Psa 133:2 Воздвигните руки ваши к святилищу, и благословите Господа.
\vs Psa 133:3 Благословит тебя Господь с Сиона, сотворивший небо и землю.
\vs Psa 134:0 Аллилуия.
\rsbpar\vs Psa 134:1 Хвалите имя Господне, хвалите, рабы Господни,
\vs Psa 134:2 стоящие в доме Господнем, во дворах дома Бога нашего.
\vs Psa 134:3 Хвалите Господа, ибо Господь благ; пойте имени Его, ибо это сладостно,
\vs Psa 134:4 ибо Господь избрал Себе Иакова, Израиля в собственность Свою.
\vs Psa 134:5 Я познал, что велик Господь, и Господь наш превыше всех богов.
\vs Psa 134:6 Господь творит все, что хочет, на небесах и на земле, на морях и во всех безднах;
\vs Psa 134:7 возводит облака от края земли, творит молнии при дожде, изводит ветер из хранилищ Своих.
\vs Psa 134:8 Он поразил первенцев Египта, от человека до скота,
\vs Psa 134:9 послал знамения и чудеса среди тебя, Египет, на фараона и на всех рабов его,
\vs Psa 134:10 поразил народы многие и истребил царей сильных:
\vs Psa 134:11 Сигона, царя Аморрейского, и Ога, царя Васанского, и все царства Ханаанские;
\vs Psa 134:12 и отдал землю их в наследие, в наследие Израилю, народу Своему.
\vs Psa 134:13 Господи! имя Твое вовек; Господи! память о Тебе в род и род.
\vs Psa 134:14 Ибо Господь будет судить народ Свой и над рабами Своими умилосердится.
\vs Psa 134:15 Идолы язычников~--- серебро и золото, дело рук человеческих:
\vs Psa 134:16 есть у них уста, но не говорят; есть у них глаза, но не видят;
\vs Psa 134:17 есть у них уши, но не слышат, и нет дыхания в устах их.
\vs Psa 134:18 Подобны им будут делающие их и всякий, кто надеется на них.
\vs Psa 134:19 Дом Израилев! благословите Господа. Дом Ааронов! благословите Господа.
\vs Psa 134:20 Дом Левиин! благословите Господа. Боящиеся Господа! благословите Господа.
\vs Psa 134:21 Благословен Господь от Сиона, живущий в Иерусалиме! Аллилуия!
\vs Psa 135:0 [Аллилуия.]
\rsbpar\vs Psa 135:1 Славьте Господа, ибо Он благ, ибо вовек милость Его.
\vs Psa 135:2 Славьте Бога богов, ибо вовек милость Его.
\vs Psa 135:3 Славьте Господа господствующих, ибо вовек милость Его;
\vs Psa 135:4 Того, Который один творит чудеса великие, ибо вовек милость Его;
\vs Psa 135:5 Который сотворил небеса премудро, ибо вовек милость Его;
\vs Psa 135:6 утвердил землю на водах, ибо вовек милость Его;
\vs Psa 135:7 сотворил светила великие, ибо вовек милость Его;
\vs Psa 135:8 солнце~--- для управления днем, ибо вовек милость Его;
\vs Psa 135:9 луну и звезды~--- для управления ночью, ибо вовек милость Его;
\vs Psa 135:10 поразил Египет в первенцах его, ибо вовек милость Его;
\vs Psa 135:11 и вывел Израиля из среды его, ибо вовек милость Его;
\vs Psa 135:12 рукою крепкою и мышцею простертою, ибо вовек милость Его;
\vs Psa 135:13 разделил Чермное море, ибо вовек милость Его;
\vs Psa 135:14 и провел Израиля посреди его, ибо вовек милость Его;
\vs Psa 135:15 и низверг фараона и войско его в море Чермное, ибо вовек милость Его;
\vs Psa 135:16 провел народ Свой чрез пустыню, ибо вовек милость Его;
\vs Psa 135:17 поразил царей великих, ибо вовек милость Его;
\vs Psa 135:18 и убил царей сильных, ибо вовек милость Его;
\vs Psa 135:19 Сигона, царя Аморрейского, ибо вовек милость Его;
\vs Psa 135:20 и Ога, царя Васанского, ибо вовек милость Его;
\vs Psa 135:21 и отдал землю их в наследие, ибо вовек милость Его;
\vs Psa 135:22 в наследие Израилю, рабу Своему, ибо вовек милость Его;
\vs Psa 135:23 вспомнил нас в унижении нашем, ибо вовек милость Его;
\vs Psa 135:24 и избавил нас от врагов наших, ибо вовек милость Его;
\vs Psa 135:25 дает пищу всякой плоти, ибо вовек милость Его.
\vs Psa 135:26 Славьте Бога небес, ибо вовек милость Его.
\vs Psa 136:0 [Давида.]
\rsbpar\vs Psa 136:1 При реках Вавилона, там сидели мы и плакали, когда вспоминали о Сионе;
\vs Psa 136:2 на вербах, посреди его, повесили мы наши арфы.
\vs Psa 136:3 Там пленившие нас требовали от нас слов песней, и притеснители наши~--- веселья: <<пропойте нам из песней Сионских>>.
\vs Psa 136:4 Как нам петь песнь Господню на земле чужой?
\vs Psa 136:5 Если я забуду тебя, Иерусалим,~--- забудь меня десница моя;
\vs Psa 136:6 прилипни язык мой к гортани моей, если не буду помнить тебя, если не поставлю Иерусалима во главе веселия моего.
\vs Psa 136:7 Припомни, Господи, сынам Едомовым день Иерусалима, когда они говорили: <<разрушайте, разрушайте до основания его>>.
\vs Psa 136:8 Дочь Вавилона, опустошительница! блажен, кто воздаст тебе за то, что ты сделала нам!
\vs Psa 136:9 Блажен, кто возьмет и разобьет младенцев твоих о камень!
\vs Psa 137:0 Давида.
\rsbpar\vs Psa 137:1 Славлю Тебя всем сердцем моим, пред богами\fns{В переводе 70-ти: пред Ангелами.} пою Тебе, [что Ты услышал все слова уст моих].
\vs Psa 137:2 Поклоняюсь пред святым храмом Твоим и славлю имя Твое за милость Твою и за истину Твою, ибо Ты возвеличил слово Твое превыше всякого имени Твоего.
\vs Psa 137:3 В день, когда я воззвал, Ты услышал меня, вселил в душу мою бодрость.
\vs Psa 137:4 Прославят Тебя, Господи, все цари земные, когда услышат слова уст Твоих
\vs Psa 137:5 и воспоют пути Господни, ибо велика слава Господня.
\vs Psa 137:6 Высок Господь: и смиренного видит, и гордого узнает издали.
\vs Psa 137:7 Если я пойду посреди напастей, Ты оживишь меня, прострешь на ярость врагов моих руку Твою, и спасет меня десница Твоя.
\vs Psa 137:8 Господь совершит за меня! Милость Твоя, Господи, вовек: дело рук Твоих не оставляй.
\vs Psa 138:0 Начальнику хора. Псалом Давида.
\rsbpar\vs Psa 138:1 Господи! Ты испытал меня и знаешь.
\vs Psa 138:2 Ты знаешь, когда я сажусь и когда встаю; Ты разумеешь помышления мои издали.
\vs Psa 138:3 Иду ли я, отдыхаю ли~--- Ты окружаешь меня, и все пути мои известны Тебе.
\vs Psa 138:4 Еще нет слова на языке моем,~--- Ты, Господи, уже знаешь его совершенно.
\vs Psa 138:5 Сзади и спереди Ты объемлешь меня, и полагаешь на мне руку Твою.
\vs Psa 138:6 Дивно для меня ведение [Твое],~--- высоко, не могу постигнуть его!
\vs Psa 138:7 Куда пойду от Духа Твоего, и от лица Твоего куда убегу?
\vs Psa 138:8 Взойду ли на небо~--- Ты там; сойду ли в преисподнюю~--- и там Ты.
\vs Psa 138:9 Возьму ли крылья зари и переселюсь на край моря,~---
\vs Psa 138:10 и там рука Твоя поведет меня, и удержит меня десница Твоя.
\vs Psa 138:11 Скажу ли: <<может быть, тьма скроет меня, и свет вокруг меня \bibemph{сделается} ночью>>;
\vs Psa 138:12 но и тьма не затмит от Тебя, и ночь светла, как день: как тьма, так и свет.
\vs Psa 138:13 Ибо Ты устроил внутренности мои и соткал меня во чреве матери моей.
\vs Psa 138:14 Славлю Тебя, потому что я дивно устроен. Дивны дела Твои, и душа моя вполне сознает это.
\vs Psa 138:15 Не сокрыты были от Тебя кости мои, когда я созидаем был в тайне, образуем был во глубине утробы.
\vs Psa 138:16 Зародыш мой видели очи Твои; в Твоей книге записаны все дни, для меня назначенные, когда ни одного из них еще не было.
\vs Psa 138:17 Как возвышенны для меня помышления Твои, Боже, и как велико число их!
\vs Psa 138:18 Стану ли исчислять их, но они многочисленнее песка; когда я пробуждаюсь, я все еще с Тобою.
\vs Psa 138:19 О, если бы Ты, Боже, поразил нечестивого! Удалитесь от меня, кровожадные!
\vs Psa 138:20 Они говорят против Тебя нечестиво; суетное замышляют враги Твои.
\vs Psa 138:21 Мне ли не возненавидеть ненавидящих Тебя, Господи, и не возгнушаться восстающими на Тебя?
\vs Psa 138:22 Полною ненавистью ненавижу их: враги они мне.
\vs Psa 138:23 Испытай меня, Боже, и узнай сердце мое; испытай меня и узнай помышления мои;
\vs Psa 138:24 и зри, не на опасном ли я пути, и направь меня на путь вечный.
\vs Psa 139:0 Псалом.
\vs Psa 139:1 Начальнику хора. Псалом Давида.
\rsbpar\vs Psa 139:2 Избавь меня, Господи, от человека злого; сохрани меня от притеснителя:
\vs Psa 139:3 они злое мыслят в сердце, всякий день ополчаются на брань,
\vs Psa 139:4 изощряют язык свой, как змея; яд аспида под устами их.
\vs Psa 139:5 Соблюди меня, Господи, от рук нечестивого, сохрани меня от притеснителей, которые замыслили поколебать стопы мои.
\vs Psa 139:6 Гордые скрыли силки для меня и петли, раскинули сеть по дороге, тенета разложили для меня.
\vs Psa 139:7 Я сказал Господу: Ты Бог мой; услышь, Господи, голос молений моих!
\vs Psa 139:8 Господи, Господи, сила спасения моего! Ты покрыл голову мою в день брани.
\vs Psa 139:9 Не дай, Господи, желаемого нечестивому; не дай успеха злому замыслу его: они возгордятся.
\vs Psa 139:10 Да покроет головы окружающих меня зло собственных уст их.
\vs Psa 139:11 Да падут на них горящие угли; да будут они повержены в огонь, в пропасти, так, чтобы не встали.
\vs Psa 139:12 Человек злоязычный не утвердится на земле; зло увлечет притеснителя в погибель.
\vs Psa 139:13 Знаю, что Господь сотворит суд угнетенным и справедливость бедным.
\vs Psa 139:14 Так! праведные будут славить имя Твое; непорочные будут обитать пред лицем Твоим.
\vs Psa 140:0 Псалом Давида.
\rsbpar\vs Psa 140:1 Господи! к Тебе взываю: поспеши ко мне, внемли голосу моления моего, когда взываю к Тебе.
\vs Psa 140:2 Да направится молитва моя, как фимиам, пред лице Твое, воздеяние рук моих~--- как жертва вечерняя.
\vs Psa 140:3 Положи, Господи, охрану устам моим, и огради двери уст моих;
\vs Psa 140:4 не дай уклониться сердцу моему к словам лукавым для извинения дел греховных вместе с людьми, делающими беззаконие, и да не вкушу я от сластей их.
\vs Psa 140:5 Пусть наказывает меня праведник: это милость; пусть обличает меня: это лучший елей, который не повредит голове моей; но мольбы мои~--- против злодейств их.
\vs Psa 140:6 Вожди их рассыпались по утесам и слышат слова мои, что они кротки.
\vs Psa 140:7 Как будто землю рассекают и дробят нас; сыплются кости наши в челюсти преисподней.
\vs Psa 140:8 Но к Тебе, Господи, Господи, очи мои; на Тебя уповаю, не отринь души моей!
\vs Psa 140:9 Сохрани меня от силков, поставленных для меня, от тенет беззаконников.
\vs Psa 140:10 Падут нечестивые в сети свои, а я перейду.
\vs Psa 141:0 Учение Давида. Молитва его, когда он был в пещере.
\rsbpar\vs Psa 141:1 Голосом моим к Господу воззвал я, голосом моим к Господу помолился;
\vs Psa 141:2 излил пред Ним моление мое; печаль мою открыл Ему.
\vs Psa 141:3 Когда изнемогал во мне дух мой, Ты знал стезю мою. На пути, которым я ходил, они скрытно поставили сети для меня.
\vs Psa 141:4 Смотрю на правую сторону, и вижу, что никто не признаёт меня: не стало для меня убежища, никто не заботится о душе моей.
\vs Psa 141:5 Я воззвал к Тебе, Господи, я сказал: Ты прибежище мое и часть моя на земле живых.
\vs Psa 141:6 Внемли воплю моему, ибо я очень изнемог; избавь меня от гонителей моих, ибо они сильнее меня.
\vs Psa 141:7 Выведи из темницы душу мою, чтобы мне славить имя Твое. Вокруг меня соберутся праведные, когда Ты явишь мне благодеяние.
\vs Psa 142:0 Псалом Давида, [когда он преследуем был сыном своим Авессаломом].
\rsbpar\vs Psa 142:1 Господи! услышь молитву мою, внемли молению моему по истине Твоей; услышь меня по правде Твоей
\vs Psa 142:2 и не входи в суд с рабом Твоим, потому что не оправдается пред Тобой ни один из живущих.
\vs Psa 142:3 Враг преследует душу мою, втоптал в землю жизнь мою, принудил меня жить во тьме, как давно умерших,~---
\vs Psa 142:4 и уныл во мне дух мой, онемело во мне сердце мое.
\vs Psa 142:5 Вспоминаю дни древние, размышляю о всех делах Твоих, рассуждаю о делах рук Твоих.
\vs Psa 142:6 Простираю к Тебе руки мои; душа моя~--- к Тебе, как жаждущая земля.
\vs Psa 142:7 Скоро услышь меня, Господи: дух мой изнемогает; не скрывай лица Твоего от меня, чтобы я не уподобился нисходящим в могилу.
\vs Psa 142:8 Даруй мне рано услышать милость Твою, ибо я на Тебя уповаю. Укажи мне, [Господи,] путь, по которому мне идти, ибо к Тебе возношу я душу мою.
\vs Psa 142:9 Избавь меня, Господи, от врагов моих; к Тебе прибегаю.
\vs Psa 142:10 Научи меня исполнять волю Твою, потому что Ты Бог мой; Дух Твой благий да ведет меня в землю правды.
\vs Psa 142:11 Ради имени Твоего, Господи, оживи меня; ради правды Твоей выведи из напасти душу мою.
\vs Psa 142:12 И по милости Твоей истреби врагов моих и погуби всех, угнетающих душу мою, ибо я Твой раб.
\vs Psa 143:0 Давида. [Против Голиафа.]
\rsbpar\vs Psa 143:1 Благословен Господь, твердыня моя, научающий руки мои битве и персты мои брани,
\vs Psa 143:2 милость моя и ограждение мое, прибежище мое и Избавитель мой, щит мой,~--- и я на Него уповаю; Он подчиняет мне народ мой.
\vs Psa 143:3 Господи! что есть человек, что Ты знаешь о нем, и сын человеческий, что обращаешь на него внимание?
\vs Psa 143:4 Человек подобен дуновению; дни его~--- как уклоняющаяся тень.
\vs Psa 143:5 Господи! Приклони небеса Твои и сойди; коснись гор, и воздымятся;
\vs Psa 143:6 блесни молниею и рассей их; пусти стрелы Твои и расстрой их;
\vs Psa 143:7 простри с высоты руку Твою, избавь меня и спаси меня от вод многих, от руки сынов иноплеменных,
\vs Psa 143:8 которых уста говорят суетное и которых десница~--- десница лжи.
\vs Psa 143:9 Боже! новую песнь воспою Тебе, на десятиструнной псалтири воспою Тебе,
\vs Psa 143:10 дарующему спасение царям и избавляющему Давида, раба Твоего, от лютого меча.
\vs Psa 143:11 Избавь меня и спаси меня от руки сынов иноплеменных, которых уста говорят суетное и которых десница~--- десница лжи.
\vs Psa 143:12 Да будут сыновья наши, как разросшиеся растения в их молодости; дочери наши~--- как искусно изваянные столпы в чертогах.
\vs Psa 143:13 Да будут житницы наши полны, обильны всяким хлебом; да плодятся овцы наши тысячами и тьмами на пажитях наших;
\vs Psa 143:14 \bibemph{да будут} волы наши тучны; да не будет ни расхищения, ни пропажи, ни воплей на улицах наших.
\vs Psa 143:15 Блажен народ, у которого это есть. Блажен народ, у которого Господь есть Бог.
\vs Psa 144:0 Хвала Давида.
\rsbpar\vs Psa 144:1 Буду превозносить Тебя, Боже мой, Царь [мой], и благословлять имя Твое во веки и веки.
\vs Psa 144:2 Всякий день буду благословлять Тебя и восхвалять имя Твое во веки и веки.
\vs Psa 144:3 Велик Господь и достохвален, и величие Его неисследимо.
\vs Psa 144:4 Род роду будет восхвалять дела Твои и возвещать о могуществе Твоем.
\vs Psa 144:5 А я буду размышлять о высокой славе величия Твоего и о дивных делах Твоих.
\vs Psa 144:6 Будут говорить о могуществе страшных дел Твоих, и я буду возвещать о величии Твоем.
\vs Psa 144:7 Будут провозглашать память великой благости Твоей и воспевать правду Твою.
\vs Psa 144:8 Щедр и милостив Господь, долготерпелив и многомилостив.
\vs Psa 144:9 Благ Господь ко всем, и щедроты Его на всех делах Его.
\vs Psa 144:10 Да славят Тебя, Господи, все дела Твои, и да благословляют Тебя святые Твои;
\vs Psa 144:11 да проповедуют славу царства Твоего, и да повествуют о могуществе Твоем,
\vs Psa 144:12 чтобы дать знать сынам человеческим о могуществе Твоем и о славном величии царства Твоего.
\vs Psa 144:13 Царство Твое~--- царство всех веков, и владычество Твое во все роды. [Верен Господь во всех словах Своих и свят во всех делах Своих.]
\vs Psa 144:14 Господь поддерживает всех падающих и восставляет всех низверженных.
\vs Psa 144:15 Очи всех уповают на Тебя, и Ты даешь им пищу их в свое время;
\vs Psa 144:16 открываешь руку Твою и насыщаешь все живущее по благоволению.
\vs Psa 144:17 Праведен Господь во всех путях Своих и благ во всех делах Своих.
\vs Psa 144:18 Близок Господь ко всем призывающим Его, ко всем призывающим Его в истине.
\vs Psa 144:19 Желание боящихся Его Он исполняет, вопль их слышит и спасает их.
\vs Psa 144:20 Хранит Господь всех любящих Его, а всех нечестивых истребит.
\vs Psa 144:21 Уста мои изрекут хвалу Господню, и да благословляет всякая плоть святое имя Его во веки и веки.
\vs Psa 145:0 [Аллилуия. \bibemph{Аггея и Захарии}.]
\rsbpar\vs Psa 145:1 Хвали, душа моя, Господа.
\vs Psa 145:2 Буду восхвалять Господа, доколе жив; буду петь Богу моему, доколе есмь.
\vs Psa 145:3 Не надейтесь на князей, на сына человеческого, в котором нет спасения.
\vs Psa 145:4 Выходит дух его, и он возвращается в землю свою: в тот день исчезают [все] помышления его.
\vs Psa 145:5 Блажен, кому помощник Бог Иаковлев, у кого надежда на Господа Бога его,
\vs Psa 145:6 сотворившего небо и землю, море и все, что в них, вечно хранящего верность,
\vs Psa 145:7 творящего суд обиженным, дающего хлеб алчущим. Господь разрешает узников,
\vs Psa 145:8 Господь отверзает очи слепым, Господь восставляет согбенных, Господь любит праведных.
\vs Psa 145:9 Господь хранит пришельцев, поддерживает сироту и вдову, а путь нечестивых извращает.
\vs Psa 145:10 Господь будет царствовать во веки, Бог твой, Сион, в род и род. Аллилуия.
\vs Psa 146:0 [Аллилуия.]
\rsbpar\vs Psa 146:1 Хвалите Господа, ибо благо петь Богу нашему, ибо это сладостно,~--- хвала подобающая.
\vs Psa 146:2 Господь созидает Иерусалим, собирает изгнанников Израиля.
\vs Psa 146:3 Он исцеляет сокрушенных сердцем и врачует скорби их;
\vs Psa 146:4 исчисляет количество звезд; всех их называет именами их.
\vs Psa 146:5 Велик Господь наш и велика крепость [Его], и разум Его неизмерим.
\vs Psa 146:6 Смиренных возвышает Господь, а нечестивых унижает до земли.
\vs Psa 146:7 Пойте поочередно славословие Господу; пойте Богу нашему на гуслях.
\vs Psa 146:8 Он покрывает небо облаками, приготовляет для земли дождь, произращает на горах траву [и злак на пользу человеку];
\vs Psa 146:9 дает скоту пищу его и птенцам ворона, взывающим \bibemph{к Нему}.
\vs Psa 146:10 Не на силу коня смотрит Он, не к \bibemph{быстроте} ног человеческих благоволит,~---
\vs Psa 146:11 благоволит Господь к боящимся Его, к уповающим на милость Его.
\vs Psa 147:0 [Аллилуия.]
\rsbpar\vs Psa 147:1 Хвали, Иерусалим, Господа; хвали, Сион, Бога твоего,
\vs Psa 147:2 ибо Он укрепляет вереи ворот твоих, благословляет сынов твоих среди тебя;
\vs Psa 147:3 утверждает в пределах твоих мир; туком пшеницы насыщает тебя;
\vs Psa 147:4 посылает слово Свое на землю; быстро течет слово Его;
\vs Psa 147:5 дает снег, как в\acc{о}лну; сыплет иней, как пепел;
\vs Psa 147:6 бросает град Свой кусками; перед морозом Его кто устоит?
\vs Psa 147:7 Пошлет слово Свое, и все растает; подует ветром Своим, и потекут воды.
\vs Psa 147:8 Он возвестил слово Свое Иакову, уставы Свои и суды Свои Израилю.
\vs Psa 147:9 Не сделал Он того никакому \bibemph{другому} народу, и судов Его они не знают. Аллилуия.
\vs Psa 148:0 [Аллилуия.]
\rsbpar\vs Psa 148:1 Хвалите Господа с небес, хвалите Его в вышних.
\vs Psa 148:2 Хвалите Его, все Ангелы Его, хвалите Его, все воинства Его.
\vs Psa 148:3 Хвалите Его, солнце и луна, хвалите Его, все звезды света.
\vs Psa 148:4 Хвалите Его, небеса небес и воды, которые превыше небес.
\vs Psa 148:5 Да хвалят имя Господа, ибо Он [сказал, и они сделались,] повелел, и сотворились;
\vs Psa 148:6 поставил их на веки и веки; дал устав, который не прейдет.
\vs Psa 148:7 Хвалите Господа от земли, великие рыбы и все бездны,
\vs Psa 148:8 огонь и град, снег и туман, бурный ветер, исполняющий слово Его,
\vs Psa 148:9 горы и все холмы, дерева плодоносные и все кедры,
\vs Psa 148:10 звери и всякий скот, пресмыкающиеся и птицы крылатые,
\vs Psa 148:11 цари земные и все народы, князья и все судьи земные,
\vs Psa 148:12 юноши и девицы, старцы и отроки
\vs Psa 148:13 да хвалят имя Господа, ибо имя Его единого превознесенно, слава Его на земле и на небесах.
\vs Psa 148:14 Он возвысил рог народа Своего, славу всех святых Своих, сынов Израилевых, народа, близкого к Нему. Аллилуия.
\vs Psa 149:0 [Аллилуия.]
\rsbpar\vs Psa 149:1 Пойте Господу песнь новую; хвала Ему в собрании святых.
\vs Psa 149:2 Да веселится Израиль о Создателе своем; сыны Сиона да радуются о Царе своем.
\vs Psa 149:3 да хвалят имя Его с ликами, на тимпане и гуслях да поют Ему,
\vs Psa 149:4 ибо благоволит Господь к народу Своему, прославляет смиренных спасением.
\vs Psa 149:5 Да торжествуют святые во славе, да радуются на ложах своих.
\vs Psa 149:6 Да будут славословия Богу в устах их, и меч обоюдоострый в руке их,
\vs Psa 149:7 для того, чтобы совершать мщение над народами, наказание над племенами,
\vs Psa 149:8 заключать царей их в узы и вельмож их в оковы железные,
\vs Psa 149:9 производить над ними суд писанный. Честь сия~--- всем святым Его. Аллилуия.
\vs Psa 150:0 [Аллилуия.]
\rsbpar\vs Psa 150:1 Хвалите Бога во святыне Его, хвалите Его на тверди силы Его.
\vs Psa 150:2 Хвалите Его по могуществу Его, хвалите Его по множеству величия Его.
\vs Psa 150:3 Хвалите Его со звуком трубным, хвалите Его на псалтири и гуслях.
\vs Psa 150:4 Хвалите Его с тимпаном и ликами, хвалите Его на струнах и органе.
\vs Psa 150:5 Хвалите Его на звучных кимвалах, хвалите Его на кимвалах громогласных.
\vs Psa 150:6 Все дышащее да хвалит Господа! Аллилуия.
\vs Psa 151:0 [\bibemph{Псалом Давида на единоборство с Голиафом}\fns{У Евреев этого псалма нет: он переведен с греческого.}.
\rsbpar\vs Psa 151:1 Я был меньший между братьями моими и юнейший в доме отца моего; пас овец отца моего.
\vs Psa 151:2 Руки мои сделали орган, персты мои настраивали псалтирь.
\vs Psa 151:3 И кто возвестил бы Господу моему?~--- Сам Господь, Сам услышал меня.
\vs Psa 151:4 Он послал вестника Своего и взял меня от овец отца моего, и помазал меня елеем помазания Своего.
\vs Psa 151:5 Братья мои прекрасны и велики, но Господь не благоволил избрать из них.
\vs Psa 151:6 Я вышел навстречу иноплеменнику, и он проклял меня идолами своими.
\vs Psa 151:7 Но я, исторгнув у него меч, обезглавил его и избавил сынов Израилевых от поношения.]

\bibbookdescr{Pro}{
  inline={\LARGE Книга\\\Huge Притчей Соломоновых},
  toc={Притчи},
  bookmark={Притчи},
  header={Притчи},
  %headerleft={},
  %headerright={},
  abbr={Притч}
}
\vs Pro 1:1 Притчи Соломона, сына Давидова, царя Израильского,
\vs Pro 1:2 чтобы познать мудрость и наставление, понять изречения разума;
\vs Pro 1:3 усвоить правила благоразумия, правосудия, суда и правоты;
\vs Pro 1:4 простым дать смышленость, юноше~--- знание и рассудительность;
\vs Pro 1:5 послушает мудрый~--- и умножит познания, и разумный найдет мудрые советы;
\vs Pro 1:6 чтобы разуметь притчу и замысловатую речь, слова мудрецов и загадки их.
\rsbpar\vs Pro 1:7 Начало мудрости~--- страх Господень; [доброе разумение у всех, водящихся им; а благоговение к Богу~--- начало разумения;] глупцы только презирают мудрость и наставление.
\vs Pro 1:8 Слушай, сын мой, наставление отца твоего и не отвергай завета матери твоей,
\vs Pro 1:9 потому что это~--- прекрасный венок для головы твоей и украшение для шеи твоей.
\vs Pro 1:10 Сын мой! если будут склонять тебя грешники, не соглашайся;
\vs Pro 1:11 если будут говорить: <<иди с нами, сделаем засаду для убийства, подстережем непорочного без вины,
\vs Pro 1:12 живых проглотим их, как преисподняя, и~--- целых, как нисходящих в могилу;
\vs Pro 1:13 наберем всякого драгоценного имущества, наполним домы наши добычею;
\vs Pro 1:14 жребий твой ты будешь бросать вместе с нами, склад один будет у всех нас>>,~---
\vs Pro 1:15 сын мой! не ходи в путь с ними, удержи ногу твою от стези их,
\vs Pro 1:16 потому что ноги их бегут ко злу и спешат на пролитие крови.
\vs Pro 1:17 В глазах всех птиц напрасно расставляется сеть,
\vs Pro 1:18 а делают засаду для их крови и подстерегают их души.
\vs Pro 1:19 Таковы пути всякого, кто алчет чужого добра: оно отнимает жизнь у завладевшего им.
\rsbpar\vs Pro 1:20 Премудрость возглашает на улице, на площадях возвышает голос свой,
\vs Pro 1:21 в главных местах собраний проповедует, при входах в городские ворота говорит речь свою:
\vs Pro 1:22 <<доколе, невежды, будете любить невежество? \bibemph{доколе} буйные будут услаждаться буйством? доколе глупцы будут ненавидеть знание?
\vs Pro 1:23 Обратитесь к моему обличению: вот, я изолью на вас дух мой, возвещу вам слова мои.
\vs Pro 1:24 Я звала, и вы не послушались; простирала руку мою, и не было внимающего;
\vs Pro 1:25 и вы отвергли все мои советы, и обличений моих не приняли.
\vs Pro 1:26 За то и я посмеюсь вашей погибели; порадуюсь, когда придет на вас ужас;
\vs Pro 1:27 когда придет на вас ужас, как буря, и беда, как вихрь, принесется на вас; когда постигнет вас скорбь и теснота.
\vs Pro 1:28 Тогда будут звать меня, и я не услышу; с утра будут искать меня, и не найдут меня.
\vs Pro 1:29 За то, что они возненавидели знание и не избрали \bibemph{для себя} страха Господня,
\vs Pro 1:30 не приняли совета моего, презрели все обличения мои;
\vs Pro 1:31 за то и будут они вкушать от плодов путей своих и насыщаться от помыслов их.
\vs Pro 1:32 Потому что упорство невежд убьет их, и беспечность глупцов погубит их,
\vs Pro 1:33 а слушающий меня будет жить безопасно и спокойно, не страшась зла>>.
\vs Pro 2:1 Сын мой! если ты примешь слова мои и сохранишь при себе заповеди мои,
\vs Pro 2:2 так что ухо твое сделаешь внимательным к мудрости и наклонишь сердце твое к размышлению;
\vs Pro 2:3 если будешь призывать знание и взывать к разуму;
\vs Pro 2:4 если будешь искать его, как серебра, и отыскивать его, как сокровище,
\vs Pro 2:5 то уразумеешь страх Господень и найдешь познание о Боге.
\vs Pro 2:6 Ибо Господь дает мудрость; из уст Его~--- знание и разум;
\vs Pro 2:7 Он сохраняет для праведных спасение; Он~--- щит для ходящих непорочно;
\vs Pro 2:8 Он охраняет пути правды и оберегает стезю святых Своих.
\vs Pro 2:9 Тогда ты уразумеешь правду и правосудие и прямоту, всякую добрую стезю.
\vs Pro 2:10 Когда мудрость войдет в сердце твое, и знание будет приятно душе твоей,
\vs Pro 2:11 тогда рассудительность будет оберегать тебя, разум будет охранять тебя,
\vs Pro 2:12 дабы спасти тебя от пути злого, от человека, говорящего ложь,
\vs Pro 2:13 от тех, которые оставляют стези прямые, чтобы ходить путями тьмы;
\vs Pro 2:14 от тех, которые радуются, делая зло, восхищаются злым развратом,
\vs Pro 2:15 которых пути кривы, и которые блуждают на стезях своих;
\vs Pro 2:16 дабы спасти тебя от жены другого, от чужой, которая умягчает речи свои,
\vs Pro 2:17 которая оставила руководителя юности своей и забыла завет Бога своего.
\vs Pro 2:18 Дом ее ведет к смерти, и стези ее~--- к мертвецам;
\vs Pro 2:19 никто из вошедших к ней не возвращается и не вступает на путь жизни.
\vs Pro 2:20 Посему ходи путем добрых и держись стезей праведников,
\vs Pro 2:21 потому что праведные будут жить на земле, и непорочные пребудут на ней;
\vs Pro 2:22 а беззаконные будут истреблены с земли, и вероломные искоренены из нее.
\vs Pro 3:1 Сын мой! наставления моего не забывай, и заповеди мои да хранит сердце твое;
\vs Pro 3:2 ибо долготы дней, лет жизни и мира они приложат тебе.
\vs Pro 3:3 Милость и истина да не оставляют тебя: обвяжи ими шею твою, напиши их на скрижали сердца твоего,
\vs Pro 3:4 и обретешь милость и благоволение в очах Бога и людей.
\vs Pro 3:5 Надейся на Господа всем сердцем твоим, и не полагайся на разум твой.
\vs Pro 3:6 Во всех путях твоих познавай Его, и Он направит стези твои.
\vs Pro 3:7 Не будь мудрецом в глазах твоих; бойся Господа и удаляйся от зла:
\vs Pro 3:8 это будет здравием для тела твоего и питанием для костей твоих.
\vs Pro 3:9 Чти Господа от имения твоего и от начатков всех прибытков твоих,
\vs Pro 3:10 и наполнятся житницы твои до избытка, и точила твои будут переливаться новым вином.
\rsbpar\vs Pro 3:11 Наказания Господня, сын мой, не отвергай, и не тяготись обличением Его;
\vs Pro 3:12 ибо кого любит Господь, того наказывает и благоволит к тому, как отец к сыну своему.
\vs Pro 3:13 Блажен человек, который снискал мудрость, и человек, который приобрел разум,~---
\vs Pro 3:14 потому что приобретение ее лучше приобретения серебра, и прибыли от нее больше, нежели от золота:
\vs Pro 3:15 она дороже драгоценных камней; [никакое зло не может противиться ей; она хорошо известна всем, приближающимся к ней,] и ничто из желаемого тобою не сравнится с нею.
\vs Pro 3:16 Долгоденствие~--- в правой руке ее, а в левой у нее~--- богатство и слава; [из уст ее выходит правда; закон и милость она на языке носит;]
\vs Pro 3:17 пути ее~--- пути приятные, и все стези ее~--- мирные.
\vs Pro 3:18 Она~--- древо жизни для тех, которые приобретают ее,~--- и блаженны, которые сохраняют ее!
\rsbpar\vs Pro 3:19 Господь премудростью основал землю, небеса утвердил разумом;
\vs Pro 3:20 Его премудростью разверзлись бездны, и облака кропят росою.
\vs Pro 3:21 Сын мой! не упускай их из глаз твоих; храни здравомыслие и рассудительность,
\vs Pro 3:22 и они будут жизнью для души твоей и украшением для шеи твоей.
\vs Pro 3:23 Тогда безопасно пойдешь по пути твоему, и нога твоя не споткнется.
\vs Pro 3:24 Когда ляжешь спать,~--- не будешь бояться; и когда уснешь,~--- сон твой приятен будет.
\vs Pro 3:25 Не убоишься внезапного страха и пагубы от нечестивых, когда она придет;
\vs Pro 3:26 потому что Господь будет упованием твоим и сохранит ногу твою от уловления.
\rsbpar\vs Pro 3:27 Не отказывай в благодеянии нуждающемуся, когда рука твоя в силе сделать его.
\vs Pro 3:28 Не говори другу твоему: <<пойди и приди опять, и завтра я дам>>, когда ты имеешь при себе. [Ибо ты не знаешь, чт\acc{о} родит грядущий день.]
\vs Pro 3:29 Не замышляй против ближнего твоего зла, когда он без опасения живет с тобою.
\vs Pro 3:30 Не ссорься с человеком без причины, когда он не сделал зла тебе.
\vs Pro 3:31 Не соревнуй человеку, поступающему насильственно, и не избирай ни одного из путей его;
\vs Pro 3:32 потому что мерзость пред Господом развратный, а с праведными у Него общение.
\vs Pro 3:33 Проклятие Господне на доме нечестивого, а жилище благочестивых Он благословляет.
\vs Pro 3:34 Если над кощунниками Он посмевается, то смиренным дает благодать.
\vs Pro 3:35 Мудрые наследуют славу, а глупые~--- бесславие.
\vs Pro 4:1 Слушайте, дети, наставление отца, и внимайте, чтобы научиться разуму,
\vs Pro 4:2 потому что я преподал вам доброе учение. Не оставляйте заповеди моей.
\vs Pro 4:3 Ибо и я был сын у отца моего, нежно любимый и единственный у матери моей,
\vs Pro 4:4 и он учил меня и говорил мне: да удержит сердце твое слова мои; храни заповеди мои, и живи.
\vs Pro 4:5 Приобретай мудрость, приобретай разум: не забывай этого и не уклоняйся от слов уст моих.
\vs Pro 4:6 Не оставляй ее, и она будет охранять тебя; люби ее, и она будет оберегать тебя.
\vs Pro 4:7 Главное~--- мудрость: приобретай мудрость, и всем имением твоим приобретай разум.
\vs Pro 4:8 Высоко цени ее, и она возвысит тебя; она прославит тебя, если ты прилепишься к ней;
\vs Pro 4:9 возложит на голову твою прекрасный венок, доставит тебе великолепный венец.
\rsbpar\vs Pro 4:10 Слушай, сын мой, и прими слова мои,~--- и умножатся тебе лета жизни.
\vs Pro 4:11 Я указываю тебе путь мудрости, веду тебя по стезям прямым.
\vs Pro 4:12 Когда пойдешь, не будет стеснен ход твой, и когда побежишь, не споткнешься.
\vs Pro 4:13 Крепко держись наставления, не оставляй, храни его, потому что оно~--- жизнь твоя.
\vs Pro 4:14 Не вступай на стезю нечестивых и не ходи по пути злых;
\vs Pro 4:15 оставь его, не ходи по нему, уклонись от него и пройди мимо;
\vs Pro 4:16 потому что они не заснут, если не сделают зла; пропадает сон у них, если они не доведут кого до падения;
\vs Pro 4:17 ибо они едят хлеб беззакония и пьют вино хищения.
\vs Pro 4:18 Стезя праведных~--- как светило лучезарное, которое более и более светлеет до полного дня.
\vs Pro 4:19 Путь же беззаконных~--- как тьма; они не знают, обо что споткнутся.
\rsbpar\vs Pro 4:20 Сын мой! словам моим внимай, и к речам моим приклони ухо твое;
\vs Pro 4:21 да не отходят они от глаз твоих; храни их внутри сердца твоего:
\vs Pro 4:22 потому что они жизнь для того, кто нашел их, и здравие для всего тела его.
\vs Pro 4:23 Больше всего хранимого храни сердце твое, потому что из него источники жизни.
\vs Pro 4:24 Отвергни от себя лживость уст, и лукавство языка удали от себя.
\vs Pro 4:25 Глаза твои пусть прямо смотрят, и ресницы твои да направлены будут прямо пред тобою.
\vs Pro 4:26 Обдумай стезю для ноги твоей, и все пути твои да будут тверды.
\vs Pro 4:27 Не уклоняйся ни направо, ни налево; удали ногу твою от зла,
\vs Pro 4:28 [потому что пути правые наблюдает Господь, а левые~--- испорчены.
\vs Pro 4:29 Он же прямыми сделает пути твои, и шествия твои в мире устроит.]
\vs Pro 5:1 Сын мой! внимай мудрости моей, и приклони ухо твое к разуму моему,
\vs Pro 5:2 чтобы соблюсти рассудительность, и чтобы уста твои сохранили знание. [Не внимай льстивой женщине;]
\vs Pro 5:3 ибо мед источают уста чужой жены, и мягче елея речь ее;
\vs Pro 5:4 но последствия от нее горьки, как полынь, остры, как меч обоюдоострый;
\vs Pro 5:5 ноги ее нисходят к смерти, стопы ее достигают преисподней.
\vs Pro 5:6 Если бы ты захотел постигнуть стезю жизни ее, то пути ее непостоянны, и ты не узнаешь их.
\vs Pro 5:7 Итак, дети, слушайте меня и не отступайте от слов уст моих.
\vs Pro 5:8 Держи дальше от нее путь твой и не подходи близко к дверям дома ее,
\vs Pro 5:9 чтобы здоровья твоего не отдать другим и лет твоих мучителю;
\vs Pro 5:10 чтобы не насыщались силою твоею чужие, и труды твои не были для чужого дома.
\vs Pro 5:11 И ты будешь стонать после, когда плоть твоя и тело твое будут истощены,~---
\vs Pro 5:12 и скажешь: <<зачем я ненавидел наставление, и сердце мое пренебрегало обличением,
\vs Pro 5:13 и я не слушал голоса учителей моих, не приклонял уха моего к наставникам моим:
\vs Pro 5:14 едва не впал я во всякое зло среди собрания и общества!>>
\rsbpar\vs Pro 5:15 Пей воду из твоего водоема и текущую из твоего колодезя.
\vs Pro 5:16 Пусть [не] разливаются источники твои по улице, потоки вод~--- по площадям;
\vs Pro 5:17 пусть они будут принадлежать тебе одному, а не чужим с тобою.
\vs Pro 5:18 Источник твой да будет благословен; и утешайся женою юности твоей,
\vs Pro 5:19 любезною ланью и прекрасною серною: груди ее да упоявают тебя во всякое время, любовью ее услаждайся постоянно.
\vs Pro 5:20 И для чего тебе, сын мой, увлекаться постороннею и обнимать груди чужой?
\vs Pro 5:21 Ибо пред очами Господа пути человека, и Он измеряет все стези его.
\vs Pro 5:22 Беззаконного уловляют собственные беззакония его, и в узах греха своего он содержится:
\vs Pro 5:23 он умирает без наставления, и от множества безумия своего теряется.
\vs Pro 6:1 Сын мой! если ты поручился за ближнего твоего и дал руку твою за другого,~---
\vs Pro 6:2 ты опутал себя словами уст твоих, пойман словами уст твоих.
\vs Pro 6:3 Сделай же, сын мой, вот что, и избавь себя, так как ты попался в руки ближнего твоего: пойди, пади к ногам и умоляй ближнего твоего;
\vs Pro 6:4 не давай сна глазам твоим и дремания веждам твоим;
\vs Pro 6:5 спасайся, как серна из руки и как птица из руки птицелова.
\rsbpar\vs Pro 6:6 Пойди к муравью, ленивец, посмотри на действия его, и будь мудрым.
\vs Pro 6:7 Нет у него ни начальника, ни приставника, ни повелителя;
\vs Pro 6:8 но он заготовляет летом хлеб свой, собирает во время жатвы пищу свою. [Или пойди к пчеле и познай, как она трудолюбива, какую почтенную работу она производит; ее труды употребляют во здравие и цари и простолюдины; любима же она всеми и славна; хотя силою она слаба, но мудростью почтена.]
\vs Pro 6:9 Доколе ты, ленивец, будешь спать? когда ты встанешь от сна твоего?
\vs Pro 6:10 Немного поспишь, немного подремлешь, немного, сложив руки, полежишь:
\vs Pro 6:11 и придет, как прохожий, бедность твоя, и нужда твоя, как разбойник. [Если же будешь не ленив, то, как источник, придет жатва твоя; скудость же далеко убежит от тебя.]
\rsbpar\vs Pro 6:12 Человек лукавый, человек нечестивый ходит со лживыми устами,
\vs Pro 6:13 мигает глазами своими, говорит ногами своими, дает знаки пальцами своими;
\vs Pro 6:14 коварство в сердце его: он умышляет зло во всякое время, сеет раздоры.
\vs Pro 6:15 Зато внезапно придет погибель его, вдруг будет разбит~--- без исцеления.
\vs Pro 6:16 Вот шесть, чт\acc{о} ненавидит Господь, даже семь, чт\acc{о} мерзость душе Его:
\vs Pro 6:17 глаза гордые, язык лживый и руки, проливающие кровь невинную,
\vs Pro 6:18 сердце, кующее злые замыслы, ноги, быстро бегущие к злодейству,
\vs Pro 6:19 лжесвидетель, наговаривающий ложь и сеющий раздор между братьями.
\rsbpar\vs Pro 6:20 Сын мой! храни заповедь отца твоего и не отвергай наставления матери твоей;
\vs Pro 6:21 навяжи их навсегда на сердце твое, обвяжи ими шею твою.
\vs Pro 6:22 Когда ты пойдешь, они будут руководить тебя; когда ляжешь спать, будут охранять тебя; когда пробудишься, будут беседовать с тобою:
\vs Pro 6:23 ибо заповедь есть светильник, и наставление~--- свет, и назидательные поучения~--- путь к жизни,
\vs Pro 6:24 чтобы остерегать тебя от негодной женщины, от льстивого языка чужой.
\vs Pro 6:25 Не пожелай красоты ее в сердце твоем, [да не уловлен будешь очами твоими,] и да не увлечет она тебя ресницами своими;
\vs Pro 6:26 потому что из-за жены блудной \bibemph{обнищевают} до куска хлеба, а замужняя жена уловляет дорогую душу.
\vs Pro 6:27 Может ли кто взять себе огонь в пазуху, чтобы не прогорело платье его?
\vs Pro 6:28 Может ли кто ходить по горящим угольям, чтобы не обжечь ног своих?
\vs Pro 6:29 То же бывает и с тем, кто входит к жене ближнего своего: кто прикоснется к ней, не останется без вины.
\vs Pro 6:30 Не спускают вору, если он крадет, чтобы насытить душу свою, когда он голоден;
\vs Pro 6:31 но, будучи пойман, он заплатит всемеро, отдаст все имущество дома своего.
\vs Pro 6:32 Кто же прелюбодействует с женщиною, у того нет ума; тот губит душу свою, кто делает это:
\vs Pro 6:33 побои и позор найдет он, и бесчестие его не изгладится,
\vs Pro 6:34 потому что ревность~--- ярость мужа, и не пощадит он в день мщения,
\vs Pro 6:35 не примет никакого выкупа и не удовольствуется, сколько бы ты ни умножал даров.
\vs Pro 7:1 Сын мой! храни слова мои и заповеди мои сокрой у себя. [Сын мой! чти Господа,~--- и укрепишься, и кроме Его не бойся никого.]
\vs Pro 7:2 Храни заповеди мои и живи, и учение мое, как зрачок глаз твоих.
\vs Pro 7:3 Навяжи их на персты твои, напиши их на скрижали сердца твоего.
\vs Pro 7:4 Скажи мудрости: <<ты сестра моя!>> и разум назови родным твоим,
\vs Pro 7:5 чтобы они охраняли тебя от жены другого, от чужой, которая умягчает слова свои.
\vs Pro 7:6 Вот, однажды смотрел я в окно дома моего, сквозь решетку мою,
\vs Pro 7:7 и увидел среди неопытных, заметил между молодыми людьми неразумного юношу,
\vs Pro 7:8 переходившего площадь близ угла ее и шедшего по дороге к дому ее,
\vs Pro 7:9 в сумерки в вечер дня, в ночной темноте и во мраке.
\vs Pro 7:10 И вот~--- навстречу к нему женщина, в наряде блудницы, с коварным сердцем,
\vs Pro 7:11 шумливая и необузданная; ноги ее не живут в доме ее:
\vs Pro 7:12 то на улице, то на площадях, и у каждого угла строит она ковы.
\vs Pro 7:13 Она схватила его, целовала его, и с бесстыдным лицом говорила ему:
\vs Pro 7:14 <<мирная жертва у меня: сегодня я совершила обеты мои;
\vs Pro 7:15 поэтому и вышла навстречу тебе, чтобы отыскать тебя, и~--- нашла тебя;
\vs Pro 7:16 коврами я убрала постель мою, разноцветными тканями Египетскими;
\vs Pro 7:17 спальню мою надушила смирною, алоем и корицею;
\vs Pro 7:18 зайди, будем упиваться нежностями до утра, насладимся любовью,
\vs Pro 7:19 потому что мужа нет дома: он отправился в дальнюю дорогу;
\vs Pro 7:20 кошелек серебра взял с собою; придет домой ко дню полнолуния>>.
\vs Pro 7:21 Множеством ласковых слов она увлекла его, мягкостью уст своих овладела им.
\vs Pro 7:22 Тотчас он пошел за нею, как вол идет на убой, [и как пес~--- на цепь,] и как олень~--- на выстрел,
\vs Pro 7:23 доколе стрела не пронзит печени его; как птичка кидается в силки, и не знает, что они~--- на погибель ее.
\vs Pro 7:24 Итак, дети, слушайте меня и внимайте словам уст моих.
\vs Pro 7:25 Да не уклоняется сердце твое на пути ее, не блуждай по стезям ее,
\vs Pro 7:26 потому что многих повергла она ранеными, и много сильных убиты ею:
\vs Pro 7:27 дом ее~--- пути в преисподнюю, нисходящие во внутренние жилища смерти.
\vs Pro 8:1 Не премудрость ли взывает? и не разум ли возвышает голос свой?
\vs Pro 8:2 Она становится на возвышенных местах, при дороге, на распутиях;
\vs Pro 8:3 она взывает у ворот при входе в город, при входе в двери:
\vs Pro 8:4 <<к вам, люди, взываю я, и к сынам человеческим голос мой!
\vs Pro 8:5 Научитесь, неразумные, благоразумию, и глупые~--- разуму.
\vs Pro 8:6 Слушайте, потому что я буду говорить важное, и изречение уст моих~--- правда;
\vs Pro 8:7 ибо истину произнесет язык мой, и нечестие~--- мерзость для уст моих;
\vs Pro 8:8 все слова уст моих справедливы; нет в них коварства и лукавства;
\vs Pro 8:9 все они ясны для разумного и справедливы для приобретших знание.
\vs Pro 8:10 Примите учение мое, а не серебро; лучше знание, нежели отборное золото;
\vs Pro 8:11 потому что мудрость лучше жемчуга, и ничто из желаемого не сравнится с нею.
\vs Pro 8:12 Я, премудрость, обитаю с разумом и ищу рассудительного знания.
\vs Pro 8:13 Страх Господень~--- ненавидеть зло; гордость и высокомерие и злой путь и коварные уста я ненавижу.
\vs Pro 8:14 У меня совет и правда; я разум, у меня сила.
\vs Pro 8:15 Мною цари царствуют и повелители узаконяют правду;
\vs Pro 8:16 мною начальствуют начальники и вельможи и все судьи земли.
\vs Pro 8:17 Любящих меня я люблю, и ищущие меня найдут меня;
\vs Pro 8:18 богатство и слава у меня, сокровище непогибающее и правда;
\vs Pro 8:19 плоды мои лучше золота, и золота самого чистого, и пользы от меня больше, нежели от отборного серебра.
\vs Pro 8:20 Я хожу по пути правды, по стезям правосудия,
\vs Pro 8:21 чтобы доставить любящим меня существенное благо, и сокровищницы их я наполняю. [Когда я возвещу то, что бывает ежедневно, то не забуду исчислить то, что от века.]
\rsbpar\vs Pro 8:22 Господь имел меня началом пути Своего, прежде созданий Своих, искони;
\vs Pro 8:23 от века я помазана, от начала, прежде бытия земли.
\vs Pro 8:24 Я родилась, когда еще не существовали бездны, когда еще не было источников, обильных водою.
\vs Pro 8:25 Я родилась прежде, нежели водружены были горы, прежде холмов,
\vs Pro 8:26 когда еще Он не сотворил ни земли, ни полей, ни начальных пылинок вселенной.
\vs Pro 8:27 Когда Он уготовлял небеса, \bibemph{я была} там. Когда Он проводил круговую черту по лицу бездны,
\vs Pro 8:28 когда утверждал вверху облака, когда укреплял источники бездны,
\vs Pro 8:29 когда давал морю устав, чтобы воды не переступали пределов его, когда полагал основания земли:
\vs Pro 8:30 тогда я была при Нем художницею, и была радостью всякий день, веселясь пред лицем Его во все время,
\vs Pro 8:31 веселясь на земном кругу Его, и радость моя \bibemph{была} с сынами человеческими.
\rsbpar\vs Pro 8:32 Итак, дети, послушайте меня; и блаженны те, которые хранят пути мои!
\vs Pro 8:33 Послушайте наставления и будьте мудры, и не отступайте \bibemph{от него}.
\vs Pro 8:34 Блажен человек, который слушает меня, бодрствуя каждый день у ворот моих и стоя на страже у дверей моих!
\vs Pro 8:35 потому что, кто нашел меня, тот нашел жизнь, и получит благодать от Господа;
\vs Pro 8:36 а согрешающий против меня наносит вред душе своей: все ненавидящие меня любят смерть>>.
\vs Pro 9:1 Премудрость построила себе дом, вытесала семь столбов его,
\vs Pro 9:2 заколола жертву, растворила вино свое и приготовила у себя трапезу;
\vs Pro 9:3 послала слуг своих провозгласить с возвышенностей городских:
\vs Pro 9:4 <<кто неразумен, обратись сюда!>> И скудоумному она сказала:
\vs Pro 9:5 <<идите, ешьте хлеб мой и пейте вино, мною растворенное;
\vs Pro 9:6 оставьте неразумие, и живите, и ходите путем разума>>.
\vs Pro 9:7 Поучающий кощунника наживет себе бесславие, и обличающий нечестивого~--- пятно себе.
\vs Pro 9:8 Не обличай кощунника, чтобы он не возненавидел тебя; обличай мудрого, и он возлюбит тебя;
\vs Pro 9:9 дай \bibemph{наставление} мудрому, и он будет еще мудрее; научи правдивого, и он приумножит знание.
\vs Pro 9:10 Начало мудрости~--- страх Господень, и познание Святаго~--- разум;
\vs Pro 9:11 потому что чрез меня умножатся дни твои, и прибавится тебе лет жизни.
\vs Pro 9:12 [Сын мой!] если ты мудр, то мудр для себя [и для ближних твоих]; и если буен, то один потерпишь. [Кто утверждается на лжи, тот пасет ветры, тот гоняется за птицами летающими: ибо он оставил пути своего виноградника и блуждает по тропинкам поля своего; проходит чрез безводную пустыню и землю, обреченную на жажду; собирает руками бесплодие.]
\rsbpar\vs Pro 9:13 Женщина безрассудная, шумливая, глупая и ничего не знающая
\vs Pro 9:14 садится у дверей дома своего на стуле, на возвышенных местах города,
\vs Pro 9:15 чтобы звать проходящих дорогою, идущих прямо своими путями:
\vs Pro 9:16 <<кто глуп, обратись сюда!>> и скудоумному сказала она:
\vs Pro 9:17 <<воды краденые сладки, и утаенный хлеб приятен>>.
\vs Pro 9:18 И он не знает, что мертвецы там, и что в глубине преисподней зазванные ею. [Но ты отскочи, не медли на месте, не останавливай взгляда твоего на ней; ибо таким образом ты пройдешь воду чужую. От воды чужой удаляйся, и из источника чужого не пей, чтобы пожить многое время, и чтобы прибавились тебе лета жизни.]
\vs Pro 10:1 Притчи Соломона. Сын мудрый радует отца, а сын глупый~--- огорчение для его матери.
\vs Pro 10:2 Не доставляют пользы сокровища неправедные, правда же избавляет от смерти.
\vs Pro 10:3 Не допустит Господь терпеть голод душе праведного, стяжание же нечестивых исторгнет.
\vs Pro 10:4 Ленивая рука делает бедным, а рука прилежных обогащает.
\vs Pro 10:5 Собирающий во время лета~--- сын разумный, спящий же во время жатвы~--- сын беспутный.
\vs Pro 10:6 Благословения~--- на голове праведника, уста же беззаконных заградит насилие.
\vs Pro 10:7 Память праведника пребудет благословенна, а имя нечестивых омерзеет.
\vs Pro 10:8 Мудрый сердцем принимает заповеди, а глупый устами преткнется.
\vs Pro 10:9 Кто ходит в непорочности, тот ходит безопасно; а кто превращает пути свои, тот будет наказан.
\vs Pro 10:10 Кто мигает глазами, тот причиняет досаду, а глупый устами преткнется.
\vs Pro 10:11 Уста праведника~--- источник жизни, уста же беззаконных заградит насилие.
\vs Pro 10:12 Ненависть возбуждает раздоры, но любовь покрывает все грехи.
\vs Pro 10:13 В устах разумного находится мудрость, но на теле глупого~--- розга.
\vs Pro 10:14 Мудрые сберегают знание, но уста глупого~--- близкая погибель.
\vs Pro 10:15 Имущество богатого~--- крепкий город его, беда для бедных~--- скудость их.
\vs Pro 10:16 Труды праведного~--- к жизни, успех нечестивого~--- ко греху.
\vs Pro 10:17 Кто хранит наставление, тот на пути к жизни; а отвергающий обличение~--- блуждает.
\vs Pro 10:18 Кто скрывает ненависть, у того уста лживые; и кто разглашает клевету, тот глуп.
\vs Pro 10:19 При многословии не миновать греха, а сдерживающий уста свои~--- разумен.
\vs Pro 10:20 Отборное серебро~--- язык праведного, сердце же нечестивых~--- ничтожество.
\vs Pro 10:21 Уста праведного пасут многих, а глупые умирают от недостатка разума.
\vs Pro 10:22 Благословение Господне~--- оно обогащает и печали с собою не приносит.
\vs Pro 10:23 Для глупого преступное деяние как бы забава, а человеку разумному свойственна мудрость.
\vs Pro 10:24 Чего страшится нечестивый, то и постигнет его, а желание праведников исполнится.
\vs Pro 10:25 Как проносится вихрь, \bibemph{так} нет более нечестивого; а праведник~--- на вечном основании.
\vs Pro 10:26 Что уксус для зубов и дым для глаз, то ленивый для посылающих его.
\vs Pro 10:27 Страх Господень прибавляет дней, лета же нечестивых сократятся.
\vs Pro 10:28 Ожидание праведников~--- радость, а надежда нечестивых погибнет.
\vs Pro 10:29 Путь Господень~--- твердыня для непорочного и страх для делающих беззаконие.
\vs Pro 10:30 Праведник во веки не поколеблется, нечестивые же не поживут на земле.
\vs Pro 10:31 Уста праведника источают мудрость, а язык зловредный отсечется.
\vs Pro 10:32 Уста праведного знают благоприятное, а уста нечестивых~--- развращенное.
\vs Pro 11:1 Неверные весы~--- мерзость пред Господом, но правильный вес угоден Ему.
\vs Pro 11:2 Придет гордость, придет и посрамление; но со смиренными~--- мудрость. [Праведник, умирая, оставляет сожаление; но внезапна и радостна бывает погибель нечестивых.]
\vs Pro 11:3 Непорочность прямодушных будет руководить их, а лукавство коварных погубит их.
\vs Pro 11:4 Не поможет богатство в день гнева, правда же спасет от смерти.
\vs Pro 11:5 Правда непорочного уравнивает путь его, а нечестивый падет от нечестия своего.
\vs Pro 11:6 Правда прямодушных спасет их, а беззаконники будут уловлены беззаконием своим.
\vs Pro 11:7 Со смертью человека нечестивого исчезает надежда, и ожидание беззаконных погибает.
\vs Pro 11:8 Праведник спасается от беды, а вместо него попадает \bibemph{в нее} нечестивый.
\vs Pro 11:9 Устами лицемер губит ближнего своего, но праведники прозорливостью спасаются.
\vs Pro 11:10 При благоденствии праведников веселится город, и при погибели нечестивых \bibemph{бывает} торжество.
\vs Pro 11:11 Благословением праведных возвышается город, а устами нечестивых разрушается.
\vs Pro 11:12 Скудоумный высказывает презрение к ближнему своему; но разумный человек молчит.
\vs Pro 11:13 Кто ходит переносчиком, тот открывает тайну; но верный человек таит дело.
\vs Pro 11:14 При недостатке попечения падает народ, а при многих советниках благоденствует.
\vs Pro 11:15 Зло причиняет себе, кто ручается за постороннего; а кто ненавидит ручательство, тот безопасен.
\vs Pro 11:16 Благонравная жена приобретает славу [мужу, а жена, ненавидящая правду, есть верх бесчестия. Ленивцы бывают скудны], а трудолюбивые приобретают богатство.
\vs Pro 11:17 Человек милосердый благотворит душе своей, а жестокосердый разрушает плоть свою.
\vs Pro 11:18 Нечестивый делает дело ненадежное, а сеющему правду~--- награда верная.
\vs Pro 11:19 Праведность \bibemph{ведет} к жизни, а стремящийся к злу \bibemph{стремится} к смерти своей.
\vs Pro 11:20 Мерзость пред Господом~--- коварные сердцем; но благоугодны Ему непорочные в пути.
\vs Pro 11:21 Можно поручиться, что порочный не останется ненаказанным; семя же праведных спасется.
\vs Pro 11:22 Что золотое кольцо в носу у свиньи, то женщина красивая и~--- безрассудная.
\vs Pro 11:23 Желание праведных \bibemph{есть} одно добро, ожидание нечестивых~--- гнев.
\vs Pro 11:24 Иной сыплет щедро, и \bibemph{ему} еще прибавляется; а другой сверх меры бережлив, и однако же беднеет.
\vs Pro 11:25 Благотворительная душа будет насыщена, и кто напояет \bibemph{других}, тот и сам напоен будет.
\vs Pro 11:26 Кто удерживает у себя хлеб, того клянет народ; а на голове продающего~--- благословение.
\vs Pro 11:27 Кто стремится к добру, тот ищет благоволения; а кто ищет зла, к тому оно и приходит.
\vs Pro 11:28 Надеющийся на богатство свое упадет; а праведники, как лист, будут зеленеть.
\vs Pro 11:29 Расстроивающий дом свой получит в удел ветер, и глупый будет рабом мудрого сердцем.
\vs Pro 11:30 Плод праведника~--- древо жизни, и мудрый привлекает души.
\vs Pro 11:31 Так праведнику воздается на земле, тем паче нечестивому и грешнику.
\vs Pro 12:1 Кто любит наставление, тот любит знание; а кто ненавидит обличение, тот невежда.
\vs Pro 12:2 Добрый приобретает благоволение от Господа; а человека коварного Он осудит.
\vs Pro 12:3 Не утвердит себя человек беззаконием; корень же праведников неподвижен.
\vs Pro 12:4 Добродетельная жена~--- венец для мужа своего; а позорная~--- как гниль в костях его.
\vs Pro 12:5 Помышления праведных~--- правда, а замыслы нечестивых~--- коварство.
\vs Pro 12:6 Речи нечестивых~--- засада для пролития крови, уста же праведных спасают их.
\vs Pro 12:7 Коснись нечестивых несчастие~--- и нет их, а дом праведных стоит.
\vs Pro 12:8 Хвалят человека по мере разума его, а развращенный сердцем будет в презрении.
\vs Pro 12:9 Лучше простой, но работающий на себя, нежели выдающий себя за знатного, но нуждающийся в хлебе.
\vs Pro 12:10 Праведный печется и о жизни скота своего, сердце же нечестивых жестоко.
\vs Pro 12:11 Кто возделывает землю свою, тот будет насыщаться хлебом; а кто идет по следам празднолюбцев, тот скудоумен. [Кто находит удовольствие в трате времени за вином, тот в своем доме оставит бесславие.]
\vs Pro 12:12 Нечестивый желает уловить в сеть зла; но корень праведных тверд.
\vs Pro 12:13 Нечестивый уловляется грехами уст своих; но праведник выйдет из беды. [Смотрящий кротко помилован будет, а встречающийся в воротах стеснит других.]
\vs Pro 12:14 От плода уст \bibemph{своих} человек насыщается добром, и воздаяние человеку~--- по делам рук его.
\vs Pro 12:15 Путь глупого прямой в его глазах; но кто слушает совета, тот мудр.
\vs Pro 12:16 У глупого тотчас же выкажется гнев его, а благоразумный скрывает оскорбление.
\vs Pro 12:17 Кто говорит то, что знает, тот говорит правду; а у свидетеля ложного~--- обман.
\vs Pro 12:18 Иной пустослов уязвляет как мечом, а язык мудрых~--- врачует.
\vs Pro 12:19 Уста правдивые вечно пребывают, а лживый язык~--- только на мгновение.
\vs Pro 12:20 Коварство~--- в сердце злоумышленников, радость~--- у миротворцев.
\vs Pro 12:21 Не приключится праведнику никакого зла, нечестивые же будут преисполнены зол.
\vs Pro 12:22 Мерзость пред Господом~--- уста лживые, а говорящие истину благоугодны Ему.
\vs Pro 12:23 Человек рассудительный скрывает знание, а сердце глупых высказывает глупость.
\vs Pro 12:24 Рука прилежных будет господствовать, а ленивая будет под данью.
\vs Pro 12:25 Тоска на сердце человека подавляет его, а доброе слово развеселяет его.
\vs Pro 12:26 Праведник указывает ближнему своему путь, а путь нечестивых вводит их в заблуждение.
\vs Pro 12:27 Ленивый не жарит своей дичи; а имущество человека прилежного многоценно.
\vs Pro 12:28 На пути правды~--- жизнь, и на стезе ее нет смерти.
\vs Pro 13:1 Мудрый сын \bibemph{слушает} наставление отца, а буйный не слушает обличения.
\vs Pro 13:2 От плода уст \bibemph{своих} человек вкусит добро, душа же законопреступников~--- зло.
\vs Pro 13:3 Кто хранит уста свои, тот бережет душу свою; а кто широко раскрывает свой рот, тому беда.
\vs Pro 13:4 Душа ленивого желает, но тщетно; а душа прилежных насытится.
\vs Pro 13:5 Праведник ненавидит ложное слово, а нечестивый срамит и бесчестит \bibemph{себя}.
\vs Pro 13:6 Правда хранит непорочного в пути, а нечестие губит грешника.
\vs Pro 13:7 Иной выдает себя за богатого, а у него ничего нет; другой выдает себя за бедного, а у него богатства много.
\vs Pro 13:8 Богатством своим человек выкупает жизнь \bibemph{свою}, а бедный и угрозы не слышит.
\vs Pro 13:9 Свет праведных весело горит, светильник же нечестивых угасает. [Души коварные блуждают в грехах, а праведники сострадают и милуют.]
\vs Pro 13:10 От высокомерия происходит раздор, а у советующихся~--- мудрость.
\vs Pro 13:11 Богатство от суетности истощается, а собирающий трудами умножает его.
\vs Pro 13:12 Надежда, долго не сбывающаяся, томит сердце, а исполнившееся желание~--- \bibemph{как} древо жизни.
\vs Pro 13:13 Кто пренебрегает словом, тот причиняет вред себе; а кто боится заповеди, тому воздается.
\vs Pro 13:14 [У сына лукавого ничего нет доброго, а у разумного раба дела благоуспешны, и путь его прямой.]
\vs Pro 13:15 Учение мудрого~--- источник жизни, удаляющий от сетей смерти.
\vs Pro 13:16 Добрый разум доставляет приятность, путь же беззаконных жесток.
\vs Pro 13:17 Всякий благоразумный действует с знанием, а глупый выставляет напоказ глупость.
\vs Pro 13:18 Худой посол попадает в беду, а верный посланник~--- спасение.
\vs Pro 13:19 Нищета и посрамление отвергающему учение; а кто соблюдает наставление, будет в чести.
\vs Pro 13:20 Желание исполнившееся~--- приятно для души; но несносно для глупых уклоняться от зла.
\vs Pro 13:21 Общающийся с мудрыми будет мудр, а кто дружит с глупыми, развратится.
\vs Pro 13:22 Грешников преследует зло, а праведникам воздается добром.
\vs Pro 13:23 Добрый оставляет наследство \bibemph{и} внукам, а богатство грешника сберегается для праведного.
\vs Pro 13:24 Много хлеба \bibemph{бывает} и на ниве бедных; но некоторые гибнут от беспорядка.
\vs Pro 13:25 Кто жалеет розги своей, тот ненавидит сына; а кто любит, тот с детства наказывает его.
\vs Pro 13:26 Праведник ест до сытости, а чрево беззаконных терпит лишение.
\vs Pro 14:1 Мудрая жена устроит дом свой, а глупая разрушит его своими руками.
\vs Pro 14:2 Идущий прямым путем боится Господа; но чьи пути кривы, тот небрежет о Нем.
\vs Pro 14:3 В устах глупого~--- бич гордости; уста же мудрых охраняют их.
\vs Pro 14:4 Где нет волов, \bibemph{там} ясли пусты; а много прибыли от силы волов.
\vs Pro 14:5 Верный свидетель не лжет, а свидетель ложный наговорит много лжи.
\vs Pro 14:6 Распутный ищет мудрости, и не находит; а для разумного знание легко.
\vs Pro 14:7 Отойди от человека глупого, у которого ты не замечаешь разумных уст.
\vs Pro 14:8 Мудрость разумного~--- знание пути своего, глупость же безрассудных~--- заблуждение.
\vs Pro 14:9 Глупые смеются над грехом, а посреди праведных~--- благоволение.
\vs Pro 14:10 Сердце знает горе души своей, и в радость его не вмешается чужой.
\vs Pro 14:11 Дом беззаконных разорится, а жилище праведных процветет.
\vs Pro 14:12 Есть пути, которые кажутся человеку прямыми; но конец их~--- путь к смерти.
\vs Pro 14:13 И при смехе \bibemph{иногда} болит сердце, и концом радости бывает печаль.
\vs Pro 14:14 Человек с развращенным сердцем насытится от путей своих, и добрый~--- от своих.
\vs Pro 14:15 Глупый верит всякому слову, благоразумный же внимателен к путям своим.
\vs Pro 14:16 Мудрый боится и удаляется от зла, а глупый раздражителен и самонадеян.
\vs Pro 14:17 Вспыльчивый может сделать глупость; но человек, умышленно делающий зло, ненавистен.
\vs Pro 14:18 Невежды получают в удел себе глупость, а благоразумные увенчаются знанием.
\vs Pro 14:19 Преклонятся злые пред добрыми и нечестивые~--- у ворот праведника.
\vs Pro 14:20 Бедный ненавидим бывает даже близким своим, а у богатого много друзей.
\vs Pro 14:21 Кто презирает ближнего своего, тот грешит; а кто милосерд к бедным, тот блажен.
\vs Pro 14:22 Не заблуждаются ли умышляющие зло? [не знают милости и верности делающие зло;] но милость и верность у благомыслящих.
\vs Pro 14:23 От всякого труда есть прибыль, а от пустословия только ущерб.
\vs Pro 14:24 Венец мудрых~--- богатство их, а глупость невежд глупость \bibemph{и есть}.
\vs Pro 14:25 Верный свидетель спасает души, а лживый наговорит много лжи.
\vs Pro 14:26 В страхе пред Господом~--- надежда твердая, и сынам Своим Он прибежище.
\vs Pro 14:27 Страх Господень~--- источник жизни, удаляющий от сетей смерти.
\vs Pro 14:28 Во множестве народа~--- величие царя, а при малолюдстве народа беда государю.
\vs Pro 14:29 У терпеливого человека много разума, а раздражительный выказывает глупость.
\vs Pro 14:30 Кроткое сердце~--- жизнь для тела, а зависть~--- гниль для костей.
\vs Pro 14:31 Кто теснит бедного, тот хулит Творца его; чтущий же Его благотворит нуждающемуся.
\vs Pro 14:32 За зло свое нечестивый будет отвергнут, а праведный и при смерти своей имеет надежду.
\vs Pro 14:33 Мудрость почиет в сердце разумного, и среди глупых дает знать о себе.
\vs Pro 14:34 Праведность возвышает народ, а беззаконие~--- бесчестие народов.
\vs Pro 14:35 Благоволение царя~--- к рабу разумному, а гнев его~--- против того, кто позорит его.
\vs Pro 15:1 [Гнев губит и разумных.] Кроткий ответ отвращает гнев, а оскорбительное слово возбуждает ярость.
\vs Pro 15:2 Язык мудрых сообщает добрые знания, а уста глупых изрыгают глупость.
\vs Pro 15:3 На всяком месте очи Господни: они видят злых и добрых.
\vs Pro 15:4 Кроткий язык~--- древо жизни, но необузданный~--- сокрушение духа.
\vs Pro 15:5 Глупый пренебрегает наставлением отца своего; а кто внимает обличениям, тот благоразумен. [В обилии правды великая сила, а нечестивые искоренятся из земли.]
\vs Pro 15:6 В доме праведника~--- обилие сокровищ, а в прибытке нечестивого~--- расстройство.
\vs Pro 15:7 Уста мудрых распространяют знание, а сердце глупых не так.
\vs Pro 15:8 Жертва нечестивых~--- мерзость пред Господом, а молитва праведных благоугодна Ему.
\vs Pro 15:9 Мерзость пред Господом~--- путь нечестивого, а идущего путем правды Он любит.
\vs Pro 15:10 Злое наказание~--- уклоняющемуся от пути, и ненавидящий обличение погибнет.
\vs Pro 15:11 Преисподняя и Аваддон \bibemph{открыты} пред Господом, тем более сердца сынов человеческих.
\vs Pro 15:12 Не любит распутный обличающих его, и к мудрым не пойдет.
\vs Pro 15:13 Веселое сердце делает лице веселым, а при сердечной скорби дух унывает.
\vs Pro 15:14 Сердце разумного ищет знания, уста же глупых питаются глупостью.
\vs Pro 15:15 Все дни несчастного печальны; а у кого сердце весело, у того всегда пир.
\vs Pro 15:16 Лучше немногое при страхе Господнем, нежели большое сокровище, и при нем тревога.
\vs Pro 15:17 Лучше блюдо зелени, и при нем любовь, нежели откормленный бык, и при нем ненависть.
\vs Pro 15:18 Вспыльчивый человек возбуждает раздор, а терпеливый утишает распрю.
\vs Pro 15:19 Путь ленивого~--- как терновый плетень, а путь праведных~--- гладкий.
\vs Pro 15:20 Мудрый сын радует отца, а глупый человек пренебрегает мать свою.
\vs Pro 15:21 Глупость~--- радость для малоумного, а человек разумный идет прямою дорогою.
\vs Pro 15:22 Без совета предприятия расстроятся, а при множестве советников они состоятся.
\vs Pro 15:23 Радость человеку в ответе уст его, и как хорошо слово вовремя!
\vs Pro 15:24 Путь жизни мудрого вверх, чтобы уклониться от преисподней внизу.
\vs Pro 15:25 Дом надменных разорит Господь, а межу вдовы укрепит.
\vs Pro 15:26 Мерзость пред Господом~--- помышления злых, слова же непорочных угодны Ему.
\vs Pro 15:27 Корыстолюбивый расстроит дом свой, а ненавидящий подарки будет жить.
\vs Pro 15:28 Сердце праведного обдумывает ответ, а уста нечестивых изрыгают зло. [Приятны пред Господом пути праведных; чрез них и враги делаются друзьями.]
\vs Pro 15:29 Далек Господь от нечестивых, а молитву праведников слышит.
\vs Pro 15:30 Светлый взгляд радует сердце, добрая весть утучняет кости.
\vs Pro 15:31 Ухо, внимательное к учению жизни, пребывает между мудрыми.
\vs Pro 15:32 Отвергающий наставление не радеет о своей душе; а кто внимает обличению, тот приобретает разум.
\vs Pro 15:33 Страх Господень научает мудрости, и славе предшествует смирение.
\vs Pro 16:1 Человеку \bibemph{принадлежат} предположения сердца, но от Господа ответ языка.
\vs Pro 16:2 Все пути человека чисты в его глазах, но Господь взвешивает души.
\vs Pro 16:3 Предай Господу дела твои, и предприятия твои совершатся.
\vs Pro 16:4 Все сделал Господь ради Себя; и даже нечестивого \bibemph{блюдет} на день бедствия.
\vs Pro 16:5 Мерзость пред Господом всякий надменный сердцем; можно поручиться, что он не останется ненаказанным. [Начало доброго пути~--- делать правду; это угоднее пред Богом, нежели приносить жертвы. Ищущий Господа найдет знание с правдою; истинно ищущие Его найдут мир.]
\vs Pro 16:6 Милосердием и правдою очищается грех, и страх Господень отводит от зла.
\vs Pro 16:7 Когда Господу угодны пути человека, Он и врагов его примиряет с ним.
\vs Pro 16:8 Лучше немногое с правдою, нежели множество прибытков с неправдою.
\vs Pro 16:9 Сердце человека обдумывает свой путь, но Господь управляет шествием его.
\vs Pro 16:10 В устах царя~--- слово вдохновенное; уста его не должны погрешать на суде.
\vs Pro 16:11 Верные весы и весовые чаши~--- от Господа; от Него же все гири в суме.
\vs Pro 16:12 Мерзость для царей~--- дело беззаконное, потому что правдою утверждается престол.
\vs Pro 16:13 Приятны царю уста правдивые, и говорящего истину он любит.
\vs Pro 16:14 Царский гнев~--- вестник смерти; но мудрый человек умилостивит его.
\vs Pro 16:15 В светлом взоре царя~--- жизнь, и благоволение его~--- как облако с поздним дождем.
\vs Pro 16:16 Приобретение мудрости гораздо лучше золота, и приобретение разума предпочтительнее отборного серебра.
\vs Pro 16:17 Путь праведных~--- уклонение от зла: тот бережет душу свою, кто хранит путь свой.
\vs Pro 16:18 Погибели предшествует гордость, и падению~--- надменность.
\vs Pro 16:19 Лучше смиряться духом с кроткими, нежели разделять добычу с гордыми.
\vs Pro 16:20 Кто ведет дело разумно, тот найдет благо, и кто надеется на Господа, тот блажен.
\vs Pro 16:21 Мудрый сердцем прозовется благоразумным, и сладкая речь прибавит к учению.
\vs Pro 16:22 Разум для имеющих его~--- источник жизни, а ученость глупых~--- глупость.
\vs Pro 16:23 Сердце мудрого делает язык его мудрым и умножает знание в устах его.
\vs Pro 16:24 Приятная речь~--- сотовый мед, сладка для души и целебна для костей.
\vs Pro 16:25 Есть пути, которые кажутся человеку прямыми, но конец их путь к смерти.
\vs Pro 16:26 Трудящийся трудится для себя, потому что понуждает его \bibemph{к тому} рот его.
\vs Pro 16:27 Человек лукавый замышляет зло, и на устах его как бы огонь палящий.
\vs Pro 16:28 Человек коварный сеет раздор, и наушник разлучает друзей.
\vs Pro 16:29 Человек неблагонамеренный развращает ближнего своего и ведет его на путь недобрый;
\vs Pro 16:30 прищуривает глаза свои, чтобы придумать коварство; закусывая себе губы, совершает злодейство; [он~--- печь злобы].
\vs Pro 16:31 Венец славы~--- седина, которая находится на пути правды.
\vs Pro 16:32 Долготерпеливый лучше храброго, и владеющий собою \bibemph{лучше} завоевателя города.
\vs Pro 16:33 В полу бросается жребий, но все решение его~--- от Господа.
\vs Pro 17:1 Лучше кусок сухого хлеба, и с ним мир, нежели дом, полный заколотого скота, с раздором.
\vs Pro 17:2 Разумный раб господствует над беспутным сыном и между братьями разделит наследство.
\vs Pro 17:3 Плавильня~--- для серебра, и горнило~--- для золота, а сердца испытывает Господь.
\vs Pro 17:4 Злодей внимает устам беззаконным, лжец слушается языка пагубного.
\vs Pro 17:5 Кто ругается над нищим, тот хулит Творца его; кто радуется несчастью, тот не останется ненаказанным [а милосердый помилован будет].
\vs Pro 17:6 Венец стариков~--- сыновья сыновей, и слава детей~--- родители их. [У верного целый мир богатства, а у неверного~--- ни обола.]
\vs Pro 17:7 Неприлична глупому важная речь, тем паче знатному~--- уста лживые.
\vs Pro 17:8 Подарок~--- драгоценный камень в глазах владеющего им: куда ни обратится он, успеет.
\vs Pro 17:9 Прикрывающий проступок ищет любви; а кто снова напоминает о нем, тот удаляет друга.
\vs Pro 17:10 На разумного сильнее действует выговор, нежели на глупого сто ударов.
\vs Pro 17:11 Возмутитель ищет только зла; поэтому жестокий ангел будет послан против него.
\vs Pro 17:12 Лучше встретить человеку медведицу, лишенную детей, нежели глупца с его глупостью.
\vs Pro 17:13 Кто за добро воздает злом, от дома того не отойдет зло.
\vs Pro 17:14 Начало ссоры~--- как прорыв воды; оставь ссору прежде, нежели разгорелась она.
\vs Pro 17:15 Оправдывающий нечестивого и обвиняющий праведного~--- оба мерзость пред Господом.
\vs Pro 17:16 К чему сокровище в руках глупца? Для приобретения мудрости \bibemph{у него} нет разума. [Кто высоким делает свой дом, тот ищет разбиться; а уклоняющийся от учения впадет в беды.]
\vs Pro 17:17 Друг любит во всякое время и, как брат, явится во время несчастья.
\vs Pro 17:18 Человек малоумный дает руку и ручается за ближнего своего.
\vs Pro 17:19 Кто любит ссоры, любит грех, и кто высоко поднимает ворота свои, тот ищет падения.
\vs Pro 17:20 Коварное сердце не найдет добра, и лукавый язык попадет в беду.
\vs Pro 17:21 Родил кто глупого,~--- себе на г\acc{о}ре, и отец глупого не порадуется.
\vs Pro 17:22 Веселое сердце благотворно, как врачевство, а унылый дух сушит кости.
\vs Pro 17:23 Нечестивый берет подарок из пазухи, чтобы извратить пути правосудия.
\vs Pro 17:24 Мудрость~--- пред лицем у разумного, а глаза глупца~--- на конце земли.
\vs Pro 17:25 Глупый сын~--- досада отцу своему и огорчение для матери своей.
\vs Pro 17:26 Нехорошо и обвинять правого, \bibemph{и} бить вельмож за правду.
\vs Pro 17:27 Разумный воздержан в словах своих, и благоразумный хладнокровен.
\vs Pro 17:28 И глупец, когда молчит, может показаться мудрым, и затворяющий уста свои~--- благоразумным.
\vs Pro 18:1 Прихоти ищет своенравный, восстает против всего умного.
\vs Pro 18:2 Глупый не любит знания, а только бы выказать свой ум.
\vs Pro 18:3 С приходом нечестивого приходит и презрение, а с бесславием~--- поношение.
\vs Pro 18:4 Слова уст человеческих~--- глубокие воды; источник мудрости~--- струящийся поток.
\vs Pro 18:5 Нехорошо быть лицеприятным к нечестивому, чтобы ниспровергнуть праведного на суде.
\vs Pro 18:6 Уста глупого идут в ссору, и слова его вызывают побои.
\vs Pro 18:7 Язык глупого~--- гибель для него, и уста его~--- сеть для души его.
\vs Pro 18:8 [Ленивого низлагает страх, а души женоподобные будут голодать.]
\vs Pro 18:9 Слова наушника~--- как лакомства, и они входят во внутренность чрева.
\vs Pro 18:10 Нерадивый в работе своей~--- брат расточителю.
\vs Pro 18:11 Имя Господа~--- крепкая башня: убегает в нее праведник~--- и безопасен.
\vs Pro 18:12 Имение богатого~--- крепкий город его, и как высокая ограда в его воображении.
\vs Pro 18:13 Перед падением возносится сердце человека, а смирение предшествует славе.
\vs Pro 18:14 Кто дает ответ не выслушав, тот глуп, и стыд ему.
\vs Pro 18:15 Дух человека переносит его немощи; а пораженный дух~--- кто может подкрепить его?
\vs Pro 18:16 Сердце разумного приобретает знание, и ухо мудрых ищет знания.
\vs Pro 18:17 Подарок у человека дает ему простор и до вельмож доведет его.
\vs Pro 18:18 Первый в тяжбе своей прав, но приходит соперник его и исследует его.
\vs Pro 18:19 Жребий прекращает споры и решает между сильными.
\vs Pro 18:20 Озлобившийся брат \bibemph{неприступнее} крепкого города, и ссоры подобны запорам з\acc{а}мка.
\vs Pro 18:21 От плода уст человека наполняется чрево его; произведением уст своих он насыщается.
\vs Pro 18:22 Смерть и жизнь~--- во власти языка, и любящие его вкусят от плодов его.
\vs Pro 18:23 Кто нашел [добрую] жену, тот нашел благо и получил благодать от Господа. [Кто изгоняет добрую жену, тот изгоняет счастье, а содержащий прелюбодейку~--- безумен и нечестив.]
\vs Pro 18:24 С мольбою говорит нищий, а богатый отвечает грубо.
\vs Pro 18:25 Кто хочет иметь друзей, тот и сам должен быть дружелюбным; и бывает друг, более привязанный, нежели брат.
\vs Pro 19:1 Лучше бедный, ходящий в своей непорочности, нежели [богатый] со лживыми устами, и притом глупый.
\vs Pro 19:2 Нехорошо душе без знания, и торопливый ногами оступится.
\vs Pro 19:3 Глупость человека извращает путь его, а сердце его негодует на Господа.
\vs Pro 19:4 Богатство прибавляет много друзей, а бедный оставляется и другом своим.
\vs Pro 19:5 Лжесвидетель не останется ненаказанным, и кто говорит ложь, не спасется.
\vs Pro 19:6 Многие заискивают у знатных, и всякий~--- друг человеку, делающему подарки.
\vs Pro 19:7 Бедного ненавидят все братья его, тем паче друзья его удаляются от него: гонится за ними, чтобы поговорить, но и этого нет.
\vs Pro 19:8 Кто приобретает разум, тот любит душу свою; кто наблюдает благоразумие, тот находит благо.
\vs Pro 19:9 Лжесвидетель не останется ненаказанным, и кто говорит ложь, погибнет.
\vs Pro 19:10 Неприлична глупцу пышность, тем паче рабу господство над князьями.
\vs Pro 19:11 Благоразумие делает человека медленным на гнев, и слава для него~--- быть снисходительным к проступкам.
\vs Pro 19:12 Гнев царя~--- как рев льва, а благоволение его~--- как роса на траву.
\vs Pro 19:13 Глупый сын~--- сокрушение для отца своего, и сварливая жена~--- сточная труба.
\vs Pro 19:14 Дом и имение~--- наследство от родителей, а разумная жена~--- от Господа.
\vs Pro 19:15 Леность погружает в сонливость, и нерадивая душа будет терпеть голод.
\vs Pro 19:16 Хранящий заповедь хранит душу свою, а нерадящий о путях своих погибнет.
\vs Pro 19:17 Благотворящий бедному дает взаймы Господу, и Он воздаст ему за благодеяние его.
\vs Pro 19:18 Наказывай сына своего, доколе есть надежда, и не возмущайся криком его.
\vs Pro 19:19 Гневливый пусть терпит наказание, потому что, если пощадишь \bibemph{его}, придется тебе еще больше наказывать его.
\vs Pro 19:20 Слушайся совета и принимай обличение, чтобы сделаться тебе впоследствии мудрым.
\vs Pro 19:21 Много замыслов в сердце человека, но состоится только определенное Господом.
\vs Pro 19:22 Радость человеку~--- благотворительность его, и бедный человек лучше, нежели лживый.
\vs Pro 19:23 Страх Господень \bibemph{ведет} к жизни, и \bibemph{кто имеет его}, всегда будет доволен, и зло не постигнет его.
\vs Pro 19:24 Ленивый опускает руку свою в чашу, и не хочет донести ее до рта своего.
\vs Pro 19:25 Если ты накажешь кощунника, то и простой сделается благоразумным; и \bibemph{если} обличишь разумного, то он поймет наставление.
\vs Pro 19:26 Разоряющий отца и выгоняющий мать~--- сын срамной и бесчестный.
\vs Pro 19:27 Перестань, сын мой, слушать внушения об уклонении от изречений разума.
\vs Pro 19:28 Лукавый свидетель издевается над судом, и уста беззаконных глотают неправду.
\vs Pro 19:29 Готовы для кощунствующих суды, и побои~--- на тело глупых.
\vs Pro 20:1 Вино~--- глумливо, сикера~--- буйна; и всякий, увлекающийся ими, неразумен.
\vs Pro 20:2 Гроза царя~--- как бы рев льва: кто раздражает его, тот грешит против самого себя.
\vs Pro 20:3 Честь для человека~--- отстать от ссоры; а всякий глупец задорен.
\vs Pro 20:4 Ленивец зимою не пашет: поищет летом~--- и нет ничего.
\vs Pro 20:5 Помыслы в сердце человека~--- глубокие воды, но человек разумный вычерпывает их.
\vs Pro 20:6 Многие хвалят человека за милосердие, но правдивого человека кто находит?
\vs Pro 20:7 Праведник ходит в своей непорочности: блаженны дети его после него!
\vs Pro 20:8 Царь, сидящий на престоле суда, разгоняет очами своими все злое.
\vs Pro 20:9 Кто может сказать: <<я очистил мое сердце, я чист от греха моего?>>
\vs Pro 20:10 Неодинаковые весы, неодинаковая мера, то и другое~--- мерзость пред Господом.
\vs Pro 20:11 Можно узнать даже отрока по занятиям его, чисто ли и правильно ли будет поведение его.
\vs Pro 20:12 Ухо слышащее и глаз видящий~--- и то и другое создал Господь.
\vs Pro 20:13 Не люби спать, чтобы тебе не обеднеть; держи открытыми глаза твои, и будешь досыта есть хлеб.
\vs Pro 20:14 <<Дурно, дурно>>, говорит покупатель, а когда отойдет, хвалится.
\vs Pro 20:15 Есть золото и много жемчуга, но драгоценная утварь~--- уста разумные.
\vs Pro 20:16 Возьми платье его, так как он поручился за чужого; и за стороннего возьми от него залог.
\vs Pro 20:17 Сладок для человека хлеб, \bibemph{приобретенный} неправдою; но после рот его наполнится дресвою.
\vs Pro 20:18 Предприятия получают твердость чрез совещание, и по совещании веди войну.
\vs Pro 20:19 Кто ходит переносчиком, тот открывает тайну; и кто широко раскрывает рот, с тем не сообщайся.
\vs Pro 20:20 Кто злословит отца своего и свою мать, того светильник погаснет среди глубокой тьмы.
\vs Pro 20:21 Наследство, поспешно захваченное вначале, не благословится впоследствии.
\vs Pro 20:22 Не говори: <<я отплачу за зло>>; предоставь Господу, и Он сохранит тебя.
\vs Pro 20:23 Мерзость пред Господом~--- неодинаковые гири, и неверные весы~--- не добро.
\vs Pro 20:24 От Господа направляются шаги человека; человеку же как узнать путь свой?
\vs Pro 20:25 Сеть для человека~--- поспешно давать обет, и после обета обдумывать.
\vs Pro 20:26 Мудрый царь вывеет нечестивых и обратит на них колесо.
\vs Pro 20:27 Светильник Господень~--- дух человека, испытывающий все глубины сердца.
\vs Pro 20:28 Милость и истина охраняют царя, и милостью он поддерживает престол свой.
\vs Pro 20:29 Слава юношей~--- сила их, а украшение стариков~--- седина.
\vs Pro 20:30 Раны от побоев~--- врачевство против зла, и удары, проникающие во внутренности чрева.
\vs Pro 21:1 Сердце царя~--- в руке Господа, как потоки вод: куда захочет, Он направляет его.
\vs Pro 21:2 Всякий путь человека прям в глазах его; но Господь взвешивает сердца.
\vs Pro 21:3 Соблюдение правды и правосудия более угодно Господу, нежели жертва.
\vs Pro 21:4 Гордость очей и надменность сердца, отличающие нечестивых,~--- грех.
\vs Pro 21:5 Помышления прилежного стремятся к изобилию, а всякий торопливый терпит лишение.
\vs Pro 21:6 Приобретение сокровища лживым языком~--- мимолетное дуновение ищущих смерти.
\vs Pro 21:7 Насилие нечестивых обрушится на них, потому что они отреклись соблюдать правду.
\vs Pro 21:8 Превратен путь человека развращенного; а кто чист, того действие прямо.
\vs Pro 21:9 Лучше жить в углу на кровле, нежели со сварливою женою в пространном доме.
\vs Pro 21:10 Душа нечестивого желает зла: не найдет милости в глазах его и друг его.
\vs Pro 21:11 Когда наказывается кощунник, простой делается мудрым; и когда вразумляется мудрый, то он приобретает знание.
\vs Pro 21:12 Праведник наблюдает за домом нечестивого: как повергаются нечестивые в несчастие.
\vs Pro 21:13 Кто затыкает ухо свое от вопля бедного, тот и сам будет вопить,~--- и не будет услышан.
\vs Pro 21:14 Подарок тайный тушит гнев, и дар в пазуху~--- сильную ярость.
\vs Pro 21:15 Соблюдение правосудия~--- радость для праведника и страх для делающих зло.
\vs Pro 21:16 Человек, сбившийся с пути разума, водворится в собрании мертвецов.
\vs Pro 21:17 Кто любит веселье, обеднеет; а кто любит вино и тук, не разбогатеет.
\vs Pro 21:18 Выкупом будет за праведного нечестивый и за прямодушного~--- лукавый.
\vs Pro 21:19 Лучше жить в земле пустынной, нежели с женою сварливою и сердитою.
\vs Pro 21:20 Вожделенное сокровище и тук~--- в доме мудрого; а глупый человек расточает их.
\vs Pro 21:21 Соблюдающий правду и милость найдет жизнь, правду и славу.
\vs Pro 21:22 Мудрый входит в город сильных и ниспровергает крепость, на которую они надеялись.
\vs Pro 21:23 Кто хранит уста свои и язык свой, тот хранит от бед душу свою.
\vs Pro 21:24 Надменный злодей~--- кощунник имя ему~--- действует в пылу гордости.
\vs Pro 21:25 Алчба ленивца убьет его, потому что руки его отказываются работать;
\vs Pro 21:26 всякий день он сильно алчет, а праведник дает и не жалеет.
\vs Pro 21:27 Жертва нечестивых~--- мерзость, особенно когда с лукавством приносят ее.
\vs Pro 21:28 Лжесвидетель погибнет; а человек, который говорит, что знает, будет говорить всегда.
\vs Pro 21:29 Человек нечестивый дерзок лицом своим, а праведный держит прямо путь свой.
\vs Pro 21:30 Нет мудрости, и нет разума, и нет совета вопреки Господу.
\vs Pro 21:31 Коня приготовляют на день битвы, но победа~--- от Господа.
\vs Pro 22:1 Доброе имя лучше большого богатства, и добрая слава лучше серебра и золота.
\vs Pro 22:2 Богатый и бедный встречаются друг с другом: того и другого создал Господь.
\vs Pro 22:3 Благоразумный видит беду, и укрывается; а неопытные идут вперед, и наказываются.
\vs Pro 22:4 За смирением следует страх Господень, богатство и слава и жизнь.
\vs Pro 22:5 Терны и сети на пути коварного; кто бережет душу свою, удались от них.
\vs Pro 22:6 Наставь юношу при начале пути его: он не уклонится от него, когда и состарится.
\vs Pro 22:7 Богатый господствует над бедным, и должник \bibemph{делается} рабом заимодавца.
\vs Pro 22:8 Сеющий неправду пожнет беду, и трости гнева его не станет. [Человека, доброхотно дающего, любит Бог, и недостаток дел его восполнит.]
\vs Pro 22:9 Милосердый будет благословляем, потому что дает бедному от хлеба своего. [Победу и честь приобретает дающий дары, и даже овладевает душею получающих оные.]
\vs Pro 22:10 Прогони кощунника, и удалится раздор, и прекратятся ссора и брань.
\vs Pro 22:11 Кто любит чистоту сердца, у того приятность на устах, тому царь~--- друг.
\vs Pro 22:12 Очи Господа охраняют знание, а слова законопреступника Он ниспровергает.
\vs Pro 22:13 Ленивец говорит: <<лев на улице! посреди площади убьют меня!>>
\vs Pro 22:14 Глубокая пропасть~--- уста блудниц: на кого прогневается Господь, тот упадет туда.
\vs Pro 22:15 Глупость привязалась к сердцу юноши, но исправительная розга удалит ее от него.
\vs Pro 22:16 Кто обижает бедного, чтобы умножить свое богатство, и кто дает богатому, тот обеднеет.
\rsbpar\vs Pro 22:17 Приклони ухо твое, и слушай слова мудрых, и сердце твое обрати к моему знанию;
\vs Pro 22:18 потому что утешительно будет, если ты будешь хранить их в сердце твоем, и они будут также в устах твоих.
\vs Pro 22:19 Чтобы упование твое было на Господа, я учу тебя и сегодня, и ты \bibemph{помни}.
\vs Pro 22:20 Не писал ли я тебе трижды в советах и наставлении,
\vs Pro 22:21 чтобы научить тебя точным словам истины, дабы ты мог передавать слова истины посылающим тебя?
\rsbpar\vs Pro 22:22 Не будь грабителем бедного, потому что он беден, и не притесняй несчастного у ворот,
\vs Pro 22:23 потому что Господь вступится в дело их и исхитит душу у грабителей их.
\vs Pro 22:24 Не дружись с гневливым и не сообщайся с человеком вспыльчивым,
\vs Pro 22:25 чтобы не научиться путям его и не навлечь петли на душу твою.
\vs Pro 22:26 Не будь из тех, которые дают руки и поручаются за долги:
\vs Pro 22:27 если тебе нечем заплатить, то для чего доводить себя, чтобы взяли постель твою из-под тебя?
\vs Pro 22:28 Не передвигай межи давней, которую провели отцы твои.
\vs Pro 22:29 Видел ли ты человека проворного в своем деле? Он будет стоять перед царями, он не будет стоять перед простыми.
\vs Pro 23:1 Когда сядешь вкушать пищу с властелином, то тщательно наблюдай, что перед тобою,
\vs Pro 23:2 и поставь преграду в гортани твоей, если ты алчен.
\vs Pro 23:3 Не прельщайся лакомыми яствами его; это~--- обманчивая пища.
\vs Pro 23:4 Не заботься о том, чтобы нажить богатство; оставь такие мысли твои.
\vs Pro 23:5 Устремишь глаза твои на него, и~--- его уже нет; потому что оно сделает себе крылья и, как орел, улетит к небу.
\vs Pro 23:6 Не вкушай пищи у человека завистливого и не прельщайся лакомыми яствами его;
\vs Pro 23:7 потому что, каковы мысли в душе его, таков и он; <<ешь и пей>>, говорит он тебе, а сердце его не с тобою.
\vs Pro 23:8 Кусок, который ты съел, изблюешь, и добрые слова твои ты потратишь напрасно.
\vs Pro 23:9 В уши глупого не говори, потому что он презрит разумные слова твои.
\vs Pro 23:10 Не передвигай межи давней и на поля сирот не заходи,
\vs Pro 23:11 потому что Защитник их силен; Он вступится в дело их с тобою.
\vs Pro 23:12 Приложи сердце твое к учению и уши твои~--- к умным словам.
\vs Pro 23:13 Не оставляй юноши без наказания: если накажешь его розгою, он не умрет;
\vs Pro 23:14 ты накажешь его розгою и спасешь душу его от преисподней.
\rsbpar\vs Pro 23:15 Сын мой! если сердце твое будет мудро, то порадуется и мое сердце;
\vs Pro 23:16 и внутренности мои будут радоваться, когда уста твои будут говорить правое.
\vs Pro 23:17 Да не завидует сердце твое грешникам, но да пребудет оно во все дни в страхе Господнем;
\vs Pro 23:18 потому что есть будущность, и надежда твоя не потеряна.
\vs Pro 23:19 Слушай, сын мой, и будь мудр, и направляй сердце твое на прямой путь.
\vs Pro 23:20 Не будь между упивающимися вином, между пресыщающимися мясом:
\vs Pro 23:21 потому что пьяница и пресыщающийся обеднеют, и сонливость оденет в рубище.
\vs Pro 23:22 Слушайся отца твоего: он родил тебя; и не пренебрегай матери твоей, когда она и состарится.
\vs Pro 23:23 Купи истину и не продавай мудрости и учения и разума.
\vs Pro 23:24 Торжествует отец праведника, и родивший мудрого радуется о нем.
\vs Pro 23:25 Да веселится отец твой и да торжествует мать твоя, родившая тебя.
\rsbpar\vs Pro 23:26 Сын мой! отдай сердце твое мне, и глаза твои да наблюдают пути мои,
\vs Pro 23:27 потому что блудница~--- глубокая пропасть, и чужая жена~--- тесный колодезь;
\vs Pro 23:28 она, как разбойник, сидит в засаде и умножает между людьми законопреступников.
\vs Pro 23:29 У кого вой? у кого стон? у кого ссоры? у кого горе? у кого раны без причины? у кого багровые глаза?
\vs Pro 23:30 У тех, которые долго сидят за вином, которые приходят отыскивать \bibemph{вина} приправленного.
\vs Pro 23:31 Не смотри на вино, как оно краснеет, как оно искрится в чаше, как оно ухаживается ровно:
\vs Pro 23:32 впоследствии, как змей, оно укусит, и ужалит, как аспид;
\vs Pro 23:33 глаза твои будут смотреть на чужих жен, и сердце твое заговорит развратное,
\vs Pro 23:34 и ты будешь, как спящий среди моря и как спящий на верху мачты.
\vs Pro 23:35 [И скажешь:] <<били меня, мне не было больно; толкали меня, я не чувствовал. Когда проснусь, опять буду искать того же>>.
\vs Pro 24:1 Не ревнуй злым людям и не желай быть с ними,
\vs Pro 24:2 потому что о насилии помышляет сердце их, и о злом говорят уста их.
\vs Pro 24:3 Мудростью устрояется дом и разумом утверждается,
\vs Pro 24:4 и с уменьем внутренности его наполняются всяким драгоценным и прекрасным имуществом.
\vs Pro 24:5 Человек мудрый силен, и человек разумный укрепляет силу свою.
\vs Pro 24:6 Поэтому с обдуманностью веди войну твою, и успех \bibemph{будет} при множестве совещаний.
\vs Pro 24:7 Для глупого слишком высока мудрость; у ворот не откроет он уст своих.
\vs Pro 24:8 Кто замышляет сделать зло, того называют злоумышленником.
\vs Pro 24:9 Помысл глупости~--- грех, и кощунник~--- мерзость для людей.
\vs Pro 24:10 Если ты в день бедствия оказался слабым, то бедна сила твоя.
\vs Pro 24:11 Спасай взятых на смерть, и неужели откажешься от обреченных на убиение?
\vs Pro 24:12 Скажешь ли: <<вот, мы не знали этого>>? А Испытующий сердц\acc{а} разве не знает? Наблюдающий над душею твоею знает это, и воздаст человеку по делам его.
\vs Pro 24:13 Ешь, сын мой, мед, потому что он приятен, и сот, который сладок для гортани твоей:
\vs Pro 24:14 таково и познание мудрости для души твоей. Если ты нашел \bibemph{ее}, то есть будущность, и надежда твоя не потеряна.
\vs Pro 24:15 Не злоумышляй, нечестивый, против жилища праведника, не опустошай места покоя его,
\vs Pro 24:16 ибо семь раз упадет праведник, и встанет; а нечестивые впадут в погибель.
\vs Pro 24:17 Не радуйся, когда упадет враг твой, и да не веселится сердце твое, когда он споткнется.
\vs Pro 24:18 Иначе, увидит Господь, и неугодно будет это в очах Его, и Он отвратит от него гнев Свой.
\vs Pro 24:19 Не негодуй на злодеев и не завидуй нечестивым,
\vs Pro 24:20 потому что злой не имеет будущности,~--- светильник нечестивых угаснет.
\vs Pro 24:21 Бойся, сын мой, Господа и царя; с мятежниками не сообщайся,
\vs Pro 24:22 потому что внезапно придет погибель от них, и беду от них обоих кто предузнает?
\vs Pro 24:23 Сказано также мудрыми: иметь лицеприятие на суде~--- нехорошо.
\vs Pro 24:24 Кто говорит виновному: <<ты прав>>, того будут проклинать народы, того будут ненавидеть племена;
\vs Pro 24:25 а обличающие будут любимы, и на них придет благословение.
\vs Pro 24:26 В уста целует, кто отвечает словами верными.
\vs Pro 24:27 Соверши дела твои вне дома, окончи их на поле твоем, и потом устрояй и дом твой.
\vs Pro 24:28 Не будь лжесвидетелем на ближнего твоего: к чему тебе обманывать устами твоими?
\vs Pro 24:29 Не говори: <<как он поступил со мною, так и я поступлю с ним, воздам человеку по делам его>>.
\vs Pro 24:30 Проходил я мимо поля человека ленивого и мимо виноградника человека скудоумного:
\vs Pro 24:31 и вот, все это заросло терном, поверхность его покрылась крапивою, и каменная ограда его обрушилась.
\vs Pro 24:32 И посмотрел я, и обратил сердце мое, и посмотрел и получил урок:
\vs Pro 24:33 <<немного поспишь, немного подремлешь, немного, сложив руки, полежишь,~---
\vs Pro 24:34 и придет, \bibemph{как} прохожий, бедность твоя, и нужда твоя~--- как человек вооруженный>>.
\vs Pro 25:1 И это притчи Соломона, которые собрали мужи Езекии, царя Иудейского.
\vs Pro 25:2 Слава Божия~--- облекать тайною дело, а слава царей~--- исследовать дело.
\vs Pro 25:3 Как небо в высоте и земля в глубине, так сердце царей~--- неисследимо.
\vs Pro 25:4 Отдели примесь от серебра, и выйдет у серебряника сосуд:
\vs Pro 25:5 удали неправедного от царя, и престол его утвердится правдою.
\vs Pro 25:6 Не величайся пред лицем царя, и на месте великих не становись;
\vs Pro 25:7 потому что лучше, когда скажут тебе: <<пойди сюда повыше>>, нежели когда понизят тебя пред знатным, которого видели глаза твои.
\vs Pro 25:8 Не вступай поспешно в тяжбу: иначе что будешь делать при окончании, когда соперник твой осрамит тебя?
\vs Pro 25:9 Веди тяжбу с соперником твоим, но тайны другого не открывай,
\vs Pro 25:10 дабы не укорил тебя услышавший это, и тогда бесчестие твое не отойдет от тебя. [Любовь и дружба освобождают: сбереги их для себя, чтобы не сделаться тебе достойным поношения; сохрани пути твои благоустроенными.]
\vs Pro 25:11 Золотые яблоки в серебряных прозрачных сосудах~--- слово, сказанное прилично.
\vs Pro 25:12 Золотая серьга и украшение из чистого золота~--- мудрый обличитель для внимательного уха.
\vs Pro 25:13 Что прохлада от снега во время жатвы, то верный посол для посылающего его: он доставляет душе господина своего отраду.
\vs Pro 25:14 Что тучи и ветры без дождя, то человек, хвастающий ложными подарками.
\vs Pro 25:15 Кротостью склоняется к милости вельможа, и мягкий язык переламывает кость.
\vs Pro 25:16 Нашел ты мед,~--- ешь, сколько тебе потребно, чтобы не пресытиться им и не изблевать его.
\vs Pro 25:17 Не учащай входить в дом друга твоего, чтобы он не наскучил тобою и не возненавидел тебя.
\vs Pro 25:18 Что молот и меч и острая стрела, то человек, произносящий ложное свидетельство против ближнего своего.
\vs Pro 25:19 Что сломанный зуб и расслабленная нога, то надежда на ненадежного [человека] в день бедствия.
\vs Pro 25:20 Что снимающий с себя одежду в холодный день, что уксус на рану, то поющий песни печальному сердцу. [Как моль одежде и червь дереву, так печаль вредит сердцу человека.]
\vs Pro 25:21 Если голоден враг твой, накорми его хлебом; и если он жаждет, напой его водою:
\vs Pro 25:22 ибо, [делая сие,] ты собираешь горящие угли на голову его, и Господь воздаст тебе.
\vs Pro 25:23 Северный ветер производит дождь, а тайный язык~--- недовольные лица.
\vs Pro 25:24 Лучше жить в углу на кровле, нежели со сварливою женою в пространном доме.
\vs Pro 25:25 Что холодная вода для истомленной жаждой души, то добрая весть из дальней страны.
\vs Pro 25:26 Что возмущенный источник и поврежденный родник, то праведник, падающий пред нечестивым.
\vs Pro 25:27 Как нехорошо есть много меду, так домогаться славы не есть слава.
\vs Pro 25:28 Что город разрушенный, без стен, то человек, не владеющий духом своим.
\vs Pro 26:1 Как снег летом и дождь во время жатвы, так честь неприлична глупому.
\vs Pro 26:2 Как воробей вспорхнет, как ласточка улетит, так незаслуженное проклятие не сбудется.
\vs Pro 26:3 Бич для коня, узда для осла, а палка для глупых.
\vs Pro 26:4 Не отвечай глупому по глупости его, чтобы и тебе не сделаться подобным ему;
\vs Pro 26:5 но отвечай глупому по глупости его, чтобы он не стал мудрецом в глазах своих.
\vs Pro 26:6 Подрезывает себе ноги, терпит неприятность тот, кто дает словесное поручение глупцу.
\vs Pro 26:7 Неровно поднимаются ноги у хромого,~--- и притча в устах глупцов.
\vs Pro 26:8 Что влагающий драгоценный камень в пращу, то воздающий глупому честь.
\vs Pro 26:9 Что \bibemph{колючий} терн в руке пьяного, то притча в устах глупцов.
\vs Pro 26:10 Сильный делает все произвольно: и глупого награждает, и всякого прохожего награждает.
\vs Pro 26:11 Как пес возвращается на блевотину свою, так глупый повторяет глупость свою.
\vs Pro 26:12 Видал ли ты человека, мудрого в глазах его? На глупого больше надежды, нежели на него.
\vs Pro 26:13 Ленивец говорит: <<лев на дороге! лев на площадях!>>
\vs Pro 26:14 Дверь ворочается на крючьях своих, а ленивец на постели своей.
\vs Pro 26:15 Ленивец опускает руку свою в чашу, и ему тяжело донести ее до рта своего.
\vs Pro 26:16 Ленивец в глазах своих мудрее семерых, отвечающих обдуманно.
\vs Pro 26:17 Хватает пса за уши, кто, проходя мимо, вмешивается в чужую ссору.
\vs Pro 26:18 Как притворяющийся помешанным бросает огонь, стрелы и смерть,
\vs Pro 26:19 так~--- человек, который коварно вредит другу своему и потом говорит: <<я только пошутил>>.
\vs Pro 26:20 Где нет больше дров, огонь погасает, и где нет наушника, раздор утихает.
\vs Pro 26:21 Уголь~--- для жара и дрова~--- для огня, а человек сварливый~--- для разжжения ссоры.
\vs Pro 26:22 Слова наушника~--- как лакомства, и они входят во внутренность чрева.
\vs Pro 26:23 Что нечистым серебром обложенный глиняный сосуд, то пламенные уста и сердце злобное.
\vs Pro 26:24 Устами своими притворяется враг, а в сердце своем замышляет коварство.
\vs Pro 26:25 Если он говорит и нежным голосом, не верь ему, потому что семь мерзостей в сердце его.
\vs Pro 26:26 Если ненависть прикрывается наедине, то откроется злоба его в народном собрании.
\vs Pro 26:27 Кто роет яму, тот упадет в нее, и кто покатит вверх камень, к тому он воротится.
\vs Pro 26:28 Лживый язык ненавидит уязвляемых им, и льстивые уста готовят падение.
\vs Pro 27:1 Не хвались завтрашним днем, потому что не знаешь, чт\acc{о} родит тот день.
\vs Pro 27:2 Пусть хвалит тебя другой, а не уста твои,~--- чужой, а не язык твой.
\vs Pro 27:3 Тяжел камень, весок и песок; но гнев глупца тяжелее их обоих.
\vs Pro 27:4 Жесток гнев, неукротима ярость; но кто устоит против ревности?
\vs Pro 27:5 Лучше открытое обличение, нежели скрытая любовь.
\vs Pro 27:6 Искренни укоризны от любящего, и лживы поцелуи ненавидящего.
\vs Pro 27:7 Сытая душа попирает и сот, а голодной душе все горькое сладко.
\vs Pro 27:8 Как птица, покинувшая гнездо свое, так человек, покинувший место свое.
\vs Pro 27:9 Масть и курение радуют сердце; так сладок \bibemph{всякому} друг сердечным советом своим.
\vs Pro 27:10 Не покидай друга твоего и друга отца твоего, и в дом брата твоего не ходи в день несчастья твоего: лучше сосед вблизи, нежели брат вдали.
\rsbpar\vs Pro 27:11 Будь мудр, сын мой, и радуй сердце мое; и я буду иметь, что отвечать злословящему меня.
\vs Pro 27:12 Благоразумный видит беду и укрывается; а неопытные идут вперед \bibemph{и} наказываются.
\vs Pro 27:13 Возьми у него платье его, потому что он поручился за чужого, и за стороннего возьми от него залог.
\vs Pro 27:14 Кто громко хвалит друга своего с раннего утра, того сочтут за злословящего.
\vs Pro 27:15 Непрестанная капель в дождливый день и сварливая жена~--- равны:
\vs Pro 27:16 кто хочет скрыть ее, тот хочет скрыть ветер и масть в правой руке своей, дающую знать о себе.
\vs Pro 27:17 Железо железо острит, и человек изощряет взгляд друга своего.
\vs Pro 27:18 Кто стережет смоковницу, тот будет есть плоды ее; и кто бережет господина своего, тот будет в чести.
\vs Pro 27:19 Как в воде лицо~--- к лицу, так сердце человека~--- к человеку.
\vs Pro 27:20 Преисподняя и Аваддон~--- ненасытимы; так ненасытимы и глаза человеческие. [Мерзость пред Господом дерзко поднимающий глаза, и неразумны невоздержанные языком.]
\vs Pro 27:21 Что плавильня~--- для серебра, горнило~--- для золота, то для человека уста, которые хвалят его. [Сердце беззаконника ищет зла, сердце же правое ищет знания.]
\vs Pro 27:22 Толк\acc{и} глупого в ступе пестом вместе с зерном, не отделится от него глупость его.
\vs Pro 27:23 Хорошо наблюдай за скотом твоим, имей попечение о стадах;
\vs Pro 27:24 потому что \bibemph{богатство} не навек, да и власть разве из рода в род?
\vs Pro 27:25 Прозябает трава, и является зелень, и собирают горные травы.
\vs Pro 27:26 Овцы~--- на одежду тебе, и козлы~--- на покупку поля.
\vs Pro 27:27 И довольно козьего молока в пищу тебе, в пищу домашним твоим и на продовольствие служанкам твоим.
\vs Pro 28:1 Нечестивый бежит, когда никто не гонится \bibemph{за ним}; а праведник смел, как лев.
\vs Pro 28:2 Когда страна отступит от закона, тогда много в ней начальников; а при разумном и знающем муже она долговечна.
\vs Pro 28:3 Человек бедный и притесняющий слабых \bibemph{то же, что} проливной дождь, смывающий хлеб.
\vs Pro 28:4 Отступники от закона хвалят нечестивых, а соблюдающие закон негодуют на них.
\vs Pro 28:5 Злые люди не разумеют справедливости, а ищущие Господа разумеют всё.
\vs Pro 28:6 Лучше бедный, ходящий в своей непорочности, нежели тот, кто извращает пути свои, хотя он и богат.
\vs Pro 28:7 Хранящий закон~--- сын разумный, а знающийся с расточителями срамит отца своего.
\vs Pro 28:8 Умножающий имение свое ростом и лихвою соберет его для благотворителя бедных.
\vs Pro 28:9 Кто отклоняет ухо свое от слушания закона, того и молитва~--- мерзость.
\vs Pro 28:10 Совращающий праведных на путь зла сам упадет в свою яму, а непорочные наследуют добро.
\vs Pro 28:11 Человек богатый~--- мудрец в глазах своих, но умный бедняк обличит его.
\vs Pro 28:12 Когда торжествуют праведники, великая слава, но когда возвышаются нечестивые, люди укрываются.
\vs Pro 28:13 Скрывающий свои преступления не будет иметь успеха; а кто сознается и оставляет их, тот будет помилован.
\rsbpar\vs Pro 28:14 Блажен человек, который всегда пребывает в благоговении; а кто ожесточает сердце свое, тот попадет в беду.
\vs Pro 28:15 Как рыкающий лев и голодный медведь, так нечестивый властелин над бедным народом.
\vs Pro 28:16 Неразумный правитель много делает притеснений, а ненавидящий корысть продолжит дни.
\vs Pro 28:17 Человек, виновный в пролитии человеческой крови, будет бегать до могилы, чтобы кто не схватил его.
\vs Pro 28:18 Кто ходит непорочно, тот будет невредим; а ходящий кривыми путями упадет на одном из них.
\vs Pro 28:19 Кто возделывает землю свою, тот будет насыщаться хлебом, а кто подражает праздным, тот насытится нищетою.
\vs Pro 28:20 Верный человек богат благословениями, а кто спешит разбогатеть, тот не останется ненаказанным.
\vs Pro 28:21 Быть лицеприятным~--- нехорошо: такой человек и за кусок хлеба сделает неправду.
\vs Pro 28:22 Спешит к богатству завистливый человек, и не думает, что нищета постигнет его.
\vs Pro 28:23 Обличающий человека найдет после б\acc{о}льшую приязнь, нежели тот, кто льстит языком.
\vs Pro 28:24 Кто обкрадывает отца своего и мать свою и говорит: <<это не грех>>, тот~--- сообщник грабителям.
\vs Pro 28:25 Надменный разжигает ссору, а надеющийся на Господа будет благоденствовать.
\vs Pro 28:26 Кто надеется на себя, тот глуп; а кто ходит в мудрости, тот будет цел.
\vs Pro 28:27 Дающий нищему не обеднеет; а кто закрывает глаза свои от него, на том много проклятий.
\vs Pro 28:28 Когда возвышаются нечестивые, люди укрываются, а когда они падают, умножаются праведники.
\vs Pro 29:1 Человек, который, будучи обличаем, ожесточает выю свою, внезапно сокрушится, и не будет \bibemph{ему} исцеления.
\vs Pro 29:2 Когда умножаются праведники, веселится народ, а когда господствует нечестивый, народ стенает.
\vs Pro 29:3 Человек, любящий мудрость, радует отца своего; а кто знается с блудницами, тот расточает имение.
\vs Pro 29:4 Царь правосудием утверждает землю, а любящий подарки разоряет ее.
\vs Pro 29:5 Человек, льстящий другу своему, расстилает сеть ногам его.
\vs Pro 29:6 В грехе злого человека~--- сеть \bibemph{для него}, а праведник веселится и радуется.
\vs Pro 29:7 Праведник тщательно вникает в тяжбу бедных, а нечестивый не разбирает дела.
\vs Pro 29:8 Люди развратные возмущают город, а мудрые утишают мятеж.
\vs Pro 29:9 Умный человек, судясь с человеком глупым, сердится ли, смеется ли,~--- не имеет покоя.
\vs Pro 29:10 Кровожадные люди ненавидят непорочного, а праведные заботятся о его жизни.
\vs Pro 29:11 Глупый весь гнев свой изливает, а мудрый сдерживает его.
\vs Pro 29:12 Если правитель слушает ложные речи, то и все служащие у него нечестивы.
\vs Pro 29:13 Бедный и лихоимец встречаются друг с другом; но свет глазам того и другого дает Господь.
\vs Pro 29:14 Если царь судит бедных по правде, то престол его навсегда утвердится.
\vs Pro 29:15 Розга и обличение дают мудрость; но отрок, оставленный в небрежении, делает стыд своей матери.
\vs Pro 29:16 При умножении нечестивых умножается беззаконие; но праведники увидят падение их.
\vs Pro 29:17 Наказывай сына твоего, и он даст тебе покой, и доставит радость душе твоей.
\rsbpar\vs Pro 29:18 Без откровения свыше народ необуздан, а соблюдающий закон блажен.
\vs Pro 29:19 Словами не научится раб, потому что, хотя он понимает \bibemph{их}, но не слушается.
\vs Pro 29:20 Видал ли ты человека опрометчивого в словах своих? на глупого больше надежды, нежели на него.
\vs Pro 29:21 Если с детства воспитывать раба в неге, то впоследствии он захочет быть сыном.
\vs Pro 29:22 Человек гневливый заводит ссору, и вспыльчивый много грешит.
\vs Pro 29:23 Гордость человека унижает его, а смиренный духом приобретает честь.
\vs Pro 29:24 Кто делится с вором, тот ненавидит душу свою; слышит он проклятие, но не объявляет о том.
\vs Pro 29:25 Боязнь пред людьми ставит сеть; а надеющийся на Господа будет безопасен.
\vs Pro 29:26 Многие ищут \bibemph{благосклонного} лица правителя, но судьба человека~--- от Господа.
\vs Pro 29:27 Мерзость для праведников~--- человек неправедный, и мерзость для нечестивого~--- идущий прямым путем.
\vs Pro 30:1 Слова Агура, сына Иакеева. Вдохновенные изречения, \bibemph{которые} сказал этот человек Ифиилу, Ифиилу и Укалу:
\vs Pro 30:2 подлинно, я более невежда, нежели кто-либо из людей, и разума человеческого нет у меня,
\vs Pro 30:3 и не научился я мудрости, и познания святых не имею.
\vs Pro 30:4 Кто восходил на небо и нисходил? кто собрал ветер в пригоршни свои? кто завязал воду в одежду? кто поставил все пределы земли? какое имя ему? и какое имя сыну его? знаешь ли?
\rsbpar\vs Pro 30:5 Всякое слово Бога чисто; Он~--- щит уповающим на Него.
\vs Pro 30:6 Не прибавляй к словам Его, чтобы Он не обличил тебя, и ты не оказался лжецом.
\rsbpar\vs Pro 30:7 Двух вещей я прошу у Тебя, не откажи мне, прежде нежели я умру:
\vs Pro 30:8 суету и ложь удали от меня, нищеты и богатства не давай мне, питай меня насущным хлебом,
\vs Pro 30:9 дабы, пресытившись, я не отрекся \bibemph{Тебя} и не сказал: <<кто Господь?>> и чтобы, обеднев, не стал красть и употреблять имя Бога моего всуе.
\vs Pro 30:10 Не злословь раба пред господином его, чтобы он не проклял тебя, и ты не остался виноватым.
\vs Pro 30:11 Есть род, который проклинает отца своего и не благословляет матери своей.
\vs Pro 30:12 Есть род, который чист в глазах своих, тогда как не омыт от нечистот своих.
\vs Pro 30:13 Есть род~--- о, как высокомерны глаза его, и как подняты ресницы его!
\vs Pro 30:14 Есть род, у которого зубы~--- мечи, и челюсти~--- ножи, чтобы пожирать бедных на земле и нищих между людьми.
\vs Pro 30:15 У ненасытимости две дочери: <<давай, давай!>> Вот три ненасытимых, и четыре, которые не скажут: <<довольно!>>
\vs Pro 30:16 Преисподняя и утроба бесплодная, земля, которая не насыщается водою, и огонь, который не говорит: <<довольно!>>
\vs Pro 30:17 Глаз, насмехающийся над отцом и пренебрегающий покорностью к матери, выклюют в\acc{о}роны дольные, и сожрут птенцы орлиные!
\vs Pro 30:18 Три вещи непостижимы для меня, и четырех я не понимаю:
\vs Pro 30:19 пути орла на небе, пути змея на скале, пути корабля среди моря и пути мужчины к девице.
\vs Pro 30:20 Таков путь и жены прелюбодейной; поела и обтерла рот свой, и говорит: <<я ничего худого не сделала>>.
\vs Pro 30:21 От трех трясется земля, четырех она не может носить:
\vs Pro 30:22 раба, когда он делается царем; глупого, когда он досыта ест хлеб;
\vs Pro 30:23 позорную женщину, когда она выходит замуж, и служанку, когда она занимает место госпожи своей.
\vs Pro 30:24 Вот четыре малых на земле, но они мудрее мудрых:
\vs Pro 30:25 муравьи~--- народ не сильный, но летом заготовляют пищу свою;
\vs Pro 30:26 горные мыши~--- народ слабый, но ставят домы свои на скале;
\vs Pro 30:27 у саранчи нет царя, но выступает вся она стройно;
\vs Pro 30:28 паук лапками цепляется, но бывает в царских чертогах.
\vs Pro 30:29 Вот трое имеют стройную походку, и четверо стройно выступают:
\vs Pro 30:30 лев, силач между зверями, не посторонится ни перед кем;
\vs Pro 30:31 конь и козел, [предводитель стада,] и царь среди народа своего.
\vs Pro 30:32 Если ты в заносчивости своей сделал глупость и помыслил злое, то \bibemph{положи} руку на уста;
\vs Pro 30:33 потому что, как сбивание молока производит масло, толчок в нос производит кровь, так и возбуждение гнева производит ссору.
\vs Pro 31:1 Слова Лемуила царя. Наставление, которое преподала ему мать его:
\vs Pro 31:2 что, сын мой? что, сын чрева моего? что, сын обетов моих?
\vs Pro 31:3 Не отдавай женщинам сил твоих, ни путей твоих губительницам царей.
\vs Pro 31:4 Не царям, Лемуил, не царям пить вино, и не князьям~--- сикеру,
\vs Pro 31:5 чтобы, напившись, они не забыли закона и не превратили суда всех угнетаемых.
\vs Pro 31:6 Дайте сикеру погибающему и вино огорченному душею;
\vs Pro 31:7 пусть он выпьет и забудет бедность свою и не вспомнит больше о своем страдании.
\vs Pro 31:8 Открывай уста твои за безгласного и для защиты всех сирот.
\vs Pro 31:9 Открывай уста твои для правосудия и для дела бедного и нищего.
\rsbpar\vs Pro 31:10 Кто найдет добродетельную жену? цена ее выше жемчугов;
\vs Pro 31:11 уверено в ней сердце мужа ее, и он не останется без прибытка;
\vs Pro 31:12 она воздает ему добром, а не злом, во все дни жизни своей.
\vs Pro 31:13 Добывает шерсть и лен, и с охотою работает своими руками.
\vs Pro 31:14 Она, как купеческие корабли, издалека добывает хлеб свой.
\vs Pro 31:15 Она встает еще ночью и раздает пищу в доме своем и урочное служанкам своим.
\vs Pro 31:16 Задумает она о поле, и приобретает его; от плодов рук своих насаждает виноградник.
\vs Pro 31:17 Препоясывает силою чресла свои и укрепляет мышцы свои.
\vs Pro 31:18 Она чувствует, что занятие ее хорошо, и~--- светильник ее не гаснет и ночью.
\vs Pro 31:19 Протягивает руки свои к прялке, и персты ее берутся за веретено.
\vs Pro 31:20 Длань свою она открывает бедному, и руку свою подает нуждающемуся.
\vs Pro 31:21 Не боится стужи для семьи своей, потому что вся семья ее одета в двойные одежды.
\vs Pro 31:22 Она делает себе ковры; виссон и пурпур~--- одежда ее.
\vs Pro 31:23 Муж ее известен у ворот, когда сидит со старейшинами земли.
\vs Pro 31:24 Она делает покрывала и продает, и поясы доставляет купцам Финикийским.
\vs Pro 31:25 Крепость и красота~--- одежда ее, и весело смотрит она на будущее.
\vs Pro 31:26 Уста свои открывает с мудростью, и кроткое наставление на языке ее.
\vs Pro 31:27 Она наблюдает за хозяйством в доме своем и не ест хлеба праздности.
\vs Pro 31:28 Встают дети и ублажают ее,~--- муж, и хвалит ее:
\vs Pro 31:29 <<много было жен добродетельных, но ты превзошла всех их>>.
\vs Pro 31:30 Миловидность обманчива и красота суетна; но жена, боящаяся Господа, достойна хвалы.
\vs Pro 31:31 Дайте ей от плода рук ее, и да прославят ее у ворот дел\acc{а} ее!

\bibbookdescr{Ecc}{
  inline={\LARGE Книга\\\Huge Екклесиаста\\или Проповедника},
  toc={Екклесиаст},
  bookmark={Екклесиаст},
  header={Екклесиаст},
  %headerleft={},
  %headerright={},
  abbr={Еккл}
}
\vs Ecc 1:1 Слова Екклесиаста, сына Давидова, царя в Иерусалиме.
\rsbpar\vs Ecc 1:2 Суета сует, сказал Екклесиаст, суета сует,~--- всё суета!
\vs Ecc 1:3 Что пользы человеку от всех трудов его, которыми трудится он под солнцем?
\vs Ecc 1:4 Род проходит, и род приходит, а земля пребывает во веки.
\vs Ecc 1:5 Восходит солнце, и заходит солнце, и спешит к месту своему, где оно восходит.
\vs Ecc 1:6 Идет ветер к югу, и переходит к северу, кружится, кружится на ходу своем, и возвращается ветер на круги свои.
\vs Ecc 1:7 Все реки текут в море, но море не переполняется: к тому месту, откуда реки текут, они возвращаются, чтобы опять течь.
\vs Ecc 1:8 Все вещи~--- в труде: не может человек пересказать всего; не насытится око зрением, не наполнится ухо слушанием.
\vs Ecc 1:9 Что было, то и будет; и что делалось, то и будет делаться, и нет ничего нового под солнцем.
\vs Ecc 1:10 Бывает нечто, о чем говорят: <<смотри, вот это новое>>; но \bibemph{это} было уже в веках, бывших прежде нас.
\vs Ecc 1:11 Нет памяти о прежнем; да и о том, что будет, не останется памяти у тех, которые будут после.
\rsbpar\vs Ecc 1:12 Я, Екклесиаст, был царем над Израилем в Иерусалиме;
\vs Ecc 1:13 и предал я сердце мое тому, чтобы исследовать и испытать мудростью все, что делается под небом: это тяжелое занятие дал Бог сынам человеческим, чтобы они упражнялись в нем.
\vs Ecc 1:14 Видел я все дела, какие делаются под солнцем, и вот, всё~--- суета и томление духа!
\vs Ecc 1:15 Кривое не может сделаться прямым, и чего нет, того нельзя считать.
\vs Ecc 1:16 Говорил я с сердцем моим так: вот, я возвеличился и приобрел мудрости больше всех, которые были прежде меня над Иерусалимом, и сердце мое видело много мудрости и знания.
\vs Ecc 1:17 И предал я сердце мое тому, чтобы познать мудрость и познать безумие и глупость: узнал, что и это~--- томление духа;
\vs Ecc 1:18 потому что во многой мудрости много печали; и кто умножает познания, умножает скорбь.
\vs Ecc 2:1 Сказал я в сердце моем: <<дай, испытаю я тебя весельем, и насладись добром>>; но и это~--- суета!
\vs Ecc 2:2 О смехе сказал я: <<глупость!>>, а о веселье: <<что оно делает?>>
\vs Ecc 2:3 Вздумал я в сердце моем услаждать вином тело мое и, между тем, как сердце мое руководилось мудростью, придержаться и глупости, доколе не увижу, что хорошо для сынов человеческих, что должны были бы они делать под небом в немногие дни жизни своей.
\rsbpar\vs Ecc 2:4 Я предпринял большие дела: построил себе домы, посадил себе виноградники,
\vs Ecc 2:5 устроил себе сады и рощи и насадил в них всякие плодовитые дерева;
\vs Ecc 2:6 сделал себе водоемы для орошения из них рощей, произращающих деревья;
\vs Ecc 2:7 приобрел себе слуг и служанок, и домочадцы были у меня; также крупного и мелкого скота было у меня больше, нежели у всех, бывших прежде меня в Иерусалиме;
\vs Ecc 2:8 собрал себе серебра и золота и драгоценностей от царей и областей; завел у себя певцов и певиц и услаждения сынов человеческих~--- разные музыкальные орудия.
\vs Ecc 2:9 И сделался я великим и богатым больше всех, бывших прежде меня в Иерусалиме; и мудрость моя пребыла со мною.
\vs Ecc 2:10 Чего бы глаза мои ни пожелали, я не отказывал им, не возбранял сердцу моему никакого веселья, потому что сердце мое радовалось во всех трудах моих, и это было моею долею от всех трудов моих.
\vs Ecc 2:11 И оглянулся я на все дела мои, которые сделали руки мои, и на труд, которым трудился я, делая \bibemph{их}: и вот, всё~--- суета и томление духа, и нет \bibemph{от них} пользы под солнцем!
\rsbpar\vs Ecc 2:12 И обратился я, чтобы взглянуть на мудрость и безумие и глупость: ибо что \bibemph{может сделать} человек после царя \bibemph{сверх того}, что уже сделано?
\vs Ecc 2:13 И увидел я, что преимущество мудрости перед глупостью такое же, как преимущество света перед тьмою:
\vs Ecc 2:14 у мудрого глаза его~--- в голове его, а глупый ходит во тьме; но узнал я, что одна участь постигает их всех.
\vs Ecc 2:15 И сказал я в сердце моем: <<и меня постигнет та же участь, как и глупого: к чему же я сделался очень мудрым?>> И сказал я в сердце моем, что и это~--- суета;
\vs Ecc 2:16 потому что мудрого не будут помнить вечно, как и глупого; в грядущие дни все будет забыто, и увы! мудрый умирает наравне с глупым.
\vs Ecc 2:17 И возненавидел я жизнь, потому что противны стали мне дела, которые делаются под солнцем; ибо всё~--- суета и томление духа!
\vs Ecc 2:18 И возненавидел я весь труд мой, которым трудился под солнцем, потому что должен оставить его человеку, который будет после меня.
\vs Ecc 2:19 И кто знает: мудрый ли будет он, или глупый? А он будет распоряжаться всем трудом моим, которым я трудился и которым показал себя мудрым под солнцем. И это~--- суета!
\vs Ecc 2:20 И обратился я, чтобы внушить сердцу моему отречься от всего труда, которым я трудился под солнцем,
\vs Ecc 2:21 потому что иной человек трудится мудро, с знанием и успехом, и должен отдать всё человеку, не трудившемуся в том, как бы часть его. И это~--- суета и зло великое!
\vs Ecc 2:22 Ибо что будет иметь человек от всего труда своего и заботы сердца своего, что трудится он под солнцем?
\vs Ecc 2:23 Потому что все дни его~--- скорби, и его труды~--- беспокойство; даже и ночью сердце его не знает покоя. И это~--- суета!
\rsbpar\vs Ecc 2:24 Не во власти человека и то благо, чтобы есть и пить и услаждать душу свою от труда своего. Я увидел, что и это~--- от руки Божией;
\vs Ecc 2:25 потому что кто может есть и кто может наслаждаться без Него?
\vs Ecc 2:26 Ибо человеку, который добр пред лицем Его, Он дает мудрость и знание и радость; а грешнику дает заботу собирать и копить, чтобы \bibemph{после} отдать доброму пред лицем Божиим. И это~--- суета и томление духа!
\vs Ecc 3:1 Всему свое время, и время всякой вещи под небом:
\vs Ecc 3:2 время рождаться, и время умирать; время насаждать, и время вырывать посаженное;
\vs Ecc 3:3 время убивать, и время врачевать; время разрушать, и время строить;
\vs Ecc 3:4 время плакать, и время смеяться; время сетовать, и время плясать;
\vs Ecc 3:5 время разбрасывать камни, и время собирать камни; время обнимать, и время уклоняться от объятий;
\vs Ecc 3:6 время искать, и время терять; время сберегать, и время бросать;
\vs Ecc 3:7 время раздирать, и время сшивать; время молчать, и время говорить;
\vs Ecc 3:8 время любить, и время ненавидеть; время войне, и время миру.
\rsbpar\vs Ecc 3:9 Что пользы работающему от того, над чем он трудится?
\vs Ecc 3:10 Видел я эту заботу, которую дал Бог сынам человеческим, чтобы они упражнялись в том.
\vs Ecc 3:11 Всё соделал Он прекрасным в свое время, и вложил мир в сердце их, хотя человек не может постигнуть дел, которые Бог делает, от начала до конца.
\vs Ecc 3:12 Познал я, что нет для них ничего лучшего, как веселиться и делать доброе в жизни своей.
\vs Ecc 3:13 И если какой человек ест и пьет, и видит доброе во всяком труде своем, то это~--- дар Божий.
\vs Ecc 3:14 Познал я, что всё, что делает Бог, пребывает вовек: к тому нечего прибавлять и от того нечего убавить,~--- и Бог делает так, чтобы благоговели пред лицем Его.
\vs Ecc 3:15 Что было, то и теперь есть, и что будет, то уже было,~--- и Бог воззовет прошедшее.
\rsbpar\vs Ecc 3:16 Еще видел я под солнцем: место суда, а там беззаконие; место правды, а там неправда.
\vs Ecc 3:17 И сказал я в сердце своем: <<праведного и нечестивого будет судить Бог; потому что время для всякой вещи и \bibemph{суд} над всяким делом там>>.
\rsbpar\vs Ecc 3:18 Сказал я в сердце своем о сынах человеческих, чтобы испытал их Бог, и чтобы они видели, что они сами по себе животные;
\vs Ecc 3:19 потому что участь сынов человеческих и участь животных~--- участь одна: как те умирают, так умирают и эти, и одно дыхание у всех, и нет у человека преимущества перед скотом, потому что всё~--- суета!
\vs Ecc 3:20 Все идет в одно место: все произошло из праха и все возвратится в прах.
\vs Ecc 3:21 Кто знает: дух сынов человеческих восходит ли вверх, и дух животных сходит ли вниз, в землю?
\rsbpar\vs Ecc 3:22 Итак увидел я, что нет ничего лучше, как наслаждаться человеку делами своими: потому что это~--- доля его; ибо кто приведет его посмотреть на то, что будет после него?
\vs Ecc 4:1 И обратился я и увидел всякие угнетения, какие делаются под солнцем: и вот слезы угнетенных, а утешителя у них нет; и в руке угнетающих их~--- сила, а утешителя у них нет.
\vs Ecc 4:2 И ублажил я мертвых, которые давно умерли, более живых, которые живут доселе;
\vs Ecc 4:3 а блаженнее их обоих тот, кто еще не существовал, кто не видал злых дел, какие делаются под солнцем.
\rsbpar\vs Ecc 4:4 Видел я также, что всякий труд и всякий успех в делах производят взаимную между людьми зависть. И это~--- суета и томление духа!
\vs Ecc 4:5 Глупый \bibemph{сидит}, сложив свои руки, и съедает плоть свою.
\vs Ecc 4:6 Лучше горсть с покоем, нежели пригоршни с трудом и томлением духа.
\rsbpar\vs Ecc 4:7 И обратился я и увидел еще суету под солнцем;
\vs Ecc 4:8 \bibemph{человек} одинокий, и другого нет; ни сына, ни брата нет у него; а всем трудам его нет конца, и глаз его не насыщается богатством. <<Для кого же я тружусь и лишаю душу мою блага?>> И это~--- суета и недоброе дело!
\rsbpar\vs Ecc 4:9 Двоим лучше, нежели одному; потому что у них есть доброе вознаграждение в труде их:
\vs Ecc 4:10 ибо если упадет один, то другой поднимет товарища своего. Но горе одному, когда упадет, а другого нет, который поднял бы его.
\vs Ecc 4:11 Также, если лежат двое, то тепло им; а одному как согреться?
\vs Ecc 4:12 И если станет преодолевать кто-либо одного, то двое устоят против него: и нитка, втрое скрученная, нескоро порвется.
\rsbpar\vs Ecc 4:13 Лучше бедный, но умный юноша, нежели старый, но неразумный царь, который не умеет принимать советы;
\vs Ecc 4:14 ибо тот из темницы выйдет на царство, хотя родился в царстве своем бедным.
\vs Ecc 4:15 Видел я всех живущих, которые ходят под солнцем, с этим другим юношею, который займет место того.
\vs Ecc 4:16 Не было числа всему народу, который был перед ним, хотя позднейшие не порадуются им. И это~--- суета и томление духа!
\rsbpar\vs Ecc 4:17 Наблюдай за ногою твоею, когда идешь в дом Божий, и будь готов более к слушанию, нежели к жертвоприношению; ибо они не думают, что худо делают.
\vs Ecc 5:1 Не торопись языком твоим, и сердце твое да не спешит произнести слово пред Богом; потому что Бог на небе, а ты на земле; поэтому слова твои да будут немноги.
\vs Ecc 5:2 Ибо, как сновидения бывают при множестве забот, так голос глупого познается при множестве слов.
\rsbpar\vs Ecc 5:3 Когда даешь обет Богу, то не медли исполнить его, потому что Он не благоволит к глупым: что обещал, исполни.
\vs Ecc 5:4 Лучше тебе не обещать, нежели обещать и не исполнить.
\vs Ecc 5:5 Не дозволяй устам твоим вводить в грех плоть твою, и не говори пред Ангелом [Божиим]: <<это~--- ошибка!>> Для чего тебе \bibemph{делать}, чтобы Бог прогневался на слово твое и разрушил дело рук твоих?
\vs Ecc 5:6 Ибо во множестве сновидений, как и во множестве слов,~--- много суеты; но ты бойся Бога.
\rsbpar\vs Ecc 5:7 Если ты увидишь в какой области притеснение бедному и нарушение суда и правды, то не удивляйся этому: потому что над высоким наблюдает высший, а над ними еще высший;
\vs Ecc 5:8 превосходство же страны в целом есть царь, заботящийся о стране.
\vs Ecc 5:9 Кто любит серебро, тот не насытится серебром, и кто любит богатство, тому нет пользы от того. И это~--- суета!
\vs Ecc 5:10 Умножается имущество, умножаются и потребляющие его; и какое благо для владеющего им: разве только смотреть своими глазами?
\vs Ecc 5:11 Сладок сон трудящегося, мало ли, много ли он съест; но пресыщение богатого не дает ему уснуть.
\vs Ecc 5:12 Есть мучительный недуг, который видел я под солнцем: богатство, сберегаемое владетелем его во вред ему.
\vs Ecc 5:13 И гибнет богатство это от несчастных случаев: родил он сына, и ничего нет в руках у него.
\vs Ecc 5:14 Как вышел он нагим из утробы матери своей, таким и отходит, каким пришел, и ничего не возьмет от труда своего, что мог бы он понести в руке своей.
\vs Ecc 5:15 И это тяжкий недуг: каким пришел он, таким и отходит. Какая же польза ему, что он трудился на ветер?
\vs Ecc 5:16 А он во все дни свои ел впотьмах, в большом раздражении, в огорчении и досаде.
\rsbpar\vs Ecc 5:17 Вот еще, что я нашел доброго и приятного: есть и пить и наслаждаться добром во всех трудах своих, какими кто трудится под солнцем во все дни жизни своей, которые дал ему Бог; потому что это его доля.
\vs Ecc 5:18 И если какому человеку Бог дал богатство и имущество, и дал ему власть пользоваться от них и брать свою долю и наслаждаться от трудов своих, то это дар Божий.
\vs Ecc 5:19 Недолго будут у него в памяти дни жизни его; поэтому Бог и вознаграждает его радостью сердца его.
\vs Ecc 6:1 Есть зло, которое видел я под солнцем, и оно часто бывает между людьми:
\vs Ecc 6:2 Бог дает человеку богатство и имущество и славу, и нет для души его недостатка ни в чем, чего не пожелал бы он; но не дает ему Бог пользоваться этим, а пользуется тем чужой человек: это~--- суета и тяжкий недуг!
\vs Ecc 6:3 Если бы какой человек родил сто \bibemph{детей}, и прожил многие годы, и еще умножились дни жизни его, но душа его не наслаждалась бы добром и не было бы ему и погребения, то я сказал бы: выкидыш счастливее его,
\vs Ecc 6:4 потому что он напрасно пришел и отошел во тьму, и его имя покрыто мраком.
\vs Ecc 6:5 Он даже не видал и не знал солнца: ему покойнее, нежели тому.
\vs Ecc 6:6 А тот, хотя бы прожил две тысячи лет и не наслаждался добром, не все ли пойдет в одно место?
\vs Ecc 6:7 Все труды человека~--- для рта его, а душа его не насыщается.
\vs Ecc 6:8 Какое же преимущество мудрого перед глупым, какое~--- бедняка, умеющего ходить перед живущими?
\vs Ecc 6:9 Лучше видеть глазами, нежели бродить душею. И это~--- также суета и томление духа!
\vs Ecc 6:10 Что существует, тому уже наречено имя, и известно, что это~--- человек, и что он не может препираться с тем, кто сильнее его.
\vs Ecc 6:11 Много таких вещей, которые умножают суету: что же для человека лучше?
\vs Ecc 6:12 Ибо кто знает, что хорошо для человека в жизни, во все дни суетной жизни его, которые он проводит как тень? И кто скажет человеку, что будет после него под солнцем?
\vs Ecc 7:1 Доброе имя лучше дорогой масти, и день смерти~--- дня рождения.
\vs Ecc 7:2 Лучше ходить в дом плача об умершем, нежели ходить в дом пира; ибо таков конец всякого человека, и живой приложит \bibemph{это} к своему сердцу.
\vs Ecc 7:3 Сетование лучше смеха; потому что при печали лица сердце делается лучше.
\vs Ecc 7:4 Сердце мудрых~--- в доме плача, а сердце глупых~--- в доме веселья.
\vs Ecc 7:5 Лучше слушать обличения от мудрого, нежели слушать песни глупых;
\vs Ecc 7:6 потому что смех глупых то же, что треск тернового хвороста под котлом. И это~--- суета!
\rsbpar\vs Ecc 7:7 Притесняя других, мудрый делается глупым, и подарки портят сердце.
\vs Ecc 7:8 Конец дела лучше начала его; терпеливый лучше высокомерного.
\vs Ecc 7:9 Не будь духом твоим поспешен на гнев, потому что гнев гнездится в сердце глупых.
\vs Ecc 7:10 Не говори: <<отчего это прежние дни были лучше нынешних?>>, потому что не от мудрости ты спрашиваешь об этом.
\vs Ecc 7:11 Хороша мудрость с наследством, и особенно для видящих солнце:
\vs Ecc 7:12 потому что под сенью ее \bibemph{то же, что} под сенью серебра; но превосходство знания в \bibemph{том, что} мудрость дает жизнь владеющему ею.
\vs Ecc 7:13 Смотри на действование Божие: ибо кто может выпрямить то, что Он сделал кривым?
\vs Ecc 7:14 Во дни благополучия пользуйся благом, а во дни несчастья размышляй: то и другое соделал Бог для того, чтобы человек ничего не мог сказать против Него.
\rsbpar\vs Ecc 7:15 Всего насмотрелся я в суетные дни мои: праведник гибнет в праведности своей; нечестивый живет долго в нечестии своем.
\vs Ecc 7:16 Не будь слишком строг, и не выставляй себя слишком мудрым; зачем тебе губить себя?
\vs Ecc 7:17 Не предавайся греху, и не будь безумен: зачем тебе умирать не в свое время?
\vs Ecc 7:18 Хорошо, если ты будешь держаться одного и не отнимать руки от другого; потому что кто боится Бога, тот избежит всего того.
\vs Ecc 7:19 Мудрость делает мудрого сильнее десяти властителей, которые в городе.
\rsbpar\vs Ecc 7:20 Нет человека праведного на земле, который делал бы добро и не грешил бы;
\vs Ecc 7:21 поэтому не на всякое слово, которое говорят, обращай внимание, чтобы не услышать тебе раба твоего, когда он злословит тебя;
\vs Ecc 7:22 ибо сердце твое знает много случаев, когда и сам ты злословил других.
\rsbpar\vs Ecc 7:23 Все это испытал я мудростью; я сказал: <<буду я мудрым>>; но мудрость далека от меня.
\vs Ecc 7:24 Далеко то, что было, и глубоко~--- глубоко: кто постигнет его?
\vs Ecc 7:25 Обратился я сердцем моим к тому, чтобы узнать, исследовать и изыскать мудрость и разум, и познать нечестие глупости, невежества и безумия,~---
\vs Ecc 7:26 и нашел я, что горче смерти женщина, потому что она~--- сеть, и сердце ее~--- силки, руки ее~--- оковы; добрый пред Богом спасется от нее, а грешник уловлен будет ею.
\vs Ecc 7:27 Вот это нашел я, сказал Екклесиаст, испытывая одно за другим.
\vs Ecc 7:28 Чего еще искала душа моя, и я не нашел?~--- Мужчину одного из тысячи я нашел, а женщины между всеми ими не нашел.
\vs Ecc 7:29 Только это я нашел, что Бог сотворил человека правым, а люди пустились во многие помыслы.
\vs Ecc 8:1 Кто~--- как мудрый, и кто понимает значение вещей? Мудрость человека просветляет лице его, и суровость лица его изменяется.
\vs Ecc 8:2 \bibemph{Я говорю}: слово царское храни, и \bibemph{это} ради клятвы пред Богом.
\vs Ecc 8:3 Не спеши уходить от лица его, и не упорствуй в худом деле; потому что он, что захочет, все может сделать.
\vs Ecc 8:4 Где слово царя, там власть; и кто скажет ему: <<что ты делаешь?>>
\rsbpar\vs Ecc 8:5 Соблюдающий заповедь не испытает никакого зла: сердце мудрого знает и время и устав;
\vs Ecc 8:6 потому что для всякой вещи есть свое время и устав; а человеку великое зло оттого,
\vs Ecc 8:7 что он не знает, что будет; и как это будет~--- кто скажет ему?
\rsbpar\vs Ecc 8:8 Человек не властен над духом, чтобы удержать дух, и нет власти у него над днем смерти, и нет избавления в этой борьбе, и не спасет нечестие нечестивого.
\vs Ecc 8:9 Все это я видел, и обращал сердце мое на всякое дело, какое делается под солнцем. Бывает время, когда человек властвует над человеком во вред ему.
\vs Ecc 8:10 Видел я тогда, что хоронили нечестивых, и приходили и отходили от святого места, и они забываемы были в городе, где они так поступали. И это~--- суета!
\vs Ecc 8:11 Не скоро совершается суд над худыми делами; от этого и не страшится сердце сынов человеческих делать зло.
\vs Ecc 8:12 Хотя грешник сто раз делает зло и коснеет в нем, но я знаю, что благо будет боящимся Бога, которые благоговеют пред лицем Его;
\vs Ecc 8:13 а нечестивому не будет добра, и, подобно тени, недолго продержится тот, кто не благоговеет пред Богом.
\vs Ecc 8:14 Есть и такая суета на земле: праведников постигает то, чего заслуживали бы дела нечестивых, а с нечестивыми бывает то, чего заслуживали бы дела праведников. И сказал я: и это~--- суета!
\vs Ecc 8:15 И похвалил я веселье; потому что нет лучшего для человека под солнцем, как есть, пить и веселиться: это сопровождает его в трудах во дни жизни его, которые дал ему Бог под солнцем.
\rsbpar\vs Ecc 8:16 Когда я обратил сердце мое на то, чтобы постигнуть мудрость и обозреть дела, которые делаются на земле, и среди которых \bibemph{человек} ни днем, ни ночью не знает сна,~---
\vs Ecc 8:17 тогда я увидел все дела Божии и \bibemph{нашел}, что человек не может постигнуть дел, которые делаются под солнцем. Сколько бы человек ни трудился в исследовании, он все-таки не постигнет этого; и если бы какой мудрец сказал, что он знает, он не может постигнуть \bibemph{этого}.
\vs Ecc 9:1 На все это я обратил сердце мое для исследования, что праведные и мудрые и деяния их~--- в руке Божией, и что человек ни любви, ни ненависти не знает во всем том, что перед ним.
\vs Ecc 9:2 Всему и всем~--- одно: одна участь праведнику и нечестивому, доброму и [злому], чистому и нечистому, приносящему жертву и не приносящему жертвы; как добродетельному, так и грешнику; как клянущемуся, так и боящемуся клятвы.
\vs Ecc 9:3 Это-то и худо во всем, что делается под солнцем, что одна участь всем, и сердце сынов человеческих исполнено зла, и безумие в сердце их, в жизни их; а после того они \bibemph{отходят} к умершим.
\vs Ecc 9:4 Кто находится между живыми, тому есть еще надежда, так как и псу живому лучше, нежели мертвому льву.
\vs Ecc 9:5 Живые знают, что умрут, а мертвые ничего не знают, и уже нет им воздаяния, потому что и память о них предана забвению,
\vs Ecc 9:6 и любовь их и ненависть их и ревность их уже исчезли, и нет им более части во веки ни в чем, что делается под солнцем.
\vs Ecc 9:7 \bibemph{Итак} иди, ешь с весельем хлеб твой, и пей в радости сердца вино твое, когда Бог благоволит к делам твоим.
\vs Ecc 9:8 Да будут во всякое время одежды твои светлы, и да не оскудевает елей на голове твоей.
\vs Ecc 9:9 Наслаждайся жизнью с женою, которую любишь, во все дни суетной жизни твоей, и которую дал тебе Бог под солнцем на все суетные дни твои; потому что это~--- доля твоя в жизни и в трудах твоих, какими ты трудишься под солнцем.
\vs Ecc 9:10 Все, что может рука твоя делать, по силам делай; потому что в могиле, куда ты пойдешь, нет ни работы, ни размышления, ни знания, ни мудрости.
\rsbpar\vs Ecc 9:11 И обратился я, и видел под солнцем, что не проворным достается успешный бег, не храбрым~--- победа, не мудрым~--- хлеб, и не у разумных~--- богатство, и не искусным~--- благорасположение, но время и случай для всех их.
\vs Ecc 9:12 Ибо человек не знает своего времени. Как рыбы попадаются в пагубную сеть, и как птицы запутываются в силках, так сыны человеческие уловляются в бедственное время, когда оно неожиданно находит на них.
\rsbpar\vs Ecc 9:13 Вот еще какую мудрость видел я под солнцем, и она показалась мне важною:
\vs Ecc 9:14 город небольшой, и людей в нем немного; к нему подступил великий царь и обложил его и произвел против него большие осадные работы;
\vs Ecc 9:15 но в нем нашелся мудрый бедняк, и он спас своею мудростью этот город; и однако же никто не вспоминал об этом бедном человеке.
\vs Ecc 9:16 И сказал я: мудрость лучше силы, и однако же мудрость бедняка пренебрегается, и слов его не слушают.
\vs Ecc 9:17 Слова мудрых, \bibemph{высказанные} спокойно, выслушиваются \bibemph{лучше}, нежели крик властелина между глупыми.
\vs Ecc 9:18 Мудрость лучше воинских орудий; но один погрешивший погубит много доброго.
\vs Ecc 10:1 Мертвые мухи портят и делают зловонною благовонную масть мироварника: то же делает небольшая глупость уважаемого человека с его мудростью и честью.
\vs Ecc 10:2 Сердце мудрого~--- на правую сторону, а сердце глупого~--- на левую.
\vs Ecc 10:3 По какой бы дороге ни шел глупый, у него \bibemph{всегда} недостает смысла, и всякому он выскажет, что он глуп.
\vs Ecc 10:4 Если гнев начальника вспыхнет на тебя, то не оставляй места твоего; потому что кротость покрывает и большие проступки.
\rsbpar\vs Ecc 10:5 Есть зло, которое видел я под солнцем, это~--- как бы погрешность, происходящая от властелина:
\vs Ecc 10:6 невежество поставляется на большой высоте, а богатые сидят низко.
\vs Ecc 10:7 Видел я рабов на конях, а князей ходящих, подобно рабам, пешком.
\vs Ecc 10:8 Кто копает яму, тот упадет в нее, и кто разрушает ограду, того ужалит змей.
\vs Ecc 10:9 Кто передвигает камни, тот может надсадить себя, и кто колет дрова, тот может подвергнуться опасности от них.
\vs Ecc 10:10 Если притупится топор, и если лезвие его не будет отточено, то надобно будет напрягать силы; мудрость умеет это исправить.
\vs Ecc 10:11 Если змей ужалит без заговаривания, то не лучше его и злоязычный.
\vs Ecc 10:12 Слова из уст мудрого~--- благодать, а уста глупого губят его же:
\vs Ecc 10:13 начало слов из уст его~--- глупость, \bibemph{а} конец речи из уст его~--- безумие.
\vs Ecc 10:14 Глупый наговорит много, \bibemph{хотя} человек не знает, что будет, и кто скажет ему, что будет после него?
\vs Ecc 10:15 Труд глупого утомляет его, потому что не знает \bibemph{даже} дороги в город.
\vs Ecc 10:16 Горе тебе, земля, когда царь твой отрок, и когда князья твои едят рано!
\vs Ecc 10:17 Благо тебе, земля, когда царь у тебя из благородного рода, и князья твои едят вовремя, для подкрепления, а не для пресыщения!
\vs Ecc 10:18 От лености обвиснет потолок, и когда опустятся руки, то протечет дом.
\vs Ecc 10:19 Пиры устраиваются для удовольствия, и вино веселит жизнь; а за все отвечает серебро.
\vs Ecc 10:20 Даже и в мыслях твоих не злословь царя, и в спальной комнате твоей не злословь богатого; потому что птица небесная может перенести слово \bibemph{твое}, и крылатая~--- пересказать речь \bibemph{твою}.
\vs Ecc 11:1 Отпускай хлеб твой по водам, потому что по прошествии многих дней опять найдешь его.
\vs Ecc 11:2 Давай часть семи и даже восьми, потому что не знаешь, какая беда будет на земле.
\vs Ecc 11:3 Когда облака будут полны, то они прольют на землю дождь; и если упадет дерево на юг или на север, то оно там и останется, куда упадет.
\vs Ecc 11:4 Кто наблюдает ветер, тому не сеять; и кто смотрит на облака, тому не жать.
\vs Ecc 11:5 Как ты не знаешь путей ветра и того, как \bibemph{образуются} кости во чреве беременной, так не можешь знать дело Бога, Который делает все.
\vs Ecc 11:6 Утром сей семя твое, и вечером не давай отдыха руке твоей, потому что ты не знаешь, то или другое будет удачнее, или то и другое равно хорошо будет.
\vs Ecc 11:7 Сладок свет, и приятно для глаз видеть солнце.
\vs Ecc 11:8 Если человек проживет \bibemph{и} много лет, то пусть веселится он в продолжение всех их, и пусть помнит о днях темных, которых будет много: все, что будет,~--- суета!
\vs Ecc 11:9 Веселись, юноша, в юности твоей, и да вкушает сердце твое радости во дни юности твоей, и ходи по путям сердца твоего и по видению очей твоих; только знай, что за все это Бог приведет тебя на суд.
\vs Ecc 11:10 И удаляй печаль от сердца твоего, и уклоняй злое от тела твоего, потому что детство и юность~--- суета.
\vs Ecc 12:1 И помни Создателя твоего в дни юности твоей, доколе не пришли тяжелые дни и не наступили годы, о которых ты будешь говорить: <<нет мне удовольствия в них!>>
\vs Ecc 12:2 доколе не померкли солнце и свет и луна и звезды, и не нашли новые тучи вслед за дождем.
\vs Ecc 12:3 В тот день, когда задрожат стерегущие дом и согнутся мужи силы; и перестанут молоть мелющие, потому что их немного осталось; и помрачатся смотрящие в окно;
\vs Ecc 12:4 и запираться будут двери на улицу; когда замолкнет звук жернова, и будет вставать \bibemph{человек} по крику петуха и замолкнут дщери пения;
\vs Ecc 12:5 и высоты будут им страшны, и на дороге ужасы; и зацветет миндаль, и отяжелеет кузнечик, и рассыплется каперс. Ибо отходит человек в вечный дом свой, и готовы окружить его по улице плакальщицы;~---
\vs Ecc 12:6 доколе не порвалась серебряная цепочка, и не разорвалась золотая повязка, и не разбился кувшин у источника, и не обрушилось колесо над колодезем.
\vs Ecc 12:7 И возвратится прах в землю, чем он и был; а дух возвратится к Богу, Который дал его.
\vs Ecc 12:8 Суета сует, сказал Екклесиаст, всё~--- суета!
\rsbpar\vs Ecc 12:9 Кроме того, что Екклесиаст был мудр, он учил еще народ знанию. Он \bibemph{все} испытывал, исследовал, \bibemph{и} составил много притчей.
\vs Ecc 12:10 Старался Екклесиаст приискивать изящные изречения, и слова истины написаны \bibemph{им} верно.
\vs Ecc 12:11 Слова мудрых~--- как иглы и как вбитые гвозди, и составители их~--- от Единого Пастыря.
\vs Ecc 12:12 А что сверх всего этого, сын мой, того берегись: составлять много книг~--- конца не будет, и много читать~--- утомительно для тела.
\rsbpar\vs Ecc 12:13 Выслушаем сущность всего: бойся Бога и заповеди Его соблюдай, потому что в этом всё для человека;
\vs Ecc 12:14 ибо всякое дело Бог приведет на суд, и все тайное, хорошо ли оно, или худо.

\bibbookdescr{Sol}{
  inline={\LARGE Книга\\\Huge Песни Песней Соломона},
  toc={Песнь Песней},
  bookmark={Песнь Песней},
  header={Песнь Песней},
  %headerleft={},
  %headerright={},
  abbr={Песн}
}
\vs Sol 1:1 Да лобзает он меня лобзанием уст своих! Ибо ласки твои лучше вина.
\vs Sol 1:2 От благовония мастей твоих имя твое~--- как разлитое миро; поэтому девицы любят тебя.
\vs Sol 1:3 Влеки меня, мы побежим за тобою;~--- царь ввел меня в чертоги свои,~--- будем восхищаться и радоваться тобою, превозносить ласки твои больше, нежели вино; достойно любят тебя!
\rsbpar\vs Sol 1:4 Дщери Иерусалимские! черна я, но красива, как шатры Кидарские, как завесы Соломоновы.
\vs Sol 1:5 Не смотрите на меня, что я смугла, ибо солнце опалило меня: сыновья матери моей разгневались на меня, поставили меня стеречь виноградники,~--- моего собственного виноградника я не стерегла.
\rsbpar\vs Sol 1:6 Скажи мне, ты, которого любит душа моя: где пасешь ты? где отдыхаешь в полдень? к чему мне быть скиталицею возле стад товарищей твоих?
\vs Sol 1:7 Если ты не знаешь этого, прекраснейшая из женщин, то иди себе по следам овец и паси козлят твоих подле шатров пастушеских.
\vs Sol 1:8 Кобылице моей в колеснице фараоновой я уподобил тебя, возлюбленная моя.
\vs Sol 1:9 Прекрасны ланиты твои под подвесками, шея твоя в ожерельях;
\vs Sol 1:10 золотые подвески мы сделаем тебе с серебряными блестками.
\vs Sol 1:11 Доколе царь был за столом своим, нард мой издавал благовоние свое.
\vs Sol 1:12 Мирровый пучок~--- возлюбленный мой у меня, у грудей моих пребывает.
\vs Sol 1:13 Как кисть кипера, возлюбленный мой у меня в виноградниках Енгедских.
\vs Sol 1:14 О, ты прекрасна, возлюбленная моя, ты прекрасна! глаза твои голубиные.
\vs Sol 1:15 О, ты прекрасен, возлюбленный мой, и любезен! и ложе у нас~--- зелень;
\vs Sol 1:16 кровли домов наших~--- кедры, потолки наши~--- кипарисы.
\vs Sol 2:1 Я нарцисс Саронский, лилия долин!
\vs Sol 2:2 Что лилия между тернами, то возлюбленная моя между девицами.
\vs Sol 2:3 Что яблоня между лесными деревьями, то возлюбленный мой между юношами. В тени ее люблю я сидеть, и плоды ее сладки для гортани моей.
\rsbpar\vs Sol 2:4 Он ввел меня в дом пира, и знамя его надо мною~--- любовь.
\vs Sol 2:5 Подкрепите меня вином, освежите меня яблоками, ибо я изнемогаю от любви.
\vs Sol 2:6 Левая рука его у меня под головою, а правая обнимает меня.
\vs Sol 2:7 Заклинаю вас, дщери Иерусалимские, сернами или полевыми ланями: не будите и не тревожьте возлюбленной, доколе ей угодно.
\rsbpar\vs Sol 2:8 Голос возлюбленного моего! вот, он идет, скачет по горам, прыгает по холмам.
\vs Sol 2:9 Друг мой похож на серну или на молодого оленя. Вот, он стоит у нас за стеною, заглядывает в окно, мелькает сквозь решетку.
\vs Sol 2:10 Возлюбленный мой начал говорить мне: встань, возлюбленная моя, прекрасная моя, выйди!
\vs Sol 2:11 Вот, зима уже прошла; дождь миновал, перестал;
\vs Sol 2:12 цветы показались на земле; время пения настало, и голос горлицы слышен в стране нашей;
\vs Sol 2:13 смоковницы распустили свои почки, и виноградные лозы, расцветая, издают благовоние. Встань, возлюбленная моя, прекрасная моя, выйди!
\vs Sol 2:14 Голубица моя в ущелье скалы под кровом утеса! покажи мне лице твое, дай мне услышать голос твой, потому что голос твой сладок и лице твое приятно.
\vs Sol 2:15 Ловите нам лисиц, лисенят, которые портят виноградники, а виноградники наши в цвете.
\rsbpar\vs Sol 2:16 Возлюбленный мой принадлежит мне, а я ему; он пасет между лилиями.
\vs Sol 2:17 Доколе день дышит \bibemph{прохладою}, и убегают тени, возвратись, будь подобен серне или молодому оленю на расселинах гор.
\vs Sol 3:1 На ложе моем ночью искала я того, которого любит душа моя, искала его и не нашла его.
\vs Sol 3:2 Встану же я, пойду по городу, по улицам и площадям, и буду искать того, которого любит душа моя; искала я его и не нашла его.
\vs Sol 3:3 Встретили меня стражи, обходящие город: <<не видали ли вы того, которого любит душа моя?>>
\vs Sol 3:4 Но едва я отошла от них, как нашла того, которого любит душа моя, ухватилась за него, и не отпустила его, доколе не привела его в дом матери моей и во внутренние комнаты родительницы моей.
\rsbpar\vs Sol 3:5 Заклинаю вас, дщери Иерусалимские, сернами или полевыми ланями: не будите и не тревожьте возлюбленной, доколе ей угодно.
\vs Sol 3:6 Кто эта, восходящая от пустыни как бы столбы дыма, окуриваемая миррою и фимиамом, всякими порошками мироварника?
\rsbpar\vs Sol 3:7 Вот одр его~--- Соломона: шестьдесят сильных вокруг него, из сильных Израилевых.
\vs Sol 3:8 Все они держат по мечу, опытны в бою; у каждого меч при бедре его ради страха ночного.
\vs Sol 3:9 Носильный одр сделал себе царь Соломон из дерев Ливанских;
\vs Sol 3:10 столпцы его сделал из серебра, локотники его из золота, седалище его из пурпуровой ткани; внутренность его убрана с любовью дщерями Иерусалимскими.
\vs Sol 3:11 Пойдите и посмотрите, дщери Сионские, на царя Соломона в венце, которым увенчала его мать его в день бракосочетания его, в день, радостный для сердца его.
\vs Sol 4:1 О, ты прекрасна, возлюбленная моя, ты прекрасна! глаза твои голубиные под кудрями твоими; волосы твои~--- как стадо коз, сходящих с горы Галаадской;
\vs Sol 4:2 зубы твои~--- как стадо выстриженных овец, выходящих из купальни, из которых у каждой пара ягнят, и бесплодной нет между ними;
\vs Sol 4:3 как лента алая губы твои, и уста твои любезны; как половинки гранатового яблока~--- ланиты твои под кудрями твоими;
\vs Sol 4:4 шея твоя~--- как столп Давидов, сооруженный для оружий, тысяча щитов висит на нем~--- все щиты сильных;
\vs Sol 4:5 два сосца твои~--- как двойни молодой серны, пасущиеся между лилиями.
\vs Sol 4:6 Доколе день дышит \bibemph{прохладою}, и убегают тени, пойду я на гору мирровую и на холм фимиама.
\rsbpar\vs Sol 4:7 Вся ты прекрасна, возлюбленная моя, и пятна нет на тебе!
\vs Sol 4:8 Со мною с Ливана, невеста! со мною иди с Ливана! спеши с вершины Аманы, с вершины Сенира и Ермона, от логовищ львиных, от гор барсовых!
\vs Sol 4:9 Пленила ты сердце мое, сестра моя, невеста! пленила ты сердце мое одним взглядом очей твоих, одним ожерельем на шее твоей.
\vs Sol 4:10 О, как любезны ласки твои, сестра моя, невеста! о, как много ласки твои лучше вина, и благовоние мастей твоих лучше всех ароматов!
\vs Sol 4:11 Сотовый мед каплет из уст твоих, невеста; мед и молоко под языком твоим, и благоухание одежды твоей подобно благоуханию Ливана!
\vs Sol 4:12 Запертый сад~--- сестра моя, невеста, заключенный колодезь, запечатанный источник:
\vs Sol 4:13 рассадники твои~--- сад с гранатовыми яблоками, с превосходными плодами, киперы с нардами,
\vs Sol 4:14 нард и шафран, аир и корица со всякими благовонными деревами, мирра и алой со всякими лучшими ароматами;
\vs Sol 4:15 садовый источник~--- колодезь живых вод и потоки с Ливана.
\vs Sol 4:16 Поднимись \bibemph{ветер} с севера и принесись с юга, повей на сад мой,~--- и польются ароматы его!~--- Пусть придет возлюбленный мой в сад свой и вкушает сладкие плоды его.
\vs Sol 5:1 Пришел я в сад мой, сестра моя, невеста; набрал мирры моей с ароматами моими, поел сотов моих с медом моим, напился вина моего с молоком моим. Ешьте, друзья, пейте и насыщайтесь, возлюбленные!
\rsbpar\vs Sol 5:2 Я сплю, а сердце мое бодрствует; \bibemph{вот}, голос моего возлюбленного, который стучится: <<отвори мне, сестра моя, возлюбленная моя, голубица моя, чистая моя! потому что голова моя вся покрыта росою, кудри мои~--- ночною влагою>>.
\vs Sol 5:3 Я скинула хитон мой; как же мне опять надевать его? Я вымыла ноги мои; как же мне марать их?
\vs Sol 5:4 Возлюбленный мой протянул руку свою сквозь скважину, и внутренность моя взволновалась от него.
\vs Sol 5:5 Я встала, чтобы отпереть возлюбленному моему, и с рук моих капала мирра, и с перстов моих мирра капала на ручки замка.
\vs Sol 5:6 Отперла я возлюбленному моему, а возлюбленный мой повернулся и ушел. Души во мне не стало, когда он говорил; я искала его и не находила его; звала его, и он не отзывался мне.
\vs Sol 5:7 Встретили меня стражи, обходящие город, избили меня, изранили меня; сняли с меня покрывало стерегущие стены.
\vs Sol 5:8 Заклинаю вас, дщери Иерусалимские: если вы встретите возлюбленного моего, что скажете вы ему? что я изнемогаю от любви.
\vs Sol 5:9 <<Чем возлюбленный твой лучше других возлюбленных, прекраснейшая из женщин? Чем возлюбленный твой лучше других, что ты так заклинаешь нас?>>
\vs Sol 5:10 Возлюбленный мой бел и румян, лучше десяти тысяч других:
\vs Sol 5:11 голова его~--- чистое золото; кудри его волнистые, черные, как ворон;
\vs Sol 5:12 глаза его~--- как голуби при потоках вод, купающиеся в молоке, сидящие в довольстве;
\vs Sol 5:13 щеки его~--- цветник ароматный, гряды благовонных растений; губы его~--- лилии, источают текучую мирру;
\vs Sol 5:14 руки его~--- золотые кругляки, усаженные топазами; живот его~--- как изваяние из слоновой кости, обложенное сапфирами;
\vs Sol 5:15 голени его~--- мраморные столбы, поставленные на золотых подножиях; вид его подобен Ливану, величествен, как кедры;
\vs Sol 5:16 уста его~--- сладость, и весь он~--- любезность. Вот кто возлюбленный мой, и вот кто друг мой, дщери Иерусалимские!
\vs Sol 6:1 <<Куда пошел возлюбленный твой, прекраснейшая из женщин? куда обратился возлюбленный твой? мы поищем его с тобою>>.
\vs Sol 6:2 Мой возлюбленный пошел в сад свой, в цветники ароматные, чтобы пасти в садах и собирать лилии.
\vs Sol 6:3 Я принадлежу возлюбленному моему, а возлюбленный мой~--- мне; он пасет между лилиями.
\rsbpar\vs Sol 6:4 Прекрасна ты, возлюбленная моя, как Фирца, любезна, как Иерусалим, грозна, как полки со знаменами.
\vs Sol 6:5 Уклони очи твои от меня, потому что они волнуют меня.
\vs Sol 6:6 Волосы твои~--- как стадо коз, сходящих с Галаада; зубы твои~--- как стадо овец, выходящих из купальни, из которых у каждой пара ягнят, и бесплодной нет между ними;
\vs Sol 6:7 как половинки гранатового яблока~--- ланиты твои под кудрями твоими.
\vs Sol 6:8 Есть шестьдесят цариц и восемьдесят наложниц и девиц без числа,
\vs Sol 6:9 но единственная~--- она, голубица моя, чистая моя; единственная она у матери своей, отличенная у родительницы своей. Увидели ее девицы, и~--- превознесли ее, царицы и наложницы, и~--- восхвалили ее.
\vs Sol 6:10 Кто эта, блистающая, как заря, прекрасная, как луна, светлая, как солнце, грозная, как полки со знаменами?
\vs Sol 6:11 Я сошла в ореховый сад посмотреть на зелень долины, поглядеть, распустилась ли виноградная лоза, расцвели ли гранатовые яблоки?
\vs Sol 6:12 Не знаю, как душа моя влекла меня к колесницам знатных народа моего.
\vs Sol 7:1 <<Оглянись, оглянись, Суламита! оглянись, оглянись,~--- и мы посмотрим на тебя>>. Что вам смотреть на Суламиту, как на хоровод Манаимский?
\vs Sol 7:2 О, как прекрасны ноги твои в сандалиях, дщерь именитая! Округление бедр твоих, как ожерелье, дело рук искусного художника;
\vs Sol 7:3 живот твой~--- круглая чаша, \bibemph{в которой} не истощается ароматное вино; чрево твое~--- ворох пшеницы, обставленный лилиями;
\vs Sol 7:4 два сосца твои~--- как два козленка, двойни серны;
\vs Sol 7:5 шея твоя~--- как столп из слоновой кости; глаза твои~--- озерки Есевонские, что у ворот Батраббима; нос твой~--- башня Ливанская, обращенная к Дамаску;
\vs Sol 7:6 голова твоя на тебе, как Кармил, и волосы на голове твоей, как пурпур; царь увлечен \bibemph{твоими} кудрями.
\vs Sol 7:7 Как ты прекрасна, как привлекательна, возлюбленная, твоею миловидностью!
\vs Sol 7:8 Этот стан твой похож на пальму, и груди твои на виноградные кисти.
\vs Sol 7:9 Подумал я: влез бы я на пальму, ухватился бы за ветви ее; и груди твои были бы вместо кистей винограда, и запах от ноздрей твоих, как от яблоков;
\vs Sol 7:10 уста твои~--- как отличное вино. Оно течет прямо к другу моему, услаждает уста утомленных.
\vs Sol 7:11 Я принадлежу другу моему, и ко мне \bibemph{обращено} желание его.
\vs Sol 7:12 Приди, возлюбленный мой, выйдем в поле, побудем в селах;
\vs Sol 7:13 поутру пойдем в виноградники, посмотрим, распустилась ли виноградная лоза, раскрылись ли почки, расцвели ли гранатовые яблоки; там я окажу ласки мои тебе.
\vs Sol 7:14 Мандрагоры уже пустили благовоние, и у дверей наших всякие превосходные плоды, новые и старые: \bibemph{это} сберегла я для тебя, мой возлюбленный!
\vs Sol 8:1 О, если бы ты был мне брат, сосавший груди матери моей! тогда я, встретив тебя на улице, целовала бы тебя, и меня не осуждали бы.
\vs Sol 8:2 Повела бы я тебя, привела бы тебя в дом матери моей. Ты учил бы меня, а я поила бы тебя ароматным вином, соком гранатовых яблоков моих.
\vs Sol 8:3 Левая рука его у меня под головою, а правая обнимает меня.
\vs Sol 8:4 Заклинаю вас, дщери Иерусалимские,~--- не будите и не тревожьте возлюбленной, доколе ей угодно.
\rsbpar\vs Sol 8:5 Кто это восходит от пустыни, опираясь на своего возлюбленного? Под яблоней разбудила я тебя: там родила тебя мать твоя, там родила тебя родительница твоя.
\vs Sol 8:6 Положи меня, как печать, на сердце твое, как перстень, на руку твою: ибо крепка, как смерть, любовь; люта, как преисподняя, ревность; стрелы ее~--- стрелы огненные; она пламень весьма сильный.
\vs Sol 8:7 Большие воды не могут потушить любви, и реки не зальют ее. Если бы кто давал все богатство дома своего за любовь, то он был бы отвергнут с презреньем.
\rsbpar\vs Sol 8:8 Есть у нас сестра, которая еще мала, и сосцов нет у нее; что нам будет делать с сестрою нашею, когда будут свататься за нее?
\vs Sol 8:9 Если бы она была стена, то мы построили бы на ней палаты из серебра; если бы она была дверь, то мы обложили бы ее кедровыми досками.
\vs Sol 8:10 Я~--- стена, и сосцы у меня, как башни; потому я буду в глазах его, как достигшая полноты.
\rsbpar\vs Sol 8:11 Виноградник был у Соломона в Ваал-Гамоне; он отдал этот виноградник сторожам; каждый должен был доставлять за плоды его тысячу сребреников.
\vs Sol 8:12 А мой виноградник у меня при себе. Тысяча пусть тебе, Соломон, а двести~--- стерегущим плоды его.
\vs Sol 8:13 Жительница садов! товарищи внимают голосу твоему, дай и мне послушать его.
\rsbpar\vs Sol 8:14 Беги, возлюбленный мой; будь подобен серне или молодому оленю на горах бальзамических!

\bibbookdescr{Wis}{
  inline={\LARGE Книга\\\Huge Премудрости Соломона\fns{Переведена с греческого.}},
  toc={Премудрость Соломона*},
  bookmark={Премудрость Соломона},
  header={Премудрость Соломона},
  %headerleft={},
  %headerright={},
  abbr={Прем}
}
\vs Wis 1:1 Любите справедливость, судьи земли, право мыслите о Господе, и в простоте сердца ищите Его,
\vs Wis 1:2 ибо Он обретается неискушающими Его и является не неверующим Ему.
\vs Wis 1:3 Ибо неправые умствования отдаляют от Бога, и испытание силы Его обличит безумных.
\vs Wis 1:4 В лукавую душу не войдет премудрость и не будет обитать в теле, порабощенном греху,
\vs Wis 1:5 ибо святый Дух премудрости удалится от лукавства и уклонится от неразумных умствований, и устыдится приближающейся неправды.
\vs Wis 1:6 Человеколюбивый дух~--- премудрость, но не оставит безнаказанным богохульствующего устами, потому что Бог есть свидетель внутренних чувств его и истинный зритель сердца его, и слышатель языка его.
\vs Wis 1:7 Дух Господа наполняет вселенную и, как все объемлющий, знает \bibemph{всякое} слово.
\vs Wis 1:8 Посему никто, говорящий неправду, не утаится, и не минет его обличающий суд.
\vs Wis 1:9 Ибо будет испытание помыслов нечестивого, и слов\acc{а} его взойдут к Господу в обличение беззаконий его;
\vs Wis 1:10 потому что ухо ревности слышит все, и ропот не скроется.
\vs Wis 1:11 Итак, хранитесь от бесполезного ропота и берегитесь от злоречия языка, ибо и тайное слово не пройдет даром, а клевещущие уста убивают душу.
\vs Wis 1:12 Не ускоряйте смерти заблуждениями вашей жизни и не привлекайте к себе погибели делами рук ваших.
\vs Wis 1:13 Бог не сотворил смерти и не радуется погибели живущих,
\vs Wis 1:14 ибо Он создал все для бытия, и все в мире спасительно, и нет пагубного яда, нет и царства ада на земле.
\vs Wis 1:15 Праведность бессмертна, а неправда причиняет смерть:
\vs Wis 1:16 нечестивые привлекли ее и руками и словами, сочли ее другом и исчахли, и заключили союз с нею, ибо они достойны быть ее жребием.
\vs Wis 2:1 Неправо умствующие говорили сами в себе: <<коротка и прискорбна наша жизнь, и нет человеку спасения от смерти, и не знают, чтобы кто освободил из ада.
\vs Wis 2:2 Случайно мы рождены и после будем как небывшие: дыхание в ноздрях наших~--- дым, и слово~--- искра в движении нашего сердца.
\vs Wis 2:3 Когда она угаснет, тело обратится в прах, и дух рассеется, как жидкий воздух;
\vs Wis 2:4 и имя наше забудется со временем, и никто не вспомнит о делах наших; и жизнь наша пройдет, как след облака, и рассеется, как туман, разогнанный лучами солнца и отягченный теплотою его.
\vs Wis 2:5 Ибо жизнь наша~--- прохождение тени, и нет нам возврата от смерти: ибо положена печать, и никто не возвращается.
\vs Wis 2:6 Будем же наслаждаться настоящими благами и спешить пользоваться миром, как юностью;
\vs Wis 2:7 преисполнимся дорогим вином и благовониями, и да не пройдет мимо нас весенний цвет жизни;
\vs Wis 2:8 увенчаемся цветами роз прежде, нежели они увяли;
\vs Wis 2:9 никто из нас не лишай себя участия в нашем наслаждении; везде оставим следы веселья, ибо это наша доля и наш жребий.
\vs Wis 2:10 Будем притеснять бедняка праведника, не пощадим вдовы и не постыдимся многолетних седин старца.
\vs Wis 2:11 Сила наша да будет законом правды, ибо бессилие оказывается бесполезным.
\vs Wis 2:12 Устроим ковы праведнику, ибо он в тягость нам и противится делам нашим, укоряет нас в грехах против закона и поносит нас за грехи нашего воспитания;
\vs Wis 2:13 объявляет себя имеющим познание о Боге и называет себя сыном Господа;
\vs Wis 2:14 он пред нами~--- обличение помыслов наших.
\vs Wis 2:15 Тяжело нам и смотреть на него, ибо жизнь его не похожа на жизнь других, и отличны пути его:
\vs Wis 2:16 он считает нас мерзостью и удаляется от путей наших, как от нечистот, ублажает кончину праведных и тщеславно называет отцом своим Бога.
\vs Wis 2:17 Увидим, истинны ли слова его, и испытаем, какой будет исход его;
\vs Wis 2:18 ибо если этот праведник есть сын Божий, то \bibemph{Бог} защитит его и избавит его от руки врагов.
\vs Wis 2:19 Испытаем его оскорблением и мучением, дабы узнать смирение его и видеть незлобие его;
\vs Wis 2:20 осудим его на бесчестную смерть, ибо, по словам его, о нем попечение будет>>.
\vs Wis 2:21 Так они умствовали, и ошиблись; ибо злоба их ослепила их,
\vs Wis 2:22 и они не познали тайн Божиих, не ожидали воздаяния за святость и не считали достойными награды душ непорочных.
\vs Wis 2:23 Бог создал человека для нетления и соделал его образом вечного бытия Своего;
\vs Wis 2:24 но завистью диавола вошла в мир смерть, и испытывают ее принадлежащие к уделу его.
\vs Wis 3:1 А души праведных в руке Божией, и мучение не коснется их.
\vs Wis 3:2 В глазах неразумных они казались умершими, и исход их считался погибелью,
\vs Wis 3:3 и отшествие от нас~--- уничтожением; но они пребывают в мире.
\vs Wis 3:4 Ибо, хотя они в глазах людей и наказываются, но надежда их полна бессмертия.
\vs Wis 3:5 И немного наказанные, они будут много облагодетельствованы, потому что Бог испытал их и нашел их достойными Его.
\vs Wis 3:6 Он испытал их как золото в горниле и принял их как жертву всесовершенную.
\vs Wis 3:7 Во время воздаяния им они воссияют как искры, бегущие по стеблю.
\vs Wis 3:8 Будут судить племена и владычествовать над народами, а над ними будет Господь царствовать во веки.
\vs Wis 3:9 Надеющиеся на Него познают истину, и верные в любви пребудут у Него; ибо благодать и милость со святыми Его и промышление об избранных Его.
\vs Wis 3:10 Нечестивые же, как умствовали, так и понесут наказание за то, что презрели праведного и отступили от Господа.
\vs Wis 3:11 Ибо презирающий мудрость и наставление несчастен, и надежда их суетна, и труды бесплодны, и дела их непотребны.
\vs Wis 3:12 Жены их несмысленны, и дети их злы, проклят род их.
\vs Wis 3:13 Блаженна неплодная неосквернившаяся, которая не познала беззаконного ложа; она получит плод при воздаянии святых душ.
\vs Wis 3:14 \bibemph{Блажен} и евнух, не сделавший беззакония рукою и не помысливший лукавого против Господа, ибо дастся ему особенная благодать веры и приятнейший жребий в храме Господнем.
\vs Wis 3:15 Плод добрых трудов славен, и корень мудрости неподвижен.
\vs Wis 3:16 Дети прелюбодеев будут несовершенны, и семя беззаконного ложа исчезнет.
\vs Wis 3:17 Если и будут они долгожизненны, но будут почитаться за ничто, и поздняя старость их будет без почета.
\vs Wis 3:18 А если скоро умрут, не будут иметь надежды и утешения в день суда;
\vs Wis 3:19 ибо ужасен конец неправедного рода.
\vs Wis 4:1 Лучше бездетность с добродетелью, ибо память о ней бессмертна: она признается и у Бога и у людей.
\vs Wis 4:2 Когда она присуща, ей подражают, а когда отойдет, стремятся к ней: и в вечности увенчанная она торжествует, как одержавшая победу непорочными подвигами.
\vs Wis 4:3 А плодородное множество нечестивых не принесет пользы, и прелюбодейные отрасли не дадут корней в глубину и не достигнут незыблемого основания;
\vs Wis 4:4 и хотя на время позеленеют в ветвях, но, не имея твердости, поколеблются от ветра и порывом ветров искоренятся;
\vs Wis 4:5 некрепкие ветви переломятся, и плод их \bibemph{будет} бесполезен, незрел для пищи и ни к чему не годен;
\vs Wis 4:6 ибо дети, рождаемые от беззаконных сожитий, суть свидетели разврата против родителей при допросе их.
\vs Wis 4:7 А праведник, если и рановременно умрет, будет в покое,
\vs Wis 4:8 ибо не в долговечности честная старость и не числом лет измеряется:
\vs Wis 4:9 мудрость есть седина для людей, и беспорочная жизнь~--- возраст старости.
\vs Wis 4:10 Как благоугодивший Богу, он возлюблен, и, как живший посреди грешников, преставлен,
\vs Wis 4:11 восхищен, чтобы злоба не изменила разума его, или коварство не прельстило души его.
\vs Wis 4:12 Ибо упражнение в нечестии помрачает доброе, и волнение похоти развращает ум незлобивый.
\vs Wis 4:13 Достигнув совершенства в короткое время, он исполнил долгие лета;
\vs Wis 4:14 ибо душа его была угодна Господу, потому и ускорил он из среды нечестия. А люди видели это и не поняли, даже и не подумали о том,
\vs Wis 4:15 что благодать и милость со святыми Его и промышление об избранных Его.
\vs Wis 4:16 Праведник, умирая, осудит живых нечестивых, и скоро достигшая совершенства юность~--- долголетнюю старость неправедного;
\vs Wis 4:17 ибо они увидят кончину мудрого и не поймут, что Господь определил о нем и для чего поставил его в безопасность;
\vs Wis 4:18 они увидят и уничтожат его, но Господь посмеется им;
\vs Wis 4:19 и после сего будут они бесчестным трупом и позором между умершими навек, ибо Он повергнет их ниц безгласными и сдвинет их с оснований, и они вконец запустеют и будут в скорби, и память их погибнет;
\vs Wis 4:20 в сознании грехов своих они предстанут со страхом, и беззакония их осудят их в лице их.
\vs Wis 5:1 Тогда праведник с великим дерзновением станет пред лицем тех, которые оскорбляли его и презирали подвиги его;
\vs Wis 5:2 они же, увидев, смутятся великим страхом и изумятся неожиданности спасения его
\vs Wis 5:3 и, раскаиваясь и воздыхая от стеснения духа, будут говорить сами в себе: <<это тот самый, который был у нас некогда в посмеянии и притчею поругания.
\vs Wis 5:4 Безумные, мы почитали жизнь его сумасшествием и кончину его бесчестною!
\vs Wis 5:5 Как же он причислен к сынам Божиим, и жребий его~--- со святыми?
\vs Wis 5:6 Итак, мы заблудились от пути истины, и свет правды не светил нам, и солнце не озаряло нас.
\vs Wis 5:7 Мы преисполнились делами беззакония и погибели и ходили по непроходимым пустыням, а пути Господня не познали.
\vs Wis 5:8 Какую пользу принесло нам высокомерие, и что доставило нам богатство с тщеславием?
\vs Wis 5:9 Все это прошло как тень и как молва быстротечная.
\vs Wis 5:10 Как после прохождения корабля, идущего по волнующейся воде, невозможно найти следа, ни стези дна его в волнах;
\vs Wis 5:11 или как от птицы, пролетающей по воздуху, никакого не остается знака ее пути, но легкий воздух, ударяемый крыльями и рассекаемый быстротою движения, пройден движущимися крыльями, и после того не осталось никакого знака прохождения по нему;
\vs Wis 5:12 или как от стрелы, пущенной в цель, разделенный воздух тотчас опять сходится, так что нельзя узнать, где прошла она;
\vs Wis 5:13 так и мы родились и умерли, и не могли показать никакого знака добродетели, но истощились в беззаконии нашем>>.
\vs Wis 5:14 Ибо надежда нечестивого исчезает, как прах, уносимый ветром, и как тонкий иней, разносимый бурею, и как дым, рассеиваемый ветром, и проходит, как память об однодневном госте.
\vs Wis 5:15 А праведники живут во веки; награда их~--- в Господе, и попечение о них~--- у Вышнего.
\vs Wis 5:16 Посему они получат царство славы и венец красоты от руки Господа, ибо Он покроет их десницею и защитит их мышцею.
\vs Wis 5:17 Он возьмет всеоружие~--- ревность Свою, и тварь вооружит к отмщению врагам;
\vs Wis 5:18 облечется в броню~--- в правду, и возложит на Себя шлем~--- нелицеприятный суд;
\vs Wis 5:19 возьмет непобедимый щит~--- святость;
\vs Wis 5:20 строгий гнев Он изострит, как меч, и мир ополчится с Ним против безумцев.
\vs Wis 5:21 Понесутся меткие стрелы молний и из облаков, как из туго натянутого лука, полетят в цель.
\vs Wis 5:22 И, как из каменометного орудия, с яростью посыплется град; вознегодует на них вода морская и реки свирепо потопят их;
\vs Wis 5:23 восстанет против них дух силы и, как вихрь, развеет их.
\vs Wis 5:24 Так беззаконие опустошит всю землю, и злодеяние ниспровергнет престолы сильных.
\vs Wis 6:1 Итак, слушайте, цари, и разумейте, научитесь, судьи концов земли!
\vs Wis 6:2 Внимайте, обладатели множества и гордящиеся пред народами!
\vs Wis 6:3 От Господа дана вам держава, и сила~--- от Вышнего, Который исследует ваши дела и испытает намерения.
\vs Wis 6:4 Ибо вы, будучи служителями Его царства, не судили справедливо, не соблюдали закона и не поступали по воле Божией.
\vs Wis 6:5 Страшно и скоро Он явится вам,~--- и строг суд над начальствующими,
\vs Wis 6:6 ибо меньший заслуживает помилование, а сильные сильно будут истязаны.
\vs Wis 6:7 Господь всех не убоится лица и не устрашится величия, ибо Он сотворил и малого и великого и одинаково промышляет о всех;
\vs Wis 6:8 но начальствующим предстоит строгое испытание.
\vs Wis 6:9 Итак, к вам, цари, слова мои, чтобы вы научились премудрости и не падали.
\vs Wis 6:10 Ибо свято хранящие святое освятятся, и научившиеся тому найдут оправдание.
\vs Wis 6:11 Итак, возжелайте слов моих, полюбите и научитесь.
\vs Wis 6:12 Премудрость светла и неувядающа, и легко созерцается любящими ее, и обретается ищущими ее;
\vs Wis 6:13 она \bibemph{даже} упреждает желающих познать ее.
\vs Wis 6:14 С раннего утра ищущий ее не утомится, ибо найдет ее сидящею у дверей своих.
\vs Wis 6:15 Помышлять о ней есть уже совершенство разума, и бодрствующий ради нее скоро освободится от забот,
\vs Wis 6:16 ибо она сама обходит и ищет достойных ее, и благосклонно является им на путях, и при всякой мысли встречается с ними.
\vs Wis 6:17 Начало ее есть искреннейшее желание учения,
\vs Wis 6:18 а забота об учении~--- любовь, любовь же~--- хранение законов ее, а наблюдение законов~--- залог бессмертия,
\vs Wis 6:19 а бессмертие приближает к Богу;
\vs Wis 6:20 поэтому желание премудрости возводит к царству.
\vs Wis 6:21 Итак, властители народов, если вы услаждаетесь престолами и скипетрами, то почтите премудрость, чтобы вам царствовать во веки.
\vs Wis 6:22 Что же есть премудрость, и как она произошла, я возвещу,
\vs Wis 6:23 и не скрою от вас тайн, но исследую от начала рождения,
\vs Wis 6:24 и открою познание ее, и не миную истины;
\vs Wis 6:25 и не пойду вместе с истаевающим от зависти, ибо таковой не будет причастником премудрости.
\vs Wis 6:26 Множество мудрых~--- спасение миру, и царь разумный~--- благосостояние народа.
\vs Wis 6:27 Итак учитесь от слов моих, и получите пользу.
\vs Wis 7:1 И я человек смертный, подобный всем, потомок первозданного земнородного.
\vs Wis 7:2 И я в утробе матерней образовался в плоть в десятимесячное время, сгустившись в крови от семени мужа и услаждения, соединенного со сном,
\vs Wis 7:3 и я, родившись, начал дышать общим воздухом и ниспал на ту же землю, первый голос обнаружил плачем одинаково со всеми,
\vs Wis 7:4 вскормлен в пеленах и заботах;
\vs Wis 7:5 ибо ни один царь не имел иного начала рождения:
\vs Wis 7:6 один для всех вход в жизнь и одинаковый исход.
\vs Wis 7:7 Посему я молился, и дарован мне разум; я взывал, и сошел на меня дух премудрости.
\vs Wis 7:8 Я предпочел ее скипетрам и престолам и богатство почитал за ничто в сравнении с нею;
\vs Wis 7:9 драгоценного камня я не сравнил с нею, потому что перед нею все золото~--- ничтожный песок, а серебро~--- грязь в сравнении с нею.
\vs Wis 7:10 Я полюбил ее более здоровья и красоты и избрал ее предпочтительно перед светом, ибо свет ее неугасим.
\vs Wis 7:11 А вместе с нею пришли ко мне все блага и несметное богатство через руки ее;
\vs Wis 7:12 я радовался всему, потому что премудрость руководствовала ими, но я не знал, что она~--- виновница их.
\vs Wis 7:13 Без хитрости я научился, и без зависти преподаю, не скрываю богатства ее,
\vs Wis 7:14 ибо она есть неистощимое сокровище для людей; пользуясь ею, они входят в содружество с Богом, посредством даров учения.
\vs Wis 7:15 Только дал бы мне Бог говорить по разумению и достойно мыслить о дарованном, ибо Он есть руководитель к мудрости и исправитель мудрых.
\vs Wis 7:16 Ибо в руке Его и мы и слова наши, и всякое разумение и искусство делания.
\vs Wis 7:17 Сам Он даровал мне неложное познание существующего, чтобы познать устройство мира и действие стихий,
\vs Wis 7:18 начало, конец и средину времен, смены поворотов и перемены времен,
\vs Wis 7:19 круги годов и положение звезд,
\vs Wis 7:20 природу животных и свойства зверей, стремления ветров и мысли людей, различия растений и силы корней.
\vs Wis 7:21 Познал я все, и сокровенное и явное, ибо научила меня Премудрость, художница всего.
\vs Wis 7:22 Она есть дух разумный, святый, единородный, многочастный, тонкий, удобоподвижный, светлый, чистый, ясный, невредительный, благолюбивый, скорый, неудержимый,
\vs Wis 7:23 благодетельный, человеколюбивый, твердый, непоколебимый, спокойный, беспечальный, всевидящий и проникающий все умные, чистые, тончайшие духи.
\vs Wis 7:24 Ибо премудрость подвижнее всякого движения, и по чистоте своей сквозь все проходит и проникает.
\vs Wis 7:25 Она есть дыхание силы Божией и чистое излияние славы Вседержителя: посему ничто оскверненное не войдет в нее.
\vs Wis 7:26 Она есть отблеск вечного света и чистое зеркало действия Божия и образ благости Его.
\vs Wis 7:27 Она~--- одна, но может все, и, пребывая в самой себе, все обновляет, и, переходя из рода в род в святые души, приготовляет друзей Божиих и пророков;
\vs Wis 7:28 ибо Бог никого не любит, кроме живущего с премудростью.
\vs Wis 7:29 Она прекраснее солнца и превосходнее сонма звезд; в сравнении со светом она выше;
\vs Wis 7:30 ибо свет сменяется ночью, а премудрости не превозмогает злоба.
\vs Wis 8:1 Она быстро распростирается от одного конца до другого и все устрояет на пользу.
\vs Wis 8:2 Я полюбил ее и взыскал от юности моей, и пожелал взять ее в невесту себе, и стал любителем красоты ее.
\vs Wis 8:3 Она возвышает \bibemph{свое} благородство тем, что имеет сожитие с Богом, и Владыка всех возлюбил ее:
\vs Wis 8:4 она таинница ума Божия и избирательница дел Его.
\vs Wis 8:5 Если богатство есть вожделенное приобретение в жизни, то что богаче премудрости, которая все делает?
\vs Wis 8:6 Если же благоразумие делает \bibemph{многое}, то какой художник лучше ее?
\vs Wis 8:7 Если кто любит праведность,~--- плоды ее суть добродетели: она научает целомудрию и рассудительности, справедливости и мужеству, полезнее которых ничего нет для людей в жизни.
\vs Wis 8:8 Если кто желает большой опытности, мудрость знает давнопрошедшее и угадывает будущее, знает тонкости слов и разрешение загадок, предузнает знамения и чудеса и последствия лет и времен.
\vs Wis 8:9 Посему я рассудил принять ее в сожитие с собою, зная, что она будет мне советницею на доброе и утешеньем в заботах и печали.
\vs Wis 8:10 Через нее я буду иметь славу в народе и честь перед старейшими, будучи юношею;
\vs Wis 8:11 окажусь проницательным в суде, и в глазах сильных заслужу удивление.
\vs Wis 8:12 Когда я буду молчать, они будут ожидать, и когда начну говорить, будут внимать, и когда продлю беседу, положат руку на уста свои.
\vs Wis 8:13 Чрез нее я достигну бессмертия и оставлю вечную память будущим после меня.
\vs Wis 8:14 Я буду управлять народами, и племена покорятся мне;
\vs Wis 8:15 убоятся меня, когда услышат обо мне страшные тираны; в народе явлюсь добрым и на войне мужественным.
\vs Wis 8:16 Войдя в дом свой, я успокоюсь ею, ибо в обращении ее нет суровости, ни в сожитии с нею скорби, но веселие и радость.
\vs Wis 8:17 Размышляя о сем сам в себе и обдумывая в сердце своем, что в родстве с премудростью~--- бессмертие,
\vs Wis 8:18 и в дружестве с нею~--- благое наслаждение, и в трудах рук ее~--- богатство неоскудевающее, и в собеседовании с нею~--- разум, и в общении слов ее~--- добрая слава,~--- я ходил и искал, как бы мне взять ее себе.
\vs Wis 8:19 Я был отрок даровитый и душу получил добрую;
\vs Wis 8:20 притом, будучи добрым, я вошел и в тело чистое.
\vs Wis 8:21 Познав же, что иначе не могу овладеть ею, как если дарует Бог,~--- и что уже было делом разума, чтобы познать, чей этот дар,~--- я обратился к Господу и молился Ему, и говорил от всего сердца моего:
\vs Wis 9:1 Боже отцов и Господи милости, сотворивший все словом Твоим
\vs Wis 9:2 и премудростию Твоею устроивший человека, чтобы он владычествовал над созданными Тобою тварями
\vs Wis 9:3 и управлял миром свято и справедливо, и в правоте души производил суд!
\vs Wis 9:4 Даруй мне приседящую престолу Твоему премудрость и не отринь меня от отроков Твоих,
\vs Wis 9:5 ибо я раб Твой и сын рабы Твоей, человек немощный и кратковременный и слабый в разумении суда и законов.
\vs Wis 9:6 Да хотя бы кто и совершен был между сынами человеческими, без Твоей премудрости он будет признан за ничто.
\vs Wis 9:7 Ты избрал меня царем народа Твоего и судьею сынов Твоих и дщерей;
\vs Wis 9:8 Ты сказал, чтобы я построил храм на святой горе Твоей и алтарь в городе обитания Твоего, по подобию святой скинии, которую Ты предуготовил от начала.
\vs Wis 9:9 С Тобою премудрость, которая знает дела Твои и присуща была, когда Ты творил мир, и ведает, что угодно пред очами Твоими и что право по заповедям Твоим:
\vs Wis 9:10 ниспошли ее от святых небес и от престола славы Твоей ниспошли ее, чтобы она споспешествовала мне в трудах моих, и чтобы я знал, что благоугодно пред Тобою;
\vs Wis 9:11 ибо она все знает и разумеет, и мудро будет руководить меня в делах моих, и сохранит меня в своей славе;
\vs Wis 9:12 и дела мои будут благоприятны, и буду судить народ Твой справедливо, и буду достойным престола отца моего.
\vs Wis 9:13 Ибо какой человек в состоянии познать совет Божий? или кто может уразуметь, что угодно Господу?
\vs Wis 9:14 Помышления смертных нетверды, и мысли наши ошибочны,
\vs Wis 9:15 ибо тленное тело отягощает душу, и эта земная храмина подавляет многозаботливый ум.
\vs Wis 9:16 Мы едва можем постигать и то, что на земле, и с трудом понимаем то, что под руками, а что на небесах~--- кто исследовал?
\vs Wis 9:17 Волю же Твою кто познал бы, если бы Ты не даровал премудрости и не ниспослал свыше святаго Твоего Духа?
\vs Wis 9:18 И так исправились пути живущих на земле, и люди научились тому, что угодно Тебе,
\vs Wis 9:19 и спаслись премудростью.
\vs Wis 10:1 Она сохраняла первозданного отца мира, который сотворен был один, и спасала его от собственного его падения:
\vs Wis 10:2 она дала ему силу владычествовать над всем.
\vs Wis 10:3 А отступивший от нее неправедный во гневе своем погиб от братоубийственной ярости.
\vs Wis 10:4 Ради него потопляемую землю опять премудрость спасла, сохранив праведника посредством малого дерева.
\vs Wis 10:5 Она же между народами, смешанными в единомыслии зла, нашла праведника и соблюла его неукоризненным пред Богом, и сохранила мужественным в жалости к сыну.
\vs Wis 10:6 Она во время погибели нечестивых спасла праведного, который избежал огня, нисшедшего на пять городов,
\vs Wis 10:7 от которых во свидетельство нечестия осталась дымящаяся пустая земля и растения, не в свое время приносящие плоды, и памятником неверной души~--- стоящий соляной столб.
\vs Wis 10:8 Ибо они, презрев премудрость, не только повредили себе тем, что не познали добра, но и оставили живущим память о своем безумии, дабы не могли скрыть того, в чем заблудились.
\vs Wis 10:9 Премудрость же спасла от бед служащих ей.
\vs Wis 10:10 Праведного, бежавшего от братнего гнева, она наставляла на правые пути, показала ему царство Божие и даровала ему познание святых, помогала ему в огорчениях и обильно вознаградила труды его.
\vs Wis 10:11 Когда из корыстолюбия обижали его, она предстала и обогатила его,
\vs Wis 10:12 сохранила его от врагов, и обезопасила от коварствовавших против него, и в крепкой борьбе доставила ему победу, дабы он знал, что благочестие всего сильнее.
\vs Wis 10:13 Она не оставила проданного праведника, но спасла его от греха:
\vs Wis 10:14 она нисходила с ним в ров и не оставляла его в узах, и потом принесла ему скипетр царства и власть над угнетавшими его, показала лжецами обвинявших его и даровала ему вечную славу.
\vs Wis 10:15 Она освободила святой народ и непорочное семя от народа угнетавших \bibemph{его},
\vs Wis 10:16 вошла в душу служителя Господня и противостала страшным царям чудесами и знамениями.
\vs Wis 10:17 Она воздала святым награду за труды их, вела их путем дивным; и днем была им покровом, а ночью~--- звездным светом.
\vs Wis 10:18 Она перевела их чрез Чермное море и провела их сквозь большую воду,
\vs Wis 10:19 а врагов их потопила и извергла их из глубины бездны.
\vs Wis 10:20 Итак, праведные завладели доспехами нечестивых и воспели святое имя Твое, Господи, и единодушно прославили поборающую руку Твою;
\vs Wis 10:21 ибо премудрость отверзла уста немых и сделала внятными языки младенцев.
\vs Wis 11:1 Она благоустроила дела их рукою святого пророка:
\vs Wis 11:2 они прошли по необитаемой пустыне, и на непроходных \bibemph{местах} поставили шатры;
\vs Wis 11:3 противостали неприятелям и отмстили врагам;
\vs Wis 11:4 томились жаждою и воззвали к Тебе, и дана им была вода из утесистой скалы и утоление жажды~--- из твердого камня.
\vs Wis 11:5 Ибо, чем наказаны были враги их,
\vs Wis 11:6 тем они, находясь в затруднении, были облагодетельствованы:
\vs Wis 11:7 вместо источника постоянно текущей реки, смрадною кровью возмущенной,
\vs Wis 11:8 в обличение их детоубийственного повеления, Ты неожиданно дал им обильную воду,
\vs Wis 11:9 показав тогда чрез жажду, как Ты наказал их противников.
\vs Wis 11:10 Ибо, когда они были испытываемы, подвергаясь, впрочем, милостивому вразумлению, тогда познали, как мучились во гневе судимые нечестивые;
\vs Wis 11:11 потому что их Ты испытывал, как отец, поучая, а тех, как гневный царь, осуждая, истязал.
\vs Wis 11:12 И отсутствовавшие и присутствовавшие одинаково пострадали:
\vs Wis 11:13 их постигла сугубая скорбь и стенание от воспоминания о прошедшем.
\vs Wis 11:14 Они, когда услышали, что чрез их наказания те были облагодетельствованы, познали Господа.
\vs Wis 11:15 Кого они прежде, как отверженного, отреклись с ругательством, Тому в последствие событий удивлялись, потерпев неодинаковую с праведными жажду.
\vs Wis 11:16 А за неразумные помышления их неправды, по которым они в заблуждении служили бессловесным пресмыкающимся и презренным чудовищам, Ты в наказание наслал на них множество бессловесных животных,
\vs Wis 11:17 чтобы они познали, что, чем кто согрешает, тем и наказывается.
\vs Wis 11:18 Не невозможно было бы для всемогущей руки Твоей, создавшей мир из необразного вещества, наслать на них множество медведей или свирепых львов,
\vs Wis 11:19 или неизвестных новосозданных лютых зверей, или дышащих огненным дыханием, или извергающих клубы дыма, или бросающих из глаз ужасные искры,
\vs Wis 11:20 которые не только повреждением могли истребить их, но и ужасающим видом погубить.
\vs Wis 11:21 Да и без этого они могли погибнуть от одного дуновения, преследуемые правосудием и рассеваемые духом силы Твоей; но Ты все расположил мерою, числом и весом.
\vs Wis 11:22 Ибо великая сила всегда присуща Тебе, и кто противостанет силе мышцы Твоей?
\vs Wis 11:23 Весь мир пред Тобою, как колебание чашки весов, или как капля утренней росы, сходящей на землю.
\vs Wis 11:24 Ты всех милуешь, потому что все можешь, и покрываешь грехи людей ради покаяния.
\vs Wis 11:25 Ты любишь все существующее, и ничем не гнушаешься, что сотворил, ибо не создал бы, если бы что ненавидел.
\vs Wis 11:26 И как могло бы пребывать что-либо, если бы Ты не восхотел? Или как сохранилось бы то, что не было призвано Тобою?
\vs Wis 11:27 Но Ты все щадишь, потому что все Твое, душелюбивый Господи.
\vs Wis 12:1 Нетленный Твой дух пребывает во всем.
\vs Wis 12:2 Посему заблуждающихся Ты мало-помалу обличаешь и, напоминая \bibemph{им}, в чем они согрешают, вразумляешь, чтобы они, отступив от зла, уверовали в Тебя, Господи.
\vs Wis 12:3 Так, возгнушавшись древними обитателями святой земли Твоей,
\vs Wis 12:4 совершавшими ненавистные дела волхвований и нечестивые жертвоприношения,
\vs Wis 12:5 и безжалостными убийцами детей, и на жертвенных пирах пожиравшими внутренности человеческой плоти и крови в тайных собраниях,
\vs Wis 12:6 и родителями, убивавшими беспомощные души,~--- Ты восхотел погубить \bibemph{их} руками отцов наших,
\vs Wis 12:7 дабы земля, драгоценнейшая всех у Тебя, приняла достойное население чад Божиих.
\vs Wis 12:8 Но и их, как людей, Ты щадил, послав предтечами воинства Твоего шершней, дабы они мало-помалу истребляли их.
\vs Wis 12:9 Хотя не невозможно было Тебе войною покорить нечестивых праведным, или истребить их страшными зверями, или грозным словом в один раз;
\vs Wis 12:10 но Ты, мало-помалу наказывая \bibemph{их}, давал место покаянию, зная, однако, что племя их негодное и зло их врожденное, и помышление их не изменится во веки.
\vs Wis 12:11 Ибо семя их было проклятое от начала, и не из опасения перед кем-либо Ты допускал безнаказанность грехов их.
\vs Wis 12:12 Ибо кто скажет: <<что Ты сделал?>> или кто противостанет суду Твоему? и кто обвинит Тебя в погублении народов, которых Ты сотворил? Или какой защитник придет к Тебе с ходатайством за неправедных людей?
\vs Wis 12:13 Ибо кроме Тебя нет Бога, который имеет попечение о всех, чтобы доказывать Тебе, что Ты несправедливо судил.
\vs Wis 12:14 Ни царь, ни властелин не в состоянии явиться к Тебе на глаза за тех, которых Ты погубил.
\vs Wis 12:15 Будучи праведен, Ты всем управляешь праведно, почитая не свойственным Твоей силе осудить того, кто не заслуживает наказания.
\vs Wis 12:16 Ибо сила Твоя есть начало правды, и то самое, что Ты господствуешь над всеми, располагает Тебя щадить всех.
\vs Wis 12:17 Силу Твою Ты показываешь не верующим всемогуществу Твоему и в не признающих Тебя обличаешь дерзость;
\vs Wis 12:18 но, обладая силою, Ты судишь снисходительно и управляешь нами с великою милостью, ибо могущество Твое всегда в Твоей воле.
\vs Wis 12:19 Но такими делами Ты поучал народ Твой, что праведному должно быть человеколюбивым, и внушал сынам Твоим благую надежду, что Ты даешь время покаянию во грехах.
\vs Wis 12:20 Ибо, если врагов сынам Твоим и повинных смерти Ты наказывал с таким снисхождением и пощадою, давая \bibemph{им} время и побуждение освободиться от зла,
\vs Wis 12:21 то с каким вниманием Ты судил сынов Твоих, которых отцам Ты дал клятвы и заветы благих обетований!
\vs Wis 12:22 Итак, вразумляя нас, Ты наказываешь врагов наших тысячекратно, дабы мы, когда судим, помышляли о Твоей благости и, когда бываем судимы, ожидали помилования.
\vs Wis 12:23 Посему-то и тех нечестивых, которые проводили жизнь в неразумии, Ты истязал собственными их мерзостями,
\vs Wis 12:24 ибо они очень далеко уклонились на путях заблуждения, обманываясь подобно неразумным детям и почитая за богов тех из животных, которые и у врагов были презренными.
\vs Wis 12:25 Посему, как неразумным детям, в посмеяние послал Ты им и наказание.
\vs Wis 12:26 Но, не вразумившись обличительным посмеянием, они испытывали заслуженный суд Божий.
\vs Wis 12:27 Ибо, что они сами терпели с досадою, то же увидев на тех, которых считали богами и чрез которых были наказываемы, они познали Бога истинного, Которого прежде отрекались знать;
\vs Wis 12:28 посему и пришло на них окончательное осуждение.
\vs Wis 13:1 Подлинно суетны по природе все люди, у которых не было ведения о Боге, которые из видимых совершенств не могли познать Сущего и, взирая на дела, не познали Виновника,
\vs Wis 13:2 а почитали за богов, правящих миром, или огонь, или ветер, или движущийся воздух, или звездный круг, или бурную воду, или небесные светила.
\vs Wis 13:3 Если, пленяясь их красотою, они почитали их за богов, то должны были бы познать, сколько лучше их Господь, ибо Он, Виновник красоты, создал их.
\vs Wis 13:4 А если удивлялись силе и действию их, то должны были бы узнать из них, сколько могущественнее Тот, Кто сотворил их;
\vs Wis 13:5 ибо от величия красоты созданий сравнительно познается Виновник бытия их.
\vs Wis 13:6 Впрочем, они меньше заслуживают порицания, ибо заблуждаются, может быть, ища Бога и желая найти Его:
\vs Wis 13:7 потому что, обращаясь к делам Его, они исследуют и убеждаются зрением, что все видимое прекрасно.
\vs Wis 13:8 Но и они неизвинительны:
\vs Wis 13:9 если они столько могли разуметь, что в состоянии были исследовать временный мир, то почему они тотчас не обрели Господа его?
\vs Wis 13:10 Но более жалки те, и надежды их~--- на бездушных, которые называют богами дела рук человеческих, золото и серебро, изделия художества, изображения животных, или негодный камень, дело давней руки.
\vs Wis 13:11 Или какой-либо древодел, вырубив годное дерево, искусно снял с него всю кору и, обделав красиво, устроил из него сосуд, полезный к употреблению в жизни,
\vs Wis 13:12 а обрезки от работы употребил на приготовление пищи и насытился;
\vs Wis 13:13 один же из обрезков, ни к чему не годный, дерево кривое и сучковатое, взяв, старательно округлил на досуге и, с опытностью знатока обделав его, уподобил его образу человека,
\vs Wis 13:14 или сделал подобным какому-нибудь низкому животному, намазал суриком и покрыл краскою поверхность его, и закрасил в нем всякий недостаток,
\vs Wis 13:15 и, устроив для него достойное его место, повесил его на стене, укрепив железом.
\vs Wis 13:16 Итак, чтобы \bibemph{произведение} его не упало, он наперед озаботился, зная, что оно само себе помочь не может, ибо это кумир и имеет нужду в помощи.
\vs Wis 13:17 Молясь же \bibemph{пред ним} о своих стяжаниях, о браке и о детях, он не стыдится говорить бездушному,
\vs Wis 13:18 и о здоровье взывает к немощному, о жизни просит мертвое, о помощи умоляет совершенно неспособное, о путешествии~--- не могущее ступить,
\vs Wis 13:19 о прибытке, о ремесле и об успехе рук~--- совсем не могущее делать руками, о силе просит самое бессильное.
\vs Wis 14:1 Еще: иной, собираясь плыть и переплывать свирепые волны, призывает на помощь дерево, слабейшее носящего его корабля;
\vs Wis 14:2 ибо стремление к приобретениям выдумало оный, а художник искусно устроил,
\vs Wis 14:3 но промысл Твой, Отец, управляет кораблем, ибо Ты дал и путь \bibemph{в море} и безопасную стезю в волнах,
\vs Wis 14:4 показывая, что Ты можешь от всего спасать, хотя бы кто отправлялся \bibemph{в море} и без искусства.
\vs Wis 14:5 Ты хочешь, чтобы не тщетны были дела Твоей премудрости; поэтому люди вверяют свою жизнь малейшему дереву и спасаются, проходя по волнам на ладье.
\vs Wis 14:6 Ибо и вначале, когда погубляемы были гордые исполины, надежда мира, управленная Твоею рукою, прибегнув к кораблю, оставила миру семя рода.
\vs Wis 14:7 Благословенно дерево, чрез которое бывает правда!
\vs Wis 14:8 А это рукотворенное проклято и само, и сделавший его~--- за то, что сделал; а это тленное названо богом.
\vs Wis 14:9 Ибо равно ненавистны Богу и нечестивец и нечестие его;
\vs Wis 14:10 и сделанное вместе со сделавшим будет наказано.
\vs Wis 14:11 Посему и на идолов языческих будет суд, так как они среди создания Божия сделались мерзостью, соблазном душ человеческих и сетью ногам неразумных.
\vs Wis 14:12 Ибо вымысл идолов~--- начало блуда, и изобретение их~--- растление жизни.
\vs Wis 14:13 Не было их вначале, и не во веки они будут.
\vs Wis 14:14 Они вошли в мир по человеческому тщеславию, и потому близкий сужден им конец.
\vs Wis 14:15 Отец, терзающийся горькою скорбью о рано умершем сыне, сделав изображение его, как уже мертвого человека, затем стал почитать его, как бога, и передал подвластным тайны и жертвоприношения.
\vs Wis 14:16 Потом утвердившийся временем этот нечестивый обычай соблюдаем был, как закон, и по повелениям властителей изваяние почитаемо было, как божество.
\vs Wis 14:17 Кого в лицо люди не могли почитать по отдаленности жительства, того отдаленное лицо они изображали: делали видимый образ почитаемого царя, дабы этим усердием польстить отсутствующему, как бы присутствующему.
\vs Wis 14:18 К усилению же почитания и от незнающих поощряло тщание художника,
\vs Wis 14:19 ибо он, желая, может быть, угодить властителю, постарался искусством сделать подобие покрасивее;
\vs Wis 14:20 а народ, увлеченный красотою отделки, незадолго пред тем почитаемого, как человека, признал теперь божеством.
\vs Wis 14:21 И это было соблазном для людей, потому что они, покоряясь или несчастью, или тиранству, несообщимое Имя прилагали к камням и деревам.
\vs Wis 14:22 Потом не довольно было для них заблуждаться в познании о Боге, но они, живя в великой борьбе невежества, такое великое зло называют миром.
\vs Wis 14:23 Совершая или детоубийственные жертвы, или скрытные тайны, или \bibemph{заимствованные} от чужих обычаев неистовые пиршества,
\vs Wis 14:24 они не берегут ни жизни, ни чистых браков, но один другого или коварством убивает, или прелюбодейством обижает.
\vs Wis 14:25 Всеми же без различия обладают кровь и убийство, хищение и коварство, растление, вероломство, мятеж, клятвопреступление, расхищение имуществ,
\vs Wis 14:26 забвение благодарности, осквернение душ, превращение полов, бесчиние браков, прелюбодеяние и распутство.
\vs Wis 14:27 Служение идолам, недостойным именования, есть начало и причина, и конец всякого зла,
\vs Wis 14:28 ибо они или веселясь неистовствуют, или прорицают ложь, или живут беззаконно, или скоро нарушают клятву.
\vs Wis 14:29 Надеясь на бездушных идолов, они не думают быть наказанными за то, что несправедливо клянутся.
\vs Wis 14:30 Но за то и другое придет на них осуждение, \bibemph{и за то}, что нечестиво мыслили о Боге, обращаясь к идолам, и \bibemph{за то}, что ложно клялись, коварно презирая святое.
\vs Wis 14:31 Ибо не сила тех, которыми они клянутся, но суд над согрешающими следует всегда за преступлением неправедных.
\vs Wis 15:1 Но Ты, Бог наш, благ и истинен, долготерпелив и управляешь всем милостиво.
\vs Wis 15:2 Если мы и согрешаем, мы~--- Твои, признающие власть Твою; но мы не будем грешить, зная, что мы признаны Твоими.
\vs Wis 15:3 Знать Тебя есть полная праведность, и признавать власть Твою~--- корень бессмертия.
\vs Wis 15:4 Не обольщает нас лукавое человеческое изобретение, ни бесплодный труд художников~--- изображения, испещренные различными красками,
\vs Wis 15:5 взгляд на которые возбуждает в безумных похотение и вожделение к бездушному виду мертвого образа.
\vs Wis 15:6 И делающие, и похотствующие, и чествующие суть любители зла, достойные таких надежд.
\vs Wis 15:7 Горшечник мнет мягкую землю, заботливо лепит всякий \bibemph{сосуд} на службу нашу; из одной и той же глины выделывает сосуды, потребные и для чистых дел и для нечистых~--- все одинаково; но какое каждого из них употребление, судья~--- тот же горшечник.
\vs Wis 15:8 И суетный труженик из той же глины лепит суетного бога, тогда как сам недавно родился из земли и вскоре пойдет туда же, откуда он взят, и взыщется с него долг души его.
\vs Wis 15:9 Но у него забота не о том, что он должен много трудиться, и не о том, что жизнь его кратка; но он соревнует художникам золотых и серебряных изделий, и подражает медникам, и вменяет себе в славу, что делает мерзости.
\vs Wis 15:10 Сердце его~--- пепел, и надежда его ничтожнее земли, и жизнь его презреннее грязи;
\vs Wis 15:11 ибо он не познал Сотворившего его и вдунувшего в него деятельную душу и вдохнувшего в него дух жизни.
\vs Wis 15:12 Они считают жизнь нашу забавою и житие прибыльною торговлею, ибо говорят, что должно же откуда-либо извлекать прибыль, хотя бы и из зла.
\vs Wis 15:13 Впрочем такой более всех знает, что он грешит, делая из земляного вещества бренные сосуды и изваяния.
\vs Wis 15:14 Самые же неразумные из всех и беднее умом самых младенцев~--- враги народа Твоего, угнетающие его,
\vs Wis 15:15 потому что они почитают богами всех идолов языческих, у которых нет употребления ни глаз для зрения, ни ноздрей для привлечения воздуха, ни ушей для слышания, ни перстов рук для осязания и которых ноги негодны для хождения.
\vs Wis 15:16 Хотя человек сделал их, и заимствовавший дух образовал их, но никакой человек не может образовать бога, как он сам.
\vs Wis 15:17 Будучи смертным, он делает нечестивыми руками мертвое, поэтому он превосходнее божеств своих, ибо он жил, а те~--- никогда.
\vs Wis 15:18 Притом они почитают животных самых отвратительных, которые по бессмыслию сравнительно хуже всех.
\vs Wis 15:19 Они даже некрасивы по виду, как \bibemph{другие} животные, чтобы могли привлекать к себе, но лишены и одобрения Божия и благословения Его.
\vs Wis 16:1 Посему они достойно были наказаны чрез подобных \bibemph{животных} и терзаемы множеством чудовищ.
\vs Wis 16:2 Вместо такого наказания Ты благодетельствовал народу Твоему: в удовлетворение прихоти их Ты приготовил им в насыщение необычайную пищу~--- перепелов,
\vs Wis 16:3 дабы те, мучимые голодом, по отвратительному виду насланных \bibemph{гадов}, отказывали и необходимому позыву на пищу, а эти, кратковременно потерпев недостаток, вкусили необычайной пищи.
\vs Wis 16:4 Ибо тех притеснителей должен был постигнуть неотвратимый недостаток, а этим только нужно было показать, как мучились враги их.
\vs Wis 16:5 И тогда, как постигла их ужасная ярость зверей и они были истребляемы угрызениями коварных змиев, гнев Твой не продолжился до конца.
\vs Wis 16:6 Но они были смущены на краткое время для вразумления, получив знамение спасения на воспоминание о заповеди закона Твоего,
\vs Wis 16:7 ибо обращавшийся исцелялся не тем, на что взирал, но Тобою, Спасителем всех.
\vs Wis 16:8 И этим Ты показал врагам нашим, что Ты~--- избавляющий от всякого зла:
\vs Wis 16:9 ибо их убивали уязвления саранчи и мух, и не нашлось врачевства для души их, потому что они достойны были мучения от сих.
\vs Wis 16:10 А сынов Твоих не одолели и зубы ядовитых змиев, ибо милость Твоя пришла на помощь и исцелила их.
\vs Wis 16:11 Хотя они и были уязвляемы в напоминание им слов Твоих, но скоро были и исцеляемы, дабы, впав в глубокое забвение \bibemph{оных}, не лишились Твоего благодеяния.
\vs Wis 16:12 Не трава и не пластырь врачевали их, но Твое, Господи, всеисцеляющее слово.
\vs Wis 16:13 Ты имеешь власть жизни и смерти и низводишь до врат ада и возводишь.
\vs Wis 16:14 Человек по злобе своей убивает, но не может возвратить исшедшего духа и не может призвать взятой души.
\vs Wis 16:15 А Твоей руки невозможно избежать,
\vs Wis 16:16 ибо нечестивые, отрекшиеся познать Тебя, наказаны силою мышцы Твоей, быв преследуемы необыкновенными дождями, градами и неотвратимыми бурями и истребляемы огнем.
\vs Wis 16:17 Но самое чудное было то, что огонь сильнее оказывал действие в воде, все погашающей, ибо самый мир есть поборник за праведных.
\vs Wis 16:18 Иногда пламя укрощалось, чтобы не сжечь животных, посланных на нечестивых, и чтобы они, видя это, познали, что преследуются судом Божиим.
\vs Wis 16:19 А иногда и среди воды жгло сильнее огня, дабы истребить произведения земли неправедной.
\vs Wis 16:20 Вместо того народ Твой Ты питал пищею ангельскою и послал им, нетрудящимся, с неба готовый хлеб, имевший всякую приятность по вкусу каждого.
\vs Wis 16:21 Ибо свойство пищи Твоей показывало Твою любовь к детям и в удовлетворение желания вкушающего изменялось по вкусу каждого.
\vs Wis 16:22 А снег и лед выдерживали огонь и не таяли, дабы они знали, что огонь, горящий в граде и блистающий в дождях, истреблял плоды врагов.
\vs Wis 16:23 Но тот же огонь, дабы напитались праведные, терял свою силу.
\vs Wis 16:24 Ибо тварь, служа Тебе, Творцу, устремляется к наказанию нечестивых и утихает для благодеяния верующим в Тебя.
\vs Wis 16:25 Посему и тогда она, изменяясь во всё, повиновалась Твоей благодати, питающей всех, по желанию нуждающихся,
\vs Wis 16:26 дабы сыны Твои, которых Ты, Господи, возлюбил, познали, что не роды плодов питают человека, но слово Твое сохраняет верующих в Тебя.
\vs Wis 16:27 Ибо неповреждаемое огнем, будучи согреваемо слабым солнечным лучом, тотчас растаявало,
\vs Wis 16:28 дабы известно было, что должно предупреждать солнце благодарением Тебе и обращаться к Тебе на восток света.
\vs Wis 16:29 Ибо надежда неблагодарного растает, как зимний иней, и выльется, как негодная вода.
\vs Wis 17:1 Велики и непостижимы суды Твои, посему ненаученные души впали в заблуждение.
\vs Wis 17:2 Ибо беззаконные, которые задумали угнетать святой народ, узники тьмы и пленники долгой ночи, затворившись в домах, скрывались от вечного Промысла.
\vs Wis 17:3 Думая укрыться в тайных грехах, они, под темным покровом забвения, рассеялись, сильно устрашаемые и смущаемые призраками,
\vs Wis 17:4 ибо и самое потаенное место, заключавшее их, не спасало их от страха, но страшные звуки вокруг них приводили их в смущение, и являлись свирепые чудовища со страшными лицами.
\vs Wis 17:5 И никакая сила огня не могла озарить, ни яркий блеск звезд не в состоянии был осветить этой мрачной ночи.
\vs Wis 17:6 Являлись им только сами собою горящие костры, полные ужаса, и они, страшась невидимого~--- призрака, представляли себе видимое еще худшим.
\vs Wis 17:7 Пали обольщения волшебного искусства, и хвастовство мудростью подверглось посмеянию,
\vs Wis 17:8 ибо обещавшиеся отогнать от страдавшей души ужасы и страхи, сами страдали позорною боязливостью.
\vs Wis 17:9 И хотя никакие устрашения не тревожили их, но, преследуемые брожениями ядовитых зверей и свистами пресмыкающихся, они исчезали от страха, боясь взглянуть даже на воздух, от которого никуда нельзя убежать,
\vs Wis 17:10 ибо осуждаемое собственным свидетельством нечестие боязливо и, преследуемое совестью, всегда придумывает ужасы.
\vs Wis 17:11 Страх есть не что иное, как лишение помощи от рассудка.
\vs Wis 17:12 Чем меньше надежды внутри, тем больше представляется неизвестность причины, производящей мучение.
\vs Wis 17:13 И они в эту истинно невыносимую и из глубин нестерпимого ада исшедшую ночь, располагаясь заснуть обыкновенным сном,
\vs Wis 17:14 то были тревожимы страшными призраками, то расслабляемы душевным унынием, ибо находил на них внезапный и неожиданный страх.
\vs Wis 17:15 Итак, где кто тогда был застигнут, делался пленником и заключаем был в эту темницу без оков.
\vs Wis 17:16 Был ли то земледелец или пастух, или занимающийся работами в пустыне, всякий, быв застигнут, подвергался этой неизбежной судьбе,
\vs Wis 17:17 ибо все были связаны одними неразрешимыми узами тьмы. Свищущий ли ветер, или среди густых ветвей сладкозвучный голос птиц, или сила быстро текущей воды, или сильный треск низвергающихся камней,
\vs Wis 17:18 или незримое бегание скачущих животных, или голос ревущих свирепейших зверей, или отдающееся из горных углублений эхо, \bibemph{все это}, ужасая их, повергало в расслабление.
\vs Wis 17:19 Ибо весь мир был освещаем ясным светом и занимался беспрепятственно делами;
\vs Wis 17:20 а над ними одними была распростерта тяжелая ночь, образ тьмы, имевшей некогда объять их; но сами для себя они были тягостнее тьмы.
\vs Wis 18:1 А для святых Твоих был величайший свет. И те, слыша голос их, а образа не видя, называли их блаженными, потому что они не страдали.
\vs Wis 18:2 А за то, что, быв прежде обижаемы ими, не мстили им, благодарили и просили прощения в том, что заставляли переносить их.
\vs Wis 18:3 Вместо того, Ты дал им указателем на незнакомом пути огнесветлый столп, а для благополучного странствования~--- безвредное солнце.
\vs Wis 18:4 Ибо те достойны были лишения света и заключения во тьме, потому что держали в заключении сынов Твоих, чрез которых имел быть дан миру нетленный свет закона.
\vs Wis 18:5 Когда определили они избить детей святых, хотя одного сына покинутого и спасли, в наказание за то Ты отнял множество их детей и самих всех погубил в сильной воде.
\vs Wis 18:6 Та ночь была предвозвещена отцам нашим, дабы они, твердо зная обетования, каким верили, были благодушны.
\vs Wis 18:7 И народ Твой ожидал как спасения праведных, так и погибели врагов,
\vs Wis 18:8 ибо, чем Ты наказывал врагов, тем самым возвеличил нас, которых Ты призвал.
\vs Wis 18:9 Святые дети добрых тайно совершали жертвоприношение и единомысленно постановили божественным законом, чтобы святые равно участвовали в одних и тех же благах и опасностях, когда отцы уже воспевали хвалы.
\vs Wis 18:10 С противной же стороны отдавался нестройный крик врагов, и разносился жалобный вопль над оплакиваемыми детьми.
\vs Wis 18:11 Одинаковым судом был наказан раб с господином, и простолюдин терпел одно и то же с царем:
\vs Wis 18:12 все вообще имели бесчисленных мертвецов, \bibemph{умерших} одинаковою смертью; и живых недоставало для погребения, так как в одно мгновение погублено было \bibemph{все} драгоценнейшее их поколение.
\vs Wis 18:13 И не верившие ничему ради чародейства, при погублении первенцев, признали, что \bibemph{этот} народ есть сын Божий,
\vs Wis 18:14 ибо, когда все окружало тихое безмолвие и ночь в своем течении достигла средины,
\vs Wis 18:15 сошло с небес от царственных престолов на средину погибельной земли всемогущее слово Твое, как грозный воин.
\vs Wis 18:16 Оно несло острый меч~--- неизменное Твое повеление и, став, наполнило все смертью: оно касалось неба и ходило по земле.
\vs Wis 18:17 Тогда вдруг сильно встревожили их мечты сновидений, и наступили неожиданные ужасы;
\vs Wis 18:18 и, будучи поражаем~--- один там, другой тут, полумертвый объявлял причину, по которой он умирал:
\vs Wis 18:19 ибо встревожившие их сновидения предварительно показали \bibemph{им} это, чтобы они не погибли, не зная того, за что терпят зло.
\vs Wis 18:20 Хотя искушение смерти коснулось и праведных, и много их погибло в пустыне, но недолго продолжался этот гнев,
\vs Wis 18:21 ибо непорочный муж поспешил защитить их; принеся оружие своего служения, молитву и умилостивление кадильное, он противостал гневу и положил конец бедствию, показав тем, что он слуга Твой.
\vs Wis 18:22 Он победил истребителя не силою телесною и не действием оружия, но словом покорил наказывавшего, воспомянув клятвы и заветы отцов.
\vs Wis 18:23 Ибо, когда уже грудами лежали мертвые одни на других, он, став в средине, остановил гнев и пресек \bibemph{ему} путь к живым.
\vs Wis 18:24 На подире его был целый мир, и славные \bibemph{имена} отцов были вырезаны на камнях в четыре ряда, и величие Твое~--- на диадиме головы его.
\vs Wis 18:25 Этому уступил истребитель, и этого убоялся: ибо довольно было одного этого испытания гневного.
\vs Wis 19:1 А над нечестивыми до конца тяготел немилостивый гнев, ибо Он предвидел и будущие их \bibemph{дела},
\vs Wis 19:2 что они, позволив им отправиться и с поспешностью выслав их, раскаются и погонятся за ними,
\vs Wis 19:3 ибо, еще имея в руках печали и рыдая над гробами мертвых, они возымели другой безумный помысл, и тех, кого с мольбою высылали, преследовали, как беглецов.
\vs Wis 19:4 Влекла же их к тому концу судьба, которой они были достойны, и она навела забвение о случившемся, дабы они восполнили наказание, недостававшее к их мучениям,
\vs Wis 19:5 и дабы народ Твой совершил славное путешествие, а они нашли себе необычайную смерть.
\vs Wis 19:6 Ибо вся тварь снова свыше преобразовалась в своей природе, повинуясь особым повелениям, дабы сыны Твои сохранились невредимыми.
\vs Wis 19:7 Явилось облако, осеняющее стан, а где стояла прежде вода, показалась сухая земля, из Чермного моря~--- беспрепятственный путь, и из бурной пучины~--- зеленая долина.
\vs Wis 19:8 Покрываемые Твоею рукою, они прошли по ней всем народом, видя дивные чудеса.
\vs Wis 19:9 Они паслись как кони и играли как агнцы, славя Тебя, Господи, Избавителя их,
\vs Wis 19:10 ибо они еще помнили о том, что случилось во время пребывания их там, как земля вместо рождения \bibemph{других} животных произвела скнипов и река вместо рыб извергла множество жаб.
\vs Wis 19:11 А после они увидели и новый род птиц, когда, увлекшись пожеланием, просили приятной пищи,
\vs Wis 19:12 ибо в утешение им налетели с моря перепелы, а грешных постигли наказания не без знамений, бывших силою молний. Они справедливо страдали за свою злобу,
\vs Wis 19:13 ибо они более сильную питали ненависть к чужеземцам: иные не принимали незнаемых странников, а эти порабощали благодетельных пришельцев.
\vs Wis 19:14 И мало этого, но еще будет суд на них за то, что те враждебно принимали чужих,
\vs Wis 19:15 а эти, с радостью приняв, потом уже пользовавшихся одинаковыми правами стали угнетать ужасными работами.
\vs Wis 19:16 Посему они поражены были слепотою, как те \bibemph{некогда} при дверях праведника, когда, будучи объяты густою тьмою, искали каждый входа в его двери.
\vs Wis 19:17 Самые стихии изменились, как в арфе звуки изменяют свой характер, всегда оставаясь теми же звуками; это можно усмотреть чрез тщательное наблюдение бывшего.
\vs Wis 19:18 Ибо земные \bibemph{животные} переменялись в водяные, а плавающие в водах выходили на землю.
\vs Wis 19:19 Огонь в воде удерживал свою силу, а вода теряла угашающее свое свойство;
\vs Wis 19:20 пламя, наоборот, не вредило телам бродящих удоборазрушимых животных, и не таял легко растаявающий снеговидный род небесной пищи.
\vs Wis 19:21 Так, Господи, Ты во всем возвеличил и прославил народ Твой, и не оставлял его, но во всякое время и на всяком месте пребывал с ним.

\bibbookdescr{Sir}{
  inline={\LARGE Книга\\\Huge Премудрости Иисуса,\\сына Сирахова\fns{Переведена с греческого.}},
  toc={Сирах*},
  bookmark={Сирах},
  header={Сирах},
  %headerleft={},
  %headerright={},
  abbr={Сир}
}
\chhdr{Предисловие\fns{Предисловие к греческому переводу, имеющееся у 70-ти и содержащееся в Славянской Библии.}}
\vs Sir 0:0 Многое и великое дано нам через закон, пророков и прочих \bibemph{писателей}, следовавших за ними, за что должно прославлять \bibemph{народ} Израильский за образованность и мудрость; и не только сами изучающие должны делаться разумными, но и находящимся вне [Палестины] усердно занимающиеся [писанием] могут приносить пользу словом и писанием. Поэтому дед мой Иисус, больше других предаваясь изучению закона, пророков и других отеческих книг и приобретя достаточный в них навык, решился и сам написать нечто, относящееся к образованию и мудрости, чтобы любители учения, вникая и в эту [книгу], еще более преуспевали в жизни по закону. Итак, прошу вас, читайте [эту книгу] благосклонно и внимательно и имейте снисхождение к тому, что в некоторых местах мы, может быть, погрешили, трудясь над переводом: ибо неодинаковый смысл имеет то, что читается по-еврейски, когда переведено будет на другой язык,~--- и не только эта [книга], но даже закон, пророчества и остальные книги имеют немалую разницу в смысле, если читать их в подлиннике. Прибыв в Египет в тридцать восьмом году при царе Евергете [Птоломее] и пробыв там, я нашел немалую разницу в образовании [между палестинскими и египетскими евреями], и счел крайне необходимым и самому приложить усердие к тому, чтобы перевести эту книгу. Много бессонного труда и знаний положил я в это время, чтобы довести книгу до конца и сделать ее доступною и тем, которые, находясь на чужбине, желают учиться и приспособляют свои нравы к тому, чтобы жить по закону.
\vs Sir 1:1 Всякая премудрость~--- от Господа и с Ним пребывает вовек.
\vs Sir 1:2 Песок морей и капли дождя и дни вечности кто исчислит?
\vs Sir 1:3 Высоту неба и широту земли, и бездну и премудрость кто исследует?
\vs Sir 1:4 Прежде всего произошла Премудрость, и разумение мудрости~--- от века.
\vs Sir 1:5 Источник премудрости~--- слово Бога Всевышнего, и шествие ее~--- вечные заповеди.
\vs Sir 1:6 Кому открыт корень премудрости? и кто познал искусство ее?
\vs Sir 1:7 Один есть премудрый, весьма страшный, сидящий на престоле Своем, Господь.
\vs Sir 1:8 Он произвел ее и видел и измерил ее
\vs Sir 1:9 и излил ее на все дела Свои
\vs Sir 1:10 и на всякую плоть по дару Своему, и особенно наделил ею любящих Его.
\vs Sir 1:11 Страх Господень~--- слава и честь, и веселие и венец радости.
\vs Sir 1:12 Страх Господень усладит сердце и даст веселие и радость и долгоденствие.
\vs Sir 1:13 Боящемуся Господа благо будет напоследок, и в день смерти своей он получит благословение. Страх Господень~--- дар от Господа и поставляет на стезях любви.
\vs Sir 1:14 Любовь к Господу~--- славная премудрость, и кому благоволит Он, разделяет ее по Своему усмотрению.
\vs Sir 1:15 Начало премудрости~--- бояться Бога, и с верными она образуется вместе во чреве. Среди людей она утвердила себе вечное основание и семени их вверится.
\vs Sir 1:16 Полнота премудрости~--- бояться Господа; она напояет их от плодов своих:
\vs Sir 1:17 весь дом их она наполнит всем, чего желают, и кладовые их~--- произведениями своими.
\vs Sir 1:18 Венец премудрости~--- страх Господень, произращающий мир и невредимое здравие; но то и другое~--- дары Бога, Который распространяет славу любящих Его.
\vs Sir 1:19 Он видел ее и измерил, пролил как дождь в\acc{е}дение и разумное знание и возвысил славу обладающих ею.
\vs Sir 1:20 Корень премудрости~--- бояться Господа, а ветви ее~--- долгоденствие.
\rsbpar\vs Sir 1:21 Страх Господень отгоняет грехи; не имеющий же страха не может оправдаться.
\vs Sir 1:22 Не может быть оправдан несправедливый гнев, ибо \bibemph{самое} движение гнева есть падение для человека.
\vs Sir 1:23 Терпеливый до времени удержится и после вознаграждается веселием.
\vs Sir 1:24 До времени он скроет слова свои, и уста верных расскажут о благоразумии его.
\vs Sir 1:25 В сокровищницах премудрости~--- притчи разума, грешнику же страх Господень ненавистен.
\vs Sir 1:26 Если желаешь премудрости, соблюдай заповеди, и Господь подаст ее тебе,
\vs Sir 1:27 ибо премудрость и знание есть страх пред Господом, и благоугождение Ему~--- вера и кротость.
\vs Sir 1:28 Не будь недоверчивым к страху пред Господом и не приступай к Нему с раздвоенным сердцем.
\vs Sir 1:29 Не лицемерь пред устами других и будь внимателен к устам твоим.
\vs Sir 1:30 Не возноси себя, чтобы не упасть и не навлечь бесчестия на душу твою, ибо Господь откроет тайны твои и уничижит тебя среди собрания за то, что ты не приступил искренно к страху Господню, и сердце твое полно лукавства.
\vs Sir 2:1 Сын мой! если ты приступаешь служить Господу Богу, то приготовь душу твою к искушению:
\vs Sir 2:2 управь сердце твое и будь тверд, и не смущайся во время посещения;
\vs Sir 2:3 прилепись к Нему и не отступай, дабы возвеличиться тебе напоследок.
\vs Sir 2:4 Все, что ни приключится тебе, принимай охотно, и в превратностях твоего уничижения будь долготерпелив,
\vs Sir 2:5 ибо золото испытывается в огне, а люди, угодные Богу,~--- в горниле уничижения.
\vs Sir 2:6 Веруй Ему, и Он защитит тебя; управь пути твои и надейся на Него.
\vs Sir 2:7 Боящиеся Господа! ожидайте милости Его и не уклоняйтесь \bibemph{от Него}, чтобы не упасть.
\vs Sir 2:8 Боящиеся Господа! веруйте Ему, и не погибнет награда ваша.
\vs Sir 2:9 Боящиеся Господа! надейтесь на благое, на радость вечную и милости.
\vs Sir 2:10 Взгляните на древние роды и посмотрите: кто верил Господу~--- и был постыжен? или кто пребывал в страхе Его~--- и был оставлен? или кто взывал к Нему, и Он презрел его?
\vs Sir 2:11 Ибо Господь сострадателен и милостив и прощает грехи, и спасает во время скорби.
\vs Sir 2:12 Горе сердцам боязливым и рукам ослабленным и грешнику, ходящему по двум стезям!
\vs Sir 2:13 Горе сердцу расслабленному! ибо оно не верует, и за то не будет защищено.
\vs Sir 2:14 Горе вам, потерявшим терпение! что будете вы делать, когда Господь посетит?
\vs Sir 2:15 Боящиеся Господа не будут недоверчивы к словам Его, и любящие Его сохранят пути Его.
\vs Sir 2:16 Боящиеся Господа будут искать благоволения Его, и любящие Его насытятся законом.
\vs Sir 2:17 Боящиеся Господа уготовят сердца свои и смирят пред Ним души свои, говоря:
\vs Sir 2:18 впадем в руки Господа, а не в руки людей; ибо, каково величие Его, такова и милость Его.
\vs Sir 3:1 Дети, послушайте меня, отца, и поступайте так, чтобы вам спастись,
\vs Sir 3:2 ибо Господь возвысил отца над детьми и утвердил суд матери над сыновьями.
\vs Sir 3:3 Почитающий отца очистится от грехов,
\vs Sir 3:4 и уважающий мать свою~--- как приобретающий сокровища.
\vs Sir 3:5 Почитающий отца будет иметь радость от детей своих и в день молитвы своей будет услышан.
\vs Sir 3:6 Уважающий отца будет долгоденствовать, и послушный Господу успокоит мать свою.
\vs Sir 3:7 Боящийся Господа почтит отца и, как владыкам, послужит родившим его.
\vs Sir 3:8 Делом и словом почитай отца твоего и мать, чтобы пришло на тебя благословение от них,
\vs Sir 3:9 ибо благословение отца утверждает домы детей, а клятва матери разрушает до основания.
\vs Sir 3:10 Не ищи славы в бесчестии отца твоего, ибо не слава тебе бесчестие отца.
\vs Sir 3:11 Слава человека~--- от чести отца его, и позор детям~--- мать в бесславии.
\vs Sir 3:12 Сын! прими отца твоего в старости \bibemph{его} и не огорчай его в жизни его.
\vs Sir 3:13 Хотя бы он и оскудел разумом, имей снисхождение и не пренебрегай им при полноте силы твоей,
\vs Sir 3:14 ибо милосердие к отцу не будет забыто; несмотря на грехи твои, благосостояние твое умножится.
\vs Sir 3:15 В день скорби твоей воспомянется о тебе: как лед от теплоты, разрешатся грехи твои.
\vs Sir 3:16 Оставляющий отца~--- то же, что богохульник, и проклят от Господа раздражающий мать свою.
\rsbpar\vs Sir 3:17 Сын мой! веди дела твои с кротостью, и будешь любим богоугодным человеком.
\vs Sir 3:18 Сколько ты велик, столько смиряйся, и найдешь благодать у Господа.
\vs Sir 3:19 Много высоких и славных, но тайны открываются смиренным,
\vs Sir 3:20 ибо велико могущество Господа, и Он смиренными прославляется.
\vs Sir 3:21 Чрез меру трудного для тебя не ищи, и, что свыше сил твоих, того не испытывай.
\vs Sir 3:22 Что заповедано тебе, о том размышляй; ибо не нужно тебе, что сокрыто.
\vs Sir 3:23 При многих занятиях твоих, о лишнем не заботься: тебе открыто очень много из человеческого знания;
\vs Sir 3:24 ибо многих ввели в заблуждение их предположения, и лукавые мечты поколебали ум их.
\vs Sir 3:25 Кто любит опасность, тот впадет в нее;
\vs Sir 3:26 упорное сердце напоследок потерпит зло:
\vs Sir 3:27 упорное сердце будет обременено скорбями, и грешник приложит грехи ко грехам.
\vs Sir 3:28 Испытания не служат врачевством для гордого, потому что злое растение укоренилось в нем.
\vs Sir 3:29 Сердце разумного обдумает притчу, и внимательное ухо есть желание мудрого.
\vs Sir 3:30 Вода угасит пламень огня, и милостыня очистит грехи.
\vs Sir 3:31 Кто воздает за благодеяния, тот помышляет о будущем и во время падения найдет опору.
\vs Sir 4:1 Сын мой! не отказывай в пропитании нищему и не утомляй ожиданием очей нуждающихся;
\vs Sir 4:2 не опечаль души алчущей и не огорчай человека в его скудости;
\vs Sir 4:3 не смущай сердца уже огорченного и не откладывай подавать нуждающемуся;
\vs Sir 4:4 не отказывай угнетенному, умоляющему о помощи, и не отвращай лица твоего от нищего;
\vs Sir 4:5 не отвращай очей от просящего и не давай человеку повода проклинать тебя;
\vs Sir 4:6 ибо, когда он в горести души своей будет проклинать тебя, Сотворивший его услышит моление его.
\vs Sir 4:7 В собрании старайся быть приятным и пред высшим наклоняй твою голову;
\vs Sir 4:8 приклоняй ухо твое к нищему и отвечай ему ласково, с кротостью;
\vs Sir 4:9 спасай обижаемого от руки обижающего и не будь малодушен, когда судишь;
\vs Sir 4:10 сиротам будь как отец и матери их~--- вместо мужа:
\vs Sir 4:11 и будешь как сын Вышнего, и Он возлюбит тебя более, нежели мать твоя.
\rsbpar\vs Sir 4:12 Премудрость возвышает сынов своих и поддерживает ищущих ее:
\vs Sir 4:13 любящий ее любит жизнь, и ищущие ее с раннего утра исполнятся радости:
\vs Sir 4:14 обладающий ею наследует славу, и, куда бы ни пошел, Господь благословит его;
\vs Sir 4:15 служащие ей служат Святому, и любящих ее любит Господь;
\vs Sir 4:16 послушный ей будет судить народы, и внимающий ей будет жить надежно;
\vs Sir 4:17 кто вверится ей, тот наследует ее, и потомки его будут обладать ею:
\vs Sir 4:18 ибо сначала она пойдет с ним путями извилистыми, наведет на него страх и боязнь
\vs Sir 4:19 и будет мучить его своим водительством, доколе не уверится в душе его и не искусит его своими уставами;
\vs Sir 4:20 но потом она выйдет к нему на прямом пути и обрадует его
\vs Sir 4:21 и откроет ему тайны свои.
\vs Sir 4:22 Если он совратится с пути, она оставляет его и отдает его в руки падения его.
\rsbpar\vs Sir 4:23 Наблюдай время и храни себя от зла~---
\vs Sir 4:24 и не постыдишься за душу твою:
\vs Sir 4:25 есть стыд, ведущий ко греху, и есть стыд~--- слава и благодать.
\vs Sir 4:26 Не будь лицеприятен против души твоей и не стыдись ко вреду твоему.
\vs Sir 4:27 Не удерживай слова, когда оно может помочь:
\vs Sir 4:28 ибо в слове познается мудрость и в речи языка~--- знание.
\vs Sir 4:29 Не противоречь истине и стыдись твоего невежества.
\vs Sir 4:30 Не стыдись исповедовать грехи твои и не удерживай течения реки.
\vs Sir 4:31 Не подчиняйся человеку глупому и не смотри на сильного.
\vs Sir 4:32 Подвизайся за истину до смерти, и Господь Бог поборет за тебя.
\vs Sir 4:33 Не будь скор языком твоим и ленив и нерадив в делах твоих.
\vs Sir 4:34 Не будь, как лев, в доме твоем и подозрителен к домочадцам твоим.
\vs Sir 4:35 Да не будет рука твоя распростертою к принятию и сжатою при отдании.
\vs Sir 5:1 Не полагайся на имущества твои и не говори: <<станет на жизнь мою>>.
\vs Sir 5:2 Не следуй влечению души твоей и крепости твоей, чтобы ходить в похотях сердца твоего,
\vs Sir 5:3 и не говори: <<кто властен в делах моих?>>, ибо Господь непременно отмстит за дерзость твою.
\vs Sir 5:4 Не говори: <<я грешил, и что мне было?>>, ибо Господь долготерпелив.
\vs Sir 5:5 При мысли об умилостивлении не будь бесстрашен, чтобы прилагать грех ко грехам
\vs Sir 5:6 и не говори: <<милосердие Его велико, Он простит множество грехов моих>>;
\vs Sir 5:7 ибо милосердие и гнев у Него, и на грешниках пребывает ярость Его.
\vs Sir 5:8 Не медли обратиться к Господу и не откладывай со дня на день:
\vs Sir 5:9 ибо внезапно найдет гнев Господа, и ты погибнешь во время отмщения.
\vs Sir 5:10 Не полагайся на имущества неправедные, ибо они не принесут тебе пользы в день посещения.
\vs Sir 5:11 Не вей при всяком ветре и не ходи всякою стезею: таков двоязычный грешник.
\vs Sir 5:12 Будь тверд в твоем убеждении, и одно да будет твое слово.
\vs Sir 5:13 Будь скор к слушанию, и обдуманно давай ответ.
\vs Sir 5:14 Если имеешь знание, то отвечай ближнему, а если нет, то рука твоя да будет на устах твоих.
\vs Sir 5:15 В речах~--- слава и бесчестие, и язык человека бывает падением ему.
\vs Sir 5:16 Не прослыви наушником, и не коварствуй языком твоим:
\vs Sir 5:17 ибо на воре~--- стыд, и на двоязычном~--- злое порицание.
\vs Sir 5:18 Не будь неразумным ни в большом ни в малом.
\vs Sir 6:1 И не делайся врагом из друга, ибо худое имя получает в удел стыд и позор; так~--- и грешник двоязычный.
\vs Sir 6:2 Не возноси себя в помыслах души твоей, чтобы душа твоя не была растерзана, как вол:
\vs Sir 6:3 листья твои ты истребишь и плоды твои погубишь, и останешься, как сухое дерево.
\vs Sir 6:4 Душа лукавая погубит своего обладателя и сделает его посмешищем врагов.
\rsbpar\vs Sir 6:5 Сладкие уста умножат друзей, и доброречивый язык умножит приязнь.
\vs Sir 6:6 Живущих с тобою в мире да будет много, а советником твоим~--- один из тысячи.
\vs Sir 6:7 Если хочешь приобрести друга, приобретай его по испытании и не скоро вверяйся ему.
\vs Sir 6:8 Бывает друг в нужное для него время, и не останется с тобой в день скорби твоей;
\vs Sir 6:9 и бывает друг, который превращается во врага и откроет ссору к поношению твоему.
\vs Sir 6:10 Бывает другом участник в трапезе, и не останется с тобою в день скорби твоей.
\vs Sir 6:11 В имении твоем он будет как ты, и дерзко будет обращаться с домочадцами твоими;
\vs Sir 6:12 но если ты будешь унижен, он будет против тебя и скроется от лица твоего.
\vs Sir 6:13 Отдаляйся от врагов твоих и будь осмотрителен с друзьями твоими.
\vs Sir 6:14 Верный друг~--- крепкая защита: кто нашел его, нашел сокровище.
\vs Sir 6:15 Верному другу нет цены, и нет меры доброте его.
\vs Sir 6:16 Верный друг~--- врачевство для жизни, и боящиеся Господа найдут его.
\vs Sir 6:17 Боящийся Господа направляет дружбу свою так, что, каков он сам, таким делается и друг его.
\rsbpar\vs Sir 6:18 Сын мой! от юности твоей предайся учению, и до седин твоих найдешь мудрость.
\vs Sir 6:19 Приступай к ней как пашущий и сеющий и ожидай добрых плодов ее:
\vs Sir 6:20 ибо малое время потрудишься в возделывании ее, и скоро будешь есть плоды ее.
\vs Sir 6:21 Для невежд она очень сурова, и неразумный не останется с нею:
\vs Sir 6:22 она будет на нем как тяжелый камень испытания, и он не замедлит сбросить ее.
\vs Sir 6:23 Премудрость соответствует имени своему, и немногим открывается.
\vs Sir 6:24 Послушай, сын мой, и прими мнение мое, и не отвергни совета моего.
\vs Sir 6:25 Наложи на ноги твои путы ее и на шею твою цепь ее.
\vs Sir 6:26 Подставь ей плечо твое, и носи ее и не тяготись узами ее.
\vs Sir 6:27 Приблизься к ней всею душею твоею, и всею силою твоею соблюдай пути ее.
\vs Sir 6:28 Исследуй и ищи, и она будет познана тобою и, сделавшись обладателем ее, не покидай ее;
\vs Sir 6:29 ибо наконец ты найдешь в ней успокоение, и она обратится в радость тебе.
\vs Sir 6:30 Путы ее будут тебе крепкою защитою, и цепи ее~--- славным одеянием;
\vs Sir 6:31 ибо на ней украшение золотое, и узы ее~--- гиацинтовые нити.
\vs Sir 6:32 Как одеждою славы ты облечешься ею, и возложишь ее на себя как венец радости.
\vs Sir 6:33 Сын мой! если ты пожелаешь ее, то научишься, и если предашься ей душею твоею, то будешь ко всему способен.
\vs Sir 6:34 Если с любовью будешь слушать \bibemph{ее}, то поймешь ее, и если приклонишь ухо твое, то будешь мудр.
\vs Sir 6:35 Бывай в собрании старцев, и кто мудр, прилепись к тому; люби слушать всякую священную повесть, и притчи разумные да не ускользают от тебя.
\vs Sir 6:36 Если увидишь разумного, ходи к нему с раннего утра, и пусть нога твоя истирает пороги дверей его.
\vs Sir 6:37 Размышляй о повелениях Господа и всегда поучайся в заповедях Его: Он укрепит твое сердце, и желание премудрости дастся тебе.
\vs Sir 7:1 Не делай зла, и тебя не постигнет зло;
\vs Sir 7:2 удаляйся от неправды, и она уклонится от тебя.
\vs Sir 7:3 Сын мой! не сей на бороздах неправды, и не будешь в семь раз более пожинать с них.
\vs Sir 7:4 Не проси у Господа власти, и у царя~--- почетного места.
\vs Sir 7:5 Не оправдывай себя пред Господом, и не мудрствуй пред царем.
\vs Sir 7:6 Не домогайся сделаться судьею, чтобы не оказаться тебе бессильным сокрушить неправду, чтобы не убояться когда-либо лица сильного и не положить тени на правоту твою.
\vs Sir 7:7 Не греши против городского общества, и не роняй себя пред народом.
\vs Sir 7:8 Не прилагай греха ко греху, ибо и за один не останешься ненаказанным.
\vs Sir 7:9 Не говори: <<Он призрит на множество даров моих, и, когда я принесу их Богу Вышнему, Он примет>>.
\vs Sir 7:10 Не малодушествуй в молитве твоей и не пренебрегай подавать милостыню.
\vs Sir 7:11 Не насмехайся над человеком, находящимся в горести души его; ибо есть Смиряющий и Возвышающий.
\vs Sir 7:12 Не выдумывай лжи на брата твоего, и не делай того же против друга.
\vs Sir 7:13 Не желай говорить какую бы то ни было ложь; ибо повторение ее не послужит ко благу.
\vs Sir 7:14 Пред собранием старших не многословь, и не повторяй слова в прошении твоем.
\vs Sir 7:15 Не отвращайся от трудной работы и от земледелия, которое учреждено от Вышнего.
\vs Sir 7:16 Не прилагайся ко множеству грешников.
\vs Sir 7:17 Глубоко смири душу твою.
\vs Sir 7:18 Помни, что гнев не замедлит,
\vs Sir 7:19 что наказание нечестивому~--- огонь и червь.
\vs Sir 7:20 Не меняй друга на сокровище, и брата однокровного~--- на золото Офирское.
\vs Sir 7:21 Не оставляй умной и доброй жены, ибо достоинство ее драгоценнее золота.
\vs Sir 7:22 Не обижай раба, трудящегося усердно, ни наемника, преданного тебе душею.
\vs Sir 7:23 Разумного раба да любит душа твоя, и не откажи ему в свободе.
\vs Sir 7:24 Есть у тебя скот? наблюдай за ним, и если он полезен тебе, пусть остается у тебя.
\vs Sir 7:25 Есть у тебя сыновья? учи их и с юности нагибай шею их.
\vs Sir 7:26 Есть у тебя дочери? имей попечение о теле их и не показывай им веселого лица твоего.
\vs Sir 7:27 Выдай дочь в замужество, и сделаешь великое дело, и подари ее мужу разумному.
\vs Sir 7:28 Есть у тебя жена по душе? не отгоняй ее.
\vs Sir 7:29 Всем сердцем почитай отца твоего и не забывай родильных болезней матери твоей.
\vs Sir 7:30 Помни, что ты рожден от них: и что можешь ты воздать им, как они тебе?
\vs Sir 7:31 Всею душею твоею благоговей пред Господом и уважай священников Его.
\vs Sir 7:32 Всею силою люби Творца твоего, и не оставляй служителей Его.
\vs Sir 7:33 Бойся Господа, и почитай священника, и давай ему часть, как заповедано тебе:
\vs Sir 7:34 начатки, и за грех, и даяние плеч, и жертву освящения, и начатки святых.
\vs Sir 7:35 И к бедному простирай руку твою, дабы благословение твое было совершенно.
\vs Sir 7:36 Милость даяния да будет ко всякому живущему, но и умершего не лишай милости.
\vs Sir 7:37 Не устраняйся от плачущих, и с сетующими сетуй.
\vs Sir 7:38 Не ленись посещать больного, ибо за это ты будешь возлюблен.
\vs Sir 7:39 Во всех делах твоих помни о конце твоем, и вовек не согрешишь.
\vs Sir 8:1 Не ссорься с человеком сильным, чтобы когда-нибудь не впасть в его руки.
\vs Sir 8:2 Не заводи тяжбы с человеком богатым, чтобы он не имел перевеса над тобою;
\vs Sir 8:3 ибо золото многих погубило, и склоняло сердца царей.
\vs Sir 8:4 Не спорь с человеком, дерзким на язык, и не подкладывай дров на огонь его.
\vs Sir 8:5 Не шути с невеждою, чтобы не подверглись бесчестию твои предки.
\vs Sir 8:6 Не укоряй человека, обращающегося от греха: помни, что все мы находимся под эпитимиями.
\vs Sir 8:7 Не пренебрегай человека в старости его, ибо и мы стареем.
\vs Sir 8:8 Не радуйся смерти человека, хотя бы он был самый враждебный тебе: помни, что все мы умрем.
\vs Sir 8:9 Не пренебрегай повестью мудрых и упражняйся в притчах их;
\vs Sir 8:10 ибо от них научишься в\acc{е}дению и~--- как служить сильным.
\vs Sir 8:11 Не удаляйся от повести старцев, ибо и они научились от отцов своих,
\vs Sir 8:12 и ты научишься от них рассудительности и~--- какой в случае надобности дать ответ.
\vs Sir 8:13 Не разжигай углей грешника, чтобы не сгореть от пламени огня его,
\vs Sir 8:14 и не восставай против наглеца, чтобы он не засел засадою в устах твоих.
\vs Sir 8:15 Не давай взаймы человеку, который сильнее тебя; а если дашь, то считай себя потерявшим.
\vs Sir 8:16 Не поручайся сверх силы твоей; а если поручишься, заботься, как обязанный заплатить.
\vs Sir 8:17 Не судись с судьею, потому что его будут судить по его почету.
\vs Sir 8:18 С отважным не пускайся в путь, чтобы он не был тебе в тягость; ибо он будет поступать по своему произволу, и ты можешь погибнуть от его безрассудства.
\vs Sir 8:19 Не заводи ссоры со вспыльчивым и не проходи с ним чрез пустыню; потому что кровь~--- как ничто в глазах его, и где нет помощи, он поразит тебя.
\vs Sir 8:20 Не советуйся с глупым, ибо он не может умолчать о деле.
\vs Sir 8:21 При чужом не делай тайного, ибо не знаешь, что он сделает.
\vs Sir 8:22 Не открывай всякому человеку твоего сердца, чтобы он дурно не отблагодарил тебя.
\vs Sir 9:1 Не будь ревнив к жене сердца твоего и не подавай ей дурного урока против тебя самого.
\vs Sir 9:2 Не отдавай жене души твоей, чтобы она не восстала против власти твоей.
\vs Sir 9:3 Не выходи навстречу развратной женщине, чтобы как-нибудь не попасть в сети ее.
\vs Sir 9:4 Не оставайся долго с певицею, чтобы не плениться тебе искусством ее.
\vs Sir 9:5 Не засматривайся на девицу, чтобы не соблазниться прелестями ее.
\vs Sir 9:6 Не отдавай души твоей блудницам, чтобы не погубить наследства твоего.
\vs Sir 9:7 Не смотри по сторонам на улицах города и не броди по пустым местам его.
\vs Sir 9:8 Отвращай око твое от женщины благообразной и не засматривайся на чужую красоту:
\vs Sir 9:9 многие совратились с пути чрез красоту женскую; от нее, как огонь, загорается любовь.
\vs Sir 9:10 Отнюдь не сиди с женою замужнею и не оставайся с нею на пиру за вином,
\vs Sir 9:11 чтобы не склонилась к ней душа твоя и чтобы ты не поползнулся духом в погибель.
\vs Sir 9:12 Не оставляй старого друга, ибо новый не может сравниться с ним;
\vs Sir 9:13 друг новый~--- то же, что вино новое: когда оно сделается старым, с удовольствием будешь пить его.
\rsbpar\vs Sir 9:14 Не завидуй славе грешника, ибо не знаешь, какой будет конец его.
\vs Sir 9:15 Не одобряй того, что одобряют нечестивые: помни, что они до \bibemph{самого} ада не исправятся.
\vs Sir 9:16 Держи себя дальше от человека, имеющего власть умерщвлять, и ты не будешь смущаться страхом смерти;
\vs Sir 9:17 а если сближаешься с ним, не ошибись, чтобы он не лишил тебя жизни:
\vs Sir 9:18 знай, что ты посреди сетей идешь и по зубцам городских стен проходишь.
\vs Sir 9:19 По силе твоей узнавай ближних и советуйся с мудрыми.
\vs Sir 9:20 Рассуждение твое да будет с разумными, и всякая беседа твоя~--- в законе Вышнего.
\vs Sir 9:21 Да вечеряют с тобою мужи праведные, и слава твоя да будет в страхе Господнем.
\vs Sir 9:22 Изделие хвалится по руке художника, а правитель народа считается мудрым по словам его.
\vs Sir 9:23 Боятся в городе дерзкого на язык, и ненавидят опрометчивого в словах.
\vs Sir 10:1 Мудрый правитель научит народ свой, и правление разумного будет благоустроено.
\vs Sir 10:2 Каков правитель народа, таковы и служащие при нем; и каков начальствующий над городом, таковы и все живущие в нем.
\vs Sir 10:3 Царь ненаученный погубит народ свой, а при благоразумии сильных устроится город.
\vs Sir 10:4 В руке Господа власть над землею, и \bibemph{человека} потребного Он вовремя воздвигнет на ней.
\vs Sir 10:5 В руке Господа благоуспешность человека, и на лице книжника Он отпечатлеет славу Свою.
\vs Sir 10:6 Не гневайся за всякое оскорбление на ближнего, и никого не оскорбляй делом.
\vs Sir 10:7 Гордость ненавистна и Господу и людям и преступна против обоих.
\vs Sir 10:8 Владычество переходит от народа к народу по причине несправедливости, обид и любостяжания.
\vs Sir 10:9 Что гордится земля и пепел?
\vs Sir 10:10 И при жизни извергаются внутренности его.
\vs Sir 10:11 Продолжительною болезнью врач пренебрегает:
\vs Sir 10:12 и вот, ныне царь, а завтра умирает.
\vs Sir 10:13 Когда же человек умрет, то наследием его становятся пресмыкающиеся, звери и черви.
\vs Sir 10:14 Начало гордости~--- удаление человека от Господа и отступление сердца его от Творца его;
\vs Sir 10:15 ибо начало греха~--- гордость, и обладаемый ею изрыгает мерзость;
\vs Sir 10:16 и за это Господь посылает на него страшные наказания и вконец низлагает его.
\vs Sir 10:17 Господь низвергает престолы властителей и посаждает кротких на место их.
\vs Sir 10:18 Господь вырывает с корнем народы и насаждает, вместо них, смиренных.
\vs Sir 10:19 Господь опустошает страны народов и разрушает их до оснований земли.
\vs Sir 10:20 Он иссушает их, и погубляет \bibemph{людей} и истребляет от земли память их.
\vs Sir 10:21 Гордость не сотворена для людей, ни ярость гнева~--- для рождающихся от жен.
\rsbpar\vs Sir 10:22 Семя почтенное какое?~--- Семя человеческое. Семя почтенное какое?~--- Боящиеся Господа.
\vs Sir 10:23 Семя бесчестное какое?~--- Семя человеческое. Семя бесчестное какое?~--- Преступающие заповеди.
\vs Sir 10:24 Старший между братьями~--- в почтении у них, так и боящиеся Господа~--- в очах Его.
\vs Sir 10:25 Богат ли кто и славен, или беден, похвала их~--- страх Господень.
\vs Sir 10:26 Несправедливо~--- бесчестить разумного бедного, и не должно прославлять мужа грешного.
\vs Sir 10:27 Почтенны вельможа, судья и властелин, но нет из них больше боящегося Господа.
\vs Sir 10:28 Рабу мудрому будут служить свободные, и разумный человек, будучи наставляем им, не будет роптать.
\vs Sir 10:29 Не умничай много, чтобы делать дело твое, и не хвались во время нужды.
\vs Sir 10:30 Лучше тот, кто трудится и имеет во всем достаток, нежели кто праздно ходит и хвалится, но нуждается в хлебе.
\vs Sir 10:31 Сын мой! кротостью прославляй душу твою и воздавай ей честь по ее достоинству.
\vs Sir 10:32 Кто будет оправдывать согрешающего против души своей? И кто будет хвалить позорящего жизнь свою?
\vs Sir 10:33 Бедного почитают за познания его, а богатого~--- за его богатство:
\vs Sir 10:34 уважаемый же в бедности насколько больше будет уважаем в богатстве? А бесславный в богатстве насколько будет бесславнее в бедности?
\vs Sir 11:1 Мудрость смиренного вознесет голову его и посадит его среди вельмож.
\vs Sir 11:2 Не хвали человека за красоту его, и не имей отвращения к человеку за наружность его.
\vs Sir 11:3 Мала пчела между летающими, но плод ее~--- лучший из сластей.
\vs Sir 11:4 Не хвались пышностью одежд и не превозносись в день славы: ибо дивны дела Господа, и сокровенны дела Его между людьми.
\vs Sir 11:5 Многие из властелинов сидели на земле, тот же, о ком не думали, носил венец.
\vs Sir 11:6 Многие из сильных подверглись крайнему бесчестию, и славные преданы были в руки других.
\vs Sir 11:7 Прежде, нежели исследуешь, не порицай; узнай прежде, и тогда упрекай.
\vs Sir 11:8 Прежде, нежели выслушаешь, не отвечай, и среди речи не перебивай.
\vs Sir 11:9 Не спорь о деле, для тебя ненужном, и не сиди на суде грешников.
\rsbpar\vs Sir 11:10 Сын мой! не берись за множество дел: при множестве дел не останешься без вины. И если будешь гнаться за ними, не достигнешь, и, убегая, не уйдешь.
\vs Sir 11:11 Иной трудится, напрягает силы, поспешает, и тем более отстает.
\vs Sir 11:12 Иной вял, нуждается в помощи, слабосилен и изобилует нищетою;
\vs Sir 11:13 но очи Господа призрели на него во благо ему, и Он восставил его из унижения его и вознес голову его, и многие изумлялись, смотря на него.
\vs Sir 11:14 Доброе и худое, жизнь и смерть, бедность и богатство~--- от Господа.
\vs Sir 11:15 Даяние Господа предоставлено благочестивым, и благоволение Его будет благопоспешно для них вовек.
\vs Sir 11:16 Иной делается богатым от осмотрительности и бережливости своей, и это часть награды его,
\vs Sir 11:17 когда он скажет: <<я нашел покой и теперь наслаждаюсь моими благами>>.
\vs Sir 11:18 И не знает он, сколько пройдет времени до того, когда он оставит их другим и умрет.
\vs Sir 11:19 Твердо стой в завете твоем и пребывай в нем и состарься в деле твоем.
\vs Sir 11:20 Не удивляйся делам грешника, веруй Господу, и пребывай в труде твоем:
\vs Sir 11:21 ибо легко в очах Господа~--- скоро и внезапно обогатить бедного.
\vs Sir 11:22 Благословение Господа~--- награда благочестивого, и в скором времени процветает он благословением Его.
\vs Sir 11:23 Не говори: <<что мне еще нужно? и какие отныне могу иметь еще блага?>>
\vs Sir 11:24 Не говори: <<довольно у меня, и какое отныне могу я потерпеть зло?>>
\vs Sir 11:25 Во дни счастья бывает забвение о несчастье, и во дни несчастья не вспомнится о счастье.
\vs Sir 11:26 Легко для Господа~--- в день смерти воздать человеку по делам его.
\vs Sir 11:27 Минутное страдание производит забвение утех, и при кончине человека открываются дела его.
\vs Sir 11:28 Прежде смерти не называй никого блаженным; человек познается в детях своих.
\rsbpar\vs Sir 11:29 Не всякого человека вводи в дом твой, ибо много козней у коварного.
\vs Sir 11:30 Как охотничья птица в западне, таково сердце надменного: он, как лазутчик, подсматривает падение;
\vs Sir 11:31 превращая добро во зло, он строит козни и на людей избранных кладет пятно.
\vs Sir 11:32 От искры огня умножаются угли, и человек грешный строит козни на кровь.
\vs Sir 11:33 Остерегайся злодея,~--- ибо он строит зло,~--- чтобы он когда-нибудь не положил на тебе пятна навек.
\vs Sir 11:34 Посели в доме твоем чужого, и он расстроит тебя смутами и сделает тебя чужим для твоих.
\vs Sir 12:1 Если ты делаешь добро, знай, кому делаешь, и будет благодарность за твои благодеяния.
\vs Sir 12:2 Делай добро благочестивому, и получишь воздаяние, и если не от него, то от Всевышнего.
\vs Sir 12:3 Нет добра для того, кто постоянно занимается злом и кто не подает милостыни.
\vs Sir 12:4 Давай благочестивому, и не помогай грешнику.
\vs Sir 12:5 Делай добро смиренному, и не давай нечестивому: запирай от него хлеб и не давай ему, чтобы он чрез то не превозмог тебя;
\vs Sir 12:6 ибо ты получил бы сугубое зло за все добро, которое сделал бы ему; ибо и Всевышний ненавидит грешников и нечестивым воздает отмщением.
\vs Sir 12:7 Давай доброму, и не помогай грешнику.
\vs Sir 12:8 Друг не познается в счастье, и враг не скроется в несчастье.
\vs Sir 12:9 При счастье человека враги его в печали, а в несчастье его и друг разойдется с ним.
\vs Sir 12:10 Не верь врагу твоему вовек, ибо, как ржавеет медь, так и злоба его:
\vs Sir 12:11 хотя бы он смирился и ходил согнувшись, будь внимателен душею твоею и остерегайся его, и будешь пред ним, как чистое зеркало, и узнаешь, что он не до конца очистился от ржавчины;
\vs Sir 12:12 не ставь его подле себя, чтобы он, низринув тебя, не стал на твое место; не сажай его по правую сторону себя, чтобы он когда-нибудь не стал домогаться твоего седалища, и ты наконец поймешь слова мои и со скорбью вспомнишь о наставлениях моих.
\vs Sir 12:13 Кто пожалеет об ужаленном заклинателе змей и обо всех, приближающихся к диким зверям? Так и о сближающемся с грешником и приобщающемся грехам его:
\vs Sir 12:14 на время он останется с тобою, но если ты поколеблешься, он не устоит.
\vs Sir 12:15 Устами своими враг усладит \bibemph{тебя}, но в сердце своем замышляет ввергнуть тебя в яму: глазами своими враг будет плакать, а когда найдет случай, не насытится кровью.
\vs Sir 12:16 Если встретится с тобою несчастье, ты найдешь его там прежде себя,
\vs Sir 12:17 и он, как будто желая помочь, подставит тебе ногу:
\vs Sir 12:18 будет кивать головою и хлопать руками, многое будет шептать, и изменит лицо свое.
\vs Sir 13:1 Кто прикасается к смоле, тот очернится, и кто входит в общение с гордым, сделается подобным ему.
\vs Sir 13:2 Не поднимай тяжести свыше твоей силы, и не входи в общение с тем, кто сильнее и богаче тебя.
\vs Sir 13:3 Какое общение у горшка с котлом? Этот толкнет его, и он разобьется.
\vs Sir 13:4 Богач обидел, и сам же грозит; бедняк обижен, и сам же упрашивает.
\vs Sir 13:5 Если ты выгоден для него, он употребит тебя; а если обеднеешь, он оставит тебя.
\vs Sir 13:6 Если ты достаточен, он будет жить с тобою и истощит тебя, а сам не поболезнует.
\vs Sir 13:7 Возымел он в тебе нужду,~--- будет льстить тебе, будет улыбаться тебе и обнадеживать тебя, ласково будет говорить с тобою и скажет: <<не нужно ли тебе чего?>>
\vs Sir 13:8 Своими угощениями он будет пристыжать тебя, доколе, два или три раза ограбив тебя, не насмеется наконец над тобою.
\vs Sir 13:9 После того он, увидев тебя, уклонится от тебя и будет кивать головою при встрече с тобою.
\vs Sir 13:10 Наблюдай, чтобы тебе не быть обманутым
\vs Sir 13:11 и не быть униженным в твоем веселье.
\vs Sir 13:12 Когда сильный будет приглашать тебя, уклоняйся, и тем более он будет приглашать тебя.
\vs Sir 13:13 Не будь навязчив, чтобы не оттолкнули тебя, и не слишком удаляйся, чтобы не забыли о тебе.
\vs Sir 13:14 Не дозволяй себе говорить с ним, как с равным тебе, и не верь слишком многим словам его; ибо долгим разговором он будет искушать тебя и, как бы шутя, изведывать тебя.
\vs Sir 13:15 Немилостив к себе, кто не удерживает себя в словах своих, и он не убережет себя от оскорбления и от уз.
\vs Sir 13:16 Будь осторожен и весьма внимателен, ибо ты ходишь с падением твоим.
\vs Sir 13:17 Услышав это во сне твоем, не засыпай.
\vs Sir 13:18 Во всю жизнь люби Господа и взывай к Нему о спасении твоем.
\vs Sir 13:19 Всякое животное любит подобное себе, и всякий человек~--- ближнего своего.
\vs Sir 13:20 Всякая плоть соединяется по роду своему, и человек прилепляется к подобному себе.
\vs Sir 13:21 Какое общение у волка с ягненком? Так и у грешника~--- с благочестивым.
\vs Sir 13:22 Какой мир у гиены с собакою? И какой мир у богатого с бедным?
\vs Sir 13:23 Ловля у львов~--- дикие ослы в пустыне, так пастбища богатых~--- бедные.
\vs Sir 13:24 Отвратительно для гордого смирение: так отвратителен для богатого бедный.
\vs Sir 13:25 Когда пошатнется богатый, он поддерживается друзьями; а когда упадет бедный, то отталкивается и друзьями.
\vs Sir 13:26 Когда подвергнется несчастью богатый, у него много помощников; сказал нелепость, и оправдали его.
\vs Sir 13:27 Подвергся несчастью бедняк, и еще бранят его; сказал разумно, и его не слушают.
\vs Sir 13:28 Заговорил богатый,~--- и все замолчали и превознесли речь его до облаков;
\vs Sir 13:29 заговорил бедный, и говорят: <<это кто такой?>> И если он споткнется, то совсем низвергнут его.
\rsbpar\vs Sir 13:30 Хорошо богатство, в котором нет греха, и зла бедность в устах нечестивого.
\vs Sir 13:31 Сердце человека изменяет лицо его или на хорошее, или на худое.
\vs Sir 13:32 Признак сердца в счастье~--- лицо веселое, а изобретение притчей соединено с напряженным размышлением.
\vs Sir 14:1 Блажен человек, который не погрешал устами своими и не уязвлен был печалью греха.
\vs Sir 14:2 Блажен, кого не зазирает душа его и кто не потерял надежды своей.
\vs Sir 14:3 Не добро богатство человеку скупому. И на что имение человеку недоброжелательному?
\vs Sir 14:4 Кто собирает, отнимая у души своей, тот собирает для других, и благами его будут пресыщаться другие.
\vs Sir 14:5 Кто зол для себя, для кого будет добр? И не будет он иметь радости от имения своего.
\vs Sir 14:6 Нет хуже человека, который недоброжелателен к самому себе, и это~--- воздаяние за злобу его.
\vs Sir 14:7 Если он и делает добро, то делает в забывчивости, и после обнаруживает зло свое.
\vs Sir 14:8 Зол, кто имеет завистливые глаза, отвращает лицо и презирает души.
\vs Sir 14:9 Глаза любостяжательного не насыщаются какою-либо частью, и неправда злого иссушает душу.
\vs Sir 14:10 Злой глаз завистлив даже на хлеб и в столе своем терпит скудость.
\vs Sir 14:11 Сын мой! по состоянию твоему делай добро себе и приношения Господу достойно приноси.
\vs Sir 14:12 Помни, что смерть не медлит, и завет ада не открыт тебе:
\vs Sir 14:13 прежде, нежели умрешь, делай добро другу, и по силе твоей простирай твою руку и давай ему.
\vs Sir 14:14 Не лишай себя доброго дня, и часть доброго желания да не пройдет мимо тебя.
\vs Sir 14:15 Не другим ли оставишь ты стяжания твои и плоды усилий твоих для раздела по жребию?
\vs Sir 14:16 Давай и принимай, и утешай душу твою,
\vs Sir 14:17 ибо в аде нельзя найти утех.
\vs Sir 14:18 Всякая плоть, как одежда, ветшает; ибо от века~--- определение: <<смертью умрешь>>.
\vs Sir 14:19 Как зеленеющие листья на густом дереве~--- одни спадают, а другие вырастают: так и род от плоти и крови~--- один умирает, а другой рождается.
\vs Sir 14:20 Всякая вещь, подверженная тлению, исчезает, и сделавший ее умирает с нею.
\rsbpar\vs Sir 14:21 Блажен человек, который упражняется в мудрости и в разуме своем поучается святому.
\vs Sir 14:22 Кто размышляет в сердце своем о путях ее, тот получит разумение и в тайнах ее.
\vs Sir 14:23 Выходи за нею, как ловчий, и строй засаду на путях ее.
\vs Sir 14:24 Кто приклоняется к окнам ее, тот послушает и при дверях ее.
\vs Sir 14:25 Кто обращается вблизи дома ее, тот вобьет гвоздь и в стенах ее, поставит палатку свою подле нее и будет обитать в жилище благ.
\vs Sir 14:26 Он положит детей своих под кровом ее и будет иметь ночлег под сенью ее.
\vs Sir 14:27 Он прикроется ею от зноя и будет жить в славе ее.
\vs Sir 15:1 Боящийся Господа будет поступать так, и твердый в законе овладеет ею.
\vs Sir 15:2 И она встретит его, как мать, и примет его к себе, как целомудренная супруга;
\vs Sir 15:3 напитает его хлебом разума, и водою мудрости напоит его.
\vs Sir 15:4 Он утвердится на ней и не поколеблется; прилепится к ней и не постыдится.
\vs Sir 15:5 И она вознесет его над ближними его, и среди собрания откроет уста его.
\vs Sir 15:6 Веселье и венец радости и вечное имя наследует он.
\vs Sir 15:7 Не постигнут ее люди неразумные, и грешники не увидят ее.
\vs Sir 15:8 Далека она от гордости, и люди лживые не подумают о ней.
\vs Sir 15:9 Неприятна похвала в устах грешника, ибо не от Господа послана она.
\vs Sir 15:10 Будет похвала произнесена мудростью, и Господь благопоспешит ей.
\rsbpar\vs Sir 15:11 Не говори: <<ради Господа я отступил>>; ибо, что Он ненавидит, того ты не должен делать.
\vs Sir 15:12 Не говори: <<Он ввел меня в заблуждение>>, ибо Он не имеет надобности в муже грешном.
\vs Sir 15:13 Всякую мерзость Господь ненавидит, и неприятна она боящимся Его.
\vs Sir 15:14 Он от начала сотворил человека и оставил его в руке произволения его.
\vs Sir 15:15 Если хочешь, соблюдешь заповеди и сохранишь благоугодную верность.
\vs Sir 15:16 Он предложил тебе огонь и воду: на что хочешь, прострешь руку твою.
\vs Sir 15:17 Пред человеком жизнь и смерть, и чего он пожелает, то и дастся ему.
\vs Sir 15:18 Велика премудрость Господа, крепок Он могуществом и видит всё.
\vs Sir 15:19 Очи Его~--- на боящихся Его, и Он знает всякое дело человека.
\vs Sir 15:20 Никому не заповедал Он поступать нечестиво и никому не дал позволения грешить.
\vs Sir 16:1 Не желай множества негодных детей и не радуйся о сыновьях нечестивых. Когда они умножаются, не радуйся о них, если нет в них страха Господня.
\vs Sir 16:2 Не надейся на их жизнь и не опирайся на их множество.
\vs Sir 16:3 Лучше один праведник, нежели тысяча \bibemph{грешников},
\vs Sir 16:4 и лучше умереть бездетным, нежели иметь детей нечестивых,
\vs Sir 16:5 ибо от одного разумного населится город, а племя беззаконных опустеет.
\vs Sir 16:6 Много такого видело око мое, и еще более того слышало ухо мое.
\vs Sir 16:7 В сборище грешников возгорится огонь, как и в народе непокорном возгорался гнев.
\vs Sir 16:8 Не умилостивился Он над древними исполинами, которые в надежде на силу свою сделались отступниками;
\vs Sir 16:9 не пощадил и живших в одном месте с Лотом, которыми возгнушался за их гордость;
\vs Sir 16:10 не помиловал народа погибельного, который надмевался грехами своими,
\vs Sir 16:11 равно как и шестисот тысяч человек, соединившихся в жестокосердии своем. И хотя бы и один был непокорный, было бы удивительно, если б он остался ненаказанным;
\vs Sir 16:12 ибо и милость и гнев~--- во власти Его: силен Он помиловать и излить гнев.
\vs Sir 16:13 Как велика милость Его, так велико и обличение Его. Он судит человека по делам его.
\vs Sir 16:14 Не убежит от Него грешник с хищением, и терпение благочестивого не останется тщетным.
\vs Sir 16:15 Всякой милостыне Он даст место, каждый получит по делам своим.
\vs Sir 16:16 Не говори: <<я скроюсь от Господа; неужели с высоты кто вспомнит обо мне?
\vs Sir 16:17 Во множестве народа меня не заметят; ибо что душа моя в неизмеримом создании?
\vs Sir 16:18 Вот, небо и небо небес~--- Божие, бездна и земля колеблются от посещения Его.
\vs Sir 16:19 Равно сотрясаются от страха горы и основания земли, когда Он взирает.
\vs Sir 16:20 И этого не может понять сердце;
\vs Sir 16:21 а пути Его кто постигнет? Как ветер, которого человек не может видеть, так и большая часть дел Его сокрыты.
\vs Sir 16:22 Кто возвестит о делах правосудия Его? или кто будет ожидать их? ибо далеко это определение>>.
\vs Sir 16:23 Скудный умом думает так, и человек неразумный и заблуждающийся размышляет так глупо.
\rsbpar\vs Sir 16:24 Слушай меня, сын мой, и учись знанию, и внимай сердцем твоим словам моим.
\vs Sir 16:25 Я показываю тебе учение обдуманное и передаю знание точное.
\vs Sir 16:26 По определению Господа дела Его от начала, и от сотворения их Он разделил части их.
\vs Sir 16:27 Навек устроил Он дела Свои, и начала их~--- в роды их. Они не алчут, не утомляются и не прекращают своих действий.
\vs Sir 16:28 Ни одно не стесняет близкого ему,
\vs Sir 16:29 и до века не воспротивятся они слову Его.
\vs Sir 16:30 И потом воззрел Господь на землю и наполнил ее Своими благами.
\vs Sir 16:31 Душа всего живущего покрыла лице ее, и в нее все возвратится.
\vs Sir 17:1 Господь создал человека из земли и опять возвращает его в нее.
\vs Sir 17:2 Определенное число дней и время дал Он им, и дал им власть над всем, что на ней.
\vs Sir 17:3 По природе их, облек их силою и сотворил их по образу Своему,
\vs Sir 17:4 и вложил страх к ним во всякую плоть, чтобы господствовать им над зверями и птицами.
\vs Sir 17:5 Он дал им смысл, язык и глаза, и уши и сердце для рассуждения,
\vs Sir 17:6 исполнил их проницательностью разума и показал им добро и зло.
\vs Sir 17:7 Он положил око Свое на сердца их, чтобы показать им величие дел Своих,
\vs Sir 17:8 да прославляют они святое имя Его и возвещают о величии дел Его.
\vs Sir 17:9 Он приложил им знание и дал им в наследство закон жизни;
\vs Sir 17:10 вечный завет поставил с ними и показал им суды Свои.
\vs Sir 17:11 Величие славы видели глаза их, и славу голоса Его слышало ухо их.
\vs Sir 17:12 И сказал Он им: <<остерегайтесь всякой неправды>>; и заповедал каждому из них обязанность к ближнему.
\vs Sir 17:13 Пути их всегда пред Ним, не скроются от очей Его.
\vs Sir 17:14 Каждому народу поставил Он вождя,
\vs Sir 17:15 а Израиль есть удел Господа.
\vs Sir 17:16 Все дела их~--- как солнце пред Ним, и очи Его всегда на путях их.
\vs Sir 17:17 Не утаились от Него неправды их, и все грехи их~--- пред Господом.
\rsbpar\vs Sir 17:18 Милостыня человека~--- как печать у Него, и благодеяние человека сохранит Он, как зеницу ока.
\vs Sir 17:19 Потом Он восстанет и воздаст им, и даяние их на голову их возвратит.
\vs Sir 17:20 Но кающимся Он давал обращение и ободрял ослабевавших в терпении.
\vs Sir 17:21 Обратись к Господу и оставь грехи;
\vs Sir 17:22 молись пред Ним и уменьши твои преткновения.
\vs Sir 17:23 Возвратись ко Всевышнему, и отвратись от неправды, и сильно возненавидь мерзость.
\vs Sir 17:24 Кто будет восхвалять Всевышнего в аде, вместо живущих и прославляющих Его?
\vs Sir 17:25 От мертвого, как от несуществующего, нет прославления:
\vs Sir 17:26 живый и здоровый восхвалит Господа.
\vs Sir 17:27 Как велико милосердие Господа и примирение с обращающимися к Нему!
\vs Sir 17:28 Не может быть всего в человеке,
\vs Sir 17:29 потому что не бессмертен сын человеческий.
\vs Sir 17:30 Что светлее солнца? но и оно затмевается. И о злом будет помышлять плоть и кровь.
\vs Sir 17:31 За силами высоких небес Он Сам наблюдает, а люди все~--- земля и пепел.
\vs Sir 18:1 Все вообще создал Живущий во веки; Господь один праведен.
\vs Sir 18:2 Никому не предоставил Он изъяснять дел\acc{а} Его.
\vs Sir 18:3 И кто может исследовать великие дела Его?
\vs Sir 18:4 Кто может измерить силу величия Его? и кто может также изречь милости Его?
\vs Sir 18:5 Невозможно ни умалить, ни увеличить, и невозможно исследовать дивных дел Господа.
\vs Sir 18:6 Когда человек окончил бы, тогда он только начинает, и когда перестанет, придет в изумление.
\vs Sir 18:7 Что есть человек и что польза его? что благо его и что зло его?
\vs Sir 18:8 Число дней человека~--- много, если сто лет: как капля воды из моря или крупинка песка, так малы лета его в дне вечности.
\vs Sir 18:9 Посему Господь долготерпелив к \bibemph{людям} и изливает на них милость Свою.
\vs Sir 18:10 Он видит и знает, что конец их очень бедствен,
\vs Sir 18:11 и потому умножает милости Свои.
\vs Sir 18:12 Милость человека~--- к ближнему его, а милость Господа~--- на всякую плоть.
\vs Sir 18:13 Он обличает и вразумляет, и поучает и обращает, как пастырь стадо свое.
\vs Sir 18:14 Он милует принимающих вразумление и усердно обращающихся к закону Его.
\rsbpar\vs Sir 18:15 Сын мой! при благотворениях не делай упреков, и при всяком даре не оскорбляй словами.
\vs Sir 18:16 Роса не охлаждает ли зноя? так слово~--- лучше, нежели даяние.
\vs Sir 18:17 Поэтому не выше ли доброго даяния слово? а у человека доброжелательного и то и другое.
\vs Sir 18:18 Глупый немилосердно укоряет, и подаяние неблагорасположенного иссушает глаза.
\vs Sir 18:19 Прежде, нежели начнешь говорить, обдумывай, и прежде болезни заботься о себе.
\vs Sir 18:20 Испытывай себя прежде суда, и во время посещения найдешь милость.
\vs Sir 18:21 Прежде, нежели почувствуешь слабость, смиряйся, и во время грехов покажи обращение.
\vs Sir 18:22 Ничто да не препятствует тебе исполнить обет благовременно, и не откладывай оправдания до смерти.
\vs Sir 18:23 Прежде, нежели начнешь молиться, приготовь себя, и не будь как человек, искушающий Господа.
\vs Sir 18:24 Припоминай о гневе в день смерти и о времени отмщения, когда Господь отвратит лице Свое.
\vs Sir 18:25 Во время сытости вспоминай о времени голода и во дни богатства~--- о бедности и нужде.
\vs Sir 18:26 От утра до вечера изменяется время, и все скоротечно пред Господом.
\vs Sir 18:27 Человек мудрый во всем будет осторожен и во дни грехов удержится от беспечности.
\vs Sir 18:28 Всякий разумный познает премудрость и нашедшему ее воздаст хвалу.
\vs Sir 18:29 Рассудительные в словах и сами умудряются, и источают основательные притчи.
\rsbpar\vs Sir 18:30 Не ходи вслед похотей твоих и воздерживайся от пожеланий твоих.
\vs Sir 18:31 Если будешь доставлять душе твоей приятное для вожделений, то она сделает тебя потехою для врагов твоих.
\vs Sir 18:32 Не ищи увеселения в большой роскоши и не привязывайся к пиршествам.
\vs Sir 18:33 Не сделайся нищим, пиршествуя на занятые деньги, когда ничего нет у тебя в кошельке.
\vs Sir 19:1 Работник, склонный к пьянству, не обогатится, и ни во что ставящий малое мало-помалу придет в упадок.
\vs Sir 19:2 Вино и женщины развратят разумных, а связывающийся с блудницами сделается еще наглее;
\vs Sir 19:3 гниль и черви наследуют его, и дерзкая душа истребится.
\vs Sir 19:4 Кто скоро доверяет, тот легкомыслен, и согрешающий грешит против души своей.
\vs Sir 19:5 Преданный сердцем удовольствиям будет осужден, а сопротивляющийся вожделениям увенчает жизнь свою.
\vs Sir 19:6 Обуздывающий язык будет жить мирно, и ненавидящий болтливость уменьшит зло.
\vs Sir 19:7 Никогда не повторяй слова, и ничего у тебя не убудет.
\vs Sir 19:8 Ни другу ни недругу не рассказывай и, если это тебе не грех, не открывай;
\vs Sir 19:9 ибо он выслушает тебя, и будет остерегаться тебя, и по времени возненавидит тебя.
\vs Sir 19:10 Выслушал ты слово, пусть умрет оно с тобою: не бойся, не расторгнет оно тебя.
\vs Sir 19:11 Глупый от слова терпит такую же муку, как рождающая~--- от младенца.
\vs Sir 19:12 Что стрела, вонзенная в бедро, то слово в сердце глупого.
\vs Sir 19:13 Расспроси друга \bibemph{твоего}, может быть, не сделал он того; и если сделал, то пусть вперед не делает.
\vs Sir 19:14 Расспроси друга, может быть, не говорил он того; и если сказал, то пусть не повторит того.
\vs Sir 19:15 Расспроси друга, ибо часто бывает клевета.
\vs Sir 19:16 Не всякому слову верь.
\vs Sir 19:17 Иной погрешает \bibemph{словом}, но не от души; и кто не погрешал языком своим?
\vs Sir 19:18 Расспроси ближнего твоего прежде, нежели грозить ему, и дай место закону Всевышнего.\rsbpar Всякая мудрость~--- страх Господень, и во всякой мудрости~--- исполнение закона.
\vs Sir 19:19 И не есть мудрость знание худого. И нет разума, где совет грешников.
\vs Sir 19:20 Есть лукавство, и это мерзость; и есть неразумный, скудный мудростью.
\vs Sir 19:21 Лучше скудный знанием, но богобоязненный, нежели богатый знанием~--- и преступающий закон.
\vs Sir 19:22 Есть хитрость изысканная, но она беззаконна, и есть превращающий \bibemph{суд}, чтобы произнести приговор.
\vs Sir 19:23 Есть лукавый, который ходит согнувшись, в унынии, но внутри он полон коварства.
\vs Sir 19:24 Он поник лицом и притворяется глухим, но он предварит тебя там, где и не думаешь.
\vs Sir 19:25 И если недостаток силы воспрепятствует ему повредить тебе, то он сделает тебе зло, когда найдет случай.
\vs Sir 19:26 По виду узнается человек, и по выражению лица при встрече познается разумный.
\vs Sir 19:27 Одежда и осклабление зубов и походка человека показывают свойство его.
\vs Sir 19:28 Бывает обличение, но не вовремя, и бывает, что иной молчит~--- и он благоразумен.
\vs Sir 20:1 Гораздо лучше обличить, нежели сердиться тайно; и обличаемый наедине предостережется от вреда.
\vs Sir 20:2 Как хорошо обличенному показать раскаяние!
\vs Sir 20:3 Ибо он избежит вольного греха.
\vs Sir 20:4 Что~--- пожелание евнуха растлить девицу, то~--- производящий суд с натяжкою.
\vs Sir 20:5 Иной молчит~--- и оказывается мудрым; а иной бывает ненавистным за многую болтливость.
\vs Sir 20:6 Иной молчит, потому что не имеет, что отвечать; а иной молчит, потому что знает время.
\vs Sir 20:7 Мудрый человек будет молчать до времени; а тщеславный и безрассудный не будет ждать времени.
\vs Sir 20:8 Многоречивый опротивеет, и кто восхищает себе право говорить, будет возненавиден.
\vs Sir 20:9 Бывает успех человеку ко злу, а находка~--- в потерю.
\vs Sir 20:10 Есть даяние, которое не будет тебе на пользу, и есть даяние, за которое бывает сугубое воздаяние.
\vs Sir 20:11 Бывает унижение для славы, а иной от унижения поднимает голову.
\vs Sir 20:12 Иной малым покупает многое и заплатит за то в семь раз больше.
\vs Sir 20:13 Мудрый в слове делается любезным, любезности же глупых останутся напрасными.
\vs Sir 20:14 Даяние безумного не будет тебе на пользу; ибо у него вместо одного много глаз для принятия.
\vs Sir 20:15 Немного даст он, а попрекать будет много, и раскроет уста свои, как глашатай. Ныне он взаем дает, а завтра потребует назад: ненавистен такой человек Господу и людям.
\vs Sir 20:16 Глупый говорит: <<нет у меня друга, и нет благодарности за мои благодеяния. Съедающие хлеб мой льстивы языком>>.
\vs Sir 20:17 Как часто и сколь многие будут насмехаться над ним!
\vs Sir 20:18 Преткновение от земли лучше, нежели от языка. Итак, скоро придет падение злых.
\vs Sir 20:19 Неприятный человек~--- безвременная басня; она всегда будет на устах невежд.
\vs Sir 20:20 Притча из уст глупого отвратительна, ибо он не скажет ее в свое время.
\vs Sir 20:21 Иной удерживается от греха скудостью, и в этом воздержании он не будет сокрушаться.
\vs Sir 20:22 Иной губит душу свою по робости, и губит ее из лицеприятия к безумному.
\vs Sir 20:23 Иной из-за стыда дает обещания другу, и без причины наживает в нем себе врага.
\vs Sir 20:24 Злой порок в человеке~--- ложь; в устах невежд она~--- всегда.
\vs Sir 20:25 Лучше вор, нежели постоянно говорящий ложь; но оба они наследуют погибель.
\vs Sir 20:26 Поведение лживого человека~--- бесчестно, и позор его всегда с ним.
\vs Sir 20:27 Мудрый в словах возвысит себя, и человек разумный понравится вельможам.
\vs Sir 20:28 Возделывающий землю увеличит свой стог, и угождающий вельможам получит помилование в случае неправды.
\vs Sir 20:29 Угощения и подарки ослепляют глаза мудрых и, как бы узда в устах, отвращают обличения.
\vs Sir 20:30 Скрытая мудрость и утаенное сокровище~--- какая польза от обоих?
\vs Sir 20:31 Лучше человек, скрывающий свою глупость, нежели человек, скрывающий свою мудрость.
\vs Sir 21:1 Сын мой! если ты согрешил, не прилагай более грехов и о прежних молись.
\vs Sir 21:2 Беги от греха, как от лица змея; ибо, если подойдешь к нему, он ужалит тебя.
\vs Sir 21:3 Зубы его~--- зубы львиные, которые умерщвляют души людей.
\vs Sir 21:4 Всякое беззаконие как обоюдоострый меч: ране от него нет исцеления.
\vs Sir 21:5 Устрашения и насилия опустошат богатство: так опустеет и дом высокомерного.
\vs Sir 21:6 Моление из уст нищего~--- \bibemph{только} до ушей его; но суд над ним поспешно приближается.
\vs Sir 21:7 Ненавидящий обличение идет по следам грешника, а боящийся Господа обратится сердцем.
\vs Sir 21:8 Издалека узнается сильный языком; но разумный видит, где тот спотыкается.
\vs Sir 21:9 Строящий дом свой на чужие деньги~--- то же, что собирающий камни для своей могилы.
\vs Sir 21:10 Сборище беззаконных~--- куча пакли, и конец их~--- пламень огненный.
\vs Sir 21:11 Путь грешников вымощен камнями, но на конце его~--- пропасть ада.
\vs Sir 21:12 Соблюдающий закон обладает своими мыслями,
\vs Sir 21:13 и совершение страха Господня~--- мудрость.
\vs Sir 21:14 Не научится тот, кто неспособен;
\vs Sir 21:15 но есть способность, умножающая горечь.
\vs Sir 21:16 Знание мудрого увеличивается подобно наводнению, и совет его,~--- как источник жизни.
\vs Sir 21:17 Сердце глупого подобно разбитому сосуду и не удержит в себе никакого знания.
\vs Sir 21:18 Если мудрое слово услышит разумный, то он похвалит его и приложит к себе. Услышал его легкомысленный, и оно не понравилось ему, и он бросил его за себя.
\vs Sir 21:19 Речь глупого~--- как бремя в пути, в устах же разумного находят приятность.
\vs Sir 21:20 Речей разумного будут искать в собрании, и о словах его будут размышлять в сердце.
\vs Sir 21:21 Как разрушенный дом, так мудрость глупому, и знание неразумного~--- бессмысленные слова.
\vs Sir 21:22 Наставление для безумных~--- оковы на ногах и как цепи на правой руке.
\vs Sir 21:23 Глупый в смехе возвышает голос свой, а муж благоразумный едва тихо улыбнется.
\vs Sir 21:24 Как золотой наряд~--- наставление для разумного, и как драгоценное украшение на правой руке.
\vs Sir 21:25 Нога глупого спешит в чужой дом, но человек многоопытный постыдится людей;
\vs Sir 21:26 неразумный сквозь дверь заглядывает в дом, а человек благовоспитанный остановится вне;
\vs Sir 21:27 невежество человека~--- подслушивать у дверей, благоразумный же огорчится таким бесстыдством.
\vs Sir 21:28 Уста многоречивых рассказывают чужое, а слова благоразумных взвешиваются на весах.
\vs Sir 21:29 В устах глупых~--- сердце их, уста же мудрых~--- в сердце их.
\vs Sir 21:30 Когда нечестивый проклинает сатану, то проклинает свою душу.
\vs Sir 21:31 Наушник оскверняет свою душу и будет ненавидим везде, где только жить будет.
\vs Sir 22:1 Грязному камню подобен ленивый: всякий освищет бесславие его.
\vs Sir 22:2 Воловьему помету подобен ленивый: всякий, поднявший его, отряхнет руку.
\vs Sir 22:3 Стыд отцу рождение невоспитанного сына, дочь же \bibemph{невоспитанная} рождается на унижение.
\vs Sir 22:4 Разумная дочь приобретет себе мужа, а бесстыдная~--- печаль родившему.
\vs Sir 22:5 Наглая позорит отца и мужа, и у обоих будет в презрении.
\vs Sir 22:6 Не вовремя рассказ~--- то же, что музыка во время печали; наказание же и учение мудрости прилично всякому времени.
\vs Sir 22:7 Поучающий глупого~--- то же, что склеивающий черепки или пробуждающий спящего от глубокого сна.
\vs Sir 22:8 Рассказывающий что-либо глупому~--- то же, что рассказывающий дремлющему, который по окончании спрашивает: <<что?>>
\vs Sir 22:9 Плачь над умершим, ибо свет исчез для него; плачь и над глупым, ибо разум исчез для него.
\vs Sir 22:10 Меньше плачь над умершим, потому что он успокоился, а злая жизнь глупого~--- хуже смерти.
\vs Sir 22:11 Плачь об умершем~--- семь дней, а о глупом и нечестивом~--- все дни жизни его.
\vs Sir 22:12 С безрассудным много не говори, и к неразумному не ходи;
\vs Sir 22:13 берегись от него, чтобы не иметь неприятности и не замарать себя столкновением с ним;
\vs Sir 22:14 уклонись от него и найдешь покой и не будешь огорчен безумием его.
\vs Sir 22:15 Что тяжелее свинца? и какое имя ему, как не глупый?
\vs Sir 22:16 Легче понести песок и соль и глыбу железа, нежели человека бессмысленного.
\vs Sir 22:17 Как деревянная связь в доме, крепко устроенная, не дает ему распадаться при сотрясении, так сердце, утвержденное на обдуманном совете, не поколеблется во время страха.
\vs Sir 22:18 Сердце, утвержденное на разумном размышлении,~--- как лепное украшение на вытесанной стене.
\vs Sir 22:19 Подпорка, поставленная на высоте, не устоит против ветра:
\vs Sir 22:20 так боязливое сердце, при глупом размышлении, не устоит против страха.
\rsbpar\vs Sir 22:21 Наносящий удар глазу вызывает слезы, а наносящий удар сердцу возбуждает чувство болезненное.
\vs Sir 22:22 Бросающий камень в птиц отгонит их; а поносящий друга расторгнет дружбу.
\vs Sir 22:23 Если ты на друга извлек меч, не отчаивайся, ибо возможно возвращение дружбы.
\vs Sir 22:24 Если ты открыл уста против друга, не бойся, ибо возможно примирение.
\vs Sir 22:25 Только поношение, гордость, обнаружение тайны и коварное злодейство могут отогнать всякого друга.
\vs Sir 22:26 Приобретай доверенность ближнего в нищете его, чтобы радоваться вместе с ним при богатстве его;
\vs Sir 22:27 оставайся с ним во время скорби, чтобы иметь участие в его наследии.
\vs Sir 22:28 Прежде пламени бывает в печи пар и дым: так прежде кровопролития~--- ссоры.
\vs Sir 22:29 Защищать друга я не постыжусь и не скроюсь от лица его;
\vs Sir 22:30 а если приключится мне чрез него зло, то всякий, кто услышит, будет остерегаться его.
\rsbpar\vs Sir 22:31 Кто даст мне стражу к устам моим и печать благоразумия на уста мои, чтобы мне не пасть чрез них и чтобы язык мой не погубил меня!
\vs Sir 23:1 Господи, Отче и Владыко жизни моей! Не оставь меня на волю их и не допусти меня пасть чрез них.
\vs Sir 23:2 Кто приставит бич к помышлениям моим и к сердцу моему наставника в мудрости, чтобы они не щадили проступков моих и не потворствовали заблуждениям их;
\vs Sir 23:3 чтобы не умножались проступки мои и не увеличивались заблуждения мои; чтобы не упасть мне пред противниками, и чтобы не порадовался надо мною враг мой?
\vs Sir 23:4 Господи, Отче и Боже жизни моей! Не дай мне возношения очей и вожделение отврати от меня.
\vs Sir 23:5 Пожелания чрева и сладострастие да не овладеют мною, и не предай меня бесстыдной душе.
\rsbpar\vs Sir 23:6 Выслушайте, дети, наставление для уст: соблюдающий его не будет уловлен своими устами.
\vs Sir 23:7 Уловлен будет ими грешник, и злоречивый и надменный преткнутся чрез них.
\vs Sir 23:8 Не приучай уст твоих к клятве
\vs Sir 23:9 и не обращай в привычку употреблять в клятве имя Святаго.
\vs Sir 23:10 Ибо, как раб, постоянно подвергающийся наказанию, не избавляется от ран, так и клянущийся непрестанно именем Святаго не очистится от греха.
\vs Sir 23:11 Человек, часто клянущийся, исполнится беззакония, и не отступит от дома его бич.
\vs Sir 23:12 Если он согрешит, грех его на нем; и если он вознерадел, то сугубо согрешит;
\vs Sir 23:13 и если он клялся напрасно, то не оправдается, и дом его наполнится несчастьями.
\vs Sir 23:14 Есть речь, облеченная смертью: да не найдется она в наследии Иакова!
\vs Sir 23:15 Ибо от благочестивых все это будет удалено, и они не запутаются во грехах.
\vs Sir 23:16 Не приучай твоих уст к грубой невежливости, ибо при ней бывают греховные слова.
\vs Sir 23:17 Помни об отце и о матери твоей, когда сидишь среди вельмож,
\vs Sir 23:18 чтобы тебе не забыться пред ними и по привычке не сделать глупости, и не пожелать, что лучше бы ты не родился, и не проклясть дня рождения твоего.
\vs Sir 23:19 Человек, привыкающий к бранным словам, во все дни свои не научится.
\vs Sir 23:20 Два качества умножают грехи, а третье навлекает гнев:
\vs Sir 23:21 душа горячая, как пылающий огонь, не угаснет, пока не истощится;
\vs Sir 23:22 человек, блудодействующий в теле плоти своей, не перестанет, пока не прогорит огонь.
\vs Sir 23:23 Блуднику сладок всякий хлеб: он не перестанет, доколе не умрет.
\vs Sir 23:24 Человек, который согрешает против своего ложа, говорит в душе своей: <<кто видит меня?
\vs Sir 23:25 Вокруг меня тьма, и стены закрывают меня, и никто не видит меня: чего мне бояться? Всевышний не воспомянет грехов моих>>.
\vs Sir 23:26 Страх его~--- только глаза человеческие,
\vs Sir 23:27 и не знает он того, что очи Господа в десять тысяч крат светлее солнца
\vs Sir 23:28 и взирают на все пути человеческие, и проникают в места сокровенные.
\vs Sir 23:29 Ему известно было все прежде, нежели сотворено было, равно как и по совершении.
\vs Sir 23:30 Такой \bibemph{человек} будет наказан на улицах города и будет застигнут там, где не думал.
\vs Sir 23:31 Так и жена, оставившая мужа и произведшая наследника от чужого:
\vs Sir 23:32 ибо, во-первых, она не покорилась закону Всевышнего, во-вторых, согрешила против своего мужа и, в-третьих, в блуде прелюбодействовала и произвела детей от чужого мужа.
\vs Sir 23:33 Она будет выведена пред собрание, и о детях ее будет исследование.
\vs Sir 23:34 Дети ее не укоренятся, и ветви ее не дадут плода.
\vs Sir 23:35 Она оставит память о себе на проклятие, и позор ее не изгладится.
\vs Sir 23:36 Оставшиеся познают, что нет ничего лучше страха Господня и нет ничего сладостнее, как внимать заповедям Господним.
\vs Sir 23:37 Великая слава~--- следовать Господу, а быть тебе принятым от Него~--- долгоденствие.
\vs Sir 24:1 Премудрость прославит себя и среди народа своего будет восхвалена.
\vs Sir 24:2 В церкви Всевышнего она откроет уста свои, и пред воинством Его будет прославлять себя:
\vs Sir 24:3 <<я вышла из уст Всевышнего и подобно облаку покрыла землю;
\vs Sir 24:4 я поставила скинию на высоте, и престол мой~--- в столпе облачном;
\vs Sir 24:5 я одна обошла круг небесный и ходила во глубине бездны;
\vs Sir 24:6 в волнах моря и по всей земле и во всяком народе и племени имела я владение:
\vs Sir 24:7 между всеми ими я искала успокоения, и в чьем наследии водвориться мне.
\vs Sir 24:8 Тогда Создатель всех повелел мне, и Произведший меня указал мне покойное жилище и сказал:
\vs Sir 24:9 поселись в Иакове и прими наследие в Израиле.
\vs Sir 24:10 Прежде века от начала Он произвел меня, и я не скончаюсь во веки.
\vs Sir 24:11 Я служила пред Ним во святой скинии и так утвердилась в Сионе.
\vs Sir 24:12 Он дал мне также покой в возлюбленном городе, и в Иерусалиме~--- власть моя.
\vs Sir 24:13 И укоренилась я в прославленном народе, в наследственном уделе Господа.
\vs Sir 24:14 Я возвысилась, как кедр на Ливане и как кипарис на горах Ермонских;
\vs Sir 24:15 я возвысилась, как пальма в Енгадди и как розовые кусты в Иерихоне;
\vs Sir 24:16 я, как красивая маслина в долине и как платан, возвысилась.
\vs Sir 24:17 Как корица и аспалаф, я издала ароматный запах и, как отличная смирна, распространила благоухание,
\vs Sir 24:18 как халвани, оникс и стакти и как благоухание ладана в скинии.
\vs Sir 24:19 Я распростерла свои ветви, как теревинф, и ветви мои~--- ветви славы и благодати.
\vs Sir 24:20 Я~--- как виноградная лоза, произращающая благодать, и цветы мои~--- плод славы и богатства.
\vs Sir 24:21 Приступите ко мне, желающие меня, и насыщайтесь плодами моими;
\vs Sir 24:22 ибо воспоминание обо мне слаще меда и обладание мною приятнее медового сота.
\vs Sir 24:23 Ядущие меня еще будут алкать, и пьющие меня еще будут жаждать.
\vs Sir 24:24 Слушающий меня не постыдится, и трудящиеся со мною не погрешат.
\vs Sir 24:25 Все это~--- книга завета Бога Всевышнего,
\vs Sir 24:26 закон, который заповедал Моисей как наследие сонмам Иаковлевым.
\vs Sir 24:27 Он насыщает мудростью, как Фисон и как Тигр во дни новин;
\vs Sir 24:28 он наполняет разумом, как Евфрат и как Иордан во дни жатвы;
\vs Sir 24:29 он разливает учение, как свет и как Гион во время собирания винограда.
\vs Sir 24:30 Первый человек не достиг полного познания ее; не исследует ее также и последний;
\vs Sir 24:31 ибо мысли ее полнее моря, и намерения ее глубже великой бездны.
\vs Sir 24:32 И я, как канал из реки и как водопровод, вышла в рай.
\vs Sir 24:33 Я сказала: полью мой сад и напою мои гряды.
\vs Sir 24:34 И вот, канал мой сделался рекою, и река моя сделалась морем.
\vs Sir 24:35 И буду я сиять учением, как утренним светом, и далеко проявлю его;
\vs Sir 24:36 и буду я изливать учение, как пророчество, и оставлю его в роды вечные>>.
\vs Sir 24:37 Видите, что я трудился не для себя одного, но для всех, ищущих \bibemph{премудрости}.
\vs Sir 25:1 Тремя я украсилась и стала прекрасною пред Господом и людьми:
\vs Sir 25:2 это~--- единомыслие между братьями и любовь между ближними, и жена и муж, согласно живущие между собою.
\vs Sir 25:3 И три рода \bibemph{людей} возненавидела душа моя, и очень отвратительна для меня жизнь их:
\vs Sir 25:4 надменного нищего, лживого богача и старика-прелюбодея, ослабевающего в рассудке.
\vs Sir 25:5 Чего не собрал ты в юности,~--- как же можешь приобрести в старости твоей?
\vs Sir 25:6 Как прилично сединам судить, и старцам~--- уметь давать совет!
\vs Sir 25:7 Как прекрасна мудрость старцев и как приличны людям почтенным рассудительность и совет!
\vs Sir 25:8 Венец старцев~--- многосторонняя опытность, и хвала их~--- страх Господень.
\vs Sir 25:9 Девять помышлений похвалил я в сердце, а десятое выскажу языком:
\vs Sir 25:10 \bibemph{это} человек, радующийся о детях и при жизни видящий падение врагов.
\vs Sir 25:11 Блажен, кто живет с женою разумною, кто не погрешает языком и не служит недостойному себя.
\vs Sir 25:12 Блажен, кто приобрел мудрость и передает ее в уши слушающих.
\vs Sir 25:13 Как велик тот, кто нашел премудрость! но он не выше того, кто боится Господа.
\vs Sir 25:14 Страх Господень все превосходит, и имеющий его с кем может быть сравнен?
\rsbpar\vs Sir 25:15 \bibemph{Можно перенести} всякую рану, только не рану сердечную, и всякую злость, только не злость женскую,
\vs Sir 25:16 всякое нападение, только не нападение от ненавидящих, и всякое мщение, только не мщение врагов;
\vs Sir 25:17 нет головы ядовитее головы змеиной, и нет ярости сильнее ярости врага.
\vs Sir 25:18 Соглашусь лучше жить со львом и драконом, нежели жить со злою женою.
\vs Sir 25:19 Злость жены изменяет взгляд ее и делает лице ее мрачным, как у медведя.
\vs Sir 25:20 Сядет муж ее среди друзей своих и, услышав \bibemph{о ней}, горько вздохнет.
\vs Sir 25:21 Всякая злость мала в сравнении со злостью жены; жребий грешника да падет на нее.
\vs Sir 25:22 Что восхождение по песку для ног старика, то сварливая жена для тихого мужа.
\vs Sir 25:23 Не засматривайся на красоту женскую и не похотствуй на жену.
\vs Sir 25:24 Досада, стыд и большой срам, когда жена будет преобладать над своим мужем.
\vs Sir 25:25 Сердце унылое и лице печальное и рана сердечная~--- злая жена.
\vs Sir 25:26 Опущенные руки и расслабленные колени~--- жена, которая не счастливит своего мужа.
\vs Sir 25:27 От жены начало греха, и чрез нее все мы умираем.
\vs Sir 25:28 Не давай воде выхода, ни злой жене~--- власти;
\vs Sir 25:29 если она не ходит под рукою твоею, то отсеки ее от плоти твоей.
\vs Sir 26:1 Счастлив муж доброй жены, и число дней его~--- сугубое.
\vs Sir 26:2 Жена добродетельная радует своего мужа и лета его исполнит миром;
\vs Sir 26:3 добрая жена~--- счастливая доля: она дается в удел боящимся Господа;
\vs Sir 26:4 с нею у богатого и бедного~--- сердце довольное и лице во всякое время веселое.
\vs Sir 26:5 Трех страшится сердце мое, а при четвертом я молюсь:
\vs Sir 26:6 городского злословия, возмущения черни и оболгания на смерть,~--- всё это ужасно.
\vs Sir 26:7 Болезнь сердца и печаль~--- жена, ревнивая к \bibemph{другой} жене,
\vs Sir 26:8 и бич языка ее, ко всем приражающийся.
\vs Sir 26:9 Движущееся туда и сюда воловье ярмо~--- злая жена; берущий ее~--- то же, что хватающий скорпиона.
\vs Sir 26:10 Большая досада~--- жена, преданная пьянству, и она не скроет своего срама.
\vs Sir 26:11 Наклонность женщины к блуду узнается по поднятию глаз и век ее.
\vs Sir 26:12 Над бесстыдною дочерью поставь крепкую стражу, чтобы она, улучив послабление, не злоупотребила собою.
\vs Sir 26:13 Берегись бесстыдного глаза, и не удивляйся, если он согрешит против тебя:
\vs Sir 26:14 как томимый жаждою путник открывает уста и пьет всякую близкую воду,
\vs Sir 26:15 так она сядет напротив всякого шатра и пред стрелою откроет колчан.
\vs Sir 26:16 Любезность жены усладит ее мужа, и благоразумие ее утучнит кости его.
\vs Sir 26:17 Кроткая жена~--- дар Господа, и нет цены благовоспитанной душе.
\vs Sir 26:18 Благодать на благодать~--- жена стыдливая,
\vs Sir 26:19 и нет достойной меры для воздержной души.
\vs Sir 26:20 Что солнце, восходящее на высотах Господних,
\vs Sir 26:21 то красота доброй жены в убранстве дома ее;
\vs Sir 26:22 что светильник, сияющий на святом свещнике, то красота лица ее в зрелом возрасте;
\vs Sir 26:23 что золотые столбы на серебряном основании, то прекрасные ноги ее на твердых пятах.
\rsbpar\vs Sir 26:24 От двух скорбело сердце мое, а при третьем возбуждалось во мне негодование:
\vs Sir 26:25 если воин терпит от бедности, и разумные мужи бывают в пренебрежении;
\vs Sir 26:26 и если кто обращается от праведности ко греху, Господь уготовит того на меч.
\vs Sir 26:27 Купец едва может избежать погрешности, а корчемник не спасется от греха.
\vs Sir 27:1 Многие погрешали ради маловажных вещей, и ищущий богатства отвращает глаза.
\vs Sir 27:2 Посреди скреплений камней вбивается гвоздь: так посреди продажи и купли вторгается грех.
\vs Sir 27:3 Если кто не удерживается тщательно в страхе Господнем, то скоро разорится дом его.
\vs Sir 27:4 При трясении решета остается сор: так нечистота человека~--- при рассуждении его.
\vs Sir 27:5 Глиняные сосуды испытываются в печи, а испытание человека~--- в разговоре его.
\vs Sir 27:6 Уход за деревом открывается в плоде его: т\acc{а}к в слове~--- помышления сердца человеческого.
\vs Sir 27:7 Прежде беседы не хвали человека, ибо она есть испытание людей.
\vs Sir 27:8 Если ты усердно будешь искать правды, то найдешь ее и облечешься ею, как подиром славы.
\vs Sir 27:9 Птицы слетаются к подобным себе, и истина обращается к тем, которые упражняются в ней.
\vs Sir 27:10 Как лев подстерегает добычу, так и грехи~--- делающих неправду.
\rsbpar\vs Sir 27:11 Беседа благочестивого~--- всегда мудрость, а безумный изменяется, как луна.
\vs Sir 27:12 Среди неразумных не трать времени, а проводи его постоянно среди благоразумных.
\vs Sir 27:13 Беседа глупых отвратительна, и смех их~--- в забаве грехом.
\vs Sir 27:14 Пустословие много клянущихся поднимет дыбом волосы, а спор их заткнет уши.
\vs Sir 27:15 Ссора надменных~--- кровопролитие, и брань их несносна для слуха.
\vs Sir 27:16 Открывающий тайны потерял доверие и не найдет друга по душе своей.
\vs Sir 27:17 Люби друга и будь верен ему;
\vs Sir 27:18 а если откроешь тайны его, не гонись больше за ним:
\vs Sir 27:19 ибо как человек убивает своего врага, так ты убил дружбу ближнего;
\vs Sir 27:20 и как ты выпустил бы из рук своих птицу, так ты упустил друга и не поймаешь его;
\vs Sir 27:21 не гонись за ним, ибо он далеко ушел и убежал, как серна из сети.
\vs Sir 27:22 Рану можно перевязать, и после ссоры возможно примирение;
\vs Sir 27:23 но кто открыл тайны, тот потерял надежду \bibemph{на примирение}.
\vs Sir 27:24 Кто мигает глазом, тот строит козни, и никто не удержит его от того;
\vs Sir 27:25 пред глазами твоими он будет говорить сладко и будет удивляться словам твоим,
\vs Sir 27:26 а после извратит уста свои и в словах твоих откроет соблазн;
\vs Sir 27:27 многое я ненавижу, но не столько, как его; и Господь возненавидит его.
\rsbpar\vs Sir 27:28 Кто бросает камень вверх, бросает его на свою голову, и коварный удар разделит раны.
\vs Sir 27:29 Кто роет яму, сам упадет в нее, и кто ставит сеть, сам будет уловлен ею.
\vs Sir 27:30 Кто делает зло, на того обратится оно, и он не узн\acc{а}ет, откуда оно пришло к нему;
\vs Sir 27:31 посмеяние и поношение от гордых и мщение, как лев, подстерегут его.
\vs Sir 27:32 Уловлены будут сетью радующиеся о падении благочестивых, и скорбь измождит их прежде смерти их.
\vs Sir 27:33 Злоба и гнев~--- тоже мерзости, и муж грешный будет обладаем ими.
\vs Sir 28:1 Мстительный получит отмщение от Господа, Который не забудет грехов его.
\vs Sir 28:2 Прости ближнему твоему обиду, и тогда по молитве твоей отпустятся грехи твои.
\vs Sir 28:3 Человек питает гнев к человеку, а у Господа просит прощения;
\vs Sir 28:4 к подобному себе человеку не имеет милосердия, и молится о грехах своих;
\vs Sir 28:5 сам, будучи плотию, питает злобу: кто очистит грехи его?
\vs Sir 28:6 Помни последнее и перестань враждовать; помни истление и смерть и соблюдай заповеди;
\vs Sir 28:7 помни заповеди и не злобствуй на ближнего;
\vs Sir 28:8 помни завет Всевышнего и презирай невежество.
\vs Sir 28:9 Удерживайся от ссоры~--- и ты уменьшишь грехи;
\vs Sir 28:10 ибо раздражительный человек возжжет ссору; человек грешник смутит друзей и поселит раздор между живущими в мире.
\vs Sir 28:11 Каково вещество огня, так он и возгорится;
\vs Sir 28:12 и какова сила человека, таков будет и гнев его, и по мере богатства усилится ярость его.
\vs Sir 28:13 Жаркий спор возжигает огонь, а жаркая ссора проливает кровь.
\vs Sir 28:14 Если подуешь на искру, она разгорится, а если плюнешь на нее, угаснет: то и другое выходит из уст твоих.
\rsbpar\vs Sir 28:15 Наушник и двоязычный да будут прокляты, ибо они погубили многих, живших в тишине;
\vs Sir 28:16 язык третий многих поколебал и изгонял их от народа к народу,
\vs Sir 28:17 и разорял укрепленные города и ниспровергал домы вельмож;
\vs Sir 28:18 язык третий изгнал доблестных жен и лишил их трудов их;
\vs Sir 28:19 внимающий ему не найдет покоя и не будет жить в тишине.
\vs Sir 28:20 Удар бича делает рубцы, а удар языка сокрушит кости;
\vs Sir 28:21 многие пали от острия меча, но не столько, сколько павших от языка;
\vs Sir 28:22 счастлив, кто укрылся от него, кто не испытал ярости его, кто не влачил ярма его и не связан был узами его;
\vs Sir 28:23 ибо ярмо его~--- ярмо железное, и узы его~--- узы медные,
\vs Sir 28:24 смерть лютая~--- смерть его, и самый ад лучше его.
\vs Sir 28:25 Не овладеет он благочестивыми, и не сгорят они в пламени его;
\vs Sir 28:26 оставляющие Господа впадут в него; в них возгорится он и не угаснет: он будет послан на них, как лев, и, как барс, будет истреблять их.
\vs Sir 28:27 Смотри, огради владение твое терновником,
\vs Sir 28:28 свяжи серебро твое и золото,
\vs Sir 28:29 и для слов твоих сделай вес и меру, и для уст твоих~--- дверь и запор.
\vs Sir 28:30 Берегись, чтобы не споткнуться ими и не пасть пред злоумышляющим.
\vs Sir 29:1 Кто оказывает милость, тот дает взаем ближнему, и кто поддерживает его своею рукою, тот соблюдает заповеди.
\vs Sir 29:2 Давай взаймы ближнему во время нужды его и сам в свое время возвращай ближнему.
\vs Sir 29:3 Твердо держи слово и будь верен ему~--- и ты во всякое время найдешь нужное для тебя.
\vs Sir 29:4 Многие считали заем находкою и причинили огорчение тем, которые помогли им.
\vs Sir 29:5 Доколе не получит, он будет целовать руку его и из-за денег ближнего смирит голос;
\vs Sir 29:6 а в срок отдачи он будет протягивать время и будет отвечать уныло и жаловаться на время.
\vs Sir 29:7 Если он будет в состоянии, то едва половину принесет~--- и это вменит ему в находку;
\vs Sir 29:8 а если будет не в состоянии, то заимодавец лишился своих денег и без причины приобрел себе врага в нем:
\vs Sir 29:9 он воздаст ему проклятиями и бранью и вместо почтения воздаст бесчестием.
\vs Sir 29:10 Многие по причине такого лукавства уклоняются \bibemph{от ссуды}, опасаясь напрасно потерпеть утрату.
\vs Sir 29:11 Но к бедному ты будь снисходителен и милостынею ему не медли;
\vs Sir 29:12 ради заповеди помоги бедному и в нужде его не отпускай его ни с чем.
\vs Sir 29:13 Трать серебро для брата и друга и не давай ему заржаветь под камнем на погибель;
\vs Sir 29:14 располагай сокровищем твоим по заповедям Всевышнего, и оно принесет тебе более пользы, нежели золото;
\vs Sir 29:15 заключи в кладовых твоих милостыню, и она избавит тебя от всякого несчастья:
\vs Sir 29:16 лучше крепкого щита и твердого копья она защитит тебя против врага.
\vs Sir 29:17 Добрый человек поручится за ближнего, а потерявший стыд оставит его.
\vs Sir 29:18 Не забывай благодеяний поручителя; ибо он дал душу свою за тебя.
\vs Sir 29:19 Грешник расстроит состояние поручителя, и неблагодарный в душе оставит своего избавителя.
\vs Sir 29:20 Поручительство привело в разорение многих достаточных людей и пошатнуло их, как волна морская;
\vs Sir 29:21 мужей могущественных изгнало из домов, и они блуждали между чужими народами.
\vs Sir 29:22 Грешник, принимающий на себя поручительство и ищущий корысти, впадет в тяжбу.
\vs Sir 29:23 Помогай ближнему по силе твоей и берегись, чтобы тебе не впасть \bibemph{в то же}.
\vs Sir 29:24 Главная потребность для жизни~--- вода и хлеб, и одежда и дом, прикрывающий наготу.
\vs Sir 29:25 Лучше жизнь бедного под дощатым кровом, нежели роскошные пиршества в чужих \bibemph{домах}.
\vs Sir 29:26 Будь доволен малым, как и многим.
\vs Sir 29:27 Худая жизнь~--- \bibemph{скитаться} из дома в дом, и где водворишься, не посмеешь и рта открыть;
\vs Sir 29:28 будешь подавать пищу и питье без благодарности, да и сверх того еще услышишь горькое:
\vs Sir 29:29 <<пойди сюда, пришлец, приготовь стол и, если есть что у тебя, накорми меня>>;
\vs Sir 29:30 <<удались, пришлец, ради почетного лица: брат пришел ко мне в гости, дом нужен>>.
\vs Sir 29:31 Тяжел для человека с чувством упрек за приют в доме и порицание за одолжение.
\vs Sir 30:1 Кто любит своего сына, тот пусть чаще наказывает его, чтобы впоследствии утешаться им.
\vs Sir 30:2 Кто наставляет своего сына, тот будет иметь помощь от него и среди знакомых будет хвалиться им.
\vs Sir 30:3 Кто учит своего сына, тот возбуждает зависть во враге, а пред друзьями будет радоваться о нем.
\vs Sir 30:4 Умер отец его~--- и как будто не умирал, ибо оставил по себе подобного себе;
\vs Sir 30:5 при жизни своей он смотрел на него и утешался, и при смерти своей не опечалился;
\vs Sir 30:6 для врагов он оставил в нем мстителя, а для друзей~--- воздающего благодарность.
\vs Sir 30:7 Поблажающий сыну будет перевязывать раны его, и при всяком крике его будет тревожиться сердце его.
\vs Sir 30:8 Необъезженный конь бывает упрям, а сын, оставленный на свою волю, делается дерзким.
\vs Sir 30:9 Лелей дитя, и оно устрашит тебя; играй с ним, и оно опечалит тебя.
\vs Sir 30:10 Не смейся с ним, чтобы не горевать с ним и после не скрежетать зубами своими.
\vs Sir 30:11 Не давай ему воли в юности и не потворствуй неразумию его.
\vs Sir 30:12 Нагибай выю его в юности и сокрушай рёбра его, доколе оно молодо, дабы, сделавшись упорным, оно не вышло из повиновения тебе.
\vs Sir 30:13 Учи сына твоего и трудись над ним, чтобы не иметь тебе огорчения от непристойных поступков его.
\rsbpar\vs Sir 30:14 Лучше бедняк здоровый и крепкий силами, нежели богач с изможденным телом;
\vs Sir 30:15 здоровье и благосостояние тела дороже всякого золота, и крепкое тело лучше несметного богатства;
\vs Sir 30:16 нет богатства лучше телесного здоровья, и нет радости выше радости сердечной;
\vs Sir 30:17 лучше смерть, нежели горестная жизнь или постоянно продолжающаяся болезнь.
\vs Sir 30:18 Сласти, поднесенные к сомкнутым устам, то же, что снеди, поставленные на могиле.
\vs Sir 30:19 Какая польза идолу от жертвы? он ни есть, ни обонять не может:
\vs Sir 30:20 так преследуемый от Господа,
\vs Sir 30:21 смотря глазами и стеная, подобен евнуху, который обнимает девицу и вздыхает.
\vs Sir 30:22 Не предавайся печали душею твоею и не мучь себя своею мнительностью;
\vs Sir 30:23 веселье сердца~--- жизнь человека, и радость мужа~--- долгоденствие;
\vs Sir 30:24 люби душу твою и утешай сердце твое и удаляй от себя печаль,
\vs Sir 30:25 ибо печаль многих убила, а пользы в ней нет.
\vs Sir 30:26 Ревность и гнев сокращают дни, а забота~--- прежде времени приводит старость.
\vs Sir 30:27 Открытое и доброе сердце заботится и о снедях своих.
\vs Sir 31:1 Бдительность над богатством изнуряет тело, и забота о нем отгоняет сон.
\vs Sir 31:2 Бдительная забота не дает дремать, и тяжкая болезнь отнимает сон.
\vs Sir 31:3 Потрудился богатый при умножении имуществ~--- и в покое насыщается своими благами.
\vs Sir 31:4 Потрудился бедный при недостатках в жизни~--- и в покое остается скудным.
\vs Sir 31:5 Любящий золото не будет прав, и кто гоняется за тлением, наполнится им.
\vs Sir 31:6 Многие ради золота подверглись падению, и погибель их была пред лицем их;
\vs Sir 31:7 оно~--- дерево преткновения для приносящих ему жертвы, и всякий несмысленный будет уловлен им.
\vs Sir 31:8 Счастлив богач, который оказался безукоризненным и который не гонялся за золотом.
\vs Sir 31:9 Кто он? и мы прославим его; ибо он сделал чудо в народе своем.
\vs Sir 31:10 Кто был искушаем \bibemph{золотом}~--- и остался непорочным? Да будет это в похвалу ему.
\vs Sir 31:11 Кто мог погрешить~--- и не погрешил, сделать зло~--- и не сделал?
\vs Sir 31:12 Прочно будет богатство его, и о милостынях его будет возвещать собрание.
\rsbpar\vs Sir 31:13 Когда ты сядешь за богатый стол, не раскрывай на него гортани твоей
\vs Sir 31:14 и не говори: <<много же на нем!>> Помни, что алчный глаз~--- злая вещь.
\vs Sir 31:15 Что из сотворенного завистливее глаза? Потому он плачет о всем, что видит.
\vs Sir 31:16 Куда он посмотрит, не протягивай руки, и не сталкивайся с ним в блюде.
\vs Sir 31:17 Суди о ближнем по себе и о всяком действии рассуждай.
\vs Sir 31:18 Ешь, как человек, что тебе предложено, и не пресыщайся, чтобы не возненавидели тебя;
\vs Sir 31:19 переставай \bibemph{есть} первый из вежливости и не будь алчен, чтобы не послужить соблазном;
\vs Sir 31:20 и если ты сядешь посреди многих, то не протягивай руки твоей прежде них.
\vs Sir 31:21 Немногим довольствуется человек благовоспитанный, и потому он не страдает одышкою на своем ложе.
\vs Sir 31:22 Здоровый сон бывает при умеренности желудка: встал рано, и душа его с ним;
\vs Sir 31:23 страдание бессонницею и холера и резь в животе бывают у человека ненасытного.
\vs Sir 31:24 Если ты обременил себя яствами, то встань из-за стола и отдохни.
\vs Sir 31:25 Послушай меня, сын мой, и не пренебреги мною, и впоследствии ты поймешь слова мои.
\vs Sir 31:26 Во всех делах твоих будь осмотрителен, и никакая болезнь не приключится тебе.
\vs Sir 31:27 Щедрого на хлебы будут благословлять уста, и свидетельство о доброте его~--- верно;
\vs Sir 31:28 против скупого на хлеб будет роптать город, и свидетельство о скупости его~--- справедливо.
\vs Sir 31:29 Против вина не показывай себя храбрым, ибо многих погубило вино.
\vs Sir 31:30 Печь испытывает крепость лезвия закалкою; так вино испытывает сердца гордых~--- пьянством.
\vs Sir 31:31 Вино полезно для жизни человека, если будешь пить его умеренно.
\vs Sir 31:32 Что за жизнь без вина? оно сотворено на веселие людям.
\vs Sir 31:33 Отрада сердцу и утешение душе~--- вино, умеренно употребляемое вовремя;
\vs Sir 31:34 горесть для души~--- вино, когда пьют его много, при раздражении и ссоре.
\vs Sir 31:35 Излишнее употребление вина увеличивает ярость неразумного до преткновения, умаляя крепость его и причиняя раны.
\vs Sir 31:36 На пиру за вином не упрекай ближнего и не унижай его во время его веселья;
\vs Sir 31:37 не говори ему оскорбительных слов и не обременяй его требованиями.
\vs Sir 32:1 Если поставили тебя старшим \bibemph{на пиру}, не возносись; будь между другими как один из них:
\vs Sir 32:2 позаботься о них и потом садись. И когда всё твое дело исполнишь, тогда займи твое место,
\vs Sir 32:3 чтобы порадоваться на них и за хорошее распоряжение получить венок.
\vs Sir 32:4 Разговор веди ты, старший,~--- ибо это прилично тебе,~---
\vs Sir 32:5 с основательным знанием, и не возбраняй музыки.
\vs Sir 32:6 Когда слушают, не размножай разговора и безвременно не мудрствуй.
\vs Sir 32:7 Что рубиновая печать в золотом украшении, то благозвучие музыки в пиру за вином;
\vs Sir 32:8 что смарагдовая печать в золотой оправе, то приятность песней за вкусным вином.
\vs Sir 32:9 Говори, юноша, если нужно тебе, едва слова два, когда будешь спрошен,
\vs Sir 32:10 говори главное, многое в немногих словах. Будь как знающий и, вместе, как умеющий молчать.
\vs Sir 32:11 Среди вельмож не равняйся с ними, и, когда говорит другой, ты много не говори.
\vs Sir 32:12 Грому предшествует молния, а стыдливого предваряет благорасположение.
\vs Sir 32:13 Вставай вовремя и не будь последним; поспешай домой и не останавливайся.
\vs Sir 32:14 Там забавляйся и делай, что тебе нравится; но не согрешай гордым словом.
\vs Sir 32:15 И за это благословляй Сотворившего тебя и Насыщающего тебя Своими благами.
\rsbpar\vs Sir 32:16 Боящийся Господа примет наставление, и с раннего утра обращающиеся к Нему приобретут благоволение Его.
\vs Sir 32:17 Ищущий закона насытится им, а лицемер преткнется в нем.
\vs Sir 32:18 Боящиеся Господа найдут суд и, как свет, возжгут правосудие.
\vs Sir 32:19 Человек грешный уклоняется от обличения и находит извинение, согласно желанию своему.
\vs Sir 32:20 Человек рассудительный не пренебрегает размышлением, а безрассудный и гордый не содрогается от страха и после того, как сделал что-либо без размышления.
\vs Sir 32:21 Без рассуждения не делай ничего, и когда сделаешь, не раскаивайся.
\vs Sir 32:22 Не ходи по пути, где развалины, чтобы не споткнуться о камень;
\vs Sir 32:23 не полагайся и на ровный путь; остерегайся даже детей твоих.
\vs Sir 32:24 Во всяком деле верь душе твоей: и это есть соблюдение заповедей.
\vs Sir 32:25 Верующий закону внимателен к заповедям, и надеющийся на Господа не потерпит вреда.
\vs Sir 33:1 Боящемуся Господа не приключится зла, но и в искушении Он избавит его.
\vs Sir 33:2 Мудрый муж не возненавидит закона, а притворно держащийся его~--- как корабль в бурю.
\vs Sir 33:3 Разумный человек верит закону, и закон для него верен, как ответ урима.
\vs Sir 33:4 Приготовь слово~--- и будешь выслушан; собери наставления~--- и отвечай.
\vs Sir 33:5 Колесо в колеснице~--- сердце глупого, и как вертящаяся ось~--- мысль его.
\vs Sir 33:6 Насмешливый друг то же, что ярый конь, который под всяким седоком ржет.
\rsbpar\vs Sir 33:7 Почему один день лучше другого, тогда как каждый дневной свет в году \bibemph{исходит} от солнца?
\vs Sir 33:8 Они разделены премудростью Господа; Он отличил времена и празднества:
\vs Sir 33:9 некоторые из них Он возвысил и освятил, а прочие положил в числе обыкновенных дней.
\vs Sir 33:10 И все люди из праха, и Адам был создан из земли;
\vs Sir 33:11 но по всеведению Своему Господь положил различие между ними и назначил им разные пути:
\vs Sir 33:12 одних из них благословил и возвысил, других освятил и приблизил к Себе, а иных проклял и унизил и сдвинул с места их.
\vs Sir 33:13 Как глина у горшечника в руке его и все судьбы ее в его произволе, так люди~--- в руке Сотворившего их, и Он воздает им по суду Своему.
\vs Sir 33:14 Как напротив зла~--- добро и напротив смерти~--- жизнь, так напротив благочестивого~--- грешник. Так смотри и на все дела Всевышнего: их по два, одно напротив другого.
\vs Sir 33:15 И я последний бодрственно потрудился, как подбиравший позади собирателей винограда,
\vs Sir 33:16 и по благословению Господа успел и наполнил точило, как собиратель винограда.
\vs Sir 33:17 Поймите, что я трудился не для себя одного, но для всех ищущих наставления.
\rsbpar\vs Sir 33:18 Послушайте меня, князья народа, и внимайте, начальники собрания:
\vs Sir 33:19 ни сыну, ни жене, ни брату, ни другу не давай власти над тобою при жизни твоей;
\vs Sir 33:20 и не отдавай другому имения твоего, чтобы, раскаявшись, не умолять о нем.
\vs Sir 33:21 Доколе ты жив и дыхание в тебе, не заменяй себя никем;
\vs Sir 33:22 ибо лучше, чтобы дети просили тебя, нежели тебе смотреть в руки сыновей твоих.
\vs Sir 33:23 Во всех делах твоих будь главным, и не клади пятна на честь твою.
\vs Sir 33:24 При скончании дней жизни твоей и при смерти передай наследство.
\vs Sir 33:25 Корм, палка и бремя~--- для осла; хлеб, наказание и дело~--- для раба.
\vs Sir 33:26 Занимай раба работою~--- и будешь иметь покой; ослабь руки ему~--- и он будет искать свободы.
\vs Sir 33:27 Ярмо и ремень согнут выю \bibemph{вола}, а для лукавого раба~--- узы и раны;
\vs Sir 33:28 употребляй его на работу, чтобы он не оставался в праздности, ибо праздность научила многому худому;
\vs Sir 33:29 приставь его к делу, как ему следует, и если он не будет повиноваться, наложи на него тяжкие оковы.
\vs Sir 33:30 Но ни на кого не налагай лишнего и ничего не делай без рассуждения.
\vs Sir 33:31 Если есть у тебя раб, то да будет он как ты, ибо ты приобрел его кровью;
\vs Sir 33:32 если есть у тебя раб, то поступай с ним, как с братом, ибо ты будешь нуждаться в нем, как в душе твоей;
\vs Sir 33:33 если ты будешь обижать его, и он встанет и убежит от тебя, то на какой дороге ты будешь искать его?
\vs Sir 34:1 Пустые и ложные надежды~--- у человека безрассудного, и сонные грезы окрыляют глупых.
\vs Sir 34:2 Как обнимающий тень или гонящийся за ветром, так верящий сновидениям.
\vs Sir 34:3 Сновидения совершенно то же, что подобие лица против лица.
\vs Sir 34:4 От нечистого что может быть чистого, и от ложного что может быть истинного?
\vs Sir 34:5 Гадания и приметы и сновидения~--- суета, и сердце наполняется мечтами, как у рождающей.
\vs Sir 34:6 Если они не будут посланы от Всевышнего для вразумления, не прилагай к ним сердца твоего.
\vs Sir 34:7 Сновидения ввели многих в заблуждение, и надеявшиеся на них подверглись падению.
\vs Sir 34:8 Закон исполняется без обмана, и мудрость в устах верных совершается.
\vs Sir 34:9 Человек ученый знает много, и многоопытный выскажет знание.
\vs Sir 34:10 Кто не имел опытов, тот мало знает; а кто странствовал, тот умножил знание.
\vs Sir 34:11 Многое я видел в моем странствовании, и я знаю больше, нежели сколько говорю.
\vs Sir 34:12 Много раз был я в опасности смерти, и спасался при помощи \bibemph{опыта}.
\vs Sir 34:13 Дух боящихся Господа поживет, ибо надежда их~--- на Спасающего их.
\vs Sir 34:14 Боящийся Господа ничего не устрашится и не убоится, ибо Он~--- надежда его.
\vs Sir 34:15 Блаженна душа боящегося Господа! кем он держится, и кто опора его?
\vs Sir 34:16 Очи Господа~--- на любящих Его. Он~--- могущественная защита и крепкая опора, покров от зноя и покров от полуденного жара, охранение от преткновения и защита от падения;
\vs Sir 34:17 Он возвышает душу и просвещает очи, дает врачевство, жизнь и благословение.
\rsbpar\vs Sir 34:18 Кто приносит жертву от неправедного \bibemph{стяжания}, того приношение насмешливое, и дары беззаконных неблагоугодны;
\vs Sir 34:19 не благоволит Всевышний к приношениям нечестивых и множеством жертв не умилостивляется о грехах их.
\vs Sir 34:20 Что заколающий на жертву сына пред отцем его, то приносящий жертву из имения бедных.
\vs Sir 34:21 Хлеб нуждающихся есть жизнь бедных: отнимающий его есть кровопийца.
\vs Sir 34:22 Убивает ближнего, кто отнимает у него пропитание, и проливает кровь, кто лишает наемника платы.
\vs Sir 34:23 Когда один строит, а другой разрушает, то что они получат для себя кроме утомления?
\vs Sir 34:24 Когда один молится, а другой проклинает, чей голос услышит Владыка?
\vs Sir 34:25 Когда кто омывается от осквернения мертвым и опять прикасается к нему, какая польза от его омовения?
\vs Sir 34:26 Так человек, который постится за грехи свои и опять идет и делает то же самое: кто услышит молитву его? и какую пользу получит он оттого, что смирялся?
\vs Sir 35:1 Кто соблюдает закон, тот умножает приношения; кто держится заповедей, тот приносит жертву спасения.
\vs Sir 35:2 Кто воздает благодарность, тот приносит семидал; а подающий милостыню приносит жертву хвалы.
\vs Sir 35:3 Благоугождение Господу~--- отступление от зла, и умилостивление \bibemph{Его}~--- уклонение от неправды.
\vs Sir 35:4 Не являйся пред лице Господа с пустыми руками, ибо всё это~--- по заповеди.
\vs Sir 35:5 Приношение праведного утучняет алтарь, и благоухание его~--- пред Всевышним;
\vs Sir 35:6 жертва праведного мужа благоприятна, и память о ней незабвенна будет.
\vs Sir 35:7 С веселым оком прославляй Господа и не умаляй начатков трудов твоих;
\vs Sir 35:8 при всяком даре имей лице веселое и в радости посвящай десятину.
\vs Sir 35:9 Давай Всевышнему по даянию Его, и с веселым оком~--- по мере приобретения рукою твоею,
\vs Sir 35:10 ибо Господь есть воздаятель и воздаст тебе всемеро.
\vs Sir 35:11 Не уменьшай даров, ибо Он не примет их: и не надейся на неправедную жертву,
\vs Sir 35:12 ибо Господь есть судия, и нет у Него лицеприятия:
\vs Sir 35:13 Он не уважит лица пред бедным и молитву обиженного услышит;
\vs Sir 35:14 Он не презрит моления сироты, ни вдовы, когда она будет изливать прошение \bibemph{свое}.
\vs Sir 35:15 Не слезы ли вдовы льются по щекам, и не вопиет ли она против того, кто вынуждает их?
\vs Sir 35:16 Служащий \bibemph{Богу} будет принят с благоволением, и молитва его дойдет до облаков.
\vs Sir 35:17 Молитва смиренного проникнет сквозь облака, и он не утешится, доколе она не приблизится \bibemph{к Богу},
\vs Sir 35:18 и не отступит, доколе Всевышний не призрит и не рассудит справедливо и не произнесет решения.
\vs Sir 35:19 И Господь не замедлит и не потерпит, доколе не сокрушит чресл немилосердых;
\vs Sir 35:20 Он будет воздавать отмщение и народам, доколе не истребит сонма притеснителей и не сокрушит скипетров неправедных,
\vs Sir 35:21 доколе не воздаст человеку по делам его, и за дела людей~--- по намерениям их,
\vs Sir 35:22 доколе не совершит суда над народом Своим и не обрадует их Своею милостью.
\vs Sir 35:23 Благовременна милость во время скорби, как дождевые облака во время засухи.
\vs Sir 36:1 Помилуй нас, Владыко, Боже всех, и призри,
\vs Sir 36:2 и наведи на все народы страх Твой.
\vs Sir 36:3 Воздвигни руку Твою на чужие народы, и да позн\acc{а}ют они могущество Твое.
\vs Sir 36:4 Как пред ними Ты явил святость Твою в нас, так пред нами яви величие Твое в них,~---
\vs Sir 36:5 и да познают они Тебя, как мы познали, что нет Бога, кроме Тебя, Господи.
\vs Sir 36:6 Возобнови знамения и сотвори новые чудеса;
\vs Sir 36:7 прославь руку и правую мышцу \bibemph{Твою}; воздвигни ярость и пролей гнев;
\vs Sir 36:8 истреби противника и уничтожь врага;
\vs Sir 36:9 ускори время и вспомни клятву, и да возвестят о великих делах Твоих.
\vs Sir 36:10 Яростью огня да будет истреблен убегающий \bibemph{от меча}, и угнетающие народ Твой да найдут погибель.
\vs Sir 36:11 Сокруши головы начальников вражеских, которые говорят: <<никого нет, кроме нас!>>
\vs Sir 36:12 Собери все колена Иакова и соделай их наследием Твоим, как было сначала.
\vs Sir 36:13 Помилуй, Господи, народ, названный по имени Твоему, и Израиля, которого Ты нарек первенцем.
\vs Sir 36:14 Умилосердись над городом святыни Твоей, над Иерусалимом, местом покоя Твоего.
\vs Sir 36:15 Наполни Сион хвалою обетований Твоих, и Твоею славою~--- народ Твой.
\vs Sir 36:16 Даруй свидетельство тем, которые от начала были достоянием Твоим, и воздвигни пророчества от имени Твоего.
\vs Sir 36:17 Даруй награду надеющимся на Тебя, и да веруют пророкам Твоим.
\vs Sir 36:18 Услышь, Господи, молитву рабов Твоих, по благословению Аарона, о народе Твоем,~---
\vs Sir 36:19 и познают все живущие на земле, что Ты~--- Господь, Бог веков.
\rsbpar\vs Sir 36:20 Желудок принимает в себя всякую пищу, но пища пищи лучше:
\vs Sir 36:21 гортань отличает пищу из дичи, так разумное сердце~--- слова ложные.
\vs Sir 36:22 Лукавое сердце причинит печаль, но человек многоопытный воздаст ему.
\vs Sir 36:23 Женщина примет всякого мужа, но девица девицы лучше:
\vs Sir 36:24 красота жены веселит лице и всего вожделеннее для мужа;
\vs Sir 36:25 если есть на языке ее приветливость и кротость, то муж ее выходит из ряда сынов человеческих.
\vs Sir 36:26 Приобретающий жену полагает начало стяжанию, приобретает соответственно ему помощника, опору спокойствия его.
\vs Sir 36:27 Где нет ограды, \bibemph{там} расхитится имение; а у кого нет жены, тот будет вздыхать скитаясь:
\vs Sir 36:28 ибо кто поверит вооруженному разбойнику, скитающемуся из города в город?
\vs Sir 36:29 Так и человеку, не имеющему оседлости и останавливающемуся для ночлега там, где он запоздает.
\vs Sir 37:1 Всякий друг может сказать: <<и я подружился с ним>>. Но бывает друг по имени только другом.
\vs Sir 37:2 Не есть ли это скорбь до смерти, когда приятель и друг обращается во врага?
\vs Sir 37:3 О, злая мысль! откуда вторглась ты, чтобы покрыть землю коварством?
\vs Sir 37:4 Приятель радуется при веселии друга, а во время скорби его будет против него.
\vs Sir 37:5 Приятель помогает другу в трудах его ради чрева, а в случае войны возьмется за щит.
\vs Sir 37:6 Не забывай друга в душе твоей и не забывай его в имении твоем.
\vs Sir 37:7 Всякий советник хвалит \bibemph{свой} совет, но иной советует в свою пользу;
\vs Sir 37:8 от советника охраняй душу твою и наперед узнай, что ему нужно; ибо, может быть, он будет советовать для самого себя;
\vs Sir 37:9 может быть, он бросит на тебя жребий и скажет тебе: <<путь твой хорош>>; а сам станет напротив тебя, чтобы посмотреть, что случится с тобою.
\vs Sir 37:10 Не советуйся с недоброжелателем твоим и от завистников твоих скрывай намерения.
\vs Sir 37:11 Не советуйся с женою о сопернице ее и с боязливым~--- о войне, с продавцом~--- о мене, с покупщиком~--- о продаже, с завистливым~--- о благодарности,
\vs Sir 37:12 с немилосердым~--- о благотворительности, с ленивым~--- о всяком деле,
\vs Sir 37:13 с годовым наемником~--- об окончании работы, с ленивым рабом~--- о большой работе:
\vs Sir 37:14 не полагайся на таких ни при каком совещании,
\vs Sir 37:15 но обращайся всегда только с мужем благочестивым, о котором узн\acc{а}ешь, что он соблюдает заповеди Господни,
\vs Sir 37:16 который своею душею~--- по душе тебе и, в случае падения твоего, поскорбит вместе с тобою.
\vs Sir 37:17 Держись совета сердца твоего, ибо нет никого для тебя вернее его;
\vs Sir 37:18 душа человека иногда более скажет, нежели семь наблюдателей, сидящих на высоком месте для наблюдения.
\vs Sir 37:19 Но при всем этом молись Всевышнему, чтобы Он управил путь твой в истине.
\rsbpar\vs Sir 37:20 Начало всякого дела~--- размышление, а прежде всякого действия~--- совет.
\vs Sir 37:21 Выражение сердечного изменения~--- лице. Четыре состояния выражаются на нем: добро и зло, жизнь и смерть, а господствует всегда язык.
\vs Sir 37:22 Иной человек искусен и многих учит, а для своей души бесполезен.
\vs Sir 37:23 Иной ухищряется в речах, а \bibemph{бывает} ненавистен,~--- такой останется без всякого пропитания;
\vs Sir 37:24 ибо не дана ему от Господа благодать, и он лишен всякой мудрости.
\vs Sir 37:25 Иной мудр для души своей, и плоды знания на устах его верны.
\vs Sir 37:26 Мудрый муж поучает народ свой, и плоды знания его верны.
\vs Sir 37:27 Мудрый муж будет изобиловать благословением, и все видящие его будут называть его блаженным.
\vs Sir 37:28 Жизнь человека определяется числом дней, а дни Израиля бесчисленны.
\vs Sir 37:29 Мудрый приобретет доверие у своего народа, и имя его будет жить вовек.
\rsbpar\vs Sir 37:30 Сын мой! в продолжение жизни испытывай твою душу и наблюдай, что для нее вредно, и не давай ей того;
\vs Sir 37:31 ибо не всё полезно для всех, и не всякая душа ко всему расположена.
\vs Sir 37:32 Не пресыщайся всякою сластью и не бросайся на разные снеди,
\vs Sir 37:33 ибо от многоядения бывает болезнь, и пресыщение доводит до холеры;
\vs Sir 37:34 от пресыщения многие умерли, а воздержный прибавит себе жизни.
\vs Sir 38:1 Почитай врача честью по надобности в нем, ибо Господь создал его,
\vs Sir 38:2 и от Вышнего~--- врачевание, и от царя получает он дар.
\vs Sir 38:3 Знание врача возвысит его голову, и между вельможами он будет в почете.
\vs Sir 38:4 Господь создал из земли врачевства, и благоразумный человек не будет пренебрегать ими.
\vs Sir 38:5 Не от дерева ли вода сделалась сладкою, чтобы познана была сила Его?
\vs Sir 38:6 Для того Он и дал людям знание, чтобы прославляли Его в чудных делах Его:
\vs Sir 38:7 ими он врачует \bibemph{человека} и уничтожает болезнь его.
\vs Sir 38:8 Приготовляющий лекарства делает из них смесь, и занятия его не оканчиваются, и чрез него бывает благо на лице земли.
\vs Sir 38:9 Сын мой! в болезни твоей не будь небрежен, но молись Господу, и Он исцелит тебя.
\vs Sir 38:10 Оставь греховную жизнь и исправь руки твои, и от всякого греха очисти сердце.
\vs Sir 38:11 Вознеси благоухание и из семидала памятную жертву и сделай приношение тучное, как бы уже умирающий;
\vs Sir 38:12 и дай место врачу, ибо и его создал Господь, и да не удаляется он от тебя, ибо он нужен.
\vs Sir 38:13 В иное время и в их руках бывает успех;
\vs Sir 38:14 ибо и они молятся Господу, чтобы Он помог им подать \bibemph{больному} облегчение и исцеление к продолжению жизни.
\vs Sir 38:15 Но кто согрешает пред Сотворившим его, да впадет в руки врача!
\vs Sir 38:16 Сын мой! над умершим пролей слезы и, как бы подвергшийся жестокому несчастию, начни плач; прилично облеки тело его и не пренебреги погребением его;
\vs Sir 38:17 горький да будет плач и рыдание теплое, и продолжи сетование о нем, по достоинству его, день или два, для избежания осуждения, и тогда утешься от печали;
\vs Sir 38:18 ибо от печали бывает смерть, и печаль сердечная истощит силу.
\vs Sir 38:19 С несчастьем пребывает и печаль, и жизнь нищего тяжела для сердца.
\vs Sir 38:20 Не предавай сердца твоего печали; отдаляй ее \bibemph{от себя}, вспоминая о конце.
\vs Sir 38:21 Не забывай о сем, ибо нет возвращения; и ему ты не принесешь пользы, а себе повредишь.
\vs Sir 38:22 <<Вспоминай о приговоре надо мною, потому что он также и над тобою; мне вчера, а тебе сегодня>>.
\vs Sir 38:23 С упокоением умершего успокой и память о нем, и утешься о нем по исходе души его.
\rsbpar\vs Sir 38:24 Мудрость книжная приобретается в благоприятное время досуга, и кто мало имеет своих занятий, может приобрести мудрость.
\vs Sir 38:25 Как может сделаться мудрым тот, кто правит плугом и хвалится бичом, гоняет волов и занят работами их, и которого разговор \bibemph{только} о молодых волах?
\vs Sir 38:26 Сердце его занято тем, чтобы проводить борозды, и забота его~--- о корме для телиц.
\vs Sir 38:27 Так и всякий плотник и зодчий, который проводит ночь, как день: кто занимается резьбою, того прилежание в том, чтобы оразнообразить форму;
\vs Sir 38:28 сердце свое он устремляет на то, чтобы изображение было похоже, и забота его~--- о том, чтоб окончить дело в совершенстве.
\vs Sir 38:29 Так и ковач, который сидит у наковальни и думает об изделии из железа: дым от огня изнуряет его тело, и с жаром от печи борется он;
\vs Sir 38:30 звук молота оглушает его слух, и глаза его устремлены на модель сосуда;
\vs Sir 38:31 сердце его устремлено на окончание дела, и попечение его~--- о том, чтобы отделать его в совершенстве.
\vs Sir 38:32 Так и горшечник, который сидит над своим делом и ногами своими вертит колесо,
\vs Sir 38:33 который постоянно в заботе о деле своем и у которого исчислена вся работа его:
\vs Sir 38:34 рукою своею он дает форму глине, а ногами умягчает ее жесткость;
\vs Sir 38:35 он устремляет сердце к тому, чтобы хорошо окончить сосуд, и забота его~--- о том, чтоб очистить печь.
\vs Sir 38:36 Все они надеются на свои руки, и каждый умудряется в своем деле;
\vs Sir 38:37 без них ни город не построится, ни жители не населятся и не будут жить в нем;
\vs Sir 38:38 и однако ж они в собрание не приглашаются, на судейском седалище не сидят и не рассуждают о судебных постановлениях, не произносят оправдания и осуждения и не занимаются притчами;
\vs Sir 38:39 но поддерживают быт житейский, и молитва их~--- об успехе художества их.
\vs Sir 39:1 Только тот, кто посвящает свою душу размышлению о законе Всевышнего, будет искать мудрости всех древних и упражняться в пророчествах:
\vs Sir 39:2 он будет замечать сказания мужей именитых и углубляться в тонкие обороты притчей;
\vs Sir 39:3 будет исследовать сокровенный смысл изречений и заниматься загадками притчей.
\vs Sir 39:4 Он будет проходить служение среди вельмож и являться пред правителем;
\vs Sir 39:5 будет путешествовать по земле чужих народов, ибо испытал доброе и злое между людьми.
\vs Sir 39:6 Сердце свое он направит к тому, чтобы с раннего утра обращаться к Господу, сотворившему его, и будет молиться пред Всевышним; откроет в молитве уста свои и будет молиться о грехах своих.
\vs Sir 39:7 Если Господу великому угодно будет, он исполнится духом разума,
\vs Sir 39:8 будет источать слова мудрости своей и в молитве прославлять Господа;
\vs Sir 39:9 благоуправит свою волю и ум и будет размышлять о тайнах Господа;
\vs Sir 39:10 он покажет мудрость своего учения и будет хвалиться законом завета Господня.
\vs Sir 39:11 Многие будут прославлять знание его, и он не будет забыт вовек;
\vs Sir 39:12 память о нем не погибнет, и имя его будет жить в роды родов.
\vs Sir 39:13 Народы будут прославлять его мудрость, и общество будет возвещать хвалу его;
\vs Sir 39:14 доколе будет жить, он приобретет б\acc{о}льшую славу, нежели тысячи; а когда почиет, увеличит ее.
\rsbpar\vs Sir 39:15 Еще размыслив, расскажу, ибо я полон, как луна в полноте своей.
\vs Sir 39:16 Выслушайте меня, благочестивые дети, и растите, как роза, растущая на поле при потоке;
\vs Sir 39:17 издавайте благоухание, как ливан;
\vs Sir 39:18 цветите, как лилия, распространяйте благовоние и пойте песнь;
\vs Sir 39:19 благословляйте Господа во всех делах; величайте имя Его и прославляйте Его хвалою Его,
\vs Sir 39:20 песнями уст и гуслями и, прославляя, говорите так:
\vs Sir 39:21 все дела Господа весьма благотворны, и всякое повеление Его в свое время исполнится;
\vs Sir 39:22 и нельзя сказать: <<что это? для чего это?>>, ибо все в свое время откроется.
\vs Sir 39:23 По слову Его стала вода, как стог, и по изречению уст Его \bibemph{явились} вместилища вод.
\vs Sir 39:24 В повелениях Его~--- всё Его благоволение, и никто не может умалить спасительность их.
\vs Sir 39:25 Пред Ним дела всякой плоти, и невозможно укрыться от очей Его.
\vs Sir 39:26 Он прозирает из века в век, и ничего нет дивного пред Ним.
\vs Sir 39:27 Нельзя сказать: <<что это? для чего это?>>, ибо все создано для своего употребления.
\vs Sir 39:28 Благословение Его покрывает, как река, и, как потоп, напояет сушу.
\vs Sir 39:29 Но и гнев Его испытывают народы, как некогда Он превратил воды в солончаки.
\vs Sir 39:30 Пути Его для святых прямы, а для беззаконных они~--- преткновения.
\vs Sir 39:31 От начала для добрых создано доброе, как для грешников~--- злое.
\vs Sir 39:32 Главное из всех потребностей для жизни человека~--- вода, огонь, железо, соль, пшеничная мука, мед, молоко, виноградный сок, масло и одежда:
\vs Sir 39:33 все это благочестивым служит в пользу, а грешникам может обратиться во вред.
\vs Sir 39:34 Есть ветры, которые созданы для отмщения и в ярости своей усиливают удары свои,
\vs Sir 39:35 во время устремления своего изливают силу и удовлетворяют ярости Сотворившего их.
\vs Sir 39:36 Огонь и град, голод и смерть~--- все это создано для отмщения;
\vs Sir 39:37 зубы зверей, и скорпионы, и змеи, и меч, мстящий нечестивым погибелью,~---
\vs Sir 39:38 обрадуются повелению Его и готовы будут на земле, когда потребуются, и в свое время не преступят слова Его.
\vs Sir 39:39 Посему я с самого начала решил, обдумал и оставил в писании,
\vs Sir 39:40 что все дела Господа прекрасны, и Он дарует все потребное в свое время;
\vs Sir 39:41 и нельзя сказать: <<это хуже того>>, ибо все в свое время признано будет хорошим.
\vs Sir 39:42 Итак, всем сердцем и устами пойте и благословляйте имя Господа.
\vs Sir 40:1 Много трудов предназначено каждому человеку, и тяжело иго на сынах Адама со дня исхода из чрева матери их до дня возвращения к матери всех.
\vs Sir 40:2 Мысль об ожидаемом и день смерти производит в них размышления и страх сердца.
\vs Sir 40:3 От сидящего на славном престоле и до поверженного на земле и во прахе,
\vs Sir 40:4 от носящего порфиру и венец и до одетого в рубище,~---
\vs Sir 40:5 \bibemph{у всякого} досада и ревность, и смущение, и беспокойство, и страх смерти, и негодование, и распря, и во время успокоения на ложе ночной сон расстраивает ум его.
\vs Sir 40:6 Мало, почти совсем не имеет он покоя, и потому и во сне он, как днем, на страже:
\vs Sir 40:7 будучи смущен сердечными своими мечтами, как бежавший с поля брани, во время безопасности своей он пробуждается и не может надивиться, что ничего не было страшного.
\vs Sir 40:8 Хотя \bibemph{это бывает} со всякою плотью, от человека до скота, но у грешников в семь крат более сего.
\vs Sir 40:9 Смерть, убийство, ссора, меч, бедствия, голод, сокрушение и удары,~---
\vs Sir 40:10 все это~--- для беззаконных; и потоп был для них.
\vs Sir 40:11 Все, что от земли, обращается в землю, и что из воды, возвращается в море.
\vs Sir 40:12 Всякий подарок и несправедливость будут истреблены, а верность будет стоять вовек.
\vs Sir 40:13 Имения неправедных, как поток, иссохнут и, как сильный гром при проливном дожде, прогремят.
\vs Sir 40:14 Кто открывает руку, тот бывает весел; а преступники вконец погибнут.
\vs Sir 40:15 Потомки нечестивых не умножат ветвей, и нечистые корни~--- на утесистой скале:
\vs Sir 40:16 осока при всякой воде и на берегу реки скашивается прежде всякой другой травы.
\vs Sir 40:17 Благотворительность, как рай, полна благословений, и милостыня пребывает вовек.
\vs Sir 40:18 Жизнь довольного своею участью \bibemph{и} труженика сладостна; но превосходит обоих тот, кто находит сокровище.
\vs Sir 40:19 Дети и построение города увековечивают имя, но превосходнее того и другого считается безукоризненная жена.
\vs Sir 40:20 Вино и музыка веселят сердце, но лучше того и другого~--- любовь к мудрости.
\vs Sir 40:21 Свирель и гусли делают приятным пение, но лучше их~--- приятный язык.
\vs Sir 40:22 Приятность и красота вожделенны для очей твоих, но более той и другой~--- зелень посева.
\vs Sir 40:23 Друг и приятель сходятся по временам, но жена с мужем~--- всегда.
\vs Sir 40:24 Братья и покровители~--- во время скорби, но вернее тех и других спасает милостыня.
\vs Sir 40:25 Золото и серебро утверждают стопы, но надежнее того и другого признаётся \bibemph{добрый} совет.
\vs Sir 40:26 Богатство и сила возвышают сердце, но выше того~--- страх Господень:
\vs Sir 40:27 в страхе Господнем нет недостатка, и нет надобности искать при нем помощи;
\vs Sir 40:28 страх Господень~--- как благословенный рай, и облекает его всякою славою.
\rsbpar\vs Sir 40:29 Сын мой! не живи жизнью нищенскою: лучше умереть, нежели просить милостыни.
\vs Sir 40:30 Кто засматривается на чужой стол, того жизнь~--- не жизнь: он унижает душу свою чужими яствами;
\vs Sir 40:31 но человек разумный и благовоспитанный предостережет себя от того.
\vs Sir 40:32 В устах бесстыдного сладким покажется прошение милостыни, но в утробе его огонь возгорится.
\vs Sir 41:1 О, смерть! как горько воспоминание о тебе для человека, который спокойно живет в своих владениях,
\vs Sir 41:2 для человека, который ничем не озабочен и во всем счастлив и еще в силах принимать пищу.
\vs Sir 41:3 О, смерть! отраден твой приговор для человека, нуждающегося и изнемогающего в силах,
\vs Sir 41:4 для престарелого и обремененного заботами обо всем, для не имеющего надежды и потерявшего терпение.
\vs Sir 41:5 Не бойся смертного приговора: вспомни о предках твоих и потомках. Это приговор от Господа над всякою плотью.
\vs Sir 41:6 Итак, для чего ты отвращаешься от того, что благоугодно Всевышнему? десять ли, сто ли, или тысяча лет,~---
\vs Sir 41:7 в аде нет исследования о \bibemph{времени} жизни.
\vs Sir 41:8 Дети грешников бывают дети отвратительные и общаются с нечестивыми.
\vs Sir 41:9 Наследие детей грешников погибнет, и вместе с племенем их будет распространяться бесславие.
\vs Sir 41:10 Нечестивого отца будут укорять дети, потому что за него они терпят бесславие.
\vs Sir 41:11 Горе вам, люди нечестивые, которые оставили закон Бога Всевышнего!
\vs Sir 41:12 Когда вы рождаетесь, то рождаетесь на проклятие; и когда умираете, то получаете в удел свой проклятие.
\vs Sir 41:13 Все, что из земли, возвратится в землю: так нечестивые~--- от проклятия в погибель.
\vs Sir 41:14 Плач людей бывает о телах их, но грешников и имя недоброе изгладится.
\vs Sir 41:15 Заботься об имени, ибо оно пребудет с тобою долее, нежели многие тысячи золота:
\vs Sir 41:16 дням доброй жизни есть число, но доброе имя пребывает вовек.
\rsbpar\vs Sir 41:17 Соблюдайте, дети, наставление в мире; а сокрытая мудрость и сокровище невидимое~--- какая в них польза?
\vs Sir 41:18 Лучше человек, скрывающий свою глупость, нежели человек, скрывающий свою мудрость.
\vs Sir 41:19 Итак, стыдитесь того, о чем я скажу,
\vs Sir 41:20 ибо не всякую стыдливость хорошо соблюдать и не всё всеми одобряется по истине.
\vs Sir 41:21 Стыдитесь пред отцом и матерью блуда, пред начальником и властелином~--- лжи;
\vs Sir 41:22 пред судьею и князем~--- преступления, пред собранием и народом~--- беззакония;
\vs Sir 41:23 пред товарищем и другом~--- неправды, пред соседями~--- кражи:
\vs Sir 41:24 стыдитесь сего и пред истиною Бога и завета Его. Стыдись и облокачивания на стол, обмана при займе и отдаче;
\vs Sir 41:25 стыдись молчания пред приветствующими, смотрения на распутную женщину, отвращения лица от родственника,
\vs Sir 41:26 отнятия доли и дара, помысла на замужнюю женщину, ухаживания за своею служанкою,
\vs Sir 41:27 и не подходи к постели ее;
\vs Sir 41:28 пред друзьями стыдись слов укорительных,~--- и после того, как ты дал, не попрекай,~---
\vs Sir 41:29 повторения слухов и разглашения слов тайных. И будешь истинно стыдлив и приобретешь благорасположение всякого человека.
\vs Sir 42:1 Не стыдись вот чего, и из лицеприятия не греши:
\vs Sir 42:2 не стыдись \bibemph{точного исполнения} закона Всевышнего и завета, и суда, чтобы оказать правосудие нечестивому,
\vs Sir 42:3 спора между товарищем и посторонними и предоставления наследства друзьям,
\vs Sir 42:4 точности в весах и мерах,~--- много ли, мало ли приобретаешь,~---
\vs Sir 42:5 беспристрастия в купле и продаже и строгого воспитания детей, и~--- окровавить ребро худому рабу.
\vs Sir 42:6 При худой жене хорошо иметь печать, и, где много рук, там запирай.
\vs Sir 42:7 Если что выдаешь, \bibemph{выдавай} счетом и весом и делай всякую выдачу и прием по записи.
\vs Sir 42:8 Не стыдись вразумлять неразумного и глупого, и престарелого, состязающегося с молодыми: и будешь истинно благовоспитанным и заслужишь одобрение от всякого человека.
\rsbpar\vs Sir 42:9 Дочь для отца~--- тайная постоянная забота, и попечение о ней отгоняет сон: в юности ее~--- как бы не отцвела, а в замужестве~--- как бы не опротивела;
\vs Sir 42:10 в девстве~--- как бы не осквернилась и не сделалась беременною в отцовском доме, в замужестве~--- чтобы не нарушила супружеской верности и в сожительстве с мужем не осталась бесплодною.
\vs Sir 42:11 Над бесстыдною дочерью усиль надзор, чтобы она не сделала тебя посмешищем для врагов, притчею в городе и упреком в народе и не осрамила тебя пред обществом.
\vs Sir 42:12 Не смотри на красоту человека и не сиди среди женщин:
\vs Sir 42:13 ибо как из одежд выходит моль, так от женщины~--- лукавство женское.
\vs Sir 42:14 Лучше злой мужчина, нежели ласковая женщина,~--- женщина, которая стыдит до поношения.
\rsbpar\vs Sir 42:15 Воспомяну теперь о делах Господа и расскажу о том, что я видел. По слову Господа \bibemph{явились} дела Его:
\vs Sir 42:16 сияющее солнце смотрит на все, и все дело его полно славы Господней.
\vs Sir 42:17 И святым не предоставил Господь провозвестить о всех чудесах Его, которые утвердил Господь Вседержитель, чтобы вселенная стояла твердо во славу Его.
\vs Sir 42:18 Он проникает бездну и сердце и видит все изгибы их; ибо Господь знает всякое в\acc{е}дение и прозирает в знамения века,
\vs Sir 42:19 возвещая прошедшее и будущее и открывая следы сокровенного;
\vs Sir 42:20 не минует Его никакое помышление и не утаится от Него ни одно слово.
\vs Sir 42:21 Он устроил великие дела Своей премудрости и пребывает прежде века и вовек;
\vs Sir 42:22 Он не увеличился и не умалился и не требовал никакого советника.
\vs Sir 42:23 Как вожделенны все дела Его, хотя мы можем видеть их как только искры!
\vs Sir 42:24 Все они живут и пребывают вовек для всяких потребностей, и все повинуются \bibemph{Ему}.
\vs Sir 42:25 Все они~--- вдвойне, одно напротив другого, и ничего не сотворил Он несовершенным:
\vs Sir 42:26 одно поддерживает благо другого,~--- и кто насытится зрением славы Его?
\vs Sir 43:1 Величие высоты, твердь чистоты, вид неба в славном явлении!
\vs Sir 43:2 Солнце, когда оно является, возвещает о них при восходе: чудное создание, дело Всевышнего!
\vs Sir 43:3 В полдень свой оно иссушает землю, и пред жаром его кто устоит?
\vs Sir 43:4 Распаляют горн для работ плавильных, но втрое сильнее солнце палит горы: дыша пламенем огня и блистая лучами, оно ослепляет глаза.
\vs Sir 43:5 Велик Господь, Который сотворил его, и по слову Его оно поспешно пробегает путь свой.
\vs Sir 43:6 И луна всем в свое время служит указанием времен и знамением века:
\vs Sir 43:7 от луны~--- указание праздника; свет ее умаляется по достижении ею полноты;
\vs Sir 43:8 месяц называется по имени ее; она дивно возрастает в своем изменении;
\vs Sir 43:9 это~--- глава вышних строев; она сияет на тверди небесной;
\vs Sir 43:10 красота неба, слава звезд, блестящее украшение, владыка на высотах!
\vs Sir 43:11 По слову Святаго \bibemph{звезды} стоят по чину и не устают на страже своей.
\vs Sir 43:12 Взгляни на радугу, и прославь Сотворившего ее: прекрасна она в сиянии своем!
\vs Sir 43:13 Величественным кругом своим она обнимает небо; руки Всевышнего распростерли ее.
\vs Sir 43:14 Повелением Его скоро сыплется снег, и быстро сверкают молнии суда Его.
\vs Sir 43:15 Отверзаются сокровищницы и вылетают из них облака, как птицы.
\vs Sir 43:16 Могуществом Своим Он укрепляет облака, и разбиваются камни града;
\vs Sir 43:17 от взора Его потрясаются горы, и по изволению Его веет южный ветер.
\vs Sir 43:18 Голос грома Его приводит в трепет землю, и северная буря и вихрь.
\vs Sir 43:19 Он сыплет снег подобно летящим вниз крылатым, и ниспадение его~--- как опускающаяся саранча;
\vs Sir 43:20 красоте белизны его удивляется глаз, и ниспадению его изумляется сердце.
\vs Sir 43:21 И как соль, рассыпает Он по земле иней, который, замерзая, делается остроконечным.
\vs Sir 43:22 Подует северный холодный ветер,~--- и из воды делается лед: он расстилается на всяком вместилище вод, и вода облекается как бы в латы;
\vs Sir 43:23 поядает горы, и пожигает пустыню, и, как огонь, опаляет траву.
\vs Sir 43:24 Но скорым исцелением всему служит туман; появляющаяся роса прохлаждает от зноя.
\vs Sir 43:25 Повелением Своим Господь укрощает бездну и насаждает на ней острова.
\vs Sir 43:26 Плавающие по морю рассказывают об опасностях на нем, и мы дивимся тому, что слышим ушами нашими:
\vs Sir 43:27 ибо там необычайные и чудные дела, разнообразие всяких животных, роды чудовищ.
\vs Sir 43:28 Чрез Него все успешно достигает своего назначения, и все держится словом Его.
\vs Sir 43:29 Многое можем мы сказать, и, однако же, не постигнем Его, и конец слов: Он есть всё.
\vs Sir 43:30 Где возьмем силу, чтобы прославить Его? ибо Он превыше всех дел Своих.
\vs Sir 43:31 Страшен Господь и весьма велик, и дивно могущество Его!
\vs Sir 43:32 Прославляя Господа, превозносите Его, сколько можете, но и затем Он будет превосходнее;
\vs Sir 43:33 и, величая Его, прибавьте силы: но не труд\acc{и}тесь, ибо не постигнете.
\vs Sir 43:34 Кто видел Его, и объяснит? и кто прославит Его, как Он есть?
\vs Sir 43:35 Много сокрыто, что гораздо больше сего; ибо мы видим малую часть дел Его.
\vs Sir 43:36 Всё сотворил Господь, и благочестивым даровал мудрость.
\vs Sir 44:1 Теперь восхвалим славных мужей и отцов нашего рода:
\vs Sir 44:2 много славного Господь являл \bibemph{чрез них}, величие Свое от века;
\vs Sir 44:3 это были господствующие в царствах своих и мужи, именитые силою; они давали разумные советы, возвещали в пророчествах;
\vs Sir 44:4 они были руководителями народа при совещаниях и в книжном обучении.
\vs Sir 44:5 Мудрые слова были в учении их; они изобрели музыкальные строи и гимны предали писанию;
\vs Sir 44:6 люди богатые, одаренные силою, они мирно обитали в жилищах своих.
\vs Sir 44:7 Все они были уважаемы между племенами своими и во дни свои были славою.
\vs Sir 44:8 Есть между ними такие, которые оставили по себе имя для возвещения хвалы их,~--- и есть такие, о которых не осталось памяти, которые исчезли, как будто не существовали, и сделались как бы небывшими, и дети их после них.
\vs Sir 44:9 Но те были мужи милости, которых праведные дела не забываются;
\vs Sir 44:10 в семени их пребывает доброе наследство; потомки их~--- в заветах;
\vs Sir 44:11 семя их будет твердо, и дети их~--- ради них;
\vs Sir 44:12 семя их пребудет до века, и слава их не истребится;
\vs Sir 44:13 тела их погребены в мире, и имена их живут в роды;
\vs Sir 44:14 народы будут рассказывать о их мудрости, а церковь будет возвещать их хвалу.
\rsbpar\vs Sir 44:15 Енох угодил Господу и был взят на небо,~--- образ покаяния для \bibemph{всех} родов.
\vs Sir 44:16 Ной оказался совершенным, праведным; во время гнева он был умилостивлением;
\vs Sir 44:17 посему сделался остатком на земле, когда был потоп;
\vs Sir 44:18 с ним заключен был вечный завет, что никакая плоть не истребится более потопом.
\vs Sir 44:19 Авраам~--- великий отец множества народов, и не было подобного ему в славе;
\vs Sir 44:20 он сохранил закон Всевышнего и был в завете с Ним,
\vs Sir 44:21 и на своей плоти утвердил завет и в испытании оказался верным;
\vs Sir 44:22 поэтому Господь с клятвою обещал ему, что в семени его благословятся все народы;
\vs Sir 44:23 обещал умножить его, как прах земли, и возвысить семя его, как звезды, и дать им наследство от моря до моря и от реки до края земли.
\vs Sir 44:24 И Исааку ради Авраама, отца его, Он также подтвердил благословение всех людей и завет;
\vs Sir 44:25 и оно же почило на голове Иакова:
\vs Sir 44:26 Он ущедрил его Своими благословениями, и дал ему в наследие \bibemph{землю}, и отделил участки ее, и разделил между двенадцатью коленами.
\rsbpar\vs Sir 44:27 И произвел от него мужа милости, который приобрел любовь в глазах всякой плоти,
\vs Sir 45:1 возлюбленного Богом и людьми Моисея, которого память благословенна.
\vs Sir 45:2 Он сравнял его в славе со святыми и возвеличил его делами на страх врагам;
\vs Sir 45:3 Он его словом прекращал чудесные знамения, прославил его пред лицем царей, давал чрез него повеления к народу его и показал ему от славы Своей.
\vs Sir 45:4 За верность и кротость его Он освятил его, избрал Себе из всех людей,
\vs Sir 45:5 сподобил его слышать голос Его, ввел его во мглу
\vs Sir 45:6 и дал ему лицем к лицу заповеди, закон жизни и в\acc{е}дения, чтобы он научил Иакова завету и Израиля~--- постановлениям Его.
\vs Sir 45:7 Он возвысил Аарона, подобного ему святого, брата его из колена Левиина,~---
\vs Sir 45:8 постановил с ним вечный завет и дал ему священство в народе; Он благословил его особым украшением и опоясал его поясом славы;
\vs Sir 45:9 Он облек его высшим украшением и облачил его в богатые одежды:
\vs Sir 45:10 в исподнюю одежду, в подир и ефод;
\vs Sir 45:11 и окружил его золотыми яблоками и весьма многими позвонками, чтобы при хождении его они издавали звук, чтобы сделать слышным в храме звон для напоминания сынам народа Его;
\vs Sir 45:12 облек его одеждою святою из золота и гиацинтовой шерсти и крученого виссона художественной работы, словом суда, уримом и туммимом,
\vs Sir 45:13 червленым тканьем искусной работы, многоценными камнями, вырезанными как на печати, в золотой оправе гранильной работы, с вырезанными на память начертаниями \bibemph{имен} по числу колен Израилевых;
\vs Sir 45:14 на кидаре его~--- золотой венец, знамение святыни, слава достоинства: величественное украшение, дело искусства, вожделенное для глаз.
\vs Sir 45:15 Прежде него не было сего от века:
\vs Sir 45:16 непринадлежащий к его племени не одевался так, только сыновья его и потомки его во все времена.
\vs Sir 45:17 Жертвы их приносятся каждый день, всегда по два раза.
\vs Sir 45:18 Моисей наполнил руки его и помазал его святым елеем:
\vs Sir 45:19 ему постановлено в вечный завет и семени его на дни неба, чтобы они служили Ему и вместе священнодействовали и благословляли народ Его именем Его;
\vs Sir 45:20 Он избрал его из всех живущих, чтобы приносить Господу жертву, курение и благоухание в память умилостивления о народе своем;
\vs Sir 45:21 Он дал ему Свои заповеди и власть в постановлениях судебных, чтобы учить Иакова откровениям и наставлять Израиля в законе Его.
\vs Sir 45:22 Восстали против него чужие, и позавидовали ему в пустыне люди, приставшие к Дафану и Авирону, и скопище Корея в ярости и гневе;
\vs Sir 45:23 Господь увидел, и Ему неугодно было это,~--- и они погибли от ярости гнева.
\vs Sir 45:24 Он сотворил над ними чудо, истребив их пламенем огня Своего.
\vs Sir 45:25 И умножил славу Аарона и дал ему наследие~--- отделил им начатки плодов:
\vs Sir 45:26 прежде всего уготовил им хлеб в насыщение, ибо они едят и жертвы Господни, которые Он дал ему и семени его;
\vs Sir 45:27 но он не должен иметь наследия в земле народа и нет ему участка между народом, ибо Он Сам удел и наследие его.
\vs Sir 45:28 Также и Финеес, сын Елеазара, третий по славе, потому что он ревновал о страхе Господнем и, при отпадении народа, устоял в добром расположении души своей и умилостивил Господа к Израилю;
\vs Sir 45:29 посему постановлен с ним завет мира, чтобы быть ему предстоятелем святых и народа своего, чтобы ему и семени его принадлежало достоинство священства навеки.
\vs Sir 45:30 Как по завету с Давидом, сыном Иессея из колена Иудина, царское наследие переходило от сына к сыну, так наследие священства \bibemph{принадлежало} Аарону и семени его.
\vs Sir 45:31 Да даст нам Бог мудрость в нашем сердце~--- судить народ Его справедливо, дабы не погибли блага их и слава их пребыла в роды их.
\vs Sir 46:1 Силен был в бранях Иисус Навин и был преемником Моисея в пророчествах.
\vs Sir 46:2 Соответственно имени своему, он был велик в спасении избранных Божиих, когда мстил восставшим врагам, чтобы ввести Израиля в наследие \bibemph{его}.
\vs Sir 46:3 Как он прославился, когда поднял руки свои и простер меч на города!
\vs Sir 46:4 Кто прежде него так стоял? Ибо он вел брани Господни.
\vs Sir 46:5 Не его ли рукою остановлено было солнце, и один день был как бы два?
\vs Sir 46:6 Он воззвал ко Всевышнему Владыке, когда со всех сторон стеснили его враги, и великий Господь услышал его:
\vs Sir 46:7 камнями града с могущественною силою бросил Он на враждебный народ и погубил противников на склоне горы,
\vs Sir 46:8 дабы язычники познали всеоружие \bibemph{его}, что война его была пред Господом, а он \bibemph{только} следовал за Всемогущим.
\vs Sir 46:9 И во дни Моисея он оказал благодеяние, он и Халев, сын Иефоннии,~--- тем, что они противостояли враждующим, удерживали народ от греха и утишали злой ропот.
\vs Sir 46:10 И они только двое из шестисот тысяч путешествовавших были спасены, чтобы ввести \bibemph{народ} в наследие~--- в землю, текущую молоком и медом.
\vs Sir 46:11 И дал Господь Халеву крепость, которая сохранилась в нем до старости, взойти на высоту земли, и семя его получило наследие,
\vs Sir 46:12 дабы видели все сыны Израилевы, что благо следовать Господу.
\vs Sir 46:13 Также и судии, каждый по своему имени, которых сердце не заблуждалось и которые не отвращались от Господа,~--- да будет память их во благословениях!
\vs Sir 46:14 Да процветут кости их от места своего,
\vs Sir 46:15 и имя их да перейдет к сынам их в прославлении их!
\vs Sir 46:16 Возлюбленный Господом своим Самуил, пророк Господень, учредил царство и помазал царей народу своему;
\vs Sir 46:17 он судил народ по закону Господню, и Господь призирал на Иакова;
\vs Sir 46:18 по вере своей он был истинным пророком, и в словах его дознана верность видения.
\vs Sir 46:19 Он воззвал ко Всемогущему Господу, когда отвсюду теснили его враги, и принес в жертву молодого агнца,~---
\vs Sir 46:20 и Господь возгремел с неба и в сильном шуме слышным сделал голос Свой,
\vs Sir 46:21 и истребил вождей Тирских и всех князей Филистимских.
\vs Sir 46:22 Еще прежде времени вечного успокоения своего он свидетельствовался пред Господом и помазанником \bibemph{Его}: <<имущества, ни даже обуви, я не брал ни от кого>>, и никто не укорил его.
\vs Sir 46:23 Он пророчествовал и по смерти своей, и предсказал царю смерть его, и в пророчестве возвысил из земли голос свой, что беззаконный народ истребится.
\vs Sir 47:1 После сего явился Нафан, чтобы пророчествовать во дни Давида.
\vs Sir 47:2 Как тук, отделенный от мирной жертвы, так Давид от сынов Израилевых.
\vs Sir 47:3 Он играл со львами, как с козлятами, и с медведями, как с ягнятами.
\vs Sir 47:4 В юности своей не убил ли он исполина, не снял ли поношение с народа,
\vs Sir 47:5 когда поднял руку с пращным камнем и низложил гордыню Голиафа?
\vs Sir 47:6 Ибо он воззвал к Господу Всевышнему, и Он дал крепость правой руке его~--- поразить человека, сильного в войне, и возвысить рог народа своего.
\vs Sir 47:7 Так прославил народ его тьмами и восхвалил его в благословениях Господа, как достойного венца славы,
\vs Sir 47:8 ибо он истребил окрестных врагов и смирил враждебных Филистимлян,~--- даже доныне сокрушил рог их.
\vs Sir 47:9 После каждого дела своего он приносил благодарение Святому Всевышнему словом хвалы;
\vs Sir 47:10 от всего сердца он воспевал и любил Создателя своего.
\vs Sir 47:11 И поставил пред жертвенником песнопевцев, чтобы голосом их услаждать песнопение.
\vs Sir 47:12 Он дал праздникам благолепие и с точностью определил времена, чтобы они хвалили святое имя Его и с раннего утра оглашали святилище.
\vs Sir 47:13 И Господь отпустил ему грехи и навеки вознес рог его и даровал ему завет царственный и престол славы в Израиле.
\vs Sir 47:14 После него восстал мудрый сын его и ради \bibemph{отца} жил счастливо.
\vs Sir 47:15 Соломон царствовал в мирные дни, потому что Бог успокоил его со всех сторон, дабы он построил дом во имя Его и приготовил святилище навеки.
\vs Sir 47:16 Как мудр был ты в юности твоей и, подобно реке, полон разума!
\vs Sir 47:17 Душа твоя покрыла землю, и ты наполнил ее загадочными притчами;
\vs Sir 47:18 имя твое пронеслось до отдаленных островов, и ты был любим за мир твой;
\vs Sir 47:19 за песни и изречения, за притчи и изъяснения тебе удивлялись страны.
\vs Sir 47:20 Во имя Господа Бога, наименованного Богом Израиля,
\vs Sir 47:21 ты собрал золото, как медь, и умножил серебро, как свинец.
\vs Sir 47:22 Но ты наклонил чресла твои к женщинам и поработился им телом твоим;
\vs Sir 47:23 ты положил пятно на славу твою и осквернил семя твое так, что навел гнев на детей твоих,~--- и они горько оплакивали твое безумие,~--- что власть разделилась надвое, и от Ефрема произошло непокорное царство.
\vs Sir 47:24 Но Господь не оставит Своей милости и не разрушит ни одного из дел Своих, не истребит потомков избранного Своего и не искоренит семени возлюбившего Его.
\vs Sir 47:25 И Он дал Иакову остаток, и Давиду~--- корень от него.
\rsbpar\vs Sir 47:26 И почил Соломон с отцами своими,
\vs Sir 47:27 и оставил по себе от семени своего безумие народу,
\vs Sir 47:28 скудного разумом Ровоама, который отвратил от себя народ чрез свое совещание,
\vs Sir 47:29 и Иеровоама, сына Наватова, который ввел в грех Израиля и Ефрему указал путь греха.
\vs Sir 47:30 И весьма умножились грехи их, так что они изгнаны были из земли своей;
\vs Sir 47:31 и посягали они на всякое зло, доколе не пришло на них мщение.
\vs Sir 48:1 И восстал Илия пророк, как огонь, и слово его горело, как светильник.
\vs Sir 48:2 Он навел на них голод и ревностью своею умалил \bibemph{число} их;
\vs Sir 48:3 словом Господним он заключил небо и три раза низводил огонь.
\vs Sir 48:4 Как прославился ты, Илия, чудесами твоими, и кто может сравниться с тобою в славе!
\vs Sir 48:5 Ты воздвиг мертвого от смерти и из ада словом Всевышнего;
\vs Sir 48:6 ты низводил в погибель царей и знатных с ложа их;
\vs Sir 48:7 ты слышал на Синае обличение \bibemph{на них} и на Хориве суды мщения;
\vs Sir 48:8 ты помазал царей на воздаяние и пророков~--- в преемники себе;
\vs Sir 48:9 ты восх\acc{и}щен был огненным вихрем на колеснице с огненными конями;
\vs Sir 48:10 ты предназначен был на обличения в свои времена, чтобы утишить гнев, прежде нежели обратится он в ярость,~--- обратить сердце отца к сыну и восстановить колена Иакова.
\vs Sir 48:11 Блаженны видевшие тебя и украшенные любовью,~--- и мы жизнью поживем.
\vs Sir 48:12 Илия сокрыт был вихрем,~--- и Елисей исполнился духом его
\vs Sir 48:13 и во дни свои не трепетал пред князем, и никто не превозмог его;
\vs Sir 48:14 ничто не одолело его, и по успении его пророчествовало тело его.
\vs Sir 48:15 И при жизни своей совершал он чудеса, и по смерти дивны были дела его.
\rsbpar\vs Sir 48:16 При всем том народ не покаялся, и не отступили от грехов своих, доколе не были пленены из земли своей и рассеяны по всей земле.
\vs Sir 48:17 И осталось весьма мало народа и князь из дома Давидова.
\vs Sir 48:18 Некоторые из них делали угодное Богу, а некоторые умножали грехи.
\vs Sir 48:19 Езекия укрепил город свой и провел внутрь его воду, пробил железом скалу и устроил хранилища для воды.
\vs Sir 48:20 Во дни его сделал нашествие Сеннахирим и послал к нему Рабсака, который поднял руку свою на Сион и много величался в гордости своей.
\vs Sir 48:21 Тогда затрепетали сердца и руки их, и они мучились, как родильницы;
\vs Sir 48:22 и воззвали они к Господу милосердому, простерши к Нему руки свои,
\vs Sir 48:23 и Святый скоро услышал их с неба и избавил их рукою Исаии;
\vs Sir 48:24 Он поразил войско Ассириян, и Ангел Его истребил их,
\vs Sir 48:25 ибо Езекия делал угодное Господу и крепко держался путей Давида, отца своего, как заповедал пророк Исаия, великий и верный в видениях своих.
\vs Sir 48:26 В его дни солнце отступило назад, и он прибавил жизни царю.
\vs Sir 48:27 Великим духом своим он провидел отдаленное будущее и утешал сетующих в Сионе;
\vs Sir 48:28 до века возвещал он будущее и сокровенное, прежде нежели оно исполнилось.
\vs Sir 49:1 Память Иосии~--- как состав фимиама, приготовленный искусством мироварника:
\vs Sir 49:2 во всяких устах она будет сладка, как мед и как музыка при угощении вином.
\vs Sir 49:3 Он успешно действовал в обращении народа и истребил мерзости беззакония;
\vs Sir 49:4 он направил к Господу сердце свое и во дни беззаконных утвердил благочестие.
\vs Sir 49:5 Кроме Давида, Езекии и Иосии, все тяжко согрешили,
\vs Sir 49:6 ибо оставили закон Всевышнего; цари Иудейские престали,
\vs Sir 49:7 ибо предали рог свой другим и славу свою~--- чужому народу.
\vs Sir 49:8 Избранный город святыни сожжен, и улицы его опустошены, как предсказал Иеремия,
\vs Sir 49:9 которого они оскорбляли, хотя он еще во чреве освящен был в пророка, чтобы искоренять, поражать и погублять, равно как строить и насаждать.
\rsbpar\vs Sir 49:10 Иезекииль видел явление славы, которую \bibemph{Бог} показал ему в херувимской колеснице;
\vs Sir 49:11 он напоминал о врагах под образом дождя и возвещал доброе тем, которые исправляли пути свои.
\vs Sir 49:12 И двенадцать пророков~--- да процветут кости их от места своего!~--- утешали Иакова и спасали их верною надеждою.
\vs Sir 49:13 Как возвеличим Зоровавеля? И он~--- как перстень на правой руке;
\vs Sir 49:14 также Иисус, сын Иоседека: они во дни свои построили дом и восстановили святый храм Господу, предназначенный к вечной славе.
\vs Sir 49:15 Велика память и Неемии, который воздвиг нам павшие стены, поставил ворота и запоры и возобновил разрушенные домы наши.
\vs Sir 49:16 Не было на земле никого из сотворенных, подобного Еноху,~--- ибо он был восх\acc{и}щен от земли,~---
\vs Sir 49:17 и не родился такой муж, как Иосиф, глава братьев, опора народа,~--- и кости его были почтены.
\vs Sir 49:18 Прославились между людьми Сим и Сиф, но выше всего живущего в творении~--- Адам.
\vs Sir 50:1 Симон, сын Онии, великий священник, при жизни своей исправил дом и во дни свои укрепил храм:
\vs Sir 50:2 им положено основание двойного возвышения~--- возведение высокой ограды храма;
\vs Sir 50:3 во дни его уменьшено водохранилище, окружность медного моря;
\vs Sir 50:4 чтобы предохранить народ свой от бедствия, он укрепил город против осады.
\vs Sir 50:5 Как величествен был он среди народа, при выходе из завесы храма!
\vs Sir 50:6 Как утренняя звезда среди облаков, как луна полная во днях,
\vs Sir 50:7 как солнце, сияющее над храмом Всевышнего, и как радуга, сияющая в величественных облаках,
\vs Sir 50:8 как цвет роз в весенние дни, как лилии при источниках вод, как ветвь ливана в летние дни,
\vs Sir 50:9 как огонь с ладаном в кадильнице,
\vs Sir 50:10 как кованый золотой сосуд, украшенный всякими драгоценными камнями,
\vs Sir 50:11 как маслина с плодами и как возвышающийся до облаков кипарис.
\vs Sir 50:12 Когда он принимал великолепную одежду и облекался во все величественное украшение, то, при восхождении к святому жертвеннику, освещал блеском окружность святилища.
\vs Sir 50:13 Также, когда он принимал \bibemph{жертвенные} части из рук священников, стоя у огня жертвенника,~---
\vs Sir 50:14 вокруг него был венец братьев, как отрасли кедра на Ливане, и они окружали его как финиковые ветви,
\vs Sir 50:15 и все сыны Аарона в славе своей, и приношение Господу в руках их пред всем собранием Израиля.
\vs Sir 50:16 В довершение служб на алтаре, чтобы увенчать приношение Всевышнему Вседержителю,
\vs Sir 50:17 он простирал свою руку к жертвенной чаше, лил в нее из винограда кровь и выливал ее к подножию жертвенника в вон\acc{ю} благоухания Вышнему Всецарю.
\vs Sir 50:18 Тогда сыны Аароновы восклицали, трубили коваными трубами и издавали громкий голос в напоминание пред Всевышним.
\vs Sir 50:19 Тогда весь народ вместе спешил падать лицем на землю, чтобы поклониться Господу своему, Вседержителю, Богу Вышнему;
\vs Sir 50:20 а песнопевцы восхваляли Его своими голосами; в пространном храме раздавалось сладостное пение,
\vs Sir 50:21 и народ молился Господу Всевышнему молитвою пред Милосердым, доколе совершалось славословие Господа,~--- и так оканчивали они службу Ему.
\vs Sir 50:22 Тогда он, сойдя, поднимал руки свои на все собрание сынов Израилевых, чтобы устами своими преподать благословение Господа и похвалиться именем Его;
\vs Sir 50:23 народ повторял поклонение, чтобы принять благословение от Всевышнего.
\vs Sir 50:24 И ныне все благословляйте Бога, Который везде совершает великие дела, Который продлил дни наши от утробы и поступает с нами по милости Своей:
\vs Sir 50:25 да даст Он нам веселие сердца, и да будет во дни наши мир в Израиле до дней века;
\vs Sir 50:26 да сохранит милость Свою к нам и в свое время да избавит нас!
\vs Sir 50:27 Двумя народами гнушается душа моя, а третий не есть народ:
\vs Sir 50:28 \bibemph{это} сидящие на горе Сеир, Филистимляне и глупый народ, живущий в Сикимах.
\vs Sir 50:29 Учение мудрости и благоразумия начертал в книге сей я, Иисус, сын Сирахов, Иерусалимлянин, который излил мудрость от сердца своего.
\vs Sir 50:30 Блажен, кто будет упражняться в сих \bibemph{наставлениях},~--- и кто положит их на сердце, тот сделается мудрым;
\vs Sir 50:31 а если будет исполнять, то все возможет; ибо свет Господень~--- путь его.
\chhdr{Молитва Иисуса, сына Сирахова.}
\vs Sir 51:1 Прославлю Тебя, Господи Царю, и восхвалю Тебя, Бога, Спасителя моего; прославляю имя Твое,
\vs Sir 51:2 ибо Ты был мне покровителем и помощником
\vs Sir 51:3 и избавил тело мое от погибели и от сети клеветнического языка, от уст сплетающих ложь; и против восставших на меня Ты был мне помощником
\vs Sir 51:4 и избавил меня, по множеству милости и ради имени Твоего, от скрежета зубов, готовых пожрать меня,
\vs Sir 51:5 от руки искавших души моей, от многих скорбей, которые я имел,
\vs Sir 51:6 от удушающего со всех сторон огня и из среды пламени, в котором я не сгорел,
\vs Sir 51:7 из глубины чрева адова, от языка нечистого и слова ложного, от клеветы пред царем языка неправедного.
\vs Sir 51:8 Душа моя близка была к смерти,
\vs Sir 51:9 и жизнь моя была близ ада преисподнего:
\vs Sir 51:10 со всех сторон окружали меня, и не было помогающего; искал я глазами заступления от людей,~--- и не было его.
\vs Sir 51:11 И вспомнил я о Твоей, Господи, милости и о делах Твоих от века,
\vs Sir 51:12 что Ты избавляешь надеющихся на Тебя и спасаешь их от руки врагов.
\vs Sir 51:13 И я вознес от земли моление мое и молился о избавлении от смерти:
\vs Sir 51:14 воззвал я к Господу, Отцу Господа моего, чтобы Он не оставил меня во дни скорби, когда не было помощи от людей надменных.
\vs Sir 51:15 Буду хвалить имя Твое непрестанно и воспевать в славословии, ибо молитва моя была услышана;
\vs Sir 51:16 Ты спас меня от погибели и избавил меня от злого времени.
\vs Sir 51:17 За это я буду прославлять и хвалить Тебя и благословлять имя Господа.
\rsbpar\vs Sir 51:18 Будучи еще юношею, прежде нежели пошел я странствовать, открыто искал я мудрости в молитве моей:
\vs Sir 51:19 пред храмом я молился о ней, и до конца буду искать ее; как бы от цвета зреющего винограда,
\vs Sir 51:20 сердце мое радуется о ней; нога моя шла прямым путем, я следил за нею от юности моей.
\vs Sir 51:21 Понемногу наклонял я ухо мое и принимал ее, и находил в ней много наставлений для себя:
\vs Sir 51:22 мне был успех в ней.
\vs Sir 51:23 Воздам славу Дающему мне мудрость.
\vs Sir 51:24 Я решился следовать ей, ревновал о добром, и не постыжусь.
\vs Sir 51:25 Душа моя подвизалась ради нее, и в делах моих я был точен;
\vs Sir 51:26 простирал руки мои к высоте и сознавал мое невежество.
\vs Sir 51:27 Я направил к ней душу мою, и сердце мое предал ей с самого начала~---
\vs Sir 51:28 и при чистоте достиг ее; посему не буду оставлен ею.
\vs Sir 51:29 И подвиглась внутренность моя, чтобы искать ее; посему я приобрел доброе приобретение.
\vs Sir 51:30 В награду мне Бог дал язык, и им я буду хвалить Его.
\vs Sir 51:31 Приблизьтесь ко мне, ненаученные, и водворитесь в доме учения,
\vs Sir 51:32 ибо вы нуждаетесь в этом и души ваши сильно жаждут.
\vs Sir 51:33 Я отверзаю уста мои и говорю: приобретайте ее себе без серебра;
\vs Sir 51:34 подклоните выю вашу под иго ее, и пусть душа ваша принимает учение; его можно найти близко.
\vs Sir 51:35 Видите своими глазами: я немного потрудился~--- и нашел себе великое успокоение.
\vs Sir 51:36 Приобретайте учение и за большое количество серебра,~--- и вы приобретете много золота.
\vs Sir 51:37 Да радуется душа ваша о милости Его, и не стыдитесь хвалить Его;
\vs Sir 51:38 делайте свое дело заблаговременно, и Он в свое время отдаст вашу награду.

\include{tex/Isa}
\include{tex/Jer}
\bibbookdescr{Lam}{
  inline={\LARGE Книга\\\Huge Плач Иеремии},
  toc={Плач Иеремии},
  bookmark={Плач Иеремии},
  header={Плач Иеремии},
  %headerleft={},
  %headerright={},
  abbr={Плач}
}
\vs Lam 1:1 Как одиноко сидит город, некогда многолюдный! он стал, как вдова; великий между народами, князь над областями сделался данником.
\vs Lam 1:2 Горько плачет он ночью, и слезы его на ланитах его. Нет у него утешителя из всех, любивших его; все друзья его изменили ему, сделались врагами ему.
\vs Lam 1:3 Иуда переселился по причине бедствия и тяжкого рабства, поселился среди язычников, и не нашел покоя; все, преследовавшие его, настигли его в тесных местах.
\vs Lam 1:4 Пути Сиона сетуют, потому что нет идущих на праздник; все ворота его опустели; священники его вздыхают, девицы его печальны, горько и ему самому.
\vs Lam 1:5 Враги его стали во главе, неприятели его благоденствуют, потому что Господь наслал на него горе за множество беззаконий его; дети его пошли в плен впереди врага.
\vs Lam 1:6 И отошло от дщери Сиона все ее великолепие; князья ее~--- как олени, не находящие пажити; обессиленные они пошли вперед погонщика.
\vs Lam 1:7 Вспомнил Иерусалим, во дни бедствия своего и страданий своих, о всех драгоценностях своих, какие были у него в прежние дни, тогда как народ его пал от руки врага, и никто не помогает ему; неприятели смотрят на него и смеются над его субботами.
\vs Lam 1:8 Тяжко согрешил Иерусалим, за то и сделался отвратительным; все, прославлявшие его, смотрят на него с презрением, потому что увидели наготу его; и сам он вздыхает и отворачивается назад.
\vs Lam 1:9 На подоле у него была нечистота, но он не помышлял о будущности своей, и поэтому необыкновенно унизился, и нет у него утешителя. <<Воззри, Господи, на бедствие мое, ибо враг возвеличился!>>
\vs Lam 1:10 Враг простер руку свою на все самое драгоценное его; он видит, как язычники входят во святилище его, о котором Ты заповедал, чтобы они не вступали в собрание Твое.
\vs Lam 1:11 Весь народ его вздыхает, ища хлеба, отдает драгоценности свои за пищу, чтобы подкрепить душу. <<Воззри, Господи, и посмотри, как я унижен!>>
\vs Lam 1:12 Да не будет этого с вами, все проходящие путем! взгляните и посмотрите, есть ли болезнь, как моя болезнь, какая постигла меня, какую наслал на меня Господь в день пламенного гнева Своего?
\vs Lam 1:13 Свыше послал Он огонь в кости мои, и он овладел ими; раскинул сеть для ног моих, опрокинул меня, сделал меня бедным и томящимся всякий день.
\vs Lam 1:14 Ярмо беззаконий моих связано в руке Его; они сплетены и поднялись на шею мою; Он ослабил силы мои. Господь отдал меня в руки, из которых не могу подняться.
\vs Lam 1:15 Всех сильных моих Господь низложил среди меня, созвал против меня собрание, чтобы истребить юношей моих; как в точиле, истоптал Господь деву, дочь Иуды.
\vs Lam 1:16 Об этом плачу я; око мое, око мое изливает воды, ибо далеко от меня утешитель, который оживил бы душу мою; дети мои разорены, потому что враг превозмог.
\vs Lam 1:17 Сион простирает руки свои, но утешителя нет ему. Господь дал повеление о Иакове врагам его окружить его; Иерусалим сделался мерзостью среди них.
\vs Lam 1:18 Праведен Господь, ибо я непокорен был слову Его. Послушайте, все народы, и взгляните на болезнь мою: девы мои и юноши мои пошли в плен.
\vs Lam 1:19 Зову друзей моих, но они обманули меня; священники мои и старцы мои издыхают в городе, ища пищи себе, чтобы подкрепить душу свою.
\vs Lam 1:20 Воззри, Господи, ибо мне тесно, волнуется во мне внутренность, сердце мое перевернулось во мне за то, что я упорно противился Тебе; отвне обесчадил меня меч, а дома~--- как смерть.
\vs Lam 1:21 Услышали, что я стенаю, а утешителя у меня нет; услышали все враги мои о бедствии моем и обрадовались, что Ты соделал это: о, если бы Ты повелел наступить дню, предреченному Тобою, и они стали бы подобными мне!
\vs Lam 1:22 Да предстанет пред лице Твое вся злоба их; и поступи с ними так же, как Ты поступил со мною за все грехи мои, ибо тяжки стоны мои, и сердце мое изнемогает.
\vs Lam 2:1 Как помрачил Господь во гневе Своем дщерь Сиона! с небес поверг на землю красу Израиля и не вспомнил о подножии ног Своих в день гнева Своего.
\vs Lam 2:2 Погубил Господь все жилища Иакова, не пощадил, разрушил в ярости Своей укрепления дщери Иудиной, поверг на землю, отверг царство и князей его, как нечистых:
\vs Lam 2:3 в пылу гнева сломил все роги Израилевы, отвел десницу Свою от неприятеля и воспылал в Иакове, как палящий огонь, пожиравший все вокруг;
\vs Lam 2:4 натянул лук Свой, как неприятель, направил десницу Свою, как враг, и убил все, вожделенное для глаз; на скинию дщери Сиона излил ярость Свою, как огонь.
\vs Lam 2:5 Господь стал как неприятель, истребил Израиля, разорил все чертоги его, разрушил укрепления его и распространил у дщери Иудиной сетование и плач.
\vs Lam 2:6 И отнял ограду Свою, как у сада; разорил Свое место собраний, заставил Господь забыть на Сионе празднества и субботы; и в негодовании гнева Своего отверг царя и священника.
\vs Lam 2:7 Отверг Господь жертвенник Свой, отвратил сердце Свое от святилища Своего, предал в руки врагов стены чертогов его; в доме Господнем они шумели, как в праздничный день.
\vs Lam 2:8 Господь определил разрушить стену дщери Сиона, протянул вервь, не отклонил руки Своей от разорения; истребил внешние укрепления, и стены вместе разрушены.
\vs Lam 2:9 Ворота ее вдались в землю; Он разрушил и сокрушил запоры их; царь ее и князья ее~--- среди язычников; не стало закона, и пророки ее не сподобляются видений от Господа.
\vs Lam 2:10 Сидят на земле безмолвно старцы дщери Сионовой, посыпали пеплом свои головы, препоясались вретищем; опустили к земле головы свои девы Иерусалимские.
\vs Lam 2:11 Истощились от слез глаза мои, волнуется во мне внутренность моя, изливается на землю печень моя от гибели дщери народа моего, когда дети и грудные младенцы умирают от голода среди городских улиц.
\vs Lam 2:12 Матерям своим говорят они: <<где хлеб и вино?>>, умирая, подобно раненым, на улицах городских, изливая души свои в лоно матерей своих.
\vs Lam 2:13 Что мне сказать тебе, с чем сравнить тебя, дщерь Иерусалима? чему уподобить тебя, чтобы утешить тебя, дева, дщерь Сиона? ибо рана твоя велика, как море; кто может исцелить тебя?
\vs Lam 2:14 Пророки твои провещали тебе пустое и ложное и не раскрывали твоего беззакония, чтобы отвратить твое пленение, и изрекали тебе откровения ложные и приведшие тебя к изгнанию.
\vs Lam 2:15 Руками всплескивают о тебе все проходящие путем, свищут и качают головою своею о дщери Иерусалима, говоря: <<это ли город, который называли совершенством красоты, радостью всей земли?>>
\vs Lam 2:16 Разинули на тебя пасть свою все враги твои, свищут и скрежещут зубами, говорят: <<поглотили мы его, только этого дня и ждали мы, дождались, увидели!>>
\vs Lam 2:17 Совершил Господь, что определил, исполнил слово Свое, изреченное в древние дни, разорил без пощады и дал врагу порадоваться над тобою, вознес рог неприятелей твоих.
\vs Lam 2:18 Сердце их вопиет к Господу: стена дщери Сиона! лей ручьем слезы день и ночь, не давай себе покоя, не спускай зениц очей твоих.
\vs Lam 2:19 Вставай, взывай ночью, при начале каждой стражи; изливай, как воду, сердце твое пред лицем Господа; простирай к Нему руки твои о душе детей твоих, издыхающих от голода на углах всех улиц.
\vs Lam 2:20 <<Воззри, Господи, и посмотри: кому Ты сделал так, чтобы женщины ели плод свой, младенцев, вскормленных ими? чтобы убиваемы были в святилище Господнем священник и пророк?
\vs Lam 2:21 Дети и старцы лежат на земле по улицам; девы мои и юноши мои пали от меча; Ты убивал их в день гнева Твоего, заколал без пощады.
\vs Lam 2:22 Ты созвал отовсюду, как на праздник, ужасы мои, и в день гнева Господня никто не спасся, никто не уцелел; тех, которые были мною вскормлены и выращены, враг мой истребил>>.
\vs Lam 3:1 Я человек, испытавший горе от жезла гнева Его.
\vs Lam 3:2 Он повел меня и ввел во тьму, а не во свет.
\vs Lam 3:3 Так, Он обратился на меня и весь день обращает руку Свою;
\vs Lam 3:4 измождил плоть мою и кожу мою, сокрушил кости мои;
\vs Lam 3:5 огородил меня и обложил горечью и тяготою;
\vs Lam 3:6 посадил меня в темное место, как давно умерших;
\vs Lam 3:7 окружил меня стеною, чтобы я не вышел, отяготил оковы мои,
\vs Lam 3:8 и когда я взывал и вопиял, задерживал молитву мою;
\vs Lam 3:9 каменьями преградил дороги мои, извратил стези мои.
\vs Lam 3:10 Он стал для меня как бы медведь в засаде, \bibemph{как бы} лев в скрытном месте;
\vs Lam 3:11 извратил пути мои и растерзал меня, привел меня в ничто;
\vs Lam 3:12 натянул лук Свой и поставил меня как бы целью для стрел;
\vs Lam 3:13 послал в почки мои стрелы из колчана Своего.
\vs Lam 3:14 Я стал посмешищем для всего народа моего, вседневною песнью их.
\vs Lam 3:15 Он пресытил меня горечью, напоил меня полынью.
\vs Lam 3:16 Сокрушил камнями зубы мои, покрыл меня пеплом.
\vs Lam 3:17 И удалился мир от души моей; я забыл о благоденствии,
\vs Lam 3:18 и сказал я: погибла сила моя и надежда моя на Господа.
\vs Lam 3:19 Помысли о моем страдании и бедствии моем, о полыни и желчи.
\vs Lam 3:20 Твердо помнит это душа моя и падает во мне.
\vs Lam 3:21 Вот что я отвечаю сердцу моему и потому уповаю:
\vs Lam 3:22 по милости Господа мы не исчезли, ибо милосердие Его не истощилось.
\vs Lam 3:23 Оно обновляется каждое утро; велика верность Твоя!
\vs Lam 3:24 Господь часть моя, говорит душа моя, итак буду надеяться на Него.
\vs Lam 3:25 Благ Господь к надеющимся на Него, к душе, ищущей Его.
\vs Lam 3:26 Благо тому, кто терпеливо ожидает спасения от Господа.
\vs Lam 3:27 Благо человеку, когда он несет иго в юности своей;
\vs Lam 3:28 сидит уединенно и молчит, ибо Он наложил его на него;
\vs Lam 3:29 полагает уста свои в прах, \bibemph{помышляя}: <<может быть, еще есть надежда>>;
\vs Lam 3:30 подставляет ланиту свою биющему его, пресыщается поношением,
\vs Lam 3:31 ибо не навек оставляет Господь.
\vs Lam 3:32 Но послал горе, и помилует по великой благости Своей.
\vs Lam 3:33 Ибо Он не по изволению сердца Своего наказывает и огорчает сынов человеческих.
\vs Lam 3:34 Но, когда попирают ногами своими всех узников земли,
\vs Lam 3:35 когда неправедно судят человека пред лицем Всевышнего,
\vs Lam 3:36 когда притесняют человека в деле его: разве не видит Господь?
\vs Lam 3:37 Кто это говорит: <<и то бывает, чему Господь не повелел быть>>?
\vs Lam 3:38 Не от уст ли Всевышнего происходит бедствие и благополучие?
\vs Lam 3:39 Зачем сетует человек живущий? всякий сетуй на грехи свои.
\vs Lam 3:40 Испытаем и исследуем пути свои, и обратимся к Господу.
\vs Lam 3:41 Вознесем сердце наше и руки к Богу, \bibemph{сущему} на небесах:
\vs Lam 3:42 мы отпали и упорствовали; Ты не пощадил.
\vs Lam 3:43 Ты покрыл Себя гневом и преследовал нас, умерщвлял, не щадил;
\vs Lam 3:44 Ты закрыл Себя облаком, чтобы не доходила молитва наша;
\vs Lam 3:45 сором и мерзостью Ты сделал нас среди народов.
\vs Lam 3:46 Разинули на нас пасть свою все враги наши.
\vs Lam 3:47 Ужас и яма, опустошение и разорение~--- доля наша.
\vs Lam 3:48 Потоки вод изливает око мое о гибели дщери народа моего.
\vs Lam 3:49 Око мое изливается и не перестает, ибо нет облегчения,
\vs Lam 3:50 доколе не призрит и не увидит Господь с небес.
\vs Lam 3:51 Око мое опечаливает душу мою ради всех дщерей моего города.
\vs Lam 3:52 Всячески усиливались уловить меня, как птичку, враги мои, без всякой причины;
\vs Lam 3:53 повергли жизнь мою в яму и закидали меня камнями.
\vs Lam 3:54 Воды поднялись до головы моей; я сказал: <<погиб я>>.
\vs Lam 3:55 Я призывал имя Твое, Господи, из ямы глубокой.
\vs Lam 3:56 Ты слышал голос мой; не закрой уха Твоего от воздыхания моего, от вопля моего.
\vs Lam 3:57 Ты приближался, когда я взывал к Тебе, и говорил: <<не бойся>>.
\vs Lam 3:58 Ты защищал, Господи, дело души моей; искуплял жизнь мою.
\vs Lam 3:59 Ты видишь, Господи, обиду мою; рассуди дело мое.
\vs Lam 3:60 Ты видишь всю мстительность их, все замыслы их против меня.
\vs Lam 3:61 Ты слышишь, Господи, ругательство их, все замыслы их против меня,
\vs Lam 3:62 речи восстающих на меня и их ухищрения против меня всякий день.
\vs Lam 3:63 Воззри, сидят ли они, встают ли, я для них~--- песнь.
\vs Lam 3:64 Воздай им, Господи, по делам рук их;
\vs Lam 3:65 пошли им помрачение сердца и проклятие Твое на них;
\vs Lam 3:66 преследуй их, Господи, гневом, и истреби их из поднебесной.
\vs Lam 4:1 Как потускло золото, изменилось золото наилучшее! камни святилища раскиданы по всем перекресткам.
\vs Lam 4:2 Сыны Сиона драгоценные, равноценные чистейшему золоту, как они сравнены с глиняною посудою, изделием рук горшечника!
\vs Lam 4:3 И чудовища подают сосцы и кормят своих детенышей, а дщерь народа моего стала жестока подобно страусам в пустыне.
\vs Lam 4:4 Язык грудного младенца прилипает к гортани его от жажды; дети просят хлеба, и никто не подает им.
\vs Lam 4:5 Евшие сладкое истаевают на улицах; воспитанные на багрянице жмутся к навозу.
\vs Lam 4:6 Наказание нечестия дщери народа моего превышает казнь за грехи Содома: тот низринут мгновенно, и руки человеческие не касались его.
\vs Lam 4:7 Князья ее \bibemph{были} в ней чище снега, белее молока; они были телом краше коралла, вид их был, как сапфир;
\vs Lam 4:8 а теперь темнее всего черного лице их; не узна\acc{ю}т их на улицах; кожа их прилипла к костям их, стала суха, как дерево.
\vs Lam 4:9 Умерщвляемые мечом счастливее умерщвляемых голодом, потому что сии истаевают, поражаемые недостатком плодов полевых.
\vs Lam 4:10 Руки мягкосердых женщин варили детей своих, чтобы они были для них пищею во время гибели дщери народа моего.
\vs Lam 4:11 Совершил Господь гнев Свой, излил ярость гнева Своего и зажег на Сионе огонь, который пожрал основания его.
\vs Lam 4:12 Не верили цари земли и все живущие во вселенной, чтобы враг и неприятель вошел во врата Иерусалима.
\vs Lam 4:13 \bibemph{Все это}~--- за грехи лжепророков его, за беззакония священников его, которые среди него проливали кровь праведников;
\vs Lam 4:14 бродили как слепые по улицам, осквернялись кровью, так что невозможно было прикоснуться к одеждам их.
\vs Lam 4:15 <<Сторонитесь! нечистый!>> кричали им; <<сторонитесь, сторонитесь, не прикасайтесь>>; и они уходили в смущении; а между народом говорили: <<их более не будет!
\vs Lam 4:16 лице Господне рассеет их; Он уже не призрит на них>>, потому что они лиц\acc{а} священников не уважают, старцев не милуют.
\vs Lam 4:17 Наши глаза истомлены в напрасном ожидании помощи; со сторожевой башни нашей мы ожидали народ, который не мог спасти нас.
\vs Lam 4:18 А они подстерегали шаги наши, чтобы мы не могли ходить по улицам нашим; приблизился конец наш, дни наши исполнились; пришел конец наш.
\vs Lam 4:19 Преследовавшие нас были быстрее орлов небесных; гонялись за нами по горам, ставили засаду для нас в пустыне.
\vs Lam 4:20 Дыхание жизни нашей, помазанник Господень пойман в ямы их, тот, о котором мы говорили: <<под тенью его будем жить среди народов>>.
\vs Lam 4:21 Радуйся и веселись, дочь Едома, обитательница земли Уц! И до тебя дойдет чаша; напьешься допьяна и обнажишься.
\vs Lam 4:22 Дщерь Сиона! наказание за беззаконие твое кончилось; Он не будет более изгонять тебя; но твое беззаконие, дочь Едома, Он посетит и обнаружит грехи твои.
\vs Lam 5:1 Вспомни, Господи, что над нами совершилось; призри и посмотри на поругание наше.
\vs Lam 5:2 Наследие наше перешло к чужим, домы наши~--- к иноплеменным;
\vs Lam 5:3 мы сделались сиротами, без отца; матери наши~--- как вдовы.
\vs Lam 5:4 Воду свою пьем за серебро, дрова наши достаются нам за деньги.
\vs Lam 5:5 Нас погоняют в шею, мы работаем, \bibemph{и} не имеем отдыха.
\vs Lam 5:6 Протягиваем руку к Египтянам, к Ассириянам, чтобы насытиться хлебом.
\vs Lam 5:7 Отцы наши грешили: их уже нет, а мы несем наказание за беззакония их.
\vs Lam 5:8 Рабы господствуют над нами, и некому избавить от руки их.
\vs Lam 5:9 С опасностью жизни от меча, в пустыне достаем хлеб себе.
\vs Lam 5:10 Кожа наша почернела, как печь, от жгучего голода.
\vs Lam 5:11 Жен бесчестят на Сионе, девиц~--- в городах Иудейских.
\vs Lam 5:12 Князья повешены руками их, лица старцев не уважены.
\vs Lam 5:13 Юношей берут к жерновам, и отроки падают под ношами дров.
\vs Lam 5:14 Старцы уже не сидят у ворот; юноши не поют.
\vs Lam 5:15 Прекратилась радость сердца нашего; хороводы наши обратились в сетование.
\vs Lam 5:16 Упал венец с головы нашей; горе нам, что мы согрешили!
\vs Lam 5:17 От сего-то изнывает сердце наше; от сего померкли глаза наши.
\vs Lam 5:18 Оттого, что опустела гора Сион, лисицы ходят по ней.
\vs Lam 5:19 Ты, Господи, пребываешь во веки; престол Твой~--- в род и род.
\vs Lam 5:20 Для чего совсем забываешь нас, оставляешь нас на долгое время?
\vs Lam 5:21 Обрати нас к Тебе, Господи, и мы обратимся; обнови дни наши, как древле.
\vs Lam 5:22 Неужели Ты совсем отверг нас, прогневался на нас безмерно?

\bibbookdescr{Epj}{
  inline={\LARGE Послание\\\Huge Иеремии\fns{Переведена с греческого.}},
  toc={Послание Иеремии*},
  bookmark={Послание Иеремии},
  header={Послание Иеремии},
  %headerleft={},
  %headerright={},
  abbr={Посл~Иер}
}
\vs Epj 1:1 Список послания, которое послал Иеремия к пленникам, отводимым в Вавилон царем Вавилонским, чтобы возвестить им, чт\acc{о} повелено ему Богом.
\rsbpar\vs Epj 1:2 За грехи, которыми вы согрешили пред Богом, будете отведены пленниками в Вавилон Навуходоносором, царем Вавилонским.
\vs Epj 1:3 Войдя в Вавилон, вы пробудете там многие годы и долгое время, даже до семи родов; после же сего Я выведу вас оттуда с миром.
\vs Epj 1:4 Теперь вы увидите в Вавилоне богов серебряных и золотых и деревянных, носимых на плечах, внушающих страх язычникам.
\vs Epj 1:5 Берегитесь же, чтобы и вам не сделаться подобными иноплеменникам, и чтобы страх пред ними не овладел и вами. Видя толпу спереди и сзади их поклоняющеюся перед ними, скажите в уме: <<Тебе должно поклоняться, Владыко!>>
\vs Epj 1:6 Ибо Ангел Мой с вами, и он защитник душ ваших.
\vs Epj 1:7 Язык их выстроган художником, и сами они оправлены в золото и серебро; но они ложные, и не могут говорить.
\vs Epj 1:8 И как бы для девицы, любящей украшение, берут они золото, и приготовляют венцы на головы богов своих.
\vs Epj 1:9 Бывает также, что жрецы похищают у богов своих золото и серебро и употребляют его на себя самих;
\vs Epj 1:10 уделяют из того и блудницам под их кровом; украшают богов золотых и серебряных и деревянных одеждами, как людей.
\vs Epj 1:11 Но они не спасаются от ржавчины и моли, хотя облечены в пурпуровую одежду.
\vs Epj 1:12 Обтирают лице их от пыли в капище, которой на них очень много.
\vs Epj 1:13 Имеет и скипетр, как человек~--- судья страны, но он не может умертвить виновного пред ним.
\vs Epj 1:14 Имеет меч в правой руке и секиру, а себя самого от войска и разбойников не защитит: отсюда познается, что они не боги; итак, не бойтесь их.
\vs Epj 1:15 Ибо, как разбитый сосуд делается бесполезным для человека, так и боги их.
\vs Epj 1:16 После того, как они поставлены в капищах, глаза их полны пыли от ног входящих.
\vs Epj 1:17 И как у нанесшего оскорбление царю заграждаются входы в жилье, когда он отводится на смерть, \bibemph{так} капища их охраняют жрецы их дверями и замками и засовами, чтобы они не были ограблены разбойниками;
\vs Epj 1:18 зажигают для них светильники, и больше, нежели для себя самих, а они ни одного из них не могут видеть.
\vs Epj 1:19 Они как бревно в доме; сердца их, говорят, точат черви земляные, и съедают их самих и одежду их,~--- а они не чувствуют.
\vs Epj 1:20 Лица их черны от курения в капищах.
\vs Epj 1:21 На тело их и на головы их налетают летучие мыши и ласточки и другие птицы, \bibemph{лазают} также по ним и кошки.
\vs Epj 1:22 Из этого уразумеете, что это не боги; итак, не бойтесь их.
\rsbpar\vs Epj 1:23 Если кто не очистит от ржавчины золота, которым они обложены для красы, то они не будут блестеть; и когда выливали их, они не чувствовали.
\vs Epj 1:24 За большую цену они куплены, а духа нет в них.
\vs Epj 1:25 Безногие, они носятся на плечах, показывая чрез то свою ничтожность людям; посрамляются же и служащие им;
\vs Epj 1:26 потому что, в случае падения их на землю, сами собою они не могут встать; также, если бы кто поставил их прямо, не могут сами собою двигаться и, если бы кто наклонил их, не могут выпрямиться; но как перед мертвыми полагают перед ними дары.
\vs Epj 1:27 Жертвы их жрецы продают и злоупотребляют ими; равно и жены их часть из них солят, и ничего не уделяют ни нищему, ни больному.
\vs Epj 1:28 К жертвам их прикасаются женщины нечистые и родильницы. Итак, познав из сего, что они не боги, не бойтесь их.
\vs Epj 1:29 Как же назвать их богами? женщины приносят жертвы этим серебряным и золотым и деревянным богам.
\vs Epj 1:30 И в капищах их сидят жрецы в разодранных одеждах, с обритыми головами и бородами и с непокрытыми головами:
\vs Epj 1:31 ревут они с воплем пред своими богами, как иные на поминках по умершим.
\vs Epj 1:32 Некоторые из одежд их жрецы берут себе и одевают ими своих жен и детей.
\vs Epj 1:33 Если испытывают от кого-либо злое или доброе, не могут воздать; не могут поставить царя, ни низложить его.
\vs Epj 1:34 Равно ни богатства, ни даже мелкой медной монеты они не могут дать. Если кто, обещав им обет, не исполнил бы его, не взыщут.
\vs Epj 1:35 От смерти человека не избавят, ни слабейшего у сильного не отнимут;
\vs Epj 1:36 человеку слепому не возвратят зрения; человеку в нужде не помогут;
\vs Epj 1:37 вдове не окажут сострадания, и сироте не сделают добра.
\vs Epj 1:38 Камням из гор подобны \bibemph{эти боги} деревянные и оправленные в золото и серебро,~--- и служащие им посрамятся.
\vs Epj 1:39 Как же можно подумать или сказать, что они боги?
\vs Epj 1:40 К тому же сами Халдеи обращаются с ними непочтительно: они, когда увидят немого, не могущего говорить, приносят его к Ваалу и требуют, чтобы он говорил, как будто он может чувствовать.
\vs Epj 1:41 И не могут они, заметив это, оставить их, потому что не имеют смысла.
\vs Epj 1:42 Женщины, обвязавшись тростниковым поясом, сидят на улицах, сожигая курение из оливковых зерен.
\vs Epj 1:43 И когда какая-либо из них, увлеченная проходящим, переспит с ним,~--- попрекает своей подруге, что та не удостоена того же, как она, и что перевязь ее не разорвана.
\vs Epj 1:44 Все, совершающееся у них, ложно. Посему как можно думать или говорить, что они боги?
\vs Epj 1:45 Устроены они художниками и плавильщиками золота; не чем иным они не делаются, как тем, чем желали их сделать художники.
\vs Epj 1:46 И те, которые приготовляют их, не бывают долговечны;
\vs Epj 1:47 как же сделанные ими могут быть богами? Они оставили по себе ложь и срам своим потомкам.
\vs Epj 1:48 Когда постигают их война и бедствия, жрецы совещаются между собою, где бы им скрыться с ними.
\vs Epj 1:49 Как же не понять, что те не боги, которые самих себя не спасают ни от войн, ни от бедствий?
\vs Epj 1:50 Так как они деревянные и оправленные в золото и серебро, то можно познать, что они ложь; всем народам и царям сделается ясным, что это не боги, а дела рук человеческих, и в них нет никакого действия божественного.
\vs Epj 1:51 Кому же после сего не понятно, что они не боги?
\vs Epj 1:52 Царя стране они не поставят, дождя людям не дадут;
\vs Epj 1:53 суда не рассудят, обидимого не защитят, будучи бессильны,
\vs Epj 1:54 как вор\acc{о}ны, находящиеся между небом и землею. Ибо и в том случае, когда подверглось бы пожару капище богов деревянных или оправленных в золото и серебро, жрецы их убегут и спасутся,~--- а они сами, как бревна в средине, сгорят.
\vs Epj 1:55 Ни царю, ни врагам они не могут противостать. Как же можно принять или подумать, что они боги?
\vs Epj 1:56 Ни от воров, ни от грабителей не могут охранить самих себя эти боги, деревянные и оправленные в серебро и золото:
\vs Epj 1:57 превосходя их силою, они снимают золото и серебро и одежды, которые на них, и уходят с добычею, а эти себе самим не в силах помочь.
\vs Epj 1:58 Поэтому лучше царь, выказывающий мужество, или полезный в доме сосуд, который употребляет хозяин, нежели ложные боги; или \bibemph{лучше} дверь в доме, охраняющая в нем имущество, нежели ложные боги; или \bibemph{лучше} деревянный столп в царском дворце, нежели ложные боги.
\vs Epj 1:59 Солнце и луна и звезды, будучи светлы и посылаемы ради потребности, благопослушны.
\vs Epj 1:60 Также и молния каждый раз, как является, ясно видима; также ветер во всякой стране веет.
\vs Epj 1:61 И облака, когда повелит им Бог пройти над всею вселенною, исполняют повеление.
\vs Epj 1:62 Тоже огонь, свыше ниспосылаемый для истребления гор и лесов, делает, что назначено; а эти не подобны им ни видом, ни силами.
\vs Epj 1:63 Почему же можно подумать или сказать, что они боги, когда они несильны ни суда рассудить, ни добра делать людям?
\vs Epj 1:64 Итак, зная, что они не боги, не бойтесь их.
\vs Epj 1:65 Царей они ни проклянут, ни благословят;
\vs Epj 1:66 знамений не покажут на небе и пред народами; не осветят, как солнце, и не осияют, как луна.
\vs Epj 1:67 Звери лучше их: они, убегая под кров, могут помочь себе.
\vs Epj 1:68 Итак, ни из чего не видно нам, что они боги; посему не бойтесь их.
\vs Epj 1:69 Как пугало в огороде ничего не сбережет, так и их деревянные, оправленные в золото и серебро боги.
\vs Epj 1:70 Равным образом их деревянные, оправленные в золото и серебро боги подобны терновому кусту в саду, на который садятся всякие птицы, также и трупу, брошенному во тьме.
\vs Epj 1:71 Из пурпура и червленицы, которые истлевают на них, вы можете уразуметь, что они не боги; да и сами они будут наконец съедены и будут позором в стране.
\rsbpar\vs Epj 1:72 Итак, лучше человек праведный, не имеющий идолов, ибо он~--- далеко от позора.

\include{tex/Bar}
\include{tex/Eze}
\include{tex/Dan}\newbookpage
\include{tex/Hos}
\include{tex/Joe}\newbookpage
\include{tex/Amo}
\include{tex/Oba}\newbookpage
\include{tex/Jon}
\include{tex/Mic}
\include{tex/Nah}\newbookpage
\include{tex/Hab}
\bibbookdescr{Zep}{
  inline={\LARGE Книга\\\Huge Пророка Софонии},
  toc={Софония},
  bookmark={Софония},
  header={Софония},
  %headerleft={},
  %headerright={},
  abbr={Соф}
}
\vs Zep 1:1 Слово Господне, которое было к Софонии, сыну Хусия, сыну Годолии, сыну Амории, сыну Езекии, во дни Иосии, сына Амонова, царя Иудейского.
\rsbpar\vs Zep 1:2 Все истреблю с лица земли, говорит Господь:
\vs Zep 1:3 истреблю людей и скот, истреблю птиц небесных и рыб морских, и соблазны вместе с нечестивыми; истреблю людей с лица земли, говорит Господь.
\vs Zep 1:4 И простру руку Мою на Иудею и на всех жителей Иерусалима: истреблю с места сего остатки Ваала, имя жрецов со священниками,
\vs Zep 1:5 и тех, которые на кровлях поклоняются воинству небесному, и тех поклоняющихся, которые клянутся Господом и клянутся царем своим,
\vs Zep 1:6 и тех, которые отступили от Господа, не искали Господа и не вопрошали о Нем.
\vs Zep 1:7 Умолкни пред лицем Господа Бога! ибо близок день Господень: уже приготовил Господь жертвенное заклание, назначил, кого позвать.
\vs Zep 1:8 И будет в день жертвы Господней: Я посещу князей и сыновей царя и всех, одевающихся в одежду иноплеменников;
\vs Zep 1:9 посещу в тот день всех, которые перепрыгивают через порог, которые дом Господа своего наполняют насилием и обманом.
\vs Zep 1:10 И будет в тот день, говорит Господь, вопль у ворот рыбных и рыдание у других ворот и великое разрушение на холмах.
\vs Zep 1:11 Рыдайте, жители нижней части города, ибо исчезнет весь торговый народ и истреблены будут обремененные серебром.
\vs Zep 1:12 И будет в то время: Я со светильником осмотрю Иерусалим и накажу тех, которые сидят на дрожжах своих и говорят в сердце своем: <<не делает Господь ни добра, ни зла>>.
\vs Zep 1:13 И обратятся богатства их в добычу и домы их~--- в запустение; они построят домы, а жить в них не будут, насадят виноградники, а вина из них не будут пить.
\rsbpar\vs Zep 1:14 Близок великий день Господа, близок, и очень поспешает: уже слышен голос дня Господня; горько возопиет тогда и самый храбрый!
\vs Zep 1:15 День гнева~--- день сей, день скорби и тесноты, день опустошения и разорения, день тьмы и мрака, день облака и мглы,
\vs Zep 1:16 день трубы и бранного крика против укрепленных городов и высоких башен.
\vs Zep 1:17 И Я стесню людей, и они будут ходить, как слепые, потому что они согрешили против Господа, и разметана будет кровь их, как прах, и плоть их~--- как помет.
\vs Zep 1:18 Ни серебро их, ни золото их не может спасти их в день гнева Господа, и огнем ревности Его пожрана будет вся эта земля, ибо истребление, и притом внезапное, совершит Он над всеми жителями земли.
\vs Zep 2:1 Исследуйте себя внимательно, исследуйте, народ необузданный,
\vs Zep 2:2 доколе не пришло определение~--- день пролетит как мякина~--- доколе не пришел на вас пламенный гнев Господень, доколе не наступил для вас день ярости Господней.
\vs Zep 2:3 Взыщите Господа, все смиренные земли, исполняющие законы Его; взыщите правду, взыщите смиренномудрие; может быть, вы укроетесь в день гнева Господня.
\vs Zep 2:4 Ибо Газа будет покинута и Аскалон опустеет, Азот будет выгнан среди дня и Екрон искоренится.
\rsbpar\vs Zep 2:5 Горе жителям приморской страны, народу Критскому! Слово Господне на вас, Хананеи, земля Филистимская! Я истреблю тебя, и не будет у тебя жителей,~---
\vs Zep 2:6 и будет приморская страна пастушьим овчарником и загоном для скота.
\vs Zep 2:7 И достанется этот край остаткам дома Иудина, и будут пасти там, и в домах Аскалона будут вечером отдыхать, ибо Господь Бог их посетит их и возвратит плен их.
\vs Zep 2:8 Слышал Я поношение Моава и ругательства сынов Аммоновых, как они издевались над Моим народом и величались на пределах его.
\vs Zep 2:9 Посему, живу Я! говорит Господь Саваоф, Бог Израилев: Моав будет, как Содом, и сыны Аммона будут, как Гоморра, достоянием крапивы, соляною рытвиною, пустынею навеки; остаток народа Моего возьмет их в добычу, и уцелевшие из людей Моих получат их в наследие.
\vs Zep 2:10 Это им за высокомерие их, за то, что они издевались и величались над народом Господа Саваофа.
\vs Zep 2:11 Страшен будет для них Господь, ибо истребит всех богов земли, и Ему будут поклоняться, каждый со своего места, все острова народов.
\vs Zep 2:12 И вы, Ефиопляне, избиты будете мечом Моим.
\vs Zep 2:13 И прострет Он руку Свою на север, и уничтожит Ассура, и обратит Ниневию в развалины, в место сухое, как пустыня,
\vs Zep 2:14 и покоиться будут среди нее стада и всякого рода животные; пеликан и еж будут ночевать в резных украшениях ее; голос их будет раздаваться в окнах, разрушение обнаружится на дверных столбах, ибо не станет на них кедровой обшивки.
\vs Zep 2:15 Вот чем будет город торжествующий, живущий беспечно, говорящий в сердце своем: <<я, и нет иного кроме меня>>. Как он стал развалиною, логовищем для зверей! Всякий, проходя мимо него, посвищет и махнет рукою.
\vs Zep 3:1 Горе городу нечистому и оскверненному, притеснителю!
\vs Zep 3:2 Не слушает голоса, не принимает наставления, на Господа не уповает, к Богу своему не приближается.
\vs Zep 3:3 Князья его посреди него~--- рыкающие львы, судьи его~--- вечерние волки, не оставляющие до утра ни одной кости.
\vs Zep 3:4 Пророки его~--- люди легкомысленные, вероломные; священники его оскверняют святыню, попирают закон.
\vs Zep 3:5 Господь праведен посреди него, не делает неправды, каждое утро являет суд Свой неизменно; но беззаконник не знает стыда.
\vs Zep 3:6 Я истребил народы, разрушены твердыни их; пустыми сделал улицы их, так что никто уже не ходит по ним; разорены города их: нет ни одного человека, нет жителей.
\vs Zep 3:7 Я говорил: <<бойся только Меня, принимай наставление!>> и не будет истреблено жилище его, и не постигнет его зло, какое Я постановил о нем; а они прилежно старались портить все свои действия.
\rsbpar\vs Zep 3:8 Итак ждите Меня, говорит Господь, до того дня, когда Я восстану для опустошения, ибо Мною определено собрать народы, созвать царства, чтобы излить на них негодование Мое, всю ярость гнева Моего; ибо огнем ревности Моей пожрана будет вся земля.
\vs Zep 3:9 Тогда опять Я дам народам уста чистые, чтобы все призывали имя Господа и служили Ему единодушно.
\vs Zep 3:10 Из заречных стран Ефиопии поклонники Мои, дети рассеянных Моих, принесут Мне дары.
\vs Zep 3:11 В тот день ты не будешь срамить себя всякими поступками твоими, какими ты грешил против Меня, ибо тогда Я удалю из среды твоей тщеславящихся твоею знатностью, и не будешь более превозноситься на святой горе Моей.
\vs Zep 3:12 Но оставлю среди тебя народ смиренный и простой, и они будут уповать на имя Господне.
\vs Zep 3:13 Остатки Израиля не будут делать неправды, не станут говорить лжи, и не найдется в устах их языка коварного, ибо сами будут пастись и покоиться, и никто не потревожит их.
\vs Zep 3:14 Ликуй, дщерь Сиона! торжествуй, Израиль! веселись и радуйся от всего сердца, дщерь Иерусалима!
\vs Zep 3:15 Отменил Господь приговор над тобою, прогнал врага твоего! Господь, царь Израилев, посреди тебя: уже более не увидишь зла.
\vs Zep 3:16 В тот день скажут Иерусалиму: <<не бойся>>, и Сиону: <<да не ослабевают руки твои!>>
\vs Zep 3:17 Господь Бог твой среди тебя, Он силен спасти тебя; возвеселится о тебе радостью, будет милостив по любви Своей, будет торжествовать о тебе с ликованием.
\vs Zep 3:18 Сетующих о торжественных празднествах Я соберу: твои они, на них тяготеет поношение.
\vs Zep 3:19 Вот, Я стесню всех притеснителей твоих в то время и спасу хромлющее, и соберу рассеянное, и приведу их в почет и именитость на всей этой земле поношения их.
\vs Zep 3:20 В то время приведу вас и тогда же соберу вас, ибо сделаю вас именитыми и почетными между всеми народами земли, когда возвращу плен ваш перед глазами вашими, говорит Господь.

\include{tex/Hag}
\include{tex/Zec}
\bibbookdescr{Mal}{
  inline={\LARGE Книга\\\Huge Пророка Малахии},
  toc={Малахия},
  bookmark={Малахия},
  header={Малахия},
  %headerleft={},
  %headerright={},
  abbr={Мал}
}
\vs Mal 1:1 Пророческое слово Господа к Израилю через Малахию.
\rsbpar\vs Mal 1:2 Я возлюбил вас, говорит Господь. А вы говорите: <<в чем явил Ты любовь к нам?>>~--- Не брат ли Исав Иакову? говорит Господь; и однако же Я возлюбил Иакова,
\vs Mal 1:3 а Исава возненавидел и предал горы его опустошению, и владения его~--- шакалам пустыни.
\vs Mal 1:4 Если Едом скажет: <<мы разорены, но мы восстановим разрушенное>>, то Господь Саваоф говорит: они построят, а Я разрушу, и прозовут их областью нечестивою, народом, на который Господь прогневался навсегда.
\vs Mal 1:5 И увидят это глаза ваши, и вы скажете: <<возвеличился Господь над пределами Израиля!>>
\vs Mal 1:6 Сын чтит отца и раб~--- господина своего; если Я Отец, то где почтение ко Мне? и если Я Господь, то где благоговение предо Мною? говорит Господь Саваоф вам, священники, бесславящие имя Мое. Вы говорите: <<чем мы бесславим имя Твое?>>
\vs Mal 1:7 Вы приносите на жертвенник Мой нечистый хлеб, а говорите: <<чем мы бесславим Тебя?>>~--- Тем, что говорите: <<трапеза Господня не стоит уважения>>.
\vs Mal 1:8 И когда приносите в жертву слепое, не худо ли это? или когда приносите хромое и больное, не худо ли это? Поднеси это твоему князю; будет ли он доволен тобою и благосклонно ли примет тебя? говорит Господь Саваоф.
\vs Mal 1:9 Итак молитесь Богу, чтобы помиловал нас; а когда такое исходит из рук ваших, то может ли Он милостиво принимать вас? говорит Господь Саваоф.
\vs Mal 1:10 Лучше кто-нибудь из вас запер бы двери, чтобы напрасно не держали огня на жертвеннике Моем. Нет Моего благоволения к вам, говорит Господь Саваоф, и приношение из рук ваших неблагоугодно Мне.
\vs Mal 1:11 Ибо от востока солнца до запада велико будет имя Мое между народами, и на всяком месте будут приносить фимиам имени Моему, чистую жертву; велико будет имя Мое между народами, говорит Господь Саваоф.
\vs Mal 1:12 А вы хулите его тем, что говорите: <<трапеза Господня не стоит уважения, и доход от нее~--- пища ничтожная>>.
\vs Mal 1:13 Притом говорите: <<вот сколько труда!>> и пренебрегаете ею, говорит Господь Саваоф, и приносите украденное, хромое и больное, и такого же свойства приносите хлебный дар: могу ли с благоволением принимать это из рук ваших? говорит Господь.
\vs Mal 1:14 Проклят лживый, у которого в стаде есть неиспорченный самец, и он дал обет, а приносит в жертву Господу поврежденное: ибо Я Царь великий, и имя Мое страшно у народов.
\vs Mal 2:1 Итак для вас, священники, эта заповедь:
\vs Mal 2:2 если вы не послушаетесь и если не примете к сердцу, чтобы воздавать славу имени Моему, говорит Господь Саваоф, то Я пошлю на вас проклятие и прокляну ваши благословения, и уже проклинаю, потому что вы не хотите приложить к тому сердца.
\vs Mal 2:3 Вот, Я отниму у вас плечо, и помет раскидаю на лица ваши, помет праздничных жертв ваших, и выбросят вас вместе с ним.
\vs Mal 2:4 И вы узнаете, что Я дал эту заповедь для сохранения завета Моего с Левием, говорит Господь Саваоф.
\vs Mal 2:5 Завет Мой с ним был \bibemph{завет} жизни и мира, и Я дал его ему для страха, и он боялся Меня и благоговел пред именем Моим.
\vs Mal 2:6 Закон истины был в устах его, и неправды не обреталось на языке его; в мире и правде он ходил со Мною и многих отвратил от греха.
\vs Mal 2:7 Ибо уста священника должны хранить ведение, и закона ищут от уст его, потому что он вестник Господа Саваофа.
\vs Mal 2:8 Но вы уклонились от пути сего, для многих послужили соблазном в законе, разрушили завет Левия, говорит Господь Саваоф.
\vs Mal 2:9 За то и Я сделаю вас презренными и униженными перед всем народом, так как вы не соблюдаете путей Моих, лицеприятствуете в делах закона.
\vs Mal 2:10 Не один ли у всех нас Отец? Не один ли Бог сотворил нас? Почему же мы вероломно поступаем друг против друга, нарушая тем завет отцов наших?
\vs Mal 2:11 Вероломно поступает Иуда, и мерзость совершается в Израиле и в Иерусалиме; ибо унизил Иуда святыню Господню, которую любил, и женился на дочери чужого бога.
\vs Mal 2:12 У того, кто делает это, истребит Господь из шатров Иаковлевых бдящего на страже и отвечающего, и приносящего жертву Господу Саваофу.
\vs Mal 2:13 И вот еще что вы делаете: вы заставляете обливать слезами жертвенник Господа с рыданием и воплем, так что Он уже не призирает более на приношение и не принимает умилостивительной жертвы из рук ваших.
\vs Mal 2:14 Вы скажете: <<за что?>> За то, что Господь был свидетелем между тобою и женою юности твоей, против которой ты поступил вероломно, между тем как она подруга твоя и законная жена твоя.
\vs Mal 2:15 Но не сделал ли того же один, и в нем пребывал превосходный дух? что же сделал этот один? он желал получить от Бога потомство. Итак берегите дух ваш, и никто не поступай вероломно против жены юности своей.
\vs Mal 2:16 Если ты ненавидишь ее, отпусти, говорит Господь Бог Израилев; обида покроет одежду его, говорит Господь Саваоф; посему наблюдайте за духом вашим и не поступайте вероломно.
\vs Mal 2:17 Вы прогневляете Господа словами вашими и говорите: <<чем прогневляем мы Его?>> Тем, что говорите: <<всякий, делающий зло, хорош пред очами Господа, и к таким Он благоволит>>, или: <<где Бог правосудия?>>
\vs Mal 3:1 Вот, Я посылаю Ангела Моего, и он приготовит путь предо Мною, и внезапно придет в храм Свой Господь, Которого вы ищете, и Ангел завета, Которого вы желаете; вот, Он идет, говорит Господь Саваоф.
\vs Mal 3:2 И кто выдержит день пришествия Его, и кто устоит, когда Он явится? Ибо Он~--- как огонь расплавляющий и как щелок очищающий,
\vs Mal 3:3 и сядет переплавлять и очищать серебро, и очистит сынов Левия и переплавит их, как золото и как серебро, чтобы приносили жертву Господу в правде.
\vs Mal 3:4 Тогда благоприятна будет Господу жертва Иуды и Иерусалима, как во дни древние и как в лета прежние.
\vs Mal 3:5 И приду к вам для суда и буду скорым обличителем чародеев и прелюбодеев и тех, которые клянутся ложно и удерживают плату у наемника, притесняют вдову и сироту, и отталкивают пришельца, и Меня не боятся, говорит Господь Саваоф.
\vs Mal 3:6 Ибо Я~--- Господь, Я не изменяюсь; посему вы, сыны Иакова, не уничтожились.
\vs Mal 3:7 Со дней отцов ваших вы отступили от уставов Моих и не соблюдаете их; обратитесь ко Мне, и Я обращусь к вам, говорит Господь Саваоф. Вы скажете: <<как нам обратиться?>>
\vs Mal 3:8 Можно ли человеку обкрадывать Бога? А вы обкрадываете Меня. Скажете: <<чем обкрадываем мы Тебя?>> Десятиною и приношениями.
\vs Mal 3:9 Проклятием вы прокляты, потому что вы~--- весь народ~--- обкрадываете Меня.
\vs Mal 3:10 Принесите все десятины в дом хранилища, чтобы в доме Моем была пища, и хотя в этом испытайте Меня, говорит Господь Саваоф: не открою ли Я для вас отверстий небесных и не изолью ли на вас благословения до избытка?
\vs Mal 3:11 Я для вас запрещу пожирающим истреблять у вас плоды земные, и виноградная лоза на поле у вас не лишится плодов своих, говорит Господь Саваоф.
\vs Mal 3:12 И блаженными называть будут вас все народы, потому что вы будете землею вожделенною, говорит Господь Саваоф.
\vs Mal 3:13 Дерзостны предо Мною слова ваши, говорит Господь. Вы скажете: <<что мы говорим против Тебя?>>
\vs Mal 3:14 Вы говорите: <<тщетно служение Богу, и что пользы, что мы соблюдали постановления Его и ходили в печальной одежде пред лицем Господа Саваофа?
\vs Mal 3:15 И ныне мы считаем надменных счастливыми: лучше устраивают себя делающие беззакония, и хотя искушают Бога, но остаются целы>>.
\vs Mal 3:16 Но боящиеся Бога говорят друг другу: <<внимает Господь и слышит это, и пред лицем Его пишется памятная книга о боящихся Господа и чтущих имя Его>>.
\vs Mal 3:17 И они будут Моими, говорит Господь Саваоф, собственностью Моею в тот день, который Я соделаю, и буду миловать их, как милует человек сына своего, служащего ему.
\vs Mal 3:18 И тогда снова увидите различие между праведником и нечестивым, между служащим Богу и не служащим Ему.
\vs Mal 4:1 Ибо вот, придет день, пылающий как печь; тогда все надменные и поступающие нечестиво будут как солома, и попалит их грядущий день, говорит Господь Саваоф, так что не оставит у них ни корня, ни ветвей.
\vs Mal 4:2 А для вас, благоговеющие пред именем Моим, взойдет Солнце правды и исцеление в лучах Его, и вы выйдете и взыграете, как тельцы упитанные;
\vs Mal 4:3 и будете попирать нечестивых, ибо они будут прахом под стопами ног ваших в тот день, который Я соделаю, говорит Господь Саваоф.
\vs Mal 4:4 Помните закон Моисея, раба Моего, который Я заповедал ему на Хориве для всего Израиля, равно как и правила и уставы.
\vs Mal 4:5 Вот, Я пошлю к вам Илию пророка пред наступлением дня Господня, великого и страшного.
\vs Mal 4:6 И он обратит сердца отцов к детям и сердца детей к отцам их, чтобы Я, придя, не поразил земли проклятием.

\bibbookdescr{1Ma}{
  inline={\LARGE Первая книга\\\Huge Маккавейская\fns{Книги Маккавейские переведены с греческого, потому что в еврейском тексте их нет.}},
  toc={1-я Маккавейская*},
  bookmark={1-я Маккавейская},
  header={1-я Маккавейская},
  %headerleft={},
  %headerright={},
  abbr={1~Мак}
}
\vs 1Ma 1:1 После того как Александр, сын Филиппа, Македонянин, который вышел из земли Киттим, поразил Дария, царя Персидского и Мидийского, и воцарился вместо него прежде над Елладою,~---
\vs 1Ma 1:2 он произвел много войн и овладел многими укрепленными местами, и убивал царей земли.
\vs 1Ma 1:3 И прошел до пределов земли и взял добычу от множества народов; и умолкла земля пред ним, и он возвысился, и вознеслось сердце его.
\vs 1Ma 1:4 Он собрал весьма сильное войско и господствовал над областями и народами и властителями, и они сделались его данниками.
\vs 1Ma 1:5 После того он слег в постель и, почувствовав, что умирает,
\vs 1Ma 1:6 призвал знатных из слуг своих, которые были воспитаны с ним от юности, и разделил им свое царство еще при жизни своей.
\rsbpar\vs 1Ma 1:7 Александр царствовал двенадцать лет и умер.
\vs 1Ma 1:8 И владычествовали слуги его каждый в своем месте.
\vs 1Ma 1:9 И по смерти его все они возложили на себя венцы, а после них и сыновья их в течение многих лет; и умножили зло на земле.
\vs 1Ma 1:10 И вышел от них корень греха~--- Антиох Епифан, сын царя Антиоха, который был заложником в Риме, и воцарился в сто тридцать седьмом году царства Еллинского.
\rsbpar\vs 1Ma 1:11 В те дни вышли из Израиля сыны беззаконные и убеждали многих, говоря: пойдем и заключим союз с народами, окружающими нас, ибо с тех пор, как мы отделились от них, постигли нас многие бедствия.
\vs 1Ma 1:12 И добрым показалось это слово в глазах их.
\vs 1Ma 1:13 Некоторые из народа изъявили желание и отправились к царю; и он дал им право исполнять установления языческие.
\vs 1Ma 1:14 Они построили в Иерусалиме училище по обычаю языческому
\vs 1Ma 1:15 и установили у себя необрезание, и отступили от святаго завета, и соединились с язычниками, и продались, чтобы делать зло.
\rsbpar\vs 1Ma 1:16 Когда Антиох увидел, что царство укрепилось, предпринял воцариться над Египтом, чтобы царствовать над двумя царствами,
\vs 1Ma 1:17 и вошел он в Египет с сильным ополчением, с колесницами, и слонами, и всадниками, и множеством кораблей;
\vs 1Ma 1:18 и вступил в сражение с Птоломеем, царем Египетским; и убоялся Птоломей от лица его и обратился в бегство, и много пало раненых.
\vs 1Ma 1:19 И овладели они укрепленными городами в земле Египетской, и взял он добычу из земли Египетской.
\rsbpar\vs 1Ma 1:20 После поражения Египта Антиох возвратился в сто сорок третьем году и пошел против Израиля, и вступил в Иерусалим с сильным ополчением;
\vs 1Ma 1:21 вошел во святилище с надменностью и взял золотой жертвенник, светильник и все сосуды его,
\vs 1Ma 1:22 и трапезу предложения, и возлияльники, и чаши, и кадильницы золотые, и завесу, и венцы, и золотое украшение, бывшее снаружи храма, и всё обобрал.
\vs 1Ma 1:23 Взял и серебро, и золото, и драгоценные сосуды, и взял скрытые сокровища, какие отыскал.
\vs 1Ma 1:24 И, взяв всё, отправился в землю свою и совершил убийства, и говорил с великою надменностью.
\vs 1Ma 1:25 Посему был великий плач в Израиле, во всех местах его.
\vs 1Ma 1:26 Стенали начальники и старейшины, изнемогали девы и юноши, и изменилась красота женская.
\vs 1Ma 1:27 Всякий жених предавался плачу, и сидящая в брачном чертоге была в скорби.
\vs 1Ma 1:28 Вострепетала земля за обитающих на ней, и весь дом Иакова облекся стыдом.
\rsbpar\vs 1Ma 1:29 По прошествии двух лет послал царь начальника податей в города Иуды, и он пришел в Иерусалим с большою толпою;
\vs 1Ma 1:30 коварно говорил им слова мира, и они поверили ему; но он внезапно напал на город и поразил его великим поражением, и погубил множество народа Израильского;
\vs 1Ma 1:31 взял добычи из города и сожег его огнем, и разрушил домы его и стены его кругом;
\vs 1Ma 1:32 и увели в плен жен и детей, и овладели скотом.
\vs 1Ma 1:33 Оградили город Давидов большою и крепкою стеною и крепкими башнями, и сделался он для них крепостью.
\vs 1Ma 1:34 И поместили там народ нечестивый, людей беззаконных, и они укрепились в ней;
\vs 1Ma 1:35 запаслись оружием и продовольствием и, собрав добычи Иерусалимские, сложили там, и сделались большою сетью.
\vs 1Ma 1:36 И было это постоянною засадою для святилища и злым диаволом для Израиля.
\vs 1Ma 1:37 Они проливали невинную кровь вокруг святилища и оскверняли святилище.
\vs 1Ma 1:38 Жители же Иерусалима разбежались ради них, и он сделался жилищем чужих и стал чужим для своего рода, и дети его оставили его.
\vs 1Ma 1:39 Святилище его запустело, как пустыня, праздники его обратились в плач, субботы его~--- в поношение, честь его~--- в уничижение.
\vs 1Ma 1:40 По мере славы его увеличилось бесчестие его, и высота его обратилась в печаль.
\rsbpar\vs 1Ma 1:41 Царь Антиох написал всему царству своему, чтобы все были одним народом
\vs 1Ma 1:42 и чтобы каждый оставил свой закон. И согласились все народы по слову царя.
\vs 1Ma 1:43 И многие из Израиля приняли идолослужение его и принесли жертвы идолам, и осквернили субботу.
\vs 1Ma 1:44 Царь послал через вестников грамоты в Иерусалим и в города Иудейские, чтобы они следовали узаконениям, чужим для сей земли,
\vs 1Ma 1:45 и чтобы не допускались всесожжения и жертвоприношения, и возлияние в святилище, чтобы ругались над субботами и праздниками
\vs 1Ma 1:46 и оскверняли святилище и святых,
\vs 1Ma 1:47 чтобы строили жертвенники, храмы и капища идольские, и приносили в жертву свиные мяса и скотов нечистых,
\vs 1Ma 1:48 и оставляли сыновей своих необрезанными, и оскверняли души их всякою нечистотою и мерзостью,
\vs 1Ma 1:49 для того, чтобы забыли закон и изменили все постановления.
\vs 1Ma 1:50 А если кто не сделает по слову царя, да будет предан смерти.
\vs 1Ma 1:51 Согласно этому писал он всему царству своему и поставил надзирателей над всем народом, и повелел городам Иудейским приносить жертвы во всяком городе.
\vs 1Ma 1:52 И собрались к ним многие из народа, все, которые оставили закон,~--- и совершили зло в земле;
\vs 1Ma 1:53 и заставили Израиля укрываться во всяком убежище его.
\rsbpar\vs 1Ma 1:54 В пятнадцатый день Хаслева, сто сорок пятого года, устроили на жертвеннике мерзость запустения, и в городах Иудейских вокруг построили жертвенники,
\vs 1Ma 1:55 и перед дверями домов и на улицах совершали курения,
\vs 1Ma 1:56 и книги закона, какие находили, разрывали и сожигали огнем;
\vs 1Ma 1:57 у кого находили книгу завета и кто держался закона, того, по повелению царя, предавали смерти.
\vs 1Ma 1:58 С таким насилием поступали они с Израильтянами, приходившими каждый месяц в города.
\vs 1Ma 1:59 И в двадцать пятый день месяца, принося жертвы на жертвеннике, который был над алтарем,
\vs 1Ma 1:60 они, по данному повелению, убивали жен, обрезавших детей своих,
\vs 1Ma 1:61 а младенцев вешали за шеи их, домы их расхищали и совершавших над ними обрезание убивали.
\vs 1Ma 1:62 Но многие в Израиле остались твердыми и укрепились, чтобы не есть нечистого,
\vs 1Ma 1:63 и предпочли умереть, чтобы не оскверниться пищею и не поругать святаго завета,~--- и умирали.
\vs 1Ma 1:64 И был весьма великий гнев над Израилем.
\vs 1Ma 2:1 В те дни восстал Маттафия, сын Иоанна, сына Симеонова, священник из сынов Иоарива из Иерусалима; жил он в Модине.
\vs 1Ma 2:2 У него было пять сыновей: Иоанн, прозываемый Гаддис,
\vs 1Ma 2:3 Симон, называемый Фасси,
\vs 1Ma 2:4 Иуда, прозываемый Маккавей,
\vs 1Ma 2:5 Елеазар, прозываемый Аваран, Ионафан, прозываемый Апфус.
\vs 1Ma 2:6 Видя богохульства, происходившие в Иудее и Иерусалиме,
\vs 1Ma 2:7 он сказал: горе мне! для чего родился я видеть разорение народа моего и разорение святаго города и оставаться здесь, когда он предан в руки врагов и святилище~--- в руки чужих?
\vs 1Ma 2:8 Храм его сделался, как муж бесславный,
\vs 1Ma 2:9 драгоценные сосуды его унесены в плен, младенцы его избиты на улицах, юноши его пали от меча врага.
\vs 1Ma 2:10 Какой народ не занимал царства его и не овладевал добычами его?
\vs 1Ma 2:11 Все украшение его отнято; из свободного он сделался рабом.
\vs 1Ma 2:12 И вот святыни наши, и благолепие наше, и слава наша опустели, и язычники осквернили их.
\vs 1Ma 2:13 Для чего нам еще жить?
\vs 1Ma 2:14 И разодрал Маттафия и сыновья его одежды свои, и облеклись во вретища, и горько плакали.
\vs 1Ma 2:15 И пришли от царя в город Модин принуждавшие к отступничеству, чтобы приносить жертвы.
\vs 1Ma 2:16 И многие из Израиля пристали к ним; а Маттафия и сыновья его устояли.
\vs 1Ma 2:17 И отвечали пришедшие от царя и сказали Маттафии: ты вождь, ты славен и велик в этом городе и имеешь опору в сыновьях и братьях.
\vs 1Ma 2:18 Итак, приступи теперь первый и исполни повеление царя, как сделали это все народы и мужи Иудейские и оставшиеся в Иерусалиме, и будешь ты и дом твой в числе друзей царских, и ты и сыновья твои будете почтены и серебром, и золотом, и многими дарами.
\vs 1Ma 2:19 И отвечал Маттафия и сказал громким голосом: если и все народы в области царства царя послушают его и отступят каждый от богослужения отцов своих, и согласятся на повеления его,
\vs 1Ma 2:20 то я и сыновья мои и братья мои будем поступать по завету отцов наших.
\vs 1Ma 2:21 Помилуй нас Бог, чтобы оставить закон и постановления!
\vs 1Ma 2:22 Не послушаем мы слов царя, чтобы отступить нам от нашего богослужения вправо или влево.
\vs 1Ma 2:23 Когда перестал он говорить эти слова, подошел муж Иудеянин пред глазами всех, чтобы принести по повелению царя идольскую жертву на жертвеннике, который был в Модине.
\vs 1Ma 2:24 Увидев это, Маттафия возревновал, и затрепетала внутренность его, и воспламенилась ярость его по законе, и он, подбежав, убил его при жертвеннике.
\vs 1Ma 2:25 И в то же время убил мужа царского, принуждавшего приносить жертву, и разрушил жертвенник.
\vs 1Ma 2:26 И возревновал он по законе, как это сделал Финеес с Замврием, сыном Салома.
\vs 1Ma 2:27 И воскликнул Маттафия в городе громким голосом: всякий, кто ревнует по законе и стоит в завете, да идет вслед за мною!
\vs 1Ma 2:28 И убежал сам и сыновья его в горы, оставив всё, что имели в городе.
\vs 1Ma 2:29 Тогда многие, преданные правде и закону, ушли в пустыню и оставались там,
\vs 1Ma 2:30 сами и сыновья их, и жены их, и скоты их, потому что умножились беды над ними.
\vs 1Ma 2:31 И возвещено было мужам царским и войску, находившемуся в Иерусалиме, городе Давидовом, что некоторые мужи, нарушив царское повеление, ушли в сокровенные места в пустыне.
\vs 1Ma 2:32 И погнались за ними многие и, настигнув их, ополчились, и выстроились к сражению против них в день субботний,
\vs 1Ma 2:33 и сказали им: теперь еще можно; выходите и сделайте по слову царя, и останетесь живы.
\vs 1Ma 2:34 Но они отвечали: не выйдем и не сделаем по слову царя, не оскверним дня субботнего.
\vs 1Ma 2:35 Тогда поспешили начать сражение против них.
\vs 1Ma 2:36 Но они не отвечали им, ни даже камня не бросили на них, ни заградили тайных убежищ своих,
\vs 1Ma 2:37 и сказали: мы все умрем в невинности нашей; небо и земля свидетели за нас, что вы несправедливо губите нас.
\vs 1Ma 2:38 Нападали на них по субботам, и умерло их, и жен их, и детей их со скотом их, до тысячи душ.
\vs 1Ma 2:39 Когда узнал о том Маттафия и друзья его, горько плакали о них;
\vs 1Ma 2:40 и говорили друг другу: если все мы будем поступать так, как поступали эти братья наши, и не будем сражаться с язычниками за жизнь нашу и постановления наши, то они скоро истребят нас с земли.
\vs 1Ma 2:41 И решили они в тот день и сказали: кто бы ни пошел на войну против нас в день субботний, будем сражаться против него, дабы нам не умереть всем, как умерли братья наши в тайных убежищах.
\vs 1Ma 2:42 Тогда собрались к ним множество Иудеев, крепкие силою из Израиля, все верные закону.
\vs 1Ma 2:43 И все, бежавшие от бедствия, присоединились к ним и сделались подкреплением для них.
\vs 1Ma 2:44 Так составили они войско и поражали в гневе своем нечестивых и в ярости своей мужей беззаконных; остальные же бежали для спасения к язычникам.
\vs 1Ma 2:45 И обходил вокруг Маттафия и друзья его, и разрушали жертвенники,
\vs 1Ma 2:46 и небоязненно обрезывали необрезанных детей, сколько находили в пределах Израильских,
\vs 1Ma 2:47 и преследовали сынов гордыни, и дело успешно шло в руках их.
\vs 1Ma 2:48 Так защищали они закон от руки язычников и от руки царей и не дали восторжествовать грешнику.
\rsbpar\vs 1Ma 2:49 Приблизились дни смерти Маттафии, и он сказал сыновьям своим: ныне усилилась гордость и испытание, ныне время переворота и гнев ярости.
\vs 1Ma 2:50 Итак, дети, возревнуйте о законе и отдайте жизнь вашу за завет отцов наших.
\vs 1Ma 2:51 Вспомните о делах отцов наших, которые они совершили во времена свои, и вы приобретете великую славу и вечное имя.
\vs 1Ma 2:52 Авраам не в искушении ли найден был верным? и это вменилось ему в праведность.
\vs 1Ma 2:53 Иосиф в стесненном положении своем сохранил заповедь и сделался господином Египта.
\vs 1Ma 2:54 Финеес, отец наш, за то, что возревновал ревностью, получил завет вечного священства.
\vs 1Ma 2:55 Иисус за исполнение слова сделался судьею над Израилем.
\vs 1Ma 2:56 Халев за свидетельство перед собранием получил в наследие землю.
\vs 1Ma 2:57 Давид за свое милосердие наследовал престол царства навеки.
\vs 1Ma 2:58 Илия за великую ревность по законе взят даже на небо.
\vs 1Ma 2:59 Анания, Азария, Мисаил верою спаслись от пламени.
\vs 1Ma 2:60 Даниил за свою невинность избавлен от челюстей львов.
\vs 1Ma 2:61 Итак, припоминайте от рода до рода, что все, надеющиеся на Него, не изнемогут.
\vs 1Ma 2:62 Не убойтесь слов мужа грешного, ибо слава его обратится в навоз и в червей.
\vs 1Ma 2:63 Сегодня он превозносится, а завтра не найдут его, ибо он обратился в прах свой, и замысел его погиб.
\vs 1Ma 2:64 Но вы, дети мои, крепитесь и мужественно стойте в законе, ибо чрез него вы прославитесь.
\vs 1Ma 2:65 Вот~--- Симон, брат ваш: знаю, что он~--- муж совета, слушайтесь его во все дни; он будет вам вместо отца.
\vs 1Ma 2:66 А Иуда Маккавей, крепкий силою от юности своей, да будет у вас начальником войска, и будет вести войну с народами.
\vs 1Ma 2:67 Итак, соберите к себе всех исполнителей закона и отмщайте за обиды народа вашего;
\vs 1Ma 2:68 воздайте воздаяние язычникам и будьте внимательны к повелениям закона.
\vs 1Ma 2:69 И благословил их и приложился к отцам своим.
\vs 1Ma 2:70 Умер же он на сто сорок шестом году; и сыновья его похоронили его в гробе отцов своих в Модине, и весь Израиль оплакивал его горьким плачем.
\vs 1Ma 3:1 И восстал вместо него Иуда, называемый Маккавей, сын его.
\vs 1Ma 3:2 И помогали ему все братья его и все, которые были привержены к отцу его, и вели войну Израиля с радостью.
\vs 1Ma 3:3 Он распространил славу народа своего; он облекался бронею, как исполин, опоясывался воинскими доспехами своими и вел войну, защищая ополчение мечом;
\vs 1Ma 3:4 он уподоблялся льву в делах своих и был как скимен, рыкающий на добычу;
\vs 1Ma 3:5 он преследовал беззаконных, отыскивая их, и возмущающих народ его сожигал.
\vs 1Ma 3:6 И смирились беззаконные из страха пред ним, и все делатели беззакония смутились пред ним, и благоуспешно было спасение рукою его.
\vs 1Ma 3:7 Он огорчил многих царей и возвеселил Иакова делами своими, и память его до века в благословении;
\vs 1Ma 3:8 прошел по городам Иудеи и истребил в ней нечестивых, и отвратил гнев от Израиля,
\vs 1Ma 3:9 и сделался именитым до последних пределов земли, и собрал погибавших.
\rsbpar\vs 1Ma 3:10 Тогда Аполлоний собрал язычников и из Самарии многочисленное войско, чтобы воевать против Израиля.
\vs 1Ma 3:11 Иуда узнал о том и вышел к нему навстречу, и поразил, и убил его; и много пало пораженных, а остальные убежали.
\vs 1Ma 3:12 И взял Иуда добычу их, и взял меч Аполлония, и сражался им во все дни.
\rsbpar\vs 1Ma 3:13 И услышал Сирон, военачальник Сирии, что Иуда собрал вокруг себя людей и сонм верных, выступающих с ним на войну,
\vs 1Ma 3:14 и сказал: сделаю себе имя и прославлюсь в царстве, и сражусь с Иудою и с теми, которые вместе с ним и которые презирают слово царское.
\vs 1Ma 3:15 И решился он идти, и пошло с ним сильное полчище нечестивых помогать ему и сделать отмщение на сынах Израиля.
\vs 1Ma 3:16 Когда они приблизились к возвышенности Вефорона, Иуда вышел к ним навстречу с очень немногими,
\vs 1Ma 3:17 которые, когда увидели идущее навстречу им войско, сказали Иуде: как можем мы в таком малом числе сражаться против такого сильного множества? И мы же совсем ослабели, еще не евши ныне.
\vs 1Ma 3:18 Но Иуда сказал им: легко и многим попасть в руки немногих, и у Бога небесного нет различия, многими ли спасти, или немногими;
\vs 1Ma 3:19 ибо не от множества войска бывает победа на войне, но с неба приходит сила.
\vs 1Ma 3:20 Они идут против нас во множестве надменности и нечестия, чтобы истребить нас и жен наших и детей наших, чтобы ограбить нас;
\vs 1Ma 3:21 а мы сражаемся за души наши и законы наши.
\vs 1Ma 3:22 Он Сам сокрушит их пред лицем нашим; вы же не страшитесь их.
\vs 1Ma 3:23 Перестав говорить, он внезапно бросился на них, и поражен был Сирон и войско его перед ним.
\vs 1Ma 3:24 И они преследовали его по спуску Вефорона до самой равнины; и пало из них до восьмисот мужей, прочие же убежали в землю Филистимскую.
\vs 1Ma 3:25 И начал страх перед Иудою и братьями его и боязнь нападать на всех окрестных язычников.
\vs 1Ma 3:26 Дошло и до царя имя его, и все народы рассказывали о битвах Иуды.
\rsbpar\vs 1Ma 3:27 Когда же услышал эти речи царь Антиох, то воспылал гневом и, послав, собрал все силы царства своего, весьма сильное ополчение;
\vs 1Ma 3:28 и открыл казнохранилище свое, и выдал войскам своим годовое жалованье, и приказал им быть готовыми на всякую надобность.
\vs 1Ma 3:29 Но увидел, что истощилось серебро в казнохранилищах, а подати страны скудны по причине волнения и разорения, которое он произвел в земле той, уничтожая законы, существовавшие от дней древних.
\vs 1Ma 3:30 И начал он опасаться, что у него недостанет, разве только на раз или два, на издержки и подарки, которые прежде раздавал щедрою рукою и превзошел в том прежних царей.
\vs 1Ma 3:31 Сильно озабоченный в душе своей, он решился идти в Персию и взять подати со стран и собрать побольше серебра.
\vs 1Ma 3:32 А дела царские от реки Евфрата до пределов Египта предоставил Лисию, человеку знаменитому, происходившему от рода царского,
\vs 1Ma 3:33 также и воспитание сына своего, Антиоха, до его возвращения;
\vs 1Ma 3:34 и передал ему половину войск и слонов, дав ему приказания о всем, чего хотел, и о жителях Иудеи и Иерусалима,
\vs 1Ma 3:35 чтобы он послал против них войско сокрушить и уничтожить могущество Израиля и остаток Иерусалима, и истребить память их от места того,
\vs 1Ma 3:36 и поселить во всех пределах их сынов иноплеменных, и разделить по жребию землю их.
\vs 1Ma 3:37 Царь же взял остальную половину войска и отправился из Антиохии, престольного города своего, в сто сорок седьмом году и, перейдя реку Евфрат, прошел верхние страны.
\vs 1Ma 3:38 Лисий избрал Птоломея, сына Дорименова, и Никанора и Горгия, мужей сильных из друзей царя,
\vs 1Ma 3:39 и послал с ними сорок тысяч мужей и семь тысяч всадников, чтобы идти в землю Иудейскую и разорить ее по слову царя.
\vs 1Ma 3:40 Они отправились со всем войском своим и, придя, расположились на равнине близ Еммаума.
\vs 1Ma 3:41 Купцы этой страны услышали имя их и, взяв весьма много серебра и золота и слуг, пришли в стан покупать сынов Израиля в рабы; к ним присоединилось и войско Сирии и земл\acc{и} иноплеменных.
\rsbpar\vs 1Ma 3:42 Увидел Иуда и братья его, что умножились бедствия и войска расположились станом в пределах их; узнали и о повелении царя, которое он приказал исполнить над народом к погублению и истреблению его.
\vs 1Ma 3:43 И говорили каждый ближнему своему: восставим низверженный народ наш и сразимся за народ наш и за святыню.
\vs 1Ma 3:44 И собрался сонм, чтобы быть готовыми к войне и помолиться, и испросить милости и сожаления.
\vs 1Ma 3:45 Иерусалим был необитаем, как пустыня; не было ни входящего в него, ни выходящего из него из природных жителей его; святилище было попрано, и сыновья инородных были в крепости его; он стал жилищем язычников; и отнято веселье у Иакова, и не слышно стало свирели и цитры.
\vs 1Ma 3:46 Итак, они собрались и пошли в Массифу, напротив Иерусалима, ибо место молитвы у Израильтян было прежде в Массифе.
\vs 1Ma 3:47 И постились в этот день, и возложили на себя вретища и пепел на головы свои, и разодрали одежды свои,
\vs 1Ma 3:48 раскрыли книгу закона из тех, которые язычники отыскивали, чтобы сделать на них изображения своих идолов,
\vs 1Ma 3:49 и принесли священнические облачения и первородных и десятины; и созвали назореев, исполнивших дни свои,
\vs 1Ma 3:50 и громко возопили к небу: что нам делать с ними и куда отвести их?
\vs 1Ma 3:51 Святилище Твое попрано и осквернено, и священники Твои в скорби и уничижении.
\vs 1Ma 3:52 И вот, собрались против нас язычники, чтобы истребить нас. Ты знаешь, что умышляют они против нас.
\vs 1Ma 3:53 Как можем мы устоять пред лицем их, если Ты не поможешь нам?
\vs 1Ma 3:54 И вострубили трубами и воскликнули громким голосом.
\rsbpar\vs 1Ma 3:55 После сего Иуда поставил вождей для народа~--- тысяченачальников, стоначальников, пятидесятиначальников и десятиначальников.
\vs 1Ma 3:56 И сказали тем, которые строили дома, обручились с женами, насадили виноградники, и людям боязливым, чтобы каждый из них, по закону, возвратился в свой дом.
\vs 1Ma 3:57 Тогда двинулось ополчение и расположилось станом на юге от Еммаума.
\vs 1Ma 3:58 И сказал Иуда: опояшьтесь и будьте мужественны и готовы к утру сразиться с этими язычниками, которые собрались против нас, чтобы погубить нас и святыню нашу.
\vs 1Ma 3:59 Ибо лучше нам умереть в сражении, нежели видеть бедствия нашего народа и святыни.
\vs 1Ma 3:60 А какая будет воля на небе, так да сотворит!
\vs 1Ma 4:1 И взял Горгий пять тысяч мужей и тысячу отборных всадников, и двинулось ополчение ночью,
\vs 1Ma 4:2 чтобы напасть на ополчение Иудеев и поразить их внезапно, а жившие в крепости служили ему проводниками.
\vs 1Ma 4:3 И услышал Иуда и выступил сам и храбрые мужи, чтобы поразить войско царя в Еммауме,
\vs 1Ma 4:4 доколе силы неприятельские были еще в отдаленности от стана.
\vs 1Ma 4:5 И пришел Горгий в стан Иуды ночью, и никого не нашел, и искал их по горам, ибо говорил: они бегут от нас.
\vs 1Ma 4:6 Но с рассветом дня Иуда явился на равнине с тремя тысячами мужей, но они не имели ни щитов, ни мечей, как того желали.
\vs 1Ma 4:7 Когда увидели они крепкое и вооруженное ополчение язычников и окружающую его конницу, обученных для войны,
\vs 1Ma 4:8 Иуда сказал бывшим с ним мужам: не бойтесь множества их и не страшитесь нападения их.
\vs 1Ma 4:9 Вспомните, как спасены были отцы наши в Чермном море, когда фараон преследовал их с войском.
\vs 1Ma 4:10 И ныне возопием на небо; может быть, Он умилосердится над нами, воспомянув завет с отцами нашими, и сокрушит ныне это ополчение перед лицем нашим;
\vs 1Ma 4:11 и все язычники познают, что есть Избавляющий и Спасающий Израиля.
\vs 1Ma 4:12 Иноплеменники, подняв глаза свои, увидели, что идут против них,
\vs 1Ma 4:13 и вышли из стана на сражение, а бывшие с Иудою затрубили,
\vs 1Ma 4:14 и сошлись, и разбиты были язычники, и побежали на равнину,
\vs 1Ma 4:15 а все остальные пали от меча; и преследовали их до Газера и до равнин Идумеи, Азота и Иамнии, и пали из них до трех тысяч мужей.
\vs 1Ma 4:16 И возвратился Иуда и войско его от преследования их
\vs 1Ma 4:17 и сказал народу: не бросайтесь на добычу, ибо война еще предстоит нам;
\vs 1Ma 4:18 Горгий и войско его на горе близ нас; станьте теперь против врагов наших и сражайтесь с ними, а после смело возьмете добычу.
\vs 1Ma 4:19 Когда еще говорил это Иуда, показалась некоторая толпа, выступавшая с горы.
\vs 1Ma 4:20 И увидел он, что их обратили в бегство и жгут лагерь; ибо поднимающийся дым показывал, что произошло.
\vs 1Ma 4:21 Когда они увидели это, очень испугались; увидев же и войско Иуды на равнине, готовое к сражению,
\vs 1Ma 4:22 все побежали в землю иноплеменников.
\vs 1Ma 4:23 Тогда Иуда обратился на добычу стана, и захватили много золота и серебра, гиацинтовых и багряных одежд и великое богатство.
\vs 1Ma 4:24 И, возвращаясь, воспевали и благословляли Господа небесного, потому что Он благ и что вовек милость Его.
\vs 1Ma 4:25 И было в тот день великое спасение Израилю.
\vs 1Ma 4:26 Уцелевшие же из иноплеменников пришли к Лисию и возвестили о всем случившемся.
\vs 1Ma 4:27 Он, услышав, уныл и опечалился, что не то случилось с Израилем, чего он хотел, и не то вышло, что повелел ему царь.
\vs 1Ma 4:28 И на следующий год Лисий собрал шестьдесят тысяч избранных мужей и пять тысяч всадников, чтобы победить их.
\vs 1Ma 4:29 И пришли они в Идумею и расположились станом в Вефсурах; а Иуда встретил их с десятью тысячами мужей.
\vs 1Ma 4:30 Увидев сильное ополчение, он молился и говорил: благословен Ты, Спаситель Израиля, сокрушивший нападение сильного рукою раба Твоего Давида и предавший полк иноплеменников в руки Ионафана, сына Саулова, и оруженосца его.
\vs 1Ma 4:31 Предай войско сие в руки народа Твоего~--- Израиля, и да будут они постыжены в силе и коннице их;
\vs 1Ma 4:32 наведи на них страх и сокруши дерзость силы их; да будут они потрясены поражением своим;
\vs 1Ma 4:33 низложи их мечом любящих Тебя, и да прославят Тебя в песнях все знающие имя Твое.
\vs 1Ma 4:34 И сразились они, и пало из войска Лисия до пяти тысяч мужей, пали перед ними.
\vs 1Ma 4:35 Лисий, увидев бегство войска своего и храбрость воинов Иуды и что они готовы или жить, или умереть отважно, отправился в Антиохию, набрал чужеземцев и, увеличив бывшее войско, думал снова идти в Иудею.
\rsbpar\vs 1Ma 4:36 Иуда же и братья его сказали: вот, враги наши сокрушены, взойдем очистить и обновить святилище.
\vs 1Ma 4:37 И собралось все ополчение, и взошли на гору Сион.
\vs 1Ma 4:38 И увидели, что святилище опустошено, жертвенник осквернен, ворота сожжены, и в притворах, как в лесу или на какой-либо горе, поросл\acc{и} растения, и хранилища разрушены,
\vs 1Ma 4:39 и разодрали они одежды свои, плакали горьким плачем и сыпали пепел на свои головы,
\vs 1Ma 4:40 и падали лицом на землю и трубили вестовыми трубами, и вопили к небу.
\vs 1Ma 4:41 Тогда отрядил Иуда мужей воевать против находившихся в крепости, доколе он очистит святилище.
\vs 1Ma 4:42 И избрал священников беспорочных, ревнителей закона.
\vs 1Ma 4:43 Они очистили святилище и оскверненные камни вынесли в нечистое место.
\vs 1Ma 4:44 Потом они рассуждали об оскверненном жертвеннике всесожжения, как поступить с ним.
\vs 1Ma 4:45 И пришла им добрая мысль разрушить его, чтобы он когда-нибудь не послужил им в поношение, так как язычники осквернили его; и разрушили они жертвенник,
\vs 1Ma 4:46 и камни сложили на горе храма в приличном месте, пока придет пророк и даст ответ о них.
\vs 1Ma 4:47 Взяли камни целые, по закону, и построили новый жертвенник по-прежнему;
\vs 1Ma 4:48 потом устроили святыни и внутренние части храма и освятили притворы;
\vs 1Ma 4:49 устроили новую священную утварь и внесли в храм свещник и алтарь всесожжений и фимиамов и трапезу;
\vs 1Ma 4:50 и воскурили на алтаре фимиам и зажгли светильники на свещнике, и осветили храм;
\vs 1Ma 4:51 и положили на трапезу хлебы, и развесили завесы, и окончили все дела, которые предприняли.
\rsbpar\vs 1Ma 4:52 В двадцать пятый день девятого месяца~--- это месяц Хаслев~--- сто сорок восьмого года встали весьма рано
\vs 1Ma 4:53 и принесли жертву по закону на новоустроенном жертвеннике всесожжений.
\vs 1Ma 4:54 В то время, в тот самый день, в который язычники осквернили жертвенник, обновлен он с песнями, с цитрами, гуслями и кимвалами.
\vs 1Ma 4:55 И весь народ падал на лицо свое, и молились и воссылали благодарение на небо Благопоспешившему им.
\vs 1Ma 4:56 Так совершали обновление жертвенника восемь дней с весельем, принося всесожжения и вознося жертву спасения и хвалы.
\vs 1Ma 4:57 И украсили переднюю сторону храма золотыми венцами и щитами и возобновили ворота и хранилища, и сделали для них двери.
\vs 1Ma 4:58 И была весьма великая радость в народе, и отвращено было поношение язычников.
\vs 1Ma 4:59 И установил Иуда и братья его и все собрание Израиля, чтобы дни обновления жертвенника празднуемы были с веселием и радостью в свое время, каждый год восемь дней, от двадцатого дня месяца Хаслева.
\vs 1Ma 4:60 В то же время обстроили гору Сион вокруг высокими стенами и крепкими башнями, чтобы язычники, придя когда-нибудь, не попрали их, как сделали это прежде.
\vs 1Ma 4:61 И расположил там Иуда войско стеречь гору, и укрепили для охранения ее Вефсуру, чтобы народ имел крепость против Идумеи.
\vs 1Ma 5:1 Когда окрестные народы услышали, что построен жертвенник и возобновлено святилище, как прежде, сильно вознегодовали;
\vs 1Ma 5:2 и решились истребить род Иакова, живший среди них, и начали убивать и истреблять людей в этом народе.
\vs 1Ma 5:3 Тогда Иуда ополчился против сынов Исава в Идумее, в Акравиме, так как они держали в осаде Израиля, и поразил их великим поражением, и смирил их, и взял добычи их.
\vs 1Ma 5:4 Вспомнил он и о злобе сынов Веана, которые были для народа сетью и претыканием, строя ему засады на дорогах.
\vs 1Ma 5:5 Хотя они заперлись от него в башнях, но он ополчился против них, предал их заклятию и сожег огнем башни их со всеми, бывшими в них.
\vs 1Ma 5:6 Потом он перешел к сынам Аммона и встретил сильное войско и многочисленный народ и Тимофея, предводителя их.
\vs 1Ma 5:7 Он имел с ними много сражений, и они были разбиты пред лицем его; он поразил их;
\vs 1Ma 5:8 взял Иазер и селения его и возвратился в Иудею.
\vs 1Ma 5:9 Тогда собрались язычники, жившие в Галааде, против Израильтян, находившихся в пределах их, чтобы истребить их; но они бежали в крепость Дафему.
\vs 1Ma 5:10 И послали письма к Иуде и братьям его и сказали: собрались против нас окружающие нас язычники, чтобы истребить нас,
\vs 1Ma 5:11 и готовятся идти и сделать нападение на крепость, в которую мы убежали, и Тимофей предводительствует войском их.
\vs 1Ma 5:12 Итак, приди и избавь нас от руки их, ибо множество из нас погибло;
\vs 1Ma 5:13 и все братья наши, бывшие в пределах Това, преданы смерти, а жен их и детей их и имущество взяли в плен, и погубили там около тысячи мужей.
\vs 1Ma 5:14 Еще читались эти письма, как вот, пришли другие вестники из Галилеи в разодранных одеждах с таким извещением:
\vs 1Ma 5:15 собрались против нас из Птолемаиды и из Тира и Сидона, и из всей Галилеи языческой, чтобы погубить нас.
\vs 1Ma 5:16 Когда услышал эти слова Иуда и народ, то собралось великое собрание для совещания, что сделать для сих братьев, находящихся в бедствии и угрожаемых войною от тех язычников?
\vs 1Ma 5:17 Тогда Иуда сказал Симону, брату своему: выбери себе мужей и иди и защити братьев твоих, находящихся в Галилее; а я и Ионафан, брат мой, пойдем в Галаад.
\vs 1Ma 5:18 И оставил он Иосифа, сына Захарии, и Азарию начальниками над народом с остатком войска в Иудее на охранение.
\vs 1Ma 5:19 И дал им повеление, сказав: управляйте народом сим, но не начинайте войны против язычников до нашего возвращения.
\vs 1Ma 5:20 Симону отделены для похода в Галилею три тысячи мужей, Иуде же~--- в Галаад восемь тысяч мужей.
\vs 1Ma 5:21 И отправился Симон в Галилею и произвел много сражений с язычниками, и разбиты им язычники.
\vs 1Ma 5:22 Он преследовал их до ворот Птолемаиды, и пало из язычников до трех тысяч мужей, и он взял добычи их.
\vs 1Ma 5:23 Также взял он с собою находившихся в Галилее и Арваттах \bibemph{Иудеев} с женами и детьми и со всем имением их и привел в Иудею с великою радостью.
\vs 1Ma 5:24 А Иуда Маккавей и Ионафан, брат его, перешли Иордан и совершили трехдневный путь в пустыне.
\vs 1Ma 5:25 Их встретили Навуфеи и приняли мирно, и рассказали им все, случившееся с братьями их в Галааде,
\vs 1Ma 5:26 и что многие из них заперты в Васаре и Восоре, в Алемах, Хасфоре, Македе и Карнаине~--- все сии города укреплены и велики~---
\vs 1Ma 5:27 и в прочих городах Галаада находятся в осаде, и что завтра назначено напасть на эти укрепления и взять их и погубить всех их в один день.
\vs 1Ma 5:28 Посему Иуда со своим войском вдруг направил путь свой в пустыню к Восору и взял этот город, и избил весь мужеский пол острием меча, и взял все добычи их, и сожег его огнем;
\vs 1Ma 5:29 а оттуда отправился ночью и шел до укрепления.
\vs 1Ma 5:30 Когда наступало утро, и подняли глаза, и вот, народ многочисленный, которому числа не было, поднимают лестницы и машины, чтобы взять укрепление, и осаждают бывших в нем.
\vs 1Ma 5:31 Увидел Иуда, что началась битва и вопль города восходил на небо трубами и громким криком,
\vs 1Ma 5:32 и сказал воинам: сражайтесь теперь за братьев ваших.
\vs 1Ma 5:33 Он обошел врагов с тыла с тремя отрядами, и затрубили трубами и воскликнули с молитвою;
\vs 1Ma 5:34 и узнало войско Тимофея, что это~--- Маккавей, и побежали от лица его, и он поразил их великим поражением, и пало из них в этот день до восьми тысяч мужей.
\vs 1Ma 5:35 Тогда поворотил он в Масфу и осадил и взял ее, избил весь мужеский пол в ней, взял добычи ее и сожег ее огнем;
\vs 1Ma 5:36 отправившись оттуда, он взял Хасфон, Макед, Восор и прочие города Галаадские.
\rsbpar\vs 1Ma 5:37 После этих событий Тимофей собрал другое войско и расположился станом перед Рафоном по ту сторону потока.
\vs 1Ma 5:38 И послал Иуда осмотреть войско, и объявили ему и сказали: собрались к ним все окружающие нас язычники~--- сила весьма многочисленная,
\vs 1Ma 5:39 и они наняли в помощь себе Аравитян и расположились станом за потоком, будучи готовы идти против тебя войною. И пошел Иуда навстречу им.
\vs 1Ma 5:40 Тогда Тимофей сказал своим военачальникам, когда Иуда и войско его приближались к потоку воды: если он перейдет к нам прежде, то мы не в силах будем устоять против него, ибо он превозможет нас.
\vs 1Ma 5:41 Если же он убоится и расположится станом по ту сторону потока, то мы перейдем к нему и превозможем его.
\vs 1Ma 5:42 Как только подошел Иуда к потоку воды, то поставил при потоке народных писцов и приказал им, сказав: не оставляйте ни одного человека в стане, но пусть все идут на сражение.
\vs 1Ma 5:43 И переправился к ним первый и весь народ за ним. И сокрушены были пред лицем его все язычники, и бросили оружие свое, и убежали в капище, которое было в Карнаине.
\vs 1Ma 5:44 Тогда взяли они этот город и сожгли огнем капище со всеми находившимися в нем; и побежден был Карнаин и не мог более противостоять Иуде.
\vs 1Ma 5:45 И собрал Иуда всех Израильтян, находившихся в Галааде, от малого до большого, и жен их, и детей их, и имение, очень большое ополчение, чтобы идти в землю Иудейскую.
\vs 1Ma 5:46 И дошли они до Ефрона. Это был большой город, весьма укрепленный, на пути; невозможно было уклониться от него ни вправо, ни влево; надобно было пройти посреди него,
\vs 1Ma 5:47 а жители заперлись в нем и ворота завалили камнями.
\vs 1Ma 5:48 Иуда послал к ним с мирным предложением: мы пройдем по земле вашей, чтобы идти нам в землю нашу, и никто не обидит вас, только ногами нашими пройдем. Но они не захотели отворить ему.
\vs 1Ma 5:49 Тогда Иуда приказал объявить в ополчении, чтобы каждый ополчился на своем месте;
\vs 1Ma 5:50 и ополчились воины и осаждали город весь тот день и всю ночь, и сдался город в руки его.
\vs 1Ma 5:51 И побил он весь мужеский пол острием меча и до основания разрушил город, и взял добычи его, и прошел через город по убитым.
\vs 1Ma 5:52 И переправились через Иордан на великую равнину против Вефсана.
\vs 1Ma 5:53 И собирал Иуда отставших и ободрял народ в продолжение всего пути, доколе не пришли в землю Иудейскую.
\vs 1Ma 5:54 И взошли на гору Сион с весельем и радостью и принесли всесожжения, потому что никто не пал из них до самого возвращения в мире.
\rsbpar\vs 1Ma 5:55 В те дни, когда Иуда и Ионафан находились в Галааде, а Симон, брат его,~--- в Галилее перед Птолемаидою,
\vs 1Ma 5:56 услышали Иосиф, сын Захарии, и Азарий, военачальники, о славных воинских подвигах, совершенных ими,
\vs 1Ma 5:57 и сказали: сделаем и мы себе имя; пойдем воевать с язычниками, окружающими нас.
\vs 1Ma 5:58 Так объявили они бывшему при них войску и пошли на Иамнию.
\vs 1Ma 5:59 И вышел Горгий из города и воины его навстречу им на сражение.
\vs 1Ma 5:60 И, обратившись в бегство, Иосиф и Азария были преследуемы до пределов Иудеи; и пали в этот день из народа Израильского до двух тысяч мужей.
\vs 1Ma 5:61 И было великое замешательство в народе Израильском, потому что не послушались Иуды и братьев его, мечтая показать храбрость,
\vs 1Ma 5:62 тогда как они не были от семени тех мужей, руке которых предоставлено спасение Израиля.
\vs 1Ma 5:63 Но муж Иуда и братья его весьма прославились перед всем Израилем и перед всеми народами, где только слышно было имя их,~---
\vs 1Ma 5:64 и собирались к ним приветствующие.
\rsbpar\vs 1Ma 5:65 После того вышел Иуда и братья его и воевали против сынов Исава в земле, лежащей к югу, и поразил Хеврон и селения его, и разрушил укрепление его, и сожег башни его вокруг него,
\vs 1Ma 5:66 и поднялся, чтобы идти в землю иноплеменников, и прошел Самарию.
\vs 1Ma 5:67 В то время пали в сражении священники, желавшие прославиться храбростью и безрассудно вышедшие на войну.
\vs 1Ma 5:68 И обратился Иуда в Азот, землю иноплеменников, разрушил жертвенники их, сожег огнем резные изображения богов их, взял добычи городов и возвратился в землю Иудейскую.
\vs 1Ma 6:1 Между тем царь Антиох, проходя верхние области, услышал, что есть в Персии город Елимаис, славящийся богатством, серебром и золотом,
\vs 1Ma 6:2 и в нем~--- храм, весьма богатый, и есть там золотые покровы, брони и оружия, которые оставил там Александр, сын Филиппа, царь Македонский,~--- первый, воцарившийся над Еллинами.
\vs 1Ma 6:3 И он пришел и старался взять этот город и ограбить его, но не мог, потому что намерение его стало известно жителям города.
\vs 1Ma 6:4 Они поднялись против него войною, и он обратился в бегство и ушел оттуда с великою скорбью, чтобы отправиться в Вавилон.
\vs 1Ma 6:5 Тогда пришел некто к нему в Персию с известием, что ополчения, ходившие в землю Иуды, обращены в бегство,
\vs 1Ma 6:6 что Лисий ходил с сильным войском впереди всех, но был поражен \bibemph{Иудеями}, и они усилились и оружием, и войском, и многими добычами, которые взяли от пораженных ими войск,
\vs 1Ma 6:7 и что они разрушили мерзость, которую он воздвиг над жертвенником в Иерусалиме, а святилище по-прежнему обнесли высокими стенами, также и Вефсуру, город его.
\vs 1Ma 6:8 Когда царь услышал слова сии, сильно испугался и встревожился, упал на постель и впал в изнеможение от печали, что не сбылось так, как он желал.
\vs 1Ma 6:9 И много дней пробыл он там, ибо возобновлялась в нем сильная печаль; он думал, что умирает.
\vs 1Ma 6:10 И созвал он всех друзей своих и сказал им: удалился сон от глаз моих, и я изнемог сердцем от печали.
\vs 1Ma 6:11 И сказал я в сердце моем: до какой скорби дошел я и до какого великого смущения, в котором нахожусь теперь! А был я полезен и любим во владычестве моем.
\vs 1Ma 6:12 Теперь же я воспоминаю о тех злодеяниях, которые я совершил в Иерусалиме, и как взял все находившиеся в нем золотые и серебряные сосуды и посылал истреблять обитающих в Иудее напрасно.
\vs 1Ma 6:13 Теперь я позна\acc{ю}, что за это постигли меня эти беды,~--- и вот, я погибаю от великой печали в чужой земле.
\vs 1Ma 6:14 И призвал он Филиппа, одного из друзей своих, и поставил его правителем над всем царством своим;
\vs 1Ma 6:15 и дал ему венец и царскую одежду свою и перстень, чтобы он руководил Антиоха, сына его, и воспитывал его для царствования.
\vs 1Ma 6:16 И умер царь Антиох в сто сорок девятом году.
\rsbpar\vs 1Ma 6:17 Когда Лисий узнал, что царь умер, то поставил вместо него на царство сына его, Антиоха, которого воспитывал в юности его, и назвал его именем Евпатора.
\vs 1Ma 6:18 Между тем находившиеся в крепости теснили Израиля вокруг святилища и всегда старались делать ему зло, а язычникам служить опорою;
\vs 1Ma 6:19 тогда Иуда решил выгнать их и созвал весь народ, чтобы осадить их.
\vs 1Ma 6:20 Все собрались и осадили их в сто пятидесятом году, и устроил он против них стрелометательные орудия и машины.
\vs 1Ma 6:21 Но некоторые из осажденных вышли, и к ним пристали некоторые из нечестивых Израильтян;
\vs 1Ma 6:22 и пошли они к царю и сказали: доколе ты не сделаешь суда и не отмстишь за братьев наших?
\vs 1Ma 6:23 Мы согласились служить отцу твоему и ходить по заповедям его и следовать повелениям его;
\vs 1Ma 6:24 а сыны народа нашего осадили крепость и за то чуждаются нас, и кого из нас находят, умерщвляют, и имущества наши расхищают,
\vs 1Ma 6:25 и не на нас только простерли они руку, но и на все пределы наши.
\vs 1Ma 6:26 И вот, теперь осадили они крепость в Иерусалиме, чтобы овладеть ею, а святилище и Вефсуру укрепили.
\vs 1Ma 6:27 Если ты не поспешишь предупредить их, то они сделают больше этого, и тогда ты не в силах будешь удержать их.
\vs 1Ma 6:28 Услышав это, царь разгневался и собрал всех друзей своих и начальников войска своего и начальников конницы;
\vs 1Ma 6:29 пришли к нему и из других царств и с морских островов войска наемные,
\vs 1Ma 6:30 так что число войск его было: сто тысяч пеших, двадцать тысяч всадников и тридцать два слона, приученных к войне.
\vs 1Ma 6:31 И прошли они через Идумею и расположились станом против Вефсуры, и сражались много дней и устроили машины; но те сделали вылазку и сожгли их огнем и сразились мужественно.
\vs 1Ma 6:32 После сего Иуда отступил от крепости и расположился станом в Вефсахаре против стана царского.
\vs 1Ma 6:33 Царь же, встав рано утром, поспешно отправился с войском своим по дороге к Вефсахаре, и приготовились войска к сражению и затрубили трубами.
\vs 1Ma 6:34 Слонам показывали кровь винограда и тутовых ягод, чтобы возбудить их к битве,
\vs 1Ma 6:35 и разделили этих животных на отряды и приставили к каждому слону по тысяче мужей в железных кольчугах и с медными шлемами на головах, сверх того по пятисот отборных всадников назначено было к каждому слону.
\vs 1Ma 6:36 Они становились заблаговременно там, где был слон, и куда он шел, шли и они вместе, не отставая от него.
\vs 1Ma 6:37 Притом на них были крепкие деревянные башни, покрывавшие каждого слона, укрепленные на них помочами, и в каждой из них по тридцати по два сильных мужей, которые сражались на них, и при слоне Индиец его.
\vs 1Ma 6:38 Остальных же всадников расставили здесь и там~--- на двух сторонах ополчения, чтобы подавать знаки и подкреплять в тесных местах.
\vs 1Ma 6:39 Когда солнце блеснуло на золотых и медных щитах, то заблистали от них горы и светились, как огненные светильники.
\vs 1Ma 6:40 Одна часть царского войска протянута была по высоким горам, а другие~--- по низменным местам; и шли они твердо и стройно.
\vs 1Ma 6:41 И смутились все, слышавшие шум множества их и шествие такого полчища и стук оружий, ибо войско было весьма великое и сильное.
\vs 1Ma 6:42 И вступил Иуда и войско его в сражение~--- и пали из ополчения царского шестьсот мужей.
\vs 1Ma 6:43 Тогда Елеазар, сын Саварана, увидел, что один из слонов покрыт бронею царскою и превосходил всех, и казалось, что на нем был царь,~---
\vs 1Ma 6:44 и он предал себя, чтобы спасти народ свой и приобрести себе вечное имя;
\vs 1Ma 6:45 и смело побежал к нему в средину отряда, поражая направо и налево, и расступались от него и в ту, и в другую сторону;
\vs 1Ma 6:46 и подбежал он под того слона, лег под него и убил его, и пал на него слон на землю, и он умер там.
\vs 1Ma 6:47 Но, увидев силу царского ополчения и стремительность войск, Иудеи уклонились от них.
\vs 1Ma 6:48 Царские же войска пошли против них на Иерусалим: царь направил войска на Иудею и на гору Сион.
\vs 1Ma 6:49 И заключил он мир с бывшими в Вефсуре, которые вышли из города, ибо не было у них продовольствия, чтобы держаться в нем в осаде, потому что был субботний год на земле.
\vs 1Ma 6:50 И овладел царь Вефсурою и оставил в ней стражу, чтобы стеречь ее.
\vs 1Ma 6:51 Потом много дней осаждал святилище и поставил там стрелометательные орудия и машины, и огнеметательные, и камнеметательные, и копьеметательные, чтобы бросать стрелы и камни.
\vs 1Ma 6:52 Но и Иудеи устроили машины против их машин и сражались много дней;
\vs 1Ma 6:53 съестных же припасов недостало в хранилищах, потому что был седьмой год, и искавшие в Иудее безопасности от язычников издержали остатки запасов;
\vs 1Ma 6:54 и осталось при святилище немного мужей, ибо одолел их голод, и разошлись каждый в свое место.
\rsbpar\vs 1Ma 6:55 Услышал Лисий, что Филипп, которому царь Антиох еще при жизни поручил воспитывать сына своего, Антиоха, для царствования,
\vs 1Ma 6:56 возвратился из Персии и Мидии и с ним ходившие с царем войска, и что он домогается принять на себя дела царства.
\vs 1Ma 6:57 Почему поспешно пошел и сказал царю, начальникам войска и вельможам: мы каждый день терпим недостаток и продовольствия у нас мало, а место, осаждаемое нами, крепко, между тем лежит на нас попечение о царстве.
\vs 1Ma 6:58 Итак, подадим правую руку этим людям и заключим с ними мир и со всем народом их,
\vs 1Ma 6:59 и предоставим им поступать по законам их, как прежде; ибо за свои законы, которые мы отменили, они раздражились и сделали всё это.
\vs 1Ma 6:60 И угодно было это слово царю и начальникам,~--- и послал он к ним, чтобы заключить мир, что они и приняли;
\vs 1Ma 6:61 и клялся им царь и начальники. После сего они вышли из крепости.
\vs 1Ma 6:62 И взошел царь на гору Сион и, осмотрев укрепленные места, пренебрег клятвою, которою клялся, и велел разорить стены кругом.
\vs 1Ma 6:63 Потом поспешно отправился, и, возвратившись в Антиохию, он нашел, что Филипп владеет городом, вступил с ним в сражение и силою взял город.
\vs 1Ma 7:1 В сто пятьдесят первом году вышел из Рима Димитрий, сын Селевка, и с немногими людьми вошел в один приморский город и там воцарился.
\vs 1Ma 7:2 Когда же он входил в царственный дом отцов своих, войско схватило Антиоха и Лисия, чтобы привести их к нему.
\vs 1Ma 7:3 Это стало известно ему, и он сказал: не показывайте мне лиц их.
\vs 1Ma 7:4 Тогда воины убили их, и воссел Димитрий на престоле царства своего.
\vs 1Ma 7:5 И пришли к нему все мужи беззаконные и нечестивые из Израильтян, и Алким предводительствовал ими, домогаясь священства;
\vs 1Ma 7:6 и обвиняли они перед царем народ, говоря: погубил Иуда и братья его друзей твоих, и нас выгнали из земли нашей.
\vs 1Ma 7:7 Итак, пошли теперь мужа, кому ты доверяешь; пусть он пойдет и увидит все разорение, которое они причинили нам и стране царя, и пусть накажет их и всех, помогающих им.
\vs 1Ma 7:8 Царь избрал Вакхида из друзей царских, который управлял по ту сторону реки, был велик в царстве и верен царю,
\vs 1Ma 7:9 и послал его и нечестивого Алкима, предоставив ему священство, и повелел ему сделать отмщение сынам Израиля.
\vs 1Ma 7:10 Они отправились и пришли в землю Иудейскую с большим войском; и он послал к Иуде и братьям его послов с мирным, но коварным предложением.
\vs 1Ma 7:11 Но они не вняли словам их, ибо видели, что они пришли с большим войском.
\vs 1Ma 7:12 К Алкиму же и Вакхиду сошлось собрание книжников искать справедливости.
\vs 1Ma 7:13 Первые из сынов Израилевых были Асидеи; они искали у них мира,
\vs 1Ma 7:14 ибо говорили: священник от племени Аарона пришел вместе с войском и не обидит нас.
\vs 1Ma 7:15 И он говорил с ними мирно и клялся им, и сказал: мы не сделаем зла вам и друзьям вашим.
\vs 1Ma 7:16 И они поверили ему, а он, захватив из них шестьдесят мужей, умертвил их в один день, как сказано в Писании:
\vs 1Ma 7:17 <<тела святых Твоих и кровь их пролили вокруг Иерусалима, и некому было похоронить их>>.
\vs 1Ma 7:18 И напал от них страх и ужас на весь народ, и говорили: нет в них истины и правды, ибо они нарушили постановление и клятву, которою клялись.
\vs 1Ma 7:19 Тогда Вакхид отступил от Иерусалима и расположился станом при Визефе, и, послав, поймал многих из бежавших от него мужей и некоторых из народа, заколол и бросил их в глубокий колодезь.
\rsbpar\vs 1Ma 7:20 Потом, поручив страну Алкиму и оставив с ним войско на помощь ему, Вакхид отправился к царю.
\vs 1Ma 7:21 Алким же домогался первосвященства.
\vs 1Ma 7:22 И собрались к нему все возмущавшие народ свой, и овладели землею Иудейскою, и произвели великое поражение в Израиле.
\vs 1Ma 7:23 И увидел Иуда все зло, какое причинил Алким со своими сообщниками сынам Израилевым,~--- больше, нежели язычники;
\vs 1Ma 7:24 и, обойдя все пределы Иудеи, сделал отмщение отступникам,~--- и они перестали входить в эту страну.
\vs 1Ma 7:25 Когда же Алким увидел, что Иуда и находящиеся с ним усилились, и понял, что не может противостоять им, возвратился к царю и жестоко обвинял их.
\vs 1Ma 7:26 Тогда царь послал Никанора, одного из славных вождей своих, ненавистника и враждебного Израилю, и приказал ему истребить этот народ.
\vs 1Ma 7:27 Никанор, придя в Иерусалим с большим войском, послал к Иуде и братьям его коварно со словами мирными:
\vs 1Ma 7:28 да не будет войны между мною и вами; я войду с немногими людьми, чтобы видеть лица ваши в мире.
\vs 1Ma 7:29 И пришел он к Иуде, и приветствовали они друг друга мирно; а между тем воины были приготовлены схватить Иуду.
\vs 1Ma 7:30 Иуде сделалось известным, что он пришел к нему с коварством, поэтому он убоялся его и не хотел более видеть лица его.
\vs 1Ma 7:31 Когда Никанор узнал, что умысел его открылся, вышел против Иуды на сражение близ Хафарсаламы.
\vs 1Ma 7:32 И пало из бывших при Никаноре около пяти тысяч мужей, а прочие убежали в город Давидов.
\vs 1Ma 7:33 После того Никанор взошел на гору Сион; и вышли из святилища некоторые из священников и старейшин народа, чтобы мирно приветствовать его и показать ему всесожжение, приносимое за царя.
\vs 1Ma 7:34 Но он осмеял их, надругался над ними и осквернил их, и говорил высокомерно,
\vs 1Ma 7:35 и, поклявшись, с гневом сказал: если не предан будет ныне Иуда и войско его в мои руки, то, когда возвращусь благополучно, сожгу дом сей. И ушел с великим гневом.
\vs 1Ma 7:36 А священники вошли и стали пред лицем жертвенника и храма, заплакали и сказали:
\vs 1Ma 7:37 Ты, Господи, избрал дом сей, чтобы на нем нарицалось имя Твое и чтобы он был домом молитвы и моления для народа Твоего.
\vs 1Ma 7:38 Сделай отмщение человеку сему и войску его, и пусть падут они от меча; вспомни злохуления их и не дай им оставаться долее.
\vs 1Ma 7:39 И вышел Никанор из Иерусалима и расположился станом при Вефороне, и пристало к нему здесь войско Сирийское.
\vs 1Ma 7:40 А Иуда с тремя тысячами мужей расположился станом при Адасе; и помолился Иуда, и сказал:
\vs 1Ma 7:41 Господи! когда посланные царя Ассирийского произносили злохуления, то пришел Ангел Твой и поразил из них сто восемьдесят пять тысяч.
\vs 1Ma 7:42 Так сокруши ныне пред нами сие полчище, да познают прочие, что они произносили хулу на святыни Твои, и суди их по злобе их.
\vs 1Ma 7:43 И вступили войска в сражение в тринадцатый день месяца Адара, и разбито было войско Никанора, и он первый пал в сражении.
\vs 1Ma 7:44 Когда же воины его увидели, что Никанор пал, то, побросав оружие свое, обратились в бегство.
\vs 1Ma 7:45 И преследовали их Израильтяне целый день, от Адаса до самой Газиры, и трубили вслед их вестовыми трубами.
\vs 1Ma 7:46 И выходили из всех окрестных селений Иудейских и окружали их,~--- и они, оборачиваясь к преследовавшим их, все пали от меча, и ни одного не осталось из них.
\vs 1Ma 7:47 И взяли \bibemph{Иудеи} добычи их и награбленное ими, и отрубили голову Никанора и правую руку его, которую он простирал надменно, и принесли и повесили перед Иерусалимом.
\vs 1Ma 7:48 Народ весьма радовался и провел тот день, как день великого веселья;
\vs 1Ma 7:49 и установили ежегодно праздновать этот день тринадцатого числа Адара.
\vs 1Ma 7:50 И успокоилась земля Иудейская на некоторое время.
\vs 1Ma 8:1 Иуда услышал о славе Римлян, что они могущественны и сильны и благосклонно принимают всех, обращающихся к ним, и кто ни приходил к ним, со всеми заключали они дружбу.
\vs 1Ma 8:2 А что они могущественны и сильны,~--- рассказывали ему о войнах их, о мужественных подвигах, которые они показали над Галатами, как они покорили их и сделали данниками;
\vs 1Ma 8:3 также о том, что сделали они в стране Испанской, чтобы овладеть находящимися там серебряными и золотыми рудниками,
\vs 1Ma 8:4 и своим благоразумием и твердостью овладели всем краем, хотя тот край весьма далеко отстоял от них, равно о царях, которые приходили против них от конца земли, и они сокрушили их и поразили великим поражением, а прочие платят им ежегодно дань;
\vs 1Ma 8:5 они также сокрушили на войне и покорили себе Филиппа и Персея, царя Китийского, и других, восставших против них,
\vs 1Ma 8:6 и Антиоха, великого царя Азии, который вышел против них на войну со ста двадцатью слонами, и с конницею, и колесницами, и весьма многочисленным войском и был разбит ими;
\vs 1Ma 8:7 они взяли его живого и заставили платить им великую дань,~--- как его, так и следующих после него царей,~--- дать заложников и допустить раздел,
\vs 1Ma 8:8 а страну Индийскую и Мидию, и Лидию, и другие из лучших областей его, взяв от него, отдали царю Евмению;
\vs 1Ma 8:9 и о том, как Еллины вознамерились прийти и истребить их,
\vs 1Ma 8:10 но это намерение сделалось им известным, и они послали против них одного военачальника и воевали против них,~--- и много из них пало пораженных, и взяли в плен жен их и детей их и разграбили их, и овладели их землею, и разорили крепости их, и поработили их до сего дня;
\vs 1Ma 8:11 и другие царства и острова, которые когда-либо восставали против них, они разорили и поработили.
\vs 1Ma 8:12 А с друзьями своими и с доверявшимися им они сохраняли дружбу; и овладели царствами ближними и дальними, и все, слышавшие имя их, боялись их.
\vs 1Ma 8:13 Если захотят кому помочь и кого воцарить, те царствуют, и кого хотят, сменяют, и они весьма возвысились;
\vs 1Ma 8:14 но при всем том никто из них не возлагал на себя венца и не облекался в порфиру, чтобы величаться ею.
\vs 1Ma 8:15 Они составили у себя совет, и постоянно каждый день триста двадцать человек совещаются обо всем, что относится до народа и благоустроения его;
\vs 1Ma 8:16 и каждый год одному человеку вверяют они начальство над собою и господство над всею землею их, и все слушают одного, и не бывает ни зависти, ни ревности между ними.
\rsbpar\vs 1Ma 8:17 Тогда избрал Иуда Евполема, сына Иоаннова, сына Аккосова, и Иасона, сына Елеазарова, и послал их в Рим, чтобы заключить с ними дружбу и союз
\vs 1Ma 8:18 и чтобы они сняли с них иго, ибо они видят, что Еллинское царство хочет поработить Израиля.
\vs 1Ma 8:19 Итак, они отправились в Рим, хотя путь был очень долгий, и вошли в собрание совета и, приступив, сказали:
\vs 1Ma 8:20 Иуда Маккавей и братья его и весь народ Иудейский послали нас к вам, чтобы заключить с вами союз и мир и чтобы вы вписали нас в число соратников и друзей ваших.
\vs 1Ma 8:21 И угодно было это слово перед ними.
\vs 1Ma 8:22 И вот список того послания, которое написали они в ответ на медных досках и послали в Иерусалим, чтобы оно служило для них там памятником мира и союза:
\vs 1Ma 8:23 <<благо да будет Римлянам и народу Иудейскому на море и на суше навеки, и меч и враг да будут далеко от них!
\vs 1Ma 8:24 Если же настанет война прежде у Римлян или у всех союзников их во всем владении их,
\vs 1Ma 8:25 то народ Иудейский должен оказать им всем сердцем помощь в войне, как потребует того время;
\vs 1Ma 8:26 и воюющим они не будут ни давать, ни доставлять ни хлеба, ни оружия, ни денег, ни кораблей, ибо так угодно Римлянам; они должны исполнять обязанность свою, ничего не получая.
\vs 1Ma 8:27 Точно так же, если прежде случится война у народа Иудейского, Римляне от души будут помогать им в войне, как потребует того время,
\vs 1Ma 8:28 и помогающим в войне не будут давать ни хлеба, ни оружия, ни денег, ни кораблей: так угодно Риму; они должны исполнять свои обязанности~--- и без обмана>>.
\vs 1Ma 8:29 На таких условиях заключили Римляне союз с народом Иудейским.
\vs 1Ma 8:30 Если же после сих условий те и другие вздумают что-нибудь прибавить или убавить, пусть сделают это по их общему произволению, и то, что они прибавят или убавят, будет иметь силу.
\vs 1Ma 8:31 А о том зле, какое делает \bibemph{Иудеям} царь Димитрий, мы написали ему так: <<для чего ты наложил тяжкое твое иго на друзей наших и союзников~--- Иудеев?
\vs 1Ma 8:32 Если они еще обратятся к нам с жалобою на тебя, то мы окажем им справедливость и будем воевать против тебя на море и на суше>>.
\vs 1Ma 9:1 Когда Димитрий услышал, что Никанор и воины его пали в сражении, послал Вакхида и Алкима во второй раз в землю Иудейскую и правое крыло с ними.
\vs 1Ma 9:2 И отправились они по дороге в Галгалы и расположились станом при Месалофе, что в Арвилах, и, овладев им, погубили множество людей.
\vs 1Ma 9:3 В первом месяце сто пятьдесят второго года расположились они станом у Иерусалима,
\vs 1Ma 9:4 но снялись и пошли к Верее с двадцатью тысячами мужей и двумя тысячами конницы.
\vs 1Ma 9:5 А Иуда расположился станом при Елеасе, и три тысячи избранных мужей с ним.
\vs 1Ma 9:6 Но, увидев множество войска, как оно многочисленно, они весьма устрашились, и многие из стана его разбежались, и осталось из них не более восьмисот мужей.
\vs 1Ma 9:7 Когда увидел Иуда, что разбежалось ополчение его, а война тревожила его, он смутился сердцем, потому что не имел времени собрать их.
\vs 1Ma 9:8 Он опечалился и сказал оставшимся: встанем и пойдем на противников наших; может быть, мы в силах будем сражаться с ними.
\vs 1Ma 9:9 Но они отклоняли его и говорили: мы не в силах, но будем теперь спасать жизнь нашу, и потом возвратимся с братьями нашими, и тогда будем сражаться против них, а теперь нас мало.
\vs 1Ma 9:10 Но Иуда сказал: нет, да не будет этого со мною, чтобы бежать от них; а если пришел час наш, то умрем мужественно за братьев наших и не оставим нарекания на славу нашу.
\vs 1Ma 9:11 И двинулось войско из стана и стало против них; и разделилась конница на две части, а впереди войска шли пращники и стрельцы и все сильные передовые воины.
\vs 1Ma 9:12 Вакхид же находился на правом крыле, и приближались отряды с обеих сторон и трубили трубами.
\vs 1Ma 9:13 Затрубили трубами и бывшие с Иудою, и поколебалась земля от шума войск, и было упорное сражение от утра до вечера.
\vs 1Ma 9:14 Когда увидел Иуда, что Вакхид и крепчайшая часть его войска находится на правой стороне, то собрались к нему все храбрые сердцем,~---
\vs 1Ma 9:15 и разбито ими правое крыло, и они преследовали их до горы Азота.
\vs 1Ma 9:16 Когда находившиеся на левом крыле увидели, что правое крыло разбито, то обратились вслед за Иудою и бывшими с ним, с тыла.
\vs 1Ma 9:17 И сражение было жестокое, и много пало пораженных с той и другой стороны,
\vs 1Ma 9:18 пал и Иуда, а прочие обратились в бегство.
\vs 1Ma 9:19 И взяли Ионафан и Симон Иуду, брата своего, и похоронили его во гробе отцов его в Модине.
\vs 1Ma 9:20 И оплакивали его и рыдали о нем сильно все Израильтяне, и печалились много дней и говорили:
\vs 1Ma 9:21 как пал сильный, спасавший Израиля?
\vs 1Ma 9:22 Прочие же дела Иуды, и сражения, и мужественные подвиги, которые совершил он, и величие его не описаны, ибо их было весьма много.
\rsbpar\vs 1Ma 9:23 По смерти же Иуды во всех пределах Израильских явились люди беззаконные, и поднялись все делатели неправды.
\vs 1Ma 9:24 В те самые дни был очень сильный голод, и страна пристала к ним.
\vs 1Ma 9:25 И выбрал Вакхид нечестивых мужей и поставил их начальниками страны.
\vs 1Ma 9:26 Они разведывали и разыскивали друзей Иуды и приводили их к Вакхиду, а он мстил им и издевался над ними.
\vs 1Ma 9:27 И была великая скорбь в Израиле, какой не бывало с того дня, как не видно стало у них пророка.
\vs 1Ma 9:28 Тогда собрались все друзья Иуды и сказали Ионафану:
\vs 1Ma 9:29 с того времени, как скончался брат твой Иуда, нет подобного ему мужа, чтобы выйти против врагов и Вакхида и против ненавистников нашего народа.
\vs 1Ma 9:30 Итак, теперь мы тебя избрали~--- быть нам вместо него начальником и вождем, чтобы вести войну нашу.
\vs 1Ma 9:31 И принял Ионафан в то время предводительство и стал на место Иуды, брата своего.
\vs 1Ma 9:32 И узнал о том Вакхид и искал убить его.
\vs 1Ma 9:33 Об этом узнали Ионафан и Симон, брат его, и все бывшие с ним и убежали в пустыню Фекое и расположились станом при водах озера Асфар.
\vs 1Ma 9:34 Вакхид, узнав о том в день субботний, переправился сам и все войско его за Иордан.
\vs 1Ma 9:35 А Ионафан отправил брата своего~--- предводителя народа~--- и просил друзей своих, Наватеев, чтобы сложить у них большой запас свой.
\vs 1Ma 9:36 Но вышли из Мидавы сыны Иамври и схватили Иоанна и все, что он имел, и ушли.
\vs 1Ma 9:37 После сих происшествий сказали Ионафану и Симону, брату его, что сыны Иамври торжественно совершают знатный брак и провожают из Надавафа с великою пышностью невесту, дочь одного из знатных вельмож Хананейских.
\vs 1Ma 9:38 Тогда вспомнили они об Иоанне, брате своем, и вышли, и скрылись под кровом горы.
\vs 1Ma 9:39 Подняв глаза свои, они увидели: вот восклицания и большое приданое; навстречу вышел жених и друзья его и братья его с тимпанами и музыкою и со многими оружиями.
\vs 1Ma 9:40 Тогда бывшие с Ионафаном поднялись на них из засады и побили их, и много пало пораженных, а остальные убежали на гору; и взяли они всю добычу их.
\vs 1Ma 9:41 И обратилось брачное торжество в печаль, и звук музыки их~--- в плач.
\vs 1Ma 9:42 Так отмстили они за кровь брата своего и возвратились к болотистому месту у Иордана.
\vs 1Ma 9:43 И услышал об этом Вакхид~--- и в день субботний пришел к берегам Иордана с большим войском.
\vs 1Ma 9:44 Тогда сказал Ионафан бывшим с ним: встанем теперь и сразимся за жизнь нашу, ибо ныне~--- не то, что вчера и третьего дня.
\vs 1Ma 9:45 Вот, неприятель и спереди нас и сзади нас, вода Иордана с той и с другой стороны, и болото и лес, и нет места, куда уклониться.
\vs 1Ma 9:46 Итак, теперь воззовите на небо, чтобы избавиться вам от руки врагов ваших.
\vs 1Ma 9:47 И началось сражение. И простер Ионафан руку свою, чтобы поразить Вакхида, но тот уклонился от него назад.
\vs 1Ma 9:48 И бросился Ионафан и бывшие с ним в Иордан и переплыли на другой берег, а те не перешли за ними Иордана.
\vs 1Ma 9:49 И пало у Вакхида в тот день до тысячи мужей.
\vs 1Ma 9:50 И возвратился он в Иерусалим и построил в Иудее крепкие города: крепость в Иерихоне, и Еммаум и Вефорон, и Вефиль и Фамнафу в Фарафоне, и Тефон с высокими стенами, воротами и запорами,
\vs 1Ma 9:51 и поставил в них стражу, чтобы враждебно действовать против Израиля.
\vs 1Ma 9:52 Укрепил также город в Вефсуре и Газару и крепость и оставил в них войско со съестными запасами,
\vs 1Ma 9:53 и взял в заложники сыновей вождей страны и поместил их в Иерусалимской крепости под стражею.
\rsbpar\vs 1Ma 9:54 В сто пятьдесят третьем году, во втором месяце, Алким велел разорить стену внутреннего двора храма и разрушить дело пророков, и уже начал разрушение.
\vs 1Ma 9:55 Но в то самое время Алким поражен был ударом, и остановились предприятия его; уста его сомкнулись, он онемел и не мог более вымолвить ни одного слова и завещать о доме своем.
\vs 1Ma 9:56 И умер Алким в то же время в тяжких мучениях.
\vs 1Ma 9:57 Когда Вакхид узнал, что Алким умер, возвратился к царю; и земля Иудейская два года оставалась в покое.
\vs 1Ma 9:58 Тогда все беззаконники совещались и говорили: вот, Ионафан и находящиеся с ним живут безопасно в покое; приведем теперь Вакхида, и он схватит всех их в одну ночь.
\vs 1Ma 9:59 Пошли и предложили ему такой совет.
\vs 1Ma 9:60 Он решился идти с большим войском и послал тайно письма всем союзникам своим, которые находились в Иудее, чтобы они схватили Ионафана и находящихся с ним, но они не могли, потому что замысел их сделался известен им.
\vs 1Ma 9:61 И поймали они из мужей страны виновников этого злодейства до пятидесяти человек и убили их.
\vs 1Ma 9:62 После сего удалились Ионафан и Симон и бывшие с ними в Вефваси, что в пустыне, и возобновили разрушенное там и укрепили город.
\vs 1Ma 9:63 Узнав об этом, Вакхид собрал все войско свое, известив и тех, которые находились в Иудее,
\vs 1Ma 9:64 пришел и осадил Вефваси, и сражался против него много дней и устроил машины.
\vs 1Ma 9:65 Ионафан же оставил в городе Симона, брата своего, а сам вышел в страну, и вышел с небольшим числом,
\vs 1Ma 9:66 и поразил Одоааррина и братьев его и сыновей Фасирона в шатрах их и начал поражать и наступать с силою.
\vs 1Ma 9:67 Тогда и Симон и бывшие с ним выступили из города и сожгли машины,
\vs 1Ma 9:68 и сражались против Вакхида, и он был разбит ими; этим они сильно опечалили его, потому что замысел его и поход остался тщетным.
\vs 1Ma 9:69 Сильно разгневался он на мужей беззаконных, которые присоветовали ему идти в эту страну, и многих из них умертвил, и решился возвратиться в землю свою.
\vs 1Ma 9:70 Узнав об этом, Ионафан послал к нему старейшин, чтобы заключить с ним мир и чтобы он отдал пленных.
\vs 1Ma 9:71 Он принял это и сделал по словам его, и поклялся не причинять ему никакого зла во все дни жизни своей,
\vs 1Ma 9:72 и отдал ему пленных, которых прежде взял в плен в земле Иудейской, и возвратился в землю свою и не приходил более в пределы их.
\vs 1Ma 9:73 И унялся меч в Израиле, и поселился Ионафан в Махмасе; и начал Ионафан судить народ и истребил нечестивых из среды Израиля.
\vs 1Ma 10:1 В сто шестидесятом году выступил Александр, сын Антиоха Епифана, и овладел Птолемаидою: и приняли его, и он воцарился там.
\vs 1Ma 10:2 Когда услышал о том царь Димитрий, собрал весьма многочисленное войско и вышел против него на войну.
\vs 1Ma 10:3 И послал Димитрий письма Ионафану с мирным предложением, как бы желая возвеличить его,
\vs 1Ma 10:4 ибо говорил: предупредим заключить с ним мир, прежде нежели он заключит с Александром против нас:
\vs 1Ma 10:5 тогда он припомнит все зло, которое мы сделали против него и братьев его и народа его.
\vs 1Ma 10:6 И он дал ему власть набирать войско и приготовлять оружия, чтобы быть союзником его, и велел отдать ему заложников, которые находились в крепости.
\vs 1Ma 10:7 Ионафан пришел в Иерусалим и прочитал письма вслух всего народа и бывших в крепости;
\vs 1Ma 10:8 и убоялись все великим страхом, услышав, что царь дал ему власть набирать войско;
\vs 1Ma 10:9 а бывшие в крепости выдали Ионафану заложников, и он возвратил их родителям их.
\vs 1Ma 10:10 И жил Ионафан в Иерусалиме; и начал строить и возобновлять город,
\vs 1Ma 10:11 и сказал производившим работы, чтобы они строили стены и вокруг горы Сиона для твердости из четырехугольных камней,~--- и делали так.
\vs 1Ma 10:12 Тогда иноплеменные, бывшие в крепостях, построенных Вакхидом, бежали:
\vs 1Ma 10:13 каждый оставил свое место и ушел в свою землю.
\vs 1Ma 10:14 Только в Вефсуре остались некоторые из тех, которые оставили закон и заповеди, ибо это место служило для них убежищем.
\vs 1Ma 10:15 И услышал царь Александр о тех обещаниях, какие Димитрий послал Ионафану, и рассказали ему о войнах и храбрых подвигах, которые совершил Ионафан и братья его, и о трудностях, понесенных ими.
\vs 1Ma 10:16 Тогда он сказал: найдем ли мы еще такого мужа, как этот? Сделаем же его нашим другом и союзником.
\vs 1Ma 10:17 И написал и послал ему письмо в таких словах:
\vs 1Ma 10:18 <<Царь Александр брату Ионафану~--- радоваться.
\vs 1Ma 10:19 Услышали мы о тебе, что ты~--- муж, крепкий силою и достойный быть нашим другом.
\vs 1Ma 10:20 Итак, мы поставляем тебя ныне первосвященником народа твоего; и ты будешь именоваться другом царя (он послал ему порфиру и золотой венец) и будешь держать нашу сторону и хранить дружбу с нами>>.
\vs 1Ma 10:21 И облекся Ионафан в священную одежду в седьмом месяце сто шестидесятого года, в праздник кущей, и собрал войско и заготовил множество оружий.
\vs 1Ma 10:22 И услышал об этом Димитрий и огорчился, и сказал:
\vs 1Ma 10:23 что это мы сделали, что Александр предупредил нас заключить дружбу с Иудеями в подкрепление себе?
\vs 1Ma 10:24 Напишу и я им слова приветствия, восхваления и обещаний, чтобы были они в помощь мне.
\vs 1Ma 10:25 И послал им письмо в таких словах: <<Царь Димитрий народу Иудейскому~--- радоваться.
\vs 1Ma 10:26 Слышали мы и радовались, что вы сохраняете договоры наши, пребываете в дружбе с нами и не склоняетесь к врагам нашим.
\vs 1Ma 10:27 Продолжайте и ныне сохранять верность к нам, и мы воздадим вам добром за то, что вы делаете для нас:
\vs 1Ma 10:28 сделаем вам многие уступки и дадим вам дары.
\vs 1Ma 10:29 Ныне же разрешаю вас и освобождаю всех Иудеев от податей и пошлины с соли и с венцов;
\vs 1Ma 10:30 и за третью часть семян и половинную часть древесных плодов, принадлежащую мне, отныне и впредь я отменяю брать с земли Иудейской и с трех областей, присоединенных к ней от Самарии и Галилеи, от нынешнего дня и на вечные времена.
\vs 1Ma 10:31 И Иерусалим да будет священным и свободным и пределы его, десятины и доходы его.
\vs 1Ma 10:32 Предоставляю и власть над крепостью Иерусалимскою и даю право первосвященнику поставить в ней людей, каких он сам изберет, для охранения ее;
\vs 1Ma 10:33 и всякого человека из Иудеев, взятого в плен из земли Иудейской, во всем царстве моем отпускаю на свободу даром: пусть все будут свободны от повинностей за себя и за скот свой.
\vs 1Ma 10:34 Все праздники и субботы и новомесячия, и дни установленные~--- три дня пред праздником и три дня после праздника,~--- все эти дни пусть будут днями льготы и свободы всем Иудеям, находящимся в моем царстве.
\vs 1Ma 10:35 Никто не будет иметь права притеснять и отягощать кого-нибудь из них ни по какому делу.
\vs 1Ma 10:36 И пусть из Иудеев записываются в царские войска до тридцати тысяч человек,~--- и им будет даваться жалованье наравне со всеми войсками царскими.
\vs 1Ma 10:37 И из них да будут поставляемы начальствующими над большими крепостями царскими, из них же да будут поставляемы и над делами царства, требующими верности, и их приставники и начальники да будут из них же, и пусть они живут по своим законам, как повелел царь в земле Иудейской.
\vs 1Ma 10:38 И три области, присоединенные к Иудее от страны Самарийской, пусть останутся присоединенными к Иудее, чтобы считаться и быть им за одну и не подлежать другой власти, кроме власти первосвященника.
\vs 1Ma 10:39 Птолемаиду с округом ее я отдаю в дар святилищу в Иерусалиме на издержки, потребные для святилища;
\vs 1Ma 10:40 я же даю ежегодно пятнадцать тысяч сиклей серебра из царских сборов с подлежащих мест.
\vs 1Ma 10:41 И все остальное, чего не отдали заведующие сборами, как в прежние годы, отныне будут отдавать на работы храма.
\vs 1Ma 10:42 Сверх того пять тысяч сиклей серебра, которые брали от доходов святилища из ежегодного сбора, и те уступаются, как принадлежащие служащим священникам.
\vs 1Ma 10:43 И все, которые убегут в храм Иерусалимский и во все пределы его по причине повинностей царских и всех других, пусть будут свободны со всем, что принадлежит им в царстве моем.
\vs 1Ma 10:44 И на строение и возобновление святилища издержки будут выдаваемы из сборов царских.
\vs 1Ma 10:45 И на построение стен Иерусалима и укрепление их вокруг издержки будут выдаваемы из доходов царских, а также на построение стен в Иудее>>.
\vs 1Ma 10:46 Ионафан и народ, выслушав эти слова, не поверили им и не приняли их, ибо вспомнили о тех великих бедствиях, которые нанес Димитрий Израильтянам, жестоко притеснив их,
\vs 1Ma 10:47 и предпочли союз с Александром, ибо он первый сделал им мирные предложения,~--- и помогали ему в войнах во все дни.
\rsbpar\vs 1Ma 10:48 Царь Александр собрал большое войско и ополчился против Димитрия.
\vs 1Ma 10:49 И вступили два царя в сражение, и войско Димитрия обратилось в бегство; Александр преследовал его, и превозмог,
\vs 1Ma 10:50 и весьма настойчиво продолжал сражение до самого захождения солнца,~--- и пал Димитрий в этот день.
\rsbpar\vs 1Ma 10:51 После того Александр отправил послов к Птоломею, царю Египетскому, с такими словами:
\vs 1Ma 10:52 <<Я возвратился в землю царства моего и воссел на престоле отцов моих, принял верховную власть, сокрушил Димитрия и стал обладателем страны нашей.
\vs 1Ma 10:53 Я вступил с ним в сражение, и он разбит нами и войско его, и воссели мы на престоле царства его.
\vs 1Ma 10:54 Итак, заключим теперь дружбу между нами, и ты дай мне дочь твою в жену, и буду я тебе зятем и дам тебе и ей дары, достойные тебя>>.
\vs 1Ma 10:55 И отвечал царь Птоломей так: <<Счастлив день, в который ты возвратился в землю отцов твоих и воссел на престоле царства их.
\vs 1Ma 10:56 Ныне я исполню для тебя то, о чем ты писал, только ты выйди ко мне в Птолемаиду, чтобы нам видеть друг друга, и я породнюсь с тобою, как ты сказал>>.
\vs 1Ma 10:57 И отправился Птоломей из Египта сам и Клеопатра, дочь его, и прибыли в Птолемаиду в сто шестьдесят втором году.
\vs 1Ma 10:58 Царь Александр встретил его, и он выдал за него Клеопатру, дочь свою, и устроил брак ее в Птолемаиде, как прилично царям, с великою пышностью.
\vs 1Ma 10:59 Писал также царь Александр Ионафану, чтобы он вышел к нему навстречу.
\vs 1Ma 10:60 И отправился Ионафан в Птолемаиду с пышностью, и представлялся обоим царям и одарил их и приближенных их серебром и золотом и многими дарами, и приобрел благоволение их.
\vs 1Ma 10:61 И собрались против него мужи зловредные из среды Израиля, мужи беззаконные, чтобы оклеветать его; но царь не внял им.
\vs 1Ma 10:62 И повелел царь снять с Ионафана одежды его и облечь его в порфиру,~--- и сделали так.
\vs 1Ma 10:63 И посадил его царь с собою и сказал своим правителям: выйдите с ним на средину города и провозгласите, чтобы никто не смел клеветать на него ни в каком деле и никто не тревожил его никаким делом.
\vs 1Ma 10:64 Когда клеветавшие увидели славу его, как он был провозглашаем и как облечен в порфиру, все разбежались.
\vs 1Ma 10:65 Так прославил его царь и вписал его в число первых друзей, и назначил его военачальником и областным правителем.
\vs 1Ma 10:66 И возвратился Ионафан в Иерусалим с миром и веселием.
\rsbpar\vs 1Ma 10:67 Но в сто шестьдесят пятом году пришел из Крита Димитрий, сын Димитрия, в землю отцов своих.
\vs 1Ma 10:68 Услышав о том, царь Александр весьма огорчился и возвратился в Антиохию.
\vs 1Ma 10:69 И поставил Димитрий военачальником Аполлония, правителя Келе-Сирии,~--- и он собрал большое войско и расположился станом при Иамнии и послал к первосвященнику Ионафану сказать:
\vs 1Ma 10:70 ты только один превозносишься над нами, я же подвергся осмеянию и посрамлению через тебя. Зачем ты противостоишь нам в горах?
\vs 1Ma 10:71 Если ты надеешься на твои военные силы, то сойди к нам на равнину, и там мы померяемся, ибо со мною войско городов.
\vs 1Ma 10:72 Спроси и узнай, кто я и прочие помогающие нам, и скажут тебе: невозможно вам устоять пред лицем нашим, ибо дважды обращены были в бегство отцы твои в земле своей.
\vs 1Ma 10:73 И ныне ты не можешь устоять против такой конницы и такого войска на равнине, где нет ни камней, ни ущелий, ни места для убежища.
\vs 1Ma 10:74 Когда Ионафан выслушал эти слова Аполлония, то подвигся духом и, избрав десять тысяч мужей, вышел из Иерусалима, и брат его Симон сошелся с ним на помощь ему.
\vs 1Ma 10:75 И расположился станом при Иоппии; но не впустили его в город, ибо в Иоппии была стража Аполлония, и они начали воевать против нее.
\vs 1Ma 10:76 Тогда устрашенные жители отворили ему город, и Ионафан овладел Иоппиею.
\vs 1Ma 10:77 Услышав о сем, Аполлоний взял три тысячи конницы и большое войско и пошел в Азот, как бы делая переход, а между тем прошел на равнину, ибо имел множество конницы и надеялся на нее.
\vs 1Ma 10:78 Ионафан же преследовал его до Азота, и вступили войска в сражение.
\vs 1Ma 10:79 Между тем Аполлоний оставил тысячу всадников в скрытном месте позади них;
\vs 1Ma 10:80 но Ионафан узнал, что есть засада сзади него. И обступили войско его и бросали в народ стрелы с утра до вечера,
\vs 1Ma 10:81 народ же стоял, как приказал Ионафан; наконец всадники утомились.
\vs 1Ma 10:82 Тогда Симон подвел войско свое и напал на отряд, ибо всадники изнемогли,~--- и были разбиты им и обратились в бегство.
\vs 1Ma 10:83 И рассеялись всадники по равнине и убежали в Азот, и вошли в Бетдагон, капище их, чтобы спастись.
\vs 1Ma 10:84 Но Ионафан сжег Азот и окрестные города и взял добычу их, и капище Дагона с убежавшими в него сжег огнем.
\vs 1Ma 10:85 И было павших от меча с сожженными до восьми тысяч мужей.
\vs 1Ma 10:86 Отправившись оттуда, Ионафан расположился станом против Аскалона; но жители города вышли к нему навстречу с великою почестью.
\vs 1Ma 10:87 И возвратился Ионафан со всеми бывшими при нем в Иерусалим, имея при себе много добычи.
\vs 1Ma 10:88 Когда царь Александр услышал о сих событиях, то вновь почтил Ионафана
\vs 1Ma 10:89 и послал ему золотую пряжку, какая по обычаю давалась царским родственникам, и подарил ему Аккарон и всю область его в наследственное владение.
\vs 1Ma 11:1 Между тем царь Египетский, собрав многочисленное войско, как песок на берегу морском, и множество кораблей, домогался овладеть царством Александра хитростью и присоединить его к своему царству.
\vs 1Ma 11:2 Он пришел в Сирию с мирными речами, и жители отворяли ему города и выходили навстречу, ибо дано было от царя Александра повеление встречать его, потому что он был тесть его.
\vs 1Ma 11:3 Когда же Птоломей входил в города, то оставлял войско для стражи в каждом городе.
\vs 1Ma 11:4 Когда приблизился он к Азоту, то показали ему сожженное капище Дагона, и Азот и окрестные города разрушенные, и тела пораженные и сожженные во время сражения, ибо сложили их в груды по пути его,
\vs 1Ma 11:5 и рассказали царю о всем, что сделал Ионафан, жалуясь на него; но царь промолчал.
\vs 1Ma 11:6 Тогда вышел Ионафан навстречу царю в Иоппию с почетом, и приветствовали друг друга и ночевали там.
\vs 1Ma 11:7 И шел Ионафан с царем до реки, называемой Елевфера, и потом возвратился в Иерусалим.
\vs 1Ma 11:8 Царь же Птоломей овладел городами на морском берегу до Селевкии приморской и составлял злые замыслы против Александра.
\vs 1Ma 11:9 И послал послов к царю Димитрию, говоря: приди сюда, заключим между собою союз, и я дам тебе дочь мою, которую имеет Александр, и ты будешь царствовать в царстве отца твоего.
\vs 1Ma 11:10 Я раскаиваюсь, что отдал ему дочь мою, ибо он старался убить меня.
\vs 1Ma 11:11 Так клеветал он на него, потому что сам домогался царства его.
\vs 1Ma 11:12 И, отняв у него дочь свою, отдал ее Димитрию, и стал чужим для Александра, и обнаружилась вражда их.
\vs 1Ma 11:13 И вошел Птоломей в Антиохию и возложил на свою голову два венца~--- Азии и Египта.
\vs 1Ma 11:14 Царь Александр находился в то время в Киликии, потому что жители тех мест отпали от него.
\vs 1Ma 11:15 Услышав об этом, Александр пошел против него воевать; тогда Птоломей вывел войско и встретил его с крепкою силою, и обратил его в бегство.
\vs 1Ma 11:16 И убежал Александр в Аравию, чтобы укрыться там; царь же Птоломей возвысился.
\vs 1Ma 11:17 Завдиил, Аравитянин, снял голову с Александра и послал ее Птоломею.
\vs 1Ma 11:18 Царь же Птоломей на третий день умер, а оставшиеся в крепостях истреблены были жителями крепостей.
\vs 1Ma 11:19 И воцарился Димитрий в сто шестьдесят седьмом году.
\rsbpar\vs 1Ma 11:20 В те дни собрал Ионафан Иудеев, чтобы завоевать крепость Иерусалимскую, и устроил перед нею множество машин.
\vs 1Ma 11:21 Но некоторые ненавистники народа своего, отступники от закона, пошли к царю и донесли, что Ионафан облагает крепость.
\vs 1Ma 11:22 Когда он услышал об этом, разгневался и, поспешно собравшись, отправился в Птолемаиду, и написал Ионафану, чтобы он не облагал крепости, а как можно скорее шел к нему навстречу в Птолемаиду, чтобы переговорить с ним.
\vs 1Ma 11:23 Но Ионафан, выслушав это, приказал продолжать осаду и, избрав из старейшин Израильских и священников, решился подвергнуться опасности.
\vs 1Ma 11:24 Взяв серебра и золота, одежды и много других даров, он пошел к царю в Птолемаиду и приобрел благоволение его.
\vs 1Ma 11:25 И хотя некоторые отступники из того же народа клеветали на него,
\vs 1Ma 11:26 но царь поступил с ним так же, как поступали с ним предшественники его, и возвысил его пред всеми друзьями своими,
\vs 1Ma 11:27 и утвердил за ним первосвященство и другие почетные отличия, какие он имел прежде, и сделал его одним из первых друзей своих.
\vs 1Ma 11:28 И просил Ионафан царя освободить от податей Иудею и три области и Самарию и обещал ему триста талантов.
\vs 1Ma 11:29 Царь согласился и написал Ионафану обо всем этом письмо такого содержания:
\vs 1Ma 11:30 <<Царь Димитрий брату Ионафану и народу Иудейскому~--- радоваться.
\vs 1Ma 11:31 Список письма, которое мы писали о вас Ласфену, родственнику нашему, посылаем и к вам, чтобы вы знали.
\vs 1Ma 11:32 Царь Димитрий Ласфену-отцу~--- радоваться.
\vs 1Ma 11:33 Народу Иудейскому, друзьям нашим, верно исполняющим свои обязанности перед нами, мы рассудили оказать благодеяние за их доброе расположение к нам.
\vs 1Ma 11:34 Итак, мы утверждаем за ними как пределы Иудеи, так и три области: Аферему, Лидду и Рамафем, которые присоединены к Иудее от Самарии, и все, принадлежащее всем жрецам их в Иерусалиме, за те царские оброки, которые прежде ежегодно получал от них царь с произрастаний земли и с плодов древесных,
\vs 1Ma 11:35 и все прочее, принадлежащее нам отныне из десятин и даней, следующих нам, соленые озера и венечный сбор, нам принадлежащий, все вполне уступаем им.
\vs 1Ma 11:36 И ничего не будет отменено из сего отныне и навсегда.
\vs 1Ma 11:37 Итак, позаботьтесь сделать список с сего, и пусть будет отдан он Ионафану и положен на святой горе в известном месте>>.
\rsbpar\vs 1Ma 11:38 И увидел царь Димитрий, что преклонилась земля пред ним и ничто не противилось ему, и отпустил все войска свои, каждого в свое место, кроме войск чужеземных, которые он нанял с островов чужих народов, за что все войска отцов его ненавидели его.
\vs 1Ma 11:39 Трифон, один из прежних приверженцев Александра, видя, что все войска ропщут на Димитрия, отправился к Емалкую Аравитянину, который воспитывал Антиоха, малолетнего сына Александрова;
\vs 1Ma 11:40 и настаивал, чтобы он выдал его ему, дабы сделать его царем вместо него; и рассказал ему обо всем, что сделал Димитрий, и о неприязни, которую имеют к нему войска его, и пробыл там много дней.
\rsbpar\vs 1Ma 11:41 И послал Ионафан к царю Димитрию, чтобы он вывел оставленных им в Иерусалимской крепости и укреплениях, ибо они нападали на Израиля.
\vs 1Ma 11:42 Димитрий послал сказать Ионафану: не только это сделаю для тебя и для народа твоего, но и почту тебя и народ твой великою честью, как скоро буду иметь благоприятное время.
\vs 1Ma 11:43 Теперь же ты справедливо поступишь, если пришлешь мне людей на помощь в войне, ибо отложились от меня все войска мои.
\vs 1Ma 11:44 И послал к нему Ионафан в Антиохию три тысячи храбрых мужей, и пришли они к царю, и обрадовался царь прибытию их.
\vs 1Ma 11:45 Граждане же, собравшись на средину города до ста двадцати тысяч человек, хотели убить царя.
\vs 1Ma 11:46 Но царь убежал во дворец, а граждане заняли все улицы города и начали осаждать его.
\vs 1Ma 11:47 Тогда царь призвал на помощь Иудеев, и все они тотчас собрались к нему, и вдруг рассыпались по городу, и умертвили в тот день в городе до ста тысяч,
\vs 1Ma 11:48 и зажгли город, и взяли в тот день много добычи, и спасли царя.
\vs 1Ma 11:49 И увидели граждане, что Иудеи овладели городом, как хотели, и упали духом, и начали взывать к царю, умоляя и говоря:
\vs 1Ma 11:50 прости нас, и пусть Иудеи перестанут нападать на нас и на город.
\vs 1Ma 11:51 И сложили оружие и заключили мир. И прославились Иудеи перед царем и перед всеми в царстве его и возвратились в Иерусалим с большою добычею.
\vs 1Ma 11:52 И воссел царь Димитрий на престоле царства своего, и успокоилась земля пред ним.
\vs 1Ma 11:53 Но он солгал во всем, что обещал, и изменил Ионафану и не воздал за сделанное ему добро и сильно оскорбил его.
\vs 1Ma 11:54 После того возвратился Трифон и с ним Антиох, еще очень юный; он воцарился и возложил на себя венец.
\vs 1Ma 11:55 И собрались к нему все войска, которые распустил Димитрий, и начали воевать с ним, и он обратился в бегство, и был поражен.
\vs 1Ma 11:56 И взял Трифон слонов и овладел Антиохиею.
\vs 1Ma 11:57 И писал юный Антиох Ионафану, говоря: предоставляю тебе первосвященство и поставляю тебя над четырьмя областями, и ты будешь в числе друзей царских.
\vs 1Ma 11:58 И послал ему золотые сосуды и домашнюю утварь и дал ему право пить из золотых сосудов и носить порфиру и золотую пряжку,
\vs 1Ma 11:59 а Симона, брата его, поставил военачальником от области Тирской до пределов Египта.
\vs 1Ma 11:60 И выступил Ионафан в поход, и проходил по ту сторону реки \bibemph{(Иордана)} и по городам, и собрались к нему на помощь все Сирийские войска; и пришел он к Аскалону, и встретили его жители города с честью.
\vs 1Ma 11:61 Оттуда пошел он в Газу; но жители Газы заперлись; и осадил он город, и сжег огнем предместья его, и опустошил их.
\vs 1Ma 11:62 И упросили жители Газы Ионафана, и он примирился с ними, только взял в заложники сыновей начальников их и отослал их в Иерусалим, и прошел страну до Дамаска.
\rsbpar\vs 1Ma 11:63 И услышал Ионафан, что пришли в Кадис, в Галилее, военачальники Димитрия с многочисленным войском, чтобы удалить его от страны.
\vs 1Ma 11:64 Но он пошел навстречу им, брата же своего, Симона, оставил в стране.
\vs 1Ma 11:65 И расположил Симон стан свой при Вефсуре, и осаждал его многие дни, и запер его.
\vs 1Ma 11:66 И просили его о мире, и он согласился, но выгнал их оттуда, и овладел городом, и поставил в нем стражу.
\vs 1Ma 11:67 А Ионафан и войско его расположились станом при водах Геннисаретских и утром стали на равнине Насор.
\vs 1Ma 11:68 И вот, войско иноплеменников встретилось с ним на равнине, оставив против него засаду в горах, само же шло навстречу ему с противной стороны.
\vs 1Ma 11:69 И вышли бывшие в засаде из своих мест, и начали сражаться: тогда все бывшие с Ионафаном обратились в бегство,
\vs 1Ma 11:70 и ни одного из них не осталось, кроме Маттафии, сына Авессаломова, и Иуды, сына Халфиева, начальников воинских отрядов.
\vs 1Ma 11:71 И разодрал Ионафан одежды свои, и посыпал землю на голову свою, и молился.
\vs 1Ma 11:72 Потом возвратился сражаться с ними и поразил их, и они бежали.
\vs 1Ma 11:73 Увидев это, убежавшие от него возвратились к нему, и с ним преследовали их до Кадиса, до самого стана их, и там остановились.
\vs 1Ma 11:74 В тот день пало от иноплеменников до трех тысяч мужей; и возвратился Ионафан в Иерусалим.
\vs 1Ma 12:1 Ионафан, видя, что время благоприятствует ему, избрал мужей и послал в Рим установить и возобновить дружбу с Римлянами,
\vs 1Ma 12:2 и к Спартанцам и в другие места послал письма о том же.
\vs 1Ma 12:3 И пришли они в Рим, и вошли в совет, и сказали: <<Ионафан-первосвященник и народ Иудейский прислали нас, чтобы возобновить дружбу с вами и союз по-прежнему>>.
\vs 1Ma 12:4 И там дали им письма к местным начальникам, чтобы проводили их в землю Иудейскую с миром.
\vs 1Ma 12:5 Вот список письма, которое писал Ионафан Спартанцам:
\vs 1Ma 12:6 <<Первосвященник Ионафан и народные старейшины и священники и остальной народ Иудейский братьям Спартанцам~--- радоваться.
\vs 1Ma 12:7 Еще прежде от Дария [Арея], царствовавшего у вас, присланы были к первосвященнику Онии письма, что вы~--- братья наши, как показывает список.
\vs 1Ma 12:8 И принял Ония посланного мужа с честью, и получил письма, в которых ясно говорилось о союзе и дружбе.
\vs 1Ma 12:9 Мы же, хотя и не имеем надобности в них, имея утешением священные книги, которые в руках наших,
\vs 1Ma 12:10 но предприняли послать к вам для возобновления братства и дружбы, чтобы не отчуждаться от вас, ибо много прошло времени после того, как вы присылали к нам.
\vs 1Ma 12:11 Мы неопустительно во всякое время, как в праздники, так и в прочие установленные дни, воспоминаем о вас при жертвоприношениях наших и молитвах, как должно и прилично воспоминать братьев.
\vs 1Ma 12:12 Мы радуемся о вашей славе;
\vs 1Ma 12:13 нас же обстоят многие беды и частые войны; ибо воевали против нас окрестные цари.
\vs 1Ma 12:14 Но мы не хотели беспокоить вас и прочих союзников и друзей наших в этих войнах,
\vs 1Ma 12:15 ибо мы имеем помощь небесную, помогающую нам; мы избавились от врагов наших, и враги наши усмирены.
\vs 1Ma 12:16 Теперь мы избрали Нуминия, сына Антиохова, и Антипатра, сына Иасонова, и послали их к Римлянам возобновить дружбу с ними и прежний союз.
\vs 1Ma 12:17 Поручили им идти и к вам, приветствовать вас и вручить вам письма от нас о возобновлении и с вами нашего братства.
\vs 1Ma 12:18 И вы хорошо сделаете, ответив нам на них>>.
\rsbpar\vs 1Ma 12:19 Вот и список писем, которые прислал Дарий [Арей]:
\vs 1Ma 12:20 <<Царь Спартанский Онии первосвященнику~--- радоваться.
\vs 1Ma 12:21 Найдено в писании о Спартанцах и Иудеях, что они~--- братья и от рода Авраамова.
\vs 1Ma 12:22 Теперь, когда мы узнали об этом, вы хорошо сделаете, написав нам о благосостоянии вашем.
\vs 1Ma 12:23 Мы же уведомляем вас: скот ваш и имущество ваше~--- наши, а что у нас есть, то ваше. И мы повелели объявить вам о том>>.
\rsbpar\vs 1Ma 12:24 И услышал Ионафан, что возвратились военачальники Димитрия с б\acc{о}льшим войском, нежели прежде, чтобы воевать против него,
\vs 1Ma 12:25 и вышел из Иерусалима, и встретил их в стране Амафитской, и не дал им времени войти в страну его.
\vs 1Ma 12:26 И послал соглядатаев в стан их, которые, возвратившись, объявили ему, что они готовятся напасть на них в эту ночь.
\vs 1Ma 12:27 Посему, когда зашло солнце, Ионафан приказал своим бодрствовать, быть в вооружении и готовиться к сражению всю ночь, и поставил вокруг стана передовых сторожей.
\vs 1Ma 12:28 И услышали неприятели, что Ионафан со своими приготовился к сражению, и устрашились, и затрепетали сердцем своим, и, зажегши огни в стане своем, ушли.
\vs 1Ma 12:29 Ионафан же и бывшие с ним не знали о том до утра, ибо видели горящие огни.
\vs 1Ma 12:30 И погнался Ионафан за ними, но не настиг их, потому что они перешли реку Елевферу.
\vs 1Ma 12:31 Тогда Ионафан обратился на Арабов, называемых Заведеями, поразил их и взял добычу их.
\vs 1Ma 12:32 Потом, возвратившись, пришел в Дамаск и прошел по всей той стране.
\vs 1Ma 12:33 И Симон вышел, и прошел до Аскалона и ближайших крепостей, и обратился в Иоппию, и овладел ею
\vs 1Ma 12:34 ибо он услышал, что \bibemph{Иоппияне} хотят сдать крепость войскам Димитрия,~--- и поставил там стражу, чтобы охранять ее.
\vs 1Ma 12:35 И возвратился Ионафан, и созвал старейшин народа, и советовался с ними, чтобы построить крепости в Иудее,
\vs 1Ma 12:36 возвысить стены Иерусалима и воздвигнуть высокую стену между крепостью и городом, дабы отделить ее от города, так чтобы она была особо и не было бы в ней ни купли, ни продажи.
\vs 1Ma 12:37 Когда собрались устроить город и дошли до стены у потока с восточной стороны, то построили так называемую Хафенафу.
\vs 1Ma 12:38 А Симон построил Адиду в Сефиле и укрепил ворота и запоры.
\rsbpar\vs 1Ma 12:39 Между тем Трифон домогался сделаться царем Азии и возложить на себя венец и поднять руку на царя Антиоха,
\vs 1Ma 12:40 но опасался, как бы не воспрепятствовал ему Ионафан и не начал против него войны; поэтому искал случая, чтобы взять Ионафана и убить, и, поднявшись, пошел в Вефсан.
\vs 1Ma 12:41 И вышел Ионафан навстречу ему с сорока тысячами избранных мужей, готовых к битве, и пришел в Вефсан.
\vs 1Ma 12:42 Когда Трифон увидел, что Ионафан идет с многочисленным войском, то побоялся поднять на него руки.
\vs 1Ma 12:43 И принял его с честью, и представил его всем друзьям своим, дал ему подарки, приказал войскам своим повиноваться ему, как себе самому.
\vs 1Ma 12:44 Потом сказал Ионафану: для чего ты утруждаешь весь этот народ, когда не предстоит нам войны?
\vs 1Ma 12:45 Итак, отпусти их теперь в домы их, а для себя избери немногих мужей, которые были бы с тобою, и пойдем со мною в Птолемаиду, и я передам ее тебе и другие крепости и остальные войска и всех, заведующих сборами, и потом возвращусь; ибо для этого я и нахожусь здесь.
\vs 1Ma 12:46 И поверил ему Ионафан, и сделал так, как он сказал, и отпустил войска, и они отправились в землю Иудейскую;
\vs 1Ma 12:47 с собою же оставил три тысячи мужей, из которых две тысячи оставил в Галилее, тысяча же отправилась с ним.
\vs 1Ma 12:48 Но как скоро вошел Ионафан в Птолемаиду, Птолемаидяне заперли ворота, и схватили его, и всех вошедших с ним убили мечом.
\vs 1Ma 12:49 Тогда Трифон послал войско и конницу в Галилею и на великую равнину, чтобы истребить всех бывших с Ионафаном.
\vs 1Ma 12:50 Но они, услышав, что Ионафан схвачен и погиб и бывшие с ним, ободрили друг друга и вышли густым строем, готовые сразиться.
\vs 1Ma 12:51 И увидели преследующие, что дело идет о жизни, и возвратились назад.
\vs 1Ma 12:52 А они все благополучно пришли в землю Иудейскую и оплакивали Ионафана и бывших с ним, и были в большом страхе, и весь Израиль плакал горьким плачем.
\vs 1Ma 12:53 Тогда все окрестные народы искали истребить их, ибо говорили: теперь нет у них начальника и поборника; итак, будем теперь воевать против них и истребим из среды людей память их.
\vs 1Ma 13:1 Услышал Симон, что Трифон собрал большое войско, чтобы идти в землю Иудейскую и разорить ее.
\vs 1Ma 13:2 И, видя, что народ в страхе и трепете, взошел в Иерусалим и собрал народ.
\vs 1Ma 13:3 И, ободряя их, говорил им: сами вы знаете, сколько я и братья мои и дом отца моего сделали ради этих законов и святыни, знаете войны и угнетения, какие мы испытали.
\vs 1Ma 13:4 Потому и погибли все братья мои за Израиля, и остался я один.
\vs 1Ma 13:5 И ныне да не будет того, чтобы я стал щадить жизнь мою во все время угнетения, ибо я не лучше братьев моих.
\vs 1Ma 13:6 Но буду мстить за народ мой и за святилище, и за жен и за детей наших, ибо соединились все народы, чтобы истребить нас по неприязни.
\vs 1Ma 13:7 И воспламенился дух народа, как только услышал он такие слова;
\vs 1Ma 13:8 и отвечали громким голосом, и сказали: ты~--- наш вождь на место Иуды и Ионафана, брата твоего.
\vs 1Ma 13:9 Веди нашу войну, и, что ты ни скажешь нам, мы всё сделаем.
\vs 1Ma 13:10 Тогда собрал он всех мужей ратных, и поспешил окончить стены Иерусалима, и со всех сторон укрепил его.
\vs 1Ma 13:11 Потом послал Ионафана, сына Авессаломова, и с ним достаточное число войска в Иоппию, и он выгнал бывших в ней и остался там.
\rsbpar\vs 1Ma 13:12 Между тем Трифон поднялся из Птолемаиды с многочисленным войском, чтобы войти в землю Иудейскую; с ним был и Ионафан под стражею.
\vs 1Ma 13:13 Симон же расположил стан при Адиде напротив равнины.
\vs 1Ma 13:14 Когда Трифон узнал, что Симон заступил место Ионафана, брата своего, и намеревается вступить в сражение с ним, то послал к нему послов сказать:
\vs 1Ma 13:15 за серебро, которым брат твой Ионафан задолжал царской казне по надобностям, какие он имел, мы удержали его.
\vs 1Ma 13:16 Итак, пришли теперь сто талантов серебра и в заложники двух сыновей его, чтобы он, быв отпущен, не отложился от нас,~--- и мы отпустим его.
\vs 1Ma 13:17 Симон понимал, что они говорят с ним коварно, но послал серебро и детей, чтобы не навлечь большой ненависти от народа,
\vs 1Ma 13:18 который сказал бы: оттого, что я не послал ему серебра и детей, \bibemph{Ионафан} погиб.
\vs 1Ma 13:19 Итак, послал детей и сто талантов; но Трифон обманул и не отпустил Ионафана.
\vs 1Ma 13:20 После сего Трифон пошел, чтобы войти в страну и разорить ее, и пошел окольным путем на Адару. Но Симон и войско его следовали за ним повсюду, куда он ни шел.
\vs 1Ma 13:21 Бывшие же в крепости послали к Трифону послов, чтобы побудить его прийти к ним чрез пустыню и прислать им съестных припасов.
\vs 1Ma 13:22 И приготовил Трифон всю свою конницу, чтобы идти в ту же ночь, но был очень большой снег, и он не пошел по причине снега, а, поднявшись, отправился в Галаад.
\vs 1Ma 13:23 Когда же приблизился к Васкаме, умертвил Ионафана, и он погребен там.
\vs 1Ma 13:24 И возвратился Трифон и ушел в землю свою.
\vs 1Ma 13:25 Тогда Симон послал и взял кости Ионафана, брата своего, и похоронил их в Модине, городе отцов своих.
\vs 1Ma 13:26 И оплакивал его весь Израиль горьким плачем, и сокрушались о нем многие дни.
\vs 1Ma 13:27 И воздвиг Симон здание над гробом отца своего и братьев своих и вывел его высоко, для благовидности, из тесаного камня с передней и задней стороны,
\vs 1Ma 13:28 и поставил на нем семь пирамид, одну против другой, отцу и матери и четырем братьям;
\vs 1Ma 13:29 сделал на них искусные украшения, поставив вокруг высокие столбы, а на столбах полное вооружение~--- на вечную память, и подле оружий~--- изваянные корабли, так что они были видимы всеми, плавающими по морю.
\vs 1Ma 13:30 Этот надгробный памятник, который сделал он в Модине, стоит до сего дня.
\rsbpar\vs 1Ma 13:31 Трифон же с коварством отправился в путь с юным царем Антиохом и убил его,
\vs 1Ma 13:32 и воцарился вместо него, и возложил на себя венец Азии, и произвел великое поражение на земле.
\vs 1Ma 13:33 А Симон строил крепости в Иудее, укрепляя их высокими башнями и большими стенами, воротами и запорами, и складывал в крепостях съестные запасы.
\vs 1Ma 13:34 Потом избрал Симон мужей и послал к царю Димитрию просить, чтобы он сделал облегчение стране, ибо все деяния Трифона были грабительские.
\vs 1Ma 13:35 И послал ему царь Димитрий ответ на эти слова и написал такое письмо:
\vs 1Ma 13:36 <<Царь Димитрий Симону, первосвященнику и другу царей, и старейшинам и народу Иудейскому~--- радоваться.
\vs 1Ma 13:37 Золотой венец и пальмовую ветвь, посланную вами, мы получили и готовы заключить с вами полный мир и написать заведующим сборами, чтобы отпустить вам дани.
\vs 1Ma 13:38 И всё, что мы постановили о вас, да будет неизменно, и крепости, которые вы построили, пусть принадлежат вам.
\vs 1Ma 13:39 Прощаем вам также неумышленные проступки ваши до сего дня и венечный сбор, который платить вы обязаны, и если другое что взимаемо было в Иерусалиме, более не будет взиматься.
\vs 1Ma 13:40 И если найдутся из вас способные быть вписанными в число состоящих при нас, пусть записываются, и да будет между нами мир>>.
\rsbpar\vs 1Ma 13:41 В сто семидесятом году снято иго язычников с Израиля;
\vs 1Ma 13:42 и народ Израильский в переписке и договорах начал писать: <<Первого года при Симоне, великом первосвященнике, вожде и правителе Иудеев>>.
\vs 1Ma 13:43 В это время Симон сделал нападение на Газу, окружил ее войском, устроил осадные машины и придвинул их к городу, разбил одну башню и овладел ею.
\vs 1Ma 13:44 А бывшие на машине вскочили в город, и произошло в городе великое смятение.
\vs 1Ma 13:45 И взошли граждане с женами и детьми на стену, разодрав одежды свои, и громко взывали, умоляя Симона дать им помилование,
\vs 1Ma 13:46 и говорили: поступи с нами не по злым делам нашим, но по милости твоей.
\vs 1Ma 13:47 И умилосердился над ними Симон, и не сражался с ними, а только выгнал их из города, и очистил домы, в которых находились идолы, и так вошел в город с славословиями и благословениями.
\vs 1Ma 13:48 И выбросил из него все нечистое, и поселил там мужей, соблюдающих закон, и укрепил его, и устроил в нем для себя жилище.
\vs 1Ma 13:49 Бывшим же в Иерусалимской крепости не позволяли ни выходить, ни вступать в страну, ни покупать, ни продавать, и они терпели сильный голод, и многие из них погибли от голода.
\vs 1Ma 13:50 Тогда воззвали они к Симону о мире, и он дал им его, но выгнал их оттуда и очистил крепость от осквернения,
\vs 1Ma 13:51 и взошел в нее в двадцать третий день второго месяца сто семьдесят первого года с славословиями, пальмовыми ветвями, с гуслями, кимвалами и цитрами, с псалмами и песнями, ибо сокрушен великий враг Израиля.
\vs 1Ma 13:52 И установил каждогодно проводить этот день с весельем, и укрепил гору храма, находящуюся близ крепости, и поселился там сам и бывшие с ним.
\vs 1Ma 13:53 И увидел Симон, что сын его Иоанн возмужал, и поставил его начальником над всеми войсками, и поселился в Газаре.
\vs 1Ma 14:1 В сто семьдесят втором году царь Димитрий собрал войска свои и отправился в Мидию, чтобы получить помощь себе для войны против Трифона.
\vs 1Ma 14:2 Но Арсак, царь Персидский и Мидийский, услышав, что Димитрий пришел в пределы его, послал одного из военачальников своих взять его живого.
\vs 1Ma 14:3 Тот отправился и разбил войско Димитрия, взял его и привел к Арсаку, который заключил его в темницу.
\rsbpar\vs 1Ma 14:4 И покоилась земля Иудейская во все дни Симона; он старался о благе народа своего, и нравилась им власть и слава его во все дни.
\vs 1Ma 14:5 И ко всей своей славе, он взял еще Иоппию для пристани и открыл вход островам морским,
\vs 1Ma 14:6 и распространил пределы народа своего, и овладел тою страною.
\vs 1Ma 14:7 Он набрал множество пленных и господствовал над Газарою и Вефсурою и над крепостью, очистил ее от осквернения, и не было противящегося ему.
\vs 1Ma 14:8 \bibemph{Иудеи} спокойно возделывали землю свою, и земля давала произведения свои и дерева в полях~--- плод свой.
\vs 1Ma 14:9 Старцы, сидя на улицах, все совещались о пользах общественных, и юноши облекались в пышные и воинские одежды.
\vs 1Ma 14:10 Городам доставлял он съестные припасы и делал их местами укрепленными, так что славное имя его произносилось до конца земли.
\vs 1Ma 14:11 Он восстановил мир в стране, и радовался Израиль великою радостью.
\vs 1Ma 14:12 И сидел каждый под виноградом своим и под смоковницею своею, и никто не страшил их.
\vs 1Ma 14:13 И не осталось никого на земле, кто воевал бы против них, и цари смирились в те дни.
\vs 1Ma 14:14 Он подкреплял всех бедных в народе своем, требовал исполнения закона и истреблял всякого беззаконника и злодея,
\vs 1Ma 14:15 украсил святилище и умножил священную утварь.
\rsbpar\vs 1Ma 14:16 Когда дошел слух до Рима и до Спарты, что Ионафан умер, они весьма опечалились.
\vs 1Ma 14:17 Когда же услышали, что Симон, брат его, сделался вместо него первосвященником и господствует над страною и находящимися в ней городами,
\vs 1Ma 14:18 то написали к нему на медных досках, чтобы возобновить с ним дружбу и союз, заключенный ими с братьями его Иудою и Ионафаном.
\vs 1Ma 14:19 Они были прочитаны в Иерусалиме пред собранием.
\vs 1Ma 14:20 Вот список с писем, присланных Спартанцами: <<Спартанские начальники и город Симону первосвященнику, старейшинам и священникам и всему народу Иудейскому, братьям нашим~--- радоваться.
\vs 1Ma 14:21 Послы, присланные к народу нашему, рассказали нам о вашей славе и чести, и мы возрадовались прибытию их
\vs 1Ma 14:22 и записали сказанное ими в народном совете так: Нуминий, сын Антиоха, и Антипатр, сын Иасона, послы Иудейские, пришли к нам возобновить с нами дружбу.
\vs 1Ma 14:23 И угодно было народу принять этих мужей с честью и внести запись слов их в открытые народные книги, на память народу Спартанскому. А список с этого мы написали для первосвященника Симона>>.
\rsbpar\vs 1Ma 14:24 После того Симон послал Нуминия в Рим с большим золотым щитом, весом в тысячу мин, чтобы заключить с ними союз.
\vs 1Ma 14:25 Когда услышал об этом народ, то сказал: какую благодарность воздадим мы Симону и сыновьям его?
\vs 1Ma 14:26 Ибо он твердо стоял и братья его и дом отца его, и отразили врагов Израиля, и доставили ему свободу.
\vs 1Ma 14:27 И написали о том на медных досках и выставили их на столбах на горе Сион. Вот список написанного: <<В восемнадцатый день Елула сто семьдесят второго года~--- это был третий год при первосвященнике Симоне~---
\vs 1Ma 14:28 в Сарамели, в великом собрании священников и народа и князей народных и старейшин страны, объявлено нам:
\vs 1Ma 14:29 так как много раз бывали войны в этой стране, то Симон, сын Маттафии, сын сынов Иарива, и братья его, подвергая себя опасности, противостали врагам народа своего, чтобы сохранить святилище его и закон, и великою славою прославили народ свой.
\vs 1Ma 14:30 Ионафан собрал народ свой и сделался первосвященником его, но он приложился к народу своему.
\rsbpar\vs 1Ma 14:31 Когда же враги их вознамерились войти в страну их, чтобы разорить страну их и простереть руки на святилище их,
\vs 1Ma 14:32 тогда восстал Симон и воевал за народ свой и издержал много собственных денег, снабжая храбрых мужей народа своего оружием и давая им жалованье.
\vs 1Ma 14:33 Он укрепил города Иудеи и Вефсуру на границах Иудеи, где прежде находились оружия неприятелей, и поставил там стражу из Иудеев.
\vs 1Ma 14:34 Также укрепил Иоппию при море и Газару на пределах Азота, в которой прежде обитали враги, и поселил там Иудеев, снабдив эти \bibemph{места} всем, что нужно было к восстановлению их.
\vs 1Ma 14:35 И видел народ деяния Симона и славу, какую старался он доставить народу своему, и поставил его своим начальником и первосвященником за то, что все это сделал он, и за справедливость и верность, которую он хранил к племени своему, всячески стараясь возвысить народ свой.
\vs 1Ma 14:36 Во дни его руками его успешно изгнаны из страны язычники и занимавшие город Давидов в Иерусалиме, которые, устроив себе крепость, выходили из нее и оскверняли все вокруг святилища и много вредили святыне.
\vs 1Ma 14:37 Он поселил в ней Иудеев и укрепил ее для безопасности страны и города и возвысил стены Иерусалима.
\vs 1Ma 14:38 Посему и царь Димитрий утвердил за ним первосвященство,
\vs 1Ma 14:39 и причислил его к друзьям своим, и почтил его великою славою.
\vs 1Ma 14:40 Ибо он услышал, что Римляне назвали Иудеев друзьями и союзниками и братьями и с честью приняли послов Симона,
\vs 1Ma 14:41 что Иудеи и священники согласились, чтобы Симон был у них начальником и первосвященником навек, доколе восстанет Пророк верный,
\vs 1Ma 14:42 чтобы он был у них военачальником и имел попечение о святых и поставлял их над работами их, и над областью, и над оружиями, и над крепостями,
\vs 1Ma 14:43 чтобы имел попечение о святилище и все слушались его, чтобы все договоры в стране писались на его имя и чтобы он одевался в порфиру и носил золотые украшения.
\vs 1Ma 14:44 И никому из народа и священников да не будет позволено отменить что-либо из сего или противоречить словам его, или без него созывать собрание в стране и одеваться в порфиру и носить золотую пряжку.
\vs 1Ma 14:45 А кто сделает что-нибудь против сего или отменит что из сего, будет повинен>>.
\vs 1Ma 14:46 И согласился весь народ подчиниться Симону и поступать по словам сим.
\vs 1Ma 14:47 Симон принял и согласился быть первосвященником и военачальником и правителем Иудеев и священников и начальствовать над всеми.
\vs 1Ma 14:48 И решили начертать запись сию на медных досках и поставить их в ограде храма на видном месте,
\vs 1Ma 14:49 а списки с них положить в сокровищнице, чтобы имел их Симон и сыновья его.
\vs 1Ma 15:1 И прислал Антиох, сын царя Димитрия, письма с островов морских к Симону, великому священнику и правителю народа Иудейского, и всему народу.
\vs 1Ma 15:2 Они были такого содержания: <<Царь Антиох Симону, первосвященнику и правителю народа, и народу Иудейскому~--- радоваться.
\vs 1Ma 15:3 Так как люди зловредные овладели царством отцов наших, то я хочу возвратить царство, чтобы восстановить его, как оно было прежде. Я набрал множество войска и приготовил военные корабли;
\vs 1Ma 15:4 и хочу пройти по области, чтобы наказать тех, которые опустошили область нашу и разорили многие города в царстве.
\vs 1Ma 15:5 Оставляю теперь за тобою все дани, какие уступали тебе цари, бывшие прежде меня, и другие дары, какие они уступали тебе;
\vs 1Ma 15:6 дозволяю тебе чеканить свою монету в стране твоей.
\vs 1Ma 15:7 Иерусалим и святилище пусть будут свободны; и все оружия, которые ты заготовил, и крепости, построенные тобою, которыми ты владеешь, пусть остаются у тебя.
\vs 1Ma 15:8 И всякий долг царский и будущие царские долги отныне и навсегда пусть будут отпущены тебе.
\vs 1Ma 15:9 Когда же мы овладеем царством нашим, тогда почтим тебя и народ твой и храм великою честью, чтобы слава ваша стала известна по всей земле>>.
\rsbpar\vs 1Ma 15:10 В сто семьдесят четвертом году вступил Антиох в землю отцов своих, и собрались к нему все войска, так что оставшихся с Трифоном было немного.
\vs 1Ma 15:11 И преследовал его царь Антиох, и он убежал в Дору, которая при море;
\vs 1Ma 15:12 ибо он увидел, что обрушились на него беды и оставили его войска.
\vs 1Ma 15:13 И пришел Антиох к Доре и с ним сто двадцать тысяч воинов и восемь тысяч конницы
\vs 1Ma 15:14 и окружил город, а корабли подошли с моря, и теснил он город с суши и моря, и не давал никому ни выйти, ни войти.
\rsbpar\vs 1Ma 15:15 Тогда пришел из Рима Нуминий и сопровождавшие его с письмами к царям и странам, в которых было написано следующее:
\vs 1Ma 15:16 <<Левкий, консул Римский, царю Птоломею~--- радоваться.
\vs 1Ma 15:17 Пришли к нам Иудейские послы, друзья наши и союзники, посланные от первосвященника Симона и народа Иудейского, возобновить давнюю дружбу и союз,
\vs 1Ma 15:18 и принесли золотой щит в тысячу мин.
\vs 1Ma 15:19 Итак, мы заблагорассудили написать царям и странам, чтобы они не причиняли им зла, и не воевали против них и городов их и страны их, и не помогали воюющим против них.
\vs 1Ma 15:20 Мы рассудили принять от них щит.
\vs 1Ma 15:21 Итак, если какие зловредные люди убежали к вам из страны их, выдайте их первосвященнику Симону, чтобы он наказал их по закону их>>.
\vs 1Ma 15:22 То же самое написал он царю Димитрию и Атталу, Ариарафе и Арсаку,
\vs 1Ma 15:23 и во все области, и Сампсаме и Спартанцам, и в Делос и в Минд, и в Сикион, и в Карию, и в Самос, и в Памфилию, и в Ликию, и в Галикарнасс, и в Родос, и в Фасилиду, и в Кос, и в Сиду, и в Арад, и в Гортину, и в Книду, и в Кипр, и в Киринию.
\vs 1Ma 15:24 Список с этих писем написали Симону первосвященнику.
\rsbpar\vs 1Ma 15:25 Царь же Антиох обложил Дору вторично, нападая на нее со всех сторон и устраивая машины, и запер Трифона так, что невозможно было ему ни войти, ни выйти.
\vs 1Ma 15:26 И послал к нему Симон две тысячи избранных мужей в помощь ему, и серебро и золото, и довольно запасов;
\vs 1Ma 15:27 но он не захотел принять это и отверг все, в чем прежде условился с ним, и отчуждился от него.
\vs 1Ma 15:28 И послал к нему Афиновия, одного из друзей своих, чтобы переговорить с ним и сказать: <<Вы владеете Иоппиею и Газарою и крепостью Иерусалимскою~--- городами царства моего;
\vs 1Ma 15:29 вы опустошили пределы их и произвели великое поражение на земле, и овладели многими местами в царстве моем.
\vs 1Ma 15:30 Итак, отдайте теперь города, которые вы взяли, и дани с тех мест, которыми вы владеете вне пределов Иудейских.
\vs 1Ma 15:31 Если же не так, то дайте за них пятьсот талантов серебра, и за опустошение, которое произвели, и за дани с городов другие пятьсот талантов; а если не дадите, то мы придем и будем сражаться с вами>>.
\vs 1Ma 15:32 И пришел Афиновий, друг царя, в Иерусалим, и когда увидел славу Симона и сокровищницу с золотою и серебряною утварью и окружающее великолепие, то изумился и объявил ему слова царя.
\vs 1Ma 15:33 Симон сказал ему в ответ: мы ни чужой земли не брали, ни господствовали над чужим, но \bibemph{владеем} наследием отцов наших, которое враги наши в одно время неправедно присвоили себе.
\vs 1Ma 15:34 Мы же, улучив время, опять возвратили себе наследие отцов наших.
\vs 1Ma 15:35 Что касается до Иоппии и Газары, которых ты требуешь, то они сами причинили много зла народу в стране нашей; за них мы дадим сто талантов. На это Афиновий ничего не отвечал;
\vs 1Ma 15:36 но, с досадою возвратившись к царю, рассказал ему эти слова и о славе Симона, и о всем, что видел, и царь сильно разгневался.
\vs 1Ma 15:37 Трифон же, сев на корабль, убежал в Орфосиаду.
\vs 1Ma 15:38 Тогда царь, сделав военачальником приморской страны Кендевея, вручил ему пешие и конные войска
\vs 1Ma 15:39 и приказал ему идти войною против Иудеи, приказал ему также построить Кедрон и укрепить ворота, и как воевать с народом; сам же царь погнался за Трифоном.
\vs 1Ma 15:40 И пришел Кендевей в Иамнию, и начал вызывать на бой народ и вторгаться в Иудею и брать народ в плен и убивать;
\vs 1Ma 15:41 и построил Кедрон, и расположил там конницу и войско, чтобы они, выходя оттуда, обходили пути Иудеи, как приказал ему царь.
\vs 1Ma 16:1 И возвратился Иоанн из Газары и рассказал Симону, отцу своему, о том, что делал Кендевей.
\vs 1Ma 16:2 Тогда Симон призвал двух старших сыновей своих, Иуду и Иоанна, и сказал им: я и братья мои и дом отца моего воевали против врагов Израиля от юности до сего дня и много раз успешно спасали руками нашими Израиля.
\vs 1Ma 16:3 Но вот, я состарился, а вы по милости \bibemph{Божией} находитесь в летах зрелых: заступите место мое и брата моего, идите и сражайтесь за народ наш, и да будет с вами помощь небесная.
\vs 1Ma 16:4 И избрал из страны двадцать тысяч воинов и всадников, и пошли они против Кендевея, и ночевали в Модине.
\vs 1Ma 16:5 Встав же утром, вышли на равнину, и вот многочисленное войско навстречу им, пешие и конные, и между ними был поток.
\vs 1Ma 16:6 И двинулся против них сам и народ его, и, видя, что народ боится переходить поток, он перешел первый, и увидели это воины, и перешли за ним.
\vs 1Ma 16:7 И разделил он народ, поставив конных среди пеших; конница же неприятелей была весьма многочисленна.
\vs 1Ma 16:8 И затрубили священными трубами; и Кендевей обратился в бегство и войско его, и пало у них много раненых, остальные же бежали в крепость.
\vs 1Ma 16:9 Тогда был ранен Иуда, брат Иоанна; но Иоанн преследовал их, доколе не пришел в Кедрон, который он построил.
\vs 1Ma 16:10 И убежали они в башни, находящиеся в области Азота, но он сжег его огнем, и погибло из них до двух тысяч мужей; и возвратился он с миром в землю Иудейскую.
\rsbpar\vs 1Ma 16:11 Птоломей же, сын Авува, поставлен был военачальником на равнине Иерихонской и имел много серебра и золота;
\vs 1Ma 16:12 ибо он был зять первосвященника.
\vs 1Ma 16:13 И надмилось сердце его, и захотел он овладеть страною, и делал коварные замыслы против Симона и сыновей его, чтобы погубить их.
\vs 1Ma 16:14 Между тем Симон, посещая города страны и заботясь о потребностях их, пришел в Иерихон, сам и Маттафия и Иуда, сыновья его, в сто семьдесят седьмом году в одиннадцатом месяце~--- это месяц Сават.
\vs 1Ma 16:15 И с коварством принял их радушно сын Авувов в небольшую крепость, называемую Док, им устроенную, и сделал для них большой пир, и спрятал там людей.
\vs 1Ma 16:16 И когда опьянел Симон и сыновья его, тогда встал Птоломей и бывшие при нем, взяли оружия свои и вошли к Симону во время пира и убили его и двух сыновей его и некоторых из служителей его.
\vs 1Ma 16:17 Так совершил он великое вероломство и воздал за добро злом.
\vs 1Ma 16:18 Птоломей написал об этом и послал к царю, чтобы прислал ему войско на помощь, и он предаст ему страну их и города.
\vs 1Ma 16:19 И некоторых послал в Газару убить Иоанна, а тысяченачальникам послал письма, чтобы они пришли к нему, и он даст им серебра и золота и подарки;
\vs 1Ma 16:20 а других послал овладеть Иерусалимом и горою храма.
\vs 1Ma 16:21 Но некто, прибежав к Иоанну в Газару, известил его, что отец его и братья умерщвлены и что \bibemph{Птоломей} послал убить и его.
\vs 1Ma 16:22 Услышав об этом, \bibemph{Иоанн} весьма смутился и, схватив мужей, пришедших погубить его, убил их, ибо узнал, что они искали погубить его.
\rsbpar\vs 1Ma 16:23 Прочие же дела Иоанна и в\acc{о}йны его и мужественные подвиги его, славно совершенные, и сооружение стен, им воздвигнутых, и другие деяния его,
\vs 1Ma 16:24 вот, они описаны в книге дней первосвященства его, с того времени, как сделался он первосвященником после отца своего.
\newbookpage
\include{tex/2Ma}
\include{tex/3Ma}
\bibbookdescr{3Ez}{
  inline={\LARGE Третья книга\\\Huge Ездры\fns{Книги этой нет ни на еврейском, ни на греческом языках. Как славянский, так и русский переводы сделаны с Вульгаты. В последней она разделена на две книги: первую составляют главы 3--14 по славянскому переводу, а вторая заключает в себе главы 1, 2, 15 и 16. В русском переводе удержан порядок глав славянского перевода.}},
  toc={3-я Ездры*},
  bookmark={3-я Ездры},
  header={3-я Ездры},
  %headerleft={},
  %headerright={},
  abbr={3~Езд}
}
\vs 3Ez 1:1 Вторая книга Ездры пророка, сына Сераии, сына Азарии, сына Хелкии, сына Шаллума, сына Садока, сына Ахитува,
\vs 3Ez 1:2 сына Ахии, сына Финееса, сына Илия, сына Амарии, сына Асиела, сына Мерайофа, сына Арна, сына Уззия, сына Ворифа, сына Авишуя, сына Финееса, сына Елеазара,
\vs 3Ez 1:3 сына Аарона от колена Левиина, который был пленником в стране Мидийской, в царствование Артаксеркса, царя Персидского.
\rsbpar\vs 3Ez 1:4 Было слово Господне ко мне:
\vs 3Ez 1:5 иди и возвести народу Моему злые дела их и сыновьям их~--- беззакония, которые они совершили против Меня, чтобы они возвестили сынам сынов своих;
\vs 3Ez 1:6 ибо грехи родителей их возросли в них; забыв Меня, они приносили жертвы богам чужим.
\vs 3Ez 1:7 Не Я ли вывел их из земли Египетской, из дома рабства? а они прогневали Меня и советы Мои презрели.
\vs 3Ez 1:8 Ты остриги волосы головы твоей, и брось на них все злое, ибо они не слушались закона моего~--- народ необузданный!
\vs 3Ez 1:9 Доколе Я буду терпеть их, которым сделал столько благодеяний?
\vs 3Ez 1:10 Ради них Я многих царей низложил; поразил фараона с рабами его и со всем войском его;
\vs 3Ez 1:11 всех язычников от лица их погубил, и на востоке народ двух областей, Тира и Сидона, рассеял и всех врагов их истребил.
\vs 3Ez 1:12 Ты же так скажи им: так говорит Господь:
\vs 3Ez 1:13 именно Я провел вас через море и по дну его проложил вам огражденную улицу, дал вам вождя Моисея и Аарона священника,
\vs 3Ez 1:14 дал вам свет в столпе огненном, и многие чудеса сотворил среди вас; а вы Меня забыли, говорит Господь.
\rsbpar\vs 3Ez 1:15 Так говорит Господь Вседержитель: перепелы были вам в знамение. Я дал вам станы для защиты, но вы и там роптали
\vs 3Ez 1:16 и не радовались во имя Мое о погибели врагов ваших, но даже доныне еще ропщете.
\vs 3Ez 1:17 Где те благодеяния, которые Я сделал вам? Не в пустыне ли, когда вы, взалкав, вопияли ко Мне,
\vs 3Ez 1:18 говоря: <<зачем Ты привел нас в эту пустыню? уморить нас? лучше нам было служить Египтянам, нежели умереть в этой пустыне>>?
\vs 3Ez 1:19 Я сжалился на стенания ваши, и дал вам манну в пищу: вы ели хлеб ангельский.
\vs 3Ez 1:20 Когда вы жаждали, не рассек ли Я камень, и потекли воды до сытости? от зноя покрывал вас листьями древесными.
\vs 3Ez 1:21 Разделил вам земли тучные; Хананеев, Ферезеев и Филистимлян изгнал от лица вашего. Что еще сделаю вам? говорит Господь.
\vs 3Ez 1:22 Так говорит Господь Вседержитель: когда вы были в пустыне, на реке Мерры, и жаждущие хулили имя Мое,
\vs 3Ez 1:23 не огонь послал Я на вас за богохульства, но вложил дерево в воду и реку сделал сладкою.
\vs 3Ez 1:24 Что сделаю тебе, Иаков? Не хотел ты повиноваться, Иуда. Переселюсь к другим народам и дам им имя Мое, чтобы соблюдали законы Мои.
\vs 3Ez 1:25 Так как вы Меня оставили, то и Я оставлю вас; просящих у Меня милости не помилую.
\vs 3Ez 1:26 Когда будете призывать Меня, Я не услышу вас, ибо вы осквернили руки ваши кровью, и ноги ваши быстры на совершение человекоубийства.
\vs 3Ez 1:27 Вы как бы не Меня оставили, а вас самих, говорит Господь.
\vs 3Ez 1:28 Так говорит Господь Вседержитель: не Я ли умолял вас, как отец сыновей и как мать дочерей и как кормилица питомцев своих,
\vs 3Ez 1:29 чтобы вы были Мне народом и Я вам Богом, чтобы вы были Мне сынами и Я вам Отцом?
\vs 3Ez 1:30 Я собрал вас, как курица птенцов своих под крылья свои. Что ныне сделаю вам? Отвергну вас от лица Моего.
\vs 3Ez 1:31 Когда принесете Мне приношение, отвращу лице Мое от вас; ибо ваши дни праздничные и новомесячия и обрезания Я отринул.
\vs 3Ez 1:32 Я послал к вам рабов Моих, пророков; вы, схватив их, умертвили и растерзали тела их. Кровь их Я взыщу, говорит Господь.
\rsbpar\vs 3Ez 1:33 Так говорит Господь Вседержитель: дом ваш пуст. Развею вас, как ветер мякину,
\vs 3Ez 1:34 и сыновья не будут иметь потомства, потому что заповедь Мою презрели и делали то, что зло предо Мною.
\vs 3Ez 1:35 Предам домы ваши людям грядущим, которые, не слышав Меня, уверуют, которые, хотя Я не показывал им знамений, исполнят то, что Я заповедал,
\vs 3Ez 1:36 не видев пророков, воспомянут о своих беззакониях.
\vs 3Ez 1:37 Завещеваю благодать людям грядущим, дети которых, не видев Меня очами плотскими, но духом веруя тому, что Я сказал, торжествуют с весельем.
\vs 3Ez 1:38 Итак теперь смотри, брат, какая слава,~--- смотри на людей, грядущих с востока,
\vs 3Ez 1:39 которым Я дам в вожди Авраама, Исаака и Иакова, и Осию, и Амоса, и Михея, и Иоиля, и Авдия, и Иону,
\vs 3Ez 1:40 и Наума, и Аввакума, Софонию, Аггея, Захарию и Малахию, который наречен и Ангелом Господним.
\vs 3Ez 2:1 Так говорит Господь: Я вывел народ сей из работы, дал им повеление через рабов Моих, пророков, которых они не захотели слушать, но отвергли Мои советы.
\vs 3Ez 2:2 Мать, которая родила их, говорит им: <<идите, дети; ибо я вдова и оставлена.
\vs 3Ez 2:3 Я воспитала вас с радостью, и отпустила с плачем и горестью, потому что вы согрешили пред Господом Богом вашим, и сделали злое пред Ним.
\vs 3Ez 2:4 Ныне же что сделаю для вас? Я вдова и оставлена: идите, дети, и просите у Господа милости>>.
\vs 3Ez 2:5 Тебя, Отче, призываю во свидетеля на мать сыновей, которые не захотели хранить завета моего.
\vs 3Ez 2:6 Предай их посрамлению и мать их~--- на расхищение, чтобы не было рода их.
\vs 3Ez 2:7 Пусть рассеются имена их по народам и изгладятся от земли, ибо они презрели завет мой.
\vs 3Ez 2:8 Горе тебе, Ассур, скрывающий у себя нечестивых! Род лукавый! вспомни, что Я сделал Содому и Гоморре.
\vs 3Ez 2:9 Земля их лежит в смоляных глыбах и холмах пепельных. Так поступлю Я с теми, которые Меня не слушались, говорит Господь Вседержитель.
\vs 3Ez 2:10 Так говорит Господь к Ездре: возвести народу Моему, что Я дам им царство Иерусалимское, которое обещал Израилю,
\vs 3Ez 2:11 и прииму славу от них и дам им обители вечные, которые приготовил для них.
\vs 3Ez 2:12 Древо жизни будет для них мастью благовонною; не будут изнуряемы трудом и не изнемогут.
\vs 3Ez 2:13 Идите и получ\acc{и}те; прос\acc{и}те себе дней малых, дабы они не замедлили. Уже готово для вас царство: бодрствуйте.
\vs 3Ez 2:14 Свидетельствуй, небо и земля, ибо Я стер злое и сотворил доброе. Живу Я! говорит Господь.
\vs 3Ez 2:15 Мать! обними сыновей твоих, воспитывай их с радостью; как голубица укрепляй ноги их, ибо Я избрал тебя, говорит Господь.
\vs 3Ez 2:16 И воскрешу мертвых от мест их и из гробов выведу их, потому что Я познал имя Мое в Израиле.
\vs 3Ez 2:17 Не бойся, мать сынов, ибо Я избрал тебя, говорит Господь.
\vs 3Ez 2:18 Я пошлю тебе в помощь рабов Моих Исаию и Иеремию, по совету которых Я освятил и приготовил тебе двенадцать дерев, обремененных различными плодами,
\vs 3Ez 2:19 и столько же источников, текущих молоком и медом, и семь гор величайших, произращающих розу и лилию, через которые исполню радостью сынов твоих.
\vs 3Ez 2:20 Оправдай вдову, дай суд бедному, помоги нищему, защити сироту, одень нагого,
\vs 3Ez 2:21 о расслабленном и немощном попекись, над хромым не смейся, безрукого защити, и слепого приведи к видению света Моего,
\vs 3Ez 2:22 старца и юношу в стенах твоих сохрани,
\vs 3Ez 2:23 мертвых, где найдешь, запечатлев, предай гробу, и Я дам тебе первое место в Моем воскресении.
\vs 3Ez 2:24 Отдыхай и покойся, народ Мой, ибо придет покой твой.
\vs 3Ez 2:25 Корми сынов твоих, добрая кормилица, укрепляй ноги их.
\vs 3Ez 2:26 Из рабов, которых Я дал тебе, никто да не погибнет, ибо Я взыщу их от тебя.
\vs 3Ez 2:27 Не ослабевай. Когда придет день печали и тесноты, другие будут плакать и сокрушаться, а ты будешь весела и изобильна.
\vs 3Ez 2:28 Язычники будут завидовать тебе, но ничего против тебя сделать не могут, говорит Господь.
\vs 3Ez 2:29 Руки Мои покроют тебя, чтобы сыны твои не видели геенны.
\vs 3Ez 2:30 Утешайся, мать, с сынами твоими, ибо Я спасу тебя.
\vs 3Ez 2:31 Помни о сынах твоих почивающих. Я выведу их от краев земли и окажу им милость, ибо Я милостив, говорит Господь Вседержитель.
\vs 3Ez 2:32 Обними детей твоих, доколе Я приду и сделаю им милость; ибо источники Мои обильны и благодать Моя не оскудеет.
\rsbpar\vs 3Ez 2:33 Я, Ездра, получил на горе Орив повеление от Господа идти к Израилю. Когда я пришел к ним, они отвергли меня и презрели заповедь Господню.
\vs 3Ez 2:34 Посему вам говорю, язычники, которые можете слышать и понимать: ожидайте Пастыря вашего, Он даст вам покой вечный, ибо близко Тот, Который придет в скончание века.
\vs 3Ez 2:35 Будьте готовы к воздаянию царствия, ибо свет немерцающий воссияет вам на вечное время.
\vs 3Ez 2:36 Избегайте тени века сего; приимите сладость славы вашей. Я открыто свидетельствую о Спасителе моем.
\vs 3Ez 2:37 Вверенный дар приимите, и наслаждайтесь, благодаря Того, Кто призвал вас в небесное царство.
\vs 3Ez 2:38 Встаньте и стойте, и смотрите, какое число знаменованных на вечери Господней,
\vs 3Ez 2:39 которые, переселившись от тени века сего, получили от Господа светлые одежды.
\vs 3Ez 2:40 Приими число твое, Сион, и заключи твоих, одетых в белые одеяния, которые исполнили закон Господень.
\vs 3Ez 2:41 Число желанных сынов твоих полно. Проси державу Господа, чтобы освятился народ твой, призванный от начала.
\rsbpar\vs 3Ez 2:42 Я, Ездра, видел на горе Сионской сонм великий, которого не мог исчислить, и все они песнями прославляли Господа.
\vs 3Ez 2:43 Посреди них был юноша величественный, превосходящий всех их, и возлагал венцы на главу каждого из них и тем более возвышался; я поражен был удивлением.
\vs 3Ez 2:44 Тогда я спросил Ангела: кто сии, господин мой?
\vs 3Ez 2:45 Он в ответ мне сказал: это те, которые сложили смертную одежду и облеклись в бессмертную и исповедали имя Божие; они теперь увенчиваются и принимают победные пальмы.
\vs 3Ez 2:46 Я спросил: а кто сей юноша, который возлагает на них венцы и вручает им пальмы?
\vs 3Ez 2:47 Он отвечал мне: Сам Сын Божий, Которого они прославляли в веке сем. И я начал славить их, мужественно стоявших за имя Господне.
\vs 3Ez 2:48 Тогда Ангел сказал мне: иди и возвести народу моему, какие видел ты дивные дела Господа Бога.
\vs 3Ez 3:1 В тридцатом году по разорении города был я в Вавилоне, и смущался, лежа на постели моей, и помышления всходили на сердце мое,
\vs 3Ez 3:2 ибо я видел опустошение Сиона и богатство живущих в Вавилоне.
\vs 3Ez 3:3 И возмутился дух мой, и я начал со страхом говорить ко Всевышнему,
\vs 3Ez 3:4 и сказал: Владыко Господи! Ты сказал от начала, когда един основал землю, и повелел персти,
\vs 3Ez 3:5 и дал Адаму тело смертное, которое было также создание рук Твоих, и вдохнул в него дух жизни, и он сделался живым пред Тобою,
\vs 3Ez 3:6 и ввел его в рай, который насадила десница Твоя, прежде нежели земля произрастила плоды;
\vs 3Ez 3:7 Ты повелел ему хранить заповедь Твою, но он нарушил ее, и Ты осудил его на смерть, и род его и происшедшие от него поколения и племена, народы и отрасли их, которым нет числа.
\vs 3Ez 3:8 Каждый народ стал ходить по своему хотению, делал пред Тобою дела неразумные и презирал заповеди Твои.
\vs 3Ez 3:9 По времени, Ты навел потоп на обитателей земли и истребил их,
\vs 3Ez 3:10 и исполнилось на каждом из них,~--- как на Адаме смерть, так на сих потоп.
\vs 3Ez 3:11 Одного из них Ты оставил~--- Ноя с семейством его, и от него произошли все праведные.
\vs 3Ez 3:12 Когда начали размножаться обитающие на земле, и умножились сыны и народы и поколения многие, и опять начали предаваться нечестию, более нежели прежние,
\vs 3Ez 3:13 когда начали делать пред Тобою беззаконие: Ты избрал Себе из них мужа, которому имя Авраам,
\vs 3Ez 3:14 и возлюбил его и открыл ему одному волю Твою,
\vs 3Ez 3:15 и положил ему завет вечный, и сказал ему, что никогда не оставишь семени его. И дал ему Исаака, и Исааку дал Иакова и Исава;
\vs 3Ez 3:16 Ты избрал Себе Иакова, Исава же отринул. И умножился Иаков чрезвычайно.
\vs 3Ez 3:17 Когда Ты вывел из Египта семя его и привел к горе Синайской,
\vs 3Ez 3:18 тогда преклонил небеса, уставил землю, поколебал вселенную, привел в трепет бездны и весь мир в смятение.
\vs 3Ez 3:19 И прошла слава Твоя в четырех \bibemph{явлениях}: в огне, землетрясении, бурном ветре и морозе, чтобы дать закон семени Иакова и радение роду Израиля,
\vs 3Ez 3:20 но не отнял у них сердца лукавого, чтобы закон Твой принес в них плод.
\vs 3Ez 3:21 С сердцем лукавым первый Адам преступил заповедь, и побежден был; так и все, от него происшедшие.
\vs 3Ez 3:22 Осталась немощь и закон в сердце народа с корнем зла, и отступило доброе, и осталось злое.
\vs 3Ez 3:23 Прошли времена и окончились лета,~--- и Ты воздвиг Себе раба, именем Давида;
\vs 3Ez 3:24 повелел ему построить город имени Твоему и в нем приносить Тебе фимиам и жертвы.
\vs 3Ez 3:25 Много лет это исполнялось, и потом согрешили населяющие город,
\vs 3Ez 3:26 во всем поступая так, как поступил Адам и все его потомки; ибо и у них было сердце лукавое.
\vs 3Ez 3:27 И Ты предал город Твой в руки врагов Твоих.
\vs 3Ez 3:28 Неужели лучше живут обитатели Вавилона и за это владеют Сионом?
\vs 3Ez 3:29 Когда я пришел сюда, видел нечестия, которым нет числа, и в этом тридцатом году пленения видит душа моя многих грешников,~--- и изныло сердце мое,
\vs 3Ez 3:30 ибо я видел, как Ты поддерживаешь сих грешников и щадишь нечестивцев, а народ Твой погубил, врагов же Твоих сохранил и не явил о том никакого знамения.
\vs 3Ez 3:31 Не понимаю, как этот путь мог измениться. Неужели Вавилон поступает лучше, нежели Сион?
\vs 3Ez 3:32 Или иной народ познал Тебя, кроме Израиля? или какие племена веровали заветам Твоим, как Иаков?
\vs 3Ez 3:33 Ни воздаяние им не равномерно, ни труд их не принес плода, ибо я прошел среди народов, и видел, что они живут в изобилии, хотя и не вспоминают о заповедях Твоих.
\vs 3Ez 3:34 Итак взвесь на весах и наши беззакония и дела живущих на земле, и нигде не найдется имя Твое, как только у Израиля.
\vs 3Ez 3:35 Когда не грешили пред Тобою живущие на земле? или какой народ так сохранил заповеди Твои?
\vs 3Ez 3:36 Между сими хотя по именам найдешь хранящих заповеди Твои, а у других народов не найдешь.
\vs 3Ez 4:1 Тогда отвечал мне посланный ко мне Ангел, которому имя Уриил,
\vs 3Ez 4:2 и сказал: сердце твое слишком далеко зашло в этом веке, что ты помышляешь постигнуть путь Всевышнего.
\vs 3Ez 4:3 Я отвечал: так, господин мой. Он же сказал мне: три пути послан я показать тебе и три подобия предложить тебе.
\vs 3Ez 4:4 Если ты одно из них объяснишь мне, то и я покажу тебе путь, который желаешь ты видеть, и научу тебя, откуда произошло сердце лукавое.
\vs 3Ez 4:5 Тогда я сказал: говори, господин мой. Он же сказал мне: иди и взвесь тяжесть огня, или измерь мне дуновение ветра, или возврати мне день, который уже прошел.
\vs 3Ez 4:6 Какой человек, отвечал я, может сделать то, чего ты требуешь от меня?
\vs 3Ez 4:7 А он сказал мне: если бы я спросил тебя, сколько обиталищ в сердце морском, или сколько источников в самом основании бездны, или сколько жил над твердью, или какие пределы у рая,
\vs 3Ez 4:8 ты, может быть, сказал бы мне: <<в бездну я не сходил, и в ад также, и на небо никогда не восходил>>.
\vs 3Ez 4:9 Теперь же я спросил тебя только об огне, ветре и дне, который ты пережил, и о том, без чего ты быть не можешь, и на это ты не отвечал мне.
\vs 3Ez 4:10 И сказал мне: ты и того, что твое и с тобою от юности, не можешь познать;
\vs 3Ez 4:11 как же сосуд твой мог бы вместить в себе путь Всевышнего и в этом уже заметно растленном веке понять растление, которое очевидно в глазах моих?
\vs 3Ez 4:12 На это сказал я: лучше было бы нам вовсе не быть, нежели жить в нечестиях и страдать, не зная, почему.
\vs 3Ez 4:13 Он же в ответ сказал мне: вот, я отправился в полевой лес, и застал дерева держащими совет.
\vs 3Ez 4:14 Они говорили: <<придите, и пойдем и объявим войну морю, чтобы оно отступило перед нами, и мы там возрастим для себя другие леса>>.
\vs 3Ez 4:15 Подобным образом и волны морские имели совещание: <<придите>>, говорили они, <<поднимемся и завоюем леса полевые, чтобы и там приобрести для себя другое место>>.
\vs 3Ez 4:16 Но замысел леса оказался тщетным, ибо пришел огонь и сжег его.
\vs 3Ez 4:17 Подобным образом кончился и замысел волн морских, ибо стал песок, и воспрепятствовал им.
\vs 3Ez 4:18 Если бы ты был судьею их, кого бы ты стал оправдывать или кого обвинять?
\vs 3Ez 4:19 Подлинно, отвечал я, замыслы их были суетны, ибо земля дана лесу, дано место и морю, чтобы носить свои волны.
\vs 3Ez 4:20 Он же в ответ сказал мне: справедливо рассудил ты; почему же ты не судил таким же образом себя самого?
\vs 3Ez 4:21 Ибо как земля дана лесу, а море волнам его, так обитающие на земле могут разуметь только то, что на земле; а обитающие на небесах могут разуметь, что на высоте небес.
\vs 3Ez 4:22 И отвечал я, и сказал: молю Тебя, Господи, да дастся мне смысл разумения.
\vs 3Ez 4:23 Не хотел я вопрошать Тебя о высшем, а о том, что ежедневно бывает у нас: почему Израиль предан на поругание язычникам? почему народ, который Ты возлюбил, отдан нечестивым племенам, и закон отцов наших доведен до ничтожества, и писанных постановлений нигде нет?
\vs 3Ez 4:24 Переходим из века сего, как саранча, жизнь наша проходит в страхе и ужасе, и мы сделались недостойными милосердия.
\vs 3Ez 4:25 Но что сделает Он с именем Своим, которое наречено на нас? вот о чем я вопрошал.
\vs 3Ez 4:26 Он же отвечал мне: чем больше будешь испытывать, тем больше будешь удивляться; потому что быстро спешит век сей к своему исходу,
\vs 3Ez 4:27 и не может вместить того, что обещано праведным в будущие времена, потому что век сей исполнен неправдою и немощами.
\vs 3Ez 4:28 А о том, о чем ты спрашивал меня, скажу тебе: посеяно зло, а еще не пришло время искоренения его.
\vs 3Ez 4:29 Посему, доколе посеянное не исторгнется, и место, на котором насеяно зло, не упразднится,~--- не придет место, на котором всеяно добро.
\vs 3Ez 4:30 Ибо зерно злого семени посеяно в сердце Адама изначала, и сколько нечестия народило оно доселе и будет рождать до тех пор, пока не настанет молотьба!
\vs 3Ez 4:31 Рассуди с собою, сколько зерно злого семени народило плодов нечестия!
\vs 3Ez 4:32 Когда будут пожаты бесчисленные колосья его, какое огромное понадобится для сего гумно!
\vs 3Ez 4:33 Как же и когда это будет? спросил я его; почему наши лета малы и несчастны?
\vs 3Ez 4:34 Не спеши подниматься, отвечал он, выше Всевышнего; ибо напрасно спешишь быть выше Его: слишком далеко заходишь.
\vs 3Ez 4:35 Не о том же ли вопрошали души праведных в затворах своих, говоря: <<доколе таким образом будем мы надеяться? И когда плод нашего возмездия?>>
\vs 3Ez 4:36 На это отвечал мне Иеремиил Архангел: <<когда исполнится число семян в вас, ибо Всевышний на весах взвесил век сей,
\vs 3Ez 4:37 и мерою измерил времена, и числом исчислил часы, и не подвинет и не ускорит до тех пор, доколе не исполнится определенная мера>>.
\vs 3Ez 4:38 Я же в ответ на это сказал ему: о, Владыко Господи! а мы все преисполнены нечестием.
\vs 3Ez 4:39 И, может быть, из-за нас не наполняются житницы праведных, и ради грехов живущих на земле.
\vs 3Ez 4:40 На это он отвечал мне: пойди, спроси беременную женщину, могут ли, по исполнении девятимесячного срока, ложесна ее удержать в себе плод?
\vs 3Ez 4:41 Я сказал: не могут. Тогда он сказал мне: подобны ложеснам и обиталища душ в преисподней.
\vs 3Ez 4:42 Как рождающая спешит родить, чтобы освободиться от болезней рождения, так и эти спешат отдать вверенное им.
\vs 3Ez 4:43 Сначала будет показано тебе то, что ты желаешь видеть.
\vs 3Ez 4:44 Если я обрел благодать пред очами твоими, отвечал я, и если это возможно и я способен к тому,
\vs 3Ez 4:45 покажи мне: имеющее прийти более ли того, что прошло, или сбывшееся более того, что будет?
\vs 3Ez 4:46 Что прошло, я это знаю, а что придет, не ведаю.
\vs 3Ez 4:47 Он сказал мне: стань на правую сторону, и я объясню тебе значение подобием.
\vs 3Ez 4:48 И я стал, и увидел: вот горящая печь проходит передо мною; и когда пламя прошло, я увидел: остался дым.
\vs 3Ez 4:49 После сего прошло предо мною облако, наполненное водою, и пролился из него сильный дождь; но как скоро стремительность дождя остановилась, остались капли.
\vs 3Ez 4:50 Тогда он сказал мне: размышляй себе: как дождь более капель, а огонь больше дыма, так мера прошедшего превысила, а остались капли и дым.
\vs 3Ez 4:51 Тогда я умолял его и сказал: думаешь ли ты, что я доживу до этих дней? и что будет в эти дни?
\vs 3Ez 4:52 На это отвечал он, и сказал: о знамениях, о которых ты спрашиваешь меня, я отчасти могу сказать тебе, а о жизни твоей я не послан говорить с тобою, да и не знаю.
\vs 3Ez 5:1 О знамениях: вот, настанут дни, в которые многие из живущих на земле, обладающие в\acc{е}дением, будут вос\-х\acc{и}\-ще\-ны, и путь истины сокроется, и вселенная оскудеет верою,
\vs 3Ez 5:2 и умножится неправда, которую теперь ты видишь и о которой издавна слышал.
\vs 3Ez 5:3 И будет, что страна, которую ты теперь видишь господствующею, подвергнется опустошению.
\vs 3Ez 5:4 А если Всевышний даст тебе дожить, то увидишь, что после третьей трубы внезапно воссияет среди ночи солнце и луна трижды в день;
\vs 3Ez 5:5 и с дерева будет капать кровь, камень даст голос свой, и народы поколеблются.
\vs 3Ez 5:6 Тогда будет царствовать тот, которого живущие на земле не ожидают, и птицы перелетят на другие места.
\vs 3Ez 5:7 Море Содомское извергнет рыб, будет издавать ночью голос, неведомый для многих; однако же все услышат голос его.
\vs 3Ez 5:8 Будет смятение во многих местах, часто будет посылаем с неба огонь; дикие звери переменят места свои, и нечистые женщины будут рождать чудовищ.
\vs 3Ez 5:9 Сладкие воды сделаются солеными, и все друзья ополчатся друг против друга; тогда сокроется ум, и разум удалится в свое хранилище.
\vs 3Ez 5:10 Многие будут искать его, но не найдут, и умножится на земле неправда и невоздержание.
\vs 3Ez 5:11 Одна область будет спрашивать другую соседнюю: <<не проходила ли по тебе правда, делающая праведным?>> И та скажет: <<нет>>.
\vs 3Ez 5:12 Люди в то время будут надеяться, и не достигнут желаемого, будут трудиться, и не управятся пути их.
\vs 3Ez 5:13 Об этих знамениях мне дозволено сказать тебе, и если снова помолишься и поплачешь, как теперь, и попостишься семь дней, то услышишь еще больше того.
\vs 3Ez 5:14 И я пришел в себя, и тело мое сильно дрожало, и душа моя изнемогла, как будто исчезала.
\vs 3Ez 5:15 Но пришедший ко мне Ангел поддержал меня и укрепил меня, и поставил на ноги.
\vs 3Ez 5:16 И было, во вторую ночь пришел ко мне Салафиил, вождь народа, и спросил меня: где ты был, и отчего лице твое так печально?
\vs 3Ez 5:17 Разве не знаешь, что тебе вверен Израиль в стране преселения его?
\vs 3Ez 5:18 Итак встань и вкуси хлеба, и не оставляй нас, как пастырь своего стада, в руках лукавых волков.
\vs 3Ez 5:19 Тогда сказал я ему: отойди от меня, и не приближайся ко мне. И он, услышав это, удалился от меня.
\vs 3Ez 5:20 А я семь дней постился, стеная и плача, как повелел мне Ангел Уриил.
\vs 3Ez 5:21 И после семи дней помышления сердца моего опять были для меня крайне тягостны;
\vs 3Ez 5:22 но душа моя прияла дух разумения, и я снова начал говорить пред Всевышним
\vs 3Ez 5:23 и сказал: о, Владыко Господи! Ты из всех лесов на земле и из всех дерев на ней избрал только одну виноградную лозу;
\vs 3Ez 5:24 Ты из всего круга земного избрал Себе одну пещеру, и из всех цветов во вселенной Ты избрал Себе одну лилию;
\vs 3Ez 5:25 Ты из всех пучин морских наполнил для Себя один источник, а из всех построенных городов освятил для Себя один Сион.
\vs 3Ez 5:26 Из всех сотворенных птиц Ты наименовал Себе одну голубицу, и из всех сотворенных скотов Ты избрал Себе одну овцу;
\vs 3Ez 5:27 из всех многочисленных народов Ты приобрел Себе один народ, и возлюбил его, дал ему закон совершенный.
\vs 3Ez 5:28 Но ныне, Господи, отчего же Ты предал одного многим, и на одном корне Ты насадил другие отрасли и рассеял Твой единственный народ между многими народами?
\vs 3Ez 5:29 И попрали его противники обетованиям Твоим и заветам Твоим не веровавшие.
\vs 3Ez 5:30 И если уже Ты сильно возненавидел народ Твой, то пусть бы он Твоими руками наказывался.
\rsbpar\vs 3Ez 5:31 Когда я произносил слова сии, послан был ко мне Ангел, который приходил ко мне прежде ночью,
\vs 3Ez 5:32 и сказал мне: послушай меня, и я научу тебя; внимай мне, и я скажу тебе еще более.
\vs 3Ez 5:33 Говори, сказал я, господин мой. И он сказал мне: ты слишком далеко зашел пытливостью ума твоего об Израиле; неужели ты больше любишь его, нежели Тот, Который сотворил его?
\vs 3Ez 5:34 Нет, господин мой, отвечал я, но говорил от великой скорби. Внутренность моя мучает меня всякий час, когда я стараюсь постигнуть путь Всевышнего и исследовать хотя часть суда Его.
\vs 3Ez 5:35 Он отвечал: не можешь. Почему же, господин мой? спросил я. Лучше бы я не родился, и утроба матерняя сделалась для меня гробом, нежели видеть угнетение Иакова и изнурение рода Израильского.
\vs 3Ez 5:36 И он сказал мне: исчисли мне, что еще не пришло, и собери мне рассеянные капли, и оживи иссохшие цветы;
\vs 3Ez 5:37 открой заключенные хранилища и выведи мне заключенные в них ветры, и покажи мне образ голоса: и тогда я покажу тебе то, что ты усиливаешься видеть.
\vs 3Ez 5:38 Владыко Господи! отвечал я, кто может знать это, разве только тот, кто не живет с человеками?
\vs 3Ez 5:39 А я безумен, и как могу говорить о том, о чем Ты спросил меня?
\vs 3Ez 5:40 Тогда Он сказал мне: как ты не можешь сделать ничего из сказанного, так не можешь познать судеб Моих, ни предела любви, которую обещал Я народу.
\vs 3Ez 5:41 Но вот, Господи, Ты близок к тем, которые к концу близятся, и что будут делать те, которые прежде меня были, или мы, или которые после нас будут?
\vs 3Ez 5:42 Он сказал мне: венцу уподоблю я суд Мой; как нет запоздания последних, так и ускорения первых.
\vs 3Ez 5:43 Отвечал я и сказал: не мог ли бы Ты соединить воедино как тех, которые сотворены были прежде, так и тех, которые существуют и которые будут, дабы скорее объявить им суд Твой?
\vs 3Ez 5:44 Он отвечал мне: не может ускорить творение Творца своего, ни век сей не может вместить в себе всех вместе, которые должны быть сотворены.
\vs 3Ez 5:45 И сказал я: как же Ты сказал рабу Твоему, что Ты дал жизнь созданному творению вкупе, и однако творение выдержало это; посему могли бы понести и ныне существующие вкупе.
\vs 3Ez 5:46 Он сказал мне: спроси женщину, и скажи ей: <<если ты рождаешь десять, то почему рождаешь по временам?>>, и проси ее, чтобы она родила десять вдруг.
\vs 3Ez 5:47 Я же сказал Ему: невозможно это, но должно быть по времени.
\vs 3Ez 5:48 Тогда Он сказал мне: и Я дал недрам земли способность посеянное на ней возращать по временам.
\vs 3Ez 5:49 Как младенец не может производить того, что свойственно старцам, так Я устроил созданный Мною век.
\vs 3Ez 5:50 Тогда я вопросил Его и сказал: когда Ты открыл мне путь, то позволь мне сказать Тебе: мать наша, о которой Ты говорил Мне, молода ли еще, или приближается к старости?
\vs 3Ez 5:51 Спроси об этом рождающую, и она скажет тебе.
\vs 3Ez 5:52 Скажи ей: <<почему рождаемые тобою ныне не подобны тем, которые рождены были прежде, но меньше их ростом?>>
\vs 3Ez 5:53 И она скажет тебе: <<одни рождены мною в крепости молодой силы, а другие рождены под старость, когда ложесна начали терять свою силу>>.
\vs 3Ez 5:54 Рассуди же ты: вы теперь меньше станом, нежели те, которые были прежде вас;
\vs 3Ez 5:55 и те, которые после вас родятся, будут еще меньше вас, так как творения, уже состаривающиеся, и крепость юноши уже миновала.
\vs 3Ez 5:56 И сказал я: если я приобрел благоволение пред очами Твоими, покажи рабу Твоему, через кого Ты посещаешь творение Твое?
\vs 3Ez 6:1 И сказал Он мне: от начала творения круга земного и прежде нежели установлены были пределы века, и прежде нежели подули ветры;
\vs 3Ez 6:2 прежде нежели услышаны были гласы громов, прежде нежели возблистали молнии, прежде нежели утвердились основания рая;
\vs 3Ez 6:3 прежде нежели показались прекрасные цветы, прежде нежели утвердились силы подвижные, и прежде нежели собрались бесчисленные воинства Ангелов;
\vs 3Ez 6:4 прежде нежели поднялись высоты воздушные, прежде нежели определились меры твердей, прежде нежели возгорелись огни на Сионе;
\vs 3Ez 6:5 прежде нежели исследованы были лета, и отделены те, которые грешат ныне, и запечатлены те, которые хранили веру, как сокровище:
\vs 3Ez 6:6 тогда Я помыслил, и сотворено было все Мною одним, а не чрез кого-либо иного; от Меня также последует и конец, а не от кого-либо иного.
\vs 3Ez 6:7 Тогда я отвечал: какое разделение времен, и когда будет конец первого и начало последнего?
\vs 3Ez 6:8 От Авраама даже до Исаака, когда родились от него Иаков и Исав, рука Иакова держала от начала пяту Исава.
\vs 3Ez 6:9 Конец сего века~--- Исав, а начало следующего~--- Иаков.
\vs 3Ez 6:10 Рука человека~--- начало его, а конец~--- пята его. О другом, Ездра, не спрашивай Меня.
\vs 3Ez 6:11 Я же в ответ сказал Ему: о, Владыко Господи! если я обрел благодать пред очами Твоими,
\vs 3Ez 6:12 молю Тебя, покажи рабу Твоему конец знамений Твоих, которых часть показал Ты мне в прошедшую ночь.
\vs 3Ez 6:13 Он отвечал мне и сказал: встань на ноги твои, и слушай голос, исполненный шума,
\vs 3Ez 6:14 и будет как бы землетрясение, но место, на котором ты стоишь, не поколеблется.
\vs 3Ez 6:15 Посему, когда будет говорить, ты не ужасайся; ибо о конце будет слово, и основания земли разумеются.
\vs 3Ez 6:16 А как речь идет о них самих, то земля вострепещет и поколеблется, ибо знает, что конец их должен измениться.
\rsbpar\vs 3Ez 6:17 И было, когда я услышал голос, встал на ноги мои, и слышал, и вот голос говорящий, и шум его, как шум вод многих,
\vs 3Ez 6:18 и сказал: вот, наступают дни, когда Я начну приближаться, чтобы посетить живущих на земле,
\vs 3Ez 6:19 когда начну Я взыскивать с тех, которые неправдою своею произвели неправедно великий вред, и когда исполнится мера уничижения Сиона.
\vs 3Ez 6:20 А когда назнаменается век, который начнет проходить, то вот знамения, которые Я покажу: книги раскроются пред лицем тверди, и все вместе увидят;
\vs 3Ez 6:21 и однолетние младенцы заговорят своими голосами, и беременные женщины будут рождать недозрелых младенцев через три и четыре месяца, и они останутся живыми и укрепятся;
\vs 3Ez 6:22 засеянные поля внезапно явятся как незасеянные, и полные житницы окажутся пустыми;
\vs 3Ez 6:23 затем вострубит труба с шумом, и когда услышат ее, все внезапно ужаснутся.
\vs 3Ez 6:24 И будет в то время, вооружатся друзья против друзей, как враги, и устрашится земля с живущими на ней, и жилы источников остановятся и три часа не будут течь.
\vs 3Ez 6:25 Всякий, кто после всего этого, о чем Я предсказал тебе, останется в живых, сам спасется, и увидит спасение Мое и конец вашего века.
\vs 3Ez 6:26 И увидят люди избранные, которые не испытали смерти от рождения своего, и изменится сердце живущих и обратится в чувство иное.
\vs 3Ez 6:27 Ибо зло истребится, и исчезнет лукавство;
\vs 3Ez 6:28 процветет вера, побеждено будет растление, явится истина, которая столько времени оставалась без плода.
\vs 3Ez 6:29 Когда Он говорил, я взглянул на того, пред которым стоял.
\vs 3Ez 6:30 И он сказал мне: я пришел показать тебе время грядущей ночи.
\vs 3Ez 6:31 Итак, если ты опять помолишься и опять семь дней попостишься, то я покажу тебе больше в день, в который я услышал тебя.
\vs 3Ez 6:32 Голос твой услышан у Всевышнего; увидел Крепкий правильное действие, увидел и чистоту, которую хранил ты от юности твоей.
\vs 3Ez 6:33 Посему Он послал меня показать тебе все это и сказать: уповай и не бойся;
\vs 3Ez 6:34 не спеши с первыми временами помышлять суетное, дабы не судить тебе с такою же поспешностью о временах последних.
\rsbpar\vs 3Ez 6:35 После сего я снова со слезами молился, и также постился семь дней, чтобы исполнить три седмицы, заповеданные мне.
\vs 3Ez 6:36 В восьмую же ночь сердце мое пришло снова в возбуждение, и я начал говорить пред Всевышним,
\vs 3Ez 6:37 ибо дух мой воспламенялся сильно, и душа моя томилась.
\vs 3Ez 6:38 И сказал я: Господи! Ты от начала творения говорил; в первый день сказал: <<да будет небо и земля>>, и слово Твое было совершившимся делом.
\vs 3Ez 6:39 Тогда носился Дух, и тьма облегала вокруг и молчание: звука человеческого голоса еще не было.
\vs 3Ez 6:40 Тогда повелел Ты из сокровищниц Твоих выйти обильному свету, чтобы явилось дело Твое.
\vs 3Ez 6:41 Во второй день сотворил Ты дух тверди и повелел ему отделить и произвести разделение между водами, чтобы некоторая часть их поднялась вверх, а прочая осталась внизу.
\vs 3Ez 6:42 В третий день Ты повелел водам собраться на седьмой части земли, а шесть частей осушил, чтобы они служили пред Тобою к обсеменению и обработанию.
\vs 3Ez 6:43 Слово Твое исходило, и тотчас являлось дело;
\vs 3Ez 6:44 вдруг явилось безмерное множество плодов и многоразличные приятности для вкуса, цветы в виде своем неизменные, с запахом, несказанно благоуханным: все это совершено было в третий день.
\vs 3Ez 6:45 В четвертый день Ты повелел быть сиянию солнца, свету луны, расположению звезд
\vs 3Ez 6:46 и повелел, чтобы они служили имеющему быть созданным человеку.
\vs 3Ez 6:47 В пятый день Ты сказал седьмой части, в которой была собрана вода, чтобы она произвела животных, летающих и рыб, что и сделалось.
\vs 3Ez 6:48 Вода немая и бездушная, по мановению Божию, произвела животных, чтобы все роды возвещали дивные дела Твои.
\vs 3Ez 6:49 Тогда Ты сохранил двух животных: одно называлось бегемотом, а другое левиафаном.
\vs 3Ez 6:50 И Ты отделил их друг от друга, потому что седьмая часть, где была собрана вода, не могла принять их вместе.
\vs 3Ez 6:51 Бегемоту Ты дал одну часть из земли, осушенной в третий день, да обитает в ней, в которой тысячи гор.
\vs 3Ez 6:52 Левиафану дал седьмую часть водяную, и сохранил его, чтобы он был пищею тем, кому Ты хочешь, и когда хочешь.
\vs 3Ez 6:53 В шестый же день повелел Ты земле произвести пред Тобою скотов, зверей и пресмыкающихся;
\vs 3Ez 6:54 а после них Ты сотворил Адама, которого поставил властелином над всеми Твоими тварями и от которого происходим все мы и народ, который Ты избрал.
\vs 3Ez 6:55 Все это сказал я пред Тобою, Господи, потому что для нас создал Ты век сей.
\vs 3Ez 6:56 О прочих же народах, происшедших от Адама, Ты сказал, что они ничто, но подобны слюне, и все множество их Ты уподобил каплям, каплющим из сосуда.
\vs 3Ez 6:57 И ныне, Господи, вот, эти народы, за ничто Тобою признанные, начали владычествовать над нами и пожирать нас.
\vs 3Ez 6:58 Мы же, народ Твой, который Ты назвал Твоим первенцем, единородным, возлюбленным Твоим, преданы в руки их.
\vs 3Ez 6:59 Если для нас создан век сей, то почему не получаем мы наследия с веком? И доколе это?
\vs 3Ez 7:1 Когда я окончил говорить эти слова, послан был ко мне Ангел, который посылаем был ко мне в прежние ночи,
\vs 3Ez 7:2 и сказал мне: встань, Ездра, и слушай слов\acc{а}, которые я пришел говорить тебе.
\vs 3Ez 7:3 Я сказал: говори, господин мой. И он сказал мне: море расположено в пространном месте, чтобы быть глубоким и безмерным;
\vs 3Ez 7:4 но вход в него находится в тесном месте, так что подобен рекам.
\vs 3Ez 7:5 Кто пожелал бы войти в море и видеть его, или господствовать над ним, тот, если не пройдет тесноты, как может дойти до широты?
\vs 3Ez 7:6 Или иное подобие: город построен и расположен на равнине, и наполнен всеми благами;
\vs 3Ez 7:7 но вход в него тесен и расположен на крутизне так, что по правую сторону огонь, а по левую глубокая вода.
\vs 3Ez 7:8 Между ними, то есть между огнем и водою, лежит лишь одна стезя, на которой может поместиться не более, как только ступень человека.
\vs 3Ez 7:9 Если город этот будет дан в наследство человеку, то как он получит свое наследство, если никогда не перейдет лежащей на пути опасности?
\vs 3Ez 7:10 Я сказал: так, Господи. И Он сказал мне: такова и доля Израиля.
\vs 3Ez 7:11 Для них Я сотворил век; но когда Адам нарушил Мои постановления, определено быть тому, что сделано.
\vs 3Ez 7:12 И сделались входы века сего тесными, болезненными, утомительными, также узкими, лукавыми, исполненными бедствий и требующими великого труда.
\vs 3Ez 7:13 А входы будущего века пространны, безопасны, и приносят плод бессмертия.
\vs 3Ez 7:14 Итак, если входящие, которые живут, не войдут в это тесное и бедственное, они не могут получить, что уготовано.
\vs 3Ez 7:15 Зачем же смущаешься, когда ты тленен, и что мятешься, когда смертен?
\vs 3Ez 7:16 Зачем не принял ты в сердце твоем того, что будущее, а принял то, что в настоящем?
\vs 3Ez 7:17 Я отвечал и сказал: Владыко Господи! вот, Ты определил законом Твоим, что праведники наследуют это, а грешники погибнут.
\vs 3Ez 7:18 Праведники потерпят тесноту, надеясь пространного, а нечестиво жившие, хотя потерпели тесноту, не увидят пространного.
\vs 3Ez 7:19 И Он сказал мне: нет судии выше Бога, нет разумеющего более Всевышнего.
\vs 3Ez 7:20 Погибают многие в этой жизни, потому что нерадят о предложенном им законе Божием.
\vs 3Ez 7:21 Ибо строго повелел Бог приходящим, когда они пришли, что делая, они будут живы, и что соблюдая, не будут наказаны.
\vs 3Ez 7:22 А они не послушались, и воспротивились Ему, утвердили в себе помышление суетное.
\vs 3Ez 7:23 Увлеклись греховными обольщениями, сказали о Всевышнем, что \bibemph{Его} нет, не познали путей Его,
\vs 3Ez 7:24 презрели закон Его, отвергли обетования Его, не имели веры к обрядовым установлениям Его, не совершали дел Его.
\vs 3Ez 7:25 И потому, Ездра, пустым пустое, а полным полное.
\vs 3Ez 7:26 Вот, придет время, когда придут знамения, которые Я предсказал тебе, и явится невеста, и являясь покажется,~--- скрываемая ныне землею.
\vs 3Ez 7:27 И всякий, кто избавится от прежде исчисленных зол, сам увидит чудеса Мои.
\vs 3Ez 7:28 Ибо откроется Сын Мой Иисус с теми, которые с Ним, и оставшиеся будут наслаждаться четыреста лет.
\vs 3Ez 7:29 А после этих лет умрет Сын Мой Христос и все люди, имеющие дыхание.
\vs 3Ez 7:30 И обратится век в древнее молчание на семь дней, подобно тому, как было прежде, так что не останется никого.
\vs 3Ez 7:31 После же семи дней восстанет век усыпленный, и умрет поврежденный.
\vs 3Ez 7:32 И отдаст земля тех, которые в ней спят, и прах тех, которые молчаливо в нем обитают, а хранилища отдадут вверенные им души.
\vs 3Ez 7:33 Тогда явится Всевышний на престоле суда, и пройдут беды, и окончится долготерпение.
\vs 3Ez 7:34 Суд будет один, истина утвердится, вера укрепится.
\vs 3Ez 7:35 Затем последует дело, откроется воздаяние, восстанет правда, перестанет господствовать неправда.
\vs 3Ez 7:(36) \fns{70 стихов, находящихся между 35 и 36 стихами 7-й главы, имеются в русском переводе в <<Толковой Библии>> А.П.Лопухина (Петербург, 1913) и в т.н. <<Брюссельской>> Библии (Брюссель, 1973). В Синодальной Библии их нет.}И откроется озеро мучения, а против него место покоя; видна будет печь геенны, а против нее рай сладости.
\vs 3Ez 7:(37) И скажет тогда Всевышний пробудившимся народам: <<посмотрите и поймите, Кого вы отвергли, Кому вы не служили и Чьи заповеди вы презрели.
\vs 3Ez 7:(38) Взгляните прямо пред собою и напротив: там сладость и покой, а тут огонь и мучения>>. Вот что скажешь Ты им в день суда.
\vs 3Ez 7:(39) Этот день таков, что не имеет ни солнца, ни луны, ни звезд,
\vs 3Ez 7:(40) ни облака, ни грома, ни молнии, ни ветра, ни дождя, ни тумана, ни мрака, ни вечера, ни утра,
\vs 3Ez 7:(41) ни лета, ни весны, ни жары, ни зимы, ни мороза, ни холода, ни града, ни дождя, ни росы,
\vs 3Ez 7:(42) ни полдня, ни ночи, ни предрассветных сумерек, ни блеска, ни ясности, ни света, кроме одного лишь сияния светлости Всевышнего, вследствие чего все могут видеть то, что пред ними.
\vs 3Ez 7:(43) Его длительность будет такая же, как седьмины лет.
\vs 3Ez 7:(44) Таков суд Мой и его порядок. Одному тебе Я открыл это.
\vs 3Ez 7:(45) И я отвечал: <<я говорил уже, и теперь скажу: блаженны живущие и исполняющие заповеданное Тобою.
\vs 3Ez 7:(46) Но я молил о следующем: найдется ли кто из живущих, чтобы не грешил, или найдется ли кто из родившихся, чтобы не нарушал Твоего завета?
\vs 3Ez 7:(47) И теперь я вижу, что будущий век принесет сладость немногим, а мучения многим.
\vs 3Ez 7:(48) Ибо внутри нас выросло сердце злое, которое удалило нас от Него и привело нас к тлению и путям смерти, показало нам тропинки погибели и удалило нас от жизни, притом не малое количество, но почти всех, кто был сотворен>>.
\vs 3Ez 7:(49) И Он отвечал мне и сказал: выслушай Меня, и Я наставлю тебя и вразумлю тебя относительно имеющего быть.
\vs 3Ez 7:(50) В виду этого Бог и сотворил не один век, а два.
\vs 3Ez 7:(51) Что же касается твоих слов, что праведных не много, но мало, тогда как нечестивых множество, то выслушай на это вот что:
\vs 3Ez 7:(52) <<если у тебя будет весьма немного драгоценных камней, то ты станешь складывать их у себя по числу их; свинца же и глины изобилие>>.
\vs 3Ez 7:(53) И я сказал: <<как же это возможно?>>
\vs 3Ez 7:(54) И Он сказал мне: <<не только это, но спроси землю, и та скажет тебе, подойди к ней с лестью, и та поведает тебе.
\vs 3Ez 7:(55) Ты скажешь ей: ты производишь золото, серебро и медь, а также железо, свинец и глину.
\vs 3Ez 7:(56) Серебра же больше, чем золота, меди больше, чем серебра, железа больше, чем меди, свинца больше, чем железа, и глины больше, чем свинца.
\vs 3Ez 7:(57) Посуди теперь сам, что драгоценно и влечет к себе, то ли, чего много, или то, что является редкостью>>.
\vs 3Ez 7:(58) И я сказал: <<Владыка Господи! Что встречается в избытке, то хуже, а что попадается реже, то драгоценнее>>.
\vs 3Ez 7:(59) И Он отвечал мне и сказал: <<взвесь про себя то, что ты подумал: кто владеет тем, что с трудом добывается, бывает рад больше того, кто обладает тем, что встречается в избытке.
\vs 3Ez 7:(60) Так обстоит дело и с обещанною Мною тварью. Я рад буду немногим спасшимся, потому что они утвердили ныне владычество Моей славы и на них наречено ныне же Мое имя.
\vs 3Ez 7:(61) Меня не будет огорчать множество погибших: ведь это те самые, которые теперь уже уподоблены пару и приравнены к огню и дыму. Вот они вспыхнули, запылали и погасли>>.
\vs 3Ez 7:(62) И я отвечал и сказал: <<о, земля! что же ты породила, если разум произошел из праха, как и остальная тварь?
\vs 3Ez 7:(63) Лучше было бы не появляться самому праху, чтобы из него не возник разум.
\vs 3Ez 7:(64) А теперь, разум возрастает вместе с нами, и из-за этого мы мучимся, так как сознательно идем к гибели.
\vs 3Ez 7:(65) Пусть рыдает род человеческий, и радуются полевые звери; пусть рыдают все, кто родился, и веселятся четвероногие и скоты.
\vs 3Ez 7:(66) Ибо им гораздо лучше, чем нам, так как они не ждут суда; им неведомы ни мучения, ни блаженство, обещанные им после смерти.
\vs 3Ez 7:(67) Что нам пользы в том, что мы будем снова жить, но будем жестоко мучиться?
\vs 3Ez 7:(68) Ведь все, кто родился, пропитаны беззакониями, полны грехов и отягчены преступлениями.
\vs 3Ez 7:(69) И быть может, лучше было бы нам, если бы нам не нужно было идти на суд>>.
\vs 3Ez 7:(70) И Он отвечал мне и сказал: <<раньше, чем Всевышний сотворил век с Адамом и всеми, происшедшими от него, Он приготовил суд и то, что относится к суду.
\vs 3Ez 7:(71) Теперь же уразумей на основании своих собственных слов; ведь ты сказал, что разум возрастает с нами.
\vs 3Ez 7:(72) Поэтому те, кто живет на земле, терпят здесь мучения, потому что, имея разум, они совершали беззакония и, получая заповеди, не исполняли их, и, будучи последователями закона, отвергали закон, полученный ими.
\vs 3Ez 7:(73) Что же имеют они сказать на суде или какой ответ дадут они в ближайшее время?
\vs 3Ez 7:(74) В самом деле, сколько времени Всевышний проявлял долготерпение к тем, кто населяет век, и не ради их самих, а ради исполнения предусмотренного Им срока>>.
\vs 3Ez 7:(75) И я отвечал и сказал: <<если я нашел благодать пред Тобою, Господи, то покажи рабу Твоему еще следующее. Будем ли мы после смерти, то есть когда каждый из нас отдаст душу свою, пребывать в покое, пока не наступят те времена, когда Ты начнешь обновлять тварь, или же тотчас будем терпеть мучения?>>
\vs 3Ez 7:(76) И Он отвечал мне и сказал: <<покажу тебе и это. Но ты не смешивай себя с теми, кто презирал, и не причисляй себя к тем, которые терпят мучения,
\vs 3Ez 7:(77) ибо у тебя есть сокровище дел, сохраняемое у Всевышнего; но оно не будет пока дано тебе до наступления последнего времени.
\vs 3Ez 7:(78) Теперь будет речь о смерти, когда выйдет от Всевышнего приговор относительно срока, чтобы умереть человеку, и когда дух выйдет из тела, чтобы снова вернуться к Тому, Кто дал его, для поклонения прежде всего славе Всевышнего.
\vs 3Ez 7:(79) И если это будут души тех, кто презирал и не сохранял путей Всевышнего, пренебрегал Его законом и ненавидел боящихся Бога,
\vs 3Ez 7:(80) то таковые души не войдут в обители, но немедленно начнут в мучениях, в постоянной скорби и печали блуждать по семи путям.
\vs 3Ez 7:(81) Первый путь это то, что они презрели закон Всевышнего.
\vs 3Ez 7:(82) Второй путь: они уже не могут принести доброе раскаяние, чтобы жить.
\vs 3Ez 7:(83) Третий путь: они увидят награду, сохраняемую для тех, кто верен заветам Всевышнего.
\vs 3Ez 7:(84) Четвертый путь: они увидят мучения, сохраняемые для них на самое последнее время.
\vs 3Ez 7:(85) Пятый путь: они видят жилища других, охраняемые в глубочайшем молчании ангелами.
\vs 3Ez 7:(86) Шестой путь: они видят, что немедленно же отсюда они перейдут на мучения.
\vs 3Ez 7:(87) Седьмой путь, превосходящий все названные выше пути, состоит в том, что они тают от смятения, их снедает стыд, они изнемогают от страха, при виде славы Всевышнего, пред которой они грешили при жизни и пред которой им предстоит суд в последние времена.
\vs 3Ez 7:(88) Что же касается тех, кто сохранял пути Всевышнего, то удел их по разлучению с тленным сосудом будет следующий:
\vs 3Ez 7:(89) во время пребывания в нем они с трудностями служили Всевышнему и каждый час подвергались опасностям, лишь бы всецело сохранить закон Законодателя.
\vs 3Ez 7:(90) Поэтому приговор о них будет такой:
\vs 3Ez 7:(91) прежде всего они увидят с великою радостью славу Того, Кто принимает их к Себе; покой же они будут вкушать семи видов.
\vs 3Ez 7:(92) Первый вид это то, что они с великим трудом вели борьбу, с целью преодолеть помышление злое, созданное вместе с ними, чтобы оно не могло отвлекать их от жизни к смерти.
\vs 3Ez 7:(93) Второй вид: они созерцают смятение, в каком блуждают души нечестивых, и наказание, предстоящее им.
\vs 3Ez 7:(94) Третий вид: они созерцают данное им их Создателем свидетельство, что они при жизни сохранили закон, вверенный им.
\vs 3Ez 7:(95) Четвертый вид: они сознают свой покой, которым они наслаждаются ныне, собравшись в своих хранилищах и оберегаемые в глубоком молчании ангелами, и прославление, ожидающее их в последние времена.
\vs 3Ez 7:(96) Пятый вид: они ликуют по поводу того, что покинули ныне тленное и получат будущее наследие; они видят кроме того ту тесноту, полную тягостей, от которой они освободились, и начинают чувствовать простор, блаженные и бессмертные.
\vs 3Ez 7:(97) Шестой вид: им показано будет, как лицо их засияет подобно солнцу и они уподобятся по блеску звездам, став тотчас же нетленными.
\vs 3Ez 7:(98) Седьмой вид, превосходящий все ранее названные: они будут ликовать с уверенностью, надеяться без посрамления и радоваться без страха, так как они спешат увидеть лицо Того, Кому они служили при жизни, и от Кого они должны получить награду, состоящую в прославлении.
\vs 3Ez 7:(99) Таков удел душ праведников, возвещаемый им тотчас же. Ранее были названы пути тех мучений, которые терпят немедленно же грешники>>.
\vs 3Ez 7:(100) И я отвечал и сказал: <<значит, душам по разлучении их с телом будет дано время, чтобы видеть то, о чем Ты мне сказал>>.
\vs 3Ez 7:(101) И Он сказал мне: <<семь дней будет длиться их свобода, чтобы они за семь дней увидели то, о чем была выше речь, а после этого они соберутся в свои жилища>>.
\vs 3Ez 7:(102) И я отвечал и сказал: <<если я нашел милость пред очами Твоими, то покажи мне, рабу Твоему, кроме того, могут ли в день суда праведники достигнуть оправдания нечестивых или молить за них Всевышнего,
\vs 3Ez 7:(103) отцы за сыновей, сыновья за родителей, братья за братьев, родственники за своих близких, или друзья за дорогих для них лиц>>.
\vs 3Ez 7:(104) Он отвечал мне и сказал: <<так как ты нашел милость пред очами Моими, то Я покажу тебе и это. День суда решительный и являет всем печать истины. Подобно тому, как ныне отец не посылает сына или сын отца, или господин раба, или друг самого дорогого для него человека с тем, чтобы тот думал за него, или спал, или ел, или лечился,
\vs 3Ez 7:(105) так никогда никто не будет за кого-либо ходатайствовать, но каждый принесет тогда свои правды или неправды>>.
\vs 3Ez 7:36 Я сказал: Авраам первый молился о Содомлянах; Моисей~--- за отцов, согрешивших в пустыне;
\vs 3Ez 7:37 Иисус после него~--- за Израиля во дни Ахана;
\vs 3Ez 7:38 Самуил и Давид~--- за погубляемых, Соломон~--- за тех, которые пришли на освящение;
\vs 3Ez 7:39 Илия~--- за тех, которые приняли дождь, и за мертвеца, чтобы он ожил;
\vs 3Ez 7:40 Езекия~--- за народ во дни Сеннахирима, и многие~--- за многих.
\vs 3Ez 7:41 Итак, если тогда, когда усилилось растление и умножилась неправда, праведные молились за неправедных, то почему же не быть тому и ныне?
\vs 3Ez 7:42 Он отвечал мне и сказал: настоящий век не есть конец; славы в нем часто не бывает, потому молились за немощных.
\vs 3Ez 7:43 День же суда будет концом времени сего и началом времени будущего бессмертия, когда пройдет тление,
\vs 3Ez 7:44 прекратится невоздержание, пресечется неверие, а возрастет правда, воссияет истина.
\vs 3Ez 7:45 Тогда никто не возможет спасти погибшего, ни погубить победившего.
\vs 3Ez 7:46 Я отвечал и сказал: вот мое слово первое и последнее: лучше было не давать земли Адаму, или, когда уже дана, удержать его, чтобы не согрешил.
\vs 3Ez 7:47 Что пользы людям~--- в настоящем веке жить в печали, а по смерти ожидать наказания?
\vs 3Ez 7:48 О, что сделал ты, Адам? Когда ты согрешил, то совершилось падение не тебя только одного, но и нас, которые от тебя происходим.
\vs 3Ez 7:49 Что пользы нам, если нам обещано бессмертное время, а мы делали смертные дела?
\vs 3Ez 7:50 Нам предсказана вечная надежда, а мы, непотребные, сделались суетными.
\vs 3Ez 7:51 Нам уготованы жилища здоровья и покоя, а мы жили худо;
\vs 3Ez 7:52 уготована слава Всевышнего, чтобы покрыть тех, которые жили кротко, а мы ходили по путям злым.
\vs 3Ez 7:53 Показан будет рай, плод которого пребывает нетленным и в котором покой и врачевство;
\vs 3Ez 7:54 но мы не войдем \bibemph{в него}, потому что обращались в местах неплодных.
\vs 3Ez 7:55 Светлее звезд воссияют лица тех, которые имели воздержание, а наши лица~--- чернее тьмы.
\vs 3Ez 7:56 Мы не помышляли в жизни, когда делали беззаконие, что по смерти будем страдать.
\vs 3Ez 7:57 Он отвечал и сказал: это~--- помышление о борьбе, которую должен вести на земле родившийся человек,
\vs 3Ez 7:58 чтобы, если будет побежден, потерпеть то, о чем ты сказал, а если победит, получить то, о чем Я говорю.
\vs 3Ez 7:59 Это та жизнь, о которой сказал Моисей, когда жил, к народу, говоря: <<избери себе жизнь, чтобы жить>>.
\vs 3Ez 7:60 Но они не поверили ему, ни пророкам после него, ни Мне, говорившему к ним,
\vs 3Ez 7:61 что не будет скорби о погибели их, как будет радость о тех, которым уготовано спасение.
\vs 3Ez 7:62 Я отвечал и сказал: знаю, Господи, что Всевышний называется милосердым, потому что помилует тех, которые еще не пришли в мир,
\vs 3Ez 7:63 и милует тех, которые провождают жизнь в законе Его.
\vs 3Ez 7:64 Он долготерпелив, ибо оказывает долготерпение к согрешившим, как к Своему творению.
\vs 3Ez 7:65 Он щедр, ибо готов давать по надобности,
\vs 3Ez 7:66 и многомилостив, ибо умножает милости Свои к живущим ныне и к жившим и к тем, которые будут жить.
\vs 3Ez 7:67 Ибо, если бы не умножал Он Своих милостей, то не мог бы век продолжать жить с теми, которые обитают в нем.
\vs 3Ez 7:68 Он подает дары; ибо если бы не даровал по благости Своей, да облегчатся совершившие нечестие от своих беззаконий, то не могла бы оставаться в живых десятитысячная часть людей.
\vs 3Ez 7:69 Он судия, и если бы не прощал тех, которые сотворены словом Его, и не истребил множества преступлений,
\vs 3Ez 7:70 может быть, из бесчисленного множества остались бы только весьма немногие.
\vs 3Ez 8:1 Он отвечал мне и сказал: этот век Всевышний сотворил для многих, а будущий для немногих.
\vs 3Ez 8:2 Скажу тебе, Ездра, подобие. Как если спросишь землю, она скажет тебе, что дает очень много вещества, из которого делаются глиняные вещи, а не много праха, из которого бывает золото, так и дела настоящего века.
\vs 3Ez 8:3 Многие сотворены, но немногие спасутся.
\vs 3Ez 8:4 Я отвечал и сказал: душа! пожри смысл и поглоти мудрость.
\vs 3Ez 8:5 Ибо ты обещала слушать, и пожелала пророчествовать, а тебе дано время только, чтобы жить.
\vs 3Ez 8:6 О, Господи! неужели Ты не позволишь рабу Твоему, чтобы мы молились пред Тобою о даровании сердцу нашему семени и разуму возделания, чтобы произошел плод, которым мог бы жить всякий растленный, кто будет носить имя человека?
\vs 3Ez 8:7 Ты един, и мы единое творение рук Твоих, как сказал Ты.
\vs 3Ez 8:8 И как же ныне во чреве матернем образуется тело, и Ты даешь члены, как сохраняется Твое творение в огне и воде, и как девять месяцев терпит в себе Твое же создание Твою тварь, которая в нем сотворена?
\vs 3Ez 8:9 И хранящее и хранимое, и то и другое сохраняются, и чрево матери в свое время отдает то сохраненное, что в нем произросло.
\vs 3Ez 8:10 Ты повелел из самих членов, то есть из сосцов, давать молоко, плод сосцов,
\vs 3Ez 8:11 да питается созданное до некоторого времени, а после передашь его Твоему милосердию.
\vs 3Ez 8:12 Ты воспитал его Твоею правдою, научил его Твоему закону, наставил его Твоим разумом,
\vs 3Ez 8:13 и умертвишь его, как Твое творение, и опять оживишь, как Твое дело.
\vs 3Ez 8:14 Если Ты погубишь созданного с таким попечением, то повелению Твоему легко устроить, чтобы и сохранялось то, что было создано.
\vs 3Ez 8:15 И ныне, Господи, я скажу: о всяком человеке Ты больше знаешь; но \bibemph{скажу} о народе Твоем, о котором болезную,
\vs 3Ez 8:16 о наследии Твоем, о котором проливаю слезы, об Израиле, о котором скорблю, об Иакове, о котором сокрушаюсь.
\vs 3Ez 8:17 Начну молиться пред Тобою за себя и за них, ибо вижу грехопадения нас, обитающих на земле.
\vs 3Ez 8:18 Но я слышал, что скоро придет Судия.
\vs 3Ez 8:19 Посему услышь мой голос, вонми словам моим, и я буду говорить пред Тобою. [Начало слов Ездры, прежде нежели он был взят.]
\vs 3Ez 8:20 Я сказал: Господи, живущий вечно, Которого очи обращены на выспреннее и небесное,
\vs 3Ez 8:21 Которого престол неоценим и слава непостижима, Которому с трепетом предстоят воинства Ангелов, служащих в ветре и огне, Которого слово истинно и глаголы непреложны,
\vs 3Ez 8:22 повеление сильно и правление страшно, Которого взор иссушает бездны, гнев расплавляет горы и истина пребывает во веки!
\vs 3Ez 8:23 Услышь молитву раба Твоего, и вонми молению создания Твоего.
\vs 3Ez 8:24 Доколе живу, буду говорить, и доколе разумею, буду отвечать. Не взирай на грехи народа Твоего, но на тех, которые Тебе в истине служат;
\vs 3Ez 8:25 не обращай внимания на нечестивые дела язычников, но на тех, которые заветы Твои сохранили среди бедствий;
\vs 3Ez 8:26 не помышляй о тех, которые пред Тобою лживо поступали, но помяни тех, которые, по воле Твоей, познали страх;
\vs 3Ez 8:27 не погубляй тех, которые жили по-скотски, но воззри на тех, которые ясно учили закону Твоему;
\vs 3Ez 8:28 не прогневайся на тех, которые признаны худшими зверей;
\vs 3Ez 8:29 но возлюби тех, которые всегда надеются на правду Твою и славу.
\vs 3Ez 8:30 Ибо мы и отцы наши такими болезнями страдаем;
\vs 3Ez 8:31 а Ты, ради нас~--- грешных, назовешься милосердым.
\vs 3Ez 8:32 Если Ты пожелаешь помиловать нас, то назовешься милосердым, потому что мы не имеем дел правды.
\vs 3Ez 8:33 Праведники же, у которых много дел приобретено, по собственным делам получат воздаяние.
\vs 3Ez 8:34 Что есть человек, чтобы Ты гневался на него, и род растленный, чтобы Ты столько огорчался им?
\vs 3Ez 8:35 Поистине, нет никого из рожденных, кто не поступил бы нечестиво, и из исповедающих \bibemph{Тебя} нет никого, кто не согрешил бы.
\vs 3Ez 8:36 В том-то и возвестится правда Твоя и благость Твоя, Господи, когда помилуешь тех, которые не имеют существа добрых дел.
\vs 3Ez 8:37 Он отвечал мне и сказал: справедливо ты сказал нечто, и по словам твоим так и будет.
\vs 3Ez 8:38 Ибо истинно не помышляю Я о делах тех созданий, которые согрешили, прежде смерти, прежде суда, прежде погибели;
\vs 3Ez 8:39 но услаждаюсь подвигами праведных, и воспоминаю, как они странствовали, как спасались и старались заслужить награду.
\vs 3Ez 8:40 Как сказал Я, так и есть.
\vs 3Ez 8:41 Как земледелец сеет на земле многие семена и садит многие растения, но не все посеянное сохранится со временем, и не все посаженное укоренится, так и те, которые посеяны в веке \bibemph{сем}, не все спасутся.
\vs 3Ez 8:42 Я отвечал и сказал: если я обрел благодать, то буду говорить.
\vs 3Ez 8:43 Как семя земледельца, если не взойдет, или не примет вовремя дождя Твоего, или повредится от множества дождя, погибает:
\vs 3Ez 8:44 так и человек, созданный руками Твоими,~--- и Ты называешься его первообразом, потому что Ты подобен ему, для которого создал все и которого Ты уподобил семени земледельца.
\vs 3Ez 8:45 Не гневайся на нас, но пощади народ Твой и помилуй наследие Твое,~--- а Ты милосерд к созданию Твоему.
\vs 3Ez 8:46 Он отвечал мне и сказал: настоящее настоящим и будущее будущим.
\vs 3Ez 8:47 Многого недостает тебе, чтобы ты мог возлюбить создание Мое более Меня, хотя Я часто приближался к тебе самому, а к неправедным никогда.
\vs 3Ez 8:48 Но и в том дивен ты пред Всевышним,
\vs 3Ez 8:49 что смирил себя, как прилично тебе, и не судил о себе так, чтобы много славиться между праведными.
\vs 3Ez 8:50 Многие и горестные бедствия постигнут тех, которые населяют век, в последнее время, потому что они ходили в великой гордыне.
\vs 3Ez 8:51 А ты заботься о себе, и подобным тебе ищи славы;
\vs 3Ez 8:52 ибо вам открыт рай, насаждено древо жизни, предназначено будущее время, готово изобилие, построен город, приготовлен покой, совершенная благость и совершенная премудрость.
\vs 3Ez 8:53 Корень зла запечатан от вас, немощь и тля сокрыты от вас, и растление бежит в ад в забвение.
\vs 3Ez 8:54 Прошли болезни, и в конце показалось сокровище бессмертия.
\vs 3Ez 8:55 Не старайся более испытывать о множестве погибающих.
\vs 3Ez 8:56 Ибо они, получив свободу, презрели Всевышнего, пренебрегли закон Его и оставили пути Его,
\vs 3Ez 8:57 а еще и праведных Его попрали,
\vs 3Ez 8:58 и говорили в сердце своем: <<нет Бога>>, хотя и знали, что они смертны.
\vs 3Ez 8:59 Как вас ожидает то, о чем сказано прежде, так и их~--- жажда и мучение, которые приготовлены. Бог не хотел погубить человека,
\vs 3Ez 8:60 но сами сотворенные обесславили имя Того, Кто сотворил их, и были неблагодарными к Тому, Кто предуготовил им жизнь.
\vs 3Ez 8:61 Посему суд Мой ныне приближается,~---
\vs 3Ez 8:62 о чем Я не всем открыл, а только тебе и немногим, тебе подобным. Я отвечал и сказал:
\vs 3Ez 8:63 вот ныне, Господи, Ты показал мне множество знамений, которые Ты начнешь творить при кончине, но не показал, в какое время.
\vs 3Ez 9:1 Он отвечал мне и сказал: измеряя измеряй время в себе самом, и когда увидишь, что прошла некоторая часть знамений, прежде указанных,
\vs 3Ez 9:2 тогда уразумеешь, что это и есть то время, в которое начнет Всевышний посещать век, Им созданный.
\vs 3Ez 9:3 Когда обнаружится в веке колебание мест, смятение народов,
\vs 3Ez 9:4 тогда уразумеешь, что об этом говорил Всевышний от дней, бывших прежде тебя, от начала.
\vs 3Ez 9:5 Как все, сотворенное в веке, имеет начало, равно и конец, и окончание бывает явно:
\vs 3Ez 9:6 так и времена Всевышнего имеют начала, открывающиеся чудесами и силами, и окончания, являемые действиями и знамениями.
\vs 3Ez 9:7 Всякий, кто спасется и возможет делами своими и верою, которою веруете, избежать от преждесказанных бед,
\vs 3Ez 9:8 останется, и увидит спасение Мое на земле Моей и в пределах Моих, которые Я освятил Себе от века.
\vs 3Ez 9:9 Тогда пожалеют отступившие ныне от путей Моих, и отвергшие их с презрением пребудут в муках.
\vs 3Ez 9:10 Те, которые не познали Меня, получая при жизни благодеяния,
\vs 3Ez 9:11 и возгнушались законом Моим, не уразумели его, но презрели, когда еще имели свободу и когда еще отверсто было им место для покаяния,
\vs 3Ez 9:12 те познают Меня по смерти в мучении.
\vs 3Ez 9:13 Ты не любопытствуй более, как нечестивые будут мучиться, но исследуй, как спасутся праведные, которым принадлежит век и ради которых век, и когда.
\vs 3Ez 9:14 Я отвечал и сказал:
\vs 3Ez 9:15 я прежде говорил, и теперь говорю, и после буду говорить, что больше тех, которые погибнут, нежели тех, которые спасутся, как волна больше капли.
\vs 3Ez 9:16 Он отвечал мне и сказал:
\vs 3Ez 9:17 какова нива, таковы и семена; каковы цветы, таковы и краски; каков делатель, таково и дело; каков земледелец, таково и возделывание; ибо то было время века.
\vs 3Ez 9:18 Когда Я уготовлял век, прежде нежели он был, для обитания тех, которые живут ныне в нем, никто Мне не противоречил.
\vs 3Ez 9:19 А ныне, когда век сей был создан, нравы сотворенных повредились при неоскудевающей жатве, при неисследимом законе.
\vs 3Ez 9:20 И рассмотрел Я век, и вот, оказалась опасность от замыслов, которые появились в нем.
\vs 3Ez 9:21 Я увидел и пощадил его, и сохранил для Себя одну ягоду из виноградной кисти и одно насаждение из множества.
\vs 3Ez 9:22 Пусть погибнет множество, которое напрасно родилось, и сохранится ягода Моя и насаждение Мое, которое Я вырастил с большим трудом.
\vs 3Ez 9:23 А ты, когда по прошествии семи дней иных, не постясь однако в них,
\vs 3Ez 9:24 выйдешь на цветущее поле, где нет построенного дома, и станешь питаться только от полевых цветов и не вкушать мяса, ни пить вина, а только цветы,
\vs 3Ez 9:25 молись ко Всевышнему непрестанно, и Я приду и буду говорить с тобою.
\vs 3Ez 9:26 И пошел я, как Он сказал мне, на поле, которое называется Ардаф, и сел там в цветах и вкушал от полевых трав, и была мне пища от них в насыщение.
\vs 3Ez 9:27 После семи дней лежал я на траве, и сердце мое опять смущалось, как прежде.
\vs 3Ez 9:28 И отверзлись уста мои, и я начал говорить пред Всевышним и сказал:
\vs 3Ez 9:29 о, Господи! являя Себя нам, Ты явился отцам нашим в пустыне непроходимой и бесплодной, когда они вышли из Египта,
\vs 3Ez 9:30 и сказал: <<слушай Меня, Израиль, и внимай словам Моим, семя Иакова.
\vs 3Ez 9:31 Вот, Я сею в вас закон Мой, и принесет в вас плод, и вы будете славиться в нем вечно>>.
\vs 3Ez 9:32 Но отцы наши, приняв закон, не исполнили его и постановлений Твоих не сохранили, и хотя плод закона Твоего не погиб и не мог погибнуть, потому что был Твой,
\vs 3Ez 9:33 но принявшие \bibemph{закон} погибли, не сохранив того, что в нем было посеяно.
\vs 3Ez 9:34 Обыкновенно бывает, что если земля приняла семя, или море корабль, или какой-либо сосуд пищу или питье, и если будет повреждено то, в чем посеяно, или то, в чем помещено,
\vs 3Ez 9:35 в таком случае погибает вместе и самое посеянное, или помещенное, или принятое, и принятого уже не остается пред нами. Но с нами не так.
\vs 3Ez 9:36 Мы, принявшие закон, согрешая, погибли, равно и сердце наше, которое приняло его;
\vs 3Ez 9:37 но закон не погиб, и остается в своей силе.
\vs 3Ez 9:38 Когда я говорил это в сердце моем, я воззрел глазами моими, и увидел на правой стороне женщину; и вот, она плакала и рыдала с великим воплем, и сильно болела душею; одежда ее была разодрана, а на голове ее пепел.
\vs 3Ez 9:39 Тогда оставил я размышления, которыми был занят, и, обратившись к ней, сказал ей:
\vs 3Ez 9:40 о чем плачешь ты, и о чем так скорбишь душею?
\vs 3Ez 9:41 Она сказала: оставь меня, господин мой, да плачу о себе и усугублю скорбь, ибо я весьма огорчена душею и весьма унижена.
\vs 3Ez 9:42 Я спросил ее: что потерпела ты? скажи мне. И она отвечала мне:
\vs 3Ez 9:43 я была неплодна, раба твоя, и не рождала, имея мужа, тридцать лет.
\vs 3Ez 9:44 Каждый час, каждый день в эти тридцать лет я молила Всевышнего непрестанно,
\vs 3Ez 9:45 и услышал меня Бог, рабу твою, после тридцати лет, увидел смирение мое, внял скорби моей и дал мне сына, и я сильно обрадовалась ему, и муж мой, и все сограждане мои, и мы много прославляли Всевышнего.
\vs 3Ez 9:46 Я вскормила его с великим трудом,
\vs 3Ez 9:47 и когда он возрос и пошел взять себе жену, я устроила день пиршества.
\vs 3Ez 10:1 Но когда сын мой вошел в брачный чертог свой, он упал, и умер.
\vs 3Ez 10:2 И опрокинули все мы светильники, и все сограждане мои поднялись утешать меня, и я почила до ночи другого дня.
\vs 3Ez 10:3 Когда же все перестали утешать меня, чтобы оставить меня в покое, я, встав ночью, побежала и пришла, как видишь, на это поле.
\vs 3Ez 10:4 И думаю уже не возвращаться в город, но оставаться здесь, ни есть, ни пить, но непрестанно плакать и поститься, доколе не умру.
\vs 3Ez 10:5 Оставив размышления, которыми занимался, я с гневом отвечал ей и сказал:
\vs 3Ez 10:6 о, безумнейшая из всех жен! не видишь ли скорби нашей и приключившегося нам,~---
\vs 3Ez 10:7 что Сион, мать наша, печалится безмерно, крайне унижена, и плачет горько?
\vs 3Ez 10:8 И теперь, когда все мы скорбим и печалимся, потому что все опечалены, будешь ли ты печалиться об одном сыне твоем?
\vs 3Ez 10:9 Спроси землю, и она скажет тебе, что ей-то должно оплакивать падение столь многих рождающихся на ней;
\vs 3Ez 10:10 ибо все рожденные из нее от начала и другие, которые имеют произойти, едва не все погибают, и толикое множество их предаются истреблению.
\vs 3Ez 10:11 Итак кто должен более печалиться, как не та, которая потеряла толикое множество, а не ты, скорбящая об одном?
\vs 3Ez 10:12 Если ты скажешь мне: <<плач мой не подобен плачу земли, ибо я лишилась плода чрева моего, который я носила с печалью и родила с болезнью;
\vs 3Ez 10:13 а земля~--- по свойству земли; на ней настоящее множество как отходит, так и приходит>>:
\vs 3Ez 10:14 и я скажу тебе, что как ты с трудом родила, так и земля дает плод свой человеку, который от начала возделывает ее.
\vs 3Ez 10:15 Посему воздержись теперь от скорби твоей и мужественно переноси случившуюся тебе потерю.
\vs 3Ez 10:16 Ибо если ты признаешь праведным определение Божие, то в свое время получишь сына, и между женами будешь прославлена.
\vs 3Ez 10:17 Итак возвратись в город к мужу твоему.
\vs 3Ez 10:18 Но она сказала: не сделаю так, не возвращусь в город, но здесь умру.
\vs 3Ez 10:19 Продолжая говорить с нею, я сказал:
\vs 3Ez 10:20 не делай этого, но послушай совета моего. Ибо сколько бед Сиону? Утешься ради скорби Иерусалима.
\vs 3Ez 10:21 Ибо ты видишь, что святилище наше опустошено, алтарь наш ниспровергнут, храм наш разрушен,
\vs 3Ez 10:22 псалтирь наш уничижен, песни умолкли, радость наша исчезла, свет светильника нашего угас, ковчег завета нашего расхищен, Святое наше осквернено, и имя, которое наречено на нас, едва не поругано, дети наши потерпели позор, священники наши избиты, левиты наши отведены в плен, девицы наши осквернены, жены наши потерпели насилие, праведники наши увлечены, отроки наши погибли, юноши наши в рабстве, крепкие наши изнемогли;
\vs 3Ez 10:23 и что всего тяжелее, знамя Сиона лишено славы своей, потому что предано в руки ненавидящих нас.
\vs 3Ez 10:24 Посему оставь великую печаль твою, и отложи множество скорбей, чтобы помиловал тебя Крепкий, и Всевышний даровал тебе успокоение и облегчение трудов.
\vs 3Ez 10:25 При сих словах моих к ней, внезапно просияло лице и взор ее, и вот, вид сделался блистающим, так что я, устрашенный ею, помышлял, что бы это было.
\vs 3Ez 10:26 И вот, она внезапно испустила столь громкий и столь страшный звук голоса, что от сего звука жены поколебалась земля.
\vs 3Ez 10:27 И я видел, и вот, жена более не являлась мне, но созидался город, и место его обозначалось на обширных основаниях, и я устрашенный громко воскликнул и сказал:
\vs 3Ez 10:28 где Ангел Уриил, который вначале приходил ко мне? ибо он привел меня в такое исступление ума, в котором цель моего стремления исчезла, и молитва моя обратилась в поношение.
\vs 3Ez 10:29 Когда я говорил это, он пришел ко мне;
\vs 3Ez 10:30 и увидел меня, и вот, я лежал, как мертвый и в бессознательном состоянии; он взял меня за правую руку, укрепил меня и, поставив на ноги, сказал мне:
\vs 3Ez 10:31 что с тобою? отчего смущены разум твой и чувства сердца твоего? отчего смущаешься?
\vs 3Ez 10:32 Оттого, отвечал я ему, что ты оставил меня, и я, поступая по словам твоим, вышел на поле, и вот увидел и еще вижу то, о чем не могу рассказать.
\vs 3Ez 10:33 А он сказал мне: стой мужественно, и я объясню тебе.
\vs 3Ez 10:34 Говори мне, господин мой, сказал я, только не оставляй меня, чтобы я не умер напрасно;
\vs 3Ez 10:35 ибо я видел, чего не знал, и слышал, чего не знаю.
\vs 3Ez 10:36 Чувство ли мое обманывает меня, или душа моя грезит во сне?
\vs 3Ez 10:37 Посему прошу тебя объяснить мне, рабу твоему, это исступление ума моего. Отвечая мне, сказал он:
\vs 3Ez 10:38 внимай мне, и я научу тебя, и изъясню тебе то, что устрашило тебя: ибо Всевышний откроет тебе многие тайны.
\vs 3Ez 10:39 Он видит правый путь твой, что ты непрестанно скорбишь о народе твоем и сильно печалишься о Сионе.
\vs 3Ez 10:40 Таково значение видения, которое пред сим явилось тебе:
\vs 3Ez 10:41 жена, которую ты видел плачущею и старался утешать,
\vs 3Ez 10:42 которая потом сделалась невидима, но явился тебе город созидаемый,
\vs 3Ez 10:43 и которая тебе рассказала о смерти сына своего, вот что значит:
\vs 3Ez 10:44 жена, которую ты видел, это Сион. А что сказала тебе та, которую ты видел, как город только что созидаемый,
\vs 3Ez 10:45 что она тридцать лет была неплодна, этим указывается на то, что в продолжение тридцати лет в Сионе еще не была приносима жертва.
\vs 3Ez 10:46 По истечении тридцати лет неплодная родила сына: это было тогда, когда Соломон создал город и принес жертвы.
\vs 3Ez 10:47 А что она сказала тебе, что с трудом воспитала его, это было обитание в Иерусалиме.
\vs 3Ez 10:48 А что сын ее, как она сказала тебе, входя в чертог свой, упал и умер, это было падение Иерусалима.
\vs 3Ez 10:49 И вот, ты видел подобие ее, и как она скорбела о сыне, старался утешать ее в случившемся: то надлежало открыть тебе о сем.
\vs 3Ez 10:50 Ныне же Всевышний, видя, что ты скорбишь душею и всем сердцем болезнуешь о нем, показал тебе светлость славы его и красоту его.
\vs 3Ez 10:51 Для сего-то я повелел тебе жить в поле, где нет дома.
\vs 3Ez 10:52 Я знал, что Всевышний покажет тебе это;
\vs 3Ez 10:53 для того и повелел, чтобы ты пришел на поле, где не положено основания здания.
\vs 3Ez 10:54 Ибо не могло дело человеческого созидания существовать там, где начинал показываться город Всевышнего.
\vs 3Ez 10:55 Итак не бойся, и да не страшится сердце твое, но войди и посмотри на светлость и великолепие созидания, сколько могут видеть глаза твои.
\vs 3Ez 10:56 После того услышишь, сколько могут слышать уши твои.
\vs 3Ez 10:57 Ты блаженнее многих и призван к Всевышнему, как немногие.
\vs 3Ez 10:58 На завтрашнюю ночь оставайся здесь,
\vs 3Ez 10:59 и Всевышний покажет тебе видение величайших дел, которые Он сотворит для обитателей земли в последние дни.
\vs 3Ez 10:60 И спал я в ту ночь и в следующую, как он повелел мне.
\vs 3Ez 11:1 И видел я сон, и вот, поднялся с моря орел, у которого было двенадцать крыльев пернатых и три головы.
\vs 3Ez 11:2 И видел я: вот, он распростирал крылья свои над всею землею, и все ветры небесные дули на него и собирались облака.
\vs 3Ez 11:3 И видел я, что из перьев его выходили другие малые перья, и из тех выходили еще меньшие и короткие.
\vs 3Ez 11:4 Головы его покоились, и средняя голова была больше других голов, но также покоилась с ними.
\vs 3Ez 11:5 И видел я: вот орел летал на крыльях своих и царствовал над землею и над всеми обитателями ее.
\vs 3Ez 11:6 И видел я, что все поднебесное было покорно ему, и никто не сопротивлялся ему, ни одна из тварей, существующих на земле.
\vs 3Ez 11:7 И вот, орел стал на когти свои и испустил голос к перьям своим и сказал:
\vs 3Ez 11:8 не бодрствуйте все вместе; спите каждое на своем месте, и бодрствуйте поочередно,
\vs 3Ez 11:9 а головы пусть сохраняются на последнее время.
\vs 3Ez 11:10 Видел я, что голос его исходил не из голов его, но из средины тела его.
\vs 3Ez 11:11 Я сосчитал малые перья его; их было восемь.
\vs 3Ez 11:12 И вот, с правой стороны поднялось одно перо и воцарилось над всею землею.
\vs 3Ez 11:13 И когда воцарилось, пришел конец его, и не видно стало места его; потом поднялось другое перо и царствовало; это владычествовало долгое время.
\vs 3Ez 11:14 Когда оно царствовало и приблизился конец его, чтобы оно так же исчезло, как и первое,
\vs 3Ez 11:15 и вот, слышен был голос, говорящий ему:
\vs 3Ez 11:16 слушай ты, которое столько времени обладало землею! вот что я возвещаю тебе, прежде нежели начнешь исчезать:
\vs 3Ez 11:17 никто после тебя не будет владычествовать столько времени, как ты, и даже половины того.
\vs 3Ez 11:18 И поднялось третье перо, и владычествовало, как и прежние, но исчезло и оно.
\vs 3Ez 11:19 Так было и со всеми другими: они владычествовали и потом исчезали навсегда.
\vs 3Ez 11:20 Я видел, что по времени с правой стороны поднимались следующие перья, чтобы и им иметь начальство, и некоторые из них начальствовали, но тотчас исчезали;
\vs 3Ez 11:21 иные же из них поднимались, но не получали начальства.
\vs 3Ez 11:22 После сего не являлись более двенадцать перьев, ни два малых пера;
\vs 3Ez 11:23 и не осталось в теле орла ничего, кроме двух голов покоящихся и шести малых перьев.
\vs 3Ez 11:24 Я видел, и вот, из шести малых перьев отделились два и остались под головою, которая была с правой стороны, а четыре оставались на своем месте.
\vs 3Ez 11:25 Потом подкрыльные перья покушались подняться и начальствовать;
\vs 3Ez 11:26 и вот, одно поднялось, но тотчас исчезло;
\vs 3Ez 11:27 а следующие исчезали еще скорее, нежели прежние.
\vs 3Ez 11:28 И видел я: вот, два остававшиеся пера покушались также царствовать.
\vs 3Ez 11:29 Когда они покушались, одна из покоящихся голов, которая была средняя, пробудилась, и она была более других двух голов.
\vs 3Ez 11:30 И видел я, что две другие головы соединились с нею.
\vs 3Ez 11:31 И эта голова, обратившись с теми, которые были соединены с нею, пожрала два подкрыльных пера, которые покушались царствовать.
\vs 3Ez 11:32 Эта голова устрашила всю землю и владычествовала над обитателями земли с великим угнетением, и удерживала власть на земном шаре более всех крыльев, которые были.
\vs 3Ez 11:33 После того я видел, что и средняя голова внезапно исчезла, как и крылья;
\vs 3Ez 11:34 оставались две головы, которые подобным образом царствовали на земле и над ее обитателями.
\vs 3Ez 11:35 И вот, голова с правой стороны пожрала ту, которая была с левой.
\vs 3Ez 11:36 И слышал я голос, говорящий мне: смотри перед собою, и размышляй о том, что видишь.
\vs 3Ez 11:37 И видел я: вот, как бы лев, выбежавший из леса и рыкающий, испустил человеческий голос к орлу и сказал:
\vs 3Ez 11:38 слушай, что я буду говорить тебе и что скажет тебе Всевышний:
\vs 3Ez 11:39 не ты ли оставшийся из числа четырех животных, которых Я поставил царствовать в веке Моем, чтобы через них пришел конец времен тех?
\vs 3Ez 11:40 И четвертое из них пришло, победило всех прежде бывших животных и держало век в большом трепете и всю вселенную в лютом угнетении, и с тягостнейшим утеснением подвластных, и столь долгое время обитало на земле с коварством.
\vs 3Ez 11:41 Ты судил землю не по правде;
\vs 3Ez 11:42 ты утеснял кротких, обижал миролюбивых, любил лжецов, разорял жилища тех, которые приносили пользу, и разрушал стены тех, которые не делали тебе вреда.
\vs 3Ez 11:43 И взошла ко Всевышнему обида твоя, и гордыня твоя~--- к Крепкому.
\vs 3Ez 11:44 И воззрел Всевышний на времена гордыни, и вот, они кончились, и исполнилась мера злодейств ее.
\vs 3Ez 11:45 Поэтому исчезни ты, орел, с страшными крыльями твоими, с гнусными перьями твоими, со злыми головами твоими, с жестокими когтями твоими и со всем негодным телом твоим,
\vs 3Ez 11:46 чтобы отдохнула вся земля и освободилась от твоего насилия, и надеялась на суд и милосердие своего Создателя.
\vs 3Ez 12:1 Когда лев говорил к орлу эти слова, я увидел,
\vs 3Ez 12:2 что не являлась более голова, которая оставалась вместе с четырьмя крыльями, которые перешли к ней и поднимались, чтобы царствовать, но которых царство было слабо и исполнено возмущений.
\vs 3Ez 12:3 И я видел, и вот они исчезли, и все тело орла сгорало, и ужаснулась земля, и я от тревоги, исступления ума и от великого страха пробудился и сказал духу моему:
\vs 3Ez 12:4 вот, ты причинил мне это тем, что испытываешь пути Всевышнего.
\vs 3Ez 12:5 Вот, я еще трепещу сердцем и весьма изнемог духом моим, и нет во мне нисколько силы от великого страха, которым я поражен в эту ночь.
\vs 3Ez 12:6 Итак ныне я помолюсь Всевышнему, чтобы Он укрепил меня до конца.
\vs 3Ez 12:7 И сказал я: Владыко Господи! если я обрел благодать пред очами Твоими, если Ты нашел меня праведным пред многими, и если молитва моя подлинно взошла пред лице Твое,
\vs 3Ez 12:8 укрепи меня и покажи мне, рабу Твоему, значение сего страшного видения, чтобы вполне успокоить душу мою:
\vs 3Ez 12:9 ибо Ты судил меня достойным, чтобы показать мне последние времена. И Он сказал мне:
\vs 3Ez 12:10 Таково значение видения сего:
\vs 3Ez 12:11 орел, которого ты видел восходящим от моря, есть царство, показанное в видении Даниилу, брату твоему;
\vs 3Ez 12:12 но ему не было изъяснено то, что ныне Я изъясню тебе.
\vs 3Ez 12:13 Вот, приходят дни, когда восстанет на земле царство более страшное, нежели все царства, бывшие прежде него.
\vs 3Ez 12:14 В нем будут царствовать, один после другого, двенадцать царей.
\vs 3Ez 12:15 Второй из них начнет царствовать, и удержит власть более продолжительное время, нежели прочие двенадцать.
\vs 3Ez 12:16 Таково значение двенадцати крыльев, виденных тобою.
\vs 3Ez 12:17 А что ты слышал говоривший голос, исходящий не от голов орла, но из средины тела его,
\vs 3Ez 12:18 это означает, что после времени того царства произойдут немалые распри, и царство подвергнется опасности падения; но оно не падет тогда и восстановится в первоначальное состояние свое.
\vs 3Ez 12:19 А что ты видел восемь малых подкрыльных перьев, соединенных с крыльями, это означает,
\vs 3Ez 12:20 что восстанут в царстве восемь царей, которых времена будут легки и годы скоротечны, и два из них погибнут.
\vs 3Ez 12:21 Когда будет приближаться среднее время, четыре сохранятся до того времени, когда будет близок конец его; а два сохранятся до конца.
\vs 3Ez 12:22 А что ты видел три головы покоящиеся, это означает,
\vs 3Ez 12:23 что в последние дни царства Всевышний воздвигнет три царства и покорит им многие другие, и они будут владычествовать над землею и обитателями ее
\vs 3Ez 12:24 с б\acc{о}льшим утеснением, нежели все прежде бывшие; поэтому они и названы головами орла,
\vs 3Ez 12:25 ибо они-то довершат беззакония его и положат конец ему.
\vs 3Ez 12:26 А что ты видел, что большая голова не являлась более, это означает, что один из царей умрет на постели своей, впрочем с мучением,
\vs 3Ez 12:27 а двух остальных пожрет меч;
\vs 3Ez 12:28 меч одного пожрет того, который с ним, но и он в последствие времени умрет от меча.
\vs 3Ez 12:29 А что ты видел, два подкрыльных пера перешли на голову, находящуюся с правой стороны,
\vs 3Ez 12:30 это те, которых Всевышний сохранил к концу царства, то есть царство скудное и исполненное беспокойств.
\vs 3Ez 12:31 Лев, которого ты видел поднявшимся из леса и рыкающим, говорящим к орлу и обличающим его в неправдах его всеми словами его, которые ты слышал,
\vs 3Ez 12:32 это~--- Помазанник, сохраненный Всевышним к концу против них и нечестий их, Который обличит их и представит пред ними притеснения их.
\vs 3Ez 12:33 Он поставит их на суд живых и, обличив их, накажет их.
\vs 3Ez 12:34 Он по милосердию избавит остаток народа Моего, тех, которые сохранились в пределах Моих, и обрадует их, доколе не придет конец, день суда, о котором Я сказал тебе вначале.
\vs 3Ez 12:35 Таков сон, виденный тобою, и таково значение его.
\vs 3Ez 12:36 Ты один был достоин знать эту тайну Всевышнего.
\vs 3Ez 12:37 Все это, виденное тобою, напиши в книге и положи в сокровенном месте;
\vs 3Ez 12:38 и научи этому мудрых из народа твоего, которых сердц\acc{а} призн\acc{а}ешь способными принять и хранить сии тайны.
\vs 3Ez 12:39 А ты пребудь здесь еще семь дней, чтобы тебе показано было, что Всевышнему угодно будет показать тебе. И отошел от меня.
\rsbpar\vs 3Ez 12:40 Когда по истечении семи дней весь народ услышал, что я не возвратился в город, собрались все от малого до большого и, придя ко мне, говорили мне:
\vs 3Ez 12:41 чем согрешили мы против тебя? И чем обидели тебя, что ты, оставив нас, сидишь на этом месте?
\vs 3Ez 12:42 Ты один из всего народа остался нам, как гроздь от винограда, как светильник в темном месте и как пристань и корабль, спасенный от бури.
\vs 3Ez 12:43 Неужели мало бедствий, приключившихся нам?
\vs 3Ez 12:44 Если ты оставишь нас, то лучше было бы для нас сгореть, когда горел Сион.
\vs 3Ez 12:45 Ибо мы не лучше тех, которые умерли там. И плакали они с громким воплем. Отвечая им, я сказал:
\vs 3Ez 12:46 надейся, Израиль, и не скорби, дом Иакова;
\vs 3Ez 12:47 ибо помнит о вас Всевышний, и Крепкий не забыл вас в напасти.
\vs 3Ez 12:48 И я не оставил вас и не ушел от вас, но пришел на это место, чтобы помолиться о разоренном Сионе и просить милосердия уничиженной святыне вашей.
\vs 3Ez 12:49 Теперь идите каждый в дом свой, и я приду к вам после сих дней.
\vs 3Ez 12:50 И пошел народ, как я сказал ему, в город,
\vs 3Ez 12:51 а я оставался в поле в продолжение семи дней, как повелено мне, и питался в те дни только цветами полевыми, и трава была мне пищею.
\vs 3Ez 13:1 И было после семи дней, я видел ночью сон:
\vs 3Ez 13:2 вот, поднялся ветер с моря, чтобы возмутить все волны его.
\vs 3Ez 13:3 Я смотрел, и вот, вышел крепкий муж с воинством небесным, и куда он ни обращал лице свое, чтобы взглянуть, все трепетало, что виднелось под ним;
\vs 3Ez 13:4 и куда ни выходил голос из уст его, загорались все, которые слышали голос его, подобно тому, как тает воск, когда почувствует огонь.
\vs 3Ez 13:5 И после этого видел я: вот, собралось множество людей, которым не было числа, от четырех ветров небесных, чтобы преодолеть этого мужа, который поднялся с моря.
\vs 3Ez 13:6 Видел я, и вот, он изваял себе большую гору и взлетел на нее.
\vs 3Ez 13:7 Я старался увидеть ту страну или место, откуда изваяна была эта гора, но не мог.
\vs 3Ez 13:8 После сего видел я, что все, которые собрались победить его, очень испугались и однако же осмелились воевать.
\vs 3Ez 13:9 Он же, когда увидел устремление идущего множества, не поднял руки своей, ни копья не держал и никакого оружия воинского;
\vs 3Ez 13:10 но только, как я видел, он испускал из уст своих как бы дуновение огня и из губ своих~--- как бы дыхание пламени и с языка своего пускал искры и бури, и все это смешалось вместе: и дуновение огня и дыхание пламени и сильная буря.
\vs 3Ez 13:11 И стремительно напал он на это множество, которое приготовилось сразиться, и сжег всех, так что ничего не видно было из бесчисленного множества, кроме праха, и только был запах от дыма; увидел я это, и устрашился.
\vs 3Ez 13:12 После сего я видел того мужа сходящим с горы и призывающим к себе другое множество, мирное.
\vs 3Ez 13:13 И многие приступали к нему, иные с лицами веселыми, а иные с печальными, иные были связаны, иных приносили,~--- и я изнемог от великого страха, пробудился и сказал:
\vs 3Ez 13:14 Ты от начала показал рабу Твоему чудеса сии и судил меня достойным, чтобы принять молитву мою;
\vs 3Ez 13:15 покажи же мне и значение сна сего,
\vs 3Ez 13:16 потому что, как я понимаю разумом моим, горе тем, которые оставлены будут до тех дней, а еще более горе тем, которые не оставлены.
\vs 3Ez 13:17 Ибо те, которые не оставлены, были печальны.
\vs 3Ez 13:18 Теперь я понимаю, что то, что отложено на последние дни, встретит их, но и тех, которые оставлены.
\vs 3Ez 13:19 Поэтому они пришли в большие опасности и большие затруднения, как показывают эти сны.
\vs 3Ez 13:20 Но легче находящемуся в опасности потерпеть это, нежели перейти подобно облаку из мира сего и не видеть того, что будет в последние времена. Он отвечал мне и сказал:
\vs 3Ez 13:21 И значение видения Я скажу тебе, и о чем ты говорил, открою тебе.
\vs 3Ez 13:22 Так как ты говорил о тех, которые оставлены, то вот объяснение:
\vs 3Ez 13:23 кто выдержит опасность в то время, тот сохранил себя, а которые впадут в опасность, это те, которые имеют дела и веру во Всемогущего.
\vs 3Ez 13:24 Итак знай, что те, которые оставлены, блаженнее умерших.
\vs 3Ez 13:25 Вот объяснение видения: так как ты видел мужа, восходящего из средины моря,
\vs 3Ez 13:26 это тот, которого Всевышний хранит многие времена, который самим собою избавит творение свое и управит тех, которые оставлены.
\vs 3Ez 13:27 А что ты видел исходивший из уст его как бы ветер, огонь и бурю,
\vs 3Ez 13:28 и что он не держал ни копья и никакого воинского оружия, но устремление его поразило множество, которое пришло, чтобы победить его, то вот объяснение:
\vs 3Ez 13:29 вот, наступают дни, когда Всевышний начнет избавлять тех, которые на земле,
\vs 3Ez 13:30 и приведет в изумление живущих на земле.
\vs 3Ez 13:31 И будут предпринимать войны одни против других, город против города, одно место против другого, народ против народа, царство против царства.
\vs 3Ez 13:32 Когда это будет и явятся знамения, которые Я показал тебе прежде, тогда откроется Сын Мой, Которого ты видел, как мужа восходящего.
\vs 3Ez 13:33 И когда все народы услышат глас Его, каждый оставит войну в своей собственной стране, которую они имеют между собою.
\vs 3Ez 13:34 И соберется в одно собрание множество бесчисленное, как бы желая идти и победить Его.
\vs 3Ez 13:35 Он же станет на верху горы Сиона.
\vs 3Ez 13:36 И Сион придет и покажется всем приготовленный и устроенный, как ты видел гору, изваянную без рук.
\vs 3Ez 13:37 Сын же Мой обличит нечестия, изобретенные этими народами, которые своими злыми помышлениями приблизили бурю и мучения, которыми они начнут мучиться,
\vs 3Ez 13:38 и которые подобны огню; и Он истребит их без труда законом, который подобен огню.
\vs 3Ez 13:39 А что ты видел, что Он собирал к себе другое, мирное общество:
\vs 3Ez 13:40 это десять колен, которые отведены были пленными из земли своей во дни царя Осии, которого отвел в плен Салманассар, царь Ассирийский, и перевел их за реку, и переведены были в землю иную.
\vs 3Ez 13:41 Они же положили в совете своем, чтобы оставить множество язычников и отправиться в дальнюю страну, где никогда не обитал род человеческий,
\vs 3Ez 13:42 чтобы там соблюдать законы свои, которых они не соблюдали в стране своей.
\vs 3Ez 13:43 Тесными входами подошли они к реке Евфрату;
\vs 3Ez 13:44 ибо Всевышний сотворил тогда для них чудеса и остановил жилы реки, доколе они проходили;
\vs 3Ez 13:45 ибо через эту страну шли они долго, полтора года; эта страна называется Арсареф.
\vs 3Ez 13:46 Там жили они до последнего времени. И ныне, когда они начнут приходить,
\vs 3Ez 13:47 Всевышний снова остановит жилы реки, чтобы они могли пройти; поэтому ты видел множество мирное.
\vs 3Ez 13:48 Но которые оставлены от народа твоего, это те, которые находятся внутри пределов Моих.
\vs 3Ez 13:49 Ибо, когда начнет Он истреблять множество собравшихся вместе народов, Он защитит народ Свой, который останется.
\vs 3Ez 13:50 И тогда покажет им множество чудес.
\vs 3Ez 13:51 Я сказал: Владыко Господи! Объясни мне это, для чего видел я мужа, восходящего из средины моря?
\vs 3Ez 13:52 И Он сказал мне: как не можешь ты исследовать и познать того, что во глубине моря, так никто не может на земле видеть Сына Моего, ни тех, которые с Ним, разве только во время дня Его.
\vs 3Ez 13:53 Вот истолкование сна, который ты видел и которым ты один здесь просвещен.
\vs 3Ez 13:54 Ты оставил дела твои и упражнялся в законе Моем, и взыскал его,
\vs 3Ez 13:55 ибо жизнь твою ты устроил в мудрости и рассудительность назвал твоею матерью.
\vs 3Ez 13:56 Поэтому Я показал тебе воздаяния у Всевышнего; после трех дней Я покажу тебе другое и открою тебе важное и чудное.
\vs 3Ez 13:57 Тогда я пошел и вышел в поле, много славя и благодаря Всевышнего за чудеса, которые Он совершал по временам,
\vs 3Ez 13:58 и что Он управляет настоящим и тем, что произойдет во времена,~--- и там я сидел три дня.
\vs 3Ez 14:1 И было после трех дней, я сидел под дубом, и вот, голос вышел из куста против меня и сказал: Ездра, Ездра!
\vs 3Ez 14:2 Я сказал: вот я, Господи. И встал на ноги мои.
\vs 3Ez 14:3 Тогда сказал Он мне: в кусте Я открылся и говорил Моисею, когда народ Мой был рабом в Египте;
\vs 3Ez 14:4 и послал его и вывел народ Мой из Египта, и привел его к горе Синаю и держал его у Себя много дней,
\vs 3Ez 14:5 и открыл ему много чудес и показал тайны времен и конец, и заповедал ему, сказав:
\vs 3Ez 14:6 <<Эти слова объяви, а прочие скрой>>.
\vs 3Ez 14:7 И ныне тебе говорю:
\vs 3Ez 14:8 знамения, которые Я показал тебе, и сны, которые ты видел, и толкования, которые слышал, положи в сердце твоем;
\vs 3Ez 14:9 потому что ты взят будешь от людей и будешь обращаться с Сыном Моим и с подобными тебе, доколе не окончатся времена.
\vs 3Ez 14:10 Ибо век потерял свою юность, и времена приближаются к старости,
\vs 3Ez 14:11 так как век разделен на двенадцать частей, и девять частей его и половина десятой части уже прошли,
\vs 3Ez 14:12 и остается то, что после половины десятой части.
\vs 3Ez 14:13 Итак ныне устрой дом твой и вразуми народ твой, утешь уничиженных и отрекись тления,
\vs 3Ez 14:14 и отпусти от себя смертные помышления, отбрось тягости людские, сними с себя немощи естества и отложи в сторону тягостные для тебя помыслы, и готовься переселиться от времен сих.
\vs 3Ez 14:15 Ибо после больше будет бедствий, нежели сколько ты видел ныне.
\vs 3Ez 14:16 Сколько будет слабеть век от старости, столько будет умножаться зло для живущих.
\vs 3Ez 14:17 Еще дальше удалится истина, и приблизится ложь; уже поспешает прийти видение, которое ты видел.
\vs 3Ez 14:18 Тогда отвечал я и сказал: вот, я~--- пред Тобою, Господи;
\vs 3Ez 14:19 я пойду, как Ты повелел мне, и вразумлю нынешний народ; но кто научит тех, которые потом родятся?
\vs 3Ez 14:20 Ибо век во тьме лежит, и живущие в нем~--- без света;
\vs 3Ez 14:21 потому что закон Твой сожжен, и оттого никто не знает, что соделано Тобою или что должно им делать.
\vs 3Ez 14:22 Но если я приобрел милость у Тебя, ниспошли на меня Духа Святаго, чтобы я написал все, что было соделано в мире от начала, что было написано в законе Твоем, дабы люди могли найти стезю и дабы те, которые захотят жить в последние времена, могли жить.
\vs 3Ez 14:23 И Он в ответ сказал мне: иди, собери народ и скажи ему, чтобы он не искал тебя в продолжение сорока дней.
\vs 3Ez 14:24 Ты же приготовь себе побольше дощечек и возьми с собою Сария, Даврия, Салемия, Ехана и Асиеля, этих пять, способных писать скоро.
\vs 3Ez 14:25 И приди сюда, и Я возжгу в сердце твоем светильник разума, который не угаснет, доколе не окончится то, что ты начнешь писать.
\vs 3Ez 14:26 И когда ты совершишь это, то иное объяви, а иное тайно передай мудрым. Завтра в этот час ты начнешь писать.
\vs 3Ez 14:27 Тогда я пошел, как Он повелел мне, и собрал весь народ и сказал:
\vs 3Ez 14:28 слушай, Израиль, слова сии:
\vs 3Ez 14:29 отцы наши были странниками в Египте, и освобождены были оттуда,
\vs 3Ez 14:30 и приняли закон жизни, которого не сохранили, который и вы после них нарушили.
\vs 3Ez 14:31 И дана была вам земля в наследие и земля Сион; но отцы ваши и вы делали беззаконие и не держались тех путей, которые Всевышний заповедал вам.
\vs 3Ez 14:32 И Он, как праведный судия, отнял у вас ныне, что даровал вам.
\vs 3Ez 14:33 И ныне вы здесь и братья ваши между вами.
\vs 3Ez 14:34 Если вы будете управлять чувством вашим и образуете сердце ваше, то сохраните жизнь и по смерти пол\acc{у}чите милость.
\vs 3Ez 14:35 Ибо по смерти настанет суд, когда мы оживем; и тогда имена праведных будут объявлены и показаны дела нечестивых.
\vs 3Ez 14:36 Никто не приходи ко мне ныне и не ищи меня до сорока дней.
\vs 3Ez 14:37 И взял я пять мужей, как Он заповедал мне, и пошли мы в поле и остались там.
\vs 3Ez 14:38 И вот, на другой день голос воззвал ко мне: Ездра! открой уста твои и выпей то, чем Я напою тебя.
\vs 3Ez 14:39 Я открыл уста мои, и вот полная чаша подана была мне, которая была наполнена как бы водою, но цвет того был подобен огню.
\vs 3Ez 14:40 И взял я и пил; и когда я пил, сердце мое дышало разумом и в груди моей возрастала мудрость, ибо дух мой подкреплялся памятью;
\vs 3Ez 14:41 уста мои были открыты и больше не закрывались.
\vs 3Ez 14:42 Всевышний даровал разум пяти мужам, и они ночью писали по порядку, что было говорено им и чего они не знали.
\vs 3Ez 14:43 Ночью они ели хлеб; а я говорил днем и не молчал ночью.
\vs 3Ez 14:44 Написаны же были в сорок дней девяносто четыре книги.
\vs 3Ez 14:45 И когда исполнилось сорок дней,
\vs 3Ez 14:46 Всевышний сказал: первые, которые ты написал, положи открыто, чтобы могли читать и достойные и недостойные,
\vs 3Ez 14:47 но последние семьдесят сбереги, чтобы передать их мудрым из народа;
\vs 3Ez 14:48 потому что в них проводник разума, источник мудрости и река знания. Так я и сделал.
\vs 3Ez 15:1 Говори вслух народа Моего слова пророчества, которые вложу Я в уста твои, говорит Господь;
\vs 3Ez 15:2 и сделай, чтобы они написаны были на хартии, потому что они верны и истинны.
\vs 3Ez 15:3 Не бойся, что будут замышлять против тебя, и да не смущает тебя неверие тех, которые будут говорить против тебя,
\vs 3Ez 15:4 ибо всякий неверующий в неверии своем умрет.
\vs 3Ez 15:5 Вот, Я наведу, говорит Господь, на круг земной бедствия: меч и голод, и смерть и пагубу
\vs 3Ez 15:6 за то, что нечестие \bibemph{людей} осквернило всю землю, и пагубные дела их переполнились.
\vs 3Ez 15:7 Посему говорит Господь:
\vs 3Ez 15:8 Я уже не буду молчать о беззакониях, которые совершают они нечестиво, и не буду терпеть в них того, что они делают преступно: вот, кровь неповинная и праведная вопиет ко Мне, и души праведных вопиют непрестанно.
\vs 3Ez 15:9 Отмщу им, говорит Господь, и возьму от них к Себе всякую кровь неповинную.
\vs 3Ez 15:10 Вот, народ Мой ведется как стадо на заклание; не потерплю более, чтобы он жил в Египте,
\vs 3Ez 15:11 но выведу его рукою сильною и мышцею высокою, и поражу Египет казнью, как прежде, и погублю всю землю его.
\vs 3Ez 15:12 Восплачет Египет и основания его, пораженные казнью и мщением, которое наведет на него Бог.
\vs 3Ez 15:13 Восплачут земледельцы, возделывающие землю, потому что оскудеют у них семена от ржавчины и от града и от страшной звезды.
\vs 3Ez 15:14 Горе веку и тем, которые живут в нем,
\vs 3Ez 15:15 ибо приблизился меч и истребление их, и восстанет народ на народ для войны, и мечи в руках их.
\vs 3Ez 15:16 Люди сделаются непостоянными и, одни других одолевая, вознерадят о царе своем, и начальники~--- о ходе дел своих в пределах своей власти.
\vs 3Ez 15:17 Пожелает человек идти в город, и не возможет,
\vs 3Ez 15:18 ибо, по причине их гордости, города возмутятся, домы будут разорены, на людей нападет страх.
\vs 3Ez 15:19 Не сжалится человек над ближним своим, предавая домы их на разорение оружием, расхищая имущество их по причине голода и многих бед.
\vs 3Ez 15:20 Вот, Я созываю, говорит Бог, всех царей земли, от востока и юга, от севера и Ливана, чтобы благоговели предо Мною и обратились к себе самим, и чтобы воздать им, что они делали тем.
\vs 3Ez 15:21 Как поступают они даже доселе с избранными Моими, так поступлю \bibemph{с ними} и воздам в недро их, говорит Господь Бог.
\vs 3Ez 15:22 Не пощадит десница Моя грешников, и меч не перестанет поражать проливающих на землю неповинную кровь.
\vs 3Ez 15:23 Исшел огонь из гнева Его и истребил основания земли и грешников, как зажженную солому.
\vs 3Ez 15:24 Горе грешникам и не соблюдающим заповедей Моих! говорит Господь.
\vs 3Ez 15:25 Не пощажу их. Удалитесь, сыновья отступников, не оскверняйте святыни Моей.
\vs 3Ez 15:26 Господь знает всех, которые грешат против Него; потому предал их на смерть и на убиение.
\vs 3Ez 15:27 На круг земной пришли уже бедствия, и вы пребудете в них. Бог не избавит вас, потому что вы согрешили против Него.
\vs 3Ez 15:28 Вот, видение грозное, и лице его от востока.
\vs 3Ez 15:29 Выступят порождения драконов Аравийских на многих колесницах и с быстротою ветра понесутся по земле, так что наведут страх и трепет на всех, которые услышат о них.
\vs 3Ez 15:30 Выйдут, как вепри из леса, Кармоняне, неистовствующие в ярости, и придут в великой силе, вступят в борьбу с ними и опустошат часть земли Ассирийской.
\vs 3Ez 15:31 Потом драконы, помнящие происхождение свое, одержат верх и, обладая великою силою, обратятся преследовать тех.
\vs 3Ez 15:32 Те смутятся, умолкнут перед силою их и обратят ноги свои в бегство.
\vs 3Ez 15:33 Но находящийся в засаде со стороны Ассириян окружит их и умертвит одного из них; в войске их произойдет страх и трепет и ропот на царей их.
\vs 3Ez 15:34 Вот, облака от востока и от севера до юга, и вид их весьма грозен, исполнен свирепости и бури.
\vs 3Ez 15:35 Они столкнутся между собою, и свергнут много звезд на землю и звезду их; и будет кровь от меча до чрева,
\vs 3Ez 15:36 и помет человеческий~--- до седла верблюда; страх и трепет великий будет на земле.
\vs 3Ez 15:37 Ужаснутся \bibemph{все}, которые увидят эту свирепость, и вострепещут.
\vs 3Ez 15:38 После того много раз будут подниматься бури от юга и севера и частью от запада,
\vs 3Ez 15:39 и ветры сильные поднимутся от востока и откроют его и облако, которое Я подвигнул во гневе; а звезда, назначенная для устрашения при восточном и западном ветре, повредится.
\vs 3Ez 15:40 И поднимутся облака, великие и сильные, полные свирепости, и звезда, чтобы устрашить всю землю и жителей ее; и прольют на всякое место, высокое и возвышенное, страшную звезду,
\vs 3Ez 15:41 огонь и град, мечи летающие и многие воды, чтобы наполнить все поля и все источники множеством вод.
\vs 3Ez 15:42 И затопят город, и стены, и горы, и холмы, и дерева в лесах, и траву в лугах, и хлебные растения их;
\vs 3Ez 15:43 и пройдут безостановочно до Вавилона и сокрушат его;
\vs 3Ez 15:44 соберутся к нему и окружат его; прольют звезду и ярость на него. И поднимется пыль и дым до самого неба, и все кругом будут оплакивать его,
\vs 3Ez 15:45 а те, которые останутся подвластными ему, будут служить тем, которые навели страх.
\vs 3Ez 15:46 И ты, Асия, соучастница в надежде Вавилона и в славе его:
\vs 3Ez 15:47 горе тебе, бедная, за то, что уподоблялась ему и украшала дочерей твоих в блудодеянии, чтобы они нравились и славились у любовников твоих, которые желали всегда блудодействовать с тобою.
\vs 3Ez 15:48 Ты подражала ненавистному во всех делах и предприятиях его.
\vs 3Ez 15:49 За то, говорит Бог, пошлю на тебя бедствия: вдовство, нищету, и голод, и меч, и язву, чтобы опустошить домы твои насилием и смертью.
\vs 3Ez 15:50 И слава могущества твоего засохнет, как цвет, когда настанет зной, посланный на тебя.
\vs 3Ez 15:51 Ты изнеможешь, как нищая, избитая и израненная женщинами, чтобы люди знатные и любовники не могли принимать тебя.
\vs 3Ez 15:52 Стал ли бы Я так ненавидеть тебя, говорит Господь,
\vs 3Ez 15:53 если бы ты не убивала избранных Моих во всякое время, поднимая руки на поражение их и глумясь над смертью их, когда ты была в опьянении?
\vs 3Ez 15:54 Украшай твое лице.
\vs 3Ez 15:55 Мзда блудодеяния твоего в недре твоем; за то и получишь ты воздаяние.
\vs 3Ez 15:56 Как поступала ты с избранными Моими, говорит Господь, так с тобою поступит Бог, и подвергнет тебя бедствиям.
\vs 3Ez 15:57 Дети твои погибнут от голода, ты падешь от меча, города твои будут разрушены, и все твои падут в поле от меча.
\vs 3Ez 15:58 А которые на горах, те погибнут от голода, и будут есть плоть свою по недостатку хлеба и пить кровь по недостатку воды.
\vs 3Ez 15:59 В несчастии пойдешь по морям,~--- и там встретишь беды.
\vs 3Ez 15:60 Во время переходов твоих они бросятся на опустошенный город, и истребят часть земли твоей, и часть славы твоей уничтожат.
\vs 3Ez 15:61 Разоренная, ты послужишь для них соломою, а они для тебя будут огнем;
\vs 3Ez 15:62 и истребят тебя, и города твои, землю твою, горы твои, все леса твои и дерева плодоносные сожгут огнем.
\vs 3Ez 15:63 Сыновей твоих уведут в плен, имущество твое захватят в добычу, и славу твою истребят.
\vs 3Ez 16:1 Горе тебе, Вавилон и Асия, горе тебе, Египет и Сирия!
\vs 3Ez 16:2 Препояшьтесь вретищем и власяницами, оплакивайте сыновей ваших, и болезнуйте, потому что приблизилась ваша погибель.
\vs 3Ez 16:3 Послан на вас меч,~--- и кто отклонит его?
\vs 3Ez 16:4 Послан на вас огонь,~--- и кто угасит его?
\vs 3Ez 16:5 Посланы на вас бедствия,~--- и кто отвратит их?
\vs 3Ez 16:6 Прогонит ли кто голодного льва в лесу, или угасит ли мгновенно огонь в соломе, когда он начнет разгораться?
\vs 3Ez 16:7 Отразит ли кто стрелу, пущенную стрелком сильным?
\vs 3Ez 16:8 Господь сильный посылает бедствия,~--- и кто отвратит их?
\vs 3Ez 16:9 Исшел огонь от гнева Его,~--- и кто угасит его?
\vs 3Ez 16:10 Он блеснет молнией,~--- и кто не убоится? Возгремит,~--- и кто не ужаснется?
\vs 3Ez 16:11 Господь воззрит грозно,~--- и кто не сокрушится до основания от лица Его?
\vs 3Ez 16:12 Содрогнулась земля и основания ее; море волнуется со дна, и волны его возмущаются и рыбы его от лица Господа и от величия силы Его.
\vs 3Ez 16:13 Ибо сильна Его десница, напрягающая лук, остры Его стрелы, пускаемые Им, не ослабеют, когда будут посылаемы до концов земли.
\vs 3Ez 16:14 Вот, посылаются бедствия, и не возвратятся, доколе не придут на землю.
\vs 3Ez 16:15 Возгорается огонь, и не угаснет, доколе не попалит основания земли.
\vs 3Ez 16:16 Как стрела, пущенная сильным стрелком, не возвращается, так не возвратятся бедствия, которые будут посланы на землю.
\vs 3Ez 16:17 Горе мне, горе мне! Кто избавит меня в те дни?
\vs 3Ez 16:18 Начнутся болезни,~--- и многие восстенают; начнется голод,~--- и многие будут гибнуть; начнутся войны,~--- и начальствующими овладеет страх; начнутся бедствия,~--- и все вострепещут.
\vs 3Ez 16:19 Что мне делать тогда, когда придут бедствия?
\vs 3Ez 16:20 Вот, голод и язва, и скорбь и теснота посланы как бичи для исправления:
\vs 3Ez 16:21 но при всем этом \bibemph{люди} не обратятся от беззаконий своих и о бичах не всегда будут помнить.
\vs 3Ez 16:22 Вот, на земле будет дешевизна во всем, и подумают, что настал мир; но тогда-то и постигнут землю бедствия~--- меч, голод и великое смятение.
\vs 3Ez 16:23 От голода погибнут очень многие жители земли, а прочие, которые перенесут голод, падут от меча.
\vs 3Ez 16:24 И трупы, как навоз, будут выбрасываемы, и некому будет оплакивать их, ибо земля опустеет, и города ее будут разрушены.
\vs 3Ez 16:25 Не останется никого, кто возделывал бы землю и сеял на ней.
\vs 3Ez 16:26 Дерева дадут плоды, и кто будет собирать их?
\vs 3Ez 16:27 Виноград созреет, и кто будет топтать его? Ибо повсюду будет великое запустение.
\vs 3Ez 16:28 Трудно будет человеку увидеть человека, или услышать голос его,
\vs 3Ez 16:29 ибо из жителей города останется не более десяти, и из поселян~--- человека два, которые скроются в густых рощах и расселинах скал.
\vs 3Ez 16:30 Как в масличном саду остаются иногда на деревах три или четыре маслины,
\vs 3Ez 16:31 или в винограднике обобранном не досмотрят несколько гроздей те, которые внимательно обирают виноград:
\vs 3Ez 16:32 так в те дни останутся трое или четверо при обыске домов их с мечом.
\vs 3Ez 16:33 Земля останется в запустении, поля ее заглохнут, дороги ее и все тропинки ее зарастут терном, потому что некому будет ходить по ним.
\vs 3Ez 16:34 Плакать будут девицы, не имея женихов; плакать будут жены, не имея мужей; плакать будут дочери их, не имея помощи.
\vs 3Ez 16:35 Женихов их убьют на войне, и мужья их погибнут от голода.
\vs 3Ez 16:36 Слушайте это, и вразумляйтесь, рабы Господни!
\vs 3Ez 16:37 Это~--- слово Господа: внимайте ему, и не верьте богам, о которых говорит Господь.
\vs 3Ez 16:38 Вот, приближаются бедствия, и не замедлят.
\vs 3Ez 16:39 Как у беременной женщины, когда в девятый месяц настанет ей пора родить сына, часа за два или за три до рождения, боли охватывают чрево ее и, при выходе младенца из чрева, не замедлят ни на одну минуту:
\vs 3Ez 16:40 так не замедлят прийти на землю бедствия, и люди того времени восстенают; боли охватят их.
\vs 3Ez 16:41 Слушай слово, народ мой: готовьтесь на брань, и среди бедствий будьте как пришельцы земли.
\vs 3Ez 16:42 Продающий пусть будет, как собирающийся в бегство, и покупающий~--- как готовящийся на погибель;
\vs 3Ez 16:43 торгующий~--- как не ожидающий никакой прибыли, и строящий дом~--- как не надеющийся жить в нем.
\vs 3Ez 16:44 Сеятель пусть думает, что не пожнет, и виноградарь,~--- что не соберет винограда;
\vs 3Ez 16:45 вступающие в брак,~--- что не будут рождать детей, и не вступающие,~--- как вдовцы.
\vs 3Ez 16:46 Посему все трудящиеся без пользы трудятся,
\vs 3Ez 16:47 ибо плодами трудов их воспользуются чужеземцы, и имущество их расхитят, домы их разрушат и сыновей их поработят, потому что в плену и в голоде они рождают детей своих.
\vs 3Ez 16:48 Кто занимается хищничеством, тех, чем дольше украшают они города и домы свои, владения и лица свои,
\vs 3Ez 16:49 тем более возненавижу за грехи их, говорит Господь.
\vs 3Ez 16:50 Как блудница ненавидит женщину честную и весьма благонравную,
\vs 3Ez 16:51 так правда возненавидит неправду, украшающую себя, и обвинит ее в лице, когда придет Тот, Кто будет защищать преследующего всякий грех на земле.
\vs 3Ez 16:52 Потому не подражайте неправде и делам ее,
\vs 3Ez 16:53 ибо еще немного, и неправда будет удалена с земли, а правда воцарится над вами.
\vs 3Ez 16:54 Пусть не говорит грешник, что он не согрешил, потому что горящие угли возгорятся на голове того, кто говорит: я не согрешил пред Господом Богом и славою Его.
\vs 3Ez 16:55 Господь знает все дела людей и начинания их, и помышления их и сердца их.
\vs 3Ez 16:56 Он сказал: <<да будет земля>>,~--- и земля явилась; <<да будет небо>>,~--- и было.
\vs 3Ez 16:57 Словом Его сотворены звезды, и Он знает число звезд.
\vs 3Ez 16:58 Он созерцает бездны и сокровенное в них, измерил море и что в нем.
\vs 3Ez 16:59 Словом Своим Он заключил море среди вод и землю повесил на водах.
\vs 3Ez 16:60 Он простер небо, как шатер, на водах основал его.
\vs 3Ez 16:61 Он поместил в пустыне источники вод и озера на вершинах гор, для низведения рек с высоких скал, чтобы напоять землю.
\vs 3Ez 16:62 Он сотворил человека и положил сердце его в средине тела, и вложил в него дух, жизнь и разум
\vs 3Ez 16:63 и дыхание Бога всемогущего, Который сотворил все и созерцает все сокровенное в сокровенных земли.
\vs 3Ez 16:64 Он знает намерение ваше и что помышляете вы в сердцах ваших, когда грешите и хотите скрыть грехи ваши.
\vs 3Ez 16:65 Потому Господь совершенно ясно видит все дела ваши, и обличит всех вас;
\vs 3Ez 16:66 и вы будете посрамлены, когда грехи ваши откроются перед людьми, и беззакония предстанут обвинителями в тот день.
\vs 3Ez 16:67 Что вы сделаете и как скроете грехи ваши пред Богом и Ангелами Его?
\vs 3Ez 16:68 Вот, Бог~--- Судия; бойтесь Его; оставьте грехи ваши и навсегда перестаньте делать беззакония, и Бог изведет вас и избавит от всякой скорби.
\vs 3Ez 16:69 Ибо вот, возгорается на вас ярость многочисленного полчища, и схватят некоторых из вас и умертвят для принесения в жертву идолам.
\vs 3Ez 16:70 Кто будет единомыслен с ними, тех подвергнут они посмеянию, поношению и попранию.
\vs 3Ez 16:71 Ибо по всем местам и в соседних городах многие восстанут против боящихся Господа.
\vs 3Ez 16:72 Будут, как исступленные, без пощады расхищать и опустошать все у боящихся Господа.
\vs 3Ez 16:73 Опустошат и расхитят имущество их, и из домов их изгонят их.
\vs 3Ez 16:74 Тогда настанет испытание избранным Моим, как золото испытывается огнем.
\vs 3Ez 16:75 Слушайте, возлюбленные Мои, говорит Господь: вот перед вами дни скорби, и от них Я избавлю вас.
\vs 3Ez 16:76 Не бойтесь и не сомневайтесь, ибо вождь ваш~--- Бог.
\vs 3Ez 16:77 Если будете исполнять заповеди и повеления Мои, говорит Господь Бог, то грехи ваши не будут бременем, подавляющим вас, и беззакония ваши не превозмогут вас.
\vs 3Ez 16:78 Горе тем, которые связаны грехами своими и покрыты беззакониями своими! Это~--- поле, которое заросло кустарником и через которое путь покрыт терном, так что человек проходить не может: оно оставляется, и обрекается огню на истребление.

\bibpart{Книги Нового Завета}{Новый Завет}{NT}
\include{tex/Mat}
\bibbookdescr{Mar}{
  inline={От Марка\\\LARGE святое благовествование},
  toc={От Марка},
  bookmark={От Марка},
  header={От Марка},
  %headerleft={},
  %headerright={},
  abbr={Мк}
}
\vs Mar 1:1 Начало Евангелия Иисуса Христа, Сына Божия,
\vs Mar 1:2 как написано у пророков: вот, Я посылаю Ангела Моего пред лицем Твоим, который приготовит путь Твой пред Тобою.
\vs Mar 1:3 Глас вопиющего в пустыне: приготовьте путь Господу, прямыми сделайте стези Ему.
\rsbpar\vs Mar 1:4 Явился Иоанн, крестя в пустыне и проповедуя крещение покаяния для прощения грехов.
\vs Mar 1:5 И выходили к нему вся страна Иудейская и Иерусалимляне, и крестились от него все в реке Иордане, исповедуя грехи свои.
\vs Mar 1:6 Иоанн же носил одежду из верблюжьего волоса и пояс кожаный на чреслах своих, и ел акриды и дикий мед.
\vs Mar 1:7 И проповедовал, говоря: идет за мною Сильнейший меня, у Которого я недостоин, наклонившись, развязать ремень обуви Его;
\vs Mar 1:8 я крестил вас водою, а Он будет крестить вас Духом Святым.
\rsbpar\vs Mar 1:9 И было в те дни, пришел Иисус из Назарета Галилейского и крестился от Иоанна в Иордане.
\vs Mar 1:10 И когда выходил из воды, тотчас увидел \bibemph{Иоанн} разверзающиеся небеса и Духа, как голубя, сходящего на Него.
\vs Mar 1:11 И глас был с небес: Ты Сын Мой возлюбленный, в Котором Мое благоволение.
\rsbpar\vs Mar 1:12 Немедленно после того Дух ведет Его в пустыню.
\vs Mar 1:13 И был Он там в пустыне сорок дней, искушаемый сатаною, и был со зверями; и Ангелы служили Ему.
\rsbpar\vs Mar 1:14 После же того, как предан был Иоанн, пришел Иисус в Галилею, проповедуя Евангелие Царствия Божия
\vs Mar 1:15 и говоря, что исполнилось время и приблизилось Царствие Божие: покайтесь и веруйте в Евангелие.
\rsbpar\vs Mar 1:16 Проходя же близ моря Галилейского, увидел Симона и Андрея, брата его, закидывающих сети в море, ибо они были рыболовы.
\vs Mar 1:17 И сказал им Иисус: идите за Мною, и Я сделаю, что вы будете ловцами человеков.
\vs Mar 1:18 И они тотчас, оставив свои сети, последовали за Ним.
\vs Mar 1:19 И, пройдя оттуда немного, Он увидел Иакова Зеведеева и Иоанна, брата его, также в лодке починивающих сети;
\vs Mar 1:20 и тотчас призвал их. И они, оставив отца своего Зеведея в лодке с работниками, последовали за Ним.
\rsbpar\vs Mar 1:21 И приходят в Капернаум; и вскоре в субботу вошел Он в синагогу и учил.
\vs Mar 1:22 И дивились Его учению, ибо Он учил их, как власть имеющий, а не как книжники.
\vs Mar 1:23 В синагоге их был человек, \bibemph{одержимый} духом нечистым, и вскричал:
\vs Mar 1:24 оставь! что Тебе до нас, Иисус Назарянин? Ты пришел погубить нас! знаю Тебя, кто Ты, Святый Божий.
\vs Mar 1:25 Но Иисус запретил ему, говоря: замолчи и выйди из него.
\vs Mar 1:26 Тогда дух нечистый, сотрясши его и вскричав громким голосом, вышел из него.
\vs Mar 1:27 И все ужаснулись, так что друг друга спрашивали: что это? что это за новое учение, что Он и духам нечистым повелевает со властью, и они повинуются Ему?
\vs Mar 1:28 И скоро разошлась о Нем молва по всей окрестности в Галилее.
\rsbpar\vs Mar 1:29 Выйдя вскоре из синагоги, пришли в дом Симона и Андрея, с Иаковом и Иоанном.
\vs Mar 1:30 Теща же Симонова лежала в горячке; и тотчас говорят Ему о ней.
\vs Mar 1:31 Подойдя, Он поднял ее, взяв ее за руку; и горячка тотчас оставила ее, и она стала служить им.
\vs Mar 1:32 При наступлении же вечера, когда заходило солнце, приносили к Нему всех больных и бесноватых.
\vs Mar 1:33 И весь город собрался к дверям.
\vs Mar 1:34 И Он исцелил многих, страдавших различными болезнями; изгнал многих бесов, и не позволял бесам говорить, что они знают, что Он Христос.
\rsbpar\vs Mar 1:35 А утром, встав весьма рано, вышел и удалился в пустынное место, и там молился.
\vs Mar 1:36 Симон и бывшие с ним пошли за Ним
\vs Mar 1:37 и, найдя Его, говорят Ему: все ищут Тебя.
\vs Mar 1:38 Он говорит им: пойдем в ближние селения и города, чтобы Мне и там проповедовать, ибо Я для того пришел.
\vs Mar 1:39 И Он проповедовал в синагогах их по всей Галилее и изгонял бесов.
\rsbpar\vs Mar 1:40 Приходит к Нему прокаженный и, умоляя Его и падая пред Ним на колени, говорит Ему: если хочешь, можешь меня очистить.
\vs Mar 1:41 Иисус, умилосердившись над ним, простер руку, коснулся его и сказал ему: хочу, очистись.
\vs Mar 1:42 После сего слова проказа тотчас сошла с него, и он стал чист.
\vs Mar 1:43 И, посмотрев на него строго, тотчас отослал его
\vs Mar 1:44 и сказал ему: смотри, никому ничего не говори, но пойди, покажись священнику и принеси за очищение твое, что повелел Моисей, во свидетельство им.
\vs Mar 1:45 А он, выйдя, начал провозглашать и рассказывать о происшедшем, так что \bibemph{Иисус} не мог уже явно войти в город, но находился вне, в местах пустынных. И приходили к Нему отовсюду.
\vs Mar 2:1 Через \bibemph{несколько} дней опять пришел Он в Капернаум; и слышно стало, что Он в доме.
\vs Mar 2:2 Тотчас собрались многие, так что уже и у дверей не было места; и Он говорил им слово.
\vs Mar 2:3 И пришли к Нему с расслабленным, которого несли четверо;
\vs Mar 2:4 и, не имея возможности приблизиться к Нему за многолюдством, раскрыли кровлю \bibemph{дома}, где Он находился, и, прокопав ее, спустили постель, на которой лежал расслабленный.
\vs Mar 2:5 Иисус, видя веру их, говорит расслабленному: чадо! прощаются тебе грехи твои.
\vs Mar 2:6 Тут сидели некоторые из книжников и помышляли в сердцах своих:
\vs Mar 2:7 что Он так богохульствует? кто может прощать грехи, кроме одного Бога?
\vs Mar 2:8 Иисус, тотчас узнав духом Своим, что они так помышляют в себе, сказал им: для чего так помышляете в сердцах ваших?
\vs Mar 2:9 Что легче? сказать ли расслабленному: прощаются тебе грехи? или сказать: встань, возьми свою постель и ходи?
\vs Mar 2:10 Но чтобы вы знали, что Сын Человеческий имеет власть на земле прощать грехи,~--- говорит расслабленному:
\vs Mar 2:11 тебе говорю: встань, возьми постель твою и иди в дом твой.
\vs Mar 2:12 Он тотчас встал и, взяв постель, вышел перед всеми, так что все изумлялись и прославляли Бога, говоря: никогда ничего такого мы не видали.
\rsbpar\vs Mar 2:13 И вышел \bibemph{Иисус} опять к морю; и весь народ пошел к Нему, и Он учил их.
\vs Mar 2:14 Проходя, увидел Он Левия Алфеева, сидящего у сбора пошлин, и говорит ему: следуй за Мною. И \bibemph{он}, встав, последовал за Ним.
\vs Mar 2:15 И когда Иисус возлежал в доме его, возлежали с Ним и ученики Его и многие мытари и грешники: ибо много их было, и они следовали за Ним.
\vs Mar 2:16 Книжники и фарисеи, увидев, что Он ест с мытарями и грешниками, говорили ученикам Его: как это Он ест и пьет с мытарями и грешниками?
\vs Mar 2:17 Услышав \bibemph{сие}, Иисус говорит им: не здоровые имеют нужду во враче, но больные; Я пришел призвать не праведников, но грешников к покаянию.
\rsbpar\vs Mar 2:18 Ученики Иоанновы и фарисейские постились. Приходят к Нему и говорят: почему ученики Иоанновы и фарисейские постятся, а Твои ученики не постятся?
\vs Mar 2:19 И сказал им Иисус: могут ли поститься сыны чертога брачного, когда с ними жених? Доколе с ними жених, не могут поститься,
\vs Mar 2:20 но придут дни, когда отнимется у них жених, и тогда будут поститься в те дни.
\vs Mar 2:21 Никто к ветхой одежде не приставляет заплаты из небеленой ткани: иначе вновь пришитое отдерет от старого, и дыра будет еще хуже.
\vs Mar 2:22 Никто не вливает вина молодого в мехи ветхие: иначе молодое вино прорвет мехи, и вино вытечет, и мехи пропадут; но вино молодое надобно вливать в мехи новые.
\rsbpar\vs Mar 2:23 И случилось Ему в субботу проходить засеянными \bibemph{полями}, и ученики Его дорогою начали срывать колосья.
\vs Mar 2:24 И фарисеи сказали Ему: смотри, чт\acc{о} они делают в субботу, чего не должно \bibemph{делать}?
\vs Mar 2:25 Он сказал им: неужели вы не читали никогда, чт\acc{о} сделал Давид, когда имел нужду и взалкал сам и бывшие с ним?
\vs Mar 2:26 как вошел он в дом Божий при первосвященнике Авиафаре и ел хлебы предложения, которых не должно было есть никому, кроме священников, и дал и бывшим с ним?
\vs Mar 2:27 И сказал им: суббота для человека, а не человек для субботы;
\vs Mar 2:28 посему Сын Человеческий есть господин и субботы.
\vs Mar 3:1 И пришел опять в синагогу; там был человек, имевший иссохшую руку.
\vs Mar 3:2 И наблюдали за Ним, не исцелит ли его в субботу, чтобы обвинить Его.
\vs Mar 3:3 Он же говорит человеку, имевшему иссохшую руку: стань на средину.
\vs Mar 3:4 А им говорит: должно ли в субботу добро делать, или зло делать? душу спасти, или погубить? Но они молчали.
\vs Mar 3:5 И, воззрев на них с гневом, скорбя об ожесточении сердец их, говорит тому человеку: протяни руку твою. Он протянул, и стала рука его здорова, как другая.
\rsbpar\vs Mar 3:6 Фарисеи, выйдя, немедленно составили с иродианами совещание против Него, как бы погубить Его.
\vs Mar 3:7 Но Иисус с учениками Своими удалился к морю; и за Ним последовало множество народа из Галилеи, Иудеи,
\vs Mar 3:8 Иерусалима, Идумеи и из-за Иордана. И \bibemph{живущие} в окрестностях Тира и Сидона, услышав, что Он делал, шли к Нему в великом множестве.
\vs Mar 3:9 И сказал ученикам Своим, чтобы готова была для Него лодка по причине многолюдства, дабы не теснили Его.
\vs Mar 3:10 Ибо многих Он исцелил, так что имевшие язвы бросались к Нему, чтобы коснуться Его.
\vs Mar 3:11 И духи нечистые, когда видели Его, падали пред Ним и кричали: Ты Сын Божий.
\vs Mar 3:12 Но Он строго запрещал им, чтобы не делали Его известным.
\rsbpar\vs Mar 3:13 Потом взошел на гору и позвал к Себе, кого Сам хотел; и пришли к Нему.
\vs Mar 3:14 И поставил \bibemph{из них} двенадцать, чтобы с Ним были и чтобы посылать их на проповедь,
\vs Mar 3:15 и чтобы они имели власть исцелять от болезней и изгонять бесов;
\vs Mar 3:16 \bibemph{поставил} Симона, нарекши ему имя Петр,
\vs Mar 3:17 Иакова Зеведеева и Иоанна, брата Иакова, нарекши им имена Воанергес, то есть <<сыны громовы>>,
\vs Mar 3:18 Андрея, Филиппа, Варфоломея, Матфея, Фому, Иакова Алфеева, Фаддея, Симона Кананита
\vs Mar 3:19 и Иуду Искариотского, который и предал Его.
\rsbpar\vs Mar 3:20 Приходят в дом; и опять сходится народ, так что им невозможно было и хлеба есть.
\vs Mar 3:21 И, услышав, ближние Его пошли взять Его, ибо говорили, что Он вышел из себя.
\vs Mar 3:22 А книжники, пришедшие из Иерусалима, говорили, что Он имеет \bibemph{в Себе} веельзевула и что изгоняет бесов силою бесовского князя.
\vs Mar 3:23 И, призвав их, говорил им притчами: как может сатана изгонять сатану?
\vs Mar 3:24 Если царство разделится само в себе, не может устоять царство т\acc{о};
\vs Mar 3:25 и если дом разделится сам в себе, не может устоять дом тот;
\vs Mar 3:26 и если сатана восстал на самого себя и разделился, не может устоять, но пришел конец его.
\vs Mar 3:27 Никто, войдя в дом сильного, не может расхитить вещей его, если прежде не свяжет сильного, и тогда расхитит дом его.
\vs Mar 3:28 Истинно говорю вам: будут прощены сынам человеческим все грехи и хуления, какими бы ни хулили;
\vs Mar 3:29 но кто будет хулить Духа Святаго, тому не будет прощения вовек, но подлежит он вечному осуждению.
\vs Mar 3:30 \bibemph{Сие сказал Он}, потому что говорили: в Нем нечистый дух.
\rsbpar\vs Mar 3:31 И пришли Матерь и братья Его и, стоя вне \bibemph{дома}, послали к Нему звать Его.
\vs Mar 3:32 Около Него сидел народ. И сказали Ему: вот, Матерь Твоя и братья Твои и сестры Твои, вне \bibemph{дома}, спрашивают Тебя.
\vs Mar 3:33 И отвечал им: кто матерь Моя и братья Мои?
\vs Mar 3:34 И обозрев сидящих вокруг Себя, говорит: вот матерь Моя и братья Мои;
\vs Mar 3:35 ибо кто будет исполнять волю Божию, тот Мне брат, и сестра, и матерь.
\vs Mar 4:1 И опять начал учить при море; и собралось к Нему множество народа, так что Он вошел в лодку и сидел на море, а весь народ был на земле, у моря.
\vs Mar 4:2 И учил их притчами много, и в учении Своем говорил им:
\vs Mar 4:3 слушайте: вот, вышел сеятель сеять;
\vs Mar 4:4 и, когда сеял, случилось, что иное упало при дороге, и налетели птицы и поклевали т\acc{о}.
\vs Mar 4:5 Иное упало на каменистое \bibemph{место}, где немного было земли, и скоро взошло, потому что земля была неглубока;
\vs Mar 4:6 когда же взошло солнце, увяло и, как не имело корня, засохло.
\vs Mar 4:7 Иное упало в терние, и терние выросло, и заглушило \bibemph{семя}, и оно не дало плода.
\vs Mar 4:8 И иное упало на добрую землю и дало плод, который взошел и вырос, и принесло иное тридцать, иное шестьдесят, и иное сто.
\vs Mar 4:9 И сказал им: кто имеет уши слышать, да слышит!
\vs Mar 4:10 Когда же остался без народа, окружающие Его, вместе с двенадцатью, спросили Его о притче.
\vs Mar 4:11 И сказал им: вам дано знать тайны Царствия Божия, а тем внешним все бывает в притчах;
\vs Mar 4:12 так что они своими глазами смотрят, и не видят; своими ушами слышат, и не разумеют, да не обратятся, и прощены будут им грехи.
\vs Mar 4:13 И говорит им: не понимаете этой притчи? Как же вам уразуметь все притчи?
\vs Mar 4:14 Сеятель слово сеет.
\vs Mar 4:15 \bibemph{Посеянное} при дороге означает тех, в которых сеется слово, но \bibemph{к которым}, когда услышат, тотчас приходит сатана и похищает слово, посеянное в сердцах их.
\vs Mar 4:16 Подобным образом и посеянное на каменистом \bibemph{месте} означает тех, которые, когда услышат слово, тотчас с радостью принимают его,
\vs Mar 4:17 но не имеют в себе корня и непостоянны; потом, когда настанет скорбь или гонение за слово, тотчас соблазняются.
\vs Mar 4:18 Посеянное в тернии означает слышащих слово,
\vs Mar 4:19 но в которых заботы века сего, обольщение богатством и другие пожелания, входя в них, заглушают слово, и оно бывает без плода.
\vs Mar 4:20 А посеянное на доброй земле означает тех, которые слушают слово и принимают, и приносят плод, один в тридцать, другой в шестьдесят, иной во сто крат.
\rsbpar\vs Mar 4:21 И сказал им: для того ли приносится свеча, чтобы поставить ее под сосуд или под кровать? не для того ли, чтобы поставить ее на подсвечнике?
\vs Mar 4:22 Нет ничего тайного, что не сделалось бы явным, и ничего не бывает потаенного, что не вышло бы наружу.
\vs Mar 4:23 Если кто имеет уши слышать, да слышит!
\vs Mar 4:24 И сказал им: замечайте, что слышите: какою мерою мерите, такою отмерено будет вам и прибавлено будет вам, слушающим.
\vs Mar 4:25 Ибо кто имеет, тому дано будет, а кто не имеет, у того отнимется и то, что имеет.
\rsbpar\vs Mar 4:26 И сказал: Царствие Божие подобно тому, как если человек бросит семя в землю,
\vs Mar 4:27 и спит, и встает ночью и днем; и к\acc{а}к семя всходит и растет, не знает он,
\vs Mar 4:28 ибо земля сама собою производит сперва зелень, потом колос, потом полное зерно в колосе.
\vs Mar 4:29 Когда же созреет плод, немедленно посылает серп, потому что настала жатва.
\rsbpar\vs Mar 4:30 И сказал: чему уподобим Царствие Божие? или какою притчею изобразим его?
\vs Mar 4:31 Оно~--- как зерно горчичное, которое, когда сеется в землю, есть меньше всех семян на земле;
\vs Mar 4:32 а когда посеяно, всходит и становится больше всех злаков, и пускает большие ветви, так что под тенью его могут укрываться птицы небесные.
\vs Mar 4:33 И таковыми многими притчами проповедовал им слово, сколько они могли слышать.
\vs Mar 4:34 Без притчи же не говорил им, а ученикам наедине изъяснял все.
\rsbpar\vs Mar 4:35 Вечером того дня сказал им: переправимся на ту сторону.
\vs Mar 4:36 И они, отпустив народ, взяли Его с собою, как Он был в лодке; с Ним были и другие лодки.
\vs Mar 4:37 И поднялась великая буря; волны били в лодку, так что она уже наполнялась \bibemph{водою}.
\vs Mar 4:38 А Он спал на корме на возглавии. Его будят и говорят Ему: Учитель! неужели Тебе нужды нет, что мы погибаем?
\vs Mar 4:39 И, встав, Он запретил ветру и сказал морю: умолкни, перестань. И ветер утих, и сделалась великая тишина.
\vs Mar 4:40 И сказал им: что вы так боязливы? как у вас нет веры?
\vs Mar 4:41 И убоялись страхом великим и говорили между собою: кто же Сей, что и ветер и море повинуются Ему?
\vs Mar 5:1 И пришли на другой берег моря, в страну Гадаринскую.
\vs Mar 5:2 И когда вышел Он из лодки, тотчас встретил Его вышедший из гробов человек, \bibemph{одержимый} нечистым духом,
\vs Mar 5:3 он имел жилище в гробах, и никто не мог его связать даже цепями,
\vs Mar 5:4 потому что многократно был он скован оковами и цепями, но разрывал цепи и разбивал оковы, и никто не в силах был укротить его;
\vs Mar 5:5 всегда, ночью и днем, в горах и гробах, кричал он и бился о камни;
\vs Mar 5:6 увидев же Иисуса издалека, прибежал и поклонился Ему,
\vs Mar 5:7 и, вскричав громким голосом, сказал: что Тебе до меня, Иисус, Сын Бога Всевышнего? заклинаю Тебя Богом, не мучь меня!
\vs Mar 5:8 Ибо \bibemph{Иисус} сказал ему: выйди, дух нечистый, из сего человека.
\vs Mar 5:9 И спросил его: как тебе имя? И он сказал в ответ: легион имя мне, потому что нас много.
\vs Mar 5:10 И много просили Его, чтобы не высылал их вон из страны той.
\vs Mar 5:11 Паслось же там при горе большое стадо свиней.
\vs Mar 5:12 И просили Его все бесы, говоря: пошли нас в свиней, чтобы нам войти в них.
\vs Mar 5:13 Иисус тотчас позволил им. И нечистые духи, выйдя, вошли в свиней; и устремилось стадо с крутизны в море, а их было около двух тысяч; и потонули в море.
\vs Mar 5:14 Пасущие же свиней побежали и рассказали в городе и в деревнях. И \bibemph{жители} вышли посмотреть, что случилось.
\vs Mar 5:15 Приходят к Иисусу и видят, что бесновавшийся, в котором был легион, сидит и одет, и в здравом уме; и устрашились.
\vs Mar 5:16 Видевшие рассказали им о том, как это произошло с бесноватым, и о свиньях.
\vs Mar 5:17 И начали просить Его, чтобы отошел от пределов их.
\vs Mar 5:18 И когда Он вошел в лодку, бесновавшийся просил Его, чтобы быть с Ним.
\vs Mar 5:19 Но Иисус не дозволил ему, а сказал: иди домой к своим и расскажи им, что сотворил с тобою Господь и \bibemph{как} помиловал тебя.
\vs Mar 5:20 И пошел и начал проповедовать в Десятиградии, что сотворил с ним Иисус; и все дивились.
\rsbpar\vs Mar 5:21 Когда Иисус опять переправился в лодке на другой берег, собралось к Нему множество народа. Он был у моря.
\vs Mar 5:22 И вот, приходит один из начальников синагоги, по имени Иаир, и, увидев Его, падает к ногам Его
\vs Mar 5:23 и усильно просит Его, говоря: дочь моя при смерти; приди и возложи на нее руки, чтобы она выздоровела и осталась жива.
\vs Mar 5:24 \bibemph{Иисус} пошел с ним. За Ним следовало множество народа, и теснили Его.
\rsbpar\vs Mar 5:25 Одна женщина, которая страдала кровотечением двенадцать лет,
\vs Mar 5:26 много потерпела от многих врачей, истощила всё, что было у ней, и не получила никакой пользы, но пришла еще в худшее состояние,~---
\vs Mar 5:27 услышав об Иисусе, подошла сзади в народе и прикоснулась к одежде Его,
\vs Mar 5:28 ибо говорила: если хотя к одежде Его прикоснусь, то выздоровею.
\vs Mar 5:29 И тотчас иссяк у ней источник крови, и она ощутила в теле, что исцелена от болезни.
\vs Mar 5:30 В то же время Иисус, почувствовав Сам в Себе, что вышла из Него сила, обратился в народе и сказал: кто прикоснулся к Моей одежде?
\vs Mar 5:31 Ученики сказали Ему: Ты видишь, что народ теснит Тебя, и говоришь: кто прикоснулся ко Мне?
\vs Mar 5:32 Но Он смотрел вокруг, чтобы видеть ту, которая сделала это.
\vs Mar 5:33 Женщина в страхе и трепете, зная, что с нею произошло, подошла, пала пред Ним и сказала Ему всю истину.
\vs Mar 5:34 Он же сказал ей: дщерь! вера твоя спасла тебя; иди в мире и будь здорова от болезни твоей.
\rsbpar\vs Mar 5:35 Когда Он еще говорил сие, приходят от начальника синагоги и говорят: дочь твоя умерла; что еще утруждаешь Учителя?
\vs Mar 5:36 Но Иисус, услышав сии слова, тотчас говорит начальнику синагоги: не бойся, только веруй.
\vs Mar 5:37 И не позволил никому следовать за Собою, кроме Петра, Иакова и Иоанна, брата Иакова.
\vs Mar 5:38 Приходит в дом начальника синагоги и видит смятение и плачущих и вопиющих громко.
\vs Mar 5:39 И, войдя, говорит им: что смущаетесь и плачете? девица не умерла, но спит.
\vs Mar 5:40 И смеялись над Ним. Но Он, выслав всех, берет с Собою отца и мать девицы и бывших с Ним и входит туда, где девица лежала.
\vs Mar 5:41 И, взяв девицу за руку, говорит ей: <<талиф\acc{а} кум\acc{и}>>, что значит: девица, тебе говорю, встань.
\vs Mar 5:42 И девица тотчас встала и начала ходить, ибо была лет двенадцати. \bibemph{Видевшие} пришли в великое изумление.
\vs Mar 5:43 И Он строго приказал им, чтобы никто об этом не знал, и сказал, чтобы дали ей есть.
\vs Mar 6:1 Оттуда вышел Он и пришел в Свое отечество; за Ним следовали ученики Его.
\vs Mar 6:2 Когда наступила суббота, Он начал учить в синагоге; и многие слышавшие с изумлением говорили: откуда у Него это? что за премудрость дана Ему, и как такие чудеса совершаются руками Его?
\vs Mar 6:3 Не плотник ли Он, сын Марии, брат Иакова, Иосии, Иуды и Симона? Не здесь ли, между нами, Его сестры? И соблазнялись о Нем.
\vs Mar 6:4 Иисус же сказал им: не бывает пророк без чести, разве только в отечестве своем и у сродников и в доме своем.
\vs Mar 6:5 И не мог совершить там никакого чуда, только на немногих больных возложив руки, исцелил \bibemph{их}.
\vs Mar 6:6 И дивился неверию их; потом ходил по окрестным селениям и учил.
\rsbpar\vs Mar 6:7 И, призвав двенадцать, начал посылать их по два, и дал им власть над нечистыми духами.
\vs Mar 6:8 И заповедал им ничего не брать в дорогу, кроме одного посоха: ни сумы, ни хлеба, ни меди в поясе,
\vs Mar 6:9 но обуваться в простую обувь и не носить двух одежд.
\vs Mar 6:10 И сказал им: если где войдете в дом, оставайтесь в нем, доколе не выйдете из того места.
\vs Mar 6:11 И если кто не примет вас и не будет слушать вас, то, выходя оттуда, отрясите прах от ног ваших, во свидетельство на них. Истинно говорю вам: отраднее будет Содому и Гоморре в день суда, нежели тому городу.
\vs Mar 6:12 Они пошли и проповедовали покаяние;
\vs Mar 6:13 изгоняли многих бесов и многих больных мазали маслом и исцеляли.
\rsbpar\vs Mar 6:14 Царь Ирод, услышав \bibemph{об Иисусе} (ибо имя Его стало гласно), говорил: это Иоанн Креститель воскрес из мертвых, и потому чудеса делаются им.
\vs Mar 6:15 Другие говорили: это Илия, а иные говорили: это пророк, или как один из пророков.
\vs Mar 6:16 Ирод же, услышав, сказал: это Иоанн, которого я обезглавил; он воскрес из мертвых.
\vs Mar 6:17 Ибо сей Ирод, послав, взял Иоанна и заключил его в темницу за Иродиаду, жену Филиппа, брата своего, потому что женился на ней.
\vs Mar 6:18 Ибо Иоанн говорил Ироду: не должно тебе иметь жену брата твоего.
\vs Mar 6:19 Иродиада же, злобясь на него, желала убить его; но не могла.
\vs Mar 6:20 Ибо Ирод боялся Иоанна, зная, что он муж праведный и святой, и берёг его; многое делал, слушаясь его, и с удовольствием слушал его.
\vs Mar 6:21 Настал удобный день, когда Ирод, по случаю \bibemph{дня} рождения своего, делал пир вельможам своим, тысяченачальникам и старейшинам Галилейским,~---
\vs Mar 6:22 дочь Иродиады вошла, плясала и угодила Ироду и возлежавшим с ним; царь сказал девице: проси у меня, чего хочешь, и дам тебе;
\vs Mar 6:23 и клялся ей: чего ни попросишь у меня, дам тебе, даже до половины моего царства.
\vs Mar 6:24 Она вышла и спросила у матери своей: чего просить? Та отвечала: головы Иоанна Крестителя.
\vs Mar 6:25 И она тотчас пошла с поспешностью к царю и просила, говоря: хочу, чтобы ты дал мне теперь же на блюде голову Иоанна Крестителя.
\vs Mar 6:26 Царь опечалился, но ради клятвы и возлежавших с ним не захотел отказать ей.
\vs Mar 6:27 И тотчас, послав оруженосца, царь повелел принести голову его.
\vs Mar 6:28 Он пошел, отсек ему голову в темнице, и принес голову его на блюде, и отдал ее девице, а девица отдала ее матери своей.
\vs Mar 6:29 Ученики его, услышав, пришли и взяли тело его, и положили его во гробе.
\rsbpar\vs Mar 6:30 И собрались Апостолы к Иисусу и рассказали Ему всё, и что сделали, и чему научили.
\vs Mar 6:31 Он сказал им: пойдите вы одни в пустынное место и отдохните немного,~--- ибо много было приходящих и отходящих, так что и есть им было некогда.
\vs Mar 6:32 И отправились в пустынное место в лодке одни.
\vs Mar 6:33 Народ увидел, \bibemph{как} они отправлялись, и многие узнали их; и бежали туда пешие из всех городов, и предупредили их, и собрались к Нему.
\vs Mar 6:34 Иисус, выйдя, увидел множество народа и сжалился над ними, потому что они были, как овцы, не имеющие пастыря; и начал учить их много.
\vs Mar 6:35 И как времени прошло много, ученики Его, приступив к Нему, говорят: место \bibemph{здесь} пустынное, а времени уже много,~---
\vs Mar 6:36 отпусти их, чтобы они пошли в окрестные деревни и селения и купили себе хлеба, ибо им нечего есть.
\vs Mar 6:37 Он сказал им в ответ: вы дайте им есть. И сказали Ему: разве нам пойти купить хлеба динариев на двести и дать им есть?
\vs Mar 6:38 Но Он спросил их: сколько у вас хлебов? пойдите, посмотрите. Они, узнав, сказали: пять хлебов и две рыбы.
\vs Mar 6:39 Тогда повелел им рассадить всех отделениями на зеленой траве.
\vs Mar 6:40 И сели рядами, по сто и по пятидесяти.
\vs Mar 6:41 Он взял пять хлебов и две рыбы, воззрев на небо, благословил и преломил хлебы и дал ученикам Своим, чтобы они раздали им; и две рыбы разделил на всех.
\vs Mar 6:42 И ели все, и насытились.
\vs Mar 6:43 И набрали кусков хлеба и \bibemph{остатков} от рыб двенадцать полных коробов.
\vs Mar 6:44 Было же евших хлебы около пяти тысяч мужей.
\rsbpar\vs Mar 6:45 И тотчас понудил учеников Своих войти в лодку и отправиться вперед на другую сторону к Вифсаиде, пока Он отпустит народ.
\vs Mar 6:46 И, отпустив их, пошел на гору помолиться.
\vs Mar 6:47 Вечером лодка была посреди моря, а Он один на земле.
\vs Mar 6:48 И увидел их бедствующих в плавании, потому что ветер им был противный; около же четвертой стражи ночи подошел к ним, идя по морю, и хотел миновать их.
\vs Mar 6:49 Они, увидев Его идущего по морю, подумали, что это призрак, и вскричали.
\vs Mar 6:50 Ибо все видели Его и испугались. И тотчас заговорил с ними и сказал им: ободритесь; это Я, не бойтесь.
\vs Mar 6:51 И вошел к ним в лодку, и ветер утих. И они чрезвычайно изумлялись в себе и дивились,
\vs Mar 6:52 ибо не вразумились \bibemph{чудом} над хлебами, потому что сердце их было окаменено.
\vs Mar 6:53 И, переправившись, прибыли в землю Геннисаретскую и пристали \bibemph{к берегу}.
\vs Mar 6:54 Когда вышли они из лодки, тотчас \bibemph{жители}, узнав Его,
\vs Mar 6:55 обежали всю окрестность ту и начали на постелях приносить больных туда, где Он, как слышно было, находился.
\vs Mar 6:56 И куда ни приходил Он, в селения ли, в города ли, в деревни ли, клали больных на открытых местах и просили Его, чтобы им прикоснуться хотя к краю одежды Его; и которые прикасались к Нему, исцелялись.
\vs Mar 7:1 Собрались к Нему фарисеи и некоторые из книжников, пришедшие из Иерусалима,
\vs Mar 7:2 и, увидев некоторых из учеников Его, евших хлеб нечистыми, то есть неумытыми, руками, укоряли.
\vs Mar 7:3 Ибо фарисеи и все Иудеи, держась предания старцев, не едят, не умыв тщательно рук;
\vs Mar 7:4 и, \bibemph{придя} с торга, не едят не омывшись. Есть и многое другое, чего они приняли держаться: наблюдать омовение чаш, кружек, котлов и скамей.
\vs Mar 7:5 Потом спрашивают Его фарисеи и книжники: зачем ученики Твои не поступают по преданию старцев, но неумытыми руками едят хлеб?
\vs Mar 7:6 Он сказал им в ответ: хорошо пророчествовал о вас, лицемерах, Исаия, как написано: люди сии чтут Меня устами, сердце же их далеко отстоит от Меня,
\vs Mar 7:7 но тщетно чтут Меня, уча учениям, заповедям человеческим.
\vs Mar 7:8 Ибо вы, оставив заповедь Божию, держитесь предания человеческого, омовения кружек и чаш, и делаете многое другое, сему подобное.
\vs Mar 7:9 И сказал им: хорошо ли, \bibemph{что} вы отменяете заповедь Божию, чтобы соблюсти свое предание?
\vs Mar 7:10 Ибо Моисей сказал: почитай отца своего и мать свою; и: злословящий отца или мать смертью да умрет.
\vs Mar 7:11 А вы говорите: кто скажет отцу или матери: корван, то есть дар \bibemph{Богу} т\acc{о}, чем бы ты от меня пользовался,
\vs Mar 7:12 тому вы уже попускаете ничего не делать для отца своего или матери своей,
\vs Mar 7:13 устраняя слово Божие преданием вашим, которое вы установили; и делаете многое сему подобное.
\vs Mar 7:14 И, призвав весь народ, говорил им: слушайте Меня все и разумейте:
\vs Mar 7:15 ничто, входящее в человека извне, не может осквернить его; но что исходит из него, то оскверняет человека.
\vs Mar 7:16 Если кто имеет уши слышать, да слышит!
\vs Mar 7:17 И когда Он от народа вошел в дом, ученики Его спросили Его о притче.
\vs Mar 7:18 Он сказал им: неужели и вы так непонятливы? Неужели не разумеете, что ничто, извне входящее в человека, не может осквернить его?
\vs Mar 7:19 Потому что не в сердце его входит, а в чрево, и выходит вон, \bibemph{чем} очищается всякая пища.
\vs Mar 7:20 Далее сказал: исходящее из человека оскверняет человека.
\vs Mar 7:21 Ибо извнутрь, из сердца человеческого, исходят злые помыслы, прелюбодеяния, любодеяния, убийства,
\vs Mar 7:22 кражи, лихоимство, злоба, коварство, непотребство, завистливое око, богохульство, гордость, безумство,~---
\vs Mar 7:23 всё это зло извнутрь исходит и оскверняет человека.
\rsbpar\vs Mar 7:24 И, отправившись оттуда, пришел в пределы Тирские и Сидонские; и, войдя в дом, не хотел, чтобы кто узнал; но не мог утаиться.
\vs Mar 7:25 Ибо услышала о Нем женщина, у которой дочь одержима была нечистым духом, и, придя, припала к ногам Его;
\vs Mar 7:26 а женщина та была язычница, родом сирофиникиянка; и просила Его, чтобы изгнал беса из ее дочери.
\vs Mar 7:27 Но Иисус сказал ей: дай прежде насытиться детям, ибо нехорошо взять хлеб у детей и бросить псам.
\vs Mar 7:28 Она же сказала Ему в ответ: так, Господи; но и псы под столом едят крохи у детей.
\vs Mar 7:29 И сказал ей: за это слово, пойди; бес вышел из твоей дочери.
\vs Mar 7:30 И, придя в свой дом, она нашла, что бес вышел и дочь лежит на постели.
\rsbpar\vs Mar 7:31 Выйдя из пределов Тирских и Сидонских, \bibemph{Иисус} опять пошел к морю Галилейскому через пределы Десятиградия.
\vs Mar 7:32 Привели к Нему глухого косноязычного и просили Его возложить на него руку.
\vs Mar 7:33 \bibemph{Иисус}, отведя его в сторону от народа, вложил персты Свои в уши ему и, плюнув, коснулся языка его;
\vs Mar 7:34 и, воззрев на небо, вздохнул и сказал ему: <<еффаф\acc{а}>>, то есть: отверзись.
\vs Mar 7:35 И тотчас отверзся у него слух и разрешились узы его языка, и стал говорить чисто.
\vs Mar 7:36 И повелел им не сказывать никому. Но сколько Он ни запрещал им, они еще более разглашали.
\vs Mar 7:37 И чрезвычайно дивились, и говорили: всё хорошо делает,~--- и глухих делает слышащими, и немых~--- говорящими.
\vs Mar 8:1 В те дни, когда собралось весьма много народа и нечего было им есть, Иисус, призвав учеников Своих, сказал им:
\vs Mar 8:2 жаль Мне народа, что уже три дня находятся при Мне, и нечего им есть.
\vs Mar 8:3 Если неевшими отпущу их в домы их, ослабеют в дороге, ибо некоторые из них пришли издалека.
\vs Mar 8:4 Ученики Его отвечали Ему: откуда мог бы кто \bibemph{взять} здесь в пустыне хлебов, чтобы накормить их?
\vs Mar 8:5 И спросил их: сколько у вас хлебов? Они сказали: семь.
\vs Mar 8:6 Тогда велел народу возлечь на землю; и, взяв семь хлебов и воздав благодарение, преломил и дал ученикам Своим, чтобы они раздали; и они раздали народу.
\vs Mar 8:7 Было у них и немного рыбок: благословив, Он велел раздать и их.
\vs Mar 8:8 И ели, и насытились; и набрали оставшихся кусков семь корзин.
\vs Mar 8:9 Евших же было около четырех тысяч. И отпустил их.
\rsbpar\vs Mar 8:10 И тотчас войдя в лодку с учениками Своими, прибыл в пределы Далмануфские.
\vs Mar 8:11 Вышли фарисеи, начали с Ним спорить и требовали от Него знамения с неба, искушая Его.
\vs Mar 8:12 И Он, глубоко вздохнув, сказал: для чего род сей требует знамения? Истинно говорю вам, не дастся роду сему знамение.
\vs Mar 8:13 И, оставив их, опять вошел в лодку и отправился на ту сторону.
\vs Mar 8:14 При сем ученики Его забыли взять хлебов и кроме одного хлеба не имели с собою в лодке.
\vs Mar 8:15 А Он заповедал им, говоря: смотрите, берегитесь закваски фарисейской и закваски Иродовой.
\vs Mar 8:16 И, рассуждая между собою, говорили: \bibemph{это значит}, что хлебов нет у нас.
\vs Mar 8:17 Иисус, уразумев, говорит им: что рассуждаете о том, что нет у вас хлебов? Еще ли не понимаете и не разумеете? Еще ли окаменено у вас сердце?
\vs Mar 8:18 Имея очи, не видите? имея уши, не слышите? и не помните?
\vs Mar 8:19 Когда Я пять хлебов преломил для пяти тысяч \bibemph{человек}, сколько полных коробов набрали вы кусков? Говорят Ему: двенадцать.
\vs Mar 8:20 А когда семь для четырех тысяч, сколько корзин набрали вы оставшихся кусков? Сказали: семь.
\vs Mar 8:21 И сказал им: как же не разумеете?
\rsbpar\vs Mar 8:22 Приходит в Вифсаиду; и приводят к Нему слепого и просят, чтобы прикоснулся к нему.
\vs Mar 8:23 Он, взяв слепого за руку, вывел его вон из селения и, плюнув ему на глаза, возложил на него руки и спросил его: видит ли что?
\vs Mar 8:24 Он, взглянув, сказал: вижу проходящих людей, как деревья.
\vs Mar 8:25 Потом опять возложил руки на глаза ему и велел ему взглянуть. И он исцелел и стал видеть все ясно.
\vs Mar 8:26 И послал его домой, сказав: не заходи в селение и не рассказывай никому в селении.
\rsbpar\vs Mar 8:27 И пошел Иисус с учениками Своими в селения Кесарии Филипповой. Дорогою Он спрашивал учеников Своих: за кого почитают Меня люди?
\vs Mar 8:28 Они отвечали: за Иоанна Крестителя; другие же~--- за Илию; а иные~--- за одного из пророков.
\vs Mar 8:29 Он говорит им: а вы за кого почитаете Меня? Петр сказал Ему в ответ: Ты Христос.
\vs Mar 8:30 И запретил им, чтобы никому не говорили о Нем.
\vs Mar 8:31 И начал учить их, что Сыну Человеческому много должно пострадать, быть отвержену старейшинами, первосвященниками и книжниками, и быть убиту, и в третий день воскреснуть.
\vs Mar 8:32 И говорил о сем открыто. Но Петр, отозвав Его, начал прекословить Ему.
\vs Mar 8:33 Он же, обратившись и взглянув на учеников Своих, воспретил Петру, сказав: отойди от Меня, сатана, потому что ты думаешь не о том, что Божие, но что человеческое.
\vs Mar 8:34 И, подозвав народ с учениками Своими, сказал им: кто хочет идти за Мною, отвергнись себя, и возьми крест свой, и следуй за Мною.
\vs Mar 8:35 Ибо кто хочет душу свою сберечь, тот потеряет ее, а кто потеряет душу свою ради Меня и Евангелия, тот сбережет ее.
\vs Mar 8:36 Ибо какая польза человеку, если он приобретет весь мир, а душе своей повредит?
\vs Mar 8:37 Или какой выкуп даст человек за душу свою?
\vs Mar 8:38 Ибо кто постыдится Меня и Моих слов в роде сем прелюбодейном и грешном, того постыдится и Сын Человеческий, когда приидет в славе Отца Своего со святыми Ангелами.
\vs Mar 9:1 И сказал им: истинно говорю вам: есть некоторые из стоящих здесь, которые не вкусят смерти, как уже увидят Царствие Божие, пришедшее в силе.
\vs Mar 9:2 И, по прошествии дней шести, взял Иисус Петра, Иакова и Иоанна, и возвел на гору высокую особо их одних, и преобразился перед ними.
\vs Mar 9:3 Одежды Его сделались блистающими, весьма белыми, как снег, как на земле белильщик не может выбелить.
\vs Mar 9:4 И явился им Илия с Моисеем; и беседовали с Иисусом.
\vs Mar 9:5 При сем Петр сказал Иисусу: Равв\acc{и}! хорошо нам здесь быть; сделаем три кущи: Тебе одну, Моисею одну, и одну Илии.
\vs Mar 9:6 Ибо не знал, что сказать; потому что они были в страхе.
\vs Mar 9:7 И явилось облако, осеняющее их, и из облака исшел глас, глаголющий: Сей есть Сын Мой возлюбленный; Его слушайте.
\vs Mar 9:8 И, внезапно посмотрев вокруг, никого более с собою не видели, кроме одного Иисуса.
\vs Mar 9:9 Когда же сходили они с горы, Он не велел никому рассказывать о том, что видели, доколе Сын Человеческий не воскреснет из мертвых.
\vs Mar 9:10 И они удержали это слово, спрашивая друг друга, что значит: воскреснуть из мертвых.
\vs Mar 9:11 И спросили Его: как же книжники говорят, что Илии надлежит прийти прежде?
\vs Mar 9:12 Он сказал им в ответ: правда, Илия должен прийти прежде и устроить всё; и Сыну Человеческому, как написано о Нем, \bibemph{надлежит} много пострадать и быть уничижену.
\vs Mar 9:13 Но говорю вам, что и Илия пришел, и поступили с ним, как хотели, как написано о нем.
\rsbpar\vs Mar 9:14 Придя к ученикам, увидел много народа около них и книжников, спорящих с ними.
\vs Mar 9:15 Тотчас, увидев Его, весь народ изумился, и, подбегая, приветствовали Его.
\vs Mar 9:16 Он спросил книжников: о чем спорите с ними?
\vs Mar 9:17 Один из народа сказал в ответ: Учитель! я привел к Тебе сына моего, одержимого духом немым:
\vs Mar 9:18 где ни схватывает его, повергает его на землю, и он испускает пену, и скрежещет зубами своими, и цепенеет. Говорил я ученикам Твоим, чтобы изгнали его, и они не могли.
\vs Mar 9:19 Отвечая ему, Иисус сказал: о, род неверный! доколе буду с вами? доколе буду терпеть вас? Приведите его ко Мне.
\vs Mar 9:20 И привели его к Нему. Как скоро \bibemph{бесноватый} увидел Его, дух сотряс его; он упал на землю и валялся, испуская пену.
\vs Mar 9:21 И спросил \bibemph{Иисус} отца его: как давно это сделалось с ним? Он сказал: с детства;
\vs Mar 9:22 и многократно \bibemph{дух} бросал его и в огонь и в воду, чтобы погубить его; но, если что можешь, сжалься над нами и помоги нам.
\vs Mar 9:23 Иисус сказал ему: если сколько-нибудь можешь веровать, всё возможно верующему.
\vs Mar 9:24 И тотчас отец отрока воскликнул со слезами: верую, Господи! помоги моему неверию.
\vs Mar 9:25 Иисус, видя, что сбегается народ, запретил духу нечистому, сказав ему: дух немой и глухой! Я повелеваю тебе, выйди из него и впредь не входи в него.
\vs Mar 9:26 И, вскрикнув и сильно сотрясши его, вышел; и он сделался, как мертвый, так что многие говорили, что он умер.
\vs Mar 9:27 Но Иисус, взяв его за руку, поднял его; и он встал.
\vs Mar 9:28 И как вошел \bibemph{Иисус} в дом, ученики Его спрашивали Его наедине: почему мы не могли изгнать его?
\vs Mar 9:29 И сказал им: сей род не может выйти иначе, как от молитвы и поста.
\rsbpar\vs Mar 9:30 Выйдя оттуда, проходили через Галилею; и Он не хотел, чтобы кто узнал.
\vs Mar 9:31 Ибо учил Своих учеников и говорил им, что Сын Человеческий предан будет в руки человеческие и убьют Его, и, по убиении, в третий день воскреснет.
\vs Mar 9:32 Но они не разумели сих слов, а спросить Его боялись.
\rsbpar\vs Mar 9:33 Пришел в Капернаум; и когда был в доме, спросил их: о чем дорогою вы рассуждали между собою?
\vs Mar 9:34 Они молчали; потому что дорогою рассуждали между собою, кто больше.
\vs Mar 9:35 И, сев, призвал двенадцать и сказал им: кто хочет быть первым, будь из всех последним и всем слугою.
\vs Mar 9:36 И, взяв дитя, поставил его посреди них и, обняв его, сказал им:
\vs Mar 9:37 кто примет одно из таких детей во имя Мое, тот принимает Меня; а кто Меня примет, тот не Меня принимает, но Пославшего Меня.
\vs Mar 9:38 При сем Иоанн сказал: Учитель! мы видели человека, который именем Твоим изгоняет бесов, а не ходит за нами; и запретили ему, потому что не ходит за нами.
\vs Mar 9:39 Иисус сказал: не запрещайте ему, ибо никто, сотворивший чудо именем Моим, не может вскоре злословить Меня.
\vs Mar 9:40 Ибо кто не против вас, тот за вас.
\vs Mar 9:41 И кто напоит вас чашею воды во имя Мое, потому что вы Христовы, истинно говорю вам, не потеряет награды своей.
\vs Mar 9:42 А кто соблазнит одного из малых сих, верующих в Меня, тому лучше было бы, если бы повесили ему жерновный камень на шею и бросили его в море.
\vs Mar 9:43 И если соблазняет тебя рука твоя, отсеки ее: лучше тебе увечному войти в жизнь, нежели с двумя руками идти в геенну, в огонь неугасимый,
\vs Mar 9:44 где червь их не умирает и огонь не угасает.
\vs Mar 9:45 И если нога твоя соблазняет тебя, отсеки ее: лучше тебе войти в жизнь хромому, нежели с двумя ногами быть ввержену в геенну, в огонь неугасимый,
\vs Mar 9:46 где червь их не умирает и огонь не угасает.
\vs Mar 9:47 И если глаз твой соблазняет тебя, вырви его: лучше тебе с одним глазом войти в Царствие Божие, нежели с двумя глазами быть ввержену в геенну огненную,
\vs Mar 9:48 где червь их не умирает и огонь не угасает.
\vs Mar 9:49 Ибо всякий огнем осолится, и всякая жертва солью осолится.
\vs Mar 9:50 Соль~--- добрая \bibemph{вещь}; но ежели соль не солона будет, чем вы ее поправите? Имейте в себе соль, и мир имейте между собою.
\vs Mar 10:1 Отправившись оттуда, приходит в пределы Иудейские за Иорданскою стороною. Опять собирается к Нему народ, и, по обычаю Своему, Он опять учил их.
\vs Mar 10:2 Подошли фарисеи и спросили, искушая Его: позволительно ли разводиться мужу с женою?
\vs Mar 10:3 Он сказал им в ответ: что заповедал вам Моисей?
\vs Mar 10:4 Они сказали: Моисей позволил писать разводное письмо и разводиться.
\vs Mar 10:5 Иисус сказал им в ответ: по жестокосердию вашему он написал вам сию заповедь.
\vs Mar 10:6 В начале же создания, Бог мужчину и женщину сотворил их.
\vs Mar 10:7 Посему оставит человек отца своего и мать
\vs Mar 10:8 и прилепится к жене своей, и будут два одною плотью; так что они уже не двое, но одна плоть.
\vs Mar 10:9 Итак, что Бог сочетал, того человек да не разлучает.
\vs Mar 10:10 В доме ученики Его опять спросили Его о том же.
\vs Mar 10:11 Он сказал им: кто разведется с женою своею и женится на другой, тот прелюбодействует от нее;
\vs Mar 10:12 и если жена разведется с мужем своим и выйдет за другого, прелюбодействует.
\rsbpar\vs Mar 10:13 Приносили к Нему детей, чтобы Он прикоснулся к ним; ученики же не допускали приносящих.
\vs Mar 10:14 Увидев \bibemph{то}, Иисус вознегодовал и сказал им: пустите детей приходить ко Мне и не препятствуйте им, ибо таковых есть Царствие Божие.
\vs Mar 10:15 Истинно говорю вам: кто не примет Царствия Божия, как дитя, тот не войдет в него.
\vs Mar 10:16 И, обняв их, возложил руки на них и благословил их.
\rsbpar\vs Mar 10:17 Когда выходил Он в путь, подбежал некто, пал пред Ним на колени и спросил Его: Учитель благий! что мне делать, чтобы наследовать жизнь вечную?
\vs Mar 10:18 Иисус сказал ему: что ты называешь Меня благим? Никто не благ, как только один Бог.
\vs Mar 10:19 Знаешь заповеди: не прелюбодействуй, не убивай, не кради, не лжесвидетельствуй, не обижай, почитай отца твоего и мать.
\vs Mar 10:20 Он же сказал Ему в ответ: Учитель! всё это сохранил я от юности моей.
\vs Mar 10:21 Иисус, взглянув на него, полюбил его и сказал ему: одного тебе недостает: пойди, всё, что имеешь, продай и раздай нищим, и будешь иметь сокровище на небесах; и приходи, последуй за Мною, взяв крест.
\vs Mar 10:22 Он же, смутившись от сего слова, отошел с печалью, потому что у него было большое имение.
\vs Mar 10:23 И, посмотрев вокруг, Иисус говорит ученикам Своим: как трудно имеющим богатство войти в Царствие Божие!
\vs Mar 10:24 Ученики ужаснулись от слов Его. Но Иисус опять говорит им в ответ: дети! как трудно надеющимся на богатство войти в Царствие Божие!
\vs Mar 10:25 Удобнее верблюду пройти сквозь игольные уши, нежели богатому войти в Царствие Божие.
\vs Mar 10:26 Они же чрезвычайно изумлялись и говорили между собою: кто же может спастись?
\vs Mar 10:27 Иисус, воззрев на них, говорит: человекам это невозможно, но не Богу, ибо всё возможно Богу.
\rsbpar\vs Mar 10:28 И начал Петр говорить Ему: вот, мы оставили всё и последовали за Тобою.
\vs Mar 10:29 Иисус сказал в ответ: истинно говорю вам: нет никого, кто оставил бы дом, или братьев, или сестер, или отца, или мать, или жену, или детей, или з\acc{е}мли, ради Меня и Евангелия,
\vs Mar 10:30 и не получил бы ныне, во время сие, среди гонений, во сто крат более домов, и братьев, и сестер, и отцов, и матерей, и детей, и земель, а в веке грядущем жизни вечной.
\vs Mar 10:31 Многие же будут первые последними, и последние первыми.
\rsbpar\vs Mar 10:32 Когда были они на пути, восходя в Иерусалим, Иисус шел впереди их, а они ужасались и, следуя за Ним, были в страхе. Подозвав двенадцать, Он опять начал им говорить о том, чт\acc{о} будет с Ним:
\vs Mar 10:33 вот, мы восходим в Иерусалим, и Сын Человеческий предан будет первосвященникам и книжникам, и осудят Его на смерть, и предадут Его язычникам,
\vs Mar 10:34 и поругаются над Ним, и будут бить Его, и оплюют Его, и убьют Его; и в третий день воскреснет.
\vs Mar 10:35 \bibemph{Тогда} подошли к Нему сыновья Зеведеевы Иаков и Иоанн и сказали: Учитель! мы желаем, чтобы Ты сделал нам, о чем попросим.
\vs Mar 10:36 Он сказал им: что хотите, чтобы Я сделал вам?
\vs Mar 10:37 Они сказали Ему: дай нам сесть у Тебя, одному по правую сторону, а другому по левую в славе Твоей.
\vs Mar 10:38 Но Иисус сказал им: не знаете, чего просите. Можете ли пить чашу, которую Я пью, и креститься крещением, которым Я крещусь?
\vs Mar 10:39 Они отвечали: можем. Иисус же сказал им: чашу, которую Я пью, будете пить, и крещением, которым Я крещусь, будете креститься;
\vs Mar 10:40 а дать сесть у Меня по правую сторону и по левую~--- не от Меня \bibemph{зависит}, но кому уготовано.
\vs Mar 10:41 И, услышав, десять начали негодовать на Иакова и Иоанна.
\vs Mar 10:42 Иисус же, подозвав их, сказал им: вы знаете, что почитающиеся князьями народов господствуют над ними, и вельможи их властвуют ими.
\vs Mar 10:43 Но между вами да не будет так: а кто хочет быть б\acc{о}льшим между вами, да будем вам слугою;
\vs Mar 10:44 и кто хочет быть первым между вами, да будет всем рабом.
\vs Mar 10:45 Ибо и Сын Человеческий не для того пришел, чтобы Ему служили, но чтобы послужить и отдать душу Свою для искупления многих.
\rsbpar\vs Mar 10:46 Приходят в Иерихон. И когда выходил Он из Иерихона с учениками Своими и множеством народа, Вартимей, сын Тимеев, слепой сидел у дороги, прося \bibemph{милостыни}.
\vs Mar 10:47 Услышав, что это Иисус Назорей, он начал кричать и говорить: Иисус, Сын Давидов! помилуй меня.
\vs Mar 10:48 Многие заставляли его молчать; но он еще более стал кричать: Сын Давидов! помилуй меня.
\vs Mar 10:49 Иисус остановился и велел его позвать. Зовут слепого и говорят ему: не бойся, вставай, зовет тебя.
\vs Mar 10:50 Он сбросил с себя верхнюю одежду, встал и пришел к Иисусу.
\vs Mar 10:51 Отвечая ему, Иисус спросил: чего ты хочешь от Меня? Слепой сказал Ему: Учитель! чтобы мне прозреть.
\vs Mar 10:52 Иисус сказал ему: иди, вера твоя спасла тебя. И он тотчас прозрел и пошел за Иисусом по дороге.
\vs Mar 11:1 Когда приблизились к Иерусалиму, к Виффагии и Вифании, к горе Елеонской, \bibemph{Иисус} посылает двух из учеников Своих
\vs Mar 11:2 и говорит им: пойдите в селение, которое прямо перед вами; входя в него, тотчас найдете привязанного молодого осла, на которого никто из людей не садился; отвязав его, приведите.
\vs Mar 11:3 И если кто скажет вам: что вы это делаете?~--- отвечайте, что он надобен Господу; и тотчас пошлет его сюда.
\vs Mar 11:4 Они пошли, и нашли молодого осла, привязанного у ворот на улице, и отвязали его.
\vs Mar 11:5 И некоторые из стоявших там говорили им: что делаете? \bibemph{зачем} отвязываете осленка?
\vs Mar 11:6 Они отвечали им, к\acc{а}к повелел Иисус; и те отпустили их.
\vs Mar 11:7 И привели осленка к Иисусу, и возложили на него одежды свои; \bibemph{Иисус} сел на него.
\vs Mar 11:8 Многие же постилали одежды свои по дороге; а другие резали ветви с дерев и постилали по дороге.
\vs Mar 11:9 И предшествовавшие и сопровождавшие восклицали: осанна! благословен Грядущий во имя Господне!
\vs Mar 11:10 благословенно грядущее во имя Господа царство отца нашего Давида! осанна в вышних!
\rsbpar\vs Mar 11:11 И вошел Иисус в Иерусалим и в храм; и, осмотрев всё, как время уже было позднее, вышел в Вифанию с двенадцатью.
\rsbpar\vs Mar 11:12 На другой день, когда они вышли из Вифании, Он взалкал;
\vs Mar 11:13 и, увидев издалека смоковницу, покрытую листьями, пошел, не найдет ли чего на ней; но, придя к ней, ничего не нашел, кроме листьев, ибо еще не время было \bibemph{собирания} смокв.
\vs Mar 11:14 И сказал ей Иисус: отныне да не вкушает никто от тебя плода вовек! И слышали т\acc{о} ученики Его.
\vs Mar 11:15 Пришли в Иерусалим. Иисус, войдя в храм, начал выгонять продающих и покупающих в храме; и столы меновщиков и скамьи продающих голубей опрокинул;
\vs Mar 11:16 и не позволял, чтобы кто пронес через храм какую-либо вещь.
\vs Mar 11:17 И учил их, говоря: не написано ли: дом Мой домом молитвы наречется для всех народов? а вы сделали его вертепом разбойников.
\vs Mar 11:18 Услышали \bibemph{это} книжники и первосвященники, и искали, как бы погубить Его, ибо боялись Его, потому что весь народ удивлялся учению Его.
\vs Mar 11:19 Когда же стало поздно, Он вышел вон из города.
\rsbpar\vs Mar 11:20 Поутру, проходя мимо, увидели, что смоковница засохла до корня.
\vs Mar 11:21 И, вспомнив, Петр говорит Ему: Равв\acc{и}! посмотри, смоковница, которую Ты проклял, засохла.
\vs Mar 11:22 Иисус, отвечая, говорит им:
\vs Mar 11:23 имейте веру Божию, ибо истинно говорю вам, если кто скажет горе сей: поднимись и ввергнись в море, и не усомнится в сердце своем, но поверит, что сбудется по словам его,~--- будет ему, что ни скажет.
\vs Mar 11:24 Потому говорю вам: всё, чего ни будете просить в молитве, верьте, что получите,~--- и будет вам.
\vs Mar 11:25 И когда стоите на молитве, прощайте, если чт\acc{о} имеете на кого, дабы и Отец ваш Небесный простил вам согрешения ваши.
\vs Mar 11:26 Если же не прощаете, то и Отец ваш Небесный не простит вам согрешений ваших.
\rsbpar\vs Mar 11:27 Пришли опять в Иерусалим. И когда Он ходил в храме, подошли к Нему первосвященники и книжники, и старейшины
\vs Mar 11:28 и говорили Ему: какою властью Ты это делаешь? и кто Тебе дал власть делать это?
\vs Mar 11:29 Иисус сказал им в ответ: спрошу и Я вас об одном, отвечайте Мне; \bibemph{тогда} и Я скажу вам, какою властью это делаю.
\vs Mar 11:30 Крещение Иоанново с небес было, или от человеков? отвечайте Мне.
\vs Mar 11:31 Они рассуждали между собою: если скажем: с небес,~--- то Он скажет: почему же вы не поверили ему?
\vs Mar 11:32 а сказать: от человеков~--- боялись народа, потому что все полагали, что Иоанн точно был пророк.
\vs Mar 11:33 И сказали в ответ Иисусу: не знаем. Тогда Иисус сказал им в ответ: и Я не скажу вам, какою властью это делаю.
\vs Mar 12:1 И начал говорить им притчами: некоторый человек насадил виноградник и обнес оградою, и выкопал точило, и построил башню, и, отдав его виноградарям, отлучился.
\vs Mar 12:2 И послал в свое время к виноградарям слугу~--- принять от виноградарей плодов из виноградника.
\vs Mar 12:3 Они же, схватив его, били, и отослали ни с чем.
\vs Mar 12:4 Опять послал к ним другого слугу; и тому камнями разбили голову и отпустили его с бесчестьем.
\vs Mar 12:5 И опять иного послал: и того убили; и многих других то били, то убивали.
\vs Mar 12:6 Имея же еще одного сына, любезного ему, напоследок послал и его к ним, говоря: постыдятся сына моего.
\vs Mar 12:7 Но виноградари сказали друг другу: это наследник; пойдем, убьем его, и наследство будет наше.
\vs Mar 12:8 И, схватив его, убили и выбросили вон из виноградника.
\vs Mar 12:9 Что же сделает хозяин виноградника?~--- Придет и предаст смерти виноградарей, и отдаст виноградник другим.
\vs Mar 12:10 Неужели вы не читали сего в Писании: камень, который отвергли строители, тот самый сделался главою угла;
\vs Mar 12:11 это от Господа, и есть дивно в очах наших.
\vs Mar 12:12 И старались схватить Его, но побоялись народа, ибо поняли, что о них сказал притчу; и, оставив Его, отошли.
\rsbpar\vs Mar 12:13 И посылают к Нему некоторых из фарисеев и иродиан, чтобы уловить Его в слове.
\vs Mar 12:14 Они же, придя, говорят Ему: Учитель! мы знаем, что Ты справедлив и не заботишься об угождении кому-либо, ибо не смотришь ни на какое лице, но истинно пути Божию учишь. Позволительно ли давать п\acc{о}дать кесарю или нет? давать ли нам или не давать?
\vs Mar 12:15 Но Он, зная их лицемерие, сказал им: что искушаете Меня? принесите Мне динарий, чтобы Мне видеть его.
\vs Mar 12:16 Они принесли. Тогда говорит им: чье это изображение и надпись? Они сказали Ему: кесаревы.
\vs Mar 12:17 Иисус сказал им в ответ: отдавайте кесарево кесарю, а Божие Богу. И дивились Ему.
\rsbpar\vs Mar 12:18 Потом пришли к Нему саддукеи, которые говорят, что нет воскресения, и спросили Его, говоря:
\vs Mar 12:19 Учитель! Моисей написал нам: если у кого умрет брат и оставит жену, а детей не оставит, то брат его пусть возьмет жену его и восстановит семя брату своему.
\vs Mar 12:20 Было семь братьев: первый взял жену и, умирая, не оставил детей.
\vs Mar 12:21 Взял ее второй и умер, и он не оставил детей; также и третий.
\vs Mar 12:22 Брали ее \bibemph{за себя} семеро и не оставили детей. После всех умерла и жена.
\vs Mar 12:23 Итак, в воскресении, когда воскреснут, которого из них будет она женою? Ибо семеро имели ее женою.
\vs Mar 12:24 Иисус сказал им в ответ: этим ли приводитесь вы в заблуждение, не зная Писаний, ни силы Божией?
\vs Mar 12:25 Ибо, когда из мертвых воскреснут, \bibemph{тогда} не будут ни жениться, ни замуж выходить, но будут, как Ангелы на небесах.
\vs Mar 12:26 А о мертвых, что они воскреснут, разве не читали вы в книге Моисея, как Бог при купине сказал ему: Я Бог Авраама, и Бог Исаака, и Бог Иакова?
\vs Mar 12:27 \bibemph{Бог} не есть Бог мертвых, но Бог живых. Итак, вы весьма заблуждаетесь.
\rsbpar\vs Mar 12:28 Один из книжников, слыша их прения и видя, что \bibemph{Иисус} хорошо им отвечал, подошел и спросил Его: какая первая из всех заповедей?
\vs Mar 12:29 Иисус отвечал ему: первая из всех заповедей: слушай, Израиль! Господь Бог наш есть Господь единый;
\vs Mar 12:30 и возлюби Господа Бога твоего всем сердцем твоим, и всею душею твоею, и всем разумением твоим, и всею крепостию твоею,~--- вот первая заповедь!
\vs Mar 12:31 Вторая подобная ей: возлюби ближнего твоего, как самого себя. Иной большей сих заповеди нет.
\vs Mar 12:32 Книжник сказал Ему: хорошо, Учитель! истину сказал Ты, что один есть Бог и нет иного, кроме Его;
\vs Mar 12:33 и любить Его всем сердцем и всем умом, и всею душею, и всею крепостью, и любить ближнего, как самого себя, есть больше всех всесожжений и жертв.
\vs Mar 12:34 Иисус, видя, что он разумно отвечал, сказал ему: недалеко ты от Царствия Божия. После того никто уже не смел спрашивать Его.
\rsbpar\vs Mar 12:35 Продолжая учить в храме, Иисус говорил: как говорят книжники, что Христос есть Сын Давидов?
\vs Mar 12:36 Ибо сам Давид сказал Духом Святым: сказал Господь Господу моему: седи одесную Меня, доколе положу врагов Твоих в подножие ног Твоих.
\vs Mar 12:37 Итак, сам Давид называет Его Господом: как же Он Сын ему? И множество народа слушало Его с услаждением.
\vs Mar 12:38 И говорил им в учении Своем: остерегайтесь книжников, любящих ходить в длинных одеждах и \bibemph{принимать} приветствия в народных собраниях,
\vs Mar 12:39 сидеть впереди в синагогах и возлежать на первом \bibemph{месте} на пиршествах,~---
\vs Mar 12:40 сии, поядающие домы вдов и напоказ долго молящиеся, примут тягчайшее осуждение.
\rsbpar\vs Mar 12:41 И сел Иисус против сокровищницы и смотрел, как народ кладет деньги в сокровищницу. Многие богатые клали много.
\vs Mar 12:42 Придя же, одна бедная вдова положила две лепты, что составляет кодрант.
\vs Mar 12:43 Подозвав учеников Своих, \bibemph{Иисус} сказал им: истинно говорю вам, что эта бедная вдова положила больше всех, клавших в сокровищницу,
\vs Mar 12:44 ибо все клали от избытка своего, а она от скудости своей положила всё, что имела, всё пропитание свое.
\vs Mar 13:1 И когда выходил Он из храма, говорит Ему один из учеников Его: Учитель! посмотри, какие камни и какие здания!
\vs Mar 13:2 Иисус сказал ему в ответ: видишь сии великие здания? всё это будет разрушено, так что не останется здесь камня на камне.
\vs Mar 13:3 И когда Он сидел на горе Елеонской против храма, спрашивали Его наедине Петр, и Иаков, и Иоанн, и Андрей:
\vs Mar 13:4 скажи нам, когда это будет, и какой признак, когда всё сие должно совершиться?
\vs Mar 13:5 Отвечая им, Иисус начал говорить: берегитесь, чтобы кто не прельстил вас,
\vs Mar 13:6 ибо многие придут под именем Моим и будут говорить, что это Я; и многих прельстят.
\vs Mar 13:7 Когда же услышите о войнах и о военных слухах, не ужасайтесь: ибо надлежит \bibemph{сему} быть,~--- но \bibemph{это} еще не конец.
\vs Mar 13:8 Ибо восстанет народ на народ и царство на царство; и будут землетрясения по местам, и будут глады и смятения. Это~--- начало болезней.
\vs Mar 13:9 Но вы смотр\acc{и}те за собою, ибо вас будут предавать в судилища и бить в синагогах, и перед правителями и царями поставят вас за Меня, для свидетельства перед ними.
\vs Mar 13:10 И во всех народах прежде должно быть проповедано Евангелие.
\vs Mar 13:11 Когда же поведут предавать вас, не заботьтесь наперед, чт\acc{о} вам говорить, и не обдумывайте; но чт\acc{о} дано будет вам в тот час, т\acc{о} и говорите, ибо не вы будете говорить, но Дух Святый.
\vs Mar 13:12 Предаст же брат брата на смерть, и отец~--- детей; и восстанут дети на родителей и умертвят их.
\vs Mar 13:13 И будете ненавидимы всеми за имя Мое; претерпевший же до конца спасется.
\vs Mar 13:14 Когда же увидите мерзость запустения, реченную пророком Даниилом, стоящую, где не должно,~--- читающий да разумеет,~--- тогда находящиеся в Иудее да бегут в горы;
\vs Mar 13:15 а кто на кровле, тот не сходи в дом и не входи взять что-нибудь из дома своего;
\vs Mar 13:16 и кто на поле, не обращайся назад взять одежду свою.
\vs Mar 13:17 Горе беременным и питающим сосцами в те дни.
\vs Mar 13:18 Мол\acc{и}тесь, чтобы не случилось бегство ваше зимою.
\vs Mar 13:19 Ибо в те дни будет такая скорбь, какой не было от начала творения, которое сотворил Бог, даже доныне, и не будет.
\vs Mar 13:20 И если бы Господь не сократил тех дней, то не спаслась бы никакая плоть; но ради избранных, которых Он избрал, сократил те дни.
\vs Mar 13:21 Тогда, если кто вам скажет: вот, здесь Христос, или: вот, там,~--- не верьте.
\vs Mar 13:22 Ибо восстанут лжехристы и лжепророки и дадут знамения и чудеса, чтобы прельстить, если возможно, и избранных.
\vs Mar 13:23 Вы же берегитесь. Вот, Я наперед сказал вам всё.
\vs Mar 13:24 Но в те дни, после скорби той, солнце померкнет, и луна не даст света своего,
\vs Mar 13:25 и звезды спадут с неба, и силы небесные поколеблются.
\vs Mar 13:26 Тогда увидят Сына Человеческого, грядущего на облаках с силою многою и славою.
\vs Mar 13:27 И тогда Он пошлет Ангелов Своих и соберет избранных Своих от четырех ветров, от края земли до края неба.
\vs Mar 13:28 От смоковницы возьмите подобие: когда ветви ее становятся уже мягки и пускают листья, то знаете, что близко лето.
\vs Mar 13:29 Так и когда вы увидите т\acc{о} сбывающимся, знайте, что близко, при дверях.
\vs Mar 13:30 Истинно говорю вам: не прейдет род сей, как всё это будет.
\vs Mar 13:31 Небо и земля прейдут, но слова Мои не прейдут.
\vs Mar 13:32 О дне же том, или часе, никто не знает, ни Ангелы небесные, ни Сын, но только Отец.
\vs Mar 13:33 Смотрите, бодрствуйте, молитесь, ибо не знаете, когда наступит это время.
\vs Mar 13:34 Подобно как бы кто, отходя в путь и оставляя дом свой, дал слугам своим власть и каждому свое дело, и приказал привратнику бодрствовать.
\vs Mar 13:35 Итак бодрствуйте, ибо не знаете, когда придет хозяин дома: вечером, или в полночь, или в пение петухов, или поутру;
\vs Mar 13:36 чтобы, придя внезапно, не нашел вас спящими.
\vs Mar 13:37 А чт\acc{о} вам говорю, говорю всем: бодрствуйте.
\vs Mar 14:1 Через два дня \bibemph{надлежало} быть \bibemph{празднику} Пасхи и опресноков. И искали первосвященники и книжники, как бы взять Его хитростью и убить;
\vs Mar 14:2 но говорили: \bibemph{только} не в праздник, чтобы не произошло возмущения в народе.
\rsbpar\vs Mar 14:3 И когда был Он в Вифании, в доме Симона прокаженного, и возлежал,~--- пришла женщина с алавастровым сосудом мира из нарда чистого, драгоценного и, разбив сосуд, возлила Ему на голову.
\vs Mar 14:4 Некоторые же вознегодовали и говорили между собою: к чему сия трата мира?
\vs Mar 14:5 Ибо можно было бы продать его более нежели за триста динариев и раздать нищим. И роптали на нее.
\vs Mar 14:6 Но Иисус сказал: оставьте ее; чт\acc{о} ее смущаете? Она доброе дело сделала для Меня.
\vs Mar 14:7 Ибо нищих всегда имеете с собою и, когда захотите, можете им благотворить; а Меня не всегда имеете.
\vs Mar 14:8 Она сделала, чт\acc{о} могла: предварила помазать тело Мое к погребению.
\vs Mar 14:9 Истинно говорю вам: где ни будет проповедано Евангелие сие в целом мире, сказано будет, в память ее, и о том, чт\acc{о} она сделала.
\rsbpar\vs Mar 14:10 И пошел Иуда Искариот, один из двенадцати, к первосвященникам, чтобы предать Его им.
\vs Mar 14:11 Они же, услышав, обрадовались, и обещали дать ему сребреники. И он искал, как бы в удобное время предать Его.
\rsbpar\vs Mar 14:12 В первый день опресноков, когда заколали пасхального \bibemph{агнца}, говорят Ему ученики Его: где хочешь есть пасху? мы пойдем и приготовим.
\vs Mar 14:13 И посылает двух из учеников Своих и говорит им: пойдите в город; и встретится вам человек, несущий кувшин воды; последуйте за ним
\vs Mar 14:14 и куда он войдет, скажите хозяину дома того: Учитель говорит: где комната, в которой бы Мне есть пасху с учениками Моими?
\vs Mar 14:15 И он покажет вам горницу большую, устланную, готовую: там приготовьте нам.
\vs Mar 14:16 И пошли ученики Его, и пришли в город, и нашли, как сказал им; и приготовили пасху.
\vs Mar 14:17 Когда настал вечер, Он приходит с двенадцатью.
\vs Mar 14:18 И, когда они возлежали и ели, Иисус сказал: истинно говорю вам, один из вас, ядущий со Мною, предаст Меня.
\vs Mar 14:19 Они опечалились и стали говорить Ему, один за другим: не я ли? и другой: не я ли?
\vs Mar 14:20 Он же сказал им в ответ: один из двенадцати, обмакивающий со Мною в блюдо.
\vs Mar 14:21 Впрочем Сын Человеческий идет, как писано о Нем; но горе тому человеку, которым Сын Человеческий предается: лучше было бы тому человеку не родиться.
\rsbpar\vs Mar 14:22 И когда они ели, Иисус, взяв хлеб, благословил, преломил, дал им и сказал: приимите, ядите; сие есть Тело Мое.
\vs Mar 14:23 И, взяв чашу, благодарив, подал им: и пили из нее все.
\vs Mar 14:24 И сказал им: сие есть Кровь Моя Нового Завета, за многих изливаемая.
\vs Mar 14:25 Истинно говорю вам: Я уже не буду пить от плода виноградного до того дня, когда буду пить новое вино в Царствии Божием.
\rsbpar\vs Mar 14:26 И, воспев, пошли на гору Елеонскую.
\vs Mar 14:27 И говорит им Иисус: все вы соблазнитесь о Мне в эту ночь; ибо написано: поражу пастыря, и рассеются овцы.
\vs Mar 14:28 По воскресении же Моем, Я предварю вас в Галилее.
\vs Mar 14:29 Петр сказал Ему: если и все соблазнятся, но не я.
\vs Mar 14:30 И говорит ему Иисус: истинно говорю тебе, что ты ныне, в эту ночь, прежде нежели дважды пропоет петух, трижды отречешься от Меня.
\vs Mar 14:31 Но он еще с б\acc{о}льшим усилием говорил: хотя бы мне надлежало и умереть с Тобою, не отрекусь от Тебя. Т\acc{о} же и все говорили.
\rsbpar\vs Mar 14:32 Пришли в селение, называемое Гефсимания; и Он сказал ученикам Своим: посидите здесь, пока Я помолюсь.
\vs Mar 14:33 И взял с Собою Петра, Иакова и Иоанна; и начал ужасаться и тосковать.
\vs Mar 14:34 И сказал им: душа Моя скорбит смертельно; побудьте здесь и бодрствуйте.
\vs Mar 14:35 И, отойдя немного, пал на землю и молился, чтобы, если возможно, миновал Его час сей;
\vs Mar 14:36 и говорил: Авва Отче! всё возможно Тебе; пронеси чашу сию мимо Меня; но не чего Я хочу, а чего Ты.
\vs Mar 14:37 Возвращается и находит их спящими, и говорит Петру: Симон! ты спишь? не мог ты бодрствовать один час?
\vs Mar 14:38 Бодрствуйте и молитесь, чтобы не впасть в искушение: дух бодр, плоть же немощна.
\vs Mar 14:39 И, опять отойдя, молился, сказав то же слово.
\vs Mar 14:40 И, возвратившись, опять нашел их спящими, ибо глаза у них отяжелели, и они не знали, чт\acc{о} Ему отвечать.
\vs Mar 14:41 И приходит в третий раз и говорит им: вы всё еще спите и почиваете? Кончено, пришел час: вот, предается Сын Человеческий в руки грешников.
\vs Mar 14:42 Встаньте, пойдем; вот, приблизился предающий Меня.
\rsbpar\vs Mar 14:43 И тотчас, как Он еще говорил, приходит Иуда, один из двенадцати, и с ним множество народа с мечами и кольями, от первосвященников и книжников и старейшин.
\vs Mar 14:44 Предающий же Его дал им знак, сказав: Кого я поцелую, Тот и есть, возьмите Его и ведите осторожно.
\vs Mar 14:45 И, придя, тотчас подошел к Нему и говорит: Равв\acc{и}! Равв\acc{и}! и поцеловал Его.
\vs Mar 14:46 А они возложили на Него руки свои и взяли Его.
\vs Mar 14:47 Один же из стоявших тут извлек меч, ударил раба первосвященникова и отсек ему ухо.
\vs Mar 14:48 Тогда Иисус сказал им: как будто на разбойника вышли вы с мечами и кольями, чтобы взять Меня.
\vs Mar 14:49 Каждый день бывал Я с вами в храме и учил, и вы не брали Меня. Но да сбудутся Писания.
\vs Mar 14:50 Тогда, оставив Его, все бежали.
\vs Mar 14:51 Один юноша, завернувшись по нагому телу в покрывало, следовал за Ним; и воины схватили его.
\vs Mar 14:52 Но он, оставив покрывало, нагой убежал от них.
\rsbpar\vs Mar 14:53 И привели Иисуса к первосвященнику; и собрались к нему все первосвященники и старейшины и книжники.
\vs Mar 14:54 Петр издали следовал за Ним, даже внутрь двора первосвященникова; и сидел со служителями, и грелся у огня.
\vs Mar 14:55 Первосвященники же и весь синедрион искали свидетельства на Иисуса, чтобы предать Его смерти; и не находили.
\vs Mar 14:56 Ибо многие лжесвидетельствовали на Него, но свидетельства сии не были достаточны.
\vs Mar 14:57 И некоторые, встав, лжесвидетельствовали против Него и говорили:
\vs Mar 14:58 мы слышали, как Он говорил: Я разрушу храм сей рукотворенный, и через три дня воздвигну другой, нерукотворенный.
\vs Mar 14:59 Но и такое свидетельство их не было достаточно.
\vs Mar 14:60 Тогда первосвященник стал посреди и спросил Иисуса: чт\acc{о} Ты ничего не отвечаешь? чт\acc{о} они против Тебя свидетельствуют?
\vs Mar 14:61 Но Он молчал и не отвечал ничего. Опять первосвященник спросил Его и сказал Ему: Ты ли Христос, Сын Благословенного?
\vs Mar 14:62 Иисус сказал: Я; и вы \acc{у}зрите Сына Человеческого, сидящего одесную силы и грядущего на облаках небесных.
\vs Mar 14:63 Тогда первосвященник, разодрав одежды свои, сказал: на что еще нам свидетелей?
\vs Mar 14:64 Вы слышали богохульство; как вам кажется? Они же все признали Его повинным смерти.
\vs Mar 14:65 И некоторые начали плевать на Него и, закрывая Ему лице, ударять Его и говорить Ему: прореки. И слуги били Его по ланитам.
\rsbpar\vs Mar 14:66 Когда Петр был на дворе внизу, пришла одна из служанок первосвященника
\vs Mar 14:67 и, увидев Петра греющегося и всмотревшись в него, сказала: и ты был с Иисусом Назарянином.
\vs Mar 14:68 Но он отрекся, сказав: не знаю и не понимаю, что ты говоришь. И вышел вон на передний двор; и запел петух.
\vs Mar 14:69 Служанка, увидев его опять, начала говорить стоявшим тут: этот из них.
\vs Mar 14:70 Он опять отрекся. Спустя немного, стоявшие тут опять стали говорить Петру: точно ты из них; ибо ты Галилеянин, и наречие твое сходно.
\vs Mar 14:71 Он же начал клясться и божиться: не знаю Человека Сего, о Котором говорите.
\vs Mar 14:72 Тогда петух запел во второй раз. И вспомнил Петр слово, сказанное ему Иисусом: прежде нежели петух пропоет дважды, трижды отречешься от Меня; и начал плакать.
\vs Mar 15:1 Немедленно поутру первосвященники со старейшинами и книжниками и весь синедрион составили совещание и, связав Иисуса, отвели и предали Пилату.
\vs Mar 15:2 Пилат спросил Его: Ты Царь Иудейский? Он же сказал ему в ответ: ты говоришь.
\vs Mar 15:3 И первосвященники обвиняли Его во многом.
\vs Mar 15:4 Пилат же опять спросил Его: Ты ничего не отвечаешь? видишь, как много против Тебя обвинений.
\vs Mar 15:5 Но Иисус и на это ничего не отвечал, так что Пилат дивился.
\vs Mar 15:6 На всякий же праздник отпускал он им одного узника, о котором просили.
\vs Mar 15:7 Тогда был в узах \bibemph{некто}, по имени Варавва, со своими сообщниками, которые во время мятежа сделали убийство.
\vs Mar 15:8 И народ начал кричать и просить \bibemph{Пилата} о том, чт\acc{о} он всегда делал для них.
\vs Mar 15:9 Он сказал им в ответ: хотите ли, отпущу вам Царя Иудейского?
\vs Mar 15:10 Ибо знал, что первосвященники предали Его из зависти.
\vs Mar 15:11 Но первосвященники возбудили народ \bibemph{просить}, чтобы отпустил им лучше Варавву.
\vs Mar 15:12 Пилат, отвечая, опять сказал им: что же хотите, чтобы я сделал с Тем, Которого вы называете Царем Иудейским?
\vs Mar 15:13 Они опять закричали: распни Его.
\vs Mar 15:14 Пилат сказал им: какое же зло сделал Он? Но они еще сильнее закричали: распни Его.
\vs Mar 15:15 Тогда Пилат, желая сделать угодное народу, отпустил им Варавву, а Иисуса, бив, предал на распятие.
\rsbpar\vs Mar 15:16 А воины отвели Его внутрь двора, то есть в преторию, и собрали весь полк,
\vs Mar 15:17 и одели Его в багряницу, и, сплетши терновый венец, возложили на Него;
\vs Mar 15:18 и начали приветствовать Его: радуйся, Царь Иудейский!
\vs Mar 15:19 И били Его по голове тростью, и плевали на Него, и, становясь на колени, кланялись Ему.
\rsbpar\vs Mar 15:20 Когда же насмеялись над Ним, сняли с Него багряницу, одели Его в собственные одежды Его и повели Его, чтобы распять Его.
\vs Mar 15:21 И заставили проходящего некоего Киринеянина Симона, отца Александрова и Руфова, идущего с поля, нести крест Его.
\vs Mar 15:22 И привели Его на место Голгофу, чт\acc{о} значит: Лобное место.
\vs Mar 15:23 И давали Ему пить вино со смирною; но Он не принял.
\vs Mar 15:24 Распявшие Его делили одежды Его, бросая жребий, кому чт\acc{о} взять.
\vs Mar 15:25 Был час третий, и распяли Его.
\vs Mar 15:26 И была надпись вины Его: Царь Иудейский.
\vs Mar 15:27 С Ним распяли двух разбойников, одного по правую, а другого по левую \bibemph{сторону} Его.
\vs Mar 15:28 И сбылось слово Писания: и к злодеям причтен.
\vs Mar 15:29 Проходящие злословили Его, кивая головами своими и говоря: э! разрушающий храм, и в три дня созидающий!
\vs Mar 15:30 спаси Себя Самого и сойди со креста.
\vs Mar 15:31 Подобно и первосвященники с книжниками, насмехаясь, говорили друг другу: других спасал, а Себя не может спасти.
\vs Mar 15:32 Христос, Царь Израилев, пусть сойдет теперь с креста, чтобы мы видели, и уверуем. И распятые с Ним поносили Его.
\rsbpar\vs Mar 15:33 В шестом же часу настала тьма по всей земле и \bibemph{продолжалась} до часа девятого.
\vs Mar 15:34 В девятом часу возопил Иисус громким голосом: Эло\acc{и}! Эло\acc{и}! ламм\acc{а} савахфан\acc{и}?~--- что значит: Боже Мой! Боже Мой! для чего Ты Меня оставил?
\vs Mar 15:35 Некоторые из стоявших тут, услышав, говорили: вот, Илию зовет.
\vs Mar 15:36 А один побежал, наполнил губку уксусом и, наложив на трость, давал Ему пить, говоря: постойте, посмотрим, придет ли Илия снять Его.
\vs Mar 15:37 Иисус же, возгласив громко, испустил дух.
\vs Mar 15:38 И завеса в храме раздралась надвое, сверху донизу.
\vs Mar 15:39 Сотник, стоявший напротив Его, увидев, что Он, т\acc{а}к возгласив, испустил дух, сказал: истинно Человек Сей был Сын Божий.
\vs Mar 15:40 Были \bibemph{тут} и женщины, которые смотрели издали: между ними была и Мария Магдалина, и Мария, мать Иакова меньшего и Иосии, и Саломия,
\vs Mar 15:41 которые и тогда, как Он был в Галилее, следовали за Ним и служили Ему, и другие многие, вместе с Ним пришедшие в Иерусалим.
\rsbpar\vs Mar 15:42 И как уже настал вечер,~--- потому что была пятница, то есть \bibemph{день} перед субботою,~---
\vs Mar 15:43 пришел Иосиф из Аримафеи, знаменитый член совета, который и сам ожидал Царствия Божия, осмелился войти к Пилату, и просил тела Иисусова.
\vs Mar 15:44 Пилат удивился, что Он уже умер, и, призвав сотника, спросил его, давно ли умер?
\vs Mar 15:45 И, узнав от сотника, отдал тело Иосифу.
\vs Mar 15:46 Он, купив плащаницу и сняв Его, обвил плащаницею, и положил Его во гробе, который был высечен в скале, и привалил камень к двери гроба.
\vs Mar 15:47 Мария же Магдалина и Мария Иосиева смотрели, где Его полагали.
\vs Mar 16:1 По прошествии субботы Мария Магдалина и Мария Иаковлева и Саломия купили ароматы, чтобы идти помазать Его.
\vs Mar 16:2 И весьма рано, в первый \bibemph{день} недели, приходят ко гробу, при восходе солнца,
\vs Mar 16:3 и говорят между собою: кто отвалит нам камень от двери гроба?
\vs Mar 16:4 И, взглянув, видят, что камень отвален; а он был весьма велик.
\vs Mar 16:5 И, войдя во гроб, увидели юношу, сидящего на правой стороне, облеченного в белую одежду; и ужаснулись.
\vs Mar 16:6 Он же говорит им: не ужасайтесь. Иисуса ищете Назарянина, распятого; Он воскрес, Его нет здесь. Вот место, где Он был положен.
\vs Mar 16:7 Но идите, скажите ученикам Его и Петру, что Он предваряет вас в Галилее; там Его увидите, как Он сказал вам.
\vs Mar 16:8 И, выйдя, побежали от гроба; их объял трепет и ужас, и никому ничего не сказали, потому что боялись.
\rsbpar\vs Mar 16:9 Воскреснув рано в первый \bibemph{день} недели, \bibemph{Иисус} явился сперва Марии Магдалине, из которой изгнал семь бесов.
\vs Mar 16:10 Она пошла и возвестила бывшим с Ним, плачущим и рыдающим;
\vs Mar 16:11 но они, услышав, что Он жив и она видела Его,~--- не поверили.
\rsbpar\vs Mar 16:12 После сего явился в ином образе двум из них на дороге, когда они шли в селение.
\vs Mar 16:13 И те, возвратившись, возвестили прочим; но и им не поверили.
\rsbpar\vs Mar 16:14 Наконец, явился самим одиннадцати, возлежавшим \bibemph{на вечери}, и упрекал их за неверие и жестокосердие, что видевшим Его воскресшего не поверили.
\vs Mar 16:15 И сказал им: идите по всему миру и проповедуйте Евангелие всей твари.
\vs Mar 16:16 Кто будет веровать и креститься, спасен будет; а кто не будет веровать, осужден будет.
\vs Mar 16:17 Уверовавших же будут сопровождать сии знамения: именем Моим будут изгонять бесов; будут говорить новыми языками;
\vs Mar 16:18 будут брать змей; и если чт\acc{о} смертоносное выпьют, не повредит им; возложат руки на больных, и они будут здоровы.
\rsbpar\vs Mar 16:19 И так Господь, после беседования с ними, вознесся на небо и воссел одесную Бога.
\vs Mar 16:20 А они пошли и проповедовали везде, при Господнем содействии и подкреплении слова последующими знамениями. Аминь.

\bibbookdescr{Luk}{
  inline={От Луки\\\LARGE святое благовествование},
  toc={От Луки},
  bookmark={От Луки},
  header={От Луки},
  %headerleft={},
  %headerright={},
  abbr={Лк}
}
\vs Luk 1:1 Как уже многие начали составлять повествования о совершенно известных между нами событиях,
\vs Luk 1:2 как передали нам т\acc{о} бывшие с самого начала очевидцами и служителями Слова,
\vs Luk 1:3 то рассудилось и мне, по тщательном исследовании всего сначала, по порядку описать тебе, достопочтенный Феофил,
\vs Luk 1:4 чтобы ты узнал твердое основание того учения, в котором был наставлен.
\rsbpar\vs Luk 1:5 Во дни Ирода, царя Иудейского, был священник из Авиевой чреды, именем Захария, и жена его из рода Ааронова, имя ей Елисавета.
\vs Luk 1:6 Оба они были праведны пред Богом, поступая по всем заповедям и уставам Господним беспорочно.
\vs Luk 1:7 У них не было детей, ибо Елисавета была неплодна, и оба были уже в летах преклонных.
\vs Luk 1:8 Однажды, когда он в порядке своей чреды служил пред Богом,
\vs Luk 1:9 по жребию, как обыкновенно было у священников, досталось ему войти в храм Господень для каждения,
\vs Luk 1:10 а всё множество народа молилось вне во время каждения,~---
\vs Luk 1:11 тогда явился ему Ангел Господень, стоя по правую сторону жертвенника кадильного.
\vs Luk 1:12 Захария, увидев его, смутился, и страх напал на него.
\vs Luk 1:13 Ангел же сказал ему: не бойся, Захария, ибо услышана молитва твоя, и жена твоя Елисавета родит тебе сына, и наречешь ему имя: Иоанн;
\vs Luk 1:14 и будет тебе радость и веселие, и многие о рождении его возрадуются,
\vs Luk 1:15 ибо он будет велик пред Господом; не будет пить вина и сикера, и Духа Святаго исполнится еще от чрева матери своей;
\vs Luk 1:16 и многих из сынов Израилевых обратит к Господу Богу их;
\vs Luk 1:17 и предъидет пред Ним в духе и силе Илии, чтобы возвратить сердца отцов детям, и непокоривым образ мыслей праведников, дабы представить Господу народ приготовленный.
\vs Luk 1:18 И сказал Захария Ангелу: по чему я узн\acc{а}ю это? ибо я стар, и жена моя в летах преклонных.
\vs Luk 1:19 Ангел сказал ему в ответ: я Гавриил, предстоящий пред Богом, и послан говорить с тобою и благовестить тебе сие;
\vs Luk 1:20 и вот, ты будешь молчать и не будешь иметь возможности говорить до того дня, как это сбудется, за т\acc{о}, что ты не поверил словам моим, которые сбудутся в свое время.
\vs Luk 1:21 Между тем народ ожидал Захарию и дивился, что он медлит в храме.
\vs Luk 1:22 Он же, выйдя, не мог говорить к ним; и они поняли, что он видел видение в храме; и он объяснялся с ними знаками, и оставался нем.
\vs Luk 1:23 А когда окончились дни службы его, возвратился в дом свой.
\vs Luk 1:24 После сих дней зачала Елисавета, жена его, и таилась пять месяцев и говорила:
\vs Luk 1:25 так сотворил мне Господь во дни сии, в которые призрел на меня, чтобы снять с меня поношение между людьми.
\rsbpar\vs Luk 1:26 В шестой же месяц послан был Ангел Гавриил от Бога в город Галилейский, называемый Назарет,
\vs Luk 1:27 к Деве, обрученной мужу, именем Иосифу, из дома Давидова; имя же Деве: Мария.
\vs Luk 1:28 Ангел, войдя к Ней, сказал: радуйся, Благодатная! Господь с Тобою; благословенна Ты между женами.
\vs Luk 1:29 Она же, увидев его, смутилась от слов его и размышляла, чт\acc{о} бы это было за приветствие.
\vs Luk 1:30 И сказал Ей Ангел: не бойся, Мария, ибо Ты обрела благодать у Бога;
\vs Luk 1:31 и вот, зачнешь во чреве, и родишь Сына, и наречешь Ему имя: Иисус.
\vs Luk 1:32 Он будет велик и наречется Сыном Всевышнего, и даст Ему Господь Бог престол Давида, отца Его;
\vs Luk 1:33 и будет царствовать над домом Иакова во веки, и Царству Его не будет конца.
\vs Luk 1:34 Мария же сказала Ангелу: к\acc{а}к будет это, когда Я мужа не знаю?
\vs Luk 1:35 Ангел сказал Ей в ответ: Дух Святый найдет на Тебя, и сила Всевышнего осенит Тебя; посему и рождаемое Святое наречется Сыном Божиим.
\vs Luk 1:36 Вот и Елисавета, родственница Твоя, называемая неплодною, и она зачала сына в старости своей, и ей уже шестой месяц,
\vs Luk 1:37 ибо у Бога не останется бессильным никакое слово.
\vs Luk 1:38 Тогда Мария сказала: се, Раба Господня; да будет Мне по слову твоему. И отошел от Нее Ангел.
\rsbpar\vs Luk 1:39 Встав же Мария во дни сии, с поспешностью пошла в нагорную страну, в город Иудин,
\vs Luk 1:40 и вошла в дом Захарии, и приветствовала Елисавету.
\vs Luk 1:41 Когда Елисавета услышала приветствие Марии, взыграл младенец во чреве ее; и Елисавета исполнилась Святаго Духа,
\vs Luk 1:42 и воскликнула громким голосом, и сказала: благословенна Ты между женами, и благословен плод чрева Твоего!
\vs Luk 1:43 И откуда это мне, что пришла Матерь Господа моего ко мне?
\vs Luk 1:44 Ибо когда голос приветствия Твоего дошел до слуха моего, взыграл младенец радостно во чреве моем.
\vs Luk 1:45 И блаженна Уверовавшая, потому что совершится сказанное Ей от Господа.
\vs Luk 1:46 И сказала Мария: величит душа Моя Господа,
\vs Luk 1:47 и возрадовался дух Мой о Боге, Спасителе Моем,
\vs Luk 1:48 что призрел Он на смирение Рабы Своей, ибо отныне будут ублажать Меня все роды;
\vs Luk 1:49 что сотворил Мне величие Сильный, и свято имя Его;
\vs Luk 1:50 и милость Его в роды родов к боящимся Его;
\vs Luk 1:51 явил силу мышцы Своей; рассеял надменных помышлениями с\acc{е}рдца их;
\vs Luk 1:52 низложил сильных с престолов, и вознес смиренных;
\vs Luk 1:53 алчущих исполнил благ, и богатящихся отпустил ни с чем;
\vs Luk 1:54 воспринял Израиля, отрока Своего, воспомянув милость,
\vs Luk 1:55 к\acc{а}к говорил отцам нашим, к Аврааму и семени его до века.
\vs Luk 1:56 Пребыла же Мария с нею около трех месяцев, и возвратилась в дом свой.
\rsbpar\vs Luk 1:57 Елисавете же настало время родить, и она родила сына.
\vs Luk 1:58 И услышали соседи и родственники ее, что возвеличил Господь милость Свою над нею, и радовались с нею.
\vs Luk 1:59 В восьмой день пришли обрезать младенца и хотели назвать его, по имени отца его, Захариею.
\vs Luk 1:60 На это мать его сказала: нет, а назвать его Иоанном.
\vs Luk 1:61 И сказали ей: никого нет в родстве твоем, кто назывался бы сим именем.
\vs Luk 1:62 И спрашивали знаками у отца его, к\acc{а}к бы он хотел назвать его.
\vs Luk 1:63 Он потребовал дощечку и написал: Иоанн имя ему. И все удивились.
\vs Luk 1:64 И тотчас разрешились уста его и язык его, и он стал говорить, благословляя Бога.
\vs Luk 1:65 И был страх на всех живущих вокруг них; и рассказывали обо всем этом по всей нагорной стране Иудейской.
\vs Luk 1:66 Все слышавшие положили это на сердце своем и говорили: чт\acc{о} будет младенец сей? И рука Господня была с ним.
\vs Luk 1:67 И Захария, отец его, исполнился Святаго Духа и пророчествовал, говоря:
\vs Luk 1:68 благословен Господь Бог Израилев, что посетил народ Свой и сотворил избавление ему,
\vs Luk 1:69 и воздвиг рог спасения нам в дому Давида, отрока Своего,
\vs Luk 1:70 к\acc{а}к возвестил устами бывших от века святых пророков Своих,
\vs Luk 1:71 что спасет нас от врагов наших и от руки всех ненавидящих нас;
\vs Luk 1:72 сотворит милость с отцами нашими и помянет святой завет Свой,
\vs Luk 1:73 клятву, которою клялся Он Аврааму, отцу нашему, дать нам,
\vs Luk 1:74 небоязненно, по избавлении от руки врагов наших,
\vs Luk 1:75 служить Ему в святости и правде пред Ним, во все дни жизни нашей.
\vs Luk 1:76 И ты, младенец, наречешься пророком Всевышнего, ибо предъидешь пред лицем Господа приготовить пути Ему,
\vs Luk 1:77 дать уразуметь народу Его спасение в прощении грехов их,
\vs Luk 1:78 по благоутробному милосердию Бога нашего, которым посетил нас Восток свыше,
\vs Luk 1:79 просветить сидящих во тьме и тени смертной, направить ноги наши на путь мира.
\rsbpar\vs Luk 1:80 Младенец же возрастал и укреплялся духом, и был в пустынях до дня явления своего Израилю.
\vs Luk 2:1 В те дни вышло от кесаря Августа повеление сделать перепись по всей земле.
\vs Luk 2:2 Эта перепись была первая в правление Квириния Сириею.
\vs Luk 2:3 И пошли все записываться, каждый в свой город.
\vs Luk 2:4 Пошел также и Иосиф из Галилеи, из города Назарета, в Иудею, в город Давидов, называемый Вифлеем, потому что он был из дома и рода Давидова,
\vs Luk 2:5 записаться с Мариею, обрученною ему женою, которая была беременна.
\vs Luk 2:6 Когда же они были там, наступило время родить Ей;
\vs Luk 2:7 и родила Сына своего Первенца, и спеленала Его, и положила Его в ясли, потому что не было им места в гостинице.
\rsbpar\vs Luk 2:8 В той стране были на поле пастухи, которые содержали ночную стражу у стада своего.
\vs Luk 2:9 Вдруг предстал им Ангел Господень, и слава Господня осияла их; и убоялись страхом великим.
\vs Luk 2:10 И сказал им Ангел: не бойтесь; я возвещаю вам великую радость, которая будет всем людям:
\vs Luk 2:11 ибо ныне родился вам в городе Давидовом Спаситель, Который есть Христос Господь;
\vs Luk 2:12 и вот вам знак: вы найдете Младенца в пеленах, лежащего в яслях.
\vs Luk 2:13 И внезапно явилось с Ангелом многочисленное воинство небесное, славящее Бога и взывающее:
\vs Luk 2:14 слава в вышних Богу, и на земле мир, в человеках благоволение!
\vs Luk 2:15 Когда Ангелы отошли от них на небо, пастухи сказали друг другу: пойдем в Вифлеем и посмотрим, чт\acc{о} там случилось, о чем возвестил нам Господь.
\vs Luk 2:16 И, поспешив, пришли и нашли Марию и Иосифа, и Младенца, лежащего в яслях.
\vs Luk 2:17 Увидев же, рассказали о том, чт\acc{о} было возвещено им о Младенце Сем.
\vs Luk 2:18 И все слышавшие дивились тому, чт\acc{о} рассказывали им пастухи.
\vs Luk 2:19 А Мария сохраняла все слова сии, слагая в сердце Своем.
\vs Luk 2:20 И возвратились пастухи, славя и хваля Бога за всё т\acc{о}, что слышали и видели, к\acc{а}к им сказано было.
\rsbpar\vs Luk 2:21 По прошествии восьми дней, когда надлежало обрезать \bibemph{Младенца}, дали Ему имя Иисус, нареченное Ангелом прежде зачатия Его во чреве.
\rsbpar\vs Luk 2:22 А когда исполнились дни очищения их по закону Моисееву, принесли Его в Иерусалим, чтобы представить пред Господа,
\vs Luk 2:23 как предписано в законе Господнем, чтобы всякий младенец мужеского пола, разверзающий ложесна, был посвящен Господу,
\vs Luk 2:24 и чтобы принести в жертву, по реченному в законе Господнем, две горлицы или двух птенцов голубиных.
\vs Luk 2:25 Тогда был в Иерусалиме человек, именем Симеон. Он был муж праведный и благочестивый, чающий утешения Израилева; и Дух Святый был на нем.
\vs Luk 2:26 Ему было предсказано Духом Святым, что он не увидит смерти, доколе не увидит Христа Господня.
\vs Luk 2:27 И пришел он по вдохновению в храм. И, когда родители принесли Младенца Иисуса, чтобы совершить над Ним законный обряд,
\vs Luk 2:28 он взял Его на руки, благословил Бога и сказал:
\rsbpar\vs Luk 2:29 Ныне отпускаешь раба Твоего, Владыко, по слову Твоему, с миром,
\vs Luk 2:30 ибо видели очи мои спасение Твое,
\vs Luk 2:31 которое Ты уготовал пред лицем всех народов,
\vs Luk 2:32 свет к просвещению язычников и славу народа Твоего Израиля.
\rsbpar\vs Luk 2:33 Иосиф же и Матерь Его дивились сказанному о Нем.
\vs Luk 2:34 И благословил их Симеон и сказал Марии, Матери Его: се, лежит Сей на падение и на восстание многих в Израиле и в предмет пререканий,~---
\vs Luk 2:35 и Тебе Самой оружие пройдет душу,~--- да откроются помышления многих сердец.
\vs Luk 2:36 Тут была также Анна пророчица, дочь Фануилова, от колена Асирова, достигшая глубокой старости, прожив с мужем от девства своего семь лет,
\vs Luk 2:37 вдова лет восьмидесяти четырех, которая не отходила от храма, постом и молитвою служа Богу день и ночь.
\vs Luk 2:38 И она в то время, подойдя, славила Господа и говорила о Нем всем, ожидавшим избавления в Иерусалиме.
\rsbpar\vs Luk 2:39 И когда они совершили всё по закону Господню, возвратились в Галилею, в город свой Назарет.
\vs Luk 2:40 Младенец же возрастал и укреплялся духом, исполняясь премудрости, и благодать Божия была на Нем.
\vs Luk 2:41 Каждый год родители Его ходили в Иерусалим на праздник Пасхи.
\vs Luk 2:42 И когда Он был двенадцати лет, пришли они также по обычаю в Иерусалим на праздник.
\vs Luk 2:43 Когда же, по окончании дней \bibemph{праздника}, возвращались, остался Отрок Иисус в Иерусалиме; и не заметили того Иосиф и Матерь Его,
\vs Luk 2:44 но думали, что Он идет с другими. Пройдя же дневной путь, стали искать Его между родственниками и знакомыми
\vs Luk 2:45 и, не найдя Его, возвратились в Иерусалим, ища Его.
\vs Luk 2:46 Через три дня нашли Его в храме, сидящего посреди учителей, слушающего их и спрашивающего их;
\vs Luk 2:47 все слушавшие Его дивились разуму и ответам Его.
\vs Luk 2:48 И, увидев Его, удивились; и Матерь Его сказала Ему: Чадо! чт\acc{о} Ты сделал с нами? Вот, отец Твой и Я с великою скорбью искали Тебя.
\vs Luk 2:49 Он сказал им: зачем было вам искать Меня? или вы не знали, что Мне должно быть в том, чт\acc{о} принадлежит Отцу Моему?
\vs Luk 2:50 Но они не поняли сказанных Им слов.
\vs Luk 2:51 И Он пошел с ними и пришел в Назарет; и был в повиновении у них. И Матерь Его сохраняла все слова сии в сердце Своем.
\vs Luk 2:52 Иисус же преуспевал в премудрости и возрасте и в любви у Бога и человеков.
\vs Luk 3:1 В пятнадцатый же год правления Тиверия кесаря, когда Понтий Пилат начальствовал в Иудее, Ирод был четвертовластником в Галилее, Филипп, брат его, четвертовластником в Итурее и Трахонитской области, а Лисаний четвертовластником в Авилинее,
\vs Luk 3:2 при первосвященниках Анне и Каиафе, был глагол Божий к Иоанну, сыну Захарии, в пустыне.
\vs Luk 3:3 И он проходил по всей окрестной стране Иорданской, проповедуя крещение покаяния для прощения грехов,
\vs Luk 3:4 как написано в книге слов пророка Исаии, который говорит: глас вопиющего в пустыне: приготовьте путь Господу, прямыми сделайте стези Ему;
\vs Luk 3:5 всякий дол да наполнится, и всякая гора и холм да понизятся, кривизны выпрямятся и неровные пути сделаются гладкими;
\vs Luk 3:6 и узрит всякая плоть спасение Божие.
\vs Luk 3:7 \bibemph{Иоанн} приходившему креститься от него народу говорил: порождения ехиднины! кто внушил вам бежать от будущего гнева?
\vs Luk 3:8 Сотворите же достойные плоды покаяния и не думайте говорить в себе: отец у нас Авраам, ибо говорю вам, что Бог может из камней сих воздвигнуть детей Аврааму.
\vs Luk 3:9 Уже и секира при корне дерев лежит: всякое дерево, не приносящее доброго плода, срубают и бросают в огонь.
\vs Luk 3:10 И спрашивал его народ: что же нам делать?
\vs Luk 3:11 Он сказал им в ответ: у кого две одежды, тот дай неимущему, и у кого есть пища, делай то же.
\vs Luk 3:12 Пришли и мытари креститься, и сказали ему: учитель! что нам делать?
\vs Luk 3:13 Он отвечал им: ничего не требуйте более определенного вам.
\vs Luk 3:14 Спрашивали его также и воины: а нам что делать? И сказал им: никого не обижайте, не клевещите, и довольствуйтесь своим жалованьем.
\vs Luk 3:15 Когда же народ был в ожидании, и все помышляли в сердцах своих об Иоанне, не Христос ли он,~---
\vs Luk 3:16 Иоанн всем отвечал: я крещу вас водою, но идёт Сильнейший меня, у Которого я недостоин развязать ремень обуви; Он будет крестить вас Духом Святым и огнем.
\vs Luk 3:17 Лопата Его в руке Его, и Он очистит гумно Свое и соберет пшеницу в житницу Свою, а солому сожжет огнем неугасимым.
\vs Luk 3:18 Многое и другое благовествовал он народу, поучая его.
\rsbpar\vs Luk 3:19 Ирод же четвертовластник, обличаемый от него за Иродиаду, жену брата своего, и за всё, что сделал Ирод худого,
\vs Luk 3:20 прибавил ко всему прочему и т\acc{о}, что заключил Иоанна в темницу.
\rsbpar\vs Luk 3:21 Когда же крестился весь народ, и Иисус, крестившись, молился: отверзлось небо,
\vs Luk 3:22 и Дух Святый нисшел на Него в телесном виде, как голубь, и был глас с небес, глаголющий: Ты Сын Мой Возлюбленный; в Тебе Мое благоволение!
\rsbpar\vs Luk 3:23 Иисус, начиная \bibemph{Своё служение}, был лет тридцати, и был, как думали, Сын Иосифов, Илиев,
\vs Luk 3:24 Матфатов, Левиин, Мелхиев, Ианнаев, Иосифов,
\vs Luk 3:25 Маттафиев, Амосов, Наумов, Еслимов, Наггеев,
\vs Luk 3:26 Маафов, Маттафиев, Семеиев, Иосифов, Иудин,
\vs Luk 3:27 Иоаннанов, Рисаев, Зоровавелев, Салафиилев, Нириев,
\vs Luk 3:28 Мелхиев, Аддиев, Косамов, Елмодамов, Иров,
\vs Luk 3:29 Иосиев, Елиезеров, Иоримов, Матфатов, Левиин,
\vs Luk 3:30 Симеонов, Иудин, Иосифов, Ионанов, Елиакимов,
\vs Luk 3:31 Мелеаев, Маинанов, Маттафаев, Нафанов, Давидов,
\vs Luk 3:32 Иессеев, Овидов, Воозов, Салмонов, Наассонов,
\vs Luk 3:33 Аминадавов, Арамов, Есромов, Фаресов, Иудин,
\vs Luk 3:34 Иаковлев, Исааков, Авраамов, Фаррин, Нахоров,
\vs Luk 3:35 Серухов, Рагавов, Фалеков, Еверов, Салин,
\vs Luk 3:36 Каинанов, Арфаксадов, Симов, Ноев, Ламехов,
\vs Luk 3:37 Мафусалов, Енохов, Иаредов, Малелеилов, Каинанов,
\vs Luk 3:38 Еносов, Сифов, Адамов, Божий.
\vs Luk 4:1 Иисус, исполненный Духа Святаго, возвратился от Иордана и поведен был Духом в пустыню.
\vs Luk 4:2 Там сорок дней Он был искушаем от диавола и ничего не ел в эти дни, а по прошествии их напоследок взалкал.
\vs Luk 4:3 И сказал Ему диавол: если Ты Сын Божий, то вели этому камню сделаться хлебом.
\vs Luk 4:4 Иисус сказал ему в ответ: написано, что не хлебом одним будет жить человек, но всяким словом Божиим.
\vs Luk 4:5 И, возведя Его на высокую гору, диавол показал Ему все царства вселенной во мгновение времени,
\vs Luk 4:6 и сказал Ему диавол: Тебе дам власть над всеми сими \bibemph{царствами} и славу их, ибо она предана мне, и я, кому хочу, даю ее;
\vs Luk 4:7 итак, если Ты поклонишься мне, то всё будет Твое.
\vs Luk 4:8 Иисус сказал ему в ответ: отойди от Меня, сатана; написано: Господу Богу твоему поклоняйся, и Ему одному служи.
\vs Luk 4:9 И повел Его в Иерусалим, и поставил Его на крыле храма, и сказал Ему: если Ты Сын Божий, бросься отсюда вниз,
\vs Luk 4:10 ибо написано: Ангелам Своим заповедает о Тебе сохранить Тебя;
\vs Luk 4:11 и на руках понесут Тебя, да не преткнешься о камень ногою Твоею.
\vs Luk 4:12 Иисус сказал ему в ответ: сказано: не искушай Господа Бога твоего.
\vs Luk 4:13 И, окончив всё искушение, диавол отошел от Него до времени.
\rsbpar\vs Luk 4:14 И возвратился Иисус в силе Духа в Галилею; и разнеслась молва о Нем по всей окрестной стране.
\vs Luk 4:15 Он учил в синагогах их, и от всех был прославляем.
\rsbpar\vs Luk 4:16 И пришел в Назарет, где был воспитан, и вошел, по обыкновению Своему, в день субботний в синагогу, и встал читать.
\vs Luk 4:17 Ему подали книгу пророка Исаии; и Он, раскрыв книгу, нашел место, где было написано:
\vs Luk 4:18 Дух Господень на Мне; ибо Он помазал Меня благовествовать нищим, и послал Меня исцелять сокрушенных сердцем, проповедовать пленным освобождение, слепым прозрение, отпустить измученных на свободу,
\vs Luk 4:19 проповедовать лето Господне благоприятное.
\vs Luk 4:20 И, закрыв книгу и отдав служителю, сел; и глаза всех в синагоге были устремлены на Него.
\vs Luk 4:21 И Он начал говорить им: ныне исполнилось писание сие, слышанное вами.
\vs Luk 4:22 И все засвидетельствовали Ему это, и дивились словам благодати, исходившим из уст Его, и говорили: не Иосифов ли это сын?
\vs Luk 4:23 Он сказал им: конечно, вы скажете Мне присловие: врач! исцели Самого Себя; сделай и здесь, в Твоем отечестве, т\acc{о}, чт\acc{о}, мы слышали, было в Капернауме.
\vs Luk 4:24 И сказал: истинно говорю вам: никакой пророк не принимается в своем отечестве.
\vs Luk 4:25 Поистине говорю вам: много вдов было в Израиле во дни Илии, когда заключено было небо три года и шесть месяцев, так что сделался большой голод по всей земле,
\vs Luk 4:26 и ни к одной из них не был послан Илия, а только ко вдове в Сарепту Сидонскую;
\vs Luk 4:27 много также было прокаженных в Израиле при пророке Елисее, и ни один из них не очистился, кроме Неемана Сириянина.
\vs Luk 4:28 Услышав это, все в синагоге исполнились ярости
\vs Luk 4:29 и, встав, выгнали Его вон из города и повели на вершину горы, на которой город их был построен, чтобы свергнуть Его;
\vs Luk 4:30 но Он, пройдя посреди них, удалился.
\rsbpar\vs Luk 4:31 И пришел в Капернаум, город Галилейский, и учил их в дни субботние.
\vs Luk 4:32 И дивились учению Его, ибо слово Его было со властью.
\vs Luk 4:33 Был в синагоге человек, имевший нечистого духа бесовского, и он закричал громким голосом:
\vs Luk 4:34 оставь; чт\acc{о} Тебе до нас, Иисус Назарянин? Ты пришел погубить нас; знаю Тебя, кто Ты, Святый Божий.
\vs Luk 4:35 Иисус запретил ему, сказав: замолчи и выйди из него. И бес, повергнув его посреди \bibemph{синагоги}, вышел из него, нимало не повредив ему.
\vs Luk 4:36 И напал на всех ужас, и рассуждали между собою: что это значит, что Он со властью и силою повелевает нечистым духам, и они выходят?
\vs Luk 4:37 И разнесся слух о Нем по всем окрестным местам.
\rsbpar\vs Luk 4:38 Выйдя из синагоги, Он вошел в дом Симона; тёща же Симонова была одержима сильною горячкою; и просили Его о ней.
\vs Luk 4:39 Подойдя к ней, Он запретил горячке; и оставила ее. Она тотчас встала и служила им.
\vs Luk 4:40 При захождении же солнца все, имевшие больных различными болезнями, приводили их к Нему и Он, возлагая на каждого из них руки, исцелял их.
\vs Luk 4:41 Выходили также и бесы из многих с криком и говорили: Ты Христос, Сын Божий. А Он запрещал им сказывать, что они знают, что Он Христос.
\rsbpar\vs Luk 4:42 Когда же настал день, Он, выйдя \bibemph{из дома}, пошел в пустынное место, и народ искал Его и, придя к Нему, удерживал Его, чтобы не уходил от них.
\vs Luk 4:43 Но Он сказал им: и другим городам благовествовать Я должен Царствие Божие, ибо на то Я послан.
\vs Luk 4:44 И проповедовал в синагогах галилейских.
\vs Luk 5:1 Однажды, когда народ теснился к Нему, чтобы слышать слово Божие, а Он стоял у озера Геннисаретского,
\vs Luk 5:2 увидел Он две лодки, стоящие на озере; а рыболовы, выйдя из них, вымывали сети.
\vs Luk 5:3 Войдя в одну лодку, которая была Симонова, Он просил его отплыть несколько от берега и, сев, учил народ из лодки.
\vs Luk 5:4 Когда же перестал учить, сказал Симону: отплыви на глубину и закиньте сети свои для лова.
\vs Luk 5:5 Симон сказал Ему в ответ: Наставник! мы трудились всю ночь и ничего не поймали, но по слову Твоему закину сеть.
\vs Luk 5:6 Сделав это, они поймали великое множество рыбы, и даже сеть у них прорывалась.
\vs Luk 5:7 И дали знак товарищам, находившимся на другой лодке, чтобы пришли помочь им; и пришли, и наполнили обе лодки, так что они начинали тонуть.
\vs Luk 5:8 Увидев это, Симон Петр припал к коленям Иисуса и сказал: выйди от меня, Господи! потому что я человек грешный.
\vs Luk 5:9 Ибо ужас объял его и всех, бывших с ним, от этого лова рыб, ими пойманных;
\vs Luk 5:10 также и Иакова и Иоанна, сыновей Зеведеевых, бывших товарищами Симону. И сказал Симону Иисус: не бойся; отныне будешь ловить человеков.
\vs Luk 5:11 И, вытащив обе лодки на берег, оставили всё и последовали за Ним.
\rsbpar\vs Luk 5:12 Когда Иисус был в одном городе, пришел человек весь в проказе и, увидев Иисуса, пал ниц, умоляя Его и говоря: Господи! если хочешь, можешь меня очистить.
\vs Luk 5:13 Он простер руку, прикоснулся к нему и сказал: хочу, очистись. И тотчас проказа сошла с него.
\vs Luk 5:14 И Он повелел ему никому не сказывать, а пойти показаться священнику и принести \bibemph{жертву} за очищение свое, к\acc{а}к повелел Моисей, во свидетельство им.
\vs Luk 5:15 Но тем более распространялась молва о Нём, и великое множество народа стекалось к Нему слушать и врачеваться у Него от болезней своих.
\vs Luk 5:16 Но Он уходил в пустынные места и молился.
\rsbpar\vs Luk 5:17 В один день, когда Он учил, и сидели тут фарисеи и законоучители, пришедшие из всех мест Галилеи и Иудеи и из Иерусалима, и сила Господня являлась в исцелении \bibemph{больных},~---
\vs Luk 5:18 вот, принесли некоторые на постели человека, который был расслаблен, и старались внести его \bibemph{в дом} и положить перед Иисусом;
\vs Luk 5:19 и, не найдя, где пронести его за многолюдством, влезли на верх дома и сквозь кровлю спустили его с постелью на средину пред Иисуса.
\vs Luk 5:20 И Он, видя веру их, сказал человеку тому: прощаются тебе грехи твои.
\vs Luk 5:21 Книжники и фарисеи начали рассуждать, говоря: кто это, который богохульствует? кто может прощать грехи, кроме одного Бога?
\vs Luk 5:22 Иисус, уразумев помышления их, сказал им в ответ: чт\acc{о} вы помышляете в сердцах ваших?
\vs Luk 5:23 Чт\acc{о} легче сказать: прощаются тебе грехи твои, или сказать: встань и ходи?
\vs Luk 5:24 Но чтобы вы знали, что Сын Человеческий имеет власть на земле прощать грехи,~--- сказал Он расслабленному: тебе говорю: встань, возьми постель твою и иди в дом твой.
\vs Luk 5:25 И он тотчас встал перед ними, взял, на чём лежал, и пошел в дом свой, славя Бога.
\vs Luk 5:26 И ужас объял всех, и славили Бога и, быв исполнены страха, говорили: ч\acc{у}дные дела видели мы ныне.
\rsbpar\vs Luk 5:27 После сего \bibemph{Иисус} вышел и увидел мытаря, именем Левия, сидящего у сбора пошлин, и говорит ему: следуй за Мною.
\vs Luk 5:28 И он, оставив всё, встал и последовал за Ним.
\vs Luk 5:29 И сделал для Него Левий в доме своем большое угощение; и там было множество мытарей и других, которые возлежали с ними.
\vs Luk 5:30 Книжники же и фарисеи роптали и говорили ученикам Его: зачем вы едите и пьете с мытарями и грешниками?
\vs Luk 5:31 Иисус же сказал им в ответ: не здоровые имеют нужду во враче, но больные;
\vs Luk 5:32 Я пришел призвать не праведников, а грешников к покаянию.
\vs Luk 5:33 Они же сказали Ему: почему ученики Иоанновы постятся часто и молитвы творят, также и фарисейские, а Твои едят и пьют?
\vs Luk 5:34 Он сказал им: можете ли заставить сынов чертога брачного поститься, когда с ними жених?
\vs Luk 5:35 Но придут дни, когда отнимется у них жених, и тогда будут поститься в те дни.
\vs Luk 5:36 При сем сказал им притчу: никто не приставляет заплаты к ветхой одежде, отодрав от новой одежды; а иначе и новую раздерет, и к старой не подойдет заплата от новой.
\vs Luk 5:37 И никто не вливает молодого вина в мехи ветхие; а иначе молодое вино прорвет мехи, и само вытечет, и мехи пропадут;
\vs Luk 5:38 но молодое вино должно вливать в мехи новые; тогда сбережется и т\acc{о} и другое.
\vs Luk 5:39 И никто, пив старое \bibemph{вино}, не захочет тотчас молодого, ибо говорит: старое лучше.
\vs Luk 6:1 В субботу, первую по втором дне Пасхи, случилось Ему проходить засеянными полями, и ученики Его срывали колосья и ели, растирая руками.
\vs Luk 6:2 Некоторые же из фарисеев сказали им: зачем вы делаете то, чего не должно делать в субботы?
\vs Luk 6:3 Иисус сказал им в ответ: разве вы не читали, что сделал Давид, когда взалкал сам и бывшие с ним?
\vs Luk 6:4 К\acc{а}к он вошел в дом Божий, взял хлебы предложения, которых не должно было есть никому, кроме одних священников, и ел, и дал бывшим с ним?
\vs Luk 6:5 И сказал им: Сын Человеческий есть господин и субботы.
\rsbpar\vs Luk 6:6 Случилось же и в другую субботу войти Ему в синагогу и учить. Там был человек, у которого правая рука была сухая.
\vs Luk 6:7 Книжники же и фарисеи наблюдали за Ним, не исцелит ли в субботу, чтобы найти обвинение против Него.
\vs Luk 6:8 Но Он, зная помышления их, сказал человеку, имеющему сухую руку: встань и выступи на средину. И он встал и выступил.
\vs Luk 6:9 Тогда сказал им Иисус: спрошу Я вас: чт\acc{о} должно делать в субботу? добро, или зло? спасти душу, или погубить? Они молчали.
\vs Luk 6:10 И, посмотрев на всех их, сказал тому человеку: протяни руку твою. Он так и сделал; и стала рука его здорова, как другая.
\vs Luk 6:11 Они же пришли в бешенство и говорили между собою, чт\acc{о} бы им сделать с Иисусом.
\rsbpar\vs Luk 6:12 В те дни взошел Он на гору помолиться и пробыл всю ночь в молитве к Богу.
\vs Luk 6:13 Когда же настал день, призвал учеников Своих и избрал из них двенадцать, которых и наименовал Апостолами:
\vs Luk 6:14 Симона, которого и назвал Петром, и Андрея, брата его, Иакова и Иоанна, Филиппа и Варфоломея,
\vs Luk 6:15 Матфея и Фому, Иакова Алфеева и Симона, прозываемого Зилотом,
\vs Luk 6:16 Иуду Иаковлева и Иуду Искариота, который потом сделался предателем.
\rsbpar\vs Luk 6:17 И, сойдя с ними, стал Он на ровном месте, и множество учеников Его, и много народа из всей Иудеи и Иерусалима и приморских мест Тирских и Сидонских,
\vs Luk 6:18 которые пришли послушать Его и исцелиться от болезней своих, также и страждущие от нечистых духов; и исцелялись.
\vs Luk 6:19 И весь народ искал прикасаться к Нему, потому что от Него исходила сила и исцеляла всех.
\vs Luk 6:20 И Он, возведя очи Свои на учеников Своих, говорил:\rsbpar Блаженны нищие духом, ибо ваше есть Царствие Божие.
\rsbpar\vs Luk 6:21 Блаженны алчущие ныне, ибо насытитесь.\rsbpar Блаженны плачущие ныне, ибо воссмеетесь.
\rsbpar\vs Luk 6:22 Блаженны вы, когда возненавидят вас люди и когда отлучат вас, и будут поносить, и пронесут имя ваше, как бесчестное, за Сына Человеческого.
\vs Luk 6:23 Возрадуйтесь в тот день и возвеселитесь, ибо велика вам награда на небесах. Так поступали с пророками отцы их.
\rsbpar\vs Luk 6:24 Напротив, горе вам, богатые! ибо вы уже получили свое утешение.
\vs Luk 6:25 Горе вам, пресыщенные ныне! ибо взалчете. Горе вам, смеющиеся ныне! ибо восплачете и возрыдаете.
\vs Luk 6:26 Горе вам, когда все люди будут говорить о вас хорошо! ибо так поступали с лжепророками отцы их.
\rsbpar\vs Luk 6:27 Но вам, слушающим, говорю: люб\acc{и}те врагов ваших, благотворите ненавидящим вас,
\vs Luk 6:28 благословляйте проклинающих вас и молитесь за обижающих вас.
\vs Luk 6:29 Ударившему тебя по щеке подставь и другую, и отнимающему у тебя верхнюю одежду не препятствуй взять и рубашку.
\vs Luk 6:30 Всякому, просящему у тебя, давай, и от взявшего твое не требуй назад.
\vs Luk 6:31 И к\acc{а}к хотите, чтобы с вами поступали люди, т\acc{а}к и вы поступайте с ними.
\vs Luk 6:32 И если любите любящих вас, какая вам за то благодарность? ибо и грешники любящих их любят.
\vs Luk 6:33 И если делаете добро тем, которые вам делают добро, какая вам за то благодарность? ибо и грешники т\acc{о} же делают.
\vs Luk 6:34 И если взаймы даёте тем, от которых надеетесь получить обратно, какая вам за то благодарность? ибо и грешники дают взаймы грешникам, чтобы получить обратно столько же.
\vs Luk 6:35 Но вы люб\acc{и}те врагов ваших, и благотворите, и взаймы давайте, не ожидая ничего; и будет вам награда великая, и будете сынами Всевышнего; ибо Он благ и к неблагодарным и злым.
\vs Luk 6:36 Итак, будьте милосерды, как и Отец ваш милосерд.
\rsbpar\vs Luk 6:37 Не суд\acc{и}те, и не будете судимы; не осуждайте, и не будете осуждены; прощайте, и прощены будете;
\vs Luk 6:38 давайте, и дастся вам: мерою доброю, утрясенною, нагнетенною и переполненною отсыплют вам в лоно ваше; ибо, какою мерою мерите, такою же отмерится и вам.
\vs Luk 6:39 Сказал также им притчу: может ли слепой водить слепого? не оба ли упадут в яму?
\vs Luk 6:40 Ученик не бывает выше своего учителя; но, и усовершенствовавшись, будет всякий, как учитель его.
\vs Luk 6:41 Что ты смотришь на сучок в глазе брата твоего, а бревна в твоем глазе не чувствуешь?
\vs Luk 6:42 Или, как можешь сказать брату твоему: брат! дай, я выну сучок из глаза твоего, когда сам не видишь бревна в твоем глазе? Лицемер! вынь прежде бревно из твоего глаза, и тогда увидишь, как вынуть сучок из глаза брата твоего.
\vs Luk 6:43 Нет доброго дерева, которое приносило бы худой плод; и нет худого дерева, которое приносило бы плод добрый,
\vs Luk 6:44 ибо всякое дерево познаётся по плоду своему, потому что не собирают смокв с терновника и не снимают винограда с кустарника.
\vs Luk 6:45 Добрый человек из доброго сокровища сердца своего выносит доброе, а злой человек из злого сокровища сердца своего выносит злое, ибо от избытка сердца говорят уста его.
\rsbpar\vs Luk 6:46 Чт\acc{о} вы зовете Меня: Господи! Господи!~--- и не делаете того, чт\acc{о} Я говорю?
\vs Luk 6:47 Всякий, приходящий ко Мне и слушающий слова Мои и исполняющий их, скажу вам, кому подобен.
\vs Luk 6:48 Он подобен человеку, строящему дом, который копал, углубился и положил основание на камне; почему, когда случилось наводнение и вода напёрла на этот дом, то не могла поколебать его, потому что он основан был на камне.
\vs Luk 6:49 А слушающий и неисполняющий подобен человеку, построившему дом на земле без основания, который, когда напёрла на него вода, тотчас обрушился; и разрушение дома сего было великое.
\vs Luk 7:1 Когда Он окончил все слова Свои к слушавшему народу, то вошел в Капернаум.
\vs Luk 7:2 У одного сотника слуга, которым он дорожил, был болен при смерти.
\vs Luk 7:3 Услышав об Иисусе, он послал к Нему Иудейских старейшин просить Его, чтобы пришел исцелить слугу его.
\vs Luk 7:4 И они, придя к Иисусу, просили Его убедительно, говоря: он достоин, чтобы Ты сделал для него это,
\vs Luk 7:5 ибо он любит народ наш и построил нам синагогу.
\vs Luk 7:6 Иисус пошел с ними. И когда Он недалеко уже был от дома, сотник прислал к Нему друзей сказать Ему: не трудись, Господи! ибо я недостоин, чтобы Ты вошел под кров мой;
\vs Luk 7:7 потому и себя самого не почел я достойным прийти к Тебе; но скажи слово, и выздоровеет слуга мой.
\vs Luk 7:8 Ибо я и подвластный человек, но, имея у себя в подчинении воинов, говорю одному: пойди, и идет; и другому: приди, и приходит; и слуге моему: сделай т\acc{о}, и делает.
\vs Luk 7:9 Услышав сие, Иисус удивился ему и, обратившись, сказал идущему за Ним народу: сказываю вам, что и в Израиле не нашел Я такой веры.
\vs Luk 7:10 Посланные, возвратившись в дом, нашли больного слугу выздоровевшим.
\rsbpar\vs Luk 7:11 После сего Иисус пошел в город, называемый Наин; и с Ним шли многие из учеников Его и множество народа.
\vs Luk 7:12 Когда же Он приблизился к городским воротам, тут выносили умершего, единственного сына у матери, а она была вдова; и много народа шло с нею из города.
\vs Luk 7:13 Увидев ее, Господь сжалился над нею и сказал ей: не плачь.
\vs Luk 7:14 И, подойдя, прикоснулся к одру; несшие остановились, и Он сказал: юноша! тебе говорю, встань!
\vs Luk 7:15 Мертвый, поднявшись, сел и стал говорить; и отдал его \bibemph{Иисус} матери его.
\vs Luk 7:16 И всех объял страх, и славили Бога, говоря: великий пророк восстал между нами, и Бог посетил народ Свой.
\vs Luk 7:17 Такое мнение о Нём распространилось по всей Иудее и по всей окрестности.
\rsbpar\vs Luk 7:18 И возвестили Иоанну ученики его о всём том.
\vs Luk 7:19 Иоанн, призвав двоих из учеников своих, послал к Иисусу спросить: Ты ли Тот, Который должен прийти, или ожидать нам другого?
\vs Luk 7:20 Они, придя к \bibemph{Иисусу}, сказали: Иоанн Креститель послал нас к Тебе спросить: Ты ли Тот, Которому должно прийти, или другого ожидать нам?
\vs Luk 7:21 А в это время Он многих исцелил от болезней и недугов и от злых духов, и многим слепым даровал зрение.
\vs Luk 7:22 И сказал им Иисус в ответ: пойдите, скажите Иоанну, чт\acc{о} вы видели и слышали: слепые прозревают, хромые ходят, прокаженные очищаются, глухие слышат, мертвые воскресают, нищие благовествуют;
\vs Luk 7:23 и блажен, кто не соблазнится о Мне!
\rsbpar\vs Luk 7:24 По отшествии же посланных Иоанном, начал говорить к народу об Иоанне: чт\acc{о} смотреть ходили вы в пустыню? трость ли, ветром колеблемую?
\vs Luk 7:25 Чт\acc{о} же смотреть ходили вы? человека ли, одетого в мягкие одежды? Но одевающиеся пышно и роскошно живущие находятся при дворах царских.
\vs Luk 7:26 Чт\acc{о} же смотреть ходили вы? пророка ли? Да, говорю вам, и больше пророка.
\vs Luk 7:27 Сей есть, о котором написано: вот, Я посылаю Ангела Моего пред лицем Твоим, который приготовит путь Твой пред Тобою.
\vs Luk 7:28 Ибо говорю вам: из рожденных женами нет ни одного пророка больше Иоанна Крестителя; но меньший в Царствии Божием больше его.
\vs Luk 7:29 И весь народ, слушавший \bibemph{Его}, и мытари воздали славу Богу, крестившись крещением Иоанновым;
\vs Luk 7:30 а фарисеи и законники отвергли волю Божию о себе, не крестившись от него.
\vs Luk 7:31 Тогда Господь сказал: с кем сравню людей рода сего? и кому они подобны?
\vs Luk 7:32 Они подобны детям, которые сидят на улице, кличут друг друга и говорят: мы играли вам на свирели, и вы не плясали; мы пели вам плачевные песни, и вы не плакали.
\vs Luk 7:33 Ибо пришел Иоанн Креститель: ни хлеба не ест, ни вина не пьет; и говорите: в нем бес.
\vs Luk 7:34 Пришел Сын Человеческий: ест и пьет; и говорите: вот человек, который любит есть и пить вино, друг мытарям и грешникам.
\vs Luk 7:35 И оправдана премудрость всеми чадами ее.
\rsbpar\vs Luk 7:36 Некто из фарисеев просил Его вкусить с ним пищи; и Он, войдя в дом фарисея, возлег.
\vs Luk 7:37 И вот, женщина того города, которая была грешница, узнав, что Он возлежит в доме фарисея, принесла алавастровый сосуд с миром
\vs Luk 7:38 и, став позади у ног Его и плача, начала обливать ноги Его слезами и отирать волосами головы своей, и целовала ноги Его, и мазала миром.
\vs Luk 7:39 Видя это, фарисей, пригласивший Его, сказал сам в себе: если бы Он был пророк, то знал бы, кто и какая женщина прикасается к Нему, ибо она грешница.
\vs Luk 7:40 Обратившись к нему, Иисус сказал: Симон! Я имею нечто сказать тебе. Он говорит: скажи, Учитель.
\vs Luk 7:41 Иисус сказал: у одного заимодавца было два должника: один должен был пятьсот динариев, а другой пятьдесят,
\vs Luk 7:42 но как они не имели чем заплатить, он простил обоим. Скажи же, который из них более возлюбит его?
\vs Luk 7:43 Симон отвечал: думаю, тот, которому более простил. Он сказал ему: правильно ты рассудил.
\vs Luk 7:44 И, обратившись к женщине, сказал Симону: видишь ли ты эту женщину? Я пришел в дом твой, и ты воды Мне на ноги не дал, а она слезами облила Мне ноги и волосами головы своей отёрла;
\vs Luk 7:45 ты целования Мне не дал, а она, с тех пор как Я пришел, не перестает целовать у Меня ноги;
\vs Luk 7:46 ты головы Мне маслом не помазал, а она миром помазала Мне ноги.
\vs Luk 7:47 А потому сказываю тебе: прощаются грехи её многие за то, что она возлюбила много, а кому мало прощается, тот мало любит.
\vs Luk 7:48 Ей же сказал: прощаются тебе грехи.
\vs Luk 7:49 И возлежавшие с Ним начали говорить про себя: кто это, что и грехи прощает?
\vs Luk 7:50 Он же сказал женщине: вера твоя спасла тебя, иди с миром.
\vs Luk 8:1 После сего Он проходил по городам и селениям, проповедуя и благовествуя Царствие Божие, и с Ним двенадцать,
\vs Luk 8:2 и некоторые женщины, которых Он исцелил от злых духов и болезней: Мария, называемая Магдалиною, из которой вышли семь бесов,
\vs Luk 8:3 и Иоанна, жена Хузы, домоправителя Иродова, и Сусанна, и многие другие, которые служили Ему имением своим.
\rsbpar\vs Luk 8:4 Когда же собралось множество народа, и из всех городов жители сходились к Нему, Он начал говорить притчею:
\vs Luk 8:5 вышел сеятель сеять семя свое, и когда он сеял, иное упало при дороге и было потоптано, и птицы небесные поклевали его;
\vs Luk 8:6 а иное упало на камень и, взойдя, засохло, потому что не имело влаги;
\vs Luk 8:7 а иное упало между тернием, и выросло терние и заглушило его;
\vs Luk 8:8 а иное упало на добрую землю и, взойдя, принесло плод сторичный. Сказав сие, возгласил: кто имеет уши слышать, да слышит!
\vs Luk 8:9 Ученики же Его спросили у Него: что бы значила притча сия?
\vs Luk 8:10 Он сказал: вам дано знать тайны Царствия Божия, а прочим в притчах, так что они видя не видят и слыша не разумеют.
\vs Luk 8:11 Вот что значит притча сия: семя есть слово Божие;
\vs Luk 8:12 а упавшее при пути, это суть слушающие, к которым пот\acc{о}м приходит диавол и уносит слово из сердца их, чтобы они не уверовали и не спаслись;
\vs Luk 8:13 а упавшее на камень, это те, которые, когда услышат слово, с радостью принимают, но которые не имеют корня, и временем веруют, а во время искушения отпадают;
\vs Luk 8:14 а упавшее в терние, это те, которые слушают слово, но, отходя, заботами, богатством и наслаждениями житейскими подавляются и не приносят плода;
\vs Luk 8:15 а упавшее на добрую землю, это те, которые, услышав слово, хранят его в добром и чистом сердце и приносят плод в терпении. Сказав это, Он возгласил: кто имеет уши слышать, да слышит!
\vs Luk 8:16 Никто, зажегши свечу, не покрывает ее сосудом, или не ставит под кровать, а ставит на подсвечник, чтобы входящие видели свет.
\vs Luk 8:17 Ибо нет ничего тайного, чт\acc{о} не сделалось бы явным, ни сокровенного, чт\acc{о} не сделалось бы известным и не обнаружилось бы.
\vs Luk 8:18 Итак, наблюдайте, как вы слушаете: ибо, кто имеет, тому дано будет, а кто не имеет, у того отнимется и т\acc{о}, чт\acc{о} он думает иметь.
\rsbpar\vs Luk 8:19 И пришли к Нему Матерь и братья Его, и не могли подойти к Нему по причине народа.
\vs Luk 8:20 И дали знать Ему: Матерь и братья Твои стоят вне, желая видеть Тебя.
\vs Luk 8:21 Он сказал им в ответ: матерь Моя и братья Мои суть слушающие слово Божие и исполняющие его.
\rsbpar\vs Luk 8:22 В один день Он вошел с учениками Своими в лодку и сказал им: переправимся на ту сторону озера. И отправились.
\vs Luk 8:23 Во время плавания их Он заснул. На озере поднялся бурный ветер, и заливало их \bibemph{волнами}, и они были в опасности.
\vs Luk 8:24 И, подойдя, разбудили Его и сказали: Наставник! Наставник! погибаем. Но Он, встав, запретил ветру и волнению воды; и перестали, и сделалась тишина.
\vs Luk 8:25 Тогда Он сказал им: где вера ваша? Они же в страхе и удивлении говорили друг другу: кто же это, что и ветрам повелевает и воде, и повинуются Ему?
\rsbpar\vs Luk 8:26 И приплыли в страну Гадаринскую, лежащую против Галилеи.
\vs Luk 8:27 Когда же вышел Он на берег, встретил Его один человек из города, одержимый бесами с давнего времени, и в одежду не одевавшийся, и живший не в доме, а в гробах.
\vs Luk 8:28 Он, увидев Иисуса, вскричал, пал пред Ним и громким голосом сказал: чт\acc{о} Тебе до меня, Иисус, Сын Бога Всевышнего? умоляю Тебя, не мучь меня.
\vs Luk 8:29 Ибо \bibemph{Иисус} повелел нечистому духу выйти из сего человека, потому что он долгое время мучил его, так что его связывали цепями и узами, сберегая его; но он разрывал узы и был гоним бесом в пустыни.
\vs Luk 8:30 Иисус спросил его: как тебе имя? Он сказал: легион,~--- потому что много бесов вошло в него.
\vs Luk 8:31 И они просили Иисуса, чтобы не повелел им идти в бездну.
\vs Luk 8:32 Тут же на горе паслось большое стадо свиней; и \bibemph{бесы} просили Его, чтобы позволил им войти в них. Он позволил им.
\vs Luk 8:33 Бесы, выйдя из человека, вошли в свиней, и бросилось стадо с крутизны в озеро и потонуло.
\vs Luk 8:34 Пастухи, видя происшедшее, побежали и рассказали в городе и в селениях.
\vs Luk 8:35 И вышли видеть происшедшее; и, придя к Иисусу, нашли человека, из которого вышли бесы, сидящего у ног Иисуса, одетого и в здравом уме; и ужаснулись.
\vs Luk 8:36 Видевшие же рассказали им, как исцелился бесновавшийся.
\vs Luk 8:37 И просил Его весь народ Гадаринской окрестности удалиться от них, потому что они объяты были великим страхом. Он вошел в лодку и возвратился.
\vs Luk 8:38 Человек же, из которого вышли бесы, просил Его, чтобы быть с Ним. Но Иисус отпустил его, сказав:
\vs Luk 8:39 возвратись в дом твой и расскажи, чт\acc{о} сотворил тебе Бог. Он пошел и проповедовал по всему городу, что сотворил ему Иисус.
\rsbpar\vs Luk 8:40 Когда же возвратился Иисус, народ принял Его, потому что все ожидали Его.
\vs Luk 8:41 И вот, пришел человек, именем Иаир, который был начальником синагоги; и, пав к ногам Иисуса, просил Его войти к нему в дом,
\vs Luk 8:42 потому что у него была одна дочь, лет двенадцати, и та была при смерти. Когда же Он шел, народ теснил Его.
\vs Luk 8:43 И женщина, страдавшая кровотечением двенадцать лет, которая, издержав на врачей всё имение, ни одним не могла быть вылечена,
\vs Luk 8:44 подойдя сзади, коснулась края одежды Его; и тотчас течение крови у ней остановилось.
\vs Luk 8:45 И сказал Иисус: кто прикоснулся ко Мне? Когда же все отрицались, Петр сказал и бывшие с Ним: Наставник! народ окружает Тебя и теснит,~--- и Ты говоришь: кто прикоснулся ко Мне?
\vs Luk 8:46 Но Иисус сказал: прикоснулся ко Мне некто, ибо Я чувствовал силу, исшедшую из Меня.
\vs Luk 8:47 Женщина, видя, что она не утаилась, с трепетом подошла и, пав пред Ним, объявила Ему перед всем народом, по какой причине прикоснулась к Нему и как тотчас исцелилась.
\vs Luk 8:48 Он сказал ей: дерзай, дщерь! вера твоя спасла тебя; иди с миром.
\vs Luk 8:49 Когда Он еще говорил это, приходит некто из дома начальника синагоги и говорит ему: дочь твоя умерла; не утруждай Учителя.
\vs Luk 8:50 Но Иисус, услышав это, сказал ему: не бойся, только веруй, и спасена будет.
\vs Luk 8:51 Придя же в дом, не позволил войти никому, кроме Петра, Иоанна и Иакова, и отца девицы, и матери.
\vs Luk 8:52 Все плакали и рыдали о ней. Но Он сказал: не плачьте; она не умерла, но спит.
\vs Luk 8:53 И смеялись над Ним, зная, что она умерла.
\vs Luk 8:54 Он же, выслав всех вон и взяв ее за руку, возгласил: девица! встань.
\vs Luk 8:55 И возвратился дух ее; она тотчас встала, и Он велел дать ей есть.
\vs Luk 8:56 И удивились родители ее. Он же повелел им не сказывать никому о происшедшем.
\vs Luk 9:1 Созвав же двенадцать, дал силу и власть над всеми бесами и врачевать от болезней,
\vs Luk 9:2 и послал их проповедовать Царствие Божие и исцелять больных.
\vs Luk 9:3 И сказал им: ничего не берите на дорогу: ни посоха, ни сум\acc{ы}, ни хлеба, ни серебра, и не имейте по две одежды;
\vs Luk 9:4 и в какой дом войдете, там оставайтесь и оттуда отправляйтесь \bibemph{в путь}.
\vs Luk 9:5 А если где не примут вас, то, выходя из того города, отрясите и прах от ног ваших во свидетельство на них.
\vs Luk 9:6 Они пошли и проходили по селениям, благовествуя и исцеляя повсюду.
\rsbpar\vs Luk 9:7 Услышал Ирод четвертовластник о всём, что делал \bibemph{Иисус}, и недоумевал: ибо одни говорили, что это Иоанн восстал из мертвых;
\vs Luk 9:8 другие, что Илия явился, а иные, что один из древних пророков воскрес.
\vs Luk 9:9 И сказал Ирод: Иоанна я обезглавил; кто же Этот, о Котором я слышу такое? И искал увидеть Его.
\rsbpar\vs Luk 9:10 Апостолы, возвратившись, рассказали Ему, чт\acc{о} они сделали; и Он, взяв их с Собою, удалился особо в пустое место, близ города, называемого Вифсаидою.
\rsbpar\vs Luk 9:11 Но народ, узнав, пошел за Ним; и Он, приняв их, беседовал с ними о Царствии Божием и требовавших исцеления исцелял.
\vs Luk 9:12 День же начал склоняться к вечеру. И, приступив к Нему, двенадцать говорили Ему: отпусти народ, чтобы они пошли в окрестные селения и деревни ночевать и достали пищи; потому что мы здесь в пустом месте.
\vs Luk 9:13 Но Он сказал им: вы дайте им есть. Они сказали: у нас нет более пяти хлебов и двух рыб; разве нам пойти купить пищи для всех сих людей?
\vs Luk 9:14 Ибо их было около пяти тысяч человек. Но Он сказал ученикам Своим: рассадите их рядами по пятидесяти.
\vs Luk 9:15 И сделали так, и рассадили всех.
\vs Luk 9:16 Он же, взяв пять хлебов и две рыбы и воззрев на небо, благословил их, преломил и дал ученикам, чтобы раздать народу.
\vs Luk 9:17 И ели, и насытились все; и оставшихся у них кусков набрано двенадцать коробов.
\rsbpar\vs Luk 9:18 В одно время, когда Он молился в уединенном месте, и ученики были с Ним, Он спросил их: за кого почитает Меня народ?
\vs Luk 9:19 Они сказали в ответ: за Иоанна Крестителя, а иные за Илию; другие же \bibemph{говорят}, что один из древних пророков воскрес.
\vs Luk 9:20 Он же спросил их: а вы за кого почитаете Меня? Отвечал Петр: за Христа Божия.
\vs Luk 9:21 Но Он строго приказал им никому не говорить о сем,
\vs Luk 9:22 сказав, что Сыну Человеческому должно много пострадать, и быть отвержену старейшинами, первосвященниками и книжниками, и быть убиту, и в третий день воскреснуть.
\rsbpar\vs Luk 9:23 Ко всем же сказал: если кто хочет идти за Мною, отвергнись себя, и возьми крест свой, и следуй за Мною.
\vs Luk 9:24 Ибо кто хочет душу свою сберечь, тот потеряет ее; а кто потеряет душу свою ради Меня, тот сбережет ее.
\vs Luk 9:25 Ибо что пользы человеку приобрести весь мир, а себя самого погубить или повредить себе?
\vs Luk 9:26 Ибо кто постыдится Меня и Моих слов, того Сын Человеческий постыдится, когда приидет во славе Своей и Отца и святых Ангелов.
\vs Luk 9:27 Говорю же вам истинно: есть некоторые из стоящих здесь, которые не вкусят смерти, как уже увидят Царствие Божие.
\rsbpar\vs Luk 9:28 После сих слов, дней через восемь, взяв Петра, Иоанна и Иакова, взошел Он на гору помолиться.
\vs Luk 9:29 И когда молился, вид лица Его изменился, и одежда Его сделалась белою, блистающею.
\vs Luk 9:30 И вот, два мужа беседовали с Ним, которые были Моисей и Илия;
\vs Luk 9:31 явившись во славе, они говорили об исходе Его, который Ему надлежало совершить в Иерусалиме.
\vs Luk 9:32 Петр же и бывшие с ним отягчены были сном; но, пробудившись, увидели славу Его и двух мужей, стоявших с Ним.
\vs Luk 9:33 И когда они отходили от Него, сказал Петр Иисусу: Наставник! хорошо нам здесь быть; сделаем три кущи: одну Тебе, одну Моисею и одну Илии,~--- не зная, чт\acc{о} говорил.
\vs Luk 9:34 Когда же он говорил это, явилось облако и осенило их; и устрашились, когда вошли в облако.
\vs Luk 9:35 И был из облака глас, глаголющий: Сей есть Сын Мой Возлюбленный, Его слушайте.
\vs Luk 9:36 Когда был глас сей, остался Иисус один. И они умолчали, и никому не говорили в те дни о том, что видели.
\rsbpar\vs Luk 9:37 В следующий же день, когда они сошли с горы, встретило Его много народа.
\vs Luk 9:38 Вдруг некто из народа воскликнул: Учитель! умоляю Тебя взглянуть на сына моего, он один у меня:
\vs Luk 9:39 его схватывает дух, и он внезапно вскрикивает, и терзает его, так что он испускает пену; и насилу отступает от него, измучив его.
\vs Luk 9:40 Я просил учеников Твоих изгнать его, и они не могли.
\vs Luk 9:41 Иисус же, отвечая, сказал: о, род неверный и развращенный! доколе буду с вами и буду терпеть вас? приведи сюда сына твоего.
\vs Luk 9:42 Когда же тот еще шел, бес поверг его и стал бить; но Иисус запретил нечистому духу, и исцелил отрока, и отдал его отцу его.
\vs Luk 9:43 И все удивлялись величию Божию.\rsbpar Когда же все дивились всему, что творил Иисус, Он сказал ученикам Своим:
\vs Luk 9:44 вложите вы себе в уши слова сии: Сын Человеческий будет предан в руки человеческие.
\vs Luk 9:45 Но они не поняли сл\acc{о}ва сего, и оно было закрыто от них, так что они не постигли его, а спросить Его о сем слове боялись.
\vs Luk 9:46 Пришла же им мысль: кто бы из них был больше?
\vs Luk 9:47 Иисус же, видя помышление сердца их, взяв дитя, поставил его пред Собою
\vs Luk 9:48 и сказал им: кто примет сие дитя во имя Мое, тот Меня принимает; а кто примет Меня, тот принимает Пославшего Меня; ибо кто из вас меньше всех, тот будет велик.
\vs Luk 9:49 При сем Иоанн сказал: Наставник! мы видели человека, именем Твоим изгоняющего бесов, и запретили ему, потому что он не ходит с нами.
\vs Luk 9:50 Иисус сказал ему: не запрещайте, ибо кто не против вас, тот за вас.
\rsbpar\vs Luk 9:51 Когда же приближались дни взятия Его \bibemph{от мира}, Он восхотел идти в Иерусалим;
\vs Luk 9:52 и послал вестников пред лицем Своим; и они пошли и вошли в селение Самарянское; чтобы приготовить для Него;
\vs Luk 9:53 но \bibemph{там} не приняли Его, потому что Он имел вид путешествующего в Иерусалим.
\vs Luk 9:54 Видя т\acc{о}, ученики Его, Иаков и Иоанн, сказали: Господи! хочешь ли, мы скажем, чтобы огонь сошел с неба и истребил их, как и Илия сделал?
\vs Luk 9:55 Но Он, обратившись к ним, запретил им и сказал: не знаете, какого вы духа;
\vs Luk 9:56 ибо Сын Человеческий пришел не губить души человеческие, а спасать. И пошли в другое селение.
\rsbpar\vs Luk 9:57 Случилось, что когда они были в пути, некто сказал Ему: Господи! я пойду за Тобою, куда бы Ты ни пошел.
\vs Luk 9:58 Иисус сказал ему: лисицы имеют норы, и птицы небесные~--- гнезда; а Сын Человеческий не имеет, где приклонить голову.
\vs Luk 9:59 А другому сказал: следуй за Мною. Тот сказал: Господи! позволь мне прежде пойти и похоронить отца моего.
\vs Luk 9:60 Но Иисус сказал ему: предоставь мертвым погребать своих мертвецов, а ты иди, благовествуй Царствие Божие.
\vs Luk 9:61 Еще другой сказал: я пойду за Тобою, Господи! но прежде позволь мне проститься с домашними моими.
\vs Luk 9:62 Но Иисус сказал ему: никто, возложивший руку свою на плуг и озирающийся назад, не благонадежен для Царствия Божия.
\vs Luk 10:1 После сего избрал Господь и других семьдесят \bibemph{учеников}, и послал их по два пред лицем Своим во всякий город и место, куда Сам хотел идти,
\vs Luk 10:2 и сказал им: жатвы много, а делателей мало; итак, молите Господина жатвы, чтобы выслал делателей на жатву Свою.
\vs Luk 10:3 Идите! Я посылаю вас, как агнцев среди волков.
\vs Luk 10:4 Не берите ни мешка, ни сум\acc{ы}, ни обуви, и никого на дороге не приветствуйте.
\vs Luk 10:5 В какой дом войдете, сперва говорите: мир дому сему;
\vs Luk 10:6 и если будет там сын мира, то почиет на нём мир ваш, а если нет, то к вам возвратится.
\vs Luk 10:7 В доме же том оставайтесь, ешьте и пейте, что у них есть, ибо трудящийся достоин награды за труды свои; не переходите из дома в дом.
\vs Luk 10:8 И если придёте в какой город и примут вас, ешьте, что вам предложат,
\vs Luk 10:9 и исцеляйте находящихся в нём больных, и говорите им: приблизилось к вам Царствие Божие.
\vs Luk 10:10 Если же придете в какой город и не примут вас, то, выйдя на улицу, скажите:
\vs Luk 10:11 и прах, прилипший к нам от вашего города, отрясаем вам; однако же знайте, что приблизилось к вам Царствие Божие.
\vs Luk 10:12 Сказываю вам, что Содому в день оный будет отраднее, нежели городу тому.
\vs Luk 10:13 Горе тебе, Хоразин! горе тебе, Вифсаида! ибо если бы в Тире и Сидоне явлены были силы, явленные в вас, то давно бы они, сидя во вретище и пепле, покаялись;
\vs Luk 10:14 но и Тиру и Сидону отраднее будет на суде, нежели вам.
\vs Luk 10:15 И ты, Капернаум, до неба вознесшийся, до ада низвергнешься.
\vs Luk 10:16 Слушающий вас Меня слушает, и отвергающийся вас Меня отвергается; а отвергающийся Меня отвергается Пославшего Меня.
\rsbpar\vs Luk 10:17 Семьдесят \bibemph{учеников} возвратились с радостью и говорили: Господи! и бесы повинуются нам о имени Твоем.
\vs Luk 10:18 Он же сказал им: Я видел сатану, спадшего с неба, как молнию;
\vs Luk 10:19 се, даю вам власть наступать на змей и скорпионов и на всю силу вражью, и ничто не повредит вам;
\vs Luk 10:20 однако ж тому не радуйтесь, что духи вам повинуются, но радуйтесь тому, что имена ваши написаны на небесах.
\vs Luk 10:21 В тот час возрадовался духом Иисус и сказал: славлю Тебя, Отче, Господи неба и земли, что Ты утаил сие от мудрых и разумных и открыл младенцам. Ей, Отче! Ибо таково было Твое благоволение.
\vs Luk 10:22 И, обратившись к ученикам, сказал: всё предано Мне Отцем Моим; и кто есть Сын, не знает никто, кроме Отца, и кто есть Отец, \bibemph{не знает никто}, кроме Сына, и кому Сын хочет открыть.
\vs Luk 10:23 И, обратившись к ученикам, сказал им особо: блаженны очи, видящие то, что вы видите!
\vs Luk 10:24 ибо сказываю вам, что многие пророки и цари желали видеть, чт\acc{о} вы видите, и не видели, и слышать, чт\acc{о} вы слышите, и не слышали.
\rsbpar\vs Luk 10:25 И вот, один законник встал и, искушая Его, сказал: Учитель! чт\acc{о} мне делать, чтобы наследовать жизнь вечную?
\vs Luk 10:26 Он же сказал ему: в законе чт\acc{о} написано? к\acc{а}к читаешь?
\vs Luk 10:27 Он сказал в ответ: возлюби Господа Бога твоего всем сердцем твоим, и всею душею твоею, и всею крепостию твоею, и всем разумением твоим, и ближнего твоего, как самого себя.
\vs Luk 10:28 \bibemph{Иисус} сказал ему: правильно ты отвечал; так поступай, и будешь жить.
\vs Luk 10:29 Но он, желая оправдать себя, сказал Иисусу: а кто мой ближний?
\vs Luk 10:30 На это сказал Иисус: некоторый человек шел из Иерусалима в Иерихон и попался разбойникам, которые сняли с него одежду, изранили его и ушли, оставив его едва живым.
\vs Luk 10:31 По случаю один священник шел тою дорогою и, увидев его, прошел мимо.
\vs Luk 10:32 Также и левит, быв на том месте, подошел, посмотрел и прошел мимо.
\vs Luk 10:33 Самарянин же некто, проезжая, нашел на него и, увидев его, сжалился
\vs Luk 10:34 и, подойдя, перевязал ему раны, возливая масло и вино; и, посадив его на своего осла, привез его в гостиницу и позаботился о нем;
\vs Luk 10:35 а на другой день, отъезжая, вынул два динария, дал содержателю гостиницы и сказал ему: позаботься о нем; и если издержишь что более, я, когда возвращусь, отдам тебе.
\vs Luk 10:36 Кто из этих троих, думаешь ты, был ближний попавшемуся разбойникам?
\vs Luk 10:37 Он сказал: оказавший ему милость. Тогда Иисус сказал ему: иди, и ты поступай так же.
\rsbpar\vs Luk 10:38 В продолжение пути их пришел Он в одно селение; здесь женщина, именем Марфа, приняла Его в дом свой;
\vs Luk 10:39 у неё была сестра, именем Мария, которая села у ног Иисуса и слушала слово Его.
\vs Luk 10:40 Марфа же заботилась о большом угощении и, подойдя, сказала: Господи! или Тебе нужды нет, что сестра моя одну меня оставила служить? скажи ей, чтобы помогла мне.
\vs Luk 10:41 Иисус же сказал ей в ответ: Марфа! Марфа! ты заботишься и суетишься о многом,
\vs Luk 10:42 а одно только нужно; Мария же избрала благую часть, которая не отнимется у неё.
\vs Luk 11:1 Случилось, что когда Он в одном месте молился, и перестал, один из учеников Его сказал Ему: Господи! научи нас молиться, как и Иоанн научил учеников своих.
\vs Luk 11:2 Он сказал им: когда м\acc{о}литесь, говорите:\rsbpar Отче наш, сущий на небесах! да святится имя Твое; да приидет Царствие Твое; да будет воля Твоя и на земле, как на небе;
\vs Luk 11:3 хлеб наш насущный подавай нам на каждый день;
\vs Luk 11:4 и прости нам грехи наши, ибо и мы прощаем всякому должнику нашему; и не введи нас в искушение, но избавь нас от лукавого.
\rsbpar\vs Luk 11:5 И сказал им: \bibemph{положим, что} кто-нибудь из вас, имея друга, придёт к нему в полночь и скажет ему: друг! дай мне взаймы три хлеба,
\vs Luk 11:6 ибо друг мой с дороги зашел ко мне, и мне нечего предложить ему;
\vs Luk 11:7 а тот изнутри скажет ему в ответ: не беспокой меня, двери уже заперты, и дети мои со мною на постели; не могу встать и дать тебе.
\vs Luk 11:8 Если, говорю вам, он не встанет и не даст ему по дружбе с ним, то по неотступности его, встав, даст ему, сколько просит.
\vs Luk 11:9 И Я скажу вам: прос\acc{и}те, и дано будет вам; ищите, и найдете; стучите, и отворят вам,
\vs Luk 11:10 ибо всякий просящий получает, и ищущий находит, и стучащему отворят.
\vs Luk 11:11 Какой из вас отец, \bibemph{когда} сын попросит у него хлеба, подаст ему камень? или, \bibemph{когда попросит} рыбы, подаст ему змею вместо рыбы?
\vs Luk 11:12 Или, если попросит яйца, подаст ему скорпиона?
\vs Luk 11:13 Итак, если вы, будучи злы, умеете даяния благие давать детям вашим, тем более Отец Небесный даст Духа Святаго просящим у Него.
\rsbpar\vs Luk 11:14 Однажды изгнал Он беса, который был нем; и когда бес вышел, немой стал говорить; и народ удивился.
\vs Luk 11:15 Некоторые же из них говорили: Он изгоняет бесов силою веельзевула, князя бесовского.
\vs Luk 11:16 А другие, искушая, требовали от Него знамения с неба.
\vs Luk 11:17 Но Он, зная помышления их, сказал им: всякое царство, разделившееся само в себе, опустеет, и дом, \bibemph{разделившийся} сам в себе, падет;
\vs Luk 11:18 если же и сатана разделится сам в себе, то к\acc{а}к устоит царство его? а вы говорите, что Я силою веельзевула изгоняю бесов;
\vs Luk 11:19 и если Я силою веельзевула изгоняю бесов, то сыновья ваши чьею силою изгоняют их? Посему они будут вам судьями.
\vs Luk 11:20 Если же Я перстом Божиим изгоняю бесов, то, конечно, достигло до вас Царствие Божие.
\vs Luk 11:21 Когда сильный с оружием охраняет свой дом, тогда в безопасности его имение;
\vs Luk 11:22 когда же сильнейший его нападет на него и победит его, тогда возьмет всё оружие его, на которое он надеялся, и разделит похищенное у него.
\vs Luk 11:23 Кто не со Мною, тот против Меня; и кто не собирает со Мною, тот расточает.
\vs Luk 11:24 Когда нечистый дух выйдет из человека, то ходит по безводным местам, ища покоя, и, не находя, говорит: возвращусь в дом мой, откуда вышел;
\vs Luk 11:25 и, придя, находит его выметенным и убранным;
\vs Luk 11:26 тогда идет и берет с собою семь других духов, злейших себя, и, войдя, живут там,~--- и бывает для человека того последнее хуже первого.
\vs Luk 11:27 Когда же Он говорил это, одна женщина, возвысив голос из народа, сказала Ему: блаженно чрево, носившее Тебя, и сосцы, Тебя питавшие!
\vs Luk 11:28 А Он сказал: блаженны слышащие слово Божие и соблюдающие его.
\rsbpar\vs Luk 11:29 Когда же народ стал сходиться во множестве, Он начал говорить: род сей лукав, он ищет знамения, и знамение не дастся ему, кроме знамения Ионы пророка;
\vs Luk 11:30 ибо к\acc{а}к Иона был знамением для Ниневитян, т\acc{а}к будет и Сын Человеческий для рода сего.
\vs Luk 11:31 Царица южная восстанет на суд с людьми рода сего и осудит их, ибо она приходила от пределов земли послушать мудрости Соломоновой; и вот, здесь больше Соломона.
\vs Luk 11:32 Ниневитяне восстанут на суд с родом сим и осудят его, ибо они покаялись от проповеди Иониной, и вот, здесь больше Ионы.
\rsbpar\vs Luk 11:33 Никто, зажегши свечу, не ставит ее в сокровенном месте, ни под сосудом, но на подсвечнике, чтобы входящие видели свет.
\vs Luk 11:34 Светильник тела есть око; итак, если око твое будет чисто, то и все тело твое будет светло; а если оно будет худо, то и тело твое будет темно.
\vs Luk 11:35 Итак, смотри: свет, который в тебе, не есть ли тьма?
\vs Luk 11:36 Если же тело твое всё светло и не имеет ни одной темной части, то будет светло всё т\acc{а}к, как бы светильник освещал тебя сиянием.
\rsbpar\vs Luk 11:37 Когда Он говорил это, один фарисей просил Его к себе обедать. Он пришел и возлег.
\vs Luk 11:38 Фарисей же удивился, увидев, что Он не умыл \bibemph{рук} перед обедом.
\vs Luk 11:39 Но Господь сказал ему: ныне вы, фарисеи, внешность чаши и блюда очищаете, а внутренность ваша исполнена хищения и лукавства.
\vs Luk 11:40 Неразумные! не Тот же ли, Кто сотворил внешнее, сотворил и внутреннее?
\vs Luk 11:41 Подавайте лучше милостыню из того, чт\acc{о} у вас есть, тогда всё будет у вас чисто.
\vs Luk 11:42 Но горе вам, фарисеям, что даете десятину с мяты, руты и всяких овощей, и нерадите о суде и любви Божией: сие надлежало делать, и того не оставлять.
\vs Luk 11:43 Горе вам, фарисеям, что любите председания в синагогах и приветствия в народных собраниях.
\vs Luk 11:44 Горе вам, книжники и фарисеи, лицемеры, что вы~--- как гробы скрытые, над которыми люди ходят и не знают того.
\vs Luk 11:45 На это некто из законников сказал Ему: Учитель! говоря это, Ты и нас обижаешь.
\vs Luk 11:46 Но Он сказал: и вам, законникам, горе, что налагаете на людей бремена неудобоносимые, а сами и одним перстом своим не дотрагиваетесь до них.
\vs Luk 11:47 Горе вам, что строите гробницы пророкам, которых избили отцы ваши:
\vs Luk 11:48 сим вы свидетельствуете о делах отцов ваших и соглашаетесь с ними, ибо они избили пророков, а вы строите им гробницы.
\vs Luk 11:49 Потому и премудрость Божия сказала: пошлю к ним пророков и Апостолов, и из них одних убьют, а других изгонят,
\vs Luk 11:50 да взыщется от рода сего кровь всех пророков, пролитая от создания мира,
\vs Luk 11:51 от крови Авеля до крови Захарии, убитого между жертвенником и храмом. Ей, говорю вам, взыщется от рода сего.
\vs Luk 11:52 Горе вам, законникам, что вы взяли ключ разумения: сами не вошли, и входящим воспрепятствовали.
\vs Luk 11:53 Когда Он говорил им это, книжники и фарисеи начали сильно приступать к Нему, вынуждая у Него ответы на многое,
\vs Luk 11:54 подыскиваясь под Него и стараясь уловить что-нибудь из уст Его, чтобы обвинить Его.
\vs Luk 12:1 Между тем, когда собрались тысячи народа, так что теснили друг друга, Он начал говорить сперва ученикам Своим: берегитесь закваски фарисейской, которая есть лицемерие.
\vs Luk 12:2 Нет ничего сокровенного, что не открылось бы, и тайного, чего не узнали бы.
\vs Luk 12:3 Посему, чт\acc{о} вы сказали в темноте, т\acc{о} услышится во свете; и чт\acc{о} говорили на ухо внутри дома, т\acc{о} будет провозглашено на кровлях.
\vs Luk 12:4 Говорю же вам, друзьям Моим: не бойтесь убивающих тело и потом не могущих ничего более сделать;
\vs Luk 12:5 но скажу вам, кого бояться: бойтесь Того, Кто, по убиении, может ввергнуть в геенну: ей, говорю вам, Того бойтесь.
\vs Luk 12:6 Не пять ли малых птиц продаются за два ассария? и ни одна из них не забыта у Бога.
\vs Luk 12:7 А у вас и волосы на голове все сочтены. Итак не бойтесь: вы дороже многих малых птиц.
\vs Luk 12:8 Сказываю же вам: всякого, кто исповедает Меня пред человеками, и Сын Человеческий исповедает пред Ангелами Божиими;
\vs Luk 12:9 а кто отвергнется Меня пред человеками, тот отвержен будет пред Ангелами Божиими.
\vs Luk 12:10 И всякому, кто скажет слово на Сына Человеческого, прощено будет; а кто скажет хулу на Святаго Духа, тому не простится.
\vs Luk 12:11 Когда же приведут вас в синагоги, к начальствам и властям, не заботьтесь, к\acc{а}к или чт\acc{о} отвечать, или чт\acc{о} говорить,
\vs Luk 12:12 ибо Святый Дух научит вас в тот час, чт\acc{о} должно говорить.
\rsbpar\vs Luk 12:13 Некто из народа сказал Ему: Учитель! скажи брату моему, чтобы он разделил со мною наследство.
\vs Luk 12:14 Он же сказал человеку тому: кто поставил Меня судить или делить вас?
\vs Luk 12:15 При этом сказал им: смотрите, берегитесь любостяжания, ибо жизнь человека не зависит от изобилия его имения.
\vs Luk 12:16 И сказал им притчу: у одного богатого человека был хороший урожай в поле;
\vs Luk 12:17 и он рассуждал сам с собою: что мне делать? некуда мне собрать плодов моих?
\vs Luk 12:18 И сказал: вот что сделаю: сломаю житницы мои и построю б\acc{о}льшие, и соберу туда весь хлеб мой и всё добро мое,
\vs Luk 12:19 и скажу душе моей: душа! много добра лежит у тебя на многие годы: покойся, ешь, пей, веселись.
\vs Luk 12:20 Но Бог сказал ему: безумный! в сию ночь душу твою возьмут у тебя; кому же достанется то, что ты заготовил?
\vs Luk 12:21 Так \bibemph{бывает с тем}, кто собирает сокровища для себя, а не в Бога богатеет.
\vs Luk 12:22 И сказал ученикам Своим: посему говорю вам,~--- не заботьтесь для души вашей, что вам есть, ни для тела, во что одеться:
\vs Luk 12:23 душа больше пищи, и тело~--- одежды.
\vs Luk 12:24 Посмотрите на воронов: они не сеют, не жнут; нет у них ни хранилищ, ни житниц, и Бог питает их; сколько же вы лучше птиц?
\vs Luk 12:25 Да и кто из вас, заботясь, может прибавить себе роста хотя на один локоть?
\vs Luk 12:26 Итак, если и малейшего сделать не можете, чт\acc{о} заботитесь о прочем?
\vs Luk 12:27 Посмотрите на лилии, как они растут: не трудятся, не прядут; но говорю вам, что и Соломон во всей славе своей не одевался так, как всякая из них.
\vs Luk 12:28 Если же траву на поле, которая сегодня есть, а завтра будет брошена в печь, Бог так одевает, то кольми паче вас, маловеры!
\vs Luk 12:29 Итак, не ищите, чт\acc{о} вам есть, или чт\acc{о} пить, и не беспокойтесь,
\vs Luk 12:30 потому что всего этого ищут люди мира сего; ваш же Отец знает, что вы имеете нужду в том;
\vs Luk 12:31 наипаче ищите Царствия Божия, и это всё приложится вам.
\vs Luk 12:32 Не бойся, малое стадо! ибо Отец ваш благоволил дать вам Царство.
\vs Luk 12:33 Продавайте имения ваши и давайте милостыню. Приготовляйте себе влагалища не ветшающие, сокровище неоскудевающее на небесах, куда вор не приближается и где моль не съедает,
\vs Luk 12:34 ибо где сокровище ваше, там и сердце ваше будет.
\rsbpar\vs Luk 12:35 Да будут чресла ваши препоясаны и светильники горящи.
\vs Luk 12:36 И вы будьте подобны людям, ожидающим возвращения господина своего с брака, дабы, когда придёт и постучит, тотчас отворить ему.
\vs Luk 12:37 Блаженны рабы те, которых господин, придя, найдёт бодрствующими; истинно говорю вам, он препояшется и посадит их, и, подходя, станет служить им.
\vs Luk 12:38 И если придет во вторую стражу, и в третью стражу придет, и найдет их так, то блаженны рабы те.
\vs Luk 12:39 Вы знаете, что если бы ведал хозяин дома, в который час придет вор, то бодрствовал бы и не допустил бы подкопать дом свой.
\vs Luk 12:40 Будьте же и вы готовы, ибо, в который час не думаете, приидет Сын Человеческий.
\vs Luk 12:41 Тогда сказал Ему Петр: Господи! к нам ли притчу сию говоришь, или и ко всем?
\vs Luk 12:42 Господь же сказал: кт\acc{о} верный и благоразумный домоправитель, которого господин поставил над слугами своими раздавать им в своё время меру хлеба?
\vs Luk 12:43 Блажен раб тот, которого господин его, придя, найдет поступающим так.
\vs Luk 12:44 Истинно говорю вам, что над всем имением своим поставит его.
\vs Luk 12:45 Если же раб тот скажет в сердце своем: не скоро придет господин мой, и начнет бить слуг и служанок, есть и пить и напиваться,~---
\vs Luk 12:46 то придет господин раба того в день, в который он не ожидает, и в час, в который не думает, и рассечет его, и подвергнет его одной участи с неверными.
\vs Luk 12:47 Раб же тот, который знал волю господина своего, и не был готов, и не делал по воле его, бит будет много;
\vs Luk 12:48 а который не знал, и сделал достойное наказания, бит будет меньше. И от всякого, кому дано много, много и потребуется, и кому много вверено, с того больше взыщут.
\vs Luk 12:49 Огонь пришел Я низвести на землю, и как желал бы, чтобы он уже возгорелся!
\vs Luk 12:50 Крещением должен Я креститься; и как Я томлюсь, пока сие совершится!
\vs Luk 12:51 Думаете ли вы, что Я пришел дать мир земле? Нет, говорю вам, но разделение;
\vs Luk 12:52 ибо отныне пятеро в одном доме станут разделяться, трое против двух, и двое против трех:
\vs Luk 12:53 отец будет против сына, и сын против отца; мать против дочери, и дочь против матери; свекровь против невестки своей, и невестка против свекрови своей.
\vs Luk 12:54 Сказал же и народу: когда вы видите облако, поднимающееся с запада, тотчас говорите: дождь будет, и бывает так;
\vs Luk 12:55 и когда дует южный ветер, говорите: зной будет, и бывает.
\vs Luk 12:56 Лицемеры! лице земли и неба распознавать умеете, как же времени сего не узнаете?
\vs Luk 12:57 Зачем же вы и по самим себе не судите, чему быть должно?
\vs Luk 12:58 Когда ты идешь с соперником своим к начальству, то на дороге постарайся освободиться от него, чтобы он не привел тебя к судье, а судья не отдал тебя истязателю, а истязатель не вверг тебя в темницу.
\vs Luk 12:59 Сказываю тебе: не выйдешь оттуда, пока не отдашь и последней полушки.
\vs Luk 13:1 В это время пришли некоторые и рассказали Ему о Галилеянах, которых кровь Пилат смешал с жертвами их.
\vs Luk 13:2 Иисус сказал им на это: думаете ли вы, что эти Галилеяне были грешнее всех Галилеян, что так пострадали?
\vs Luk 13:3 Нет, говорю вам, но, если не покаетесь, все т\acc{а}к же погибнете.
\vs Luk 13:4 Или думаете ли, что те восемнадцать человек, на которых упала башня Силоамская и побила их, виновнее были всех, живущих в Иерусалиме?
\vs Luk 13:5 Нет, говорю вам, но, если не покаетесь, все т\acc{а}к же погибнете.
\vs Luk 13:6 И сказал сию притчу: некто имел в винограднике своем посаженную смоковницу, и пришел искать плода на ней, и не нашел;
\vs Luk 13:7 и сказал виноградарю: вот, я третий год прихожу искать плода на этой смоковнице и не нахожу; сруби ее: на что она и землю занимает?
\vs Luk 13:8 Но он сказал ему в ответ: господин! оставь ее и на этот год, пока я окопаю ее и обложу навозом,~---
\vs Luk 13:9 не принесет ли плода; если же нет, то в следующий \bibemph{год} срубишь ее.
\rsbpar\vs Luk 13:10 В одной из синагог учил Он в субботу.
\vs Luk 13:11 Там была женщина, восемнадцать лет имевшая духа немощи: она была скорчена и не могла выпрямиться.
\vs Luk 13:12 Иисус, увидев ее, подозвал и сказал ей: женщина! ты освобождаешься от недуга твоего.
\vs Luk 13:13 И возложил на нее руки, и она тотчас выпрямилась и стала славить Бога.
\vs Luk 13:14 При этом начальник синагоги, негодуя, что Иисус исцелил в субботу, сказал народу: есть шесть дней, в которые должно делать; в те и приход\acc{и}те исцеляться, а не в день субботний.
\vs Luk 13:15 Господь сказал ему в ответ: лицемер! не отвязывает ли каждый из вас вола своего или осла от яслей в субботу и не ведет ли поить?
\vs Luk 13:16 сию же дочь Авраамову, которую связал сатана вот уже восемнадцать лет, не надлежало ли освободить от уз сих в день субботний?
\vs Luk 13:17 И когда говорил Он это, все противившиеся Ему стыдились; и весь народ радовался о всех славных делах Его.
\rsbpar\vs Luk 13:18 Он же сказал: чему подобно Царствие Божие? и чему уподоблю его?
\vs Luk 13:19 Оно подобно зерну горчичному, которое, взяв, человек посадил в саду своем; и выросло, и стало большим деревом, и птицы небесные укрывались в ветвях его.
\vs Luk 13:20 Ещё сказал: чему уподоблю Царствие Божие?
\vs Luk 13:21 Оно подобно закваске, которую женщина, взяв, положила в три меры муки, доколе не вскисло всё.
\rsbpar\vs Luk 13:22 И проходил по городам и селениям, уча и направляя путь к Иерусалиму.
\rsbpar\vs Luk 13:23 Некто сказал Ему: Господи! неужели мало спасающихся? Он же сказал им:
\vs Luk 13:24 подвизайтесь войти сквозь тесные врата, ибо, сказываю вам, многие поищут войти, и не возмогут.
\vs Luk 13:25 Когда хозяин дома встанет и затворит двери, тогда вы, стоя вне, станете стучать в двери и говорить: Господи! Господи! отвори нам; но Он скажет вам в ответ: не знаю вас, откуда вы.
\vs Luk 13:26 Тогда станете говорить: мы ели и пили пред Тобою, и на улицах наших учил Ты.
\vs Luk 13:27 Но Он скажет: говорю вам: не знаю вас, откуда вы; отойдите от Меня все делатели неправды.
\vs Luk 13:28 Там будет плач и скрежет зубов, когда увидите Авраама, Исаака и Иакова и всех пророков в Царствии Божием, а себя изгоняемыми вон.
\vs Luk 13:29 И придут от востока и запада, и севера и юга, и возлягут в Царствии Божием.
\vs Luk 13:30 И вот, есть последние, которые будут первыми, и есть первые, которые будут последними.
\rsbpar\vs Luk 13:31 В тот день пришли некоторые из фарисеев и говорили Ему: выйди и удались отсюда, ибо Ирод хочет убить Тебя.
\vs Luk 13:32 И сказал им: пойдите, скажите этой лисице: се, изгоняю бесов и совершаю исцеления сегодня и завтра, и в третий \bibemph{день} кончу;
\vs Luk 13:33 а впрочем, Мне должно ходить сегодня, завтра и в последующий день, потому что не бывает, чтобы пророк погиб вне Иерусалима.
\vs Luk 13:34 Иерусалим! Иерусалим! избивающий пророков и камнями побивающий посланных к тебе! сколько раз хотел Я собрать чад твоих, как птица птенцов своих под крылья, и вы не захотели!
\vs Luk 13:35 Се, оставляется вам дом ваш пуст. Сказываю же вам, что вы не увидите Меня, пока не придет время, когда скажете: благословен Грядый во имя Господне!
\vs Luk 14:1 Случилось Ему в субботу прийти в дом одного из начальников фарисейских вкусить хлеба, и они наблюдали за Ним.
\vs Luk 14:2 И вот, предстал пред Него человек, страждущий водяною болезнью.
\vs Luk 14:3 По сему случаю Иисус спросил законников и фарисеев: позволительно ли врачевать в субботу?
\vs Luk 14:4 Они молчали. И, прикоснувшись, исцелил его и отпустил.
\vs Luk 14:5 При сем сказал им: если у кого из вас осёл или вол упадет в колодезь, не тотчас ли вытащит его и в субботу?
\vs Luk 14:6 И не могли отвечать Ему на это.
\rsbpar\vs Luk 14:7 Замечая же, как званые выбирали первые места, сказал им притчу:
\vs Luk 14:8 когда ты будешь позван кем на брак, не садись на первое место, чтобы не случился кто из званых им почетнее тебя,
\vs Luk 14:9 и звавший тебя и его, подойдя, не сказал бы тебе: уступи ему место; и тогда со стыдом должен будешь занять последнее место.
\vs Luk 14:10 Но когда зван будешь, придя, садись на последнее место, чтобы звавший тебя, подойдя, сказал: друг! пересядь выше; тогда будет тебе честь пред сидящими с тобою,
\vs Luk 14:11 ибо всякий возвышающий сам себя унижен будет, а унижающий себя возвысится.
\vs Luk 14:12 Сказал же и позвавшему Его: когда делаешь обед или ужин, не зови друзей твоих, ни братьев твоих, ни родственников твоих, ни соседей богатых, чтобы и они тебя когда не позвали, и не получил ты воздаяния.
\vs Luk 14:13 Но, когда делаешь пир, зови нищих, увечных, хромых, слепых,
\vs Luk 14:14 и блажен будешь, что они не могут воздать тебе, ибо воздастся тебе в воскресение праведных.
\vs Luk 14:15 Услышав это, некто из возлежащих с Ним сказал Ему: блажен, кто вкусит хлеба в Царствии Божием!
\vs Luk 14:16 Он же сказал ему: один человек сделал большой ужин и звал многих,
\vs Luk 14:17 и когда наступило время ужина, послал раба своего сказать званым: идите, ибо уже всё готово.
\vs Luk 14:18 И начали все, как бы сговорившись, извиняться. Первый сказал ему: я купил землю и мне нужно пойти посмотреть ее; прошу тебя, извини меня.
\vs Luk 14:19 Другой сказал: я купил пять пар волов и иду испытать их; прошу тебя, извини меня.
\vs Luk 14:20 Третий сказал: я женился и потому не могу прийти.
\vs Luk 14:21 И, возвратившись, раб тот донес о сем господину своему. Тогда, разгневавшись, хозяин дома сказал рабу своему: пойди скорее по улицам и переулкам города и приведи сюда нищих, увечных, хромых и слепых.
\vs Luk 14:22 И сказал раб: господин! исполнено, как приказал ты, и еще есть место.
\vs Luk 14:23 Господин сказал рабу: пойди по дорогам и изгородям и убеди прийти, чтобы наполнился дом мой.
\vs Luk 14:24 Ибо сказываю вам, что никто из тех званых не вкусит моего ужина, ибо много званых, но мало избранных.
\rsbpar\vs Luk 14:25 С Ним шло множество народа; и Он, обратившись, сказал им:
\vs Luk 14:26 если кто приходит ко Мне и не возненавидит отца своего и матери, и жены и детей, и братьев и сестер, а притом и самой жизни своей, тот не может быть Моим учеником;
\vs Luk 14:27 и кто не несет креста своего и идёт за Мною, не может быть Моим учеником.
\vs Luk 14:28 Ибо кто из вас, желая построить башню, не сядет прежде и не вычислит издержек, имеет ли он, что нужно для совершения ее,
\vs Luk 14:29 дабы, когда положит основание и не возможет совершить, все видящие не стали смеяться над ним,
\vs Luk 14:30 говоря: этот человек начал строить и не мог окончить?
\vs Luk 14:31 Или какой царь, идя на войну против другого царя, не сядет и не посоветуется прежде, силен ли он с десятью тысячами противостать идущему на него с двадцатью тысячами?
\vs Luk 14:32 Иначе, пока тот еще далеко, он пошлет к нему посольство просить о мире.
\vs Luk 14:33 Так всякий из вас, кто не отрешится от всего, что имеет, не может быть Моим учеником.
\vs Luk 14:34 Соль~--- добрая вещь; но если соль потеряет силу, чем исправить ее?
\vs Luk 14:35 ни в землю, ни в навоз не годится; вон выбрасывают ее. Кто имеет уши слышать, да слышит!
\vs Luk 15:1 Приближались к Нему все мытари и грешники слушать Его.
\vs Luk 15:2 Фарисеи же и книжники роптали, говоря: Он принимает грешников и ест с ними.
\vs Luk 15:3 Но Он сказал им следующую притчу:
\vs Luk 15:4 кто из вас, имея сто овец и потеряв одну из них, не оставит девяноста девяти в пустыне и не пойдет за пропавшею, пока не найдет ее?
\vs Luk 15:5 А найдя, возьмет ее на плечи свои с радостью
\vs Luk 15:6 и, придя домой, созовет друзей и соседей и скажет им: порадуйтесь со мною: я нашел мою пропавшую овцу.
\vs Luk 15:7 Сказываю вам, что так на небесах более радости будет об одном грешнике кающемся, нежели о девяноста девяти праведниках, не имеющих нужды в покаянии.
\vs Luk 15:8 Или какая женщина, имея десять драхм, если потеряет одну драхму, не зажжет свеч\acc{и} и не станет мести комнату и искать тщательно, пока не найдет,
\vs Luk 15:9 а найдя, созовет подруг и соседок и скажет: порадуйтесь со мною: я нашла потерянную драхму.
\vs Luk 15:10 Так, говорю вам, бывает радость у Ангелов Божиих и об одном грешнике кающемся.
\rsbpar\vs Luk 15:11 Еще сказал: у некоторого человека было два сына;
\vs Luk 15:12 и сказал младший из них отцу: отче! дай мне следующую \bibemph{мне} часть имения. И \bibemph{отец} разделил им имение.
\vs Luk 15:13 По прошествии немногих дней младший сын, собрав всё, пошел в дальнюю сторону и там расточил имение свое, живя распутно.
\vs Luk 15:14 Когда же он прожил всё, настал великий голод в той стране, и он начал нуждаться;
\vs Luk 15:15 и пошел, пристал к одному из жителей страны той, а тот послал его на поля свои пасти свиней;
\vs Luk 15:16 и он рад был наполнить чрево свое рожк\acc{а}ми, которые ели свиньи, но никто не давал ему.
\vs Luk 15:17 Придя же в себя, сказал: сколько наемников у отца моего избыточествуют хлебом, а я умираю от голода;
\vs Luk 15:18 встану, пойду к отцу моему и скажу ему: отче! я согрешил против неба и пред тобою
\vs Luk 15:19 и уже недостоин называться сыном твоим; прими меня в число наемников твоих.
\vs Luk 15:20 Встал и пошел к отцу своему. И когда он был еще далеко, увидел его отец его и сжалился; и, побежав, пал ему на шею и целовал его.
\vs Luk 15:21 Сын же сказал ему: отче! я согрешил против неба и пред тобою и уже недостоин называться сыном твоим.
\vs Luk 15:22 А отец сказал рабам своим: принесите лучшую одежду и оденьте его, и дайте перстень на руку его и обувь на ноги;
\vs Luk 15:23 и приведите откормленного теленка, и заколите; станем есть и веселиться!
\vs Luk 15:24 ибо этот сын мой был мертв и ожил, пропадал и нашелся. И начали веселиться.
\vs Luk 15:25 Старший же сын его был на поле; и возвращаясь, когда приблизился к дому, услышал пение и ликование;
\vs Luk 15:26 и, призвав одного из слуг, спросил: что это такое?
\vs Luk 15:27 Он сказал ему: брат твой пришел, и отец твой заколол откормленного теленка, потому что принял его здоровым.
\vs Luk 15:28 Он осердился и не хотел войти. Отец же его, выйдя, звал его.
\vs Luk 15:29 Но он сказал в ответ отцу: вот, я столько лет служу тебе и никогда не преступал приказания твоего, но ты никогда не дал мне и козлёнка, чтобы мне повеселиться с друзьями моими;
\vs Luk 15:30 а когда этот сын твой, расточивший имение своё с блудницами, пришел, ты заколол для него откормленного теленка.
\vs Luk 15:31 Он же сказал ему: сын мой! ты всегда со мною, и всё мое твое,
\vs Luk 15:32 а о том надобно было радоваться и веселиться, что брат твой сей был мертв и ожил, пропадал и нашелся.
\vs Luk 16:1 Сказал же и к ученикам Своим: один человек был богат и имел управителя, на которого донесено было ему, что расточает имение его;
\vs Luk 16:2 и, призвав его, сказал ему: что это я слышу о тебе? дай отчет в управлении твоем, ибо ты не можешь более управлять.
\vs Luk 16:3 Тогда управитель сказал сам в себе: что мне делать? господин мой отнимает у меня управление домом; копать не могу, просить стыжусь;
\vs Luk 16:4 знаю, что сделать, чтобы приняли меня в домы свои, когда отставлен буду от управления домом.
\vs Luk 16:5 И, призвав должников господина своего, каждого порознь, сказал первому: сколько ты должен господину моему?
\vs Luk 16:6 Он сказал: сто мер масла. И сказал ему: возьми твою расписку и садись скорее, напиши: пятьдесят.
\vs Luk 16:7 Потом другому сказал: а ты сколько должен? Он отвечал: сто мер пшеницы. И сказал ему: возьми твою расписку и напиши: восемьдесят.
\vs Luk 16:8 И похвалил господин управителя неверного, что догадливо поступил; ибо сыны века сего догадливее сынов света в своем роде.
\vs Luk 16:9 И Я говорю вам: приобретайте себе друзей богатством неправедным, чтобы они, когда обнищаете, приняли вас в вечные обители.
\vs Luk 16:10 Верный в малом и во многом верен, а неверный в малом неверен и во многом.
\vs Luk 16:11 Итак, если вы в неправедном богатстве не были верны, кто поверит вам истинное?
\vs Luk 16:12 И если в чужом не были верны, кто даст вам ваше?
\vs Luk 16:13 Никакой слуга не может служить двум господам, ибо или одного будет ненавидеть, а другого любить, или одному станет усердствовать, а о другом нерадеть. Не можете служить Богу и маммоне.
\rsbpar\vs Luk 16:14 Слышали всё это и фарисеи, которые были сребролюбивы, и они смеялись над Ним.
\vs Luk 16:15 Он сказал им: вы выказываете себя праведниками пред людьми, но Бог знает сердц\acc{а} ваши, ибо что высоко у людей, т\acc{о} мерзость пред Богом.
\vs Luk 16:16 Закон и пророки до Иоанна; с сего времени Царствие Божие благовествуется, и всякий усилием входит в него.
\vs Luk 16:17 Но скорее небо и земля прейдут, нежели одна черта из закона пропадет.
\vs Luk 16:18 Всякий, разводящийся с женою своею и женящийся на другой, прелюбодействует, и всякий, женящийся на разведенной с мужем, прелюбодействует.
\rsbpar\vs Luk 16:19 Некоторый человек был богат, одевался в порфиру и виссон и каждый день пиршествовал блистательно.
\vs Luk 16:20 Был также некоторый нищий, именем Лазарь, который лежал у ворот его в струпьях
\vs Luk 16:21 и желал напитаться крошками, падающими со стола богача, и псы, приходя, лизали струпья его.
\vs Luk 16:22 Умер нищий и отнесен был Ангелами на лоно Авраамово. Умер и богач, и похоронили его.
\vs Luk 16:23 И в аде, будучи в муках, он поднял глаза свои, увидел вдали Авраама и Лазаря на лоне его
\vs Luk 16:24 и, возопив, сказал: отче Аврааме! умилосердись надо мною и пошли Лазаря, чтобы омочил конец перста своего в воде и прохладил язык мой, ибо я мучаюсь в пламени сем.
\vs Luk 16:25 Но Авраам сказал: чадо! вспомни, что ты получил уже доброе твое в жизни твоей, а Лазарь~--- злое; ныне же он здесь утешается, а ты страдаешь;
\vs Luk 16:26 и сверх всего того между нами и вами утверждена великая пропасть, так что хотящие перейти отсюда к вам не могут, также и оттуда к нам не переходят.
\vs Luk 16:27 Тогда сказал он: так прошу тебя, отче, пошли его в дом отца моего,
\vs Luk 16:28 ибо у меня пять братьев; пусть он засвидетельствует им, чтобы и они не пришли в это место мучения.
\vs Luk 16:29 Авраам сказал ему: у них есть Моисей и пророки; пусть слушают их.
\vs Luk 16:30 Он же сказал: нет, отче Аврааме, но если кто из мертвых придет к ним, покаются.
\vs Luk 16:31 Тогда \bibemph{Авраам} сказал ему: если Моисея и пророков не слушают, то если бы кто и из мертвых воскрес, не поверят.
\vs Luk 17:1 Сказал также \bibemph{Иисус} ученикам: невозможно не прийти соблазнам, но горе тому, через кого они приходят;
\vs Luk 17:2 лучше было бы ему, если бы мельничный жернов повесили ему на шею и бросили его в море, нежели чтобы он соблазнил одного из малых сих.
\vs Luk 17:3 Наблюдайте за собою. Если же согрешит против тебя брат твой, выговори ему; и если покается, прости ему;
\vs Luk 17:4 и если семь раз в день согрешит против тебя и семь раз в день обратится, и скажет: каюсь,~--- прости ему.
\rsbpar\vs Luk 17:5 И сказали Апостолы Господу: умножь в нас веру.
\vs Luk 17:6 Господь сказал: если бы вы имели веру с зерно горчичное и сказали смоковнице сей: исторгнись и пересадись в море, то она послушалась бы вас.
\vs Luk 17:7 Кто из вас, имея раба п\acc{а}шущего или пасущего, по возвращении его с поля, скажет ему: пойди скорее, садись за стол?
\vs Luk 17:8 Напротив, не скажет ли ему: приготовь мне поужинать и, подпоясавшись, служи мне, пока буду есть и пить, и потом ешь и пей сам?
\vs Luk 17:9 Станет ли он благодарить раба сего за то, что он исполнил приказание? Не думаю.
\vs Luk 17:10 Так и вы, когда исполните всё повеленное вам, говорите: мы рабы ничего не стоящие, потому что сделали, чт\acc{о} должны были сделать.
\rsbpar\vs Luk 17:11 Идя в Иерусалим, Он проходил между Самариею и Галилеею.
\vs Luk 17:12 И когда входил Он в одно селение, встретили Его десять человек прокаженных, которые остановились вдали
\vs Luk 17:13 и громким голосом говорили: Иисус Наставник! помилуй нас.
\vs Luk 17:14 Увидев \bibemph{их}, Он сказал им: пойдите, покажитесь священникам. И когда они шли, очистились.
\vs Luk 17:15 Один же из них, видя, что исцелен, возвратился, громким голосом прославляя Бога,
\vs Luk 17:16 и пал ниц к ногам Его, благодаря Его; и это был Самарянин.
\vs Luk 17:17 Тогда Иисус сказал: не десять ли очистились? где же девять?
\vs Luk 17:18 как они не возвратились воздать славу Богу, кроме сего иноплеменника?
\vs Luk 17:19 И сказал ему: встань, иди; вера твоя спасла тебя.
\rsbpar\vs Luk 17:20 Быв же спрошен фарисеями, когда придет Царствие Божие, отвечал им: не придет Царствие Божие приметным образом,
\vs Luk 17:21 и не скажут: вот, оно здесь, или: вот, там. Ибо вот, Царствие Божие внутрь вас есть.
\vs Luk 17:22 Сказал также ученикам: придут дни, когда пожелаете видеть хотя один из дней Сына Человеческого, и не увидите;
\vs Luk 17:23 и скажут вам: вот, здесь, или: вот, там,~--- не ходите и не гоняйтесь,
\vs Luk 17:24 ибо, как молния, сверкнувшая от одного края неба, блистает до другого края неба, так будет Сын Человеческий в день Свой.
\vs Luk 17:25 Но прежде надлежит Ему много пострадать и быть отвержену родом сим.
\vs Luk 17:26 И как было во дни Ноя, так будет и во дни Сына Человеческого:
\vs Luk 17:27 ели, пили, женились, выходили замуж, до того дня, как вошел Ной в ковчег, и пришел потоп и погубил всех.
\vs Luk 17:28 Т\acc{а}к же, к\acc{а}к было и во дни Лота: ели, пили, покупали, продавали, садили, строили;
\vs Luk 17:29 но в день, в который Лот вышел из Содома, пролился с неба дождь огненный и серный и истребил всех;
\vs Luk 17:30 так будет и в тот день, когда Сын Человеческий явится.
\vs Luk 17:31 В тот день, кто будет на кровле, а вещи его в доме, тот не сходи взять их; и кто будет на поле, также не обращайся назад.
\vs Luk 17:32 Вспоминайте жену Лотову.
\vs Luk 17:33 Кто станет сберегать душу свою, тот погубит ее; а кто погубит ее, тот оживит ее.
\vs Luk 17:34 Сказываю вам: в ту ночь будут двое на одной постели: один возьмется, а другой оставится;
\vs Luk 17:35 две будут молоть вместе: одна возьмется, а другая оставится;
\vs Luk 17:36 двое будут на поле: один возьмется, а другой оставится.
\vs Luk 17:37 На это сказали Ему: где, Господи? Он же сказал им: где труп, там соберутся и орлы.
\vs Luk 18:1 Сказал также им притчу о том, что должно всегда молиться и не унывать,
\vs Luk 18:2 говоря: в одном городе был судья, который Бога не боялся и людей не стыдился.
\vs Luk 18:3 В том же городе была одна вдова, и она, приходя к нему, говорила: защити меня от соперника моего.
\vs Luk 18:4 Но он долгое время не хотел. А после сказал сам в себе: хотя я и Бога не боюсь и людей не стыжусь,
\vs Luk 18:5 но, как эта вдова не дает мне покоя, защищу ее, чтобы она не приходила больше докучать мне.
\vs Luk 18:6 И сказал Господь: слышите, что говорит судья неправедный?
\vs Luk 18:7 Бог ли не защитит избранных Своих, вопиющих к Нему день и ночь, хотя и медлит защищать их?
\vs Luk 18:8 сказываю вам, что подаст им защиту вскоре. Но Сын Человеческий, придя, найдет ли веру на земле?
\rsbpar\vs Luk 18:9 Сказал также к некоторым, которые уверены были о себе, что они праведны, и уничижали других, следующую притчу:
\vs Luk 18:10 два человека вошли в храм помолиться: один фарисей, а другой мытарь.
\vs Luk 18:11 Фарисей, став, молился сам в себе так: Боже! благодарю Тебя, что я не таков, как прочие люди, грабители, обидчики, прелюбодеи, или как этот мытарь:
\vs Luk 18:12 пощусь два раза в неделю, даю десятую часть из всего, чт\acc{о} приобретаю.
\vs Luk 18:13 Мытарь же, стоя вдали, не смел даже поднять глаз на небо; но, ударяя себя в грудь, говорил: Боже! будь милостив ко мне грешнику!
\vs Luk 18:14 Сказываю вам, что сей пошел оправданным в дом свой более, нежели тот: ибо всякий, возвышающий сам себя, унижен будет, а унижающий себя возвысится.
\rsbpar\vs Luk 18:15 Приносили к Нему и младенцев, чтобы Он прикоснулся к ним; ученики же, видя то, возбраняли им.
\vs Luk 18:16 Но Иисус, подозвав их, сказал: пустите детей приходить ко Мне и не возбраняйте им, ибо таковых есть Царствие Божие.
\vs Luk 18:17 Истинно говорю вам: кто не примет Царствия Божия, как дитя, тот не войдет в него.
\rsbpar\vs Luk 18:18 И спросил Его некто из начальствующих: Учитель благий! что мне делать, чтобы наследовать жизнь вечную?
\vs Luk 18:19 Иисус сказал ему: что ты называешь Меня благим? никто не благ, как только один Бог;
\vs Luk 18:20 знаешь заповеди: не прелюбодействуй, не убивай, не кради, не лжесвидетельствуй, почитай отца твоего и матерь твою.
\vs Luk 18:21 Он же сказал: все это сохранил я от юности моей.
\vs Luk 18:22 Услышав это, Иисус сказал ему: еще одного недостает тебе: все, что имеешь, продай и раздай нищим, и будешь иметь сокровище на небесах, и приходи, следуй за Мною.
\vs Luk 18:23 Он же, услышав сие, опечалился, потому что был очень богат.
\vs Luk 18:24 Иисус, видя, что он опечалился, сказал: как трудно имеющим богатство войти в Царствие Божие!
\vs Luk 18:25 ибо удобнее верблюду пройти сквозь игольные уши, нежели богатому войти в Царствие Божие.
\vs Luk 18:26 Слышавшие сие сказали: кто же может спастись?
\vs Luk 18:27 Но Он сказал: невозможное человекам возможно Богу.
\rsbpar\vs Luk 18:28 Петр же сказал: вот, мы оставили все и последовали за Тобою.
\vs Luk 18:29 Он сказал им: истинно говорю вам: нет никого, кто оставил бы дом, или родителей, или братьев, или сестер, или жену, или детей для Царствия Божия,
\vs Luk 18:30 и не получил бы гораздо более в сие время, и в век будущий жизни вечной.
\rsbpar\vs Luk 18:31 Отозвав же двенадцать учеников Своих, сказал им: вот, мы восходим в Иерусалим, и совершится все, написанное через пророков о Сыне Человеческом,
\vs Luk 18:32 ибо предадут Его язычникам, и поругаются над Ним, и оскорбят Его, и оплюют Его,
\vs Luk 18:33 и будут бить, и убьют Его: и в третий день воскреснет.
\vs Luk 18:34 Но они ничего из этого не поняли; слова сии были для них сокровенны, и они не разумели сказанного.
\rsbpar\vs Luk 18:35 Когда же подходил Он к Иерихону, один слепой сидел у дороги, прося милостыни,
\vs Luk 18:36 и, услышав, что мимо него проходит народ, спросил: что это такое?
\vs Luk 18:37 Ему сказали, что Иисус Назорей идет.
\vs Luk 18:38 Тогда он закричал: Иисус, Сын Давидов! помилуй меня.
\vs Luk 18:39 Шедшие впереди заставляли его молчать; но он еще громче кричал: Сын Давидов! помилуй меня.
\vs Luk 18:40 Иисус, остановившись, велел привести его к Себе: и, когда тот подошел к Нему, спросил его:
\vs Luk 18:41 чего ты хочешь от Меня? Он сказал: Господи! чтобы мне прозреть.
\vs Luk 18:42 Иисус сказал ему: прозри! вера твоя спасла тебя.
\vs Luk 18:43 И он тотчас прозрел и пошел за Ним, славя Бога; и весь народ, видя это, воздал хвалу Богу.
\vs Luk 19:1 Потом \bibemph{Иисус} вошел в Иерихон и проходил через него.
\vs Luk 19:2 И вот, некто, именем Закхей, начальник мытарей и человек богатый,
\vs Luk 19:3 искал видеть Иисуса, кто Он, но не мог за народом, потому что мал был ростом,
\vs Luk 19:4 и, забежав вперед, взлез на смоковницу, чтобы увидеть Его, потому что Ему надлежало проходить мимо нее.
\vs Luk 19:5 Иисус, когда пришел на это место, взглянув, увидел его и сказал ему: Закхей! сойди скорее, ибо сегодня надобно Мне быть у тебя в доме.
\vs Luk 19:6 И он поспешно сошел и принял Его с радостью.
\vs Luk 19:7 И все, видя то, начали роптать, и говорили, что Он зашел к грешному человеку;
\vs Luk 19:8 Закхей же, став, сказал Господу: Господи! половину имения моего я отдам нищим, и, если кого чем обидел, воздам вчетверо.
\vs Luk 19:9 Иисус сказал ему: ныне пришло спасение дому сему, потому что и он сын Авраама,
\vs Luk 19:10 ибо Сын Человеческий пришел взыскать и спасти погибшее.
\rsbpar\vs Luk 19:11 Когда же они слушали это, присовокупил притчу: ибо Он был близ Иерусалима, и они думали, что скоро должно открыться Царствие Божие.
\vs Luk 19:12 Итак сказал: некоторый человек высокого рода отправлялся в дальнюю страну, чтобы получить себе царство и возвратиться;
\vs Luk 19:13 призвав же десять рабов своих, дал им десять мин\fns{Фунтов серебра.} и сказал им: употребляйте их в оборот, пока я возвращусь.
\vs Luk 19:14 Но граждане ненавидели его и отправили вслед за ним посольство, сказав: не хотим, чтобы он царствовал над нами.
\vs Luk 19:15 И когда возвратился, получив царство, велел призвать к себе рабов тех, которым дал серебро, чтобы узнать, кто что приобрел.
\vs Luk 19:16 Пришел первый и сказал: господин! мина твоя принесла десять мин.
\vs Luk 19:17 И сказал ему: хорошо, добрый раб! за то, что ты в малом был верен, возьми в управление десять городов.
\vs Luk 19:18 Пришел второй и сказал: господин! мина твоя принесла пять мин.
\vs Luk 19:19 Сказал и этому: и ты будь над пятью городами.
\vs Luk 19:20 Пришел третий и сказал: господин! вот твоя мина, которую я хранил, завернув в платок,
\vs Luk 19:21 ибо я боялся тебя, потому что ты человек жестокий: берешь, чего не клал, и жнешь, чего не сеял.
\vs Luk 19:22 \bibemph{Господин} сказал ему: твоими устами буду судить тебя, лукавый раб! ты знал, что я человек жестокий, беру, чего не клал, и жну, чего не сеял;
\vs Luk 19:23 для чего же ты не отдал серебра моего в оборот, чтобы я, придя, получил его с прибылью?
\vs Luk 19:24 И сказал предстоящим: возьмите у него мину и дайте имеющему десять мин.
\vs Luk 19:25 И сказали ему: господин! у него есть десять мин.
\vs Luk 19:26 Сказываю вам, что всякому имеющему дано будет, а у неимеющего отнимется и то, что имеет;
\vs Luk 19:27 врагов же моих тех, которые не хотели, чтобы я царствовал над ними, приведите сюда и избейте предо мною.
\vs Luk 19:28 Сказав это, Он пошел далее, восходя в Иерусалим.
\rsbpar\vs Luk 19:29 И когда приблизился к Виффагии и Вифании, к горе, называемой Елеонскою, послал двух учеников Своих,
\vs Luk 19:30 сказав: пойдите в противолежащее селение; войдя в него, найдете молодого осла привязанного, на которого никто из людей никогда не садился; отвязав его, приведите;
\vs Luk 19:31 и если кто спросит вас: зачем отвязываете? скажите ему так: он надобен Господу.
\vs Luk 19:32 Посланные пошли и нашли, как Он сказал им.
\vs Luk 19:33 Когда же они отвязывали молодого осла, хозяева его сказали им: зачем отвязываете осленка?
\vs Luk 19:34 Они отвечали: он надобен Господу.
\vs Luk 19:35 И привели его к Иисусу, и, накинув одежды свои на осленка, посадили на него Иисуса.
\vs Luk 19:36 И, когда Он ехал, постилали одежды свои по дороге.
\vs Luk 19:37 А когда Он приблизился к спуску с горы Елеонской, все множество учеников начало в радости велегласно славить Бога за все чудеса, какие видели они,
\vs Luk 19:38 говоря: благословен Царь, грядущий во имя Господне! мир на небесах и слава в вышних!
\vs Luk 19:39 И некоторые фарисеи из среды народа сказали Ему: Учитель! запрети ученикам Твоим.
\vs Luk 19:40 Но Он сказал им в ответ: сказываю вам, что если они умолкнут, то камни возопиют.
\vs Luk 19:41 И когда приблизился к городу, то, смотря на него, заплакал о нем
\vs Luk 19:42 и сказал: о, если бы и ты хотя в сей твой день узнал, что служит к миру твоему! Но это сокрыто ныне от глаз твоих,
\vs Luk 19:43 ибо придут на тебя дни, когда враги твои обложат тебя окопами и окружат тебя, и стеснят тебя отовсюду,
\vs Luk 19:44 и разорят тебя, и побьют детей твоих в тебе, и не оставят в тебе камня на камне за т\acc{о}, что ты не узнал времени посещения твоего.
\vs Luk 19:45 И, войдя в храм, начал выгонять продающих в нем и покупающих,
\vs Luk 19:46 говоря им: написано: дом Мой есть дом молитвы, а вы сделали его вертепом разбойников.
\vs Luk 19:47 И учил каждый день в храме. Первосвященники же и книжники и старейшины народа искали погубить Его,
\vs Luk 19:48 и не находили, что бы сделать с Ним; потому что весь народ неотступно слушал Его.
\vs Luk 20:1 В один из тех дней, когда Он учил народ в храме и благовествовал, приступили первосвященники и книжники со старейшинами,
\vs Luk 20:2 и сказали Ему: скажи нам, какою властью Ты это делаешь, или кто дал Тебе власть сию?
\vs Luk 20:3 Он сказал им в ответ: спрошу и Я вас об одном, и скажите Мне:
\vs Luk 20:4 крещение Иоанново с небес было, или от человеков?
\vs Luk 20:5 Они же, рассуждая между собою, говорили: если скажем: с небес, то скажет: почему же вы не поверили ему?
\vs Luk 20:6 а если скажем: от человеков, то весь народ побьет нас камнями, ибо он уверен, что Иоанн есть пророк.
\vs Luk 20:7 И отвечали: не знаем откуда.
\vs Luk 20:8 Иисус сказал им: и Я не скажу вам, какою властью это делаю.
\rsbpar\vs Luk 20:9 И начал Он говорить к народу притчу сию: один человек насадил виноградник и отдал его виноградарям, и отлучился на долгое время;
\vs Luk 20:10 и в свое время послал к виноградарям раба, чтобы они дали ему плодов из виноградника; но виноградари, прибив его, отослали ни с чем.
\vs Luk 20:11 Еще послал другого раба; но они и этого, прибив и обругав, отослали ни с чем.
\vs Luk 20:12 И еще послал третьего; но они и того, изранив, выгнали.
\vs Luk 20:13 Тогда сказал господин виноградника: что мне делать? Пошлю сына моего возлюбленного; может быть, увидев его, постыдятся.
\vs Luk 20:14 Но виноградари, увидев его, рассуждали между собою, говоря: это наследник; пойдем, убьем его, и наследство его будет наше.
\vs Luk 20:15 И, выведя его вон из виноградника, убили. Что же сделает с ними господин виноградника?
\vs Luk 20:16 Придет и погубит виноградарей тех, и отдаст виноградник другим. Слышавшие же это сказали: да не будет!
\vs Luk 20:17 Но Он, взглянув на них, сказал: что значит сие написанное: камень, который отвергли строители, тот самый сделался главою угла?
\vs Luk 20:18 Всякий, кто упадет на тот камень, разобьется, а на кого он упадет, того раздавит.
\vs Luk 20:19 И искали в это время первосвященники и книжники, чтобы наложить на Него руки, но побоялись народа, ибо поняли, что о них сказал Он эту притчу.
\vs Luk 20:20 И, наблюдая за Ним, подослали лукавых людей, которые, притворившись благочестивыми, уловили бы Его в каком-либо слове, чтобы предать Его начальству и власти правителя.
\vs Luk 20:21 И они спросили Его: Учитель! мы знаем, что Ты правдиво говоришь и учишь и не смотришь на лице, но истинно пути Божию учишь;
\vs Luk 20:22 позволительно ли нам давать подать кесарю, или нет?
\vs Luk 20:23 Он же, уразумев лукавство их, сказал им: что вы Меня искушаете?
\vs Luk 20:24 Покажите Мне динарий: чье на нем изображение и надпись? Они отвечали: кесаревы.
\vs Luk 20:25 Он сказал им: итак, отдавайте кесарево кесарю, а Божие Богу.
\vs Luk 20:26 И не могли уловить Его в слове перед народом, и, удивившись ответу Его, замолчали.
\rsbpar\vs Luk 20:27 Тогда пришли некоторые из саддукеев, отвергающих воскресение, и спросили Его:
\vs Luk 20:28 Учитель! Моисей написал нам, что если у кого умрет брат, имевший жену, и умрет бездетным, то брат его должен взять его жену и восставить семя брату своему.
\vs Luk 20:29 Было семь братьев, первый, взяв жену, умер бездетным;
\vs Luk 20:30 взял ту жену второй, и тот умер бездетным;
\vs Luk 20:31 взял ее третий; также и все семеро, и умерли, не оставив детей;
\vs Luk 20:32 после всех умерла и жена;
\vs Luk 20:33 итак, в воскресение которого из них будет она женою, ибо семеро имели ее женою?
\vs Luk 20:34 Иисус сказал им в ответ: чада века сего женятся и выходят замуж;
\vs Luk 20:35 а сподобившиеся достигнуть того века и воскресения из мертвых ни женятся, ни замуж не выходят,
\vs Luk 20:36 и умереть уже не могут, ибо они равны Ангелам и суть сыны Божии, будучи сынами воскресения.
\vs Luk 20:37 А что мертвые воскреснут, и Моисей показал при купине, когда назвал Господа Богом Авраама и Богом Исаака и Богом Иакова.
\vs Luk 20:38 Бог же не есть \bibemph{Бог} мертвых, но живых, ибо у Него все живы.
\vs Luk 20:39 На это некоторые из книжников сказали: Учитель! Ты хорошо сказал.
\vs Luk 20:40 И уже не смели спрашивать Его ни о чем. Он же сказал им:
\vs Luk 20:41 к\acc{а}к говорят, что Христос есть Сын Давидов,
\vs Luk 20:42 а сам Давид говорит в книге псалмов: сказал Господь Господу моему: седи одесную Меня,
\vs Luk 20:43 доколе положу врагов Твоих в подножие ног Твоих?
\vs Luk 20:44 Итак, Давид Господом называет Его; как же Он Сын ему?
\vs Luk 20:45 И когда слушал весь народ, Он сказал ученикам Своим:
\vs Luk 20:46 остерегайтесь книжников, которые любят ходить в длинных одеждах и любят приветствия в народных собраниях, председания в синагогах и предвозлежания на пиршествах,
\vs Luk 20:47 которые поедают д\acc{о}мы вдов и лицемерно долго молятся; они примут тем большее осуждение.
\vs Luk 21:1 Взглянув же, Он увидел богатых, клавших дары свои в сокровищницу;
\vs Luk 21:2 увидел также и бедную вдову, положившую туда две лепты,
\vs Luk 21:3 и сказал: истинно говорю вам, что эта бедная вдова больше всех положила;
\vs Luk 21:4 ибо все те от избытка своего положили в дар Богу, а она от скудости своей положила все пропитание свое, какое имела.
\rsbpar\vs Luk 21:5 И когда некоторые говорили о храме, что он украшен дорогими камнями и вкладами, Он сказал:
\vs Luk 21:6 придут дни, в которые из того, что вы здесь видите, не останется камня на камне; все будет разрушено.
\vs Luk 21:7 И спросили Его: Учитель! когда же это будет? и какой признак, когда это должно произойти?
\vs Luk 21:8 Он сказал: берегитесь, чтобы вас не ввели в заблуждение, ибо многие придут под именем Моим, говоря, что это Я; и это время близко: не ходите вслед их.
\vs Luk 21:9 Когда же услышите о войнах и смятениях, не ужасайтесь, ибо этому надлежит быть прежде; но не тотчас конец.
\vs Luk 21:10 Тогда сказал им: восстанет народ на народ, и царство на царство;
\vs Luk 21:11 будут большие землетрясения по местам, и глады, и моры, и ужасные явления, и великие знамения с неба.
\vs Luk 21:12 Прежде же всего того возложат на вас руки и будут гнать \bibemph{вас}, предавая в синагоги и в темницы, и поведут пред царей и правителей за имя Мое;
\vs Luk 21:13 будет же это вам для свидетельства.
\vs Luk 21:14 Итак положите себе на сердце не обдумывать заранее, что отвечать,
\vs Luk 21:15 ибо Я дам вам уста и премудрость, которой не возмогут противоречить ни противостоять все, противящиеся вам.
\vs Luk 21:16 Преданы также будете и родителями, и братьями, и родственниками, и друзьями, и некоторых из вас умертвят;
\vs Luk 21:17 и будете ненавидимы всеми за имя Мое,
\vs Luk 21:18 но и волос с головы вашей не пропадет,~---
\vs Luk 21:19 терпением вашим спасайте души ваши.
\vs Luk 21:20 Когда же увидите Иерусалим, окруженный войсками, тогда знайте, что приблизилось запустение его:
\vs Luk 21:21 тогда находящиеся в Иудее да бегут в горы; и кто в городе, выходи из него; и кто в окрестностях, не входи в него,
\vs Luk 21:22 потому что это дни отмщения, да исполнится все написанное.
\vs Luk 21:23 Горе же беременным и питающим сосцами в те дни; ибо великое будет бедствие на земле и гнев на народ сей:
\vs Luk 21:24 и падут от острия меча, и отведутся в плен во все народы; и Иерусалим будет попираем язычниками, доколе не окончатся времена язычников.
\vs Luk 21:25 И будут знамения в солнце и луне и звездах, а на земле уныние народов и недоумение; и море восшумит и возмутится;
\vs Luk 21:26 люди будут издыхать от страха и ожидания \bibemph{бедствий}, грядущих на вселенную, ибо силы небесные поколеблются,
\vs Luk 21:27 и тогда увидят Сына Человеческого, грядущего на облаке с силою и славою великою.
\vs Luk 21:28 Когда же начнет это сбываться, тогда восклонитесь и поднимите головы ваши, потому что приближается избавление ваше.
\vs Luk 21:29 И сказал им притчу: посмотрите на смоковницу и на все деревья:
\vs Luk 21:30 когда они уже распускаются, то, видя это, знаете сами, что уже близко лето.
\vs Luk 21:31 Так, и когда вы увидите то сбывающимся, знайте, что близко Царствие Божие.
\vs Luk 21:32 Истинно говорю вам: не прейдет род сей, как все это будет;
\vs Luk 21:33 небо и земля прейдут, но слова Мои не прейдут.
\vs Luk 21:34 Смотрите же за собою, чтобы сердца ваши не отягчались объядением и пьянством и заботами житейскими, и чтобы день тот не постиг вас внезапно,
\vs Luk 21:35 ибо он, как сеть, найдет на всех живущих по всему лицу земному;
\vs Luk 21:36 итак бодрствуйте на всякое время и молитесь, да сподобитесь избежать всех сих будущих \bibemph{бедствий} и предстать пред Сына Человеческого.
\rsbpar\vs Luk 21:37 Днем Он учил в храме, а ночи, выходя, проводил на горе, называемой Елеонскою.
\vs Luk 21:38 И весь народ с утра приходил к Нему в храм слушать Его.
\vs Luk 22:1 Приближался праздник опресноков, называемый Пасхою,
\vs Luk 22:2 и искали первосвященники и книжники, как бы погубить Его, потому что боялись народа.
\vs Luk 22:3 Вошел же сатана в Иуду, прозванного Искариотом, одного из числа двенадцати,
\vs Luk 22:4 и он пошел, и говорил с первосвященниками и начальниками, как Его предать им.
\vs Luk 22:5 Они обрадовались и согласились дать ему денег;
\vs Luk 22:6 и он обещал, и искал удобного времени, чтобы предать Его им не при народе.
\rsbpar\vs Luk 22:7 Настал же день опресноков, в который надлежало заколать пасхального \bibemph{агнца},
\vs Luk 22:8 и послал \bibemph{Иисус} Петра и Иоанна, сказав: пойдите, приготовьте нам есть пасху.
\vs Luk 22:9 Они же сказали Ему: где велишь нам приготовить?
\vs Luk 22:10 Он сказал им: вот, при входе вашем в город, встретится с вами человек, несущий кувшин воды; последуйте за ним в дом, в который войдет он,
\vs Luk 22:11 и скажите хозяину дома: Учитель говорит тебе: где комната, в которой бы Мне есть пасху с учениками Моими?
\vs Luk 22:12 И он покажет вам горницу большую устланную; там приготовьте.
\vs Luk 22:13 Они пошли, и нашли, как сказал им, и приготовили пасху.
\rsbpar\vs Luk 22:14 И когда настал час, Он возлег, и двенадцать Апостолов с Ним,
\vs Luk 22:15 и сказал им: очень желал Я есть с вами сию пасху прежде Моего страдания,
\vs Luk 22:16 ибо сказываю вам, что уже не буду есть ее, пока она не совершится в Царствии Божием.
\vs Luk 22:17 И, взяв чашу и благодарив, сказал: приимите ее и разделите между собою,
\vs Luk 22:18 ибо сказываю вам, что не буду пить от плода виноградного, доколе не придет Царствие Божие.
\vs Luk 22:19 И, взяв хлеб и благодарив, преломил и подал им, говоря: сие есть тело Мое, которое за вас предается; сие творите в Мое воспоминание.
\vs Luk 22:20 Также и чашу после вечери, говоря: сия чаша \bibemph{есть} Новый Завет в Моей крови, которая за вас проливается.
\vs Luk 22:21 И вот, рука предающего Меня со Мною за столом;
\vs Luk 22:22 впрочем, Сын Человеческий идет по предназначению, но горе тому человеку, которым Он предается.
\vs Luk 22:23 И они начали спрашивать друг друга, кто бы из них был, который это сделает.
\vs Luk 22:24 Был же и спор между ними, кто из них должен почитаться б\acc{о}льшим.
\vs Luk 22:25 Он же сказал им: цари господствуют над народами, и владеющие ими благодетелями называются,
\vs Luk 22:26 а вы не так: но кто из вас больше, будь как меньший, и начальствующий~--- как служащий.
\vs Luk 22:27 Ибо кто больше: возлежащий, или служащий? не возлежащий ли? А Я посреди вас, как служащий.
\vs Luk 22:28 Но вы пребыли со Мною в напастях Моих,
\vs Luk 22:29 и Я завещаваю вам, как завещал Мне Отец Мой, Царство,
\vs Luk 22:30 да ядите и пиете за трапезою Моею в Царстве Моем, и сядете на престолах судить двенадцать колен Израилевых.
\vs Luk 22:31 И сказал Господь: Симон! Симон! се, сатана просил, чтобы сеять вас как пшеницу,
\vs Luk 22:32 но Я молился о тебе, чтобы не оскудела вера твоя; и ты некогда, обратившись, утверди братьев твоих.
\vs Luk 22:33 Он отвечал Ему: Господи! с Тобою я готов и в темницу и на смерть идти.
\vs Luk 22:34 Но Он сказал: говорю тебе, Петр, не пропоет петух сегодня, как ты трижды отречешься, что не знаешь Меня.
\vs Luk 22:35 И сказал им: когда Я посылал вас без мешка и без сум\acc{ы} и без обуви, имели ли вы в чем недостаток? Они отвечали: ни в чем.
\vs Luk 22:36 Тогда Он сказал им: но теперь, кто имеет мешок, тот возьми его, также и сум\acc{у}; а у кого нет, продай одежду свою и купи меч;
\vs Luk 22:37 ибо сказываю вам, что должно исполниться на Мне и сему написанному: и к злодеям причтен. Ибо то, что о Мне, приходит к концу.
\vs Luk 22:38 Они сказали: Господи! вот, здесь два меча. Он сказал им: довольно.
\rsbpar\vs Luk 22:39 И, выйдя, пошел по обыкновению на гору Елеонскую, за Ним последовали и ученики Его.
\vs Luk 22:40 Придя же на место, сказал им: молитесь, чтобы не впасть в искушение.
\vs Luk 22:41 И Сам отошел от них на вержение камня, и, преклонив колени, молился,
\vs Luk 22:42 говоря: Отче! о, если бы Ты благоволил пронести чашу сию мимо Меня! впрочем не Моя воля, но Твоя да будет.
\vs Luk 22:43 Явился же Ему Ангел с небес и укреплял Его.
\vs Luk 22:44 И, находясь в борении, прилежнее молился, и был пот Его, как капли крови, падающие на землю.
\vs Luk 22:45 Встав от молитвы, Он пришел к ученикам, и нашел их спящими от печали
\vs Luk 22:46 и сказал им: что вы спите? встаньте и молитесь, чтобы не впасть в искушение.
\rsbpar\vs Luk 22:47 Когда Он еще говорил это, появился народ, а впереди его шел один из двенадцати, называемый Иуда, и он подошел к Иисусу, чтобы поцеловать Его. Ибо он такой им дал знак: Кого я поцелую, Тот и есть.
\vs Luk 22:48 Иисус же сказал ему: Иуда! целованием ли предаешь Сына Человеческого?
\vs Luk 22:49 Бывшие же с Ним, видя, к чему идет дело, сказали Ему: Господи! не ударить ли нам мечом?
\vs Luk 22:50 И один из них ударил раба первосвященникова, и отсек ему правое ухо.
\vs Luk 22:51 Тогда Иисус сказал: оставьте, довольно. И, коснувшись уха его, исцелил его.
\vs Luk 22:52 Первосвященникам же и начальникам храма и старейшинам, собравшимся против Него, сказал Иисус: как будто на разбойника вышли вы с мечами и кольями, чтобы взять Меня?
\vs Luk 22:53 Каждый день бывал Я с вами в храме, и вы не поднимали на Меня рук, но теперь ваше время и власть тьмы.
\rsbpar\vs Luk 22:54 Взяв Его, повели и привели в дом первосвященника. Петр же следовал издали.
\vs Luk 22:55 Когда они развели огонь среди двора и сели вместе, сел и Петр между ними.
\vs Luk 22:56 Одна служанка, увидев его сидящего у огня и всмотревшись в него, сказала: и этот был с Ним.
\vs Luk 22:57 Но он отрекся от Него, сказав женщине: я не знаю Его.
\vs Luk 22:58 Вскоре потом другой, увидев его, сказал: и ты из них. Но Петр сказал этому человеку: нет!
\vs Luk 22:59 Прошло с час времени, еще некто настоятельно говорил: точно и этот был с Ним, ибо он Галилеянин.
\vs Luk 22:60 Но Петр сказал тому человеку: не знаю, что ты говоришь. И тотчас, когда еще говорил он, запел петух.
\vs Luk 22:61 Тогда Господь, обратившись, взглянул на Петра, и Петр вспомнил слово Господа, как Он сказал ему: прежде нежели пропоет петух, отречешься от Меня трижды.
\vs Luk 22:62 И, выйдя вон, горько заплакал.
\rsbpar\vs Luk 22:63 Люди, державшие Иисуса, ругались над Ним и били Его;
\vs Luk 22:64 и, закрыв Его, ударяли Его по лицу и спрашивали Его: прореки, кто ударил Тебя?
\vs Luk 22:65 И много иных хулений произносили против Него.
\rsbpar\vs Luk 22:66 И как настал день, собрались старейшины народа, первосвященники и книжники, и ввели Его в свой синедрион
\vs Luk 22:67 и сказали: Ты ли Христос? скажи нам. Он сказал им: если скажу вам, вы не поверите;
\vs Luk 22:68 если же и спрошу вас, не будете отвечать Мне и не отпустите \bibemph{Меня};
\vs Luk 22:69 отныне Сын Человеческий воссядет одесную силы Божией.
\vs Luk 22:70 И сказали все: итак, Ты Сын Божий? Он отвечал им: вы говорите, что Я.
\vs Luk 22:71 Они же сказали: какое еще нужно нам свидетельство? ибо мы сами слышали из уст Его.
\vs Luk 23:1 И поднялось все множество их, и повели Его к Пилату,
\vs Luk 23:2 и начали обвинять Его, говоря: мы нашли, что Он развращает народ наш и запрещает давать подать кесарю, называя Себя Христом Царем.
\vs Luk 23:3 Пилат спросил Его: Ты Царь Иудейский? Он сказал ему в ответ: ты говоришь.
\vs Luk 23:4 Пилат сказал первосвященникам и народу: я не нахожу никакой вины в этом человеке.
\vs Luk 23:5 Но они настаивали, говоря, что Он возмущает народ, уча по всей Иудее, начиная от Галилеи до сего места.
\vs Luk 23:6 Пилат, услышав о Галилее, спросил: разве Он Галилеянин?
\vs Luk 23:7 И, узнав, что Он из области Иродовой, послал Его к Ироду, который в эти дни был также в Иерусалиме.
\vs Luk 23:8 Ирод, увидев Иисуса, очень обрадовался, ибо давно желал видеть Его, потому что много слышал о Нем, и надеялся увидеть от Него какое-нибудь чудо,
\vs Luk 23:9 и предлагал Ему многие вопросы, но Он ничего не отвечал ему.
\vs Luk 23:10 Первосвященники же и книжники стояли и усильно обвиняли Его.
\vs Luk 23:11 Но Ирод со своими воинами, уничижив Его и насмеявшись над Ним, одел Его в светлую одежду и отослал обратно к Пилату.
\vs Luk 23:12 И сделались в тот день Пилат и Ирод друзьями между собою, ибо прежде были во вражде друг с другом.
\vs Luk 23:13 Пилат же, созвав первосвященников и начальников и народ,
\vs Luk 23:14 сказал им: вы привели ко мне человека сего, как развращающего народ; и вот, я при вас исследовал и не нашел человека сего виновным ни в чем том, в чем вы обвиняете Его;
\vs Luk 23:15 и Ирод также, ибо я посылал Его к нему; и ничего не найдено в Нем достойного смерти;
\vs Luk 23:16 итак, наказав Его, отпущу.
\vs Luk 23:17 А ему и нужно было для праздника отпустить им одного \bibemph{узника}.
\vs Luk 23:18 Но весь народ стал кричать: смерть Ему! а отпусти нам Варавву.
\vs Luk 23:19 Варавва был посажен в темницу за произведенное в городе возмущение и убийство.
\vs Luk 23:20 Пилат снова возвысил голос, желая отпустить Иисуса.
\vs Luk 23:21 Но они кричали: распни, распни Его!
\vs Luk 23:22 Он в третий раз сказал им: какое же зло сделал Он? я ничего достойного смерти не нашел в Нем; итак, наказав Его, отпущу.
\vs Luk 23:23 Но они продолжали с великим криком требовать, чтобы Он был распят; и превозмог крик их и первосвященников.
\vs Luk 23:24 И Пилат решил быть по прошению их,
\vs Luk 23:25 и отпустил им посаженного за возмущение и убийство в темницу, которого они просили; а Иисуса предал в их волю.
\rsbpar\vs Luk 23:26 И когда повели Его, то, захватив некоего Симона Киринеянина, шедшего с поля, возложили на него крест, чтобы нес за Иисусом.
\vs Luk 23:27 И шло за Ним великое множество народа и женщин, которые плакали и рыдали о Нем.
\vs Luk 23:28 Иисус же, обратившись к ним, сказал: дщери Иерусалимские! не плачьте обо Мне, но плачьте о себе и о детях ваших,
\vs Luk 23:29 ибо приходят дни, в которые скажут: блаженны неплодные, и утробы неродившие, и сосцы непитавшие!
\vs Luk 23:30 тогда начнут говорить горам: падите на нас! и холмам: покройте нас!
\vs Luk 23:31 Ибо если с зеленеющим деревом это делают, то с сухим что будет?
\rsbpar\vs Luk 23:32 Вели с Ним на смерть и двух злодеев.
\vs Luk 23:33 И когда пришли на место, называемое Лобное, там распяли Его и злодеев, одного по правую, а другого по левую сторону.
\vs Luk 23:34 Иисус же говорил: Отче! прости им, ибо не знают, что делают. И делили одежды Его, бросая жребий.
\vs Luk 23:35 И стоял народ и смотрел. Насмехались же вместе с ними и начальники, говоря: других спасал; пусть спасет Себя Самого, если Он Христос, избранный Божий.
\vs Luk 23:36 Также и воины ругались над Ним, подходя и поднося Ему уксус
\vs Luk 23:37 и говоря: если Ты Царь Иудейский, спаси Себя Самого.
\vs Luk 23:38 И была над Ним надпись, написанная словами греческими, римскими и еврейскими: Сей есть Царь Иудейский.
\vs Luk 23:39 Один из повешенных злодеев злословил Его и говорил: если Ты Христос, спаси Себя и нас.
\vs Luk 23:40 Другой же, напротив, унимал его и говорил: или ты не боишься Бога, когда и сам осужден на то же?
\vs Luk 23:41 и мы \bibemph{осуждены} справедливо, потому что достойное по делам нашим приняли, а Он ничего худого не сделал.
\vs Luk 23:42 И сказал Иисусу: помяни меня, Господи, когда приидешь в Царствие Твое!
\vs Luk 23:43 И сказал ему Иисус: истинно говорю тебе, ныне же будешь со Мною в раю.
\rsbpar\vs Luk 23:44 Было же около шестого часа дня, и сделалась тьма по всей земле до часа девятого:
\vs Luk 23:45 и померкло солнце, и завеса в храме раздралась по средине.
\vs Luk 23:46 Иисус, возгласив громким голосом, сказал: Отче! в руки Твои предаю дух Мой. И, сие сказав, испустил дух.
\vs Luk 23:47 Сотник же, видев происходившее, прославил Бога и сказал: истинно человек этот был праведник.
\vs Luk 23:48 И весь народ, сшедшийся на сие зрелище, видя происходившее, возвращался, бия себя в грудь.
\vs Luk 23:49 Все же, знавшие Его, и женщины, следовавшие за Ним из Галилеи, стояли вдали и смотрели на это.
\rsbpar\vs Luk 23:50 Тогда некто, именем Иосиф, член совета, человек добрый и правдивый,
\vs Luk 23:51 не участвовавший в совете и в деле их; из Аримафеи, города Иудейского, ожидавший также Царствия Божия,
\vs Luk 23:52 пришел к Пилату и просил тела Иисусова;
\vs Luk 23:53 и, сняв его, обвил плащаницею и положил его в гробе, высеченном \bibemph{в скале}, где еще никто не был положен.
\vs Luk 23:54 День тот был пятница, и наступала суббота.
\vs Luk 23:55 Последовали также и женщины, пришедшие с Иисусом из Галилеи, и смотрели гроб, и как полагалось тело Его;
\vs Luk 23:56 возвратившись же, приготовили благовония и масти; и в субботу остались в покое по заповеди.
\vs Luk 24:1 В первый же день недели, очень рано, неся приготовленные ароматы, пришли они ко гробу, и вместе с ними некоторые другие;
\vs Luk 24:2 но нашли камень отваленным от гроба.
\vs Luk 24:3 И, войдя, не нашли тела Господа Иисуса.
\vs Luk 24:4 Когда же недоумевали они о сем, вдруг предстали перед ними два мужа в одеждах блистающих.
\vs Luk 24:5 И когда они были в страхе и наклонили лица \bibemph{свои} к земле, сказали им: что вы ищете живого между мертвыми?
\vs Luk 24:6 Его нет здесь: Он воскрес; вспомните, как Он говорил вам, когда был еще в Галилее,
\vs Luk 24:7 сказывая, что Сыну Человеческому надлежит быть предану в руки человеков грешников, и быть распяту, и в третий день воскреснуть.
\vs Luk 24:8 И вспомнили они слова Его;
\vs Luk 24:9 и, возвратившись от гроба, возвестили всё это одиннадцати и всем прочим.
\vs Luk 24:10 То были Магдалина Мария, и Иоанна, и Мария, \bibemph{мать} Иакова, и другие с ними, которые сказали о сем Апостолам.
\vs Luk 24:11 И показались им слова их пустыми, и не поверили им.
\vs Luk 24:12 Но Петр, встав, побежал ко гробу и, наклонившись, увидел только пелены лежащие, и пошел назад, дивясь сам в себе происшедшему.
\rsbpar\vs Luk 24:13 В тот же день двое из них шли в селение, отстоящее стадий на шестьдесят от Иерусалима, называемое Эммаус;
\vs Luk 24:14 и разговаривали между собою о всех сих событиях.
\vs Luk 24:15 И когда они разговаривали и рассуждали между собою, и Сам Иисус, приблизившись, пошел с ними.
\vs Luk 24:16 Но глаза их были удержаны, так что они не узнали Его.
\vs Luk 24:17 Он же сказал им: о чем это вы, идя, рассуждаете между собою, и отчего вы печальны?
\vs Luk 24:18 Один из них, именем Клеопа, сказал Ему в ответ: неужели Ты один из пришедших в Иерусалим не знаешь о происшедшем в нем в эти дни?
\vs Luk 24:19 И сказал им: о чем? Они сказали Ему: что было с Иисусом Назарянином, Который был пророк, сильный в деле и слове пред Богом и всем народом;
\vs Luk 24:20 как предали Его первосвященники и начальники наши для осуждения на смерть и распяли Его.
\vs Luk 24:21 А мы надеялись было, что Он есть Тот, Который должен избавить Израиля; но со всем тем, уже третий день ныне, как это произошло.
\vs Luk 24:22 Но и некоторые женщины из наших изумили нас: они были рано у гроба
\vs Luk 24:23 и не нашли тела Его и, придя, сказывали, что они видели и явление Ангелов, которые говорят, что Он жив.
\vs Luk 24:24 И пошли некоторые из наших ко гробу и нашли так, как и женщины говорили, но Его не видели.
\vs Luk 24:25 Тогда Он сказал им: о, несмысленные и медлительные сердцем, чтобы веровать всему, что предсказывали пророки!
\vs Luk 24:26 Не так ли надлежало пострадать Христу и войти в славу Свою?
\vs Luk 24:27 И, начав от Моисея, из всех пророков изъяснял им сказанное о Нем во всем Писании.
\vs Luk 24:28 И приблизились они к тому селению, в которое шли; и Он показывал им вид, что хочет идти далее.
\vs Luk 24:29 Но они удерживали Его, говоря: останься с нами, потому что день уже склонился к вечеру. И Он вошел и остался с ними.
\vs Luk 24:30 И когда Он возлежал с ними, то, взяв хлеб, благословил, преломил и подал им.
\vs Luk 24:31 Тогда открылись у них глаза, и они узнали Его. Но Он стал невидим для них.
\vs Luk 24:32 И они сказали друг другу: не горело ли в нас сердце наше, когда Он говорил нам на дороге и когда изъяснял нам Писание?
\vs Luk 24:33 И, встав в тот же час, возвратились в Иерусалим и нашли вместе одиннадцать \bibemph{Апостолов} и бывших с ними,
\vs Luk 24:34 которые говорили, что Господь истинно воскрес и явился Симону.
\vs Luk 24:35 И они рассказывали о происшедшем на пути, и как Он был узнан ими в преломлении хлеба.
\rsbpar\vs Luk 24:36 Когда они говорили о сем, Сам Иисус стал посреди них и сказал им: мир вам.
\vs Luk 24:37 Они, смутившись и испугавшись, подумали, что видят духа.
\vs Luk 24:38 Но Он сказал им: что смущаетесь, и для чего такие мысли входят в сердца ваши?
\vs Luk 24:39 Посмотрите на руки Мои и на ноги Мои; это Я Сам; осяжите Меня и рассмотр\acc{и}те; ибо дух плоти и костей не имеет, как видите у Меня.
\vs Luk 24:40 И, сказав это, показал им руки и ноги.
\vs Luk 24:41 Когда же они от радости еще не верили и дивились, Он сказал им: есть ли у вас здесь какая пища?
\vs Luk 24:42 Они подали Ему часть печеной рыбы и сотового меда.
\vs Luk 24:43 И, взяв, ел пред ними.
\vs Luk 24:44 И сказал им: вот то, о чем Я вам говорил, еще быв с вами, что надлежит исполниться всему, написанному о Мне в законе Моисеевом и в пророках и псалмах.
\vs Luk 24:45 Тогда отверз им ум к уразумению Писаний.
\vs Luk 24:46 И сказал им: так написано, и так надлежало пострадать Христу, и воскреснуть из мертвых в третий день,
\vs Luk 24:47 и проповедану быть во имя Его покаянию и прощению грехов во всех народах, начиная с Иерусалима.
\vs Luk 24:48 Вы же свидетели сему.
\vs Luk 24:49 И Я пошлю обетование Отца Моего на вас; вы же оставайтесь в городе Иерусалиме, доколе не облечетесь силою свыше.
\rsbpar\vs Luk 24:50 И вывел их вон \bibemph{из города} до Вифании и, подняв руки Свои, благословил их.
\vs Luk 24:51 И, когда благословлял их, стал отдаляться от них и возноситься на небо.
\vs Luk 24:52 Они поклонились Ему и возвратились в Иерусалим с великою радостью.
\vs Luk 24:53 И пребывали всегда в храме, прославляя и благословляя Бога. Аминь.

\include{tex/Joh}
\include{tex/Act}\newbookpage
\bibbookdescr{Jam}{
  inline={Соборное Послание\\\LARGE Святого Апостола Иакова},
  toc={Иакова},
  bookmark={Иакова},
  header={Иакова},
  %headerleft={},
  %headerright={},
  abbr={Иак}
}
\vs Jam 1:1 Иаков, раб Бога и Господа Иисуса Христа, двенадцати коленам, находящимся в рассеянии,~--- радоваться.
\vs Jam 1:2 С великою радостью принимайте, братия мои, когда впадаете в различные искушения,
\vs Jam 1:3 зная, что испытание вашей веры производит терпение;
\vs Jam 1:4 терпение же должно иметь совершенное действие, чтобы вы были совершенны во всей полноте, без всякого недостатка.
\vs Jam 1:5 Если же у кого из вас недостает мудрости, да просит у Бога, дающего всем просто и без упреков,~--- и дастся ему.
\vs Jam 1:6 Но да просит с верою, нимало не сомневаясь, потому что сомневающийся подобен морской волне, ветром поднимаемой и развеваемой.
\vs Jam 1:7 Да не думает такой человек получить что-нибудь от Господа.
\vs Jam 1:8 Человек с двоящимися мыслями не тверд во всех путях своих.
\rsbpar\vs Jam 1:9 Да хвалится брат униженный высотою своею,
\vs Jam 1:10 а богатый~--- унижением своим, потому что он прейдет, как цвет на траве.
\vs Jam 1:11 Восходит солнце, \bibemph{настает} зной, и зноем иссушает траву, цвет ее опадает, исчезает красота вида ее; так увядает и богатый в путях своих.
\rsbpar\vs Jam 1:12 Блажен человек, который переносит искушение, потому что, быв испытан, он получит венец жизни, который обещал Господь любящим Его.
\vs Jam 1:13 В искушении никто не говори: Бог меня искушает; потому что Бог не искушается злом и Сам не искушает никого,
\vs Jam 1:14 но каждый искушается, увлекаясь и обольщаясь собственною похотью;
\vs Jam 1:15 похоть же, зачав, рождает грех, а сделанный грех рождает смерть.
\rsbpar\vs Jam 1:16 Не обманывайтесь, братия мои возлюбленные.
\vs Jam 1:17 Всякое даяние доброе и всякий дар совершенный нисходит свыше, от Отца светов, у Которого нет изменения и ни тени перемены.
\vs Jam 1:18 Восхотев, родил Он нас словом истины, чтобы нам быть некоторым начатком Его созданий.
\rsbpar\vs Jam 1:19 Итак, братия мои возлюбленные, всякий человек да будет скор на слышание, медлен на слова, медлен на гнев,
\vs Jam 1:20 ибо гнев человека не творит правды Божией.
\vs Jam 1:21 Посему, отложив всякую нечистоту и остаток злобы, в кротости примите насаждаемое слово, могущее спасти ваши души.
\vs Jam 1:22 Будьте же исполнители слова, а не слышатели только, обманывающие самих себя.
\vs Jam 1:23 Ибо, кто слушает слово и не исполняет, тот подобен человеку, рассматривающему природные черты лица своего в зеркале:
\vs Jam 1:24 он посмотрел на себя, отошел и тотчас забыл, каков он.
\vs Jam 1:25 Но кто вникнет в закон совершенный, \bibemph{закон} свободы, и пребудет в нем, тот, будучи не слушателем забывчивым, но исполнителем дела, блажен будет в своем действии.
\vs Jam 1:26 Если кто из вас думает, что он благочестив, и не обуздывает своего языка, но обольщает свое сердце, у того пустое благочестие.
\vs Jam 1:27 Чистое и непорочное благочестие пред Богом и Отцем есть то, чтобы призирать сирот и вдов в их скорбях и хранить себя неоскверненным от мира.
\vs Jam 2:1 Братия мои! имейте веру в Иисуса Христа нашего Господа славы, не взирая на лица.
\vs Jam 2:2 Ибо, если в собрание ваше войдет человек с золотым перстнем, в богатой одежде, войдет же и бедный в скудной одежде,
\vs Jam 2:3 и вы, смотря на одетого в богатую одежду, скажете ему: тебе хорошо сесть здесь, а бедному скажете: ты стань там, или садись здесь, у ног моих,~---
\vs Jam 2:4 то не пересуживаете ли вы в себе и не становитесь ли судьями с худыми мыслями?
\vs Jam 2:5 Послушайте, братия мои возлюбленные: не бедных ли мира избрал Бог быть богатыми верою и наследниками Царствия, которое Он обещал любящим Его?
\vs Jam 2:6 А вы презрели бедного. Не богатые ли притесняют вас, и не они ли влекут вас в суды?
\vs Jam 2:7 Не они ли бесславят доброе имя, которым вы называетесь?
\vs Jam 2:8 Если вы исполняете закон царский, по Писанию: возлюби ближнего твоего, как себя самого,~--- хорошо делаете.
\vs Jam 2:9 Но если поступаете с лицеприятием, то грех делаете, и перед законом оказываетесь преступниками.
\vs Jam 2:10 Кто соблюдает весь закон и согрешит в одном чем-нибудь, тот становится виновным во всем.
\vs Jam 2:11 Ибо Тот же, Кто сказал: не прелюбодействуй, сказал и: не убей; посему, если ты не прелюбодействуешь, но убьешь, то ты также преступник закона.
\vs Jam 2:12 Т\acc{а}к говорите и т\acc{а}к поступайте, как имеющие быть судимы по закону свободы.
\vs Jam 2:13 Ибо суд без милости не оказавшему милости; милость превозносится над судом.
\rsbpar\vs Jam 2:14 Чт\acc{о} пользы, братия мои, если кто говорит, что он имеет веру, а дел не имеет? может ли эта вера спасти его?
\vs Jam 2:15 Если брат или сестра наги и не имеют дневного пропитания,
\vs Jam 2:16 а кто-нибудь из вас скажет им: <<идите с миром, грейтесь и питайтесь>>, но не даст им потребного для тела: что пользы?
\vs Jam 2:17 Так и вера, если не имеет дел, мертва сама по себе.
\vs Jam 2:18 Но скажет кто-нибудь: <<ты имеешь веру, а я имею дела>>: покажи мне веру твою без дел твоих, а я покажу тебе веру мою из дел моих.
\vs Jam 2:19 Ты веруешь, что Бог един: хорошо делаешь; и бесы веруют, и трепещут.
\vs Jam 2:20 Но хочешь ли знать, неосновательный человек, что вера без дел мертва?
\vs Jam 2:21 Не делами ли оправдался Авраам, отец наш, возложив на жертвенник Исаака, сына своего?
\vs Jam 2:22 Видишь ли, что вера содействовала делам его, и делами вера достигла совершенства?
\vs Jam 2:23 И исполнилось слово Писания: <<веровал Авраам Богу, и это вменилось ему в праведность, и он наречен другом Божиим>>.
\vs Jam 2:24 Видите ли, что человек оправдывается делами, а не верою только?
\vs Jam 2:25 Подобно и Раав блудница не делами ли оправдалась, приняв соглядатаев и отпустив их другим путем?
\vs Jam 2:26 Ибо, как тело без духа мертво, так и вера без дел мертва.
\vs Jam 3:1 Братия мои! не многие делайтесь учителями, зная, что мы подвергнемся большему осуждению,
\vs Jam 3:2 ибо все мы много согрешаем. Кто не согрешает в слове, тот человек совершенный, могущий обуздать и все тело.
\vs Jam 3:3 Вот, мы влагаем удила в рот коням, чтобы они повиновались нам, и управляем всем телом их.
\vs Jam 3:4 Вот, и корабли, как ни велики они и как ни сильными ветрами носятся, небольшим рулем направляются, куда хочет кормчий;
\vs Jam 3:5 так и язык~--- небольшой член, но много делает. Посмотри, небольшой огонь как много вещества зажигает!
\vs Jam 3:6 И язык~--- огонь, прикраса неправды; язык в таком положении находится между членами нашими, что оскверняет все тело и воспаляет круг жизни, будучи сам воспаляем от геенны.
\vs Jam 3:7 Ибо всякое естество зверей и птиц, пресмыкающихся и морских животных укрощается и укрощено естеством человеческим,
\vs Jam 3:8 а язык укротить никто из людей не может: это~--- неудержимое зло; он исполнен смертоносного яда.
\vs Jam 3:9 Им благословляем Бога и Отца, и им проклинаем человеков, сотворенных по подобию Божию.
\vs Jam 3:10 Из тех же уст исходит благословение и проклятие: не должно, братия мои, сему так быть.
\vs Jam 3:11 Течет ли из одного отверстия источника сладкая и горькая \bibemph{вода}?
\vs Jam 3:12 Не может, братия мои, смоковница приносить маслины или виноградная лоза смоквы. Также и один источник не \bibemph{может} изливать соленую и сладкую воду.
\rsbpar\vs Jam 3:13 Мудр ли и разумен кто из вас, докажи это на самом деле добрым поведением с мудрою кротостью.
\vs Jam 3:14 Но если в вашем сердце вы имеете горькую зависть и сварливость, то не хвалитесь и не лгите на истину.
\vs Jam 3:15 Это не есть мудрость, нисходящая свыше, но земная, душевная, бесовская,
\vs Jam 3:16 ибо где зависть и сварливость, там неустройство и всё худое.
\vs Jam 3:17 Но мудрость, сходящая свыше, во-первых, чиста, потом мирна, скромна, послушлива, полна милосердия и добрых плодов, беспристрастна и нелицемерна.
\vs Jam 3:18 Плод же правды в мире сеется у тех, которые хранят мир.
\vs Jam 4:1 Откуда у вас вражды и распри? не отсюда ли, от вожделений ваших, воюющих в членах ваших?
\vs Jam 4:2 Желаете~--- и не имеете; убиваете и завидуете~--- и не можете достигнуть; препираетесь и враждуете~--- и не имеете, потому что не пр\acc{о}сите.
\vs Jam 4:3 Пр\acc{о}сите, и не получаете, потому что пр\acc{о}сите не на добро, а чтобы употребить для ваших вожделений.
\vs Jam 4:4 Прелюбодеи и прелюбодейцы! не знаете ли, что дружба с миром есть вражда против Бога? Итак, кто хочет быть другом миру, тот становится врагом Богу.
\vs Jam 4:5 Или вы думаете, что напрасно говорит Писание: <<до ревности любит дух, живущий в нас>>?
\vs Jam 4:6 Но тем б\acc{о}льшую дает благодать; посему и сказано: Бог гордым противится, а смиренным дает благодать.
\rsbpar\vs Jam 4:7 Итак покоритесь Богу; противостаньте диаволу, и убежит от вас.
\vs Jam 4:8 Приблизьтесь к Богу, и приблизится к вам; очистите руки, грешники, исправьте сердца, двоедушные.
\vs Jam 4:9 Сокрушайтесь, плачьте и рыдайте; смех ваш да обратится в плач, и радость~--- в печаль.
\vs Jam 4:10 Смиритесь пред Господом, и вознесет вас.
\rsbpar\vs Jam 4:11 Не злословьте друг друга, братия: кто злословит брата или судит брата своего, тот злословит закон и судит закон; а если ты судишь закон, то ты не исполнитель закона, но судья.
\vs Jam 4:12 Един Законодатель и Судия, могущий спасти и погубить; а ты кто, который судишь другого?
\rsbpar\vs Jam 4:13 Теперь послушайте вы, говорящие: <<сегодня или завтра отправимся в такой-то город, и проживем там один год, и будем торговать и получать прибыль>>;
\vs Jam 4:14 вы, которые не знаете, что случится завтра: ибо что такое жизнь ваша? пар, являющийся на малое время, а потом исчезающий.
\vs Jam 4:15 Вместо того, чтобы вам говорить: <<если угодно будет Господу и живы будем, то сделаем то или другое>>,~---
\vs Jam 4:16 вы, по своей надменности, тщеславитесь: всякое такое тщеславие есть зло.
\vs Jam 4:17 Итак, кто разумеет делать добро и не делает, тому грех.
\vs Jam 5:1 Послушайте вы, богатые: плачьте и рыдайте о бедствиях ваших, находящих на вас.
\vs Jam 5:2 Богатство ваше сгнило, и одежды ваши изъедены молью.
\vs Jam 5:3 Золото ваше и серебро изоржавело, и ржавчина их будет свидетельством против вас и съест плоть вашу, как огонь: вы собрали себе сокровище на последние дни.
\vs Jam 5:4 Вот, плата, удержанная вами у работников, пожавших поля ваши, вопиет, и вопли жнецов дошли до слуха Господа Саваофа.
\vs Jam 5:5 Вы роскошествовали на земле и наслаждались; напитали сердца ваши, как бы на день заклания.
\vs Jam 5:6 Вы осудили, убили Праведника; Он не противился вам.
\rsbpar\vs Jam 5:7 Итак, братия, будьте долготерпеливы до пришествия Господня. Вот, земледелец ждет драгоценного плода от земли и для него терпит долго, пока получит дождь ранний и поздний.
\vs Jam 5:8 Долготерп\acc{и}те и вы, укрепите сердца ваши, потому что пришествие Господне приближается.
\vs Jam 5:9 Не сетуйте, братия, друг на друга, чтобы не быть осужденными: вот, Судия стоит у дверей.
\vs Jam 5:10 В пример злострадания и долготерпения возьмите, братия мои, пророков, которые говорили именем Господним.
\vs Jam 5:11 Вот, мы ублажаем тех, которые терпели. Вы слышали о терпении Иова и видели конец \bibemph{оного} от Господа, ибо Господь весьма милосерд и сострадателен.
\rsbpar\vs Jam 5:12 Прежде же всего, братия мои, не клянитесь ни небом, ни землею, и никакою другою клятвою, но да будет у вас: <<да, да>> и <<нет, нет>>, дабы вам не подпасть осуждению.
\rsbpar\vs Jam 5:13 Злостраждет ли кто из вас, пусть молится. Весел ли кто, пусть поет псалмы.
\vs Jam 5:14 Болен ли кто из вас, пусть призовет пресвитеров Церкви, и пусть помолятся над ним, помазав его елеем во имя Господне.
\vs Jam 5:15 И молитва веры исцелит болящего, и восставит его Господь; и если он соделал грехи, простятся ему.
\rsbpar\vs Jam 5:16 Признавайтесь друг пред другом в проступках и молитесь друг за друга, чтобы исцелиться: много может усиленная молитва праведного.
\vs Jam 5:17 Илия был человек, подобный нам, и молитвою помолился, чтобы не было дождя: и не было дождя на землю три года и шесть месяцев.
\vs Jam 5:18 И опять помолился: и небо дало дождь, и земля произрастила плод свой.
\rsbpar\vs Jam 5:19 Братия! если кто из вас уклонится от истины, и обратит кто его,
\vs Jam 5:20 пусть тот знает, что обративший грешника от ложного пути его спасет душу от смерти и покроет множество грехов.

\bibbookdescr{1Pe}{
  inline={Первое Соборное Послание\\\LARGE Святого Апостола Петра},
  toc={1-е Петра},
  bookmark={1-е Петра},
  header={1-е Петра},
  %headerleft={},
  %headerright={},
  abbr={1~Пет}
}
\vs 1Pe 1:1 Петр, Апостол Иисуса Христа, пришельцам, рассеянным в Понте, Галатии, Каппадокии, Асии и Вифинии, избранным,
\vs 1Pe 1:2 по предведению Бога Отца, при освящении от Духа, к послушанию и окроплению Кровию Иисуса Христа: благодать вам и мир да умножится.
\rsbpar\vs 1Pe 1:3 Благословен Бог и Отец Господа нашего Иисуса Христа, по великой Своей милости возродивший нас воскресением Иисуса Христа из мертвых к упованию живому,
\vs 1Pe 1:4 к наследству нетленному, чистому, неувядаемому, хранящемуся на небесах для вас,
\vs 1Pe 1:5 силою Божиею через веру соблюдаемых ко спасению, готовому открыться в последнее время.
\vs 1Pe 1:6 О сем радуйтесь, поскорбев теперь немного, если нужно, от различных искушений,
\vs 1Pe 1:7 дабы испытанная вера ваша оказалась драгоценнее гибнущего, хотя и огнем испытываемого золота, к похвале и чести и славе в явление Иисуса Христа,
\vs 1Pe 1:8 Которого, не видев, любите, и Которого доселе не видя, но веруя в Него, радуетесь радостью неизреченною и преславною,
\vs 1Pe 1:9 достигая наконец верою вашею спасения душ.
\vs 1Pe 1:10 К сему-то спасению относились изыскания и исследования пророков, которые предсказывали о назначенной вам благодати,
\vs 1Pe 1:11 исследуя, на которое и на какое время указывал сущий в них Дух Христов, когда Он предвозвещал Христовы страдания и последующую за ними славу.
\vs 1Pe 1:12 Им открыто было, что не им самим, а нам служило то, что ныне проповедано вам благовествовавшими Духом Святым, посланным с небес, во что желают проникнуть Ангелы.
\rsbpar\vs 1Pe 1:13 Посему, (возлюбленные,) препоясав чресла ума вашего, бодрствуя, совершенно уповайте на подаваемую вам благодать в явлении Иисуса Христа.
\vs 1Pe 1:14 Как послушные дети, не сообразуйтесь с прежними похотями, бывшими в неведении вашем,
\vs 1Pe 1:15 но, по примеру призвавшего вас Святаго, и сами будьте святы во всех поступках.
\vs 1Pe 1:16 Ибо написано: будьте святы, потому что Я свят.
\vs 1Pe 1:17 И если вы называете Отцем Того, Который нелицеприятно судит каждого по делам, то со страхом провод\acc{и}те время странствования вашего,
\vs 1Pe 1:18 зная, что не тленным серебром или золотом искуплены вы от суетной жизни, преданной вам от отцов,
\vs 1Pe 1:19 но драгоценною Кровию Христа, как непорочного и чистого Агнца,
\vs 1Pe 1:20 предназначенного еще прежде создания мира, но явившегося в последние времена для вас,
\vs 1Pe 1:21 уверовавших чрез Него в Бога, Который воскресил Его из мертвых и дал Ему славу, чтобы вы имели веру и упование на Бога.
\rsbpar\vs 1Pe 1:22 Послушанием истине чрез Духа, очистив души ваши к нелицемерному братолюбию, постоянно люб\acc{и}те друг друга от чистого сердца,
\vs 1Pe 1:23 \bibemph{как} возрожденные не от тленного семени, но от нетленного, от слова Божия, живаго и пребывающего вовек.
\vs 1Pe 1:24 Ибо всякая плоть~--- как трава, и всякая слава человеческая~--- как цвет на траве: засохла трава, и цвет ее опал;
\vs 1Pe 1:25 но слово Господне пребывает вовек; а это есть то слово, которое вам проповедано.
\vs 1Pe 2:1 Итак, отложив всякую злобу и всякое коварство, и лицемерие, и зависть, и всякое злословие,
\vs 1Pe 2:2 как новорожденные младенцы, возлюб\acc{и}те чистое словесное молоко, дабы от него возрасти вам во спасение;
\vs 1Pe 2:3 ибо вы вкусили, что благ Господь.
\vs 1Pe 2:4 Приступая к Нему, камню живому, человеками отверженному, но Богом избранному, драгоценному,
\vs 1Pe 2:5 и сами, как живые камни, устрояйте из себя дом духовный, священство святое, чтобы приносить духовные жертвы, благоприятные Богу Иисусом Христом.
\vs 1Pe 2:6 Ибо сказано в Писании: вот, Я полагаю в Сионе камень краеугольный, избранный, драгоценный; и верующий в Него не постыдится.
\vs 1Pe 2:7 Итак Он для вас, верующих, драгоценность, а для неверующих камень, который отвергли строители, но который сделался главою угла, камень претыкания и камень соблазна,
\vs 1Pe 2:8 о который они претыкаются, не покоряясь слову, на что они и оставлены.
\vs 1Pe 2:9 Но вы~--- род избранный, царственное священство, народ святой, люди, взятые в удел, дабы возвещать совершенства Призвавшего вас из тьмы в чудный Свой свет;
\vs 1Pe 2:10 некогда не народ, а ныне народ Божий; \bibemph{некогда} непомилованные, а ныне помилованы.
\vs 1Pe 2:11 Возлюбленные! прошу вас, как пришельцев и странников, удаляться от плотских похотей, восстающих на душу,
\vs 1Pe 2:12 и провождать добродетельную жизнь между язычниками, дабы они за то, за что злословят вас, как злодеев, увидя добрые дела ваши, прославили Бога в день посещения.
\vs 1Pe 2:13 Итак будьте покорны всякому человеческому начальству, для Господа: царю ли, как верховной власти,
\vs 1Pe 2:14 правителям ли, как от него посылаемым для наказания преступников и для поощрения делающих добро,~---
\vs 1Pe 2:15 ибо такова есть воля Божия, чтобы мы, делая добро, заграждали уста невежеству безумных людей,~---
\vs 1Pe 2:16 как свободные, не как употребляющие свободу для прикрытия зла, но как рабы Божии.
\vs 1Pe 2:17 Всех почитайте, братство любите, Бога бойтесь, царя чтите.
\rsbpar\vs 1Pe 2:18 Слуги, со всяким страхом повинуйтесь господам, не только добрым и кротким, но и суровым.
\vs 1Pe 2:19 Ибо то угодно Богу, если кто, помышляя о Боге, переносит скорби, страдая несправедливо.
\vs 1Pe 2:20 Ибо что за похвала, если вы терпите, когда вас бьют за проступки? Но если, делая добро и страдая, терпите, это угодно Богу.
\vs 1Pe 2:21 Ибо вы к тому призваны, потому что и Христос пострадал за нас, оставив нам пример, дабы мы шли по следам Его.
\vs 1Pe 2:22 Он не сделал никакого греха, и не было лести в устах Его.
\vs 1Pe 2:23 Будучи злословим, Он не злословил взаимно; страдая, не угрожал, но предавал то Судии Праведному.
\vs 1Pe 2:24 Он грехи наши Сам вознес телом Своим на древо, дабы мы, избавившись от грехов, жили для правды: ранами Его вы исцелились.
\vs 1Pe 2:25 Ибо вы были, как овцы блуждающие (не имея пастыря), но возвратились ныне к Пастырю и Блюстителю душ ваших.
\vs 1Pe 3:1 Также и вы, жены, повинуйтесь своим мужьям, чтобы те из них, которые не покоряются слову, житием жен своих без слова приобретаемы были,
\vs 1Pe 3:2 когда увидят ваше чистое, богобоязненное житие.
\vs 1Pe 3:3 Да будет украшением вашим не внешнее плетение волос, не золотые уборы или нарядность в одежде,
\vs 1Pe 3:4 но сокровенный сердца человек в нетленной \bibemph{красоте} кроткого и молчаливого духа, что драгоценно пред Богом.
\vs 1Pe 3:5 Так некогда и святые жены, уповавшие на Бога, украшали себя, повинуясь своим мужьям.
\vs 1Pe 3:6 Так Сарра повиновалась Аврааму, называя его господином. Вы~--- дети ее, если делаете добро и не смущаетесь ни от какого страха.
\rsbpar\vs 1Pe 3:7 Также и вы, мужья, обращайтесь благоразумно с женами, как с немощнейшим сосудом, оказывая им честь, как сонаследницам благодатной жизни, дабы не было вам препятствия в молитвах.
\rsbpar\vs 1Pe 3:8 Наконец будьте все единомысленны, сострадательны, братолюбивы, милосерды, дружелюбны, смиренномудры;
\vs 1Pe 3:9 не воздавайте злом за зло или ругательством за ругательство; напротив, благословляйте, зная, что вы к тому призваны, чтобы наследовать благословение.
\vs 1Pe 3:10 Ибо, кто любит жизнь и хочет видеть добрые дни, тот удерживай язык свой от зла и уста свои от лукавых речей;
\vs 1Pe 3:11 уклоняйся от зла и делай добро; ищи мира и стремись к нему,
\vs 1Pe 3:12 потому что очи Господа \bibemph{обращены} к праведным и уши Его к молитве их, но лице Господне против делающих зло (чтобы истребить их с земли).
\vs 1Pe 3:13 И кто сделает вам зло, если вы будете ревнителями доброго?
\vs 1Pe 3:14 Но если и страдаете за правду, то вы блаженны; а страха их не бойтесь и не смущайтесь.
\rsbpar\vs 1Pe 3:15 Господа Бога святите в сердцах ваших; \bibemph{будьте} всегда готовы всякому, требующему у вас отчета в вашем уповании, дать ответ с кротостью и благоговением.
\vs 1Pe 3:16 Имейте добрую совесть, дабы тем, за что злословят вас, как злодеев, были постыжены порицающие ваше доброе житие во Христе.
\vs 1Pe 3:17 Ибо, если угодно воле Божией, лучше пострадать за добрые дела, нежели за злые;
\vs 1Pe 3:18 потому что и Христос, чтобы привести нас к Богу, однажды пострадал за грехи наши, праведник за неправедных, быв умерщвлен по плоти, но ожив духом,
\vs 1Pe 3:19 которым Он и находящимся в темнице духам, сойдя, проповедал,
\vs 1Pe 3:20 некогда непокорным ожидавшему их Божию долготерпению, во дни Ноя, во время строения ковчега, в котором немногие, то есть восемь душ, спаслись от воды.
\vs 1Pe 3:21 Так и нас ныне подобное сему образу крещение, не плотской нечистоты омытие, но обещание Богу доброй совести, спасает воскресением Иисуса Христа,
\vs 1Pe 3:22 Который, восшед на небо, пребывает одесную Бога и Которому покорились Ангелы и Власти и Силы.
\vs 1Pe 4:1 Итак, как Христос пострадал за нас плотию, то и вы вооружитесь тою же мыслью; ибо страдающий плотию перестает грешить,
\vs 1Pe 4:2 чтобы остальное во плоти время жить уже не по человеческим похотям, но по воле Божией.
\vs 1Pe 4:3 Ибо довольно, что вы в прошедшее время жизни поступали по воле языческой, предаваясь нечистотам, похотям (мужеложству, скотоложству, помыслам), пьянству, излишеству в пище и питии и нелепому идолослужению;
\vs 1Pe 4:4 почему они и дивятся, что вы не участвуете с ними в том же распутстве, и злословят вас.
\vs 1Pe 4:5 Они дадут ответ Имеющему вскоре судить живых и мертвых.
\vs 1Pe 4:6 Ибо для того и мертвым было благовествуемо, чтобы они, подвергшись суду по человеку плотию, жили по Богу духом.
\vs 1Pe 4:7 Впрочем близок всему конец.\rsbpar Итак будьте благоразумны и бодрствуйте в молитвах.
\vs 1Pe 4:8 Более же всего имейте усердную любовь друг ко другу, потому что любовь покрывает множество грехов.
\vs 1Pe 4:9 Будьте страннолюбивы друг ко другу без ропота.
\vs 1Pe 4:10 Служите друг другу, каждый тем даром, какой получил, как добрые домостроители многоразличной благодати Божией.
\vs 1Pe 4:11 Говорит ли кто, \bibemph{говори} как слова Божии; служит ли кто, \bibemph{служи} по силе, какую дает Бог, дабы во всем прославлялся Бог через Иисуса Христа, Которому слава и держава во веки веков. Аминь.
\rsbpar\vs 1Pe 4:12 Возлюбленные! огненного искушения, для испытания вам посылаемого, не чуждайтесь, как приключения для вас странного,
\vs 1Pe 4:13 но как вы участвуете в Христовых страданиях, радуйтесь, да и в явление славы Его возрадуетесь и восторжествуете.
\vs 1Pe 4:14 Если злословят вас за имя Христово, то вы блаженны, ибо Дух Славы, Дух Божий почивает на вас. Теми Он хулится, а вами прославляется.
\vs 1Pe 4:15 Только бы не пострадал кто из вас, как убийца, или вор, или злодей, или как посягающий на чужое;
\vs 1Pe 4:16 а если как Христианин, то не стыдись, но прославляй Бога за такую участь.
\vs 1Pe 4:17 Ибо время начаться суду с дома Божия; если же прежде с нас \bibemph{начнется}, то какой конец непокоряющимся Евангелию Божию?
\vs 1Pe 4:18 И если праведник едва спасается, то нечестивый и грешный где явится?
\vs 1Pe 4:19 Итак страждущие по воле Божией да предадут Ему, как верному Создателю, души свои, делая добро.
\vs 1Pe 5:1 Пастырей ваших умоляю я, сопастырь и свидетель страданий Христовых и соучастник в славе, которая должна открыться:
\vs 1Pe 5:2 пасите Божие стадо, какое у вас, надзирая за ним не принужденно, но охотно и богоугодно, не для гнусной корысти, но из усердия,
\vs 1Pe 5:3 и не господствуя над наследием \bibemph{Божиим}, но подавая пример стаду;
\vs 1Pe 5:4 и когда явится Пастыреначальник, вы получите неувядающий венец славы.
\vs 1Pe 5:5 Также и младшие, повинуйтесь пастырям; все же, подчиняясь друг другу, облекитесь смиренномудрием, потому что Бог гордым противится, а смиренным дает благодать.
\rsbpar\vs 1Pe 5:6 Итак смиритесь под крепкую руку Божию, да вознесет вас в свое время.
\vs 1Pe 5:7 Все заботы ваши возлож\acc{и}те на Него, ибо Он печется о вас.
\vs 1Pe 5:8 Трезвитесь, бодрствуйте, потому что противник ваш диавол ходит, как рыкающий лев, ища, кого поглотить.
\vs 1Pe 5:9 Противостойте ему твердою верою, зная, что такие же страдания случаются и с братьями вашими в мире.
\vs 1Pe 5:10 Бог же всякой благодати, призвавший нас в вечную славу Свою во Христе Иисусе, Сам, по кратковременном страдании вашем, да совершит вас, да утвердит, да укрепит, да соделает непоколебимыми.
\vs 1Pe 5:11 Ему слава и держава во веки веков. Аминь.
\rsbpar\vs 1Pe 5:12 Сие кратко написал я вам чрез Силуана, верного, как думаю, вашего брата, чтобы уверить вас, утешая и свидетельствуя, что это истинная благодать Божия, в которой вы стоите.
\rsbpar\vs 1Pe 5:13 Приветствует вас избранная, подобно \bibemph{вам, церковь} в Вавилоне и Марк, сын мой.
\vs 1Pe 5:14 Приветствуйте друг друга лобзанием любви. Мир вам всем во Христе Иисусе. Аминь.

\bibbookdescr{2Pe}{
  inline={Второе Соборное Послание\\\LARGE Святого Апостола Петра},
  toc={2-е Петра},
  bookmark={2-е Петра},
  header={2-е Петра},
  %headerleft={},
  %headerright={},
  abbr={2~Пет}
}
\vs 2Pe 1:1 Симон Петр, раб и Апостол Иисуса Христа, принявшим с нами равно драгоценную веру по правде Бога нашего и Спасителя Иисуса Христа:
\vs 2Pe 1:2 благодать и мир вам да умножится в познании Бога и Христа Иисуса, Господа нашего.
\rsbpar\vs 2Pe 1:3 Как от Божественной силы Его даровано нам все потребное для жизни и благочестия, через познание Призвавшего нас славою и благостию,
\vs 2Pe 1:4 которыми дарованы нам великие и драгоценные обетования, дабы вы через них соделались причастниками Божеского естества, удалившись от господствующего в мире растления похотью:
\vs 2Pe 1:5 то вы, прилагая к сему все старание, покажите в вере вашей добродетель, в добродетели рассудительность,
\vs 2Pe 1:6 в рассудительности воздержание, в воздержании терпение, в терпении благочестие,
\vs 2Pe 1:7 в благочестии братолюбие, в братолюбии любовь.
\vs 2Pe 1:8 Если это в вас есть и умножается, то вы не останетесь без успеха и плода в познании Господа нашего Иисуса Христа.
\vs 2Pe 1:9 А в ком нет сего, тот слеп, закрыл глаза, забыл об очищении прежних грехов своих.
\vs 2Pe 1:10 Посему, братия, более и более старайтесь делать твердым ваше звание и избрание; так поступая, никогда не преткнетесь,
\vs 2Pe 1:11 ибо так откроется вам свободный вход в вечное Царство Господа нашего и Спасителя Иисуса Христа.
\rsbpar\vs 2Pe 1:12 Для того я никогда не перестану напоминать вам о сем, хотя вы то и знаете, и утверждены в настоящей истине.
\vs 2Pe 1:13 Справедливым же почитаю, доколе нахожусь в этой \bibemph{телесной} храмине, возбуждать вас напоминанием,
\vs 2Pe 1:14 зная, что скоро должен оставить храмину мою, как и Господь наш Иисус Христос открыл мне.
\vs 2Pe 1:15 Буду же стараться, чтобы вы и после моего отшествия всегда приводили это на память.
\vs 2Pe 1:16 Ибо мы возвестили вам силу и пришествие Господа нашего Иисуса Христа, не хитросплетенным басням последуя, но быв очевидцами Его величия.
\vs 2Pe 1:17 Ибо Он принял от Бога Отца честь и славу, когда от велелепной славы принесся к Нему такой глас: Сей есть Сын Мой возлюбленный, в Котором Мое благоволение.
\vs 2Pe 1:18 И этот глас, принесшийся с небес, мы слышали, будучи с Ним на святой горе.
\vs 2Pe 1:19 И притом мы имеем вернейшее пророческое слово; и вы хорошо делаете, что обращаетесь к нему, как к светильнику, сияющему в темном месте, доколе не начнет рассветать день и не взойдет утренняя звезда в сердцах ваших,
\vs 2Pe 1:20 зная прежде всего то, что никакого пророчества в Писании нельзя разрешить самому собою.
\vs 2Pe 1:21 Ибо никогда пророчество не было произносимо по воле человеческой, но изрекали его святые Божии человеки, будучи движимы Духом Святым.
\vs 2Pe 2:1 Были и лжепророки в народе, как и у вас будут лжеучители, которые введут пагубные ереси и, отвергаясь искупившего их Господа, навлекут сами на себя скорую погибель.
\vs 2Pe 2:2 И многие последуют их разврату, и через них путь истины будет в поношении.
\vs 2Pe 2:3 И из любостяжания будут уловлять вас льстивыми словами; суд им давно готов, и погибель их не дремлет.
\vs 2Pe 2:4 Ибо, если Бог ангелов согрешивших не пощадил, но, связав узами адского мрака, предал блюсти на суд для наказания;
\vs 2Pe 2:5 и если не пощадил первого мира, но в восьми душах сохранил семейство Ноя, проповедника правды, когда навел потоп на мир нечестивых;
\vs 2Pe 2:6 и если города Содомские и Гоморрские, осудив на истребление, превратил в пепел, показав пример будущим нечестивцам,
\vs 2Pe 2:7 а праведного Лота, утомленного обращением между людьми неистово развратными, избавил
\vs 2Pe 2:8 (ибо сей праведник, живя между ними, ежедневно мучился в праведной душе, видя и слыша дела беззаконные)~---
\vs 2Pe 2:9 то, конечно, знает Господь, как избавлять благочестивых от искушения, а беззаконников соблюдать ко дню суда, для наказания,
\vs 2Pe 2:10 а наипаче тех, которые идут вслед скверных похотей плоти, презирают начальства, дерзки, своевольны и не страшатся злословить высших,
\vs 2Pe 2:11 тогда как и Ангелы, превосходя их крепостью и силою, не произносят на них пред Господом укоризненного суда.
\vs 2Pe 2:12 Они, как бессловесные животные, водимые природою, рожденные на уловление и истребление, злословя то, чего не понимают, в растлении своем истребятся.
\vs 2Pe 2:13 Они получат возмездие за беззаконие, ибо они полагают удовольствие во вседневной роскоши; срамники и осквернители, они наслаждаются обманами своими, пиршествуя с вами.
\vs 2Pe 2:14 Глаза у них исполнены любострастия и непрестанного греха; они прельщают неутвержденные души; сердце их приучено к любостяжанию: это сыны проклятия.
\vs 2Pe 2:15 Оставив прямой путь, они заблудились, идя по следам Валаама, сына Восорова, который возлюбил мзду неправедную,
\vs 2Pe 2:16 но был обличен в своем беззаконии: бессловесная ослица, проговорив человеческим голосом, остановила безумие пророка.
\vs 2Pe 2:17 Это безводные источники, облака и мглы, гонимые бурею: им приготовлен мрак вечной тьмы.
\vs 2Pe 2:18 Ибо, произнося надутое пустословие, они уловляют в плотские похоти и разврат тех, которые едва отстали от находящихся в заблуждении.
\vs 2Pe 2:19 Обещают им свободу, будучи сами рабы тления; ибо, кто кем побежден, тот тому и раб.
\vs 2Pe 2:20 Ибо если, избегнув скверн мира чрез познание Господа и Спасителя нашего Иисуса Христа, опять запутываются в них и побеждаются ими, то последнее бывает для таковых хуже первого.
\vs 2Pe 2:21 Лучше бы им не познать пути правды, нежели, познав, возвратиться назад от преданной им святой заповеди.
\vs 2Pe 2:22 Но с ними случается по верной пословице: пес возвращается на свою блевотину, и: вымытая свинья \bibemph{идет} валяться в грязи.
\vs 2Pe 3:1 Это уже второе послание пишу к вам, возлюбленные; в них напоминанием возбуждаю ваш чистый смысл,
\vs 2Pe 3:2 чтобы вы помнили слова, прежде реченные святыми пророками, и заповедь Господа и Спасителя, преданную Апостолами вашими.
\vs 2Pe 3:3 Прежде всего знайте, что в последние дни явятся наглые ругатели, поступающие по собственным своим похотям
\vs 2Pe 3:4 и говорящие: где обетование пришествия Его? Ибо с тех пор, как стали умирать отцы, от начала творения, всё остается так же.
\vs 2Pe 3:5 Думающие так не знают, что вначале словом Божиим небеса и земля составлены из воды и водою:
\vs 2Pe 3:6 потому тогдашний мир погиб, быв потоплен водою.
\vs 2Pe 3:7 А нынешние небеса и земля, содержимые тем же Словом, сберегаются огню на день суда и погибели нечестивых человеков.
\rsbpar\vs 2Pe 3:8 Одно т\acc{о} не должно быть сокрыто от вас, возлюбленные, что у Господа один день, как тысяча лет, и тысяча лет, как один день.
\vs 2Pe 3:9 Не медлит Господь \bibemph{исполнением} обетования, как некоторые почитают то медлением; но долготерпит нас, не желая, чтобы кто погиб, но чтобы все пришли к покаянию.
\vs 2Pe 3:10 Придет же день Господень, как тать ночью, и тогда небеса с шумом прейдут, стихии же, разгоревшись, разрушатся, земля и все дела на ней сгорят.
\vs 2Pe 3:11 Если так всё это разрушится, то какими должно быть в святой жизни и благочестии вам,
\vs 2Pe 3:12 ожидающим и желающим пришествия дня Божия, в который воспламененные небеса разрушатся и разгоревшиеся стихии растают?
\vs 2Pe 3:13 Впрочем мы, по обетованию Его, ожидаем нового неба и новой земли, на которых обитает правда.
\rsbpar\vs 2Pe 3:14 Итак, возлюбленные, ожидая сего, потщитесь явиться пред Ним неоскверненными и непорочными в мире;
\vs 2Pe 3:15 и долготерпение Господа нашего почитайте спасением, как и возлюбленный брат наш Павел, по данной ему премудрости, написал вам,
\vs 2Pe 3:16 как он говорит об этом и во всех посланиях, в которых есть нечто неудобовразумительное, что невежды и неутвержденные, к собственной своей погибели, превращают, как и прочие Писания.
\vs 2Pe 3:17 Итак вы, возлюбленные, будучи предварены о сем, берегитесь, чтобы вам не увлечься заблуждением беззаконников и не отпасть от своего утверждения,
\vs 2Pe 3:18 но возрастайте в благодати и познании Господа нашего и Спасителя Иисуса Христа. Ему слава и ныне и в день вечный. Аминь.
\newbookpage
\include{tex/1Jo}
\bibbookdescr{2Jo}{
  inline={Второе Соборное Послание\\\LARGE Святого Апостола Иоанна Богослова},
  toc={2-е Иоанна},
  bookmark={2-е Иоанна},
  header={2-е Иоанна},
  %headerleft={},
  %headerright={},
  abbr={2~Ин}
}
\vs 2Jo 1:1 Старец~--- избранной госпоже и детям ее, которых я люблю по истине, и не только я, но и все, познавшие истину,
\vs 2Jo 1:2 ради истины, которая пребывает в нас и будет с нами вовек.
\vs 2Jo 1:3 Да будет с вами благодать, милость, мир от Бога Отца и от Господа Иисуса Христа, Сына Отчего, в истине и любви.
\rsbpar\vs 2Jo 1:4 Я весьма обрадовался, что нашел из детей твоих, ходящих в истине, как мы получили заповедь от Отца.
\vs 2Jo 1:5 И ныне прошу тебя, госпожа, не как новую заповедь предписывая тебе, но ту, которую имеем от начала, чтобы мы любили друг друга.
\vs 2Jo 1:6 Любовь же состоит в том, чтобы мы поступали по заповедям Его. Это та заповедь, которую вы слышали от начала, чтобы поступали по ней.
\vs 2Jo 1:7 Ибо многие обольстители вошли в мир, не исповедующие Иисуса Христа, пришедшего во плоти: такой \bibemph{человек} есть обольститель и антихрист.
\vs 2Jo 1:8 Наблюдайте за собою, чтобы нам не потерять того, над чем мы трудились, но чтобы получить полную награду.
\vs 2Jo 1:9 Всякий, преступающий учение Христово и не пребывающий в нем, не имеет Бога; пребывающий в учении Христовом имеет и Отца и Сына.
\vs 2Jo 1:10 Кто приходит к вам и не приносит сего учения, того не принимайте в дом и не приветствуйте его.
\vs 2Jo 1:11 Ибо приветствующий его участвует в злых делах его.
\rsbpar\vs 2Jo 1:12 Многое имею писать вам, но не хочу на бумаге чернилами, а надеюсь прийти к вам и говорить устами к устам, чтобы радость ваша была полна.
\vs 2Jo 1:13 Приветствуют тебя дети сестры твоей избранной. Аминь.

\bibbookdescr{3Jo}{
  inline={Третье Соборное Послание\\\LARGE Святого Апостола Иоанна Богослова},
  toc={3-е Иоанна},
  bookmark={3-е Иоанна},
  header={3-е Иоанна},
  %headerleft={},
  %headerright={},
  abbr={3~Ин}
}
\vs 3Jo 1:1 Старец~--- возлюбленному Гаию, которого я люблю по истине.
\rsbpar\vs 3Jo 1:2 Возлюбленный! молюсь, чтобы ты здравствовал и преуспевал во всем, как преуспевает душа твоя.
\vs 3Jo 1:3 Ибо я весьма обрадовался, когда пришли братия и засвидетельствовали о твоей верности, как ты ходишь в истине.
\vs 3Jo 1:4 Для меня нет б\acc{о}льшей радости, как слышать, что дети мои ходят в истине.
\rsbpar\vs 3Jo 1:5 Возлюбленный! ты как верный поступаешь в том, что делаешь для братьев и для странников.
\vs 3Jo 1:6 Они засвидетельствовали перед церковью о твоей любви. Ты хорошо поступишь, если отпустишь их, как должно ради Бога,
\vs 3Jo 1:7 ибо они ради имени Его пошли, не взяв ничего от язычников.
\vs 3Jo 1:8 Итак мы должны принимать таковых, чтобы сделаться споспешниками истине.
\rsbpar\vs 3Jo 1:9 Я писал церкви; но любящий первенствовать у них Диотреф не принимает нас.
\vs 3Jo 1:10 Посему, если я приду, то напомню о делах, которые он делает, понося нас злыми словами, и не довольствуясь тем, и сам не принимает братьев, и запрещает желающим, и изгоняет из церкви.
\vs 3Jo 1:11 Возлюбленный! не подражай злу, но добру. Кто делает добро, тот от Бога; а делающий зло не видел Бога.
\vs 3Jo 1:12 О Димитрии засвидетельствовано всеми и самою истиною; свидетельствуем также и мы, и вы знаете, что свидетельство наше истинно.
\rsbpar\vs 3Jo 1:13 Многое имел я писать; но не хочу писать к тебе чернилами и тростью,
\vs 3Jo 1:14 а надеюсь скоро увидеть тебя и поговорить устами к устам.
\vs 3Jo 1:15 Мир тебе. Приветствуют тебя друзья; приветствуй друзей поименно. Аминь.
\newbookpage
\bibbookdescr{Jud}{
  inline={Соборное Послание\\\LARGE Святого Апостола Иуды},
  toc={Иуды},
  bookmark={Иуды},
  header={Иуды},
  %headerleft={},
  %headerright={},
  abbr={Иуд}
}
\vs Jud 1:1 Иуда, раб Иисуса Христа, брат Иакова, призванным, которые освящены Богом Отцем и сохранены Иисусом Христом:
\vs Jud 1:2 милость вам и мир и любовь да умножатся.
\rsbpar\vs Jud 1:3 Возлюбленные! имея все усердие писать вам об общем спасении, я почел за нужное написать вам увещание~--- подвизаться за веру, однажды преданную святым.
\vs Jud 1:4 Ибо вкрались некоторые люди, издревле предназначенные к сему осуждению, нечестивые, обращающие благодать Бога нашего в \bibemph{повод к} распутству и отвергающиеся единого Владыки Бога и Господа нашего Иисуса Христа.
\rsbpar\vs Jud 1:5 Я хочу напомнить вам, уже знающим это, что Господь, избавив народ из земли Египетской, потом неверовавших погубил,
\vs Jud 1:6 и ангелов, не сохранивших своего достоинства, но оставивших свое жилище, соблюдает в вечных узах, под мраком, на суд великого дня.
\vs Jud 1:7 Как Содом и Гоморра и окрестные города, подобно им блудодействовавшие и ходившие за иною плотию, подвергшись казни огня вечного, поставлены в пример,~---
\vs Jud 1:8 так точно будет и с сими мечтателями, которые оскверняют плоть, отвергают начальства и злословят высокие власти.
\vs Jud 1:9 Михаил Архангел, когда говорил с диаволом, споря о Моисеевом теле, не смел произнести укоризненного суда, но сказал: <<да запретит тебе Господь>>.
\vs Jud 1:10 А сии злословят то, чего не знают; что же по природе, как бессловесные животные, знают, тем растлевают себя.
\vs Jud 1:11 Горе им, потому что идут путем Каиновым, предаются обольщению мзды, как Валаам, и в упорстве погибают, как Корей.
\vs Jud 1:12 Таковые бывают соблазном на ваших вечерях любви; пиршествуя с вами, без страха утучняют себя. Это безводные облака, носимые ветром; осенние деревья, бесплодные, дважды умершие, исторгнутые;
\vs Jud 1:13 свирепые морские волны, пенящиеся срамотами своими; звезды блуждающие, которым блюдется мрак тьмы на веки.
\vs Jud 1:14 О них пророчествовал и Енох, седьмой от Адама, говоря: <<се, идет Господь со тьмами святых Ангелов Своих~---
\vs Jud 1:15 сотворить суд над всеми и обличить всех между ними нечестивых во всех делах, которые произвело их нечестие, и во всех жестоких словах, которые произносили на Него нечестивые грешники>>.
\vs Jud 1:16 Это ропотники, ничем не довольные, поступающие по своим похотям (нечестиво и беззаконно); уста их произносят надутые слова; они оказывают лицеприятие для корысти.
\vs Jud 1:17 Но вы, возлюбленные, помните предсказанное Апостолами Господа нашего Иисуса Христа.
\vs Jud 1:18 Они говорили вам, что в последнее время появятся ругатели, поступающие по своим нечестивым похотям.
\vs Jud 1:19 Это люди, отделяющие себя (от единства веры), душевные, не имеющие духа.
\vs Jud 1:20 А вы, возлюбленные, назидая себя на святейшей вере вашей, молясь Духом Святым,
\vs Jud 1:21 сохраняйте себя в любви Божией, ожидая милости от Господа нашего Иисуса Христа, для вечной жизни.
\vs Jud 1:22 И к одним будьте милостивы, с рассмотрением,
\vs Jud 1:23 а других страхом спасайте, исторгая из огня, обличайте же со страхом, гнушаясь даже одеждою, которая осквернена плотью.
\rsbpar\vs Jud 1:24 Могущему же соблюсти вас от падения и поставить пред славою Своею непорочными в радости,
\vs Jud 1:25 Единому Премудрому Богу, Спасителю нашему чрез Иисуса Христа Господа нашего, слава и величие, сила и власть прежде всех веков, ныне и во все веки. Аминь.

\include{tex/Rom}
\bibbookdescr{1Co}{
  inline={Первое Послание\\к Коринфянам\\\LARGE Святого Апостола Павла},
  toc={1-е Коринфянам},
  bookmark={1-е Коринфянам},
  header={1-е Коринфянам},
  %headerleft={},
  %headerright={},
  abbr={1~Кор}
}
\vs 1Co 1:1 Павел, волею Божиею призванный Апостол Иисуса Христа, и Сосфен брат,
\vs 1Co 1:2 церкви Божией, находящейся в Коринфе, освященным во Христе Иисусе, призванным святым, со всеми призывающими имя Господа нашего Иисуса Христа, во всяком месте, у них и у нас:
\vs 1Co 1:3 благодать вам и мир от Бога Отца нашего и Господа Иисуса Христа.
\rsbpar\vs 1Co 1:4 Непрестанно благодарю Бога моего за вас, ради благодати Божией, дарованной вам во Христе Иисусе,
\vs 1Co 1:5 потому что в Нем вы обогатились всем, всяким словом и всяким познанием,~---
\vs 1Co 1:6 ибо свидетельство Христово утвердилось в вас,~---
\vs 1Co 1:7 так что вы не имеете недостатка ни в каком даровании, ожидая явления Господа нашего Иисуса Христа,
\vs 1Co 1:8 Который и утвердит вас до конца, \bibemph{чтобы вам быть} неповинными в день Господа нашего Иисуса Христа.
\vs 1Co 1:9 Верен Бог, Которым вы призваны в общение Сына Его Иисуса Христа, Господа нашего.
\rsbpar\vs 1Co 1:10 Умоляю вас, братия, именем Господа нашего Иисуса Христа, чтобы все вы говорили одно, и не было между вами разделений, но чтобы вы соединены были в одном духе и в одних мыслях.
\vs 1Co 1:11 Ибо от \bibemph{домашних} Хлоиных сделалось мне известным о вас, братия мои, что между вами есть споры.
\vs 1Co 1:12 Я разумею то, что у вас говорят: <<я Павлов>>; <<я Аполлосов>>; <<я Кифин>>; <<а я Христов>>.
\vs 1Co 1:13 Разве разделился Христос? разве Павел распялся за вас? или во имя Павла вы крестились?
\vs 1Co 1:14 Благодарю Бога, что я никого из вас не крестил, кроме Криспа и Гаия,
\vs 1Co 1:15 дабы не сказал кто, что я крестил в мое имя.
\vs 1Co 1:16 Крестил я также Стефанов дом; а крестил ли еще кого, не знаю.
\vs 1Co 1:17 Ибо Христос послал меня не крестить, а благовествовать, не в премудрости слова, чтобы не упразднить креста Христова.
\vs 1Co 1:18 Ибо слово о кресте для погибающих юродство есть, а для нас, спасаемых,~--- сила Божия.
\vs 1Co 1:19 Ибо написано: погублю мудрость мудрецов, и разум разумных отвергну.
\vs 1Co 1:20 Где мудрец? где книжник? где совопросник века сего? Не обратил ли Бог мудрость мира сего в безумие?
\vs 1Co 1:21 Ибо когда мир \bibemph{своею} мудростью не познал Бога в премудрости Божией, то благоугодно было Богу юродством проповеди спасти верующих.
\vs 1Co 1:22 Ибо и Иудеи требуют чудес, и Еллины ищут мудрости;
\vs 1Co 1:23 а мы проповедуем Христа распятого, для Иудеев соблазн, а для Еллинов безумие,
\vs 1Co 1:24 для самих же призванных, Иудеев и Еллинов, Христа, Божию силу и Божию премудрость;
\vs 1Co 1:25 потому что немудрое Божие премудрее человеков, и немощное Божие сильнее человеков.
\rsbpar\vs 1Co 1:26 Посмотрите, братия, кто вы, призванные: не много \bibemph{из вас} мудрых по плоти, не много сильных, не много благородных;
\vs 1Co 1:27 но Бог избрал немудрое мира, чтобы посрамить мудрых, и немощное мира избрал Бог, чтобы посрамить сильное;
\vs 1Co 1:28 и незнатное мира и уничиженное и ничего не значащее избрал Бог, чтобы упразднить значащее,~---
\vs 1Co 1:29 для того, чтобы никакая плоть не хвалилась пред Богом.
\vs 1Co 1:30 От Него и вы во Христе Иисусе, Который сделался для нас премудростью от Бога, праведностью и освящением и искуплением,
\vs 1Co 1:31 чтобы \bibemph{было}, как написано: хвалящийся хвались Господом.
\vs 1Co 2:1 И когда я приходил к вам, братия, приходил возвещать вам свидетельство Божие не в превосходстве слова или мудрости,
\vs 1Co 2:2 ибо я рассудил быть у вас незнающим ничего, кроме Иисуса Христа, и притом распятого,
\vs 1Co 2:3 и был я у вас в немощи и в страхе и в великом трепете.
\vs 1Co 2:4 И слово мое и проповедь моя не в убедительных словах человеческой мудрости, но в явлении духа и силы,
\vs 1Co 2:5 чтобы вера ваша \bibemph{утверждалась} не на мудрости человеческой, но на силе Божией.
\rsbpar\vs 1Co 2:6 Мудрость же мы проповедуем между совершенными, но мудрость не века сего и не властей века сего преходящих,
\vs 1Co 2:7 но проповедуем премудрость Божию, тайную, сокровенную, которую предназначил Бог прежде веков к славе нашей,
\vs 1Co 2:8 которой никто из властей века сего не познал; ибо если бы познали, то не распяли бы Господа славы.
\vs 1Co 2:9 Но, как написано: не видел того глаз, не слышало ухо, и не приходило то на сердце человеку, что приготовил Бог любящим Его.
\vs 1Co 2:10 А нам Бог открыл \bibemph{это} Духом Своим; ибо Дух все проницает, и глубины Божии.
\vs 1Co 2:11 Ибо кто из человеков знает, чт\acc{о} в человеке, кроме духа человеческого, живущего в нем? Т\acc{а}к и Божьего никто не знает, кроме Духа Божия.
\vs 1Co 2:12 Но мы приняли не духа мира сего, а Духа от Бога, дабы знать дарованное нам от Бога,
\vs 1Co 2:13 что и возвещаем не от человеческой мудрости изученными словами, но изученными от Духа Святаго, соображая духовное с духовным.
\vs 1Co 2:14 Душевный человек не принимает того, чт\acc{о} от Духа Божия, потому что он почитает это безумием; и не может разуметь, потому что о сем \bibemph{надобно} судить духовно.
\vs 1Co 2:15 Но духовный судит о всем, а о нем судить никто не может.
\vs 1Co 2:16 Ибо кто познал ум Господень, чтобы \bibemph{мог} судить его? А мы имеем ум Христов.
\vs 1Co 3:1 И я не мог говорить с вами, братия, как с духовными, но как с плотскими, как с младенцами во Христе.
\vs 1Co 3:2 Я питал вас молоком, а не \bibemph{твердою} пищею, ибо вы были еще не в силах, да и теперь не в силах,
\vs 1Co 3:3 потому что вы еще плотские. Ибо если между вами зависть, споры и разногласия, то не плотские ли вы? и не по человеческому ли \bibemph{обычаю} поступаете?
\vs 1Co 3:4 Ибо когда один говорит: <<я Павлов>>, а другой: <<я Аполлосов>>, то не плотские ли вы?
\vs 1Co 3:5 Кто Павел? кто Аполлос? Они только служители, через которых вы уверовали, и притом поскольку каждому дал Господь.
\vs 1Co 3:6 Я насадил, Аполлос поливал, но возрастил Бог;
\vs 1Co 3:7 посему и насаждающий и поливающий есть ничто, а \bibemph{все} Бог возращающий.
\vs 1Co 3:8 Насаждающий же и поливающий суть одно; но каждый получит свою награду по своему труду.
\vs 1Co 3:9 Ибо мы соработники у Бога, \bibemph{а} вы Божия нива, Божие строение.
\rsbpar\vs 1Co 3:10 Я, по данной мне от Бога благодати, как мудрый строитель, положил основание, а другой строит на \bibemph{нем}; но каждый смотри, к\acc{а}к строит.
\vs 1Co 3:11 Ибо никто не может положить другого основания, кроме положенного, которое есть Иисус Христос.
\vs 1Co 3:12 Строит ли кто на этом основании из золота, серебра, драгоценных камней, дерева, сена, соломы,~---
\vs 1Co 3:13 каждого дело обнаружится; ибо день покажет, потому что в огне открывается, и огонь испытает дело каждого, каково оно есть.
\vs 1Co 3:14 У кого дело, которое он строил, устоит, тот получит награду.
\vs 1Co 3:15 А у кого дело сгорит, тот потерпит урон; впрочем сам спасется, но т\acc{а}к, как бы из огня.
\rsbpar\vs 1Co 3:16 Разве не знаете, что вы храм Божий, и Дух Божий живет в вас?
\vs 1Co 3:17 Если кто разорит храм Божий, того покарает Бог: ибо храм Божий свят; а этот \bibemph{храм}~--- вы.
\rsbpar\vs 1Co 3:18 Никто не обольщай самого себя. Если кто из вас думает быть мудрым в веке сем, тот будь безумным, чтобы быть мудрым.
\vs 1Co 3:19 Ибо мудрость мира сего есть безумие пред Богом, как написано: уловляет мудрых в лукавстве их.
\vs 1Co 3:20 И еще: Господь знает умствования мудрецов, что они суетны.
\vs 1Co 3:21 Итак никто не хвались человеками, ибо все ваше:
\vs 1Co 3:22 Павел ли, или Аполлос, или Кифа, или мир, или жизнь, или смерть, или настоящее, или будущее,~--- все ваше;
\vs 1Co 3:23 вы же~--- Христовы, а Христос~--- Божий.
\vs 1Co 4:1 Итак каждый должен разуметь нас, как служителей Христовых и домостроителей таин Божиих.
\vs 1Co 4:2 От домостроителей же требуется, чтобы каждый оказался верным.
\vs 1Co 4:3 Для меня очень мало значит, к\acc{а}к судите обо мне вы или \bibemph{к\acc{а}к судят} другие люди; я и сам не сужу о себе.
\vs 1Co 4:4 Ибо \bibemph{хотя} я ничего не знаю за собою, но тем не оправдываюсь; судия же мне Господь.
\vs 1Co 4:5 Посему не суд\acc{и}те никак прежде времени, пока не придет Господь, Который и осветит скрытое во мраке и обнаружит сердечные намерения, и тогда каждому будет похвала от Бога.
\rsbpar\vs 1Co 4:6 Это, братия, приложил я к себе и Аполлосу ради вас, чтобы вы научились от нас не мудрствовать сверх того, что написано, и не превозносились один перед другим.
\vs 1Co 4:7 Ибо кто отличает тебя? Что ты имеешь, чего бы не получил? А если получил, что хвалишься, как будто не получил?
\vs 1Co 4:8 Вы уже пресытились, вы уже обогатились, вы стали царствовать без нас. О, если бы вы \bibemph{и в самом деле} царствовали, чтобы и нам с вами царствовать!
\vs 1Co 4:9 Ибо я думаю, что нам, последним посланникам, Бог судил быть как бы приговоренными к смерти, потому что мы сделались позорищем для мира, для Ангелов и человеков.
\vs 1Co 4:10 Мы безумны Христа ради, а вы мудры во Христе; мы немощны, а вы крепки; вы в славе, а мы в бесчестии.
\vs 1Co 4:11 Даже доныне терпим голод и жажду, и наготу и побои, и скитаемся,
\vs 1Co 4:12 и трудимся, работая своими руками. Злословят нас, мы благословляем; гонят нас, мы терпим;
\vs 1Co 4:13 хулят нас, мы молим; мы как сор для мира, \bibemph{как} прах, всеми \bibemph{попираемый} доныне.
\rsbpar\vs 1Co 4:14 Не к постыжению вашему пишу сие, но вразумляю вас, как возлюбленных детей моих.
\vs 1Co 4:15 Ибо, хотя у вас тысячи наставников во Христе, но не много отцов; я родил вас во Христе Иисусе благовествованием.
\vs 1Co 4:16 Посему умоляю вас: подражайте мне, как я Христу.
\vs 1Co 4:17 Для сего я послал к вам Тимофея, моего возлюбленного и верного в Господе сына, который напомнит вам о путях моих во Христе, как я учу везде во всякой церкви.
\vs 1Co 4:18 Как я не иду к вам, то некоторые \bibemph{у вас} возгордились;
\vs 1Co 4:19 но я скоро приду к вам, если угодно будет Господу, и испытаю не слова возгордившихся, а силу,
\vs 1Co 4:20 ибо Царство Божие не в слове, а в силе.
\vs 1Co 4:21 Чего вы хотите? с жезлом прийти к вам, или с любовью и духом кротости?
\vs 1Co 5:1 Есть верный слух, что у вас \bibemph{появилось} блудодеяние, и притом такое блудодеяние, какого не слышно даже у язычников, что некто \bibemph{вместо жены} имеет жену отца своего.
\vs 1Co 5:2 И вы возгордились, вместо того, чтобы лучше плакать, дабы изъят был из среды вас сделавший такое дело.
\vs 1Co 5:3 А я, отсутствуя телом, но присутствуя \bibemph{у вас} духом, уже решил, как бы находясь у вас: сделавшего такое дело,
\vs 1Co 5:4 в собрании вашем во имя Господа нашего Иисуса Христа, обще с моим духом, силою Господа нашего Иисуса Христа,
\vs 1Co 5:5 предать сатане во измождение плоти, чтобы дух был спасен в день Господа нашего Иисуса Христа.
\vs 1Co 5:6 Нечем вам хвалиться. Разве не знаете, что малая закваска квасит все тесто?
\vs 1Co 5:7 Итак очистите старую закваску, чтобы быть вам новым тестом, так как вы бесквасны, ибо Пасха наша, Христос, заклан за нас.
\vs 1Co 5:8 Посему станем праздновать не со старою закваскою, не с закваскою порока и лукавства, но с опресноками чистоты и истины.
\rsbpar\vs 1Co 5:9 Я писал вам в послании~--- не сообщаться с блудниками;
\vs 1Co 5:10 впрочем не вообще с блудниками мира сего, или лихоимцами, или хищниками, или идолослужителями, ибо иначе надлежало бы вам выйти из мира \bibemph{сего}.
\vs 1Co 5:11 Но я писал вам не сообщаться с тем, кто, называясь братом, остается блудником, или лихоимцем, или идолослужителем, или злоречивым, или пьяницею, или хищником; с таким даже и не есть вместе.
\vs 1Co 5:12 Ибо чт\acc{о} мне судить и внешних? Не внутренних ли вы судите?
\vs 1Co 5:13 Внешних же судит Бог. Итак, извергните развращенного из среды вас.
\vs 1Co 6:1 Как смеет кто у вас, имея дело с другим, судиться у нечестивых, а не у святых?
\vs 1Co 6:2 Разве не знаете, что святые будут судить мир? Если же вами будет судим мир, то неужели вы недостойны судить маловажные \bibemph{дела}?
\vs 1Co 6:3 Разве не знаете, что мы будем судить ангелов, не тем ли более \bibemph{дела} житейские?
\vs 1Co 6:4 А вы, когда имеете житейские тяжбы, поставляете \bibemph{своими судьями} ничего не значащих в церкви.
\vs 1Co 6:5 К стыду вашему говорю: неужели нет между вами ни одного разумного, который мог бы рассудить между братьями своими?
\vs 1Co 6:6 Но брат с братом судится, и притом перед неверными.
\vs 1Co 6:7 И то уже весьма унизительно для вас, что вы имеете тяжбы между собою. Для чего бы вам лучше не оставаться обиженными? для чего бы вам лучше не терпеть лишения?
\vs 1Co 6:8 Но вы \bibemph{сами} обижаете и отнимаете, и притом у братьев.
\vs 1Co 6:9 Или не знаете, что неправедные Царства Божия не наследуют? Не обманывайтесь: ни блудники, ни идолослужители, ни прелюбодеи, ни малакии, ни мужеложники,
\vs 1Co 6:10 ни воры, ни лихоимцы, ни пьяницы, ни злоречивые, ни хищники~--- Царства Божия не наследуют.
\vs 1Co 6:11 И такими были некоторые из вас; но омылись, но освятились, но оправдались именем Господа нашего Иисуса Христа и Духом Бога нашего.
\rsbpar\vs 1Co 6:12 Все мне позволительно, но не все полезно; все мне позволительно, но ничто не должно обладать мною.
\vs 1Co 6:13 Пища для чрева, и чрево для пищи; но Бог уничтожит и то и другое. Тело же не для блуда, но для Господа, и Господь для тела.
\vs 1Co 6:14 Бог воскресил Господа, воскресит и нас силою Своею.
\rsbpar\vs 1Co 6:15 Разве не знаете, что тел\acc{а} ваши суть члены Христовы? Итак отниму ли члены у Христа, чтобы сделать \bibemph{их} членами блудницы? Да не будет!
\vs 1Co 6:16 Или не знаете, что совокупляющийся с блудницею становится одно тело \bibemph{с нею}? ибо сказано: два будут одна плоть.
\vs 1Co 6:17 А соединяющийся с Господом есть один дух с Господом.
\vs 1Co 6:18 Бегайте блуда; всякий грех, какой делает человек, есть вне тела, а блудник грешит против собственного тела.
\vs 1Co 6:19 Не знаете ли, что тел\acc{а} ваши суть храм живущего в вас Святаго Духа, Которого имеете вы от Бога, и вы не свои?
\vs 1Co 6:20 Ибо вы куплены \bibemph{дорогою} ценою. Посему прославляйте Бога и в телах ваших и в душах ваших, которые суть Божии.
\vs 1Co 7:1 А о чем вы писали ко мне, то хорошо человеку не касаться женщины.
\vs 1Co 7:2 Но, \bibemph{во избежание} блуда, каждый имей свою жену, и каждая имей своего мужа.
\vs 1Co 7:3 Муж оказывай жене должное благорасположение; подобно и жена мужу.
\vs 1Co 7:4 Жена не властна над своим телом, но муж; равно и муж не властен над своим телом, но жена.
\vs 1Co 7:5 Не уклоняйтесь друг от друга, разве по согласию, на время, для упражнения в посте и молитве, а \bibemph{потом} опять будьте вместе, чтобы не искушал вас сатана невоздержанием вашим.
\vs 1Co 7:6 Впрочем это сказано мною как позволение, а не как повеление.
\vs 1Co 7:7 Ибо желаю, чтобы все люди были, как и я; но каждый имеет свое дарование от Бога, один так, другой иначе.
\rsbpar\vs 1Co 7:8 Безбрачным же и вдовам говорю: хорошо им оставаться, как я.
\vs 1Co 7:9 Но если не \bibemph{могут} воздержаться, пусть вступают в брак; ибо лучше вступить в брак, нежели разжигаться.
\vs 1Co 7:10 А вступившим в брак не я повелеваю, а Господь: жене не разводиться с мужем,~---
\vs 1Co 7:11 если же разведется, то должна оставаться безбрачною, или примириться с мужем своим,~--- и мужу не оставлять жены \bibemph{своей}.
\vs 1Co 7:12 Прочим же я говорю, а не Господь: если какой брат имеет жену неверующую, и она согласна жить с ним, то он не должен оставлять ее;
\vs 1Co 7:13 и жена, которая имеет мужа неверующего, и он согласен жить с нею, не должна оставлять его.
\vs 1Co 7:14 Ибо неверующий муж освящается женою верующею, и жена неверующая освящается мужем верующим. Иначе дети ваши были бы нечисты, а теперь святы.
\vs 1Co 7:15 Если же неверующий \bibemph{хочет} развестись, пусть разводится; брат или сестра в таких \bibemph{случаях} не связаны; к миру призвал нас Господь.
\vs 1Co 7:16 Почему ты знаешь, жена, не спасешь ли мужа? Или ты, муж, почему знаешь, не спасешь ли жены?
\vs 1Co 7:17 Только каждый поступай так, как Бог ему определил, и каждый, как Господь призвал. Так я повелеваю по всем церквам.
\vs 1Co 7:18 Призван ли кто обрезанным, не скрывайся; призван ли кто необрезанным, не обрезывайся.
\vs 1Co 7:19 Обрезание ничто и необрезание ничто, но \bibemph{всё} в соблюдении заповедей Божиих.
\vs 1Co 7:20 Каждый оставайся в том звании, в котором призван.
\vs 1Co 7:21 Рабом ли ты призван, не смущайся; но если и можешь сделаться свободным, то лучшим воспользуйся.
\vs 1Co 7:22 Ибо раб, призванный в Господе, есть свободный Господа; равно и призванный свободным есть раб Христов.
\vs 1Co 7:23 Вы куплены \bibemph{дорогою} ценою; не делайтесь рабами человеков.
\vs 1Co 7:24 В каком \bibemph{звании} кто призван, братия, в том каждый и оставайся пред Богом.
\rsbpar\vs 1Co 7:25 Относительно девства я не имею повеления Господня, а даю совет, как получивший от Господа милость быть \bibemph{Ему} верным.
\vs 1Co 7:26 По настоящей нужде за лучшее призна\acc{ю}, что хорошо человеку оставаться т\acc{а}к.
\vs 1Co 7:27 Соединен ли ты с женой? не ищи развода. Остался ли без жены? не ищи жены.
\vs 1Co 7:28 Впрочем, если и женишься, не согрешишь; и если девица выйдет замуж, не согрешит. Но таковые будут иметь скорби по плоти; а мне вас жаль.
\rsbpar\vs 1Co 7:29 Я вам сказываю, братия: время уже коротко, так что имеющие жен должны быть, как не имеющие;
\vs 1Co 7:30 и плачущие, как не плачущие; и радующиеся, как не радующиеся; и покупающие, как не приобретающие;
\vs 1Co 7:31 и пользующиеся миром сим, как не пользующиеся; ибо проходит образ мира сего.
\vs 1Co 7:32 А я хочу, чтобы вы были без забот. Неженатый заботится о Господнем, как угодить Господу;
\vs 1Co 7:33 а женатый заботится о мирском, как угодить жене. Есть разность между замужнею и девицею:
\vs 1Co 7:34 незамужняя заботится о Господнем, как угодить Господу, чтобы быть святою и телом и духом; а замужняя заботится о мирском, как угодить мужу.
\vs 1Co 7:35 Говорю это для вашей же пользы, не с тем, чтобы наложить на вас узы, но чтобы вы благочинно и непрестанно \bibemph{служили} Господу без развлечения.
\vs 1Co 7:36 Если же кто почитает неприличным для своей девицы то, чтобы она, будучи в зрелом возрасте, оставалась так, тот пусть делает, как хочет: не согрешит; пусть \bibemph{таковые} выходят замуж.
\vs 1Co 7:37 Но кто непоколебимо тверд в сердце своем и, не будучи стесняем нуждою, но будучи властен в своей воле, решился в сердце своем соблюдать свою деву, тот хорошо поступает.
\vs 1Co 7:38 Посему выдающий замуж свою девицу поступает хорошо; а не выдающий поступает лучше.
\vs 1Co 7:39 Жена связана законом, доколе жив муж ее; если же муж ее умрет, свободна выйти, за кого хочет, только в Господе.
\vs 1Co 7:40 Но она блаженнее, если останется так, по моему совету; а думаю, и я имею Духа Божия.
\vs 1Co 8:1 О идоложертвенных \bibemph{яствах} мы знаем, потому что мы все имеем знание; но знание надмевает, а любовь назидает.
\vs 1Co 8:2 Кто думает, что он знает что-нибудь, тот ничего еще не знает так, как должно знать.
\vs 1Co 8:3 Но кто любит Бога, тому дано знание от Него.
\vs 1Co 8:4 Итак об употреблении в пищу идоложертвенного мы знаем, что идол в мире ничто, и что нет иного Бога, кроме Единого.
\vs 1Co 8:5 Ибо хотя и есть так называемые боги, или на небе, или на земле, так как есть много богов и господ много,~---
\vs 1Co 8:6 но у нас один Бог Отец, из Которого все, и мы для Него, и один Господь Иисус Христос, Которым все, и мы Им.
\vs 1Co 8:7 Но не у всех \bibemph{такое} знание: некоторые и доныне с совестью, \bibemph{признающею} идолов, едят \bibemph{идоложертвенное} как жертвы идольские, и совесть их, будучи немощна, оскверняется.
\vs 1Co 8:8 Пища не приближает нас к Богу: ибо, едим ли мы, ничего не приобретаем; не едим ли, ничего не теряем.
\vs 1Co 8:9 Берегитесь однако же, чтобы эта свобода ваша не послужила соблазном для немощных.
\vs 1Co 8:10 Ибо если кто-нибудь увидит, что ты, имея знание, сидишь за столом в капище, то совесть его, как немощного, не расположит ли и его есть идоложертвенное?
\vs 1Co 8:11 И от знания твоего погибнет немощный брат, за которого умер Христос.
\vs 1Co 8:12 А согрешая таким образом против братьев и уязвляя немощную совесть их, вы согрешаете против Христа.
\vs 1Co 8:13 И потому, если пища соблазняет брата моего, не буду есть мяса вовек, чтобы не соблазнить брата моего.
\vs 1Co 9:1 Не Апостол ли я? Не свободен ли я? Не видел ли я Иисуса Христа, Господа нашего? Не мое ли дело вы в Господе?
\vs 1Co 9:2 Если для других я не Апостол, то для вас \bibemph{Апостол}; ибо печать моего апостольства~--- вы в Господе.
\vs 1Co 9:3 Вот мое защищение против осуждающих меня.
\vs 1Co 9:4 Или мы не имеем власти есть и пить?
\vs 1Co 9:5 Или не имеем власти иметь спутницею сестру жену, как и прочие Апостолы, и братья Господни, и Кифа?
\vs 1Co 9:6 Или один я и Варнава не имеем власти не работать?
\vs 1Co 9:7 Какой воин служит когда-либо на своем содержании? Кто, насадив виноград, не ест плодов его? Кто, пася стадо, не ест молока от стада?
\vs 1Co 9:8 По человеческому ли только \bibemph{рассуждению} я это говорю? Не то же ли говорит и закон?
\vs 1Co 9:9 Ибо в Моисеевом законе написано: не заграждай рта у вола молотящего. О волах ли печется Бог?
\vs 1Co 9:10 Или, конечно, для нас говорится? Так, для нас это написано; ибо, кто пашет, должен пахать с надеждою, и кто молотит, \bibemph{должен молотить} с надеждою получить ожидаемое.
\vs 1Co 9:11 Если мы посеяли в вас духовное, велико ли то, если пожнем у вас телесное?
\vs 1Co 9:12 Если другие имеют у вас власть, не паче ли мы? Однако мы не пользовались сею властью, но все переносим, дабы не поставить какой преграды благовествованию Христову.
\vs 1Co 9:13 Разве не знаете, что священнодействующие питаются от святилища? что служащие жертвеннику берут долю от жертвенника?
\vs 1Co 9:14 Т\acc{а}к и Господь повелел проповедующим Евангелие жить от благовествования.
\vs 1Co 9:15 Но я не пользовался ничем таковым. И написал это не для того, чтобы т\acc{а}к было для меня. Ибо для меня лучше умереть, нежели чтобы кто уничтожил похвалу мою.
\vs 1Co 9:16 Ибо если я благовествую, то нечем мне хвалиться, потому что это необходимая \bibemph{обязанность} моя, и горе мне, если не благовествую!
\vs 1Co 9:17 Ибо если делаю это добровольно, то \bibemph{буду} иметь награду; а если недобровольно, то \bibemph{исполняю только} вверенное мне служение.
\vs 1Co 9:18 За чт\acc{о} же мне награда? За т\acc{о}, что, проповедуя Евангелие, благовествую о Христе безмездно, не пользуясь моею властью в благовествовании.
\vs 1Co 9:19 Ибо, будучи свободен от всех, я всем поработил себя, дабы больше приобрести:
\vs 1Co 9:20 для Иудеев я был как Иудей, чтобы приобрести Иудеев; для подзаконных был как подзаконный, чтобы приобрести подзаконных;
\vs 1Co 9:21 для чуждых закона~--- как чуждый закона,~--- не будучи чужд закона пред Богом, но подзаконен Христу,~--- чтобы приобрести чуждых закона;
\vs 1Co 9:22 для немощных был как немощный, чтобы приобрести немощных. Для всех я сделался всем, чтобы спасти по крайней мере некоторых.
\vs 1Co 9:23 Сие же делаю для Евангелия, чтобы быть соучастником его.
\vs 1Co 9:24 Не знаете ли, что бегущие на ристалище бегут все, но один получает награду? Так бегите, чтобы получить.
\vs 1Co 9:25 Все подвижники воздерживаются от всего: те для получения венца тленного, а мы~--- нетленного.
\vs 1Co 9:26 И потому я бегу не так, как на неверное, бьюсь не так, чтобы только бить воздух;
\vs 1Co 9:27 но усмиряю и порабощаю тело мое, дабы, проповедуя другим, самому не остаться недостойным.
\vs 1Co 10:1 Не хочу оставить вас, братия, в неведении, что отцы наши все были под облаком, и все прошли сквозь море;
\vs 1Co 10:2 и все крестились в Моисея в облаке и в море;
\vs 1Co 10:3 и все ели одну и ту же духовную пищу;
\vs 1Co 10:4 и все пили одно и то же духовное питие: ибо пили из духовного последующего камня; камень же был Христос.
\vs 1Co 10:5 Но не о многих из них благоволил Бог, ибо они поражены были в пустыне.
\vs 1Co 10:6 А это были образы для нас, чтобы мы не были похотливы на злое, как они были похотливы.
\vs 1Co 10:7 Не будьте также идолопоклонниками, как некоторые из них, о которых написано: народ сел есть и пить, и встал играть.
\vs 1Co 10:8 Не станем блудодействовать, как некоторые из них блудодействовали, и в один день погибло их двадцать три тысячи.
\vs 1Co 10:9 Не станем искушать Христа, как некоторые из них искушали и погибли от змей.
\vs 1Co 10:10 Не ропщите, как некоторые из них роптали и погибли от истребителя.
\vs 1Co 10:11 Все это происходило с ними, \bibemph{как} образы; а описано в наставление нам, достигшим последних веков.
\vs 1Co 10:12 Посему, кто думает, что он сто\acc{и}т, берегись, чтобы не упасть.
\vs 1Co 10:13 Вас постигло искушение не иное, как человеческое; и верен Бог, Который не попустит вам быть искушаемыми сверх сил, но при искушении даст и облегчение, так чтобы вы могли перенести.
\rsbpar\vs 1Co 10:14 Итак, возлюбленные мои, убегайте идолослужения.
\vs 1Co 10:15 Я говорю \bibemph{вам} как рассудительным; сами рассуд\acc{и}те о том, что говорю.
\vs 1Co 10:16 Чаша благословения, которую благословляем, не есть ли приобщение Крови Христовой? Хлеб, который преломляем, не есть ли приобщение Тела Христова?
\vs 1Co 10:17 Один хлеб, и мы многие одно тело; ибо все причащаемся от одного хлеба.
\vs 1Co 10:18 Посмотрите на Израиля по плоти: те, которые едят жертвы, не участники ли жертвенника?
\vs 1Co 10:19 Что же я говорю? То ли, что идол есть что-нибудь, или идоложертвенное значит что-нибудь?
\vs 1Co 10:20 \bibemph{Нет}, но что язычники, принося жертвы, приносят бесам, а не Богу. Но я не хочу, чтобы вы были в общении с бесами.
\vs 1Co 10:21 Не можете пить чашу Господню и чашу бесовскую; не можете быть участниками в трапезе Господней и в трапезе бесовской.
\vs 1Co 10:22 Неужели мы \bibemph{решимся} раздражать Господа? Разве мы сильнее Его?
\rsbpar\vs 1Co 10:23 Все мне позволительно, но не все полезно; все мне позволительно, но не все назидает.
\vs 1Co 10:24 Никто не ищи своего, но каждый \bibemph{пользы} другого.
\vs 1Co 10:25 Все, что продается на торгу, ешьте без всякого исследования, для \bibemph{спокойствия} совести;
\vs 1Co 10:26 ибо Господня земля, и чт\acc{о} наполняет ее.
\vs 1Co 10:27 Если кто из неверных позовет вас, и вы захотите пойти, то все, предлагаемое вам, ешьте без всякого исследования, для \bibemph{спокойствия} совести.
\vs 1Co 10:28 Но если кто скажет вам: это идоложертвенное,~--- то не ешьте ради того, кто объявил вам, и ради совести. Ибо Господня земля, и чт\acc{о} наполняет ее.
\vs 1Co 10:29 Совесть же разумею не свою, а другого: ибо для чего моей свободе быть судимой чужою совестью?
\vs 1Co 10:30 Если я с благодарением принимаю \bibemph{пищу}, то для чего порицать меня за то, за что я благодарю?
\vs 1Co 10:31 Итак, едите ли, пьете ли, или иное что делаете, все делайте в славу Божию.
\vs 1Co 10:32 Не подавайте соблазна ни Иудеям, ни Еллинам, ни церкви Божией,
\vs 1Co 10:33 так, как и я угождаю всем во всем, ища не своей пользы, но \bibemph{пользы} многих, чтобы они спаслись.
\vs 1Co 11:1 Будьте подражателями мне, как я Христу.
\rsbpar\vs 1Co 11:2 Хвалю вас, братия, что вы все мое помните и держите предания так, как я передал вам.
\vs 1Co 11:3 Хочу также, чтобы вы знали, что всякому мужу глава Христос, жене глава~--- муж, а Христу глава~--- Бог.
\vs 1Co 11:4 Всякий муж, молящийся или пророчествующий с покрытою головою, постыжает свою голову.
\vs 1Co 11:5 И всякая жена, молящаяся или пророчествующая с открытою головою, постыжает свою голову, ибо \bibemph{это} то же, как если бы она была обритая.
\vs 1Co 11:6 Ибо если жена не хочет покрываться, то пусть и стрижется; а если жене стыдно быть остриженной или обритой, пусть покрывается.
\vs 1Co 11:7 Итак муж не должен покрывать голову, потому что он есть образ и слава Божия; а жена есть слава мужа.
\vs 1Co 11:8 Ибо не муж от жены, но жена от мужа;
\vs 1Co 11:9 и не муж создан для жены, но жена для мужа.
\vs 1Co 11:10 Посему жена и должна иметь на голове своей \bibemph{знак} власти \bibemph{над нею}, для Ангелов.
\vs 1Co 11:11 Впрочем ни муж без жены, ни жена без мужа, в Господе.
\vs 1Co 11:12 Ибо как жена от мужа, так и муж через жену; все же~--- от Бога.
\vs 1Co 11:13 Рассудите сами, прилично ли жене молиться Богу с непокрытою \bibemph{головою}?
\vs 1Co 11:14 Не сама ли природа учит вас, что если муж растит волосы, то это бесчестье для него,
\vs 1Co 11:15 но если жена растит волосы, для нее это честь, так как волосы даны ей вместо покрывала?
\vs 1Co 11:16 А если бы кто захотел спорить, то мы не имеем такого обычая, ни церкви Божии.
\rsbpar\vs 1Co 11:17 Но, предлагая сие, не хвалю \bibemph{вас}, что вы собираетесь не на лучшее, а на худшее.
\vs 1Co 11:18 Ибо, во-первых, слышу, что, когда вы собираетесь в церковь, между вами бывают разделения, чему отчасти и верю.
\vs 1Co 11:19 Ибо надлежит быть и разномыслиям между вами, дабы открылись между вами искусные.
\vs 1Co 11:20 Далее, вы собираетесь, \bibemph{так, что это} не значит вкушать вечерю Господню;
\vs 1Co 11:21 ибо всякий поспешает прежде \bibemph{других} есть свою пищу, \bibemph{так что} иной бывает голоден, а иной упивается.
\vs 1Co 11:22 Разве у вас нет домов на то, чтобы есть и пить? Или пренебрегаете церковь Божию и унижаете неимущих? Чт\acc{о} сказать вам? похвалить ли вас за это? Не похвалю.
\vs 1Co 11:23 Ибо я от \bibemph{Самого} Господа принял т\acc{о}, что и вам передал, что Господь Иисус в ту ночь, в которую предан был, взял хлеб
\vs 1Co 11:24 и, возблагодарив, преломил и сказал: приимите, ядите, сие есть Тело Мое, за вас ломимое; сие творите в Мое воспоминание.
\vs 1Co 11:25 Также и чашу после вечери, и сказал: сия чаша есть новый завет в Моей Крови; сие творите, когда только будете пить, в Мое воспоминание.
\vs 1Co 11:26 Ибо всякий раз, когда вы едите хлеб сей и пьете чашу сию, смерть Господню возвещаете, доколе Он придет.
\vs 1Co 11:27 Посему, кто будет есть хлеб сей или пить чашу Господню недостойно, виновен будет против Тела и Крови Господней.
\vs 1Co 11:28 Да испытывает же себя человек, и таким образом пусть ест от хлеба сего и пьет из чаши сей.
\vs 1Co 11:29 Ибо, кто ест и пьет недостойно, тот ест и пьет осуждение себе, не рассуждая о Теле Господнем.
\vs 1Co 11:30 Оттого многие из вас немощны и больны и немало умирает.
\vs 1Co 11:31 Ибо если бы мы судили сами себя, то не были бы судимы.
\vs 1Co 11:32 Будучи же судимы, наказываемся от Господа, чтобы не быть осужденными с миром.
\vs 1Co 11:33 Посему, братия мои, собираясь на вечерю, друг друга ждите.
\vs 1Co 11:34 А если кто голоден, пусть ест дома, чтобы собираться вам не на осуждение. Прочее устрою, когда приду.
\vs 1Co 12:1 Не хочу оставить вас, братия, в неведении и о \bibemph{дарах} духовных.
\vs 1Co 12:2 Знаете, что когда вы были язычниками, то ходили к безгласным идолам, так, как бы вели вас.
\vs 1Co 12:3 Потому сказываю вам, что никто, говорящий Духом Божиим, не произнесет анафемы на Иисуса, и никто не может назвать Иисуса Господом, как только Духом Святым.
\vs 1Co 12:4 Дары различны, но Дух один и тот же;
\vs 1Co 12:5 и служения различны, а Господь один и тот же;
\vs 1Co 12:6 и действия различны, а Бог один и тот же, производящий все во всех.
\vs 1Co 12:7 Но каждому дается проявление Духа на пользу.
\vs 1Co 12:8 Одному дается Духом слово мудрости, другому слово знания, тем же Духом;
\vs 1Co 12:9 иному вера, тем же Духом; иному дары исцелений, тем же Духом;
\vs 1Co 12:10 иному чудотворения, иному пророчество, иному различение духов, иному разные языки, иному истолкование языков.
\vs 1Co 12:11 Все же сие производит один и тот же Дух, разделяя каждому особо, как Ему угодно.
\vs 1Co 12:12 Ибо, как тело одно, но имеет многие члены, и все члены одного тела, хотя их и много, составляют одно тело,~--- так и Христос.
\vs 1Co 12:13 Ибо все мы одним Духом крестились в одно тело, Иудеи или Еллины, рабы или свободные, и все напоены одним Духом.
\vs 1Co 12:14 Тело же не из одного члена, но из многих.
\vs 1Co 12:15 Если нога скажет: я не принадлежу к телу, потому что я не рука, то неужели она потому не принадлежит к телу?
\vs 1Co 12:16 И если ухо скажет: я не принадлежу к телу, потому что я не глаз, то неужели оно потому не принадлежит к телу?
\vs 1Co 12:17 Если все тело глаз, то где слух? Если все слух, то где обоняние?
\vs 1Co 12:18 Но Бог расположил члены, каждый в \bibemph{составе} тела, как Ему было угодно.
\vs 1Co 12:19 А если бы все были один член, то где \bibemph{было бы} тело?
\vs 1Co 12:20 Но теперь членов много, а тело одно.
\vs 1Co 12:21 Не может глаз сказать руке: ты мне не надобна; или также голова ногам: вы мне не нужны.
\vs 1Co 12:22 Напротив, члены тела, которые кажутся слабейшими, гораздо нужнее,
\vs 1Co 12:23 и которые нам кажутся менее благородными в теле, о тех более прилагаем попечения;
\vs 1Co 12:24 и неблагообразные наши более благовидно покрываются, а благообразные наши не имеют \bibemph{в том} нужды. Но Бог соразмерил тело, внушив о менее совершенном большее попечение,
\vs 1Co 12:25 дабы не было разделения в теле, а все члены одинаково заботились друг о друге.
\vs 1Co 12:26 Посему, страдает ли один член, страдают с ним все члены; славится ли один член, с ним радуются все члены.
\vs 1Co 12:27 И вы~--- тело Христово, а порознь~--- члены.
\vs 1Co 12:28 И иных Бог поставил в Церкви, во-первых, Апостолами, во-вторых, пророками, в-третьих, учителями; далее, \bibemph{иным дал} силы \bibemph{чудодейственные}, также дары исцелений, вспоможения, управления, разные языки.
\vs 1Co 12:29 Все ли Апостолы? Все ли пророки? Все ли учители? Все ли чудотворцы?
\vs 1Co 12:30 Все ли имеют дары исцелений? Все ли говорят языками? Все ли истолкователи?
\vs 1Co 12:31 Ревнуйте о дарах б\acc{о}льших, и я покажу вам путь еще превосходнейший.
\vs 1Co 13:1 Если я говорю языками человеческими и ангельскими, а любви не имею, то я~--- медь звенящая или кимвал звучащий.
\vs 1Co 13:2 Если имею \bibemph{дар} пророчества, и знаю все тайны, и имею всякое познание и всю веру, так что \bibemph{могу} и горы переставлять, а не имею любви,~--- то я ничто.
\vs 1Co 13:3 И если я раздам все имение мое и отдам тело мое на сожжение, а любви не имею, нет мне в том никакой пользы.
\vs 1Co 13:4 Любовь долготерпит, милосердствует, любовь не завидует, любовь не превозносится, не гордится,
\vs 1Co 13:5 не бесчинствует, не ищет своего, не раздражается, не мыслит зла,
\vs 1Co 13:6 не радуется неправде, а сорадуется истине;
\vs 1Co 13:7 все покрывает, всему верит, всего надеется, все переносит.
\vs 1Co 13:8 Любовь никогда не перестает, хотя и пророчества прекратятся, и языки умолкнут, и знание упразднится.
\vs 1Co 13:9 Ибо мы отчасти знаем, и отчасти пророчествуем;
\vs 1Co 13:10 когда же настанет совершенное, тогда то, что отчасти, прекратится.
\vs 1Co 13:11 Когда я был младенцем, то по-младенчески говорил, по-младенчески мыслил, по-младенчески рассуждал; а как стал мужем, то оставил младенческое.
\vs 1Co 13:12 Теперь мы видим как бы сквозь \bibemph{тусклое} стекло, гадательно, тогда же лицем к лицу; теперь знаю я отчасти, а тогда позн\acc{а}ю, подобно как я познан.
\vs 1Co 13:13 А теперь пребывают сии три: вера, надежда, любовь; но любовь из них больше.
\vs 1Co 14:1 Достигайте любви; ревнуйте о \bibemph{дарах} духовных, особенно же о том, чтобы пророчествовать.
\vs 1Co 14:2 Ибо кто говорит на \bibemph{незнакомом} языке, тот говорит не людям, а Богу; потому что никто не понимает \bibemph{его}, он тайны говорит духом;
\vs 1Co 14:3 а кто пророчествует, тот говорит людям в назидание, увещание и утешение.
\vs 1Co 14:4 Кто говорит на \bibemph{незнакомом} языке, тот назидает себя; а кто пророчествует, тот назидает церковь.
\vs 1Co 14:5 Желаю, чтобы вы все говорили языками; но лучше, чтобы вы пророчествовали; ибо пророчествующий превосходнее того, кто говорит языками, разве он притом будет и изъяснять, чтобы церковь получила назидание.
\vs 1Co 14:6 Теперь, если я приду к вам, братия, и стану говорить на \bibemph{незнакомых} языках, то какую принесу вам пользу, когда не изъяснюсь вам или откровением, или познанием, или пророчеством, или учением?
\vs 1Co 14:7 И бездушные \bibemph{вещи}, издающие звук, свирель или гусли, если не производят раздельных тонов, как распознать т\acc{о}, чт\acc{о} играют на свирели или на гуслях?
\vs 1Co 14:8 И если труба будет издавать неопределенный звук, кто станет готовиться к сражению?
\vs 1Co 14:9 Так если и вы языком произносите невразумительные слова, то как узн\acc{а}ют, чт\acc{о} вы говорите? Вы будете говорить на ветер.
\vs 1Co 14:10 Сколько, например, различных слов в мире, и ни одного из них нет без значения.
\vs 1Co 14:11 Но если я не разумею значения слов, то я для говорящего чужестранец, и говорящий для меня чужестранец.
\vs 1Co 14:12 Так и вы, ревнуя о \bibemph{дарах} духовных, старайтесь обогатиться \bibemph{ими} к назиданию церкви.
\vs 1Co 14:13 А потому, говорящий на \bibemph{незнакомом} языке, молись о даре истолкования.
\vs 1Co 14:14 Ибо когда я молюсь на \bibemph{незнакомом} языке, то хотя дух мой и молится, но ум мой остается без плода.
\vs 1Co 14:15 Что же делать? Стану молиться духом, стану молиться и умом; буду петь духом, буду петь и умом.
\vs 1Co 14:16 Ибо если ты будешь благословлять духом, то стоящий на месте простолюдина к\acc{а}к скажет: <<аминь>> при твоем благодарении? Ибо он не понимает, чт\acc{о} ты говоришь.
\vs 1Co 14:17 Ты хорошо благодаришь, но другой не назидается.
\vs 1Co 14:18 Благодарю Бога моего: я более всех вас говорю языками;
\vs 1Co 14:19 но в церкви хочу лучше пять слов сказать умом моим, чтобы и других наставить, нежели тьму слов на \bibemph{незнакомом} языке.
\rsbpar\vs 1Co 14:20 Братия! не будьте дети умом: на злое будьте младенцы, а по уму будьте совершеннолетни.
\vs 1Co 14:21 В законе написано: иными языками и иными устами буду говорить народу сему; но и тогда не послушают Меня, говорит Господь.
\vs 1Co 14:22 Итак языки суть знамение не для верующих, а для неверующих; пророчество же не для неверующих, а для верующих.
\vs 1Co 14:23 Если вся церковь сойдется вместе, и все станут говорить \bibemph{незнакомыми} языками, и войдут к вам незнающие или неверующие, то не скажут ли, что вы беснуетесь?
\vs 1Co 14:24 Но когда все пророчествуют, и войдет кто неверующий или незнающий, то он всеми обличается, всеми судится.
\vs 1Co 14:25 И таким образом тайны сердца его обнаруживаются, и он падет ниц, поклонится Богу и скажет: истинно с вами Бог.
\rsbpar\vs 1Co 14:26 Итак чт\acc{о} же, братия? Когда вы сходитесь, и у каждого из вас есть псалом, есть поучение, есть язык, есть откровение, есть истолкование,~--- все сие да будет к назиданию.
\vs 1Co 14:27 Если кто говорит на \bibemph{незнакомом} языке, \bibemph{говорите} двое, или много трое, и т\acc{о} порознь, а один изъясняй.
\vs 1Co 14:28 Если же не будет истолкователя, то молчи в церкви, а говори себе и Богу.
\vs 1Co 14:29 И пророки пусть говорят двое или трое, а прочие пусть рассуждают.
\vs 1Co 14:30 Если же другому из сидящих будет откровение, то первый молчи.
\vs 1Co 14:31 Ибо все один за другим можете пророчествовать, чтобы всем поучаться и всем получать утешение.
\vs 1Co 14:32 И духи пророческие послушны пророкам,
\vs 1Co 14:33 потому что Бог не есть \bibemph{Бог} неустройства, но мира. Т\acc{а}к \bibemph{бывает} во всех церквах у святых.
\vs 1Co 14:34 Жены ваши в церквах да молчат, ибо не позволено им говорить, а быть в подчинении, как и закон говорит.
\vs 1Co 14:35 Если же они хотят чему научиться, пусть спрашивают \bibemph{о том} дома у мужей своих; ибо неприлично жене говорить в церкви.
\vs 1Co 14:36 Разве от вас вышло слово Божие? Или до вас одних достигло?
\rsbpar\vs 1Co 14:37 Если кто почитает себя пророком или духовным, тот да разумеет, чт\acc{о} я пишу вам, ибо это заповеди Господни.
\vs 1Co 14:38 А кто не разумеет, пусть не разумеет.
\vs 1Co 14:39 Итак, братия, ревнуйте о том, чтобы пророчествовать, но не запрещайте говорить и языками;
\vs 1Co 14:40 только всё должно быть благопристойно и чинно.
\vs 1Co 15:1 Напоминаю вам, братия, Евангелие, которое я благовествовал вам, которое вы и приняли, в котором и утвердились,
\vs 1Co 15:2 которым и спасаетесь, если преподанное удерживаете так, как я благовествовал вам, если только не тщетно уверовали.
\vs 1Co 15:3 Ибо я первоначально преподал вам, что и \bibemph{сам} принял, \bibemph{то есть}, что Христос умер за грехи наши, по Писанию,
\vs 1Co 15:4 и что Он погребен был, и что воскрес в третий день, по Писанию,
\vs 1Co 15:5 и что явился Кифе, потом двенадцати;
\vs 1Co 15:6 потом явился более нежели пятистам братий в одно время, из которых б\acc{о}льшая часть доныне в живых, а некоторые и почили;
\vs 1Co 15:7 потом явился Иакову, также всем Апостолам;
\vs 1Co 15:8 а после всех явился и мне, как некоему извергу.
\vs 1Co 15:9 Ибо я наименьший из Апостолов, и недостоин называться Апостолом, потому что гнал церковь Божию.
\vs 1Co 15:10 Но благодатию Божиею есмь то, что есмь; и благодать Его во мне не была тщетна, но я более всех их потрудился: не я, впрочем, а благодать Божия, которая со мною.
\vs 1Co 15:11 Итак я ли, они ли, мы так проповедуем, и вы так уверовали.
\rsbpar\vs 1Co 15:12 Если же о Христе проповедуется, что Он воскрес из мертвых, то к\acc{а}к некоторые из вас говорят, что нет воскресения мертвых?
\vs 1Co 15:13 Если нет воскресения мертвых, то и Христос не воскрес;
\vs 1Co 15:14 а если Христос не воскрес, то и проповедь наша тщетна, тщетна и вера ваша.
\vs 1Co 15:15 Притом мы оказались бы и лжесвидетелями о Боге, потому что свидетельствовали бы о Боге, что Он воскресил Христа, Которого Он не воскрешал, если, \bibemph{то есть}, мертвые не воскресают;
\vs 1Co 15:16 ибо если мертвые не воскресают, то и Христос не воскрес.
\vs 1Co 15:17 А если Христос не воскрес, то вера ваша тщетна: вы еще во грехах ваших.
\vs 1Co 15:18 Поэтому и умершие во Христе погибли.
\vs 1Co 15:19 И если мы в этой только жизни надеемся на Христа, то мы несчастнее всех человеков.
\vs 1Co 15:20 Но Христос воскрес из мертвых, первенец из умерших.
\vs 1Co 15:21 Ибо, как смерть через человека, \bibemph{так} через человека и воскресение мертвых.
\vs 1Co 15:22 Как в Адаме все умирают, так во Христе все оживут,
\vs 1Co 15:23 каждый в своем порядке: первенец Христос, потом Христовы, в пришествие Его.
\vs 1Co 15:24 А затем конец, когда Он предаст Царство Богу и Отцу, когда упразднит всякое начальство и всякую власть и силу.
\vs 1Co 15:25 Ибо Ему надлежит царствовать, доколе низложит всех врагов под ноги Свои.
\vs 1Co 15:26 Последний же враг истребится~--- смерть,
\vs 1Co 15:27 потому что все покорил под ноги Его. Когда же сказано, что \bibemph{Ему} все покорено, то ясно, что кроме Того, Который покорил Ему все.
\vs 1Co 15:28 Когда же все покорит Ему, тогда и Сам Сын покорится Покорившему все Ему, да будет Бог все во всем.
\vs 1Co 15:29 Иначе, что делают крестящиеся для мертвых? Если мертвые совсем не воскресают, то для чего и крестятся для мертвых?
\vs 1Co 15:30 Для чего и мы ежечасно подвергаемся бедствиям?
\vs 1Co 15:31 Я каждый день умираю: свидетельствуюсь в том похвалою вашею, братия, которую я имею во Христе Иисусе, Господе нашем.
\vs 1Co 15:32 По \bibemph{рассуждению} человеческому, когда я боролся со зверями в Ефесе, какая мне польза, если мертвые не воскресают? Станем есть и пить, ибо завтра умрем!
\vs 1Co 15:33 Не обманывайтесь: худые сообщества развращают добрые нравы.
\vs 1Co 15:34 Отрезвитесь, как должно, и не грешите; ибо, к стыду вашему скажу, некоторые из вас не знают Бога.
\rsbpar\vs 1Co 15:35 Но скажет кто-нибудь: как воскреснут мертвые? и в каком теле придут?
\vs 1Co 15:36 Безрассудный! то, что ты сеешь, не оживет, если не умрет.
\vs 1Co 15:37 И когда ты сеешь, то сеешь не тело будущее, а голое зерно, какое случится, пшеничное или другое какое;
\vs 1Co 15:38 но Бог дает ему тело, как хочет, и каждому семени свое тело.
\vs 1Co 15:39 Не всякая плоть такая же плоть; но иная плоть у человеков, иная плоть у скотов, иная у рыб, иная у птиц.
\vs 1Co 15:40 Есть тела небесные и тела земные; но иная слава небесных, иная земных.
\vs 1Co 15:41 Иная слава солнца, иная слава луны, иная звезд; и звезда от звезды разнится в славе.
\vs 1Co 15:42 Так и при воскресении мертвых: сеется в тлении, восстает в нетлении;
\vs 1Co 15:43 сеется в уничижении, восстает в славе; сеется в немощи, восстает в силе;
\vs 1Co 15:44 сеется тело душевное, восстает тело духовное. Есть тело душевное, есть тело и духовное.
\vs 1Co 15:45 Так и написано: первый человек Адам стал душею живущею; а последний Адам есть дух животворящий.
\vs 1Co 15:46 Но не духовное прежде, а душевное, потом духовное.
\vs 1Co 15:47 Первый человек~--- из земли, перстный; второй человек~--- Господь с неба.
\vs 1Co 15:48 Каков перстный, таковы и перстные; и каков небесный, таковы и небесные.
\vs 1Co 15:49 И как мы носили образ перстного, будем носить и образ небесного.
\rsbpar\vs 1Co 15:50 Но то скажу \bibemph{вам}, братия, что плоть и кровь не могут наследовать Царствия Божия, и тление не наследует нетления.
\vs 1Co 15:51 Говорю вам тайну: не все мы умрем, но все изменимся
\vs 1Co 15:52 вдруг, во мгновение ока, при последней трубе; ибо вострубит, и мертвые воскреснут нетленными, а мы изменимся.
\vs 1Co 15:53 Ибо тленному сему надлежит облечься в нетление, и смертному сему облечься в бессмертие.
\vs 1Co 15:54 Когда же тленное сие облечется в нетление и смертное сие облечется в бессмертие, тогда сбудется слово написанное: поглощена смерть победою.
\vs 1Co 15:55 Смерть! где твое жало? ад! где твоя победа?
\vs 1Co 15:56 Жало же смерти~--- грех; а сила греха~--- закон.
\vs 1Co 15:57 Благодарение Богу, даровавшему нам победу Господом нашим Иисусом Христом!
\vs 1Co 15:58 Итак, братия мои возлюбленные, будьте тверды, непоколебимы, всегда преуспевайте в деле Господнем, зная, что труд ваш не тщетен пред Господом.
\vs 1Co 16:1 При сборе же для святых поступайте так, как я установил в церквах Галатийских.
\vs 1Co 16:2 В первый день недели каждый из вас пусть отлагает у себя и сберегает, сколько позволит ему состояние, чтобы не делать сборов, когда я приду.
\vs 1Co 16:3 Когда же приду, то, которых вы изберете, тех отправлю с письмами, для доставления вашего подаяния в Иерусалим.
\vs 1Co 16:4 А если прилично будет и мне отправиться, то они со мной пойдут.
\rsbpar\vs 1Co 16:5 Я приду к вам, когда пройду Македонию; ибо я иду через Македонию.
\vs 1Co 16:6 У вас же, может быть, поживу, или и перезимую, чтобы вы меня проводили, куда пойду.
\vs 1Co 16:7 Ибо я не хочу видеться с вами теперь мимоходом, а надеюсь пробыть у вас несколько времени, если Господь позволит.
\vs 1Co 16:8 В Ефесе же я пробуду до Пятидесятницы,
\vs 1Co 16:9 ибо для меня отверста великая и широкая дверь, и противников много.
\rsbpar\vs 1Co 16:10 Если же придет к вам Тимофей, смотр\acc{и}те, чтобы он был у вас безопасен; ибо он делает дело Господне, как и я.
\vs 1Co 16:11 Посему никто не пренебрегай его, но провод\acc{и}те его с миром, чтобы он пришел ко мне, ибо я жду его с братиями.
\vs 1Co 16:12 А что до брата Аполлоса, я очень просил его, чтобы он с братиями пошел к вам; но он никак не хотел идти ныне, а придет, когда ему будет удобно.
\rsbpar\vs 1Co 16:13 Бодрствуйте, стойте в вере, будьте мужественны, тверды.
\vs 1Co 16:14 Все у вас да будет с любовью.
\rsbpar\vs 1Co 16:15 Прошу вас, братия (вы знаете семейство Стефаново, что оно есть начаток Ахаии и что они посвятили себя на служение святым),
\vs 1Co 16:16 будьте и вы почтительны к таковым и ко всякому содействующему и трудящемуся.
\vs 1Co 16:17 Я рад прибытию Стефана, Фортуната и Ахаика: они восполнили для меня отсутствие ваше,
\vs 1Co 16:18 ибо они мой и ваш дух успокоили. Почитайте таковых.
\rsbpar\vs 1Co 16:19 Приветствуют вас церкви Асийские; приветствуют вас усердно в Господе Акила и Прискилла с домашнею их церковью.
\vs 1Co 16:20 Приветствуют вас все братия. Приветствуйте друг друга святым целованием.
\rsbpar\vs 1Co 16:21 Мое, Павлово, приветствие собственноручно.
\vs 1Co 16:22 Кто не любит Господа Иисуса Христа, анафема, мар\acc{а}н-аф\acc{а}\fns{Да будет отлучен до пришествия Господа.}.
\vs 1Co 16:23 Благодать Господа нашего Иисуса Христа с вами,
\vs 1Co 16:24 и любовь моя со всеми вами во Христе Иисусе. Аминь.

\bibbookdescr{2Co}{
  inline={Второе Послание\\к Коринфянам\\\LARGE Святого Апостола Павла},
  toc={2-е Коринфянам},
  bookmark={2-е Коринфянам},
  header={2-е Коринфянам},
  %headerleft={},
  %headerright={},
  abbr={2~Кор}
}
\vs 2Co 1:1 Павел, волею Божиею Апостол Иисуса Христа, и Тимофей брат, церкви Божией, находящейся в Коринфе, со всеми святыми по всей Ахаии:
\vs 2Co 1:2 благодать вам и мир от Бога Отца нашего и Господа Иисуса Христа.
\rsbpar\vs 2Co 1:3 Благословен Бог и Отец Господа нашего Иисуса Христа, Отец милосердия и Бог всякого утешения,
\vs 2Co 1:4 утешающий нас во всякой скорби нашей, чтобы и мы могли утешать находящихся во всякой скорби тем утешением, которым Бог утешает нас самих!
\vs 2Co 1:5 Ибо по мере, как умножаются в нас страдания Христовы, умножается Христом и утешение наше.
\vs 2Co 1:6 Скорбим ли мы, \bibemph{скорбим} для вашего утешения и спасения, которое совершается перенесением тех же страданий, какие и мы терпим.
\vs 2Co 1:7 И надежда наша о вас тверда. Утешаемся ли, \bibemph{утешаемся} для вашего утешения и спасения, зная, что вы участвуете как в страданиях наших, так и в утешении.
\rsbpar\vs 2Co 1:8 Ибо мы не хотим оставить вас, братия, в неведении о скорби нашей, бывшей с нами в Асии, потому что мы отягчены были чрезмерно и сверх силы, так что не надеялись остаться в живых.
\vs 2Co 1:9 Но сами в себе имели приговор к смерти, для того, чтобы надеяться не на самих себя, но на Бога, воскрешающего мертвых,
\vs 2Co 1:10 Который и избавил нас от столь \bibemph{близкой} смерти, и избавляет, и на Которого надеемся, что и еще избавит,
\vs 2Co 1:11 при содействии и вашей молитвы за нас, дабы за дарованное нам, по ходатайству многих, многие возблагодарили за нас.
\rsbpar\vs 2Co 1:12 Ибо похвала наша сия есть свидетельство совести нашей, что мы в простоте и богоугодной искренности, не по плотской мудрости, но по благодати Божией, жили в мире, особенно же у вас.
\vs 2Co 1:13 И мы пишем вам не иное, как то, что вы читаете или разумеете, и что, как надеюсь, до конца уразумеете,
\vs 2Co 1:14 так как вы отчасти и уразумели уже, что мы будем вашею похвалою, равно и вы нашею, в день Господа нашего Иисуса Христа.
\vs 2Co 1:15 И в этой уверенности я намеревался прийти к вам ранее, чтобы вы вторично получили благодать,
\vs 2Co 1:16 и через вас пройти в Македонию, из Македонии же опять прийти к вам; а вы проводили бы меня в Иудею.
\vs 2Co 1:17 Имея такое намерение, легкомысленно ли я поступил? Или, чт\acc{о} я предпринимаю, по плоти предпринимаю, так что у меня то <<да, да>>, то <<нет, нет>>?
\vs 2Co 1:18 Верен Бог, что слово наше к вам не было то <<да>>, то <<нет>>.
\vs 2Co 1:19 Ибо Сын Божий, Иисус Христос, проповеданный у вас нами, мною и Силуаном и Тимофеем, не был <<да>> и <<нет>>; но в Нем было <<да>>,~---
\vs 2Co 1:20 ибо все обетования Божии в Нем <<да>> и в Нем <<аминь>>,~--- в славу Божию, через нас.
\vs 2Co 1:21 Утверждающий же нас с вами во Христе и помазавший нас \bibemph{есть} Бог,
\vs 2Co 1:22 Который и запечатлел нас и дал залог Духа в сердца наши.
\rsbpar\vs 2Co 1:23 Бога призываю во свидетели на душу мою, что, щадя вас, я доселе не приходил в Коринф,
\vs 2Co 1:24 не потому, будто мы берем власть над верою вашею; но мы споспешествуем радости вашей: ибо верою вы тверды.
\vs 2Co 2:1 Итак я рассудил сам в себе не приходить к вам опять с огорчением.
\vs 2Co 2:2 Ибо если я огорчаю вас, то кто обрадует меня, как не тот, кто огорчен мною?
\vs 2Co 2:3 Это самое и писал я вам, дабы, придя, не иметь огорчения от тех, о которых мне надлежало радоваться: ибо я во всех вас уверен, что моя радость есть \bibemph{радость} и для всех вас.
\vs 2Co 2:4 От великой скорби и стесненного сердца я писал вам со многими слезами, не для того, чтобы огорчить вас, но чтобы вы познали любовь, какую я в избытке имею к вам.
\vs 2Co 2:5 Если же кто огорчил, то не меня огорчил, но частью,~--- чтобы не сказать много,~--- и всех вас.
\vs 2Co 2:6 Для такого довольно сего наказания от многих,
\vs 2Co 2:7 так что вам лучше уже простить его и утешить, дабы он не был поглощен чрезмерною печалью.
\vs 2Co 2:8 И потому прошу вас оказать ему любовь.
\vs 2Co 2:9 Ибо я для того и писал, чтобы узнать на опыте, во всем ли вы послушны.
\vs 2Co 2:10 А кого вы в чем прощаете, того и я; ибо и я, если в чем простил кого, простил для вас от лица Христова,
\vs 2Co 2:11 чтобы не сделал нам ущерба сатана, ибо нам не безызвестны его умыслы.
\rsbpar\vs 2Co 2:12 Придя в Троаду для благовествования о Христе, хотя мне и отверста была дверь Господом,
\vs 2Co 2:13 я не имел покоя духу моему, потому что не нашел \bibemph{там} брата моего Тита; но, простившись с ними, я пошел в Македонию.
\vs 2Co 2:14 Но благодарение Богу, Который всегда дает нам торжествовать во Христе и благоухание познания о Себе распространяет нами во всяком месте.
\vs 2Co 2:15 Ибо мы Христово благоухание Богу в спасаемых и в погибающих:
\vs 2Co 2:16 для одних запах смертоносный на смерть, а для других запах живительный на жизнь. И кто способен к сему?
\vs 2Co 2:17 Ибо мы не повреждаем слова Божия, как многие, но проповедуем искренно, как от Бога, пред Богом, во Христе.
\vs 2Co 3:1 Неужели нам снова знакомиться с вами? Неужели нужны для нас, как для некоторых, одобрительные письма к вам или от вас?
\vs 2Co 3:2 Вы~--- наше письмо, написанное в сердцах наших, узнаваемое и читаемое всеми человеками;
\vs 2Co 3:3 вы показываете собою, что вы~--- письмо Христово, через служение наше написанное не чернилами, но Духом Бога живаго, не на скрижалях каменных, но на плотяных скрижалях сердца.
\vs 2Co 3:4 Такую уверенность мы имеем в Боге через Христа,
\vs 2Co 3:5 не потому, чтобы мы сами способны были помыслить чт\acc{о} от себя, как бы от себя, но способность наша от Бога.
\vs 2Co 3:6 Он дал нам способность быть служителями Нового Завета, не буквы, но духа, потому что буква убивает, а дух животворит.
\vs 2Co 3:7 Если же служение смертоносным буквам, начертанное на камнях, было так славно, что сыны Израилевы не могли смотреть на лице Моисеево по причине славы лица его преходящей,~---
\vs 2Co 3:8 то не гораздо ли более должно быть славно служение духа?
\vs 2Co 3:9 Ибо если служение осуждения славно, то тем паче изобилует славою служение оправдания.
\vs 2Co 3:10 То прославленное даже не оказывается славным с сей стороны, по причине преимущественной славы \bibemph{последующего}.
\vs 2Co 3:11 Ибо, если преходящее славно, тем более славно пребывающее.
\vs 2Co 3:12 Имея такую надежду, мы действуем с великим дерзновением,
\vs 2Co 3:13 а не так, как Моисей, \bibemph{который} полагал покрывало на лице свое, чтобы сыны Израилевы не взирали на конец преходящего.
\vs 2Co 3:14 Но умы их ослеплены: ибо то же самое покрывало доныне остается неснятым при чтении Ветхого Завета, потому что оно снимается Христом.
\vs 2Co 3:15 Доныне, когда они читают Моисея, покрывало лежит на сердце их;
\vs 2Co 3:16 но когда обращаются к Господу, тогда это покрывало снимается.
\vs 2Co 3:17 Господь есть Дух; а где Дух Господень, там свобода.
\vs 2Co 3:18 Мы же все открытым лицем, как в зеркале, взирая на славу Господню, преображаемся в тот же образ от славы в славу, как от Господня Духа.
\vs 2Co 4:1 Посему, имея по милости \bibemph{Божией} такое служение, мы не унываем;
\vs 2Co 4:2 но, отвергнув скрытные постыдные \bibemph{дела}, не прибегая к хитрости и не искажая слова Божия, а открывая истину, представляем себя совести всякого человека пред Богом.
\vs 2Co 4:3 Если же и закрыто благовествование наше, то закрыто для погибающих,
\vs 2Co 4:4 для неверующих, у которых бог века сего ослепил умы, чтобы для них не воссиял свет благовествования о славе Христа, Который есть образ Бога невидимого.
\vs 2Co 4:5 Ибо мы не себя проповедуем, но Христа Иисуса, Господа; а мы~--- рабы ваши для Иисуса,
\vs 2Co 4:6 потому что Бог, повелевший из тьмы воссиять свету, озарил наши сердца, дабы просветить \bibemph{нас} познанием славы Божией в лице Иисуса Христа.
\rsbpar\vs 2Co 4:7 Но сокровище сие мы носим в глиняных сосудах, чтобы преизбыточная сила была \bibemph{приписываема} Богу, а не нам.
\vs 2Co 4:8 Мы отовсюду притесняемы, но не стеснены; мы в отчаянных обстоятельствах, но не отчаиваемся;
\vs 2Co 4:9 мы гонимы, но не оставлены; низлагаемы, но не погибаем.
\vs 2Co 4:10 Всегда носим в теле мертвость Господа Иисуса, чтобы и жизнь Иисусова открылась в теле нашем.
\vs 2Co 4:11 Ибо мы живые непрестанно предаемся на смерть ради Иисуса, чтобы и жизнь Иисусова открылась в смертной плоти нашей,
\vs 2Co 4:12 так что смерть действует в нас, а жизнь в вас.
\vs 2Co 4:13 Но, имея тот же дух веры, как написано: я веровал и потому говорил, и мы веруем, потому и говорим,
\vs 2Co 4:14 зная, что Воскресивший Господа Иисуса воскресит через Иисуса и нас и поставит перед \bibemph{Собою} с вами.
\vs 2Co 4:15 Ибо всё для вас, дабы обилие благодати тем б\acc{о}льшую во многих произвело благодарность во славу Божию.
\vs 2Co 4:16 Посему мы не унываем; но если внешний наш человек и тлеет, то внутренний со дня на день обновляется.
\vs 2Co 4:17 Ибо кратковременное легкое страдание наше производит в безмерном преизбытке вечную славу,
\vs 2Co 4:18 когда мы смотрим не на видимое, но на невидимое: ибо видимое временно, а невидимое вечно.
\vs 2Co 5:1 Ибо знаем, что, когда земной наш дом, эта хижина, разрушится, мы имеем от Бога жилище на небесах, дом нерукотворенный, вечный.
\vs 2Co 5:2 Оттого мы и воздыхаем, желая облечься в небесное наше жилище;
\vs 2Co 5:3 только бы нам и одетым не оказаться нагими.
\vs 2Co 5:4 Ибо мы, находясь в этой хижине, воздыхаем под бременем, потому что не хотим совлечься, но облечься, чтобы смертное поглощено было жизнью.
\vs 2Co 5:5 На сие самое и создал нас Бог и дал нам залог Духа.
\vs 2Co 5:6 Итак мы всегда благодушествуем; и как знаем, что, водворяясь в теле, мы устранены от Господа,~---
\vs 2Co 5:7 ибо мы ходим верою, а не в\acc{и}дением,~---
\vs 2Co 5:8 то мы благодушествуем и желаем лучше выйти из тела и водвориться у Господа.
\vs 2Co 5:9 И потому ревностно стараемся, водворяясь ли, выходя ли, быть Ему угодными;
\vs 2Co 5:10 ибо всем нам должно явиться пред судилище Христово, чтобы каждому получить \bibemph{соответственно тому}, чт\acc{о} он делал, живя в теле, доброе или худое.
\rsbpar\vs 2Co 5:11 Итак, зная страх Господень, мы вразумляем людей, Богу же мы открыты; надеюсь, что открыты и вашим совестям.
\vs 2Co 5:12 Не снова представляем себя вам, но даем вам повод хвалиться нами, дабы имели вы \bibemph{чт\acc{о} сказать} тем, которые хвалятся лицем, а не сердцем.
\vs 2Co 5:13 Если мы выходим из себя, то для Бога; если же скромны, то для вас.
\vs 2Co 5:14 Ибо любовь Христова объемлет нас, рассуждающих так: если один умер за всех, то все умерли.
\vs 2Co 5:15 А Христос за всех умер, чтобы живущие уже не для себя жили, но для умершего за них и воскресшего.
\vs 2Co 5:16 Потому отныне мы никого не знаем по плоти; если же и знали Христа по плоти, то ныне уже не знаем.
\vs 2Co 5:17 Итак, кто во Христе, \bibemph{тот} новая тварь; древнее прошло, теперь все новое.
\vs 2Co 5:18 Все же от Бога, Иисусом Христом примирившего нас с Собою и давшего нам служение примирения,
\vs 2Co 5:19 потому что Бог во Христе примирил с Собою мир, не вменяя \bibemph{людям} преступлений их, и дал нам слово примирения.
\vs 2Co 5:20 Итак мы~--- посланники от имени Христова, и как бы Сам Бог увещевает через нас; от имени Христова просим: примиритесь с Богом.
\vs 2Co 5:21 Ибо не знавшего греха Он сделал для нас \bibemph{жертвою за} грех, чтобы мы в Нем сделались праведными пред Богом.
\vs 2Co 6:1 Мы же, как споспешники, умоляем вас, чтобы благодать Божия не тщетно была принята вами.
\vs 2Co 6:2 Ибо сказано: во время благоприятное Я услышал тебя и в день спасения помог тебе. Вот, теперь время благоприятное, вот, теперь день спасения.
\vs 2Co 6:3 Мы никому ни в чем не полагаем претыкания, чтобы не было порицаемо служение,
\vs 2Co 6:4 но во всем являем себя, как служители Божии, в великом терпении, в бедствиях, в нуждах, в тесных обстоятельствах,
\vs 2Co 6:5 под ударами, в темницах, в изгнаниях, в трудах, в бдениях, в постах,
\vs 2Co 6:6 в чистоте, в благоразумии, в великодушии, в благости, в Духе Святом, в нелицемерной любви,
\vs 2Co 6:7 в слове истины, в силе Божией, с оружием правды в правой и левой руке,
\vs 2Co 6:8 в чести и бесчестии, при порицаниях и похвалах: нас почитают обманщиками, но мы верны;
\vs 2Co 6:9 мы неизвестны, но нас узнают; нас почитают умершими, но вот, мы живы; нас наказывают, но мы не умираем;
\vs 2Co 6:10 нас огорчают, а мы всегда радуемся; мы нищи, но многих обогащаем; мы ничего не имеем, но всем обладаем.
\rsbpar\vs 2Co 6:11 Уста наши отверсты к вам, Коринфяне, сердце наше расширено.
\vs 2Co 6:12 Вам не тесно в нас; но в сердцах ваших тесно.
\vs 2Co 6:13 В равное возмездие,~--- говорю, как детям,~--- распространитесь и вы.
\rsbpar\vs 2Co 6:14 Не преклоняйтесь под чужое ярмо с неверными, ибо какое общение праведности с беззаконием? Что общего у света с тьмою?
\vs 2Co 6:15 Какое согласие между Христом и Велиаром? Или какое соучастие верного с неверным?
\vs 2Co 6:16 Какая совместность храма Божия с идолами? Ибо вы храм Бога живаго, как сказал Бог: вселюсь в них и буду ходить \bibemph{в них}; и буду их Богом, и они будут Моим народом.
\vs 2Co 6:17 И потому выйдите из среды их и отделитесь, говорит Господь, и не прикасайтесь к нечистому; и Я прииму вас.
\vs 2Co 6:18 И буду вам Отцем, и вы будете Моими сынами и дщерями, говорит Господь Вседержитель.
\vs 2Co 7:1 Итак, возлюбленные, имея такие обетования, очистим себя от всякой скверны плоти и духа, совершая святыню в страхе Божием.
\rsbpar\vs 2Co 7:2 Вместите нас. Мы никого не обидели, никому не повредили, ни от кого не искали корысти.
\vs 2Co 7:3 Не в осуждение говорю; ибо я прежде сказал, что вы в сердцах наших, так чтобы вместе и умереть и жить.
\vs 2Co 7:4 Я много надеюсь на вас, много хвалюсь вами; я исполнен утешением, преизобилую радостью, при всей скорби нашей.
\vs 2Co 7:5 Ибо, когда пришли мы в Македонию, плоть наша не имела никакого покоя, но мы были стеснены отовсюду: отвне~--- нападения, внутри~--- страхи.
\vs 2Co 7:6 Но Бог, утешающий смиренных, утешил нас прибытием Тита,
\vs 2Co 7:7 и не только прибытием его, но и утешением, которым он утешался о вас, пересказывая нам о вашем усердии, о вашем плаче, о вашей ревности по мне, так что я еще более обрадовался.
\vs 2Co 7:8 Посему, если я опечалил вас посланием, не жалею, хотя и пожалел было; ибо вижу, что послание т\acc{о} опечалило вас, впрочем на время.
\vs 2Co 7:9 Теперь я радуюсь не потому, что вы опечалились, но что вы опечалились к покаянию; ибо опечалились ради Бога, так что нисколько не понесли от нас вреда.
\vs 2Co 7:10 Ибо печаль ради Бога производит неизменное покаяние ко спасению, а печаль мирская производит смерть.
\vs 2Co 7:11 Ибо то самое, что вы опечалились ради Бога, смотр\acc{и}те, какое произвело в вас усердие, какие извинения, какое негодование \bibemph{на виновного}, какой страх, какое желание, какую ревность, какое взыскание! По всему вы показали себя чистыми в этом деле.
\vs 2Co 7:12 Итак, если я писал к вам, то не ради оскорбителя и не ради оскорбленного, но чтобы вам открылось попечение наше о вас пред Богом.
\vs 2Co 7:13 Посему мы утешились утешением вашим; а еще более обрадованы мы радостью Тита, что вы все успокоили дух его.
\vs 2Co 7:14 Итак я не остался в стыде, если чем-либо о вас похвалился перед ним, но как вам мы говорили все истину, так и перед Титом похвала наша оказалась истинною;
\vs 2Co 7:15 и сердце его весьма расположено к вам, при воспоминании о послушании всех вас, как вы приняли его со страхом и трепетом.
\vs 2Co 7:16 Итак радуюсь, что во всем могу положиться на вас.
\vs 2Co 8:1 Уведомляем вас, братия, о благодати Божией, данной церквам Македонским,
\vs 2Co 8:2 ибо они среди великого испытания скорбями преизобилуют радостью; и глубокая нищета их преизбыточествует в богатстве их радушия.
\vs 2Co 8:3 Ибо они доброхотны по силам и сверх сил~--- я свидетель:
\vs 2Co 8:4 они весьма убедительно просили нас принять дар и участие \bibemph{их} в служении святым;
\vs 2Co 8:5 и не только то, чего мы надеялись, но они отдали самих себя, во-первых, Господу, \bibemph{потом} и нам по воле Божией;
\vs 2Co 8:6 поэтому мы просили Тита, чтобы он, как начал, так и окончил у вас и это доброе дело.
\rsbpar\vs 2Co 8:7 А к\acc{а}к вы изобилуете всем: верою и словом, и познанием, и всяким усердием, и любовью вашею к нам,~--- т\acc{а}к изобилуйте и сею добродетелью.
\vs 2Co 8:8 Говорю это не в виде повеления, но усердием других испытываю искренность и вашей любви.
\vs 2Co 8:9 Ибо вы знаете благодать Господа нашего Иисуса Христа, что Он, будучи богат, обнищал ради вас, дабы вы обогатились Его нищетою.
\vs 2Co 8:10 Я даю на это совет: ибо это полезно вам, которые не только начали делать сие, но и желали того еще с прошедшего года.
\vs 2Co 8:11 Совершите же теперь самое дело, дабы, чего усердно желали, то и исполнено было по достатку.
\vs 2Co 8:12 Ибо если есть усердие, то оно принимается смотря по тому, кто что имеет, а не по тому, чего не имеет.
\vs 2Co 8:13 Не \bibemph{требуется}, чтобы другим \bibemph{было} облегчение, а вам тяжесть, но чтобы была равномерность.
\vs 2Co 8:14 Ныне ваш избыток в \bibemph{восполнение} их недостатка; а после их избыток в \bibemph{восполнение} вашего недостатка, чтобы была равномерность,
\vs 2Co 8:15 как написано: кто собрал много, не имел лишнего; и кто мало, не имел недостатка.
\rsbpar\vs 2Co 8:16 Благодарение Богу, вложившему в сердце Титово такое усердие к вам.
\vs 2Co 8:17 Ибо, хотя и я просил его, впрочем он, будучи очень усерден, пошел к вам добровольно.
\vs 2Co 8:18 С ним послали мы также брата, во всех церквах похваляемого за благовествование,
\vs 2Co 8:19 и притом избранного от церквей сопутствовать нам для сего благотворения, которому мы служим во славу Самого Господа и \bibemph{в соответствие} вашему усердию,
\vs 2Co 8:20 остерегаясь, чтобы нам не подвергнуться от кого нареканию при таком обилии приношений, вверяемых нашему служению;
\vs 2Co 8:21 ибо мы стараемся о добром не только пред Господом, но и пред людьми.
\vs 2Co 8:22 Мы послали с ними и брата нашего, которого усердие много раз испытали во многом и который ныне еще усерднее по великой уверенности в вас.
\vs 2Co 8:23 Что касается до Тита, это~--- мой товарищ и сотрудник у вас; а что до братьев наших, это~--- посланники церквей, слава Христова.
\vs 2Co 8:24 Итак перед лицем церквей дайте им доказательство любви вашей и того, что мы \bibemph{справедливо} хвалимся вами.
\vs 2Co 9:1 Для меня впрочем излишне писать вам о вспоможении святым,
\vs 2Co 9:2 ибо я знаю усердие ваше и хвалюсь вами перед Македонянами, что Ахаия приготовлена еще с прошедшего года; и ревность ваша поощрила многих.
\vs 2Co 9:3 Братьев же послал я для того, чтобы похвала моя о вас не оказалась тщетною в сем случае, но чтобы вы, как я говорил, были приготовлены,
\vs 2Co 9:4 \bibemph{и} чтобы, когда придут со мною Македоняне и найдут вас неготовыми, не остались в стыде мы,~--- не говорю <<вы>>,~--- похвалившись с такою уверенностью.
\vs 2Co 9:5 Посему я почел за нужное упросить братьев, чтобы они наперед пошли к вам и предварительно озаботились, дабы возвещенное уже благословение ваше было готово, как благословение, а не как побор.
\rsbpar\vs 2Co 9:6 При сем скажу: кто сеет скупо, тот скупо и пожнет; а кто сеет щедро, тот щедро и пожнет.
\vs 2Co 9:7 Каждый \bibemph{уделяй} по расположению сердца, не с огорчением и не с принуждением; ибо доброхотно дающего любит Бог.
\vs 2Co 9:8 Бог же силен обогатить вас всякою благодатью, чтобы вы, всегда и во всем имея всякое довольство, были богаты на всякое доброе дело,
\vs 2Co 9:9 как написано: расточил, раздал нищим; правда его пребывает в век.
\vs 2Co 9:10 Дающий же семя сеющему и хлеб в пищу подаст обилие посеянному вами и умножит плоды правды вашей,
\vs 2Co 9:11 так чтобы вы всем богаты были на всякую щедрость, которая через нас производит благодарение Богу.
\vs 2Co 9:12 Ибо дело служения сего не только восполняет скудость святых, но и производит во многих обильные благодарения Богу;
\vs 2Co 9:13 ибо, видя опыт сего служения, они прославляют Бога за покорность исповедуемому вами Евангелию Христову и за искреннее общение с ними и со всеми,
\vs 2Co 9:14 молясь за вас, по расположению к вам, за преизбыточествующую в вас благодать Божию.
\vs 2Co 9:15 Благодарение Богу за неизреченный дар Его!
\vs 2Co 10:1 Я же, Павел, который лично между вами скромен, а заочно против вас отважен, убеждаю вас кротостью и снисхождением Христовым.
\vs 2Co 10:2 Прошу, чтобы мне по пришествии моем не прибегать к той твердой смелости, которую думаю употребить против некоторых, помышляющих о нас, что мы поступаем по плоти.
\vs 2Co 10:3 Ибо мы, ходя во плоти, не по плоти воинствуем.
\vs 2Co 10:4 Оружия воинствования нашего не плотские, но сильные Богом на разрушение твердынь: \bibemph{ими} ниспровергаем замыслы
\vs 2Co 10:5 и всякое превозношение, восстающее против познания Божия, и пленяем всякое помышление в послушание Христу,
\vs 2Co 10:6 и готовы наказать всякое непослушание, когда ваше послушание исполнится.
\vs 2Co 10:7 На личность ли см\acc{о}трите? Кто уверен в себе, что он Христов, тот сам по себе суди, что, как он Христов, так и мы Христовы.
\vs 2Co 10:8 Ибо если бы я и более стал хвалиться нашею властью, которую Господь дал нам к созиданию, а не к расстройству вашему, то не остался бы в стыде.
\vs 2Co 10:9 Впрочем, да не покажется, что я устрашаю вас \bibemph{только} посланиями.
\vs 2Co 10:10 Так как \bibemph{некто} говорит: в посланиях он строг и силен, а в личном присутствии слаб, и речь \bibemph{его} незначительна,~---
\vs 2Co 10:11 такой пусть знает, что, каковы мы на словах в посланиях заочно, таковы и на деле лично.
\vs 2Co 10:12 Ибо мы не смеем сопоставлять или сравнивать себя с теми, которые сами себя выставляют: они измеряют себя самими собою и сравнивают себя с собою неразумно.
\vs 2Co 10:13 А мы не без меры хвалиться будем, но по мере удела, какой назначил нам Бог в такую меру, чтобы достигнуть и до вас.
\vs 2Co 10:14 Ибо мы не напрягаем себя, как не достигшие до вас, потому что достигли и до вас благовествованием Христовым.
\vs 2Co 10:15 Мы не без меры хвалимся, не чужими трудами, но надеемся, с возрастанием веры вашей, с избытком увеличить в вас удел наш,
\vs 2Co 10:16 так чтобы и далее вас проповедовать Евангелие, а не хвалиться готовым в чужом уделе.
\vs 2Co 10:17 Хвалящийся хвались о Господе.
\vs 2Co 10:18 Ибо не тот достоин, кто сам себя хвалит, но кого хвалит Господь.
\vs 2Co 11:1 О, если бы вы несколько были снисходительны к моему неразумию! Но вы и снисходите ко мне.
\vs 2Co 11:2 Ибо я ревную о вас ревностью Божиею; потому что я обручил вас единому мужу, чтобы представить Христу чистою девою.
\vs 2Co 11:3 Но боюсь, чтобы, как змий хитростью своею прельстил Еву, так и ваши умы не повредились, \bibemph{уклонившись} от простоты во Христе.
\vs 2Co 11:4 Ибо если бы кто, придя, начал проповедовать другого Иисуса, которого мы не проповедовали, или если бы вы получили иного Духа, которого не получили, или иное благовестие, которого не принимали,~--- то вы были бы очень снисходительны \bibemph{к тому}.
\vs 2Co 11:5 Но я думаю, что у меня ни в чем нет недостатка против высших Апостолов:
\vs 2Co 11:6 хотя я и невежда в слове, но не в познании. Впрочем мы во всем совершенно известны вам.
\vs 2Co 11:7 Согрешил ли я тем, что унижал себя, чтобы возвысить вас, потому что безмездно проповедовал вам Евангелие Божие?
\vs 2Co 11:8 Другим церквам я причинял издержки, получая \bibemph{от них} содержание для служения вам; и, будучи у вас, хотя терпел недостаток, никому не докучал,
\vs 2Co 11:9 ибо недостаток мой восполнили братия, пришедшие из Македонии; да и во всем я старался и постараюсь не быть вам в тягость.
\vs 2Co 11:10 По истине Христовой во мне \bibemph{скажу}, что похвала сия не отнимется у меня в странах Ахаии.
\vs 2Co 11:11 Почему же \bibemph{так поступаю}? Потому ли, что не люблю вас? Богу известно! Но как поступаю, так и буду поступать,
\vs 2Co 11:12 чтобы не дать повода ищущим повода, дабы они, чем хвалятся, в том оказались \bibemph{такими же}, как и мы.
\vs 2Co 11:13 Ибо таковые лжеапостолы, лукавые делатели, принимают вид Апостолов Христовых.
\vs 2Co 11:14 И неудивительно: потому что сам сатана принимает вид Ангела света,
\vs 2Co 11:15 а потому не великое дело, если и служители его принимают вид служителей правды; но конец их будет по делам их.
\rsbpar\vs 2Co 11:16 Еще скажу: не почти кто-нибудь меня неразумным; а если не так, то примите меня, хотя как неразумного, чтобы и мне сколько-нибудь похвалиться.
\vs 2Co 11:17 Чт\acc{о} скажу, т\acc{о} скажу не в Господе, но как бы в неразумии при такой отважности на похвалу.
\vs 2Co 11:18 Как многие хвалятся по плоти, то и я буду хвалиться.
\vs 2Co 11:19 Ибо вы, люди разумные, охотно терпите неразумных:
\vs 2Co 11:20 вы терпите, когда кто вас порабощает, когда кто объедает, когда кто обирает, когда кто превозносится, когда кто бьет вас в лицо.
\vs 2Co 11:21 К стыду говорю, что \bibemph{на это} у нас недоставало сил. А если кто смеет \bibemph{хвалиться} чем-либо, то (скажу по неразумию) смею и я.
\vs 2Co 11:22 Они Евреи? и я. Израильтяне? и я. Семя Авраамово? и я.
\vs 2Co 11:23 Христовы служители? (в безумии говорю:) я больше. Я гораздо более \bibemph{был} в трудах, безмерно в ранах, более в темницах и многократно при смерти.
\vs 2Co 11:24 От Иудеев пять раз дано мне было по сорока \bibemph{ударов} без одного;
\vs 2Co 11:25 три раза меня били палками, однажды камнями побивали, три раза я терпел кораблекрушение, ночь и день пробыл во глубине \bibemph{морской};
\vs 2Co 11:26 много раз \bibemph{был} в путешествиях, в опасностях на реках, в опасностях от разбойников, в опасностях от единоплеменников, в опасностях от язычников, в опасностях в городе, в опасностях в пустыне, в опасностях на море, в опасностях между лжебратиями,
\vs 2Co 11:27 в труде и в изнурении, часто в бдении, в голоде и жажде, часто в посте, на стуже и в наготе.
\vs 2Co 11:28 Кроме посторонних \bibemph{приключений}, у меня ежедневно стечение \bibemph{людей}, забота о всех церквах.
\vs 2Co 11:29 Кто изнемогает, с кем бы и я не изнемогал? Кто соблазняется, за кого бы я не воспламенялся?
\vs 2Co 11:30 Если должно мне хвалиться, то буду хвалиться немощью моею.
\vs 2Co 11:31 Бог и Отец Господа нашего Иисуса Христа, благословенный во веки, знает, что я не лгу.
\vs 2Co 11:32 В Дамаске областной правитель царя Ареты стерег город Дамаск, чтобы схватить меня; и я в корзине был спущен из окна по стене и избежал его рук.
\vs 2Co 12:1 Не полезно хвалиться мне, ибо я приду к видениям и откровениям Господним.
\vs 2Co 12:2 Знаю человека во Христе, который назад тому четырнадцать лет (в теле ли~--- не знаю, вне ли тела~--- не знаю: Бог знает) восхищен был до третьего неба.
\vs 2Co 12:3 И знаю о таком человеке (\bibemph{только} не знаю~--- в теле, или вне тела: Бог знает),
\vs 2Co 12:4 что он был восхищен в рай и слышал неизреченные слова, которых человеку нельзя пересказать.
\vs 2Co 12:5 Таким \bibemph{человеком} могу хвалиться; собою же не похвалюсь, разве только немощами моими.
\vs 2Co 12:6 Впрочем, если захочу хвалиться, не буду неразумен, потому что скажу истину; но я удерживаюсь, чтобы кто не подумал о мне более, нежели сколько во мне видит или слышит от меня.
\vs 2Co 12:7 И чтобы я не превозносился чрезвычайностью откровений, дано мне жало в плоть, ангел сатаны, удручать меня, чтобы я не превозносился.
\vs 2Co 12:8 Трижды молил я Господа о том, чтобы удалил его от меня.
\vs 2Co 12:9 Но \bibemph{Господь} сказал мне: <<довольно для тебя благодати Моей, ибо сила Моя совершается в немощи>>. И потому я гораздо охотнее буду хвалиться своими немощами, чтобы обитала во мне сила Христова.
\vs 2Co 12:10 Посему я благодушествую в немощах, в обидах, в нуждах, в гонениях, в притеснениях за Христа, ибо, когда я немощен, тогда силен.
\rsbpar\vs 2Co 12:11 Я дошел до неразумия, хвалясь; вы меня \bibemph{к сему} принудили. Вам бы надлежало хвалить меня, ибо у меня ни в чем нет недостатка против высших Апостолов, хотя я и ничто.
\vs 2Co 12:12 Признаки Апостола оказались перед вами всяким терпением, знамениями, чудесами и силами.
\vs 2Co 12:13 Ибо чего у вас недостает перед прочими церквами, разве только того, что сам я не был вам в тягость? Простите мне такую вину.
\vs 2Co 12:14 Вот, в третий раз я готов идти к вам, и не буду отягощать вас, ибо я ищу не вашего, а вас. Не дети должны собирать имение для родителей, но родители для детей.
\vs 2Co 12:15 Я охотно буду издерживать \bibemph{свое} и истощать себя за души ваши, несмотря на то, что, чрезвычайно любя вас, я менее любим вами.
\vs 2Co 12:16 Положим, \bibemph{что} сам я не обременял вас, но, будучи хитр, лукавством брал с вас.
\vs 2Co 12:17 Но пользовался ли я \bibemph{чем} от вас через кого-нибудь из тех, кого посылал к вам?
\vs 2Co 12:18 Я упросил Тита и послал с ним одного из братьев: Тит воспользовался ли чем от вас? Не в одном ли духе мы действовали? Не одним ли путем ходили?
\rsbpar\vs 2Co 12:19 Не думаете ли еще, что мы \bibemph{только} оправдываемся перед вами? Мы говорим пред Богом, во Христе, и все это, возлюбленные, к вашему назиданию.
\vs 2Co 12:20 Ибо я опасаюсь, чтобы мне, по пришествии моем, не найти вас такими, какими не желаю, также чтобы и вам не найти меня таким, каким не желаете: чтобы \bibemph{не найти у вас} раздоров, зависти, гнева, ссор, клевет, ябед, гордости, беспорядков,
\vs 2Co 12:21 чтобы опять, когда приду, не уничижил меня у вас Бог мой и \bibemph{чтобы} не оплакивать мне многих, которые согрешили прежде и не покаялись в нечистоте, блудодеянии и непотребстве, какое делали.
\vs 2Co 13:1 В третий уже раз иду к вам. При устах двух или трех свидетелей будет твердо всякое слово.
\vs 2Co 13:2 Я предварял и предваряю, как бы находясь \bibemph{у вас} во второй раз, и теперь, отсутствуя, пишу прежде согрешившим и всем прочим, что, когда опять приду, не пощажу.
\vs 2Co 13:3 Вы ищете доказательства на то, Христос ли говорит во мне: Он не бессилен для вас, но силен в вас.
\vs 2Co 13:4 Ибо, хотя Он и распят в немощи, но жив силою Божиею; и мы также, \bibemph{хотя} немощны в Нем, но будем живы с Ним силою Божиею в вас.
\rsbpar\vs 2Co 13:5 Испытывайте самих себя, в вере ли вы; самих себя исследуйте. Или вы не знаете самих себя, что Иисус Христос в вас? Разве только вы не т\acc{о}, чем должны быть.
\vs 2Co 13:6 О нас же, надеюсь, узн\acc{а}ете, что мы т\acc{о}, чем быть должны.
\vs 2Co 13:7 Молим Бога, чтобы вы не делали никакого зла, не для того, чтобы нам показаться, чем должны быть; но чтобы вы делали добро, хотя бы мы казались и не тем, чем должны быть.
\vs 2Co 13:8 Ибо мы не сильны против истины, но сильны за истину.
\vs 2Co 13:9 Мы радуемся, когда мы немощны, а вы сильны; о сем-то и молимся, о вашем совершенстве.
\vs 2Co 13:10 Для того я и пишу сие в отсутствии, чтобы в присутствии не употребить строгости по власти, данной мне Господом к созиданию, а не к разорению.
\rsbpar\vs 2Co 13:11 Впрочем, братия, радуйтесь, усовершайтесь, утешайтесь, будьте единомысленны, мирны,~--- и Бог любви и мира будет с вами.
\vs 2Co 13:12 Приветствуйте друг друга лобзанием святым. Приветствуют вас все святые.
\rsbpar\vs 2Co 13:13 Благодать Господа нашего Иисуса Христа, и любовь Бога Отца, и общение Святаго Духа со всеми вами. Аминь.

\include{tex/Gal}
\include{tex/Eph}
\include{tex/Phi}
\include{tex/Col}\newbookpage
\bibbookdescr{1Th}{
  inline={Первое Послание\\к Фессалоникийцам\\\LARGE Святого Апостола Павла},
  toc={1-е Фессалоникийцам},
  bookmark={1-е Фессалоникийцам},
  header={1-е Фессалоникийцам},
  %headerleft={},
  %headerright={},
  abbr={1~Фес}
}
\vs 1Th 1:1 Павел и Силуан и Тимофей~--- церкви Фессалоникской в Боге Отце и Господе Иисусе Христе: благодать вам и мир от Бога Отца нашего и Господа Иисуса Христа.
\rsbpar\vs 1Th 1:2 Всегда благодарим Бога за всех вас, вспоминая о вас в молитвах наших,
\vs 1Th 1:3 непрестанно памятуя ваше дело веры и труд любви и терпение упования на Господа нашего Иисуса Христа пред Богом и Отцем нашим,
\vs 1Th 1:4 зная избрание ваше, возлюбленные Богом братия;
\vs 1Th 1:5 потому что наше благовествование у вас было не в слове только, но и в силе и во Святом Духе, и со многим удостоверением, как вы \bibemph{сами} знаете, каковы были мы для вас между вами.
\vs 1Th 1:6 И вы сделались подражателями нам и Господу, приняв слово при многих скорбях с радостью Духа Святаго,
\vs 1Th 1:7 так что вы стали образцом для всех верующих в Македонии и Ахаии.
\vs 1Th 1:8 Ибо от вас пронеслось слово Господне не только в Македонии и Ахаии, но и во всяком месте прошла \bibemph{слава} о вере вашей в Бога, так что нам ни о чем не нужно рассказывать.
\vs 1Th 1:9 Ибо сами они сказывают о нас, какой вход имели мы к вам, и как вы обратились к Богу от идолов, \bibemph{чтобы} служить Богу живому и истинному
\vs 1Th 1:10 и ожидать с небес Сына Его, Которого Он воскресил из мертвых, Иисуса, избавляющего нас от грядущего гнева.
\vs 1Th 2:1 Вы сами знаете, братия, о нашем входе к вам, что он был не бездейственный;
\vs 1Th 2:2 но, прежде пострадав и быв поруганы в Филиппах, как вы знаете, мы дерзнули в Боге нашем проповедать вам благовестие Божие с великим подвигом.
\vs 1Th 2:3 Ибо в учении нашем нет ни заблуждения, ни нечистых \bibemph{побуждений}, ни лукавства;
\vs 1Th 2:4 но, как Бог удостоил нас того, чтобы вверить \bibemph{нам} благовестие, так мы и говорим, угождая не человекам, но Богу, испытующему сердца наши.
\vs 1Th 2:5 Ибо никогда не было у нас перед вами ни слов ласкательства, как вы знаете, ни видов корысти: Бог свидетель!
\vs 1Th 2:6 Не ищем славы человеческой ни от вас, ни от других:
\vs 1Th 2:7 мы могли явиться с важностью, как Апостолы Христовы, но были тихи среди вас, подобно как кормилица нежно обходится с детьми своими.
\vs 1Th 2:8 Так мы, из усердия к вам, восхотели передать вам не только благовестие Божие, но и души наши, потому что вы стали нам любезны.
\vs 1Th 2:9 Ибо вы помните, братия, труд наш и изнурение: ночью и днем работая, чтобы не отяготить кого из вас, мы проповедовали у вас благовестие Божие.
\vs 1Th 2:10 Свидетели вы и Бог, как свято и праведно и безукоризненно поступали мы перед вами, верующими,
\vs 1Th 2:11 потому что вы знаете, как каждого из вас, как отец детей своих,
\vs 1Th 2:12 мы просили и убеждали и умоляли поступать достойно Бога, призвавшего вас в Свое Царство и славу.
\rsbpar\vs 1Th 2:13 Посему и мы непрестанно благодарим Бога, что, приняв от нас слышанное слово Божие, вы приняли не \bibemph{к\acc{а}к} слово человеческое, но \bibemph{как} слово Божие,~--- каково оно есть по истине,~--- которое и действует в вас, верующих.
\vs 1Th 2:14 Ибо вы, братия, сделались подражателями церквам Божиим во Христе Иисусе, находящимся в Иудее, потому что и вы то же претерпели от своих единоплеменников, что и те от Иудеев,
\vs 1Th 2:15 которые убили и Господа Иисуса и Его пророков, и нас изгнали, и Богу не угождают, и всем человекам противятся,
\vs 1Th 2:16 которые препятствуют нам говорить язычникам, чтобы спаслись, и через это всегда наполняют меру грехов своих; но приближается на них гнев до конца.
\rsbpar\vs 1Th 2:17 Мы же, братия, быв разлучены с вами на короткое время лицем, а не сердцем, тем с б\acc{о}льшим желанием старались увидеть лице ваше.
\vs 1Th 2:18 И потому мы, я Павел, и раз и два хотели прийти к вам, но воспрепятствовал нам сатана.
\vs 1Th 2:19 Ибо кто наша надежда, или радость, или венец похвалы? Не и вы ли пред Господом нашим Иисусом Христом в пришествие Его?
\vs 1Th 2:20 Ибо вы~--- слава наша и радость.
\vs 1Th 3:1 И потому, не терпя более, мы восхотели остаться в Афинах одни,
\vs 1Th 3:2 и послали Тимофея, брата нашего и служителя Божия и сотрудника нашего в благовествовании Христовом, чтобы утвердить вас и утешить в вере вашей,
\vs 1Th 3:3 чтобы никто не поколебался в скорбях сих: ибо вы сами знаете, что так нам суждено.
\vs 1Th 3:4 Ибо мы и тогда, как были у вас, предсказывали вам, что будем страдать, как и случилось, и вы знаете.
\vs 1Th 3:5 Посему и я, не терпя более, послал узнать о вере вашей, чтобы как не искусил вас искуситель и не сделался тщетным труд наш.
\vs 1Th 3:6 Теперь же, когда пришел к нам от вас Тимофей и принес нам добрую весть о вере и любви вашей, и что вы всегда имеете добрую память о нас, желая нас видеть, как и мы вас,
\vs 1Th 3:7 то мы, при всей скорби и нужде нашей, утешились вами, братия, ради вашей веры;
\vs 1Th 3:8 ибо теперь мы живы, когда вы стоите в Господе.
\vs 1Th 3:9 Какую благодарность можем мы воздать Богу за вас, за всю радость, которою радуемся о вас пред Богом нашим,
\vs 1Th 3:10 ночь и день всеусердно молясь о том, чтобы видеть лице ваше и дополнить, чего недоставало вере вашей?
\vs 1Th 3:11 Сам же Бог и Отец наш и Господь наш Иисус Христос да управит путь наш к вам.
\vs 1Th 3:12 А вас Господь да исполнит и преисполнит любовью друг к другу и ко всем, какою мы исполнены к вам,
\vs 1Th 3:13 чтобы утвердить сердца ваши непорочными во святыне пред Богом и Отцем нашим в пришествие Господа нашего Иисуса Христа со всеми святыми Его. Аминь.
\vs 1Th 4:1 За сим, братия, просим и умоляем вас Христом Иисусом, чтобы вы, приняв от нас, как должно вам поступать и угождать Богу, более в том преуспевали,
\vs 1Th 4:2 ибо вы знаете, какие мы дали вам заповеди от Господа Иисуса.
\vs 1Th 4:3 Ибо воля Божия есть освящение ваше, чтобы вы воздерживались от блуда;
\vs 1Th 4:4 чтобы каждый из вас умел соблюдать свой сосуд в святости и чести,
\vs 1Th 4:5 а не в страсти похотения, как и язычники, не знающие Бога;
\vs 1Th 4:6 чтобы вы ни в чем не поступали с братом своим противозаконно и корыстолюбиво: потому что Господь~--- мститель за все это, как и прежде мы говорили вам и свидетельствовали.
\vs 1Th 4:7 Ибо призвал нас Бог не к нечистоте, но к святости.
\vs 1Th 4:8 Итак непокорный непокорен не человеку, но Богу, Который и дал нам Духа Своего Святаго.
\rsbpar\vs 1Th 4:9 О братолюбии же нет нужды писать к вам; ибо вы сами научены Богом любить друг друга,
\vs 1Th 4:10 ибо вы так и поступаете со всеми братиями по всей Македонии. Умоляем же вас, братия, более преуспевать
\vs 1Th 4:11 и усердно стараться о том, чтобы жить тихо, делать свое \bibemph{дело} и работать своими собственными руками, как мы заповедовали вам;
\vs 1Th 4:12 чтобы вы поступали благоприлично перед внешними и ни в чем не нуждались.
\rsbpar\vs 1Th 4:13 Не хочу же оставить вас, братия, в неведении об умерших, дабы вы не скорбели, как прочие, не имеющие надежды.
\vs 1Th 4:14 Ибо, если мы веруем, что Иисус умер и воскрес, то и умерших в Иисусе Бог приведет с Ним.
\vs 1Th 4:15 Ибо сие говорим вам словом Господним, что мы живущие, оставшиеся до пришествия Господня, не предупредим умерших,
\vs 1Th 4:16 потому что Сам Господь при возвещении, при гласе Архангела и трубе Божией, сойдет с неба, и мертвые во Христе воскреснут прежде;
\vs 1Th 4:17 потом мы, оставшиеся в живых, вместе с ними восхищены будем на облаках в сретение Господу на воздухе, и так всегда с Господом будем.
\vs 1Th 4:18 Итак утешайте друг друга сими словами.
\vs 1Th 5:1 О временах же и сроках нет нужды писать к вам, братия,
\vs 1Th 5:2 ибо сами вы достоверно знаете, что день Господень так придет, как тать ночью.
\vs 1Th 5:3 Ибо, когда будут говорить: <<мир и безопасность>>, тогда внезапно постигнет их пагуба, подобно как мука родами \bibemph{постигает} имеющую во чреве, и не избегнут.
\vs 1Th 5:4 Но вы, братия, не во тьме, чтобы день застал вас, как тать.
\vs 1Th 5:5 Ибо все вы~--- сыны света и сыны дня: мы~--- не \bibemph{сыны} ночи, ни тьмы.
\vs 1Th 5:6 Итак, не будем спать, как и прочие, но будем бодрствовать и трезвиться.
\vs 1Th 5:7 Ибо спящие спят ночью, и упивающиеся упиваются ночью.
\vs 1Th 5:8 Мы же, будучи \bibemph{сынами} дня, да трезвимся, облекшись в броню веры и любви и в шлем надежды спасения,
\vs 1Th 5:9 потому что Бог определил нас не на гнев, но к получению спасения через Господа нашего Иисуса Христа,
\vs 1Th 5:10 умершего за нас, чтобы мы, бодрствуем ли, или спим, жили вместе с Ним.
\vs 1Th 5:11 Посему увещавайте друг друга и назидайте один другого, как вы и делаете.
\rsbpar\vs 1Th 5:12 Просим же вас, братия, уважать трудящихся у вас, и предстоятелей ваших в Господе, и вразумляющих вас,
\vs 1Th 5:13 и почитать их преимущественно с любовью за дело их; будьте в мире между собою.
\vs 1Th 5:14 Умоляем также вас, братия, вразумляйте бесчинных, утешайте малодушных, поддерживайте слабых, будьте долготерпеливы ко всем.
\vs 1Th 5:15 Смотрите, чтобы кто кому не воздавал злом за зло; но всегда ищите добра и друг другу и всем.
\vs 1Th 5:16 Всегда радуйтесь.
\vs 1Th 5:17 Непрестанно молитесь.
\vs 1Th 5:18 За все благодарите: ибо такова о вас воля Божия во Христе Иисусе.
\vs 1Th 5:19 Духа не угашайте.
\vs 1Th 5:20 Пророчества не уничижайте.
\vs 1Th 5:21 Все испытывайте, хорошего держитесь.
\vs 1Th 5:22 Удерживайтесь от всякого рода зла.
\vs 1Th 5:23 Сам же Бог мира да освятит вас во всей полноте, и ваш дух и душа и тело во всей целости да сохранится без порока в пришествие Господа нашего Иисуса Христа.
\vs 1Th 5:24 Верен Призывающий вас, Который и сотворит \bibemph{сие}.
\vs 1Th 5:25 Братия! молитесь о нас.
\rsbpar\vs 1Th 5:26 Приветствуйте всех братьев лобзанием святым.
\vs 1Th 5:27 Заклинаю вас Господом прочитать сие послание всем святым братиям.
\rsbpar\vs 1Th 5:28 Благодать Господа нашего Иисуса Христа с вами. Аминь.

\include{tex/2Th}
\include{tex/1Ti}\newbookpage
\bibbookdescr{2Ti}{
  inline={Второе Послание к Тимофею\\\LARGE Святого Апостола Павла},
  toc={2-е Тимофею},
  bookmark={2-е Тимофею},
  header={2-е Тимофею},
  %headerleft={},
  %headerright={},
  abbr={2~Тим}
}
\vs 2Ti 1:1 Павел, волею Божиею Апостол Иисуса Христа, по обетованию жизни во Христе Иисусе,
\vs 2Ti 1:2 Тимофею, возлюбленному сыну: благодать, милость, мир от Бога Отца и Христа Иисуса, Господа нашего.
\rsbpar\vs 2Ti 1:3 Благодарю Бога, Которому служу от прародителей с чистою совестью, что непрестанно вспоминаю о тебе в молитвах моих днем и ночью,
\vs 2Ti 1:4 и желаю видеть тебя, вспоминая о слезах твоих, дабы мне исполниться радости,
\vs 2Ti 1:5 приводя на память нелицемерную веру твою, которая прежде обитала в бабке твоей Лоиде и матери твоей Евнике; уверен, что она и в тебе.
\vs 2Ti 1:6 По сей причине напоминаю тебе возгревать дар Божий, который в тебе через мое рукоположение;
\vs 2Ti 1:7 ибо дал нам Бог духа не боязни, но силы и любви и целомудрия.
\vs 2Ti 1:8 Итак, не стыдись свидетельства Господа нашего Иисуса Христа, ни меня, узника Его; но страдай с благовестием Христовым силою Бога,
\vs 2Ti 1:9 спасшего нас и призвавшего званием святым, не по делам нашим, но по Своему изволению и благодати, данной нам во Христе Иисусе прежде вековых времен,
\vs 2Ti 1:10 открывшейся же ныне явлением Спасителя нашего Иисуса Христа, разрушившего смерть и явившего жизнь и нетление через благовестие,
\vs 2Ti 1:11 для которого я поставлен проповедником и Апостолом и учителем язычников.
\vs 2Ti 1:12 По сей причине я и страдаю так; но не стыжусь. Ибо я знаю, в Кого уверовал, и уверен, что Он силен сохранить залог мой на оный день.
\vs 2Ti 1:13 Держись образца здравого учения, которое ты слышал от меня, с верою и любовью во Христе Иисусе.
\vs 2Ti 1:14 Храни добрый залог Духом Святым, живущим в нас.
\rsbpar\vs 2Ti 1:15 Ты знаешь, что все Асийские оставили меня; в числе их Фигелл и Ермоген.
\vs 2Ti 1:16 Да даст Господь милость дому Онисифора за то, что он многократно покоил меня и не стыдился уз моих,
\vs 2Ti 1:17 но, быв в Риме, с великим тщанием искал меня и нашел.
\vs 2Ti 1:18 Да даст ему Господь обрести милость у Господа в оный день; а сколько он служил мне в Ефесе, ты лучше знаешь.
\vs 2Ti 2:1 Итак укрепляйся, сын мой, в благодати Христом Иисусом,
\vs 2Ti 2:2 и что слышал от меня при многих свидетелях, то передай верным людям, которые были бы способны и других научить.
\vs 2Ti 2:3 Итак переноси страдания, как добрый воин Иисуса Христа.
\vs 2Ti 2:4 Никакой воин не связывает себя делами житейскими, чтобы угодить военачальнику.
\vs 2Ti 2:5 Если же кто и подвизается, не увенчивается, если незаконно будет подвизаться.
\vs 2Ti 2:6 Трудящемуся земледельцу первому должно вкусить от плодов.
\vs 2Ti 2:7 Разумей, что я говорю. Да даст тебе Господь разумение во всем.
\rsbpar\vs 2Ti 2:8 Помни Господа Иисуса Христа от семени Давидова, воскресшего из мертвых, по благовествованию моему,
\vs 2Ti 2:9 за которое я страдаю даже до уз, как злодей; но для слова Божия нет уз.
\vs 2Ti 2:10 Посему я все терплю ради избранных, дабы и они получили спасение во Христе Иисусе с вечною славою.
\vs 2Ti 2:11 Верно слово: если мы с Ним умерли, то с Ним и оживем;
\vs 2Ti 2:12 если терпим, то с Ним и царствовать будем; если отречемся, и Он отречется от нас;
\vs 2Ti 2:13 если мы неверны, Он пребывает верен, ибо Себя отречься не может.
\rsbpar\vs 2Ti 2:14 Сие напоминай, заклиная пред Господом не вступать в словопрения, что нимало не служит к пользе, а к расстройству слушающих.
\vs 2Ti 2:15 Старайся представить себя Богу достойным, делателем неукоризненным, верно преподающим слово истины.
\vs 2Ti 2:16 А непотребного пустословия удаляйся; ибо они еще более будут преуспевать в нечестии,
\vs 2Ti 2:17 и слово их, как рак, будет распространяться. Таковы Именей и Филит,
\vs 2Ti 2:18 которые отступили от истины, говоря, что воскресение уже было, и разрушают в некоторых веру.
\vs 2Ti 2:19 Но твердое основание Божие сто\acc{и}т, имея печать сию: <<познал Господь Своих>>; и: <<да отступит от неправды всякий, исповедующий имя Господа>>.
\vs 2Ti 2:20 А в большом доме есть сосуды не только золотые и серебряные, но и деревянные и глиняные; и одни в почетном, а другие в низком употреблении.
\vs 2Ti 2:21 Итак, кто будет чист от сего, тот будет сосудом в чести, освященным и благопотребным Владыке, годным на всякое доброе дело.
\vs 2Ti 2:22 Юношеских похотей убегай, а держись правды, веры, любви, мира со всеми призывающими Господа от чистого сердца.
\vs 2Ti 2:23 От глупых и невежественных состязаний уклоняйся, зная, что они рождают ссоры;
\vs 2Ti 2:24 рабу же Господа не должно ссориться, но быть приветливым ко всем, учительным, незлобивым,
\vs 2Ti 2:25 с кротостью наставлять противников, не даст ли им Бог покаяния к познанию истины,
\vs 2Ti 2:26 чтобы они освободились от сети диавола, который уловил их в свою волю.
\vs 2Ti 3:1 Знай же, что в последние дни наступят времена тяжкие.
\vs 2Ti 3:2 Ибо люди будут самолюбивы, сребролюбивы, горды, надменны, злоречивы, родителям непокорны, неблагодарны, нечестивы, недружелюбны,
\vs 2Ti 3:3 непримирительны, клеветники, невоздержны, жестоки, не любящие добра,
\vs 2Ti 3:4 предатели, наглы, напыщенны, более сластолюбивы, нежели боголюбивы,
\vs 2Ti 3:5 имеющие вид благочестия, силы же его отрекшиеся. Таковых удаляйся.
\vs 2Ti 3:6 К сим принадлежат те, которые вкрадываются в домы и обольщают женщин, утопающих во грехах, водимых различными похотями,
\vs 2Ti 3:7 всегда учащихся и никогда не могущих дойти до познания истины.
\vs 2Ti 3:8 Как Ианний и Иамврий противились Моисею, так и сии противятся истине, люди, развращенные умом, невежды в вере.
\vs 2Ti 3:9 Но они не много успеют; ибо их безумие обнаружится перед всеми, как и с теми случилось.
\vs 2Ti 3:10 А ты последовал мне в учении, житии, расположении, вере, великодушии, любви, терпении,
\vs 2Ti 3:11 в гонениях, страданиях, постигших меня в Антиохии, Иконии, Листрах; каковые гонения я перенес, и от всех избавил меня Господь.
\vs 2Ti 3:12 Да и все, желающие жить благочестиво во Христе Иисусе, будут гонимы.
\vs 2Ti 3:13 Злые же люди и обманщики будут преуспевать во зле, вводя в заблуждение и заблуждаясь.
\vs 2Ti 3:14 А ты пребывай в том, чему научен и что тебе вверено, зная, кем ты научен.
\vs 2Ti 3:15 Притом же ты из детства знаешь священные писания, которые могут умудрить тебя во спасение верою во Христа Иисуса.
\vs 2Ti 3:16 Все Писание богодухновенно и полезно для научения, для обличения, для исправления, для наставления в праведности,
\vs 2Ti 3:17 да будет совершен Божий человек, ко всякому доброму делу приготовлен.
\vs 2Ti 4:1 Итак заклинаю тебя пред Богом и Господом нашим Иисусом Христом, Который будет судить живых и мертвых в явление Его и Царствие Его:
\vs 2Ti 4:2 проповедуй слово, настой во время и не во время, обличай, запрещай, увещевай со всяким долготерпением и назиданием.
\vs 2Ti 4:3 Ибо будет время, когда здравого учения принимать не будут, но по своим прихотям будут избирать себе учителей, которые льстили бы слуху;
\vs 2Ti 4:4 и от истины отвратят слух и обратятся к басням.
\vs 2Ti 4:5 Но ты будь бдителен во всем, переноси скорби, совершай дело благовестника, исполняй служение твое.
\rsbpar\vs 2Ti 4:6 Ибо я уже становлюсь жертвою, и время моего отшествия настало.
\vs 2Ti 4:7 Подвигом добрым я подвизался, течение совершил, веру сохранил;
\vs 2Ti 4:8 а теперь готовится мне венец правды, который даст мне Господь, праведный Судия, в день оный; и не только мне, но и всем, возлюбившим явление Его.
\rsbpar\vs 2Ti 4:9 Постарайся прийти ко мне скоро.
\vs 2Ti 4:10 Ибо Димас оставил меня, возлюбив нынешний век, и пошел в Фессалонику, Крискент в Галатию, Тит в Далматию; один Лука со мною.
\vs 2Ti 4:11 Марка возьми и приведи с собою, ибо он мне нужен для служения.
\vs 2Ti 4:12 Тихика я послал в Ефес.
\vs 2Ti 4:13 Когда пойдешь, принеси фелонь, который я оставил в Троаде у Карпа, и книги, особенно кожаные.
\vs 2Ti 4:14 Александр медник много сделал мне зла. Да воздаст ему Господь по делам его!
\vs 2Ti 4:15 Берегись его и ты, ибо он сильно противился нашим словам.
\rsbpar\vs 2Ti 4:16 При первом моем ответе никого не было со мною, но все меня оставили. Да не вменится им!
\vs 2Ti 4:17 Господь же предстал мне и укрепил меня, дабы через меня утвердилось благовестие и услышали все язычники; и я избавился из львиных челюстей.
\vs 2Ti 4:18 И избавит меня Господь от всякого злого дела и сохранит для Своего Небесного Царства, Ему слава во веки веков. Аминь.
\rsbpar\vs 2Ti 4:19 Приветствуй Прискиллу и Акилу и дом Онисифоров.
\vs 2Ti 4:20 Ераст остался в Коринфе; Трофима же я оставил больного в Милите.
\vs 2Ti 4:21 Постарайся прийти до зимы. Приветствуют тебя Еввул, и Пуд, и Лин, и Клавдия, и все братия.
\rsbpar\vs 2Ti 4:22 Господь Иисус Христос со духом твоим. Благодать с вами. Аминь.

\bibbookdescr{Tit}{
  inline={Послание к Титу\\\LARGE Святого Апостола Павла},
  toc={к Титу},
  bookmark={к Титу},
  header={к Титу},
  %headerleft={},
  %headerright={},
  abbr={Тит}
}
\vs Tit 1:1 Павел, раб Божий, Апостол же Иисуса Христа, по вере избранных Божиих и познанию истины, \bibemph{относящейся} к благочестию,
\vs Tit 1:2 в надежде вечной жизни, которую обещал неизменный в слове Бог прежде вековых времен,
\vs Tit 1:3 а в свое время явил Свое слово в проповеди, вверенной мне по повелению Спасителя нашего, Бога,~---
\vs Tit 1:4 Титу, истинному сыну по общей вере: благодать, милость и мир от Бога Отца и Господа Иисуса Христа, Спасителя нашего.
\rsbpar\vs Tit 1:5 Для того я оставил тебя в Крите, чтобы ты довершил недоконченное и поставил по всем городам пресвитеров, как я тебе приказывал:
\vs Tit 1:6 если кто непорочен, муж одной жены, детей имеет верных, не укоряемых в распутстве или непокорности.
\vs Tit 1:7 Ибо епископ должен быть непорочен, как Божий домостроитель, не дерзок, не гневлив, не пьяница, не бийца, не корыстолюбец,
\vs Tit 1:8 но страннолюбив, любящий добро, целомудрен, справедлив, благочестив, воздержан,
\vs Tit 1:9 держащийся истинного слова, согласного с учением, чтобы он был силен и наставлять в здравом учении и противящихся обличать.
\rsbpar\vs Tit 1:10 Ибо есть много и непокорных, пустословов и обманщиков, особенно из обрезанных,
\vs Tit 1:11 каковым должно заграждать уста: они развращают целые домы, уча, чему не должно, из постыдной корысти.
\vs Tit 1:12 Из них же самих один стихотворец сказал: <<Критяне всегда лжецы, злые звери, утробы ленивые>>.
\vs Tit 1:13 Свидетельство это справедливо. По сей причине обличай их строго, дабы они были здравы в вере,
\vs Tit 1:14 не внимая Иудейским басням и постановлениям людей, отвращающихся от истины.
\vs Tit 1:15 Для чистых все чисто; а для оскверненных и неверных нет ничего чистого, но осквернены и ум их и совесть.
\vs Tit 1:16 Они говорят, что знают Бога, а делами отрекаются, будучи гнусны и непокорны и не способны ни к какому доброму делу.
\vs Tit 2:1 Ты же говори то, что сообразно с здравым учением:
\vs Tit 2:2 чтобы старцы были бдительны, степенны, целомудренны, здравы в вере, в любви, в терпении;
\vs Tit 2:3 чтобы старицы также одевались прилично святым, не были клеветницы, не порабощались пьянству, учили добру;
\vs Tit 2:4 чтобы вразумляли молодых любить мужей, любить детей,
\vs Tit 2:5 быть целомудренными, чистыми, попечительными о доме, добрыми, покорными своим мужьям, да не порицается слово Божие.
\vs Tit 2:6 Юношей также увещевай быть целомудренными.
\vs Tit 2:7 Во всем показывай в себе образец добрых дел, в учительстве чистоту, степенность, неповрежденность,
\vs Tit 2:8 слово здравое, неукоризненное, чтобы противник был посрамлен, не имея ничего сказать о нас худого.
\vs Tit 2:9 Рабов \bibemph{увещевай} повиноваться своим господам, угождать им во всем, не прекословить,
\vs Tit 2:10 не красть, но оказывать всю добрую верность, дабы они во всем были украшением учению Спасителя нашего, Бога.
\vs Tit 2:11 Ибо явилась благодать Божия, спасительная для всех человеков,
\vs Tit 2:12 научающая нас, чтобы мы, отвергнув нечестие и мирские похоти, целомудренно, праведно и благочестиво жили в нынешнем веке,
\vs Tit 2:13 ожидая блаженного упования и явления славы великого Бога и Спасителя нашего Иисуса Христа,
\vs Tit 2:14 Который дал Себя за нас, чтобы избавить нас от всякого беззакония и очистить Себе народ особенный, ревностный к добрым делам.
\rsbpar\vs Tit 2:15 Сие говори, увещевай и обличай со всякою властью, чтобы никто не пренебрегал тебя.
\vs Tit 3:1 Напоминай им повиноваться и покоряться начальству и властям, быть готовыми на всякое доброе дело,
\vs Tit 3:2 никого не злословить, быть не сварливыми, но тихими, и оказывать всякую кротость ко всем человекам.
\vs Tit 3:3 Ибо и мы были некогда несмысленны, непокорны, заблуждшие, были рабы похотей и различных удовольствий, жили в злобе и зависти, были гнусны, ненавидели друг друга.
\vs Tit 3:4 Когда же явилась благодать и человеколюбие Спасителя нашего, Бога,
\vs Tit 3:5 Он спас нас не по делам праведности, которые бы мы сотворили, а по Своей милости, банею возрождения и обновления Святым Духом,
\vs Tit 3:6 Которого излил на нас обильно через Иисуса Христа, Спасителя нашего,
\vs Tit 3:7 чтобы, оправдавшись Его благодатью, мы по упованию соделались наследниками вечной жизни.
\vs Tit 3:8 Слово это верно; и я желаю, чтобы ты подтверждал о сем, дабы уверовавшие в Бога старались быть прилежными к добрым делам: это хорошо и полезно человекам.
\vs Tit 3:9 Глупых же состязаний и родословий, и споров и распрей о законе удаляйся, ибо они бесполезны и суетны.
\vs Tit 3:10 Еретика, после первого и второго вразумления, отвращайся,
\vs Tit 3:11 зная, что таковой развратился и грешит, будучи самоосужден.
\rsbpar\vs Tit 3:12 Когда пришлю к тебе Артему или Тихика, поспеши прийти ко мне в Никополь, ибо я положил там провести зиму.
\vs Tit 3:13 Зину законника и Аполлоса позаботься отправить так, чтобы у них ни в чем не было недостатка.
\vs Tit 3:14 Пусть и наши учатся упражняться в добрых делах, \bibemph{в удовлетворении} необходимым нуждам, дабы не были бесплодны.
\rsbpar\vs Tit 3:15 Приветствуют тебя все находящиеся со мною. Приветствуй любящих нас в вере. Благодать со всеми вами. Аминь.

\include{tex/Phm}
\include{tex/Heb}
\include{tex/Rev}
\bibpart{Апокрифа}{Апокрифа}{Apo}
\bibbookdescr{1En}{
  inline={Первая книга Еноха},
  toc={1-я Еноха},
  bookmark={1-я Еноха},
  header={1-я Еноха},
  abbr={1~Ено}
}
\vs 1En 1:1
Слова благословения Еноха, которыми он благословил избранных и
праведных, которые будут жить в день скорби, когда все злые и нечестивые
будут отвержены.
И отвечал и сказал Енох,~--- праведный муж, которому были открыты
Богом очи,~--- что он видел на небесах святое видение: Его показали мне
ангелы, и от них я слышал всё, и уразумел, что видел, но не для этого рода,
а для родов отдалённых, которые явятся.
Об избранных говорил я и о них беседовал со Святым и Великим, с
Богом мира, Который выйдет из Своего жилища.
И оттуда Он придёт на гору Синай, и явится со Своими воинствами, и в
силе Своего могущества явится с неба.
И всё устрашится, и стражи содрогнутся, и великий страх и трепет
обоймёт их до пределов земли.
Поколеблются возвышенные горы, и высокие холмы опустятся, и растают,
как сотовый мёд от пламени.
Земля погрузится, и всё, что на земле, погибнет, и совершится суд
над всем и над всеми праведными.
Но праведным Он уготовит мир и будет охранять избранных, и милость
будет господствовать над ними; они все будут Божии, и хорошо им будет, и они
будут благословлены, и свет Божий будет светить им.
И вот Он идёт с мириадами святых,  чтобы совершить суд над ними,  и
Он уничтожит нечестивых, и будет судиться со всякою плотью относительно
всего, что грешники и нечестивые сделали и совершили против Него.
Я наблюдал всё, что происходит на небе: как светила, которые на
небе, не изменяют своих путей, как все они восходят и заходят по порядку,
каждое в своё время, и не преступают своих законов.
Взгляните на землю и обратите внимание на вещи, которые на ней, от
первой до последней, как каждое произведение Божие правильно обнаруживает
себя!
Взгляните на лето и зиму, как тогда (зимою) вся земля изобилует
водою, и тучи, и роса, и дождь стелются над нею!
Я наблюдал и видел, как зимою все деревья кажутся, будто они
высохли, и все листья их опали, кроме четырнадцати деревьев, которые не
обнажаются, но ожидают, оставаясь со старой листвой, появления новой в
течение двух--трёх лет.
И опять я наблюдал дни летние, как тогда солнце стоит над нею
(землёю), прямо против неё, а вы ищете прохладных мест и тени от солнечной
жары, и как тогда даже земля горит от зноя, а вы не можете ступить ни на
землю, ни на скалу (камень) вследствие их жара.
Я наблюдал, как деревья покрываются зеленью листьев и приносят
плоды; и вы обратите внимание на всё и узнайте, что всё это для вас сотворил
Тот, Который живёт вечно; посмотрите, как Его произведения существуют пред
Ним в каждом новом году и все Его произведения служат Ему и не изнемогают,
но как установил Бог, так всё и происходит!
И посмотрите, как моря и реки все вместе выполняют своё дело!
А вы не претерпели до конца и не выполнили закона Господня; но
преступили его и надменными, хульными словами поносили Его величие из своих
нечестивых уст; вы, жестокосердые, не обретёте никакого мира!
И посему вы проклянёте ваши дни, и годы вашей жизни прекратятся;
велико будет вечное осуждение, и вы не обретёте никакой милости.
В те дни вы лишитесь мира, чтобы быть вечным проклятием для всех
праведных, и они будут всегда проклинать вас как грешников,~--- вас вместе со
всеми грешниками.
Для избранных же настанет свет, и радость, и мир, и они наследуют
землю; а для вас, нечестивые, наступит проклятие.
Тогда избранным будет дана мудрость и они все будут жить и не
согрешат опять ни по небрежности, ни по надменности, но будут смирёнными, не
согрешая опять, так как имеют мудрость.
И они будут наказаны в продолжение своей жизни, и не умрут в муках
и в гневном осуждении, но окончат число дней своей жизни, а состареются в
мире, и годы их счастья будут многими: они будут пребывать в вечном
наслаждении и в мире в продолжение всей своей жизни.
\vs 1En 2:1
И случилось,~--- после того как сыны человеческие умножились в те
дни, у них родились красивые и прелестные дочери.
И ангелы, сыны неба, увидели их, и возжелали их, и сказали друг
другу: "давайте выберем себе жён в среде сынов человеческих и родим себе
детей"!
И Семъйяза,  начальник их,  сказал им: "Я боюсь, что вы не захотите
привести в исполнение это дело и тогда я  один  должен  буду  искупать
этот великий грех".
Тогда все они ответили ему и сказали: "Мы все поклянёмся клятвою и
обяжемся друг другу заклятиями не оставлять этого намерения, но привести его в
исполнение".
Тогда  поклялись  все  они вместе и обязались в этом все друг другу
заклятиями: было же их всего двести.
И они спустились на Ардис,  который есть вершина горы Ермон;  и они
назвали  её  горою  Ермон,  потому что поклялись на ней и изрекли друг другу
заклятия.
И вот имена их начальников:  Семъйяза, их начальник, Уракибарамеел,
Акибеел,  Тамиел,  Рамуел,  Данел, Езекеел, Саракуйял, Азаел, Батраал, Анани,
Цакебе, Самсавеел, Сартаел, Турел, Иомъйяел, Аразъйял. Это управители двухсот
ангелов, и другие все были с ними.
И они взяли себе жён, и каждый выбрал для себя одну; и они начали
входить к ним и смешиваться с ними, и научили их волшебству и заклятиям, и
открыли им срезывания корней и деревьев.
Они зачали и родили великих исполинов, рост которых был в три тысячи
локтей.
Они поели всё приобретение людей, так что люди уже не могли
прокармливать их.
Тогда исполины обратились против самих людей, чтобы пожирать их.
И они стали согрешать по отношению к птицам и зверям, и тому, что
движется, и рыбам, и стали пожирать друг с другом их мясо и пить из него кровь.
Тогда сетовала земля на нечестивых.
И Азазел научил людей делать мечи, и ножи, и щиты, и панцири, и
научил их видеть, что было позади них, и научил их искусствам: запястьям, и
предметам украшения, и употреблению белил и румян, и украшению бровей, и
украшению драгоценнейших и превосходнейших камней, и всяких цветных материй и
металлов земли.
И явилось великое нечестие и много непотребств, и люди согрешали, и
все пути их развратились.
Амезарак научил всяким заклинаниям и срезыванию корней, Армарос~---
расторжению заклятий, Баракал~--- наблюдению над звёздами, Кокабел~--- знамениям;
и Темел научил наблюдению над звёздами, и Астрадел научил движению Луны.
И когда люди погибли, они возопили и голос их проник к небу.
Тогда взглянули Михаил, Гавриил, Суръйян и Уръйян с неба и
увидели много крови, которая текла на земле, и всю неправду, которая
совершалась на земле.
И они сказали друг другу: "Голос вопля их (людей) достиг от
опустошённой земли до врат неба.
И ныне к вам, о святые неба, обращаются с мольбою души людей, говоря:
испросите нам правду у Всевышнего".
И они сказали своему Господу Царю: "Господь господей, Бог богов, Царь
царей!
Престол Твоей славы существует во все роды мира: Ты прославлен и
восхвалён!
Ты всё сотворил, и владычество над всем Тебе принадлежит: всё пред
Тобою обнаружено и открыто, и Ты видишь всё, и ничто не могло сокрыться пред
Тобою.
Так посмотри же, что сделал Азазел, как он научил на земле всякому
нечестию и открыл небесные тайны мира.
И заклинания открыл Семъйяза, которому ты дал власть быть вождём его
сообщников.
И пришли они (стражи) друг с другом к дочерям человеческими переспали
с ними, с этими жёнами, и осквернились, и открыли им эти грехи.
Жёны же родили исполинов, и чрез это вся земля наполнилась кровью и
нечестием.
И вот теперь разлученные души сетуют и вопиют к вратам неба и их
воздыхание возносится: они не могут убежать от нечестия, которое совершается
на земле.
И Ты знаешь всё, прежде чем это случилось, и Ты знаешь это и их дела,
и, однако же, ничего не говоришь нам.
Что мы теперь должны сделать с ними за это?
Тогда стал говорить Всевышний, Великий и Святый, и послал
Арсъйялалйюра к сыну Лемеха (Ною) и сказал ему: "Скажи ему Моим именем:
"скройся"!
и объяви ему предстоящий конец!
Ибо вся земля погибнет, и вода потопа готовится прийти на всю землю,
и то, что есть на ней, погибнет.
И теперь научи его, чтобы он спасся и его семя сохранилось для всей
земли"!
И сказал опять Господь Рафуилу: "Свяжи Азазела по рукам и ногам и
положи его во мрак; сделай отверстие в пустыне, которая находится в Дудаеле, и
опусти его туда.
И положи на него грубый и острый камень, и покрой его мраком, чтобы
он оставался там навсегда, и закрой ему лицо, чтобы он не смотрел на свет!
И в великий день суда он будет брошен в жар (в геенну).
И исцели землю, которую развратили ангелы, и возвести земле
исцеление, что Я исцелю её и что не все сыны человеческие погибнут чрез тайну
всего того, что сказали стражи и чему научили сыновей своих; и вся земля
развратилась чрез научения делам Азазела: ему припиши все грехи"!
И Гавриилу Бог сказал: "Иди к незаконным детям, и любодейцам, и к
детям любодеяния и уничтожь детей любодеяния и детей стражей из среды людей;
выведи их и выпусти, чтобы они сами погубили себя чрез избиения друг друга:
ибо они не должны иметь долгой жизни.
И  все они будут просить тебя, но отцы их (исполинов) ничего не
добьются для них (в пользу их), хотя они и надеются на вечную жизнь и на то,
что каждый из них проживёт пятьсот лет".
И Михаилу Бог сказал: "Извести Семъйязу и его соучастников, которые
соединились с жёнами, чтобы развратиться с ними во всей их нечистоте.
Когда все сыны их взаимно будут избивать друг друга и они увидят
погибель своих любимцев, то крепко свяжи их под холмами земли на семьдесят
родов до дня суда над ними и до окончания родов, пока не совершится последний
суд над всею вечностью.
В те дни их бросят в огненную бездну; на муку и в узы они будут
заключены на всю вечность.
И немедленно Семъйяза сгорит и отныне погибнет с ними; они будут
связаны друг с другом до окончания всех родов.
И уничтожь все сладострастные души и детей стражей, ибо они дурно
поступили с людьми.
Уничтожь всякое насилие с лица земли, и всякое злое деяние должно
прекратиться; и явится растение справедливости и правды, и всякое дело будет
сопровождаться благословением; справедливость и правда будут насаждать полную
радость в века.
И теперь в смирении будут поклоняться все праведные и будут пребывать
в жизни, пока не родят тысячу детей, и все дни своей юности и свои субботы они
окончат в мире.
В те дни вся земля будет обработана в справедливости, и будет вся
обсажена деревьями, и исполнятся благословения.
Всякие деревья веселия насадятся на ней, и виноградники насадят на
ней; виноградник, который будет насажен на ней, принесёт плод в изобилии, и от
всякого семени, которое будет на ней посеяно, одна мера принесёт десять тысяч,
и мера маслин даст десять пресов елея.
И ты очисть землю от всякого насилия, и от всякой неправды, и от
всякого греха, и от всякой нечистоты, какая совершается на земле, уничтожь их
с земли.
И все сыны человеческие должны сделаться праведными, и все народы
будут оказывать Мне почесть и прославлять Меня, и все будут поклоняться Мне.
И земля будет очищена от всякого развращения, и от всякого греха, и
от всякого наказания, и от всякого мучения; И Я никогда не пошлю опять на неё
потопа, от рода до рода вовек.
В те дни Я открою сокровищницы благословения, которые на небе,
чтобы низвести их на землю, на произведение и на труд сынов человеческих.
Мир и правда соединятся тогда на все дни мира и на все роды земли.
\vs 1En 3:1
И прежде чем всё это случилось, Енох был сокрыт, и никто из
людей не знал, где он сокрыт, и где он пребывает, и что с ним стало.
И вся его деятельность в течение земной жизни была со святыми и со
стражами.
--- И едва я, Енох, прославил великого Господа и Царя мира, как меня
призвали стражи,~--- меня, Еноха, писца,~--- и сказали мне: "Енох, писец правды!
Иди, возвести стражам неба, которые оставили вышнее небо и святые
вечные места, и развратились с жёнами, и поступили так, как делают сыны
человеческие, и взяли себе жён,  и погрузились на земле в великое
развращение: они не будут иметь на земле ни мира, ни прощение грехов: ибо они
не могут радоваться своим детям.
Избиение своих любимцев увидят они, и о погибели своих детей будут
воздыхать; и будут умолять, но милосердия и мира не будет для них".
И Енох пошёл и сказал Азазелу: "Ты не будешь иметь мира; тяжкий
суд учинён над тобою, чтобы взять тебя, связать тебя, и облегчение, ходатайство
и милосердие не будут долею для тебя за то насилие, которому ты научил, и за
все дела хулы, насилия и греха, которые ты показал сынам человеческим".
Тогда я пошёл далее и сказал всем им вместе; и они устрашились все,
страх и трепет объял их.
И они просили меня написать за них просьбу, чтобы чрез это они обрели
прощение, и вознести их просьбу на небо к Богу.
Ибо сами они не могли отныне ни говорить с Ним, ни поднять очей своих
к небу от стыда за свою греховную вину, за которую они были наказаны.
Тогда я составил им письменную просьбу и мольбу относительно
состояния их духа и их отдельных поступков и относительно того, о чём они
просили, чтобы чрез это получили они прощение и долготерпение.
И я пошёл, и сел при водах Дана в области Дан (т.е. к югу) от
западной стороны Ермона, и читал их просьбу, пока не заснул.
И вот нашёл на меня сон, и напало на меня видение; и я видел видение
суда, которое я должен был возвестить сынам неба и сделать им порицание.
И как только я пробудился от сна, то пришёл к ним; и все они сидели
печальные с закрытыми лицами, собравшись в Ублес-йяеле, который лежит между
Ливаном и Сенезером.
И я рассказал им все видения, которые видел во время своего сна, и
начал говорить те слова правды и порицать стражей неба.
То, что здесь далее написано, есть слово правды и наставления,
данное мне вечными стражами, как повелел им Святый и Великий в том видении.
Я видел во время видения моего сна то, что я буду теперь рассказывать
моим плотским языком и моим дыханием, которое Великий вложил в уста людям,
чтобы они говорили им и понимали это сердцем (мыслию).
Как сотворил Он всех людей и даровал им понимание слова благоразумия,
так Он сотворил и меня и дал мне право порицать стражей~--- сынов неба.
"Я написал вашу просьбу, и мне было открыто в видении, что именно
ваша просьба не будет для вас исполнена до всей вечности, дабы совершился над
вами суд, и ничто не будет для вас исполнено.
И отныне вы не взойдёте уже на небо до всей вечности и на земле вас
должны связать на все дни мира: такой произнесён приговор.
Но прежде этого вы увидите уничтожение ваших возлюбленных сынов, и вы
будете обладать ими, но они падут пред вами от меча.
Ваша просьба за них не будет исполнена для вас, как и та (моя)
просьба за вас; вы не можете даже в плаче и воздыхании произносить устами ни
одного слова из писания, которое я написал".
И видение мне явилось таким образом: вот тучи звали меня в видении и
облако звало меня; движение звёзд и молний гнало и влекло меня; и ветры в
видении дали мне крылья и гнали меня.
Они вознесли меня на небо, и я приблизился к одной стене, которая
была устроена из кристалловых камней и окружена огненным пламенем; и она стала
устрашать меня.
И я вошёл в огненное пламя, и приблизился к великому дому, который
был устроен из кристалловых камней; стены этого дома были подобны наборному
полу (паркет или мозаика) из кристалловых камней, и почвою его был кристалл.
Его крыша была подобна пути звёзд и молний с огненными херувимами
между нею (крышею) и водным небом.
Пылающий огонь окружал стены дома, и дверь его горела огнём.
И я вступил в тот дом, который был горяч как огонь и холоден как лёд;
не было в нём ни веселия, ни жизни~--- страх покрыл меня и трепет объял меня.
И так как я был потрясён и трепетал, то упал на своё лицо; и я видел
в видении.
И вот там был другой дом, больший, нежели тот; все врата его стояли
предо мной отворёнными, и он был выстроен из огненного пламени.
И во всём было так преизобильно: во славе, в великолепии и величии,
что я не могу дать описания вам его величия и его славы.
Почвою же дома был огонь, а поверх его была молния и путь звёзд, и
даже его крышею был пылающий огонь.
И я взглянул и увидел в нём возвышенный престол; его вид был как
иней, и вокруг него было как бы блистающее солнце и херувимские голоса.
И из-под великого престола выходили реки пылающего огня, так что
нельзя было смотреть на него.
И Тот, Кто велик во славе, сидел на нём; одежда Его была блестящее,
чем само солнце, и белее чистого снега.
Ни ангел не мог вступить сюда, ни смертный созерцать вид лица самого
Славного и Величественного.
Пламень пылающего огня был вокруг Него, и великий огонь находился
пред Ним, и никто не мог к Нему приблизиться из тех, которые находились около
Него: тьмы тем были пред Ним, но Он не нуждался в святом совете.
И святые, которые были вблизи Его, не удалялись ни днём, ни ночью и
никогда не отходили от Него.
И  я с тех пор имел покрывало на своём челе, потому что трепетал;
тогда позвал меня Господь собственными устами и сказал мне: "Пойди, Енох, сюда
и к Моему святому слову"!
И Он повелел подняться мне и подойти к вратам~--- я же опустил своё
лицо.
И Он отвечал и сказал мне Своим словом: Слушай!
Не страшись, Енох, ты праведный муж и писец правды; подойди сюда и
выслушай Моё слово!
И ступай, скажи стражам неба, которые послали тебя, чтобы ты просил
за них: вы должны попросить за людей, а не люди за вас.
Зачем вы оставили вышнее, святое, вечное небо, и преспали с жёнами,
и осквернились с дочерьми человеческими, и взяли себе жён, и поступали как сыны
земли, и родили сынов-исполинов?
Будучи духовными, святыми, в наслаждении вечной жизни, вы
осквернились с жёнами, кровию плотской родили детей, возжелали крови людей и
произвели плоть и кровь, как производят те, которые смертны и тленны.
Ради того-то Я им и дал жён, чтобы они оплодотворяли их, и чрез них
рождали бы детей, как это обыкновенно происходит на земле.
Но вы были прежде духовны, призваны к наслаждению вечной, бессмертной
жизни на все роды мира.
Посему Я не сотворил для вас жён, ибо духовные имеют своё жилище на
небе.
И теперь исполины, которые родились от тела и плоти, будут называться
на земле злыми духами и на земле будет их жилище.
Злые существа выходят из тела их; так как они сотворены свыше и их
начало и первое происхождение было от святых стражей, то они будут на земле
злыми духами, и будут называться злыми духами.
А духи неба имеют своё жилище на небе, а духи земли, родившиеся на
земле, имеют своё жилище на земле.
И духи исполинов, которые устремляются на облака, погибнут, и будут
низринуты, и станут совершать насилие, и производить разрушения на земле, и
причинять бедствия; они не будут принимать пищи, и не будут жаждать, и будут
невидимы.
И те существа не восстанут против сынов человеческих и против жён,
так как они произошли от них.
В дни избиения и погибели и смерти исполинов, лишь только души
выйдут из тел, их тело должно предаться тлению без суда; так будут погибать
они до того дня, когда великий суд совершится над великим миром,~--- над стражами
и нечестивыми людьми.
И теперь скажи стражам, которые послали тебя, чтобы ты просил за них,
и которые жили прежде на небе, теперь скажи им: "Вы были на небе, и хотя
сокровенные вещи не были ещё открыты вам, однако вы узнали незначительную тайну
и рассказали её в своём жестокосердии жёнам, и чрез эту тайну жёны и мужья
причиняют земле много зла".
Скажи им: "Для вас нет мира".
\vs 1En 4:1
И они (ангелы) унесли меня в одно место, где были фигуры, как
пылающий огонь, и когда они хотели, то казались людьми.
И они привели меня к месту бури и на одну гору, конец вершины которой
доходил до неба.
И я увидел ярко блестящие места и гром на краях их; в глубине этого
огненный лук стрелы и колчан для них, и огненный меч, и все молнии.
И они донесли меня до так называемой воды и до огня запада, который
принимает в себя каждый вечер заходящее солнце.
И я пришёл к огненной реке, огонь которой жидкий, как вода, и которая
впадает в великое море к западу.
И я видел все великие реки, и дошёл до великого мрака, и пришёл туда,
где шествуют все смертные.
И  я  видел горы мрачных туч зимнего времени и место, куда впадает
вода целой бездны.
И я видел устье всех рек земли и устье бездны.
И я видел хранилища всех ветров, и видел, как Он изукрасил этим
всё творение, и видел основание земли.
И я видел краеугольный камень земли, и видел четыре ветра, которые
носят землю и основание неба.
И я видел, как ветры растягивают высоты неба, и они носятся между
небом и землёю~--- это столпы неба.
И я видел ветры, которые кружат небо, которые несут солнечный круг и
все звёзды к заходу.
И я видел ветры на земле, которые носят тучи; и видел пути ангелов,
и видел в конце земли вверху основание неба.
И я пошёл далее к югу, который горит день и ночь,~--- туда, где
находятся семь гор из драгоценных камней,~--- три к востоку и три к югу: и
именно те, которые к востоку, одна из цветных камней, и одна из перловых
камней, и одна из сурьмы; а те, которые к югу, из красных камней.
Средняя же, достигавшая до неба, как престол Божий, была из
алебастра, и вершина престола из сапфира.
И я видел пылающий огонь, который был во всех горах.
И я видел там одно место по ту сторону великой земли: там собирались
воды.
И я видел глубокую расселину в земле со столбами небесного огня; и я
видел между ними ниспадающие столбы небесного огня, которые нельзя было
сосчитать ни в направлении к верху, ни к низу.
И над тою расселиной я видел одно место, которое не имело ни небесной
тверди над собою, ни земного основания под собою; на нём не было ни воды, ни
птиц, но это было пустое место.
И было ужасно то, что я видел там: семь звёзд, как великие горящие
горы и как духи, которые просили меня.
Ангел сказал мне: "Это то место, где оканчивается небо и земля; оно
служит темницей для звёзд небесных и для воинства небесного.
И эти звёзды, которые катятся над огнём, суть те самые, которые
преступили повеление Божие пред своим восходом, так как они пришли не в своё
определённое время.
И Он разгневался на них и связал их до времени, когда окончится их
вина,~--- в год тайны".
И Уриил сказал мне: "Здесь будут находиться духи ангелов,
которые соединились с жёнами и, принявши различные виды, осквернили людей и
соблазнили их, чтобы они приносили жертвы демонам, как богам,~--- будут
находиться именно в тот день, когда над ними будет произведён великий суд, пока
не постигнет их конечная участь.
Так же и с жёнами их, которые соблазнили ангелов неба, будет
поступлено точно так же, как и с друзьями их.
И только я, Енох, созерцал пределы всего, и ни один человек не видел
их так, как видел их я.
И вот имена святых ангелов, которые стерегут: Уриил, один из
святых ангелов, ангел грома и колебания; Руфаил, один из святых ангелов, ангел
духов людей; Рагуил, один из святых ангелов, который карает мир и светила;
Михаил, один из святых ангелов, поставленный над лучшею частью людей,~--- над
избранным народом; Саракаел, один из святых ангелов, который поставлен над
душами сынов человеческих, склонивших духов к греху; Гавриил, один из святых
ангелов, который поставлен над змеями, и над раем, и над херувимами.
И я обошёл кругом до одного места, где не было никакой вещи.
И я видел там нечто страшное, ни небо возвышенное и ни землю
утверждённую, но одно пустое (пустынное) место, величественное и страшное.
И здесь я видел семь звёзд небесных, вместе связанных в этом месте,
подобным великим горам и пылающим как бы огнём.
На этот раз я сказал: "За какой грех они связаны и за что они сюда
изгнаны"?
Тогда мне сказал Уриил, один из святых ангелов, который был при мне
как мой путеводитель: "Енох, для чего ты разведываешь, и для чего разузнаёшь,
и спрашиваешь, и любопытствуешь?
Это те звёзды, которые преступили повеление Всевышнего Бога, и они
связаны здесь до тех пор, пока не окончится тьма миров,~--- число дней их вины".
И отсюда я пошёл в другое место, которое было ещё страшнее, чем это,
и увидел нечто страшное: там был великий огонь, который пылал и горел, и он
имел разделения; он был ограничен (окружён) совершенною пропастью; великие
огненные столбы низвергались туда; но его (огня) протяжения и величины я не мог
рассмотреть, и не в состоянии был даже взглянуть, откуда он происходит.
Тогда я сказал: "Как страшно это место и как мучительно осматривать
его!"
Тогда отвечал мне Уриил, один из святых ангелов, который был при мне;
он отвечал мне и сказал: "Енох, к чему такой страх и трепет в тебе на этом
ужасном месте и при виде этого мучения?"
И он сказал мне: "Это место~--- темница ангелов, и здесь они будут
содержаться заключёнными до вечности".
\vs 1En 5:1
Отсюда я пошёл в другое место, и он (Руфаил) показал мне на
западе большой высокий горный хребет, твёрдые скалы и четыре прекрасных места.
И между ними (последними) были глубокие, и обширные, и совершенно
выглаженные настолько гладко, как нечто, что катится, и глубокое, и мрачное
на вид.
На этот раз ответил мне Руфаил, один из святых ангелов, который был
со мною, и сказал мне: "Эти прекрасные места назначены для того, чтобы на них
собирались духи,~--- души умерших; для них они созданы, чтобы все души сынов
человеческих собирались здесь.
Места эти созданы для них местами жилища до дня их суда и до
определённого для них срока, и срок этот велик: он продолжится дотоле, пока не
совершится над ними великий суд".
И я видел духов сынов человеческих, которые умерли, и их голос
проникал до неба и сетовал.
На этот раз я спросил ангела Руфаила, который был со мною, и сказал
ему: "Чей это там дух, голос которого так проникает вверх и сетует?"
И он отвечал мне и сказал мне так: "Это дух, который вышел из Авеля,
убитого своим братом Каином; и он жалуется на него, пока семя его (Каина) не
будет изглажено с лица земли и из семени людей не будет уничтожено его семя".
И поэтому я спросил тогда о нём (об Авеле) и о суде над всеми и
сказал: "Почему одно место отделено от другого?"
И он отвечал мне и сказал мне: "Эти три остальные отделения сделаны
для того, чтобы разделять души умерших.
И души праведных отделены таким образом: там есть источник воды, над
которым свет.
Точно также сделано такое отделение и для грешников, когда они
умирают и погребаются на земле без того, что суд над ними ещё не произведён при
их жизни.
Здесь отделены их души, в этом великом мучении, пока не наступит
великий день суда и наказания, и мучения для хулителей до вечности, и мщения
для их душ; и он (ангел наказания) связал их здесь до вечности.
И если это было пред вечностью, тогда это (последнее) отделение
сделано для душ тех, которые сетуют и возвещают о своей погибели, так как они
были умерщвлены в дни грешников.
Таким образом, это отделение сделано для душ людей, которые были не
праведными, а грешниками, скончавшись в вине; они будут находиться возле
виновных и подобны им, но их души не умрут до дня суда и не выйдут отсюда.
Тогда я прославил Господа славы и сказал: "Будь прославлен, Господь
мой, Господь славы и справедливости, всё направляющий в вечность!"
Оттуда я пошёл в другое место к западу, к пределам земли.
И я видел здесь горящий огонь, который тёк беспрерывно, и ни днём, ни
ночью не прекращал своего течения, но равномерно тёк.
И я спросил Рагуила, говоря: "Что это такое там, что не имеет покоя?"
На этот раз отвечал мне Рагуил, один из святых ангелов, который был
со мною, и сказал мне: "Этот горящий огонь на западе, течение которого ты
видел, есть огонь всех светил небесных".
Оттуда я пошёл в другое место земли, и он (Михаил) показал мне
там горный хребет огненный, который горел день и ночь.
И я взошёл на него и увидел семь великолепных гор, из которых каждая
отделена от другой, и великолепные (драгоценные), прекрасные камни; всё было
великолепно и славного вида и прекрасной видимости; три горы расположены к
востоку, одна над другой укреплена, и три к югу, одна над другой укреплена;
здесь были и глубокие вьющиеся долины, из которых ни одна не примыкала к
другой.
И седьмая гора была между ними; в своей же вышине они все были
подобны тронному седалищу, которое было окружено благовонными деревьями.
И  между ними было одно дерево с благоуханием, которого я ещё никогда
не обонял ни от тех, ни от других деревьев; и никакой другой запах не был похож
на его запах; его листья, цветы, ствол не гниют вечно, и плод его прекрасен; а
его плод подобен грозду пальмы.
На этот раз я сказал: "Посмотри на это прекрасное дерево: прекрасны
на вид и приятны его листья (ветви), и его плод очень приятен для ока".
Тогда отвечал мне Михаил, один из святых и почитаемых ангелов, бывший
со мною, который был поставлен над этим.
И он сказал мне: "Енох, что ты спрашиваешь меня о запахе этого
дерева и стремишься узнать?"
Тогда я, Енох, отвечал ему, говоря: "Обо всём желал бы я узнать, но
особенно об этом дереве".
И он отвечал мне, говоря: "Эта высокая гора, которую ты видел, и
вершина, которая подобна престолу Господа, есть Его престол, где воссядет
Святый и Великий, Господь славы, вечный Царь, когда Он сойдёт, чтобы посетить
землю с милостью.
И к этому дереву с драгоценным запахом не позволено прикасаться ни
одному из смертных до времени великого суда; когда всё будет искуплено и
окончено для вечности, оно будет отдано праведным и смиренным.
От его плода будет дана жизнь избранным; оно будет пересажено на
север к святому месту,~--- к храму Господа, великого Царя.
Тогда они будут радоваться полною радостью и ликовать в Святом; они
будут воспринимать запах его в свои кости, и продолжительную жизнь они будут
жить на земле, как жили их отцы; и в дни их жизни не коснётся их ни печаль, ни
горе, ни труд, ни мучение".
Тогда я прославил Господа славы, вечного Царя, за то, что Он уготовал
это для праведных людей, и создал такое, и обещал дать им.
И оттуда я пошёл в средину земли, и видел благословенное и
плодородное место, где были ветви, которые укоренялись и вырастали из
срубленного дерева.
И там я видел святую гору, и под горой~--- к востоку от неё~--- воду,
которая текла к югу.
И я видел к востоку другую гору такой же вышины, и между ними обоими
глубокую долину, но неширокую; в ней также текла вода возле горы.
И на западе от неё была другая гора, ниже той и невысокая, и внизу
её, между ними (горами) обоими, была долина; и другие долины глубокие и сухие
были в конце всех трёх.
И все долины были глубокие, но не широкие, из твёрдого скалистого
камня; и деревья были насажены на них.
И я удивился скалам, и удивился долине, и удивился чрезвычайно.
Тогда я сказал: "Для чего эта благословенная страна, которая
вся наполнена деревьями, и в промежутке (между горами) эта проклятая долина?"
Тогда отвечал мне Уриил, один из святых ангелов, который был со мною,
и сказал мне: "Эта проклятая долина для тех, которые прокляты до вечности;
здесь должны собраться все те, которые говорят своими устами непристойные речи
против Бога, и дерзко говорят о Его славе; здесь соберут их, и здесь место их
наказания.
И в последнее время будет зрелище праведного суда над ними пред лицом
праведных навсегда в вечности; за это те, которые обрели милосердие, будут
прославлять Господа славы, вечного Царя.
И в дни суда над ними (грешниками) они (праведные) прославят Его за
милосердие, по которому он назначил им такой жребий".
Тогда и я прославил Господа славы, и говорил к Нему, и вспоминал Его
величие, как подобает.
Оттуда я пошёл к востоку, в самую средину горного хребта,
(находящегося в) пустыне, и здесь я не видел ничего, кроме одной равнины.
Но она была наполнена деревьями тех же семян, и вода струилась на неё
сверху.
Можно было видеть, насколько орошение, которое она поглощала, было
обильное, можно было видеть и то, что как на севере, так и на западе и как
повсюду, так и здесь поднималась вода и роса.
И я пошёл в другое место, прочь от пустыни, приближаясь к
горному хребту на востоке.
И там я видел деревья суда, особенно же такие, которые издают запах
ладана и мирры и которые были не похожи на обыкновенные деревья.
И над этим, высоко над ними (деревьями), над восточною горою и
недалеко от неё, видел я другое место, именно~--- долины с водой, которые не
иссякают.
И я видел прекрасное дерево, запах которого, как запах мастикса.
И по сторонам тех долин я видел благовонную корицу.
И я поднялся вверх над ними (долинами или деревьями), направляясь
ближе к востоку.
И я видел другую гору с деревьями, из которой текла вода и из
которой выходило нечто подобное нектару, что называют сарира и гальбан.
И  над той горой я видел другую гору, на которой были алойные
деревья; и те деревья изобиловали миндалеподобным твёрдым веществом.
И если взять тот плод, то он был лучше, чем всякие благовония.
И после этих благовоний, как только я взглянул к северу выше тех
гор, я увидел там ещё семь гор, изобиловавших драгоценными нардами и
благовонными деревьями, корицей и перцем.
Оттуда я пошёл на вершину тех гор далеко к востоку, и подвинулся
далее, пройдя над Эритрейским морем, и ушёл далеко от него, и прошёл над
ангелом Цутелем.
И я пришёл в сад правды и увидел разнообразное множество тех
деревьев; там росло много больших деревьев,~--- благовонных, великих, очень
прекрасных и великолепных,~--- и дерево мудрости, доставляющее великую мудрость
тем, которые вкушают от него.
И оно похоже на кератонию; его плод, подобный виноградной кисти,
очень прекрасен; запах дерева распространяется и проникает далеко.
И я сказал: "Как прекрасно это дерево и как прекрасен и прелестен его
вид!"
И святой ангел Руфаил, который был со мною, отвечал мне и сказал:
"Это то самое дерево мудрости, от которого твои предки, твой старый отец и
старая мать вкусили и обрели познание мудрости, и у них открылись очи, и они
узнали, что были наги и были изгнаны из сада".
Оттуда я пошёл к пределам земли и увидел там великих зверей, из
которых каждый был отличен от другого, а также птиц, разнообразных по наружной
красоте и по голосу, из которых каждая была отлична от другой.
И на востоке от тех зверей я видел пределы земли, на которых покоится
небо, и открытые врата неба.
И я видел, как выходят звёзды небесные, и сосчитал врата, из которых
они выходят, и записал все выходы их,~--- о каждой из них особо, по числу их, их
именам, их связи, их положению, их времени и их месяцам,~--- так, как показал мне
это ангел Уриил, который был со мною.
Всё показал он мне и записал мне; их имена он также записал для меня,
и их законы и их отправления.
Оттуда я пошёл к северу к пределам земли, и там я видел великое
и славное чудо на пределах всей земли.
Здесь я видел трое открытых небесных врат на небе; из них выходят
северные ветры; если там (из них) дует, то бывает холод, град, иней, снег, роса
и дождь.
Из одних врат (средних) дует ко благу; но если они (ветры) дуют чрез
двое других врат, то бывает бурно и на землю приносится бедствие, и они дуют
тогда бурно.
Оттуда я пошёл к западу к пределам земли и увидел тогда трое
открытых врат подобно тому, как я видел их на востоке,~--- одинаковые врата и
одинаковые выходы.
Оттуда я пошёл на юг к пределам земли и видел там двое открытых
врат неба; из них выходит южный ветер, а с ним~--- роса, дождь и ветер.
Оттуда я пошёл к востоку к пределам неба и видел здесь трое восточных
небесных врат открытых и над ними маленькие врата.
Чрез каждые маленькие врата проходят звёзды небесные и бегут к вечеру
(к западу) на колеснице, которая им назначена.
И как только я увидел это, то прославил Господа, и таким образом я
всякий раз прославлял Господа славы, который сотворил великие и славные чудеса,
чтобы показать величие Своего творения ангелам и душам людей, дабы они
восхваляли Его творение и дабы все Его твари видели дело Его могущества,
восхваляли великое дело Его рук и славили Его довеку.
\vs 1En 6:1
Второе видение мудрости, которое видел Енох, сын Иареда, сына
Малелеила, сына Каинана, сына Еноса, сына Сифа, сына Адама.
И вот начало речи мудрости,  которую я начал говорить и высказывать
живущим на земле; слушайте вы, древнейшие, и обратите внимание, потомки, на
святые слова, которые я буду говорить пред Господом духов.
Справедливо назвать тех (древних) прежде всего, но и потомков мы не
будем удерживать от начала премудрости.
И до сегодня никогда не была дарована от Господа духов кому-либо та
мудрость, которую я получил по моему разумению, по благоволению
Господа духов, от которого мне назначен жребий вечной жизни.
Три притчи были долею для меня, и я начал их рассказывать тем, которые
населяют твердь.
\vs 1En 7:1
Первая притча.
Когда откроется общество праведных, и грешники будут судимы за свои
грехи, и будут изгнаны с лица земли, и когда Праведный явится пред очами
избранных праведников, дела которых взвешены Господом духов, и свет откроется
праведным и избранным, живущим на земле,~--- то где тогда будет жилище грешников
и убежище тех, которые отвергли Господа духов?
Было бы лучше для них, если б они никогда не рождались.
И когда тайны праведных будут открыты, тогда грешники будут судимы и
нечестивые будут отвергнуты от праведных и избранных.
И отныне не будут более сильными и вознесёнными те, которые владеют
землёю, и не будут в состоянии видеть лицо святых, ибо свет Господа духов будет
сиять на лица святых, и праведных, и избранных.
И сильные цари погибнут в то время и будут преданы в руки праведных и
святых.
И с тех пор никто не будет (иметь возможности) молить Господа духов о
милости, ибо жизнь их (людей) окончится.
И это случится в те дни, когда избранные и святые дети сойдут с
высоких небес и их семя соединится с сынами человеческими.
В те дни Енох получил книги гнева и ярости и книги беспокойства и
смятения, и в это самое время меня унесла прочь от земли туча и буря, и
принесла меня к пределам неба.
И здесь я видел другое видение, именно~--- жилища праведных и ложа
святых.
Здесь мои очи видели жилища возле ангелов и их ложа возле святых,
видел, как они молились, и просили, и умоляли за сынов человеческих, и правда
текла пред ними, как вода, и милосердие, как роса на земле: так бывает между
ними от века до века.
И в те дни мои очи видели место избранных правды и веры, и как правда
господствует в те дни, и как неисчислимо велико множество праведных и избранных
пред Ним от века до века.
И я видел жилища их под крыльями Господа духов, и видел, как все
праведные и избранные украшены пред Ним как бы огненным сиянием, и их уста
полны славословия, и их губы хвалят имя Господа духов, и правда не преходит
пред Ним.
Здесь желал я жить, и моя душа стремилась к тому жилищу; здесь уже
прежде была уготована мне участь, ибо так постановлено относительно меня у
Господа духов.
В те дни я хвалил и превозносил имя Господа духов благословениями и
славословиями, ибо Он определил мне благословение и славу.
Долго рассматривали мои очи то место, и я прославил Его (Господа),
говоря: Хвала Ему и да прославится Он от начала до вечности!
Пред Ним нет прехождения; Он знает, прежде чем создан мир, что он
такое и что будет от рода до рода.
Тебя славят те, которые не спят они стоят пред Тобою славою, и
прославляют, хвалят и превозносят Тебя, говоря: "свят, свят, свят
Господь духов, Он наполняет землю духами!"
И здесь мои очи видели всех тех, которые не спят, как они стоят пред
Ним, и прославляют, и говорят: "Будь прославлен Ты и да будет прославлено имя
Господа от века до века!"
И моё лицо изменилось, так что я не мог больше видеть.
И после этого я видел тысячу тысяч и тьму тем, несметно и
неисчислимо многих, стоящих пред славою Господа духов.
Я видел, и на четырёх сторонах престола Господа духов я заметил
четыре лица, отличные от тех, которые стояли там (1 ст.), и я узнал имена их,
так как ангел, пришедший со мною (или ко мне), открыл мне имена их и показал
мне все сокровенные вещи.
И я слышал глас тех четырёх лиц, как они пели хвалу пред Господом
славы.
Первый голос прославляет Господа духов от века и до века.
И другой голос слышал я, прославляет Избранного и избранных, которые
взвешены Господом духов.
И третий голос слышал я, просит, и молится за живущих на земле, и
умоляет во имя Господа духов.
И слышал я четвёртый голос, как он отражал врагов и не дозволял им
приступить к Господу духов, чтобы клеветать или жаловаться на живущих на земле.
После этого я спросил ангела мира, шедшего со мною, который показал
мне всё, что сокрыто, и сказал ему: "Кто эти четыре лица, которых я видел и
глас, которых я слышал и записал?"
И он сказал мне: "Этот первый~--- есть милосердный и долготерпеливый
святой Михаил; и другой, поставленный над всеми болезнями и над всеми ранами
сынов человеческих, есть Руфаил; и третий, поставленный над всеми силами, есть
святой Гавриил; и четвёртый, поставленный над покаянием и надеждою тех,
которые получат в наследие вечную жизнь, есть Фануил".
И вот четыре ангела всевышнего Бога, и четыре голоса их я слышал в те
дни.
И после этого я видел все тайны неба, и как разделено царство, и
как дела людей взвешены на весах.
Там видел я жилища избранных и жилища святых; и мои очи видели там,
как изгоняются оттуда все грешники, которые отвергли имя Господа духов, как
отражают их, и для них там нет места вследствие наказания, которое исходит от
Господа духов.
И там мои очи видели тайны молний и грома, и тайны ветров, как они
распределены,  чтобы дуть на землю, и тайны туч и росы; и там видел я,
откуда они выходят в том самом месте и как отсюда насыщается пыль
земная.
И там видел я замкнутые хранилища, из которых распределяются ветра,
и хранилища града, и хранилища тумана и туч, и Его туча, которая носится над
землёю до вечности.
И я видел хранилища Солнца и Луны, откуда они выходят и куда
возвращаются, и их славное возвращение; и я видел, как одно (т.е. Солнце) имеет
преимущество перед другой, видел и их определённое движение, как они не
преступают пути, ничего не прибавляя к своему пути и ничего не убавляя от него,
и соблюдают верность между собой, сохраняя клятву.
Прежде всего, выходит Солнце и совершает свой путь по Вселенной
Господа духов, и могущественно имя Его от века до века; за ним следует видимый
и невидимый путь Луны; и я видел, как она оканчивает движение по своему пути в
том месте днём и ночью, одно (светило, т.е. Луна), противостоя другому
(Солнцу), пред Господом духов; и они благодарят, и прославляют, и не
успокаиваются, так как их благодарение служит для них покоем.
Ибо сияющее Солнце совершает много обращений для благословения и для
проклятия; и движение Луны по её пути есть свет для праведных и для грешников
во имя Господа, Который положил разделение между светом и тьмою, и разделил
души людей, и утвердил души праведных во имя Своей правды.
Ибо ни ангел не нарушает этого, и никакая сила не может нарушить
этого (установленного Богом), но Судья видит их все (души людей) и судит их все
пред Собою.
Мудрость не нашла на земле места, где бы ей жить, и потому
жилище её стало на небесах.
Пришла мудрость, чтобы жить между сынами человеческими, не нашла себе
места; тогда мудрость возвратилась назад в своё место, и заняла своё положение
между ангелами.
И неправда вышла из своих хранилищ: не искавшая его (приёма), она
нашла его и жила между людьми, как дождь в пустыне и как роса в земле жаждущей.
И видел я опять молнии и звёзды небесные, как Он призывал их
всех отдельно по именам и они внимали Ему.
И я видел, как они взвешены правильными весами по мере их света, по
обширности их мест и времени их появления и обращения (видел, как одна молния
рождает другую), и их обращение по числу ангелов, и как они сохраняют между
собой верность.
И я спросил ангела, который шёл со мною и показал мне, что сокрыто:
"Кто это"?
И он сказал мне: "Образ их показал тебе Господь духов: это имена
праведных, которые живут на земле и веруют во имя Господа духов во всю
вечность".
И иное также видел я относительно молний, как они возникают из
звёзд, и становятся молниями, и ничего не могут удержать при себе.
\vs 1En 8:1
И  вот вторая притча относительно тех, которые отвергают имя
жилища святых и имя Господа духов.
Они не взойдут на небо, и на землю не придут они: таков будет жребий
грешников, которые отвергают имя Господа духов и которые сохраняются, таким
образом, на день страданий и скорби.
В тот день Избранный сядет на престоле славы и произведёт выбор между
делами их (людей) и местами без числа, и дух их сделается сильным в их
внутренности, ибо они увидят моего Избранного и тех, которые умаляли Моё святое
и славное имя.
И в тот день Я пошлю Моего Избранного жить между ними, и преобразую
небо, и приготовлю его для вечного благословения и света.
И я изменю землю, и приготовлю её для благословения,  и поселю  на
ней Моих избранных; грех же и преступления исчезнут на ней,~--- они не появятся.
Ибо Я увидел и насытил миром Моих праведных, и поставил их пред Собою;
для грешников же у Меня предстоит суд, дабы уничтожить их с лица земли.
И там я видел Единого, имевшего главу дней (престарелую  главу), и
Его глава была бела, как руно; и при Нём был другой, лице которого было подобно
виду человека, и Его лице полно было прелести и подобно одному из святых
ангелов.
И я спросил одного из ангелов, который шёл со мною и показывал мне все
сокровенные вещи, о том Сыне человеческом, кто Он, и откуда Он, и почему Он
идёт с Главою дней?
И он отвечал мне и сказал: "Это Сын человеческий, Который имеет правду,
при Котором живёт правда, и Который открывает все сокровища того, что сокрыто,
ибо Господь духов избрал Его, и жребий Его пред Господом духов превзошёл всё,
благодаря праведности.
И этот Сын человеческий, Которого ты видел, поднимет царей и
могущественных с их лож и сильных с их престолов, и развяжет узы сильных, и
зубы грешников сокрушит.
И Он изгонит царей с их престолов и из их царств, ибо они не
превозносят Его, и не прославляют Его, и не признают с благодарностью, откуда
досталось им царство.
И лицо сильных Он отвергнет, и краска стыда покроет их; мрак будет их
жилищем, и слёзы их ложем, и они не будут иметь надежды встать со своих лож,
так как они не превозносят имя Господа духов.
И это те, которые осуждают звёзды небесные и возвышают свои руки
против Всевышнего, и попирают землю и на ней живут; все дела их неправда, и они
открывают неправду; сила их основывается на богатстве, и вера их относится к
богам, сделанным их же руками; и они отвергли Господа духов.
И они изгоняются из домов их общественного собрания и из домов
верующих, которые взвешены во имя Господа духов.
И в те дни восходит молитва праведных и кровь праведного от земли
к Господу духов.
В те дни святые ангелы, живущие вверху на небесах, соединившись
вместе,
будут единым гласом просить, и молить, и прославлять, и благодарить, и
восхвалять имя Господа духов ради крови праведных, которая пролита, и ради
молитв праведных, что она не может быть тщетной пред Господом духов и что
совершён суд для них, и им не нужно терпеть (или дожидаться суда) вечного.
И в те дни я видел Главу дней как Он восседал на престоле своей славы
и книги живых были раскрыты пред Ним, и видели всё Его воинство, которое
находится вверху и на небесах и окружает Его, предстоя пред Ним.
И сердца святых были полны радостью, ибо исполнилось число правды, и
молитвы праведных услышана, и кровь праведного искуплена (или отомщена) пред
Господом духов.
И в том месте я видел источник правды, который был неисчерпаем;
его окружали вокруг многие источники мудрости, и все жаждущие пили из них и
исполнялись мудростью, и имели свои жилища около праведных, и святых, и
избранных.
И в тот час был назван тот Сын человеческий возле Господа духов и Его
имя пред Главою дней.
И прежде чем Солнце и знамения были сотворены, прежде чем звёзды
небесные были созданы, Его имя было названо пред Господом духов.
Он будет жезлом для праведных и святых, чтобы они опёрлись на Него и
не падали; и Он будет светом народов и чаянием тех, которые опечалены в своём
сердце.
Пред Ним упадут и поклонятся все живущие на земле, и будут хвалить и
прославлять, и петь хвалу имени Господа духов.
И посему Он был избран и сокрыт пред Ним, прежде даже чем создан мир;
и Он будет пред Ним до вечности.
И премудрость Господа духов открыла Его святым и избранным, ибо Он
охраняет жребий праведных, так как они возненавидели и презрели этот мир
неправды, и все его произведения и пути возненавидели во имя Господа духов; ибо
во имя Его они спасаются, и Он становится мстителем за их жизнь.
И в те дни потупили взор цари земли и сильные, владеющие твердью,
страшась за дела своих рук, ибо в день своей печали и бедствия они не спасут
своих душ.
И Я предал их в руки Моих избранных: как солома в огне и как свинец в
воде, они сгорят пред лицом праведных и потонут пред лицом святых, и никакого
следа более не останется от них.
И в день их бедствия водворится покой на земле; они падут пред Ним и
не восстанут опять; не будет никого, кто бы взял их в свои руки и поднял: ибо
они отвергли Господа духов и Его Помазанника.
Имя Господа духов будет прославлено.
Ибо мудрость излилась на Сына человеческого, как вода, и слава не
прекращается пред Ним от века до века.
Ибо Он силён во всех тайнах правды, и неправда прейдёт пред Ним, как
тень, и не будет иметь постоянства, так как Избранный восстал пред Господом
духов; и Его слава от века до века, и Его могущество от рода до рода.
В Нём живёт дух мудрости и дух Того, Кто даёт проницательность, и дух
учения и силы, и дух тех, которые почили в правде.
И Он будет судить сокровенные вещи, и никто не осмелится вести пред
Ним пустую речь, ибо Он избран пред Господом духов, и Его благоволению.
И в те дни совершится перемена со святыми и избранными: свет дней
будет обитать пред ними, и слава, и честь будут дарованы святым.
И в день бедствия соберётся нечестие на грешников, праведные же
победят во имя Господа духов; и Он покажет это другим, чтобы они принесли
покаяние и оставили дела своих рук.
Они не будут иметь чести пред Господом духов, но будут спасены во имя
Его; И Господь духов умилосердится над ними, ибо Его милосердие велико.
И праведен Он в Своём суде и пред Его славою, и на Его суде не устоит
неправда: кто не принесёт покаяния пред Ним, тот погибнет.
Но отныне Я не буду более милосердным к ним, говорит Господь духов.
И в те дни земля возвратит вверенное ей и царство мёртвых
возвратит вверенное ему, что оно получило, и преисподняя отдаст назад то, что
обязана отдать.
И Он изберёт между ними праведных и святых, ибо пришёл день, чтобы
спастись им.
И Избранный в те дни сядет на престоле Своём, и все тайны мудрости
будут истекать из мыслей Его уст, ибо Господь духов даровал Ему это и прославил
Его.
И в те дни горы будут скакать, как овны, и холмы будут прыгать, как
агнцы, насытившиеся молоком; и все они сделаются ангелами на небе.
Их лицо будет сиять от радости, так как в те дни восстанет Избранный;
и земля возрадуется, и на ней будут жить праведные, и избранные будут ходить и
шествовать по ней.
И после тех дней, в том месте, где я видел все видения
относительно того, что сокрыто,~--- я был восхищён в вихре ветра и приведён к
западу,~--- там очи мои видели сокровенные предметы неба, всё, что произойдёт на
земле: одну гору из железа, одну из меди, одну из серебра, одну из золота,
одну из жидкого металла и одну из свинца.
И я спросил ангела, который шёл со мною, говоря: "Что это за предметы,
которые я видел в сокровенном месте?"
И он сказал мне: "Все эти предметы, которые ты видел, служат
владычеству Его Помазанника, дабы Он был сильным и могущественным на земле".
И отвечал мне тот ангел мира, говоря: "Подожди немного, тогда ты
увидишь и тебе будет открыто всё, что сокровенно и что насадил Господь духов.
И те горы, которые ты видел: гора из железа, гора из меди, гора из
серебра, гора из золота, гора из жидкого металла и гора из свинца~--- все они
будут пред Избранным, как сотовый мёд пред огнём и как та вода, которая стекает
сверху на эти горы, и они окажутся слабыми под Его ногами.
И случится в те дни, что нельзя будет спастись ни золотом, ни
серебром: нельзя будет тогда ни спастись, ни убежать.
И не будет дано тогда для битвы ни железа, ни панцирной одежды; руда
не будет пригодна ни на что, и олово не будет пригодным ни на что и не пойдёт в
прок, и свинец не будет добываться.
Все эти вещи исчезнут и уничтожатся с поверхности земли, когда
появится Избранный пред лицом Господа духов.
И там мои очи видели глубокую долину, устье которой было открыто;
и все живущие на тверди, и в море, и на островах принесут Ему дары, и подарки,
и знаки верности, но та глубокая долина не наполнится.
И они совершают преступление своими руками, и всё, что они, грешники,
добывают, то преступным образом пожирают сами, так они, грешники, погибнут пред
лицом Господа, и будут изгнаны с лица Его земли без прекращения на всю
вечность.
Ибо я видел ангелов наказания, как они шли и готовили сатане все
орудия.
И я спросил ангела мира, шедшего со мною: "Те орудия,~--- для кого они
их готовят?"
И он сказал мне: "Они готовят их для царей и для сильных земли сей,
чтобы уничтожить их чрез это.
И после этого Праведный и Избранный откроет дом Своего общественного
собрания, которое отныне не должно быть стесняемо, во имя Господа духов.
И эти горы будут пред Его лицом, как земля и холмы будут, как водный
источник; и праведники будут иметь покой при унижении грешников".
И я взглянул и обратился к другой стороне земли, и увидел там
глубокую долину с пылающим огнём.
И они (ангелы наказания) принесли царей и сильных и положили их в
глубокую долину.
И там мои очи видели, как сделали для них орудия,~--- железные цепи
безмерного веса.
И я спросил ангела мира, говоря: "Эти цепи-орудия,~--- для кого они
приготовлены?"
И он сказал мне: "Они приготовлены для отрядов Азазела, чтобы взять их
и бросить в преисподний ад: и челюсти их будут покрыты грубыми камнями, как
повелел Господь духов.
Михаил и Гавриил, Руфаил и Фануил схватят их в тот великий день суда и
бросят в этот день в печь с пылающим огнём, дабы Господь духов отмстил им за их
неправду,~--- за то, что они покорились сатане и прельстили живущих на земле.
И в те дни наступит осуждение Господа духов, и откроется хранилище
вод, которые сверху на небесах, и, кроме них, те источники, которые под
небесами и внизу в земле.
И все воды на земле соединятся с водами, которые в верху на небесах;
вода же, которая вверху на небе, есть мужская, и вода, которая внизу на земле,
есть женская.
И тогда будут уничтожены все, которые живут на земле и которые живут
между пределами неба.
И чрез это они узнают всю неправду, которую они совершили на земле и
за которую погибают.
И после этого раскаялся Глава дней и сказал: "Напрасно Я погубил
всех живущих на земле".
И Он поклялся Своим великим именем: "Отныне Я не буду более поступать
так с живущими на земле; и Я положу знамение на небе: оно будет залогом между
Мною и ими до вечности, пока существует небо над землёю.
И тогда произойдёт по Моему повелению: когда Я в Моём гневе и в Моём
осуждении решу схватить их рукою ангелов в день скорби и печали, то Мой гнев и
Моё осуждение будут оставаться над ними навсегда,~--- говорит Бог, Господь духов.
Вы, могущественные цари, которые будете жить на земле, вы должны
увидеть Моего Избранного, как Он сидит на престоле Моей славы и судит Азазела,
и всё его сообщество, и все его отряды, во имя Господа духов".
И я видел там воинство идущих ангелов наказания, которые держали
верёвки из железа и руды.
И я спросил ангела мира, шедшего со мною, говоря: "К кому идут те,
которые держат верёвки?"
И он сказал мне: "Каждый к своим избранным и возлюбленным, чтобы
бросить их в глубокую пропасть долины.
И тот час та долина наполниться избранными и возлюбленными, и день их
жизни окончится, и день их обольщения не будет с тех пор более считаться".
И в те дни соберутся ангелы, и их начальники направятся к востоку к
Парфеянам и Мидянам~--- приготовить там возмущение между царями, чтобы нашёл на
них дух возмущения; и они поднимутся со своих престолов, чтобы выступить в
середину их стада, как львы из своих логовищ и как голодные волки.
И они поднимутся и обступят землю их избранных, и земля Его избранных
будет пред ними гумном и тропою.
Но город Моих праведных будет преградой для их коней; и они начнут
борьбу друг с другом, и их правая рука будет сильна против них самих, и никто
не будет знать своего ближнего и брата, ни сын своего отца и своей матери, пока
не будет достаточно трупов вследствие из смерти, и осуждение над ними не будет
тщетным.
И в те дни царство мёртвых откроет свою пасть, и они будут опущены в
него; и вот их погибель: царство мёртвых поглотит грешников пред лицом
избранных.
И случилось после этого: там опять я увидел отряд колесниц, на
которых ехали люди, и они шли на крыльях ветра от восхода и захода к полудню.
И был слышен шум их колесниц; и как только это смятение произошло,
святые ангелы заметили это с неба; и столпы земли подвинулись со своих мест, и
это было слышно от пределов земли до пределов неба, в один день.
И они все упадут и поклонятся Господу духов.
И это конец второй притчи.
\vs 1En 9:1
И я начал говорить третью притчу о праведных и избранных.
Будьте блаженными вы, праведные и избранные, ибо жребий ваш будет
славен!
И праведные будут жить в свете солнца и избранные в свете вечной жизни;
дни вечной жизни их не кончаются, и дни святых бесчисленны.
И они будут искать света и обретут правду у Господа духов: мир будут
иметь праведные у Господа мира.
И после этого будет сказано святым, чтобы они искали на небе тайны
справедливости и наследие веры, ибо оно стало ясно, как сияние солнца на земле,
и мрак исчез.
И не прекращаемый свет будет существовать, и дни, в которые они будут
жить, бесчисленны, ибо мрак заранее будет уничтожен, и силен будет свет пред
Господом духов, и свет праведности будет силен во век пред Господом духов.
И в те дни очи мои видели тайны молний и массы света, и их правду;
и они блестят для благословения и для проклятия, как желает этого Господь
духов.
И там я видел тайны грома, и слышал, как раздаётся глас его, когда он
гремит вверху на небе, и они (ангелы проводники) показали мне места жилищ на
земле и глас грома, как он служит для благополучия и благословения или для
проклятия, по слову Господа духов.
И после этого мне были показаны все тайны масс света и молний, как они
блестят для благословения и для насыщения.
\vs 1En 10:1
В пятисотый год, в седьмой месяц, в четырнадцатый день месяца
жизни Еноха.
В той притче я видел, как небо небес поколебалось от сильного трепета,
и воинство Всевышнего, и тысячи тысяч и тьмы тем ангелов были потрясены
вследствие сильного волнения.
И тот час я увидел Главу дней, сидящего на престоле Своей славы, и
ангелов и праведных, стоящих вокруг Него.
И меня объял сильный трепет, и страх охватил меня; моё бедро согнулось
и ослабело, всё моё существо сплавилось, и я упал на своё лицо.
Тогда святой Михаил послал другого святого ангела~--- одного из святых
ангелов~--- и он поднял меня; и как только он меня поднял, мой дух обратился
назад, ибо я не мог вынести вида этого воинства, и колебания и трепета неба.
И сказал мне святой Михаил: "что за вид так взволновал тебя?
До сего дня был день Его милосердия, ибо Он был милосерден и
долготерпелив к населяющим почву земную.
Но вот придет день, и власть, и наказание, и суд, что приготовил
Господь духов для тех, которые преклоняются пред праведным судом, и для тех,
которые отвергают праведный суд, и для тех, которые напрасно употребляют Его
имя; и тот день будет для избранных защитою, а для грешников расследованием.
И в тот день будут распределены два чудовища: женское чудовище,
называемое Левияфа, чтобы оно жило в бездне моря над источниками вод, мужеское
же называется Бегемотом, который своею грудью занимает необитаемую пустыню,
называемую Дендаин, находящуюся на востоке сада, где живут избранные и
праведные и куда взят мой дед, седьмой от Адама~--- первого человека, которого
сотворил Господь духов.
И я молил того другого ангела, чтобы он показал мне власть тех
чудовищ, как они разделены в один день, и одно было поставлено в глубину моря,
а другое на твердую почву пустыни.
И он сказал мне: "ты, сын человеческий,~--- ты добиваешься здесь узнать,
что сокрыто".
И сказал мне другой ангел, который шел со мною и показал мне, что
находится в сокровенных местах, первое и последнее, что на небе в высоте и на
земле в глубине, и что на пределах неба, и в хранилищах при основании неба, и в
хранилищах ветров; и он показал, как распределены духи, и как возвышаются
(явления в природе), и как исчислены источники и ветры по силе духа, и какова
сила лунного света, и как все это есть сила правды, и (показал) отделения звезд
по их именам, и как все отделения разделены; и он показал громы по местам их
падения, и все отделения; которые сделаны между молниям, чтобы они сверкали и
их отряды тотчас бы повиновались (следовали за ними); ибо гром имеет места
отдыха и ему определено выжидать свой удар; и они оба~--- гром и молния~---
неотделимы; и хотя они не одно, однако оба чрез посредство духа идут вместе и
не разделяются.
Ибо, когда сверкает молния, то и гром дает свой глас, и дух
задерживает во время удара и одинаково делает разделение между ними; ибо запас
их ударов, как песок, и каждый в отдельности из них удерживается при своем
ударе уздою, и силою духа они возвращаются назад, и таким образом посылаются
далее соразмерно с множеством стран земли.
И дух моря есть мужественный и сильный; и соразмерно с крепостью своей
силы он притягивает его (море) назад уздою; и таким образом оно перегоняется
вперед и разливается во все горы земли.
И дух инея есть его (собственный, особенный) ангел, и дух града есть
добрый ангел.
И духа снега Он назначил ради его силы, и он (снег) имеет особенного
духа; и то, что поднимается из него, есть как бы дым и его имя мороз.
Но дух облака не соединён с ними (духами инея, града и снега) в их
хранилищах, а имеет особое хранилище; ибо его движение бывает при ясности и
свете и при мраке, и зимой и летом, и его хранилище есть свет; и он (дух
облака) есть его ангел.
И дух росы имеет свое жилище на пределах неба, и оно связано с
хранилищем дождя, и его движение бывает зимою и летом; и его тучи и тучи
дождевого облака находятся в связи и сообщаются друг с другом.
И когда дух дождя выходит из своего хранилища, приходят ангелы, и
открывают хранилище, и выпускают его, и тогда он рассевается по всей суше и
таким образом соединяется с водою на земле.
Ибо воды существуют для живущих на земле, так как они составляют пищу
для земли от Всевышнего, Который существует на небе; посему дождь имеет меру,
и ангелы владеют им.
Я видел все эти вещи вплоть до сада праведных.
И ангел мира, который был со мною, сказал мне: "эти два чудовища
приготовлены сообразно с величием Божиим для того, чтобы быть накормленными,
дабы осуждение Божие не было тщетным; и будут умерщвлены сыны со своими
матерями и дети со своими отцами.
Когда осуждение Господа духов будет пребывать над ними, то будет
пребывать для того, чтобы осуждение Господа духов не сделалось тщетным по
отношению к ним; после этого будет суд по Его милосердию и терпению.
И я видел в те самые дни, как даны были тем ангелам длинные
веревки, и они подняли крылья и полетели, и достигли севера.
И я спросил ангела, говоря: "для чего они держали те длинные веревки и
удалились?"
И он сказал мне: "они ушли, чтобы измерять".
И ангел, шедший со мною, сказал мне: "они несут меры праведных и
канаты праведных, чтобы они опирались на имя Господа духов навсегда и навеки.
И начнут и будут жить избранные с избранными, и эти меры будут даны
вере и будут укреплять слова правды.
И эти меры откроют всё сокровенное в глубине земли, и погибших по
пустыням, и пожранных рыбами морскими и зверями, чтобы они возвратились и
оперлись на день Избранного; ибо никто не погибнет пред Господом духов, и никто
не может погибнуть.
И сохранили повеления все те, которые вверху на небе, и одна сила,
один голос и один свет, подобный огню, был дан им.
И Того, прежде всего, прославили, и возвеличили, и восхвалили они с
мудростью, и показали себя мудрыми в слове и духе жизни.
И Господь духов посадил Избранного на престол Своей славы, и Он будет
судить все деяния святых ангелов на небе и взвесит их поступки на весах.
И когда Он поднимает Свое лицо, чтобы судить их сокрытые пути по слову
имени Господа духов и их стезю по пути праведного суда всевышнего Бога, тогда
все они возглаголят одним гласом, и прославят, и восхвалят, и вознесут, и будут
хвалить имя Господа духов.
И будет взывать все воинство небесное и все святые, которые вверху, и
воинство Божие,~--- херувимы и серафимы, и офанимы, и все ангелы власти, и все
ангелы господства, и Избранный, и другие силы, которые на тверди и над водою,~---
все они будут взывать в тот день и будут возносить одним гласом, и прославлять,
и восхвалят, и хвалить, и превозносить в духе веры, и в духе мудрости и
терпения, и в духе милосердия, и в духе правды и мира, и в духе благости; и
будут все говорить одним гласом: "славь Его, и да будет прославлено имя Господа
духов во веки и до века!"
Его будут хвалить все, которые не спят вверху на небе; Его будут
прославлять все Его святые, которые на небе, и все избранные, живущие в саду
жизни, и каждый дух света, способный прославлять и восхвалять, и превозносить,
и святить Твое имя, и всякая плоть, которая будет чрезмерно прославлять и
восхвалять Твое имя вовеки.
Ибо велико милосердие Господа духов, и Он долготерпелив, и все Свои
творения и всю Свою силу,~--- так много Он сотворил,~--- Он открыл праведным и
избранным, во имя Господа духов.
И Господь духов так повелел царям, и сильным, и вознесенным, и
населяющим землю, и сказал: "откройте свои глаза и вознесите ваши роги, ибо вы
можете узнать Избранного!"
И Господь духов сел на престол славы, и дух правды изливался на Него,
и слово уст Его умертвило всех грешников и всех неправедных, и они погибли
перед лицом Его.
И будут стоять в тот день все цари, и сильные, и вознесенные, и
владеющие твердью, и увидят Его и узнают, как Он сидит на престоле Своей славы,
и пред Ним судятся праведные в правде и никакая пустая речь не говорится пред
Ним.
Тогда постигнет их боль, как жену, которая в родильных потугах и
которой трудно бывает родить, когда ее сын входит в проход утробы, и которая
имеет боли при родах.
И одна часть из них будет смотреть на другую, и они устрашатся и
потупят свой взор, и боль обоймёт их, когда они увидят того Сына жены, сидящим
на престоле Своей славы.
И цари, и сильные, и все владеющие землею будут восхвалять, и
прославлять, и превозносить Владычествующего над всем, Который был сокрыт.
Ибо прежде Сын человеческий был сокрыт, и Всевышний сохранял Его пред
Своим могуществом, и открыл Его избранным; и будет посеяно общество святых и
избранных, и будут стоять пред Ним в тот день все избранные.
И все могущественные цари, и вознесенные, и господствующие над
твердью, упадут пред Ним на свое лицо, и поклонятся, и возложат на того Сына
человеческого свою надежду, и будут умолять Его и просить у Него милосердия.
И Господь духов будет теперь теснить их, чтобы они немедленно
удалились прочь от Его лица; и их лица исполнятся стыдом, и мрак соберется на
них.
И ангелы наказания возьмут их, чтобы совершить над ними возмездие за
то, что они притеснили Его детей и избранных.
И они сделаются зрелищем для праведных и избранных Его: они
(праведные) будут радоваться, взирая на них, ибо гнев Господа духов будет
пребывать на них, и меч Господа духов упьется ими.
И праведные и избранные будут спасены в тот день, и не будут более
видеть отныне лица грешников и неправедных.
И Господь духов будет обитать над ними, и они будут жить вместе с тем
Сыном человеческим, и есть, и ложиться, и вставать, от века до века.
И праведные и избранные будут вознесены от земли, и перестанут
опускать свой взор, и будут облечены в одежду жизни.
И это будет одежда жизни у Господа духов.
В те дни могущественные цари, владеющие твердью, будут вымаливать
у Его ангелов наказания, которым они преданы,~--- даровать им немного успокоения,
и просить, чтобы им можно было пасть ниц перед Господом духов и поклониться, и
сознаться перед Ним в своих грехах.
И они будут прославлять и восхвалять Господа духов, и говорить: "да
будет прославлен Он, Господь духов и Господь царей, Господь сильных и Господь
властителей, Господь славы и Господь мудрости, пред которым всякая тайна ясна.
И Твое могущество от рода до рода, и Твоя слава от века до века;
глубоки все Твои тайны и бесчисленны, и слава Твоя неисчислима.
Теперь узнали мы, что нам нужно восхвалять и прославлять Господа царей
и Того, Кто царь над всеми царями".
И они скажут: "о, если бы нам дали успокоение, чтобы мы восхвалили
Его, и возблагодарили Его, и прославили Его, и уверовали пред Его славой!
И теперь мы домогаемся небольшого успокоения, но не находим его: мы
прогнаны, и не получили его, свет исчез пред нами, и мрак служит нашим жилищем
навсегда и навеки.
Ибо мы не уверовали в Него, и не восхвалили имя Господа царей за
всякое Его дело, и наша надежда была бы на скипетр нашего владычества и на наше
величие.
И в тот день нашего страдания и нашей печали Он не спасет нас, и мы не
найдем успокоения, дабы уверовать, что Господь наш истинен во всяком Своем
деле, и во всех Своих судах, и в Своей правде, и суды Его не лицеприятны.
И мы погибнем пред Его лицом за свои дела, и все грехи наши исчислены
по справедливости".
Теперь они скажут себе: "душа наша насытилась неправедным стяжанием,
но оно не отвратит того, что мы будем низвергнуты в пламя адского мучения".
И после этого их лицо исполнится мраком перед тем Сыном человеческим и
они будут отвергнуты от Его лица, и меч будет жить между ними перед Его лицом.
И Господь духов так сказал: "вот повеление и суд над сильными, и
царями, и вознесёнными, и владеющими твердью, пред Господом духов".
Также и другие виды я видел в том сокровенном месте.
Я слышал глас ангела, как он сказал: "это ангелы, которые сошли с неба
на землю и открыли сынам человеческим то, что было сокрыто, и соблазнили сынов
человеческих совершать грехи".
\vs 1En 11:1
И в те дни Ной увидел землю, как она согнулась, и ее гибель была
близка.
И он направил оттуда свои стоны и пришел к пределам земли, и вскликнул
к своему деду Еноху; и Ной трижды сказал опечаленным голосом: "послушай меня,
послушай меня, послушай меня!"
И он (Ной) сказал ему: "скажи мне, что это такое происходит на земле
что земля так ослабела и поколебалась?
как бы я не погиб вместе с нею!"
И после этого мгновения было великое колебание на земле, и голос был
слышен с неба, и я упал на свое лицо.
И пришел мой дед Енох, и встал около меня, и сказал мне: "Почему ты
восклицал ко мне опечаленным криком и плачем?
От лица Господа вышло повеление относительно живущих на тверди, что
должен наступить их конец, так как они знают все тайны ангелов, и всю власть
дьяволов, и всю их сокровенную силу, и всю силу тех, которые совершают
волшебства, и силу заклинаний, и силу тех, которые льют для всей земли
изображения идолов; и хорошо также знают, как серебро производится из праха
земли, и как жидкий металл образуется на земле.
Ибо свинец и олово не так производятся из земли, как первое (серебро):
существует особый источник, производящий их, и ангел, стоящий в нем, и он
преимущественно тот ангел".
И после этого дед Енох обнял меня своей рукою, поднял меня и сказал
мне: "иди, ибо я спрашивал Господа духов об этом колебании на земле.
И он сказал мне: за их нечестие над ними совершен суд, и он уже не
вычисляется предо Мною ради месяцев, которые они расследовали и через это
узнали, что земля и живущие на ней погибнут.
И для них (ангелов) не будет убежища вовеки, так как они показали им
(людям) то, что сокрыто, и они осуждены; но не так ты, мой сын: Господь духов
знает, что ты чист и свободен от этой укоризны за тайны.
И Он утвердил твое имя между святыми, и сохранит тебя между живущими
на тверди; и Он определил в правде твое семя для царей и для великой славы, и
из твоего семени произойдет источник праведных и святых без числа во веки".
И после этого он показал мне ангелов наказания, готовых идти и
выпустить все силы воды, которая внизу на земле, чтобы принести суд и погибель
всем, покоящимся и живущим на тверди.
И Господь духов дал повеление ангелам, вышедшим теперь, чтобы они не
простирали рук, а дожидались: ибо те ангелы были поставлены над силами вод.
И я удалился от лица Еноха.
И в те дни было слово Господа ко мне, и Он сказал мне: "Ной!
вот твой жребий предстал предо Мною, жребий без порока, жребий любви
и милосердия.
И теперь ангелы делают деревянное здание; и так как они вышли на это
дело, то и Я приложу к нему Свою руку и буду охранять его (ковчег); и выйдет из
него семя жизни, и земля должна подвергнуться превращению, чтобы ей не остаться
пустою.
И Я укреплю твое семя предо Мною на всю вечность, и живущие с тобою
распространятся по поверхности земли, и оно (семя) будет благословенно и
умножится на земле во имя Господа".
И они заключат тех ангелов, показавших неправду, в ту пылающую долину
на западе, которую показал мне прежде дед Енох, возле гор золота, и серебра, и
железа, и жидкого металла, и свинца.
И я видел ту долину, в которой было великое колебание и волнение вод.
И когда всё это случилось, то из той огненной металлической лавы и от
колебания, которое их (воды) колебало, в том месте (в долине) явился серный
запах, и он соединился с теми водами; и та долина ангелов, которые прельстили
людей, разгоралась все далее под тою землею.
И через долины этой самой земли проходят реки огня,~--- именно там, где
осуждены пребывать те ангелы, которые соблазнили живущих на тверди.
Но те воды будут служить в те дни для царей, и сильных, и вознесённых,
и для живущих на тверди к исцелению души и тела и к наказанию духа, так как
дух их исполнен сладострастием, чтобы они были наказаны со своим телом, ибо
они отвергли Господа духов; и они изо дня в день видят свое будущее наказание и
однако не веруют в Его имя.
И в той самой мере, насколько становятся сильными жар их тела, будет
происходить изменение и в их духе (от века до века), ибо не может быть сказано
пред Господом духов пустое слово.
Ибо придет суд на них, так как они веруют в сладострастии своего тела
и отвергают дух Господа.
И те воды сами в те дни претерпят изменение: ибо, когда те ангелы
будут наказаны в те дни, будет изменяться жар тех водных источников, и когда
ангелы будут подниматься, та вода источников будет изменяться и охлаждаться.
И я слышал святого Михаила, когда он отвечал и говорил: "этот суд,
которым осуждены ангелы, есть свидетельство для царей, и сильных, и владеющих
твердью.
Ибо эти воды суда служат к исцелению ангелов и для смерти их тела; но
они (владыки) не увидят того и не уверуют, что те воды изменятся и превратятся
в огонь, который горит вовек".
И после этого мой дед Енох дал мне в книге знамения всех тайн и
притч, которые ему были даны, и собрал их для меня в словах книги притчей.
И в тот день отвечал святой Михаил Руфаилу, говоря: "сила духа увлекает
меня и возбуждает меня, и строгость суда тайн, суда над ангелами, поражает
меня; кто может вынести строгость суда, который совершен и до сих пор пребывает
и от которого они распаляются?"
И опять отвечал и сказал святой Михаил Руфаилу: "есть ли кто такой,
который не размягчился бы сердцем и почки которого не содрогнулись бы от этого
слова?
суд вышел относительно них, относительно тех, которых выгнали они
таким образом".
И случилось, когда святой Михаил стоял пред Господом духов, то он
сказал Руфаилу так: "и я не буду представительствовать за них пред очами
Господа, ибо Господь духов разгневался на них, потому что они действуют так,
как если бы были равны Богу.
Посему на них грядет суд, который сокрыт, от века до века; ибо ни
ангел, ни человек не получат своей доли, но только они получат свой суд, от
века до века".
И после этого суда они навлекут на них гнев и ярость, так как они
показали это живущим на тверди.
И вот имена тех ангелов, и эти имена их: первый из них Семъяйза,
второй Арестикифа, третий Армен, четвертый Кокабаел, пятый Тураел, шестой
Румъйял, седьмой Данел, восьмой Нукаел, девятый Баракел, десятый Азазел:
одиннадцатый Армерс, двенадцатый Батаръйял, тринадцатый Базазаел, четырнадцатый
Ананел, пятнадцатый Турхйял, шестнадцатый Симанизиел, семнадцатый Иетарел,
восемнадцатый Тумаел, девятнадцатый Тарел, двадцатый Румаел, двадцать первый
Изезеел.
И это главы их ангелов и имена их предводителей над сотнею,
пятьюдесятью и десятью.
Имя первому Иекун; это тот, который соблазнил всех детей святых
ангелов, и свел их на землю, и соблазнил их чрез дочерей человеческих.
И имя другому Асбеел: этот внушил детям святых ангелов злой совет, и
соблазнил их, чтобы они осквернили свои тела с дочерьми человеческими.
И имя третьему Гадрел: это тот, который показал сынам человеческим все
смертоносные удары, и он соблазнил Еву, и показал сынам человеческим орудия
смерти, и панцырь, и щит, и меч для битвы, и показал сынам человеческим все
орудия смерти.
И из его руки они перешли к живущим на тверди, от того часа до века.
И имя четвертому Пенемуэ: этот показал сынам человеческим горькое и
сладкое, и показал им все тайны их мудрости.
Он научил людей письму чернилами и употреблению бумаги, и чрез это
многие согрешили от века до века и до сего дня.
Ибо люди сотворены не для того, чтобы они, таким образом, тростью и
чернилами закрепляли свою верность (свое слово).
Ибо люди сотворены не иначе, чем ангелы, чтобы им пребывать праведными
и чистыми, и смерть, которая губит всех, не касалась бы их, но они погибают
чрез это свое знание, и чрез эту силу она пожирает меня.
И имя пятому Касдейя: этот показал людям все злые удары духов и
демонов, и улары рождения в утробе матери, дабы устранить его, и удары души,
укушения змей, и удары, случающиеся в полдень,~--- сына змеи, именуемого Табает.
И это число Кесбеела, который показал святым главу клятвы, когда он
жил высоко вверху во славе, и имя ее (клятвы) Бека.
И этот ангел сказал святому Михаилу, чтобы он показал им сокровенное
имя Божие, дабы они видели то сокровенное имя и упоминали его при клятве, чтобы
содрогались пред тем именем и клятвой те, которые показали сынам человеческим
все, что было сокрыто.
И такова сила той клятвы, ибо она сильна и могущественна, и Он положил
эту клятву Акаэ в руку святого Михаила.
И таковы тайны клятвы, и они (тайны мира) утверждены чрез его клятву,
и силою его небо повешено, прежде чем был создан мир, и до века.
И чрез нее была основана земля на воде, и силою ее
выходит из сокровищ гор прекрасная вода для живущих от сотворения мира
до века.
И чрез ту клятву было сотворено море, и, как его основание, Он положил
ему на время ярости песок, и оно не должно преступать его от сотворения мира до
века.
И чрез ту клятву основания земли утверждены, и стоят и не движутся со
своего места от века до века.
И чрез ту клятву совершают свое движение солнце и луна, и не отступают
от предписанного им от века до века.
И чрез клятву звезды совершают свое движение, и Он зовет их по именам
и они отвечают Ему от века до века; и точно также духи воды, ветров и всего
воздуха, и их пути по всем соединениям духов.
И в ней (силою клятв) сберегаются хранилища гласов грома и света
молний; и в ней сберегаются хранилища града и инея, и хранилища дождя и росы.
И они все веруют и воссылают благодарение Господу духов, и восхваляют
всею своею силою, и их пища состоит в громких благодарениях; они благодарят, и
прославляют, и превозносят имя Господа духов от века до века.
И могущественна над ними эта клятва, и они сохраняются чрез неё, и их
пути сохраняются, и их движения не нарушаются.
И было для них (для праведников) великою радостью, и прославляли, и
восхваляли за то, что им было открыто имя того Сына человеческого.
И Он сел на престол Своей славы и весь суд был предан Ему, Сыну
человеческому, и Он допустил прийти и погибнуть с лица земли грешникам и тем,
которые соблазнили мир.
Они связаны ценою и заключены в своих сборных местах разврата, и все
дела их исчезают с земли.
И отныне не будет более там ничего тленного, ибо Он, Сын мужа, явился
и сел на престоле Своей славы: и всякое зло исчезнет и прейдет пред Его лицом;
слово же того Сына мужа будет иметь силу пред Господом духов.
Это третья притча Еноха.
\vs 1En 12:1
И случилось после этого: вот его Еноха имя было вознесено при
жизни к тому Сыну человеческому, к Господу духов, от живущих на тверди.
И оно было вознесено на колесницах духа, и имя его вышло среди людей.
И с того дня я не входил в их среду; и Он посадил меня между двумя
ветрами, между севером и западом,~--- там, где ангелы взяли веревки, чтобы
измерить около меня место для избранных и праведных.
И там я видел первых отцов праведных, от древнейшего времени живущих в
том месте.
И после того случилось, что мой дух был сокрыт (восхищен) и
вознесен на небеса; там я видел сынов ангелов, как они ходят по огненному
пламени; и их одежды и их одеяния белы, и свет лица их как кристалл.
И я видел две реки из огня, и свет того огня блистал, как гиацинт: и я
пал на свое лицо пред Господом духов.
И ангел Михаил, один из архангелов, взял меня за правую руку и поднял
меня, и привел меня ко всем тайнам милосердия и правды.
И он показал мне все тайны пределов неба и все хранилища всех звезд и
светил, откуда они выходят пред святых.
И дух восхитил Еноха на небо небес, и я видел там, в средине того
света, нечто такое, что было устроено из кристалловых камней, и между теми
камнями было пламя живого огня.
И мой дух видел, как вокруг того дома обходил огонь, на четырех же
сторонах его реки, наполненные живым огнем, и видел, как они окружают тот дом.
И вокруг были серафимы, херувимы и офанимы: это те которые не спят и
охраняют престол его славы.
И я видел ангелов, которые не могут быть исчислены, тысячу тысяч и
тьму тем, окружающих тот дом: и Михаил и Руфаил, Гавриил и Фануил, и святые
ангелы, которые вверху на небесах, выходят и входят в тот дом.
И вышли из того дома Михаил и Гавриил, Руфаил и Фануил, и многие
святые ангелы без числа, и с ними Глава дней; Его глава была чиста как волна
(руно) И Его одежда неописуема.
И я упал на свое лицо, и все мое тело сплавилось, и мой дух изменился:
и я воскликнул громким голосом, духом силы, и прославил и восхвалил и
превознес.
И эти прославления, которые вышли из моих уст, были приятны для того
Главы дней.
И сам Глава дней шел с Михаилом и Гавриилом, Руфаилом и Фануилом, и с
тысячами и со тьмами тысяч, с ангелами без числа.
И тот ангел пришел ко мне, и приветствовал меня своим гласом, и
сказал: "ты~--- сын человеческий, рожденный для правды", и правда обитает над
тобою, и правда Главы дней не оставляет тебя".
И он сказал мне: "Он призывает тебе мира, во имя будущего мира, ибо
оттуда исходит мир со времени сотворения вселенной, и таким образом ты будешь
иметь его во веки и от века до века.
И все, которые в будущем пойдут по твоему пути,~--- ты, которого правда
не оставляет вовек, жилища тех будут возле тебя и наследие их около тебя, и они
не будут отделены от тебя во век и от века до века.
И таким образом возле того Сына человеческого будет долгая жизнь, и
мир наступит для праведных, во имя Господа духов от века до века.
\vs 1En 13:1
Книга об обращении светил небесных, как это обращение происходит
с каждым из них, по их классам, по их господству и их времени, по их именам и
местам происхождения, и по их месяцам, которые показал мне их путеводитель,
святой ангел Уриил, бывший при мне; и он показал мне все их описание, что с
ними происходит со всеми годами мира и до века, пока не создано новое творение,
которое продолжится во век.
И вот первый закон светил: светило солнце имеет свой восход в восточных
вратах неба и свой заход в западных вратах неба.
И я видел шесть врат, в которых солнце заходит; луна также восходит и
заходит чрез те же врата, путеводители звёзд вместе со своими путеводными
восходят и заходят там же: шесть врат на востоке и шесть на западе, следующих
друг за другом в строго соответствующем порядке, а также много окон направо и
налево от тех врат.
Прежде всего, выходит великое светило, называемое солнцем, его
окружность как окружность неба, и оно совершенно наполнено блистающим и
согревающим огнем.
Колесницы, в которых оно поднимается, гонит ветер, и солнце, заходя,
исчезает с неба и возвращается назад через север, чтобы достигнуть востока; и
оно направляется таким образом, что приходит к соответствующим вратам и светит
на небе.
Таким образом, оно восходит в первый месяц в великих вратах и
именно оно восходит через четвёртые из тех шести восточных врат.
И при тех четвёртых вратах, через которые солнце восходит в первый
месяц, находятся двенадцать оконных отверстий, из которых выходит пламя, когда
они в свое время открываются.
Когда солнце поднимается на небо, то оно выходит чрез те четвёртые
врата в продолжение тридцати утров, и заходит прямо, напротив, в четвёртых
вратах на западе неба.
И в этот период день становится день за днем длиннее, и ночь становится
ночь за ночью короче до тридцатого утра.
И в тот день, день бывает длиннее на две части, чем ночь, и день
включает ровно десять частей и ночь восемь частей.
И солнце восходит из тех четвёртых врат и заходит в четвёртых, и
возвращается к пятым вратам востока в продолжение тридцати утров, и восходит из
них, и заходит в пятых вратах.
Тогда день становится длиннее на две части и заключает одиннадцать
частей, и ночь становится короче и заключает семь частей.
И солнце возвращается к востоку, и вступает в шестые врата, и восходит
и заходит в шестых вратах в продолжение тридцати одного утра ради их знака.
И в тот день, день становится длиннее ночи настолько, что заключает
двойное число частей ночи,~--- именно двенадцать частей, и ночь делается короче и
заключает шесть частей.
И поднимается солнце, чтобы день стал короче и ночь длиннее, и солнце
возвращается к востоку и вступает в шестые врата, и восходит из них и заходит в
продолжение тридцати утров.
И когда пройдет тридцать утров, день уменьшается ровно на одну часть,
и заключает одиннадцать частей и ночь семь частей.
И солнце выступает на западе их тех шести врат и идёт к востоку, и
восходит в пятых вратах в продолжение тридцати утров, и опять заходит на западе
в пятых западных вратах.
В тот день, день уменьшится на две части и заключает десять частей и
ночь восемь частей.
И солнце выходит из тех пятых врат запада, и поднимается в четвёртых
вратах ради их знака тридцать одно утро, и заходит на западе.
В тот день сравнивается день с ночью, и они становятся одинаково
длинными, и ночь заключает девять частей и день девять частей.
И солнце восходит из тех врат и заходит на западе, возвращается к
востоку и восходит в третьих вратах тридцать утров, и заходит на западе в
третьих вратах.
И в тот день ночь становится длиннее дня до тридцатого утра, и день
становится ежедневно короче до тридцатого дня, и ночь заключает ровно десять
частей и день восемь частей.
И солнце восходит из тех третьих врат, и заходит в третьих вратах на
западе, возвращается к востоку и восходит во вторых вратах востока в
продолжение тридцати утров, и точно также заходит во вторых вратах на западе
неба.
И в тот день ночь заключает одиннадцать частей и день из тех вторых
врат и заходит на западе во вторых вратах, и возвращается к востоку в первые
врата в продолжение тридцати одного утра и заходит на западе в первых вратах.
И в тот день ночь становится настолько длинною, что включает двойное
число частей дня; ночь заключает ровно двенадцать частей и день шесть частей.
Этим солнце закончило свои путевые становища, и оно опять поворачивает
на эти же становища, и вступает в те первые врата в продолжение тридцати утров,
и заходит также на западе напротив них.
И в тот день ночь уменьшается в продолжительности на одну часть, и она
заключает одиннадцать частей в день семь частей.
И солнце возвращается и вступает во вторые врата востока, и
возвращается на те свои путевые становища в продолжение тридцати утров, восходя
и заходя.
И в тот день ночь уменьшается в продолжительности, и ночь заключает
десять частей и день восемь частей.
И в тот день солнце восходит из тех вторых врат и заходит на западе,
потом возвращается к востоку поднимается в третьих вратах в продолжение
тридцати одного утра, и заходит на западе неба.
В тот день ночь уменьшается и заключает десять частей и день девять
частей, и ночь сравнивается с днем, и год заключает ровно триста шестьдесят
четыре дня, и продолжительность дня и ночи, и краткость дня и ночи в следствие
движения солнца становятся различными.
По причине этого дневное движение ежедневно становится длиннее, и его
ночное движение становится каждоночно короче.
И таков закон и движение солнца и его возвращение, насколько оно часто
возвращается: шестьдесят раз возвращается и восходит оно, именно то великое
вечное светило, которое навеки именуется солнцем.
И то, что таким образом восходит, есть великое светило, как оно
называется по своему появлению в силу повеления Господа.
И таким образом оно восходит и заходит, и не уменьшается и не
покоится, но движется день и ночь в колеснице, и его свет в семь раз светлее
лунного, но по величине они оба одинаковы.
\vs 1En 14:1
И после этого закона я видел другой закон, касающийся малого
светила, который называется луною.
Ее окружность подобна окружности неба, и ее колесница, в которой она
идет, гонится ветром; и ей дается свет по определённой мере.
В каждый месяц изменяется ее восход и заход; её дни как дни солнца; и
если ее свет равномерен (полон), то она содержит седьмую часть солнечного
света.
И она восходит таким образом: и ее начало на востоке выступает в
тридцатое утро; в тот день она становится видимою, и тогда бывает для всех
начало луны, в тридцатое утро, одинаково с солнцем в тех же вратах, где
восходит солнце.
И одна половина ее выступает на одну седьмую часть, и весь ее круг
бывает пуст, без света, кроме одной седьмой части из ее четырнадцати частей
света.
И когда она получает одну седьмую часть с половиной от своего света, то
ее свет заключает одну седьмую и седьмую часть с половиной.
Она заходит в новолуние вместе с солнцем, и когда солнце восходит,
восходит и луна вместе с ним, и получает половину одной седьмой части света, и
в ту ночь, в начале ее утра, луна заходит в первый день месяца вместе с
солнцем, и бывает невидима в ту ночь семью и семью частями с половиной.
И она в тот день становится видимою ровно одной седьмой частью, и
восходит и отклоняется от солнца, и дает света в остальные дни семь и семь (14)
частей.
И я видел другой закон и движение её, как она по тому закону
совершает своё месячное обращение.
И всё показал мне святой ангел Уриил, который служит вождём всех их
(светил); и я описал все её (луны) положения, и показал их мне, и описал ее
месяцы, как они бывают, и появление её света до истечения пятнадцати дней.
В каждых семи частях весь ее свет делается полным на востоке, и в
каждых седьми частях весь ее мрак делается полным на западе.
И в определенные месяцы она изменяет свой заход, и в определенные
месяцы она идет своим особенным (от солнца) движением.
И в двоих вратах луна заходит вместе с солнцем,~--- в тех двоих вратах,
в третьих и четвертых вратах.
Именно,~--- она выходит в продолжение семи дней и поворачивает, и
возвращается опять через врата, где восходит солнце; и в них ее свет делается
полным; и она отклоняется от солнца, и вступает в течение 8 дней в шестые
врата, из которых выходит солнце.
И когда солнце выходит из четвертых врат, она выходит семь дней, так
что она выходит из пятых, и возвращается опять в течение семи дней в четвёртые
врата, и весь ее свет делается полным, и она отклоняется и вступает в первые
врата в течение восьми дней.
И опять она возвращается в течение семи дней в четвертые врата, из
которых выходит солнце.
Так видел я их положения, как солнце восходит и заходит по порядку
своих месяцев.
И между теми днями, если взять вместе пять лет, солнце имеет излишку
тридцать дней; которые приходятся на один из тех пяти лет, если они полны,
составляют триста шестьдесят четыре дня.
И излишек солнца и звезд простирается до шести дней; а в пять лет, в
каждый по шести, до тридцати дней, и луна отстает от солнца и звезд на тридцать
дней.
И луна точно ведет все года, так что их положение вовек ни поспешает,
ни запаздывает ни на один день, но действительно правильно совершает годовую
смену в триста шестьдесят четыре дня.
Три года имеют тысячу девяносто два дня и пять лет тысячу восемьсот
двадцать дней, так что на восемь лет приходится две тысячи девятьсот двенадцать
дней.
На луну же приходится в три года тысяча шестьдесят два дня, и в пять
лет она отстаёт на пятьдесят дней; именно с суммою этого нужно прибавить к
шестидесяти двум дням.
И на пять лет приходится тысяча семьсот семьдесят пять дней, так что
лунные дни в восемь лет составляют две тысячи восемьсот тридцать два дня.
Именно ее отставание образует в восемь лет восемьдесят дней, и всех
дней, на которые она отстает в восемь лет, восемьдесят.
И правильный год достигает конца сообразно с положением их (фаз луны?)
и с положением солнца, так как оно восходит из врат, из которых оно восходит и
заходит тридцать дней.
И путеводители глав тысячей, которые поставлены над всем
творением и над всеми звёздами, существует с четырьмя добавочными днями,
которые не могут быть отделены от своего места сообразно со всем исчислением
года; и эти путеводители служат для четырёх дней, которые не считаются при
исчисление года.
И из-за них люди ошибаются в том (в исчислении), ибо те светила
действительно служат для положения мира, одно в первых, одно в третьих, одно в
четвертых и одно в шестых вратах; и точность движения мира оканчивается всегда
чрез триста шестьдесят четыре положения его.
Ибо знаки, времена, и годы и дни показал мне ангел Уриил, которого
вечный Господь славы поставил над всеми небесными светилами на небе и в мире,
чтобы они управляли на поверхности неба, и явились над землею, и были
путеводителями для дня и ночи, именно солнце, луна и звезды, и все служебные
творения, которые совершают свое обращение во всех колесницах неба.
Точно так же Уриил дал мне увидеть двенадцать дверных отверстие в
кругу солнечных колесниц на небе, из которых пробиваются лучи солнца; и от них
исходит теплота на землю, когда они открываются в определённые времена.
Такие же отверстия есть также для ветров для духа росы, когда они по
временам открываются, стоя открытыми в небесах на пределах.
И я видел двенадцать врат на небе на приделах земли, из которых
солнце, луна и звёзды, и все произведения неба выходят на востоке и на западе.
И много оконных отверстий находится направо и налево от них, и каждое
окно выбрасывает в свое время тепло, соответствуя тем вратам, из которых
выходят звезды по повелению, которое Он дал им, и в которые они заходят,
соответствуя их числу.
И я видел на небе колесницы, как они неслись в мире,~--- вверху и внизу
от тех врат,~--- в которых обращаются никогда не заходящие звезды.
И одна их них больше всех их, и она проходит чрез весь мир.
\vs 1En 15:1
И на пределах земли я видел открытыми для всех ветров двенадцать
врат, из которых выходят ветры и дуют на землю.
Трое из них открыты на лице неба (на востоке), и трое на заходе, и трое
на правой стороне неба и трое на левой.
И трое первых лежат к востоку, и трое к северу, и трое, противостоящих
им налево, к югу, и трое на западе.
Чрез четверо из них выходят ветры благословения и благополучия, а из
тех (из остальных) восьми выходят ветры бедствия; когда они посылаются, то
производят разрушение на всей земле, и в воде, существующей на ней, и во всех
тварях, живущих на ней, и во всем, что находится в воде и на суше.
И первый ветер, дующий из тех врат и называющийся восточным,
выходит в первых восточных вратах, склоняющихся к югу; из них выходит
разрушение, сухость, зной и гибель.
И чрез вторые врата, что лежат в средине, выходит правильное смещение,
и именно~--- из них выходит дождь и плодородие, и благополучие, и роса; и чрез
третьи врата, которые лежат к северу, выходит холод и сухость.
И после этих, выходят южные ветры чрез трое врат: во-первых, через
первые из них, которые склоняются к востоку, выходит жгучий ветер.
И через прилежащие к ним средние врата выходят благовония, и роса, и
дождь, и благополучие, и здоровье.
И чрез третьи врата, лежащие к западу, выходит роса, и дождь, и саранча,
и разрушение.
И после этих северные ветры: из седьмых врат, которые на восточной
стороне склоняются к югу выходит роса и дождь, саранча и разрушение.
И из средних врат на прямом направлении выходит дождь, и роса, и
здоровье, и благополучие; и через третьи врата на северо-западной стороне
выходит туман, и иней, и снег, и дождь, и роса, и саранча.
И после этих западные ветры: чрез первые врата, склоняющиеся к северу,
выходит роса, и дождь, и иней, и холод, и снег, и мороз.
И из средних врат выходит роса и дождь, благополучие и благословение;
и чрез последние врата, лежащие к югу, выходит сухость и разрушение, жар и
гибель.
Этим оканчиваются двенадцать врат четырех небесных стран; и все их
законы, и все их бедствия, и все их благодеяния я показал тебе, мой сын
Мафусаил.
Первый ветер называют восточным, так как он передний (первый); и
второй ветер называется южным ветром, так как там нисходит Всевышний; и там
предпочтительнее всего сходит Тот, Который да будет прославлен вовеки.
И западный ветер называется ветром уменьшения, так как там небесные
светила уменьшаются и опускаются.
И четвертый ветер называется северным: он разделяется на три части:
первая из них назначена для жилища людей, вторая для водных морей и с долинами,
и лесами, и реками, и мраком, и туманом; и третья часть с садом правды.
Я видел семь высоких гор, выше всех гор, находящихся на земле; оттуда
выходит иней; и приходят и исчезают дни, времена и годы.
Семь рек видел я на земле, больше всех других; одна из них, текущая с
запада, изливает свою воду в великое море.
И две из них текут с севера к морю, и изливают свою воду в эритрейское
море на востоке.
И четыре остальные вытекают на северной стране к своему морю, две к
эритрейскому морю, и две имеют устье в великом море, по другим~--- в пустыне.
Семь великих островов я видел на море и на суше: два на суше и пять на
великом море.
Имена солнца следующие: первое Оререс, второе Томас.
И луна имеет четыре имени: первое Азонъйя, второе Эбла, третье Беназэ,
и четвертое Эраэ.
Это оба великие светила: их окружность, как окружность неба, и по
величине они оба равны.
В кругу солнца находится одна седьмая часть света, в которою
прибавляется свет луне, и именно~--- в определенной мере прибавляется он, пока не
истощится седьмая часть солнца.
И они заходят и входят в западные врата, и совершают обращение через
север, и через восточные врата они выходят на поверхность неба.
И когда луна поднимается, то она появляется на небе имея в себе света
половину одной седьмой части; и в течение четырнадцати дней весь ее свет
делается полным.
В нее прибавляется также трижды пять (15) частей света, так что к
пятнадцатому дню свет ее становится полным по знаку года, и составляется трижды
пять частей, и луна рождается чрез половину одной седьмой части.
И при своем ущербе она уменьшается в первый день до четырнадцати своих
частей света, во второй до тринадцати, в третий до двенадцати, в четвертый до
одиннадцати, в пятый до десяти, в шестой до девяти, в седьмой до восьми, в
восьмой до семи, в девятый до шести, в десятый до пяти, в одиннадцатый до
четырех, в двенадцатый до трех, в тринадцатый до двух, в четырнадцатый до
половины одной седьмой части: и ее свет, который оставался от целого,
совершенно исчезает в пятнадцатый день.
И в определенные месяцы месяц имеет по двадцати девяти дней и один раз
двадцать восемь.
Также и другое установление показал мне Уриил относительно того, когда
прибавляется луне свет и на которой стороне он прибавляется ей от солнца.
Во все время, когда луна усиливается в своем свете, она лежит по
отношению к солнцу напротив; к четырнадцатому дню ее свет становится полным на
небе; и когда она вся освещена, ее свет бывает полным на небе.
И в первый день она называется новолунием, ибо в тот день начинается в
ней свет.
И она становится полною ровно в тот день (в 15-ый), когда солнце
заходит на западе, а она ночью восходит с востока и светит целую ночь, пока
солнце не взойдет напротив нее, и она бывает видима напротив солнца.
На той стороне, где бывает свет луны, она также опять уменьшается,
пока не исчезнет весь ее свет, и дни месяца оканчиваются, и ее круг остается
пустым без света.
И в продолжение трех месяцев она делает тридцать дней в свое время, и
в продолжение трех месяцев она делает по двадцати девяти дней, в которых
происходит ее ущерб в первое время и в первых вратах в течение ста семидесяти
семи дней.
И во время своего восхода она показывается в продолжение трех месяцев
по тридцать дней, и в продолжение трех месяцев по двадцати девяти дней.
Ночью она показывается приблизительно в течение двадцати дней как муж,
и днем как небо, ибо нет ничего другого в ней, кроме ее света.
И теперь, мой сын Мафусаил, я показал тебе все, и весь закон
звезд (светил) небесных окончен.
И он (Уриил) показал мне весь закон их для каждого дня, для каждого
времени (года), для каждого господства, и для каждого года, и его выход по Его
предписанию для каждого месяца и каждой недели; и он показал ущерб луны,
который происходит в шестых вратах; именно~--- в этих шестых вратах оканчивается
весь ее свет, и после этого там бывает начало месяца; и он показал ущерб,
который происходит в первых вратах в свое время, пока не пройдет сто семьдесят
семь дней, а по исчислению по неделям~--- двадцать недель и два дня; и он
показал, как она отстает от солнца и от порядка звезд ровно на пять дней в одно
время, и когда это место, которое ты видишь, оканчивается.
Таков образ, и описание каждого светила, как их показал мне вождь их~---
великий ангел Уриил.
И в те дни отвечал мне Уриил и сказал мне: "вот я показал тебе
все, о Енох, и открыл тебе все, чтобы ты увидел это, это солнце, и эту луну, и
путеводителей звезд небесных, и всех тех, которые вращают их, их соотношения, и
времена, и выходы.
И в дни грешников годы будут укорочены, и их посев будет запаздывать в
их странах и на их пастбищах (полях), и все вещи на земле изменятся и не будут
являться в свое время; дождь будет задержан, и небо удержит его.
И в те времена плоды земли будут запаздывать и не будут вырастать в
свое время; и плоды деревьев будут задержаны от созревания в свое время.
И луна изменит свой порядок и не будет являться в свое время.
И в те дни будет видимо на небе, как приходит великое неплодородие, на
самой крайней колеснице на западе; и оно (небо или солнце) будет светить ярче,
чем по обыкновенному порядку света.
И многие главы начальственных звезд будут ошибаться и они нарушат свои
пути и отправления, и подчинённые им не будут появляться в свои времена.
И весь порядок звезд сокрыт для грешников, и мысли тех, которые живут
на земле, будут ошибаться из-за них, и они уклонятся от всех своих путей, и
будут грешить и станут считать их (звезды) за богов.
И много зол придет на них, и осуждение придет на них, чтобы уничтожить
их всех".
И он сказал мне: "о Энох, рассмотри писание небесных скрижалей и
прочитай, что на них написано, и заметь для себя все в отдельности".
И я рассмотрел все на небесных скрижалях, и прочитал все, что на них,
и заметил для себя все, и прочитал книгу и все, что было на ней, все дела людей
и всех телесно-рожденных, которые будут на земле до самых отдаленных родов.
И после этого я тотчас прославил Господа, вечного Царя славы, за то,
что Он сотворил все произведения мира и восхвалил Господа за Его терпение, и
благословил Его за детей мира.
И в тот час я сказал: "блажен муж, который умирает как праведный и
благой, о котором не написано никакое писание неправды и против которого не
найдено вины"!
И те трое святых ангелов принесли меня и поставили меня на землю пред
дверями моего дома, и сказали мне: "возвести все своему сыне Мафусаилу и открой
всем своим детям, что ни один из смертных не праведен пред Господом, ибо Он
Творец их.
На один год мы оставим тебя при твоих детях,~--- пока ты не укрепишься
снова,~--- чтобы ты научил своих детей, и записал им это, и засвидетельствовал
им, всем твоим детям, и на другой год ты будешь взят из среды их.
Ибо добрые будут возвещать правду; праведный будет радоваться с
праведными, и они будут благожелать друг другу.
Грешник же умрет с грешником и отпадший потонет с отпадшим.
И те, которые сохранят справедливость, умрут ради дел людей и будут
соединены ради деяния нечестивых".
И в те дни они перестали говорить со мною.
И я пришел к своим домочадцам, прославляя Господа мира.
И теперь, сын мой Мафусаил, я рассказываю тебе все эти вещи и
записываю тебе; и я открыл тебе все и дал тебе писание обо всех них (светилах);
итак, сохрани же, мой сын Мафусаил, писания ради твоего отца, и передай их
грядущим родам.
Мудрость я дал тебе и твоим детям, и тем твоим детям, которые еще
придут, чтобы они передали ее своим детям и грядущим родам до вечности,~---
именно эту мудрость, превышающую их мысли.
И разумеющие ее не будут спать, и будут прислушиваться своим ухом,
чтобы научиться этой мудрости, ибо она понравится тем, которые кушают от неё,
лучше приятной пищи.
Блаженны все праведные, блаженны все, ходящие по пути правды, и не
погрешающие, подобно грешникам, в исчислении всех своих дней, в течение которых
солнце ходит на небе, входя и выходя через врата по тридцати дней вместе с
главами над тысячью этого порядка звезд, именно~--- вместе с четырьмя, которые
прибавляются и разделяют четыре части года, которые их направляют, и с ними
входят четыре дня.
И из-за них люди будут ошибаться, и не будут считать их при исчислении
целого движения мира; напротив люди будут ошибаться в них и не узнают их в
точности.
Ибо они (добавочные дни) относятся к исчислению года и действительно
отмечены навсегда~--- один в первых вратах, и один в третьих, и один в четвертых,
и один в шестых; и год завершается в 364 дня.
И рассказ об этом праведен, и точно указано исчисление этого
(т.е.года) ибо светила, и месяцы, и праздники, и годы, и дни мне показал и
внушил Уриил, которому Господь всего мироздания дал повеление ради меня
относительно воинства небесного; и он имеет власть над ночью и днем на небе,
чтобы заставлять свет светить над людьми,~--- солнце, луну и звёзды, и все силы
небесные, которые вращаются в своих кругах.
И таковы порядки звезд, которые заходят в своих местах и в свое время,
и праздники и месяцы.
И таковы имена тех, которые путеводят их (звезды) и которые
бодрствуют, чтобы он вступил в определенные им времена, в своих порядках, в
свои сроки, и месяцы, и времена господства, и по своим местам.
Четыре их путеводителя, которые разделяют четыре части года, вступают
прежде всех и после них двенадцать путеводителей порядков, которые разделяют
месяцы и год на 364 дня, рядом с главами над тысячью (хилиархами), которые
делают дни; и для четырех добавочных дней существуют те же путеводители,
которые разделяют четыре части года.
И из тех начальников над тысячью один расположен между путеводителем и
путеводимым позади мест, но только путеводители их делают разделение.
И вот имена путеводителей, разделяющих четыре установленные части
года: Мелкеел, и Гелеммелех, и Мелейял, и Нарел, И имена тех, которых они
ведут: Аднарел, и Ийязузаел, и Ийелумиел.
Эти трое следуют за путеводителями порядков, и один следует за троими
путеводителями порядков, следующими за теми место начальниками (топархами),
которые разделяют четыре части года.
В начале года первым восходит и управляет Мелкейял, который называется
Таммани и солнцем; и всего времени его господства, в продолжение которого он
управляет, девяносто один день.
И вот признаки дней, которые должны появляться на земле во время его
господства: пот, и жар, и тоска; все деревья тогда производят плоды, и листва
появляется на всех деревьях, и бывает жатва пшеницы и расцвет роз, и все цветы
тогда цветут на поле, но зимние деревья становятся сухими.
И вот имена подчиненных им (топархам) путеводителей: Беркеел,
Цалбезаел и еще другой, который присоединяется,~--- глава над тысячью, называемый
Голойязеф, и дни господства этого заканчиваются.
Другой путеводитель (топарх), который следует за ними, есть
Гелеммелек, которого называют также светящим солнцем; и все время его света
девяносто один день, и вот признаки дней на земле в то время: жар и сухость, и
плоды деревьев становятся зрелыми и спелыми, и плоды их сохнут; и овцы тогда
спариваются и становятся суягными; и тогда собираются все плоды земли и все,
что есть на полях, и бывает выжимание винограда: все это происходит во дни его
господства.
И вот имена, и порядки, и подчиненные им путеводители тех глав над
тысячью: Гедаел, и Кеел, и Геел, и имя начальника над тысячью, который
присоединяется к ним, Асфаел; и оканчиваются дни его господства.
\vs 1En 16:1
И теперь, мой сын Мафусаил, я хочу открыть тебе все видения,
которые я видел, рассказавши тебе их.
Два видения видел я, прежде чем взял жену, и они не похожи одно на
другое; в первый раз, когда я изучал писание, и во второй раз, прежде чем взять
твою мать, я видел страшные видения: и из-за них я молил Господа.
Я лег в доме моего деда Малелеила, и тогда я увидел в видении, как небо
опустилось и уменьшилось, и упало к земле.
И когда оно упало к земле.
И когда оно упало на землю, я увидел землю, как она была поглощена
великою бездною, и горы опустились на горы, и холмы погрузились на холмы, и
высокие деревья оторвались от своих стволов (корней), и низверглись и потонули
в бездне.
И от этого в моих устах обрелась речь, и я начал восклицать и сказал: "
погибла земля"!
И мой дед Малелеил разбудил меня, ибо я лежал около него, и сказал мне:
"отчего ты восклицаешь так, мой сын, и от чего ты так сетуешь"?
Тогда я рассказал ему видение, которые видел, и он сказал мне: "ужасно
то, что ты видел, мой сын!
И твое сновидение обнимает тайну всех грехов земли: она должна
погрузиться в бездну и потерпеть насильственную гибель.
И теперь, сын мой, встань и молись Господу славы,~--- ибо ты верующий,~---
чтобы остаток сохранился на земле целым и чтобы Он истребил не всю землю.
Сын мой!
С неба все это придет на землю, и на земле совершится насильственная
гибель".
После этого я встал, и просил, и умолял, и записал свою молитву для
грядущих родов, и я все покажу тебе, сын мой Мафусаил.
И когда я вышел вниз (т.е.из дому), и увидел небо и солнце, восходящее
на востоке, и луну, опускающуюся на западе, и все, как Он узнал это в начале,
то я прославил Господа суда, и превознес Его, ибо Он повелел солнцу выходить из
окон востока, чтобы оно поднималось, и восходило на плоскости неба, и
возносилось, и проходило теперь путь, который ему указан.
И я воздвиг руки свои в правде, и прославил Святого и Великого, и
говорил дыханием моих уст и телесным языком, который сотворил Бог для сынов
человеческих, чтобы они говорили им, и дал им дыхание, и язык, и уста, чтобы
они говорили благодаря этому.
"Будь прославлен Ты, о Господи, Царь Великий и Могущественный в Своем
величии, Господь всего небесного творения, Царь царей, и Бог всего мира!
И Твое божество, и царство, и величие пребывает во век и от века до
века, и Твое господство~--- чрез все роды, и все небеса служат Тебе престолом
вовек, и вся земля~--- подножием Твоих ног вовек и от века до века.
Ибо Ты сотворил и господствуешь над всем, и для Тебя совершенно ничего
нет трудного, и никакая мудрость не ускользнет от Тебя; она не отвращается от
своего престола,~--- Твоего престола,~--- ни от Твоего лица; и Ты знаешь, и видишь,
и слышишь все, и нет ничего, чтобы было сокровенно для Тебя, ибо Ты видишь все.
И теперь ангелы Твоего неба беззаконнуют, и гнев Твой пребывает на
плоти людей до дня великого суда.
И теперь, о Боже и Господи, и великий Царь, я молю и прошу, чтобы Ты
исполнил для меня мою просьбу, прошу оставить мне на земле потомство целым, и
не истреблять всю плоть человеческую, и не делать землю безлюдною, чтобы была
вечная гибель, и теперь, Господь мой, истреби от земли плоть, которая
разгневала Тебя, но плоть правды и праведности утверди как растение семени
навсегда, и не отвращай Твоего лица от молитвы раба Твоего, о Господи"!
\vs 1En 17:1
И после этого я видел другой сон, и я вполне открою его тебе, мой
сын.
И Енох начал и сказал своему сыну Мафусаилу: "тебе я буду говорить,
мой сын; слушай речь мою и приклони ухо свое к сновидению твоего отца!
Прежде чем я взял твою мать Едну, я видел в видении на своем ложе, и
вот телец вышел из земли, и тот телец был белый; и за ним вышло женское рогатое
животное, и вместе с ним вышли другие рогатые животные: одно из них было
черное, другое красное.
И то черное рогатое животное бодало красное и преследовало его на
земле; и скоро я не мог более видеть того красного рогатого животного.
Но то черное рогатое животное выросло и к нему пришло женское рогатое
животное, и я видел, как многие тельцы, которые были похожи на него и следовали
за ним, вышли от него.
И та корова,~--- та первая,~--- вышла от лица того первого тельца, чтобы
искать то красное животное, но не нашла его, и тотчас подняла великий жалобный
вопль, и искала его.
И я видел, как пришел к ней тот первый телец и успокоил ее, и с того
часа она более не ревела.
После этого она родила другого белого тельца, а после него родила
многих других тельцов и черных коров.
И я видел в моем сновидении, как тот белый вол также вырос и сделался
большим белым волом, и от него произошло много белых тельцов, которые были
похожи на него, И они стали производить белых тельцов, которые были похожи на
них, следуя один за другим.
И я опять видел своими очами, в то время как спал, и увидел
вверху небо, и вот одна звезда упала с неба, и она поднялась, и ела, и паслась
между теми тельцами.
И после этого я видел больших и черных тельцов, и вот они все
переменили свои загороди, и пастбища, и своих рогатых животных, и начали
сетовать друг с другом.
И я опять видел в видении, и посмотрел на небо, и вот я увидел много
звезд, как они упали и были низвергнуты с неба к той первой звезде и в среду
тех рогатых животных и тельцов; и вот они были с теми и паслись в среде их.
И я посмотрел на них и увидел, и вот все они обнаружили свои срамные
члены, как кони, и начали подниматься на тельцовых коров; и все они стали
стельными, и родили слонов, верблюдов и ослов.
И все тельцы устрашились и испугались их; и они начали кусаться своими
зубами и пожирать, и бодать своими рогами.
И они начали теперь поедать тех тельцов; и вот все дети земли начали
трепетать пред ними, и дрожать, и спасаться бегством.
И я опять видел их, как они начали бодаться сами между собою и
пожирать друг друга, и земля стала взывать.
И я опять поднял свои очи к небу и увидел в видении: и вот там вышли
из неба имевшие вид белых людей; из того места вышел один и вместе с ним трое.
И те трое, которые вышли после, взяли меня за руку и подняли меня
прочь от рода земли, и вознесли меня на высокое место, и показали мне башню,
высоко стоящую над землей, и все холмы были ниже ее.
И они сказали мне: "оставайся здесь, чтобы видеть всё, что произойдет
со всеми теми слонами, и верблюдами, и ослами, со звездами, и со всем
тельцами"!
И я видел одного из тех четверых, которые вышли прежде, как он
схватил звезду, прежде всех ниспадшую с неба, связал ей руки и ноги, и положил
ее в пропасть; пропасть же та была тесна и глубока, ужасна и мрачна.
И один из них обнажил свой меч и отдал его тем слонам, и верблюдам, и
ослам; тогда они начали поражать друг друга, так что вся земля дрожала
вследствие этого.
И когда я видел в видении,~--- вот там бросился теперь с неба вниз один
из тех четверых, которые спустились, и собрал и взял великие звезды, срамные
члены которых были как срамные члены коней, и связал их всех по рукам и ногам,
и положил их в ущелье земли.
И один из тех четверых пришел к тем белым тельцам, и научал его
(одного из них) тайне, в то время как он трепетал: он был рожден подобно тельцу
и сделался человеком, и выстроил себе большое судно и поселился в нем; вместе с
ним расположились также в том судне трое тельцов; и оно было закрыто над ними.
И я опять поднял свои очи к небу и увидел высокую крышу с семью
шлюзами на ней, и те шлюзы изливали много воды во двор.
И я видел опять, и вот, тогда открылись источники на почве в том
великом дворе, и эта самая вода начала волноваться и подниматься выше почвы, и
сделала тот двор невидимым, так что вся почва его закрылась водою.
И выростала на нем (дворе) вода, мрак и облако; и тогда я посмотрел на
высоту той воды, как она поднялась выше того двора, и текла поверх него, и
остановилась на земле.
И все тельцы того двора столпились вместе, так что я тотчас увидел,
как они потонули, и были поглощены и погибли в той воде.
Само же судно плавало по воде, между тем как все тельцы, и слоны, и
верблюды, и ослы на земле погрузились вместе со всем скотом, так что я не мог
более видеть их, и они не могли выйти, но потонули и погрузились в бездне.
И я опять видел в видении, как те шлюзы отложились от той высокой
крыши, и источники земли иссякли, и другие бездны открылись.
Тогда вода начала стекать в них, пока земля не сделалась видимою; а то
судно твердо встало на земле, и отступил мрак, и просиял свет.
А тот белый телец, который стал мужем, вышел из того судна и три
тельца с ним; и один из трех был белый, подобно тому тельцу, и один из них был
красный, как кровь, и один черный; и этот самый,~--- тот белый телец, отошел от
них.
И они начали рождать диких зверей и птиц, так что от всех их вместе
произошло разнообразное множество видов,~--- львы, тигры, псы, волки, шакалы,
дикие свиньи, соколы, коршуны, ястребы, орлы и вороны; и в среде их родился
белый телец.
И они начали грызться друг с другом: но тот белый телец, родившийся в
среде их, произвел дикого осла и вместе с ним белого вола; и дикий осел
умножился.
А тот телец, родившийся от него, произвел черную дикую свинью и белую
овцу; и та дикая свинья произвела многих свиней, та овца произвела двенадцать
овец.
И когда те двенадцать овец выросли, они передали одну из своей среды
ослам, и эти ослы опять передали ту овцу волкам, и та овца росла между волками.
И Господь привел одиннадцать овец~--- жить вместе с нею и пастись при
ней среди волков, и они размножились и выросли во многие овечьи стада.
И волки начали бояться их, и притесняли их, так что, наконец, стали
лишать жизни их агнцев; и они бросали их агнцев в многоводную реку; а те овцы
начали кричать о своих агнцах и жаловаться своему Господу.
И одна овца, которая была спасена от волков, убежала и ушла к диким
ослам; и я видел овец, как они сетовали, и кричали, и просили своего Господа
изо всех сил, пока тот Господь не сошел из высокого покоя на зов овец, и не
пошёл к ним и не посетил их.
И Он позвал ту овцу, удалившуюся от волков, и говорил с нею
относительно волков, чтобы она уговорила их не трогать овец.
И овца пошла к волкам по слову Господа, и другая овца сошлась с той
овцой и пошла с нею, и они обе вместе одна с другой пришли на сборище тех
волков, и говорили с ними, и увещевали их отныне не трогать впредь более овец.
При этом я видел волков, и как они стали еще более смирять овец всею
своею силою; и овцы кричали.
И Господь их пришел к овцам и начал бить тех волков; тогда волки
начали сетовать, овцы же сделались спокойными и тотчас не стали более кричать.
И я видел овец, как они ушли от волков; у волков же глаза были
ослеплены, и те волки вышли для преследования овец со всею своею силою.
И Господь овец шел с ними, предводительствуя ими, и все Его овцы
следовали за Ним; лицо же Его было блестящее, и вид Его страшен и величествен.
А волки стали преследовать тех овец, пока не настигли их при водном
озере.
И это самое водное озеро разделилось, и вода остановилась пред ними по
обеим сторонам; и их Господь, Который вел их, встал между ними и волками.
И так как те волки не стали уже видеть овец, то они вошли в средину
того водного озера и преследовали овец, и те волки погнались за ними в водном
озере.
И когда они увидели Господа овец, то воротились, чтобы убежать от
Него, но то водное озеро соединилось, внезапно приняло свою природу, и вода
поднялась и возвысилась, так что покрыла тех волков.
И я видел, как все волки, преследовавшие тех овец, погибли и потонули.
Но овцы вышли из той воды и перешли пустыню, где не было воды и травы;
и они начали открывать свои глаза и видеть; и я видел Господа овец, как Он пас
их и дал им воды и травы, и как та овца шла и вела их.
И та овца поднялась на вершину высокой скалы; и Господь овец послал ее
к ним.
И после этого я видел Господа овец, стоящего пред ними; и Его вид был
величествен и чрезмерно велик, и все те овцы видели Его и устрашились пред Его
лицом.
И все они устрашились и трепетали пред Ним, и кричали после ухода той
овцы, которая была при Нем, к другой овце, находившейся между ними: "мы не
можем вынести этого пред нашим Господом и взирать на Него".
И та овца, которая вела их, возвратилась и поднялась на вершину той
скалы; но овцы начали слепнуть и уклоняться от пути, который она показала им;
между тем та овца ничего не знала об этом.
И Господь овец сильно разгневался на них, и та овца узнала это и
спустилась с вершины скалы, и пришла к овцам, и нашла самую большую часть из
них ослепленною и уклонившеюся от своего пути.
И как только они увидели ее, устрашились и затрепетали пред ее лицом,
и пожелали возвратиться в свои загороди.
И та овца взяла с собою других овец и пришла к тем уклонившимся овцам,
и при этом начала умерщвлять их, и овцы устрашились пред ее лицом; и таким
образом та овца направила уклонившихся овец, и они возвратились в свои
загороди.
И я видел там видение, как та овца сделалась мужем, и выстроила
Господу овец дом, и повелела всем овцам стоять в том доме.
И я видел, как овца, сошедшаяся с той овцою, которая вела их, заснула;
и я видел, как все большие овцы погибли, и малые направились к своему месту, и
они пошли на пастбище и приблизились к водной реке.
Тогда отделилась от них та овца, которая вела их и которая сделалась
мужем, и заснула; и все овцы искали ее и подняли по ней великий вопль.
И я видел, как они прекратили вопль по той овце и переправились через
ту водную реку; и стояли всегда овцы, ведшие их, на месте тех, которые заснули
и которые вели их.
И я видел, как овцы пришли в хорошее место и в вожделенную и
великолепную страну, и видел, как те овцы насытились; а тот дом стоял между
ними в вожделенной стране.
И глаза их то открывались, то ослеплялись, пока не восстала другая
овца, и не повела их, и не направила их всех, и глаза их открылись.
И псы, и лисицы, и дикие свиньи начали пожирать тех овец, пока не
восстала другая овца,~--- баран из их среды, который вел их.
И тот баран начал бодать на обе стороны тех псов, лисиц и диких
свиней, пока не уничтожил их всех.
И у той овцы раскрылись глаза, и она увидела того барана, бывшего
между овцами, как он отрекся от своего достоинства и начал бодать тех овец, и
попирал их, и действовал непристойно.
И Господь овец послал овцу к другой овце, и возвысил ее (последнюю),
чтобы она была бараном и вела овец вместо той овцы, которая оказалась неверной
в своем достоинстве.
И она пошла к ней и говорила только с ней, и поставила ее бараном, и
сделала ее царем и вождем овец; а между всем этим псы притесняли овец.
И первый баран преследовал того второго барана, и тот второй баран
встал и убежал от него; и я увидел, как те псы низвергли того первого барана.
И тот второй баран возвысился и вел малых овец; и тот баран родил
многих овец и заснул; и малая овца сделалась бараном вместо него, и стала
вождём и царем тех овец.
И выросли и размножились те овцы, и все псы, и лисицы, и дикие свиньи
устрашились и разбежались от него; и тот баран бодал и убивал диких зверей; и
те дикие звери не могли уже осилить овец, и никогда уже не похищали у них
ничего.
И тот дом стал великим и широким, и тем овцам была выстроена высокая
башня над тем домом для Господа овец; и тот дом был низок, а башня была
возвышена и высока; и Господь овец стоял на той башне, и пред Ним поставили
полный стол.
И я видел опять тех овец, как они опять заблудились и пошли
многоразличными путями, и оставили тот свой дом; и Господь овец призвал
некоторых из овец и послал их к овцам, но овцы начали умерщвлять их.
И одна из них спаслась и не была умерщвлена, и она убежала и кричала
об овцах, и он хотел ее умертвить; но Господь овец спас ее из рук их и возвел
ее ко мне, и позволил ей жить там.
И многих других овец Он посылал к тем овцам, чтобы свидетельствовать
(или увещевать) и сетовать о них.
И после этого я видел: вот они оставили дом Господа овец и его башню;
они уклонились совершенно и их глаза ослепли; и я видел Господа овец, как он
допустил много поражений над ними в их отдельных стадах, так что те овцы начали
жаловаться на такие поражения и переменили место.
И Он предал их в руки львов и тигров, и волков, и шакалов, и в руки
лисиц и всех диких зверей; и дикие звери стали разрывать тех овец.
И я видел, что Он оставил тот дом их и их башню, и предал их всех в
руки львов, в руки всех диких зверей, чтобы они разрывали их и пожирали.
И я начал кричать изо всех сил, и призывать Господа овец, и
представлять Ему относительно овец, что они пожираются всеми дикими зверями.
Но Он оставался спокойным, когда видел это, и радовался, что они
пожираются, и истребляются и расхищаются; и Он оставил их в руках всех диких
зверей на съедение.
И Он призвал семьдесят пастырей,~--- и отверг тех овец,~--- чтобы они
пасли их, и сказал пастырям и их товарищам: "каждый из вас должен отныне пасти
овец, и все, что Я вам прикажу, то делайте!
И Я передаю их вам по числу, и буду вам объявлять: кто из их должен
погибнуть, тех истребляйте"!
И Он предал им тех овец.
И Он призвал другого и сказал ему: "замечай и смотри за всем, что
будут делать пастыри с этими овцами: ибо они будут губить их более, чем Я им
повелел.
И всякий излишек и уничтожение, которое будет совершаемо пастухами,
запиши,~--- именно сколько губят они по Моему повелению и сколько по своей
собственной воле; и запиши о каждом пастыре в отдельности все, что они губят.
И прочитай это предо Мною по числу (с указанием числа), сколько они
погубили по собственной воле и сколько предано им на погибель, чтобы это было
для Меня свидетельством против них, дабы я знал всякое действие пастырей, чтобы
передать их суду; и смотри, что они делают,~--- пребывают ли в Моем повелении,
которое Я им дал, или нет.
Но они не должны открывать им этого и наставлять их на путь, но запиши
только все, что они погубят, всякий раз о каждом в отдельности, и представь все
Мне"!
И я видел, как те пастыри пасли в определенное им время, и они начали
умерщвлять и погубят более чем им было повелено, и предали тех овец в руки
львов.
И львы и тигры пожирали и истребляли большую часть тех овец, и дикие
свиньи пожирали вместе с ними; и они сожгли ту башню и разрушили тот дом.
И я сильно опечалился из-за башни, так как самый дом овец был
разрушен; и после этого я не мог уже видеть тех овец, входили ли они в тот дом.
И пастыри и их товарищи предали тех овец всем диким зверям, чтобы они
пожирали их; и каждый в отдельности из них получил в своё время определенное
число, и о каждом в отдельности записал другой в книгу, сколько он погубил.
И каждый из них умертвил и погубил гораздо более чем ему было
позволено; и я начал плакать и сильно сетовать о тех овцах.
И я видел в видении того писца, как он записал каждую овцу, погибшую
от тех пастырей, день за днем, и всю книгу вознес к Господу овец, и представил
и показал, что они сделали, и всех, которых каждый из них уничтожил, и всех,
которых они предали погибели.
И книга была прочитана пред Господом овец, и он взял книгу в свои
руки, и прочитал ее, и сложил ее.
И тотчас я увидел, как пастыри пасли в продолжение двенадцати часов; и
вот три из тех овец возвратились, и пришли, и приступили, и начали строить все,
что было разрушено в том доме; но дикие свиньи помешали им, так что они не
могли продолжать этого.
И они начали опять строить, как прежде, и возвели ту башню, и она была
названа высокой башней; и они начали опять ставить стол пред башнею, но весь
хлеб на нем был скверен и нечист.
И по отношению ко всему глаза у тех овец были ослеплены, так что они
не видели, а также и у пастырей их, весьма многие из них были преданы пастырям
на погибель, и они попирали овец своими ногами и пожирали их.
И Господь овец оставался спокойным, пока все овцы не рассеялись по
полю и не перемешались с ними (диким зверями), и они (пастыри) не спасли их от
рук зверей.
И тот, который писал книгу, вознес ее к обителям Господа овец, и
показал ее, и умолял Его за их, и просил Его, показав Ему всю деятельность
пастырей их, и представил Ему свидетельство против всех пастырей.
И он взял книгу, сложил ее у Него и вышел.
И я смотрел до тех пор, пока таким образом не приняли паству
тридцать семь пастырей, и они все окончили каждый свое время, как первые; и
другие приняли их (овец) в свою власть, чтобы каждый пас их по определенному им
времени,~--- каждый пастырь в свое время.
И после этого я видел в видении, как пришли птицы небесные,~--- орлы,
коршуны, ястреба, вороны; орлы же предводительствовали всеми птицами; и они
начали пожирать тех овец, и выклевывать им глаза, и пожирать их мясо.
И овцы кричали, так как их мясо было пожираемо птицами, и я восклицал
и жаловался во время моего сна на того пастыря, который пас овец.
И я видел, как те овцы были пожраны псами, и орлами, и ястребами, и
они не оставили им ни мяса, ни кожи, ни сухожилий, так что от них остался
только остов, но и остов их упал на землю, и овец стало мало.
И я смотрел до тех пор, пока не приняли паству двадцать три пастыря,
и окончили, управляя каждый по определенному им времени, пятьдесят восемь
времен.
Но от тех белых овец родились малые агнцы, и они стали открывать свои
глаза, и видеть, и кричать овцам.
И овцы не кричали им и не слышали, что и сказали им, но были
чрезвычайно глухи, и их глаза были слишком и чрезмерно ослеплены.
И я видел в видении, как вороны налетели на тех агнцев и взяли одного
из тех агнцев, овец же разорвали и пожрали.
И я видел, как у тех агнцев выросли рога, и вороны низвергли их рога;
и я видел, как выскочил один великий рог,~--- одна из тех овец; и их глаза
открылись.
И я смотрел за ним, и глаза их раскрылись; и она кричала к овцам, и
юнцы увидели ее и все побежали к ней.
И между всем тем те орлы, и коршуны, и вороны, и ястреба все еще
разрывали овец беспрестанно, и слетались, на них и пожирали их; но овцы
оставались покойными, и юнцы сетовали и кричали.
И те вороны сражались и боролись с ними, и хотели сломить его рог, но
ничего не могли сделать с ним.
И я видел их, пока не пришли пастыри, и орлы, и те коршуны и ястреба,
и они кричали воронам, чтобы они сломили рог того юнца; и они боролись и
сражались с ними, и он боролся с ним, и кричал, чтобы пришла к нему помощь.
И я видел, как пришел тот муж, который записывал имена пастырей и
представлял Господу овец, и он помог тому юнцу, и показал ему все, чтобы пришла
его помощь.
И я видел, как тот Господь овец пришел к ним во гневе, и все видевшие
Его убежали, и упали все в Его тени пред лицом Его.
Все орлы, и коршуны, и вороны, и ястребы собрались и привели с собою
всех полевых овец, и все они сошлись и помогали друг другу, сломить тот рог
юнца.
И я видел того мужа, который писал книгу по повелению Господа, как он
развернул ту книгу умертвления, которое совершили те двенадцать последних
пастырей, и он показал пред Господом овец, что они умертвили гораздо более, чем
предшествовавшие.
И я видел, как пришел к ним (к хищным птицам и зверям) Господь овец,
и взял в Свою руку посох гнева, и ударил в землю, так что она расторгалась, и
все звери и птицы небесные упали с овец, и погрузились в землю, и она
замкнулась над ними.
И я видел, как овцам дан был великий меч: тогда овцы выступили против
тех полевых зверей, чтобы умертвить их, и все звери и птицы небесные
разбежались от их лица.
И я видел, как был воздвигнут престол в любимой земле, и Господь овец
воссел на нем; и он взял все запечатанные книги и раскрыл их пред Господом
овец.
И Господь призвал тех шесть (или семь) первых белых, чтобы они
принесли к Нему, начиная от первой звезды, пришедшей вперёд, все звезды, у
которых срамные члены были как срамные члены коней, и первую звезду, которая
ниспала прежде всех; и они принесли их все к Нему.
И Он сказал тому мужу, который писал пред Ним и который был одним из
тех шести (или семи) белых, и сказал ему: "возьми тех семьдесят пастырей,
которым Я предал овец, и которые взяли их и умертвили из них более, чем Я им
повелел, самовластно"!
И вот я видел их всех связанными, и они все стояли пред Ним.
И суд совершился, прежде всего, над звездами, и они были судимы и
оказались виновными, и пришли к месту осуждения, и их бросили в глубокое место,
наполненное огнем, пылающее и наполненное огненными столбами.
И те семьдесят пастырей были судимы и оказались виновными, и точно
также были брошены в ту огненную пропасть.
И я видел тогда, как была открыта подобная пропасть в средине земли,
наполненная огнем, и как принесли тех ослепленных овец, и они все были судимы и
найдены виновными, и брошены в ту огненную пропасть, и они сгорели: а пропасть
эта была направо от того дома.
И я видел, как сгорели те овцы, и кости их сгорели.
И я встал, чтобы видеть, как Он украшал тот древний дом: и выломали в
нем все столбы, и все балки и украшения этого дома были завернуты вместе с
ними; и выломали их совсем, и положили их в одно место на юге страны.
И я видел Господа овец, как он принес новый дом больше и выше того
первого, и поставил его на месте первого, который был завернут; все его столбы
были новы и больше, чем украшения первого древнего, который Он выломал; и все
овцы были в нем.
И я видел всех овец, которые остались целыми, и всех зверей на земле
и всех птиц небесных, как они пали ниц и воздавали честь тем овцам, и умоляли
их, и слушались их в каждом слове.
И после этого меня взяли те трое, одетые в белом, которые подняли
меня прежде, за мою руку, и в то время, как рука того юнца взяла меня, они
подняли меня и посадили меня посреди тех овец, прежде чем совершился суд.
А те овцы были все белы и их шерсть была большая и чистая.
И все и все погибшие и рассеянные овцы, и все звери полевые, и все
птицы небесные собрались в том доме, и у Господа овец была великая радость, так
как все они были добры и возвратились к Его дому.
И я видел, как они сложили тот меч, который был дан овцам, и принесли
назад в Его дом, и он был запечатан пред лицом Господа; и все овцы были
заключены в тот дом, и он не вмещал их.
И у них у всех были открыты глаза, так что они видели доброе, и не
было между ними ни одной, которая бы не сделалась видящею.
И я видел, что тот дом был велик и широк, и весьма наполнен.
И я видел, что родился белый телец с большими рогами, и все звери
полевые и все птицы небесные устрашились его и умоляли его все время.
И я видел, как весь род их изменился, и все они стали белыми
тельцами; и первый между ними (был Слово, и это Слово сделалось) сделался
великим зверем, и имел большие черные рога на своей голове; и Господь овец
радовался, взирая на них и на всех тельцов.
И я спал в среде их, затем пробудился и видел все.
Таково видение, которое я видел в то время, как спал, и я пробудился
и прославил Господа правды, и воздал Ему хвалу.
И после этого я поднял великий вопль, и мои слезы не останавливались,
так как я не мог более удержаться; когда я смотрел, то у меня лились слезы по
поводу того, что я видел, ибо все придет и исполнится; и всякое деяние людей
мне было показано по порядку.
И в ту ночь я вспомнил о моем первом сне; также и из-за этого я
плакал и трепетал, ибо я видел то видение.
\vs 1En 18:1
И теперь, мой сын Мафусаил, призови ко мне всех своих братьев, и
собери ко мне всех сыновей твоей матери; ибо слово побуждает меня и дух излился
на меня, чтобы я открыл вам все, что придет на вас до вечности.
После этого Мафусаил пошел и призвал всех своих братьев к себе, и
собрал своих родственников.
И он (Енох) говорил со всеми своими детьми о правде, и сказал:
"вслушайтесь, сыны мои, каждую речь вашего отца и должным образом внемли гласу
моих уст, ибо я увещеваю вам, возлюбленные мои: любите праведность и ходите в
ней.
И не приближайтесь к праведности с двояким сердцем, и не
присоединяйтесь к тем, у которых двоякое сердце, но ходите в правде, сыны мои;
и она приведет вас на добрые пути, и правда будет вашей помощницей.
Ибо я знаю, что дела насилия возьмут верх на земле, и великое осуждение
совершится на земле; и всякая неправда прекратится и будет отделена от своих
корней, и все здание ее исчезнет.
И неправда опять повторится, и все дела неправды и все дела насилия и
беззакония вторично совершатся на земле.
И так как тогда усилится неправда, и грех, и хула, и насилие, и другого
рода действия, и увеличится отпадение, и беззаконие, и нечистота, то придет
великое осуждение с неба на всех них, и святой Господь выйдет с гневом и
наказанием, чтобы совершить суд на земле.
В те дни насилие будет отделено от своих корней, и корни неправды
погибнут вместе с ложью, и они исчезнут из-под неба.
И все идолы язычников будут преданы погибели; башни будут сожжены
огнем, и их уберут со всей земли; и они будут брошены по осуждению в огнь, и
погибнут в гневе и жестоком осуждении, которое продолжится вовек.
И восстанет тогда праведный от сна, и мудрость восстанет и будет дана
им.
И после того корни неправды будет отделены, и грешники погибнут от
меча, у клеветников будут отделены корни во всяком месте, и те, которые
замышляют насилие и произносят хулу, погибнут от острия меча.
И после этого будет другая седмина~--- восьмая, седмина правды; и будет
дан ей меч, чтобы судить и справедливость исполнить над теми, которые поступают
насильственно, и грешники будут преданы в руки праведных.
И в конце ее они приобретут домы своею спаведливостю, и создастся дом
великому Царю в прославление навсегда и навечно.
И после этого в девятую седмину откроется всему миру праведный суд, и
все деяния нечестивых исчезнут со всей земли; и мир будет присужден к погибели,
и все люди будут взирать на путь праведности.
И после этого в десятую седьмину, в седьмую ее часть, будет суд на
вечность, который совершится над стражами, и явится великое небо,
произрастающее из среды ангелов.
И прежнее небо уменьшится и исчезнет, и явится новое небо, и все силы
небесные седмерицею будут светить вовек.
И после этого будет много седьмин без числа в вечность во благо и в
правду, и с тех пор грех не будет более именоваться до вечности.
И теперь я говорю вам, мои сыны, и указываю вам пути правды и пути
насилия, и я укажу вам их опять, чтобы вы знали, что придет.
И теперь послушайте, мои сыны, и ходите в путях правды, и не ходите по
путям насилия, ибо навеки погибнут все, ходящие путями неправды.
\vs 1En 19:1
Написанное Енохом писцом пространное учение мудрости,~---
которое заслуживает прославления от всех людей и есть судья всей земли,~---
для всех моих детей, которые будут жить на земле, и для будущих родов,
которые будут ходить в праведности и мире.
Да не смущается дух ваш из-за времен, ибо Святой и Великий всему
положил дни.
И праведный восстанет от сна, восстанет и пойдет по пути правды, и весь
его путь и стезя будут в вечном благе и милости для праведного, и даст
господство, и он будет жить во благе и правде, будет ходить в вечном свете.
И погибнет грех во мраке навсегда и навечно, и более уже не появится от
того дня до вечности.
И после этого Енох начал возвещать из книг.
И сказал Енох: "о детях правды, и об избранных мира, и о растении
справедливости и праведности, говорю я это вам, мои сыны,~--- я Енох,~--- согласно с
тем, что мне открыто в небесном видении, и что я знаю чрез слово святых ангелов
и что узнал из скрижалей небесных".
И Енох начал теперь повествовать из книг и сказал: "я родился седьмым
в первую седьмину, когда суд и правда еще медлили.
И после меня во вторую седьмину восстанет великая злоба и произрастет
обман, и во время нее будет первый конец, и во время ее спасется один муж; и
после того, как он (конец) совершится, возрастет неправда, и Он даст закон
грешникам.
И после этого в третью седьмину, в конце ее, будет избран в растение
праведного суда один муж, и после него явится растение правды навсегда и
навечно.
И после этого в четвертую седьмину, в конце ее, будут видимы видения
святых и праведных, и закон для всех будущих родов и двор будет сделан (дан)
им, И после этого в пятую седьмину, в конце ее, будет устроен дом славы и
господства навсегда и навечно.
И после этого в шестую седмину все, которые будут жить во время ее,
будут ослеплены, и все они погрузятся своею мыслью в неразумие, забыв мудрость;
и во время нее будет взят вверх один муж; и в конце его господства будет сожжен
огнем, и весь род избранного корня будет рассеян.
И после этого в седьмую седьмину восстанет отпадший (или развращенный)
род, и много будет деяний его, и все его деяния будут отпадением.
И в конце ее будут награждены избранные и праведные от вечного
растения правды, между тем как им будет дано седьмикратное наставление обо всем
Его творении.
Ибо есть ли где-нибудь сын человеческий, который услышал бы голос
Святого и не был бы потрясен?
И есть ли где-нибудь такой, кто мог бы мыслить его мысли?
И где есть такой, кто мог бы видеть все произведения неба?
И как мог бы существовать тот, кто узнал бы произведения неба, и
увидел бы Его дыхание, и Его дух, и подсказал бы о том, или вошел бы наверх и
увидел все концы (буквально~--- крылья) их (небес), и мог бы придумать их, или
сделать что подобное им?
И есть ли где-нибудь такой муж, который мог бы знать, какова широта и
длина земли, и кому открыта мера всего этого?
И найдется ли кто-нибудь, который мог бы знать длину неба, и какова
его высота, и на чем оно утверждено, и как велико число звезд, и где покоятся
все светила?
И теперь я говорю вам, мои сыны, любите правду и ходите в ней,
ибо пути правды достойны, чтобы принять их; а пути неправды исчезают внезапно и
погибают.
И некоторым людям из грядущих родов будут открыты пути насилия и
смерти, и они будут держать себя далеко от них, и не будут им следовать.
И теперь я говорю вам~--- праведным: ходите не по злому пути и не в
насилии, и не по путям смерти, и не приближайтесь к ним, чтобы вам не
погибнуть.
Но ищите и изберите себе правду и приятную для Бога жизнь, и ходите по
путям мира, чтобы выжили и имели радость.
И держите в мыслях вашего сердца и не допускайте, чтобы речь моя
искоренилась из вашего сердца, ибо я знаю, что грешники соблазнят людей~---
унижать мудрость, и она не приобретет нигде места, и искушения всякого рода не
уменьшатся.
Горе тем, которые созидают неправду и насилие, и полагают основание
обману; ибо они внезапно будут искоренены и не будут иметь мира.
Горе тем, которые строят свои дома грехом, ибо они будут искоренены до
основания и падут от меча; и приобретающие золото и серебро внезапно погибнут
на суде.
Горе вам, богатые, ибо вы положитесь на ваше богатство, и вы лишитесь
своего богатства, так как вы не думали о Всевышнем в дни своего богатства.
Вы творили хулу и неправду, и приготовили себя ко дню кровопролития, и
ко дню мрака, и ко дню великого суда.
Это я говорю вам, что вас истребит до основания Тот, Кто сотворил вас:
и не будет никакого сострадания к вашему падению; и ваш Творец будет радоваться
вашей погибели.
И ваши праведники в те дни будут служить поношением для грешников и
нечестивых.
О, если бы мои очи были водной тучей, чтобы плакать о вас, и
излить мои слезы как водную тучу, дабы я получил успокоение для своего сердца
от печали!
Кто позволил вам совершать ненависть и злобу?
Так пусть же постигнет вас, грешники, суд!
Не страшитесь грешников, вы~--- праведные, ибо Господь опять предаст их
в ваши руки, чтобы вы совершили над ними суд, как желаете.
Горе вам, изрекающим проклятие, чтобы проклинать неразрешимо; и ваше
исцеление должно быть далеко от вас вследствие ваших грехов!
Горе вам, воздающим своему ближнему злом, ибо вам будет уготовано по
вашим делам!
Горе вам лжесвидетелям и тем, которые показывают неправду, ибо вы
внезапно погибнете!
Горе вам, грешникам, так как вы преследуете праведных; ибо вы будете
преданы и преследуемы, вы~--- люди неправды, и тяжело будет на вас их (праведных)
ярмо.
Вы, праведные, надейтесь, ибо грешники внезапно погибнут пред
вами, и вы будете господствовать над ними, как желаете!
И в день страдания грешников ваши юнцы вознесутся и взлетят, как орлы,
и выше, чем у коршуна, будет ваше гнездо, вознесетесь; и как кролики вы
проникнете в ущелье земли и в расселины скал навсегда пред праведными; а они
будут воздыхать из-за вас и плакать, как лесные духи.
Но и вы не бойтесь,~--- вы страдающие, ибо для вас будет исцеление, и
будет светить вам блестящий свет, и призыв к покою вы услышите с неба.
Горе вам, вы~--- грешники, ибо ваше богатство позволяет вам казаться
праведным, но ваше сердце изобличает вас, что вы грешники, и эта речь будет
свидетельствовать против вас для напоминания о ваших злодеяниях.
Горе вам, которые едите тук пшеницы и пьете силу корня источника, и
попираете своею силою приниженных!
Горе вам, которые всегда пьете воду, ибо вам внезапно будет воздано, и
вы завянете и иссохнете, так как вы оставили источник жизни!
Горе вам, совершающим неправду, и обман, и хулу: это будет памятью
против вас к вашему злу!
Горе вам, сильные, поражающие своею силою праведного, ибо придет день
вашей погибели, в то время много хороших дней придет для праведных день~---
вашего суда.
Веруйте вы, праведные, ибо грешники будут позором для вас и
погибнут в день неправды.
Да будет вам (грешникам) известно, что Всевышний думает о вашей
погибели, и ангелы радуются вашей погибели.
Что будете вы делать, грешники, и куда убежите в тот день суда, когда
услышите голос молитвы праведных?
И для вас не будет того, что для них,~--- для вас, против которых будет
свидетельством это слово: "вы сделались союзниками грешников".
И в те дни молитва праведных проникнет к Господу, и для вас наступят
дни вашего суда.
И все ваши неправедные речи будут прочитаны пред Великим и Святым, и
ваше лицо пристыдится, и всякое дело, основанное на неправде, будет отринуто.
Горе вам, грешникам, в средине моря и на суше, воспоминание которых о вас
недоброе!
Горе вам, приобретающим себе серебро и золото не по правде и
говорящим: "мы сделались богатыми и имеем сокровища, и владеем всем, чего
хотим; и теперь мы исполним все то, что нам думается, ибо мы собрали серебра и
наполнили наши кладовые, и как воды много у нас оберегающих наши дома".
как вода, разольется ваша ложь, ибо богатство не сохранится у вас, но
внезапно будет у вас отнято, так как вы все приобрели неправдою, и вы сами
подвергнетесь великому осуждению.
И теперь я клянусь вам, мудрым и безумным; ибо вы много
переживете (или увидите) на земле.
Ибо вы, мужи, будете возлагать на себя украшений более, нежели жены, и
разноцветного более, чем дева, в царском достоинстве и величии и власти, и в
серебре, и в золоте, и в пурпуре, и в почести, и в пище они разольются, как
вода.
Посему им не достает учения и мудрости, и чрез то они погибнут вместе
со своими сокровищами, и со всею своею силою и почестью; и в позоре, и в
умертвлении, и в великой бедности их дух будет брошен в огненную печь.
Я клянусь вам, грешники: как гора не была и не будет рабой, ни
возвышенность служанкой жены, так точно и грех не был послан на землю, но люди
произвели его из своей головы; и великому осуждению подпадут те, которые
совершают его.
И неплодие не дано было жене, но ради дела своих рук она умирает без
детей, Я клянусь вам, грешники, Святым и Великим, что всякое злое дело ваше
открыто на небесах, и ни одно из ваших деяний насилия не утаено или прикрыто.
И не думайте в своем духе и не говорите в своем сердце,~--- вы не знаете
и не видите, что каждый грех записывается ежедневно на небе пред Всевышним.
Отныне вы знайте, что все ваше насилие, которое вы совершаете,
записывается каждый день до дня вашего суда, Горе вам, безумные, ибо вы
погибнете чрез ваше безумие; и так как вы не слушаетесь мудрых, то ничто доброе
не будет вашим уделом.
И теперь знайте, что вы приготовлены на день погибели, и не надейтесь,
что вы будете жить,~--- вы грешники,~--- но вы погибнете и умрете, так как вы не
знаете никакого выкупа: ибо вы приготовлены на день великого суда, и на день
страдания и великого позора для вашего духа.
Горе вам,~--- вы ожесточенные, которые делаете зло и едите кровь!
Откуда у вас хорошая пища, и питье, и насыщение?
От всякого блага, которое наш Господь, Всевышний в изобилии послал на
землю: посему вы не должны иметь мира.
Горе вам, любящим свои деяния неправды!
Почему вы чаете блага для себя?
Знайте, что вы будете преданы в руки праведных; они перережут ваши шеи
и умертвят вас, и не будут иметь сострадания к вам.
Горе вам, радующимся страданию праведных, ибо для вас не будет вырыта
могила!
Горе вам, для которых слова праведных только пустые речи, ибо для вас
не будет надежды на жизнь!
Горе вам, записывающим лживые речи и беззаконные слова; ибо они
записывают свою ложь, чтобы их слушали и не забывали их безумия; так не будет
же для них мира, но они умрут внезапной смертью!
Горе тем, которые совершают нечестие, и похваляют и сохраняют в
уважении лживые речи: вы погибнете чрез это и для вас нет хорошей жизни!
Горе вам, искажающим слова праведности!
И они отпадут от вечного закона и сами себя делают тем, чем не были,
именно~--- грешниками; они будут попираемы на земле.
В те дни вы, праведные, приготовьтесь вознести свои мысленные молитвы,
вы представите их как свидетельство ангелам, чтобы они представили беззакония
грешников Всевышнему в напоминание.
В те дни народы придут в смятение, и поколения народов восстанут ко
дню погибели.
И в те дни выйдет плод материного чрева, и они (матери) растерзают
своих собственных детей; они оттолкнут от себя своих детей, и у них выпадет
недоношенный плод; грудных детей они оттолкнут от себя, и не возвратятся опять
к ним, и не сжалятся над своими любимцами.
Опять клянусь вам, грешники, что грех уготован на день беспрерывного
кровопролития.
И они будут поклоняться камням, и другие будут делать изображения из
золота и серебра, и из дерева и глины; и другие будут поклоняться нечистым
духам, и демонам, и разного рода идолам в идольских капищах: между тем у них
(идолов) нельзя найти никакой помощи.
И они погрузятся в неведение вследствие безумия своего сердца, и их
очи будут ослеплены страхом их сердца и сновидениями.
Чрез них они впадут в неведенье и страх, ибо они все свои дела
совершают во лжи, и поклоняются камням; и они погибнут все разом.
Но в те дни блаженны все те, которые принимают слова мудрости, и знают
ее, и исполняют пути Всевышнего, и ходят по пути правды, и с безбожными: ибо
они будут спасены.
Горе вам, распространяющим зло между своими ближними, ибо вы будете
умерщвленны в геенне.
Горе вам, полагающим основание греху и лжи, и вызывающим ожесточение
на земле: ибо за это их постигнет конец.
Горе вам, которые строите свои дома потом других и у которых
строительный материал есть не что иное, как черепица и камень греха; я говорю
вам, что для вас нет мира.
Горе тем, которые отвергают меру и наследие своих отцов, пребывающее
вечно, и прилепляют свои души к идолам: ибо для них не будет покоя.
Горе тем, которые делают неправду, и помогают насилию, и умерщвляют
своих ближних, в день великого суда: ибо Он низринет вашу славу, и положит вам
злобу на сердце, и возбудит дух Своего гнева, чтобы погубить вас всех мечом; и
все праведные и святые припомнят ваши грехи.
И в те дни будут умерщвлены в одном месте отцы вместе со своими
сынами, и братья друг с другом упадут от смерти, пока их кровь не потечет
подобно потоку.
Ибо муж не будет из сострадания удерживать свою руку от своих сынов и
от своих внуков, убивая их; и грешник не будет сдерживать своей руки от своего
почетнейшего брата; от утренней зари до солнечного захода они будут умерщвлять
друг друга.
И конь будет по самую грудь ходить в крови грешников, и колесница
потонет до своего верха.
И в те дни ангелы сойдут в убежища грешников и соберут в одно место
всех тех, которые помогали греху; и Всевышний восстанет в тот день, чтобы
произвести великий суд над всеми грешниками.
Но над всеми праведными и святыми Он поставит стражами святых ангелов,
чтобы они берегли их, как зеницу ока, пока не наступит конец всякой злобе и
всякого греха; и если даже праведные спят продолжительным сном, то и тогда они
не должны ничего бояться.
И кто мудр между людьми, тот увидит истину, и дети земли поймут все
слова этой книги, и узнают, что их богатство не может спасти их при погибели их
греха.
Горе вам, грешники, если вы мучите праведных,~--- в день жестокого
страдания,~--- и сжигаете их огнем: вам будет воздано по вашим делам.
Горе вам, развращенные сердцем, заботящиеся о том, чтобы измышлять
злое; на вас нападет страх, и никто не поможет вам.
Горе вам, грешники, ибо вы будете гореть в озере огненного пламени за
слова своих уст и за дела своих рук, которыми вы действуете нечестиво.
И теперь знайте, что ангелы на небе будут выведывать о ваших делах у
солнца, и луны, и звезд,~--- выведывать о ваших греховных делах, ибо вы
совершаете на земле суд над праведными.
И Он сделает свидетелями против вас каждую тучу, и облако, и росу, и
дождь, ибо все они задерживаются вами, чтобы не ходить на вас; и не должны ли
они думать о ваших грехах?
И теперь дайте дары дождю, чтобы он не был задержан от снисхождения
на вас, а также не была бы задержана роса, если она получила от вас золото и
серебро.
Когда будут падать на вас иней и снег вместе с их холодом, и все
снежные ветры со всеми своими бедствиями, то вы не устоите в те дни против них.
Рассмотрите небо, все дети земли, и каждое произведение
Всевышнего, и устрашайтесь пред Ним, и не делайте пред Ним ничего злого!
Если бы Он закрыл окна небесные и задержал из-за вас дождь и росу,
чтобы они не падали на землю, то что вы стали бы тогда делать?
И если Он посылает Свой гнев на вас и на все ваши произведения, то
можете ли вы не поклоняться Ему, так как вы высказываете надменные и бесстыдные
речи против Его правды, и для вас не будет мира.
И не видите ли вы управителей кораблей, как их корабли бросаются
волнами, качаются ветрами, и подвергаются опасности; и они вследствие этого
впадают в страх, так как они взяли с собою в море самое лучшее из своего
имения, и они беспокоятся в своем сердце, как бы море не поглотило их и как бы
они не погибли в нем?
Все море, и все его воды, и все его движение~--- не есть ли творение
Всевышнего, и не запечатал ли Он все Свое дело и не заключил ли его совсем в
песок?
Оно засыхает от Его угроз и устрашается, и все его рабы и все, что
есть в нем, умирают: и вы, грешники, живущие на земле, не боитесь Его.
Не сотворил ли Он небо, и землю, и все, что есть на них?
И кто дал учение и мудрость всем, которые движутся на земле и которые
живут в море?
Не боятся ли моря все цари кораблей?
А грешники не боятся Всевышнего.
В те дни, когда Он пошлет на вас мучительный огонь, куда вы
убежите, и где спасетесь?
И когда Он пошлет на вас Свое слово, не будете ли вы поражены тогда и
не устрашитесь ли?
Все светила потрясутся тогда от великого страха, и вся земля будет
поражена, и она задрожит и устрашится.
И все ангелы выполнят данные им повеления и будут стараться укрыться
пред Тем, Кто велик во славе, и дети земли задрожат и затрепещут; и вы, о
грешники, будете прокляты навеки, и пусть не будет для вас мира!
--- Не страшитесь вы, души праведных, и уповайте на день своей смерти в
правде!
И не печальтесь, что ваша душа нисходит в царство мертвых в великой
скорби, в горе, и воздыхании, и печали, и что ваше тело не обрело в вашей жизни
того, чего заслужила ваша благость, скорее теперь в день, когда вы стали
одинаковыми с грешниками, и в день проклятия и осуждения.
И когда вы умираете, грешники говорят над вами: "праведники умирают,
как и мы, и какая для них польза от их дел?
Вот они, как и мы, умерли в печали и мраке, и какое преимущество они
имеют пред нами?
отныне мы одинаковы.
И чего они достигнут этим, и что они увидят в вечности?
Ибо вот они также умерли и отныне не увидят света до века".
Я говорю вам, грешники: для вас достаточно есть, и пить, и обнажать
человека, и расхищать, и согрешать, и приобретать силу, и видеть хорошие дни.
Видели ли вы праведных, как конец их был мирен, ибо никакого рода
насилия не было в них по день их смерти, "И они погибли, как бы и не
существовали, и их души в печали сошли в царство мертвых".
И теперь я клянусь вам праведным Его великою славою и честью, и
Его достохвальным царством, и Его владычеством я клянусь вам: я знаю эту тайну
и прочитал ее на небесных скрижалях, и видел книгу святых, и нашел написанное и
отмеченное в ней относительно них, что для них уготовано всякое благо, и
радость и почесть; и я нашел записанное относительно духов тех, которые умерли
в правде; и узнали, что вам будет воздано многими благами за ваши труды, и ваша
участь лучше, чем участь живущих.
И будут жить ваши духи,~--- вы, умершие в правде; и будут радоваться и
ликовать их духи, и память о них будет пред лицом Великого на все роды мира:
так не страшитесь же их поношения!
Горе вам, грешники, когда вы умираете в своих грехах и подобные вам
говорят о вас: "блаженны грешники, они видели все свои дни; и теперь они умерли
в счастье и в богатстве, и не видели в своей жизни ни горести, ни убийства; в
славе они умерли, и во время их жизни суд не совершился над ними".
Но знаете ли вы, что души их должны сойти в царство мертвых, и они
найдут его невыносимым, и велика будет печаль их?
И во время великого суда ваш дух сойдет во мраке, и в сети, и в
плавающее пламя, и великий суд будет для всех родов до века: горе вам, ибо для
вас нет мира!
Не говорите праведным и добрым, которые еще живут: "в дни нашего
бедствия мы трудились, и побеждали всякую нужду, и встречались со всякими
бедствиями; мы не могли ничего сделать против врагов ни словом, ни делом, и
совершенно ничего не достигли; мы мучились и погибали, и не могли надеяться
видеть жизнь день за днем.
Мы надеялись быть главою, а сделались хвостом; мы измучились в
работах и не получили плодов своего труда, мы сделались пищею для грешников,
неправедные сделали для нас тяжким свое ярмо.
Владыками над нами были те, которые ненавидели нас и били нас: и мы
должны были склонять свои головы пред ненавидящими нас, и они не имели
сострадания к нам.
Мы старались ускользнуть от них, чтобы убежать и получить успокоение,
но мы не находили, куда бежать нам и спастись от них.
Мы жаловались на них в своей горести властителям, и сетовали на тех,
которые поедали нас; но они не взирали на наш вопль и не хотели слышать нашего
голоса.
И они помогали тем, которые обкрадывали нас и поедали, и тем, которые
принижали нас; и они утаивали их притеснения, так что не снимали с нас их ярма,
но поедали нас, и прогоняли, и убивали: и они утаивали умерщвление нас, и не
думали о том, что они подняли свои руки против нас".
Я клянусь вам, праведные, что ангелы на небе напоминают о вас
пред славою Великого к вашему благу, и ваши имена записаны пред славою
Великого.
Надейтесь вы, праведные, ибо прежде вы были в позоре, и несчастии, и
бедствии, а теперь вы будете светить, как светила небесные, и будете видимы, и
врата небесные отверзнутся для вас.
И ваш вопль о суде продолжается: он откроется для вас, ибо
властителям отомстится за ваше страдание, и всем помощникам тех, которые
обкрадывали вас.
Надейтесь и не покидайте свои надежды: ибо вы будете иметь великую
радость, как ангелы небесные, Так как вам предстоит таковое, то вы не будете
скрываться в день великого суда, и не будете найдены подобными грешникам, и от
вас далеко будет вечное осуждение, на все роды мира.
И теперь вы не бойтесь, праведные, когда видите грешников
усиливающимися и услаждающимися в своем веселие, и не имейте никакого общения с
ними, но держитесь в отдалении, ибо вы должны быть союзниками небесных воинств.
Вы грешники, хотя и говорите: "вы не можете разузнать этого и наши
грехи не записаны все", однако же они (ангелы) каждый день записывают ваши
грехи.
И теперь я открываю вам, что свет и мрак, день и ночь видят все ваши
грехи.
Не будьте нечестивыми в своем сердце, и не лгите, не изменяйте слов
праведности (или истины), и не выдавайте за ложь слов Святого и Великого, и не
прославляйте своих идолов; ибо вся ваша ложь и ваше нечестие служит не к
правде, а к великому греху.
И теперь я знаю эту тайну, что многие грешники изменят слова
праведности (или истины) и отпадут от них, и будут говорить двойные речи, и
говорить ложь, и творить великие (греховные) дела, и писать книги о своих
речах.
Но когда они все мои слова пишут правильно на своих языках, и ничего
не изменяют и не пропускают из моих слов, но все пишут правильно,~--- все, что я
прежде утверждал относительно них; то я знаю другую тайну, что именно только
праведным и мудрым даны книги к радости, и к праведности, и к великой мудрости,
и им даны книги, и они уверуют в них и возрадуются о них; и получат награду все
праведные, научившиеся из них знать все пути праведности.
"И в те дни, говорит Господь, они (праведные) должны воззвать к
сынам земли и представить свидетельство относительно мудрости их (книг);
покажите им их,~--- ибо вы их вожди,~--- и награды для всей земли.
Ибо Я и Мой Сын соединимся с ними навсегда и навечно на путях
праведности в их жизни.
И мир будет с вами: радуйтесь, вы~--- дети праведности, воистину"!
\vs 1En 20:1
И после некоторого времени мой сын Мафусаил взял своему сыну
Лемеху жену, и она зачала от него и родила сына.
Тело его было бело, как снег, и красно, как роза, и его волосы головные
и темянные были, как волна (руно), и его глаза были прекрасны; и когда он
открыл свои глаза, то они осветили весь дом подобно солнцу, так что весь дом
сделался необычайно светлым.
И как только он был взят из руки повивальной бабки, то открыл свои уста
и начал говорить к Господу правды.
И его отец Ламех устрашился этого, и удалился, и пришел к своему отцу
Мафусаилу.
И он сказал ему: "я родил необыкновенного сына; он не как человек, а
похож на детей небесных ангелов, ибо он родился иначе, нежели мы: его глаза
подобны лучам солнца и его лицо блестящее.
И мне кажется, что он происходит не от меня, а от ангелов; и я боюсь,
как бы в его дни не произошло на земле чудо.
И теперь, мой отец, я здесь с неотступною просьбою к тебе о том, чтобы
ты отправился к нашему отцу Еноху и выведал от него истину, ибо он имеет свое
жилище возле ангелов".
И когда Мафусаил слушал речь своего сына, то пришел ко мне к пределам
земли,~--- ибо он слышал, что я там,~--- и воскликнул; и я услышал его голос, и
пришел к нему, и сказал ему: "вот я здесь, мой сын, ибо ты пришел ко мне".
И он отвечал мне и сказал: "ради важного дела я пришел к тебе, и из-за
тревожного случая я приблизился сюда.
И теперь, отец мой, выслушай меня: у моего сына Ламеха родился сын,
образ и вид которого не как вид человека; его цвет белее, нежели снег, и
краснее розы, и его головные волосы белее, чем белое руно, и его глаза, как
лучи солнца; и он открыл свои глаза, и вот они осветили весь дом.
И взятый из руки повивальной бабки он открыл свои уста и прославил
Господа неба.
Тогда устрашился отец его Ламех и прибежал ко мне; и он не верит, что
он произошел от него, но что будто он подобие ангелов небесных; и вот я пришел
к тебе, чтобы ты открыл мне истину".
И я, Енох, отвечал и сказал ему: "Господь совершит на земле новое, и
это я знаю, и я видел в видении, и открыл тебе, что в век моего отца Иареда
некоторые ангелы, сошедшие с высоты неба преступили слово Господне.
И вот они совершили грех, и преступили закон, и соединились с женами,
и совершили с ними грех, и взяли жен из них, и родили с ними детей.
И великая погибель придет на всю землю, придет потоп, и будет великая
погибель в продолжение года.
Этот сын, родившийся у вас, останется на земле, и три его сына
спасутся вместе с ним; когда все люди живущие на земле, умрут, он и его сыновья
спасутся.
[Они рождают на земле исполинов не по духу, а по плоти, и за это
придет великое наказание на землю, земля будет вполне омыта от всей нечистоты].
И теперь извести сына своего Ламеха, что родившийся есть действительно
его сын, и нареки ему имя Ной, ибо он будет для вас остатком; и он и его
сыновья спасутся от уничтожения, которое придет на землю за все грехи и за
всякую неправду, которые совершаются на земле в его дни.
И после того неправда будет еще гораздо больше, чем та, которая
совершалась на земле прежде, ибо я знаю тайны святых,
так как Он~--- Господь~--- дозволил мне видеть их и открыл их мне, и я почитал их на скрижалях небесных.
И я видел написанное на них, что род за родом будет
беззаконовать, пока не восстанет род правды, и беззаконие будет обречено на
погибель, и грех исчезнет с земли, и все доброе появится на ней.
И теперь, мой сын, иди и возвести своему сыну Ламеху, что этот
родившийся сын есть действительно его сын, и это не ложь".
И когда Мафусаил выслушал речь своего отца Еноха,~--- ибо все тайные
вещи он открыл ему,~--- то возвратился, увидевшись с ним (Енохом), назад, и нарёк
тому сыну имя Ной, ибо он утешит землю в вознаграждение за всю погибель.
Другое писание, которое Енох написал для своего сына Мафусаила и
для всех, которые придут после него и будут сохранять закон в последние дни.
Вы, исполнившие его и теперь ожидающие, как в те дни совершится конец
над теми, которые делают злое, и сила беззаконников окончится,~--- вы ожидайте
только, когда минует грех, ибо имя их (грешников) будет изглажено из книг
святых и семя их погибнет навсегда и навечно, и их духи будут умерщвлены, и они
будут восклицать и взывать в пустом необитаемом месте и гореть в огне, где нет
земли.
И я видел там нечто похожее на облако, чего нельзя было узнать, ибо
вследствие глубины его (этого места) я не мог взглянуть на него; и я увидел там
ярко~--- пылающее пламя огня, и там кружились предметы, как блестящие горы, и
двигались туда и сюда.
И я спросил одного из святых ангелов, бывших при мне, и сказал ему:
"что это такое блестящее?
ибо это не небо, а только пламя пылающего огня и звуки вопля, и плача,
и сетования, и жестокого страдания".
28 И он сказал мне: "в это место, которое ты видишь,~--- сюда приносятся
духи грешников, и хулителей, и тех, которые делают злое и изменяют всё, что Бог
сказал устами пророков о будущем.
Ибо об этом есть писания и начертания вверху на небе, чтобы ангелы
читали их и знали, что случится с грешниками и духами покорных и тех, которые
умерщвляли свою плоть и за это получили от Бога награду, и тех, которые были
обесчещены злыми людьми; которые любили Бога, не любили ни золота, ни серебра,
ни всех благ мира, но предавали свое тело мучению; и которые, со времени своего
бытия, домогались не земных явлений, а считали самих себя за преходящее дыхание
и сообразно с этим жили, и были многократно испытываемы Господом, но их души
были обретены в чистоте, чтобы прославить Его имя.
Все благословения, которые они получают, я представил в книгах; и Он
назначал им за это награду, ибо они обрелись возлюбившими более вечное небо,
чем свою жизнь, и в то время, как были попираемы злыми людьми, и должны были
выслушывать от них оскорбления и хуления, и были обесчещиваемы, они прославили
Меня".
И теперь Я призову духов добрых людей из поколения света, и произведу
перемену с теми, которые родились во тьме и которые в своей плоти не были
награждены почестью, как надлежало за их верность.
И Я введу в блистающий свет любивших Мое святое имя, и посажу каждого
из них отдельно на престоле почести,~--- его почести.
И они будут блистать в продолжение бесчисленных времен, ибо
справедливость есть суд Божий и верным Он даст верность в жилище праведных
путей.
И они (праведные) увидят, как родившиеся во тьме будут брошены во
тьму, между тем как праведные будут блистать.
И грешники воскликнут и увидят, как они блистают: и они также пойдут
туда, где им написаны дни и времена.

\bibbookdescr{2En}{
  inline={Вторая книга Еноха},
  toc={2-я Еноха},
  bookmark={2-я Еноха},
  header={2-я Еноха},
  abbr={2~Ено}
}
\vs 2En 1:1
Мужа мудрого, великого книжника, которого взял Господь, дабы он увидел и возлюбил высшее житие, непреходящее царство премудрого и великого Бога Вседержителя,
\vs 2En 1:2
дабы стал он свидетелем превеликого, многоочитого и непоколебимого престола Господа, пресветлого предстояния слуг Господа и степеней их господства,
\vs 2En 1:3
геенны огненной, неисчислимого состава воинства небесного, многого множества стихий и различных видений, несказанного пения воинства Херувов, и света безмерного.
\vs 2En 1:4
Когда мне исполнилось сто шестьдесят пять лет, сказал Енох, у меня родился сын Мафусаил. Затем я прожил еще двести лет. А вся моя жизнь продолжалась триста шестьдесят пять лет.
\vs 2En 1:5
В первый месяц, в известный день первого месяца я, Енох, был в доме своем один.
\vs 2En 1:6
И когда лежал я на ложе своем и спал, обильная скорбь охватила сердце мое, и я сказал: Вот, очи мои испускают слезы (ибо во сне я не мог понять, что означает сия скорбь). Что будет со мною?"
\vs 2En 1:7
И явились мне два мужа столь великих, каких никогда не видел я на земле: лица их сияли подобно солнцу, а очи их были словно свечи горящие,
\vs 2En 1:8
из уст их исходил как бы огонь, и одеяния их были как струящаяся пена, светлее золота крылья их, и руки их белее снега.
\vs 2En 1:9
И стали они у изголовья моего и позвали меня по имени.
\vs 2En 1:10
И пробудился я от сна моего, и вот, мужи те стоят предо мною наяву.
\vs 2En 1:11
И встал я поспешно и поклонился им, и вспыхнуло лицо мое от страха перед увиденным.
\vs 2En 1:12
И сказали мне мужи те: Ободрись, Енох, не бойся, Господь вечный послал нас к тебе, в день сей восходишь ты с нами на небо.
\vs 2En 1:13
Скажи же сынам своим все, что нужно им сделать на земле; и пусть никто из дома твоего не ищет тебя до тех пор, пока не возвратит тебя к ним Господь".

\vs 2En 2:1
И послушался я их, и пошел, и призвал сыновей своих Мафуселу и Ригима, и поведал им то, что сказали мне мужи те:
\vs 2En 2:2
И вот я знаю, дети, что я не знаю, куда иду и что встретит меня.
\vs 2En 2:3
Вы же, дети мои, не отступайте от Бога, и пред лицем Господним ходите, и соблюдайте суды Его,
\vs 2En 2:4
не отвергайте жертв спасения вашего и не отвергнет Господь труд рук ваших;
\vs 2En 2:5
не лишайте даров Господа и не лишит Господь приращений Своих в хранилищах ваших!
\vs 2En 2:6
Благословляйте Господа первенцами от стад скота вашего и будете благословенны перед Господом во веки.
\vs 2En 2:7
Не отступайте от Господа и не поклоняйтесь богам суетным, не сотворившим ни небес, ни земли;
\vs 2En 2:8
и они погибнут, и те, что поклонятся им.
\vs 2En 2:9
Да утвердит Господь сердца ваши в страхе своем!
\vs 2En 2:10
И ныне, дети мои, пусть никто не ищет меня, доколе Господь не возвратит меня к вам.

\vs 2En 3:1
И было, когда говорил я сыновьям своим, позвали меня мужи те и взяли на крылья свои.
\vs 2En 3:2
И вознесли меня на первое небо, и поставили меня там.
\vs 2En 3:3
И привели пред лице мое верховных владык чинов звездных, и показали мне путь и движение их от года до года.
\vs 2En 3:4
И показали мне двести ангелов, которые управляют звездами и составом небес.
\vs 2En 3:5
И показали мне там море огромное, большее моря земного.
\vs 2En 3:6
И вокруг ангелы летали на крыльях своих.
\vs 2En 3:7
И показали мне хранилища снега и льда и грозных ангелов, стражей хранилищ тех.
\vs 2En 3:8
И показали мне там хранилища облаков, откуда они выходят и куда входят.
\vs 2En 3:9
И показали мне хранилища росы, подобной елею масличному; и ангелов, стерегущих сокровища те, и вид их как все цветы земные.

\vs 2En 4:1
И взяли меня мужи те, и поставили меня на втором небе, и показали мне узников, соблюдаемых для суда безмерного.
\vs 2En 4:2
И там видел я ангелов осужденных, плачущих, и спросил я мужей, которые были со мною: За что они мучимы?
\vs 2En 4:3
И отвечали мне мужи те: Это отступники от Господа, не послушавшиеся гласа Господня, но своею волею державшие совет.
\vs 2En 4:4
И опечалился я о них. И ангелы те поклонились мне, и сказали: Муж Божий, помолись бы о нас ко Господу.
\vs 2En 4:5
И я отвечал им, и сказал: Кто я, человек смертный, чтобы молиться об ангелах; кто знает, куда иду или что встретит меня, или кто помолится обо мне?

\vs 2En 5:1
И взяли меня оттуда мужи те, и возвели на третье небо, и поставили меня посреди рая.
\vs 2En 5:2
И место то невыразимо красотою вида его: всякое дерево цветами украшено, и всякий плод зрел, и всякие яства вечно изобилуют, всякое дуновение благовонно.
\vs 2En 5:3
И четыре реки протекают там покойным течением.
\vs 2En 5:4
И всякий злак, который рождается в пищу, прекрасен.
\vs 2En 5:5
И древо жизни на месте том, и на нем почивает Господь, когда входит Господь в рай, и древо то несказанно прекрасно благоуханием.
\vs 2En 5:6
И рядом другое древо масличное, постоянно источающее елей.
\vs 2En 5:7
И всякое дерево благоплодно, и нет там дерева безплодного; и все место то благовонно.
\vs 2En 5:8
И Ангелы, охраняющие рай, светлы весьма, непрестанным гласом сладкопения своего служат Богу во все дни.
\vs 2En 5:9
И сказал я: Сколь благо это место весьма!
\vs 2En 5:10
Отвечали мне мужи те: Место это, Енох, уготовано праведникам, которые претерпят напасти в этой жизни, и душам которых причинят зло, и которые отвратят очи свои от неправды и сотворят суд праведный
\vs 2En 5:11
чтобы дать хлеб алчущим и покрыть нагого одеждой, и поднять падшего, и помочь обиженным;
\vs 2En 5:12
которые пред лицем Господа ходят и Ему одному служат, тем уготовано сие в наследие вечное.

\vs 2En 6:1
И взяли меня оттуда мужи те, и вознесли меня на север неба, и показали мне там место весьма страшное:
\vs 2En 6:2
всякое томление и мучение на месте том, и тьма, и мгла, и нет там света, но огонь мрачный разгорается всегда на месте том, и река огненная растекается на все места те;
\vs 2En 6:3
лед холодный, и темницы, и ангелы лютые и неистовые, носящие оружие и мучающие без милости.
\vs 2En 6:4
И сказал я: Как страшно место это весьма!
\vs 2En 6:5
И отвечали мне мужи те: Это место, Енох, уготовано нечестивым, творящим безбожное на земле,
\vs 2En 6:6
тем, которые творят колдовство и волхвование, и похваляются делами своими, и тайно крадут души, и разрешают бремя связанное,
\vs 2En 6:7
тем, которые богатеют в ущерб имуществу чужому, и умерщвляют голодом алчущего, дабы самим насытиться; и имея возможность одеть нагих, раздевают;
\vs 2En 6:8
тем, которые не познали Творца своего, но поклонялись богам суетным, создавая идолов и поклоняясь творению рук своих.
\vs 2En 6:9
И всем тем уготовано это место в удел вечный.

\vs 2En 7:1
И взяли меня оттуда мужи те и подняли на четвертое небо, и показали мне там все движение солнца и луны и все лучи их.
\vs 2En 7:2
И измерил я путь их, и рассчитал свет их,
\vs 2En 7:3
и видел я: солнце имеет свет, в семь раз больший луны.
\vs 2En 7:4
И видел я круг их и колесницы, на которых ездит каждый из них, как ходит ветер,
\vs 2En 7:5
и нет им покоя, день и ночь ходящим и возвращающимся.
\vs 2En 7:6
И я видел четыре звезды великих, висящих справа от колесницы солнца, и четыре слева от солнца всегда.
\vs 2En 7:7
И ангелы движутся перед колесницей солнечной, духи летающие;
\vs 2En 7:8
двенадцать крыльев у каждого ангела, что мчат колесницу солнца, неся росу и зной, когда повелит им Господь сойти на землю с лучами солнечными.

\vs 2En 8:1
И отнесли меня мужи те на восток неба, и показали мне врата, из которых выходит солнце в положенные времена и по обращениям луны всего года,
\vs 2En 8:2
и при убавлении, и при возрастании дня, сообразно уменьшению и возрастанию дня и ночи:
\vs 2En 8:3
шесть ворот одинаковых, отверстых в тридцать одну стадию ровно, и я измерил величину их, и не мог постичь величины их.
\vs 2En 8:4
И те врата ими восходит солнце и идет на запад:
\vs 2En 8:5
первыми вратами выходит оно сорок два дня, вторыми тридцать пять дней,
\vs 2En 8:6
третьими тридцать пять дней, четвертыми тридцать пять дней,
\vs 2En 8:7
пятыми тридцать пять дней, шестыми сорок два дня,
\vs 2En 8:8
и снова возвращается шестыми вратами по истечению срока своего.
\vs 2En 8:9
И оно входит пятыми воротами тридцать пять дней, четвертыми воротами тридцать пять дней,
\vs 2En 8:10
третьими воротами тридцать пять дней, вторыми тридцать пять дней,
\vs 2En 8:11
и заканчиваются дни года по обращению времен.

\vs 2En 9:1
И возвели меня мужи те на запад неба, и показали мне там шесть великих ворот отверстых, поставленных против ворот восточных.
\vs 2En 9:2
И ими заходит солнце по обращении по небу из восточных ворот: по восходу из восточных ворот и по числу дней так же заходит в западные ворота.
\vs 2En 9:3
И когда изойдет оно из западных ворот, берут четыре ангела венец его и возносят ко Господу,
\vs 2En 9:4
а солнце поворачивает колесницу свою и идет без света, и там возлагают на него венец.
\vs 2En 9:5
И вот, показали они мне порядок солнца и ворот, которыми оно восходит и заходит, ибо эти ворота сотворил Господь. И солнце смену времен года указывает.
\vs 2En 9:7
А лунный порядок другой. И показали они мне весь путь ее и все обращения ее,
\vs 2En 9:8
и показали мне мужи те врата, и показали мне двенадцать ворот на востоке и двенадцать ворот на западе,
\vs 2En 9:9
и круги, по которым восходит и заходит луна по установленному времени:
\vs 2En 9:10
первыми воротами на востоке тридцать один день точно, вторыми тридцать пять дней точно,
\vs 2En 9:11
третьими тридцать один день ровно, четвертыми тридцать дней точно,
\vs 2En 9:12
пятыми тридцать один день особо, шестыми тридцать один день точно,
\vs 2En 9:13
седьмыми тридцать дней точно, восьмыми тридцать один день особо,
\vs 2En 9:14
девятыми тридцать один день определенно, десятыми тридцать дней точно,
\vs 2En 9:15
одиннадцатыми тридцать один день ровно, двенадцатыми воротами входит двадцать два дня точно.
\vs 2En 9:16
Также и западными воротамим по обращению и по числу восточных ворот.
\vs 2En 9:17
Так входит она и западными воротами и совершает год в триста шестьдесят пять дней \ldots
\vs 2En 9:18
Когда заканчиваются западные ворота, возвращается и идет к восточным со светом своим.
\vs 2En 9:19
И так ходит кругом день и ночь, и круг ее подобен небу.
\vs 2En 9:20
И колесницу, на которую она восходит, влечет ветер.
\vs 2En 9:21
И движется колесница ее летящими духами, у каждого из ангелов по шесть крыльев. Таков порядок лунный.
\vs 2En 9:22
И видел я посреди неба воинов вооруженных, служащих Богу непрестанным гласом тимпанов и органов, и я наслаждался, слушая их.

\vs 2En 10:1
И взяли меня оттуда мужи те, и вознесли меня на пятое небо.
\vs 2En 10:2
И я видел там множество Бодрствующих, видел я двести.
\vs 2En 10:3
Видом своим как люди, величиной же больше чудес великих, лица их печальны, уста их молчат, и не было там служения.
\vs 2En 10:4
И сказал я у мужам, бывшим со мною: Почему столь печальны и унылы лица их, и уста их молчат, и нет службы на небе этом?
\vs 2En 10:5
И отвечали мне мужи те: Это Бодрствующие, которые отпали от Господа: двое князей и двести ходящих вслед князей тех;
\vs 2En 10:6
и сошли они на землю и исполнили обет свой на хребте горы Ермон, чтобы оскверниться с женами человеческими.
\vs 2En 10:7
И за осквернение то осудил их Господь, и вот, рыдают они о братии своей, и о бывшей укоризне.
\vs 2En 10:8
И сказал я Бодрствующим: Я видел братию вашу и дела их познал, и вот мольбу их видел, и молился о них.
\vs 2En 10:9
И вот осудил их Господь под землю, доколе не придет конец неба и земли.
\vs 2En 10:10
Зачем же ждете вы братьев своих и не служите пред лицем Господа?
\vs 2En 10:11
Восстановите прежнее служение, служите во имя Господне! Ведь если разгневаете Господа, Бога вашего, свергнет Он вас с места этого.
\vs 2En 10:12
И вняли они увещанию моему и стали на небе по четырем чинам;
\vs 2En 10:13
и пока я стоял там, вострубили одновременно в четыре трубы и стали служить Бодрствующие, и поднялся голос их единым гласом к лицу Господа.

\vs 2En 11:1
И подняли меня оттуда мужи те, и вознесли меня на шестое небо.
\vs 2En 11:2
И увидел я там семь ангелов, стоящих вместе, светлых и славных весьма: лица их как лучи солнечные блистают, и нет различия ни в лицах, ни во власти, ни в содержании власти их.
\vs 2En 11:3
Они устрояют и преподают благой порядок миру: движению звезд, солнца и луны, и ангелов, возящих их, и небесным гласам, и умиротворяют все бытие небес;
\vs 2En 11:4
и устрояют заповеди и поучения, и сладкогласное пение, и всякую хвалу и славу.
\vs 2En 11:5
И есть ангелы над временами и годами, и ангелы над реками и морями, и ангелы над плодами и травою, и над всем прозябающим;
\vs 2En 11:6
и ангелы всех народов, управляющие всею жизнью их и записывающие ее перед лицом Господа.
\vs 2En 11:7
И среди них семь Фениксов, и семь Херувов и семь Серафов, единогласных голосами своими и пением своим, и неизъяснимо пение их.
\vs 2En 11:8
И радуется Господь подножию Своему.

\vs 2En 12:1
И подняли меня оттуда мужи те и вознесли меня на седьмое небо.
\vs 2En 12:2
И увидел я свет великий и все огненное воинство безплотных: архангелов, ангелов, и светозарное стояние Офанов.
\vs 2En 12:3
И я устрашился, и вострепетал, и взяли меня иужи те, и поставили среди них, говоря мне: Ободрись, Енох, не бойся!
\vs 2En 12:4
И показали мне издалека Господа, сидящего на престоле Своем, и все воинства небесные, соединенные по чину, приступали и кланялись Господу,
\vs 2En 12:5
и снова отходили, и шли на места свои в радости и веселии, и в свете безмерном.
\vs 2En 12:6
И Славные служат ему неотступно день и ночь, стоя перед лицем Господа и творя волю Его.
\vs 2En 12:7
И все воинство Херувов вокруг престола Его неотступно, и Серафы покрывают престол Его, воспевая пред лицем Господа.
\vs 2En 12:8
И когда я увидел все это, то отошли от меня мужи те, и больше я не видел их.
\vs 2En 12:9
И они поставили меня одного на краю неба, и испугался я, и пал на лице свое.
\vs 2En 12:10
И послал Господь одного из Славных своих ко мне, Гавриила, и он сказал мне. Дерзай, Енох, не бойся! Встань и пойди со мной, и стань перед лицем Господним во веки.
\vs 2En 12:11
Я же отвечал ему, говоря: Увы мне, господин, душа моя отступила из меня от страха.
\vs 2En 12:12
Позови ко мне мужей, приведших меня на место сие, ибо к ним я имею доверие и с ними пришел я пред лице Господне.

\vs 2En 13:1
И восхитил меня Гавриил, как восхищается лист ветром, и погнал меня, и поставил меня пред лицем Господним.
\vs 2En 13:2
И увидел я Господа, и лицо его могущественно, и преславно и страшно.
\vs 2En 13:3
Но кто я, чтобы поведать, объять подлинное лице Господа, могущественное и весьма страшное,
\vs 2En 13:4
или изречь о хорах окрест Его, многоочитых и многогласных, о престоле Его, весьма великом и нерукотворенном,
\vs 2En 13:5
или стояние, которое пред Ним, воинства Херувов и Серафов, или неизменное, неисповедимое, неумолкающее славное служение Ему?
\vs 2En 13:6
И пал я на лице свое, и поклонился Господу.
\vs 2En 13:7
И Господь устами Своими воззвал ко мне: Дерзай, Енох, не бойся! Восстань и стань пред лицем Моим во веки.
\vs 2En 13:8
И поднял меня Михаил, архангел Господень, и привел меня пред лице Господа.
\vs 2En 13:9
И испытал Господь слуг своих, сказав им: Да вступит Енох, чтобы стоять пред лицем Моим во веки.
\vs 2En 13:10
И Славные поклонились Ему и сказали: Да вступит.
\vs 2En 13:11
И сказал Господь Михаилу: Возьми Еноха, и сними с него земные одежды, и помажь елеем многоценным, и облеки в ризы славы.
\vs 2En 13:12
И снял Михаил одежды мои с меня, и помазал меня елеем благим.
\vs 2En 13:13
И вид елея ярче света великого, и умащение им словно роса добрая, и благоухание его подобно мирре, и лучи его как солнечные.
\vs 2En 13:14
И оглядел я всего себя: и стал я, как один из Славных, и не было на вид различия.

\vs 2En 14:1
И призвал Господь Веревеила, одного из архангелов Своих, который был мудр и записывал все дела Господни.
\vs 2En 14:2
И сказал Господь Веревеилу: Возьми книги из хранилищ и дай Еноху трость и прочти ему книги.
\vs 2En 14:3
И поспешил Веревеил и принес мне книги, изукрашенные смирной.
\vs 2En 14:4
И дал мне трость из руки своей, и стал рассказывать мне все дела Господни: о земле, о море, о всех стихиях, о движении всех планет и бытии их,
\vs 2En 14:5
о смене лет и движении дней, о земных заповедях и наставлениях, о сладкогласном пении, о входах облаков и исходах ветров,
\vs 2En 14:6
о всяком народе, и о новой песне вооруженного воинства все, чему следовало меня научить, поведал мне Веревеил.
\vs 2En 14:7
Тридцать дней и тридцать ночей говорили уста его, не умолкая.
\vs 2En 14:8
И я не спал тридцать дней и тридцать ночей, записывая все знаками.
\vs 2En 14:9
Когда же закончил, сказал мне Веревеил: Сядь, напиши то, что я поведал тебе.
\vs 2En 14:10
И я, просидев еще тридцать дней и тридцать ночей, записал все подробно и отчетливо, и поведал это в трехсот и шестидесяти книгах.

\vs 2En 15:1
И призвал меня Господь, и поставил меня слева от Себя рядом с Гавриилом, и я поклонился Господу.
\vs 2En 15:2
И сказал мне Господь: Все, что ты видел, Енох, неподвижное и движущееся, сотворено Мною, и Я возвещу тебе о том от начала.
\vs 2En 15:3
Прежде, когда не было всего, что Я привел из небытия в бытие, и из невидимого в видимое, и ангелам Моим не возвестил Я тайны Моей, и не поведал им, как сотворил их, и не постигли они бесконечного Моего и непостижимого творения, тебе же Я возвещаю сегодня.
\vs 2En 15:4
Прежде, когда не было всего видимого, открылся свет, и Я среди света, будучи невидим, один проезжал, как ездит солнце от востока до запада и от запада на восток,
\vs 2En 15:5
но солнце находит покой, Я же не обрел покоя, поскольку все было несотворенным.
\vs 2En 15:6
И помыслил Я поставить основание, сотворить тварь видимую. И повелел Я в глубине, да взойдет одно из невидимых в видимое.
\vs 2En 15:7
И вышла Божественная вечность, весьма великая, и вот, имела она во чреве своем великий век.
\vs 2En 15:8
И Я сказал ей: Разрешись от бремени, о Божественная вечность, и да будет видимо разрешаемое из тебя.
\vs 2En 15:9
И разрешилась она, и вышел из нее великий век, и таким образом изнесло все творение, которое Я восхотел сотворить. И увидел Я, что это хорошо.
\vs 2En 15:10
И поставил Я Себе престол, и сел на нем, свету же сказал: Взойди ввысь, и утвердись, и будь основанием горнему. И нет превыше света ничего иного.
\vs 2En 15:11
И увидев это, Я восклонился с престола Моего, и воззвал в глубине во второй раз, и сказал: Да произыдет из невидимого твердь, и да станет видима!
\vs 2En 15:12
И произошло основание тверди, тяжелое и весьма мрачное. И увидел Я, что это хорошо.
\vs 2En 15:13
И Я сказал ему: Сойди вниз, и утвердись, и будь основанием дольнему.
\vs 2En 15:14
И сошло оно, и утвердилось, и стало основанием дольнему. И ниже тьмы нет ничего иного.
\vs 2En 15:15
И облек Я эфир светом, уплотнил Я его и простер Я его над тьмою, а из вод утвердил камни великие.
\vs 2En 15:16
Водам же бездны повелел Я высохнуть, а впадины Я назвал безднами.
\vs 2En 15:17
Море Я собрал в одно место, связал его узами, и дал морю границу вечную, и не исторгнется из вод.
\vs 2En 15:18
И поставил Я твердь и основал ее поверх вод.
\vs 2En 15:19
И помимо всего воинства небесного, образовал Я на небесах солнце от света великого, и поставил его на небе, дабы светило оно на землю.
\vs 2En 15:20
Из камня Я высек огонь великий и из огня сотворил все воинства безплотные и все воинства звезд. И Херувов, и Серафов, и Офанов и это все Я высек из огня.
\vs 2En 15:21
Земле же Я повелел произрастить всякое дерево, и всякую гору, и всякую траву живую, и всякое семя живое, сеющее семя, прежде, чем сотворить души живые, Я приготовил пищу им.
\vs 2En 15:22
И морю повелел Я породить в себе рыб и всяких гадов, ползающих по земле, и всякую птицу парящую.
\vs 2En 15:23
И когда закончил все, повелел Я Премудрости Своей сотворить человека.

\vs 2En 16:1
И ныне, то, что Я рассказал тебе, и то, что ты видел на небесах, и то, что видел на земле, и то, что ты написал в книгах, создал Я Премудростью Своею и искусством Своим, и сотворил от нижнего основания до высшего.
\vs 2En 16:2
И до скончания их нет Мне советника, ни помощника.
\vs 2En 16:3
Сам Я вечен, нерукотворен. Неизменна мысль моя, советник Мой, и слово Мое есть дело, и очи Мои следят за всем: если отверну лице Мое все погибнут, если же призираю утверждаются.
\vs 2En 16:4
Положи, Енох, ум свой, и познай Говорящего с тобою, и возьми книги, которые ты написал.
\vs 2En 16:5
И даю тебе Семеила и Рагуила, возведших тебя ко Мне, и сойди на землю и расскажи сынам своим то, что Я говорил тебе, и то, что видел ты от нижнего неба и до престола Моего,
\vs 2En 16:6
все воинства Я сотворил, и нет противящегося Мне или не покоряющегося: все покоряются Моему единовластию и работают одной Моей власти.
\vs 2En 16:7
И дай им книги, соделанные рукою твоею, и прочтут они и познают Творца своего, и уразумеют и они, что нет иного Творца, кроме Меня.
\vs 2En 16:8
И передай книги, соделанные рукою твоею, детям и детям детей твоих, и дай наставления ближним из рода в род.
\vs 2En 16:9
Ибо Я дам тебе ходатая, о Енох, архистратига моего Михаила, чтобы написанное рукою твоею и написанное рукою отцов твоих, Адамом и Сифом, не погибло до века последнего, 10 Ибо Я заповедал ангелам Моим, Ариоху и Мариоху, которых поставил Я над землею, дабы хранили ее и повелевали временами,
\vs 2En 16:11
дабы соблюли они написанное рукою твоею и написанное рукою отцов твоих, и не погибло это в грядущий потоп, который Я сотворю в роде твоем.
\vs 2En 16:12
Я знаю злобу человеческую, что они не вынесут бремени, которое Я возложил на них, и не будут сеять семя, которое Я дал им, но отвергнут бремя Мое, и иное бремя примут,
\vs 2En 16:13
и посеют семена пустые, и поклонятся богам суетным, и отринут единовластие Мое, и вся земля согрешит неправдами, и обидами, и прелюбодейством, и идолослужением.
\vs 2En 16:14
Тогда наведу Я потоп на землю, и земля сама сокрушится в бездну великую.
\vs 2En 16:15
И Я оставлю мужа праведного из племени твоего, со всем домом его, который сотворит по воле Моей.
\vs 2En 16:16
И от семени их востанет род последний, многочисленный и весьма ненасытный.
\vs 2En 16:17
Тогда при исходе рода того явятся книги, написанные рукою твоею и отцов твоих, которые стражи земные покажут мужам верным, и расскажут роду тому и будут они почитаемы впоследствии больше, чем прежде.

\vs 2En 17:1
Ныне же, Енох, даю тебе срок ожидания тридцать дней, чтобы сотворил ты в доме твоем и рассказал сыновьям твоим и домочадцам твоим от Меня.
\vs 2En 17:2
И всякий, кто блюдет сердце свое, да прочтет и уразумеет, что нет никого, кроме Меня.
\vs 2En 17:3
И спустя тридцать дней Я пошлю ангелов за тобою, и возьмут тебя ко Мне от земли и от сыновей твоих;
\vs 2En 17:4
возьмут тебя ко Мне, ибо уготовано тебе место, и ты будешь перед лицем Моим отныне и до века.
\vs 2En 17:5
И будешь созерцать тайны Мои, и будешь книжником над рабами Моими,
\vs 2En 17:6
ибо будешь записывать все дела земные и о пребывающих на земле и на небесах, и будешь Мне свидетелем Суда Великого Века.
\vs 2En 17:7
Все сие говорил мне Господь, как говорит муж ближнему своему.

\vs 2En 18:1
И ныне, чада мои, услышьте голос отца вашего и то, что я заповедаю вам сегодня:
\vs 2En 18:2
ходите пред лицем Господним, и все, чему должно произойти по воле Господа.
\vs 2En 18:3
Ибо я послан от уст Господа к вам, дабы сказать вам, что есть и что будет до Дня Судного.
\vs 2En 18:4
И ныне, чада мои, не от уст моих вещаю вам сегодня, но от уст Господа, пославшего меня к вам.
\vs 2En 18:5
И вы слышите слова мои из уст моих, подобно вам созданного человека; я же слышал из уст Господа, огненных, ибо уста Господа как печь огненная, и слова его как пламя огненное исходят.
\vs 2En 18:6
Вы, чада мои, видите лице мое, подобно вам созданного человека, я же видел лице Господа, как железо, раскаленное огнем, испускающее искры.
\vs 2En 18:7
И вы видите глаза, подобно вам созданного человека, я же видел очи Господа, светящиеся, как лучи солнца, ужасающие глаза человеческие.
\vs 2En 18:8
И вы, дети, видите десницу мою, подающую вам знаки, подобно вам сотворенного человека, я же видел руку Господа, подающую знак мне, наполняющую небо.
\vs 2En 18:9
И вы видите охват тела моего, подобного вашему, я же видел объятие Господа, безграничное и несравненное, которому нет конца.
\vs 2En 18:10
И вы слышите слова из уст моих, я же слышал глаголы Господа, как гром великий, приводящие в непрестанное движение облака.
\vs 2En 18:11
Теперь, чада мои, услышьте беседующего о царе земном страшно и трепетно стоять перед лицем царя земного, страшно, потому что воля царя смерть, и воля царя жизнь.
\vs 2En 18:12
Каково же стоять перед лицем Царя Небесного? кто выдержит этот безмерный страх и жар великий?
\vs 2En 18:13
Но призвал Господь одного из ангелов своих верховных, грозного, и поставил рядом со мною,
\vs 2En 18:14
видом же ангел был, как снег, а руки его как лед, и он остудил лице мое, потому что не стерпел бы я страха и жара огненного.
\vs 2En 18:15
И тогда сказал мне Господь все слова Свои.

\vs 2En 19:1
И ныне, чада мои, я знаю все: одно из уст Господа, другое глаза мои видели; от начала и до конца, и от конца до нового обращения все я узнал.
\vs 2En 19:2
И записал я в книгах обо всем наполняющем небеса до краев их, я измерил путь их, и воинство их я узнал, и записал звезд многое множество бесчисленное.
\vs 2En 19:3
Кто из людей знает о круговом движении их и пути их, и обращении их, или о том, кто ведет их, или о ведомых?
\vs 2En 19:4
Ангелы не знают числа их, я же имена их записал.
\vs 2En 19:5
И солнечный круг я измерил, и лучи сосчитал, и весь путь его, и входы его и исходы его, и имена их записал.
\vs 2En 19:6
И лунный круг я измерил, и движение ее во все дни исчислил, и свет ее на всякий день и час, и в книгах имена ее записал.
\vs 2En 19:7
И жилища облаков, и устав их, и крылья их, и дождь их, и капли их я изследовал.
\vs 2En 19:8
И описал грохот грома и блеск молнии, и показали мне хранителей ключей их, и восходы их;
\vs 2En 19:9
ходят же по мере: узами поднимаются и узами опускаются, дабы тяжелым напором не обрушили облака, и не погубили то, что на земле.
\vs 2En 19:10
Я написал о сокровищницах снега и хранилищах льда и воздуха холодного;
\vs 2En 19:11
наблюдал я, как время от времени хранители ключей их наполняют ими облака, но не истощаются сокровищницы.
\vs 2En 19:12
Написал я о ложе ветров, смотрел я и увидел, как хранители ключей их, носящие весы с собой и меры, на одну чашу весов кладут сокровища, во вторую же меру, и по мере отпускают их на всю землю, дабы чрезмерным ветром не поколебать землю.

\vs 2En 20:1
Оттуда сведен я был вниз, и пришел на место судное, и я видел ад отверстый, и видел там тех, кому хуже, чем узникам, суд безмерный.
\vs 2En 20:2
И спустился я, и написал о всяком суде осужденных, и все вопрошения их увидел, и воздохнул, и заплакал о погибели нечестивых.
\vs 2En 20:3
И сказал я в сердце своем: Блажен, кто еще не родился, или родившийся, но не согрешивший перед лицем Господа, дабы не попал на место это и дабы не понес бремени места этого.
\vs 2En 20:4
И видел я хранителей ключей ада, стоящих у весьма великих ворот, подобных аспидам огромным,
\vs 2En 20:5
лица их как свечи потухшие, глаза их как пламя померкшее, и зубы их обнажены до персей их.
\vs 2En 20:6
И сказал я в лице их: Ушел бы я и не видел вас, ибо вы здесь за деяния ваши. И да не придет никто из племени моего к вам.

\vs 2En 21:1
И оттуда взошел я в рай праведных, и видел там место благословенное, и вся тварь благословенна, и все живут в радости и веселии, и в свете безмерном, и в жизни вечной.
\vs 2En 21:2
Тогда сказал я: Чада мои, говорю вам: блажен, кто боится имени Господа, и перед лицем Его будет служить всегда, и приготовит дары и приношения \ldots и жизнию поживет, и умрет.
\vs 2En 21:3
Блажен, кто будет творить суд праведный: нагого оденет в одежды и голодному даст хлеб.
\vs 2En 21:4
Блажен, кто судит суд праведный: сироте, и вдовице, и всякому обиженному поможет.
\vs 2En 21:5
Блажен, кто сойдет с пути временного и пойдет путями праведными.
\vs 2En 21:6
Блажен, кто сеет семена праведные, ибо и пожнет их седмерицею.
\vs 2En 21:7
Блажен, в ком есть истина, да говорит истину ближнему своему.
\vs 2En 21:8
Блажен, у кого в устах его милость истинная и кротость.
\vs 2En 21:9
Блажен, кто разумеет дела Господа, ибо по делам Его познается Создатель.
\vs 2En 21:10
И вот, дети мои, я обозрел землю до краев ее, и записал я все: все года сложил, и из лет разделил месяцы, и в месяце рассчитал дни, дни разделил на часы, часы же измерил.
\vs 2En 21:11
И описал всякое семя на земле, и разделил каждую меру, и каждые весы правильные я измерил, и описал.
\vs 2En 21:12
И как год от года разнится в достоинстве, так и человек от человека в чести:
\vs 2En 21:13
кто благодаря большому богатству, кто благодаря сердечной мудрости, а кто благодаря остроте ума или молчанию уст.
\vs 2En 21:14
Но нет никого более боящегося Господа, ибо боящиеся Господа славны будут во век.
\vs 2En 21:15
Господь руками Своими создал человека в подобие лица Своего, малого и великого сотворил Господь.
\vs 2En 21:16
Оскорбляющий лице человеческое оскорбляет лице Господа, гнушающийся лица человеческого гнушается лица Господа,
\vs 2En 21:17
презирающий лице человека презирает лице Господа; гнев и Суд Великий тем, кто плюет в лицо человеку.

\vs 2En 22:1
Блажен, кто приготовит себя всякому человеку: кто помогает осуждаемому, и кто поднимает упавшего, и кто подает просящему,
\vs 2En 22:2
ибо в день Суда Великого всякое дело человека обновится писанием.
\vs 2En 22:3
Блажен, у кого будет мера праведная, и весы праведные, и гири праведные,
\vs 2En 22:4
так как в день Суда Великого каждая мера, и каждые весы, и каждая гиря словно при покупке приложатся, и узнает каждый меру свою и по ней примет мзду.
\vs 2En 22:5
Тому, кто всегда творит пред лицем Господа, управит Господь приобретения его.
\vs 2En 22:6
Тому, кто умножает светильники пред лицем Господа, умножит Господь хранилища его.
\vs 2En 22:7
Разве нужны Господу хлеб или свеча, или овен, или телец? но этим испытывает Господь сердце человека.
\vs 2En 22:8
Ибо когда Господь пошлет свой свет великий во тьму и будет Суд, кто тогда утаиться?
\vs 2En 22:9
Ныне же, чада мои, положите помышление на сердцах ваших и внемлите словам отца вашего, ибо то, что я вещаю вам от уст Господних.
\vs 2En 22:10
Возьмите книги эти книги, написанные рукою отца вашего, и прочтите их, и из них узнаете дела Господа, и что нет никого, кроме Господа единого,
\vs 2En 22:11
Который поставил основания на неведомом, и простер небеса на невидимом, землю поставил, на водах ее основав непостоянных,
\vs 2En 22:12
Который безчисленную тварь сотворил один (а кто исчислил прах земной или песок морской, или капли облаков?),
\vs 2En 22:13
Который землю и море соединил нерушимыми узами,
\vs 2En 22:14
Который немыслимую красоту из огня высек и украсил небо,
\vs 2En 22:15
Который из невидимого в видимое все сотворил, Сам будучи невидимым.
\vs 2En 22:16
И раздайте эти книги детям вашим и детям детей ваших.
\vs 2En 22:17
И все ближние ваши, и все сродники ваши, которые знают и боятся Господа, да примут их, и да будут они им нужнее всякой пищи доброй, и да прочтут и прилепятся к ним!
\vs 2En 22:18
А неразумные, не знающие Господа, не примут их, но отвергнут, ибо отягчат они бремя их.
\vs 2En 22:19
Блажен, кто понесет бремя их, примет его, ибо обретет его в день Суда Великого.
\vs 2En 22:20
Ибо я клянусь вам, чада мои, что еще прежде, чем быть человеку, место Судное уготовано ему, и мера, и весы, которыми будет испытан человек, там прежде уготованы.
\vs 2En 22:21
Я же дело всякого человека в письменах изложу, и никто не сможет укрыться.

\vs 2En 23:1
И ныне, чада мои, пребывайте в терпении и кротости число дней ваших, да наследуете безконечный век будущий.
\vs 2En 23:2
И всякое бедствие, и всякое страдание, и зной, и всякое слово злое, если найдет на вас, потерпите Господа ради.
\vs 2En 23:3
И хотя можете отплатить расплатой, не воздавайте ближнему, ибо один Господь воздает, и будет отмщающим за вас в День Суда Великого.
\vs 2En 23:4
Золотом и серебром пожертвуйте ради брата, дабы принять сокровище плоти в День Судный.
\vs 2En 23:5
И к сироте и вдове прострите руки ваши, и по силе помогите бедному, и обретете покровительство во время всякого труда.
\vs 2En 23:6
Если найдет на вас скорбь и печаль, ради Господа отриньте, и обретете воздаяние в День Судный.
\vs 2En 23:7
Утром, и в полдень, и вечером благо есть ходить в дом Господень прославлять Творца всего.
\vs 2En 23:8
Блажен, кто раскрывает сердце свое для хвалы и хвалит Господа.
\vs 2En 23:9
Проклят раскрывающий сердце свое для хулы и клеветы на ближнего.
\vs 2En 23:10
Блажен, кто раскрывает уста свои, благословляя и прославляя Господа.
\vs 2En 23:11
Проклят раскрывающий уста свои для проклятия и хулы в лице Господа.
\vs 2En 23:12
Блажен прославляющий все дела Господни.
\vs 2En 23:13
Проклят оскорбляющий творение Господа.
\vs 2En 23:14
Блажен созидающий и воздвигающий трудом рук своих.
\vs 2En 23:15
Проклят стремящийся уничтожить труды чужие.
\vs 2En 23:16
Блажен хранящий устои отцов до конца.
\vs 2En 23:17
Проклят нарушающий установления и законы отцов своих.
\vs 2En 23:18
Благословен насаждающий мир.
\vs 2En 23:19
Проклят уничтожающий мирное.
\vs 2En 23:20
Благословен говорящий мир и имеющий мир в сердце своем.
\vs 2En 23:21
Проклят говорящий то, но не имеющий мира в сердце своем.
\vs 2En 23:22
Все это на весах и в книгах изобличится в День Суда Высшего.

\vs 2En 24:1
И ныне, чада мои, блюдите сердца ваши от всякой неправды, да унаследуете подножие света во веки.
\vs 2En 24:2
Не говорите, дети мои: отец наш с Господом и умолит нас о грехе.
\vs 2En 24:3
Знайте, что все дела всякого человека я записываю, и никто не может уничтожить написанного рукою моею, потому что Господь все видит.
\vs 2En 24:4
И теперь, чада мои, усвойте все слова отца вашего, которые я говорю вам, да будут они вам в достояние покоя.
\vs 2En 24:5
И книги, которые я дал вам, не отриньте их, но всем желающим растолкуйте их, может быть узнают дела Господа.
\vs 2En 24:6
И вот, чада мои, приближается назначенный день года, и время подходит установленное, и ангелы, которые идут со мною, стоят пред лицем моим.
\vs 2En 24:7
И утром я поднимусь на небо высшее, в вечное мое достояние, и потому заповедаю вам, дети мои, делайте все то, что благословенно пред лицем Господа.

\vs 2En 25:1
И отвечал Мафусела отцу своему Еноху: Что угодно очам твоим, отец? Да приготовим пищу пред лицем твоим;
\vs 2En 25:2
да благословишь дома наши и сыновей своих, и всех домочадцев своих, и прославишь народ свой, и после этого уйдешь.
\vs 2En 25:3
И ответил Енох сыну своему, говоря: Слушай, сын мой, от того дня, как помазал меня Господь елеем славы Своей, вострепетал я, и не услаждает меня пища, и не хочется мне ничего из земных яств.
\vs 2En 25:4
Но позови братьев своих и всех домочадцев наших, и старейшин народа, дабы я говорил с ними, и тогда отойду.
\vs 2En 25:5
И поспешил Мафусела, и позвал братьев своих Регима, и Ариима, и Ахазухана, и Харимиона, и старейшин народа, и привел их пред лице отца своего Еноха.
\vs 2En 25:6
И поклонились ему, и принял их Енох, и благословил их, и отвечал ним, говоря:
\vs 2En 25:7
Послушайте, чада! Во дни отца вашего Адама сошел Господь на землю, чтобы посетить ее и все сотворенное Им, которое Сам создал.
\vs 2En 25:8
И призвал Господь всех зверей земных и всех гадов земных, и всех птиц пернатых, и привел их пред лице отца вашего Адама, чтобы нарек он имена всем на земле.
\vs 2En 25:9
И оставил их Господь у него, и подчинил ему всех, сделав вторыми по меньшинству, и притупил весь разум их, дабы повиновались человеку.
\vs 2En 25:10
Ибо Господь сотворил человека над всем владением Своим, и за это не будет Суда никакой душе живой, но одному человеку.
\vs 2En 25:11
Всем душам скотов уготовано одно место, и предел один, и пастбище одно в Веке Великом.
\vs 2En 25:12
И не укроется ни одна душа живая, которую сотворил Господь, до Суда, и все души, которых оклевещут до Суда.
\vs 2En 25:13
И тот, кто плохо заботится о душе своей, сделает свою душу беззаконною.
\vs 2En 25:14
А приносящий жертву из чистого скота, имеет исцеление, он исцеляет душу свою.
\vs 2En 25:15
Умерщвляющий же скот всякий, не связав его, преступает закон, он предает душу свою беззаконию.
\vs 2En 25:16
И творящий злое животным в тайне, преступает закон, он предает душу свою беззаконию.
\vs 2En 25:17
Творящий злое душе человеческой, творит злое душе своей, и нет ему исцеления во веки.
\vs 2En 25:18
Толкающий человека в сеть, сам в ней увязнет, и нет ему исцеления во веки.
\vs 2En 25:19
И подвергающий человека осуждению, непременно будет осужден во веки.

\vs 2En 26:1
И ныне, чада мои, храните сердца ваши от всякой неправды, которую возненавидит Господь, более же всего по отношению ко всякой душе живой, которую создал Господь.
\vs 2En 26:2
И что просит человек для своей души от Господа, пусть тоже сотворит всякой душе живой.
\vs 2En 26:3
Потому что в Веке Великом многие обители уготованы человеку: обители добрые весьма и бесчисленные обители злые.
\vs 2En 26:4
Блажен, кто отойдет в благословенные обители, ибо из злых нет возвращения.
\vs 2En 26:5
Когда положит человек слово в сердце своем принести дар пред лицем Господа, а руки его того не сделают, тогда отвергнет Господь труд рук его, и не обретет ничего.
\vs 2En 26:6
И если сотворят руки его, но сердце его будет роптать, то не прекратится болезнь сердца его, ибо роптание его поспешно.
\vs 2En 26:7
Блажен человек, который в терпении своем принесет дар пред лицем Господа, ибо обретет воздаяние.
\vs 2En 26:8
И если человек назначит устами своими определенное время для принесения дара пред лицем Господа и совершит это то обретет воздаяние;
\vs 2En 26:9
если же пройдет назначенное время, и возвратит слова свои, то даже если раскается, не будет ему благословения.
\vs 2En 26:10
Потому что всякое промедление порождает искушение.
\vs 2En 26:11
Человек, который прикроет нагого и алчущему даст хлеб, обретет воздаяние.
\vs 2En 26:12
Если же станет роптать сердце его, то погубит себя и не будет ему воздаяния.
\vs 2En 26:13
И если нищий, когда насытится сердце его, возгордится, то погубит все добрые дела свои и не обретет воздаяния, ибо мерзок Господу всякий муж возгордившийся.

\vs 2En 27:1
И было, когда говорил Енох сыновьям своим и князьям народа, услышал его весь народ и все близкие его, что призывает Еноха Господь,
\vs 2En 27:2
и, посовещавшись, сказали: Идем и приветствуем Еноха.
\vs 2En 27:3
И собралось около двух тысяч мужей, и пришли на место Азухань, где был Енох и сыновья его, и старейшины народа, и приветствовали Еноха, говоря:
\vs 2En 27:4
Благословен ты у Господа, Царя Вечного, ныне же благослови народ свой и прославь его пред лицем Господа, ибо тебя избрал Господь в пророки, чтобы ты отнял грехи наши.
\vs 2En 27:5
И отвечал Енох народу своему, говоря: Слушайте, чада мои. Сначала, когда ничего не было, прежде, чем появилось все творение, создал Господь век сотворенный,
\vs 2En 27:6
и после этого сотворил все творение Свое, видимое и невидимое, и после всего этого создал человека по образу Своему,
\vs 2En 27:7
и вложил ему глаза, чтобы видеть, и уши, чтобы слышать, и сердце, чтобы разуметь, и ум, чтобы размышлять.
\vs 2En 27:8
Тогда освободил Господь век ради человека, и разделил его на времена и часы,
\vs 2En 27:9
да размышляет человек о смене времен, и о конце и начале лет, и окончании месяцев, и дней, и часов, да предаст ему свою жизнь и смерть.
\vs 2En 27:10
Когда же перестанет существовать все творение, которое сотворил Господь, и всякий человек придет на Суд Господа Великий,
\vs 2En 27:11
тогда исчезнут времена, и лет больше не будет, и ни месяцы, ни дни, ни часы более не будут сосчитываться, но настанет век единый.
\vs 2En 27:12
И все праведники, которые избегнут Суда Господня Великого, соединятся в Веке Великом, вместе соединятся праведники, и будут пребывать вечно.
\vs 2En 27:13
И более не будет у них ни труда, ни болезни, ни скорби, ни ожидания невзгод, ни тягот, ни ночи, ни тьмы; но свет великий будет для них всегда.
\vs 2En 27:14
И стена неразрушимая в раю великом будет защитой их жилища вечного.
\vs 2En 27:15
Блаженны праведники, которые избегнут Суда Господня Великого, ибо озарятся лица их подобно солнцу.
\vs 2En 27:16
Ныне же, чада мои, оберегайте души ваши от всякой неправды, которую возненавидел Господь;
\vs 2En 27:17
пред лицем Господа ходите и Ему одному служите, и всякое приношение приносите пред лице Господа.
\vs 2En 27:18
Если и посмотрит человек ввысь то там Господь, ибо Господь сотворил небеса;
\vs 2En 27:19
если посмотрит на землю и на море и подумает о том, что под землей, то и там Господь, ибо Господь сотворил все, и не скроется ни какое дело от лица Господня.
\vs 2En 27:20
Вы же с долготерпением и с кротостью, сквозь страдания и мучения, пройдете болезненный век сей.

\vs 2En 28:1
И когда говорил это Енох народу своему, послал Господь мрак на землю, и была тьма, и покрыла тьма стоящих с Енохом мужей.
\vs 2En 28:2
И поспешили ангелы, и взяли ангелы Еноха, и вознесли его на небо вышнее.
\vs 2En 28:3
И принял его Господь, и поставил его пред лицем Своим во веки.
\vs 2En 28:4
И отступила тьма от земли, и стал свет.
\vs 2En 28:5
И увидел народ, и понял, что взят был Енох, и, прославив Бога, пошли в дома свои.
\vs 2En 28:6
И поспешил Мафусела и братья его, сыновья Еноха, и сделали жертвенник на месте Азухань, откуда взят был Енох, и, взяв овнов и тельцов, принесли их в жертву пред лицем Господа.
\vs 2En 28:7
И созвали всех людей, дабы пришли к ним на пир.
\vs 2En 28:8
И принесли люди дары сыновьям Еноха.
\vs 2En 28:9
И радовались и веселились три дня.

\vs 2En 29:1
И в третий день, во время вечернее, сказали старейшины народа Мафуселе, говоря:
\vs 2En 29:2
Иди и встань пред лицем Господа и пред лицем народа своего пред алтарем Господним и будешь славен в народе твоем.
\vs 2En 29:3
И отвечал Мафусела народу своему: Господь Бог отца моего Еноха, Сам изберет священника над народом Своим.
\vs 2En 29:4
И ждал народ всю ночь ту на месте Азухань.
\vs 2En 29:5
И пребывал Мафусела у алтаря, и молился Господу, говоря: Господь всего века, сего ли сына отца нашего Еноха избрал ты?
\vs 2En 29:6
Господи, яви священника народу Своему, да в неразумии сердца их боятся славы Твоей, и сотвори все по воле Твоей!
\vs 2En 29:7
И уснул Мафусела, и явился ему Господь в видении ночном, и сказал ему:
\vs 2En 29:8
Слушай Мафусела, Я Господь Бог отца твоего Еноха, услышал Я глас народа Своего.
\vs 2En 29:9
Встань же пред ними и перед алтарем Моим, и прославлю тебя перед этим народом Моим во все дни жизни твоей.
\vs 2En 29:10
И востал Мафусела от сна своего, и благословил Явившегося ему.
\vs 2En 29:11
И утром пришли старейшины народа к Мафуселе, и направил Господь Бог сердце Мафуселы послушаться голоса народа,
\vs 2En 29:12
и сказал им: Да сотворит Господь Бог наш благое в глазах Его для этого народа Своего.
\vs 2En 29:13
И поспешили Сарсан, и Хармий, и Заза, старейшины народа, и облекли Мафуселу в одежду превосходную, и возложили венец светлый на голову его.
\vs 2En 29:14
И поспешно привел народ овнов и тельцов, и из птиц все, что положено, чтобы принес Мафусела жертву пред лицем Господа и пред лицем народа.
\vs 2En 29:15
И поднялся Мафусела к жертвеннику Господню, подобно восходящей деннице, и весь народ шел за ним.
\vs 2En 29:16
И стал Мафусела у алтаря, и весь народ вокруг алтаря.
\vs 2En 29:17
И старейшины народа, взяв овнов и тельцов, связали их по четыре ноги и положили на возглавие алтаря.
\vs 2En 29:18
И сказал народ Мафуселе: Возьми нож и заколи назначенное перед лицем Господа.
\vs 2En 29:19
И простер Мафусела руки свои к небу и призвал Господа, говоря:
\vs 2En 29:20
Увы мне, Господи, кто я, чтобы стоять у возглавия жертвенника Твоего, во главе всего народа Твоего, и для всего познания?
\vs 2En 29:21
Яви благодать рабу Твоему пред лицем народа сего, да знают, что это Ты! Назначь священника народу Своему!
\vs 2En 29:22
И было, когда молился Мафусела, сотрясся алтарь, и поднялся нож с алтаря, и вскочил нож в руки Мафуселе перед лицем всего народа.
\vs 2En 29:23
И объял всех людей трепет, и прославили Господа.
\vs 2En 29:24
И был почитаем Мафусела в глазах Господа и в глазах всего народа с того дня.
\vs 2En 29:25
И взял Мафусела нож, и совершил заклание всех приношений народа.
\vs 2En 29:26
И возрадовался народ, и возвеселился пред лицем Господа и пред лицем Мафуселы в тот день, и после этого разошлись по домам своим.
\vs 2En 29:27
И стоял Мафусела у возглавия алтаря и во главе всего народа с того дня триста девяносто два года.
\vs 2En 29:28
И благословил Господь Мафуселу в жертвах, и в дарах его, и во всем служении, которое совершал он пред лицем Господа.

\vs 2En 30:1
И когда приблизились к концу дни Мафуселы, явился ему Господь в видении ночном и сказал ему:
\vs 2En 30:2
Слушай, Мафусела, Я Бог отца твоего Еноха. Познай волю Мою, ибо кончились дни жизни твоей, и приблизился день отдохновения твоего.
\vs 2En 30:3
Ибо приблизились времена погибели всей земли, и всякого человека, и всего, что движется по земле, ибо во дни сии велико нестроение на земле.
\vs 2En 30:4
Ибо возненавидел человек ближнего своего, и люди на людей нападают, и народ против народа возбуждает брань, и наполнилась земля кровью и пагубным безпорядком.
\vs 2En 30:5
И ко всему этому они оставили Творца Своего, и стали поклоняться тверди небесной, и ходящим по земле, и волнам морским.
\vs 2En 30:6
И возвеличился Сатана, и радуется делам их.
\vs 2En 30:7
И к негодованию Моему, вся земля восприняла перемену устройства своего, и всякий плод, и всякая трава изменили пору свою, ибо предчувствуют время погибели.
\vs 2En 30:8
И все народы изменились на земле, к сожалению Моему.
\vs 2En 30:9
И тогда Я повелю бездне низринуться на землю, и запасы вод небесных устремятся на землю.
\vs 2En 30:10
И погибнет весь состав земной, и сотрясется вся земля, и лишится силы своей с того дня.
\vs 2En 30:11
Тогда Я сохраню Ноя, первородного сына Ламеха, сына твоего, и воссоздам от семени его иной мир, и семя его пребудет во веки.
\vs 2En 30:12
И пробудился Мафусела от сна своего, и весьма опечалился о сне этом;
\vs 2En 30:13
и призвал всех старейшин народа и поведал им все, что сказал ему Господь, и о видении, явившемся ему от Господа.
\vs 2En 30:14
И опечалились люди из-за видения его, и ответили ему: Господь властен творить по воле Своей.
\vs 2En 30:15
И когда говорил Мафусела народу, взволновался дух его, и преклонил он колени свои, и простер руки свои к небу, и молился Господу, и когда молился он, отошел дух его.

\bibbookdescr{Jub}{
  inline={Книга Юбилеев},
  toc={Книга Юбилеев},
  bookmark={Книга Юбилеев},
  header={Книга Юбилеев},
  abbr={Юбл}
}
\vs Jub 0:0
Вот слова деления дней по закону и свидетельству,
по событиям годов,
по их седминам, по их юбилеям, на все годы мира,
согласно с тем, что говорил он с Моисеем на горе Синай.

\vs Jub 1:1
Случилось в первый год по выходе сынов Израиля из Египта, в третий месяц, в
шестнадцатый день его, тогда сказал Бог Моисею, говоря: <<Взойди ко Мне на
гору, чтобы Я дал тебе две каменные скрижали закона и все заповеди, которые Я
написал, дабы ты возвестил их им (сынам Израиля)!>> И Моисей взошел на гору
Господню, и слава Господня обитала на горе Синай, и облако осеняло ее шесть
дней. И Он воззвал Моисея в седьмой день среди облака. И он видел славу Божию,
как пылающий огонь, на горе Синай, когда взошел, чтобы получить каменные
скрижали закона и заповедей, по слову Господа, как Он сказал ему:
<<Поднимись на вершину горы!>> И Моисей был на горе сорок дней и сорок
ночей, и Господь научил его относительно того, что было прежде и что случится в
будущем; Он изъяснил ему деление дней закона и свидетельства и сказал:
<<Внимай всем словам, которые Я тебе говорю, и запиши их в книгу, дабы их
роды (потомки) видели, как Я оставил их за все зло, какое сделали они,
уклонившись от завета, который Я утверждаю ныне между Мною и тобою на горе
Синай для будущих родов их. И будет это слово, когда придут все наказания,
свидетельствовать против них, и они познают, что Я справедливее, нежели они во
всей их правде и во всяком их деле, и узнают, что Я был с ними. И ты запиши
себе все слова, которые Я тебе возвещаю ныне,~--- ибо Я знаю их противление
и жестоковыйность,~--- прежде чем приведу их в землю, о которой клялся
Аврааму, Исааку и Иакову, когда сказал: <<Вашему семени Я дам землю,
текущую молоком и медом>>. И они будут есть, и насыщаться, и уклоняться к
чуждым богам, к тем, которые их не спасли от всей их тяготы. И будет это
свидетельство услышано во свидетельство им: ибо они будут забывать Мои
заповеди, все, что Я заповедаю им, и пойдут вослед язычников и за их нечистотою
и мерзостию, и будут служить их богам, и эти (боги) сделаются для них
претыканием в бедствие, и страдание, и сетию. И многие погибнут, и будут
пленены, и впадут в руки врага, так как они забудут Мои постановления, и Мои
заповеди, и Мои праздники, Мой завет, и Мои субботы, и Мою святыню, которую Я
освящу Себе между ними, и Мою скинию, и Мое святилище, которое Я освящу Себе в
стране, чтобы положить на нем Свое имя, дабы оно обитало там. И они будут
делать себе изображения из камня и из дерева, и будут преклоняться пред ними,
чтобы впадать в грехи, и будут приносить своих сыновей в жертву демонам и
предаваться всем делам заблуждения своего сердца. И Я буду посылать к
ним свидетелей, чтобы дать им свидетельство, но они не послушают их и будут
убивать Моих свидетелей; и также тех, которые следуют закону, они будут убивать
и преследовать, и отвергнут его (закон) совершенно, и начнут делать то, что
есть зло пред Моими очами. Тогда Я сокрою Свое лице от них, и предам их
язычникам в пленение, и в узы, и на истребление, и изгоню их из земли
(Ханаанской), и рассею между язычниками. И они забудут весь Мой закон, и
все Мои заповеди, и всю Мою правду, и не будут более хранить ни новолуния, ни
субботы, и никакого праздника и юбилейного года, и никакого установления. После
сего они опять обратятся ко Мне из среды язычников всем сердцем и всею
душою и всеми своими силами. И Я соберу их всех из среды язычников; и они опять
будут искать Меня, чтобы Я явил им Себя. Когда же они будут искать Меня всем
сердцем и всею душою, Я открою им великий мир с правдою и восставлю их как
растение праведности от всего Моего сердца и от всей души; и они будут
во благословение, а не в проклятие, и соделаются главою, а не хвостом.

И Я воссоздам Мое святилище между ними и буду обитать с ними, и буду их
Богом, и они будут Моим народом воистину и вправду; и Я не оставлю их, не
отрекусь от них, ибо Я Господь, Бог их>>.

И Моисей пал на свое лице, и молился, и говорил: <<Господи, Боже мой! не
оставляй Твоего народа и Твоего наследия, чтобы не ходить им в заблуждении
своего сердца, и не предавай их в руки врагов-язычников, чтобы они
владычествовали над ними; не допусти их до сего, чтобы им не потерять Тебя!
Простри, Господи, Свое милосердие над народом Своим, и дух правый соделай в
них, и не допусти духа Велиара владычествовать над ними, чтобы он
клеветал на них пред Тобою и совращал их со всех путей правды, дабы они
погибли пред лицем Твоим! Ибо они Твой народ и наследие, который Ты великою
силою освободил из рук египтян; соделай в них чистое сердце и святой дух и не
допусти их, чтобы они были доведены до падения чрез свои грехи, отныне и до
века!>>

И Бог сказал Моисею: <<Я знаю их противление, и их помышления, и их
жестоковыйность; они не покорятся, пока не познают своих грехов и грехов
отцов их. И после сего они обратятся ко Мне во всей праведности, и от всего
сердца, и от всей души, и Я обрежу крайнюю плоть их сердца и крайнюю плоть
сердца их семени, и соделаю в них дух святой, и очищу их, чтобы они более не
отвращались от Меня с того дня и до века. И их душа прилепится ко Мне и ко всем
Моим заповедям, и они будут исполнять Мои повеления, и Я буду их отцом, и они
будут Моим сыном, и все будут именоваться сынами Божиими и все сынами
Духа. И тогда откроется, что они сыны Мои и Я отец их в праведность и
правду, и что Я люблю их. Ты же запиши себе все эти слова, которые Я возвещаю
тебе на этой горе, первое и последнее, и грядущее, согласно со всем делением
времени под законом и свидетельством и по седминам юбилейных годов, до века,
пока Я не сойду и не буду жить с ними от века до века>>.

И Он сказал Ангелу лица: <<Запиши для Моисея события с первого
творения до того времени, когда Мое святилище будет устроено между ними,
навсегда и навечно, и Бог откроется для очей каждого, и всякий познает, что Я
Бог Израиля, и Отец всех детей Иакова, и Царь на горе Сионе, от века до века. И
Сион Иерусалим будет святым>>. И Ангел лица, который шел пред станом
израильтян, взял скрижали деления лет от творения, седмин и юбилеев, закона и
свидетельства, каждый год по его числу и юбилеи по годам, со дня нового
творения, когда были сотворены небо и земля и все их произведения, равно как и
небесные силы и все творение земли, до того времени, когда будет создано
святилище Господа в Иерусалиме на горе Сионе, и все светила будут обновлены к
освящению, и к миру, и к благословению для всех избранных Израиля, чтобы сие
пребывало так от того дня в продолжение всех дней земли.

\vs Jub 2:1
И Ангел лица сказал Моисею по слову Господа, говоря: <<Напиши все
повествование о творении, как Господь Бог совершил в шесть дней все Свои
произведения, которые Он сотворил, и в седьмой день соблюдал субботу, и освятил
ее на все века, и утвердил ее в знамение для всех Своих творений>>.

Ибо в первый день Он сотворил небеса, которые вверху, и землю, и воды, и
всех духов, которые Ему служат, и Ангелов лица, и Ангелов прославления, и
Ангелов духа огня, и Ангелов духа ветров, и Ангелов облачных духов мрака, и
града и инея, и Ангелов долин, и громов и молний, и Ангелов духов холода и
зноя, зимы и весны, осени и лета, и Ангелов всех духов Его творений на небе, и
на земле, и во всех долинах, и духов мрака и света, и утренней зари, и вечера,
которые Он приготовил по предвидению Своей премудрости. И тогда мы увидели Его
произведения, и прославили Его, и восхвалили Его за все произведения Его, ибо
семь великих произведений Он сотворил в первый день.

И во второй день Он сотворил твердь между водами; и разделились воды в тот
день: половина их поднялась вверх, и половина опустилась вниз под твердь,
которая в середине, на поверхность всей земли. И это единственное произведение,
которое Он сотворил во второй день.

И в третий день сотворил Он, как сказал водам: да стекут они с поверхности
всей земли в одно место и да явится суша. И Он сделал таким образом с водами,
как сказал им. И они стекли с поверхности земли в одно место, вне тверди, и
явилась суша. И в тот день Он создал для нее (воды) бездны морей по их
отдельным вместилищам, и все реки, и вместилища вод в горах и во всей земле, и
все озера, и всякую росу земную, и семя, которое сеется по роду своему, и все,
что употребляется в пищу, и плодовые и лесные деревья, и сад Едем для веселия.
Все эти четыре великие творения Он сотворил в третий день.

И в четвертый день Он сотворил солнце, и луну, и звезды, и поставил их на
тверди небесной, чтобы они светили на всю землю, и повелел им управлять днем и
ночью, и разделять между светом и между тьмою. И Бог сделал солнце великим
знамением на земле для дней, и суббот, и годов, и юбилеев, и для всех времен
года, и повелел ему разделять между светом и между тьмою, и
предназначил его для роста, чтобы росло все, что прозябает и
произрастает на земле. Эти три рода творения Он создал в четвертый
день.

И в пятый день Он сотворил больших морских животных в глубинах вод,~---
ибо они были созданы Его рукою прежде всего,~--- всякую плоть, и все, что
движется в водах, рыб, и все, что летает, птиц и весь их род. И солнце взошло
над ними для развития, и над всем, что существует на земле, и над всем, что
прозябает из земли, и над всеми плодовыми деревьями, и над всякою плотью. Все
эти три рода Он сотворил именно в пятый день.

И в шестой день Он создал всех зверей земных, и всякий скот, и все, что
движется на земле. И после всего этого Он сотворил человека, одного, мужа и
жену сотворил их, и поставил его владыкою над всем, что на земле и что в морях,
и над тем, что летает, и над зверями, и над скотом, и над всем, что движется на
земле, и над всею землею: надо всем этим Он сделал его господином. И эти четыре
рода творений Он сотворил в шестой день.

И было всего сотворено в шесть дней двадцать два рода. И Он закончил все
Свои произведения в шестой день~--- все, что на небе, и на земле, и в морях,
и в долинах, во свете и во тьме и всюду. И Он дал нам (Ангелам) великое
знамение~--- день субботний, чтобы мы в продолжение шести дней делали дела и
в седьмой день соблюдали субботу ото всех дел, все Ангелы лица и все Ангелы
прославления. Нам, этим двум великим родам, сказал Он, чтобы мы хранили с Ним
субботу на небе и на земле. И Он сказал нам: <<Вот Я выделю Себе народ из
среды всех народов, чтобы и они (он?) праздновали субботу; и Я освящу
его Себе в Свой народ, и благословлю его, как Я освятил день субботний и
посвятил их (субботы) Себе; так благословлю Я его; и они будут Моим народом, и
Я буду их Богом. И Я избрал семя Иакова между всеми из тех, которых Я увидел, и
написал Его у Себя перворожденным сыном, и освятил его для Себя навсегда и
навечно. И Я возвещу им о субботнем дне, чтобы они хранили в него субботу ото
всех дел своих>>. Так положил Он знамение в нем, дабы и они праздновали с
нами субботу в седьмой день, чтобы есть и пить, и прославлять Того, Кто
сотворил все, как и Он благословил сие и освятил Себя для Своего народа, чтобы
это было явлено пред всеми народами и чтобы они (потомки Иакова?) одинаково с
нами праздновали субботу. И Он установил, чтобы Его повеления возносились пред
Него, как благовоние, которое было бы приятно Ему, во все дни двадцати двух
глав людей от Адама до Иакова. И двадцать два рода произведений были сотворены
до сего седьмого дня. Этот благословлен и освящен, и тот (Иаков) также
благословлен и освящен. И этот вместе с тем служит к освящению и благословению.
И сему (Иакову и его потомкам) даровано было, чтобы они были всегда
благословенными и святыми в свидетельстве и законе, как прежде седьмой день Он
освятил и благословил быть седьмым днем (т.е. субботою). Он сотворил небо
и землю и все, что создано в шесть дней, и Господь установил святой праздник
для всех Своих тварей. Посему Он дал повеление относительно него всем Своим
творениям, что нарушители седьмого дня должны умереть: если кто
осквернит его, тот да умрет смертию. И ты с своей стороны скажи сынам
Израилевым, чтобы они соблюдали этот день, святили его, и никакого дела не
делали в него, и не оскверняли его; ибо он святее, нежели все другие
дни, и всякий, кто оскверняет его, должен умереть смертию; и всякий, кто делает
в него какое-либо дело, должен умереть смертию, навсегда и навечно чтобы сыны
Израиля соблюдали этот день в своих родах и не были истреблены на земле. Ибо
это святой день и благословенный день, и всякий человек, соблюдающий его и
празднующий в него субботу от всякого своего дела, будет свят и благословен
всегда, как мы (Ангелы). И ты возвести и изъясни сынам Израилевым закон этого
дня, чтобы они праздновали в него субботу и не забывали бы его в заблуждении
своего сердца, чтобы они не делали в него ничего из своих нужд, не приготовляли
в него чего-либо из пищи и питья, ни воды не черпали, ни какой-либо ноши не
вносили и не выносили в него чрез свои врата, если бы им не пришлось
приготовить себе чего-нибудь в течение шести дней в своих домах. И они не
должны ничего выносить и вносить в этот день из одного дома в другой, ибо он
святее и благословеннее всех юбилейных дней юбилейного года. В него мы
праздновали субботу, прежде чем кому-либо из смертных сделалось известным
празднование в него на земле субботы. И Творец всех вещей благословил его; но
Он освятил не всех людей и не все народы праздновать в него субботу, а
только Израиля; ему только предназначил Он есть и пить, и праздновать в него
субботу на земле. И благословил его Творец всех вещей, создавший этот день к
благословению, и к освящению, и к прославлению пред всеми другими днями.
Этот закон и свидетельство даны сынам Израилевым как вечный закон для
всех родов.

\vs Jub 3:1
И в шесть дней второй субботы (седмицы) мы по повелению Господа привели к
Адаму всех зверей, и всякий скот, и всех птиц, и все, что движется на земле, и
все, что движется в воде, по их родам и видам, именно~--- зверей в первый
день, скот во второй, птиц в третий, все, что движется по земле, в четвертый,
все, что движется в воде, в пятый день; и Адам дал им всем имена, и как он
назвал их, так и было им имя. И в продолжение этих пяти дней Адам видел все
это, самца и самку в каждом роде, что есть на земле, между тем как только он
был одинок, и он не мог найти себе никого подобного, кто был бы ему
помощником.

И Господь сказал мне: <<Нехорошо быть человеку одному: создадим ему
помощника, подобного ему>>. И Господь Бог наш навел на него усыпление,
чтобы он заснул. И Он взял для жены одно из ребер его, как вещество для жены, и
создал плоть вместо него; и Он создал жену и пробудил Адама от сна. И когда
Адам пробудился, поднялся в шестой день, и взял ее к себе, и узнал ее, и сказал
ей: <<Это кость от моей кости, и плоть от моей плоти; она назовется моею
женою; ибо от своего мужа она взята. Посему муж и жена да будут одно, и посему
он оставит отца своего и матерь свою и прилепится к жене своей, и будут они
одною плотью>>. И в первую седмицу был создан Адам и его жена, и во
вторую седмицу Он (Бог) поставил ее пред ним. И ради сего дана
заповедь~--- семь дней для мальчика, а для девочки дважды семь дней пребывать
женщине в ее нечистоте.

И после того как Адам пробыл сорок дней в стране, где он был сотворен, мы
привели его в сад Едем. Ради сего на небесных скрижалях рожденных предписано:
<<Если она родила дитя мужеского пола, то должна оставаться в своей
нечистоте семь дней, соответственно первой неделе, и тридцать три дня должна
оставаться в крови своего очищения, и не должна прикасаться ни к чему святому,
ни вступать в святилище, пока она не окончит этих дней, та, которая родила
мужеского пола. А родившая младенца женского пола должна две недели,
соответственно двум первым неделям, пребывать в своей нечистоте и шестьдесят
шесть дней в крови очищения; и будет для нее всего восемьдесят дней. И когда
жена (Ева) окончила восемьдесят дней, мы привели ее в сад Едем, ибо он свят во
всей земле, и каждое дерево, которое насаждено в нем, свято. Ради сего для
рождающей мальчика или девочку установлен закон этих дней, чтобы она не
прикасалась ни к чему святому, ни в святилище не входила, пока не окончатся эти
дни для мальчика или девочки. Это закон и свидетельство, написанное для
израильтян, чтобы они соблюдали это всегда.

И в начале первого юбилея Адам и жена его были в саду Едем семь лет,
возделывая и храня его. И мы дали ему занятие и научили его все видимое
употреблять в дело, и он трудился. Он же был наг, не зная сего и не стыдясь. И
он охранял сад от птиц, и зверей, и скота, и собирал плоды сада и ел, и
сберегал остаток для себя и своей жены, и делал запас.

И по истечении семи лет, которые он там провел, ровно семи лет, во второй
месяц в семнадцатый день его пришел змий и приблизился к жене. И сказал змий
жене: <<Разве Бог запретил вам все плоды деревьев, которые в раю, чтобы вы
не ели от них?>> И она сказала ему: <<От всех плодов деревьев, которые
в раю, сказал нам Бог, можно нам есть, но от плода дерева, которое в средине
рая, сказал нам Бог, мы не должны есть и прикасаться к нему, дабы нам не
умереть>>. И змий сказал жене: <<Вы не умрете смертию; напротив, Бог
знает, что в день, в который вы вкусите от него, откроются очи ваши, и
вы будете как боги и будете знать доброе и злое>>. И вот жена увидела
дерево, что оно было хорошо и приятно для глаза, и его плод хорош для пищи,
тотчас взяла его и ела. И она первая покрыла свою срамоту смоковничным листом;
и она дала его (плод) Адаму, и он ел, и его глаза открылись, и он увидел, что
был наг, и взял смоковничных листьев, и сшил их, и сделал себе препоясание, и
покрыл свою срамоту. И Господь проклял змия и разгневался на него навсегда. И
на жену также разгневался Он, так как она послушалась голоса змия и ела. И Он
сказал ей: <<Я умножу твои болезни и твое страдание; с болезнями ты будешь
рождать детей, и у своего мужа будешь находить свою защиту, и он будет
твоим господином>>. И Адаму Он сказал: <<Так как ты послушался гласа
жены своей и ел от того дерева, от которого Я запретил тебе есть, то земля
будет проклята из-за тебя; тернии и волчцы будут произрастать тебе, и свой хлеб
ты будешь есть в поте лица твоего, пока не возвратишься в землю, из которой ты
взят. Ибо ты на земле и в землю возвратишься>>. И Он сделал им кожаные
одежды и одел их ими, и изгнал их из рая Едем. И в тот день, когда Адам вышел
из рая Едем, он принес в приятное благоухание жертву благовонную: ладан, и
халван, и стакти, и Сенегал, утром с восходом солнца, в день, когда он покрыл
свою срамоту. И в тот день заключились уста всех зверей, и скота, и птиц, и
того, что ходит (ногами), и того, что движется, так что они не могли более
говорить, ибо до сего все они говорили между собою одними устами и одним
языком. И Он изгнал из сада Едем всякую плоть, которая была в саду Едем; и
рассеялась всякая плоть по своим породам и видам в места, которые для них были
созданы (удобны). Только Адаму Он повелел покрывать свою срамоту~--- ему
одному между всеми зверями и скотом. Ради сего Он на скрижалях повелел всем,
знающим правду закона, покрывать свою срамоту и не обнажаться, как обнажаются
язычники.

И в новолуние четвертого месяца Адам и его жена вышли из рая Едем, и жили в
земле Елдад, в той земле, где они были созданы. И Адам дал имя жене своей Ева.
И у них не было ни одного сына до первого юбилейного года. И после сего он
познал ее. Он же обрабатывал свою землю, как был научен в саду Едем.

\vs Jub 4:1
И в третью седмину во второй юбилей родила она Каина, и в четвертую родила
Авеля, и в пятую родила дочь свою Аван. И в первую седмину третьего юбилея Каин
убил Авеля, ибо Он (Бог) принял дар от руки его милостиво, а от руки Каина
жертву плодов не милостиво. И он убил его на поле, и его кровь вопиет от земли
к небу, восклицая, что он убит. И Бог наказал Каина за Авеля, которого он убил,
и сделал его проклятым на земле за кровь его брата, и проклял его на
земле, ради чего на небесных скрижалях написано так: <<Да будет проклят,
кто убивает своего ближнего по злобе, и все видящие это должны говорить: да
будет так! И человек, который видит и не объявит сего, да будет проклят, как
он!>> Ради сего мы являемся к Господу, Богу нашему, возвещать все грехи,
которые совершаются на небе и на земле, во свете и во тьме, и всюду.

И Адам и его жена скорбели об Авеле четыре седмины. И в четвертый год пятой
седмины он утешился, и опять познал жену свою, и она родила ему сына, и он
нарек ему имя~--- Сиф; ибо он сказал: <<Господь восставил нам другое
семя на земле вместо Авеля, ибо Каин убил его>>. В шестую седмину он родил
свою дочь Азуру. И Каин взял себе свою сестру Аван в жены, и она родила ему
Еноха в конце четвертого юбилея. И в первый год первой седмины пятого, юбилея
были построены дома на земле, и Каин построил город и назвал его по имени сына
своего Енох. И Адам познал свою жену Еву, и она родила еще девять сыновей. И в
пятую седмину сего юбилея Сиф взял себе в жены свою сестру Азуру, и она родила
ему в четвертый год Эноса. И он первый начал призывать имя Господне на земле. И
в седьмой юбилей в третью седмину Энос взял свою сестру Ноаму в жены, и она
родила ему сына в третий год пятой седмины, и он нарек ему имя Каинан. И в
восьмой юбилей в конце его Каинан взял свою сестру Муалелиту в жены, и
она родила ему сына в девятый юбилей, в первую седмину, в третий год той
седмины, и он нарек ему имя Малалел. И во вторую седмину десятого юбилея
Малалел взял себе в жены Дину, дочь Боракиэла, дочь сестры его отца,~---
себе в жены,~--- и она родила ему сына в третью седмину в шестой год, и он
нарек ему имя Иаред; ибо в его дни сошли на землю Ангелы Господни, которые
назывались стражами, чтобы научить сынов человеческих совершать на земле правду
и справедливость.

И в одиннадцатый юбилей Иаред взял себе жену, по имени Барака, дочь
Разузаила, дочь сестры его отца, в четвертую седмину сего юбилея. И она родила
ему сына в пятую седмину, в четвертый год юбилея, и он нарек ему имя
Енох. Он был первый из сынов человеческих, рожденных на земле, который научился
письму, и знанию, и мудрости; и он описал знамения неба по порядку их месяцев в
книге, чтобы сыны человеческие могли знать время годов в порядке их отдельных
месяцев. Он прежде всех записал свидетельство, и дал сынам человеческим
свидетельство о родах земли, и изъяснил им седмины юбилеев, и возвестил им дни
годов, и распределил в порядке месяцы, и изъяснил субботние годы, как мы ему
возвестили их. И что было, и что будет, он видел в своем сне, как произойдет
это с сынами детей человеческих в их поколениях до дня суда. Все видел и узнал
он, и записал во свидетельство, и положил сие, как свидетельство, на земле для
всех сынов детей человеческих и для их родов. И в двенадцатый юбилей в седьмую
седмину взял он себе жену именем Адни, дочь Даниала, дочь сестры его отца. И в
шестой год этой седмины она родила ему сына, и он нарек ему имя Мефусалаг. И
вот он был с Ангелами Божиими в продолжение шести лет, и они показали ему все,
что на земле и на небесах, господство солнца; и он записал все. И он дал
свидетельство стражам, которые согрешили с дочерьми человеческими. Ибо они
стали смешиваться, чтобы оскверняться с дочерьми человеческими. И Енох дал
свидетельство против всех них. И он был взят из среды сынов детей человеческих,
и мы привели его в рай Едем к славе и почести. И вот здесь он записывает суд и
вечное наказание, и всякое зло сынов детей человеческих. И ради него Он (Бог)
послал потоп на землю; ибо он был поставлен в знамение, и чтобы дать
свидетельство против всех сынов детей человеческих, чтобы объявлять все деяния
родов до дня суда. И он принес в жертву курение..., которое было приятно Богу,
на горе полудня; ибо четыре места Божий существуют на земле: рай Едем, и гора
востока, и эта гора, на которой ты теперь,~--- гора Синай, и гора Сион,
которая будет освящена в новом творении для освящения земли; чрез нее земля
освятится от всей своей вины и нечистоты навсегда и навечно.

И в четырнадцатый юбилей взял Мефусалаг Адину, дочь Азраела, дочь сестры его
отца, себе в жены, в третью седмину в первый год, и он родил сына и нарек ему
имя Ламех. И в пятнадцатый юбилей в третью седмину взял себе Ламех жену, по
имени Битанос, дочь Баракела, дочь сестры его отца, себе в жены; и в эту
седмину она родила ему сына, и он назвал его Ноем, говоря: <<Он утешит меня
о всех моих трудах и о земле, которую проклял Бог>>.

И в конце девятнадцатого юбилея в седьмую седмину в шестой год ее умер Адам,
и все сыны его погребли его в стране, где он был сотворен. И он был погребен
прежде всех в земле. И он жил на семьдесят лет меньше тысячи лет, ибо тысяча
лет как один день по небесному свидетельству. Ради сего о древе познания
написано: <<В день, когда вы вкусите от него, вы умрете>>. Посему он не
окончил годы этого дня, но умер в этот день.

В конце этого юбилея был убит Каин после него (Адама) в том же году. Его дом
упал на него, и он умер посреди своего дома, и погиб под его камнями. Ибо
камнем он убил Авеля, и камнем был убит по праведному суду. Сего ради на
небесных скрижалях предписано: <<Орудием, которым муж убил своего ближнего,
должен быть и он убит; как ранил он его, так должны они сделать и
ему>>.

И в двадцать пятый юбилей Ной взял себе жену, по имени Емзараг, дочь
Ракиела, дочь его сестры (?), себе в жены, в первый год в пятую седмину. И в
третий год ее она родила ему Сима, и в пятый год родила ему Хама, и в первый
год в шестую седмину родила ему Иафета.

\vs Jub 5:1
И случилось, когда сыны детей человеческих начали умножаться на поверхности
всей земли и у них родились дочери, Ангелы Господни увидели в один год этого
юбилея, что они были прекрасны на вид. И они взяли их себе в жены, выбрав их из
всех; и они родили им сыновей, которые сделались исполинами. И неправда
усилилась на земле, и всякая плоть извратила свой путь, от людей до скота, и до
зверей, и до птиц, и до всего, что ходит по земле. Все извратили свой путь и
свой порядок, и начали пожирать друг друга. И неправда усилилась на земле, и
все помышления разума сынов человеческих сделались столь злыми во всякое время.
И Господь воззрел на землю, и вот она извратилась, и всякая плоть извратила
свой порядок, и они совершали всякое зло пред Его очами~--- все, что было на
земле. И Он сказал, что Он уничтожит людей и всякую плоть, которую Он сотворил
на земле.

И только Ной обрел милость пред Его очами. И на Ангелов Своих, которых Он
посылал на землю, Он весьма разгневался, так что решил истребить их. И Он
сказал нам, чтобы мы связали их в пропастях земли. И вот они были связаны в них
и разобщены. И относительно детей их вышло повеление от Его лица, чтобы
поразить их мечом и умертвить их под небом. И Он сказал: <<Моему духу не
вечно пребывать на людях, ибо они плоть, и дней их пусть будет сто двадцать
лет!>> И Он послал Свой меч в среду их, чтобы они умертвили друг друга. И
они начали убивать друг друга, пока не пали все от меча и не были уничтожены с
земли на глазах своих отцов. После сего они (их отцы) были связаны в пропастях
земных до дня великого суда, когда придет наказание на всех, извративших свои
пути и свои дела пред Господом. И Он уничтожил все их пристанища, и ни один из
них не остался, которого Он не уничтожил бы за все их зло. И Он соделал для
всех Своих творений новое и праведное естество, чтобы они не согрешали вовек по
всему своему естеству, и каждый был бы праведен чрез свою отрасль. И наказание
всех их определено и записано на небесных скрижалях без неправды. И все,
преступившие путь, который им определен, чтобы ходить по нему, если не ходят по
нему, то наказание написано для каждого естества и для каждого рода. И ничто
не избежит его, что на небе и на земле, во свете и во тьме, в царстве
мертвых, и в пропасти, и в мрачном месте. Все наказания их определены, и
записаны, и начертаны для всех. Великого Он будет судить по его величию, и
малого~--- по его малости, и каждого отдельно~--- по его пути. И Он не
примет никаких даров, ибо говорит, что будет совершать суд над каждым отдельно.
И если бы кто-нибудь дал Ему все, что есть на земле, то Он не обратит лица
Своего и не примет этого от него: ибо Он Судия. И о сынах Израиля написано и
определено: если они обратятся к Нему в справедливости, то Он отпустит им
всякую вину и все грехи их простит. Написано и определено, что милосердие будет
оказано всем, которые обратятся от всякого своего злодеяния, однажды в год. Но
всем тем, которые свои пути и свое стремление извратили пред потопом, не дано
снисхождения, кроме только Ноя, ибо Господь призрел на лице его ради
сыновей, которых Он спас из-за него от потопа. Ибо сердце его было праведно во
всех путях его, как было повелено ему. И он ничего не преступил из того, что
было ему предписано.

И Господь сказал: <<Да будет истреблено все, что на суше, от всякого
скота до диких зверей и птиц и до всего, что движется на земле!>> И Он
повелел Ною сделать себе ковчег, чтобы спастись в нем от потопа. И Ной сделал
ковчег для всех тварей, как Он повелел ему, в (двадцать седьмой)
юбилей, в пятую седмину, в пятый год. И он вошел в него в шестой год ее, в
другой месяц, в новолуние другого месяца. До шестнадцатого дня его вошел в
ковчег он и все, что мы привели к нему. И Господь затворил его снаружи в
семнадцатый день вечером. И Бог открыл семь окон небесных, чтобы они
изливали воду с неба на землю в продолжение сорока дней и сорока ночей. И
источники бездны также изливали воду, так что весь мир наполнился водою. И
поднялись воды на земле: на пятнадцать локтей поднялась вода над всеми высокими
горами. И ковчег носился над землею и плавал на поверхности воды. И вода стояла
на поверхности земли пять месяцев, сто пятьдесят дней. И он (ковчег) пришел и
остановился на вершине Любара, одной из гор Арарата. И в четвертый месяц
замкнулись источники великой бездны и хляби небесные затворились. И в новолуние
седьмого месяца все отверстия пропастей земли открылись, и вода стала стекать в
преисподнюю бездну. И в новолуние десятого месяца показались вершины гор. И в
новолуние первого месяца обнаружилась земля, и вода стекла с земли в пятую
седмину в седьмой год ее. И в семнадцатый день второго месяца просохла земля. И
в двадцать седьмой день его он отворил ковчег и выпустил из него зверей,
птиц и что двигалось.

\vs Jub 6:1
И в новолуние третьего месяца вышел он из ковчега, и устроил жертвенник на
этой горе, и показался на земле. И он взял молодого козла и пролил кровь его в
искупление за всю вину земли, ибо все, что существовало на ней, было
истреблено, кроме тех, которые были в ковчеге с Ноем. И он положил тук его на
жертвенник, и взял тельца, и овна, и овцу, и козлов, и соли, и горлицу, и
молодого голубя, и принес всесожжение на жертвеннике, и примешал к сему
испеченные в масле жертвенные плоды, и возлил кровь и вино, и положил на все
фимиам, и вознес приятное благоухание, которое было приятно Господу. И Господь
обонял приятное благоухание и заключил с ним завет, что не придет более потоп,
который погубил бы землю, что во все дни земли сеяние и жатва не прекратятся,
мороз и жар, лето и зима, день и ночь не изменят своего порядка и не
прекратятся. <<И вы растите и плодитесь на земле, и размножайтесь на ней, и
будьте во благословение на ней! Ваш страх и трепет Я положу на все, что на
земле и в море. И вот Я всех диких зверей, и всякий скот, и все, что летает, и
все, что движется на земле, и рыб в водах, и все~--- дал вам в пищу, как
зелень травную дал Я вам все, чтобы вы ели. Только плоть, в которой живая душа,
вы не должны вкушать с кровию, ибо душа всякой плоти есть кровь, да не взыщется
кровь вашей души. От каждого человека, от каждого Я взыщу кровь человека; кто
проливает человеческую кровь, того кровь пусть прольется от руки человеческой,
ибо по образу Божию Он сотворил Адама. А вы раститесь и умножайтесь на
земле!>> И дети его поклялись, что они не будут есть крови, которая в
какой-либо плоти. И он заключил завет пред Господом Богом навечно, на все роды
земли, в этом месяце.

Ради сего Он говорил с тобою, чтобы и ты с сынами Израиля в этом месяце на
горе заключил завет с клятвою и окропил их кровию ради всех слов завета,
который Господь заключил с ними на все время. И это свидетельство предписано
им, дабы и вы соблюдали это во все дни, чтобы вам никогда не есть крови
зверей (...). И человек, который ест кровь дикого зверя, и скота, и птиц, пока
стоит земля, будет истреблен на земле~--- он и его семя. И Он повелел сынам
Израиля не есть крови, дабы они и их семя существовали пред Господом, Богом
нашим, всегда. И для сего закона нет конца времени; вечно они должны соблюдать
его вместе с потомками, чтобы непрерывно кровию за вас испрашивалось прощение
пред жертвенником; ежедневно, утром и вечером, должно испрашивать у Господа
прощение за них, чтобы они соблюдали это и не были истреблены.

И Он дал Ною и его сыновьям знамение, что не придет опять потоп на землю. Он
поставил Свою радугу в облаках в знамение вечного завета, что потоп более не
придет на землю для истребления ее, пока стоит земля. Посему определено и
написано на небесных скрижалях, чтобы они соблюдали праздник седмиц в этом
месяце однажды в год, чтобы возобновлять завет каждый год. И всего времени, в
течение которого праздновался этот праздник на небе, от дней творения до дней
Ноя было двадцать семо юбилеев и пять седмин. И Ной праздновал его в
продолжение семи юбилеев и одной седмины до дня своей смерти; а сыны Ноя
оскверняли его до дней Авраама и ели кровь. Только Авраам соблюдал его, и
сыновья его Исаак и Иаков соблюдали его до твоих дней. И в твои дни сыны
Израиля забыли его, пока я не обновил их при этой горе. И ты сделай также
повеление сынам Израиля, чтобы они соблюдали этот праздник во всех своих родах,
как закон для себя. Один день в году в этом месяце пусть празднуют они
праздник. Ибо это праздник седмиц, и это праздник первого творения; праздник
этот имеет двоякого рода значение и установлен для двух родов
сообразно тому, что об этом написано и начертано. Ибо я записал это в книге
первого закона, в той, которую я написал тебе, да празднуешь ты всякий раз по
одному дню в году. Я изъяснил тебе и жертвенные дары в него, дабы они хранились
в памяти, и сыны Израиля праздновали бы его в своих родах в этом месяце по
одному дню в год.

И новолуния первого, четвертого, седьмого и десятого месяцев суть дни
воспоминания и праздничные дни в четыре времени года. Они записаны и
установлены к ежегодному свидетельству. И Ной назначил их себе в праздники для
будущих родов, чтобы иметь в них праздник воспоминания. В новолуние первого
месяца было сказано ему, чтобы он сделал ковчег; и в этот день земля стала
сухою, и он отворил ковчег и увидел землю. В новолуние четвертого месяца
заключилось отверстие преисподней глубины земли. И в новолуние седьмого месяца
все отверстия и глубины бездны открылись и воды стали стекать в них. И в
новолуние десятого месяца показались вершины гор, и Ной возрадовался. Посему он
определил их себе в праздники воспоминания навек, и так они утверждены. И они
внесены на небесные скрижали: двенадцать (?) суббот имеет каждое из них, от
одного новолуния до другого (т.е. от первого до четвертого) идет
их воспоминание, от первого до второго, от второго до третьего, от третьего до
четвертого. И всех дней, которые предписаны, пятьдесят две субботы дней; этим
весь год исполняется. Так начертано и установлено на небесных скрижалях, и не
бывает пропуска, ежегодно, из года в год.

И ты скажи сынам Израилевым, чтобы они содержали годы по сему числу, триста
шестьдесят четыре дня: и это будет полный год, и определенное время дней и
праздники года не будут извращены; ибо все совершается в нем (в году) согласно
тому, что утверждено относительно сего, и они не должны опускать ни одного дня
и не должны нарушать ни одного праздника. Если же они преступят и не будут
поступать по его повелениям, то они враз все определенные времена извратят и
годы будут подвинуты с мест. И они будут преступать свой порядок; и все сыны
Израиля забудут путь годов, и не обретут более, и забудут новолуние и его время
и субботы, и заблудятся относительно всего порядка годов. Ибо я знаю это и
отныне возвещаю тебе сие, и это не по моему разумению, но так, как написано в
книге у меня, и на небесных скрижалях определено деление дней, ибо они не
должны забывать праздников завета, и не должны соблюдать праздников язычников,
и ходить по их заблуждениям и по их мыслям. И это будет с людьми, которые будут
наблюдать над луною, они именно извратят времена, и каждый год уйдет вперед на
десять дней. И из-за этого они извратят будущий год, и сделают мнимый день за
день свидетельства и нечистый день за день праздничный. И каждый будет
смешивать святой день с нечистым и нечистый со святым; ибо они будут
заблуждаться в месяцах, и субботах, и праздниках, и юбилейных годах. Посему я
повелеваю и подтверждаю тебе, чтобы ты засвидетельствовал им,~--- так как
после твоей смерти твои дети (?) извратят это,~--- что они должны считать
год только в триста шестьдесят четыре дня. Из-за сего они будут заблуждаться в
новолунии, и субботе, и в дне торжества и праздника и будут всегда есть плоть в
крови.

\vs Jub 7:1
И в седьмую седмину в первый год ее в этом
юбилее Ной насадил виноградные деревья на горе,
на которой остановился ковчег, называемой Лубар,
на одной из гор Арарата. И они принесли плод на
четвертом году. И он берег свои плоды, и снял их в
том году в седьмом месяце, и сделал из них вино, и
влил его в сосуд, и держал его даже до пятого года,
до первого дня, т.е. до новолуния первого месяца.
И он принес всесожжение для Господа, молодого
тельца, и овна, и семь однолетних агнцев, и
молодого козла, чтобы испросить прощение себе и
своим сыновьям. И он приготовил прежде всего
козла, и принес его кровь к (...) алтаря, который он
сделал, и весь тук его положил на алтарь, где он
приготовил всесожжение, и от тельца, и от овна, и
от агнцев он взял все мясо на жертвенник и
возложил на него все плодовые жертвы, какие
принадлежали к ней, смешанные с елеем. Тогда он
возлил прежде всего вино в огонь на жертвеннике,
и положил фимиам на жертвенник, и вознес доброе,
приятное благоухание, чтобы оно вознеслось пред
Господа, Бога его. И он возрадовался, и испил от
этого вина~--- он и его дети, исполненные радости. И
настал вечер; тогда он вошел в свой шатер, и лег
опьяненный и заснул, обнажился во время сна в
своем шатре. И Хам увидел своего отца Ноя нагого,
и вышел, и рассказал своим двум братьям. И Сим
взял свою одежду и поднялся вместе с Иафетом, и
они сняли свою одежду с своих плеч, обратив лицо
назад, и покрыли срамоту своего отца, обративши лицо
назад. И когда Ной пробудился от своего сна, то
узнал все, что сделал с ним его младший сын. И он
проклял его сына и сказал: <<Проклят Ханаан,
послушнейшим рабом да будет он своим братьям!>>
И он благословил Сима: <<Да будет прославлен
Господь Бог Сима, и Ханаан да будет его рабом! Да
распространит Господь Иафета, и да живет Господь
в жилище Сима, и Ханаан будет его рабом!>>

И Хам узнал, что его отец проклял его сына, и
отделился от своего отца, он и его сыновья с ним, в
Хуш, и Мистрем, и Фуд, и Ханаан. И он выстроил себе
город и назвал его по имени своей жены
Неелатамек. И Иафет увидел это, и позавидовал
своему брату, и также выстроил город, и назвал его
по имени своей жены Адотанелек. Но Сим жил со
своим отцом Ноем, и выстроил город близ города
своего отца при горе, и он также назвал его по
имени своей жены Седукательбаб. Вот три города
близ горы Лубар: Седукательбаб пред горою на ее
восточной стороне, и Неелтамаук на южной стороне,
Адатанезес (?) к западу. И вот сыновья Сима: Елам,
Асур, Арфаскад [...].

В двадцать восьмой юбилей Ной начал учить своих
внуков всем постановлениям и заповедям, которые
он знал, и закону; и дал свидетельство своим
сыновьям, чтобы они делали справедливость, и
покрывали срамоту своего тела, и прославляли
Того, Кто сотворил их, и почитали отца и матерь,
чтобы любили друг друга и ограждали свои души от
всякого любодеяния и нечистоты и от всякой
несправедливости. Ибо за эти три вины пришел на
землю потоп, именно~--- за любодеяние, которым
стражи вопреки предписаниям их закона блудили с
дочерьми человеческими и взяли себе жен из всех,
которые им понравились: они положили начало
нечистоте. И их сыны, Нефилимы и все другие стали
разногласить друг с другом, и один пожирал
другого: исполин убивал Нефила, и Нефил убивал
Елъйо, и Елъйо сынов человеческих, и один человек
другого. И каждый был [...], чтобы делать неправду и
проливать неповинную кровь; и земля наполнилась
нечестием. И за ними последовали все дикие звери,
и птицы, и что движется, и что ходит по земле; и
пролилось много крови на земле. И все помышление
и стремление людей было пустое и злое. И Господь
истребил все с поверхности земли; за лукавство их
дел и за кровь, которую они пролили на земле, Он
истребил все. И я, и вы, мои сыны, и все, что с нами
вошло в кочег, сохранилось целым. И вот я вижу
прежде всего ваши дела, как вы ходите не в
справедливости, но начали ходить по пути
развращения, и отделяться друг от друга, и быть
завистливыми Друг к другу, один к другому, и как
вы не единодушны, мои сыны, брат с своим братом.
Ибо я вижу, что демоны начали обольщать вас и
ваших сыновей. И теперь я страшусь за вас, чтобы
вы, когда я умру, не стали проливать на земле
кровь человеческую, а чтобы и вы не были
истреблены с поверхности земли. Ибо каждый, кто
проливает человеческую кровь, и каждый, кто ест
кровь какой-либо плоти, будет истреблен из среды
всех с лица земли, и ни одного человека не
останется на земле, который ест кровь и проливает
кровь на земле; и не останется у него семени и
потомства под небом; но они пойдут в царство
мертвых и сойдут в место осуждения; все они
погрузятся в мрак бездны через мучительную
смерть~--- каждый из вас, кто от всякой крови не
принесет за себя для очищения; т.е. как только вы
заколете зверя, или скот, или что летает на земле,
то делайте доброе дело за себя кровию, где только
она проливается на земле. И никто из вас не должен
есть плоть с кровию; удерживайте, чтобы не ели
кровь пред вами. Закапывайте кровь, ибо так было
заповедано мне; я свидетельствую о сем как вам,
так и вашим сыновьям, вместе со всякою плотию. И
не ешьте душу с плотию, да не взыщется ваша кровь,
которая есть ваша душа, от всякой плоти, которая
проливает ее на земле. Ибо земля будет нечиста от
крови со времени ее пролития на ней, но только через
кровь того, кто пролил ее, земля будет чистою в
продолжение всех своих родов. И теперь, мои
сыновья, послушайте меня, творите правду и
справедливость, чтобы вы были насаждены в
справедливости на всем лице земли, и да
вознесется ваша слава к Богу моему, Который спас
меня от потопа. И вот вы пойдете и выстроите себе
города, и разведете в них всякие растения,
которые на земле. И теперь от всех плодовых
деревьев в продолжение трех лет не должен
собираться плод ни от какого дерева, чтобы
есть его, и в четвертый год их плод должен быть
освящен, и начаток плодов [...] должен быть
принесен пред Господа, Всевышнего, Который
создал небо и землю и все, чтобы с лучшим начатком
плодов принести вино и елей на жертвенник
Господа, который Он изберет; и что останется,
слуги дома Божия должны съесть пред
жертвенником, который Он изберет. И в пятый год
сделайте обнародование, чтобы вы обнародовали
это в справедливости и праведности, и вы будете
праведными, и все ваши растения умножатся. Ибо
так заповедал Енох, отец вашего отца Мефусалага,
своему сыну, и Мефусалаг своему сыну Ламеху, и
Ламех заповедал мне все, что заповедали ему отцы
его. И я также заповедую вам это, мои сыны, как
Енох заповедал своему сыну в его первый юбилей,
когда он был еще жив, седьмой в своем роде, он
заповедовал и свидетельствовал это своему сыну и
сыновьям его сыновей до дня своей смерти.

\vs Jub 8:1
И в двадцать девятый юбилей в первую седмину в
первый год Арфаскад взял себе жену, по имени
Разуйю, дочь Сусаны, дочери Елама, себе в жены, и
она родила ему сына в третий год этой седмины, и
он нарек ему имя Каинам. И его сын возрос, и его
отец научил его писанию, и он пошел искать себе
место, где бы основать себе город. И он нашел
надписание, которое праотцы начертали на скале; и
он прочитал, что было на ней, и перевел это, и
нашел, что на ней было знание, которому научили
стражи, о колесницах солнца, и луны, и звезд, и обо
всех замениях неба. И он записал это, но ничего о
сем не рассказал, ибо он боялся рассказать о сем
Ною, чтобы он не разгневался на него за это.

И в тридцатый юбилей во вторую седмину в первый
год ее взял он себе жену, по имени Мелку, дочь
Абадая, сына Иафета. И в четвертый год она родила
ему сына, и он нарек ему имя Сала, ибо сказал: <<Я
отпущен>>. В четвертый год родился Сала, и он
возрос и взял себе жену по имени Муак, дочь
Кеседа, брата его отца, себе в жены. И в тридцать
первый юбилей в пятую седмину в первый год она
родила ему сына [...], и он нарек ему имя Ебор. И он
взял ему жену по имени Ацурад, дочь Неброда, и
именно в тридцать второй юбилей в седьмую
седмину в третий год. И в шестой год она родила
ему сына, и он нарек ему имя Фалек. Ибо во Дни,
когда он родился, дети Ноя начали делить землю
между собою; и ради этого он нарек ему имя Фалек. А
они делили между собою лукаво, и об этом было
сказано Ною.

И в начале тридцать третьего юбилея они
разделили землю на три части~--- Симу, Хаму и Иафету,
по их наследственным частям в первый год первой
седмины; в то время Ангел, один из нас, посланных к
ним, был при этом. И он (Ной) призвал своих сыновей,
и они приблизились к нему~--- они со своими
сыновьями~--- и он разделил землю по жребию, что
должны были получить три его сына, и они
распростерли руки, и взяли жребий из пазухи
своего отца Ноя.

И на жребий Сима вышла средина земли, которую он
должен был получить как наследие для своих
сыновей и потомков вовек, от средины горы Рафу,
где изливается вода из реки Тоны; и идет его
наследие к западу чрез средину той реки, и идет,
пока не подойдешь к водному бассейну, из которого
выходит эта река, и река эта вытекает и изливает
свою воду в море Миот, и идет эта река до великого
моря. И все, что к югу от него, принадлежит Симу; и
идет его наследие, пока не подойдешь к Карасо,
т.е. до залива перешейка, который смотрит к югу. И
идет его наследие к великому морю и выходит
прямо, пока не подойдешь к западу перешейка,
который смотрит к югу. Ибо это море называется
египетским морским заливом. И оттуда
направляется на юг к устью великого моря до
берегов воды, и идет к Аравии в Офру, и идет, пока
не достигнет воды потока Гигон, и на юг от воды
Гигон, вдоль берега этой реки, и идет на юг, пока
не подойдет к раю Едем на юг от него и на восток от
всей страны Едем [...]; и обращается на восток от
него, и идет, так что подходит к востоку горы,
которая называется Рафа, и спускается к берегу
устья реки Тины. Это наследие досталось по жребию
Симу и его детям, чтобы владеть им (наследием), и
его потомкам до века. И Ной возрадовался, что это
наследие досталось Симу и его детям, и он
размышлял обо всем, что он сказал своими устами в
своем пророчестве, когда говорил: <<Да будет
прославлен Господь, Бог Сима, и да вселится
Господь в жилищах Сима!>> И он знал, что рай Едем
есть святейшая из святынь и жилище Господа и что
гора Сион, центр пустыни, и гора Синай, центр пупа
земли, эти три, одна против другой, созданы были
святынями земли. И он прославил Бога богов,
который вложил речь Господа в уста его [...]. И он
познал, что блаженное и благословенное наследие
Симу и его детям будет уделом для вечных родов;
именно~--- вся страна Эритрейского моря, и вся
страна востока и Индия (и при Эритрейском море) и
горы ее, и вся страна Бала, и вся страна Либанос, и
острова Кафтор, и весь горный хребет Санер и Амар,
и горный хребет Ассур на севере, и вся страна
Елам, Ассур, и Бабель, и Сузан, и Мадай, и вся
страна Арарат, и вся страна по ту сторону горного
хребта Ассур к северу~--- благословенная и обширная
страна, и все, что в ней, очень хорошо.

И Хаму досталась вторая наследственная часть,
по ту сторону Гигона, к югу, направо от рая, и она
идет к югу. И направляется она к огненным горам
и к западу к морю Атил, и направляется на запад,
пока не подойдет к морю бассейна, того, в котором
погибает все, что бы ни стекало, и идет к северу к
пределу Гадит, и идет до берегов моря по ту
сторону великого моря, пока не подойдет к потоку
Гигон, [...], направо от рая Едем. И эта страна,
которая досталась Хаму как наследственная часть,
которой он должен владеть вовек,~--- ему и его
сыновьям в их родах вовек.

И Иафету вышла третья наследственная часть, по
ту сторону реки Тины, к северным странам истока
ее воды, и идет к северо-востоку вся область Лага
и все восточные страны ее; и идет на крайний
север, и простирается до гор Кильта к северу, и к
морю Маук, и идет на восток Гадира, до берегов
моря; и направляется, пока не подойдет к западу
Пары, и обращается назад к Аферагу, и
направляется к востоку, к воде моря Миот, и
направляется вдоль реки Тины, к востоку севера,
пока не подойдет к границе ее воды, к горе Рафы, и
обходит кругом к северу. Это страна, доставшаяся
Иафету и его сыновьям как его наследие, которым
он должен владеть вовек,~--- ему и его сыновьям в их
родах до века: пять великих островов и великая
страна на севере, только она холодная, а страна
Хама жаркая. Но земля Сима не имеет ни жары, ни
мороза, а в ней холод и тепло смешаны.

\vs Jub 9:1
И Хам разделил свою часть между своими
сыновьями. И вышла первая наследственная часть
для всех к востоку и западу Фуду, и запад ее
Ханаану, и к западу моря. И Сим также разделил
между сыновьями. И вышла первая наследственная
часть Еламу и его сыновьям, к востоку от реки
Тигр, пока не подойдешь к стране востока, вся
страна Индия и страна при Эритрейском море, и
воды Дудина, и все горы и Ила (Ела), и вся страна
Сузан, и все, что находится к стороне Фарнака, до
Эритрейского моря и до реки Тины. И Ассуру вышла
вторая наследственная часть: страна Ассур, и
Ниневе, и Синаар, и до границ Индии, и она идет
вверх к реке. И Арфаскаду вышла третья
наследственная часть: вся страна владения
Халдеев, к востоку от Евфрата, вблизи
Эритрейского моря, и все воды пустыни, пока не
придешь к морскому заливу, который смотрит к
Египту, вся страна Либаноса, и Сапера, и Амано, до
соседства с Евфратом. И Араму вышла четвертая
наследственная часть: вся страна Месопотамия,
между Тигром и Евфратом, на север от Халдеев, пока
не придешь к горному хребту Ассур, и все
отдельные страны, до великого моря, и
приближается к востоку к своему брату Ассуру.

И Иафет также разделил страну наследия между
своими сыновьями. И вышел первый жребий Гомеру к
востоку, от севера до реки Тины. И на севере
Магогу досталась вся внутренность севера, пока
не придешь к морю Миот. И Мадаю вышел удел, чтобы
он владел им, на запад от обоих его братьев, до
островов и до границ островов. И Ийоайону вышел
четвертый удел~--- весь остров и острова к Адлуду. И
Толбелу вышел пятый удел, между перешейком,
который подходит к Уда, уделу Луда, до другого
перешейка, внутрь в третий перешеек. И Месеку
вышел шестой удел, и все по ту сторону третьего
перешейка, пока не придешь к востоку Гадира. И
Терасу вышел седьмой удел: он имел великие
острова в средине моря, которые принадлежали к
наследию Хама, и острова Каматури. И детям
Арфаскада вышло блаженное наследие.

Так разделили дети Ноя уделы своим сыновьям
пред Ноем, своим отцом, и он велел им поклясться,
заклиная клятвою каждого, который пытался бы
получить удел, не доставшийся ему по жребию. И все
сказали: <<Да будет так!>> И да будет это так
для них и их сыновей до века, в их родах, до дня
суда, в который Господь Бог будет судить их мечом
и огнем за все лукавство и нечистоту их деяний,
так как они наполнили землю злодеянием,
нечестием, блудодеянием и грехом.

\vs Jub 10:1
И в третью седмину этого юбилея начали нечистые
демоны обольщать сыновей Ноя, чтобы ослеплять их
и губить. И дети Ноя пришли к своему отцу и
рассказали ему о демонах, которые соблазняют
сыновей их сыновей, ослепляют и умерщвляют их. И
он молился Господу Богу своему и сказал: <<Боже
духов всякой плоти, являющий Свое милосердие, и
спасший меня и моих детей от воды потопа, и не
допустивший меня погибнуть, как поступил Ты с
сынами погибели, ибо велика милость Твоя ко мне и
велико Твое милосердие к моей душе: яви милость
Твою на сынах Твоих, не допусти злых духов
господствовать над ними, дабы они не истребили их
от земли! Вот Ты благословил меня и моих сыновей,
чтобы мы возрастали, и умножались, и наполняли
землю. Ты знаешь, как Твои стражи, отцы этих духов,
поступили в мои дни. И этих духов, которые живы,
также заключи и свяжи в месте осуждения, чтобы
они не производили развращения между сынами
Твоего раба, Боже мой, ибо они злобны и созданы на
погибель! Не допусти их господствовать над
духами живущих, ибо Ты один знаешь суд их; и не
допусти их иметь власть над детьми
справедливости отныне и до века!>>

И Бог наш сказал нам, чтобы мы связали всех.
Тогда пришел высший из духов Мастема и сказал:
<<Господи, нельзя ли некоторым из них остаться у
меня, чтобы они слушались моего голоса и делали
все, что я скажу им? Ибо если ни одного из них не
останется у меня, то я не могу являть могущества
своей воли над сынами человеческими; ибо они
существуют для того, чтоб развращать и обольщать
по моему повелению под моим господством, так как
злоба людей велика>>. И он сказал: <<Десятая
часть их пусть останется у меня, и девять частей
пусть сойдут в место суда!>> И один из нас
сказал: <<Мы научим Ноя всем целебным
средствам>>; ибо он знал, что они ходят не в
справедливости, и будут вести борьбу не в
праведности. И мы сделали по Его повелению: всех
злых, лютых духов мы связали в месте
наказания, и десятую часть из них мы оставили,
чтобы они предстали пред Сатаною на земле. И
целебные средства от всех их (т.е. причиняемых
демонами) болезней вместе с их способами
обольщения мы сказали Ною, как излечивать себя
растениями земли. И Ной записал все, как мы
научили его, в книгу, о каждом роде лекарств. Так
злые духи были отделены в заключение от детей
Ноя.

И он дал все писания, которые написал, своему
старейшему сыну Симу, ибо он любил его больше из
всех своих сыновей. И Ной почил с своими отцами и
был погребен на горе Лубар в земле Арарат.
Девятьсот пятьдесят лет он окончил в своей жизни,
девятнадцать юбилеев, две седмины, пять лет. И его
жизнь на земле была знаменитее, чем жизнь остальных сынов человеческих,
ради его справедливости, в которой он усовершился, кроме только Еноха; ибо
история Еноха была предназначена во свидетельство для родов вечности, чтобы
показать все, что случится с родами родов до дня суда.

И в тридцать четвертый юбилей, в первый год
второй седмины, Фалек взял себе жену по имени
Ломна, дочь Синаара. И она родила ему сына в
четвертый год этой седмины, и он нарек ему имя
Рагев, ибо сказал: <<Вот сыны человеческие
сделались дурными через гнусный замысел, что они
построят себе город и башню в земле Синаар, ибо
они переселились от Арарата к востоку в
Синаар>>. Ибо в его дни они построили город и
башню, говоря: <<Мы поднимемся по ней на небо>>.
И они начали строить в четвертую седмину, и
обжигали огнем (кирпичи), и кирпичи служили им
вместо камня, и цементом, которым они укрепляли
промежутки, был асфальт из моря и из водных
источников в стране Синаар. И они строили это в
продолжение сорока трех лет. И Господь Бог наш
сказал нам: <<Вот, это один народ, и он начал
делать это! И ныне Я не отступлю от них! Вот, мы
сойдем и смешаем языки их, чтобы они не понимали
друг друга и рассеялись в страны и народы, и да не
осуществится никогда их замысел до дня суда!>> И
Господь сошел, и мы сошли с Ним, видеть город и
башню, которую строили сыны человеческие; и Он
расторг каждое слово их языка, и никто уже не
понимал слово другого. И вот они отказались
строить город и башню. Ради сего вся страна
Синаар была названа Бабель (Вавилон). Ибо так
расторг Бог все языки сынов человеческих; и
оттуда они рассеялись в свои города по их языкам
и народам. И Бог послал сильный ветер на их башню
и поверг ее на землю. И вот она стояла между
страной Ассур и Вавилоном в земле Синаар; и
нарекли ей имя развалины.

В первый год четвертой седмины тридцать пятого
юбилея они рассеялись в стране Синаар. И Хам с
своими сыновьями ушел в страну, которая стала его
собственностью и которая досталась ему при
разделе, в страну юга. А Ханаан увидел страну
Либаноса, до ручья Египетского, что она очень
хороша, и пошел не в страну своего наследия, на
запад от моря, но жил в стране Либанос, на востоке
и на западе от сынов народа Либаноса и вдоль моря.
И отец его Хам, и Куш, и Мицраим, его братья,
сказали ему: <<Ты поселился в стране, которая не
принадлежит тебе и которая по жребию не
досталась нам. Ты не должен так делать. Ибо если
ты сделаешь так, то погибнешь, так же как и твои
сыновья, в стране, как подвергшийся проклятию,
силой оружия; ибо вы силой оружия поселились, и
силой оружия падут твои сыновья, и ты будешь
истреблен вовек. Не живи в месте обитания Сима,
ибо оно досталось по жребию Симу и его детям.
Проклят ты и проклят будешь пред всеми сыновьями
Ноя проклятием, которым мы обязались в клятве
пред святым Судиею и пред нашим отцом Ноем>>. Но
он не послушал их и жил в стране Либанос, от
Гамафа до начала Египта, он и его сыновья до
нынешнего дня. И посему та страна была названа
Ханаан. Но Иафет и его сыновья пошли на запад и
жили в стране своего наследия. И Мадай увидел
страну моря, и она понравилась ему, и он выпросил
ее себе у Елама и Ассура и Арфаскада, брата его
жены, и жил в стране Мидакин (Мидийской стране)
вблизи брата своей жены до сего дня; и он назвал
свое место обитания и место обитания своих детей Медекин, по имени их отца
Мадая.

\vs Jub 11:1
И в тридцать пятый юбилей в третью седмину в
первый год ее Рагев взял жену по имени Ара, дочь
сына Кеседа. И она родила ему сына, и он нарек ему
имя Серуг в седьмой год этой седмины и этого
юбилея. И сыны Ноя начали вести борьбу друг с
другом; они стали друг друга брать и убивать,
проливать кровь человеческую на земле, и есть
кровь, и строить укрепленные города, и стены и
башни, и помимо того возноситься над народом, и
повсюду основывать царство, и вести войну~--- один
народ против другого, и народы против народов, и
город против города, и подвергать все порче, и
делать себе оружие, и учить своих детей войне. И
они начали покорять города, и продавать
невольников и невольниц.

И Ур, сын Кеседа, построил город Ару Халдейскую
и назвал его по имени себя и по имени отца своего.
И он делал им звезды и поклонялся каждому идолу,
которого он лил себе. И они начали делать
изваяния, и статуи, и нечистое, и духи нечистые
помогали в этом и обольщали их совершать грех и
нечистоту. И князь Мастема прилагал свою власть,
чтобы делали это, и побуждал чрез духов, которые
были отданы в его руки, совершать различного рода
злодеяния, и грехи и всякое развращение, чтобы
развращать, и губить, и проливать на земле кровь.
Посему ему было наречено имя Серух, ибо он
удалился, чтобы свободнее совершать грех и
злодеяние. И он сделался великим, и жил в Уре
Халдейском, вблизи родителей (своей матери), и
поклонялся идолам. И он взял себе жену в тридцать
шестой юбилей, в пятую седмину, в первый год, по
имени Мелка, дочь Кгебера, дочь (сестры) его отца.
И она родила ему Накгора в первый год этой
седмины; и он возрос и жил в Уре, в Уре Халдейском.
И его отец, мудрец Халдейский, научил его
предсказанию и гаданию по знамениям неба.

И в тридцать седьмой юбилей, в шестую седмину, в
первый год взял он себе жену по имени Ийосака,
дочь Нестега Халдейского, и она родила ему сына
Фарага в седьмой год этой седмины. И князь
Мастема послал воронов и птиц, чтобы они пожирали
семя, посеянное на земле, чтобы произвести порчу
на земле, чтобы они расхищали у сынов
человеческих их произведения. Ибо прежде чем они
запахивали семя, вороны подбирали его с
поверхности земли. Посему он нарек ему имя Фараг,
так как вороны и птицы обкрадывали их и пожирали
у них семя их. И годы стали делаться неурожайными
от птиц; и все древесные плоды они пожирали с
деревьев [...]. Только с великим трудом можно было в
их дни спасти кое-что от всех плодов земли.

И в тридцать девятый юбилей во вторую седмину, в
первый год, взял себе Фараг жену, по имени Една,
дочь Арема, сестрину дочь его отца, себе в жены. И
в седьмой год этой седмины она родила сына, и он
нарек ему имя Аврам, по имени отца его матери; ибо
он умер, прежде чем приобретен был ее и его сын. И
дитя начало замечать греховность земли, как она
была соблазнена к греху чрез изваяния и
нечистоту. И его отец научил его писать. И когда
он был двух седмин, то отдалился от своего отца,
чтобы не поклоняться вместе с ним идолам. И он
начал молиться Творцу всех вещей, чтобы Он спас
его от обольщения сынов человеческих и чтобы его
наследие, после того как он стал праведным, не
впало в греховность и нечестие.

И пришло время посева для тех, кто засевает
землю. И они вышли все вместе, чтобы стеречь свои
семена от воронов. И Аврам вышел с другими, будучи
дитею четырнадцати лет. И налетело облако (стая)
воронов, чтобы пожирать семена. Но Аврам побежал
к ним, прежде чем они сели на землю, и закричал на
них, прежде нежели они сели на землю, чтобы
пожирать семена, и сказал: <<Не смейте
спускаться, воротитесь в то место, откуда
прилетели!>> И они воротились. И они (?) сделали
так в тот день с семью стаями воронов. И из всех
воронов ни один не сел где-либо на пашню, где был
сам Аврам,~--- даже ни один. И все, бывшие около него
на той пашне, видели, как он закричал и сказал:
<<Воротитесь, вороны!>> И его имя сделалось
великим во всей стране Халдейской. К нему
приходили в этот год все, которые сеяли; и он
ходил с ними, пока не прошло время сеяния. И они
засеяли свою землю и собрали в том году хлеб, так
что ели и были сыты.

И в первый год пятой седмины Аврам научил тех,
которые делают воловью упряжь,~--- плотников, и они
сделали прибор над землею против деревянной дуги
плуга, чтобы класть на него семена и выбрасывать
их оттуда в семенную борозду, чтобы они
скрывались в земле. И они не боялись более
воронов и делали так у всех дуг плугов нечто над
землею. И они засеяли и обработали всю страну
вполне так, как им велел Аврам; и они не боялись
более воронов.

\vs Jub 12:1
И случилось в шестую седмину в седьмой год ее,
сказал Аврам отцу своему Фарагу, говоря: <<Отец,
отец мой!>> И он сказал: <<Вот я здесь, мой
сын!>> И он сказал: <<Что нам за помощь и
услаждение от всех идолов [...], что ты
поклоняешься им? Ибо в них совсем нет духа (души);
но они, которых вы почитаете, суть проклятие и
соблазн сердца. Почитайте Бога небесного,
Который низводит на землю дождь и росу, и все
совершает на земле, и все сотворил Своим словом, и
вся жизнь пред Его лицем! Зачем вы почитаете тех,
которые не имеют духа? ибо они нечто сделанное, и
на своих плечах вы носите их, и не имеете от них
никакой помощи, но они служат великим поношением
для тех, которые делают их к соблазну сердца и
почитают их. Не почитайте их!>> И отец его сказал
ему: <<И я знаю это, сын мой. Но что я сделаю с
моим родством, которое заставило меня служить им?
Если я скажу им истину, то они убьют меня, ибо их
душа прилепилась к ним, чтобы почитать и
прославлять их. Молчи, сын мой, чтобы они не убили
тебя!>> И он сказал эту речь своим двум братьям,
и они разгневались на него. Тогда он замолчал.

И в сороковой юбилей во вторую седмину в
седьмой год ее Аврам взял жену по имени Сора, дочь
его отца (?), и она сделалась его женою. Аран, брат
его, взял себе жену в [...] год третьей седмины,
и она родила ему сына в седьмой год этой седмины;
и он нарек ему имя Лот. И его брат Накгор также
взял себе жену.

И (в шестидесятый) год жизни Аврама, т.е. в
четвертый год четвертой седмины, встал Аврам
ночью, и сожег капище идолов и все, что было в нем,
так что люди ничего не знали об этом. И они встали
ночью и хотели спасти своих идолов из огня. И Аран
поспешил сюда, чтобы спасти их; тогда пламя
бросилось на него, и он сгорел в огне и умер в Уре.
Халдейском, прежде своего отца Фарага; и они
погребли его в Уре Халдейском.

И Фараг вышел из Ура Халдейского, он и дети его,
чтобы идти в страну Либаноса и в страну Ханаан; и
он жил в стране Харран . И Аврам жил со своим отцом
Фарагом в Харране две седмины.

И в шестую седмину в пятый год ее встал Аврам и
сидел в течение ночи, в новолуние седьмого
месяца, чтобы наблюдать звезды, от вечера до утра,
чтобы видеть, что будет с погодою в этот год. И он
был один, когда сидел и наблюдал. И пришло на его
мысль слово, и он сказал: <<Все знамения звезд и
знамения солнца и луны в руке Господа. Зачем мне
исследовать их? Когда Он хочет, то посылает дождь
рано и поздно, и когда хочет, то изливает потоки
(дождя), и все в Его руке>>. И он молился в эту
ночь и сказал: <<Боже мой, Боже мой! Ты всевышний
Бог, Ты один только Бог мой, и Ты все сотворил и
все есть дело рук Твоих; и Тебя, Твое божество
избрал я. Спаси меня от руки злых духов, которые
сильны над помышлениями человеческого сердца,
чтобы они не отвратили меня от Тебя, Боже мой! И
соделай, чтобы я и мое семя вовек не отвращались от
Тебя, отныне и до века!>> И он сказал:
<<Возвратиться ли мне в Ур Халдеев, которые
ищут моего лица, чтобы я возвратился к ним, или
оставаться мне здесь в этом месте? Укажи рабу
Твоему правый путь пред Тобою, чтобы исполнять
его, и чтобы я не ходил в обольщении моего сердца,
Боже мой!>> И когда он окончил речь и молитву,
вот тогда было послано чрез меня слово Господа к
нему, говоря: <<Поднимись из земли твоей и из
рода твоего и из дома отца твоего в землю, которую
Я тебе покажу! И Я произведу от тебя великий и
бесчисленный народ, и благословлю тебя, и сделаю
твое имя великим. И ты будешь благословлен на
земле, и в тебе благословятся все народы земли;
благословляющих тебя Я благословлю и
проклинающих тебя прокляну; и Я буду Богом тебе, и
твоим сыновьям, и сыновьям сынов твоих, и всему
твоему семени; и буду за тобою, Я Бог твой. Не
бойся, отныне до всех родов земли я Бог твой>>. И
Господь Бог сказал мне: <<Открой его уста, и его
уши, и его губы!>> И я начал говорить по-еврейски
на его коренном языке. И он взял книги своего
отца, которые были написаны по-еврейски, и списал
их. Тогда он начал изучать их, и я объяснял ему
все, чего он не понимал, и он изучал их в
продолжение шести дождливых месяцев.

И был седьмой год шестой седмины. Тогда говорил
он со своим отцом и возвестил ему, что он выйдет
из Харрана, чтобы идти в землю Ханаан, что он
осмотрит ее и возвратится к нему. И отец Фараг
сказал ему: <<Иди в мире, Бог мира да соделает
путь твой правый, и да будет Господь с тобою, и
хранит тебя от всех зол, и да даст тебе милость, и
благоволение, и милосердие пред теми, которые
увидят тебя, чтобы никакой человек не возымел
силы над тобою, чтобы предпринять что-либо против
тебя! Иди в мире! И если ты найдешь страну угодною
очам твоим, чтобы жить там, то возьми и меня с
собою; и возьми с собою Лота, сына Арана, брата
твоего, как своего сына! И Господь да будет с
тобою!>>

\vs Jub 13:1
И Аврам вышел из Харрана и взял с собою жену
свою Сору и Лота, сына брата своего Харрана, в
страну Ханаан. И он пошел [...] и прошел до Сикимона,
близ высокого дуба. И Господь сказал ему: <<Тебе
и твоему семени Я дам эту страну!>> И он устроил
там жертвенник и принес на нем Господу, Который
явился ему, всесожжение, И оттуда он поднялся к
горному хребту Бетель (Вефиль), который был от
него на запад и Ай (Гай) на восток, и разбил там
шатер свой. И он увидел, что земля была очень
обширна и хороша, и что в ней росло все:
виноградные лозы, смоквы, гранаты, дубы, и твердые
деревья, и теревинфы, и масличные деревья, и
кедры, и кипарисы, и ливанские деревья, и все
деревья полевые, и что она имела воду на горах. И
он благословил Господа, Который привел его из Ура
Халдейского на эту гору.

И случилось, в первый год, в седьмую седмину, в
новолуние первого месяца он устроил на этой горе
жертвенник и призвал имя Господа: <<Ты, Боже мой,
вечный Бог>>. И он принес на жертвеннике Божием
всесожжение, чтобы Он был с ним и не оставлял его
в течение его жизни. И он поднялся оттуда и пошел
(на юг), и достиг Хеврона; и Хеврон был тогда
построен. И он оставался там два года [...]. Тогда
пошел Аврам в Египет в третий год седмины, и жил
в Египте пять лет, прежде чем у него была похищена
жена. Санай же был тогда построен в Египте чрез
семь лет после Хеврона. И случилось, когда Фараон
похитил Сору, жену Аврама, Господь поразил
Фараона и весь дом его тяжкими бедствиями за
Сору, жену Аврама. И Аврам был очень обогащен
овцами, и рогатым скотом, и ослами, и конями, и
верблюдами, и рабами, и служанками, и серебром, и
золотом вполне; и Лот, сын его брата, также был
обогащен. И когда Фараон возвратил Сору, его жену,
он переселился из земли Египетской, и пришел в
одно место, на восток от Вефиля, и прославил
Господа Бога своего, который вывел его обратно с
миром.

И случилось, в (сорок первый) юбилей в
третий год первой седмины он возвратился в это
место, и принес там всесожжение, и призвал имя
Господне и сказал: <<Ты Господь, Бог всевышний,
Бог мой вовек!>> И в четвертый год седмины Лот
отделился от него. И Лот жил в Содоме, но жители
содомские были очень злы. И он (Аврам) опечалился
в сердце своем, что его племянник отделился от
него, ибо он не имел детей. В тот год этой седмины,
когда Лот был пленен, Господь говорил Авраму,
после того как Лот отделился от него, и сказал
ему: <<Возведи очи твои от места, где ты живешь, к
(северу), и к югу, и к утру, ибо всю страну,
которую ты видишь, Я дам тебе и дам семени твоему
вовек. И Я сделаю твое семя, как песок при море, (и
как человек не может сосчитать песок при море),
так нельзя исчислить и твоего семени. Встань и
пройди ее по длине и широте, и посмотри все, ибо Я
дам ее твоему семени>>.

И Аврам пошел в Хеврон и жил там. И в этот год
пришли Колодогомер, царь еламский, и Амалфал,
царь синаарский, и Ариох, царь селасарский, и
Тергал, языческий царь; и они поразили царя
Гоморры, и царь Содома бежал, и многие,
обратившиеся в Сиддин, солончатую страну, в
Содом, Адом и Севоим, пали. И они взяли в плен Лота,
племянника Аврама, со всем его имуществом, и
отвели в Дан. И пришел один, который спасся
бегством, и рассказал Авраму, что племянник его
взят в плен [...]. И его раб принес в умилостивление
за Аврама и его семя десятину начатков Господу. И
Господь сделал отсюда постановление навсегда,
чтобы давать ее (десятину) священникам, которые
служат пред Его лицем, дабы они пользовались ею
вовек. И это установление не на день, но Он
утвердил его на вечные роды, чтобы давать Господу
десятину от семян, и вина, и масла, и рогатого
скота, и овец. И Он дал ее Своим священникам, чтобы
они ели от нее с радостью и пили пред Ним.

И вышел к нему царь содомский, и пал пред ним
ниц, и сказал: <<Господин мой Аврам, отдай мне
людей, которых ты освободил; добыча же пусть
будет твоя!>> И Аврам сказал ему: <<Я воздвигаю
руки мои к всевышнему Богу: ни нитки, ни
башмачного ремня я не возьму из всего, что
принадлежит тебе, дабы ты не сказал: <<Я сделал
Аврама богатым>>, кроме того, что съели отроки. И
мужи, ходившие со мною, Аунан, и Ескол, и Мамре,
должны взять свою долю>>.

\vs Jub 14:1
И после сего события, в четвертый год этой
седмины, в новолуние третьего месяца, было слово
Господне к Авраму в сновидении, говорящее: <<Не
бойся, Аврам, Я твоя защита, и награда твоя будет
чрезмерна>>. И он сказал: <<Господи, Господи,
что Ты дашь мне? Вот я иду туда без детей, и сын
Месек мой раб тот Дамаск Елиезер, будет
наследником мне; а мне Ты не дал семени>>. И Он
сказал ему: <<Он не наследит тебе, но
происшедший от плоти твоей будет тебе
наследником>>. И Он вывел его и сказал ему:
<<Взгляни на небо и сосчитай звезды небесные:
можешь ли ты сосчитать их?>> И он взглянул на
небо и увидел звезды. И Он сказал ему: <<Так
будет твое семя>>. И он поверил Господу, и это
было вменено ему в праведность. И Он сказал: <<Я
Господь Бог твой, выведший тебя из Ура Халдейского,
чтобы дать тебе в вечное владение землю
Ханаанитов, и чтобы Я был твоим Богом и Богом
твоего семени>>. И он сказал: <<Господи,
Господи!>> И он сказал: <<Господи, по чему я
узнаю, что наследую ее?>> И Он сказал ему:
<<Принеси Мне трехлетнюю телицу, и трехлетнюю
козу, и трехлетнюю овцу, и трехлетнюю горлицу, и
голубя>>. И он взял все это в средине месяца. И он
жил при дубе Мамре, который близ Хеврона. Там
устроил он жертвенник, и заколол все, возлил
кровь их на жертвенник, и разделил их пополам, и
положил их друг против друга; но птиц он не
касался. И птицы спустились на куски, но Аврам
отгонял их, и не давал птицам прикасаться к ним. И
было, когда солнце зашло, бессилие напало на
Аврама, и вот сильный страх мрака напал на него. И
было сказано: <<Аврам, знай, что твое семя будет
странником в чужой земле и его будут порабощать и
угнетать в продолжение четырехсот лет. Но Я
произведу суд над народом, которому они будут
служить; после того они выйдут оттуда с большим
имуществом. И ты в мире отойдешь к своим отцам, и
будешь погребен в доброй старости. И в четвертом
роде оно (твое семя) возвратится сюда, ибо грех
Аморреев доселе еще не наполнился>>.

И он пробудился от своего сна и встал, и солнце
было зашедшим. Тогда появилось пламя, и вот~--- печь
дымилась, и огненное пламя прошло между кусками.
И в ту ночь Бог заключил завет с Аврамом, сказав:
<<Твоему семени Я отдам эту землю, от реки
Египетской до великой реки, реки Евфрат,~--- Кенеев,
Кенезеев, Ферезеев, Рафейн, [...], Евеев, Аморреев,
Канаанеев, Гергесеев>>. И Он отошел. И Аврам
принес куски, и птиц, и жертву плодовую, и жертву
возлияния, которые принадлежали к сему, и огонь
пожрал их.

И в эту ночь Он заключил завет с Аврамом,
согласно завету, который мы заключили в этом
месяце с Ноем. И Аврам возобновил его в праздник и
в постановление для себя, до века. И Аврам
возрадовался и рассказал все это происшествие
своей жене Соре. И он поверил, что у него будет
семя, но она не рождала. Тогда Сора посоветовала
своему мужу Авраму и сказала ему: <<Войди к моей
служанке Агари, египтянке; быть может, я
произведу тебе от нее семя>>. И Аврам послушался
голоса жены своей Соры и сказал ей: <<Сделай
это>>. Тогда Сора взяла египетскую служанку
Агарь и дала ее своему мужу Авраму, чтобы она была
его женою. И он вошел к ней, и она сделалась
беременной, и родила сына, и он нарек ему имя
Измаил в пятый год этой седмины. В том году был
восемьдесят шестой год жизни Аврама.

\vs Jub 15:1
И в пятый год четвертой седмины этого юбилея, в
третий месяц, в средине месяца, Аврам праздновал
праздник начатков жатвы хлеба и принес свежую
хлебную жертву; к жертвам начатков хлеба для
Господа (он присоединил) тельца, и овна, и овцу на
жертвенник вместе с благовонным курением. И
Господь явился ему и сказал Авраму: <<Я~--- Бог
Владыка, благоугождай предо Мною и будь
благочестив. И Я заключу завет между Мною и тобою
и сделаю тебя весьма великим>>. И Аврам пал на
свое лице. И Господь говорил с ним и сказал:
<<Вот завет Мой с тобою, и Я сделаю тебя отцом
многих народов, и ты не будешь более называться
Аврам отныне до века; но Авраам будет тебе имя, ибо
Я сделал тебя отцом многих народов, и сделаю тебя
весьма великим, и произведу от тебя народов и
царей. И я поставлю завет Мой между тобою и Мною, и
между твоим семенем после тебя, в их родах, в
вечное установление, чтобы Я был твоим Богом и
Богом твоего семени после тебя во всех родах. И
Я дам тебе и семени твоему после тебя землю~---
ибо ты пришлец в ней~--- землю Ханаанскую, чтобы ты
был господином над нею навсегда. И Я буду им
Богом>>. И Господь сказал Аврааму: <<И храни
Мой завет ты и твое семя после тебя, и обрезывайте
все ваши крайние плоти. И это будет знамением
Моего вечного установления между Мною и тобою и
для родов (потомков). В осьмой день вы должны
обрезывать все мужеское, в ваших родах,
рожденного дома и купленного вами за золото у
всех сыновей чужеземцев, что приобрели вы. Кто от
твоего семени, тот да будет обрезан, рожденный
дома и купленный за золото да будет обрезан. И Мой
завет на теле вашем пусть будет в вечное
установление.

И кто не обрезан, всякий мужеского пола между
вами, крайняя плоть которого не обрезана в
восьмой день, душа та да истребится из рода
вашего, ибо она нарушила завет Мой>>. И Господь
сказал Аврааму: <<Сора, жена твоя, не будет более
называться Сорою, но Сара~--- имя ее; и Я благословлю
ее, и дам тебе от нее сына; и Я благословлю его, и
произведу от него народ, и цари над народами
произойдут от него>>.

И Авраам пал на лице свое, и возрадовался, и
сказал в сердце своем: <<У меня ли, имеющего сто
лет, родится сын, и Сара девяноста лет родит ли
сына?>> И Авраам сказал Господу: <<Хотя бы
Измаил остался жив пред Тобою!>> И Господь
сказал: <<Да! но и Сара родит тебе сына, и ты
наречешь ему имя Исаак. И Я восстановлю завет Мой
с ними, завет Мой вечный, и с его семенем после
него. И о Измаиле Я услышал тебя, и вот я
благословлю и умножу его, и сделаю его весьма
многочисленным. И двенадцать царей произведет
он; и Я произведу от него великий народ; но завет
Мой Я поставлю с Исааком, которого родит тебе
Сара около сего времени на другой год>>. И после
того, как Он кончил говорить с ним, Господь
восшел.

И он (Авраам) взял своего сына Измаила и всех
своих рожденных дома (рабов) и купленных за
золото, весь мужеский пол, который был в его доме,
и обрезал плоть их члена. И в этот день был
обрезан Авраам, и люди его дома были обрезаны, и
также все, которых он купил за золото у сынов
иноплеменников, были обрезаны вместе с ним. И
этот закон~--- для всех родов вовек. И нельзя
изменять дней, ни пропускать одного из восьми
дней, ибо это вечное благословение, утвержденное
и записанное на небесных скрижалях. И каждый
рожденный, крайняя плоть которого не обрезана до
восьмого дня, не принадлежит к сынам завета,
который Господь заключил с Авраамом, но к сынам
погибели, и вот он не имеет знака на себе, что он
Господень; он предназначен к погибели, и
уничтожению, и истреблению от земли, ибо он
нарушил завет Господа нашего Бога. Ибо Он освятил
Израиля, чтобы он был со всеми Его Ангелами лица,
и со всеми Ангелами прославления, и со святыми
Его Ангелами. И ты повели также сынам Израиля,
чтобы они хранили знак сего завета в своих родах,
как вечное установление, чтобы не быть им
истребленными от земли. Ибо постановление cue
утверждено для завета, чтобы оно соблюдалось
навсегда между всеми сынами Израиля. Ибо Измаила,
и сыновей его и братьев, и Исава не приблизил
Господь и не избрал их; но сынов Авраама познал Он
и избрал Израиля, чтобы они были Его народом, и
освятил его, и собрал его из всех сынов
человеческих. Ибо много народов, и бесчисленны
люди, и все принадлежат Ему, и над всеми Он
поставил духов вместо Господа, чтобы они
отвращали их от Него. Над Израилем же Он никого не
поставил господом~--- ни Ангела, ни духа, но Он
единый их Владыка, и Он охраняет их и ведет тяжбы
их против Своих Ангелов, и Своих духов, и против
всего. И если они будут хранить все Его повеления,
то Он благословит их, и они будут Его сынами, и Он
будет их Отцом отныне до века. И теперь я
предсказываю тебе, что сыны Израиля будут
поступать вопреки этому установлению, и их сыны
не будут обрезываться согласно всему этому
закону. Ибо на плоти своего обрезания они не
будут совершать оного обрезания своих сыновей, и
они все, сыны Велиара, будут оставлять своих
сыновей необрезанными, как они родились. И гнев
Господа на детей Израиля будет велик, ибо они
оставили завет Его, и уклонились от Его слова, и
возбудили Его на гнев, и восхулили Его, и не
сделали сего знака по их закону, но оставили свою
плоть необрезанною подобно язычникам, чтобы
быть уничтоженными и истребленными с земли. И они
впредь не обретут прощения и помилования, чтобы
быть прощенными и помилованными во всех своих
грехах за сие отступление вовек.

\vs Jub 16:1
И в новолуние четвертого месяца явились мы
Аврааму при дубе Мамврийском и беседовали с ним.
И мы также возвестили ему, что у него родится сын
от жены его Сары. Тогда Сара рассмеялась, ибо она
слышала, что мы говорили эту речь Аврааму. И мы
заметили ей; но она испугалась и стала отрицать,
что она смеялась над нашими словами. И мы
сказали ей имя его сына, как определено и
написано было на небесных скрижалях, именно
Исаак. И когда мы возвратимся к ней в
определенное время, тогда она будет беременной
сыном.

И в этот месяц Господь совершил суд над Содомом,
и Гоморрою, и Севоимом, и всею страною Иорданскою,
и сожег их огнем и серой, и предал их погибели до
сего дня; согласно тому, как мы рассказывали тебе
о всех их делах, что они были гнусными и весьма
греховными и что они осквернялись, и
блудодействовали, и делали мерзость на земле~---
согласно сему Бог совершил суд; во гневе и ярости
за нечистоту Содома совершил Он суд над Содомом.
И мы спасли Лота, ибо Господь вспомнил об Аврааме
и вывел его (Лота) из разрушения. Но и он, и дочери
его совершили на земле грех, какого не было на
земле от Адама до того времени; ибо муж переспал с
своею дочерью. И вот, относительно всего его
семени определено и начертано на скрижалях,
чтобы уничтожить и истребить его, и совершить суд
над ним, как над Содомом, и не оставить ему семени
на земле ко дню осуждения.

И в этот месяц поднялся Авраам от Хеврона и
пошел и жил между Кадетом и Суром на горах
Герарона. И в средине пятого месяца он поднялся
оттуда и жил при клятвенном колодезе. И в средине
шестого месяца Господь посетил Сару, и сотворил
ей, как сказал, и она сделалась беременною. И она
родила ему сына в третий месяц, в средине месяца,
как сказал Бог Аврааму. В праздник начатков жатвы
родился Исаак, и Авраам обрезал своего сына в
восьмой день. Он первый был обрезан согласно
завету, как определено навечно.

И в шестой год четвертой седмины пришли мы к
Аврааму к клятвенному колодезю и явились ему, как
сказали Саре, что придем к ней. А она сделалась
беременною сыном, и мы возвратились в седьмой
месяц, и нашли Сару беременною пред нами, и
благословили Сару, и рассказали Саре все, что
было повелено нам относительно него (т.е.
Авраама), что он не умрет, пока не родит шесть
сыновей, и что он увидит их, прежде чем умрет, но
что в Исааке будет наречено имя его и семя, и что
все семя его сыновей будет язычниками и
причтется к язычникам; но только семя от сыновей
Исаака будет святым, и не причтется к язычникам;
ибо оно будет наследием Всевышнего, и все его
семя будет между теми, которые почитают Бога,
чтобы быть для Господа драгоценным украшением
пред всеми народами и быть царством и народом
святым. И мы прошли наш путь, и передали Саре все,
что мы сказали ему (Аврааму). И они оба друг с
другом были в великой радости. И он устроил там
жертвенник Господу, Который спас его и
возвеселил его в стране его странствования, и
праздновал торжество в этом месяце в течение
семи дней близ жертвенника, который он устроил
при клятвенном колодезе, и устроил кущи для себя
и своих рабов к этому празднику. И он праздновал этот
праздник в первый раз на земле; и в эти семь
дней он приносил каждодневно на жертвеннике
Господу всесожжение: семь волов, двух молодых
козлов, двух овнов, семь овец; одного козла в
жертву за грех, чтобы искупить ею себя и свое
семя; и в жертву благодарения семь овнов, семь
молодых козлов, семь овец, семь тельцов вместе с
плодовою жертвою и возлиянием, которые
относились к сему. Над всем их туком он воскурял
на жертвеннике избранное всесожжение в приятное
благовоние. Утром и вечером он воскурял ладан, и
халван, и стакти, и нард, и мирру, и Сенегал и кост;
все эти семь веществ он приносил
истолченными, смешанными между собою по равной
части и очищенными. И он праздновал этот праздник
в течение семи дней, радуясь в своем сердце и всею
душою,~--- он и все, бывшие в его доме; и ни одного
чужеземца не было с ним, и ни одного
незаконнорожденного. И он прославлял своего
Творца, Который создал его в его роде, ибо Он по
Своему благоволению создал его. Ибо он знал и
уразумел, что от него придет растение
праведности для будущих родов и что равным
образом от Него придет святое семя, от Него,
который все создал. И он прославил Его, и нарек
имя этому празднику~--- праздник Господень, и
радовался радостию, которая была приятна
Всевышнему Богу. И мы благословили его вовек и
все его семя после него на все роды земли, ибо он
праздновал тогда этот праздник по свидетельству
небесных скрижалей. Посему на небесных скрижалях
определено для Израиля, чтобы они праздновали
праздник кущей в течение семи дней с радостию, в
седьмой месяц, дабы это было приятно Господу, в
вечный закон для родов их, на все века и годы; и
нет для сего установления конца дней, но
навек определено относительно Израиля, чтобы они
праздновали его, и жили в кущах, и полагали венки
на свои головы. И как они берут от ручья покрытую
листьями ивовую ветвь, так брал и Авраам сережки
от пальмовых ветвей и хорошие древесные плоды, и
обходил каждый день с ветвями вокруг жертвенника
семь раз в день, и утром он восхвалял и благодарил
Бога своего за все с радостию.

\vs Jub 17:1
И в первый год пятой седмины этого юбилея Исаак
был отнят от груди, и Авраам сделал большой пир на
третий месяц, в день, когда сын его Исаак был
отнят от груди. И Измаил, сын египтянки Агари, был
пред лицем отца своего Авраама на своем месте. И
Авраам радовался и прославлял Бога, что он увидел
от себя сыновей и не умер без сыновей. И он
вспомнил слово, как Он говорил с ним в тот день,
когда Лот отделился от него. И он радовался, что
Бог дал ему семя на земле, чтобы получить в
наследие страну. И он прославил громким голосом
Творца всех вещей. И когда Сара увидела Измаила,
как он был весел и плясал и что даже Авраам
радовался при этом, то почувствовала зависть при
взгляде на Измаила и сказала Аврааму: <<Выгони
эту служанку и ее сына; сын этой служанки не
должен наследовать с моим сыном Исааком>>. И это
показалось неприятным Аврааму ради его служанки
и его сына, что он должен выгнать их от себя. И
Господь сказал Аврааму: <<Не нужно тебе
огорчаться из-за отрока и рабыни; все, что сказала
тебе Сара, послушайся ее слова и исполни его, ибо
в Исааке наречется тебе имя и семя. Сына же этой
рабыни Я сделаю великим народом, ибо он~--- твой
род>>. И Авраам собрался рано утром, взял хлеба и
мех с водою, и положил их на плечи Агари вместе с
отроком, и отослал их. И она пошла, блуждая в
пустыне Вирсавии. И не стало воды в мехе; и отрок
истомился от жажды, и не мог идти и упал. И мать
взяла его, и пошла и бросила под масличное дерево.
И она пошла дальше, и села против него, удалившись
на выстрел из лука, ибо сказала: <<Я не могу
смотреть на смерть моего сына>>. И вот она села и
плакала. Тогда Ангел Божий, один из святых, сказал
ей: <<Что ты плачешь, Агарь? встань, подними
отрока и возьми его своею рукою, ибо Господь
услышал твой голос>>. И когда она увидела
отрока, подняла свои глаза и увидела колодезь с
водою, и пошла туда, наполнила свой мех водою и
напоила свое дитя. И она встала и пошла к Фараону.
И отрок вырос и сделался стрелком из лука, и
Господь был с ним. И мать его взяла ему жену из
дочерей Египетских, и она родила ему сына. И он
нарек ему имя Навайвоф, ибо она сказала: <<Бог
был близ меня, когда я призывала его>>.

И случилось в седьмую седмину в первый год в
первый месяц этого юбилея, в двенадцатый день
сего месяца, были сказаны на небесах некоторые
слова об Аврааме, что он верен во всем, что
Господь говорит ему, и что он любит Его и верен во
всяком искушении. Тогда пришел начальный Мастема
и сказал пред Богом: <<Вот Авраам любит и
дорожит своим сыном Исааком больше всего; скажи
ему, чтобы он принес его во всесожжение на
жертвеннике, и Ты увидишь, исполнит ли он это
повеление, чтобы узнать Тебе, верен ли он во всем,
чем Ты его испытываешь>>. И Бог знал, что Авраам
верен во всех испытаниях, которые Он назначает
ему, ибо Он искушал его царством царей, и затем
женою его, когда она была похищена у него, и далее
Измаилом и Агарью, его служанкою, когда он
отослал их, и во всем, чем Он искушал его, он
оказался верным, и его душа не была мятежною, и не
медлил он исполнять сие, ибо был верен и любил
Бога.

\vs Jub 18:1
И Господь сказал Аврааму: <<Авраам!>> И он
сказал: <<Вот я!>> И Он сказал ему: <<Возьми
возлюбленного твоего сына Исаака, и пойди на
высокую гору, и принеси его в жертву на одной
из гор, которую Я тебе покажу>>. И он собрался
оттуда утром на рассвете, и оседлал свою ослицу, и
взял двух своих рабов с собою и своего сына
Исаака, и наколол дров для жертвы. И он шел к назначенному
месту три дня и увидел то место издали. И он
пришел к колодезю с водою и сказал своим рабам:
<<Останьтесь здесь с ослицею; я и отрок пойдем и,
когда помолимся, возвратимся к вам>>. И он взял
дрова для жертвы и возложил на плечи сыну своему
Исааку, и взял в руки огонь и нож, и они пошли оба
вместе к тому месту. И Исаак сказал своему отцу:
<<Отец!>> И он сказал: <<Вот я, сын мой>>.
И он сказал: <<Вот здесь нож и дрова, где же овца
для всесожжения, отец мой?>> И он сказал:
<<Господь усмотрит себе овцу для всесожжения,
сын мой>>. И он пошел к месту горы Божией, и
устроил жертвенник, и положил дрова на
жертвенник, и поднял сына своего Исаака, и
положил его на дрова на жертвенник, и простер
руку свою взять нож, чтобы заколоть сына своего
Исаака. И я (Ангел) стал пред ним (пред Богом?) и
пред высшим Мастемой. И Господь сказал: <<Скажи
ему, чтобы он не возлагал руки своей на отрока и
не делал ему никакого вреда, ибо Я знаю, что он
богобоязнен>>. И я воззвал к нему с неба и
сказал: <<Авраам, Авраам!>> И он убоялся и
сказал: <<Вот я>>. И Он сказал ему: <<Не
возлагай руки своей на отрока и не делай ему
никакого вреда, ибо теперь Я знаю, что ты
богобоязнен и сам не пожалел твоего
перворожденного сына предо Мною>>. И посрамился
высший Мастема. И Авраам возвел очи свои и увидел,
и вот там был овен, зацепившийся своими рогами. И
Авраам пошел, и взял овна, и принес его во
всесожжение вместо сына своего. И Авраам назвал
то место: <<Господь усмотрел сие>>, так что
говорят: <<Господь усмотрел сие>>, т.е.
гора Сион.

И Господь вторично воззвал Авраама по имени с
неба, как Он возвестил мне, чтобы я говорил с ним
во имя Господа. И Он сказал: <<Моею главою Я
поклялся, говорит Господь: так как ты сделал это,
и твоего перворожденного сына, которого любишь,
ты не пожалел предо Мною, то Я поистине
благословлю тебя и умножу семя твое, как звезды
небесные и как песок на берегу моря. Твое семя
получит в наследие города врагов своих, и
благословятся в семени твоем все народы земли за
то, что ты послушался гласа Моего и показал всем,
что ты верен Мне во всем, что Я возложил на тебя.
Иди в мире!>>

И Авраам пошел к своим рабам, и встали, и пошли
они вместе в Вирсавию, и Авраам жил при
клятвенном колодезе. И он соблюдал сей праздник
ежегодно в течение семи дней с радостию, и назвал
его праздником 1Ъсподним соответственно семи
дням, в продолжение которых он ходил и
возвратился в мире. И так утверждено сие и
записано на небесных скрижалях относительно
Израиля и его семени, чтобы они праздновали этот
праздник в течение семи дней с радостию.

\vs Jub 19:1
И в первый год первой седмины сорок второго
юбилея возвратился Авраам и жил против Хеврона,
т.е. Каръяфарбока. Во вторую седмину в третий год
этого юбилея окончились дни жизни Сары, и она
умерла в Хевроне. И пришел Авраам оплакать и
погребсти ее. И мы испытывали его, покорен ли дух
его и не произнесет ли он устами своими мятежного
слова, но он и здесь оказался покорным и не
возмущался, а с спокойным духом говорил с детьми
Киту (т.е. Хета), чтобы они дали ему место, на
котором он похоронил бы свою умершую. И Господь
наградил его благоволением пред всеми, которые
видели его, и он просил, полный смирения, детей
Хета, и они дали ему землю двойной пещеры против
Мамре, т.е. Хеврона, за сорок серебреников. Но они
просили его, говоря: <<Мы отдадим вам это
даром>>. Но он не взял у них даром, а отдал им
цену за место~--- хорошее серебро, и поклонился им
дважды. И после сего он похоронил свою умершую в
двойной пещере. И всех дней жизни Сары было сто
двадцать семь лет, т.е. два юбилея четыре седмины
и один год. Это годы жизни Сары. И это было десятое
испытание, которым был искушаем Авраам; и он
обнаружил верный и покорный дух. И он не сказал
никакого слова о том, что Бог обещал ему дать
страну ему и его семени после него, но он просил
там только о местах, чтобы похоронить свою
умершую. Так оказался он верным и покорным, и
записан был, как друг Господа, на небесных
скрижалях.

И в четвертый год ее (второй седмины) взял он
сыну своему Исааку жену по имени Ревекка, дочь
Вафуила, сына Нахорова, брата Авраама. И Авраам
взял себе третью жену по имени Кетура, из дочерей
своих домашних рабов; ибо Агарь умерла прежде
Сары; и она родила ему шесть сыновей: Ценбари, и
Якзана, и Мадая, и Ийясбока, и Зигийю.

Во второй год шестой седмины Ревекка родила
Исааку двух сыновей~--- Иакова и Исава. И Иаков был
благочестив, а Исав~--- муж грубый, земледелец и
волосатый; и Иаков жил в шатрах. И юноши подросли:
и Исав научился, так как он был земледельцем и
охотником, войне и всякому грубому затятию. И
Иакова любил Авраам, а Исава Исаак. И Авраам видел
занятие Исава и уразумел, что в Иакове будет
наречено ему имя и семя. И он призвал Ревекку и
дал ей повеление относительно Иакова, ибо он
видел, что она также любила гораздо более Иакова,
нежели Исава. И он сказал ей: <<Дочь моя! береги
сына моего Иакова, ибо он будет вместо меня на
земле в благословление между сынами
человеческими и всему своему семени имя его будет
во славу. Ибо я знаю, что Господь произведет от
него народ и он будет предпочтен пред всеми,
которые на лице земли. И вот, сын мой Исаак любит
Исава более, нежели Иакова, и я вижу, что ты
действительно любишь Иакова. Так сделай ему еще
больше добра, и да будет он твоим возлюбленным сыном,
ибо он будет мне в благословение на земле
отныне до всех родов века. Да укрепятся руки твои
и да возрадуешься ты о сыне твоем Иакове, ибо я
люблю его более всех моих сыновей; ибо он будет
благословен вовек, и семя его наполнит всю землю.
Ибо как не может человек сосчитать пыль земную,
так не может быть исчислено и семя его. И все
благословения, которыми Господь благословил
меня и мое семя, будут также уделом Иакову и его
семени во все дни. И в его семени будет
благословлено мое имя, и имя моих отцов Сима, и
Ноя, и Еноха, и Малалела, и Сифа, и Адама. Да
послужат они к тому, чтобы основать небо, и
утвердить землю, и обновить светила, которые на
тверди небесной>>.

И он призвал Иакова пред очи матери его Ревекки,
и поцеловал его, и благословил его, и сказал:
<<Возлюбленный сын мой Иаков, которого
возлюбила душа моя! Да благословит тебя Бог с
высоты тверди небесной, и да даст тебе все
благословения, которыми Он благословил Адама, и
Еноха, и Ноя, и Сима; и все, что Он говорил со мною,
и все, что он обещал дать только мне, да пошлет Он
на тебя и на твое семя до века, пока небо
существует над землею. И да не владычествуют над
тобою и над твоим семенем духи Мастемы, чтобы
отвращать тебя от Господа, Который есть Бог твой,
отныне до века! И да будет Господь твоим Богом и
твоим отцом, а ты Его первородным сыном и Его
народом во все дни! Иди, сын мой, в мире!>> И они
все (?) вместе вышли от Авраама. И Ревекка любила
Иакова всем сердцем и всею душою и гораздо
больше, чем Исава. И Исаак любил гораздо больше
Исава, нежели Иакова.

\vs Jub 20:1
И в сорок второй юбилей в первый год седьмой
седмины призвал Авраам Измаила и двенадцать его
сыновей, и Исаака и обоих его сыновей, и шесть
сыновей Кетуры и детей их, и заповедал им хранить
пути Господа, чтобы поступали по справедливости
и любили друг друга, чтобы поступали таким же
образом во всякой войне, чтобы против каждого,
кто будет против них, они выходили все вместе, и
совершали правду и справедливость на земле,
чтобы они своих сыновей обрезывали по завету,
который Он заключил с ними, и не уклонялись бы ни
направо, ни налево от всех путей, <<которые
Господь заповедал нам>>, и соблюдали бы себя от
всякой мерзости, и избегали бы всякой мерзости и
блуда. <<И если какая-либо женщина или девица
совершит прелюбодеяние между вами, то сожгите ее
огнем, и не блудите вслед за нею очами и
сердцем>>. И пусть они не берут себе жен из
дочерей Ханаанских, ибо семя Ханаана будет
истреблено на земле. И он говорил им о суде над
исполинами и суде над Содомом, как они были
наказаны за их порочность, и блудодеяние, и
нечистоту, и взаимное развращение. За
блудодеяние погибли они, но вы воздерживайтесь
от всякого любодеяния и мерзости, и от всякого
осквернения грехами и мерзостию их, чтобы вам не
сделать имя наше проклятием и всю жизнь вашу позором,
и не предать бы всех сыновей ваших погибели
от меча, и чтобы проклятие ваше не было как Содом
и остаток ваш как сыны Гоморры. Я свидетельствую
вам, сыны мои: любите Бога небес и покоряйтесь
всем Его заповедям, и не обращайтесь к идолам и
мерзостям их (язычников); и не делайте себе ни
литых идолов, ни изваяний, ибо они ничтожны, и не
имеют души, но они суть дело рук, и все, которые
полагаются на них, не получают помощи,~--- все,
которые положились на них. Не почитайте их и не
поклоняйтесь им, а почитайте Бога Всевышнего, и
поклоняйтесь всегда Ему, и надейтесь на Твое
лице, о Господи, (?) во всякое время, и совершайте
правду, и справедливость, и праведность пред Ним,
чтобы Он имел благоволение к вам, и являл вам Свое
милосердие, и ниспосылал дождь утром и вечером, и
благословлял всякий труд ваш и все, над чем вы
трудитесь на земле; и ваш посев, и твою (?) воду, и
семя твоей плоти, и семя твоей земли, и твои стада
и овец благословит Он, и ты будешь во
благословение на земле, и все народы земли будут
иметь благоволение к вам и благословлять сынов
ваших именем моим, дабы они были благословлены,
как я>>.

И он дал Измаилу и его сыновьям и сыновьям
Кетуры подарки, и отослал их от своего сына
Исаака. И Измаил с сыновьями своими и сыновья
Кетуры с сыновьями их пошли вместе, и жили от
Фармона (вероятно, Фаран), пока не придешь к
Вавилону, во всей области, которая лежит к
востоку против пустыни, И они соединились вместе,
и были названы арабами и измаильтянами.

\vs Jub 21:1
И в шестой год седьмой седмины этого юбилея
Авраам призвал сына своего Исаака и заповедал
ему, говоря: <<Я стар и не знаю, когда умру, ибо я
пресытился днями своими. И вот мне сто семьдесят
пять лет, и в продолжение всей жизни моей я
помышлял о Господе, и от всего сердца стремился
исполнять волю Бога моего и ходить право по всем
Его путям. Идолов ненавидела душа моя, дабы быть
внимательным к исполнению воли Того, Кто
сотворил меня; ибо Он~--- Бог живый, и свят, и верен, и
праведен во всем, и нет неправды в Нем, чтобы
взирать на лице и принимать дары; но Он есть Бог
правды, совершающий наказание над всеми, которые
преступают его заповеди и нарушают Его завет. И
ты также, сын мой, соблюдай Его заповеди, Его
установление и правду, и не ходите (?) вслед за
мерзостию язычников, и изваяниями и литыми
изображениями, и не ешьте крови ни<b> </b>зверей, ни
скота, ни различных птиц, которые летают на небе.
И если ты закалываешь, то закалывай в жертву мира,
которая приятна Богу: закалывай ее, и кровь ее
выливай к жертвеннику, с мукою и плодовыми
жертвами, смешанными с маслом, вместе с жертвою
возлияния. Принеси все эта на жертвеннике
всесожжения в приятное благоухание пред
Господом. Как при жертве благодарения, положи
куски тука на огонь жертвенника, именно~--- тук
чрева, и тук внутренностей, и обе почки и весь тук
на них, и тук на стегнах, и печень вместе с
прилежащими к ней почками, И ты принесешь все в
доброе благоухание, которое приятно Господу, с
первыми жертвами и возлияниями, которые к сему
относятся, в доброе благоухание, как хлеб
всесожжения для Господа. Мясо же сей жертвы ешь
в этот и в следующий дни, и не дай солнцу во второй
день зайти над ним, пока оно не съедено. И ничего
не должно оставлять на третий день, ибо это
неприятно и неугодно Господу и его нельзя уже
съедать. Все, которые будут есть его, понесут на
себе грех; ибо так нашел я написанным о сем в
книге моих праотцев, в словах Еноха и Ноя. На свою
плодовую жертву ты должен положить соли, и без
соли завета не должны быть оставляемы все твои
плодовые жертвы пред Господом.

И в отношении к жертвенным дровам ты должен
остерегаться, чтобы не принести какое-нибудь
другое жертвенное дерево, как только кипарис, и
ель, и миндаль, и сосна, и пихта, и кедр, и
можжевельник, и лимон, и маслина, и мирт, и лавр, и
кедр, называемый арбот, и бальзамовый кустарник.
Из этих пород деревьев полагай под всесожжение
на жертвеннике, после того как ты рассмотришь их
наружность, и не клади [...] разрушенного дерева; но
твердое и безукоризненное, лучшее и
новорастущее, и не старое, ибо запах у него исчез
и его нет уже в нем, как прежде. Кроме этих дров не
клади других, ибо они не имеют запаха. И да
вознесется от тебя воня благоухания их к небу.
Соблюдай сию заповедь и исполняй ее, сын мой,
чтобы поступать право во всяком своем деле.

И всякий раз будь чист своим телом и омывайся
водою, прежде чем приступишь принести жертву на
жертвеннике; омой руки и ноги, прежде чем
приблизиться к жертвеннику. И когда ты
приготовишь жертвоприношение, то опять омой руки
и ноги, чтобы не оказалось следов крови ни на вас,
ни на ваших одеждах. Будь очень осторожен, сын
мой, с кровью, будь очень осторожен. Закопай ее в
землю, и не ешьте крови, ибо она есть душа; совсем
не ешь крови.

И не бери выкупа за кровь какого-либо человека,
чтобы она не была пролита даром без наказания;
ибо эта кровь, которая проливается, делает землю
греховною, и она не может быть очищена от крови,
как только кровью того, кто пролил ее. И не
принимай выкупа и дара за человеческую кровь:
кровь за кровь; тогда она вас сделает угодными
Господу, всевышнему Богу, и он будет хранителем
блага, чтобы сохранять тебя от всякого зла и
спасать тебя от всякой смерти. Я вижу, сын мой, все
дела сынов человеческих, что они~--- грех и зло; и
всякое дело их~--- мерзость, и жестоковыйность, и
осквернение, и нет правды в нем. Берегись, не ходи
по их путям, и не следуй по стезям их, и не
совершай смертного греха пред всевышним Богом, а
не то отвратит Он лице Свое от тебя, и вменит тебе
вину твою, и истребит тебя в сей стране и твое
семя под небом, чтобы имя твое и семя твое исчезли
на всей земле. Удаляйся от всех дел их и всякой
мерзости их, и храни защиту Бога всевышнего, и
исполняй волю Его и поступай право во всем. Тогда
Он благословит тебя во всех твоих делах, и
произведет от тебя растение правды для всей
земли, на все роды земли. И будут знать имя мое и
имя твое под небом во все дни. Иди, сын мой, в мире;
да укрепит тебя всевышний Бог, Бог мой и Бог твой,
исполнять Его волю; да благословит Он все семя
твое и остаток твоего семени на вечные роды всеми
благословениями правды, дабы ты был благословен
на всей земле!>> И он вышел от него, исполненный
радости.

\vs Jub 22:1
И было в первую седмину сорок третьего юбилея
во второй год, т.е. в тот год, когда умер Авраам,
пришли Исаак и Измаил от клятвенного колодезя,
чтобы праздновать семидневный праздник, т.е.
праздник начатков жатвы, со своим отцом Авраамом.
И Авраам обрадовался, что пришли два его сына.
Именно, Исаак имел много имущества в Вирсавии, и
ходил туда, чтобы осмотреть свое имущество, и
возвратился теперь к своему отцу. И в эти дни
пришел Измаил, чтобы видеть своего отца; и они
пришли оба вместе. Тогда Исаак заколол жертву во
всесожжение, и принес ее на жертвеннике своего
отца, устроенном им в Хевроне, и принес жертву, и
сделал торжественный пир своему брату Измаилу. И
Ревекка приготовила новый хлеб из нового жита; и
она дала его Иакову, своему предпочтенному
сыну, чтобы он отнес своему отцу Аврааму первый
плод земли, дабы он ел и благословил Творца всех
вещей, прежде чем умрет. И Исаак также послал чрез
Иакова, предпочтенного, Аврааму от
благодарственной жертвы, чтобы он ел и пил. И он
ел и пил, и благословлял всевышнего Бога, Который
создал и небо и землю, и распростер всю землю, и
дал сынам человеческим пищу и питие. И он благословил
своего Творца: <<И ныне благодарю Тебя, Боже
мой, что Ты удостоил меня видеть сей день. Вот я
теперь ста семидесяти пяти лет, седой и
престарелый. И все мои дни~--- суд мира: меч
ненавистника не победил меня; [...] и во всем, что Ты
давал мне и моим детям во все дни жизни моей до
сего дня. Боже мой, да будет милость Твоя на рабе
Твоем и на семени его сыновей, чтобы оно было для
Тебя избранным народом и наследием пред всеми
народами земли, отныне до всех дней родов земли,
во все века>>.

И он подозвал Иакова и сказал ему: <<Сын мой
Иаков! да благословит тебя Бог всех вещей, и да
укрепит тебя~--- совершать правду и волю Его [...], и
изберет тебя и семя твое, чтобы вы были Ему
народом, как наследие Его, согласно Его воле! И
подойди сюда, сын мой Иаков, и поцелуй меня!>> И
он подошел и поцеловал его.

Тогда он сказал: <<Да будут благословлены
Иаков и все сыны его Господом, Всевышним, во все
века! Да даст тебе Господь семя правды от сынов
твоих, которое святило бы Его по всей земле!
Да послужат тебе и падут пред семенем твоим все
народы! Будь силен пред людьми! И так как ты
уподобишься во всем семени Сифа, то да будут пути
твои и пути сыновей твоих правыми, чтобы народ
твой был свят. Бог, Всевышний, да даст тебе все те
благословения, которыми Он благословил меня и
которыми благословил Ноя и Адама! Да покоятся они
на священном темени (главе) твоего потомства на
все роды и до всей вечности! И да сохранит тебя
Господь чистым от всякого мерзкого осквернения,
чтобы получить тебе прощение во всякой вине,
которую ты по неведению совершишь; и да укрепит
Он тебя и да благословит тебя, чтобы ты
наследовал всю землю. Да восстановит Он завет
Свой с тобою, чтобы ты был Ему народом наследия
Его во все века! И да будет Он тебе и семени твоему
Богом, в действительность и истину, во все дни
земли! Помни же, сын мой Иаков, слово мое и храни
заповедь Авраама, отца твоего! Не сообщайся с
народами, и не ешь с ними, и не поступай по делам
их, и не вступай в родство с ними, ибо (всякое) дело
их нечисто, и все пути их осквернены и суть
мерзость. Свои жертвы они закалают мертвым, и
почитают демонов, и едят на могилах; они лишены
мудрости, чтобы разуметь, и очи их ничего не
видят; как еще погрешать им, если они говорят
дереву. <<Ты бог мой>> и камню: <<Ты господь
мой и спаситель мой>>, тогда как они (дерево и
камень) не имеют разума? И ты, сын мой Иаков,~--- Бог,
Всевышний, да вспомоществует тебе, и Бог небесный
да благословит тебя и да удалит тебя от нечистоты
их и от всей греховности их! Берегись, сын мой
Иаков, чтобы не брать жены из всего семени
дочерей Ханаана; ибо семя его предназначено к
истреблению на земле; ибо за вину Хама и за
проступок Ханаана будет уничтожено и все семя
его, и весь остаток его, и что избегло гибели. И
все поклоняющиеся идолам и все упорствующие не
имеют надежды в земле живых, но они сойдут в
царство мертвых, и пойдут к месту осуждения, и не
оставят по себе памяти на земле. Как сыны Содома
были истреблены на земле, так будут истреблены
все, поклоняющиеся идолам. Не бойся, сын мой
Иаков, и не страшись! Бог, Всевышний, будет
охранять тебя от погибели, и от всякого пути
греховного Он спасет тебя. Здесь в этой стране построишь
мне дом, чтобы я положил имя мое на нем,
предназначено тебе и семени твоему вовек, и он
будет называться домом Авраама. Это
предназначено тебе и семени твоему вовек, ибо ты
построишь дом мой и имя мое поддержишь пред
Богом. Вовек будет пребывать семя твое и имя твое
во все роды земли>>. И он перестал изрекать
заповеди и благословения.

И они легли оба вместе на одно ложе, и Иаков
заснул при персях своего деда Авраама. И его душа
семь раз прижимала его к сердцу, и любовь его и
сердце его радовались о нем, и он благословил его
от всего сердца и сказал: <<Бог, Всевышний, Бог
всех вещей и Творец всего, изведший меня из Ура
Халдейского, чтобы дать мне эту страну, дабы я
владел ею вовек и воскресил святое семя, чтобы
оно было благословенно вовек! Благослови и сына
моего Иакова, о котором я радуюсь всем сердцем
моим и любовию моею! Твоя милость и Твоя благость
да пребудут на нем и на семени его всегда! Не
оставляй его и не покидай его отныне до века! И да
будут очи Твои открытыми на него и на его семя,
чтобы охранять его, и благослови его и освяти его
в народ наследия Твоего! Благослови его всеми
благословениями Твоими, отныне до всех дней
вечности; и восстанови завет Твой и милость Твою
с ним и семенем его, и всю волю Твою восстанови с
ним на все роды земли!>>

\vs Jub 23:1
И он положил два перста Иакова на свои очи, и
прославил Бога богов, и закрыл свои глаза. И он
простер ноги свои, и уснул вечным сном, и
приложился к отцам своим. И во все это время Иаков
лежал при его персях, не зная, что отец его Авраам
умер. И Иаков пробудился от своего сна, и вот~---
Авраам был холоден как лед; и он сказал: <<Отец,
отец!>>, но он не говорил. Тогда он узнал, что
Авраам умер, и он встал, и побежал, и известил о
сем мать свою Ревекку. И Ревекка пошла ночью к
Исааку, и известила его о сем. И они пошли вместе с
Иаковом, который имел светильник в руке своей; и
когда они вошли, то нашли лежащее тело Авраама. И
Исаак пал на лице отца своего, и плакал, и
благословлял его, и лобызал его; и вопль раздался
в доме Авраама. Тогда встал сын его Измаил, и
пришел к отцу своему Аврааму, и оплакивал отца
своего Авраама, он и весь дом Авраама; и они
подняли великий плач. И сыновья его Исаак и
Измаил погребли его в двойной пещере вместе с
женою его Сарой. И оплакивали его в продолжение
сорока дней все домочадцы его, Исаак и Измаил со
всеми сыновьями своими и сыновья Кетуры в местах
своего поселения. Потом окончился плач по
Аврааме.

И он жил три юбилея и четыре седмины, сто
семьдесят пять лет, и дни его окончились. Ибо дни
предков простирались до девятнадцати юбилеев; но
после дней потопа они начали уменьшаться и
становиться короче девятнадцати юбилеев. И они
(люди) стали скоро достигать старости и
пресыщаться жизнью вследствие многих несчастий
и вследствие неправды своих путей, исключая
Авраама; ибо Авраам был совершенным во всяком
своем деле с Богом и благоугоден, и (ходил) в
правде в продолжение своей жизни. И вот он не
окончил четырех юбилеев в своей жизни, так что
состарился ради неправды и пресытился жизнью. И
все роды, которые явятся отныне до дня великого
суда, будут скоро достигать старости, прежде чем
достигнут двух юбилеев. И так как и знание их
будет оставлять их по причине их престарелости,
то уменьшится все знание их. И в те дни, если кто
проживет полтора юбилея, то об нем будут
говорить: <<Он жил долго>>, но большая часть
его жизни пройдет в несчастии, и труде, и
страдании, и без мира; ибо наказание последует за
наказанием, мучение за мучением, нужда за нуждой,
зло за злом, болезнь за болезнью, и все таковые
злые наказания одно за другим: болезнь, и резь в
животе, и град, и лед, и снег [...], и несчастие, и
оцепенение, и неплодородие, и смерть, и меч, и
пленение, и всякие наказания и несчастия. Все это
придет на злой род, который наполнит беззаконием
землю чрез нечистоту блудодеяния и скверноты и
чрез мерзость своих деяний. И тогда будут
говорить: <<Жизнь предков продолжалась до
тысячи лет, и она была хороша; а дней нашей жизни,
если человек проживет долго, семьдесят лет, и
если они сильны, восемьдесят лет, и вся она
нехороша>>. И не будет мира во дни того злого
рода. И в том роде дети будут злословить своих
отцов и старцев за греховность и нечистоту, и за
речи их уст, и за великие нечестия, которые они
совершают, и за то, что они оставили завет,
который Господь заключил между ними и Собою, дабы
они соблюдали и хранили все Его заповеди, и
постановления, и весь закон Его, не уклоняясь ни
налево, ни направо; так что все они совершают
злое, и все уста их говорят беззаконное, и всякое
дело их нечистота и мерзость, и все пути их
осквернение, и нечистота, и погибель. Вот земля
погибнет ради всех дел их; и не будет более семени
от вина и елея, так как все дела их~--- полное
нечестие; и все вместе погибнут: дикие звери, и
скот, и птицы, и все морские рыбы~--- из-за сынов
человеческих. И они будут враждовать друг с
другом, эти с теми, юноши со старейшинами, и
старейшины с юношами, бедные с богатыми, и
униженные с великими, и нищий с князьями~--- именно,
относительно закона и завета; ибо они забыли Его
заповеди, и завет, и праздники, и новолуния, и
субботы, и юбилеи, и всякую правду. И они будут
восставать с мечами и луком, чтобы привести их
обратно на путь, но они не возвратятся, пока не
прольется много крови на земле; один будет против
другого, и те, которые останутся, не возвратятся
на путь правды от своего нечестия. Ибо все они
будут восставать, чтобы расхищать богатство, и
брать, что принадлежит другому, и приобретать
себе великое имя, но не для правды и истины; и
святое святых осквернят они мерзостью своего
осквернения. И придет великое осуждение ради дел
того рода от Господа, и Он предаст их мечу, и на
осуждение, и пленение, и расхищение, и пожрание. И
Он возбудит на них грешников~--- которые не знают
сострадания и милости и не взирают на лицо ни
старого, ни юного, ни другого кого-либо, но на злых
и могущественных людей,~--- чтобы они поступали
яростнее, чем все сыны человеческие, и совершали
насилие против Израиля и делали беззаконие
Иакову. И много крови прольется на земле. И не
будет никого собирающего и никого ближнего. В те
дни будут они издавать вопль, и взывать, и
умолять, чтобы их освободили от руки греховных
язычников, но не будет никого спасающего. И
головы детей будут белыми от седых волос, и
трехнедельное дитя будет казаться старым, как
столетний; и их существование будет приведено к
погибели чрез страдание и бедствие.

И в те дни дети начнут оставлять свои
(греховные) законы, и стремиться к заповедям, и
возвращаться на путь правды. И дни начнут
возрастать, и сыны человеческие будут достигать
большей старости, от рода до рода и от дня до дня,
так что время жизни их продлится тысячу лет. И не
будет старого и пресыщенного жизнью, но все они
будут как дети и отроки, и скончаются все дни их в
мире и радости, и будут они жить так, что тогда не
будет сатаны и какого-либо губителя; ибо все дни
их будут днями благословения и спасения. В то
время Господь исцелит своих слуг; и они
вознесутся и будут наслаждаться глубоким миром,
и опять преследовать своих врагов. И они увидят
это и будут благодарить и радоваться радостью до
века. И они увидят на своих врагах все наказания
их и все проклятие их; и хотя кости их будут
покоиться в земле, но для духа их будет многая
радость, и они познают, что это Господь,
совершающий суд и являющий милость на сотнях и
тысячах, и на всех, которые любят Его. И ты, Моисей,
запиши сие слово; ибо так начертано на
свидетельстве небесных скрижалей для вечных
родов.

\vs Jub 24:1
И было, после того как Авраам умер, Бог
благословил сына его Исаака. И он поднялся от
Хеврона и пошел дальше, и жил при кладезе видения,
в первый год третьей седмины этого юбилея, в
продолжение семи лет.

И в первый год четвертой седмины начался голод
в стране, сверх прежнего, который был во время
Авраама. И Иаков сварил чечевичное кушанье; тогда
пришел Исав с поля голодный. И он сказал брату
своему Иакову: <<Дай мне от этого кушанья
плод!>> И Иаков сказал ему: <<Передай мне твое
первородство, тогда я дам тебе хлеба и также плод
от этого кушанья>>. И Исав сказал в сердце своем:
<<Я должен умереть: что мне за польза быть
первородным?>> И он сказал Иакову: <<Я отдаю
тебе его>>. И Иаков сказал: <<Поклянись мне!>>
И он поклялся ему. И Иаков дал брату своему Исаву
хлеба и кушанья; и он ел, пока не насытился. Так
пренебрег Исав первородством; посему Исав
называется также Едомом ради плода кушанья,
которое Иаков дал ему за его первородство. И
Иаков сделался старшим; Исав же потерял свое
преимущество.

И был голод в стране; тогда Исаак пошел, чтобы
спуститься в Египет, во второй год этой седмины. И
он пошел к царю Филистимскому в Герарон к
Авимелеху. И Господь явился ему и сказал ему:
<<Не ходи в Египет, оставайся в стране, которую Я
указываю, и будь чужеземцем в оной стране; Я буду
с тобою и благословлю тебя. Ибо тебе и твоему
семени Я дам всю эту землю, и исполню клятву Свою,
которою Я поклялся отцу твоему Аврааму, и умножу
семя твое, как звезды небесные, и дам всю эту
землю твоему семени. И благословятся в твоем
семени все народы земли~--- за то, что отец твой
слушался гласа Моего, и соблюдал слово Мое, и Мои
заповеди, и законы, и установления, и завет. И
теперь и ты слушайся гласа Моего и Моих заповедей
и живи в этой стране!>>

И он жил в Герароне три седмины. И Авимелех
отдал приказание относительно него и
относительно всего имущества его и сказал:
<<Всякий, кто прикоснется к нему или к
чему-нибудь из его имения, умрет смертию>>. И
Исаак стал великим в Филистимской земле, и
приобрел много волов и овец, и верблюдов, и много
имущества. И они сеяли в земле филистимлян, и
получили прибыли во сто крат. И Исаак сделался
очень великим. И филистимляне стали завидовать
ему; и все колодези, которые отроки Авраама
выкопали при его жизни, филистимляне засыпали
после смерти Авраама и завалили их землею. И
Авимелех сказал Исааку: <<Дались от нас, ибо ты
стал великим для нас!>> И Исаак вышел оттуда в
первый год седьмой седмины и странствовал в
долинах Герарона. И они опять выкопали колодези,
которые отроки отца его Авраама выкопали, и
филистимляне после смерти отца его Авраама
засыпали. И он назвал их именем, которое нарек им
отец его Авраам. И отроки Исаака выкопали
колодези в долине и нашли источник воды. И
пастухи герарские спорили с пастухами Исаака,
говоря: <<Вода принадлежит нам>>. И Исаак
назвал место сего колодезя: <<противный>>,
<<ибо они враждовали с нами>>. И они выкопали
другой колодезь и спорили из-за него, и Исаак дал
ему имя~--- <<теснота>>. И он вышел оттуда, и они
выкопали другой колодезь, и о нем они не спорили;
тогда он дал ему имя~--- <<пространный>>. И Исаак
сказал: <<Теперь Господь распространил нас>>.
И он усилился в земле той и пришел к кладезю
клятвенному в первый год первой седмины в сорок
четвертый юбилей.

И Господь явился ему в ту ночь, в новолуние
первого месяца, и сказал ему: <<Я Бог Авраама,
отуа твоего; не бойся, ибо Я с тобою. И Я
благословлю тебя и умножу семя твое, как песок
морской, ради раба Моего Авраама>>. И он устроил
там жертвенник, где прежде устроил отец его
Авраам, и призвал имя Господа, и принес жертву
Богу отца своего Авраама. И они выкопали колодезь
и нашли источник воды. И отроки Исаака выкопали
еще колодезь и не нашли воды. И они пришли и
сказали Исааку, что не нашли воды. И Исаак сказал:
<<Я поклялся ныне филистимлянам, и это есть
причина (безводности кладезя)>>. И он дал имя
месту сему <<клятвенный колодезь>>. Ибо здесь
он поклялся Авимелеху и Акофу, другу его, и
Фиколу, надзирателю его. И Исаак познал в тот
день, что они ложно поклялись~--- хранить с ними мир.
И Исаак проклял в тот день филистимлян и сказал:
<<Да будут прокляты филистимляне в день гнева и
ярости всеми народами! Пусть сделает их Господь
посмешищем, и проклятием, и гневом, и яростью в
руках грешных язычников и истребит их рукою
Хеттеев. И что избегнет меча врагов и Хеттеев, то
да истребит народ праведных судом праведным под
небом. Ибо врагами и ненавистниками будут они для
сынов моих во дни их и в земле их. И никто из
них не останется и никто не спасется в день
гневного суда. Но погибнет, и истребится, и будет
уничтожено в стране сей все семя филистимлян, и
от них не останется более и потомства их на земле.
Если бы оно взошло даже на небо, то да
низвергнется оттуда; и если бы оно утвердилось на
земле, то да будет исторгнуто, и если бы укрылось
между народами, то да будет истреблено и отсюда, и
если бы оно взошло в царство мертвых, то и там да
будет велико его наказание, и пусть не будет ему
там мира, и если бы оно странствовало в плену, то
да будет умерщвлено теми, которые подстерегают
на пути его душу. Не оставляй ему Ты, Который да
будешь прославлен, имени и семени на всей земле, и
да сопровождает его вечное проклятие!>> И
относительно него написано и начертано на
небесных скрижалях, чтобы так было поступлено с
ним в день суда, дабы оно было истреблено на
земле.

\vs Jub 25:1
И во второй год этой седмины в этом юбилее
Ревекка призвала сына своего Иакова и беседовала
с ним, говоря: <<Сын мой, не бери себе жену из
дочерей Ханаана, как брат твой Исав, который взял
себе двух жен из семени Ханаана; и они поразили
мой дух всеми делами своими, нечистотою блуда и
брака, и нет правды в них, но злы дела их. И я
весьма люблю тебя, сын мой; моя нежность
благословляет тебя каждый час и стражу нощную. И
ныне послушай гласа моего, и исполни волю матери
твоей, и не бери себе жену из дочерей сей страны, а
только из дома отца твоего и из рода отца твоего.
Возьми себе жену из дома отца моего; и Бог
всевышний благословит тебя и сынов твоих сделает
праведным родом и семя твое святым>>. После сего
Иаков говорил с матерью своей Ревеккою и сказал
ей: <<Вот, мать моя, мне девять седмин, и я не знаю
жены; ни одна не прикасалась ко мне и не
обручалась мне, и я не думаю брать себе жену из
какого-либо семени дочерей Ханаана, ибо я помню
слова отца нашего Авраама и что он заповедал мне,
что я не должен брать жены из всего семени дома
Ханаана. Но я хочу взять себе жену из семени дома
отца моего и из рода моего. Я слышал прежде, что
брат твой Лаван имеет в потомстве дочерей. На них
обратил я свое сердце, чтобы из них взять жену. И
посему я остерегался в своем духе, чтобы не
согрешить и не развратиться на всех путях моих,
во все дни жизни моей. Ибо относительно брака и
блуда отец мой Авраам дал мне много заповедей. И
вопреки всему тому, что он заповедал мне, брат мой
теперь прекословит мне в продолжение двадцати
двух лет и говорит часто мне, говоря: <<Брат мой,
возьми в жены сестру моих двух жен!>> Но я не
хочу делать так, как делает брат мой. Я клянусь
пред тобою, что я в продолжение всей моей жизни не
возьму себе жену из семени всех дочерей Ханаана и
не поступлю худо, как поступил брат мой. Не бойся,
мать моя! Поверь мне, что я исполню волю твою, и
буду ходить в праведности, и не извращу моих
путей вовек!>>

После сего она возвела лицо свое на небо, и
распростерла персты своей руки, и открыла уста
свои, и прославила Бога всевышнего, сотворившего
небо и землю, и принесла Ему благодарение и хвалу,
и сказала: <<Да будет прославлен Господь, Бог
мой, и да прославится имя Его вовек, что Он дал мне
Иакова, невинного сына и святое семя; ибо он Твой,
и семя его всегда Твое во все роды вовек.
Благослови его, Владыка, и вложи благословение
правды в уста мои, чтобы я благословила его!>> В
тот самый час дух святой сошел в уста ее, и она
положила руки свои на главу Иакова и сказала:
<<Будь прославлен Ты, Господь правды и Бог
миров, и да прославляют Тебя люди всех родов! Да
дарует Он тебе, сын мой, путь правды, и да откроет
семени твоему правду, и умножит сынов твоих во
время жизни твоей, и восставит их по числу
месяцев года! И да умножатся сыны их, и будут
бесчисленны, как звезды небесные, и да будет
число их больше, чем песок морской! Да даст Он тебе
эту плодоносную землю, как сказал Он, что Он
даст ее Аврааму и семени его после него навсегда
и что они вечно будут владеть ею. И да увижу я в
тебе, сын мой, благословенного сына во время
жизни моей; и все семя твое да будет святым
семенем! И как покоил тебя дух твоей матери во
время жизни ее на лоне родившей тебя, так
благословляет тебя моя нежность, и перси мои
благословляют тебя, и уста мои, и язык мой
прославляют тебя. Умножайся, и возрастай, и
распространяйся на земле, и да будет семя твое
совершенным в небесной и земной радости во всю
вечность! И да ликует семя твое, и в великие дни
мира да будет ему уделом мир твоего имени! И да
пребывает семя твое во всю вечность; и Бог
всевышний да будет их Богом, и Бог всевышний да
обитает с ними, и да будет устроено между ними
святилище Его на все века! Благословляющий тебя
да будет благословен, и всякая плоть,
проклинающая тебя напрасно, да будет
проклята!>> И она поцеловала его и сказала ему:
<<Господь мира да возлюбит тебя, как сердце
матери твоей и нежность ее; да возрадуется Он о
тебе и да благословит тебя!>> И вот она
перестала благословлять его.

\vs Jub 26:1
И в седьмой год этой седмины Исаак призвал
старшего сына своего Исава и сказал ему: <<Сын
мой, я стар, и вот очи мои притупились для зрения,
и я не знаю, когда умру. И теперь возьми свои
охотничьи орудия, свой колчан и лук, и выйди в
поле, и добудь дичи для меня, и налови мне
что-нибудь, сын мой, и приготовь мне кушанье, как
любит душа моя, и принеси его мне, чтобы я ел, и
душа моя благословила тебя, прежде чем я умру>>.
И Ревекка услышала речь его, когда Исаак говорил
Исаву. И Исав вышел рано в поле, чтобы добыть дичи,
и наловить, и принести своему отцу.

И Ревекка позвала сына своего Иакова и сказала
ему: <<Вот я слышала, как отец твой Исаак говорил
с братом твоим Исавом, говоря: <<Налови мне
какой-нибудь дичи, и приготовь мне кушанье, и
принеси его мне, чтобы я благословил тебя пред
Господом, прежде чем я умру>>. И теперь выслушай
слово мое, сын мой, что я тебе велю! Ступай в свое
стадо и принеси мне двух хороших козлят, я
приготовлю из них кушанье, как он любит. И ты
отнесешь его отцу твоему поесть, дабы он
благословил тебя пред Господом, прежде чем умрет,
и ты будешь благословен!>> И Иаков сказал матери
своей Ревекке: <<Мать, я ничего не жалею, что
отец мой может есть н что ему угодно. Только я
боюсь, мать моя, как бы он не узнал моего голоса и
не ощупал меня; ты знаешь ведь, что я гладкий, а
брат мой Исав волосат; и как бы мне не явиться в
его очах преступником, и не сделать чего-либо
такого, чего он не повелел мне, и как бы он не
разгневался на меня, и я навлеку на себя
проклятие, а не благословение>>. И мать его
Ревекка сказала ему: <<Пусть на меня придет твое
проклятие, сын мой; скорее послушайся гласа
моего!>>

И Иаков послушался гласа матери своей Ревекки,
и пошел, и сходил за двумя хорошими тучными
козлятами, и принес их матери своей, и мать его
приготовила их, как он любил. И Ревекка взяла
одежды старшего сына своего Исава, самые дорогие,
какие были у нее в доме, и одела у себя в них
Иакова, и кожею козлят обложила его руки и
открытые части его тела. И она дала кушанье и
обед, который приготовила, в руки сыну своему
Иакову; и он вошел к своему отцу и сказал: <<Я,
сын твой, сделал, что ты сказал мне; встань и сядь,
и поешь того, что я наловил, отец, чтобы душа твоя
благословила меня>>. И Исаак сказал сыну своему:
<<Как это ты так скоро наловил дичи, сын мой?>>
И Иаков сказал: <<Пославший мне это, Бог твой,
был со мною>>. И Исаак сказал ему: <<Подойди
сюда, чтобы я тебя ощупал, сын мой, сын ли ты мой
Исав или нет>>. И Иаков подошел к отцу своему
Исааку, и он ощупал его и сказал: <<Голос~--- голос
Иакова, но руки Исава>>; и он не узнал его; ибо
было соизволение (послание) с неба, которое
отняло чувство его. И Исаак не узнал его, ибо руки
его были, как руки того, и волосаты, как руки
Исава, дабы он благословил его. И он сказал:
<<Сын ли ты мой?>> И он сказал: <<Я сын твой>>.
И он сказал: <<Подай мне поесть того, что наловил
ты, сын мой, дабы душа моя благословила тебя!>> И
он поднес ему, и он ел; и он подал ему вина, и он
пил. И отец его Исаак сказал: <<Подойди и поцелуй
меня, сын мой!>> И он подошел и поцеловал его; и
он ощутил запах одежды его. И он благословил его и
сказал: <<Вот запах от сына моего, как запах от
поля, которое благословил Господь. Да даст тебе
Господь и сделает жребием твоим много росы
небесной и плодородия земли, и много хлеба; и
масла да даст тебе Он в изобилии! И да послужат
тебе народы, и люди (племена) да поклонятся тебе!
Ты будешь господином над братьями своими, и сыны
матери твоей да поклонятся тебе! И все
благословения, которыми Господь благословил
меня и отца моего Авраама, да будут на тебе и
семени твоем до века! Проклинающий тебя да будет
проклят, и благословляющий тебя да будет
благословен!>>

И после того, как Исаак кончил благословлять
сына своего Иакова, и Иаков вышел от отца своего
Исаака и скрылся, пришел его брат Исав с охоты; и
он также приготовил кушанье, и принес его отцу
своему, и сказал отцу своему: <<Отец мой, встань
и поешь моей дичи, чтобы душа твоя благословила
меня!>> И отец его Исаак сказал ему: <<Кто
ты?>> И он сказал: <<Я первенец твой Исав, я
сделал, как ты повелел мне>>. И Исаак
вострепетал великим трепетом и сказал: <<Кто же
тот, который мне добыл дичи, и наловил, и принес,
чтобы я ел от всего, прежде чем ты пришел, и я
благословил его? Да будет благословен он и все
семя его вовек!>> И когда Исав услышал речь отца
своего Исаака, то поднял громкий вопль, горько
сетуя, и сказал отцу своему: <<Благослови и меня,
отец!>> И он сказал ему: <<Твой брат пришел
хитростью и восхитил благословения твои>>. И
Исав сказал: <<Теперь я знаю, почему он
называется Иаковом; дважды он теперь запнул меня:
в первый раз он взял мое первородство, а теперь он
взял у меня мое благословение>>. И он сказал:
<<Разве ты не оставил для меня благословения,
отец?>> И Исаак отвечал и сказал Исаву: <<Вот я
сделал его господином над тобою и над всеми его
братьями, и дал их ему в рабы; изобилием хлеба и
масла я сделал его сильным; что я теперь сделаю
тебе, сын мой?>> И Исав сказал отцу своему
Исааку: <<Разве у тебя только одно
благословение, отец? Благослови и меня, отец!>> И
Исав возвысил голос свой и заплакал. И Исаак
отвечал и сказал ему: <<Вот от тучности земли
будет благословение твое и от росы небесной
свыше; своим мечом будешь питаться ты и будешь
служить твоему брату. И будет, если ты сделаешься
великим и свергнешь ярмо его с выи твоей, то
совершишь смертный грех, и все твое семя будет
истреблено под небом>>. И Исав разгневался на
Иакова за благословение, которым отец его
благословил его, и сказал в сердце своем:
<<Теперь придут дни плача по отце моем, и я убью
брата моего Иакова>>.

\vs Jub 27:1
Тогда было открыто во сне Ревекке слово Исава,
старшего ее сына. И Ревекка послала и призвала
Иакова, старшего сына своего, и сказала ему:
<<Вот брат твой Исав замышляет мщение, чтобы
убить тебя. И ныне послушайся гласа моего, встань
и беги к брату моему Лавану, и оставайся у него
некоторое время, пока не переменится гнев брата
твоего, и он не оставит гнева своего против тебя,
и забудет все, что ты сделал ему, и тогда я пошлю и
вызову тебя оттуда>>. И Иаков сказал: <<Я не
боюсь; если он хочет убить меня, то я сам убью
его>>. И она сказала: <<Так я лишилась бы в один
день обоих моих сыновей>>. И Иаков сказал своей
матери Ревекке: <<Вот ты знаешь, что мой отец
стар, и я вижу, что очи его ослабели; и если я
покину его, то это будет злом пред очами его, что я
оставлю его и уйду от вас; и отец мой разгневается
и проклянет меня. Я не могу идти; только если он
меня отошлет, чтобы я шел, то я пойду>>. И Ревекка
сказала Иакову: <<Я войду и скажу ему это, чтобы
он отпустил тебя>>. И Ревекка вошла и сказала
Исааку: <<Мне стала противною моя жизнь из-за
обеих дочерей Хета, которых Исав взял себе в жены,
и если Иаков возьмет себе жену между дочерями этой
страны, которые такие же, как и те, то зачем
мне еще жить? ибо дочери земли Ханаанской злы>>.
И Исаак призвал своего сына Иакова, и благословил
его, и увещевал его, и сказал ему: <<Не бери себе
жену из всех дочерей Ханаана; соберись, иди в
Месопотамию, в дом Бефуела, отца твоей матери, и
возьми себе оттуда жену из дочерей Лавана, брата
твоей матери. И Бог небесный да благословит тебя,
и возрастит тебя, и умножит тебя, чтобы ты
сделался обществом народов. И да даст Он тебе
благословения отца моего Авраама, тебе и твоему
семени после тебя, дабы ты наследовал землю
твоего странствования и всю землю, которую
Господь дал Аврааму. Иди, сын мой, с миром!>> И
Исаак отпустил Иакова, и он пошел в Месопотамию к
Лавану, сыну Бефуела, сирийцу, брату Ревекки,
матери Иакова.

И было, когда Иаков собрался идти в Месопотамию,
Ревекка опечалилась о своем сыне и плакала. И
Исаак сказал Ревекке: <<Сестра моя! не плачь о
моем сыне Иакове, ибо с миром он пойдет и с миром
возвратится. Бог всевышний охранит его от
всякого зла, и будет с ним, и не оставит его во все
дни; ибо я знаю, что Господь даст успех в путях
его, повсюду, где он пойдет, пока не возвратится к
нам с миром, и мы увидим его в мире. Не бойся за
него, сестра моя, ибо путь его прямой, и он муж
благочестивый и верный, и посему не погибнет.
Не плачь!>> И Исаак утешал Ревекку о сыне ее
Иакове и благословил его.

И Иаков вышел от клятвенного колодезя, чтобы
идти в Харран, в первый год второй седмины сорок
четвертого юбилея, и пришел в Лозу на горе, т.е. в
Вефиль, в новолуние первого месяца, в эту седмину;
и дошел до некоторого места, когда был вечер.
И он уклонился несколько к западу от дороги в ту
ночь и заснул здесь, ибо солнце зашло. И он взял
один из камней того места и положил его под
дерево,~--- ибо он странствовал один~--- и заснул, и
видел сон в ту ночь. И вот лестница была
утверждена на земле, вершина которой досягала до
неба. И вот Ангелы Господни поднимались и
опускались по ней, и сам Господь стоял на ней. И
Господь говорил с Иаковом и сказал: <<Я Бог отца
твоего Авраама и Бог Исаака; землю, на которой ты
стоишь, Я дам тебе и твоему семени после тебя; и
твое семя будет как пыль земная, и ты
размножишься к западу и востоку, и югу, и северу, и
благословятся в тебе и твоем семени все страны
народов. И вот Я буду с тобою, и буду охранять тебя
повсюду, где ты будешь ходить, и возвращу тебя в
мире в эту землю. Ибо Я не оставлю тебя, пока не
исполню все, что Я сказал тебе>>. И Иаков спал
(пробудился) и сказал: <<Точно, это место~--- дом
Господа, и я не знал сего>>. И он убоялся и
сказал: <<Священно это место, на котором ничего
нет иного, как только дом Господень; и это врата
небесные>>. И утром рано встал Иаков и взял
камень, который он положил себе в изголовье, и
поставил его памятником в знамение на этом месте,
и возлил на него сверху елей, и нарек имя тому
месту Вефиль. Прежде же оно называлось Луз, как
страна. И Иаков дал Богу обет, говоря: <<Если
Господь будет со мною, и сохранит меня на том
пути, в который я иду, и даст мне хлеб в пищу и
одежду для одеяния, и я возвращусь в мире в
дом отца моего, то да будет Господь моим Богом, и
также камень этот, который я поставил в этом
месте памятником в знамение, да будет домом
Господним! И из всего, что Ты дашь мне, я дам
десятую часть Тебе, Боже мой!>>

\vs Jub 28:1
И он встал и пошел в Месопотамию, в землю Лавана,
брата Ревекки, лежащую к востоку. И он оставался у
него и служил ему за Рахиль, одну из дочерей его. И
в первый год третьей седмины сказал он ему:
<<Дай мне мою жену, за которую я служил тебе семь
лет>>. И Лаван сказал Иакову: <<Я отдам тебе
твою жену>>. И Лаван устроил пир, и взял Лию, свою
старшую дочь, и отдал ее Иакову в жены, и дал ей
свою рабу Залафу в служанки. И Иаков не заметил
этого, ибо он думал, что это Рахиль. И он вошел к
ней, и вот это была Лия. Тогда Иаков разгневался
на Лавана и сказал ему: <<Зачем ты сделал так? Не
служил ли я тебе за Рахиль, а не за Лию? Зачем ты
обидел меня? Возьми свою дочь и отпусти меня, ибо
ты нехорошо поступил со мною>>. А Иаков любил
Рахиль больше, чем Лию. Ибо глаза Лии были слабы,
но лицом она была очень красива. Рахиль же имела
прекрасные глаза, и лицом она была очень красива
и привлекательна. И Лаван сказал Иакову: <<В
нашей стране нет такого обычая, чтобы выдавать
младшую дочь прежде старшей, и несправедливо
делать это. Ибо так сие определено и написано в
небесных скрижалях, и неправеден тот, кто делает
это, ибо это нехорошее дело пред Господом. И ты
также с своей стороны скажи сынам Израиля, чтобы
они не делали этого, и не позволяли брать и
выдавать младшую, прежде чем выдадут старшую; ибо
это очень нехорошо>>. И Лаван сказал Иакову;
<<Пусть пройдут семь дней пиршества, тогда я дам
тебе Рахиль, чтобы ты служил мне другие семь лет,
чтобы ты пас моих овец, как ты служил в течение
первой седмины (семилетия). [Когда же семь дней
пиршества Лии прошли], Лаван дал Иакову Рахиль,
чтобы он служил ему другие семь лет. И Рахили он
дал в служанки Баллу, сестру Залафы. И он служил
еще семь лет за Рахиль. [...].

И Господь отверз утробу Лии, и она зачала, и
родила Иакову сына, и он дал ему имя Робел, в
четырнадцатый день девятого месяца в первый год
третьей седмины. Утроба же Рахили была заключена,
ибо Господь видел, что Лия была ненавидима, а
Рахиль любима. И Иаков опять вошел к Лии, и она
зачала, и родила Иакову второго сына, и он дал ему
имя Симеон в двадцать первый день десятого
месяца в третий год этой седмины. И Иаков опять
вошел к Лии, и она зачала, и родила ему третьего
сына, и он дал ему имя Левий в новолуние первого
месяца в шестой год этой седмины. И Иаков опять
вошел к Лии, и она зачала, и родила ему четвертого
сына, и он дал ему имя Иуда в пятнадцатый день
третьего месяца в первый год четвертой седмины. И
ради всего этого Рахиль позавидовала Лии, так как
сама она не рождала. И она сказала Иакову: <<Дай
мне сына!>> И Иаков сказал: <<Разве я задержал
плод тебе, плод утробы твоей, разве я покинул
тебя?>> И когда Рахиль увидела, что Лия родила
Иакову четверых детей~--- Робела, Симеона, Левия и
Иуду,~--- то Рахиль сказала ему: <<Войди к моей
служанке Балле, чтобы она зачала и родила мне
сына!>> И он вошел, и она зачала и родила ему
сына, и она нарекла ему имя Дан в девятый день
шестого месяца в шестой год третьей седмины. И
Иаков опять вошел к Балле, и она зачала и родила
Иакову второго сына, и Рахиль дала ему имя
Наффали в пятый день седьмого месяца во второй
год четвертой седмины. И когда Лия увидела, что
она стала неплодною и не рождала более, то
позавидовала, и дала точно так же свою служанку
Залафу Иакову в жены; и она зачала и родила сына, и
она дала ему имя Асер в двенадцатый день восьмого
месяца в третий год четвертой седмины. И опять он
вошел к ней, и она зачала и родила ему второго
сына; и Лия дала ему имя Исашар во второй день
одиннадцатого месяца в пятый год четвертой
седмины. И Иаков вошел к Лии, и она зачала и родила
Иакову сына, и он дал ему имя Заблон в четвертый
день пятого месяца в четвертый год четвертой
седмины; и она передала его няньке. И Иаков опять
вошел к ней, и она зачала и родила двоих детей,
сына и дочь, и дала имя ему Заблон и дочери Дина в
седьмой день седьмого месяца в шестой год
четвертой седмины. И Господь умилостивился над
Рахилью и отверз утробу ее, и она зачала и родила
сына, и дала ему имя Иосиф в новолуние четвертого
месяца в шестой год этой четвертой седмины.

И когда родился Иосиф, Иаков сказал Лавану:
<<Дай мне моих жен и детей, чтобы идти мне к отцу
моему Исааку и чтобы он (?) сделал мне дом; ибо я
кончил годы, которые должен был служить тебе за
двух твоих дочерей, и я хочу идти в дом моего
отца>>. И Лаван сказал Иакову: <<Останься у
меня за вознаграждение, и паси опять у меня мои
стада, и возьми себе вознаграждение>>. И они
договорились друг с другом, чтобы он дал ему в
вознаграждение из овец и коз всех, которые [...]. И
овцы опять родили других, подобных себе, и все
были со знаком Иакова, и ни одна со знаком Лавана.
И имущество Иакова очень увеличилось. И он
приобрел себе рогатого скота, и овец, и ослов, и
верблюдов, и рабов, и служанок. И Лаван вместе с
своими сыновьями стали завидовать Иакову. И
Лаван отнял у него овец и замышлял злое против
него.

\vs Jub 29:1
И случилось, когда Рахиль родила Иосифа, пошел
Лаван стричь своих овец на расстояние трех дней
пути. И Иаков увидел, что Лаван пошел стричь своих
овец, и призвал Баллу и Рахиль, и уговаривал их
идти с ним в землю Ханаанскую; он рассказал им
именно все, что он видел во сне, и все, что Он
говорил с ним, чтобы он возвратился в дом своего
отца. И они сказали: <<Мы пойдем в то место; куда пойдешь
ты, пойдем и мы с тобою>>. И Иаков прославил
Бога отца своего Исаака и Бога деда своего
Авраама, и собрался, и посадил на верблюдов своих
жен и детей, и взял все свое имущество, и
переправился через реку, и пришел в землю
Гилеадскую. Но Иаков скрыл свой замысел от Лавана
и ничего не сказал ему об этом. В седьмой год
четвертой седмины Иаков возвратился в Гилеад, в
двадцать первый день первого месяца. И Лаван
преследовал его и настиг Иакова на горе Гилеад в
тринадцатый день третьего месяца. Но Господь не
допустил, чтобы он причинил вред Иакову; ибо Он
явился ему во сне ночью. И Лаван говорил с
Иаковом. И в пятнадцатый день того месяца сделал
Иаков Лавану и всем, которые пришли с ним,
пиршество. И Иаков поклялся Лавану в тот день, и
Лаван Иакову, что они не перейдут гору Гилеад с
злым умыслом друг против друга. И он устроил там
большой каменный холм во свидетельство; посему
дано имя тому месту~--- <<каменный холм
свидетельства>>. [...]. Прежде же звали землю
Гилеад землею Рефаил, ибо она была страною
Рефаимов, и рождались там Рефаимы-исполины,
которые были высотою до десяти, девяти, восьми,
семи локтей, и жилища которых простирались от
земли Аммонитян до горы Гермон, и главные города
которых были Хоронаим, и Астарос, и Эдрао, и Мисур,
и Беон. И Господь истребил их за нечестие их дел,
ибо они были очень мерзкими. И они оставили ее
(страну) вместо себя Аморреям, злому и греховному
народу; и нет ныне никакого другого народа, который
совершил бы все их грехи; посему они не имеют
долгой жизни на земле.

И Иаков отпустил Лавана в Месопотамию, в
восточную страну; и Иаков с своей стороны
направился в землю Гилеадскую и перешел Иаббок в
девятый месяц в одиннадцатый день его. И в этот
день пришел к нему брат его Исав, и они прекратили
свою распрю. И он ушел от него в землю Сеир, а
Иаков жил в шатрах. И в первый год в пятую седмину
в этот юбилей перешел он Иордан, и жил по ту
сторону Иордана, и пас свои стада от моря [...] до
Бефазона, и Дафаама, и Акрабита. И он посылал отцу
своему Исааку от всего своего имения одеяние, и
пищи, и мяса, и питья, и молока, и масла, и сыра, и
плодов от всяких пальм долины; и также матери своей
Ревекке он посылал четыре раза в год, между
месячными периодами, между пашней и жатвой, между
весной и дождем, между зимой и летом. И он (Исаак)
жил в башне Авраама; ибо Исаак возратился от
клятвенного колодезя и пошел в башню своего отца
Авраама, и жил здесь без (далеко от) Исава, своего
сына. К тому времени, когда Иаков отправился в
Месопотамию, Исав взял себе Маалиф, дочь Измаила,
в жены, и собрал все стада своего отца и своих жен,
и поднялся, и жил в горе Сеир, и оставил отца
своего Исаака одного при клятвенном колодезе. И
Исаак поднялся от клятвенного колодезя, и жил в
башне Авраама, отца своего, в горе Хеврон. И сюда
посылал Иаков все, что он от времени до времени
посылал своему отцу и своей матери, чтобы
облегчить всякую их скорбь. И они
благословляли Иакова от всего сердца и от всей
души.

\vs Jub 30:1
И в первый год шестой седмины поднялся он в
Салем, который находится на востоке от Сихема, с
миром, в четвертый месяц. И там увезли они Дину,
дочь Иакова, в дом Сихема, сына Емора, Гевитянина,
владетеля страны; и он спал с нею и обесчестил ее.
И она была маленькая девушка двенадцати лет. И он
просил ее отца, чтобы она была отдана ему в жены, и
у ее братьев он просил ее себе. Но Иаков и его
сыновья разгневались на сихемских мужей, которые
обесчестили их сестру Дину. И они замыслили между
собою нечто злое, и перехитрили, и обманули их. И
Симеон и Левий пришли тайно в Сихем, и совершили
наказание над всеми сихемскими мужьями, и убили
всех мужей, которых нашли в нем, и не оставили в
нем ни одного. Всех предали они мучительной
смерти, так как они обесчестили сестру их Дину.

И вы не должны отныне более делать так~---
бесчестить дочерей Израиля! Ибо на небе было
определено против них наказание, чтобы истребить
мечом всех сихемских мужей, так как они причинили
дочери Израиля бесчестие. И Господь предал их в
руки сыновей Иакова, чтобы они истребили их мечом
и совершили над ними наказание. И пусть никогда
не случится более в Израиле что-либо подобное
тому, чтобы бесчестили израильскую девицу. И если
муж в Израиле отдаст свою дочь или сестру
какому-либо мужу от семени язычников, или отдал,
то да умрет он смертию, и его должно побить
камнями, ибо он совершил грех и бесчестие
Израилю. И жену должно сожечь огнем, ибо она
осквернила имя дома своего отца, и она должна
быть истреблена из Израиля. И да не обретется
мерзость и блуд во Израиле во все роды земли, ибо
Израиль свят Господу. И каждый человек, который
совершит мерзость, должен умереть смертию и быть
побитым камнями. Ибо так утверждено это и
написано на небесных скрижалях относительно
всего семени Израиля: кто совершит мерзость, тот
должен умереть, смертию и быть побитым камнями. И
для сего закона нет конца дней, и прекращения, и
послабления, но непременно должен быть истреблен
тот муж, который осквернил свою дочь, во всем
Израиле, ибо он от своего семени дал Молоху и
совершил вину, осквернив его (семя). И ты, Моисей,
скажи сынам Израиля и положи свидетельство
против них, чтобы они не отдавали дочерей своих
язычникам и не брали дочерей языческих; ибо это
преступно пред Господом. Посему я записал тебе во
всех словах закона все деяние Сихемлян, что они
сделали с Диной, и как сыновья Иакова совещались,
говоря: <<Мы не отдадим нашу дочь (?) мужам
необрезанным, ибо это~--- поношение для нас и для
Израиля, если отдать ее или если взять из дочерей
языческих; ибо это нечисто и преступно для
Израиля, и Израиль не был бы чистым>>. И за эту
нечистоту, что один имеет жену из дочерей
языческих или что один отдает из своих дочерей
мужу из разного рода язычников, придут мучение за
мучением, и проклятие за проклятием, и все
наказания, и мучения, и проклятия. И если ты это
сделаешь, а он (народ) будет закрывать свои глаза, чтобы
не видеть тех, которые совершают мерзость, и
делают нечистым храм Господень, и оскверняют
святое имя, то весь народ должен быть наказан за
всю эту мерзость и осквернение, и не должно быть
допускаемо никакого лицеприятия и никакого
снисхождения, и не должны быть принимаемы от его
рук плоды, и плодовые жертвы, и всесожжения, и тук,
и жертвы курения в добрую воню, чтобы он
(согрешивший) был угоден. Так да будет с каждым
мужем и женщиною во Израиле, которые оскверняют
храм Его. Ради сего я дал тебе повеление, говоря:
<<Засвидетельствуй Израилю, что было
засвидетельствовано: вот как поступлено с
Сихемлянами и их сыновьями, как они были преданы
в руки двоих сыновей Иакова и были преданы
мучительной смерти. И это послужило им к правде, и
семя Левиино было избрано во священники и левиты,
чтобы они служили пред Господом, как мы, во все
дни. И да будет благословен Левий с его сыновьями
вовек, ибо они возревновали, чтобы совершить
правду, и суд, и мщение в отношении ко всем,
которые восстают против Израиля. И таким образом
отмечаются мужу в свидетельстве небесных
скрижалей благословение и справедливость пред
Ним, Богом всех вещей; и мы также будем вспоминать
правду, которую он совершил в своей жизни, во все
времена до тысячи родов. Благословение
предначертано ему, и оно придет на него~--- на него и
его род после него; и он будет записан, как друг и
праведник, на небесных скрижалях. И все это
событие я записал тебе, и дал тебе повеление,
чтобы ты открыл его сынам Израиля, дабы они не
совершали вины и не преступали закона, и не
разрушали завета, заключенного с ними, дабы они
хранили его, и были записаны друзьями. Если же они
преступят и будут ходить по всем путям нечистоты,
то будут написаны на небесных скрижалях врагами,
чтобы быть изглаженными из книги живых и
записанными в книгу тех, которые будут
уничтожены, и вместе с теми, которые будут
истреблены в стране>>. В тот день, когда сыновья
Иакова умертвили Сихемлян, было начертано им это
в книге на небе, что они совершили
справедливость, и правду, и мщение в отношении к
грешникам, и было записано им в благословение.

И они увели сестру свою Дину из дома Сихема. И
они увели в добычу все, что было в Сихеме: овец их,
и рогатый скот, и ослов, и все их имущество, и все
стада их~--- и привели все это к своему отцу Иакову.
И он совещался с ними о разрушении города; ибо они
страшились жителей страны~--- Хананеев и Ферезеев.
Но пришел страх Божий на все города вокруг
Сихемлян, так что они не решились изгнать сыновей
Иакова, ибо напало на них смущение.

\vs Jub 31:1
И в новолуние [...] месяца говорил Иаков со всеми
своими домочадцами, говоря: <<Очиститесь и
смените ваши одежды; соберемтесь и пойдемте в
Вефиль, где я дал обет в тот день, когда бежал от
моего брата Исава, ибо Он был со мною и возвратил
меня в мире в эту страну. И удалите чуждых богов,
которые у вас, и бросьте чуждых богов, и что
имеете на шеях и в ушах, и идола, которого Рахиль
украла у своего отца Лавана!>> И она (?) отдала
все Иакову и [...]. И он разбил и уничтожил их, и
оставил их под теревинфом в стране Сихемлян.

И в новолуние седьмого месяца пошел он в Вефиль
и устроил жертвенник на том месте, где он спал, и
поставил памятник. И он послал к отцу своему
Исааку, чтобы он пришел к нему к его жертве, а
также и к своей матери Ревекке. И Исаак сказал:
<<Пусть придет сын мой Иаков, чтобы я увидел его,
прежде чем умру>>. И Иаков пошел к своему отцу
Исааку и к своей матери Ревекке в дом отца его
Авраама, и взял с собою двоих из своих сыновей~---
Левия и Иуду, и пришел к отцу своему Исааку и к
своей матери Ревекке. И Ревекка вышла из башни к
нему, чтобы поцеловать и обнять Иакова; ибо ожил
дух ее, как только она услышала: <<Вот пришел сын
твой Иаков>>. И она поцеловала его, и увидела
двоих сыновей его, и узнала их, и сказала ему:
<<Это твои сыновья, сын мой?>> И она обняла их, и
поцеловала их, и благословила их, говоря: <<Да
будет чрез вас честь семени Авраама, и да будете
вы во благословение на земле!>> И Иаков вошел к
отцу своему Исааку в покой его, где он спал, и двое
сыновей его с ним. И он взял руку отца своего, и
наклонился, и поцеловал его; и Исаак пал на шею
сыну своему Иакову и плакал на шее его. И мрак сошел
с очей Исаака, и он увидел обоих сыновей Иакова
- Левия и Иуду, и сказал: <<Это твои сыновья, сын
мой? ибо они похожи на тебя>>. И он сказал ему,
что они действительно его сыновья, и
<<действительно я видел, что они истинно мои
сыновья>>. И они подошли и обернулись к нему, и
он поцеловал их и обнял их обоих вместе. И дух
пророчества нисшел в уста его. И он взял Левия за
правую руку, и обратился к Левию, и начал прежде
благословлять его, и сказал: <<Да благословит
прежде всего тебя Господь миров~--- тебя и твоих
сыновей во весь век! И да прославит Господь тебя и
твое семя великой честью; и да благоволит Он,
чтобы из всякой плоти ты и твое семя приступали к
Нему для служения Ему в Его святилище; как Ангелы
лица и как святые, так да будет семя твоих сыновей
в честь, и достоинство, и освящение! И да соделает
Он их великими во все века; и владыками, и
князьями, и начальниками да будут они над всем
семенем сыновей Иакова; да изрекают они слово
Господне с искренностью, и Его правду да
исполняют они по всей справедливости, и да
повествуют они о моих путях Иакову и об
откровении Израиля; благословение Господне да
будет вложено в уста их, дабы все семя их
благословляло тебя, возлюбленный! И мать твоя
нарекла тебе имя Левий, и справедливо она назвала
тебя так: ты будешь стоять ближе всех к Господу и
будешь иметь долю у всех детей Иакова; его стол да
будет твоим, и ты и сыновья твои должны питаться
от него; во все роды да изобилует стол твой, и твое
пропитание да не умалится никогда во все века!
Все ненавидящие тебя за что-либо погибнут пред
тобою, и все твои враги да будут истреблены и да
погибнут! Благословляющие тебя да будут
благословлены, и все люди, проклинающие тебя, да
будут прокляты!>>

И Иуде сказал он: <<Да даст тебе Господь силу и
крепость низложить всех, ненавидящих тебя! Будь
господином, ты и один из сыновей твоих, над сынами
Иакова! Твое имя и имя сыновей твоих да пойдет и
распространится по всей земле и по городам! Тогда
устрашатся язычники пред лицем твоим, и все
народы будут поражены, и все люди будут поражены.
Да придет Иакову чрез тебя помощь Его, и да
обретет Израиль чрез тебя избавление! И когда ты
будешь восседать на престоле славы, да
возвеличится справедливость твоя! Мир да будет
всему семени сыновей возлюбленного!
Благословляющий тебя да будет благословлен, и
все ненавидящие тебя, и гнетущие, и проклинающие
тебя да истребятся и погибнут от земли, и да будут
прокляты!>> И он обратился, и поцеловал его
опять, и обнял его, и очень радовался, что увидел
сыновей Иакова, который был его истинным сыном.

И он (Иаков) отошел от его лона, и пал ниц, и
склонился пред ним, и так он благословил их. И он
оставался там у своего отца Исаака в ту ночь, и
они ели и пили, исполненные радости. И он поставил
обоих сыновей Иакова, одного направо, другого
налево от себя, и это было вменено ему в
праведность. И Иаков рассказал своему отцу все в
ту ночь, как являл Господь к нему великую милость
и давал ему на всех его путях счастие и охранял
его от всякого зла. И Исаак благословил Бога отца
своего Авраама, Который Своим милосердием и
справедливостью не отступил от сына раба Своего
Исаака. И утром Иаков открыл отцу своему Исааку
обет, какой он дал Господу, и видение, какое он
видел, и сказал, что он устроил жертвенник и что
все приготовлено к жертве, чтобы совершить ее
пред Господом, как он дал обет, и что он пришел,
чтобы посадить его на осла. И Исаак сказал Иакову:
<<Я не могу идти с тобою, ибо я стар, и не могу
перенести путешествия. Иди, сын мой, с миром; ибо
мне теперь сто шестьдесят пять лет, я не могу
путешествовать. Посади на осла твою мать,
чтобы она шла с тобою. Но я знаю, сын мой, что ты
пришел ради меня; и да будет благословен этот
день, в который ты увидел меня живым и я тебя, сын
мой! Будь счастлив и исполни обет, который ты дал,
и не откладывай своего обета, (ибо это радостный
обет). И теперь поспеши исполнить его, и да примет
его Сотворивший все, Которому ты дал обет!>> И он
сказал Ревекке: <<Иди ты с своим сыном
Иаковом!>> И Ревекка пошла с Иаковом и Девора с
ней; и они пришли в Вефиль.

И Иаков вспомнил о молитве, которою отец
благословил его и двоих его сыновей~--- Левия и
Иуду, и возрадовался, и прославил Бога отцов
своих~--- Авраама и Исаака,~--- и сказал: <<Теперь я
знаю, что у меня есть вечная надежда~--- у меня и
моих сыновей~--- пред Богом всех вещей; и это
определено относительно них обоих, и начертано
это для них в вечное свидетельство па небесных
скрижалях~--- так, как благословил Исаак>>.

\vs Jub 32:1
И он оставался в ту ночь в Вефиле. И Левий видел
во сне, что он и его сыновья поставлены и
определены навек ко священству для Бога
всевышнего. И он пробудился от сна и прославил
Бога. И Иаков собрался рано утром в четырнадцатый
день этого месяца и дал десятину от всего, что
прибыло с ним, от человека до скота, от золота до
сосудов и одежд, и дал десятину от всего. И в те
дни Рахиль была беременною Вениамином, своим
сыном; и Иаков исчислил, начиная с него, своих
сыновей; и жребий Господа пал на Левия. Тогда он
одел его в священнические одежды и наполнил руки
его. И в пятнадцатый день этого месяца он принес
на жертвеннике четырнадцать тельцов из рогатого
скота, и двадцать восемь овнов, и сорок девять
овец, и шестьдесят агнцев, и двадцать девять
молодых козлят, как всесожжение на жертвеннике и
как благоприятный дар в добрую воню пред
Господом. Это была дань его ради обета, который он
дал~--- отделить десятину~--- вместе с плодовыми
жертвами и жертвами возлияния, которые
относились сюда. И когда огонь пожрал их, он
воскурил над ними фимиам на огне; и в жертву
благодарения он принес двух тельцов, и
четырех овнов, и двух годовых ягнят, и десять
телят, и четырех овец и двух молодых козлят. Так
делал он, давая свою дань, в продолжение семи
дней. И он ел там со всеми своими сыновьями и
людьми в радости в течение семи дней, и прославил
и благодарил при сем Господа, Который спас его от
всякого зла и исполнил на нем Свое обетование. И
он отделил десятину от всего чистого скота и
совершил всесожжение. А нечистый скот он отдал
сыну своему Левию, и людей отдал он ему. И Левий
исполнил в Вефиле священнические обязанности
пред своим отцом Иаковом, будучи предпочтен
своим десяти братьям, и был там священником. И
Иаков отдал ему свой обет. И таким образом он
отдал вторую десятину Господу и посвятил ее, и
она стала посвященною ему. И посему определено
это как закон на небе~--- давать вторую десятину,
чтобы есть ее пред Господом в том месте, которое
избрано, чтобы имя Его обитало там, во все годы. И
для сего закона нет конца дней; навечно записано
то постановление, чтобы делать это ежегодно, именно~---
есть вторую десятину пред Господом в том
месте, которое избрано. И ничего не должно быть
оставляемо от нее на следующий год, но в том же
году должно быть съедаемо семя до следующего
года, от дней начатков года, семени, и вина, и
масла, опять до этих же дней. И все, что останется
от нее и сделается устаревшим, должно считать
оскверненным и сожечь огнем, ибо это нечистое. И
так они должны вместе есть ее во святом доме и не
давать ей залеживаться. И все десятины от
рогатого скота и овец суть святы Господу, и Его
священникам должны принадлежать они, чтобы они
ели пред Ним из года в год. Ибо так это определено
и начертано на небесных скрижалях относительно
десятины.

И в следующую ночь, в двадцать второй день этого
месяца, Иаков решил обстроить то место, и обнести
место стенами, и посвятить, и сделать его святым
навечно для себя и своих детей после себя до века.
И Господь явился ему ночью, и благословил его, и
сказал ему: <<Твое имя не должно быть только
Иаков, но должно быть наречено имя тебе
Израиль>>. И Он опять сказал ему: <<Я Господь
Бог твой, сотворивший небо и землю. Я возращу
тебя, и весьма умножу тебя, и цари произойдут от
тебя, и будут они господствовать всюду, где
только ступит нога сынов человеческих. И Я дам
твоему семени всю землю, которая под небом, и они
будут по своей воле господствовать над всеми
народами; и после этого они завладеют всею землею
и наследуют ее навеки>>. И Он окончил Свою
беседу с ним и поднялся от него. И Иаков видел, как
Он вознесся на небо; и он видел ночью в видении, и
вот Ангел сошел с неба с семью скрижалями в своих
руках, и он дал их Иакову, и он читал их и прочитал
все, что было написано на них, что случится с ним и
с его сыновьями во все века. И он показал ему все,
что было написано на скрижалях, и сказал ему:
<<Ты не должен строить на этом месте и делать
святыню навечно, и Он не хочет обитать здесь, ибо
не это Его место. Иди в дом Авраама, отца
твоего, и живи в доме отца твоего Исаака до дня
смерти твоего отца. Ибо в Египте ты умрешь в мире,
и будешь погребен в этой стране с честью в гробах
твоих отцов с Авраамом и Исааком. Не бойся! ибо
как ты видел и прочитал, так все и случится. И
запиши все, как ты видел это и прочитал>>. И
Иаков сказал: <<Как я упомню все так, как видел
это и прочитал?>> И он сказал ему: <<Я опять
приведу тебе все на память>>. И он поднялся от
него.

И он пробудился от сна своего, и вспомнил все,
что читал и видел, и записал всю речь, которую
читал и видел. И он остался там еще на один день и
принес в этот день жертву совершенно так же, как в
прежние дни, и назвал его~--- <<прибавление>>. Ибо
тот день прибавлен. И прежние дни он назвал
праздником. И так ему было открыто, что должно
случиться, и это написано на небесных скрижалях.
И ради того было ему это открыто, чтобы он хранил
его, и прибавлял его таким образом к семи
праздничным дням. И он назван был прибавлением,
как заканчивающий в мире праздничные дни по
числу дней года.

И в ночь на двадцать третий день этого месяца
умерла Девора, нянька Ревекки, и они похоронили
ее внизу города под дубом реки, и он нарек имя той
реке~--- <<река Деворы>> и дубу~--- <<дуб плача
Деворы>>. И Ревекка пошла и возвратилась в дом к
его отцу Исааку. И Иаков послал ему чрез нее
барана, и телят, и овец, чтобы она приготовила его
отцу кушанье, как он любил. И после отправления своей
матери он пошел дальше, пока не пришел в страну
Кебрафан, и жил там. И Рахиль родила ночью сына и
назвала его: <<мой сын болезни>>, ибо она имела
тяжелые роды. А отец его назвал его Вениамином в
десятый день восьмого месяца в первый год шестой
седмины этого юбилея. И Рахиль умерла там и была
погребена в стране Ефрафе, т.е. Вифлееме. И Иаков
устроил на могиле Рахили памятник при дороге, над
ее могилою.

\vs Jub 33:1
И Иаков пошел дальше и жил к северу в Магд-Ладре
Еф(рафа). И он пошел к своему отцу Исааку, он и его
жена Лия, в новолуние десятого месяца, и Робел
увидел Баллу, служанку Рахили, наложницу своего
отца, когда она купалась в воде в уединенном
месте, возымел любовь к ней, и спрятался ночью, и
вошел в жилище Баллы, и нашел ее одну ночью
лежащей на своей постели и спящей в своем жилище.
И он лег к ней на ложе, и открыл покрывало ее; и она
схватила его и вскрикнула. И когда она узнала его,
что это был Робел, застыдилась его, и отняла свою
руку от него, и убежала, и очень скорбела о
случившемся, но не сказала ничего ни одному
человеку. И когда Иаков пришел и отыскивал ее, она
сказала ему: <<Я не чиста для тебя, но обесчещена
для тебя, ибо Робел обесчестил меня, и лег со мною
ночью, когда я спала у себя, и я не узнала его, пока
он не открыл моего покрывала, и он спал со
мною>>. И Иаков сильно разгневался на Робела,
что он спал с Валлою и открыл покров своего
отца. И Иаков не приближался более к ней, так как
Робел обесчестил ее, и пред всеми людьми открыл
покров своего отца. Ибо поступок его был очень
нехорош; это постыдно пред Господом.

Посему написано и определено на небесных
скрижалях, что муж не должен лежать с женою
своего отца и открывать покров своего отца, ибо
это мерзость. Смертию должен умереть преступный
муж, который ляжет с женою своего отца, а также и
жена: ибо они мерзость совершили на земле. И пред
нашим Господом не должно быть ничего мерзкого в
народе, который Он избрал Себе в царское
достояние. И еще написано: да будет проклят, если
кто лежит с женою своего отца, за то, что он
открывает срамоту своего отца. И все святые
Господа пусть скажут: <<Аминь, аминь!>> И ты,
Моисей, скажи сынам Израиля, чтобы они соблюдали
сие слово, ибо за него угрожает наказание
смертию, и это мерзость, и нет за это прощения,
чтобы можно было искупить мужа, который совершит
сие зло, кроме наказания смертию, и умерщвления, и
побиения камнями, и истребления из народа нашего
Бога. Ни одного дня не должен жить на земле муж,
который совершит это во Израиле, ибо это
преступно и мерзко. И не должно говорить, что
Робел остался в живых и получил прощение, хотя он
лежал с наложницей своего отца, в то время как муж
ее, отец его Иаков, еще был жив. Ибо Он тогда же
вполне открыл всем постановление, и правду, и
закон, который существует вовек. Но во все дни
твои он должен иметь силу закона, с его дней и
есть вечный закон для вечных родов. И этот закон
не прекратится, и никакое прощение не будет
уделом такового, кроме того, что оба вместе будут
истреблены из народа; в тот день, когда они
совершили это, должно умертвить их. И ты, Моисей,
напиши это для Израиля, чтобы они соблюдали сие и
поступали по сему слову и не совершали смертного
греха, ибо Господь Бог наш есть судия
нелицеприятный и неподкупный. И скажи им это
постановление, чтобы они слушались, и
оберегались, и внимали сему, и не погибли бы, и не
были бы истреблены на земле. Ибо нечисты, и
мерзки, и преступны, и скверны все они,
совершающие сие на земле пред нашим Господом. И
нет большего греха на земле, как любодеяние,
которым они любодействуют; ибо Израиль есть
народ, святый Господу, и народ наследия для
своего Бога, и народ священства и царства, и
достояние Божие. И никто не должен
существовать, кто является столь нечистым среди
святого народа.

И в третий год этой шестой седмины вышел Иаков и
все его сыновья, и жили в доме Авраама у своего
отца Исаака и своей матери Ревекки. И вот имена
сыновей Иакова: его первенец Робел, Симеон, Левий,
Иуда, Исашар, Завулон~--- сыновья Лии; и сыновья
Рахили: Иосиф и Вениамин; и сыновья Баллы: Дан и
Наффали; и сыновья Залафы: Гад и Асер; и Дина, дочь
Лии, единственная дочь Иакова. И они пошли и
поклонились Исааку и Ревекке. И когда последние
увидели их, благословили Иакова и всех его детей.
И Исаак очень обрадовался, что увидел сыновей
Иакова, своего младшего сына, и благословил их.

\vs Jub 34:1
И в шестой год этой седмины сорок четвертого
юбилея Иаков отослал своих сыновей~--- пасти его
овец~--- и своих рабов с ними на поля Сихемские. И
собрались против них семь царей аморрейских,
чтобы умертвить их, укрывшись под деревьями, и
увести их скот. Но жены их, и Иаков, и Левий, и Иуда,
и Иосиф оставались дома у своего отца Исаака, ибо
дух его был прискорбен, и он не хотел отпустить
их; Вениамин, как юнейший, оставался с своим
отцом. И пришли цари Фафы и Арезы, и Сарагана, и
Село, и Гаиза, и царь Бефорона, и Маанизакира, и
все, живущие на тех горах и обитающие в лесах
страны Ханаанской. И известили Иакова: <<Цари
аморрейские окружили твоих сыновей и похитили их
стада>>. И отправился из своего дома он, и его
три сына, и все рабы его отца, и его рабы, и вышли
против них в числе восьмисот мужей, носивших
мечи; и они поразили их на поле Сихемском, и
преследовали бегущих, и убили Арезу, и Фафу, и
Сарагана, и Аманискино, и Гаганиса. И он собрал
свои стада, и был могущественнее их, и наложил на
них дань, чтобы они давали плоды своей страны. И
они построили Робел и Фамнафарес. И он
возвратился благополучно, и заключил с ним мир, и
они сделались его рабами, пока он не ушел с своими
сыновьями в Египет.

И в седьмой год этой седмины послал он Иосифа,
чтобы он осведомился о состоянии своих братьев,
из своего дома в Сихем. И он нашел их в стране
Дуфаим. И они подстерегали его, и сделали против
него умысел убить его. И когда они изменили свое
намерение, то продали его измаильским
странствующим купцам. И они отвели его в Египет и
продали его Питфаре, евнуху Фараона, главному
повару, жрецу Гелиопольскому. И сыновья Иакова
закололи козленка, и омочили одежду Иосифа в
его крови, и послали ее Иакову в десятый день
седьмого месяца. [...]. И они принесли ее ему, и он
занемог болезнию от печали о его смерти. И он
сказал: <<Дикий зверь пожрал Иосифа>>. И все
его домочадцы были при нем в этот день; и его
сыновья, и его дочери собрались утешать его; но он
оставался безутешным о своем сыне. И в тот день
услышала Балла, что Иосиф потерян, и умерла от
печали по нем, в то время как она была в Караффифе.
И дочь его Дина также умерла, после того как Иосиф
был потерян. Эта тройная скорбь пришла на Израиля
в один месяц. И они похоронили Баллу напротив
могилы Рахили, а также и дочь его Дину похоронили
там. И он скорбел об Иосифе год и не переставал
печалиться; ибо он сказал: <<Я сойду в могилу,
печалясь об Иосифе>>. Ради сего определено
между сынами Израиля, чтобы скорбеть в десятый
день седьмого месяца, в тот день, когда пришло
печальное известие об Иосифе к его отцу Иакову,
чтобы испрашивать в оный день прощение чрез
козла, в десятый день седьмого месяца, один раз в
год, в своих грехах; ибо они превратили любовь
своего отца к его сыну Иосифу в печаль о нем. И
этот день установлен, чтобы они в течение его
скорбели о своих грехах, и о всякой своей вине, и о
своем проступке, дабы очищаться в этот день
однажды в год.

И после того как Иосиф был потерян, сыновья
Иакова взяли себе жен. 1)Имя жены Робела~--- Ада;
2)жены Симеона~--- Адиба, хананеянка;
3)жены Левия~--- Мелха, из дочерей Аррама, из семени сыновей
Фарана; 4)жены Иуды~--- Бефазуел, хананеянка; 5)жены
Исашара~--- Гизека; 6)жены Дана~--- Эгла; 7)жены
Завулона~--- Нииман; 8)жены Наффалима~--- Разуу из
Месопотамии; 9)жены Гада~--- Михи; 10)жены Асера~---
Ийона; 11)жены Иосифа~--- Асанеф, египтянка; 12)жены
Витамина~--- Ийоска. И Симеон изменил намерение, и
взял вторую жену из Месопотамии, как и его братья.

\vs Jub 35:1
И в первый год первой седмины сорок пятого
юбилея призвала Ревекка сына своего Иакова и
дала ему повеление относительно его отца и брата,
чтобы он почитал их во все дни жизни своей. Он
сказал: <<Я буду поступать так, как ты повелела
мне, ибо это будет для меня честью, и
достоинством, и праведностью пред Господом, что я
почитаю их. Ты же знаешь от дня моего рождения до
сего дня каждое мое деяние и всякое мое
помышление, что я всегда благожелаю всем. Как же
мне не исполнять того, что ты заповедала мне,~--- именно
почитать моего отца и брата? Скажи мне, мать
моя, какое зло ты заметила во мне? Я же и далек от
него (от Исава), и между нами существует
доброе согласие>>. И она сказала ему: <<Сын мой,
в продолжение всей своей жизни я не видела в тебе
ничего предосудительного, а только
справедливое. Я говорю тебе, сын мой: в этом году я
кончу свою жизнь. Ибо я видела во сне день
моей смерти, что я не проживу более ста
пятидесяти лет. И вот я кончила все дни своей
жизни, которые надлежало мне прожить>>. И Иаков
усмехнулся над словами своей матери, что мать
сказала ему, будто она умрет, между тем как она
сидела против него в полной силе, не ослабевшая;
ибо она входила и выходила, и видела, и зубы ее
были здоровы, и никакая болезнь не коснулась ее в
течение всей ее жизни. И Иаков сказал ей: <<Я
буду счастлив, мать моя, если моя жизнь
сравняется по продолжительности с твоей жизнью и
если я так же сохранюсь в полной своей силе, как
ты. Ты не умрешь, и напрасно говоришь со мною о
своей смерти>>.

И она вошла к Исааку и сказала ему: <<Я имею к
тебе просьбу: заставь поклясться Исава, что он не
причинит обиды Иакову и никогда не будет
преследовать его. Ибо ты знаешь нрав Исава, что он
груб от юности, и нет в нем добродушия; ибо он
замышляет после твоей смерти убить его. И ты
знаешь все, что совершил он во все дни от того дня,
когда брат его Иаков пошел в Харран, до сего дня;
как он оставил нас всем своим сердцем и сделал
нам злое; как он присвоил себе твои стада и все
достояние твое похитил пред лицем твоим. И когда
мы умоляли и просили о нашем достоянии, он
действовал подобно человеку, как бы оказывающему
нам свою милость. И он гневается на тебя, ибо ты
благословил своего благочестивого и праведного
сына Иакова; ибо в нем нет ничего злого, но одно
только доброе. И с того времени, как он
возвратился из Харрана, до сего дня он не обидел
нас ни в малейшем; но мы все получаем от него вовремя
и всегда; и он радуется от всего сердца, если мы
что-нибудь принимаем от него, и благословляет
нас; и он не отделился от нас с того времени, как
пришел из Харрана, до сего дня. И он живет всегда с
нами в доме, почитая нас>>. И Исаак сказал ей:
<<Знаю и я, и вижу отношение Иакова к нам, что он
нас почитает во всем. Я раньше любил более Исава,
чем Иакова, ибо он родился прежде; но теперь я
люблю Иакова больше, чем Исава, так как он
оказался в своих делах весьма дурным и в нем нет
никакой справедливости. Ибо все пути его~---
несправедливость и насилие, и нет в нем
справедливости. Мое сердце также потрясено
теперь из-за всех его дел, и ему и семени его не
будет счастия, но они погибнут на земле и будут
истреблены под небом. Ибо он оставил Бога Авраама
и последовал за своими женами, за мерзостию и
соблазном их~--- он и его сыновья. И ты говоришь мне,
чтобы я заставил его поклясться, что он не убьет
Иакова; но если он и поклялся бы, то это будет
бесполезно, и он будет совершать не добро, а
только зло. И если он захочет умертвить своего
брата Иакова, то будет предан в руки Иакова, и не
избегнет рук его, но впадет в руки его. И ты не
бойся за Иакова: ибо хранитель Иакова~---
могущественный, и досточтимый, и
достопоклоняемый всеми>>. [...]

И Ревекка послала и призвала Исава; и он пришел
к ней. И она сказала ему: <<У меня есть просьба к
тебе, сын мой, и ты обещай, что исполнишь то, что я
скажу тебе, сын мой!>> И он сказал: <<Я сделаю
все, что ты скажешь мне, и не откажу в твоей
просьбе>>. И она сказала ему: <<Я прошу тебя,
чтобы ты, когда я умру, перенес меня и похоронил с
Сарой, матерью отца твоего, и чтобы вы любили друг
друга, ты и брат твой Иаков, и никто не
предпринимал бы никакого зла против своего
брата, а оказывал бы только взаимную любовь, дабы
вы были счастливы, сыновья мои, и были почитаемы
на земле, и никакой враг не восторжествовал бы
над вами, и вы были бы достойными милосердия пред
очами тех, которые любят вас>>. И он сказал: <<Я
все исполню, что ты сказала мне, и похороню тебя,
когда ты умрешь, вместе с Сарой, матерью отца
моего, так как ты любишь кости ее, чтобы они были с
твоими костями. И также брата моего Иакова я буду
любить больше, чем всякую плоть; у меня на всей
земле нет брата, кроме его одного; и нет ничего
великого (трудного) для меня в том, чтобы любить
его, ибо он брат мой, и мы вместе были посеяны в
твоем чреве и вместе вышли из твоих недр. И если
не любить мне своего брата, то кого же мне любить?
И я также прошу тебя, чтобы ты сделала увещание
Иакову относительно меня и моих детей, так как я
знаю, что он как царь будет господствовать надо
мною и над моими сыновьями. Ибо в тот день, когда
мой отец благословил его, он сделал его высшим, а
меня подчиненным. И я клянусь тебе, что я буду
любить его и ничего злого не замыслю против него
в продолжение всей моей жизни, а только
доброе>>. И он подтвердил клятвою все эти слова.
И она призвала Иакова пред очи Исава и дала Иакову
повеление согласно беседе, какую она вела с
Исавом, и он сказал: <<Я исполню твою волю,
ручаясь за то, что от меня и моих сыновей не
выйдет ничего злого против моего брата Исава, и
только лишь одну любовь встретит он>>. И они ели
и пили, она и ее сыновья, в эту ночь. И она умерла,
трех юбилеев одной седмины и одного года, в эту
ночь. И оба ее сына Исав и Иаков похоронили ее в
пещере около Сары, матери их отца.

\vs Jub 36:1
И в шестой год этой седмины призвал Исаак обоих
своих сыновей~--- Исава и Иакова, и они пришли к
нему, и он сказал им: <<Сыны мои, я иду по пути
моих отцов в вечное жилище, где отцы мои.
Похороните меня с моим отцом Авраамом в двойной
пещере на полях Эфрона Хеттеянина, которые
Авраам купил для могилы; там похороните меня! И я
заповедую вам, сыны мои, совершать на земле
справедливость и правду, чтобы Господь послал
вам все, что обещал сделать Аврааму и семени его.
И любите друг друга, как братья, сыны мои, так, как
каждый любит самого себя, и стараясь сделать
лучшее для другого, действуя единодушно на земле
и каждый любя другого, как самого себя. И
относительно идолов я заповедую вам, чтобы вы
отвергали их, и ненавидели, и не любили их; ибо они
исполнены соблазна для тех, которые почитают их,
и для тех, которые поклоняются им. Памятуйте, сыны
мои, о Господе, Боге Авраама, отца вашего, как и я
после него почитал Его и служил Ему воистину,
дабы Он умножил вас в радости и возрастил семя
ваше~--- умножил вас в числе, как звезды небесные, и
насадал бы на земле вас и всякое растение правды,
которое не будет истреблено во все роды века. И
ныне я заклинаю вас великою клятвою~--- ибо нет
большей клятвы, как клятва славным, и честнейшим,
и великим именем Того, Кто сотворил небо и землю и
все в совокупности,~--- чтобы вы страшились Его и
почитали и чтобы каждый любил своего брата нежно
и искренно, и не желал бы своему брату зла отныне
до века, во все дни вашей жизни, дабы вы были
счастливы во всех своих делах и не погибли. И если
кто из вас предпримет что-либо злое против своего
брата, то знайте отныне, что всякий, замышляющий
что-либо злое против своего брата, падет от его
руки и будет истреблен из страны живых, и семя его
также погибнет под небом. И в день проклятия и
власти Он сожжет пылающим и поедающим огнем и
его страну, и город, и все принадлежащее ему,
подобно тому как Он сожег Содом; и он будет
изглажен из книги наставления сынов
человеческих и не будет записан в книге жизни. Но
он погибнет и подпадет вечному осуждению, чтобы
их наказание беспрерывно возобновлялось чрез
ненависть, и проклятие, и гнев, и мучение, и злобу,
и муки, и болезнь, вовек. Я говорю и возвещаю вам,
сыны мои, суд, как он придет на мужа, который
захочет сделать что-нибудь дурное против своего
брата>>.

И он разделил все свое имущество между ними
обоими в тот день. И он дал преимущество тому, кто
был рожден прежде, и отдал ему башню, и все
кругом ее, и все, что Авраам приобрел у
клятвенного колодезя. И он сказал:
<<Преимущество это должен иметь тот, кто рожден
прежде>>. И Исав сказал: <<Я продал и передал
свое старшинство Иакову; пусть будет отдано это
Иакову! И я не буду более говорить ему об этом, ибо
так случилось это>>. И Исаак сказал: <<Да
покоится, сыны мои, благословение на вас и на
вашем семени в этот день, что вы остались
спокойными и не огорчили меня из-за старшинства,
что вы не допускаете ничего постыдного из-за
него! Господь, Всевышний, да благословит того
мужа, который совершает справедливость, его и
семя его вовек!>> И он перестал давать заповеди
и благословлять их. И они ели и пили вместе пред
ним, и он радовался, что между ними совершилось
примирение. И они вышли от него, и отдыхали в тот
день, и спали.

И Исаак почил в тот день на своем ложе, полный
радости, и почил вечным сном, и умер ста
восьмидесяти лет, окончив двадцать пять седмин и
пять лет. И оба сына его, Исав и Иаков, похоронили
его. И Исав пошел в страну Едом, на горе Сеир, и
оставался там. И Иаков жил на горе Хеврон в башне
страны странствования отца своего Авраама; и он
почитал Господа от всего сердца и по Его заповеди
[...].

И жена его Лия умерла в четвертый год второй
седмины сорок пятого юбилея; и он похоронил ее в
двойной пещере возле своей матери Ревекки,
налево от могилы Сары, матери отца его. И все ее и
его сыновья пришли оплакивать вместе с ним Лию,
жену его, и утешать его в скорби по ней. Ибо он
скорбел об ней, так как любил ее еще более после
того, как умерла сестра ее Рахиль. Ибо она была
благочестива и праведна во всех путях своих и
почитала Иакова. И в течение всего времени, как
она жила с ним, он не слышал из уст ее никакого
грубого слова; ибо она была кротка, и миролюбива,
и праведна, и досточтима. И он вспомнил ее дела,
какие она делала во время своей жизни, и очень
оплакивал ее, ибо он чрезвычайно любил ее от
всего сердца и от всей души.

\vs Jub 37:1
И когда Исаак, отец Иакова и Исава, умер,
услышали сыновья Исава, что Исаак отдал
первенство своему младшему сыну Иакову, и
разгневались чрезмерно, и препирались с своим
отцом, говоря: <<Почему, когда ты старший, а
Иаков~--- младший, твой отец отдал первенство
Иакову, и тебя поставил ниже?>> И он сказал им:
<<Потому что я свое первородство продал за
немногое~--- за чечевичное кушанье. И в тот день,
когда мой отец послал меня на охоту~--- наловить
чего-нибудь и принести к нему, чтобы он ел и
благословил меня, пришел он (Иаков) хитростью и
принес моему отцу есть и пить, и мой отец
благословил его, а меня отдал в его руки. И вот
отец наш заставил нас поклясться, меня и его, что
мы ничего злого не замыслим друг против друга, и
будем жить друг с другом в любви и мире, и не
извратим наших путей>>. И они сказали ему: <<Мы
не послушаемся тебя в том, чтобы поддерживать с
ним мир, ибо мы сильнее, нежели он, и мы преодолеем
его. Мы выйдем против него, и умертвим его, и
истребим его сыновей. И если ты не пойдешь с нами,
мы причиним зло и тебе. Послушай же нас теперь: в
Араме, и Филистее, и Моаве, и Аммоне мы наберем
себе отборных людей, которые способны к войне, и
пойдем против него, и сразимся с ним, и истребим
его в стране, прежде чем он приобретет силу>>. И
отец их сказал им: <<Не ходите, и не начинайте с
ним войны, дабы вам не пасть от него>>. И они
сказали ему: <<Неужели тебе от юности и до сего
дня только и делать, чтобы склонять свою выю под
его ярмо? Мы не послушаемся сих слов>>. И они
послали в Арам и к Адураму, другу своего отца, и
наняли себе у них тысячу способных к войне мужей
и отборных воинов. И пришли к ним от Моава и от
сынов Аммона нанятых тысяча отборных воинов, и от
филистимлян тысяча отборных воинов, и от Эдома и
хореев тысяча отборных ратников, и от хетитов
сильные, способные к войне мужи. И они сказали
своему отцу: <<Выходи, веди нас! а иначе мы убьем
тебя>>. И он разгневался и пришел в ярость, когда
увидел, как сыновья употребляли в отношении к
нему насилие, чтобы он был предводителем их и вел
их против своего брата Иакова.

После сего ему вспомнилось все то зло, которое
лежало сокрытым внутри его против его брата
Иакова, и он не вспомнил о клятве, которую он дал
своему отцу и своей матери, что он не предпримет
ничего злого против своего брата Иакова во всю
свою жизнь.

И в продолжение всего этого времени Иаков
ничего не знал о том, что они выступают против
него войною,~--- он сильно скорбел о своей жене Лии,~---
пока они не подошли к башне против него~--- четыре
тысячи способных к войне, сильных, воинственных,
отборных мужей. И жители Хеврона послали к нему
сказать: <<Вот брат твой пришел на тебя, чтобы
победить тебя, с четырьмя тысячами мужей,
препоясанных мечами и носящих щит и оружие>>.
Они любили Иакова более, чем Исава, поэтому и
сказали ему это; ибо Иаков был муж милостивый и
более любвеобильный, чем Исав. И Иаков не поверил
этому, пока они не приблизились к самой башне. И
он взошел на башню, и говорил с своим братом
Исавом, и сказал: <<Приносишь ли ты мне доброе
утешение? Пришел ли ты ко мне ради моей умершей
жены? Это ли клятва, которою ты дважды поклялся
твоим родителям пред их смертию? Ты нарушил
клятву, и тем, чем ты поклялся своему отцу, ты
осужден>>. Тогда Исав отвечал и сказал ему:
<<Никогда не клянутся между сынами
человеческими и между зверями земли истинною
клятвою до века; но в тот самый день они уже
замышляют злое друг против друга, и враг ищет
убить своего врага. И ты также ненавидишь меня и
моих сыновей до века, и с тобою нельзя сохранять братской
любви. Слушай эти слова мои, которые я скажу
тебе. Если бы я мог изменить кожу и щетину свиньи,
чтобы она (щетина) стала шерстью, и если бы на ее
голове выросли рога, подобно рогам овец, тогда я
поддерживал бы с тобою братскую любовь. И если
грудь у матери отделится~--- ибо ты отселе мне не
брат,~--- и если волки заключат мир с ягнятами, что
они не будут пожирать и похищать их, и если сердце
их склонится к тому, чтобы делать друг другу
добро, тогда я буду иметь в своем сердце мир с
тобою. И если лев сделается другом вола, и будет
запрягаться с ним в одно ярмо, и будет пахать с
ним, тогда я заключу мир с тобою. И если вороны
сделаются белыми, как рис, тогда я узнаю, что я
люблю тебя и храню мир с тобою. Ты должен быть
истреблен, и сыновья твои должны быть истреблены,
и да не будет с тобою мира!>> И Иаков увидел в тот
час, что он замыслил против него злое [...], чтобы
убить его, и что он пришел, стремясь как дикий
зверь, бросающийся на копье, которое пронзает и
убивает его самого, и он не отступает от него.
Тогда он сказал домочадцам и своим рабам, чтобы
они напали на него~--- на него и на всех его
соучастников.

\vs Jub 38:1
И после сего Иуда говорил со своим отцом
Иаковом и сказал ему: <<Отец! Натяни лук свой, и
пусти стрелу свою, и порази злодея, и убей врага.
Да будет у тебя сила на это, ибо мы не хотим
убивать твоего брата!>> [...]. И Иаков тотчас
натянул лук свой, и пустил он стрелу свою, и
поразил брата своего Исава, и убил его. И еще
пустил он стрелу свою и попал в арамеянина Адрона
в его левый грудной сосок, и обратил его в
бегство, и убил его. После сего сыновья Иакова
выступили со своими рабами и распределились на
четырех сторонах башни. Вперед вышел Иуда с
Наффали, и Гадом, и пятьюдесятью рабами на
северной стороне башни, и они умертвили все, что
было пред ними, и никто не спасся от них, даже ни
один. И Левий, и Дан, и Асер выступили на восточной
башне с пятьюдесятью мужами и убили ратников
моавитян и аммонитян. И Робел с Исашаром и
Завулоном выступили на южной стороне башни с
пятьюдесятью мужами и убили воинов филистимлян.
И Симеон, и Вениамин, и Енох, сын Робела, выступили
на западной стороне башни с пятьюдесятью мужами
и перебили из едомитян и хореев (триста) сильных
воинственных мужей; и семьсот убежали. И четыре
сына Исава бежали с ними, и оставили отца своего
убитого, как он пал на холме, который находится в
Адураме. И сыновья Иакова преследовали их до горы
Сеир; а Иаков похоронил своего брата на холме,
который находится в Адураме, и возвратился в свой
дом. И сыновья Иакова стеснили сыновей Исава на
горе Сеир, и согнули их выю, так что они стали
рабами сыновей Иакова. И они послали к своему
отцу спросить, заключить ли мир с ними или
умертвить их. И Иаков велел сказать своим
сыновьям, чтобы они заключили мир. И они
заключили мир с ними и наложили на них ярмо
рабства, чтобы они платили Иакову и его сыновьям
дань всякий год. И они, не переставая, платили
Иакову дань до того дня, когда он ушел в Египет
[...].

И вот цари, которые владычествовали над Едомом,
- прежде чем стал владычествовать царь над сынами
Израиля,~--- до сего дня в стране Едом. И был царем в
Едоме Балак, сын Беора, и имя его города было
Динаба. И Балак умер, и вместо него стал царем
Иобаб, сын Зары из Базуры. И вместо него стал
царем Адафа, сын Барада, который поразил
Мидианитян на поле Моав; и имя его города Авуф. И
Адафа умер, и вместо него стал царем Салман из
Амелека. И Салман умер, и вместо него стал царем
Суал из Робаофа при реке. И Суал умер, и вместо
него стал царем Беулуман, сын Акбура. И Беулуман,
сын Акбура, умер, и вместо него стал царем Адафа, и
имя жены его было Майя-Тобиф, дочь Матрифы, дочери
Мимифбид-Цаобы. Вот цари, которые управляли в
стране Едом.

\vs Jub 39:1
И Иаков жил в земле странствования отца своего,
в стране Ханаанской. Вот роды Иакова. Иосиф был
семнадцати лет, когда они отвели его в Египет, и
Питфаран, евнух Фараона, главный повар, купил его.
И он поставил Иосифа над всем своим домом. И
благословение Господа было над домом египтянина
ради Иосифа, и во всем, что он делал, Господь давал
ему успех. И египтянин предоставил Иосифу все,
что было у него, ибо видел, что Господь был с
ним, и во всем, что он делал, давал ему успех. Иосиф
же был красив и весьма миловиден лицом. И жена
господина его обратила на него свои взоры, и
увидела Иосифа, и почувствовала любовь к нему, и
просила его, чтобы он лег с нею. Но он не предал ей
свою душу, и вспомнил о Господе и о словах,
которые отец его Иаков читал в словах Авраама,
что никто не должен прелюбодействовать с женою,
имеющей мужа, и что для такового определено
наказание смертию на небесах пред Господом
всевышним, и что грех будет записан за ним в
книгах, которые до века всегда существуют пред
Господом. И Иосиф вспомнил эти слова, и не хотел
лечь с нею. И она просила его в течение года, но он
отказывал ей, и не хотел слушаться ее. Но она
обняла его и схватила его в доме, чтобы принудить
его лечь с нею, и заперла двери дома. Но он
вырвался из рук ее, и оставил в руках ее свою
одежду, и разломал запор, и выбежал от нее. И когда
та жена увидела, что он не хочет лечь с нею,
очернила его пред своим господином, говоря:
<<Твой еврейский раб, которого ты любишь, хотел
причинить мне насилие, чтобы лечь со мною; но
когда я возвысила голос свой, он убежал, и оставил
свою одежду в моих руках, как только я схватила
его, и разломал запор>>. И египтянин увидел
одежду Иосифа и также запор, который был
разломан; и послушался слов жены своей, и посадил
Иосифа в темницу в одно место, где сидели
заключенные, которых царь велел заключить в
темницу. И он оставался там в темнице. И Господь
дал Иосифу милость в глазах главного темничного
стража и милосердие в глазах его. Ибо он видел,
что Господь был с ним и во всем, что он делал,
давал ему успех. И он передал ему все, и главный
темничный страж не смотрел ни за чем; ибо все, что
делал Иосиф, совершал Господь. И он оставался там
два года.

И в те дни Фараон, царь египетский, разгневался
на двух своих евнухов, на главного кравчего и на
главного хлебника, и посадил их в темницу в доме
главного повара~--- в темницу, где был заключен
Иосиф. И главный темничный страж приказал Иосифу,
чтобы он служил им; и он служил им. И они оба
видели сон~--- кравчий и хлебник, и рассказали его
Иосифу. И как он истолковал его им, так с ними и
случилось. Главного кравчего Фараон опять
приставил к его должности, а главного хлебника
предал смерти~--- как он истолковал им. И главный
кравчий забыл Иосифа в темнице, хотя он возвестил
ему, что с ним случится; и он не думал о том, чтобы
объявить Фараону, как Иосиф сказал ему; но он
забыл о нем.

\vs Jub 40:1
И в те дни Фараон видел два сна в одну ночь о
голоде, который придет на всю землю. И он
пробудился от сна своего, и призвал всех
снотолкователей, которые были в Египте, и
волхвов, и рассказал им оба свои сна; но они не
могли ничего узнать. После этого главный кравчий
вспомнил об Иосифе и сказал о нем царю. И он велел
привести его из темницы и рассказал ему оба свои
сна. И он сказал Фараону: <<Два сна означают одно
и то же>>. И он сказал ему: <<В продолжение семи
лет будет изобилие во всем Египте, и после этого в
продолжение семи лет голод, подобного которому
не было на всей земле. Теперь назначь, Фараон, во
всей земле Египетской житницы, чтобы в них
собирали пищу в каждом городе в продолжение лет
изобилия, чтобы иметь пищу на семь лет голода, ибо
он будет весьма велик>>. И Господь дал Иосифу
милость и милосердие в очах Фараона. И Фараон
сказал своим слугам: <<Мы не найдем столь мудрого
и разумного мужа, как он, ибо дух Господа с ним>>.
И он поставил его вторым над всем своим царством,
и сделал его господином над всем Египтом, и велел
везти на своей второй колеснице, и одел его в
виссонную одежду, и возложил ему золотую цепь на
шею, и велел возвещать впереди него: <<Ел Ел
Ваабрир>>. И он надел (кольцо) на руку его, и
сделал его господином над всем своим домом, и
возвеличил его, и сказал ему: <<Только престолом
одним я буду больше тебя>>. И Иосиф был
господином над всею Египетскою страною. И любили
его все князья Фараона, и все слуги его, и все
исполнявшие царские дела, ибо он ходил в
праведности и без гордости и надменности и был
нелицеприятным и неподкупным, но по
справедливости судил все народы страны. И страна
была хорошо управляема Фараоном благодаря
Иосифу, ибо Господь был с ним и дал ему милость и
благоволение на весь его род в глазах всех,
которые его знали и о нем слышали. И царство
Фараона было благоустроено: ни злоумышленника,
ни злодея не было там. И царь нарек имя Иосифу
Сафанфи-фанс и дал Иосифу в жены дочь Патифарана,
дочь жреца Гелиопольского, главного повара. И в
тот день, когда Иосиф стоял пред Фараоном, ему
было (тридцать) лет, когда он стоял пред
Фараоном. И в тот самый год умер Исаак. И сбылось
так, как Иосиф сказал в толковании его сна: и
пришли семь лет изобилия на всю Египетскую
страну~--- на одну меру тысяча восемьсот мер. И
Иосиф собирал пищу в каждом городе, пока они
не наполнились хлебом, так что нельзя было уже
считать его и мерить по причине великого
изобилия.

\vs Jub 41:1
И в сорок пятый юбилей во вторую седмину во
второй год взял Иуда своему первенцу Еру жену из
дочерей Арама, по имени Фамарь. Но он ненавидел
ее, и не спал с нею, так как мать его была из
дочерей Ханаанских, и он хотел взять себе жену из
родства своей матери, но отец его Иуда не
позволил ему этого. И этот первенец его был
дурной, и Господь лишил его жизни. И Иуда сказал
сыну своему Онану: <<Войди к жене брата твоего, и
соверши с нею брак ужичества, и восстанови свое
семя брату твоему!>> И Онан знал, что это было бы
семя не его, а его брата, и вошел к жене своего
брата, и излил свое семя на землю. И это было злом
пред очами Господа, и Он лишил его жизни. И Иуда
сказал своей невестке Фамари: <<Оставайся в
доме отца твоего вдовою, пока сын мой Шела не
подрастет; тогда я отдам тебя ему в жены>>. И он
подрос. Но Бефзуел, жена Иуды, не допускала, чтобы
сын ее Шела женился на ней. И Бефзуел, жена Иуды,
умерла в пятый год этой седмины.

И в шестой год отправился Иуда стричь своих
овец в Фимнафу. И она сняла вдовьи одежды, и
надела покрывало, и нарядилась, и села при
воротах на дороге в Фимнафу. И когда Иуда вошел,
он встретил ее, и принял ее за блудницу, и сказал
ей: <<Я войду к тебе>>. И она сказала:
<<Войди!>> И он вошел. И она сказала: <<Дай мне
плату блудницы>>. И он сказал: <<Я ничего не
имею при себе, кроме кольца на пальце, и серег, и
трости, которая у меня в руке>>. И она сказала
ему: <<Дай их мне, пока ты не пришлешь мне плату
блудницы>>. И он сказал ей: <<Я пришлю тебе
козленка>>, и отдал их ей. И она зачала от него; и
Иуда пошел к своим овцам, а она в дом отца своего.
И Иуда послал чрез пастуха едолламского
козленка, но он не нашел ее. И он спрашивал людей той
местности, и сказал им: <<Где блудница,
которая была там?>> И они сказали: <<У нас
нет блудницы>>. И он возвратился и известил его,
что он не встретил ее, и сказал ему: <<Я
спрашивал людей того места, и они сказали мне:
<<Нет там блудницы>>>>. И он сказал:
<<Пойдемте, чтобы не быть осмеянными>>. И когда
прошло три месяца, она узнала, что зачала; и они
известили об этом Иуду, говоря: <<Вот твоя
невестка Фамарь сделалась беременной от
блуда>>. И Иуда пошел в дом отца ее, и сказал ее
родителям и братьям: <<Выведите ее, чтобы она
была сожжена, ибо она совершила нечто нечистое в
Израиле>>. И вот, когда они вывели ее, чтобы
сжечь, она послала своему свекру кольцо, и серьгу,
и трость, говоря: <<Узнай, кому принадлежит
это: ибо от того я зачала>>. И Иуда узнал, и
сказал: <<Фамарь правее меня>>. И они не сожгли
ее. И посему она не была отдана Шеле. И он уже не
приближался больше к ней. И после сего она родила
двоих сыновей~--- Фареса и Зару, в седьмой год этой
второй седмины. И тогда окончились семь лет
плодородия, о которых Иосиф сказал Фараону.

И Иуда сознал, что это было дурное дело, которое
он совершил, так как он преспал с своею невесткою,
и нашел это неправым пред своими очами, и сознал,
что он совершил вину и согрешил, так как открыл
покров своего сына. И он стал скорбеть и умолять
Господа о милосердии к своей вине. И мы сказали
ему в сновидении, что она прощена ему, ибо он
неотступно просил о милости, и скорбел, и вновь не
совершил сего. И он получил прощение, ибо он
обратился от своего греха и неразумия. Ибо велика
эта вина пред нашим Господом; всякого, кто делает
так, и всякого, кто преспит с своею тещею, должно
сожечь огнем, чтобы он сгорел в нем. Ибо мерзость
и осквернение лежит на них; огнем должно сожечь
их. И ты также скажи сынам Израиля, чтобы не было
между ними мерзости; огнем должно сожечь мужа,
который преспит с нею, и также жену, дабы Он
отвратил Свой гнев и Свое наказание от Израиля. И
Иуде мы сказали, что так как два его сына не
сочетались браком, то семя его восстановлено для
другого рода, и оно не будет истреблено; ибо он
пришел по своему неведению и желал наказания;
именно по закону Авраама, который он заповедал
своим детям, Иуда хотел сожечь ее огнем.

\vs Jub 42:1
И в первый год третьей седмины сорок пятого юбилея
настал в стране голод; и на земле не было дождя,
так что совсем ничего не падало, и земля
сделалась бесплодною. И только в стране
Египетской была пища, так как Иосиф собрал, чтобы
можно было давать им пищу. И Иосиф собирал в
течение семи лет плодородия семя в стране и
сберегал его. И египтяне пришли к Иосифу, чтобы он
дал им пищи; и он открыл житницы, где был хлеб от
первого года, и продавал его жителям страны за
золото.

И Иаков услышал, что в Египте была пища; (тогда
он послал своих сыновей в Египет приобрести
хлеба), но Вениамина не послал. И они пришли
вместе с сопровождавшими их; и Иосиф узнал их,
но они его не узнали. И он беседовал с ними, и
спрашивал их, и говорил им: <<Не соглядатаи ли
вы, и не пришли ли, чтобы разузнать след
страны?>> И он заключил их; потом он освободил
их, и оставил одного только Симеона, и его девять
братьев отпустил, и наполнил мешки их хлебом; а их
золото он положил им в их мешки, но так, что они не
знали. И он повелел им привести своего младшего
брата, ибо они сказали ему, что их отец и младший
брат живы. И они вышли из страны Египетской, и
пришли в землю Ханаанскую, и рассказали своему
отцу все, что с ними случилось, и как правитель
страны говорил с ними, и как он посадил Симеона в
заключение, пока они не привезут к нему
Вениамина. И Иаков сказал: <<Вы похитили у меня
моих детей; Иосифа нет более, и Симеона нет, и
Вениамина еще хотите взять? Ваши дурные действия
ложатся тяготою на мне>>. И он сказал: <<Сын мой
не пойдет с вами; он может заболеть во время
пути. Ибо мать их родила только двоих; один
из них потерян, и еще этого хотите у меня взять? С
ним может случиться в путешествии болезнь, и вы
доведете до смерти мою седую старость от горя>>.
Ибо он видел, что золото их принесено назад в их
мешках, и посему он боялся послать его с ними.

И усилился голод, и сделался великим в стране
Ханаанской и во всех странах, кроме только земли
Египетской. Ибо многие из египтян собирали себе
семена в пищу, после того как увидели, что Иосиф
собирает семена, и кладет их в житницы, и
сберегает на голодные годы. И жители Египта
прокармливались этим в первый год голода. И когда
Израиль увидел, что голод в стране очень
усилился, и не было более спасения, он сказал
своим сыновьям: <<Идите опять, и приобретите
себе пищи, чтобы нам не умереть>>. И они сказали:
<<Мы не пойдем; если наш младший брат не пойдет с
нами, то мы не пойдем>>. И (Иаков) увидел, что
если он не пошлет его с ними, то все они погибнут
от голода. И Робел сказал: <<Передай мне его в
мои руки, и если я его не приведу к тебе назад, то
умертви двух моих сыновей за его душу>>. Но он
сказал: <<Он не пойдет с тобою>>. И Иуда подошел
и сказал: <<Отпусти его со мною, и если я его не
приведу к тебе назад, то буду пред тобою
преступником во все дни моей жизни>>. И он
отпустил его с ними во второй год седмины в
новолуние, и они пришли в Египетскую страну
вместе со всеми другими, шедшими туда, с дарами в
своих руках, с стираксой (стакти), и миндальными
орехами, и фисташками, и чистым медом. И они
пришли и предстали пред Иосифа, и он увидел
Вениамина, своего брата, и узнал их, и сказал им:
<<Это ваш младший брат?>> И они сказали ему:
<<Это он>>. И он сказал: <<Да будет милость
Господня с тобою, сын мой!>> И он послал их в свой
дом, и выдал им также Симеона, и приготовил им
обед. И они передали ему дар, который они привезли
для него. И они ели пред ним, и он дал каждому из
них по части, но часть Вениамина была в семь раз
больше, чем часть остальных. И они ели, и пили, и
встали, и оставались у своих ослов. И Иосиф
придумал способ, посредством которого он мог бы
узнать их помышления, господствуют ли в них
человеческие помышления. И он сказал мужу,
который управлял его домом: <<Наполни им все их
мешки хлебом, положи им также назад их золото в их
хранилища, и мою серебряную чашу, из которой я
пью, положи в мешок младшего, и отпусти их>>.

\vs Jub 43:1
И он сделал, как сказал ему Иосиф, и наполнил
мешки их пищею, и золото их положил также в их
мешки, и чашу в мешок Вениамина. И рано утром они
отправились. И когда они выехали оттуда, Иосиф
сказал мужу: <<Гонись за ними, беги и обличи их,
говоря: <<Вы отплатили злом за добро, и похитили
серебряную чашу, из которой пьет господин мой>>.
И приведи назад ко мне их младшего брата, и
приведи его немедленно, прежде чем я займусь
своими делами>>. И он побежал за ними и сказал им
по его словам. И они сказали ему: <<Да будет это
далеко от рабов твоих; они не сделают ничего
подобного, и не украдут никакого имущества из
дома твоего господина. И даже золото, которое мы в
первый раз нашли в наших мешках, мы, рабы твои,
принесли назад из земли Ханаанской. Украдем ли мы
какое-нибудь имущество? Вот мы здесь и мешки наши:
ищи, и тот из нас, в мешке которого ты найдешь
чашу, пусть будет наказан смертию, и мы с своими
ослами будем в подчинении у твоего господина>>.
И он сказал им: <<Нет; мужа, у которого я найду,
его одного только возьму я в рабы; а вы идите с
миром>>. И когда он искал в их сосудах, он начал
со старшего и кончил младшим, и она была найдена в
мешке Вениамина, младшего. И они пришли в ужас, и
разорвали свои одежды, и навьючили своих ослов, и
возвратились назад в город. И они пришли в дом
Иосифа, и пали все пред ним на свое лице на землю.
И Иосиф сказал им: <<Вы сделали это>>. И они
сказали: <<Что нам сказать и как оправдаться,
когда наш господин нашел вину за своими рабами?
Вот мы рабы господина нашего вместе с нашими
ослами>>. И Иосиф сказал им: <<Я страшусь
Господа, и вы пойдете домой; но ваш брат будет
принадлежать мне, ибо вы сделали злое. Вы не
знаете, что муж, как я, пьющий из этой чаши,
дорожит своею чашею? И вы похитили ее у меня>>. И
Иуда сказал: <<Да будет позволено мне, господин
мой, сказать слово в уши господина моего. Двоих
братьев мать моя родила нашему отцу, рабу твоему:
один ушел и погиб, так что его уже не нашли; и
только тот один остался от своей матери, и раб
твой, отец наш, любит его, и душа его привязалась к
этой душе. И будет, что если мы возвратимся к рабу
твоему, отцу нашему, и младшего не будет с нами, то
он умрет, и мы погубим нашего отца, и он умрет от
печали. Лучше я буду рабом твоим вместо дитяти,
рабом моего господина; но юноше позволь идти с
его братьями, ибо я поручился за него пред рабом
твоим, отцом нашим; и если ты не отдашь его, то раб
твой будет всегда виновным пред нашим отцом>>.

И Иосиф увидел, что все они были единодушными и
благожелательными друг к другу; и он не мог более
удерживаться, и сказал им, что он~--- Иосиф, и
разговаривал с ними по-еврейски, и пал им на шею, и
плакал; и они не узнали его. Теперь и они начали
плакать. И он сказал им: <<Не плачьте из-за меня.
Поспешите и приведите ко мне отца моего, чтобы я
увидел моего отца, прежде чем умру [...]. Ибо вот это
второй год голода, и еще предстоят пять лет, когда
не будет жатвы, и плода с деревьев, и никаких
растений. Поспешите с вашими домочадцами, чтобы
вам не погибнуть от голода и не быть в
беспокойстве за себя и за свое имущество. Ибо
Господь послал меня вам как вашего питателя,
чтобы остались в живых многие. И расскажите отцу
моему, что я жив еще. Вы сами видите, что Господь
поставил меня отцом Фараону и господином в доме
его и над всею страною Египетскою. И расскажите
отцу моему о всей моей славе и о всем богатстве и
славе, которые дал мне Господь>>. И он дал им по
повелению Фараона колесницы и пищу на дорогу и
дал им цветные одежды и серебро; и отцу их также
Фараон послал одежд, и серебра, и десять ослов,
которые везли хлеб. И он отпустил их, и они пошли и
рассказали своему отцу, что он жив, и что он всем
народам земли отпускает хлеб, и что он поставлен
господином над всею Египетскою землею. И отец их
не поверил этому, ибо он был поражен в своей душе.
И после сего он увидел колесницы, которые прислал
Иосиф; тогда опять ожил вновь дух его. И он сказал:
<<Довольно для меня, что Иосиф жив; я пойду и
увижу его, прежде чем умру>>.

\vs Jub 44:1
И Израиль пошел из своего жилища Хеврона в
новолуние третьего месяца, и зашел к клятвенному
колодезю, и принес жертву Богу отца своего Исаака
в седьмой день этого месяца. И Иаков вспомнил сон,
который он видел в Вефиле, и убоялся идти в
Египет. И, подумав, он хотел известить Иосифа,
чтобы он пришел к нему, и что он сам не пойдет; он
оставался там семь дней, ожидая, не увидит ли
он, быть может, видение о том, оставаться ли ему
или идти. И он совершил праздник жатвы~--- начатков
хлеба~--- со старым хлебом, ибо во всей стране
Ханаанской не было и пригоршни семян, но был
голод для всех зверей, и скота, и птиц, и людей. И в
шестнадцатый день явился ему Господь и сказал:
<<Иаков, Иаков!>> И он сказал: <<Вот я
здесь>>. И Он сказал ему: <<Я Бог отцов твоих,
Авраама и Исаака; не бойся и иди в Египет! Ибо Я
сделаю тебя там великим народом; Я пойду с тобою,
и приведу (возвращу) тебя в эту землю, чтобы ты был
погребен здесь. И Иосиф закроет своими руками
твои глаза. Не бойся, иди в Египет!>>

И они собрались, его дети и дети его детей, и
посадили своего отца и положили свое имущество
на колесницы. И Израиль пошел от клятвенного
колодезя в шестнадцатый день этого третьего
месяца и отправился в страну Египет. И Израиль
послал сына своего Иуду вперед себя к Иосифу,
чтобы осмотреть страну Гесем. Ибо сюда~--- так
сказал Иосиф братьям~--- они должны были прийти,
чтобы жить здесь, дабы быть им вблизи его. И это
хорошая страна в земле Египте; и она была
недалеко от него.

Вот имена сыновей Израиля, которые пошли с
своим отцом Иаковом в Египет. Иаков, отец их.
Робел, перворожденный Израиля. И вот имена его
сыновей: Енох, Фалус, Есером, Карами~--- пятеро.
Симеон и его сыновья; и вот имена его сыновей:
Иямуел, Иямин, Аод, Ияким, Саар, Саул, сын
Сефенсеянки~--- семеро. Левий и его сыновья; вот
имена сыновей его: Гедеон, Кааф и Мерари~--- четверо.
Иуда и его сыновья; и вот имена его сыновей: Селом,
Фарес, Зара~--- [четверо]. Исашар и его сыновья; и
вот имена его сыновей: Фола, Фуа, Иясоб, и Сам~---
пятеро. Заблон и его сыновья; и вот имена его
сыновей: Саор, и Елом, и Иялиел~--- четверо. И вот
сыновья Иакова, которых родила Иакову Лия в
Месопотамии, шесть сыновей и одна сестра их Дина.

И всех душ детей Лии и их детей, которые пошли со
своим отцом Иаковом в Египет, было двадцать
девять; с отцом их Иаковом было тридцать. И дети
Залафы, служанки Лии, жены Иакова, которых она
родила Иакову, суть Гад и Асер. И вот имена их
детей, которые пошли с ними в Египет. Дети Гада:
Сафион, Агафи, Суни, Асон, Араби, Аради~--- восьмеро.
И дети Асера: Иямна, Иесуа, Баръя и Сара, их сестра.
Всего четырнадцать душ. И всех детей Лии было
сорок четыре. И дети Рахили, жены Иакова,~--- Иосиф и
Вениамин. И у Иосифа родились в Египте, прежде чем
отец его пришел в Египет, сыновья, которых родила
ему Ассенеф, дочь Питфары Гелиопольского,~---
Манассе и Ефрем~--- трое. Дети Вениамина: Лаубаел,
Асбел, Гуав, Нееман, Абродио, Раифес, Ианини, Афим,
Яам, Гаам~--- одиннадцать. И всех детей Рахили было
четырнадцать. И дети Баллы, служанки Рахили, жены
Иакова, которых она родила Иакову,~--- Дан и
Неффалим. И вот имена их детей, которые пошли с
ними в Египет. Дети Дана: Куси, Самой, Асуд, Иясек,
Саломон~--- шестеро. И они умерли в Египте в тот год,
в который пришли, и у Дана остался только Куси. И
вот имена детей Неффалима: Иязиел, Гахан, Асаар,
Якум, Ау~--- шестеро. И умер Ау, родившийся после
первого голодного года. И всех детей Рахили
вместе было двадцать шесть. И всех душ Иакова,
пришедших в Египет, было семьдесят душ. Вот его
дети и дети его детей~--- всего семьдесят. И пятеро
из них умерли в Египте при Иосифе, не имея детей. И
в стране Ханаанской умерли у Иуды два его сына~---
Ер и Онан, не имея детей. И сыновья Израиля
похоронили тех, которые умерли, и они входят в
число семидесяти человек.

\vs Jub 45:1
И Израиль пришел в Египетскую землю, в страну
Гесем, в новолуние четвертого месяца во второй
год третьей седмины сорок пятого юбилея. Иосиф
вышел навстречу своему отцу Иакову, в страну
Гесем, и пал отцу на шею, и плакал. И Израиль
сказал Иосифу: <<Теперь я умру спокойно, так
как увидел тебя. И ныне да будет прославлен
Господь, Бог Израилев, Бог Авраама, и Бог Исаака,
Который не отвратил Своего милосердия и
благоволения от раба Своего Иакова! Довольно для
меня, что я увидел лицо твое, пока я жив. Да,
истинно видение, которое я видел в Вефиле. Да
будет прославлен Господь, Бог мой, во весь век!>>
И Иосиф и братья его ели пред очами своего отца
хлеб, и пили вино; и Иаков был исполнен великой
радости, что видел Иосифа, как он с братьями
своими пред его глазами ел и пил. И он прославил
Творца всех вещей, Который сохранил его, и
сохранил ему двенадцать его сыновей. И Иосиф дал
своему отцу и своим братьям в дар страну Гесем,
чтобы они жили в ней и в Рамизифино и во всей ее
области, чтобы они владели ею пред глазами
Фараона. И Израиль жил с своими сыновьями в
стране Гесем, лучшей части земли Египетской.
Израиль же был ста тридцати лет, когда он пришел в
Египет. И Иосиф снабжал своего отца, и своих
братьев, и их домочадцев съестными припасами,
насколько они нуждались в них, в продолжение семи
лет голода. И земля Египетская страдала от
голода. И Иосиф подчинил всю страну Египет
Фараону за хлеб, и также людей и скот; все
приобрел Фараон.

И кончились неурожайные годы, и Иосиф дал
народам, жившим в стране, семян и съестных
продуктов, чтобы они посеяли их в восьмой год; ибо
река наводнила всю страну Египет. Именно в семь
лет неурожая она орошала только отдельные места
около берега реки; теперь же она переполнилась. И
египтяне засеяли страну, и она принесла в том
году много хлеба, и это был первый год четвертой
седмины сорок литого юбилея. И Иосиф взял из
хлеба, который они засевали, пятую часть для царя,
и четыре (части) оставил им в пищу и для посева. И
Иосиф сделал это законом для Египетской земли до
сего дня.

И Израиль жил в стране Египте семнадцать лет, и
всей его жизни было три юбилея, сто сорок семь
лет. И он умер в четвертый год пятой седмины сорок
пятого юбилея. И Израиль благословил своих
сыновей пред своею смертью, и сказал им все, что
случится с ними в последние дни; все возвестил он
им, н благословил их. И он дал Иосифу две части в
стране. И он почил с своими отцами, и был погребен
в двойной пещере в земле Ханаанской, рядом с
своим отцом Авраамом, в могиле, которую он
выкопал для себя, в двойной пещере, в стране
Хеврон. И он отдал все свои книги и книги своих
отцов сыну своему Левию, чтобы он хранил их и
возобновлял их для своих детей до сего дня.

\vs Jub 46:1
И было, после того как Иаков умер, умножились
сыны Израиля в стране Египетской и сделались
многочисленными; и они были все единодушными в
своих мыслях, так что брат любил своего брата, и
каждый помогал своему брату; и они умножились
чрезмерно. И было всей жизни Иосифа десять
седмин, которые он прожил после своего отца. И
Иосиф не имел зложелателя, и не случилось с ним
чего-либо худого во все время его жизни, которую
он прожил после отца своего Иакова. Ибо все
Египтяне почитали сынов Израиля в продолжение
всего времени, пока жил Иосиф. И Иосиф умер ста
десяти лет; семнадцать лет он пробыл в стране
Ханаанской, и десять лет был слугою, и три гада
пробыл в темнице, и восемьдесят лет у царя
управлял всею страною Египетскою, И он умер, и все
его братья, и весь тот род.

И он завещал сынам Израиля перед смертью, чтобы
они взяли с собою его кости, когда они выйдут из
Египта. И он взял с них клятву относительно
костей своих; ибо он знал, что Египтяне не отнесут
его тело и не похоронят его в свое время в
стране Ханаанской, так как Ханаанский царь
Мемкерон, владевший страною Ассур, сражался в
долине с царем Египетским, и убил там его, и
преследовал Египтян до ворот Эромона. Но он не
мог вступить в Египет, ибо восстал другой
новый царь над Египтом для управления, и был
могущественнее его. И он возвратился в страну
Ханаанскую, а ворота Египта были заперты, и никто
не приходил в Египет.

И Иосиф умер в сорок шестой юбилей в шестую
седмину во второй год, и они похоронили его в
стране Египетской. И все братья его умерли после
него. И царь Египетский выступил, чтобы сразиться
с царем Ханаанским, в сорок седьмой юбилей во
вторую седмину во второй год. И сыны Израиля
вынесли кости всех сыновей Иакова, кроме Иосифа,
и похоронили их на поле, в двойной пещере на горе.
И большинство возвратилось в Египет; и только
немногие из них остались на горе Хеврон, и твой
отец [Амрам] остался с ними. И царь Ханаанский
победил царя Египетского, и запер ворота Египта.

И он (царь Египетский) замыслил недоброе дело
против сынов Израиля~--- притеснять их, и сказал
египтянам: <<Вот народ сынов Израиля возрос и
сделался многочисленнее нас; употребим же против
них хитрость, прежде чем они слишком размножатся,
и будем притеснять их рабскою работою, прежде чем
нас постигнет поражение и они победят нас в
битве. А не то они вступят в союз с врагами и
выйдут из нашей страны; ибо их сердце и лицо
обращено к стране Ханаанской>>. И он поставил
над ними смотрителей за работами, чтобы они
притесняли их рабскою работою. И они должны были
строить крепкие города для Фараона~--- Питофо и
Рамзе, и должны были строить всякие стены и
оплоты, которые обрушились в городах египетских,
и они сильно притесняли их. Но чем хуже поступали
они с ними, тем больше умножались и увеличивались
они. И египтяне считали сынов Израиля нечистыми.

\vs Jub 47:1
И в седьмую седмину в седьмой год сорок
седьмого юбилея пришел отец твой из страны
Ханаанской, и ты родился в четвертую седмину, в
шестой год, в сорок восьмой юбилей, когда были дни
преследования сынов Израиля. И царь Фараон
Египетский дал повеление относительно них, чтобы
детей их~--- всякое дитя мужеского пола, которое
родится,~--- бросали в реку. И они бросали их в
течение семи месяцев до того месяца, когда ты был
рожден. И твоя мать скрывала тебя три месяца, и на
нее донесли. Тогда она сделала для тебя корзину, и
осмолила ее смолою и асфальтом, и положила ее в
траву на берегу реки, и клала тебя в нее в течение
семи дней. И мать твоя приходила ночью и кормила
тебя грудью; и днем тебя стерегла от птиц сестра
твоя Мария.

И в те дни пришла дочь Фараона Фармуф
искупаться в реке. И она услышала твой голос,
когда ты плакал, и сказала своей служанке, чтобы
она принесла тебя. И она принесла тебя к ней. И она
вынула тебя из корзинки, и сжалилась над тобою. А
сестра твоя сказала ей: <<Не пойти ли мне, и не
призвать ли к тебе одну из евреек, чтобы она
воспитала это дитя для тебя и кормила грудью?>>
И она пошла, и призвала твою мать Ийокабиф, и она
дала ей плату, чтобы она ходила за тобою. И после
сего ты возрос, и тебя привели в дом Фараона, и ты
сделался отроком. И твой отец (Амбран) научил тебя
писанию. И после того как ты окончил три седмины,
он привел тебя в царский дворец, и ты был при
дворе три седмины до того времени, когда ты вышел
из царского дворца и увидел египтянина, который
бил твоего друга из сынов Израиля. И ты убил его и
скрыл его в песке. И в следующий день ты встретил
двоих из сынов Израиля, которые ссорились, и
сказал обидчику: <<Зачем ты бьешь своего
брата?>> И он разгневался, и озлобился, и сказал:
<<Кто поставил тебя начальником и судьею над
нами, разве ты хочешь убить меня, как ты убил
египтянина?>> И ты испугался и убежал
вследствие этих слов.

\vs Jub 48:1
И в шестой год третьей седмины сорок девятого
юбилея ты ушел и оставался (вне Египта) шесть
седмин и один год. И ты возвратился в Египет во
вторую седмину во второй год в пятидесятый
юбилей. И ты знаешь, что Он говорил с тобою у горы
Синай, и что высший Мастема хотел сделать с тобою
на пути, когда ты возвращался в Египет, в праздник
кущей. Не хотел ли он всеми силами умертвить тебя
и спасти египтян от рук твоих, когда увидел, как
ты был послан совершить над египтянами суд и
мщение? И я спас тебя от руки его и совершил
знамения и чудеса, которые ты был послан
совершить в Египте пред Фараоном, и всем его
домом, и рабами его, и его народом. И Господь
совершил мщение над ними, тяжкое мщение за
Израиля, и поражал, и умерщвлял их чрез кровь, и
чрез жаб, и мошек, и песьих мух, и злокачественные
воспалительные нарывы, и их скот Он поразил смертию,
и градом~--- чрез это Он истребил все, что росло у
них,~--- и саранчой, которая поела остаток,
оставшийся от града, и тьмою; и их первенцев из
людей и скота Он истребил. И всем их идолам
отметил Господь и сожег их огнем. И все это
послано было чрез твою руку, чтобы ты совершил
это, [...]. И ты говорил с царем египетским, и пред
всеми его служителями, и пред его народом; и все
случилось по твоему слову; десять великих и
страшных наказаний пришли на страну Египетскую,
чтобы чрез них отметить за Израиля. И все это
совершил Господь за Израиля и согласно завету,
который Он заключил с Авраамом, чтобы отметить
им, ибо они жестоко притесняли их. И высший
Мастема восстал против тебя, и хотел предать тебя
в руки Фараона, и содействовал египетским
волхвам, и помогал им, чтобы и они сделали это
пред твоими глазами. Хотя мы и допустили их
произвести зло, но, однако, не позволили им
врачебных средств, чтобы они воспользовались ими
своими руками. И Господь поразил их (волхвов)
злокачественными нарывами, чтобы они не могли
противостоять ему; ибо мы погубили их, чтобы они
не могли совершить ни одного знамения. Но
несмотря на все знамения и чудеса, высший Мастема
не смутился, ибо он приложил все силы и воззвал к
египтянам, чтобы они преследовали тебя всеми
силами Египта, с своими колесницами и конями, и со
всем множеством народов Египта. И я встал между
тобою и ими, между египтянами и израильтянами, и
спас израильтян от руки их, от руки египтян. И
Господь провел их чрез море, как по сухой земле; и
всех людей, которые выступали для преследования
Израиля, Господь Бог наш ввергнул в море, в
глубину бездны, вместо детей Израиля, за то, что
египтяне бросали их в реку сотня за сотней; за это
совершено над ними мщение, и тысяча сильных мужей
[...] была истреблена за одного погибшего младенца
из детей твоего народа, брошенного ими в реку. В
четырнадцатый, и в пятнадцатый, шестнадцатый,
семнадцатый и восемнадцатый дни высший Мастема
был связан и заключен позади сынов Израиля, чтобы
он не мог обвинять их (пред египтянами). А в
девятнадцатый день мы освободили его, чтобы он
помогал египтянам и чтобы они преследовали сынов
Израиля. И он очерствил сердце их, и ожесточил их,
и стал могущественным над ними по воле Господа,
Бога нашего, чтобы поразить египтян и ввергнуть
их в море. И в пятнадцатый день мы связали его,
чтобы он не обвинял сынов Израиля, в тот день,
когда они требовали у египтян утварь и одежды,
утварь серебряную, золотую и медную, чтобы
обобрать египтян за то, что когда они служили им,
они сильно притесняли их; и мы не допустили, чтобы
сыны Израиля вышли из Египта с пустыми руками.

\vs Jub 49:1
Вспомни заповедь, которую дал тебе Господь
относительно пасхи, чтобы ты соблюдал ее в свое
время, в четырнадцатый день первого месяца, чтобы
ты заколол его (агнца), прежде чем наступит вечер,
и чтобы ели его ночью, в вечер пятнадцатого дня, с
солнечного захода. Ибо день этот есть первый
праздник и первый день пасхи. И вы ели пасху в
Египте, в то время как все силы Мастемы были
освобождены, чтобы умерщвлять всякого первенца в
стране Египетской, от первенца фараонова до
первенца пленной рабыни на мельнице и до екота. И
вот знамение, которое дал им Бог. В каждый дом, у
которого дверной косяк был обрызган кровью
агнца, в этот дом они не должны были входить для
избиения находящихся в нем, так что все, бывшие в
этом доме, спаслись, потому что на двери был знак
крови. И силы Господин сделали все, что только
Господь повелел им, и прошли мимо всех сынов
Израиля. И на них не простерлось бедствие, чтобы
погубить из них чью-либо душу, ни на скот, ни на
человека, ни даже на собаку. В Египте же бедствие
было очень велико, и не было дома, в котором не
было бы умершего, и плача, и сетования. И весь
Израиль спокойно вкушал пасхальное мясо, и пил
вино, и хвалил, и благодарил, и прославлял
Господа, Бога отцов своих, и приготовлялся к
исходу из-под ига. рабства и из злого Египта. И ты
помни этот день во все дни твоей жизни, раз в год,
в свой (определенный) день, согласно со всем
законом относительно сего, и не смешивай этого
дня с другими и этого месяца с другим. Ибо это~---
вечное установление, и оно начертано на небесных
скрижалях для сынов Израиля, чтобы они каждый год
соблюдали праздник, один раз в год, во все свои
роды; и нет предела времени сему, но он (праздник)
утвержден навек. И муж, если он чист и не придет
совершить его в назначенный день, чтобы принести
дар, угодный Господу, в день Его праздника и чтобы
есть и пить пред Господом в день Его праздника,
тот муж должен быть истреблен, если он чист и
находится недалеко, ибо не принес дар Господень в
назначенное время. И грех примет на себя тот муж.
Сыны Израиля, грядущие, должны праздновать пасху
в назначенное для нее время, в четырнадцатый день
первого месяца вечером, в третью часть дня до
третьей части ночи. Ибо две части дня назначены
для света и третья~--- для вечера. Вот то, что
повелел Господь, чтобы ты совершал это в исходе
вечера. И не должно совершаться это утром в
какой-либо час света, но в вечернее время. И они
должны вкушать его в вечернее время до третьей
части ночи, и что останется от всего мяса после
третьей части ночи, они опять должны сожечь
огнем. И они не должны варить его в воде, и не
должны его есть сырым, но тщательно испекши на
огне и изжарив на огне. Его голову, со
внутренностями и ногами его, они должны изжарить
на огне и не раздроблять ему костей. Ради сего
Господь повелел сынам Израиля, чтобы они
праздновали пасху в назначенный для нее день и не
раздробляли у него (агнца) костей; ибо это
праздничный день и назначенный для празднования
день, и нельзя уклоняться от него на день или на
месяц, но в свой праздничный день он должен
праздноваться. И ты скажи сынам Израиля, чтобы
они соблюдали пасху в ее день, ежегодно, один раз
в год, в определенный день, чтобы это было
воспоминанием, которое будет приятно для
Господа, и чтобы не случилось с ними в том году
никакого бедствия и они не были бы умерщвлены и
поражены. Если они будут праздновать пасху в свое
время, соблюдая все, как заповедано, то они не
должны вкушать ее вне святилища Господня; пред
всем народом общества Израилева должны
соблюдать ее в свое время все люди, которые
явились в день ее, чтобы вкушать пред Господом в
святилище вашего Бога, кто имеет двадцать лет и
выше. Ибо так написано и определено, чтобы ели ее
в доме святилища Господня. И когда сыны Израиля
придут в страну, которою они будут владеть, в
страну Ханаанскую, и устроят скинию Господню в
сей стране, в одном из своих отрядов (колен), так
что святилище Господа будет устроено в стране, то
они должны приходить и праздновать пасху среди
скинии Господней, и закалать ее пред Господом из
года в год. И во дни, когда будет устроен дом во
имя Господне в стране их наследия, они должны
ходить туда и закалать пасху вечером, когда
зайдет солнце, в третью часть дня, и должны
окропить кровью порог алтаря, и тук положить на
огонь, который на жертвеннике, мясо же его,
изжаренное на огне, есть в преддверии дома
святилища во имя Господне. И они не должны
совершать пасху в своих городах и в своих местах,
а только пред скиниею Господнею, или пред Его
домом, так как имя Его живет в нем, дабы им не
согрешить пред Господом. И ты, Моисей, скажи сынам
Израиля, чтобы они соблюдали постановление о
пасхе, как повелено тебе, что вы должны соблюдать
ее ежегодно в день ее и также праздник
опресноков, чтобы они ели пресное в продолжение
семи дней, соблюдая праздник Его и принося для
Него ежедневно дар, в те семь пасхальных дней,
пред Господом, на жертвеннике вашего Бога. Ибо
этот праздник вы праздновали с боязливою
робостию, когда вы вышли из Египта, пока не
перешли чрез море в пустыню Сур; ибо на берегу
моря вы окончили его.

\vs Jub 50:1
И потом после сего закона я возвестил тебе о
субботних днях в пустыне Синая, которая
находится между Еломом и Синаем. И также о
субботах земли я сказал тебе на горе Синай и о
юбилейных годах вместе с субботними годами.

Но год его мы не сказали тебе, пока ты не придешь
в страну, которою вы будете владеть. Тогда и
страна должна праздновать свои субботы, когда
они будут жить в ней, и они узнают год юбилея.
Посему я определил тебе седмины и юбилейные годы:
сорок девять юбилейных годов от дней Адама до
сего дня и одна седмина и два года. И еще
предлежат тебе сорок лет, чтобы узнать заповеди
Господа, пока они не переправятся чрез Иордан к
западу. И юбилеи прекратятся, когда Израиль
очистится от всякого блуда, и вины, и нечистоты, и
осквернения, и греха, и злодеяния, и спокойно
будет жить во всей стране, и против него не
восстанет более ни сатана, ни какой-либо
ненавистник, и земля будет с тех пор чистою
всегда.

И вот я записал тебе также повеление
относительно суббот, и все установления законов
относительно них: шесть дней делай дела, и в
седьмой день суббота для Господа Бога вашего. Вы
не должны делать в нее никакого дела, вы, и ваши
сыновья, и ваши рабы, и служанки, и весь ваш скот, и
чужеземец, который у тебя. И человек, который
делает какое-либо дело, должен умереть. Всякий,
кто оскверняет этот день, кто спит с своею женою,
и кто говорит о том, что он хочет предпринять в
нее (в субботу) путешествие или о разного рода
купле и продаже, и кто черпает воду, не приготовив
ее себе в шестой день, и кто поднимает ношу, чтобы
перенести ее из своего шатра или из своего дома,
тот должен умереть. Вы не должны делать никакого
дела в субботу, которого вы не приготовили себе в
шестой день, чтобы есть, и пить, и покоиться, и
соблюдать субботу от всякого дела в этот день, и
прославлять Господа Бога вашего, Который дал ее
вам в праздник. И днем святым, и даем святого
царства для всего Израиля должен быть этот день в
вашей жизни непрестанно. Ибо велика честь,
которой Господь удостоил Израиля, чтобы они ели,
и пили, и насыщались в этот праздничный день, и
отдыхали от всякого дела, которое относится к
человеческим делам, кроме воскурения фимиама и
принесения даров и жертв пред Господом в субботы.
Только это дело пусть совершается в субботы, во
дни дома святилища Господа Бога вашего, чтобы
приносить в умилостивление за Израиля
непрестанно и ежедневно дары в воспоминание,
которое приятно и которое делает их угодными
пред Господом, каждый день года, как повелено
тебе. Но каждый человек, который совершает дело, и
предпринимает путешествие, и ухаживает за своим
скотом, будь это дома или в другом месте, и кто
зажигает огонь, или едет верхом на каком-нибудь
животном, или путешествует на корабле по морю, и
каждый, кто убивает и умерщвляет кого-либо, и кто
закалывает животное или птицу, и кто ловит зверя,
или птицу, или рыбу, и кто постится, и кто ведет
войну в субботний день; всякий, кто делает
что-нибудь из этого в субботний день, тот должен
умереть, чтобы дети Израиля хранили субботу по
заповедям о субботах земли, как это списано с
небесных скрижалей, которые Он дал мне в мои руки,
дабы я написал тебе законы времени и время по
делению его дней.

\chhdr{Отрывки из Книги Юбилеев, сохранившиеся у греческих церковных писателей}
\chhdr{1. Св. Епифаний Кипрский}
Но в Книге Юбилеев, называемой также и Малым
Бытием, можно найти, что эта книга содержит имена
жен Каиновой и Сифовой, чтобы всяким образом были
посрамлены эти слагатели басен для жизни (т.е.
Сифияне). Когда Адам родил сынов и дочерей, было
необходимостью в то время, чтобы его сыновья
вступили в брак с собственными сестрами; ибо это
не было беззаконным, потому что иного рода не
было. Да и сам Адам, можно сказать, был в
супружестве почти с собственной дочерью,
образованною из его тела и созданною Богом для
супружества с ним, и это не было беззаконным. Так
и сыновья его вступили в брак~--- Каин с старшею
сестрою, так называемой Савою, а Сиф, третий сын,
рожденный после Авеля, с сестрою своею,
называемой Азурою. У Адама родились, как
описывает Малое Бытие, и другие девять сыновей,
после тех трех, так что у него было две дочери, а
детей мужеского пола двенадцать: один убитый, а
одиннадцать оставшихся в живых. Ты имеешь
указание на это в Бытии мира и первой книге
Моисеевой, где говорится так: <<и поживе Адам
лет девятьсот тридесять, и роди сыны и дщери, и
умре>>. (Ср. Кн. Юбил., IV).

\chhdr{2. Иоанн Зонара}
Действительно я знаю записанное в Малом Бытии,
что в первый день и небесные силы прежде прочего
были созданы Творцом вселенной; но так как это
Малое Бытие не отнесено к книгам еврейской
мудрости, написанным божественными отцами, то я
ничего, что в ней написано, не считаю достаточно
твердым и не соглашаюсь с этим учением (ср. Кн. Юбил., II).

\chhdr{3. Георгий Синкелл}
В первосозданные сутки, по-еврейскому, в первый
день первого месяца Нисана, как указано прежде,
по-римскому, в двадцать пятый день месяца марта и,
по-египетскому, в двадцать девятый день Фаменофа,
в день божественный, именно в первую неделю, Бог
сотворил небо и землю, мрак и воды, дух и свет, и
сутки, всего семь творений. Во вторые сутки
явилась твердь~--- одно творение. В третьи сутки
было четыре творения~--- появление земли и осушение
ее, рай, многоразличные деревья, травы и семена. В
четвертый день Бог сотворил солнце, и луну, и
звезды. В пятый день Бог сотворил пресмыкающихся
и всех плавающих, великих морских животных и рыб,
и все, что в водах, а также пернатых~--- всего три
творения. В шестой день Бог сотворил
четвероногих и пресмыкающихся на земле, зверей и
человека~--- четыре творения. Вместе всех творений~---
двадцать два, соответственно двадцати двум
еврейским буквам, затем двадцати двум еврейским
книгам и наконец двадцати двум генерациям от
Адама до Иакова, как говорится в Малом Бытии,
которое называют иные откровением Моисея. Эта же
книга говорит, что небесные силы были сотворены в
первый день (Кн.Юбил.,II).

Необходимость побудила меня сообщить нечто и
из того, что и другими историками, записавшими
иудейские древности и христианские
повествования, заимствуется о сем из Малого
Бытия и так называемой Жизни Адама~--- хотя она и не
считается божественною,~--- чтобы исследующие это
не впали в нелепейшие вымыслы. Итак, в известной
под именем Жизни Адама указывается число дней,
когда было наименование животных, и образование
жены, и вход Адама в рай, и заповедь Божия к нему о
пище с дерева, и вход Евы после сего в рай, также
обстоятельства преступления заповеди и
последствия преступления, как далее следует.

В первый день недели, который был третьим от
сотворения Адама, восьмой первого месяца Нисана,
первый месяца апреля и, по-египетскому, шестой
месяца Фармуфи, Адам по некоему божественному
благоизволению наименовал диких зверей; во
второй день второй недели он дал имена скотам; в
третий день второй недели он наименовал
пернатых; в четвертый день второй недели он
наименовал пресмыкающихся; в пятый день второй
недели он наименовал плавающих. В шестой день
второй недели, который, по-римскому, был шестой
день апреля, а по-египетскому, одиннадцатый
месяца Фармуфи, Бог, взявши некую часть ребра
Адамова, образовал жену. В сорок шестой день от
сотворения мира, в четвертый день седьмой недели,
четырнадцатого Пахона, девятого мая, когда
Солнце было в знаке Тельца и Луна против
созвездия Скорпиона, в восход Плеяд, Бог ввел
Адама в рай в сороковой день после его
сотворения. В пятидесятый день от сотворения
мира, в сорок четвертый от сотворения Адама, день
божественный, восемнадцатого Пахона,
тринадцатого мая, через три дня после входа его в
рай, когда Солнце было в знаке Тельца и Луна в
знаке Козерога, Бог заповедал Адаму не вкушать от
древа познания.

В девяносто третий день творения, во второй
день четырнадцатой недели, во время летнего
поворота Солнца, когда и Солнце и Луна были в
созвездии Рака, в двадцать пятый день месяца
июня, первого Епифи, введена была Богом в рай
помощница Адама Ева, в восьмидесятый день по
сотворении ее. Взяв ее, Адам дал ей имя~--- Ева, что
значит жизнь. Посему Бог повелел чрез Моисея в
книге Левит, именно, ради дней пребывания их вне
рая по сотворении, чтобы она (женщина) оставалась
нечистою при рождении мальчика сорок, а при
рождении девочки восемьдесят дней; посему и Адам
в сороковой день по сотворении введен был в рай,
ради чего и новорожденных в сороковой день
приносят в храм по закону. При рождении же
девочки она должна быть нечистою восемьдесят
дней, ради того, что она (Ева) вошла в рай в
восьмидесятый день, и ради женской нечистоты в
отношении к мужу; даже и находящаяся в месячном
очищении не входит в храм до семи дней по
божественному закону.

Это я ради любознания в сокращении заимствовал
из так называемой Жизни Адама (Кн.Юбил.,III).

Из Малого Бытия:

В седьмой год он согрешил, и в осьмой они были
изгнаны из рая, как говорит (Малое Бытие), чрез
сорок пять дней после падения, в восход Плеяд.

Пробыл же Адам в раю седмину трехсот
шестидесяти пяти дней; и изгнан был с женою Евою
за преступление заповеди в десятый день месяца
мая.

Звери, и четвероногие и пресмыкающиеся, говорит
Иосиф и Малое Бытие, до падения говорили одним
языкам с первосотворенными; посему, говорит, змей
беседовал с Евою человеческим голосом, что,
кажется, невозможно.

В восьмой год (говорит) Адам познал Еву, жену
свою.

В восьмидесятый род родился у них первородньм
сын Каин.

В семьдесят седьмой год, говорят, родился
праведный Авель.

В восемьдесят пятый год родилась у них дочь, и
они дали ей имя Асуам.

В девяносто седьмой год Каин принес жертву.

В девяносто девятый год Авель принес жертву
Богу, имея от роду двадцать два года, в полнолуние
седьмого еврейского месяца, то есть в праздник
кущей.

Достойно примечания, что Писание называет
жертву Каина принесением плодов, а жертву Авеля
дарами, обозначая сим настроение каждого.

В тот же девяносто девятый год Каин убил Авеля,
и первозданные оплакивали его четыре седмины, то
есть двадцать восемь лет.

В сто двадцать седьмой год Адам и Ева
прекратили свой плач. В сто тридцать пятый год
Каин взял собственную сестру Асавнан, которой
было пятьдесят лет; а сам он был шестидесяти пяти
лет (Кн.Юбил.,III,IV).

В двести тридцать четвертом году он родил дочь,
которой дал имя Азуран (Кн.Юбил.,IV).

В четыреста двадцать пятом году Сиф взял в жены
собственную сестру Азуран; Сиф же был девяносто
одного года (Кн.Юбил.,IV).

В том же девятьсот тридцатом году умер и Каин от
обрушившегося на него дома; ибо и сам он камнями
убил Авеля (Кн.Юбил.,IV).

В этом 2251 году, как говорят, Ной насадил
виноградник на горе Лувар в Армении (Кн.Юбил.,VII).

Ангел, говоривший с Моисеем, сказал ему: я
научил Авраама еврейскому языку, каким он был от
начала творения, чтобы он говорил на нем, как на
природном, о чем говорится в Малом Бытии (XII).

В 3373 году от сотворения мира, когда Аврааму был
61 год, сожег Авраам идолов отца своего, и вместе с
ними сожжен был Арран, хотевший тушить огонь
ночью. И вышел Фарра с Авраамом, чтобы идти в
землю Ханаанскую, и, переменив намерение, жил в
Харране, предаваясь идолопоклонству до своей
смерти (Кн.Юбил.,XII).

В сто пятьдесят третьем году жизни Исаака,
Иаков возвратился к нему из Месопотамии. И Исаак,
возведя очи и увидя сыновей Иакова, благословил
Левия, как первосвященника, и Иуду, как царя и
начальника. Ревекка побудила Исаака, уже бывшего
в старости, чтобы он внушил Исаву и Иакову любить
друг друга. И он, увещевая их, предсказал, что если
Исав восстанет на Иакова, то впадет в руки его. И
вот после смерти Исаака, Исав, возмущаемый своими
сыновьями, собрав людей, вышел войною против
Иакова и его сыновей. Иаков, заперев ворота башни,
увещевал Исава вспомнить родительские
завещания. Когда же он не склонился на увещания, а
напротив, стал оскорблять и поносить его, Иаков,
побуждаемый Иудой, натянул лук и поверг Исава,
поразив его в правый сосок груди. После его
смерти сыновья Иакова, открыв ворота, перебили
весьма многих. Это говорится в Малом Бытии
(XXXVII,XXXVIII).

\chhdr{4. Михаил Глика}
Не потому, что он (змий) имел прежде ноги, как
говорит Иосиф и так называемое Малое Бытие,
теперь Бог объявляет, что он будет ходить на
чреве; но, как объясняет Златоустый Иоанн, прежде
он благодаря прямому положению имел такую
смелость, что приблизился к самому уху Евы и
разговаривал с ней, а теперь осужден, и
совершенно справедливо, ползать по земле
(Кн.Юбил.,III).
Малое Бытие говорит, что Адам неосмотрительно
взял от древа и ел и не обратил полного внимания
на слова Евы, потому что изнемог от труда и
голода. Но об этом, возлюбленный, лучше умолчать,
ибо, как сказано выше, бывает нечто достойное и
молчания; разве только и ты хочешь говорить, что
Адам взял жену, чтобы не обратиться на других
животных. Змий стал пресмыкающимся из скота, и
имел руки и ноги; но это было отнято ради того, что
он дерзновенно вошел в рай и посему первый взял
от древа и ел. Адам отгонял птиц и пресмыкающихся,
собирал плод в раю и ел его с своею женою. Вот
это-то, чтобы не сказать, и еще гораздо большее из
подобного, содержит Малое Бытие. Но оставь это;
ибо иначе относящимся к Священному Писанию (это)
покажется, напротив, смешным и забавным (Кн.Юбил.,III).

\chhdr{5. Георгий Кедрин}
В Малом Бытии говорится, что Мастифат,
начальник демонов, приблизясь к Богу, сказал Ему:
если Авраам любит Тебя, пусть принесет Тебе в
жертву сына своего (Кн.Юбил.,XVII).
Ревекка, приготовив кушанье, отдала его Иакову
и ввела его вместе с другими дарами для Исаака к
Аврааму; взяв его на свое лоно и многообразно
благословив его, Авраам, почивши, умер, на
пятнадцатом году жизни Иакова (Кн.Юбил.,IXX).
В Малом Бытии говорится, что израильские дети
были бросаемы в реку только в течение десяти
месяцев, пока Моисей не был поднят царицею.
Посему на египтян были посланы десять казней в
течение десяти месяцев, и наконец они были
ввергнуты в море, по образу того, как они погубили
в реке еврейских детей,~--- за одного израильского
мальчика тысяча погубленных сильных мужей из
египтян. Самого же Моисея дочь Фараона усыновила
в царском достоинстве, но, конечно, не освободила
израильтян от порученной им работы
(ср.Кн.Юбил.,XLVII,XLVIII).
Моисей первый написал законы для иудеев.
Оставив занятия, соответственные Египту, Моисей
в пустыне изучал мудрость, получая откровения от
архангела Гавриила о происхождении мира и
первого человека, о бывшем после него, о потопе, о
смешении и многообразии языков, о событиях из
жизни первого человека, о происшествиях до его
времени, о законе, который он должен был дать
народу иудейскому, также о положении звезд, о
стихиях, арифметике, геометрии и всякой мудрости,
как говорится в Малом Бытии (ср.Кн.Юбил.,I).

\bibbookdescr{Asn}{
  inline={Книга Иосифа и Асенефи},
  toc={Иосиф и Асенефь},
  bookmark={Иосиф и Асенефь},
  header={Иосиф и Асенефь},
  abbr={Асн}
}
\vs Asn 1:1
И было в 1-ый год 7-ми лет изобилия, в 3-ий месяц, в 5-ый день месяца.
\vs Asn 1:2
И послал фараон Иосифа обойти всю страну Египетскую.
\vs Asn 1:3
И в 4-ом месяце 1-го года, в 18-ый день месяца
он прибыл в пределы Илиополя, и он собрал
пшеницы полей края того, как песок морской.
\vs Asn 1:4
И был муж в том городе, сатрап фараона;
и был он поставлен над всеми сатрапами, и превосходил разумом
всех вельмож фараоновых.
\vs Asn 1:5
И был муж тот весьма богат,
и был он советником фараона,
и было имя ему Потифер, жрец илиопольский.
\vs Asn 1:6
У него была дочь, около
18-ти лет от роду, дева высокого роста и прекрасная лицом,
превосходившая бывших на земле.
\vs Asn 1:7
В ней не было никакого
сходства с дщерями египтян; она во всём походила на дочерей Еврейских:
была она высока, как Сарра, и благообразна,
как Ревекка, и прекрасна, как Рахиль.
И было имя девы той Асенефь.
\vs Asn 1:8
И слава о красоте её прошла по всей земле той,
и даже до пределов земли той; искали её руки и сыновья всех
сатрапов, и сыновья вельмож, и все царственные юноши, и военачальники;
\vs Asn 1:9
и разделяла их всех ревность и вражда из-за Асенефи,
и они готовы были из-за неё воевать между собою.
\vs Asn 1:10
И первородный сын фараона,
услышав о ней, стал просить отца своего дать её ему в жёны,
и говорил отцу:
дай мне в жёны Асенефь, дочь Потифера, жреца илиопольского.
\vs Asn 1:11
И ответил ему отец его, фараон:
зачем домогаешься ты жены ниже тебя?
не ты ли царь всей вселенной?
Ведь обручена уже с тобою дочь моавитского царя,
царевна красоты отменной; её и бери в жёны.

\vs Asn 2:1
И Асенефь уничижала и презирала всякого мужа
и была очень горда и надменна в отношении всех.
Никакой муж никогда не видел её.
\vs Asn 2:2
При доме Потифера была башня,
весьма великая и высокая,
и в ней горница, имевшая 10 комнат,
где она и жила, никем не видимая.
\vs Asn 2:3
И была 1-ая комната велика и благолепна:
пол её был выложен каменьями порфировыми;
\vs Asn 2:4
и была та горница убрана мрамором;
её стены были унизаны драгоценными блестящими камнями;
под кровом её поставлены были боги египетские,
золотые и серебряные, без числа.
\vs Asn 2:5
И всех их почитала Асенефь,
боялась их,
всегда приносила им жертвы всесожжения и фимиам.
\vs Asn 2:6
И 2-ая комната была хранилищем всего убранства
Асенефи и всех ларцов её с золотом, серебром,
златоткаными ризами,
превосходными дорогими камнями,
всем девичьим её убранством.
\vs Asn 2:7
И 3-ья комната содержала все блага земные
и служила Асенефи кладовой.
\vs Asn 2:8
Остальные же 7 комнат отданы были 7-ми девам,
по одной каждой.
И девы эти служили Асенефи, одного года,
родившиеся в одну ночь с нею.
\vs Asn 2:9
И все они были прелестны, как звёзды небесные.
С ними никогда не говорил муж, ни даже дитя мужского пола.
\vs Asn 2:10
В комнате, где охранялось девство Асенефи,
были 3 больших окна.
И 1-ое, самое большое, выходившее на двор,
было обращено на восток;
2-ое глядело на юг,
а 3-ье на север, где прямая дорога.
\vs Asn 2:11
В комнате, выходившей на восток,
утверждено было золотое ложе,
убранное золотой пурпуровой тканью,
украшенное иакинфом и виссоном.
\vs Asn 2:12
На ложе том почивала Асенефь,
и не сидел на ложе том ни один муж с женой,
кроме одной Асенефи.
\vs Asn 2:13
Обширный двор окружал комнаты,
а двор высокие четырёхугольные стены из больших камней.
\vs Asn 2:14
Входили во двор 3-мя железными воротами,
которые охранялись 8-ью сильными вооруженными мужами.
\vs Asn 2:15
На дворе, вдоль стены,
росли различные красивые
плодовые деревья со спелыми на них плодами,
ибо наступила пора урожая.
\vs Asn 2:16
На правой стороне двора был большой источник,
в\acc{о}ды которого, стекались в водоём; от водоёма исходил ручей,
бегущий посреди двора, орошая находившиеся там деревья.

\vs Asn 3:1
И было на 1-ом году 7-ми лет изобилия,
в 4-ый месяц, в 18-ый день месяца,
когда Иосиф вступил в пределы илиопольские
для собирания хлеба во время изобилия.
\vs Asn 3:2
Приблизившись к тому городу,
Иосиф послал перед лицом своим 12 мужей к жрецу Потиферу сказать:
\vs Asn 3:3
Сегодня я остановлюсь у тебя,
ибо вот полдень, час трапезы,
и солнечный жар усиливается,
и отдохну под сенью твоего дома.
\vs Asn 3:4
Потифер, услышав это,
возрадовался радостью великой, и сказал:
Да будет благословен Бог Иосифов,
внушивший ему посетить нас!
\vs Asn 3:5
И Потифер, призвав
домоправителя своего, сказал ему:
Поспеши, и устрой дом мой, и приготовить
большой обед, ибо Иосиф, сильный бог, ныне придёт к нам,.
\vs Asn 3:6
Асенефь, услышав, что отец
её и мать возвратились с поля наследия её, обрадовалась и сказала:
\vs Asn 3:7
пойду и увижу отца моего и
матерь, возвратившихся с поля наследия моего; то было время жатвы.
\vs Asn 3:8
И Асенефь поспешно надела на себя виссонную ризу златотканную,
шитую нитями иакинфовыми; опоясалась золотым поясом,
надела обручи на руки и обручи на ноги, дорогое ожерелье на шею и
усыпанную различными камнями обувь на ноги.
\vs Asn 3:9
И на всём её убранстве были начертаны
имена богов египетских, на ожерелье же её и на драгоценных камнях
вырезаны были лица идолов.
\vs Asn 3:10
И возложила она на голову венец,
и замкнула повязку вокруг висков своих,
а сверху покрылась летним покрывалом.

\vs Asn 4:1
И поспешила она, и спустилась
по лестнице из своей горницы навстречу отцу и матери
и поклонилась им с приветствием.
\vs Asn 4:2
И возрадовались Потифер и жена его радостью великой,
глядя на дочь свою; ибо видели её родители
её нарядившейся как невесту бога.
\vs Asn 4:3
И вынесли они всё добро,
что принесли они с поля наследия их, и дали дочери своей.
\vs Asn 4:4
И возрадовалась Асенефь о добре том,
при виде всех плодов винограда, смоквы и финика, и о гранатовых
яблоках, ибо всё было в поре той.
\vs Asn 4:5
И сказал Потифер дочери своей Асенефи: Дитя моё!
\vs Asn 4:6
---~Вот я, господин мой!
\vs Asn 4:7
И он сказал: Пойди, сядь между нами и скажу тебе слова мои.
\vs Asn 4:8
И села Асенефь между отцом своим и матерью.
\vs Asn 4:9
И взял Потифер правую руку дочери и, поцеловав её, сказал: Дитя моё!
\vs Asn 4:10
---~Да говорит господин мой и отец мой!
\vs Asn 4:11
И сказал Потифер:
Вот, Иосиф, сильный бог, сегодня придёт к нам: он повелитель всей страны
Египетской, ибо фараон поставил его над всеми своими владениями,
\vs Asn 4:12
и он спаситель всей нашей земли,
ибо доставляет хлеб всей стране нашей,
чем и избавит людей от предстоящего голода.
\vs Asn 4:13
Иосиф муж благочестивый, целомудренный,
скромный, и девственник, как ты ныне, муж, сильный в премудрости
и знании, ибо с ним дух Божий и благодать Господня.
\vs Asn 4:14
Итак, дитя моё,
приди и я отдам тебя ему в жёны и он будет тебе мужем навсегда.
\vs Asn 4:15
Асенефь, услышав слова отца
своего, побледнела и разлился по ней пот кровавый, обильный.
\vs Asn 4:16
С гневом посмотрев на отца, она сказала:
отец, господин мой!
Неужели по этим словам ты, как рабу,
отдашь меня человеку чужому, беглому, проданному в рабство?
\vs Asn 4:17
Не сын ли он пастуха из земли ханаанской?
Не он ли был уличён в том, что лёг с госпожою своею,
за что господин его бросил его в мрачную темницу,
откуда вывел его царь,
потому что тот истолковал его сон,
как толкуют старицы египетские?
\vs Asn 4:18
Нет, но я сочетаюсь с первородным сыном фараона,
ибо он царь всей земли.
\vs Asn 4:19
Услышав это, не стал
Потифер продолжать разговор с своею дочерью об Иосифе,
так как она ответила ему дерзко и гневно.

\vs Asn 5:1
И пришёл к Потиферу один из
отроков его, и говорит: вот, Иосиф у ворот двора нашего!
\vs Asn 5:2
И убежала Асенефь от лица
отца своего и матери, как только услышала, что они хотят отдать её за Иосифа,
взошла в горницу и вступила в свою комнату.
\vs Asn 5:3
И стала она у большого своего окна,
выходящего на восток, чтобы видеть Иосифа, входящего в дом отца её.
\vs Asn 5:4
И вышли Потифер, и жена его,
и все рабы его, и все слуги дома его Иосифу навстречу, и отверзли восточные
ворота двора.
\vs Asn 5:5
И въехал Иосиф, восседая на 2-ой колеснице фараоновой,
запряжённой 4-мя белоснежными конями,
все в золотых удилах; и вся колесница была из цельного золота.
\vs Asn 5:6
И Иосиф был облачён в белую прекрасную одежду
с пурпуровой накидкой из златотканого виссона,
с золотым венцом на главе.
\vs Asn 5:7
Вокруг венца вделаны были 12 драгоценных камней,
и на камнях 12 блестящих лучей из золота.
\vs Asn 5:8
В левой руке у Иосифа был жезл,
а в правой масленичные ветви с тучными плодами.
\vs Asn 5:9
И он вступил во двор,
и затворены были за ним все ворота.
\vs Asn 5:10
И муж\acc{и} и жёны остались за
воротами, ибо привратники заложили их и никому не дали входить.
\vs Asn 5:11
И пришли Потифер, и жена его,
и все сродники его, кроме дочери его Асенефи,
и пали на лицо своё и поклонились Иосифу.
\vs Asn 5:12
И Иосиф сошел с колесницы
своей, и они приняли его в свои объятия.
\vs Asn 6:1
И увидела Асенефь Иосифа и полюбила его сильною любовью:
и сокрушилось сердце её, и подкосились колени её,
и дрожь напала на всё тело её,
и великий страх напал на Асенефь,
и ужас овладел ею, и она сказала со вздохом:
\vs Asn 6:2
куда пойду я и куда сокроюсь от лица его?
или как взглянет на меня Иосиф, сын Божий?
ибо худое говорила я о нём.
куда бегу и укроюсь?
\vs Asn 6:3
Ибо всё сокрытое видит он
и ничто тайное не утаится от него
по причине великого света, пребывающего в нём.
\vs Asn 6:4
И ныне милостив будь ко мне, Бог Иосифа,
ибо в неведении говорила я слова лукавые.
\vs Asn 6:5
Что сделаю теперь я, несчастная?
Давно ли с презрением говорили о нём со мною отец мой и мать,
что идёт к нам сын пастуха из земли ханаанской~--- так
они отзывались об Иосифе!
\vs Asn 6:6
Ныне же само солнце с неба приходит
к нам в колеснице его, и вступает в наш дом.
\vs Asn 6:7
И я, неразумная, дерзкая,
негодная, с презрением дурно говорила о нём,
не ведая, что Иосиф сын богов;
\vs Asn 6:8
ибо невозможно родиться человеку с такой красотой,
и какая утроба произведёт такого светозарного человека!
\vs Asn 6:9
Я же, злополучная и неразумная,
худое говорила о нём со своим отцом!
\vs Asn 6:10
И теперь господин мой удалил меня от него;
ибо я по неведению худо отозвалась о нём;
пусть теперь мой отец отдаст меня
к нему в рабы в вечное услужение.

\vs Asn 7:1
И вступил Иосиф в дом Потифера и сел на седалище.
\vs Asn 7:2
И омыли ноги его, и приготовили ему трапезу особо:
ибо Иосиф не ел с египтянами,
считая осквернением вкушать с ними.
\vs Asn 7:3
И говорит Иосиф Потиферу и всем его сродникам:
кто эта женщина, которая стоит в горнице у окна?
пусть она удалится отсюда, из этого дома.
\vs Asn 7:4
Ибо Иосиф опасался беспокойства от неё;
ибо досаждали ему все жёны и дочери вельмож египетских,
желавшие возлечь с ним.
\vs Asn 7:5
При виде его они воспламенялись страстью к нему;
но Иосиф презирал их; и посланцев,
которых жёны египетские посылали
к нему с золотом и серебром и богатыми дарами,
он отсылал с бранью и угрозой.
\vs Asn 7:6
И говорил он перед Господом:
нет, не сотворю греха перед лицом Бога Израилева.
\vs Asn 7:7
И он всегда имел перед глазами образ отца своего,
Иакова, и не забывал заповедей отца своего,
который говорил Иосифу и всем сыновьям своим:
\vs Asn 7:8
берегитесь, сыны мои, жён иноплемённых,
не имейте с ними общения;
ибо общение с ними гибель для вас и осквернение.
\vs Asn 7:9
Вот почему Иосиф сказал:
Пусть та женщина удалится из этого дома.
\vs Asn 7:10
---~Господин! Та, которую ты видел в горнице,
не чужая женщина, но дочь наша и раба твоя:
\vs Asn 7:11
она дева, не видевшая мужа,
и никто из мужей ещё не видел её, кроме тебя сегодня.
\vs Asn 7:12
Если желаешь, она придёт поклониться тебе,
ибо дочь наша тебе сестра.
\vs Asn 7:13
И возрадовался Иосиф радостью великой,
когда Потифер сказал, что она дева
и что она ещё не видела мужа.
\vs Asn 7:14
Он подумал в мыслях своих, сказав сам себе:
если она дева, то должна ненавидеть всякого мужа
и не будет обременять меня.
\vs Asn 7:15
И говорит Иосиф Потиферу и всем сродникам его:
если дочь твоя дева, пусть она придёт, и так как она
сестра мне, то отныне я готов любить её как сестру свою.
\vs Asn 8:1
И взошла мать её в горницу,
и привела Асенефь, и поставила её перед Иосифа.
\vs Asn 8:2
И сказал Потифер Асенефи:
Дочь моя!
Приветствуй брата твоего; ибо он подобно тебе целомудрен по сей день
и ненавидит всякую жену чужую, как и ты всякого чужого мужа.
\vs Asn 8:3
И Асенефь сказала Иосифу:
радуйся, господин, благословенный Всевышнего Бога!
\vs Asn 8:4
И говорит Иосиф Асенефи:
да благословит тебя Господь, дающий жизнь всему!
\vs Asn 8:5
И сказал Потифер: Дочь моя!
Подойди и поцелуй брата своего.
\vs Asn 8:6
И когда подошла Асенефь поцеловать Иосифа,
простёр Иосиф десницу свою и, положив её на грудь её, сказал:
\vs Asn 8:7
Не подобает мужу богобоязненному,
который благословляет Бога живого устами своими,
который вкушает хлеб благословенный и животворящий,
который пьёт благословенную чашу бессмертия,
помазуется помазанием нетления,
\vs Asn 8:8
лобызать жену иноплемённую,
благословляющую своими устами мёртвых и немых идолов,
вкушающую с жертвенников их удавленину,
и пьющую на возлияниях их из чаши вино обмана,
и помазующуюся помазанием погибели.
\vs Asn 8:9
Но мужу богобоязненному надлежит лобызать
своих благочестивых, возлюбленных мать и сестру,
и всех из своего племени и народа,
и жену, делящую с ним ложе,
устами своими благословляющих Бога Живого.
\vs Asn 8:10
Так же и жене богобоязненной не подобает
лобызать чужого мужа, ибо это скверна перед Богом.
\vs Asn 8:11
И когда услышала Асенефь слова Иосифа,
сильно опечалилась; и стала она воздыхать,
и смотрела на Иосифа со страхом, и глаза её наполнились слезами.
\vs Asn 8:12
При виде этого Иосиф сжалился над нею,
ибо был Иосиф кроток и милостив, и боялся Бога.
\vs Asn 8:13
И он поднял десницу свою и
возложил её на голову её, и сказал:
\vs Asn 8:14
Господь, Бог отца моего
Израиля, сильный и Вышний Бог Иакова!
\vs Asn 8:15
Ты, который из мрака вызвал всё существующее к свету!
\vs Asn 8:16
Ты, который вывел из заблуждения к истине, из смерти к жизни,
\vs Asn 8:17
Господи, животвори и благослови деву сию, и обнови её духом твоим,
\vs Asn 8:18
и воссоздай её невидимой твоею рукою, и сообщи ей новую жизнь.
\vs Asn 8:19
И да вкушает она хлеб жизни, и да пьёт она от чаши благословения:
\vs Asn 8:20
приобщи её к народу твоему,
избранному тобою прежде мироздания,
\vs Asn 8:21
и да войдёт она в покой твой, уготованный тобою твоим возлюбленным,
\vs Asn 8:22
и да живёт она жизнью вечною!

\vs Asn 9:1
И возрадовалась Асенефь радостью великой
при благословении Иосифа,
и поспешно возвратилась в уединённую свою горницу,
и пала на своё ложе с воздыханиями.
\vs Asn 9:2
Ибо нашли на неё и радость,
и печаль, и страх, и трепет,
и сильный пот, когда услышала она те слова Иосифа,
что говорил он ей во имя Бога Всевышнего,
\vs Asn 9:3
и плакала она плачем великим и горьким:
раскаяние объяло её сердце при мысли о своих богах,
которым она служила; и она возненавидела всех своих идолов.
\vs Asn 9:4
Так она пробыла до наступления вечера.
\vs Asn 9:5
И Иосиф ел и пил, и по окончании трапезы приказал своим отрокам:
\vs Asn 9:6
Запрягите коней в колесницу:
вот, отхожу в путь и обойду город и страну эту.
\vs Asn 9:7
И сказал Потифер Иосифу:
Отдохни здесь, господин мой,
под кровом сим этот день, завтра поедешь в путь свой.
\vs Asn 9:8
И отвечал Иосиф:
Нет, отойду сегодня же,
ибо в сей день Бог начал творить свои создания.
\vs Asn 9:9
В 7-ой же день, когда снова наступит этот день,
возвращусь и я к вам и отдохну под кровом этим.

\vs Asn 9:10
И Иосиф отправился в путь,
а Потифер со всем своим семейством отправился в поле наследия своего.
\vs Asn 10:1
И в доме осталась одна Асенефь с 7-ью девицами:
тосковала она и плакала до заката солнца, не ела хлеба, и воды не пила.
\vs Asn 10:2
И когда наступила ночь, и все бывшие в доме заснули,
не спала одна только Асенефь.
\vs Asn 10:3
И вспоминая Иосифа, она плакала и сильно била себя в перси;
великий страх напал на неё, и начала она сильно дрожать.

\vs Asn 10:4
И когда всюду водворилась тишина,
Асенефь открыла дверь свою и спустилась с ложа своего, и сошла из
горницы тихонько по лестнице,
\vs Asn 10:5
и пришла к мельнице, и нашла
мельника спящим вместе с своими сыновьями, и поспешно сняла с дверей шерстяную
завесу,
\vs Asn 10:6
и насыпала в неё пепел из печи,
и понесла в горницу, и положила на пол,
и заперла дверь железным запором.
\vs Asn 10:7
И она начала громко рыдать и плакать.
\vs Asn 10:8
И услышали кормилица и сверстница её,
которую она любила больше всех дев, стенание госпожи своей,
\vs Asn 10:9
и пробудились от сна прочие девы,
и подошли к дверям Асенефи и нашли их запертыми.
\vs Asn 10:10
До них доходили плач и рыдания,
и они спросили: Что с тобою, госпожа наша Асенефь?
Чем ты огорчена?
 Отвори нам, чтоб мы увидели, что с тобою случилось.
\vs Asn 10:11
И Асенефь, не отпирая дверей, отвечала им изнутри, говоря:
Голова моя отяжелела и не нахожу покоя на ложе своём,
\vs Asn 10:12
нет силы отворить вам,
ослабели все члены мои, ни встать не могу, ни отпереть;
\vs Asn 10:13
разойдитесь по своим комнатам,
успокойтесь и дайте мне также успокоиться
и отдохнуть немного.
\vs Asn 10:14
И девы по слову её разошлись по своим комнатам.
\vs Asn 10:15
И встала Асенефь, тихонько отперла дверь
и пошла в другую комнату, где хранились ларцы с убранством её;
\vs Asn 10:16
и открыла ковчежец, и вынула из него
чёрное траурное платье,
которое она надевала,
оплакивая смерть первородного брата своего.
\vs Asn 10:17
И принесла Асенефь то траурное платье
в свою комнату и, положив его, заперла дверь запором.
\vs Asn 10:18
И поспешно совлекла Асенефь с себя
царственное своё одеяние и виссон,
и златотканную порфиру, и облеклась в чёрное;
\vs Asn 10:19
и развязала золотой пояс, и препоясалась вервием;
\vs Asn 10:20
и сложила венец и повязку с головы своей,
и запястья с рук и ног.
\vs Asn 10:21
И взяла она всё это и выбросила в окно,
выходившее на север.
\vs Asn 10:22
И поспешила Асенефь,
и взяла также золотых и серебряных богов,
которым не было числа, и разбила их намелко,
и выбросила их в окно из горницы нищим.
\vs Asn 10:23
И взяла Асенефь царственный свой ужин,
хлеб и рыбу, и мясо тельца,
и жертвы для богов своих, и чаши для вина,
в которых она совершала возлияния, и выбросила в окно.
\vs Asn 10:24
И бросила она всю пищу чужим собакам на съедение,
дабы ужин её, мясо агнцев, приготовленный для
идолов, не сделался пищею её собственных собак.
\vs Asn 10:25
Тогда Асенефь распорола шерстяную завесу,
наполненную пеплом, и посыпала им пол;
\vs Asn 10:26
и взяла вретище, и препоясала чресла свои;
и сняла покрывало с главы своей и расплела свои волосы,
и посыпала главу пеплом,
лежавшим на полу, и накрылась им, и пала ниц в пепел.
\vs Asn 10:27
И начала часто бить себя в перси руками,
рыдать и проливать горькие слёзы всю ночь до утра.

\vs Asn 10:28
И когда рассвело, восстала Асенефь и увидела,
и вот пепел под нею стал от слёз её как грязь болотная.
\vs Asn 10:29
И снова она пала на лицо своё
в пепел и пролежала до вечера, до заката солнца.
\vs Asn 10:30
И так делала Асенефь 7 дней,
в продолжение которых она
не переставала мучить и терзать себя:
7 дней она не вкусила хлеба и не пила воды.

\vs Asn 11:1
И было в 8-ой день, на рассвете,
когда начали кричать петухи
и собаки лаять на проходящих,
она подняла голову свою от пола
\vs Asn 11:2
(ибо члены её расслабели от
непринятия пищи в продолжение 7-ми дней),
и пала на колена и, опершись рукой о пол,
поникла головой.
\vs Asn 11:3
Волосы на голове у неё были
распущены, взъерошены, покрыты густым пеплом.
\vs Asn 11:4
Сложив руки, Асенефь оплакивала свою голову,
била себя в грудь, издавала глубокие вздохи, рвала себе
волосы, посыпая их пеплом.
\vs Asn 11:5
Таким-то образом Асенефь,
утруждая себя, изнемогла, лишилась сил
\vs Asn 11:6
и, обратившись к стене, села у окна,
выходящего на восток, и наклонила голову на грудь,
и положила руки на колена и оставалась безмолвною,
\vs Asn 11:7
ибо не нашлось слова на устах её:
в тесноте своей она в продолжение 7-ми дней не раскрывала уст.
\vs Asn 11:8
И сказала Асенефь в сердце своём:
Что мне делать?
Кто будет моим прибежищем?
К кому я обращусь?
\vs Asn 11:9
Я дева и сирота, всеми покинутая:
отец и мать меня возненавидели, потому что я возненавидела их богов,
я уничтожила, я бросила их на попрание людям;
за это возненавидели меня отец, и мать, и все мои сродники.
\vs Asn 11:10
Отец мой сказал:
Отныне Асенефь не назовётся нашей дочерью,
потому что она уничтожила золотых и серебряных богов наших.
\vs Asn 11:11
И вот, я стала ненавистной
в глазах людей, ибо надмевалась над всеми,
за коих сватали меня.
И теперь все обрадовались моему горю.
\vs Asn 11:12
Об этом только она думала и говорила:
\vs Asn 11:13
Господь Всевышний, Бог Иосифа!
Ты ненавидишь чествующих идолов мёртвых, немых и бездыханных;
ибо ты Бог мстительный и страшный богам чуждым.
\vs Asn 11:14
За это и меня возненавидел Бог,
что я чествовала идолов немых, бездыханных,
за то, что я восхваляла их, что я ела от жертвенного их мяса,
\vs Asn 11:15
уста мои осквернены их трапезой,
и я не имею права взывать к Господу, Богу неба и земли,
к Всевышнему Избавителю Иосифа.
\vs Asn 11:16
Ибо душа моя осквернена
жертвоприношениями и всесожжениями идолам.
\vs Asn 11:17
Слышала я, как говорили,
что Бог евреев Бог истинный, Бог живой,
Бог милостивый, долготерпеливый, многомилостивый,
не вменяющий человеку грехи, терпеливый к кающемуся,
не обличающий человека в тесноте его.
\vs Asn 11:18
Итак, дерзну, обращусь к нему,
сделаю его своим прибежищем,
исповедую ему все грехи мои,
изолью мольбы свои пред ним,
и он помилует меня.
\vs Asn 11:19
Быть может, он взглянет на горе
моё и сжалится надо мною, покинутою;
быть может, он, видя мои рыдания,
поможет мне,
\vs Asn 11:20
ибо он отец сирот и помощник угнетённых,
дерзну и я воззвать к нему~--- быть может, простит меня.

\vs Asn 11:21
И отвернулась Асенефь от стены
и обратилась к окну,
выходящему на восток, и стала на колена свои,
и подняла руки к небу;
\vs Asn 11:22
но страх напал на Асенефь,
и она не могла раскрыть уста свои и произнести имя Бога.
\vs Asn 11:23
И снова обратилась к стене
и села, и стала бить себя руками
в грудь и в голову неоднократно.
\vs Asn 11:24
И говорила она в сердце своём,
не раскрывая уст:
Несчастная я сирота уста мои осквернены жертвенным
мясом идолов и хвалением богов египетских.
\vs Asn 11:25
И хотя проливаю теперь слёзы
и покрываю голову пеплом,
но не могу устами своими хвалить
святое и страшное имя Бога,
боясь гнева его за призывание его имени.
\vs Asn 11:26
Итак, что делать мне, злосчастной?
Дерзну, обращусь к нему:
\vs Asn 11:27
если он в гневе своём низвергнет меня,
то он властен восстановить; если накажет, то может утешить;
при наказании может возобновить меня своею милостью;
\vs Asn 11:28
если грехи мои огорчат его,
то примириться со мною и отпустить все мои грехи.
\vs Asn 11:29
Итак, дерзну, открою уста свои,
обращусь к нему, может быть, сжалится и простит мои прегрешения.

\vs Asn 12:1
И встала Асенефь, отвернулась от стены,
стала на колена, воздела руки свои к востоку,
взглянула на небо и произнесла:
\vs Asn 12:2
Господь, Бог веков!
Ты создал всё, ты оживил всех тварей,
\vs Asn 12:3
ты вывел всё из небытия,
всё видимое из невидимого на свет,
\vs Asn 12:4
ты поднял небо и основал его
на ветрах, и землю утвердил на водах;
\vs Asn 12:5
ты поставил над бездною великие горы,
которые не тонут, но держатся на водах, как дубовый лист:
\vs Asn 12:6
горы те живые, ибо внимают гласу твоему,
Господи, ибо ты сообщаешь жизнь всем созданиям твоим.
\vs Asn 12:7
Господь, Бог мой!
На тебя уповаю, к тебе простираю мольбы мои,
тебе исповедаю грехи мои
и пред тобою открою беззакония мои:
\vs Asn 12:8
пощади меня, Господи,
ибо я во всём согрешила,
совершила преступления перед тобою, Господи!
\vs Asn 12:9
Я произносила недостойные речи,
оскверняла уста свои жертвенным мясом
и от трапезы богов египетских.
\vs Asn 12:10
Согрешила я, Господи,
согрешила и лукавое сотворила,
почитая идолов глухих и мёртвых по неведению;
\vs Asn 12:10
поэтому я недостойна к тебе обратиться с мольбами,
по причине прегрешений моих.
\vs Asn 12:11
Согрешила я, Господи, перед лицом твоим,
я, Асенефь, дочь жреца Потифера,
некогда гордая, надменная,
стоявшая выше всех богатством,
теперь стою как сирота, покинутая всеми.
\vs Asn 12:12
Тебе приношу, Господи, моление моё,
и к тебе взываю: спаси меня от гонящих меня
\vs Asn 12:14
Как испуганное дитя бежит к отцу,
тянется к нему, чтобы тот поднял его с пола,
и, раз уже в его объятиях,
оно крепко обхватывает руками
его шею и тут успокаивается;
\vs Asn 12:15
так и я, преследуемая со всех сторон,
к тебе, Господи, прибегаю:
\vs Asn 12:16
простри руку твою надо мною,
как отец чадолюбивый и милостивый,
и возьми меня с лица земли.
\vs Asn 12:17
Ибо вот старый свирепый лев преследует меня,
ибо он отец богов египетских, а идолы народов~--- дети львов;
\vs Asn 12:18
я же выбросила всех богов,
уничтожила их, а лев отец их,
разгневанный, хочет поглотить меня.
\vs Asn 12:19
Избавь меня, Господи, от когтей его
и спаси из пасти его, дабы он не схватил меня как волк,
\vs Asn 12:20
и не растерзал, и не бросил в огонь печи,
из огня в вихрь, который, охватив,
лишит меня зрения и низвергнет в бездну морскую,
\vs Asn 12:21
где поглотит меня великое чудовище морское,
существующее изначала, и где я погибну на вечные времена.
\vs Asn 12:22
Господи!
Спаси меня, прежде чем постигнет меня всё это;
\vs Asn 12:23
спаси и укрепи меня покинутую,
ибо отец и мать отреклись от меня, сказав:
Асенефь не дочь нам, она уничтожила богов наших,
отвергла их.
\vs Asn 12:24
Ты один, Господи, надежда моя, на тебя уповаю,
ибо ты отец сирот и защитник гонимых, и притесняемых
покровитель.
\vs Asn 12:25
Помилуй меня, деву.
Ты милостив, как отец;
ты жалостлив, как мать;
ты долготерпелив как никто.
\vs Asn 12:26
Ибо вот, всё наследие,
данное мне отцом моим Потифером,
тленно и скоротечно;
твои же дары непреходящи, вечны.
\vs Asn 12:27
Теперь я отреклась от всего
и ото всех, покинула все блага земные.
\vs Asn 12:28
Тебя одного сделала своей надеждой.
\vs Asn 12:29
Одевшись во вретище,
покрывшись пеплом, оплакиваю грехи мои.
\vs Asn 12:30
Призри на сиротство моё,
Господи, ибо к тебе прибегла я.
\vs Asn 12:31
Вот, бросила я царственную
ризу свою златотканную,
виссон и серьги драгоценные и облеклась в хитон чёрный.
\vs Asn 12:32
Вот, сняла с себя золотой пояс
и препоясалась вервием и вретищем.
\vs Asn 12:33
Вот, бросила с головы венец
и покрыла главу мою пеплом.
\vs Asn 12:34
Вот, мраморные полы моего жилища,
прежде убранные разноцветными каменьями и пурпуром,
блестящие чистотой,
теперь омочены моими слезами и омрачены пеплом.
\vs Asn 12:35
Вот, Господь мой, от пепла и слёз плача моего
грязь великая соделалась в чертоге моём,
как на пути проезжем.
\vs Asn 12:36
Вот, я отказалась от царского моего ужина
и бросила его чужим собакам на съедение.
\vs Asn 12:37
И вот, 7 дней и 7 ночей не ела я хлеба, не пила воды;
\vs Asn 12:38
уста у меня высохли,
как кожа тимпана;
язык мой стал как рог;
губы мои сделались как черепица,
лицо моё осунулось;
очи опухли от беспрерывных слёз,
все силы оставили меня.
\vs Asn 12:39
Теперь, узнав, что боги,
чествуемые мною по неведению,
были немые и глухие идолы,
я отдала их на попрание,
серебро и золото растаскали воры и унесли с глаз моих;
\vs Asn 12:40
потому что надежду свою я положила на тебя, Господь, Бог Иосифа!
\vs Asn 12:41
Прости меня, ибо всё сделала я по неведению,
я порицала господина моего Иосифа, не зная,
что он сын возлюбленный у тебя.
\vs Asn 12:42
Мне сказали легкомысленные люди про него,
что он сын пастуха земли ханаанской, и я поверила им.
\vs Asn 12:43
И заблуждалась, и с презрением отнеслась
к избраннику твоему и стала говорить о нём непочтительно,
не зная, что он сын твой.
\vs Asn 12:45
Ибо кто из людей породил такую красоту,
и кто есть другой столь мудрый и сильный, как Иосиф?
\vs Asn 12:46
Ты одарил его дивной красотой, великой мудростью и добродетелью.
\vs Asn 12:45
Но, Господь мой, тебе
поручаю его, ибо возлюбила его я больше души моей.
\vs Asn 12:46
Сохрани его премудростью и благодатью твоею
и предай меня в рабы ему, и буду я омывать ноги его и оправлять
ему ложе, и служить ему во все дни жизни моей.

\chhdr{Исповедь Асенефи перед Богом.}
\vs Asn 13:1
Согрешила я перед тобою, согрешила, Господи!
Совершила беззаконие я, Асенефь, дочь Потифера,
жреца илиопольского, главного смотрителя над всеми богами.
\vs Asn 13:2
Согрешила я перед тобой, Господи, согрешила,
совершила беззаконие, почитала богов, коим нет числа,
и ела от жертвенного их мяса.
\vs Asn 13:3
Согрешила я, согрешила, Господи, перед тобой,
совершила беззаконие; ибо я была дева гордая, надменная.
\vs Asn 13:4
Согрешила я, Господи, согрешила перед тобой,
совершила беззаконие: я ела хлеб удушающий,
пила чашу сетей, вкушая от стола смерти.
\vs Asn 13:5
Согрешила я, Господи, согрешила перед тобой,
совершила беззаконие: не знала я Господа, Бога небес, не
надеялась на Всевышнего живого Бога веков.
\vs Asn 13:6
Согрешила я, Господи, согрешила перед тобой,
совершила беззаконие: я надеялась на величие своей
славы, на красоту свою; была я горда и надменна.
\vs Asn 13:7
Согрешила я, Господи, согрешила перед тобой,
совершила беззаконие, презирая всех людей, из которых ни
одного не считала я человеком.
\vs Asn 13:8
Согрешила я, Господи, согрешила перед тобой,
совершила беззаконие; много раз говорила, что нет на земле
князя, достойного развязать девственный мой пояс.
\vs Asn 13:9
Согрешила я, Господи, согрешила перед тобой,
совершила беззаконие: я ненавидела всех желавших взять
меня в жёны, я презирала их и порицала.
\vs Asn 13:10
Согрешила я, Господи, согрешила перед тобой,
совершила беззаконие, но по твоему милосердию я
сделаюсь невестой сына великого государя~--- отпусти мне грехи мои.
\vs Asn 13:11
Когда пришёл Божий воин Иосиф,
он низложил мою гордость, он усмирил меня,
и уловил красотою своею,
и мудростью своею поймал меня, как рыбку в сети,
\vs Asn 13:12
душою своею предложил мне лекарство жизни,
силою своею утвердил меня и посвятил меня Богу веков.
\vs Asn 13:13
Он дал мне есть хлеб жизни и пить чашу бессмертия.
\vs Asn 13:14
Он напоил меня, и я стала его невестой вовеки.

\vs Asn 14:1
И когда Асенефь прекратила беседу с Господом,
вот, взошла денница на небеса от страны восточной.
\vs Asn 14:2
И увидела её Асенефь, и возрадовавшись, сказала:
Внял Бог молитве моей; ибо вот, светило,
предвестник великого дня, явилось.
\vs Asn 14:3
И вот видит Асенефь подле денницы разверзлось небо,
и показался свет неизреченный.
\vs Asn 14:4
Видя это, она пала лицом на пепел.
\vs Asn 14:5
И сошло с неба подобие мужа,
и стал он перед головой Асенефи, и стал звать её: Асенефь!
\vs Asn 14:6
И сказала она: Кто это звал меня?
Двери моей горницы заперты, башня моя высока,
кто это осмелился войти в мой чертог!
\vs Asn 14:7
И тот муж вторично позвал её, и сказал: Асенефь, Асенефь!
\vs Asn 14:8
И спросила Асенефь: Возвести мне, кто ты?
\vs Asn 14:9
---~Я князь Израилев и архистратиг
всего воинства Всевышнего.
Встань, становись на ноги, и я поведаю слово.
\vs Asn 14:10
И Асенефь, подняв голову, увидела мужа,
и вот, во всём подобен он Иосифу:
и одеждою, и венцом, и царским жезлом;
\vs Asn 14:11
лицо же его словно молния,
глаза как солнечные лучи,
волосы на голове как пламя,
а руки как раскалённое железо,
от рук и ног его сыпались искры,
как от пламенеющего огня.
\vs Asn 14:12
При виде этого Асенефь пала на лицо своё на землю,
и ужас объял её, и дрожь проникла до костей членов её.
\vs Asn 14:13
И тот муж сказал ей:
Мужайся, Асенефь, не бойся,
но встань на ноги свои, и я обращу к тебе мои слова.
\vs Asn 14:14
И сказал тот муж:
Пойди, сними с себя хитон чёрный,
выражение печали, и отложи вретище с чресл твоих,
и отряхни прах с головы твоей;
\vs Asn 14:15
и омой лицо своё живою водою,
и облекись в ризу новую, нетронутую,
и опояшь чресла твои двойным
золотым поясом девства твоего,
и приди, и тогда я скажу тебе слово.

\vs Asn 14:16
И Асенефь поспешно
удалилась во 2-ую свою комнату,
где были корзины с её убранствами.
\vs Asn 14:17
И открыла она ковчежец свой,
и вынула полотняное дорогое платье,
до которого никто ещё не касался,
и сняла с себя чёрное траурное платье
и надела платье новое.
\vs Asn 14:18
И сняла она вервие и вретище с чресл своих,
и опоясала 2-мя поясами своего девства:
одним стан свой, а другим грудь свою с сосцами.
\vs Asn 14:19
И отряхнула прах с головы своей,
и умыла руки и лицо своё водою чистою;
и взяла она чистый полотняный покров
и покрыла им свою голову.

\vs Asn 15:1
И пришла она к мужу тому в 1-ую комнату,
и муж тот, увидев её, сказал:
\vs Asn 15:2
Сними с головы своей покрывало,
зачем ты надела его сегодня?
ибо ты до сей поры чистая и скромная дева,
и голова твоя как голова юноши.
\vs Asn 15:3
И сняла Асенефь покрывало с головы своей.
И сказал ей тот муж:
\vs Asn 15:4
Мужайся, чистая дева Асенефь!
Вот я внял исповеди твоей и молитвам,
вот я увидел 7-дневные тяжкие твои лишения;
\vs Asn 15:5
теперь я своими глазами вижу тебя,
стоящую на пепле и проливающую слёзы:
мужайся, чистая дева Асенефь!
\vs Asn 15:6
Ибо имя твоё вот уже написано
на небесах Божьими перстами
в книги живых с именами изначала вписанных
и пребудет оно там неизгладимо во веки веков.
\vs Asn 15:7
Отныне ты обновишься:
ты вкусишь от животворящего,
благословенного хлеба
и пить будешь от чаши благословения
и бессмертия;
и умастишь себя чистым елеем.
\vs Asn 15:8
Мужайся, чистая дева Асенефь!
Вот я отдаю ныне тебя Иосифу в невесты навсегда,
и отныне имя тебе будет не Асенефь, но Город убежища;
\vs Asn 15:9
ибо через тебя многие народы прибегнут к Господу,
Богу небес, и под сенью крыл твоих укроются
уповающие на Господа Бога;
\vs Asn 15:10
и за стенами твоими будут защищены,
притекающие к Всевышнему через покаяние,
ибо покаяние есть дщерь Всевышнего,
и она предстательствует на всякий час
пред Всевышним за тебя и за всех кающихся.
\vs Asn 15:11
Ибо он есть Отец покаяния,
она же есть матерь дев, и на всякий час молит его о кающихся;
\vs Asn 15:12
ибо возлюбленных своих
вознесет Бог, который есть податель даров, подкрепитель всех дев.
\vs Asn 15:13
Ему угодно девство, он ищет его,
он заботится о нём всегда.
\vs Asn 15:14
Кающихся он принимает под сень свою,
готовит для них на небесах место покоя, и они навеки будут под
покровом его.
\vs Asn 15:15
И есть покаяние весьма прекрасная дева,
чистая, и непорочная, и кроткая, и Бог Всевышний любит её, и
все ангелы почитают её.
\vs Asn 15:16
И вот, я иду к Иосифу,
и буду говорить с ним о тебе, и он войдёт к тебе сегодня, и увидит тебя, и
возлюбит тебя, и будет женихом твоим, ты же будешь ему невестой.
\vs Asn 15:17
Итак, внимай, о ты, дева Асенефь!
Облекись в брачную ризу, в ризу древнюю, приготовленную для тебя из
начала в покое твоем;
\vs Asn 15:18
и надень на себя всякий убор твой избранный,
и укрась себя нарядами, как добрую невесту, и ты пойдёшь навстречу Иосифу.
\vs Asn 15:19
Ибо вот, он сегодня приблизится к тебе,
и увидит тебя, и возрадуется.
\vs Asn 15:20
И когда тот муж кончил речь свою,
Асенефь возрадовалась радостью великою,
и пала она на лицо своё,
поклонилась ему и сказала:
\vs Asn 15:21
Благословен Бог Всевышний,
пославший тебя избавить меня от мрака и возвести к свету, и
извлёкший меня из глубины бездны.
\vs Asn 15:22
Скажи имя твое, господин!
Поведай мне, дабы я могла благословлять его навеки.
\vs Asn 15:23
Отвечает ей муж тот:
Имя моё написано на небесах в книге
Всевышнего Божьими перстами прежде,
чем были написаны все имена.
\vs Asn 15:24
Я Князь Всевышнего,
и имена, вписанные в книгу Всевышнего,
не подлежат ни исследованию, ни слуху, ни
зрению человека в этом мире.
\vs Asn 15:25
Асенефь сказала: Если я
обрела благодать пред тобою и разумею все слова, сказанные мне тобою, то
позволь рабе твоей сказать пред тобою слово.
\vs Asn 15:26
И сказал тот муж: Говори.
\vs Asn 15:27
И говорит Асенефь: Прошу тебя, господин.
\vs Asn 15:28
Произнося эти слова, она приблизилась к его руке
и с такой мольбой: Сядь на малое время на этом ложе:
оно чисто и не осквернено, ибо на нём не сидели ещё муж и жена.
\vs Asn 15:29
Я поставлю пред тобою стол
и принесу из моей кладовой хлеб, и вкусишь ты от него;
\vs Asn 15:30
и старое вино, благовоние
которого до небес, и будешь ты пить от него,
и отправишься в путь свой.
\vs Asn 16:1
И сказал ей муж тот: Иди и принеси скорее.
\vs Asn 16:2
И Асенефь поспешно принесла
и поставила перед ним пустой стол,
и выйдя от него, она готова была уже войти в
кладовую за хлебом,
\vs Asn 16:3
когда он сказал: Принеси мне мёд сотовый.
\vs Asn 16:4
Асенефь остановилась, и она опечалилась,
потому что не было у неё сотового мёда в кладовой.
\vs Asn 16:5
И спросил её тот муж: Что же ты остановилась?
\vs Asn 16:6
И отвечала Асенефь: Пошлю отрока за город,
недалеко отсюда поле наследия нашего, он тотчас принесет оттуда
медовые соты, и я поставлю их перед тобою, господин.
\vs Asn 16:7
И сказал ей тот муж: Войди в свою кладовую
и ты найдешь медовые соты на столе, возьми их и принеси.
\vs Asn 16:8
И взошла Асенефь в свою кладовую и нашла на столе соты.
\vs Asn 16:9
И были ячейки тех сотов большие,
и белые как снег, и полные мёдом.
И ячейки эти подобны были небесной
росе, и запах от них благоухание жизни.
\vs Asn 16:10
Асенефь изумилась и подумала:
уж не из уст ли этого мужа вышли эти соты,
так как запах от них, как от этого мужа?
\vs Asn 16:11
И взяла Асенефь медовые соты,
принесла и поставила их на пустой стол перед тем мужем.
\vs Asn 16:12
И спросил тот муж: Как же это ты сказала,
что нет у меня в кладовой медовых сотов, а между тем,
вот, принесены оттуда эти соты?
\vs Asn 16:13
И, смутившись, сказала Асенефь:
У меня, господин, не было сотов медовых в кладовой; но в то время,
как ты заговорил, быть может, из уст твоих они изошли;
ибо запах от них как запах от уст твоих.
\vs Asn 16:14
И улыбнулся тот муж, видя разумность Асенефи.
\vs Asn 16:15
И подозвав её к себе, он простёр правую руку свою к голове её.
\vs Asn 16:16
И Асенефь испугалась руки того мужа,
ибо искры сыпались из уст его, как от раскалённого железа,
\vs Asn 16:17
она, устремив глаза,
смотрела на его руку, а он при виде этого, улыбнувшись, сказал:
\vs Asn 16:18
Блаженна ты, Асенефь;
ибо неизречённые тайны Всевышнего Бога открылись тебе!
Блаженны и те, кои предстанут пред Господа с покаянием;
\vs Asn 16:19
они вкусят от медовых этих
сотов, дающих жизнь; ибо их приготовили пчёлы рая места сладости;
\vs Asn 16:20
они приготовили их из росы
живых райских роз, и ангелы Божьи вкушают от него, и все сыны Всевышнего, и
вкусивший от этих сотов не умрёт вовеки.
\vs Asn 16:21
И простёр тот муж правую руку,
отломил частицу от сотов,
и вкусил сам, и частицу рукою своею вложил ей в уста,
\vs Asn 16:22
говоря: Вот ты, Асенефь,
вкусила хлеб жизни, и пила чашу бессмертия,
и умастилась елеем непорочности.
\vs Asn 16:23
Отныне тело твоё
распускаться будет подобно цветку, выросшему на земле Всевышнего; кости твои
утучнятся подобно кедрам, растущим в раю сладости,
\vs Asn 16:24
и сила проникнет всю тебя,
и молодость твоя не увидит старости, и красота не покинет тебя вовеки,
\vs Asn 16:25
и будешь ты как город,
окружённый бойницами во имя Господа Бога, царя веков.
\vs Asn 16:26
И простёр тот муж руку к
отломанной части сотов, и соты сделались целы, как прежде.
\vs Asn 16:27
И он снова протянул правую
руку свою и перстом коснулся края сотов,
обращенного на восток, и обратил его на сторону,
выходящую на запад; и путь перста его принял кровавый вид.
\vs Asn 16:28
И он, простерши руку в другой раз,
коснулся ею края сотов, обращенного на север,
и обратил его на сторону, выходящую на юг,
и путь перста его имел вид кровавый.
\vs Asn 16:29
И стояла Асенефь слева, и
смотрела и видела всё, что делал он.
\vs Asn 16:30
И сказал тот муж медовым сотам: Приблизьтесь сюда.
\vs Asn 16:31
И вот из твердых сотовых
ячеек поднялись тысячи и тьмы белоснежных
пчёл с длинными пурпуровыми крыльями,
\vs Asn 16:32
у иных же крылья были как
виссон, унизанный дорогими камнями,
и как гиацинт, и как нити златые; их головки
были украшены золотыми венцами.
\vs Asn 16:33
Пчёлы эти были прекрасны на
вид и жала их были изострены.
\vs Asn 16:34
И покрыли Асенефь все пчёлы
роем с головы до ног; и иные пчёлы,
большие, как бы царицы роя, присели к устам Асенефь.
\vs Asn 16:35
Поднялись они из ячеек
своих, облепили всё лицо Асенефи
и стали работать на её лице, и отверстия ячеек
приходились к устам Асенефи.
\vs Asn 16:36
И тот муж сказал пчёлам:
Ступайте по своим местам.
\vs Asn 16:37
И поднялись все пчёлы и
улетели по направлению к небу.
\vs Asn 16:38
И те из них, которые жалить
хотели Асенефь, падали мёртвые.
\vs Asn 16:39
И тот муж, жезлом своим
прикоснувшись к мёртвым пчелам, сказал:
И вы восстаньте и ступайте на свои места.
\vs Asn 16:40
И встрепенулись они, и
полетели перед домом Асенефи,
и уселись на плодовых деревьях.
\vs Asn 17:1
И сказал тот муж Асенефи:
Видела ли слово сие?
\vs Asn 17:2
И отвечала она: Так, господин: всё сие я видела.
\vs Asn 17:3
И он сказал: Таковы будут слова мои ныне.
\vs Asn 17:4
И он в 3-ий раз простёр правую руку свою
к части медовых сотов и съел её, не повредив.
\vs Asn 17:5
И поднялся огонь от стола, и пожрал соты,
и запах от сжигаемых сотов наполнил собою горницу,
и запах был весьма приятен.
\vs Asn 17:6
И сказала Асенефь мужу:
Есть у меня 7 девиц однолеток, служащих мне, воспитанных со мною от
младенчества моего, родившихся в одну ночь со мною: я их люблю как сестёр;
позову их сюда, чтобы ты благословил их, как ты благословил меня.
\vs Asn 17:7
И сказал муж: Зови.
И, будучи позваны, они стали перед ним.
\vs Asn 17:8
И тот муж сказал:
Да благословит вас Бог Всевышний;
да будете вы 7-ью столпами для этого города,
и пусть почиет на вас Господне благословение вовеки.
\vs Asn 17:9
И сказал он Асенефи:
Переставь отсюда этот стол.
\vs Asn 17:10
И когда обратилась Асенефь,
чтобы поставить стол на его прежнее место,
муж тот сделался невидим.
\vs Asn 17:11
И увидела Асенефь подобие
колесницы, несущейся на восток; и колесница подобна была огню, кони её как
молнии, и на колеснице стоял муж тот.
\vs Asn 17:12
И сказала Асенефь:
Я, неразумная и дерзкая, позволила себе сказать смело,
что человек пришёл в мою горницу, не ведая,
что пришедший сегодня ко мне был
Господь небесный, который вот возвращается на своё место,
\vs Asn 17:13
и она присовокупила:
Помилуй, Господи, и сжалься надо мною рабою твоею, ибо в неведении я говорила
перед лицом твоим дерзновенно и неразумно.

\vs Asn 18:1
И между тем как Асенефь
погружена была в эти размышления, прибежал отрок из числа рабов Потифера и
сказал:
\vs Asn 18:2
Вот Иосиф, бог сильный,
едет к нам: его колесница уже перед нашим двором.
\vs Asn 18:3
Асенефь поспешно позвала
кормилицу свою, заведовавшую всем её имуществом, и сказала:
\vs Asn 18:4
Пойди, займись поскорее
убранством нашего дома и приготовь лучшую вечерю для бога сильного Иосифа, ныне
едущего к нам.
\vs Asn 18:5
Тут кормилица заметила, что
у Асенефи ланиты впали по причине 7-дневного воздержания от пищи, ей грустно
стало, и она заплакала, и, взяв её за правую руку, поцеловала,
\vs Asn 18:6
и спросила: Что с тобой, дитя?
Отчего такие впалые у тебя ланиты?
\vs Asn 18:7
Отвечала Асенефь:
Сильная головная боль посетила меня,
ночь провела без сна, вот отчего изменилась в лице.
\vs Asn 18:8
И пошла её кормилица убирать дом и готовить вечерю;
Асенефь же вспомнила слова того мужа, и поспешила
во 2-ую свою комнату, где в хранилищах лежали её наряды.
\vs Asn 18:9
И открыв большой ковчежец,
она вынула из него брачные свои одежды, превосходные, блестящие, и надела их.
\vs Asn 18:10
И опоясалась она золотым царским поясом,
украшенным различными камнями многоценными;
\vs Asn 18:11
и надела на руки и ноги
золотые запястья, на шею дорогие ожерелья, унизанные бесчисленными дорогими
каменьями;
\vs Asn 18:12
и надела на голову золотой
венец, унизанный с обеих сторон у чела 12-ью большими камнями;
\vs Asn 18:13
и набросила на голову
лёгкое покрывало, как подобает невесте; и взяла царский жезл в руку.
\vs Asn 18:14
И вспомнила Асенефь слова
своей кормилицы, что печально выражение лица твоего, воздохнула и с грустью
сказала: лицо моё горит, если Иосиф заметит это, ему не понравится.
\vs Asn 18:15
И, обратившись к подругам
своим, сказала: Принесите мне чистой, ключевой воды, умою лицо своё.
Принесли и налили воды в рукомойницу.
\vs Asn 18:16
И она наклонилась и
увидела в воде лицо своё, подобное солнцу, глаза свои как восходящую утреннюю
звезду, прелестные ланиты свои как части граната, уста свои как
распустившуюся розу и зубы, блестящие белизной.
\vs Asn 18:17
И Асенефь, созерцая себя в
воде в таком виде, возрадовалась радостью великой и стала умывать лицо своё.
\vs Asn 18:18
И когда пришла кормилица с
донесением об исполнении данных ей приказаний, взглянув на Асенефь, удивилась
так, что не могла опомниться в продолжение 2-ух часов:
так велико было её изумление!
\vs Asn 18:19
Она, став на колени, спросила:
Откуда эта великая, дивная красота, госпожа моя?
Вижу сам Господь, Бог небесный, избрал тебя быть невестой Иосифа.

\vs Asn 19:1
Между тем как они
беседовали, пришёл отрок из рабов и сказал Асенефи: Вот, Иосиф стоит у врат
двора нашего.
\vs Asn 19:2
Асенефь поспешно спустилась
по лестнице в сопровождении 7-ми дев навстречу Иосифу и стала в проходе дома.
\vs Asn 19:3
Иосиф вступил на двор;
ворота затворились, и чужой народ остался за воротами.
\vs Asn 19:4
Тогда Асенефь вышла из
прохода навстречу Иосифу.
\vs Asn 19:5
И увидев её, Иосиф был
поражен великой её красотой и спросил: Скажи, кто ты?
\vs Asn 19:6
И она ответила: Я раба
твоя, Асенефь, которая по повелению твоему выбросила всех своих идолов и
уничтожила.
\vs Asn 19:7
Сегодня приходил ко мне
некий муж, который дал мне хлеб жизни и вино благословения, сказав: вот Я отдаю
тебя как вечную невесту Иосифу, который будет твоим женихом навсегда;
\vs Asn 19:8
к этому он прибавил: отныне
ты будешь называться не Асенефь, а Городом прибежища; ибо через тебя
прибегнут многие народы к Богу Всевышнему.
\vs Asn 19:9
Он прибавил: я пойду к
Иосифу и поведаю в слух его слова мои о тебе.
\vs Asn 19:10
Ты знаешь уже, господин
мой; ибо тот муж приходил к тебе и говорил обо мне.
\vs Asn 19:11
И сказал Иосиф Асенефи:
Всевышний Бог благословил тебя; ибо Господь Бог утвердил стены твои на высоте,
стены же твои адамантовые, они стены жизни;
\vs Asn 19:12
ибо многие сыны
человеческие жить будут в твоем Городе прибежища, и Господь Бог воцарится в
нём вовеки.
\vs Asn 19:13
Итак, приди ко мне, дева
чистая; зачем так далеко стоишь от меня! Ибо благую весть о тебе принёс мне от
небес муж, поведавший мне всё, что было с тобою.
\vs Asn 19:14
И он, подняв руку, подозвал
Асенефь. И она подошла к Иосифу и пала ему в объятия.
\vs Asn 19:15
И оживились души у них и
исполнились радостью; и Иосиф, дав Асенефи лобзание, сообщил ей дух жизни, дух
премудрости, и дух истины.
\vs Asn 19:16
И, обнявшись, они долго
лобызали друг друга.
\vs Asn 20:1
Наконец, Асенефь сказала:
Пойди сюда, господин мой, взойди в наш дом; ибо я убрала наш дом и приготовила
великолепную вечерю.
\vs Asn 20:2
Она взяла его за руку правую
и ввела в дом свой, посадила на седалище отца своего и принесла воды, чтобы
омыть ноги его.
\vs Asn 20:3
И Иосиф сказал ей: Пусть
придёт одна из дев и умоет мои ноги.
\vs Asn 20:4
И отвечала Асенефь: Нет,
господин мой, отныне я раба твоя. С чего ты взял, что другая будет умывать
ноги твои? Ноги твои мои ноги, тело твоё моё тело.
\vs Asn 20:5
И она настояла на своём и
омыла ему ноги.
\vs Asn 20:6
И посмотрел Иосиф на её руки
и не мог наглядеться на их жизненность: пальцы у неё ходили, как у скорописца.
\vs Asn 20:7
Затем Асенефь, взяв его за
правую руку, облобызала её; а он поцеловал её в голову. И она села по правую
руку его.
\vs Asn 20:8
И пришли отец её и мать с
поля наследия своего, пришли и все сродники её и увидели Асенефь, как бы
окружённую светом (красота её была словно небесная), сидящую с Иосифом и
одетую в ризу брачную.
\vs Asn 20:9
При виде этого они
ужаснулись, поражённые её красотой, и они воздали славу Богу, который
животворит все.
\vs Asn 20:10
После этого они ели, и пили, и веселились.
\vs Asn 20:11
И сказал Потифер Иосифу:
Завтра ты пригласишь сановников и вельмож египетских, и я устрою свадьбу
вашу, и ты возьмёшь дочь мою Асенефь себе в жёны.
\vs Asn 20:12
И ответил Иосиф: Нет,
прежде я отправлюсь к фараону; ибо он для меня как отец, он меня поставил
князем над этой страной. Поведаю его слуху об Асенефи,
и он отдаст мне Асенефь в жёны.
\vs Asn 20:13
На это Потифер ответил:
Ступай с миром.
\vs Asn 20:14
Иосиф остался тот день у
Потифера и не вошёл к Асенефи;
\vs Asn 20:15
ибо говорил: Не подобает
мужу богобоязненному прежде брака почивать с женою своею.

\vs Asn 21:1
На утро Иосиф отправился к
фараону и сказал ему:
Отдай мне Асенефь, дочь Потифера,
илиопольского жреца, в жёны.
\vs Asn 21:2
Фараон сказал: Ведь она
твоя невеста и с давних пор обручена.
\vs Asn 21:3
И он послал вестников за
Потифером, который пришёл с Асенефью, и представил её перед фараона.
\vs Asn 21:4
И изумился фараон при виде
красоты её и сказал:
Да благословит тебя, дитя моё,
Бог Иосифа, избравшего тебя в невесту себе!
Да не покинет тебя красота твоя!
\vs Asn 21:5
Справедлив Господь,
избравший тебя для Иосифа, и как говорится, отныне наречёшься ты дочерью
Всевышнего, и будет тебе Иосиф женихом навеки.
\vs Asn 21:6
И фараон возложил на Иосифа
и Асенефь золотые венцы, взятые из царской сокровищницы.
\vs Asn 21:7
И фараон, поставив Асенефь
по правую руку Иосифа, возложил на их головы свои руки, правую руку на голову
Асенефь,
\vs Asn 21:8
и сказал: Да благословит
вас Всевышний Бог, и да прославит на вечные времена.
\vs Asn 21:9
И фараон повернул их лицом к
лицу, подвинул их близко и принудил лобызаться.
\vs Asn 21:10
После сего фараон сотворил брак,
устроил роскошную вечерю и пир, и винопитие великое в продолжение 7-ми
дней; и были приглашены все князья египетские, вельможи,
все цари соседних народов.
\vs Asn 21:11
И приказал царь возвестить
по всей земле Египетской, что если кто в продолжение 7-ми дней бракосочетания
Иосифа и Асенефи будет работать, тот умрёт горькою смертью.
\vs Asn 21:12
И было после сего, когда
совершилось торжество брачное и закончилось пиршество, Иосиф вошёл к Асенефи,
она зачала и родила Манассию в доме Иосифа.

\vs Asn 22:1
После этого прошло 7 лет изобилия, и настало 7 лет голода.
\vs Asn 22:2
И услышал Иаков о сыне своём Иосифе,
и прибыл в Египет со всеми сродниками своими на 2-ом году голода, в
21-ый день месяца нисана, и поселился в земле Гесем.
\vs Asn 22:3
И сказала Асенефь Иосифу:
Пойду я посетить отца твоего, ибо отец твой Израиль для меня как Бог.
\vs Asn 22:4
И сказал Иосиф: Ты пойдёшь со мною и увидишь отца моего.
\vs Asn 22:5
И пришли Иосиф и Асенефь в
страну Гесем; и встретились ему братья его,
и пали на лицо своё и поклонились
ему, в особенности же Асенефи.
\vs Asn 22:6
И вошли они к Иакову,
который сидел на ложе своём: и был он сед и очень стар.
\vs Asn 22:7
И Асенефь сильно поразил вид
его; ибо Иаков, несмотря на седину, был благовиден, как прекрасный юноша;
\vs Asn 22:8
глава его была бела как
снег, кудрявые волосы его были густы, как у хушитянина;
белая красивая борода его покрывала всю грудь его;
\vs Asn 22:9
глаза его веселы,
блестящие и красивые; грудь его, и плечи, и мышцы, и пальцы на руках как у
сильного ангела; бёдра и голени его как у нефилима.
\vs Asn 22:10
И представлялся Иаков как
муж боговидный.
\vs Asn 22:11
Асенефь, видя его, пришла в
ужас и пала пред ним на землю на лицо своё.
\vs Asn 22:12
И спросил Иаков: Эта ли
моя невестка, жена твоя? Да благословит её Всевышний Бог!
\vs Asn 22:13
И, подозвав её к себе,
облобызал и благословил её; также и Асенефь простёрла руки свои, и обвила ими
шею Иакова и повисла на раменах отца мужа своего, как бы возвратившегося с войны
целым и невредимым, и лобызала его.
\vs Asn 22:14
После того ели они и пили.
\vs Asn 22:15
И отправились Иосиф и
Асенефь в дом свой, и взяли с собою Симеона и Левия.
\vs Asn 22:16
Но не все сыновья Зелфы и
Валлы, Лии и Рахили провожали их, потому что завидовали им и были их врагами.
\vs Asn 22:17
С правой стороны Асенефи
шел Левий, а с левой Симеон. И Асенефь держала за руку Левия,
\vs Asn 22:18
ибо более всех братьев
Иосифа Асенефь возлюбила Левия, как мужа пророчествующего, и благочестивого, и
богобоязненного.
\vs Asn 22:19
Потому что он читал
письмена, написанные на небесах, и разумел их, и знал тайны Всевышнего, которые
он открывал ей в словах таинственных.
\vs Asn 22:20
И Левий сильно любил
Асенефь. Он видел место её упокоения в вышних, и окружавшие его стены были
словно адамантовые, и основания его как каменные основы 3-го неба.
\vs Asn 23:1
И было, когда возвращались
Иосиф и Асенефь путем своим, первородный сын фараона увидел её с высоты стены и
сильно возмутился духом при виде великой красоты её.
\vs Asn 23:2
И сказал он: Нет, не быть этому!
\vs Asn 23:3
И сын фараона отправил
гонцов призвать к себе Симеона и Левия, которые, пришедши, предстали
пред лицом его.
\vs Asn 23:4
И сказал им первородный сын
фараона: Я знаю, что вы силою превосходите всех людей на земле: вашею десницею
был сокрушён город Сихем, и 2-мя мечами вашими были истреблены 30000
мужей воинственных.
\vs Asn 23:5
И вот призываю я вас,
придите на помощь мне! Вот я приму вас в товарищи себе: дам я вам много золота
и серебра, и рабов, и рабынь, и дома, и большие уделы, и богатства;
только помогите мне, исполните это моё слово.
\vs Asn 23:6
Окажите мне милость, ибо я
поруган братом вашим Иосифом: он отнял у меня Асенефь, которая первоначально
была обручена со мною.
\vs Asn 23:7
И ныне будьте со мною,
воздвигните брань на брата вашего Иосифа, тогда я убью его мечом своим и возьму
Асенефь себе в жену;
\vs Asn 23:8
этим вы докажете верность
вашу и будете мне братьями и друзьями даже до конца, исполните только это моё
слово немедленно.
\vs Asn 23:9
Но если, выслушав моё
предложение, вы пренебрежёте им~--- знайте, что вас ожидает этот меч.
\vs Asn 23:10
Говоря это, обнажил он меч свой и показал им.
\vs Asn 23:11
Услышав такую надменную
речь из уст сына фараонова, Симеон и Левий пришли в негодование.
\vs Asn 23:12
Симеон же был муж смелый и
решительный; он готов был схватиться за рукоятку своего меча, вынуть его из
ножен и поразить им сына фараонова за оскорбительные слова;
\vs Asn 23:13
но Левий провидел его
намерение и, как муж, одарённый даром пророчества, духовным оком прозрел, что
было изображено у него на сердце, и ногою своею наступил ему на правую ногу и
тем дал ему знак, чтобы тот укротил гнев свой.
\vs Asn 23:14
И сказал Левий сыну
фараона, и сказал смело, и без гнева, и с сердцем кротким и ликом ясным:
\vs Asn 23:15
Зачем ты, господин наш,
произносишь такие речи? Мы мужи богобоязненные, отец наш раб Бога
Всевышнего, и брат наш Иосиф возлюблен Богом;
\vs Asn 23:16
как же мы можем творить
злое дело и грешить перед Богом, перед отцом нашим Иаковом и братом нашим
Иосифом?
\vs Asn 23:17
Итак, слушай: мужу
богобоязненному ни в каком случае не подобает творить беззаконие потому только,
что имеет он в руках меч, поэтому воздержись говорить недоброе о нашем брате
Иосифе;
\vs Asn 23:18
ибо если ты будешь
упорствовать в злом твоём намерении, то вот готовы обнаженные мечи наши в
правых руках наших перед лицом твоим.
\vs Asn 23:19
С этими словами Симеон и
Левий извлекли мечи свои из ножен, и сказали: Видишь ли ты в наших руках эти
мечи?
\vs Asn 23:20
посредством этих мечей
отомстил Господь сихемлянам за оскорбление, нанесённое сынам израилевым в лице
сестры нашей Дины, которую обесчестил Сихем, сын Еммора.
\vs Asn 23:21
При виде меча обоюдоострого
сын фараонов испугался, и задрожали кости его; ибо мечи те блистали, как пламя
огня.
\vs Asn 23:22
Потемнело в глазах у сына
фараонова, и упал он на лицо своё под ноги их.
\vs Asn 23:23
Тогда Левий, протянув руки,
ухватил его и сказал: Встань, не бойся; смотри за собой; впредь остерегайся
говорить злое слово о брате нашем.
\vs Asn 23:24
И сказав это, вышли от сына
фараонова Симеон и Левий. Ужас и печаль овладели сыном фараона; ибо он боялся
Симеона.

\vs Asn 24:1
Тяжело было ему от любви к Асенефи; великая, безмерная тоска
напала на него.
\vs Asn 24:2
Тогда рабы его стали нашептывать ему, говоря: Вот, сыны Валлы
и сыны Зелфы, рабынь Лии и Рахили, жен иаковлевых, враги Иосифу и Асенефи,
которым они завидуют: они-то будут тебе покорны и исполнят твою волю.
\vs Asn 24:3
И сын фараонов послал вестников призвать их к себе. И пришли
они ночью и стали перед ним.
\vs Asn 24:4
И сказал им сын фараонов: Речь свою обращаю к вам, как к
мужам сильным.
\vs Asn 24:5
И говорят ему старшие братья Дан и Гад: Пусть говорит теперь
господин наш, что хочет, и мы, рабы твои, услышим и исполним волю твою.
\vs Asn 24:6
И сын фараонов возрадовался радостью великой и сказал своим
рабам: Отойдите от меня немного, ибо этим мужам я имею сообщить нечто втайне,
и все они отошли.
\vs Asn 24:7
И сказал сын фараонов Дану и Гаду: Перед лицом вашим
благословение и смерть; выбирайте скорее благословение, нежели смерть; вы
мужи сильные, вы не должны умереть как женщина, мужайтесь и отмстите врагам
вашим.
\vs Asn 24:8
Ибо я сам слышал, как брат ваш Иосиф говорил о вас отцу моему
фараону, что вы сыновья рабыни его матери, а не братья его: не дождусь смерти
моего отца, чтобы их истребить вместе с их родом, дабы они, дети служанки, не
могли участвовать в наследии.
\vs Asn 24:9
Это они продали меня измаильтянам, и я отплачу им за зло, мне
причинённое: пусть только умрёт отец мой.
\vs Asn 24:10
И похвалил его отец мой, фараон, говоря: Ты хорошо сказал,
возьми у меня 1000 человек воинов, и выйди на них тайно, и сотвори им, как
сотворили они тебе, и я буду твоим помощником.
\vs Asn 24:11
Когда же те мужи услышали
слова сына фараонова, возмутились духом, и опечалились, и сказали сыну
фараонову: Просим тебя, господин, помоги нам.
\vs Asn 24:12
И тот в ответ: Я помогу
вам, если вы послушаетесь меня.
\vs Asn 24:13
И они сказали: Стоим перед
тобою мы, рабы твои; прикажи, и мы исполним твою волю.
\vs Asn 24:14
И говорит им фараонов сын:
Нынешнею ночью я убью отца моего, потому что фараон стал отцом Иосифа;
\vs Asn 24:15
вы же убейте Иосифа, и я
возьму себе в жёны Асенефь, а вы и братья ваши будете моими сонаследниками
исполните только моё слово.
\vs Asn 24:16
И сказали ему Дан и Гад:
мы рабы твои: сегодня же исполним твоё приказание;
\vs Asn 24:17
ибо ныне мы слышали, как
Иосиф говорил Асенефи: пойди завтра в поле наследия нашего, ибо настало время
сбора винограда.
\vs Asn 24:18
600 сильных воинов
будут сопровождать её и 60 ей предшествовать. Теперь послушай, что мы
тебе скажем. И открыли они ему свои мысли.
\vs Asn 24:19
И сын фараонов дал каждому
из 4-ёх мужей по 500 воинов, назначив их князьями и начальниками.
\vs Asn 24:20
И говорят ему Дан и Гад:
Нынешнею ночью мы пойдем и сядем в засаде у потока в зарослях тростника;
\vs Asn 24:21
ты же возьми с собой 50 стрелков всадников и поезжай вперед.
\vs Asn 24:22
Как только покажется
Асенефь, мы предадим мечу воинов, сопровождающих её, тогда бросится она на
колеснице вперёд, и попадёт тебе в руки Асенефь, и сотворишь с ней, как хочет
душа твоя.
\vs Asn 24:23
После этого мы умертвим
Иосифа, погружённого в печаль, и сыновей его пред глазами его.
\vs Asn 24:24
И обрадовался сын фараонов,
услышав слова сии, и дал им 2000 воинов.
\vs Asn 24:25
И пришли они к потоку, и
укрылись в зарослях тростника, и 500 засели впереди, и заняли широкую
переправу с той и с другой стороны потока.

\vs Asn 25:1
И сын фараонов встал и отправился в ту ночь в дом отца своего.
\vs Asn 25:2
И пришёл сын фараонов к ложу
отца своего, чтобы убить его мечом; но телохранители не позволили ему доступ к
отцу его и спросили: Какая тебе надобность, господин?
\vs Asn 25:3
И отвечал сын фараона: Хочу
видеть отца моего, так как отправляюсь на сбор винограда в новый виноградник.
\vs Asn 25:4
И сказали ему телохранители:
Страданием страдает отец твой, всю ночь не мог заснуть, и ныне немного
успокоился. Приказал он никого не впускать.
\vs Asn 25:5
И отошёл он в ярости, и
отправился сын фараонов к своим воинам и при рассвете засел в засаде, как
посоветовали ему Дан и Гад.
\vs Asn 25:6
Услышав об этом, сыновья
Иакова, Неффалим и Асир, младшие братья, сказали Дану и Гаду: За что вы
задумываете ещё злые козни против нашего отца, Израиля, и брата нашего, Иосифа?
\vs Asn 25:7
Разве Господь не бережёт его
как зеницу ока? Не вы ли некогда продали его?
\vs Asn 25:8
А ныне он царствует над
страной, он раздаёт по доброй воле пшеницу, которой питается народ, он спасает
многим жизнь.
\vs Asn 25:9
Если вы сегодня попытаетесь
причинить ему зло, то умолит он Бога Израилева и появится на небе и настигнет
вас огонь, который пожрёт вас, и ангелы Божьи будут сражаться за него и явятся к
нему на помощь.
\vs Asn 25:10
Дан и Гад разгневались на
своих братьев и сказали им: В противном случае мы умрём как женщины!
Да не будет того.
\vs Asn 25:11
И вышли навстречу Иосифу и Асенефь.

\vs Asn 26:1
На рассвете Асенефь встала и
сказала Иосифу: Как ты сказал, я отправлюсь в поле наследия нашего для сбора
винограда; но я боюсь, как бы кто-нибудь, придя, не похитил бы меня у тебя.
\vs Asn 26:2
И сказал ей Иосиф: Мужайся,
не бойся ничего, но спеши идти; Господь будет с тобою, и он убережёт тебя как
зеницу ока и сохранит тебя от всякого зла.
\vs Asn 26:3
Я же пойду на труд свой и
раздавать буду хлебные припасы в городе, чтобы кормить народ и принять меры,
дабы от голода никто не погиб в стране.
\vs Asn 26:4
И отошла Асенефь своим путем;
и Иосиф отошёл на труд свой и раздавал хлеб.
\vs Asn 26:5
И приблизилась Асенефь в
сопровождении 600 воинов к месту, где была ложбина.
\vs Asn 26:6
И внезапно вышли из засады воины Дана и Гада
и напали на воинов Асенефи, и завязался бой с сильными Асенефи,
\vs Asn 26:7
и убили из них около 50 всадников,
ехавших впереди, а Асенефь бежала на своей колеснице.
\vs Asn 26:8
И Левий узнал всё сие, и
известил своих братьев, сынов Лии, об измене.
\vs Asn 26:9
И каждый из них обнажил меч
свой при бедре своём, и щит свой на плече своём, и взял копье в правую руку. И
побежали они поспешно вслед Асенефи.

\vs Asn 26:10
И Асенефь бежала, и вот, сын фараона, сопровождаемый
50-ью всадниками, навстречу ей.
\vs Asn 26:11
И увидела его Асенефь, и испугалась, и вострепетала. Тогда
призвала она имя Господа, Бога Всевышнего.
\vs Asn 27:1
И Вениамин был с нею в её колеснице; и был он отрок сильный,
красивый, богобоязненный и весьма храбрый.
\vs Asn 27:2
И он сошел с колесницы, и
набрал у потока полные руки гладких камней, и бросил их в сына фараонова, и
поразил его в левый висок, и причинил ему жестокую рану, и сын фараона упал с
коня своего и лежал на земле.
\vs Asn 27:3
И после того Вениамин
поднялся поспешно на высокую скалу и сказал вознице Асенефи: Достань мне
гладких камней из потока.
\vs Asn 27:4
И тот достал ему 48 гладких камней.
И бросил те камни Вениамин, и убил 48 мужей,
сопровождавших сына фараонова.

\vs Asn 27:5
И сыны Лии Рувим, Симеон,
Левий, Иуда, Иссахар и Завулон погнались за мужами, сидевшими в засаде в
кустах, и напали на них неожиданно; и шестеро их убили их всех.
\vs Asn 27:6
И братья их, Дан и Гад,
сыновья Валлы и Зелфы, убежали при виде их, говоря:
\vs Asn 27:7
Мы не устояли перед нашими
братьями, и сын фараона побеждён и ранен смертельно Вениамином,
и все, бывшие с ним, погибли от руки его.
Пойдём же, убьём Асенефь, и скроемся в зарослях тростника.
\vs Asn 27:8
И пришли они, держа в руке
мечи свои обнаженными и полными крови.
\vs Asn 27:9
И увидела их Асенефь и
сказала: Господь, Бог мой, ты, который спас меня от смерти и который сказал
мне: живи вовеки! избавь меня от меча этих нечестивых мужей.
\vs Asn 27:10
И внял Бог гласу её, и
тотчас мечи их выпали из рук их на землю и рассыпались в прах.
\vs Asn 28:1
Видя это, сыны Валлы и Зелфы
устрашились и сказали: Истинно Господь воюет против нас за Асенефь.
\vs Asn 28:2
И пали на лица свои на землю
и бросились к ногам Асенефи и сказали ей:
\vs Asn 28:3
Ты наша госпожа и царица;
мы согрешили пред тобою, и Бог воздал нам по делам нашим.
\vs Asn 28:4
Мы, рабы твои, умоляем тебя,
помилуй нас и спаси нас от рук братьев наших, ибо они идут отмстить нам и мечи
их изострены на нас.
\vs Asn 28:5
И сказала Асенефь:
Мужайтесь и не страшитесь братьев ваших, ибо они мужи богобоязненные; идите в
эти тростниковые заросли, пока не умолю я за вас и не усмирю гнева их; ибо
велика дерзость ваша против них.
\vs Asn 28:6
Мужайтесь и не бойтесь, и да
рассудит Господь между мною и вами! И убежали в заросли тростника Дан и Гад.
\vs Asn 28:7
И вот, прибежали сыновья
Левия, подобно стаду оленей, и сошла Асенефь с закрытой колесницы своей и
встретила их со слезами.
\vs Asn 28:8
Они же, пав на землю,
поклонились ей, и плакали громко, и искали братьев своих.
\vs Asn 28:9
И сказала Асенефь: Пощадите
братьев ваших и не делайте им зла, ибо Господь явился мне защитником против них
и сокрушил мечи их, и, как воск от огня, растаяли они на земле.
\vs Asn 28:10
И этого будет с них, ибо
Господь воюет против них, а вы пощадите их, ведь они братья ваши и кровь отца
вашего Израиля.
\vs Asn 28:11
И сказал ей Симеон: Зачем
госпожа наша говорит доброе о врагах наших? Нет, мы перебьём их мечами нашими,
так как они замышляли об отце нашем Израиле,
и о брате нашем Иосифе, уже 2-жды;
а ныне и на тебя
\vs Asn 28:12
И, простёрши руку свою,
Асенефь коснулась бороды его и поцеловала её, говоря: Ты никогда этого не
сделаешь, брат мой, и не отплатишь злом за зло ближнему своему; ибо Господь
судит обиду сию; а ведь они братья ваши и чада отца вашего и убежали от лица
вашего.
\vs Asn 28:13
И преклонился Симеон пред
Асенефь. И подошёл к ней Левий, и облобызал ей руки, и благословил её; и понял
он, что она желает спасти братьев его.
\vs Asn 28:14
И находились они в зарослях
тростника; и узнал он это от братьев их, но не дал знать им о том, ибо опасался,
как бы они в пылу не перебили их.
\vs Asn 29:1
И сын фараонов поднялся от
земли и сел, выплевывая кровь из уст своих, ибо кровь текла из виска его в уста
его.
\vs Asn 29:2
И подбежал к нему Вениамин,
и взял меч его, и вынул его из ножен (ибо не носил Вениамин меча при бедре
своём) и хотел убить его и поразить сына фараона в грудь.
\vs Asn 29:3
И подошел к нему Левий, и
взял его за руку, и сказал: Брат мой, не делай этого, ведь мы мужи
богобоязненные и не подобает мужу богобоязненному воздавать злом за зло, ни
попирать поверженного или добивать до смерти попавшего в руки врага.
\vs Asn 29:4
И теперь вложи свой меч в
ножны и помоги мне обвязать раны его, и если жив будет, сделается нашим другом;
как и фараон нам как отец.
\vs Asn 29:5
И поднял Левий сына
фараонова, и отер кровь с лица его, и наложил повязку на рану его, и принял его
на коня своего, и повёз его к отцу его, и рассказал ему обо всем случившемся.

\vs Asn 29:6
И поднялся фараон с престола своего и поклонился Левию.
\vs Asn 29:7
И на 3-ий день умер сын
фараонов от раны, причинённой камнем Вениаминовым.
\vs Asn 29:8
И оплакивал фараон сына
своего первородного, и впал от печали в недуг.
\vs Asn 29:9
И умер фараон, имея 109 лет от роду, и оставил диадему свою Иосифу.

\vs Asn 29:10
И царствовал Иосиф в земле
египетской 48 лет, а после того предал Иосиф диадему внуку фараона.
\vs Asn 29:11
И был Иосиф в земле
египетской, как отец его.
\vs Asn 29:12
И так хранил его Бог от
нежной юности даже до конца жизни его, ибо был он семенем избранных мужей
праведных, Авраама, Исаака и Иакова, и молитвы их шли перед ним;
\vs Asn 29:13
и солнце и звёзды
преклонились пред Иосифом, знаменуя, что быть ему царем.

\include{tex/Vis}
\bibbookdescr{2Ba}{
  inline={Вторая Книга Пророка Варуха\fns{Переведена с сирийского.}},
  toc={2-я Варуха},
  bookmark={2-я Варуха},
  header={2-я Варуха},
  abbr={2~Вар}
}
\vs 2Ba 1:1
В 25-й год Иехонии, царя Иуды, слово ЯХВЕ было обращено к Варуху, сыну Нерии, и ему было сказано:
\vs 2Ba 1:2
Ты видел всё, что сделал Мне этот народ. Зло, совершенное оставшимися двумя коленами, превосходит зло, совершенное десятью коленами, которые были уведены в плен.
\vs 2Ba 1:3
Ибо те первые колена были вовлечены в грех своими царями, эти же два увлекли и вынудили ко греху своих царей.
\vs 2Ba 1:4
Поэтому вот, Я посылаю злое на этот город и на живущих в нём, и он отнимется от Меня на некоторое время. Я разсею Мой народ среди народов, дабы он был посланием для них.
\vs 2Ba 1:5
И Мой народ будет наказан, и придет время, когда они пожалеют о бывшем благоденствии.

\vs 2Ba 2:1
Я сказал тебе эти слова, чтобы ты передал их Иеремии и всем, кто подобен вам, с тем, чтобы вы удалились из этого города.
\vs 2Ba 2:2
Ибо ваши дела для этого города словно прочный столп, и ваши молитвы словно укрепленная стена.

\vs 2Ba 3:1
И я сказал: ЯХВЕ, Бог мой, для того ли я пришел в этот мир, чтобы видеть грехи моей матери? Нет, ЯХВЕ!
\vs 2Ba 3:2
Если я обрел милость в Твоих глазах, отними сперва мой дух, дабы мне отойти к моим отцам и не быть при погибели моей матери.
\vs 2Ba 3:3
Ибо я весьма страшусь двух этих вещей: Тебе противиться не могу, но и видеть грехи моей матери также не вынесет душа моя.
\vs 2Ba 3:4
Одно только я скажу пред Тобою, ЯХВЕ:
\vs 2Ba 3:5
Что же будет после всего этого? Ибо если Ты разрушишь Твой город и предашь Твою землю ненавидящим нас, как еще останется память имени Израиля?
\vs 2Ba 3:6
Или как вознесётся слава Тебе? Кому будут объяснять то, что содержится в Твоём Законе?
\vs 2Ba 3:7
Или мир возвратится к своей изначальной природе и век вернется к молчанию, бывшему искони? Или множество душ будет унесено, и человеческая природа не будет более называться своим именем? И что станет со всем, что сказано Тобою о нас Мойсею?

\vs 2Ba 4:1
И ЯХВЕ сказал мне: Этот город будет предан на время, и народ на время наказан, но мир не будет предан забвению.
\vs 2Ba 4:2
Или ты думаешь, что об этом городе Я сказал: Я вырезал тебя на ладонях рук Моих?
\vs 2Ba 4:3
Нет, это здание, что возвышается сейчас посреди вас, не то, которое будет открыто у Меня, уготованное здесь от времени, когда Мне на мысль пришло создать Рай. Я показал его Адаму до того, как он согрешил. Когда же он преступил повеление, он утратил его, также, как и Рай.
\vs 2Ba 4:4
Я показал его также и Аврааму ночью между разсеченными жертвами.
\vs 2Ba 4:5
И Моисею Я тоже показал его на горе Синай, когда Я открыл ему образ Скинии и всех её сосудов.
\vs 2Ba 4:6
И вот теперь Я храню его у Себя, равно как и Рай.
\vs 2Ba 4:7
Итак, иди и исполни то, что Я повелел тебе.

\vs 2Ba 5:1
И я отвечал, говоря: Так, на меня возложена скорбь о Сионе оттого, что Твои враги придут на это место и осквернят Твоё Святилище, и уведут в плен Твоё наследие. Они воцарятся над теми, кого Ты возлюбил. И они возвратятся к местам своих идолов и будут хвалиться перед ними. Но что Ты сделаешь ради Твоего великого Имени?
\vs 2Ba 5:2
И ЯХВЕ сказал мне: Мое Имя и Моя слава во веки веков; и Мой суд удерживает Свою праведность до времени.
\vs 2Ba 5:3
И ты увидишь это своими глазами: не вражеское войско разрушит Сион и предаст огню Иеросалим, но слуги Судьи в свое время.
\vs 2Ba 5:4
Ты же иди и исполни всё, что Я сказал тебе.
\vs 2Ba 5:5
И я пошел и взял с собою Иеремию, Аддо и Серайю, и Иавеша и Годолию, а также всех почтенных мужей из народа. Я привел их к потоку Кедронскому и рассказал им всё, что было сказано мне.
\vs 2Ba 5:6
И они возвысили свой голос и все заплакали.
\vs 2Ba 5:7
И мы сидели там, соблюдая пост до вечера.

\vs 2Ba 6:1
Назавтра войско Халдеев окружило город. К вечеру я, Варух, оставил народ и ушел, и стал у дуба.
\vs 2Ba 6:2
Я плакал о Сионе и сокрушался о плене, который выпал народу,
\vs 2Ba 6:3
и вдруг сильный дух поднял меня и перенес через стену Иеросалима.
\vs 2Ba 6:4
И я увидел: вот, четыре ангела стоят в четырех углах города, и каждый из них держал в руках горящий факел.
\vs 2Ba 6:5
Еще один ангел сошел с неба и сказал им: Держите факелы, но не зажигайте пожар, доколе я не скажу вам.
\vs 2Ba 6:6
Ибо я послан сказать прежде слово земле и передать ей то, что повелено мне от ЯХВЕ Элиона.
\vs 2Ba 6:7
И я видел, как он сошел на Святое Святых и взял завесу, святой ефод, седалище искупления, две скрижали и священное облачение священников, алтарь для курений, сорок восемь драгоценных камней, что носят священники, и святые сосуды Скинии.
\vs 2Ba 6:8
И он сказал земле громким голосом: Земля, земля, земля, слушай слово Эл Шаддаи, прими предметы, которые я вручаю тебе, и храни их до последних времен. Тогда лишь, когда ты получишь повеление, ты вернешь их. И так чужие не овладеют ими.
\vs 2Ba 6:9
Ибо настало время, когда Иеросалим будет предан ненадолго, пока не придет повеление возстановить его навечно.
\vs 2Ba 6:10
И земля разверзла свою пасть и поглотила их.

\vs 2Ba 7:1
И после этого я услышал, как ангел сказал тем ангелам, что держали факелы: Теперь сокрушите и уничтожьте эту стену до её основания, чтобы враги не могли хвалиться, говоря: Это мы разрушили стены Сионские, это мы сожгли место Эл Шаддаи.
\vs 2Ba 7:2
Возьмите то место, на котором я стоял прежде.

\vs 2Ba 8:1
Ангелы сделали так, как он им велел. И когда они уничтожили углы стен, и когда рухнула стена, из Храма раздался голос, и он сказал:
\vs 2Ba 8:2
Входите, враги, и вы, ненавистники, идите сюда. Охранявший этот Дом покинул его.
\vs 2Ba 8:3
И я, Варух, ушел.
\vs 2Ba 8:4
И после этого вошло войско Халдеев и заняло Дом и всё, что вокруг него.
\vs 2Ba 8:5
И оно увело народ в плен, убив иных из него; они связали цепью царя Седекию и отправили его к царю Вавилона.

\vs 2Ba 9:1
И я, Варух, возвратился вместе с Иеремией, сердце которого было найдено чистым от греха и который не был взят в плен при взятии города.
\vs 2Ba 9:2
И мы разодрали наши одежды и плакали, и оделись во вретище, и постились семь дней.

\vs 2Ba 10:1
Семь дней спустя, слово Божие было обращено ко мне, и мне было сказано:
\vs 2Ba 10:2
Скажи Иеремии, чтобы он шел в Вавилон укреплять пленный народ.
\vs 2Ba 10:3
Сам же ты оставайся здесь, на развалинах Сиона, и после этих дней Я покажу тебе то, что случится в конце дней.
\vs 2Ba 10:4
Я передал Иеремии повеление ЯХВЕ,
\vs 2Ba 10:5
и он ушел вместе с народом. Я же, Варух, возвратился и сел у дверей Храма. И я оплакал Сион плачем, говоря так:
\vs 2Ba 10:6
Счастлив тот, кто никогда не рождался, или родился для того, чтобы умереть тотчас же.
\vs 2Ba 10:7
Но горе нам, живым, видевшим муки Сиона и участь Иеросалима.
\vs 2Ba 10:8
Я буду взывать к морским сиренам и вы, лилиты, спешите сюда из пустыни; демоны и драконы, поспешайте из леса, пробудитесь и препояшьтесь вретищем. Воспойте вместе со мною погребальную песнь, стенайте вместе со мною.
\vs 2Ba 10:9
Вы, земледельцы перестаньте сеять; земля, к чему тебе приносить плоды урожаев? Храни в своем чреве сладость твоей пищи.
\vs 2Ba 10:10
К чему, виноградная лоза, продолжать тебе дарить вино? Ибо никогда более его не принесут на Сион, никогда больше здесь не предложат в жертву начатков.
\vs 2Ba 10:11
Вы, небеса, удерживайте росу и не открывайте дождевые хранилища.
\vs 2Ba 10:12
Солнце, удержи свет твоих лучей, и ты, луна, погаси яркость твоего сияния. Зачем вновь рождаться дню, когда затмился свет Сиона?
\vs 2Ba 10:13
Женихи, не входите в брачный чертог и не давайте невестам украшаться венками. Женщины, не молитесь о том, чтобы зачать.
\vs 2Ba 10:14
Ибо безплодная весьма возрадуется, и те, у которых нет сыновей, будут считать себя счастливыми, а имеющие сыновей возскорбят.
\vs 2Ba 10:15
Зачем рожать в муках и затем хоронить в скорби?
\vs 2Ba 10:16
Или зачем мужам рождать потомство, или зачем их семя будет вновь получать имя, когда мать брошена в одиночестве, а сыновья уведены в плен?
\vs 2Ba 10:17
Не говорите отныне о красоте и не разсуждайте об изяществе.
\vs 2Ba 10:18
Но вы, священники, возьмите ключи от Святилища и забросьте их высоко в небо, верните их ЯХВЕ, говоря: Охраняй Сам Свой Дом, ибо мы были найдены неверными управителями.
\vs 2Ba 10:19
И вы, девы, вплетающие в виссон и шелк Офирское золото, поспешите, возьмите всё это и бросьте в огонь, дабы он возвратил эти вещи Сотворившему их, и пламя вернуло их Творцу, дабы враги не овладели ими.

\vs 2Ba 11:1
Но и против тебя, Вавилон, буду говорить я, Варух: даже если ты благоденствовал, и Сион жил в своей славе, велика была бы наша скорбь видеть тебя равным Сиону.
\vs 2Ba 11:2
Но теперь наша скорбь безгранична и стенания наши безмерны, ибо ты благоденствуешь, а Сион опустошен.
\vs 2Ba 11:3
Кто будет судьею, видя это? Или кому мы пожалуемся на то, что отяготило нас? Как ты вынес всё это, ЯХВЕ?
\vs 2Ba 11:4
Отцы наши уснули без страданий, и все праведники покоятся в земле с миром.
\vs 2Ba 11:5
Они не познали нынешних тягот, они не слышали о горе, выпавшем нам.
\vs 2Ba 11:6
Земля, если бы ты имела уши; пыль, если бы ты имела сердце, вы бы могли пойти возвестить Шеолу и сказать мёртвым: Вы гораздо блаженнее нас, живущих.

\vs 2Ba 12:1
Но я выскажу мои мысли, и возвышу голос против тебя, земля благоденствующая:
\vs 2Ba 12:2
Не всегда пылает жара полуденная, и лучи солца не постоянно дают свет.
\vs 2Ba 12:3
Не думай и не надейся, что ты во всякое время будешь благоденствовать и радоваться. Не возвышайся чрезмерно и не надмевайся, повергая в рабство.
\vs 2Ba 12:4
Ибо воистину, в свою пору проснется гнев на тебя, пока что удерживаемый как уздою долготерпением.
\vs 2Ba 12:5
И закончив говорить, я постился семь дней.

\vs 2Ba 13:1
И после этого я, Варух, стоял на горе Сион, и вот, голос пришел с высоты и сказал мне:
\vs 2Ba 13:2
Встань, Варух, и слушай слово Эл Шаддаи.
\vs 2Ba 13:3
Поскольку ты изумлен участью Сиона, ты останешься до конца времен свидетельствовать о ней.
\vs 2Ba 13:4
И когда вдруг эти процветающие города спросят: Почему Эл Шаддаи навлек на нас эту кару?
\vs 2Ba 13:5
скажи им, ты и подобные тебе, вынесшие эту катастрофу и кары, которые обрушились на вас и ваш народ в определенное время, скажи им, что народы будут тяжко наказаны.
\vs 2Ba 13:6
И они будут упорствовать во зле.
\vs 2Ba 13:7
И если они тогда скажут: Когда это будет?
\vs 2Ba 13:8
ответь им: Вы пили оцеженное вино, выпейте и осадок. Ибо таков суд Элиона, Который не взирает на лица.
\vs 2Ba 13:9
Поэтому он не пощадил Своих сыновей, но прежде поразил их, как Своих врагов, ибо они согрешили.
\vs 2Ba 13:10
Поэтому они подверглись карам лишь для того, чтобы снискать прощение.
\vs 2Ba 13:11
Но вы, народы и языки, вы поистине виновны, ибо вы всегда попирали ногами землю и неправедно обходились с творением.
\vs 2Ba 13:12
Я всегда творил для вас добро, и всегда вы противились Моему благосердию.

\vs 2Ba 14:1
И я отвечал, говоря: Вот, Ты показал мне последовательность времён и то, что придёт после. Ты сказал мне воздаяние, о котором Ты возвестил, что оно постигнет народы.
\vs 2Ba 14:2
Теперь я знаю, сколь многие согрешили. Пожив в благоденствии, они покинули этот мир, где в эти времена останется мало народов, которые услышат слова, сказанные Тобою.
\vs 2Ba 14:3
Какой в этом прок? Или какое зло должны мы ожидать более того, что мы видели, поразившим нас?
\vs 2Ba 14:4
Но я снова буду говорить пред Тобою.
\vs 2Ba 14:5
Какое преимущество получат знающие пред Тобою, не ходившие в тщеславии, подобно остальным народам? Они не сказали тому, в чем нет жизни: Дай нам жизнь, но всегда боялись Тебя и никогда не удалялись от Твоих путей.
\vs 2Ba 14:6
И вот, несмотря на их ревность, Ты не пощадил Сион.
\vs 2Ba 14:7
И если одни творили злые дела, надо было простить Сион, ради добрых дел, которые сотворили другие и не губить его ради дел неправедных.
\vs 2Ba 14:8
Но кто, ЯХВЕ, Бог мой, проникнет в Твой суд, кто изследует глубину Твоего пути или возвестит вес Твоей стези?
\vs 2Ba 14:9
Или же кто способен понять Твой непостижимый совет? Или кто из тех, что рожден, когда-нибудь найдет начало и конец Твоей премудрости?
\vs 2Ba 14:10
Ибо все мы стали подобны дыханию.
\vs 2Ba 14:11
Ибо так же, как дыхание поднимается само собою и умирает во вне, так и природа человека: они не отходят по своей воле и не знают, какова будет их участь в конце.
\vs 2Ba 14:12
Праведные справедливо ожидают конца и без страха покидают свои жилища, ибо они у Тебя обладают могучим хранилищем своих дел.
\vs 2Ba 14:13
Поэтому они без страха оставляют этот мир и с радостью доверяются надежде стяжать мир, обещанный Тобою.
\vs 2Ba 14:14
Но горе нам, уже покрытым поношением и ожидающим лишь злого в будущем.
\vs 2Ba 14:15
Но Ты точно знаешь, что Ты сделал с Твоими рабами, ибо мы неспособны понять, что есть добро так, как Ты, наш Создатель.
\vs 2Ba 14:16
Но я буду опять говорить пред Тобою, ЯХВЕ, Бог мой.
\vs 2Ba 14:17
Прежде чем мир начал быть с его обитателями, Ты разсудил и произнес слово, и сейчас же творения стали пред Тобою.
\vs 2Ba 14:18
И Ты сказал, что создашь человека управителем Твоих дел в Твоем мире, дабы каждый знал, что не он был создан для мира, но мир для него.
\vs 2Ba 14:19
На деле же я вижу, что творение, созданное ради нас, пребывает, тогда как мы, для которых оно создано, погибаем.

\vs 2Ba 15:1
ЯХВЕ отвечал мне, говоря: Справедливо ты изумлен тем, что люди преходящи, но ты неверно разсудил о зле, поражающем грешников,
\vs 2Ba 15:2
когда ты сказал: Праведные отняты, а нечестивые благоденствуют;
\vs 2Ba 15:3
и когда ты прибавил: Никто не познал Твои суды.
\vs 2Ba 15:4
Посему слушай Меня, и Я буду говорить тебе; будь внимателен и услышишь от Меня Мои слова.
\vs 2Ba 15:5
Человек по праву мог бы пренебречь Моим судом, если бы не получил от Меня Закон, и если бы Я не увещевал его к уразумению.
\vs 2Ba 15:6
Но поскольку он грешил сознательно, так же сознательно он будет наказан.
\vs 2Ba 15:7
Что же до сказанного тобою о праведниках, будто бы мир начался для них, мир грядущий будет еще более принадлежать им.
\vs 2Ba 15:8
Ибо этот мир есть борьба и труд с великою скорбью, но мир грядущий будет венцом, соединенным с великою славой.

\vs 2Ba 16:1
Я отвечал, говоря: ЯХВЕ, Бог мой, вот нынешние годы кратки и злы. Кто же сможет за столь краткое время приобрести то, что не имеет меры?

\vs 2Ba 17:1
ЯХВЕ отвечал мне, говоря: У Элиона ничего не значит ни долгота, ни краткость времени.
\vs 2Ba 17:2
Что дала Адаму жизнь длиною в девятьсот тридцать лет, когда он преступил данную заповедь?
\vs 2Ba 17:3
Безполезным было прожитое им долгое время. Он ввел в мир смерть и сократил годы родившихся от него.
\vs 2Ba 17:4
Оттого ли и Моисей не потерпел урон, прожив только сто двадцать лет в послушании своему Создателю? Он передал Закон потомству Иакова и зажег светоч для народа Израилева.

\vs 2Ba 18:1
Я отвечал, говоря: Тот, кто зажег свет, принял его от Света, однако мало таких, кто последовал ему, и многие из тех, кого он осветил, взяли от тьмы Адама и не возрадовались свету Светоча.

\vs 2Ba 19:1
И Он отвечал мне, говоря: Посему в то время он заключил с ними завет и сказал: Вот, я поставил перед вами жизнь и смерть. И он взял в свидетели перед ними небо и землю,
\vs 2Ba 19:2
ибо он знал, что его время кратко, но что небеса и земля пребудут вовек.
\vs 2Ba 19:3
Они, однако, после его смерти, согрешили и пренебрегли, хорошо зная, что у них есть Закон, который будет их обвинять, Свет, который ничто не обманет, сферы, которые будут свидетельствовать,
\vs 2Ba 19:4
и Я Сам, Судящий всё, что есть. Ты же не заботься об этом и не печалься о прошлом.
\vs 2Ba 19:5
Вот, теперь надлежит придти исполнению времен, дел и благоденствия, а также унижению, а не началу всего этого.
\vs 2Ba 19:6
Если человек, преуспевавший в начале, в старости покрывается поношением, он забывает о своем бывшем благоденствии.
\vs 2Ba 19:7
И напротив, преуспев позднее, после того, как в начале был в ничтожестве, он уже не помнит о своем унижении.
\vs 2Ba 19:8
Слушай еще: если всё это время, с того дня, когда смерть была суждена людям, преходящим в этом мире, каждый в начале преуспевал бы, дабы погибнуть в конце, всё было бы тщетою.

\vs 2Ba 20:1
Поэтому вот, придут дни, когда времена будут проходить быстрее, чем в прошлом. И все сроки будут истекать скорее, чем сейчас, быстрее нынешних будут течь годы.
\vs 2Ba 20:2
Поэтому Я и уничтожил Сион, дабы ускорить время Моего посещения этого мира.
\vs 2Ba 20:3
Ныне же сохраняй всем сердцем всё, что Я повелеваю тебе, и запечатай это в самой глубине ума.
\vs 2Ba 20:4
Тогда Я открою Мой всевластный суд и Мои непроницаемые пути.
\vs 2Ba 20:5
Иди же и освятись в течение семи дней: ни хлеба не ешь, ни воды не пей, и не говори ни с кем.
\vs 2Ba 20:6
И по окончании этого срока возвращайся на это место, и Я явлюсь тебе, и Я скажу тебе истины и дам тебе наставления о последовании времён; ибо они приближаются и не замедлят.

\vs 2Ba 21:1
Молитва Варуха, сына Нерии. И я покинул это место, и ушел, и сел у потока Кедронского в подземной пещере. Там я освятился: хлеба не ел и не был голоден, воды не пил и не испытывал жажды. Там я оставался до седьмого дня, как Он заповедал мне.
\vs 2Ba 21:2
Затем я пришел на то место, где Он разговаривал со мною.
\vs 2Ba 21:3
И на закате солнца душа моя была охвачена множеством мыслей, и я обратился к Шаддаи
\vs 2Ba 21:4
и сказал: Ты, сотворивший землю, выслушай меня; Ты, утвердивший небесный свод Твоим словом и укрепивший высоту неба духом; Ты, от начала мира призывающий то, что еще не существует, и всё тебе повинуется;
\vs 2Ba 21:5
Ты, одним мановением повелевающий воздуху и видящий будущее словно прошлое;
\vs 2Ba 21:6
Ты, великим советом управляющий силами, которые предстоят Тебе, и в гневе направляющий безчисленные святые создания, сотворенные Тобою из огня и пламени, окружающие Твой престол;
\vs 2Ba 21:7
Ты Единый, имеющий власть исполнить во всякое время всё, что Ты пожелаешь;
\vs 2Ba 21:8
Ты, Кто пролил каждую каплю дождя, упавшую на землю; Ты, Единый, знающий конец времен прежде их начала, внемли моей молитве.
\vs 2Ba 21:9
Ты Один можешь поддерживать тех, кто есть, тех, кто ушел и тех, кто придет, грешников и тех, кто оправдан, ибо Ты Живой и Непостижимый.
\vs 2Ba 21:10
Ты Единый, Живой, Безсмертный и Непостижимый; Ты знаешь число людей,
\vs 2Ba 21:11
многие ли согрешили в своё время, и были ли оправданы другие, не менее многочисленные.
\vs 2Ba 21:12
Ты знаешь место, которое Ты приготовил в конце одним, тем, кто согрешил; а также место свершения других, тех, что были оправданы.
\vs 2Ba 21:13
Ибо если бы для всех людей была лишь одна жизнь здесь, не было бы ничего горше.
\vs 2Ba 21:14
Ибо к чему сила, которая обращается в слабость, пища вдоволь, если она обращается в голод, красота, если она становится ненавистною?
\vs 2Ba 21:15
Ибо без конца изменяется природа человека.
\vs 2Ba 21:16
Мы уже не те, чем мы были прежде, и в будущем мы не останемся тем, что мы есть сегодня.
\vs 2Ba 21:17
Если бы всему этому не надлежало окончиться, тщетно было бы и самое начало.
\vs 2Ba 21:18
Но дай мне знать, что исходит от Тебя, и просвети меня во всём, что я спрошу у Тебя.
\vs 2Ba 21:19
Доколе будет пребывать растленное и доколе будет процветающим время смертных? Будут ли еще сильнее оскверняться преходящие в этом мире?
\vs 2Ba 21:20
Дай же повеление в Твоём милосердии и соверши то, чему Ты обещал нам исполниться, дабы явилось Твоё могущество тем, кто думает, будто Твоё долготерпение от слабости.
\vs 2Ba 21:21
Покажи тем, кто видел несчастье, падшее на нас и на наш город, не разумеющим того, что в долготерпении Твоей мощи Ты назвал нас ради Твоего Имени Своим возлюбленным народом.
\vs 2Ba 21:22
Всякое существо ныне смертно по природе.
\vs 2Ba 21:23
Посему удержи ангела смерти, и да возсияет Слава Твоя, да явится величие Твоей красоты. И пусть Шеол будет запечатан, пусть отныне он не принимает более умерших, и хранилища душ отпустят тех, кто в них заключен.
\vs 2Ba 21:24
Ибо многочисленны для нас года со дней Авраама, Исаака и Иакова и всех, кто подобен им и спит в земле. И Ты сказал, что ради них Ты сотворил мир.
\vs 2Ba 21:25
Ныне поспеши явить Твою Славу и не замедли исполнением Твоих обетований.
\vs 2Ba 21:26
И закончив на этом слова молитвы, я был в изнеможении.

\vs 2Ba 22:1
И после этого небеса открылись, и мне было видение. Мне дана была сила, и голос раздался свыше и сказал мне:
\vs 2Ba 22:2
Варух, Варух, почему ты в смятении?
\vs 2Ba 22:3
Отправившийся в путешествие не завершает ли его? И мореплаватель может ли утешиться, прежде чем достигнет гавани?
\vs 2Ba 22:4
Или обещавший сделать дар, но не сделавший, не крадет ли?
\vs 2Ba 22:5
Также и человек, засеявший землю, не теряет ли всё, если не соберет урожай в благоприятное время?
\vs 2Ba 22:6
И насадивший готовится ли собирать плоды раньше, чем они созреют?
\vs 2Ba 22:7
Или женщина, зачав и родив не в срок, не причиняет ли смерть своему ребёнку?
\vs 2Ba 22:8
Тот же, кто строит дом, может ли назвать своё строение домом прежде, чем покроет его крышей и закончит? Ответь Мне вначале на этот вопрос.

\vs 2Ba 23:1
И я отвечал, говоря: Нет, ЯХВЕ, Бог мой.
\vs 2Ba 23:2
И Он отвечал, говоря: Почему тогда ты волнуешься о том, чего не знаешь и возмущаешься тем, о чем тебе ничего не известно?
\vs 2Ba 23:3
Ибо так же, как Я не забыл о людях, которые живут сейчас и которые отошли, так же Я вспомню о тех, кто придет.
\vs 2Ba 23:4
Когда Адам согрешил, и смерть стала приговором для всех, кто родится, множество имеющих родиться было сочтено, и для этого числа были уготованы места, где будут жить живые и храниться мёртвые.
\vs 2Ba 23:5
И пока предустановленное число не исполнится, творение не будет спасено, ибо Мой дух созидает жизнь, и Шеол принимает умерших.
\vs 2Ba 23:6
Послушай же еще о том, что придет после этих времен.
\vs 2Ba 23:7
Ибо пришествие Моего искупления поистине близко и не отстоит уже далеко, как некогда.

\vs 2Ba 24:1
Вот, грядут дни, и откроются книги, в которых записаны грехи всех грешников, а также собраны все сокровища праведности оправданных в творении.
\vs 2Ba 24:2
В эти времена ты узнаешь, ты и многие с тобою, каково было долготерпение Элиона из поколения в поколение, и насколько Он был терпелив ко всем рожденным, как грешникам, так и праведным.
\vs 2Ba 24:3
И я отвечал, говоря: Но, ЯХВЕ, нет никого, кто бы знал число того, что произошло и того, что будет. И я сам, хотя и знаю, что было с нами, я не ведаю того, что будет с нашими врагами, и времени, когда Ты посетишь Твоих рабов.

\vs 2Ba 25:1
И Он отвечал, говоря: Ты также пребудешь до того времени, ради знамения, которое Элион сделает для населяющих землю в конце дней.
\vs 2Ba 25:2
Знамение будет таким:
\vs 2Ba 25:3
когда оцепенение охватит населяющих землю, они впадут во многие безпокойства и снова в жестокие муки.
\vs 2Ba 25:4
И будет, что от великих безпокойств они станут думать: Шаддаи не помнит более о земле. И когда они утратят надежду, тогда и настанет это время.

\vs 2Ba 26:1
И я отвечал, говоря: Долго ли продлятся эти безпокойства и растянется ли нужда на многие годы?

\vs 2Ba 27:1
И Он отвечал мне, говоря: То время разделено на двенадцать частей, и каждая предназначается для предписанного ей.
\vs 2Ba 27:2
В первой части начало волнений,
\vs 2Ba 27:3
во второй убийство знатных,
\vs 2Ba 27:4
в третьей падение великого множества,
\vs 2Ba 27:5
в четвертой послан меч,
\vs 2Ba 27:6
в пятой голод и засуха,
\vs 2Ba 27:7
в седьмой землетрясения и ужасы,
\vs 2Ba 27:9
в восьмой множество призраков и нападения демонов,
\vs 2Ba 27:10
в девятой падение огня,
\vs 2Ba 27:11
в десятой кражи и великое притеснение,
\vs 2Ba 27:12
в одиннадцатой нечестие и страсти,
\vs 2Ba 27:13
а в двенадцатой смута от смешения всего перечисленного.
\vs 2Ba 27:14
Ибо эти части времени соблюдаются для того, чтобы быть смешанными друг с другом и послужить одна другой,
\vs 2Ba 27:15
ибо одни из них превзойдут свой удел и возьмут у других, иные исполнят свои свойства, а также свойства других, чтобы находящиеся на земле не поняли, что с этими днями настал конец времен.

\vs 2Ba 28:1
И, однако, будет премудр тот, кто уразумеет.
\vs 2Ba 28:2
Мерою же счета будут две части: недели по семь недель.
\vs 2Ba 28:3
И я отвечал, говоря: Хорошо человеку достичь этого времени и быть его зрителем. Лучше, однако, не достигать ему этого, дабы не пасть.
\vs 2Ba 28:4
Но я прибавлю еще: нетленный презрит ли тленных и их удел, взирая лишь на нетленное?
\vs 2Ba 28:6
ЯХВЕ, если поистине события, предсказанные мне Тобою, должны совершиться, и если я обрел благодать в Твоих глазах, открой мне еще и это: в одном ли месте или в одной ли части земли они совершатся, или весь мир испытает их?

\vs 2Ba 29:1
И Он отвечал мне, говоря: То, что произойдет тогда, произойдет по всей земле.
\vs 2Ba 29:2
Поэтому их испытают все живущие. Я пощажу лишь тех, кто в эти дни окажется на этой земле.
\vs 2Ba 29:3
Когда исполнится предусмотренное для этих частей земли, Мессия начнет открываться.
\vs 2Ba 29:4
И Бегемот появится из своего места, и Левиафан возстанет из моря оба эти гигантских чудовища, которых Я создал в пятый день творения и которых Я сберегал для этого времени в пищу всем, кто останется.
\vs 2Ba 29:5
И земля принесёт свои плоды мириадократно. Каждый виноградник принесет тысячу лоз, каждая лоза принесет тысячу гроздей, каждая гроздь будет насчитывать по тысяче ягод, и каждая ягода даст кор вина.
\vs 2Ba 29:6
И те, кто голоден, возрадуются и каждый день будут видеть чудеса.
\vs 2Ba 29:7
Ветры изойдут от Моего лица, обдувая каждое утро благоуханием ароматных плодов, а вечерами будут поднимать облака, которые испустят целительную росу.
\vs 2Ba 29:8
В это время манна, хранящаяся в хранилище, будет падать вновь, и они будут есть её в эти годы, ибо они достигли конца времен.

\vs 2Ba 30:1
И после этого, когда исполнится время пришествия Мессии, и Он вернется в славе, все, кто почил в надежде на Него, воскреснут.
\vs 2Ba 30:2
В это время будут распечатаны хранилища, содержащие души праведников; они выйдут, и множество душ явится в единомысленном собрании. Первые возрадуются, последние не познают тревоги.
\vs 2Ba 30:3
Они истинно узнают, что пришел день, о котором было предсказано, как о скончании времен.
\vs 2Ba 30:4
Но души злых, увидев всё это, истлеют полностью, ибо они знают, что их ожидают мучения, и что настала их погибель.

\vs 2Ba 31:1
И после этого я пошел к народу и сказал им: Соберите ко мне всех старейшин, и я буду говорить с вами.
\vs 2Ba 31:2
Все собрались у потока Кедронского.
\vs 2Ba 31:3
И я отвечал, говоря: Слушай, Израиль, и я буду говорить тебе; и ты, семя Иакова, открой слух, и я буду учить тебя.
\vs 2Ba 31:4
Не забывайте Сион, но помните о скорби Иеросалима.
\vs 2Ba 31:5
Ибо вот, грядут дни, когда всё станет добычей тления и сделается как не бывшее вовсе.

\vs 2Ba 32:1
Вы же, если приготовите ваши сердца к тому, чтобы посеять в них плоды Закона, Шаддай пощадит вас в то время, когда Он потрясет всё создание.
\vs 2Ba 32:2
Ибо через краткое время Здание Сиона будет потрясено, а затем возстановлено.
\vs 2Ba 32:3
Это новое Здание будет также временным и после некоторого времени также будет разрушено и останется в развалинах до времени.
\vs 2Ba 32:4
Затем ему должно будет обновиться в славе и придти к вечному исполнению.
\vs 2Ba 32:5
Не надо скорбеть чрезмерно, видя бедствие, постигшее нас ныне, как и то, которое еще будет.
\vs 2Ba 32:6
Ибо еще более ужасным, чем эти два потрясения, будет испытание, в котором Шаддай обновит творение.
\vs 2Ba 32:7
А теперь не приближайтесь ко мне несколько дней и не приходите ко мне, прежде чем я приду к вам.
\vs 2Ba 32:8
И когда я сказал им эти слова, я, Варух, пошел своим путем, и когда народ увидел, что я ухожу, все возвысили голос и возстенали, говоря: Куда ты уходишь от нас, Варух? Неужто и ты покидаешь нас, словно отец, бросающий своих детей, оставляющий их сиротами?

\vs 2Ba 33:1
Таково ли повеление, данное тебе твоим спутником пророком Иеремией, когда он говорил тебе:
\vs 2Ba 33:2
Смотри за этим народом, тогда как я ухожу укреплять в Вавилон остаток наших братьев, приговоренных идти в плен?
\vs 2Ba 33:3
Теперь если и ты нас оставишь, лучше было бы всем нам погибнуть прежде, а потом уж уходи от нас.

\vs 2Ba 34:1
Я отвечал народу, говоря: Далека от меня мысль покинуть вас или скрыться от вас. Я лишь иду к Святому Святых, дабы просить у Шаддаи большего просвещения ради вас и ради Сиона. И после этого я вернусь к вам.

\vs 2Ba 35:1
И я, Варух, пошел на Святое место и сел на его развалинах. Я заплакал и сказал:
\vs 2Ba 35:2
Да будут мои глаза источником, и мои ресницы ключом слёз,
\vs 2Ba 35:3
ибо доколе мне скорбеть о Сионе и доколе мне оплакивать Иеросалим?
\vs 2Ba 35:4
Ибо на этом месте, где я ныне простерся, некогда первосвященник приносил святые жертвы, возлагая на них благовонные курения.
\vs 2Ba 35:5
Но ныне наша слава обратилась в прах, и наслаждение наших душ в пепел.

\vs 2Ba 36:1
И сказав эти слова, я заснул там, и ночью мне было видение: Вот, это был лес деревьев, выросший в долине, окруженной высокими горами и крутыми скалами. Лес простирался на обширном пространстве.
\vs 2Ba 36:3
И вот напротив него вырос виноградник, из под которого мирно струился источник.
\vs 2Ba 36:4
Потом этот источник достиг леса и стал большим потоком, и этот поток затопил лес и в одно мгновение вырвал с корнем все деревья и ниспроверг все горы вокруг.
\vs 2Ba 36:5
Так был унижен возвысившийся лес, и вершины гор опустились: источник был столь силен, что остался только один кедр.
\vs 2Ba 36:6
Когда он сломал и его и вырвал с корнем и уничтожил весь лес так, что от него не осталось ничего, и место его стало неузнаваемо, тогда виноградник вместе с источником поднялся в великом спокойствии, и он подошел близко к месту кедра. И ему принесли кедр, который был сломлен.
\vs 2Ba 36:7
И мне было видение: виноградник открыл уста свои и сказал кедру: Ты, кедр, единственный уцелевший из злого леса. Из-за тебя нечестие утвердилось и распространялось все эти годы, но никогда ничего доброго. Ты завладел тем, что тебе не принадлежало, и никогда не жалел твой собственный удел.
\vs 2Ba 36:8
Ты простирал свою власть на тех, кто был далеко от тебя; ты завлекал в сети твоего лукавства тех, кто был рядом с тобою. Каждый миг ты превозносился, будто бы тебя нельзя вырвать с корнем.
\vs 2Ba 36:9
Но теперь твой исход приблизился быстро, и твой час настал.
\vs 2Ba 36:10
Иди же, кедр, вслед за лесом, который исчез перед тобою, стань, как и он, пеплом, и пусть смешается ваш прах. Спите ныне в муках, покойтесь в страдании до пришествия последних времен, когда ты возвратишься и будешь наказан еще тяжелее.

\vs 2Ba 37:1
После этого я видел кедр в огне, а виноградник возросшим, долина же вокруг покрылась неувядающими цветами. И я проснулся и встал.

\vs 2Ba 38:1
И я помолился, говоря: ЯХВЕ, Бог мой, Ты во всякое время просвещаешь тех, кто водится разумом.
\vs 2Ba 38:2
Твой закон есть жизнь, и Твоя премудрость правота.
\vs 2Ba 38:3
Открой же мне истолкование этого видения.
\vs 2Ba 38:4
Ибо Ты знаешь, что моя душа всегда ходила в Твоем Законе, и от рождения я не удалялся от Твоей Премудрости.

\vs 2Ba 39:1
И Он отвечал мне, говоря: Варух, вот истолкование видения, которое ты видел.
\vs 2Ba 39:2
Так же, как ты видел большой лес, окруженный высокими и крутыми горами вот значение этого
\vs 2Ba 39:3
точно так же наступают дни, когда царство, некогда уничтожившее Сион, само будет уничтожено и покорено пришедшему вслед за ним.
\vs 2Ba 39:4
Оно также будет скоро разрушено, и возстанет третье, которе будет главенствовать в свое время и тоже исчезнет.
\vs 2Ba 39:5
После этого возстанет четвёртое царство, власть которого будет суровей и злее власти предыдущих царств, и оно будет править долго, как лес в долине, и побеждать века и вознесётся как кедр Ливанский.
\vs 2Ba 39:6
Истина скроется от него, а те, кто запятнан нечестием, будут прибегать к нему, как злые звери прибегают в лес и прячутся в нем.
\vs 2Ba 39:7
И когда приблизится время его конца и время его падения, будет открыта власть Моего Мессии. Она будет подобна источнику и винограднику, и открывшись, выдернет с корнем их собравшееся множество.
\vs 2Ba 39:8
Что же до высокого кедра, уцелевшего в одиночестве из леса, и слов, которые ему были сказаны виноградником и которые ты слышал, вот значение их:

\vs 2Ba 40:1
Последний вождь, который уцелеет в то время уничтожения множества его собраний, будет связан и приведен на гору Сион. Мой Мессия обличит его во всяком нечестиии, соберёт и поставит перед ним всякое деяние его воинств.
\vs 2Ba 40:2
И потом Он казнит его и будет покровителем для остатка Моего народа, который обретётся в месте, избранном Мною.
\vs 2Ba 40:3
И Его власть пребудет вовек, доколе не прекратится этот мир тления, и не исполнятся предсказанные времена.
\vs 2Ba 40:4
Вот твое видение, и вот его истолкование.

\vs 2Ba 41:1
И я отвечал, говоря: Для кого и для скольких будут эти события? Кто из них удостоится быть спасенным в это время?
\vs 2Ba 41:2
Я выскажу пред Тобою все мои мысли и спрошу у Тебя о том, над чем я размышляю.
\vs 2Ba 41:3
Вот, я вижу многих, удалившихся от Твоего завета, и отбросивших ярмо Твоего Закона.
\vs 2Ba 41:4
Я видел и других, кто, напротив, прибег под сень Твоих крыльев.
\vs 2Ba 41:5
Какова будет их участь? И что готовит им последнее время?
\vs 2Ba 41:6
Будет ли тщательно взвешена долгота их жизни, и будут ли они осуждены по тому, куда склонятся весы?

\vs 2Ba 42:1
Он отвечал мне, говоря: И это Я также покажу тебе.
\vs 2Ba 42:2
Ты спрашиваешь, для кого предназначено это, и сколько их будет? Те, кто уверовал, получать обещанные блага; презревшим же будет противное.
\vs 2Ba 42:3
Ты также говорил о некоторых, котороые приблизились и удалились. Вот слово о них.
\vs 2Ba 42:4
Для тех, кто вначале послушался, а потом отдалился и смешался с семенем перемешавшихся народов, первая часть их жизни была [прежде] и будет сочтена за нечто [возвышенное].
\vs 2Ba 42:5
Тем же, кто прежде не знал, а после познал жизнь и примешался к семени, отделенному от народов, первая часть их времени [была позже и] будет сочтена за нечто [возвышенное].
\vs 2Ba 42:6
Одни времена сменяют другие, эпохи сменяются эпохами. Они не примут одни от других и в конце уравняются по мере времени и часов эпох.
\vs 2Ba 42:7
Тление возьмет своих, а жизнь своих.
\vs 2Ba 42:8
И прах будет призван, и ему будет сказано: Верни то, что тебе не принадлежит; яви то, что ты сохранял до времени.

\vs 2Ba 43:1
Но ты, Варух, укрепи твое сердце ради сказанного тебе и уразумей то, что тебе показала благодать в видениях, ибо вот ты был весьма утешен навек.
\vs 2Ba 43:2
Сейчас ты покинешь это место, и видимые его окрестности; ты забудешь всё тленное и никогда не вспомнишь о том, что бывает среди смертных.
\vs 2Ba 43:3
Иди же, отдай повеления твоему народу, а после возвращайся сюда, и постись семь дней, а потом Я приду к тебе и буду говорить с тобою.

\vs 2Ba 44:1
И я, Варух, ушел оттуда и пришел к моему народу и призвал своего первородного сына, Годолию, [моих друзей] и семерых из старейшин народа,
\vs 2Ba 44:2
и я сказал им: Вот, я иду к моим отцам, по пути всей земли,
\vs 2Ba 44:3
вы же не удаляйтесь от пути Закона, но и сохраняйте и увещевайте уцелевший народ, дабы они не удалялись от заповедей Элиона.
\vs 2Ba 44:4
Ибо вы видите, что Тот, Кому мы служим, справедлив; Тот, Кто образовал нас, не взирает на лица.
\vs 2Ba 44:5
Посмотрите, что было с Сионом, что произошло с Иеросалимом;
\vs 2Ba 44:6
суд Шаддая познается так же, как Его пути, неизследимые и прямые.
\vs 2Ba 44:7
Если вы будете терпеливы и пребудете в Его страхе, и не будете забывать Его Закона, времена для вас изменятся к лучшему, и вы увидите утешение для Сиона.
\vs 2Ba 44:8
То, что существует ныне, есть ничто, грядущее же будет весьма великим.
\vs 2Ba 44:9
Тленное проходит, и смертное уносится, и ни о чем из настоящего уже не вспомнят, не вспомнят более об этом времени, оскверненном злом.
\vs 2Ba 44:10
Бегущий ныне бежит тщетно; и успевающий скоро падет и будет унижен.
\vs 2Ba 44:11
Ибо желанно будущее, и в грядущем наша надежда.
\vs 2Ba 44:12
И есть час, который не прейдет, грядет эпоха, которая пребудет вовек, новый мир, который не приведет к тлению идущих под его властью и не поведет к гибели тех, кто спасается в нем.
\vs 2Ba 44:13
Для этих будет наследством возвещенное время, они унаследуют обетованную эпоху.
\vs 2Ba 44:14
Приготовившим хранилища премудрости и тем, у кого были найдены сокровищницы разума, не удалившимся от милосердия и сохранившим истину Закона будет отдан грядущий мир; жилище же других, весьма многих, будет в огне.

\vs 2Ba 45:1
Вы же, насколько сможете, наставляйте народ, ибо в этом ваш труд. 2 Ибо научая их Закону, вы спасёте их.

\vs 2Ba 46:1
Мой сын и старейшины народа отвечали мне, говоря: Неужели Шаддаи унизит нас настолько, что скоро отнимет тебя у нас?
\vs 2Ba 46:2
Воистину, мы будем во тьме; не будет более света для уцелевшего народа.
\vs 2Ba 46:3
Где еще будем мы искать Закон и кто различит для нас смерть от жизни?
\vs 2Ba 46:4
И я сказал им: Я не могу противиться престолу Шаддаи. Однако Израиль не останется без премудрого, и без Закона семя Иакова.
\vs 2Ba 46:5
Вы лишь приготовьте ваши сердца к слушанию Закона и подчинению тем, кто в страхе обладает премудростью и разумом. Расположите ваши души так, чтобы вам не уклониться от этого.
\vs 2Ba 46:6
Если вы будете делать это, благо, которое я предсказал вам, сбудется вам, и вы не подвергнетесь мучению, которое я возвещал вам прежде.
\vs 2Ba 46:7
Но о слове, которое возвестило мне удалиться, я не открыл никому, даже моему сыну.

\vs 2Ba 47:1
С тем я ушел и отослал их, и покинул их, говоря: Вот, я иду к Хеврону, туда посылает меня Шаддаи.
\vs 2Ba 47:2
И я пришел на то место, о котором мне было сказано, и я сел там и постился семь дней.

\chhdr{Молитва Варуха.}
\vs 2Ba 48:1
И после седьмого дня я помолился Шаддаи, говоря:
\vs 2Ba 48:2
О ЯХВЕ, Ты призываешь приход времён, и они встают перед Тобою. Мощь столетий исчезает от Тебя. Ты располагаешь череды эпох, и они повинуются Тебе.
\vs 2Ba 48:3
Ты Один знаешь счет поколений, и не многим Ты открываешь Твои тайны.
\vs 2Ba 48:4
Ты показываешь изобилие огня, и Ты взвешиваешь легкость ветра.
\vs 2Ba 48:5
Ты изследуешь пределы высот; Ты проницаешь глубины тьмы.
\vs 2Ba 48:6
Ты определяешь число тех, кто преходят, и тех, кого надо сохранить; и Ты готовишь жилище для тех, кто будет.
\vs 2Ba 48:7
Ты помнишь начала, которые Ты создал; и о будущем уничтожении Ты не забываешь.
\vs 2Ba 48:8
Страшными и яростными знамениями Ты повелеваешь пламени, и оно испаряется. Словом Ты вызываешь то, чего нет, и могучею властью Ты удерживаешь то, что еще не пришло.
\vs 2Ba 48:9
Ты научаешь творения Твоему разуму, Ты наделяешь Сферы мудростью служения Тебе.
\vs 2Ba 48:10
Безчисленные воинства предстоят Тебе и служат в покое по своему чину по одному Твоему мановению.
\vs 2Ba 48:11
Послушай раба Твоего и приклони ухо к моему молению.
\vs 2Ba 48:12
Ибо мы рождены на краткое время и скоро уходим.
\vs 2Ba 48:13
У Тебя же часы как эпохи, и дни как поколения.
\vs 2Ba 48:14
Посему не гневайся на человека, ибо он ничто; не изследуй наши дела.
\vs 2Ba 48:15
Ибо что мы такое? По Твоему дару мы пришли в мир, и вошли в него без нашего согласия.
\vs 2Ba 48:16
Мы не говорили нашим родителям: Зачните нас!, ни посылали к Шеолу с вестью: Прими нас!
\vs 2Ba 48:17
Сильны ли мы выдержать Твой гнев? Способны ли мы вынести Твой суд?
\vs 2Ba 48:18
Прикрой нас Твоим милосердием и Твоею благостью помоги нам.
\vs 2Ba 48:19
Взгляни на смиренных, служащих Тебе, и спаси всех, приближающихся к Тебе. Не разрушай надежду нашего народа и не сокращай времени Твоей помощи.
\vs 2Ba 48:20
Ибо это народ, избранный Тобою, эти люди народ, подобного которому нет в очах Твоих.
\vs 2Ba 48:21
Но и теперь я буду говорить прямо пред Тобою и высказывать мысли моего сердца.
\vs 2Ba 48:22
Мы уповаем на Тебя, ибо вот, с нами Твой Закон. Мы знаем, что покуда мы соблюдаем Твои заповеди, мы не падём.
\vs 2Ba 48:23
На всё время и во всяком деле на нас благословение, доколе мы не смешиваемся с народами.
\vs 2Ba 48:24
Ибо мы единственный прославленный народ, получивший Закон от Единого. И этот Закон, который посреди нас, поможет нам; его превосходная премудрость поддержит нас.
\vs 2Ba 48:25
И когда я произнес слова этой молитвы, я был в полном изнеможении.
\vs 2Ba 48:26
И Он отвечал мне, говоря: Ты молился в простоте, Варух, и все твои слова были услышаны.
\vs 2Ba 48:27
Но Мой суд взыскует своё, и Мой Закон взывает к своим правам.
\vs 2Ba 48:28
По твоим словам Я отвечу тебе и по твоей молитве Я буду говорить с тобою.
\vs 2Ba 48:29
Ибо так и есть: подлежащий тлению есть ничто. Он творит зло, как если бы он мог делать что-либо, и ни о Моей благости не помнит, ни постигает Моего долготерпения.
\vs 2Ba 48:30
Поэтому он отнимется, как Я говорил тебе прежде. Время, о котором я предсказал тебе, придет; время потрясения исполнится.
\vs 2Ba 48:31
Оно придет и пройдет в силе и ярости, посреди смятения, в буйстве и возмущении.
\vs 2Ba 48:32
В эти дни все живущие на земле поднимутся друг против друга, не зная, что Мой суд приблизился.
\vs 2Ba 48:33
Ибо в эти дни не найдется много мудрых; и разумные будут редки. Более того, те, кто знают больше всего, будут молчать.
\vs 2Ba 48:34
Поднимется много слухов и немало новостей; будут наблюдать явления призрачные, сообщать много пророчеств, иные из которых окажутся тщетными, иные подтвердятся.
\vs 2Ba 48:35
Честь обратится в позор, и сила будет унижена в презрение, уверенность ослабнет, красота станет отвратительной.
\vs 2Ba 48:36
И многие скажут друг другу в это время: Где спряталось изобилие разума? Куда сокрылось множество премудрости?
\vs 2Ba 48:37
И пока они будут размышлять об этом, ревность проявится против тех, кто об этом не думал. Спокойный будет обуреваем страстями, и многие будут в гневе вредить многим. Они поднимут войска для кровопролития, и в конце все они погибнут.
\vs 2Ba 48:38
И в это самое время каждому станет ясно, что времена меняются. Потому что во всё это время они осквернялись, творя жестокие дела, и каждый ходил по своим деяниям, и не вспоминал о Законе Элиона.
\vs 2Ba 48:39
Поэтому огонь пожрет их помыслы, и пламя испытает помышления их печени. Судья придет и не замедлит.
\vs 2Ba 48:40
Ибо каждый из живущих на земле творит зло сознательно; и по своей гордыне они пренебрегли Моим Законом.
\vs 2Ba 48:41
Многие тогда искренне заплачут о живых сильнее, чем о мертвых.
\vs 2Ba 48:42
И я отвечал, говоря: О Адам, что ты сделал для всех, кто родился от тебя? Что будет сказано Евве, которая первою послушала змея?
\vs 2Ba 48:43
Всё это множество идет к своей погибели; и нет счета тем, кого пожрет огонь.
\vs 2Ba 48:44
Но я буду еще говорить пред Тобою,
\vs 2Ba 48:45
ЯХВЕ, Бог мой. Ты знаешь, что есть Твое создание,
\vs 2Ba 48:46
ибо Ты некогда повелел праху образовать Адама, и Ты знаешь число родившихся от него, и то, сколь многие из бывших некогда, согрешили пред Твоим лицем, отказавшись признать Тебя своим Создателем.
\vs 2Ba 48:47
И их конец станет их обвинением, и Твой Закон, который они преступили, будет для них отмщением в Твой день.
\vs 2Ba 48:48
Но теперь оставим нечестивых и спросим о праведных.
\vs 2Ba 48:49
И я расскажу об их благополучии и не устану превозносить уготованную им славу.
\vs 2Ba 48:50
Ибо поистине, как в то краткое время, что вы живете в преходящем мире, вы понесли множество трудов, так и в мире, который не скончается, вы получите великий свет.

\vs 2Ba 49:1
И всё же я буду еще просить Тебя, о Шаддаи, и умолять Твоё милосердие, Создавший всё.
\vs 2Ba 49:2
В каком обличье будут жить те, кто увидит Твой День? Или что станет с блеском тех, кто переживет его?
\vs 2Ba 49:3
Обретут ли они вновь свой нынешний вид? Примут ли снова эти члены пленения, ныне преданные злу и через которые совершается зло? Или же Ты изменишь бывших в мире настолько же, насколько и самый мир?

\vs 2Ba 50:1
Он мне отвечал, говоря: Выслушай это слово, Варух, и запечатлей в памяти сердца всё, что узнаешь.
\vs 2Ba 50:2
Тогда земля возвратит мёртвых, которых она ныне принимает на хранение. Она не изменит ничего в их обличьи и вернет их такими, какими приняла, и какими Я их отдаю ей, такими она их воскресит.
\vs 2Ba 50:3
Ибо тогда надо будет показать живым, что мёртвые ожили вновь, и что те, кто ушел, возвратился вновь.
\vs 2Ba 50:4
И когда ныне знакомые узнaют друг друга, тогда суд обретёт силу, и предсказанное состоится.

\vs 2Ba 51:1
И когда пройдет этот назначенный день, тогда лишь изменятся обличья тех, кто будет осужден, и слава тех, кто будет оправдан.
\vs 2Ba 51:2
Обличье тех, кто ныне делает злое, явится худшим, чем было, ибо им надлежит испытывать мучения.
\vs 2Ba 51:3
Также и слава тех, кого ныне оправдывает Мой Закон, кто явит себя разумным в жизни и укоренил в своём сердце корень премудрости, их блеск прославится во время преображения, и образ их лиц изменится светлою красотой, чтобы они могли получить и принять мир, который не умирает и который обетован им на то время.
\vs 2Ba 51:4
Те, кто придут тогда, весьма возстенают оттого, что презрели Мой Закон и заткнули свой слух, чтобы не слышать премудрость и не принять разум.
\vs 2Ba 51:5
Когда же они увидят вознесенными и прославленными тех, над которыми они ныне так возносятся, а также то, что изменятся те и другие, одни в ангельское сияние, а они сами в страшные привидения, они будут полностью подавлены.
\vs 2Ba 51:6
Сначала они претерпят это зрелище, а после пойдут на муки.
\vs 2Ba 51:7
Но тем, кто будет спасен своими делами, и чаяния которых в Законе, надежды на разум, чья вера в премудрости, тем в их время будут явлены чудеса.
\vs 2Ba 51:8
Они увидят тот мир, что незрим ныне, они узрят время, которое ныне скрыто от них,
\vs 2Ba 51:9
время, которое уже не будет старить их.
\vs 2Ba 51:10
Они будут жить на вершинах этого мира, они будут подобны ангелам и сравниваемы со звездами. И они смогут принимать по желанию любое обличье, от красоты до сияния, и от света до торжества славы.
\vs 2Ba 51:11
Ибо пред ними откроются просторы Рая, и им будет явлена величественная красота живых существ под престолом, а также все ангельские воинства, ныне удерживаемые Моим словом от проявления и привязанные Моим повелением к своим местам, доколе им не придти.
\vs 2Ba 51:12
Но тогда праведникам будет принадлежать преимущество перед ангелами.
\vs 2Ba 51:13
Ибо первые примут последних, которых они ждали, а последние тех, о ком они привыкли слышать, как о предваривших их.
\vs 2Ba 51:14
Они освободились от этого мира скорби и отложили груз страданий.
\vs 2Ba 51:15
Почему же те люди привели свою жизнь к погибели? На что они, бывшие на земле, променяли свою душу?
\vs 2Ba 51:16
Ибо они тогда предпочли выбрать время, которое не может пройти без скорби и исход которого в печали и зле. Они отвергли мир, в котором не стареют те, кто в нём пребывает; они презрели время славы. Поэтому они и не достигли чести, которую Я предсказал тебе.

\vs 2Ba 52:1
И я отвечал, говоря: Намного ли они заблудили, те, кому уготовано проклятие?
\vs 2Ba 52:2
И почему мы носим еще траур по тем, кто умер, и оплакиваем тех, кто идет в Шеол?
\vs 2Ba 52:3
Лучше отложить плач до начала кары и удержать слезы до пришествия погибели.
\vs 2Ba 52:4
Но я скажу пред лицем этого:
\vs 2Ba 52:5
что ныне будут делать праведные?
\vs 2Ba 52:6
Радуйтесь в страданиях, которые вы испытываете ныне: зачем вам ожидать падения ваших врагов?
\vs 2Ba 52:7
Готовьтесь к тому, что уготовано вам и расположите ваши души для награды, вам предназначенной.
\vs 2Ba 52:8
И сказав так, я заснул на этом месте.

\vs 2Ba 53:1
И мне было видение. Вот, из моря поднялось огромное облако. Я смотрел на него, и оно было полно водою, черною и светлою, и подобие сильной молнии было видно над его вершиной.
\vs 2Ba 53:2
Я увидел, как быстро и скоро это облако подошло и покрыло всю землю.
\vs 2Ba 53:3
И потом это облако пролило на землю воды, бывшие в нём.
\vs 2Ba 53:4
И я увидел, что воды, падая на землю, были непохожи одна на другую.
\vs 2Ba 53:5
Сперва они были все черными до времени [и более плотными]. Затем я увидел, что они стали светлыми, но не плотными. Затем я увидел их черными, потом светлыми, потом черными и снова светлыми.
\vs 2Ba 53:6
Так повторялось двенадцать раз, но всякий раз черные воды были более плотными, чем светлые.
\vs 2Ba 53:7
И вот облако истаяло, пролившись дождем черных вод, еще более темных, чем прежде и смешавшихся с огнем. И куда падали воды, там они причиняли опустошение и разрушение.
\vs 2Ba 53:8
И потом я увидел, как молния, которую я видел на вершине облака, схватила его и спустила на землю.
\vs 2Ba 53:9
Эта молния была настолько яркою, что осветила всю землю и возстановила те места, что разрушили, падая, последние воды.
\vs 2Ba 53:10
И она заняла всю землю и овладела ею.
\vs 2Ba 53:11
И после этого я увидел: вот, двенадцать рек поднялись из моря и окружили молнию и служили ей.
\vs 2Ba 53:12
и я проснулся, охваченный страхом.

\vs 2Ba 54:1
Молитва Варуха. И я взмолился к Шаддаи, говоря: Ты один, ЯХВЕ, знающий заранее глубины мира. И Ты повелеваешь, и в своё время бывает то или иное по Твоему слову. Против дел тех, кто живет на земле, Ты ускоряешь начала времён, и Ты один знаешь концы эпох.
\vs 2Ba 54:2
Ты, для Кого ничто не трудно, совершаешь всё с легкостью, одним мановением.
\vs 2Ba 54:3
Пропасти и высоты обращаются к Тебе, и начала веков повинуются Твоему слову.
\vs 2Ba 54:4
Ты открываешь боящимся Тебя то, что ожидает их; и так Ты утешаешь их.
\vs 2Ba 54:5
Ты показываешь великие деяния тем, кто не знает; Ты отнимаешь занавес перед невеждами; Ты освещаешь тьму, и Ты открываешь тайное непорочным, тем, кто в вере послушен Тебе и Твоему Закону.
\vs 2Ba 54:6
Ты показал это видение Твоему рабу; дай мне также и истолкование.
\vs 2Ba 54:7
Ибо я знаю истинно, что о тех вещах, о которых я взыскивал ответ у Тебя, Ты открывал мне, как и то, каким гласом хвалить Тебя и какими членами возносить похвалу и аллилуйю.
\vs 2Ba 54:8
Ибо если все мои члены были бы устами, и все волосы на моей голове имели бы голос, даже и тогда я не смог бы воздать Тебе хвалу и возславить Тебя как должно, и не смог бы произнести Тебе хваления, ни прославить величие Твоей красоты.
\vs 2Ba 54:9
Ибо что я среди людей? И чем мне сочетаться с теми, кто превосходит меня? Ведь я слышал столько чудесного из уст Элиона и безчисленные вести от Того, Кто создал меня.
\vs 2Ba 54:10
Блаженна мать моя среди зачинающих! Да прославится между женами та, что произвела меня на свет!
\vs 2Ba 54:11
Я же не перестану восхвалять Шаддаи; я поведаю Его чудеса гласом хваления.
\vs 2Ba 54:12
Ибо кто может сравниться с Тобою в чудесах, как Твои, о Боже? Кто изследует глубину Твоего помысла о жизни?
\vs 2Ba 54:13
Ибо Ты управляешь в Твоей премудрости всем сотворенным Твоею десницею. Ты поставил рядом с Собою всякий источник света, и Ты уготовал ниже Твоего престола хранилища премудрости.
\vs 2Ba 54:14
Праведно погибают те, кто не возлюбил Твой Закон, и мука осуждения ожидает непокорившихся Твоему могуществу.
\vs 2Ba 54:15
Ибо хотя первым согрешил Адам и навел смерть на всех, кого еще не было в его время, однако и каждый из рожденых от него приготовил для себя грядущую муку или же избрал вечную славу.
\vs 2Ba 54:16
Ибо поистине, кто верит, получит свою награду.
\vs 2Ba 54:17
Но вы, творящие нечестивое ныне, возвращайтесь в истление: вы будете тяжко наказаны за то, что некогда отвергли познание Элиона.
\vs 2Ba 54:18
Его деяния ничему не научили вас, ни неизменное искусство Его творения не убедило вас.
\vs 2Ba 54:19
Ибо Адам повинен только в своей душе, но каждый из нас Адам для своей собственной души.
\vs 2Ba 54:20
Но Ты, ЯХВЕ, объясни мне открытое Тобою; наставь меня в вопросах, которые я задал Тебе.
\vs 2Ba 54:21
Ибо в конце света будет возмездие нечестивым по их нечестию, и Ты прославишь тех, кто будет верен в Твоей вере.
\vs 2Ba 54:22
Ибо Ты ведешь тех, кто принадлежит Тебе, и Ты вырываешь грешников из среды Твоего владения.

\vs 2Ba 55:1
И когда я окончил эту молитву, я сел под дерево отдохнуть в тени его ветвей.
\vs 2Ba 55:2
Я был поражен и изумлен, возвращаясь мыслию к величию благости, которое грешники на земле удалили от себя, и к тяжести муки, которую они презирают, зная, что они будут мучаться за свои грехи.
\vs 2Ba 55:3
И когда я думал об этом, и о другом, подобном этому, вот ангел Ремиил, вождь истинных видений, был послан ко мне. И он сказал мне:
\vs 2Ba 55:4
Отчего волнуется твое сердце, Варух? Почему смущаются твои мысли?
\vs 2Ba 55:5
Если одно лишь предвозвещение суда так тревожит тебя, то что будет, когда ты увидишь его совершающимся открыто на твоих глазах?
\vs 2Ba 55:6
Если ты столь подавлен ожиданием Дня Шаддаи, то что будет, когда ты достигнешь его пришествия?
\vs 2Ba 55:7
Если слово, возвещающее казнь падших, так тяготит тебя, каково же будет, когда чудеса начнут сбываться?
\vs 2Ba 55:8
Если, услышав имя благ и зол, грядущих в то время, ты затрепетал, то что будет, когда ты увидишь то, что откроет Величие, Которое обвинит одних и возвеселит других?

\vs 2Ba 56:1
Однако, поскольку ты просил Элиона открыть тебе истолкование увиденного тобою видения, я был послан говорить с тобою.
\vs 2Ba 56:2
Ибо Шаддаи особо показал тебе последовательность времен, которые прошли и еще имеют пройти в мире от начала его сотворения и до его завершения: среди них иные ложь, и иные истина.
\vs 2Ba 56:3
Ты видел большое облако, поднявшееся из моря и затем покрывшее землю, которому подобна долгота века, созданного Шаддаи, когда Он решил сотворить мир.
\vs 2Ba 56:4
И когда слово вышло от Него, долгота века пришла в бытие, и была крайне малою, и устроилась согласно изобилию разума Пославшего её.
\vs 2Ba 56:5
Подобно черным водам, которые ты видел вначале наверху облака и которые первыми пали на землю, было прегрешение Адама, первого человека.
\vs 2Ba 56:6
Ибо когда он преступил, безвременно явилась смерть, скорбь получила имя, боль была уготована, страдание было создано, труд достиг своего предела. И гордыня начала утверждаться. Шеол потребовал своего обновления в крови и похищал детей. И страсть родителей была создана. Величие человечества было унижено, и доброта зачахла.
\vs 2Ba 56:7
Что могло оказаться чернее и мрачнее этого?
\vs 2Ba 56:8
Таково начало черных вод, которые ты видел.
\vs 2Ba 56:9
Из этих черных вод в свой черед рождались другие черные воды, и тьма явилась над тьмою.
\vs 2Ba 56:10
Тот, кто был опасен для своей души, стал опасностью и для ангелов.
\vs 2Ba 56:11
Ибо когда он был создан, те наслаждались свободою.
\vs 2Ba 56:12
И когда он стал опасен, некоторые из них сошли и смешались с женщинами.
\vs 2Ba 56:13
И те, кто сделал это, были мучимы в цепях.
\vs 2Ba 56:14
Но остаток неисчислимого множества ангелов уцелел.
\vs 2Ba 56:15
А те, кто населял землю, погибли вместе в водах потопа. Это первые черные воды.

\vs 2Ba 57:1
После этого ты видел светлые воды. Это корень Авраамов и его потомство, пришествие его сына, и сына его сына, и тех, кто подобен им.
\vs 2Ba 57:2
Ибо в их время среди них был призываем неписанный Закон и исполнялись заповеди. Тогда родилась вера в будущий суд, и утвердилось упование на обновление мира, и обетование грядущей жизни укоренилось в сердцах.
\vs 2Ba 57:3
Это светлые воды, которые ты видел.

\vs 2Ba 58:1
Черные воды, которые ты видел в-третьих смесь всех грехов, соделанных народами после смерти этих праведников: нечестие земли Мицрейской, зло, которое они совершили, обратив в рабство их сынов.
\vs 2Ba 58:2
И, однако, в конце они также погибли.

\vs 2Ba 59:1
Светлые воды, которые ты видел в-четвертых, это пришествие Мойсея, Аарона, Мариами, Иошуа, сына Нунова, Халева и всех, подобных им.
\vs 2Ba 59:2
В эти дни светоч Вечного Закона осветил всех, пребывавших во тьме, означая для уверовавших обетование награды и для отступников уготованную им огненную кару.
\vs 2Ba 59:3
Но в эти времена также небеса закрылись для всякой земли, и стоявшие ниже престола Шаддаи пошатнулись, когда Он принял к Себе Мойсея.
\vs 2Ba 59:4
Он показал ему, как и тебе, множество наставлений, и вместе с ними Закон и исполнение времён, а также образ и размеры Сиона, который будет построен по образу нынешнего Святилища.
\vs 2Ba 59:5
Он показал ему также меру огня и глубину бездны, вес ветров, число капель дождя,
\vs 2Ba 59:6
повеление гневом, изобилие долготерпения, праведность суда,
\vs 2Ba 59:7
корень премудрости, богатство разума, источник знания;
\vs 2Ba 59:8
высоту воздуха, величие Рая, конец веков, начало Судного Дня,
\vs 2Ba 59:9
число приношений земли, что еще не возникли,
\vs 2Ba 59:10
пасть геенны, союз с местью, место веры, жилище надежды,
\vs 2Ba 59:11
образ будущей муки, неисчисимое множество ангелов, мощь пламени, блеск молний, голос грома, перемены времён и глубокое изучение Закона.
\vs 2Ba 59:12
Это светлые воды, которые ты видел в-четвертых.

\vs 2Ba 60:1
Черные воды, которые ты видел нисшедшими дождем на мир в-пятых, это дела Аморреев и их колдовские взывания, нечестие их таинств и осквернение их нечистотою.
\vs 2Ba 60:2
Но Израиль также осквернился грехом в дни Судий, несмотря на то, что видел множество знамений, совершенных Тем, Кто создал его.

\vs 2Ba 61:1
Светлые воды, которые ты видел в-шестых, это время, в которое родились Давыд и Соломон.
\vs 2Ba 61:2
В это время был возведен Сион и было освящено Святилище, была пролита кровь многих народов, грешивших тогда, и много приношений было сделано во время освящения Святилища.
\vs 2Ba 61:3
И мир и покой были в это время, и премудрость была слышна в собрании,
\vs 2Ba 61:4
и богатые разумом были прославляемы на сходбищах.
\vs 2Ba 61:5
И святые празднества соблюдались с великим приличием и радостью.
\vs 2Ba 61:6
И суд правителей выносился без коварства, и праведность заповедей исполнялась в истине.
\vs 2Ba 61:7
И земля была тогда возлюбленною ЯХВЕ, и, поскольку жившие на ней не грешили, она была прославлена над всеми землями и странами.
\vs 2Ba 61:8
Это светлые воды, которые ты видел.

\vs 2Ba 62:1
Черные воды, которые ты видел в-седьмых, это извращенность духа Иеровоама, решившегося воздвигнуть двух золотых тельцов,
\vs 2Ba 62:2
а также всё нечестие, сотворённое бывшими после него царями,
\vs 2Ba 62:3
проклятие Иезавели и идолопоклонство Израиля в то время,
\vs 2Ba 62:4
засуха и голод, столь великие, что женщины ели плод чрева своего;
\vs 2Ba 62:5
наконец, время плена, наставшее для девяти с половиною колен, которые предавались множеству грехов.
\vs 2Ba 62:6
Тогда пришел Салманасар, царь Ассирийцев, и увел их в плен.
\vs 2Ba 62:7
Но и о народах говорить тягостно, столь велики всегда были их нечестие и лукавство, и никогда они не явили праведности.
\vs 2Ba 62:8
Это черные воды, которые ты видел в-седьмых.

\vs 2Ba 63:1
Светлые воды, которые ты видел в-восьмых, это праведность и правда Езекии, царя Иуды, и милость, которая была ему дарована,
\vs 2Ba 63:2
когда Синнахериб был увлечен к своей погибели и ослеплён гневом настолько, что повел на гибель великое множество бывших у него народов.
\vs 2Ba 63:3
Когда Езекия узнал о намерениях царя Ассура захватить его, погубить оставшиеся два с половиною колена, а также разрушить Сион, Езекия положился на свои дела и понадеялся на свою праведность. Он сказал такое слово к Шаддаи:
\vs 2Ba 63:4
Смотри, вот Синнахериб готов нас погубить. Он будет хвалиться и превозноситься, разрушив Сион.
\vs 2Ba 63:5
И Шаддаи услышал его, ибо Езекия был премудр, и Он внял его молитве, ибо Езекия был праведен.
\vs 2Ba 63:6
Шаддаи дал тогда повеление Своему ангелу Ремиилу, который говорит с тобою.
\vs 2Ba 63:7
И я пошел погубить множество, одних вождей которого было сто восемьдесят тысяч, и каждый из них имел столько же воинов.
\vs 2Ba 63:8
В это время я спалил их тела изнури, оставляя нетронутым всё снаружи, вплоть до одежды и оружия. И так более явными открылись великие деяния Шаддаи, Имя Которого произносится по всей земле.
\vs 2Ba 63:9
И Сион был спасен, и Иеросалим освобожден, и Израиль избавлен от тревоги.
\vs 2Ba 63:10
Все, бывшие в святой земле, возрадовались, и Имя Шаддаи было прославлено и наречено по всей земле.
\vs 2Ba 63:11
Это светлые воды, которые ты видел.

\vs 2Ba 64:1
Черные воды, которые ты видел в-девятых, это всеобщее нечестие во времена Манассии, сына Езекии.
\vs 2Ba 64:2
Он содеял зло в изобилии. Он убивал праведников, он извращал суды, он проливал невинную кровь, осквернял и насиловал замужних женщин, ниспроверг жертвенники, прекратил жертвоприношения и изгнал священников, дабы они не служили более в Святилище.
\vs 2Ba 64:3
Он поставил идола с пятью лицами. Четыре из них смотрели на четыре ветра; пятое же венчало идола, как противника ревности Шаддаи.
\vs 2Ba 64:4
Тогда гнев вышел от Шаддаи, сильный искоренить Сион, как это было в ваши дни.
\vs 2Ba 64:5
И против двух с половиною колен вышло определение быть уведёнными в плен в точности так, как ты видел.
\vs 2Ba 64:6
Нечестие Манассии достигло таких пределов, что Слава Шаддаи удалилась от Святилища.
\vs 2Ba 64:7
Вот почему Манассия был назван нечестивым и в конце концов его жилищем стал огонь.
\vs 2Ba 64:8
Ибо, хотя его молитва и достигла Шаддаи, когда он был брошен в бронзового коня, и бронзовый конь раскололся, и в тот час ему было знамение,
\vs 2Ba 64:9
он не жил в совершенстве и не оказался достоин. Но с этого дня он будет знать, Кем в конце концов он будет мучим.
\vs 2Ba 64:10
Ибо могущий благотворить, может и карать.

\vs 2Ba 65:1
Так Манассия творил злое и всю свою жизнь думал, что Шаддаи не взыщет ни за что.
\vs 2Ba 65:2
Это черные воды, которые ты видел в-девятых.

\vs 2Ba 66:1
Светлые воды, которые ты видел в-десятых, это чистота поколений Иосии, царя Иуды, единственного в свое время явившегося покорным Шаддаи всем сердцем своим и всею душею своей.
\vs 2Ba 66:2
Он очистил землю от идолов и освятил все сосуды, которые были осквернены. Он возобновил приношения на жертвенниках, он вознёс рог святых, возвысил праведных, прославил всех премудрых за их разум; он вернул священникам их служение; он искоренил и уничтожил с земли волхвов, прорицателей и гадателей.
\vs 2Ba 66:3
Он не только умертивил нечестивцев, которые жили, но и вынул из гробов кости умерших и предал их огню.
\vs 2Ba 66:4
Он устроил празднества и субботы в их святости. Он сжег огнём осквернившихся. И лжепророков, обманывавших народ, он также предал огню. И толпу, покорную тем, он сбросил живьем в поток Кедронский и собрал камни над ними.
\vs 2Ba 66:5
Всею душею он предался ревности Шаддаи. Единственный в свое время, он был тверд в Законе, не терпя никого необрезанного, ни одного, кто творил бы нечестивое по всей земле, во все дни своей жизни.
\vs 2Ba 66:6
Таков тот, кто стяжал вечную награду; он будет прославлен у Шаддаи над многими в последнее время.
\vs 2Ba 66:7
Для него и для подобных ему были созданы и уготованы прекраснейшие почести; они были указаны тебе прежде.
\vs 2Ba 66:8
Это светлые воды, которые ты видел.

\vs 2Ba 67:1
Черные воды, которые ты видел в-одиннадцатых, это бич, поразивший ныне Сион.
\vs 2Ba 67:2
Ты думаешь, ангелы не опечалены пред Элионом, видя Сион преданным так, на виду у народов, которые похваляются в своём сердце и собираются перед своими идолами, говоря: Вот, ныне попираема ногами та, что так часто попирала других и низвергнута в рабство покорявшая?
\vs 2Ba 67:3
Ты думаешь, Элион радуется этому, или что Его Имя прославляется от этого?
\vs 2Ba 67:4
Но что тогда было бы с Его праведным судом?
\vs 2Ba 67:5
Однако после этого волнение охватит тех, кто разсеян среди народов; на всяком месте они будут жить в позоре.
\vs 2Ba 67:6
Ибо в то время, когда Сион предан, Иеросалим разрушен, а идолы благоденствуют в городах народов, ароматный дым благовоний затух на Сионе, и вот, повсюду в области Сионской возносится дым нечестия.
\vs 2Ba 67:7
И еще также царь Вавилонский, разрушивший ныне Сион, возвысится; он будет похваляться над народом. В своём сердце он произнесёт слова гордыни пред Элионом.
\vs 2Ba 67:8
Но и он падёт в конце.
\vs 2Ba 67:9
Это черные воды.

\vs 2Ba 68:1
Светлые воды, которые ты видел двенадцатыми, вот слово о них.
\vs 2Ba 68:2
Настанет время, когда на твой народ найдет столь великое испытание, что он едва не погибнет весь сразу.
\vs 2Ba 68:3
Однако, он будет спасен, и его враги падут перед ним.
\vs 2Ba 68:4
Некоторое время он пребудет в великой радости.
\vs 2Ba 68:5
В это время после краткого промежутка Сион будет отстроен вновь; и в нём снова принесут жертву, священники вернутся к своему служению, и народы придут воздать ему почести, но не так единодушно, как в первые времена.
\vs 2Ba 68:7
Но после этого падение поразит многие народы.
\vs 2Ba 68:8
Это светлые воды, которые ты видел.

\vs 2Ba 69:1
Последние воды, которые ты видел, более черные, чем все предыдущие, пришедшие после первых двенадцати вместе взятых, относятся ко всему миру.
\vs 2Ba 69:2
Ибо Элион отделил их с самого начала, и Он один знает, что будет.
\vs 2Ba 69:3
Он предвидел, что нечестивые злодейства пройдут пред Ним в шести образах, не считая того, что совершится в конце света.
\vs 2Ba 69:5
Вот почему Он не примешал черные воды к черным водам и светлые воды к светлым. Ибо это конец.

\vs 2Ba 70:1
Выслушай значение последних черных вод, которым надлежит придти после черных же, вот слово о них.
\vs 2Ba 70:2
Вот, грядут дни, и когда время достигнет зрелости, и придет пора собирать урожай злых и добрых семян. Элион приведет на землю, на живущих на земле и на её правителей, смятение духа и оцепенение сердца.
\vs 2Ba 70:3
Они будут ненавидеть друг друга и вызывать друг друга на бой, и низкие будут подчинять себе достойных, презренные возвысятся над почтенными.
\vs 2Ba 70:4
Великое множество будет предано малому числу. Бывшие ничем подчинят себе сильных, бедные превзойдут богатых в изобилии, нечестивые возобладают над героями.
\vs 2Ba 70:5
Мудрецы умолкнут, и глупцы будут говорить. Ни мысли человеческие, ни совет могучих не упрочатся, ни надежда уповающих не утвердится.
\vs 2Ba 70:6
Когда придет то, что предсказано, смятение распространится среди людей: одни падут на войне, другие умрут в скорбях, иные будут обманом захвачены своими ближними.
\vs 2Ba 70:7
Элион откроет народам уготованное Им заранее, и они придут сразиться с вождями, которые уцелеют тогда.
\vs 2Ba 70:8
И тот, кто убежит от войны, погибнет от мятежа; кого пощадит мятеж, тот сгорит в огне, а убежавший от огня, погибнет от голода.
\vs 2Ba 70:9
Те же, кто в конце выйдут целыми и невредимыми из предсказанных бед, победители и побежденные, будут живыми преданы в руки Моего Мессии. Ибо вся земля пожрет живущих на ней.

\vs 2Ba 71:1
Но святая земля помилует своих и пощадит живущих на ней в это время.
\vs 2Ba 71:2
Вот виденное тобою видение, и вот его истолкование.
\vs 2Ba 71:3
Я же пришел рассказать тебе это, ибо твоя молитва услышана ЯХВЕ.

\vs 2Ba 72:1
Выслушай еще о светлых водах, которые появятся в конце после этих черных вод.
\vs 2Ba 72:2
Как только явятся преждереченные знамения, народы смятутся, и придёт Мой Мессия. Он созовёт все народы, из них же иные спасет, а иные погубит.
\vs 2Ba 72:3
Вот что будет для народов, которые Он спасёт.
\vs 2Ba 72:4
Всякий народ, не познавший Израиля и не попиравший ногами семя Иакова, будет спасён.
\vs 2Ba 72:5
Так будет потому, что среди всех народов они повиновались твоему народу.
\vs 2Ba 72:6
Но те, кто правил вами или кто познал вас, будут преданы мечу.

\vs 2Ba 73:1
И когда Он унизит весь мир и возсядет в мире навеки на свой царственный трон, тогда откроется радость, и явится спокойствие;
\vs 2Ba 73:2
тогда здоровье снизойдет как роса, и болезнь удалится. Заботы, скорби и стоны уйдут далеко от людей; радость распространится по всей земле.
\vs 2Ba 73:3
Не будут более умирать преждевременно; никакая беда не поразит внезапно.
\vs 2Ba 73:4
Суды, обвинения, борьба, месть, преступление, ревность, ненависть и всё, что подобно им, будет выкорчевано и пойдёт получать своё осуждение.
\vs 2Ba 73:5
Ибо это они исполнили землю злом, и через них жизнь человеческая была весьма поколеблена.
\vs 2Ba 73:6
Дикие звери выйдут из леса и будут служить людям. Змеи и драконы вылезут из своих нор и будут послушны младенцу.
\vs 2Ba 73:7
Женщины не будут больше страдать вынашивая и не будут мучаться, давая жизнь плоду чрева своего.

\vs 2Ba 74:1
В эти дни жнецы не будут более знать усталости, ни утомления строители. Работа будет совершаться сама собою, скоро и спокойно, в согласии с теми, кто трудится.
\vs 2Ba 74:2
Ибо в это время прекратится тлен и начнется нетление.
\vs 2Ba 74:3
Поэтому исполнение предсказанного совершится в нём, и поэтому оно удалено от злых и совсем близко для тех, кто никогда не умрет.
\vs 2Ba 74:4
Это последние светлые воды, явившиеся за последними черными водами.

\vs 2Ba 75:1
И я отвечал, говоря: Кто сравнится с Тобою в благости, ЯХВЕ? Ибо она непревосходима.
\vs 2Ba 75:2
Кто изследует Твоё безконечное милосердие?
\vs 2Ba 75:3
Или кто постигнет Твой разум?
\vs 2Ba 75:4
Или кто может поведать мысли Твоего ума?
\vs 2Ba 75:5
Есть ли один среди рождённых, кто может надеяться на это, кроме того, над кем Ты сжалишься и кому Ты покровительствуешь?
\vs 2Ba 75:6
Не будь этой жалости Твоей к людям, которых прикрывает Твоя десница, они не смогут этого, кроме тех, кто назван в числе избранных по имени.
\vs 2Ba 75:7
Но мы, живые, если мы знаем причину, по которой мы пришли, и если мы повинуемся Тому, Кто вывел нас из Мицры, при нашем возвращении мы вспомним о прошлом и возрадуемся тому, что было.
\vs 2Ba 75:8
Если же, напротив, мы не знаем, почему мы пришли и не признаём власть Того, Кто поднял нас из Мицры, при нашем возвращении мы будем сожалеть о том, что происходит ныне и мы будем тяжко страдать о прошлом.

\vs 2Ba 76:1
Он отвечал мне, говоря: Поскольку по твоей молитве тебе изъяснилось откровение этого видения, выслушай слова Шаддаи и узнай, чтo будет после этих событий.
\vs 2Ba 76:2
Ты покинешь эту землю, но не умрешь, а будешь сохранен до [скончания] времён.
\vs 2Ba 76:3
Взойди же на вершину этой горы. Все места земли пройдут перед тобою: образ мира, вершины гор, глубины долин, бездны моря и великое множество рек. Так ты увидишь то, что ты оставляешь и то, куда ты идешь.
\vs 2Ba 76:4
Это случится через сорок дней.
\vs 2Ba 76:5
Теперь же, в эти дни, иди и научи народ, насколько ты сможешь, дабы они знали, что они не умрут в эти последние времена, но что они доживут до скончания времён.

\vs 2Ba 77:1
И я, Варух, оставил это место и ушел к народу. Я собрал их всех, от самого великого до самого малого, и сказал им:
\vs 2Ba 77:2
Слушайте, сыны Израиля, посмотрите, сколько вас уцелело из двенадцати колен Израилевых.
\vs 2Ba 77:3
Вам и вашим отцам ЯХВЕ дал Закон преимущественно перед всеми народами.
\vs 2Ba 77:4
Но поскольку ваши братья преступили заповеди Элиона, Он навлёк на вас, как и на них, отмщение. Он не пощадил первых, но последних Он также отдал в плен и не оставил остатка.
\vs 2Ba 77:5
И вот вы здесь, на этом месте со мною.
\vs 2Ba 77:6
Если вы исправите ваши пути, вы не уйдёте, как ушли ваши братья; скорее, они вернутся к вам.
\vs 2Ba 77:7
Ибо Он милосерд, Тот, Кому вы служите; Он Покровитель, Тот, на Кого вы надеетесь; Он верен, творящий доброе, а не злое.
\vs 2Ba 77:8
Не видели ли вы, что было с Сионом?
\vs 2Ba 77:9
Не думаете ли вы, что это место было разграблено потому, что оно согрешило, и земля сотворила неразумное и за это была предана?
\vs 2Ba 77:10
Не знаете ли вы, что из-за вас, грешники, она была разграблена и из-за нечестивых она была предана врагам, не сотворившая неразумного?
\vs 2Ba 77:11
И весь народ отвечал, говоря: Насколько мы в силах, мы помним благодеяния, оказанные нам Элионом. Те же, о которых мы забыли, Он помнит в Своем милосердии.
\vs 2Ba 77:12
Однако сделай милость для нас, твоего народа. Напиши и нашим братьям в Вавилон письмо учения, свиток надежды, дабы утвердить их, прежде, чем ты покинешь нас.
\vs 2Ba 77:13
Ибо пастыри Израиля погибли, и светочи, просвещавшие его, погасли. Изсяк источник, из которого мы пили. Мы остались во тьме, в густом лесу, в жажду пустыни.
\vs 2Ba 77:15
И я отвечал им, говоря: Пастыри, светочи и источники происходят от Закона. И если мы преходим, то Закон пребывает.
\vs 2Ba 77:16
Если разсматривая Закон, вы соблюдаете благоразумие в премудрости, светоч не отнимется от вас, ни пастырь не падёт, ни источник не изсякнет.
\vs 2Ba 77:17
Всё же я напишу вашим братьям в Вавилон, как вы просили меня, и пошлю с людьми, и напишу даже и девяти с половиною коленам и отправлю к ним с птицею.
\vs 2Ba 77:18
В двадцать восьмой день восьмого месяца я, Варух, пришел и сел под дубом в тени его ветвей. Никого не было со мною; я был один.
\vs 2Ba 77:19
И я написал эти два письма. Я отправил одно из них с орлом девяти с половиною коленам, и я послал другое с тремя людьми тем, кто был в Вавилоне.
\vs 2Ba 77:20
Я призвал орла и сказал ему такие слова:
\vs 2Ba 77:21
Элион создал тебя над прочими птицами.
\vs 2Ba 77:22
Ныне лети, не садясь нигде, ни в твоём гнезде и ни на каком дереве, прежде чем пролетишь над широкими водами Ефрата. Лети к народу, что проживает там. Оставь им это письмо.
\vs 2Ba 77:23
И помни, что во время потопа оливковую ветвь Ною принес голубь, которого он трижды выпускал из ковчега.
\vs 2Ba 77:24
Также и вoроны служили Илии, принося ему пищу, как он им приказывал.
\vs 2Ba 77:25
Даже Соломон во время своего царствования, когда хотел передать послание или спросить что-нибудь, посылал птицу, и та повиновалась ему неукоснительно.
\vs 2Ba 77:26
Теперь же не противься, не отклоняйся ни направо, ни налево. Но лети самым прямым путем и так исполни повеление Шаддаи, как я сказал тебе.

\vs 2Ba 78:1
Письмо, которое Варух, сын Нерии написал девяти с половиною коленам. Вот слова письма, которое Варух, сын Нерии, послал девяти с половиною коленам в изгнании за рекою, в котором было написано:
\vs 2Ba 78:2
Так говорит Варух, сын Нерии, братьям, уведённым в плен: милосердие и мир да будут с вами.
\vs 2Ba 78:3
Я храню в памяти любовь Создавшего нас, Возлюбившего нас искони; никогда мы не были ненавистны Ему, но превыше всего Он наставлял нас.
\vs 2Ba 78:4
Я истинно знаю, что мы, все двенадцать колен, связаны одними узами, так же как мы рождены от одного отца.
\vs 2Ba 78:5
Поэтому я весьма забочусь о том, чтобы оставить вам слова этого послания, прежду чем я умру: будьте утешены во зле, которое выпало вам, испытайте новую скорбь пред лицем испытаний, поразивших ваших братьев и воздайте справедливость приговору Того, Кто осудил вас на плен. Ибо то, что вы вынесли ничто по сравнению с тем, что вы сделали, лишь бы в последние времена вам оказаться достойными ваших отцов.
\vs 2Ba 78:6
Вот почему, если вы размыслите о том, что претерпели ныне для вашего блага, дабы не подвергнуться конечному осуждению и каре, вы обретёте вечное в надежде, если только вы упраздните из ваших сердец тщетное заблуждение, из-за которого вам пришлось уйти отсюда.
\vs 2Ba 78:7
Ибо если вы так сделаете, Он вспомнит верно о вас, Тот, Кто искони обещал ради нас лучшим, чем мы, никогда не забывать и не оставлять наше семя, но в Своём великом милосердии собрать вновь разсеянных.

\vs 2Ba 79:1
Теперь, братья мои, узнайте, чтo случилось с Сионом. Навуходресар, царь Вавилона, поднялся против нас.
\vs 2Ba 79:2
Поистине мы согрешили против Того, Кто создал нас, и не сохранили заповеди, которые Он дал нам. Но Он не наказал нас так, как мы заслужили.
\vs 2Ba 79:3
Ибо выпавшее вам претерпели и мы, и сильнее, ибо это же выпало и нам.

\vs 2Ba 80:1
И ныне, братья, я открываю вам, что когда город был окружен врагами, ангелы Элиона были посланы низвергнуть его стены и укрепления и уничтожить прочные углы стен, которые были выдернуты с корнем.
\vs 2Ba 80:2
Однако, они скрыли священные сосуды, дабы уберечь их от вражеского осквернения.
\vs 2Ba 80:3
Совершив это, они отдали врагам скрытые укрепления, Дом ограбленный, Храм сожженный и народ побеждённый, ибо отданный. И так враги не смогли сказать в своей гордыне: Наша мощь была такова, что мы разрушили войною самый Дом Элиона.
\vs 2Ba 80:4
И ваших братьев они также заковали в цепи и отвели их в Вавилон и поселили там.
\vs 2Ba 80:5
Нас осталось здесь весьма мало.
\vs 2Ba 80:6
Вот потрясение, о котором я пишу вам.
\vs 2Ba 80:7
Ибо я знаю поистине, каким утешением было для вас то, что Сион был заселен, и знание о том, что он благоденствует, было для вас более печали, которую вы испытывали далеко от него.

\vs 2Ba 81:1
Но выслушайте и слово облегчения.
\vs 2Ba 81:2
Ибо я оплакивал Сион и просил о милосердии Шаддаи, говоря:
\vs 2Ba 81:3
Будет ли это для нас безконечно длительным? Навечно ли постигли нас эти беды?
\vs 2Ba 81:4
И Шаддаи соделал в изобилии Своего милосердия, Элион по величию Своей благости. Он открыл мне слово для утешения, Он показал мне ведение, дабы не печалиться больше. Он явил мне таинства времён, и Он дал мне знать приходы эпох.

\vs 2Ba 82:1
Итак, братья, я написал вам, как обрести утешение во множестве ваших скорбей.
\vs 2Ba 82:2
Знайте, что Создавший нас отмстит за нас всем врагам по тому, что они нам сделали. И особенно знайте, что конец, дело Шаддаи, близок, так же, как и грядущее милосердие. Исполнение Его суда недалеко.
\vs 2Ba 82:3
Ибо мы стоим ныне при изобилии и процветании народов, в то время как они творят нечестивое.
\vs 2Ba 82:4
Мы видим величие их мощи и вместе их гнусности, но они будут подобны капле.
\vs 2Ba 82:5
Мы видим уверенность их силы и их непрестающее противление Шаддаи, но они будут сочтены за плевок.
\vs 2Ba 82:6
Мы думаем о славе их величия, в то время как они не соблюдают заповеди Элиона, и однако они пройдут как дым.
\vs 2Ba 82:7
Мы размышляем о блеске их красоты, тогда как они живут в скверне, но они изсохнут, как увядшая трава.
\vs 2Ba 82:8
Мы глядим на их силу и жестокость, хотя они не вспоминают о конце, но они исчезнут, как пробежавшая волна.
\vs 2Ba 82:9
Мы ясно видим гордыню их мощи, но они отвергают милость Того, Кто дал им эту мощь, и как проходит облако, пройдут и они.

\vs 2Ba 83:1
Ибо Элион ускорит явственно Своё время, и наведёт Свой век,
\vs 2Ba 83:2
и Он будет судить всех, кто есть в Его мире, Он поистине посетит всё и вся, потому как все их дела греховны.
\vs 2Ba 83:3
Он изследует сокровенные мысли и всё, что хранится в тайниках всех членов человеческих, и Он явит всё это перед всеми во время исправления.
\vs 2Ba 83:4
Пусть ничто из этого нынешнего не всходит вам на сердце, но будем ждать, ибо обетованное грядет.
\vs 2Ba 83:5
Не будем смотреть на нынешнее наслаждение народов, но будем помнить о том, что нам обетовано в конце.
\vs 2Ba 83:6
Ибо рубежи времен преходят, эпохи и всё, что в них, вместе.
\vs 2Ba 83:7
Конец мира откроет великое могущество Того, Кто управляет им, тогда как всякая вещь идет на суд.
\vs 2Ba 83:8
Вы же утверждайте ваши сердца в ожидании того, во что вы верили искони, дабы не оказаться вам далеко от обоих миров: здесь вы были уведены в плен, и будете мучимы там.
\vs 2Ba 83:9
Во всём, что есть ныне, во всём, что прошло, во всём, что будет, во всём этом ни зло не есть полностью зло, ни добро не есть полностью добро.
\vs 2Ba 83:10
Ибо всё, что ныне есть здоровье, становится болезнью,
\vs 2Ba 83:11
всякая сила ныне становится слабостью, всякая уверенность сегодня обращается в ничтожество.
\vs 2Ba 83:12
Крепость молодости обращается в старость и истление. Красота, ныне блестящая, увядает и делается отвратительною.
\vs 2Ba 83:13
Набухшая гордыня скоро обращается в уничижение и позор.
\vs 2Ba 83:14
Слово всякого превозношения ныне превращается в смятение и немоту, всякая нынешняя похвальба и наглость ведут к падению и нищете.
\vs 2Ba 83:15
Всякое нынешнее удовольствие и всякое удовлетворение превращаются в червей и тлен.
\vs 2Ba 83:16
Крики гордыни изменяются в прах и безмолвие.
\vs 2Ba 83:17
Всякое приобретение богатств ныне возвращает в одиночестве к Шеолу.
\vs 2Ba 83:18
Всё похищенное из жадности ныне ведет к нежеланной смерти, всякое вожделение приводит к осуждению на муку.
\vs 2Ba 83:19
Всякое лукавое притворство будет призвано на суд во имя Истины.
\vs 2Ba 83:20
Всякая нынешняя сладость умащений уступает место суду и осуждению.
\vs 2Ba 83:21
Всякая притворная дружба падёт в безмолвный стыд.
\vs 2Ba 83:22
И неужто вы думаете, что всё то, что творится у нас на глазах, останется без отмщения?
\vs 2Ba 83:23
Всякая вещь, достигая своего исполнения, приводит к Истине.

\vs 2Ba 84:1
Я же учил вас этому, пока я жил. Я сказал вам познавать заповеди Шаддаи, которым я наставлял вас, и прежде чем умереть, я представлю перед вашими глазами некоторые из заповедей Его судилища.
\vs 2Ba 84:2
Помните, что некогда Мойсей взял небо и землю в свидетели против вас: Если вы нарушите Закон, вы будете разсеяны, и если вы будете его соблюдать, вы укрепитесь.
\vs 2Ba 84:3
И он сказал вам и другие слова, когда вы были двенадцатью коленами, собранными вместе в пустыне.
\vs 2Ba 84:4
А после его смерти вы отвергли их. И так пророчества исполнились против вас.
\vs 2Ba 84:5
То, что Мойсей объяснил вам некогда еще до того, как это сбылось, вот, совершилось теперь, ибо вы оставили Закон.
\vs 2Ba 84:6
Я тоже говорю вам, после ваших страданий, что если вы уверуете в то, что сказано вам, вы получите от Шаддаи всё, что отложено и сохранено для вас.
\vs 2Ba 84:7
Пусть это письмо будет свидетелем между мною и вами, дабы вы помнили о заповедях Шаддаи. И так я смогу оправдаться перед Пославшим меня.
\vs 2Ba 84:8
Помните о Законе, о Сионе, а также о святой земле и ваших братьях. Не забывайте Завета, заключенного с вашими отцами, праздников и суббот.
\vs 2Ba 84:9
Передавайте это письмо и предания Закона вашим детям после вас, как вам их передали ваши отцы.
\vs 2Ba 84:10
Во всякое время верно взывайте и ревностно молитесь всею душею, чтобы Шаддаи сжалился над вами и не смотрел на множество ваших грехов, но, напротив, вспомнил о праведности ваших отцов. Ибо если Он не станет судить нас по изобилию Своего милосердия, горе нам, рожденным.

\vs 2Ba 85:1
Затем знайте, что в прежние времена и в прежние роды наши отцы имели поддержкою праведников и святых пророков.
\vs 2Ba 85:2
Но и мы были на нашей земле, и они помогали нам, когда мы грешили. Они молились за нас Создавшему нас, ибо они полагались на свои дела. Шаддаи внимал им и являл нам Свою милость.
\vs 2Ba 85:4
Но теперь праведники умерли, пророки упокоились и мы также покинули нашу землю; Сион был отнят у нас. У нас есть лишь Шаддаи и его Закон.
\vs 2Ba 85:4
Если же мы исправимся и настроим наши сердца, мы вновь обретём то, что мы утратили, и намного более, гораздо более того, что мы утратили.
\vs 2Ba 85:5
Ибо то, что мы утратили, было подвержено истлению; то, что мы получим, нетленно.
\vs 2Ba 85:6
И такие же слова я написал нашим братьям в Вавилон, чтобы засвидетельствать им всё это.
\vs 2Ba 85:7
Пусть все эти предсказания остаются перед вашими глазами на всякое время, ибо до сих пор мы живы и владеем нашей свободой.
\vs 2Ba 85:8
Шаддаи также терпелив к нам; Он открыл нам будущее и не скрыл грядущего в конце.
\vs 2Ba 85:9
И прежде чем Его суд не потребовал своего, и истина должного ей по праву, мы приготовим наши души к тому, чтобы принять и не быть отнятыми, уповать и не быть посрамлёнными, упокоиться вместе с нашими отцами и не быть мучимыми вместе с теми, кто нас ненавидит.
\vs 2Ba 85:10
Ибо юность мира прошла; сила творения ныне потребилась. Мало что остается еще до свершения проходящих времен. Кувшин у колодца, корабль близ гавани. Дорога завершается у города, а жизнь приближается к концу.
\vs 2Ba 85:11
Снова приготовьте ваши души к тому, чтобы, окончив плавание и сойдя с корабля, вам отдохнуть, и, прибыв на место, вам не оказаться осуждёнными.
\vs 2Ba 85:12
Ибо вот, Элион наведёт эти события. Тогда не будет больше места покаянию, рубежей времени, долготы веков, перемен к облегчению. Не будет более места мольбе, дара любви, [покаяния душе], ни ходатайства за прегрешения, ни умолений отцов, ни молитв пророков, ни помощи от праведных.
\vs 2Ba 85:13
Но будет только приговор к истязанию, путь в огонь, стезя в пекло.
\vs 2Ba 85:14
Вот почему един Закон, данный Единым, и един мир. И для всех, кто в нём, настаёт конец.
\vs 2Ba 85:15
Тогда Он спасёт тех, кого Он обретёт, и Он простит их. И тогда же Он погубит тех, кто оскверняется грехом.

\vs 2Ba 86:1
И когда вы получите письмо, читайте его прилежно в синагогах
\vs 2Ba 86:2
и размышляйте над ним, главным образом, в дни поста.
\vs 2Ba 86:3
И вспоминайте обо мне ради этого письма, так же как и я держу вас в памяти сейчас, когда и пишу, и каждую минуту.

\vs 2Ba 87:1
И когда я исполнил все слова этого письма и внимательно дописал его до конца, я свернул его, тщательно запечатал и укрепил на шее орла. Я освободил его и отправил.
\chhdr{Конец книги Варуха, сына Нерии.}

\bibbookdescr{3Ba}{
  inline={Третья Книга Пророка Варуха\fns{Переведена с греческого.}},
  toc={3-я Варуха},
  bookmark={3-я Варуха},
  header={3-я Варуха},
  abbr={3~Вар}
}
\vs 3Ba 1:1
Откровение Варуха, который стал у реки Гел, плача о пленении Иерусалима, когда Авимелех сохранен был рукой Божией в садах Агриппы. И так сидел он у Красных дверей, где пребывало Святое святых. Я, Варух, плакал в помышлении моем о народе, как же позволил Бог царю Навуходресару разрушить город Его, говоря:
\vs 3Ba 1:2
"Яхве, зачем выжег Ты виноградник Твой и опустошил его? Зачем сделал Ты это? И зачем, Яхве, не воздал Ты нам другим наказанием, но предал нас язычникам, чтобы надругались они, говоря: Где Бог их?"
\vs 3Ba 1:3
И вот, когда плакал я и говорил это, вижу я, Ангел Яхве пришел и говорит мне:
\vs 3Ba 1:4
"Слушай, муж желаний, не тревожься так о спасении Иеросалима, ибо вот что говорит Яхве, Бог Вседержитель: послал Он меня пред лице твое, чтобы возвестил я и явил тебе все Божественное, ибо молитва твоя услышана пред Ним и вошла в уши Адонаи Яхве".
\vs 3Ba 1:5
И когда он сказал мне это, успокоился я.
\vs 3Ba 1:6
И говорит мне Ангел: "Перестань раздражать Бога, и покажу я тебе другие тайны, большие этих".
\vs 3Ba 1:7
И сказал я, Варух: "Жив Адонаи Яхве, если покажешь мне и услышу я слова твои, уже не буду я больше говорить; да умножит Бог в день суда суд надо мною, если скажу что-либо впредь".
\vs 3Ba 1:8
И сказал мне Ангел сил: "Идем, покажу я тебе тайны Божии".

\vs 3Ba 2:1
И взяв меня, отнес он меня туда, где утверждено небо и где была река, которую никому не пересечь ни одному странствующему дуновению, из всех, что создал Бог.
\vs 3Ba 2:2
И взяв меня, отнес он меня к первому небу и показал мне превеликие врата. И сказал мне: "Войдем через них".
\vs 3Ba 2:3
И вошли мы словно на крыльях, преодолев расстояние примерно в тридцать дней пути.
\vs 3Ba 2:4
И показал он мне равнину, бывшую внутри этого неба, и были люди, жившие на ней: лица бычьи, рога оленьи, ноги козьи, а чресла бараньи.
\vs 3Ba 2:5
И вопросил я, Варух, Ангела: "Возвести мне, прошу тебя, какова толщина неба, где мы держим путь, и каково расстояние его от земли и что это за равнина, чтобы и я мог возвестить это сынам человеческим".
\vs 3Ba 2:6
И сказал мне Ангел, которому имя было Фамаил: "Врата, которые видишь ты, ведут на небо, и сколько от земли до него, такова и толщина его, и каково расстояние от севера до юга, такова длина равнины, которую ты увидел".
\vs 3Ba 2:7
И снова говорит мне Ангел сил: "Се, покажу я тебе и большие тайны".
\vs 3Ba 2:8
Сказал же я: "Прошу тебя, объясни мне, что это за люди?"
\vs 3Ba 2:9
И сказал он мне: "Это те, кто построили богопротивную башню, и за то удалил их Яхве".

\vs 3Ba 3:1
И взяв меня, отнес меня Ангел Яхве ко второму небу.
\vs 3Ba 3:2
И показал он мне и там врата, подобные первым, и сказал: "Войдем через них".
\vs 3Ba 3:3
И вошли мы, поднятые на крыльях, преодолев расстояние в шестьдесят дней пути, и показал мне он там равнину, и была она полна людей, видом же они походили на собак, а ноги оленьи.
\vs 3Ba 3:4
И вопросил я Ангела: "Прошу тебя, господин, скажи мне, кто эти люди?"
\vs 3Ba 3:5
И сказал он: "Это те, кто дали совет построить башню.
\vs 3Ba 3:6
Сами они, кого ты видишь, выгнали множество мужчин и женщин для изготовления кирпичей.
\vs 3Ba 3:7
Женщине одной, делавшей кирпичи, когда пришло ей время родить, не позволили они уйти, но, делая кирпичи, родила она и ребенка своего носила в полотенце, и делала кирпичи.
\vs 3Ba 3:8
И явившись им, Яхве изменил языки их, когда башня достигала высоты в триста шестьдесят три локтя.
\vs 3Ba 3:9
И взяв бурав, стали они стараться пробуравить небо, говоря: "Посмотрим, глиняное небо, медное или железное".
\vs 3Ba 3:10
Увидев это, Бог не позволил им, но поразил их слепотой и разноязычием и оставил их как ты их видишь".

\vs 3Ba 4:1
И сказал я, Варух: "Се, господин, великое и чудесное показал ты мне. И сейчас покажи мне все ради Яхве".
\vs 3Ba 4:2
И сказал мне Ангел: "Отправимся дальше".
\vs 3Ba 4:3
И отправился я с Ангелом дальше от места этого примерно на сто восемьдесят пять дней пути, и показал он мне равнину и змея, длиной, как мне показалось, около двух сотен плетров.
\vs 3Ba 4:4
И показал он мне ад, и вид его был мрачный и непотребный.
\vs 3Ba 4:5
И сказал я: "Что это за дракон, и что за дикость вокруг него?"
\vs 3Ba 4:6
И сказал Ангел: "Дракон этот есть пожирающий тела живущих неправедной жизнью, ими же он питается, то же, что вокруг него ад, который сам подобен ему, где он пьет из моря примерно с локоть, а воды нисколько не убывает".
\vs 3Ba 4:7
Сказал Варух: "Как же так?"
\vs 3Ba 4:8
И сказал Ангел: "Слушай: Яхве Бог сотворил триста шестьдесят рек, из которых из всех первые Алфей, Авир и Гирик. И, беря от них, не убывает вода в море".

\vs 3Ba 5:1
И сказал я, Варух, Ангелу: "Спрошу я тебя об одном, господин: когда уж сказал ты мне, что выпивает дракон из моря локоть, скажи мне: какова глубина чрева его?"
\vs 3Ba 5:2
И сказал Ангел: "Чрево его ад, и сколько пролетает свинец, пущенный тремястами мужами, таково и чрево его. Идем, и я покажу тебе дела еще большие этих".

\vs 3Ba 6:1
И взяв меня, отнес он меня туда, где начинает свой путь Солнце.
\vs 3Ba 6:2
И показал он мне колесницу с четверной упряжью, и вырывался из-под нее огонь, и сидел на колеснице муж в огненном венце, и влекли ту колесницу сорок Ангелов.
\vs 3Ba 6:3
И вот, впереди Солнца кружила птица величиной с девять гор.
\vs 3Ba 6:4
И сказал я Ангелу: "Что это за птица?" И говорит он мне: "Она хранитель вселенной".
\vs 3Ba 6:5
И сказал я: "Господин, как это хранитель вселенной? Объясни мне".
\vs 3Ba 6:6
И сказал мне Ангел: "Птица эта летит вместе с Солнцем и, раскинув крылья, принимает лучи его, которые подобны языкам пламени. И если бы не принимала она их, не уцелел бы род человеческий, и вообще ничто живое, но приставил Бог эту птицу".
\vs 3Ba 6:7
И раскинула она крылья свои, и увидел я на правом крыле ее буквы весьма великие, каждая словно гумно, величиной около четырех тысяч модиев, и были те буквы золотые.
\vs 3Ba 6:8
И сказал мне Ангел: "Прочти их".
\vs 3Ba 6:9
И прочел я, и гласили они: "Не земля рождает меня и не небо, а рождают меня крылья огненные".
\vs 3Ba 6:10
И сказал я: "Господин, что это за птица, и как имя ее?" И сказал мне Ангел: "Феникс имя ее".
\vs 3Ba 6:11
И сказал я: "А что ест она?" И сказал он мне: "Манну небесную и росу земную".
\vs 3Ba 6:12
И сказал я: "Испражняется ли птица эта?" И сказал он мне: "Испражняется червяком, а из испражнений червяка получается корица та, что употребляют цари и правители. Но помедли, и увидишь славу Божию".
\vs 3Ba 6:13
И пока говорил он, послышался словно бы раскат грома, и поколебалось место, на котором мы стояли.
\vs 3Ba 6:14
И вопросил я Ангела: "Господин мой, что это за шум?" И сказал мне Ангел: "Сейчас открывают Ангелы триста шестьдесят пять врат небесных, и выходит через них свет из тьмы".
\vs 3Ba 6:15
И пришел голос, говорящий: "Податель света, дай миру свет!"
\vs 3Ba 6:16
И услышав звук, издаваемый птицей, сказал я: "Господин, что это за звук?"
\vs 3Ba 6:17
И сказал он: "Звук, который пробуждает на земле петухов, ибо петух, подобно вторым устам, оповещает мир о наступлении утра своей песней. Ангелы приготовили Солнце вот и кричит петух".

\vs 3Ba 7:1
И сказал я: "А где пребывает Солнце после того, как кричит петух?"
\vs 3Ba 7:2
И сказал мне Ангел: "Слушай, Варух: все, что показал я тебе, находится на первом и втором небе. И проходит Солнце по третьему небу, и дает миру свет. Но подожди, и увидишь славу Божию".
\vs 3Ba 7:3
И в тот самый миг, когда говорил он, вижу я птицу, и вновь появилась она предо мною, и мало-помалу увеличивалась она и вырастала, а позади нее блистающее Солнце и с ним Ангелы, несущие венец над головой его, вида и лицезрения которого я не мог вынести.
\vs 3Ba 7:4
И только засияло Солнце, как раскинул Феникс крылья свои. Я же, увидев подобную славу, умалился от великого страха и бежал, и спрятался среди крыльев Ангела.
\vs 3Ba 7:5
И сказал мне Ангел: "Не бойся, Варух, но подожди, и тогда увидишь еще и заход их".

\vs 3Ba 8:1
И взяв меня, отнес он меня к Западу.
\vs 3Ba 8:2
И когда пришло время захода Солнца, снова вижу я впереди летящую птицу.
\vs 3Ba 8:3
И только приблизилась она, вижу я Ангелов, и убрали они венец от головы Солнца.
\vs 3Ba 8:4
Птица же стала, присмирев, и сложила крылья свои.
\vs 3Ba 8:5
И увидев это, сказал я: "Господин, зачем убрали они венец от головы Солнца, и отчего так присмирела птица?"
\vs 3Ba 8:6
И сказал мне Ангел: "Венец Солнца, после того, как прошло оно свой дневной путь, забирают четыре Ангела и уносят на небо, и обновляют его, ибо осквернился он и лучи его на земле. Да и вовсе каждый день обновляется он подобным образом".
\vs 3Ba 8:7
И сказал я, Варух: "Господин, а из-за чего оскверняются лучи его на земле?"
\vs 3Ba 8:8
И сказал мне Ангел: "Взирая на людские беззакония и неправедности: блуд, разврат, кражи, разбой, идолопоклонство, пьянство, убийства, вражду, ревность, злословие, ропот, наушничество, гадание и другие вещи, неугодные Богу.
\vs 3Ba 8:9
Из-за них оно оскверняется и поэтому обновляется.
\vs 3Ba 8:10
О птице же, почему она так присмирела: это потому, что сдерживает лучи Солнца, из-за огня и жара в течение всего дня вот из-за чего она присмирела.
\vs 3Ba 8:11
Ведь если бы ее крылья, о чем уже говорил я тебе, не прикрывали кругом солнечных лучей, не уцелело бы ни одно дыхание".

\vs 3Ba 9:1
И когда сложила она крылья, настала ночь с Луной и со звездами.
\vs 3Ba 9:2
И сказал я, Варух: "Господин, покажи мне и Луну, прошу тебя, то, как восходит она и как заходит, и в каком виде идет по небу".
\vs 3Ba 9:3
И сказал Ангел: "Дождись увидишь и ее спустя короткое время".
\vs 3Ba 9:4
И день спустя вижу я и ее в виде женщины, сидящей на колесе колесницы.
\vs 3Ba 9:5
И были впереди нее тельцы и агнцы запряжены в колесницу, и сонм Ангелов также.
\vs 3Ba 9:6
И сказал я: "Господин, кто эти тельцы и агнцы?" И сказал он мне: "И они тоже Ангелы".
\vs 3Ba 9:7
И сказал я: "А почему не светит она всегда, но только ночью?"
\vs 3Ba 9:8
И сказал Ангел: "Слушай: как челядь не может смело говорить в присутствии царя, так пред лицом Солнца не могут воссиять Луна и звезды. Ибо звезды висят всегда, однако прикрыты Солнцем.
\vs 3Ba 9:9
И Луна, оставаясь целой и невредимой, истощается солнечным жаром".

\vs 3Ba 10:1
И когда узнал я это все от Архангела, взяв, отнес он меня на четвертое небо.
\vs 3Ba 10:2
И увидел я плоскую равнину, и посреди нее озеро вод.
\vs 3Ba 10:3
И были там сонмища птиц всех родов, и не были птицы эти похожи на тех, что здесь, но увидел я журавля величиной с больших тельцов, превосходящего всех крупных животных, какие существуют в мире.
\vs 3Ba 10:4
И вопросил я Ангела: "Что это за равнина, и озеро, и что за множество птиц вокруг него?"
\vs 3Ba 10:5
И сказал Ангел: "Слушай, Варух: равнина эта вмещает озеро и прочие чудеса, которые есть на ней. Тут, водя бесконечные хороводы, ходят и беседуют друг с другом души праведников.
\vs 3Ba 10:6
Вода же та, беря которую, облака проливаются на землю дождем, и возрастают плоды".
\vs 3Ba 10:7
И снова сказал я Ангелу Яхве: "А птицы?"
\vs 3Ba 10:8
И сказал он мне: "Они те, которые ежечасно прославляют Яхве".
\vs 3Ba 10:9
И сказал я: "Господин, как же люди говорят, что из моря вода, проливающаяся на землю?"
\vs 3Ba 10:10
И сказал Ангел: "Дождевая вода та, что из моря, и из земных вод, и вот эта. Та же ее часть, которая дает рост плодам от этой воды. И еще узнай: от нее то, что люди называют росой небесной".

\vs 3Ba 11:1
И после того, взяв меня, отнес меня Ангел на пятое небо. И были заперты врата.
\vs 3Ba 11:2
И сказал я: "Господин, разве не отворятся ворота эти, чтобы мы вошли?"
\vs 3Ba 11:3
И сказал мне Ангел: "Не можем мы войти, пока не пришел Михаил, хранитель ключей от Царства Небесного. Подожди же, и увидишь славу Божию".
\vs 3Ba 11:4
И раздался голос громкий, как раскат грома. И сказал я: "Господин, что это за голос?"
\vs 3Ba 11:5
И сказал он мне: "Сейчас сойдет архистратиг Михаил, чтобы принять молитвы людей".
\vs 3Ba 11:6
И вот, пришел голос: "Да отворятся врата!" И отворились они, и раздался скрежет громкий, как при ударе грома.
\vs 3Ba 11:7
И пришел Михаил, и выступил навстречу ему Ангел, бывший со мной, и поклонился ему, и сказал: "Радуйся, архистратиг мой и всего нашего войска!"
\vs 3Ba 11:8
И сказал архистратиг Михаил: "Радуйся, брат наш и тот, кто толкует откровения живущим праведной жизнью".
\vs 3Ba 11:9
И так приветствовав друг друга встали они.
\vs 3Ba 11:10
И увидел я архистратига Михаила держащим великую чашу: глубина ее сколько есть расстояния от неба до земли, ширина ее сколько от севера и до юга.
\vs 3Ba 11:11
И сказал я: "Господин, что это держит Михаил архангел?"
\vs 3Ba 11:12
И сказал он мне: "Это чаша, куда приходят добродетели праведников и все благие поступки, совершаемые ими, которые затем доставляются пред лицем Бога Небесного".

\vs 3Ba 12:1
И еще говорил я с ними, как вот, пришли Ангелы, неся корзины, наполненные цветами. И отдали они их Михаилу.
\vs 3Ba 12:2
И вопросил я Ангела: "Господин, кто они и что есть приносимое ими?" И сказал он мне: "Это Ангелы-власти". И взяв, опрокинул архангел корзины в чашу.
\vs 3Ba 12:3
И говорит мне Ангел: "Цветы эти добродетели праведников".
\vs 3Ba 12:4
И увидел я других Ангелов, несущих корзины пустые и ненаполненные.
\vs 3Ba 12:5
И шли они печальные, и не осмелились приблизиться, потому что не имели наград совершенных.
\vs 3Ba 12:6
И воззвал Михаил, говоря: "Ну же и вы, Ангелы, несите, что принесли".
\vs 3Ba 12:7
И огорчился Михаил и Ангел, бывший со мной, потому что не наполнили они чашу.

\vs 3Ba 13:1
И так же затем пришли другие Ангелы, плача и сетуя, и со страхом говорили: "Взгляни, как почернели мы, господин, ибо преданы мы дурным людям и желаем уйти от них".
\vs 3Ba 13:2
И сказал Михаил: "Не можете вы уйти от них, чтобы враг не завладел ими окончательно. Но скажите мне, чего вы просите?"
\vs 3Ba 13:3
И сказали они: "Просим тебя, Михаил, архистратиг наш, переместить нас от них, ибо не в силах мы оставаться при людях дурных и безрассудных, ибо нет в них ничего доброго, но всяческая неправедность и корыстолюбие.
\vs 3Ba 13:4
Ведь ни разу не видели мы их входящими в собрание, или чтобы сделать что-то во благо, но где убийство они тут как тут, и где блуд, разврат, кражи, злословие, клятвопреступления, зависть, пьянство, вражда, ревность, ропот, наушничество, идолопоклонство, гадание и все тому подобное, там и они делатели таких вот дел и других, еще худших. Потому просим мы позволения уйти от них".
\vs 3Ba 13:5
И сказал Михаил Ангелам: "Подождите, пока я узнаю у Яхве, чему быть".

\vs 3Ba 14:1
И в тот самый миг отошел Михаил, и затворились врата. И был голос словно раскат грома.
\vs 3Ba 14:2
И вопросил я Ангела: "Что это за голос?"
\vs 3Ba 14:3
И сказал он мне: "Сейчас приносит Михаил добродетели человеческие Богу".

\vs 3Ba 15:1
И в то самое мгновение пришел Михаил, и отворились врата. И принес он елей.
\vs 3Ba 15:2
И Ангелам, принесшим полные корзины, наполнил он их елеем, говоря: "Отнесите, воздайте стократ друзьям нашим и тем, кто в трудах совершил благие дела. Ибо посеявшие как должно, собирают должную жатву".
\vs 3Ba 15:3
И говорит он и тем, что держат пустые корзины: "Давайте же и вы, заберите мзду по тому, что принесли, и воздайте сынам человеческим".
\vs 3Ba 15:4
И говорит он потом принесшим полные и принесшим пустые: "Пойдите и воздайте хвалу друзьям нашим и скажите им так:
\vs 3Ba 15:5
Вот что говорит Яхве: в малом верны вы Ему, надо многими поставит Он вас, войдите в радость Господа вашего".

\vs 3Ba 16:1
И повернувшись, говорит он и тем, которые ничего не принесли: "Вот что говорит Яхве: Не будьте унылы и не плачьте, не оставьте же и сынов человеческих, но когда прогневили они Меня делами своими, пойдя, не дайте им покоя, и прогневите их, и огорчите народ неразумный.
\vs 3Ba 16:2
Еще вдобавок к этому нашлите гусеницу, ливень, ржавчину, саранчу и град с молниями и громом, и рассеките их надвое мечом и смертью, и детей их демонами, ибо не услышали они голоса Моего, не соблюли заповедей Моих и не сделали по ним, но презрели заповеди Мои и оскорбили священников, возвещающих им слова Мои".

\vs 3Ba 17:1
И при этих словах затворились врата, и мы отступили.
\vs 3Ba 17:2
И взяв меня, опустил он меня на то место, с какого отправились мы в путь.
\vs 3Ba 17:3
И придя в себя, принялся я возносить славу Богу, удостоившему меня столь великой чести.

\include{tex/Mis}
\bibbookdescr{Azp}{
  inline={Апокалипсис Софонии},
  toc={Апокалипсис Софонии},
  bookmark={Апокалипсис Софонии},
  header={Апокалипсис Софонии},
  abbr={Ап~Соф}
}
\chhdr{Явление на 5-м небе.}
\vs Azp 0:0
И Дух взял меня и вознес меня на 5-е небо.
И я увидел ангелов, которых называют Господствами.
И диадема была возложена на них в Святом Духе,
и трон каждого из них блестел в 7 раз больше,
чем свет восходящего солнца.
И они жили в храмах спасения и пели гимны невыразимому Богу.
\chhdr{Саидский фрагмент\\Пророческое видение души в мучении.}
\vs Azp 1:1
Я увидел душу, которую 5000 ангелов наказывали и стерегли.
\vs Azp 1:2
Они взяли её на Востоке и принесли её на Запад.
Они били её, они давали ей 100 ударов плетью, каждый ежедневно.
\vs Azp 1:3
Я испугался, и я пал на лицо своё,
так что мои составы распались.
\vs Azp 1:4
Ангел помог мне.
Он сказал мне:
<<Крепись, о тот, кто победит и восторжествует,
потому что ты восторжествуешь над обвинителем и взойдёшь из Шеола.>>
\vs Azp 1:5
И после того, как я поднялся, я сказал:
<<Кто это, кого они наказывают?>>
\vs Azp 1:6
Он сказал мне:
<<Это душа, которая была найдена в её беззаконии.
И прежде, чем она пришла в раскаяние,
она была посещена и взята из её тела.>>
\vs Azp 1:7
Истинно, я, Цефания, видел эти вещи в моём видении.

\chhdr{Явление на широком месте.}
\vs Azp 1:8
И ангел Яхве пошёл со мной.
Я видел большое широкое место,
тысячи тысяч окружали его на его левой стороне,
и тьмы тем на его правой стороне.
Вид каждого был различным.
\vs Azp 1:9
Их волосы были распущены как это свойственно женщинам.
Их зубы были похожи на зубы \ldots
\chhdr{Ахмимский текст\\Фрагмент о погребении.}
\vs Azp 1:10
<<\ldots\ мёртвый.
Мы похороним его подобно всякому человеку.
\vs Azp 1:11
Каждый раз, когда он умирает, мы будем выносить его,
играя на кифаре перед ним и воспевая псалмы и оды над его телом.>>
\chhdr{Явления сверху города пророка.}
\vs Azp 2:1
Вот, я пошёл с ангелом Яхве,
и он вознёс меня над всем моим городом.
Сначала ничего не было перед моими глазами.
\vs Azp 2:2
Потом я увидел двух мужей, шедших вместе по одной дороге.
Я наблюдал за ними как они разговаривали.
\vs Azp 2:3
И, кроме того, я также увидел двух женщин,
мелющих вместе на мельнице.
И я наблюдал за ними как они разговаривали.
\vs Azp 2:4
И я также увидел двоих на постели,
каждого из них делавшего себе взаимно~--- на постели.
\vs Azp 2:5
И я увидел всю вселенную, висящую как капля воды,
которая свешивается с ведра, когда оно поднимается из колодца.
\vs Azp 2:6
Я сказал ангелу Яхве:
<<Разве тьмы или мрака нет в этом месте?>>
\vs Azp 2:7
Он сказал мне:
<<Нет, потому что нет тьмы в том месте,
где праведные и святые, а вернее они всегда пребывают во свете.>>
\vs Azp 2:8
И я увидел все души человеческие, как они пребывали в наказании.
\vs Azp 2:9
И я воскликнул к Господу Всемогущему:
<<O Боже, если ты обитаешь со святыми, ты сожалеешь о мире и о душах,
которые в этом наказании!>>
\chhdr{Ангелы с горы Сеир, делающие запись.}
\vs Azp 3:1
Ангел Яхве сказал мне:
<<Приди, я покажу тебе место праведности.>>
\vs Azp 3:2
И он вознёс меня на гору Сеир, и он показал мне 3-х мужей,
как 2 ангела шли с ними, радуясь и ликуя о них.
\vs Azp 3:3
Я сказал ангелу:
<<Какого рода они?>>
\vs Azp 3:4
Он сказал мне:
<<Они~--- 3 сына Йотама священника,
которые не хранят заповедь своего отца,
ни соблюдают повелений Яхве.>>
\vs Azp 3:5
Тогда я увидел, что 2 других ангела плакали
по 3-м сыновьям Йотама священника.
\vs Azp 3:6
Я сказал:
<<O ангел, кто они?>>
Он сказал:
<<Они~--- ангелы Господа Всемогущего.
Они записывают все хорошие дела праведных в свои свитки,
так как они бодрствуют у врат небесных.
\vs Azp 3:7
И я беру их из их рук и возношу их перед Господом Всемогущим.
Он записывает их имя в Книгу Жизни.
\vs Azp 3:8
Подобно ангелы обвинителя, который на земле:
они также записывают все грехи людей в свои свитки.
\vs Azp 3:9
Они также сидят у врат небесных.
Они сообщают обвинителю и он записывает их в свой свиток,
для того чтобы обвинить их, когда они выйдут из мира туда.
\chhdr{Отвратительные ангелы уносят души безбожников.}
\vs Azp 4:1
Потом я шёл с ангелом Яхве.
Я взглянул перед собой и я увидел там место.
\vs Azp 4:2
Тысячи тысяч и тьмы тем ангелов проходили через него.
\vs Azp 4:3
Их лица были подобны леопарду,
их клыки были снаружи их пасти~--- как у диких кабанов.
\vs Azp 4:4
Их глаза были смешаны с кровью.
Их волосы были распущены как волосы у женщин,
и огненные бичи были в их руках.
\vs Azp 4:5
Когда я увидел их, я испугался.
Я сказал тому ангелу, который шёл со мной:
<<Какого рода они?>>
\vs Azp 4:6
Он сказал мне:
<<Они~--- служители всей твари, которые приходят
к душам безбожников и приносят их,
и оставляют их в этом месте.
\vs Azp 4:7
Прежде чем они приносят их и бросают их
на их вечное наказание, они проводят с ними 3 дня,
ходя всюду в воздухе.>>
\vs Azp 4:8
Я сказал:
<<Я умоляю тебя, о господин, не дай им власти прийти ко мне.
\vs Azp 4:9
Ангел сказал:
<<Не бойся; я не позволю им прийти к тебе,
потому что ты чист перед Яхве.
Я не позволю им прийти к тебе,
потому что Господь Всемогущий послал меня к тебе,
потому что ты чист перед ним.
\vs Azp 4:10
Тогда он дал знак им, и они удалились, и они бежали от меня.
\chhdr{Небесный город.}
\vs Azp 5:1
Однако я пошёл с ангелом Яхве,
и я взглянул перед собой, и я увидел ворота.
\vs Azp 5:2
Вот, когда я приблизился к ним, я обнаружил,
что эти ворота были бронзовыми.
\vs Azp 5:3
Ангел коснулся их и они открылись перед ним.
Я вошёл с ним и нашёл целый квартал,
подобный красивому городу, и я шёл посреди него.
\vs Azp 5:4
Тогда ангел Яхве преобразился возле меня в том месте.
\vs Azp 5:5
\ldots\ Вот, я взглянул на них, и я обнаружил,
что эти ворота были бронзовые, и замки бронзовые,
а засовы железные.
\vs Azp 5:6
Вот, мои уста были заключены в них.
Я заметил бронзовые ворота впереди меня
как бы извергающие огонь приблизительно на 50 стадий.
\chhdr{Еремиэл~--- ангел и обвинитель в Шеоле.}
\vs Azp 6:1
Опять я возвратился и шёл,
и я увидел великое море.
\vs Azp 6:2
Но я думал, что это было море водяное.
Я обнаружил, что это море было полностью из огня,
подобно слизи, которая изливает множество пламени
и чьи волны жгут кал и асфальт.
\vs Azp 6:3
Они начали приближаться ко мне.
\vs Azp 6:4
Тогда я подумал, что Господь Всемогущий пришёл посетить меня.
\vs Azp 6:5
Вот, когда я увидел, я пал на лицо моё перед ним,
чтобы поклониться ему.
\vs Azp 6:6
Я был очень сильно напуган, и я умолял его,
чтобы он спас меня от этого бедствия.
\vs Azp 6:7
Я воскликнул, говоря:
<<Элои, Яхве, Адонай, Цебаот!
Я молю тебя, спаси меня от этого бедствия,
потому что это происходит со мной.>>
\vs Azp 6:8
В тот же самый момент я встал,
и я увидел великого ангела передо мной.
Его волосы были распростёрты как у львиц.
Его зубы были снаружи его пасти как у медведя.
Его волосы были распростёрты как у женщин.
Его тело было подобно змеиному,
когда он хотел поглотить меня.
\vs Azp 6:9
И когда я увидел его, я так испугался его,
что все части моего тела ослабли, и я пал на лицо моё.
\vs Azp 6:10
Я не мог стоять, и я молился перед Господом Всемогущим:
<<Спаси меня от этого бедствия.
Ты тот, кто спас Израиля от руки Фараона, царя Египетского.
Ты спас Сусанну от руки старцев неправедности.
Ты спас 3-х святых мужей, Шадрака, Мешака, Абед-Него,
из печи, горящей огнем.
Я прошу тебя спасти меня от этого бедствия.>>

\vs Azp 6:11
Тогда я поднялся и стал, и я увидел,
что другой великий ангел стоял передо мной своим лицом,
сияющим подобно лучам солнца в его славе,
так как его лицо похоже на то,
которое совершенно в своей славе.
\vs Azp 6:12
И он был опоясан, как если бы золотой пояс был на его груди.
Его ноги были подобны бронзе, которая расплавляется в огне.
\vs Azp 6:13
И когда я увидел его, я обрадовался, ибо я подумал,
что Господь Всемогущий пришёл посетить меня.
\vs Azp 6:14
Я пал на лицо моё, и я поклонился ему.
\vs Azp 6:15
Он сказал мне:
<<<Берегись. Не поклоняйся мне.
Я~--- не Господь Всемогущий, но великий ангел Еремиэл,
который поставлен над Абаддоном и Шеолом, тем,
в котором все души заключены с конца Потопа,
нашедшего на землю, до сего дня.>>
\vs Azp 6:16
Тогда я спросил у ангела:
<<Что это за место, куда я пришел?>>
Он сказал мне:
<<Это Шеол.>>
\vs Azp 6:17
Тогда я спросил его:
<<Кто тот великий ангел, стоящий так, которого я видел?>>
Он сказал:
<<Это тот, кто обвиняет людей перед лицом Яхве.>>
\chhdr{Два свитка.}
\vs Azp 7:1
Тогда я взглянул, и я увидел его со свитком в его руке.
Он начал разворачивать его.
\vs Azp 7:2
Вот, после того, как он развернул его,
я прочитал его на моём собственном языке.
Я нашёл, что все мои грехи, которые я сделал,
были написаны в нём,~--- те,
которые я сделал от моей юности до сего дня.
\vs Azp 7:3
Все они были написаны на том свитке моём,
не было ложного слова в нём.
\vs Azp 7:4
Если я не ходил посетить больного или вдову,
я нашёл это записанным в мою рукопись как проступок.
\vs Azp 7:5
Если я не посещал сироту, это было найдено записанным
в мой свиток как проступок.
\vs Azp 7:6
День, в который я не постился или не умолял во время молитвы,
я нашёл записанным в мой свиток как падение.
\vs Azp 7:7
И день, когда я не обращался
к сынам Израиля~--- так как это проступок~--- я нашёл
записанным в мой свиток,
\vs Azp 7:8
так что я пал на лицо моё и молился перед Господом Всемогущим:
<<Простри на меня твою милость и изгладь мой свиток,
потому что милость твоя приходит, чтобы пребывать
во всяком месте, и наполняет всякое место.>>
\vs Azp 7:9
Потом я поднялся и встал,
и я увидел великого ангела передо мной,
говорящего мне:
<<Торжествуй, побеждай, потому что ты превозмог
и восторжествовал над обвинителем,
и ты поднимешься из Шеола и Абаддона.
Ныне ты перейдёшь место перехода.>>
\vs Azp 7:10
Опять он принёс другой свиток,
который был написан рукой.
\vs Azp 7:11
Он начал разворачивать его,
и я читал его,
и нашёл его написанным на моём собственном языке \ldots
\chhdr{Оставление Ада.}
\vs Azp 8:1
Они помогли мне и поставили меня на ту лодку.
\vs Azp 8:2
Тысячи тысяч и тьмы тем ангелов воздали хвалу передо мной.
\vs Azp 8:3
Сам я облёкся ангельским одеянием.
Я видел всех тех ангелов молящимися.
\vs Azp 8:4
Сам я молился вместе с ними.
\vs Azp 8:5
Я понимал их язык, на котором они говорили со мной \ldots
\vs Azp 8:6
<<Вот, более того, сыновья мои, это искушение, потому что нужно,
чтобы добро и зло были взвешены на весах.>>
\chhdr{Первая труба: торжество и посещение праведников.}
\vs Azp 9:1
Потом явился великий ангел,
имея золотой рог в руке своей,
и он протрубил им 3 раза над моей головой, сказав:
<<Мужайся, о тот, кто торжествует!
Превозмогай, о тот, кто превозмогает!
Ибо ты торжествуешь над обвинителем,
и ты избегаешь Абаддона и Шеола.
\vs Azp 9:2
Hыне ты перейдёшь место перехода.
Ибо твоё имя написано в Книге Жизни.
\vs Azp 9:3
Я хотел обнять его, но я не мог обнять великого ангела,
потому что его слава велика.
\vs Azp 9:4
Потом он побежал ко всем праведникам,
то есть Аврааму и Исааку, и Иакову, и Еноху, и Илии, и Давиду.
\vs Azp 9:5
Он беседовал с ними как друг говорит с другом.
\chhdr{Вторая труба: открытие небес и д\acc{у}ши в мучении.}
\vs Azp 10:1
После этого великий ангел пришёл ко мне
с золотым рогом в своей руке,
и он затрубил в него к небесам.
\vs Azp 10:2
Небеса открылись с того места, где солнце восходит,
до места, где оно заходит, и с севера на юг.
\vs Azp 10:3
Я увидел море, которое я видел на дне Шеола.
Его волны поднимались к облакам.
\vs Azp 10:4
Я увидел все души, тонущие в нём.
Я увидел некоторых,
руки которых были привязаны к их шее,
их руки и ноги были скованы.
\vs Azp 10:5
Я сказал:
<<Кто они?>>
Он сказал мне:
<<Они те, кто был подкуплен,
и им давали золото и серебро,
пока души людей были введены в заблуждение.>>
\vs Azp 10:6
И я увидел других, покрытых огненными рогожами.
\vs Azp 10:7
Я сказал:
<<Кто они?>>
Он сказал мне:
<<Они те, кто давал деньги в рост, и они получали лихву за лихву.>>
\vs Azp 10:8
И я также увидел некоторых слепых вопиющих.
И я был изумлен, когда я увидел все эти дела Божьи.
\vs Azp 10:9
Я сказал:
<<Кто они?>>
Он сказал мне:
<<Они обученные, которые услышали слово Божье,
но не усовершенствовались в деле, о котором они услышали.>>
\vs Azp 10:10
И я сказал ему:
<<Значит они не имеют здесь покаяния?>>
Он сказал:
<<Да>>.
\vs Azp 10:11
Я сказал:
<<Как долго?>>
Он сказал мне:
<<До того дня, когда Яхве будет судить.>>
\vs Azp 10:12
И я увидел иных с их волосами на них.
\vs Azp 10:13
Я сказал:
<<Зачем волосы и тело в этом месте?>>
\vs Azp 10:14
Он сказал:
<<Да, Яхве даёт тело и волосы им, как он хочет.>>
\chhdr{Заступничество святых за тех, кто в мучении.}
\vs Azp 11:1
И я также увидел множества.
Он породил их.
\vs Azp 11:2
Так как они смотрели на все мучения,
они взывали, молясь перед Господом Всемогущим,
говоря:
<<Мы молимся тебе за тех, кто пребывает
во всех этих мучениях, чтобы ты пощадил всех их.>>
\vs Azp 11:3
И когда я увидел их, я сказал ангелу, который говорил со мной:
<<Кто они?>>
\vs Azp 11:4
Он сказал:
<<Они те, кто умоляет Яхве,~--- Авраам, Иссак и Иаков.
\vs Azp 11:5
Тогда на некоторое время они ежедневно являются с великим ангелом.
Он даёт трубный глас к небесам, а другой возглашает на землю.
\vs Azp 11:6
Все праведники слышат глас.
Они сбегаются, ежедневно молясь к Господу Всемогущему
за тех, кто пребывает во всех этих мучениях.
\chhdr{Другой рог: грядущий гнев Божий.}
\vs Azp 12:1
И опять великий ангел явился с золотым рогом в своей руке,
трубя над землёй.
\vs Azp 12:2
Они услышали его с места восхода солнца до места заката,
и от южных областей до северных областей.
\vs Azp 12:3
И опять он трубит им к небесам, и слышен его глас.
\vs Azp 12:4
Я сказал:
<<O господин, почему бы тебе не оставить меня,
пока я не увижу их всех?>>
\vs Azp 12:5
Он сказал мне:
<<Я не имею власти показать их тебе,
пока Господь Всемогущий не восстанет в своём гневе,
чтобы уничтожить землю и небо.
\vs Azp 12:6
Они увидят и будут встревожены,
и все они воскликнут, говоря:
<<Всякую плоть, которая предана тебе,
мы вверим тебе в день Яхве.>>
\vs Azp 12:7
Кто устоит перед лицом его,
когда он восстанет в своём гневе,
чтобы уничтожить небо и землю?
\vs Azp 12:8
Всякое дерево, которое растёт на земле,
будет исторгнуто с его корнями и падёт.
И всякая высокая башня и птицы летающие падут \ldots

\bibbookdescr{Ars}{
  inline={Письмо Аристея},
  toc={Письмо Аристея},
  bookmark={Письмо Аристея},
  header={Письмо Аристея},
  abbr={Арист}
}
\chhdr{Аристей Филократу.}
\vs Ars 1:1
Так как у нас имеется заслуживающее внимания повествование о посольстве к иудейскому первосвященнику Елеазару, а ты, Филократ, при всяком случае напоминал, что считаешь важным знать, для чего и почему мы были посланы, то я, зная твою любознательность, попытался изобразить тебе.
\vs Ars 1:2
Самое важное для человека это всегда учиться и приобретать что-либо новое, путем ли исторических повествований, или путем собственного опыта. Ибо чистое настроение души приобретается в том случаe, если она, усвоив прекраснейшее и одобрив то, что важнее всего, устраивает благочестие при помощи твердого правила.
\vs Ars 1:3
Имея склонность к тщательному размышлению о божественном, мы посвятили себя посольству к выше упомянутому мужу, который своим благородством и славой снискал особую честь как у сограждан, так и иноземцев, и принес величайшую пользу иудеям Палестины и других мест переводом божественного Закона, потому что написан у них на пергаменте еврейскими буквами.
\vs Ars 1:4
Это-то мы и выполнили со всяким тщанием. Следует сообщить тебе и о том, что мы говорили царю получив удобный случай о переселенных в Египет из Иудеи отцом царя, прежнем владельце столицы и владыке Египта.
\vs Ars 1:5
Я убежден, что ты, имея значительное расположениe к нравственной чистоте и душевной настроенности мужей, живущих согласно священному законодательству, охотно услышишь о том, что мы желаем сообщить, так как ты недавно приходил к нам с острова и выражал желание узнать о том, что способствует исправлению души.
\vs Ars 1:6
И ранее я отправил тебе описание замечательного, по моему мнению, об иудейском народе, полученное нами от ученейших жрецов в Египте.
\vs Ars 1:7
А так как ты любознателен в том, что может принести пользу душе, то необходимо передать по преимуществу всем единомышленникам, а тем более тебе, ибо у тебя подлинное расположение; ты брат не только по родству, но и по настроенности, влечение к прекрасному у нас одно и то же.
\vs Ars 1:8
Ведь удовольствие от золота, или какое-либо иное имущество, почитаемое пустыми, не приносит столько пользы, как образование и забота о нем. Дабы не впасть в мнoгocлoвиe, удлиняя предисловие, мы вернемся к дальнейшему ходу повествования.
\vs Ars 1:9
Димитрий Фалирей, заведующий царской библиотекой, получил крупные суммы на то, чтобы собрать, по возможности, все книги мира. Скупая и снимая копии, он, по мере сил, довел до конца желание царя.
\vs Ars 1:10
Однажды в нашем присутствии он был спрошен, сколько у него тысяч книг, и ответил: свыше двухсот тысяч, царь, а в непродолжительном времени я позабочусь об остальных, чтобы довести до пятисот тысяч. Но мне сообщают, что и законы иудеев заслуживают того, чтобы их переписать и иметь в твоей библиотеке.
\vs Ars 1:11
Что же препятствует тeбе, спросил, сделать это? Ведь в твоем распоряжении есть всё, касающееся этого дела!. Димитрий ответил: необходим еще перевод, так как среди иудеев пользуются особым письмом, подобно тому как египтяне своим расположением букв, почему имеют и особый язык. Предполагают, что говорят на сирийском, но их не этот, а иного типа. Узнав обо всем, царь повелел написать иудейскому первосвященнику, чтобы привел в исполнение этот план.
\vs Ars 1:12
А я, считая настоящий момент удобным, просил начальников телохранителей Сосивия тарентинца и Андрея о том же, о чем часто об освобождении переселенных из иудеи отцом царя действительно, он столь же удачно, как и храбро напал на всю территорию нижней Сирии и Финикии и одних переселил, а других взял в плен, всё подчинив, благодаря страху. В это время он и переселил около ста тысяч из Иудеи в Египет.
\vs Ars 1:13
Около тридцати тысяч из них, лучших воинов, он, вооружив, поселил в крепостях своей страны (хотя много и раньше прибыло с персидским царем, а до этого и иные были отправлены на помощь Псаммитиху, чтобы сражаться против эфиопского царя, но их прибыло не так много, как переселил Птолемей, сын Лага).
\vs Ars 1:14
Выбрав, как мы сказали, цветущих возрастом и отличающихся силою, он вооружил, а остальную массу старцев, юношей, а также женщин, он обратил в рабство, не столько по собственному желанию, сколько по требованию воинов, за услуги, которые они оказали на войнe. А так как мы, о чем ранее сказано, получили известный предлог к освобождению их, то обратились к царю с такими словами:
\vs Ars 1:15
Царь, не будь настолько безразсуден, чтобы тебя обличали сами факты. Ведь законодательство, которое мы намереваемся не только переписать, но и перевести, имеет силу для всех иудеев; какое же основание у нас будет для отправления, если в твоем царстве огромная масса находится в рабстве? Освободи же, по совершенству и богатству души, угнетаемых бедствиями, ибо, как я тщательно изследовал, Бог, управляющий твоим царством, даровал закон и им.
\vs Ars 1:16
Они, царь, чтут Зрителя всяческих и создателя Бога, Которого почитают и все, а мы иначе называем Его Зевсом и Дием. Древние дали это удачное наименование Тому, Кем оживотворяется и создано всё; Он же управляет и владычествует над всем. А так как величием души ты превосходишь всех людей, то освободи находящихся в рабстве.
\vs Ars 1:17
Подождав немного, когда мы в душе молились Богу, чтобы Он внушил ему мысль об освобождении всех, ибо человеческий род, творение Бога, Он переменяет и снова изменяет. Поэтому я часто и многообразно призывал Владыку сердец, чтобы Он побудил исполнить то, чего я просил.
\vs Ars 1:18
A выступая с речью о спасении людей, я твердо надеялся, что Бог исполнит просимое; ибо, если люди делают по благочестию то, что, по их мнению относится к справедливости и попечению о прекрасном, то владычествующий над всем Бог руководит их действиями и намерениями он, подняв голову и милостиво взглянув, спросил:
\vs Ars 1:19
сколько будет тысяч, по твоему мнению? Присутствовавший Андрей ответил: немногим более ста тысяч. Царь сказал: немногого же просит у нас Аристей. А Сосивий и некоторые из присутствующих ответили это: действительно, величия твоей души достойно принести величайшему Богу в качестве благодарственной жертвы освобождение их. Так как Владыкой всяческих ты удостоен высочайшей чести и прославлен более твоих предков, то тебе следует принести в благодарность и величайшую жертву.
\vs Ars 1:20
Сильно обрадованный, приказал добавить к жалованью и за каждого человека получать по двадцать драхм; издать об этом указ, а списки изготовить немедленно. Он обнаружил величайшее расположение, ибо Бог исполнил все наши желания и побудил его освободить не только тех, которые пришли с войском его отца, но и тех, которые жили ранее, или впоследствие были приведены в государство, хотя ему и заявляли, что дар обойдется более четырехсот талантов.
\vs Ars 1:21
A копия указа была сделана, по моему мнению, не напрасно, ведь великодушие царя будет гораздо яснее и очевиднее, ибо Бог дал ему возможность послужить спасению множества. Содержание же указа таково:
\vs Ars 1:22
По приказанию царя, те из соратников нашего отца в Сирии и Финикии, которые при нападении на Иудею захватили пленников иудеев и переселили их в столицу и страну, или продали иным, точно также, если некоторые жили ранее, или впоследствие были приведены оттуда, владеющие должны немедленно отпустить на свободу, получив тотчас же по двадцать драхм за человека: воины при выдаче жалованья, а остальные из царской казны.
\vs Ars 1:23
Ибо по нашему мнению они были взяты в плен и вопреки воле отца нашего и вопреки благородству, а страна их была опустошена и иудеи были переселены в Египет вследствие запальчивости воинов. Ведь добыча, захваченная воинами на поле брани, была достаточно велика, почему и порабощение этих людей совершенно несправедливо.
\vs Ars 1:24
Итак, воздавая, по общему мнению, справедливое всем людям, а особенно угнетаемым неразумно, и во всём стремясь к полному coглacию со справедливостью и благочестием в отношении всех, мы определяем всех иудеев нашего государства, каким бы то ни было образом в рабстве, отпустить на свободу, уплатив владельцам назначенную сумму. Никто не должен медлить исполнением этого; а списки доставить назначенным для этого в течение трех дней со времени издания настоящего указа, предъявляя вместе с тем и самих людей.
\vs Ars 1:25
Ибо мы решили, что осуществление этого полезно и нам и государству. А о неповинующихся должен доносить всякий желающий, с условием, что он станет господином того, кто окажется виновным, а имущество таковых будет взято в царскую казну.
\vs Ars 1:26
Когда этот указ, в котором находилось всё, кроме: и если некоторые жили ранее, или впоследствие были приведены оттуда, был подан царю для просмотра, то он, по своему благородству и великодушию, это добавил сам и приказал дать назначение казначеям легионов и царским менялам на всю сумму издержек.
\vs Ars 1:27
В таком виде это постановление было утверждено в течение семи дней, а выкупная сумма достигла более шестисот талантов, ибо много и грудных детей было освобождено вместе с их матерями. Когда же к царю обратились с запросом, выдавать ли и за них по двадцать драхм, то царь приказал делать и это, выполняя во всём его волю полностью.
\vs Ars 1:28
Когда это было исполнено, приказал Димитрию сделать доклад относительно копии иудейских книг. (Ибо эти цари управляли всем при посредстве указов и с великой осмотрительностью. Вот почему я помещаю копии доклада и писем, также количество отправленного и устройство их, так как все они отличались роскошью и искусством.) Копия доклада такова:
\vs Ars 1:29
Великому царю от Димитрия.
Так как ты, царь, для пополнения отсутствующих в твоей библиотеке книг приказал собрать, а распавшиеся надлежащим образом исправить, то я, тщательно потрудившись над этим, доношу тебе следующее:
\vs Ars 1:30
отсутствуют, наряду с немногими другими, книги иудейского закона. Они написаны еврейскими буквами и языком, но, как сообщают знающие, слишком небрежно и не так, как должно, ибо не привлекали царского внимания.
\vs Ars 1:31
Teбе необходимо иметь у себя и эти, но тщательно исправив, ибо это законодательство, как божественное, чисто и исполнено мудрости? Поэтому, прозаики, поэты и многие историки были далеки от упоминания о названных книгах и о мужах, которые управлялись на основании их, так как, по словам Экатея Авдиритского, учение в них чисто и священно.
\vs Ars 1:32
Итак, если, царь, угодно, пусть напишут первосвященнику в Иерусалиме, чтобы он прислал старцев особенно добродетельной жизни, сведущих в своём Законе, по шести от каждого колена, чтобы, достигнув coгласия по большинству и получив точный перевод, мы положили на видном месте, достойно и самого дела и твоего намерения. Будь счастлив всегда.
\vs Ars 1:33
После этого доклада царь приказал написать об этом Елеазару, сообщив и об освобождении пленников. А для изготовления сосудов, бокалов, трапезы и чаш для возлияния он дал золота весом пятьдесят талантов, серебра семьдесят талантов и достаточное количество драгоценных камней (ибо он приказал хранителям сокровищ, чтобы они предоставили мастерам выбирать, что те пожелают) и денег для жертв и на остальное около ста талантов.
\vs Ars 1:34
Об изготовлении мы скажем после того, как передадим копии писем. Письмо царя было такого содержания:
\vs Ars 1:35
Царь Птолемей первосвященнику Елеазару радоваться и здравствовать!
Так как в нашу страну было переселено много иудеев, силою уведенных из Иеросалима персами во время их господства, а кроме того пленников прибыло в Египет и вместе с отцом нашим
\vs Ars 1:36
(большинство их он зачислил в войско на большое жалованье, равным образом и тем, которые жили paнee, он, по доверию к ним, поручил охрану построенных им крепостей, чтобы, таким образом, египтяне были в безопасности; а мы, получив царскую власть, проявили большее человеколюбие по отношению ко всем, а особенно твоим согражданам),
\vs Ars 1:37
то мы освободили более ста тысяч пленников, уплатив их господам следуемую денежную плату и исправляя вместе с тем зло, причиненное им яростью черни. Мы решили, что этим поступаем благочестиво и приносим благодарственную жертву величайшему Богу, Который сохраняет наше царство в мире и величайшей славе во всей вселенной. Зрелых возрастом мы зачислили в войско, а пригодных для нашей службы и заслуживающих доверия при дворе мы определили на должности.
\vs Ars 1:38
Желая сделать угодное и им, и иудеям всего миpa и последующим, мы предрешили перевести ваш Закон греческими буквами с букв, называемых у вас еврейскими, чтобы в нашей библиотеке, наряду с другими царскими книгами, находились и эти.
\vs Ars 1:39
Поэтому ты поступишь прекрасно и согласно нашему желанию, если выберешь старцев добродетельной жизни, сведущих в Законе и сильных в переводе, по шести от каждого колена, чтобы достигнуть согласия по большинству, ибо изследование касается очень важных предметов. Мы полагаем, что, исполнив это, ты приобретешь себе великую славу.
\vs Ars 1:40
Для этого мы посылаем Андрея, начальника телохранителей, и Аристея, которые пользуются у нас почетом; они будут вести с тобою переговоры и доставят начатки приношений в храм, а для жертв и на остальное сто талантов серебра. А сообщив нам о желаниях, ты приобретешь благосклонность и поступишь согласно дружбе, так как мы возможно скорее исполним то, что тебе угодно. Будь здоров.
\vs Ars 1:41
На это письмо Елеазар тотчас же ответил следующее:
Первосвященник Елеазар царю Птолемею, истинному другу радоваться!
Нам приятно было бы, если бы ты, царица Арсиноя, твоя сестра и дети были здоровы; этого мы и желаем, а мы здоровы.
\vs Ars 1:42
Получив от тебя письмо, мы весьма возрадовались твоему намерению и прекрасному желанию; собрав весь народ, мы прочли ему, чтобы он знал о твоем благоговении к нашему Богу. Мы показали и присланные тобою бокалы, двадцать золотых и тридцать серебряных, пять сосудов, трапезу для возношения и сто талантов серебра для принесения жертв и необходимых исправлений в храме.
\vs Ars 1:43
Это доставили пользующиеся у тебя почетом Андрей и Аристей, мужи добрые, прекрасные, отличающиеся образованием и во всём достойные твоего настроения и справедливости. Они-то и передали нам твоё, на что и с нашей стороны услышали соответствующее твоему письму.
\vs Ars 1:44
Мы исполним всё, что полезно для тебя, даже если бы это было противно природе (ведь это свидетельствует о дружбе и любви), ибо и ты оказал нашим согражданам великие, разнообразные и никогда не забываемые благодеяния.
\vs Ars 1:45
Поэтому мы тотчас же принесли жертвы за тебя, твою сестру, детей и любезных), и весь народ молился, чтобы исполнилось всё, что тебе угодно, чтобы владычествующий над всем Бог сохранил твоё царство в мире и славе и чтобы перевод святого Закона был сделан с пользою для тебя и тщательно.
\vs Ars 1:46
В присутствии всех мы избрали старцев, мужей добрых и благородных, из каждого колена по шести; их мы отправили вместе с Законом. А ты, праведный царь, прекрасно поступишь, приказав, чтобы эти мужи, по окончании перевода книг, снова безпрепятственно вернулись к нам. Будь здоров.
\vs Ars 1:47
Из первого колена: Иосиф, Езекия, Захария, Иоанн, Езекия, Елисей.
Из второго: Иуда, Симон, Самуил, Адей, Матафия, Есхлемия.
Из третьего: Неемия, Иосиф, Феодосий, Васея, Орния, Дакис.
\vs Ars 1:48
Из четвертого: Ионафан, Аврей, Елисей, Анания, Хаврий, 3ахария.
Из пятого: Исаак, Иаков, Иесуа, Савватий, Симон, Левий.
Из шестого: Иуда, Иосиф, Симон, Захария, Самуил, Шелемия.
\vs Ars 1:49
Из седьмого: Савватий, Седекия, Иаков, Исайя, Иесия, Натфей.
Из восьмого: Феодосий, Иасон, Иесуа, Феодот, Иоанн, Ионафан.
Из десятого: Феофил, Авраам, Арсам, Иасон, Эндемия, Даниил.
\vs Ars 1:50
Из десятого: Иеремия, Елеазар, Захария, Ванея, Елисей, Дафей.
Из одиннадцатого: Самуил, Иосиф, Иуда, Ионафан, Хавев, Досифей.
Из двенадцатого: Исаил, Иоанн, Феодосий, Арсам, Авиит, Иезекииль.
Всего семьдесят два.
\vs Ars 1:51
Таков был ответ Елеазара и его приближенных на письмо царя.
Согласно обещанию, я опишу тебе также приготовленное. Они были сделаны с необыкновенным искусством, так как царь отпустил большие средства и всегда наблюдал за мастерами. Поэтому они ничего не могли упустить из виду и сделать небрежно.
\vs Ars 1:52
Прежде всего я опишу тебе устройство трапезы. Царь желал сделать это сооружение огромных размеров; но он приказал справиться у местных, какова величина уже существующей и стоящей в Иеросалимском храме.
\vs Ars 1:53
Когда же сообщили её размеры, он снова спросил, можно ли делать большую? Некоторые из священников и другие говорили, что нет препятствий, но он сказал, что желает сделать в пять раз большую, однако опасается, что она окажется непригодной для богослужения.
\vs Ars 1:54
А он, конечно, не хотел, чтобы приготовленная им только стояла на месте; ему будет гораздо приятнее, если соответствующие службы будут совершаться, как и должно, назначенными для этого на приготовленной им.
\vs Ars 1:55
Для прежней трапезы были указаны меньшие размеры не по недостатку золота, но, сказал, она была сделана таких размеров по известным, как кажется, основаниям. А если бы оказалось необходимым увеличить её, то ни в чем не было бы недостатка. Поэтому не следует ни уменьшать, ни увеличивать удачно избранные.
\vs Ars 1:56
Итак приказал широко пользоваться различными искусствами, ибо он всё замышлял в величественных чертах и обладал природной способностью представлять предметы в их готовом виде. Что не было записано, он приказал делать согласно с красотой, а что было указано, в этом следовать размерам.
\vs Ars 1:57
Из чистого золота было сделано массивное сооружение длиною в два локтя, шириною в один локоть и высотою в полтора локтя; говорю же не о накладном золоте, но о том, что была положена массивная доска.
\vs Ars 1:58
Вокруг был сделан ободок, шириною в ладонь, с витыми бортами, украшенными рельефной плетеной резьбой, удивительно искуссно выгравированной с трех сторон, ибо был треугольным.
\vs Ars 1:59
На каждой стороне работа была выполнена одинаково, так что, в какую бы сторону ни поворачивать, вид был один и тот же. Но в то время, как художественная работа под ободком была обращена к трапезе, наружная поверхность была видима приходящему.
\vs Ars 1:60
Поэтому верхний край с обоих сторон был острым, ибо, как сказано ранее, был сделан треугольным (в какую бы сторону его ни поворачивать). Посредине плетения в него были вставлены различные драгоценные камни, прикрепленные один к другому с неподражаемым искусством.
\vs Ars 1:61
Все они для безопасности были укреплены в отверстиях золотыми гвоздями, а на углах для прочности оправы связывались вместе.
\vs Ars 1:62
По бокам у ободка в верхней части кругом было сделано всё усеянное драгоценными камнями подобие яиц), изображенное выступами при помощи сплошного барельефа в виде полос, плотно прилегающих одна к другой вокруг всей трапезы.
\vs Ars 1:63
A под изображенным из драгоценных камней подобием яиц художники превосходно и очень отчетливо сделали венок, изобилующий всякими плодами: виноградными кистями, колосьями, финиками, масличными ягодами, гранатовыми яблоками и т. п. Для изображения этих плодов они употребили камни, соответствующие цвету каждого плода и прикрепили их к золотому кольцу вокруг всей трапезы, сбоку её.
\vs Ars 1:64
Украсив ободок, они внизу под изображением подобия яиц устроили таким же образом и остальные части рельефных украшений и резьбы, так что трапеза была сделана для пользования с обоих сторон, с какой угодно. Были сделаны и борты и ободок в нижней части у ножек.
\vs Ars 1:65
По всей ширине трапезы они сделали массивную доску в четыре пальца толщиною, в неё были вставлены ножки и под ободком укреплены шипами, находящимися в углублениях, чтобы можно было пользоваться с какой угодно стороны. Это можно было видеть на верхней доске, так как это произведение было устроено для употребления с обоих сторон.
\vs Ars 1:66
На самой же трапезе превосходно и рельефно изобразили мэандр), посредине которого находилось множество драгоценных камней разных пород: рубины, смарагды, ониксы и другие породы камней превосходного качества.
\vs Ars 1:67
Под изображением мэандра находилась сделанная удивительно искусно сетка, посредине имеющая узор в форме ромба. В него были вставлены горный хрусталь и так называемый янтарь, производя необычайное впечатление на зрителей.
\vs Ars 1:68
Ножки были сделаны в форме головок лилий, под трапезой лилии загибались, а с лицевой стороны имели прямые листья.
\vs Ars 1:69
Основание ножек на полу из рубина и всюду имело четыре пальца, с лицевой стороны имея форму башмака в восемь пальцев ширины. На нем и удерживалась вся тяжесть ножек.
\vs Ars 1:70
Вырезали из камня плющ, обвитый тернием и виноградной лозой, которая вместе с виноградными кистями, вытесанными из камня, окружала ножки до верху. Таково было устройство четырех ножек. Всё было сделано и выступало ясно; опытность и искусство неизменно превосходили действительность, так что при дуновении воздуха листья начинали двигаться, ибо всё было сделано так, чтобы изображать действительность.
\vs Ars 1:71
Переднюю сторону трапезы сделали из трех частей, как бы в форме триптиха, и по толщине сооружения скрепили одну с другой шипами в форме гусиной лапки, так что соединение скреп не было видно и нельзя было найти. А толщина всей трапезы была не меньше полулоктя, так что на всё сооружение пошло много талантов.
\vs Ars 1:72
А так как царь не запрещал увеличивать размеров, то, если нужно было издержать больше приготовленного, царь отпускал на это и больше. Согласно его желанию всё было исполнено удивительно и достойным образом, безподобно со стороны искусства и безукоризненно в отношении красоты.
\vs Ars 1:73
Два сосуда были сделаны из золота; от основания и до середины они были покрыты чешуйчатой резьбой, а между чешуей были весьма искуссно устроены скрепы из драгоценных камней.
\vs Ars 1:74
Далее лежал мэандр высотою в локоть; он был рельефно изображен при помощи драгоценных камней различного цвета, свидетельствуя как о зрелости искусства, так и о тщательности. За ним рельефное украшение в виде ромба, которое до отверстия имело форму сети.
\vs Ars 1:75
Впечатление красоты дополняли небольшие щиты величиною не меньше четырех пальцев из различных драгоценных камней и расположенные посредине один подле другого. А по краю отверстия, кругом, были изображены лилии с цветками и виноградные лозы, переплетающиеся с виноградными кистями.
\vs Ars 1:76
Таково было устройство золотых сосудов, которые вмещали более двух метритов). Что касается серебряных, то они были сделаны гладкими, вроде зеркала, и уже это было удивительно, так как в них гораздо яснее, чем в зеркале, отражалось всё, что подносили.
\vs Ars 1:77
По сравнению с отражаемой ими действительностью их действие описать невозможно. Когда всё было окончено и предметы были поставлены один подле другого, то есть сначала серебряный сосуд, затем золотой, снова серебряный и золотой, то действие их вида было совершенно неописуемо, так что приходившие посмотреть на них не могли уйти вследствие их необычайного блеска и прелести для взоров.
\vs Ars 1:78
Впечатление от их наружного вида было разнообразно. Когда смотрели на работу из золота, являлась радость с удивлением, так как внимание непрерывно устремлялось на каждое из этих художественных произведений. А если кто, напротив, хотел взглянуть на серебряные сосуды, то они всюду и кругом начинали блестеть, где бы кто ни стоял, и доставляли зрителю еще большее удовольствие. Таким образом, изящество их совершенно нельзя описать.
\vs Ars 1:79
На золотых бокалах посредине были выгравированы венки виноградной лозы, а по краям вырезали венок, сплетенный из плюща, мирты и маслины, вставив различные драгоценные камни. Остальные части граверной работы они сделали из различных узоров, усердно стремясь всё сделать для большей славы царя.
\vs Ars 1:80
Вообще, таких роскошных и художественных предметов нет не только в царских сокровищницах, но и в ничьих других. Ибо славолюбивый царь не мало подумал над тем, чтобы всё было исполнено прекрасно.
\vs Ars 1:81
Часто он оставлял публичные аудиенции и внимательно следил за художниками, чтобы они выполняли свою работу достойно того места, куда отправлялись их произведения. Поэтому всё делалось великолепно и достойно, как царя, отправляющего, так и первосвященника, управляющего этим местом.
\vs Ars 1:82
На работу пошло великое множество драгоценных камней, притом большой величины, не менее пяти тысяч и всё отличалось художественностью исполнения, так что количество драгоценных камней и работа ювелиров стоили в пять раз дороже золота.
\vs Ars 1:83
Я сообщил тебe описание их, предполагая, что это необходимо. Дальнейшее содержит наше путешествие к Елеазару. Сначала я опишу устройство всей страны. Когда мы прибыли на место, то увидели город, лежащий посредине всех иудеев, на очень высокой гopе.
\vs Ars 1:84
На краю был построен храм превосходного вида, три стены, высотой более семидесяти локтей, а их ширина и длина соответствовали устройству храма, так как всё было построено с необыкновенными во всех отношениях великолепием и роскошью.
\vs Ars 1:85
Очевидно было, что на двери, на прикрепление их к косякам и укрепление притолоков затрачены были также огромные суммы.
\vs Ars 1:86
Устройство завесы во всем было совершено подобно дверям. Особенно приятный вид, от которого с трудом можно было оторваться, она получала при дуновении ветра, когда ткань приходила в непрерывное движение, так как дуновение от основания передавалось по складкам до верхнего края.
\vs Ars 1:87
Устройство жертвенника отвечало месту и сожигаемым на огне жертвам, точно также и подъем к нему; место это, вследствие необходимого благоприличия, имело наклон, так как священники совершали служение одетыми до пят в льняные хитоны.
\vs Ars 1:88
Храм лицом был обращен к востоку, а задней стороной на запад. Весь пол был вымощен камнем, а для стока воды от замывания жертвенной крови имел в соответствующих местах наклон; ибо в праздничные дни приводились для жертв тысячи скота.
\vs Ars 1:89
Скопление же воды было неисчерпаемо, так как внутри протекал обильный естественный источник, а под землею, кроме того, находились удивительные и неописуемые водоемы. И показывали на пять стадий вокруг основания храма безчисленные галереи каждого из них, так как потоки в каждой части соединялись друг с другом.
\vs Ars 1:90
Всё это на полу и стенах было обложено свинцом, а поверх этого покрыто толстым слоем штукатурки, так что всё было сделано прочно. Частые отверстия в полу не были известны никому, кроме служащих, как будто всё множество жертвенной крови очищалось одним движением и мановением.
\vs Ars 1:91
Объясню, насколько я, по моему убеждению, сам удостоверился, и устройство водоемов. Меня вывели за город дальше, чем на четыре стадии, и приказали, наклонившись в известном месте, прислушаться к шуму от встречи вод. Вследствие этого мне, как сказано, ясной стала величина водоемов.
\vs Ars 1:92
Служение священников по силе, а также настроению благоприличия и тишины несравненно. Все усердно трудятся по доброй воле и с великим напряжением; каждый же заботится о порученном. Они непрерывно работают: одни доставляют дрова, другие масло, иные крупинчатую муку; иные ароматы; другие сожигают части жертвенного мяса, обнаруживая необыкновенную силу.
\vs Ars 1:93
Взяв обоими руками ноги телят, каждая из которых весит почти более двух талантов, они удивительно ловко и без промаха бросают их обоими руками на значительную высоту, точно также и овец и коз, отличающихся значительным весом и тучностью. Назначенные для этого выбирают безпорочных и особенно тучных и совершается вышеуказанное.
\vs Ars 1:94
Для отдыха им назначено место, где сидят отдыхающие. В это время пробуждаются те из отдыхавших, которые имеют желание, хотя никто не приказывает им служить.
\vs Ars 1:95
А тишина такова, что можно подумать, будто здесь нет никого, хотя служащих находится около семи тысяч (количество приносящих жертвы также очень велико), но всё совершается в страхе и достойно великого Божества.
\vs Ars 1:96
Нас охватило великое изумление, когда мы увидели Елеазара в служении, его облачение и славу, которая обнаруживалась в носимом им хитоне и камнях на нем. Вокруг его подира были золотые позвонки, которые издавали своеобразные гармонические звуки, а около каждого из них разноцветные гранатовые яблочки поразительной окраски.
\vs Ars 1:97
Он быль опоясан превосходным и великолепным поясом, вытканным из красивейших цветов. На груди он носит так называемый наперсник судный, в который были вставлены оправленные в золото двенадцать камней различной породы, с расположенными согласно первоначальному порядку именами начальников колен. Каждый сверкал своим характерным и неописуемым природным блеском.
\vs Ars 1:98
На голове имеет так называемый кидар, а на нем безподобная митра, то есть святая диадема, на которой над бровями священным шрифтом на золотом листке было вырезано имя Божие, полное славы. В таком виде выходит тот, кто был признан быть достойным этого, на служение.
\vs Ars 1:99
Всё это вместе вызывало страх и трепет, так что казалось, будто приходишь в иное место, вне этого миpa. И я утверждаю, что каждый человек, приходя посмотреть на это, повергался в изумление и невыразимое удивление, так как мысль его изменялась вследствие святого устройства во всём.
\vs Ars 1:100
Чтобы узнать всё, мы производили осмотр, поднявшись на лежащую около города крепость. Она расположена на самом высоком месте и укреплена множеством башен, так как они доверху выстроены из больших каменных плит, для охраны, как мы понимаем, мест около храма
\vs Ars 1:101
(чтобы, в случаe какого-либо нападения, бунта, или вторжения неприятелей, никто не мог проникнуть за стены, окружающие храм, так как на башнях крепости есть метательные машины и разные снаряды, а место это лежит выше упомянутых ранее стен),
\vs Ars 1:102
так как эти башни охраняются наиболее надежными мужами, давшими отечеству великие доказательства. Им разрешается выходить из крепости только по праздникам, притом по частям. Точно также никого не разрешается и впускать.
\vs Ars 1:103
Большую осторожность соблюдают они, если начальник дал разрешение впустить кого-либо для осмотра. Это случилось и с нами. С трудом безоружных нас двух впустили, чтобы посмотреть на принесение жертв.
\vs Ars 1:104
Говорили, что они обязались в этом клятвою. Bcе они, числом пятьсот, поклялись (конечно, при клятве дело по необходимости выполняется по-божески) не впускать в крепость болee пяти человек одновременно. Ведь крепость является единственной защитой храма и строитель укрепил её так для охраны указанного ранее.
\vs Ars 1:105
Город средней величины, так как стена, насколько можно догадываться, имеет около сорока стадий. Башни расположены в нем в форме театра; в нижних входы не видны, а в верхних заметны; в них и выходы. Местность имеет подъем, так как город построен на горе.
\vs Ars 1:106
Ко входам лестницы; одни вверху, а другие внизу, и очень удалены от дороги, чтобы те, которые живут в чистоте, не соприкасались с недозволенным.
\vs Ars 1:107
И начальники города неслучайно построили его симметрично, но по мудром размышлении. Так как эта страна обширна и прекрасна, и одни части её ровны, как например по направлению к Самарии и граничащие с Идумеей, а другие, как например посредине страны, гористы, то необходимо постоянно возделывать и обрабатывать землю, чтобы таким путем и эти стали плодородными. Если делать это, то во всей указанной стране всё приносит обильные плоды.
\vs Ars 1:108
В городах, отличающихся своей величиной и соответствующим благоденствием, население многочисленно, а страна оставляется в пренебрежении, так как все склоняются к жизненным радостям, ибо все люди склонны к yдoвольcтвиям.
\vs Ars 1:109
Это имеет место и в Александрии, превосходящей все города своей величиной и благоденствием. Именно, те из поселян, которые, прибыв в неё погостить, остаются надолго, отвыкают от земледельческого труда.
\vs Ars 1:110
Поэтому, чтобы они не задерживались, царь разрешил оставаться не более двадцати дней. Соответственно этому он сделал письменное распоряжение чиновникам: если необходимо вызвать кого-либо, то разбираться в течение пяти дней.
\vs Ars 1:111
В виду важности дела он назначил для каждого округа судей и их помощников, чтобы земледельцы и поверенные, получая доходы, не уменьшали городских кладовых; говорю же я о земледельческих налогах.
\vs Ars 1:112
Мы уклонились в сторону, потому что Елеазар прекрасно на примерах разъяснил нам вышеизложенное. Действительно, труд при обработке земли велик. И страна их изобилует масличными деревьями, хлебными плодами, овощами, а кроме того виноградом и массою меда (других плодовых деревьев и финиковых пальм у них нет), множеством различного скота и обилием пастбищ для него.
\vs Ars 1:113
Поэтому они прекрасно обратили внимание на то, что эта страна требует многочисленного населения, и установили надлежащее отношение между городом и деревнями.
\vs Ars 1:114
Кроме того, сюда доставляется арабами масса благовоний, различных драгоценных камней и золота. Страна эта, удобная для земледелия, пригодна и для торговли, а город для: занятия различными ремеслами. Она не имеет недостатка ни в чем, что доставляется морем.
\vs Ars 1:115
Есть в ней и удобные гавани доставляющие: у Аскалона, Яфы, Газы, а также у основанной царем Птолемаиды, которая находится посредине первых, на небольшом от них разстоянии. Страна эта имеет всё в изобилии, так как всюду хорошо орошается и прочно защищена.
\vs Ars 1:116
Её окружает река, называемая Иорданом, которая никогда не пересыхает (первоначально страна эта была не менее шестидесяти миллионов арур), но впоследствие, когда соседи были вытеснены из неё, шестьсот тысяч мужей получили в удел сто арур). Разливаясь, подобно Нилу, она около времени жатвы увлажняет большую часть страны.
\vs Ars 1:117
Он впадает в другую реку, в стране Птолемеев, а эта выходит в море. Текут и иные горные потоки, охватывающие окрестности Газы и местность Азота.
\vs Ars 1:118
Охраняется самою природой, будучи недоступна для вторжения и непроходима для больших масс, так как дороги узки, её окружают утесы и глубокие ущелья, a кромe того все горы вокруг этой страны скалисты.
\vs Ars 1:119
Говорили также, что ранее в соседних горах Аравии существовали медные и железные рудники, но во время господства персов они были оставлены, так как начальники того времени пустили ложные слухи, будто разработка их безполезна и дорого обходится,
\vs Ars 1:120
чтобы вследствие добывания оных не погибла страна и, при их тирании, не отпала, тогда как путем этой клеветы они получили предлог к доступу в эту местность. Итак, брат Филократ, я указал тебе главное и сколько нужно было об этом; далее же мы изложим то, что касается перевода.
\vs Ars 1:121
Елеазар выбрал лучших мужей, отличающихся образованием и знатностью рода, которые приобрели навык не только в иудейской литературе, но тщательно позаботились и об изучении греческой.
\vs Ars 1:122
Поэтому они были пригодны для посольства и в необходимых случаях исполняли его; они обладали большими дарованиями к беседам и изследованию в области Закона, стремясь к среднему положению (ибо оно прекраснее всего); они оставили грубость и неотделанность мысли, а также пренебрегли самомнением и своим превосходством над другими; в беседах они были примером для других, как своим умением слушать, так и отвечать каждому должное; все они соблюдают это, желая в этом более всего превосходить друг друга, все быть достойными своего начальника и его добродетели.
\vs Ars 1:123
А что они любили Елеазара, видно было, как они с неохотой покидали его. И сам он не только царю написал о возвращении их, но настойчиво просил Андрея и нас содействовать, насколько можем.
\vs Ars 1:124
И хотя мы обещали внимательно заботиться о них, он говорил, что сильно безпокоится. Действительно, он знал, что любящий доброе и добрых царь выше всего ставит приглашеше таких людей, которые где-либо признаются, как отличающиеся от других своим образованием и разумом.
\vs Ars 1:125
Царь, я полагаю, прекрасно говорит, что, имея около себя мужей праведных и мудрых, он приобретет лучшую охрану для своего царства, так как друзья с полной откровенностью советуют ему полезное. А этим именно и обладали посланные Елеазара.
\vs Ars 1:126
И он клятвенно уверял, что он не отпустил бы этих людей, если бы того требовало какое-либо иное личное его дело, но он отправляет их для общего исправления всех граждан.
\vs Ars 1:127
Ибо добродетельная жизнь заключается в соблюдении законов, а это гораздо лучше достигается путем слушания, чем путем чтения. Итак, предлагая их и подобное им, Елеазар ясно показывал свое расположение к ним.
\vs Ars 1:128
Следует вкратце упомянуть и о том, что Елеазар ответил нам на наши вопросы (ибо я полагаю, что многиe серьезно интересуются некоторыми из законов о пище и питье, а также о животных, признаваемых нечистыми).
\vs Ars 1:129
Итак, когда мы спросили, почему, несмотря на одинаковое происхождение, одни считаются нечистыми для еды, а другие и для прикосновения (ибо большинство из Закона отличается суеверием, а в этих частях полным), то он начал на это следующее.
\vs Ars 1:130
Ты видишь, сказал он, как влияют образ жизни и знакомства; поддерживая знакомство с порочными, люди совращаются и становятся несчастными на всю жизнь; а если они живут в обществе мудрых и разумных, то из неведения вступают в жизнь лучшую.
\vs Ars 1:131
Поэтому наш законодатель, определив прежде всего то, что относится к благочестию и справедливости, научив всему этому не только в форме запрещений, но и путем разъяснений, показав вредные последствия, а также наказания, посылаемые Богом виновникам этого
\vs Ars 1:132
(ибо прежде всего он и указал, что Бог един и сила Его очевидна всюду, так как всякое место полно Его господства и от Него не скроется ни одно из тайных дел людей на земле, но Ему видно всё, что делает и намеревается делать человек),
\vs Ars 1:133
тщательно выполнив это и сделав вполне очевидным, он показал, что, если бы кто и замыслил сделать дурное, то не не скрыл, но даже и не сделал бы, на протяжении всего законодательства указывая могущество Божие.
\vs Ars 1:134
Итак, положив такое начало и показав, что все остальные люди, кроме нас, почитают многих богов, хотя сами гораздо сильнее тех, кого безразсудно чтут
\vs Ars 1:135
(действительно, сделав из дерева и камней статуи, они говорят, что это образы тех, которые изобрели нечто полезное для их жизни; им они покланяются, сразу обнаруживая глупость.
\vs Ars 1:136
Разве они не [обнаружили бы] полное неразумие, если бы на этом основании, вследствие изобретения, кто-либо был обоготворен? Действительно, взяв одну из тварей, они лишь заметили и указали пользу, но не создали её устройства.
\vs Ars 1:137
Поэтому тщетно и безразсудно обоготворять подобных. И теперь ведь еще есть много людей более ученых и изобретательных, чем прежде, и не доходят до того, чтобы поклоняться им. Создавшие эти образы и составители мифов считают себя самыми мудрыми из греков.
\vs Ars 1:138
Что же говорить о других, более безразсудных, о египтянах и подобных им? Они останавливаются на зверях, на большинстве гадов и животных, покланяются им и приносят жертвы, как живым, так и павшим),
\vs Ars 1:139
и вот, имея в виду всё это, мудрый законодатель, которого Бог наделил способностями к познанию всего, огородил нас частоколом, которого нельзя прорубить, и железными стенами, чтобы мы ни в чем не смешивались с другими народами, пребывая чистыми по телу и душе, свободными от пустых учений, выше всех тварей почитая единого и могущественного Бoгa.
\vs Ars 1:140
На этом основании начальники египтян, их жрецы, постигшие многое и знакомые с письменами, называют нас людьми Божиими. А это неприложимо к остальным, если они не почитают истинного Бога, но являются людьми пищи, питья и одежды.
\vs Ars 1:141
Действительно, к этому направлено всё настроение их души, а у наших это вменяется ни во что; мы в течение всей жизни занимаемся изследованием божественного правления.
\vs Ars 1:142
Поэтому, чтобы мы ни с кем не смешивались и, имея общение с порочными, не испортились, всюду оградил нас законами о чистоте, в пище, питье, прикосновениях, в том, что мы слышим и видим.
\vs Ars 1:143
Вообще, всё подобное согласно с естественным разумом, так как установлено одной Силой и каждое в отдельности о том, почему мы воздерживаемся и пользуемся, имеет глубокое основание.
\vs Ars 1:144
Для примера я кратко объясню тебе одно или два, чтобы ты не впал в опровергнутое мнение, будто Моисей заповедует это, заботясь о мышах, куницах и тому подобных. Напротив, все важные определения сделаны ради справедливости, с целью чистых размышлений и образования нравов.
\vs Ars 1:145
Все птицы, которыми мы питаемся, ручные, чистые, питающиеся овощами и пшеницей, как например голуби, горлицы, куры, куропатки, гуси и прочие, подобные им.
\vs Ars 1:146
А в ряду запрещенных птиц ты найдешь диких, плотоядных, порабощающих благодаря своей силе остальных и несправедливо пожирающих названных выше ручных. Да и не только этих; они похищают даже ягнят, козлят и причиняют вред людям, как мертвым, так и живым.
\vs Ars 1:147
Поэтому, назвав еще нечистыми, обозначил этим, что те, для кого назначено законодательство, должны быть справедливыми по душе, никого не угнетать, полагаясь на свою силу, и ничего не похищать, но управлять своею жизнью согласно справедливости, подобно тому, как названные выше ручные птицы питаются овощами, растущими на земле, и не пользуются своей силой для угнетения более слабых и родственных.
\vs Ars 1:148
И вот, посредством такого законодатель дал знамение разумным, чтобы они были справедливыми, не насильничали и, полагаясь на свою силу, не угнетали других.
\vs Ars 1:149
А если этих, вследствие их природных особенностей, не должно даже и касаться, то как же не охранять себя всячески от того, чтобы наши нравы не извратились в этом.
\vs Ars 1:150
Итак, всё о разрешении этих и животных изложены нам в форме символов. Так, раздвоенность копыт и разделение когтей является символом того, чтобы разграничивать каждое из дел, стремясь к прекрасному.
\vs Ars 1:151
Ибо сила всего тела и деятельность его опору имеет в плечах и бедрах. Поэтому, этим символом принуждает всё направлять к справедливости с разделением, а также, что мы отличаемся от всех людей.
\vs Ars 1:152
Действительно, большинство остальных людей оскверняют себя совокуплениями, совершая тяжкую несправедливость, и этим хвалятся целые страны и города; они не только вступают в сношения с мужчинами, но оскверняют матерей и дочерей. Мы же воздерживаемся от этого.
\vs Ars 1:153
Но кто владеет указанным выше способом различения, тот, как он указал, владеет им и в отношении памяти. Ибо все, раздвояющие копыта, отрыгают и жвачку, ясно показывая размышляющим свойства памяти;
\vs Ars 1:154
ведь жвачность есть ничто иное, как воспоминание о жизни и устройстве, и полагает, что жизнь поддерживается благодаря питанию.
\vs Ars 1:155
Поэтому и чрез писание он предписывает следующее: помни ЯХВЕ, Бога твоего, сотворившего среди тебя великое и чудное. Действительно, если подумать, то великим и славным окажется прежде всего скрепление тела, потребление пищи и разделение каждого члена.
\vs Ars 1:156
Еще более безконечной мудрости заключает устройство чувств, деятельность мысли и невидимое движение, а также быстрота действия в каждом и изобретение искусств.
\vs Ars 1:157
Поэтому предписывает помнить, что указанное выше вместе с устройством хранится божественной силой. Всякое время и место он определил для постоянного памятования о Боге, владыке и хранителе.
\vs Ars 1:158
Поэтому он повелевает, чтобы при пище и питье сначала принести начатки Богу и затем пользоваться. Далее он дал нам знак воспоминания и на покровах); точно также, для памяти о том, что Бог есть, он приказал нам поместить изречения на дверях и воротах).
\vs Ars 1:159
И на руках он ясно приказал привесить знак), ясно показывая, что всякое действие должно совершать справедливо, памятуя о своем устройстве, а более всего питая страх к Богу.
\vs Ars 1:160
Призывает также, отходя ко сну, вставая и путешествуя, изучать творения Божии и не только на словах, но и в мыслях рассматривать свои движения и представления, когда мы отходим ко сну и когда пробуждаемся, так как смена этого божественна и непостижима.
\vs Ars 1:161
Итак, тебе показано превосходное учение в отношении к различию и памяти, как мы истолковали раздвоенность копыт и жвачность. Это заповедуется душе не безцельно и случайно, но ради истины и руководства к здравому учению.
\vs Ars 1:162
Дав предписания относительно пищи, питья, а также прикосновений, приказывает ничего не делать и слушать необдуманно и, пользуясь силою разума, не обращать его на неправду.
\vs Ars 1:163
То же можно найти и в отношении животных. Куницы, мыши и подобные им, сколько указано, вредны.
\vs Ars 1:164
Мыши всё оскверняют и портят, не только для собственного питания, но и делают совершенно безполезным для человека всё, чему бы они ни начали вредить.
\vs Ars 1:165
А куницы оригинальны. Кроме указанного выше они имеют постыдное устройство: они зачинают ушами, а детей рождают через рот.
\vs Ars 1:166
Поэтому такой характер нечист для людей. То, что они воспринимают слухом, это воплощают в слове и погружают других во зло, производя не случайную нечистоту, но пятная себя всюду осквернением нечестия. Ваш царь прекрасно делает, убивая, как мы слышим, таковых.
\vs Ars 1:167
А я сказал: я полагаю, ты говоришь о доносчиках, так как их он всегда подвергает побоям и мучительной смерти.
Он: и я говорю о них. Действительно, подкарауливать погибель людей нечестиво.
\vs Ars 1:168
А наш закон предписывает никому не вредить ни словом, ни делом.
Итак, тебе показано, насколько это можно было сделать вкратце, что все определения имеют в виду справедливость и писание не предписывает ничего безцельного или баснословного, но чтобы в течение всей жизни в своих действиях мы упражнялись в справедливости по отношению ко всем людям, помня о господствующем Боге.
\vs Ars 1:169
Поэтому всё разсуждение о пище и о нечистых гадах и животных относится к справедливости и справедливым отношениям к людям.
\vs Ars 1:170
По моему мнению он прекрасно защищал каждое. А относительно приносимых в жертву телят, баранов и козлов он говорил, что, взяв из стада быков и овец ручных, их должно приносить в жертву, но не диких, чтобы приносящие жертву, воспользовавшись указаниями законодателя, ничем не гордились и знали свою природу, ибо приносящий жертву приносит в жертву всё настроение своей души.
\vs Ars 1:171
Итак, я полагаю, что и в этом отношении его беседы заслуживают внимания. Поэтому, вследствие твоей любознательности, у меня, Филократ, были побуждения объяснить тебе святость Закона и его согласие с природой.
\vs Ars 1:172
И Елеазар, совершив жертвоприношение и избрав посланцев и приготовив много даров для
\vs Ars 1:173
царя, отправил нас в путь в великой безопасности. И когда мы достигли Александрии, царю тотчас же доложили о нашем прибытии. Будучи приняты во дворце, Андрей и я тепло приветствовали
\vs Ars 1:174
царя и передали ему письмо, написанное Елеазаром. Царь весьма безпокоился о том, чтобы принять посланных, и повелел, чтобы все прочие чиновники вышли, а посланные
\vs Ars 1:175
тотчас же были приведены к нему. Это вызвало всеобщее изумление, ибо обычно те, кто добивается быть допущенным пред царем по важным делам, ждут пять дней, а послы царя или больших городов с трудом добиваются придворного приема на третий день; но этих людей он счел достойными больших почестей, поскольку он имел столь великое почтение к их наставнику, и так он отослал тех, чье присутствие он счел излишним, и прогуливался, пока они не вошли и он смог приветствовать их.
\vs Ars 1:176
Когда они вошли с дарами, которые были посланы с ними, и драгоценными пергаментами, на которых был золотыми еврейскими письменами записан Закон (ибо пергамент был чудно изготовлен и соединение страниц было сделано так, что было невидимо), царь, как только
\vs Ars 1:177
увидел их, стал спрашивать их о книгах. И когда они вынули свитки из коробов и развернули их, царь долго простоял в безмолвии и поклонившись семь раз, он сказал: Благодарю вас, друзья, и еще более благодарю того, кто послал вас,
\vs Ars 1:178
превыше же всего Бога, вещавшего это. И когда все, посланные и прочие, бывшие там, вместе воскликнули в один голос: Бог да хранит царя!, он пролил слёзы радости. Ибо в душе его восторг и переполняющее ощущение оказанной ему чести
\vs Ars 1:179
побудили его плакать от счастья. Он повелел свернуть свитки обратно и затем, поклонившись этим мужам, сказал: Достойно было, люди Божии, мне уделить прежде всего почтение книгам, ради которых я призвал вас сюда, и теперь, после того, как я это сделал, протянуть вам десницу моей дружбы. Ради этого я
\vs Ars 1:180
сделал это прежде всего. Я дал указ о том, чтобы этот день, в который вы прибыли, стал считаться великим днем и стал ежегодно торжественно справляться в течение всей моей жизни. Вышло так, что это еще и годовщина
\vs Ars 1:181
моей победы на море над Антигоном. Поэтому я буду рад пировать с вами сегодня. Все, что вам может потребоваться, сказал он, будет приготовлено как подобает и вместе с вами и для меня тоже. И они выразили своё восхищение, и он повелел отвести их в лучший квартал, прилежащий к цитадели) и готовить пир.
\vs Ars 1:182
И Никанор вызвал главного дворцового распорядителя Дорофея, чиновника, особо назначенного смотреть за евреями, и приказал ему приготовить всё необходимое для каждого из них. Ибо так было установлено царем, и это установление вы увидите соблюденным сегодня. Ибо поскольку многие города имеют свои обычаи в том, что касается еды, питья и возлежания, есть особые чиновники, назначение которых узнавать, что им требуется. И всякий раз, когда те приходят к царю, для них готовят, соблюдая их собственные обычаи, так чтобы они не испытывали безпокойства, наслаждаясь посещением. Та же предусмотрительность была соблюдена и для еврейских посланцев. Дорофей же, назначенный старшим приставником при еврейских гостях, был
\vs Ars 1:183
человеком весьма тщательным. Ради такого пира он открыл все хранилища, бывшие под его надзором и державшиеся особо для подобных гостей. Он расположил сидения в два ряда согласно царскому повелению. Ибо он повелел ему усадить половину мужей справа от себя, а остальных позади, так чтобы он не лишил их величайшей из возможных чести. Когда они заняли сидения, он повелел Дорофею всё делать,
\vs Ars 1:184
сообразуясь с обычаями, принятыми среди еврейских гостей. Поэтому он прибег к услугам священных глашатаев и священников, приносящих жертвы, и прочих, кто привык возносить молитвы, и призвал одного из нашего числа, по имени Елеазар, старейшего из еврейских священников, вознести молитву вместо [себя]. И тот поднялся и сотворил превосходную молитву: Да обогатит
\vs Ars 1:185
тебя Всемогущий Бог, о царь, всяким благом, созданным Им, и да наградит Он тебя и твою жену, и твоих детей и твоих товарищей) непрерывным владычеством их во всю вашу жизнь. При этих словах поднялось громкое и радостное одобрение, длившееся весьма долго, и потом
\vs Ars 1:186
они обратились к наслаждению приготовленным пиром. Все распоряжения за столом совершались согласно внушениям Дорофея. Среди прислуживающих были юноши из свиты царя и иные, занимавшие почетные должности при царском дворе.
\vs Ars 1:187
Воспользовавшись моментом, когда пир приостановился, царь спросил посланца, занимавшего почетное место (ибо их расположили по старшинству), как ему сохранить царство
\vs Ars 1:188
неослабным до конца. Поразмыслив немного, тот отвечал: Лучше всего ты утвердишь его безопасность, если ты будешь подражать безконечной Божьей благости. Ибо если ты будешь являть милость и налагать кроткие наказания на тех, кто их заслуживает сообразно сделанному ими, ты
\vs Ars 1:189
обратишь их от зла и приведешь их к покаянию.
Царь похвалил ответ и затем спросил у еще одного мужа, как может совершать самое лучшее во всём. И тот отвечал: Если муж держится правильного отношения ко всему, он всегда будет поступать правильно во всяком случае, помня, что всякая мысль известна Богу. Если страх Божий станет для тебя исходною чертою, ты никогда не пройдешь мимо цели.
\vs Ars 1:190
Царь похвалил и этого мужа и спросил у иного, как приобрести друзей, мыслящих одинаково с собою. Тот отвечал: Если они будут видеть, что ты ревнуешь о нуждах множества, которым ты правишь, ты сам увидишь, как Бог одаривает Своими благодеяниями
\vs Ars 1:191
человеческий род, подавая им здоровье и пищу и всё остальное в должное время.
Выразив согласие с ответом, царь спросил следующего, как, принимая просящих и вынося суд, он может стяжать похвалу даже от тех, кто не добился исполнения иска. И тот сказал: Если твоя речь будет прилична для всех равно и ты никогда не будешь поступать надменно или как тиранн с
\vs Ars 1:192
преступниками. И ты будешь делать это, если рассмотришь образ деяний Божьих. Прошения достойных всегда исполняются, тогда как те, кто не получает ответа на свои молитвы, уведоляются снами или событиями о том, что в их просьбах было вредное, и что Бог не поражает их по их грехам или по величию Своей силы, но долготерпит им.
\vs Ars 1:193
Царь горячо похвалил мужа за его ответ и спросил следующего за ним, как он может стать непобедимым в делах войны. И тот ответил, что если он не будет полагаться только на множество своих сил и их воинственность, но будет непрестанно взывать к Богу привести начатое к счастливому исполнению, а сам
\vs Ars 1:194
же будет исполнять все свои обязанности в духе праведности.
Поблагодарив за ответ, он спросил другого, как он может сделаться грозен для своих врагов. И тот ответил, что если сохраняя мощный запас оружия и войска, он будет помнить, что этим нельзя добиться постоянного и окончательного итога. Ибо даже Бог внушает страх в умы людей, откладывая исполнение и лишь являя величие Своего могущества.
\vs Ars 1:195
Этого мужа царь похвалил и спросил следующего, что есть высшее благо в жизни. И тот ответил: Познать, что Бог есть Господь Вселенной, и что в конечном исполнении всех наших дел не мы добиваемся успеха, но Бог, Своею властью приводящий всё к завершению и ведущий нас к цели.
\vs Ars 1:196
Царь воскликнул, что этот муж ответил хорошо и затем спросил следующего, как он может сохранить всё, чем владеет в целости и в конце передать своим наследникам таким же. И тот ответил: Постоянно моля Бога вдохновить тебя высокою целью во всех твоих начинаниях и предупреждая твоих наследников не ослепляться молвою или богатством, ибо все эти дары подает Бог, а люди никогда сами по себе не достигают превосходства.
\vs Ars 1:197
Царь выразил своё согласие с ответом и осведомился у следующего гостя, как ему суметь переносить с душевным спокойствием всё, что выпадет ему. И тот сказал: Если ты будешь иметь твердое понимание того, что всем людям надлежит от Бога иметь часть как в величайшем зле, так и в величайшем благе, поскольку невозможно для человека уйти от этого. Но Бог, Которому мы все обязаны молитвою, вселяет в нас мужество претерпевать.
\vs Ars 1:198
Восхищенный ответами мужей, царь сказал, что все их ответы были хороши. Я задам еще один вопрос одному мужу, прибавил он, и тогда я прервусь на время, чтобы мы могли обратить наше внимание
\vs Ars 1:199
к наслаждению пиром и провести время с удовольствием. И тогда он спросил мужа, какова истинная цель мужества. И тот ответил: Если правильный замысел исполняется в час опасности в согласии с первоначальным намерением. Ибо всё совершается Богом к твоему преимуществу, о царь, когда твой умысел благ.
\vs Ars 1:200
Когда все своим одобрением выразили согласие с ответом, царь сказал философам (ибо их там было немало): По моему мнению, эти мужи сияют добродетелями и владеют необычайным знанием, ибо в мгновение они дали правильные ответы на те вопросы, что я им задавал, и все положили Бога источником своих слов.
\vs Ars 1:201
И Менедем, философ из Эритреи, сказал: Истинно, о царь, ибо вселенная управляется провидением, и поскольку мы верно постигаем то, что человек есть создание Бога, отсюда следует,
\vs Ars 1:202
что всякая сила и красота слова исходит от Бога. Когда царь показал, что он согласен с этим чувством, разговор прекратился, и они предались удовольствию. Когда наступил вечер, пир закончился.
\vs Ars 1:203
На следующий день они вновь сели за стол и продолжили пир в прежнем распорядке. Когда царь счел, что настал подходящий момент, чтобы предложить изследование, он стал задавать вопросы тем мужам, которые
\vs Ars 1:204
сидели вслед за отвечавшими накануне. Он приступил к началу беседы с одиннадцатым мужем, ибо десять уже отвечали на прежние вопросы. Когда установилось
\vs Ars 1:205
молчание, он спросил как он сможет и далее оставаться богатым. После краткого раздумья, муж, которому был задан вопрос, отвечал, что если он никогда не делал ничего недостойного своего звания, никогда не вел себя распутно, никогда не расточительствовал ради пустого и тщетного, но своею благотворительностью располагал подданных к себе. Ибо Бог Творец всяческого блага и
\vs Ars 1:206
Ему человек обязан послушанием.
Царь воздал этому хвалу и затем спросил у другого, как ему соблюсти истину. В ответ на вопрос тот сказал: Признав, что ложь наводит великий позор на всех людей, и еще более на царей. Ибо поскольку они имеют власть делать, что хотят, зачем им прибегать ко лжи? В прибавление к этому ты должен помнить, о царь, что Бог любит истину.
\vs Ars 1:207
Царь принял ответ с великим удовольствием и, глядя на другого, сказал: Что есть научение истине. И тот отвечал: Если ты не хочешь, чтобы зло случилось с тобою, но хочешь быть причастником всего благого, тогда ты должен соблюдать одно и то же в отношении к твоим подданным и к преступникам, и ты должен кротко увещевать благородного и доброго. Ибо Бог привлекает к Себе всех людей Своей благостью.
\vs Ars 1:208
Царь похвалил его и спросил у следующего по порядку, как ему быть другом людей. И тот ответил: Наблюдая то, что человеческий род возрастает и рождается во многом смятении и великом страдании. Посему ты не должен легкомысленно наказывать или подвергать их пыткам, поскольку ты знаешь, что человеческая жизнь состоит из боли и наказаний. Ибо если ты поймешь всё, ты преисполнишься жалости, ибо Бог также преисполнен жалости.
\vs Ars 1:209
Царь принял ответ с одобрением и спросил у следующего: Что есть основное отличие правления? Блюсти себя, отвечал тот, свободным от пьянства и соблюдать трезвость в течение большей части жизни, почитать праведность превыше всего и делать своими друзьями людей подобного рода. Ибо Бог любит также праведность.
\vs Ars 1:210
Выказав своё одобрение, царь спросил у другого: Что есть истинный признак благочестия? И тот ответил: Постигать то, что Бог непрестанно творит во Вселенной и знает всё, и ни один человек, делающий несправедливое и творящий развращенное, не может избежать Его взора. Поскольку Бог благотворит всему миру, также и ты должен подражать Ему и быть свободным от преступлений.
\vs Ars 1:211
Царь выказал своё согласие и сказал другому: Что есть сущность царствования? И тот ответил: Хорошо управлять собою и не уклоняться ради славы или богатства к неумеренным или непристойным желаниям, вот истинный путь правления, если ты хорошо уразумеваешь дело. Ибо всё, что тебе нужно, у тебя есть, и Бог свободен от нужд и добросердечен. Пусть твои мысли будут такими, какие достойны мужа, и желай немногого, но только того, что необходимо для правления.
\vs Ars 1:212
Царь похвалил его и спросил у другого мужа, как его разсуждения могут привести к наилучшему. И тот ответил, что если он будет постоянно полагать пред собою праведность во всём и думать, что неправедность равносильна лишению жизни. Ибо Бог всяческих обещает высочайшее благословение праведному.
\vs Ars 1:213
Похвалив его, царь спросил следующего, как он может быть свободен от безпокойных мыслей во время сна. И тот отвечал: Ты задал мне вопрос, на который очень трудно ответить, ибо мы не можем руководить собою в часы сна, но твердо удерживаемся в них
\vs Ars 1:214
воображением, которое не может управляться разумом. Ибо наши души обладают чувствами, которые на самом деле видят то, что входит в наше сознание во время сна. Но мы ошибаемся, если мы полагаем, что мы на самом деле плывем на корабле по морю или летаем по воздуху или путешествуем по другим странам или что-нибудь иное в этом роде. И всё-таки мы на самом деле воображаем, что
\vs Ars 1:215
эти вещи происходят. Насколько я могу решить, я достиг следующего решения. Ты должен, о царь, управлять своими словами и делами в законе благочестия всяким возможным образом, так чтобы ты мог сознавать, что ты соблюдаешь добродетель и никогда не решал вознаграждать себя за расточение разума и никогда не превышать своей власти до того, чтобы
\vs Ars 1:216
презирать праведность. Ибо ум по большей части занимается во сне тем же, чем он занят, бодрствуя. И тот, кто направил все свои мысли и дела к самым благородным целям, утверждает себя в праведности и во время бодрствования и во время сна. Благодаря этому ты можешь пребывать неуклонно в постоянном самоблюдении.
\vs Ars 1:217
Царь воздал хвалу мужу и сказал другому: Поскольку ты отвечаешь десятым, то когда ты выскажешься, мы предадимся пиру. И затем он задал вопрос:
\vs Ars 1:218
Как я могу избегать недостойных дел сам по себе. И тот ответил: Всегда обращай внимание на твою славу и твоё верховное звание, так чтобы ты мог говорить и думать только то,
\vs Ars 1:219
что совместимо с ними, зная что все твои подданные думают и говорят о тебе. Ибо ты не должен казаться хуже актеров, которые тщательно изучают свои роли, которые им надо сыграть, и сообразовывают с ними все свои дела. Ты не играешь роль, но ты истинный царь, поскольку Бог наделил тебя царскою властью, чтобы ты соблюдал её вместе с твоей славой.
\vs Ars 1:220
Когда царь похвалил громко и долго и весьма любезно, гостей стали побуждать отдохнуть. И так, когда прекратился разговор, они предались течению пира.
\vs Ars 1:221
На следующий день всё было устроено по-прежнему, и когда царь уловил подходящий момент, чтобы задавать вопросы мужам, он спросил первого из тех, кто оставался
\vs Ars 1:222
неспрошенным: Что есть высший образ правления? И тот ответил: Управлять собою и не поддаваться порывам. Ибо каждй человек от природы обладает особым умственным увлечением.
\vs Ars 1:223
Возможно, большинство людей склонны к еде, питью и наслаждениям, а царь увлекается приобретением земель и великою славой. Но хорошо, когда во всём этом соблюдается умеренность. Что Бог дает, то мы должны принимать и хранить, но никогда не стремиться приобретать то, чего мы не в силах достичь.
\vs Ars 1:224
Довольный этими словами царь спросил у следующего, как ему быть свободным от зависти. И после краткого молчания тот ответил: Если ты прежде всего будешь смотреть на то, что славою и великим богатством всех царей наделяет Бог, а не сам царь своею властью. Все люди хотят иметь такую славу, но не могут, потому что это дар Божий.
\vs Ars 1:225
Царь похвалил мужа в длинной речи и затем спросил у другого, как ему научиться презирать врагов. И тот ответил: Если ты будешь выказывать добродушие ко всем и добиваться их дружбы, тебе не надо будеть никого бояться. Пользоваться всеобщею любовью это наилучший из даров, которые можно получить от Бога.
\vs Ars 1:226
Похвалив этот ответ, царь велел следующему мужу ответить на вопрос, как он может сохранить свою высокую славу. И тот отвечал так: Если ты будешь щедр и открыт сердцем, оделяя других добродушием и благодеяниями, ты никогда не утратишь твою славу, но если ты хочешь, чтобы и впредь милость пребывала с тобою, ты должен постоянно призывать Бога.
\vs Ars 1:227
Царь изъявил своё согласие и спросил у следующего, кому надлежит мужу являть щедрость. И тот ответил: Всеми признано, что нам надлежит являть щедрость по отношению к тем, кто благорасположен к нам, но я думаю, что нам должно являть то же самое щедрое расположение духа и к тем, кто враждебен нам, чтобы мы могли привлечь их к правде и выгоде для них самих. Но мы должны молить Бога о том, чтобы это совершилось, ибо Он управляет умами всех.
\vs Ars 1:228
Выразив своё согласие с ответом, царь просил шестого мужа ответить на вопрос, кому мы должны выказывать благодарность. И тот ответил: Нашим родителям постоянно, ибо Бог дал нам важнейшую из заповедей относительно должного почитания родителей. Затем Он поместил отношение к друзьям, ибо Он говорит: друг словно твоя душа. Хорошо тебе постараться сделать всех твоими друзьями.
\vs Ars 1:229
Царь сказал ему ласковое слово и затем спросил следующего, что по цене подобно красоте. И тот сказал: Благочестие, ибо оно есть преимущественный образ красоты, и его сила в любви, она же Божий дар. Ты уже приобрел это, и вместе все благословения жизни.
\vs Ars 1:230
Царь с великою любезностью одобрил ответ и спросил у другого, как ему, совершив ошибку, вновь вернуть своё имя на прежнюю степень. И тот сказал: Невозможно тебе совершить ошибку, ибо во всех людях ты взыскал семена благодарности, приносящие урожай благожелательности,
\vs Ars 1:231
который могущественней самого сильного оружия и обезпечивает величайшую безопасность. Но если кто-либо совершает ошибку, он никогда не должен повторять того, что привело к ней, но ему надлежит сообразовываться с дружбою и творить справедливость. Ибо дар от Бога быть способным творить дела добра, а не противные ему.
\vs Ars 1:232
Восхищенный этими словами царь спросил у другого, как ему быть свободным от печали. И тот ответил: Если ты никогда никому не причинял вреда, но творил добро и следовал стезею
\vs Ars 1:233
праведности, ибо её плоды дают свободу от печали. Но нам надлежит молить Бога о том, чтобы нежданное зло вроде смерти, болезни, скорби или чего-либо в этом роде не нашло на нас и не принесло вреда. Но поскольку ты предался благочестию, никакое из этих несчастий никогда не постигнет тебя.
\vs Ars 1:234
Царь одарил его великою похвалой и спросил десятого, что есть высший образ славы. И тот ответил: Почитать Бога, и делать это не [только] дарами и жертвоприношениями, но в чистоте души и святом убеждении, поскольку всё образовано и управляется Богом по Его воле. Этой цели ты добиваешься неотменно, как это видно для всех в твоих свершениях в прошлом и настоящем.
\vs Ars 1:235
Громким голосом царь поблагодарил их всех и говорил к ним милостиво и выразил своё одобрение всем, кто был тут, особенно же философам. Ибо те превосходили их и поведением и доводами, поскольку они положили Бога источником себе. Затем царь, чтобы показать свои добрые чувства, начал пить за здоровье гостей.
\vs Ars 1:236
На следующий день пир был приготовлен так же, и царь, как только представился момент, стал задавать вопросы мужам, которые сидели вслед за теми, кто уже отвечал; и у первого он спросил: Можно ли научить мудрости? И тот сказал: Душа устроена так, что может божественною силою принять всё доброе и отвергнуть противное.
\vs Ars 1:237
Царь выразил одобрение и спросил следующего мужа: Что самое полезное для здоровья? И тот сказал: Умеренность, а её невозможно обрести, доколе Бог не внушит расположение к ней.
\vs Ars 1:238
Царь сказал ему ласковое слово и спросил другого: Как человек может достойно заплатить долг благодарности родителям? И тот сказал: Никогда не причиняя им скорби, а это невозможно, доколе Бог не расположит ум к поиску самых благородных целей.
\vs Ars 1:239
Царь выразил согласие и спросил следующего, как ему сделаться жаждущим слушателем. И тот сказал: Помня, что всякое знание полезно, потому что с Божьей помощью оно делает тебя способным во время опасности выбрать что-либо из того, что ты изучил и применить против нашедшей на тебя напасти. И усилия человеческие в этом исполняются через присутствие Божие.
\vs Ars 1:240
Царь похвалил его и спросил другого, как ему избежать делать что-либо противное закону. И тот сказал: Если ты признаешь, что в сердца законодателей помышления охранять человеческие жизни вложил Бог, ты последуешь им.
\vs Ars 1:241
Царь подтвердил ответ этого мужа и сказал другому: В чем преимущество родства? И тот ответил: Если мы обратим внимание на то, что мы сами поражаемся невзгодами, выпадающими нашим ближним, и если их страдания делаются нашими, тогда сразу
\vs Ars 1:242
же становится явною сила родства, ибо лишь явив такие чувства, мы приобретем честь и достоинство в их глазах. Ибо помощь, когда она связана с добротою, есть сама в себе связь, разорвать которую никак нельзя. И в день их процветания мы не должны вожделеть того, чем они владеют, но молить Бога даровать им блага всякого рода.
\vs Ars 1:243
И вознаградив его тою же похвалою, что и прочих, царь спросил другого, как ему достичь свободы от страха. И тот отвечал: Когда ум сознает, что он не творил зла, и когда Бог направляет его на всякий благородный замысел.
\vs Ars 1:244
Царь выразил одобрение и спросил другого, как ему всегда приходить к правильному суждению. И тот отвечал, что если он будет постоянно держать перед глазами выпадающие людям невзгоды и признавать, что Бог иных лишает благоденствия, а других приводит к великим почестям и славе.
\vs Ars 1:245
Царь оказал мужу ласковый прием и попросил другого ответить на вопрос: Как ему избегать жизни в праздности и наслаждениях? И тот отвечал, что если он будет постоянно помнить о том, что он правитель великого царства и господин над великим множеством, и что его ум не должен заниматься иными вещами, но должен всегда изследовать, как ему наилучшим образом обезпечить их благоденствие. Он также должен молиться Богу о том, чтобы ничто из должного не было в пренебрежении.
\vs Ars 1:246
Воздав ему хвалу, царь спросил десятого, как ему распознать тех, кто поступает с ним лукаво. И тот ответил на вопрос: Если он будет наблюдать, насколько естественно их отношение к нему, и придерживаются ли они правил старшинства во время приемов и совета, а в ежедневном общении выходят ли когда-нибудь за пределы
\vs Ars 1:247
приличий в поздравлениях и в прочих манерах. Но Бог направит твой ум, о царь, ко всему благородному. Когда царь выразил своё громкое одобрение и похвалил всех по-одному (и вместе их похвалили все, кто был там), они обратились к радостям пира.
\vs Ars 1:248
И на следующий день, когда подошло время, царь спросил следующего мужа: Что есть грубейший вид пренебрежения? И тот ответил: Если человек не заботится о своих детях и не прилагает всяческих усилий к их воспитанию. Ибо мы всегда молим Бога не столько о самих себе, сколько о наших детях, чтобы им стяжать всякое благословение. Наши пожелания о том, чтобы наши дети сумели владеть собою, может быть исполнено только Божьей властью.
\vs Ars 1:249
Царь сказал, что он говорил хорошо, и потом спросил другого, как ему любить отечество. Сохраняя в уме, отвечал тот, мысль о том, что хорошо жить и умереть для своей страны. Пребывание на чужбине наводит презрение на бедняка и позор на богача, как если бы они были изгнаны за преступление. Если ты даруешь благодеяния всем, как ты это делаешь постоянно, Бог даст тебе милость во всём, и ты будешь признан любящим отечество.
\vs Ars 1:250
Выслушав этого мужа, царь спросил у следующего по порядку, как ему жить в дружбе с женою. И тот ответил: Признавая, что женщины по природе упорны и деятельны, когда добиваются исполнения своих желаний, и склонны резко менять свои суждения от ложных разсуждений, и их природа слаба по сути своей. Необходимо быть мудрым в обращении с ними
\vs Ars 1:251
и не порождать споров. Для того, чтобы пройти жизнь успешно, кормчий должен знать цель, к которой ему надлежит направиться. Лишь призыванием Божьей помощи люди будут соблюдать истинное направление жизни во всякое время.
\vs Ars 1:252
Царь выразил своё согласие и спросил следующего, как ему быть свободным от заблуждений. И тот отвечал: Если ты всегда будешь действовать с разсуждением и никогда не давать веры клевете, но сам испытывать то, что тебе говорят, и решать своим собственным судом просьбы, которые тебе приносят, и приводить всё на свет твоего суждения, ты будешь свободен от заблуждения, о царь! Но познание и исполнение этого есть дело Божественной силы.
\vs Ars 1:253
Восхищенный этими словами царь спросил другого, как ему быть свободным от гнева. И тот сказал в ответ на вопрос, что если он будет признавать, что он имеет власть надо всеми вплоть до предания их смерти, если он даст место гневу и будет безполезно и жалко, если он, потому что он повелитель,
\vs Ars 1:254
лишит жизни многих. Что за нужда гневаться, когда все покорны и никто не противится ему? Надлежит признавать, что Бог правит всем миром в духе добросердечия и без гнева во всём, и ты, о царь, сказал он, необходимо должен следовать Его примеру.
\vs Ars 1:255
Царь сказал, что он отвечал хорошо и потом спросил у следующего мужа: Что есть добрый совет? Делать добро во всякое время и с должным разсуждением, объяснил тот, сравнивая то, что выгодно для твоих дел с уроном, который может быть следствием принятия противного решения, и таким образом взвешивая каждый шаг, мы можем умудряться, и наши намерения могут быть исполнены. И самый важный из всех твой замысел Божьей силою найдет своё завершение потому, что ты соблюдаешь благочестие.
\vs Ars 1:256
Царь сказал, что этот муж отвечал хорошо, и спросил другого: Что такое философия? И тот объяснил: Правильное обсуждение всякого возникшего вопроса с тем, чтобы никогда не увлекаться порывами, но взвешивать всякий ущерб, причиняемый страстями, и действовать в правильной сообразности с тем, как того требуют обстоятельства, соблюдая умеренность. Но мы должны молиться Богу вселить в наш ум почтительное отношение к этому.
\vs Ars 1:257
Царь выразил своё согласие и спросил другого, как ему встречаться с признательностью, путешествуя в чужих странах. Будучи любезным ко всем, отвечал тот, казаться ниже, нежели выше тех, с кем ты путешествуешь. Ибо признано правило, что Бог по Своей истинной природе приемлет смиренного. А человеческий род любит тех, кто охотно подчиняется им.
\vs Ars 1:258
Выразив своё одобрение с этим ответом, царь спросил другого, как ему строить так, чтобы построенное сохранилось после него. И тот ответил на вопрос, что если сделанное им будет принадлежать к числу прекрасного и благородного, так что обладатели сохранят это ради его красоты, и он никогда не лишит себя тех, кто делает подобные вещи, и никогда не будет вынуждать других служить его
\vs Ars 1:259
нуждам без вознаграждения. Ибо наблюдая, как Бог печется о человеческом роде, наделяя его здоровьем и умственными способностями и иными дарами, он сам должен следовать Его примеру, воздавая людям вознаграждение за их тяжкий труд. Ибо дела, творимые в праведности, пребывают всегда.
\vs Ars 1:260
Царь сказал, что этот муж также отвечал правильно и спросил у десятого: Что есть плод мудрости? И тот ответил: То, что муж должен сознавать в себе, что он не делал зла
\vs Ars 1:261
и что ему надлежит прожить жизнь в истине, ибо от этого, о могучий царь, величайшая радость и стойкость души и крепкая вера в Бога умножатся в тебе, если ты будешь править твоим царством в благочестии. И когда они выслушали ответ, они все приветствовали его громкими восклицаниями, и после этого царь, преисполнившись радости, стал пить за их здоровье.
\vs Ars 1:262
И на следующий день пир продолжился также, как и в предыдущие, и когда подошел момент, царь стал задавать вопросы оставшимся гостям, и
\vs Ars 1:263
сказал первому: Как человеку уберечься от гордыни? И тот ответил: Если он остается уравновешенным и во всех обстоятельствах помнит, что он муж, правящий людьми; и: Бог низводит гордых в ничто, и возносит слабых и смиренных.
\vs Ars 1:264
Царь сказал ему милостивое слово и спросил следующего: Кого мужу следует избирать себе в советники? И тот ответил: Тех, кто был испытан во многих делах и сохранил непревратной благожелательность по отношению к нему и разделял с ним его намерения. И Бог Сам является тем, кто достоин исполнения этих целей.
\vs Ars 1:265
Царь похвалил его и спросил другого: Чем прежде всего надлежит владеть царю? Дружбою и любовью своих подданных, отвечал тот, ибо через них узы доброжелательности делаются нерасторжимыми. И Бог делает прочным это настолько, чтобы оно происходило согласно твоему желанию.
\vs Ars 1:266
Царь похвалил его и спросил у другого: Что есть цель слова? И тот ответил: Убедить твоего противника, показав ему его ошибки хорошо и правильно выстроенными доводами. Ибо так ты приобретешь слушателя, не споря с ним, но хваля его с тем, чтобы убедить его. А убеждение совершается силою Божьей.
\vs Ars 1:267
Царь сказал, что он дал хороший ответ, и спросил другого, как ему жить в дружбе со множеством различных племен, составляющих население его царства. Делая то, что надлежит в отношении каждого из них, отвечал тот, и беря себе праведность вождем, как ты это делаешь ныне с помощью проницательности, которою тебя наделил Бог.
\vs Ars 1:268
Царь был восхищен этим ответом и спросил у другого: При каких обстоятельствах человеку надлежит выносить скорбь? В невзгодах, которые выпадают нашим друзьям, ответил тот, когда мы видим, что они длительны и неизгонимы. Разум не дает нам печалиться о тех, кто умер и освободился от зла, но все люди печалятся о них, потому что думают, лишь о себе и своей собственной выгоде. Одною лишь силою Божьей мы можем избежать всякого зла.
\vs Ars 1:269
Царь сказал, что тот дал приличный ответ, и спросил другого: Как утрачивают доброе имя? И тот отвечал: Когда гордыня и необузданная самоуверенность держат власть, рождаются позор и потеря доброго имени. Ибо Бог есть Господь доброй славы и наделяет ею, когда хочет.
\vs Ars 1:270
Царь подтвердил ответ и спросил следующего мужа, кому люди должны доверяться. Тем, отвечал тот, кто служит тебе по доброй воле, а не из страха или своего интереса, помышляя о собственной наживе. Ибо знак любви это одно, а другое признак дурной воли и временного услужения. Ибо человек, который всегда ищет собственной наживы, в сердце своём предатель. Но ты владеешь привязанностью всех твоих подданных с помощью разсудительности, которою тебя одарил Бог.
\vs Ars 1:271
Царь сказал, что тот отвечал мудро, и спросил у другого, чем сохраняется царство. И тот ответил на этот вопрос: Заботой и предусмотрительностью о том, чтобы никакое зло не было сделано теми, кто стоит у власти над народом, и тем, что ты всегда делаешь с Божьей помощью, внушающей тебе важные суждения.
\vs Ars 1:272
Царь сказал ему одобряющее слово и спросил следующего: Чем соблюдаются благодарность и честь? И тот ответил: Добродетелью, ибо она есть творец добрых дел, и она уничтожается злом, даже если ты выказываешь благородство характера ко всем благодаря Божьему дару, которым Он наделил тебя.
\vs Ars 1:273
Царь милостиво принял ответ и спросил одиннадцатого (поскольку их было семьдесят и еще двое), как во время войны ему сохранять душевное спокойствие. И тот отвечал: Помня, что он не сделал зла никому из подданных, и что все они в ответ будут сражаться за него ради благодеяний, которые они получили, зная, что даже если они потеряют жизнь, ты позаботишься о тех,
\vs Ars 1:274
кто находится на их иждивении. Ибо ты никогда не забудешь о таком воздаянии, таково добросердечие, которым тебя одарил Бог.
Царь громко похвалил их всех и говорил с ними милостиво и потом много пил за здоровье каждого, сам предаваясь радости и осыпая своих гостей самыми щедрыми и радостными выражениями дружбы.
\vs Ars 1:275
На седьмой день были сделаны самые обильные приготовления и собраны и другие мужи от различных городов (и среди них большое число посланников). Когда подошел момент, царь спросил у первого из тех, кто еще не был спрошен, как ему избегать
\vs Ars 1:276
обмана неверным разсуждением. И тот ответил: Обращая тщательное внимание на говорящего, на то, что говорится и на обсуждаемый предмет, и задавая те же самые вопросы через некоторое время по-другому. Но обладать зорким умом и уметь выносить здравое суждение в каждом случае есть один из благих даров от Бога, и ты имеешь его, о царь!
\vs Ars 1:277
Царь громко одобрил ответ и спросил у другого: Почему большинство людей никогда не бывает добродетельно? Потому что, отвечал тот, все люди неумеренны по природе и склонны
\vs Ars 1:278
к наслаждениям. Отсюда возникает неправедность и потоком любостяжание. Обычай добродетели препятствие для тех, кто отдает жизнь на наслаждения, потому что он обязывается их предпочитать умеренности и праведности. Ибо хозяин сего Бог.
\vs Ars 1:279
Царь сказал, что он отвечал хорошо, и спросил [следующего], чему должны повиноваться цари. И тот сказал: Законам, так чтобы праведными поступками они могли возвращать людям жизнь. Когда ты делашь это в повиновении Божественным заповедям, ты откладываешь для себя запас в хранилищах вечной памяти.
\vs Ars 1:280
Царь сказал, что этот муж также говорил хорошо, и спросил у следующего: Кого мы должны назначать наместниками? И тот ответил: Всех тех, кто ненавидит порочность и, подражая твоему поведению, творит праведность с тем, чтобы ему поддерживать постоянно своё доброе имя. Ибо это то, что делаешь ты, о могучий царь, сказал он, и Бог возложил на тебя венец праведности.
\vs Ars 1:281
Царь громко приветствовал этот ответ и затем, глядя на следующего мужа, сказал: Кого мы должны назначать начальниками над войсками? И тот объяснил: Тех, кто блистает храбростью и праведностью и тех, кто заботится более о том, чтобы беречь своих людей, чем о том, чтобы добиться победы, опрометчиво подвергая опасности их жизнь. Ибо так же, как Бог действует ко благу всех людей, так и ты, последуя Ему, творишь благо для всех твоих подданных.
\vs Ars 1:282
Царь сказал, что он дал хороший ответ и спросил у другого: Какой человек достоин восхищения? И тот ответил: Человек, наделенный доброю славой и богатством и силой, и душа которого относится одинаково ко всему. Ты сам своими деяниями являешь себя наидостойнейшим восхищения благодаря Божьей помощи, направляющей твоё внимание к этому.
\vs Ars 1:283
Царь выразил своё одобрение и сказал следующему: Чему царь должен уделять наибольшее время? И тот отвечал: Чтению и изучению записей путешествий твоих чиновников, описывающих различные царства для того, чтобы изменять и охранять твоих подданных. И благодаря таким деяниям ты достиг славы, которой не приближался никто другой с Божьей помощью, исполняющей все твои желания.
\vs Ars 1:284
Царь сказал воодушевленное слово этому мужу и спросил у следующего, чем заниматься человеку в часы отдыха и развлечения. И тот отвечал: Смотреть те игры), в которых можно соблюсти уместность и представлять перед глазами сцены, взятые из жизни и показанные
\vs Ars 1:285
достойно и прилично, и полезно и благопристойно. Ибо даже в таких развлечениях можно найти некое наставление, потому что часто какой-нибудь нужный урок извлекается из самого незначительного житейского обстоятельства. Но соблюдая наивысшую пристойность во всех твоих деяниях, ты явил себя философом и был почтен у Бога ради твоей добродетели.
\vs Ars 1:286
Царь, обрадованный весьма хорошо сказанными словами, сказал девятому мужу: Как надлежит вести себя человеку на пирах? И тот ответил: Тебе надлежит собирать у себя ученых мужей и тех, кто способен указать тебе нечто полезное касательно дел твоего царства и жизни твоих подданных (ибо тебе не найти предмета, более подходящего или более
\vs Ars 1:287
наставительного, чем это), поскольку эти люди угодны Богу, потому что они обучили свой ум созерцать самые благородные предметы, как и ты сам, конечно, делаешь это, поскольку все твои дела направляются Богом.
\vs Ars 1:288
Восхищенный ответом, царь спросил у следующего мужа: Что для народа лучше всего? То ли, что царем над ним может стать частный гражданин или же член царской семьи? И тот
\vs Ars 1:289
ответил: Лучший по природе. Ибо цари, происходящие от царской крови, часто грубы и суровы к своим подданным. И всё же часто бывает с некоторыми из тех, кто вознесся из рядов частных граждан, что испытав злое и пожив некое время
\vs Ars 1:290
в бедности, они, управляя многими, становятся свирепее безбожных тираннов. Но, как я сказал, добрая природа, получив пристойное обучение, способна к управлению, и ты великий царь, не столько потому, что блистаешь славою твоего правления и твоими богатствами, сколько оттого, что ты превзошел всех людей в милости и человеколюбии, благодаря Богу, одарившему тебя этими достоинствами.
\vs Ars 1:291
Царь некоторое время восхвалял этого мужа, а после спросил самого последнего: Что есть величайшее свершение в правлении царством? И тот отвечал: То, когда все подданные могут непрерывно жить в мире, и правосудие быстро разрешает споры.
\vs Ars 1:292
Это может быть достигнуто благодаря силе правителя, когда это муж, ненавидящий зло и любящий благо и отдающий свои устремления спасению человеческой жизни, точно также, как ты считаешь несправедливость худшею разновидностью зла и своим управлением создал себе безсмертное доброе имя, поскольку Бог даровал тебе ум чистый и незапятнанный злом.
\vs Ars 1:293
И когда он закончил, раздались длительные, громкие и радостные голоса одобрения. Когда они замолкли, царь взял чашу и сказал слово в честь всех своих гостей и тех слов, что они произнесли. И в конце он сказал: Я извлек величайшую выгоду из вашего прихода,
\vs Ars 1:294
я многое получил от мудрого учения, преподанного вами мне об искусстве правления. Потом он повелел дать каждому по три таланта серебра и указал одному из своих рабов раздать деньги. Тут же все восклицанием выразили своё согласие, и пир превратился в зрелище радости, в то время как царь предался непрерывной череде застолья.
\vs Ars 1:295
Я писал долго и должен просить тебя о прощении, Филострат. Я был безмерно изумлен этими мужами и тем, как они мгновенно давали ответы,
\vs Ars 1:296
требующие поистине длительного размышления. Ибо несмотря на то, что спрашивающий предлагал каждому сложную мысль в каждом вопросе, отвечавшие один за другим давали свои ответы уже готовыми тотчас же и так, что мне и всем, кто был там, а особенно философам, они казались достойными восхищения. И я полагаю, что это покажется невероятным тем, кто
\vs Ars 1:297
прочтет мой рассказ в будущем. Но не подобает искажать то, что записано в государственных архивах. И неправедным с моей стороны было бы извращать подобный предмет. Я рассказываю эту историю так, как она происходила, тщательно избегая любой ошибки. Сила их высказываний запечатлелась во мне настолько, что я особенно переговорил с теми, чья обязанность состоит в том, чтобы
\vs Ars 1:298
записывать всё, что происходит на царских приемах и пирах. Ибо существует, как ты знаешь, обычай с того момента, как царь затевает какое-либо дело и вплоть до того, как он удаляется на покой, вести запись всего, что он говорит или делает, устроение превосходное и полезное.
\vs Ars 1:299
Ибо на следующий день всё по времени, что говорилось и делалось накануне, читается прежде начала дела, и если находится в этом какая-нибудь неправильность, её тотчас же исправляют.
\vs Ars 1:300
И благодаря этому, как уже сказано, я добился точных сведений из государственных архивов и расположил события в правильном порядке, поскольку я знаю, сколь жадно ты стремишься получать полезные сведения.
\vs Ars 1:301
Через три дня Димитрий собрал этих мужей, проведя их вдоль дамбы за семь стадий к острову, перешел мост и направился к северным кварталам Фароса. Там он собрал их в доме, построенном на дамбе, весьма красивом и уединенном, и пригласил их приступить к переводу, поскольку всё, что было им нужно для этого
\vs Ars 1:302
находилось в их распоряжении. И так они принялись за работу, сравнивая сделанное каждым и согласовываясь между собою, и всё согласованное надлежащим образом копировалось под руководством Димитрия.
\vs Ars 1:303
И работа длилась до девятого часа, после чего они могли свободно исполнять свои
\vs Ars 1:304
телесные нужды. Всё, в чем они нуждались, им доставлялось весьма щедро. В дополнение к этому Дорофей приготовлял для них ежедневно то же, что и для самого царя, ибо таково было повеление царя. Каждое утро они рано появлялись во дворце, и,
\vs Ars 1:305
поклонившись царю, возвращались на своё место. И по еврейскому обычаю они мыли руки в море и молились Богу и затем предавались чтению и
\vs Ars 1:306
переводу того или иного начатого места; и я спросил у них, почему они моют руки перед тем, как молятся. И они объяснили, что это делается в знак того, что они не делают зла (ибо всё, что ни делается, делается руками), поскольку благородно и свято видят во всем символ праведности и истины.
\vs Ars 1:307
Как я уже сказал, они собирались ежедневно на месте, восхитительном из-за его спокойствия и веселости, и принимались за работу. И так случилось, что весь перевод был исполнен за семьдесят два дня, будто бы преднамерено заранее.
\vs Ars 1:308
И когда работа была закончена, Димитрий собрал всё еврейское население туда, где работали переводчики, и прочитал им перевод вслух в присутствии переводчиков, которых также, как и народ, принял с почестями из-за великих благодеяний, которые они привлекли
\vs Ars 1:309
на него. Они горячо похвалили Димитрия и побуждали его переписать весь Закон целиком и передать список их начальникам.
\vs Ars 1:310
После того, как книга была прочитана, священники и старейшие из переводчиков, и еврейская община, и начальники народа поднялись и сказали, что теперь, когда совершен столь превосходный и столь тщательный перевод, будет только справедливо, если он останется таким, каков он есть и никакое
\vs Ars 1:311
изменение не будет внесено в него. И когда все выразили согласие с этим, они просили произнести проклятие согласно их обычаю на тех, кто внесет туда какое-нибудь изменение, или прибавив что-нибудь или изменив как-нибудь хоть одно слово из написанного, или опустив что-нибудь. Это было весьма мудрою предосторожностью для того, чтобы наверняка сохранить эту книгу неизменной на всё время в будущем.
\vs Ars 1:312
Когда об этом доложили царю, он весьма обрадовался, ибо он чувствовал, что его желание было исполнено в целости. Вся книга была прочитана ему вслух, и он весьма изумился духу законодателя. И он сказал Димитрию: Как это никто из историков или поэтов не подумал когда-либо посвятить хотя бы слово столь чудесному
\vs Ars 1:313
деянию? И Димитрий ответил: Потому что Закон свят и исходит от Бога. И некоторые из имевших намерение прикоснуться к нему были поражены Богом и отступили от
\vs Ars 1:314
своих замыслов. Он сказал, что слышал от Феопомпа, как тот на тридцать дней потерял разсудок, потому что попытался вставить в свою историю какие-то события из раннего и малодостоверного перевода Закона. Когда он немного пришел
\vs Ars 1:315
в себя, он просил Бога открыть ему, за что ему выпала эта напасть. И ему было открыто во сне, что из пустого любопытства он хотел передать священную истину обычным людям, и что если он отступится, он обретет вновь здоровье. Я также слышал из уст
\vs Ars 1:316
Феодекта, одного из тех поэтов, что пишут трагедии, что когда он попытался переделать кое-что из происшествий, записанных в этой книге для своей трагедии, оба его глаза покрылись бельмами. Когда он понял причину своего несчастья, он много дней молил Бога и затем обрел здоровье.
\vs Ars 1:317
И после того, как царь, как я уже сказал, получил от Димитрия объяснение этому вопросу, он поклонился и повелел, чтобы с книгами обращались с великою тщательностью, и чтобы их
\vs Ars 1:318
хранили как святыню. И он горячо приглашал переводчиков часто навещать его после того, как они возвратятся в Иудею, ибо, говорил он, лишь так будет справедливо, если теперь он отпустит их домой. Но когда они придут снова, он
\vs Ars 1:319
примет их как своих друзей, как подобает, и они получат богатые подарки от него. Он повелел сделать приготовления к их возвращению домой и явил к ним великую щедрость. Он подарил каждому три превосходнейших одежды, два золотых таланта, сундук весом в талант, всё, что необходимо для трех лож.)
\vs Ars 1:320
А в сопровождение он послал Елеазару десять лож на серебряных ножках и всё необходимое к ним, сундук ценою в тридцать талантов, десять одежд, багряницу и великолепный венец, и сто штук тончайшей шерсти, а также кубки и блюда, и две золотых чаши для посвящения Богу.
\vs Ars 1:321
Он просил его также в письме о том, что если кто из этих мужей захочет вернуться к нему, не препятствовать ему. Ибо он считал за честь наслаждаться обществом столь ученых мужей, предпочитая расточать свои богатства на них, нежели на суетное.
\vs Ars 1:322
И теперь, Филострат, ты получил полный рассказ, согласно моему обещанию. Я полагаю, ты испытаешь большее удовольствие от этого, чем от сочинений баснописцев. Ибо ты предан тому, что может принести пользу душе, и уделяешь этому много времени. Я постараюсь рассказать о каких-нибудь еще событиях, стоящих записывания, так что внимательно читая их, ты сможешь удостоверить высочайшую награду за твоё усердие.

\bibbookdescr{Tjb}{
  inline={Завещание Иова,\\непорочного, жертвы, завоевателя во многих соревнованиях},
  toc={Завещание Иова},
  bookmark={Завещание Иова},
  header={Завещание Иова},
  abbr={Зав~Иов}
}
\vs Tjb 1:1
Однажды он стал больным, и, зная, что он должен будет оставить свою телесную обитель, он призвал своих семерых сыновей (их имена: Терси, Хор, Гион, Никэ, Фор, Фиф, Фруон) и своих трех дочерей вместе и сказал им так:
\vs Tjb 1:2
Составьте круг около меня, дети, и послушайте, и я разскажу вам, что Господь сделал для меня, и всё, что происходило со мною.
\vs Tjb 1:3
Ибо я~--- Иов, ваш отец.
\vs Tjb 1:4
К тому же знайте, мои дети, что вы~--- род избранный, и примите во внимание ваше благородное рождение.
\vs Tjb 1:5
Ибо я~--- из сынов Исава. Мой брат Нерос, и ваша мать Дина. Чрез неё я стал отцом вашим.
\vs Tjb 1:6
Ибо моя первая жена умерла с моими другими десятью детьми горькой смертью.
\vs Tjb 1:7
Послушайте теперь, дети, и я открою вам, что происходило со мною.
\vs Tjb 1:8
Я был очень богатым человеком, живущим на Востоке в земле Уц, и прежде, чем Господь назвал меня Иовом, я был назван Иовав.
\vs Tjb 1:9
Начало моего испытания было таким. Возле моего дома был идол одного поклоняющегося [ему] народа; и я постоянно видел жертвоприношения ему как богу.
\vs Tjb 1:10
Тогда я подумал и сказал в себе: Тот ли это, который сотворил небо и землю, море и нас всех? Как я узнаю истину?
\vs Tjb 1:11
И в ту ночь, когда я лег спать, пришел глас и позвал: Иовав! Иовав! Встань, и я поведаю тебе, Кто~--- Тот, Кого ты пожелал узнать.
\vs Tjb 1:12
Тот, впрочем, кому люди приносят всесожжения и возлияния,~--- не Бог, но это~--- сила и действо Соблазнителя, которыми он обманывает людей.
\vs Tjb 1:13
И когда я слушал это, я пал на землю и я простерся, говоря:
\vs Tjb 1:14
O господин мой, который говорит для спасения души моей! Я прошу тебя, если это~--- идол Сатаны, я прошу тебя, позволь мне пойти отсюда и уничтожить его и очистить это место.
\vs Tjb 1:15
Ибо вот нет ни одного, кто может запретить мне сделать это, потому что я~--- царь этой земли; так чтобы те, что живут в ней, больше не вводились в заблуждение.
\vs Tjb 1:16
И глас, который говорил из пламени, ответил мне: Ты можешь очистить это место.
\vs Tjb 1:17
Но вот, я возвещаю тебе то, что Господь повелел мне, чтобы я сообщил тебе, ибо я~--- архангел Божий.
\vs Tjb 1:18
И я сказал: То, что будет сказано Его рабу, я буду слушать.
\vs Tjb 1:19
И архангел сказал мне: Так говорит Господь: если ты решишься уничтожить и убрать образ Сатаны, он примется с гневом вести войну против тебя, и он покажет на тебе всю свою злобу.
\vs Tjb 1:20
Он принесет тебе много жестоких бедствий и отнимет у тебя всё, что ты имел.
\vs Tjb 1:21
Он отнимет твоих детей и причинит много зла тебе.
\vs Tjb 1:22
Тогда ты должен бороться подобно борцу и сопротивляться боли, уверенный в своей награде, преодолевать испытания и бедствия.
\vs Tjb 1:23
Но когда ты претерпишь, Я сделаю твое имя известным среди всех поколений земли до кончины мира.
\vs Tjb 1:24
И Я возвращу тебе всё, что ты имел; и вдвойне от того, что ты потерял, дам тебе, чтобы ты мог знать, что Бог нелицеприятен, но дает каждому, кто заслужил, благо.
\vs Tjb 1:25
И тебе также будет дано оно, и ты наденешь диадему амарантовую,
\vs Tjb 1:26
и в воскресение ты пробудишься для жизни вечной. Тогда ты узнаешь, что Он~--- Господь праведный и истинный и Всемогущий.
\vs Tjb 1:27
После чего, дети мои, я ответил: Я буду из любви Божьей терпеть до смерти всё, что снизойдет на меня, и я не уклонюсь вспять.
\vs Tjb 1:28
Тогда ангел положил свою печать на мне и покинул меня.

\vs Tjb 2:1
После этого я встал ночью и взял пятьдесят рабов, и пошел к храму идола и разрушил его до основания.
\vs Tjb 2:2
И так я возвратился в мой дом и дал наказы, чтобы дверь его крепко затворили, говоря моим привратницам:
\vs Tjb 2:3
Если кто-нибудь будет спрашивать обо мне, не доносите никакого сообщения ко мне, но скажите ему: Он занят срочными делами, он~--- внутри.
\vs Tjb 2:4
Тогда Сатана притворился нищим и сильно стучал в двери, говоря привратнице:
\vs Tjb 2:5
Сообщите Иову и скажите, что я желаю встретиться с ним.
\vs Tjb 2:6
И привратница пошла и сказала мне это, но услышала от меня, что я занят.
\vs Tjb 2:7
Велиал, потерпев неудачу в этом, ушел и взял на своё плечо старую порванную корзину, и пришел и сказал привратнице, говоря: Скажите Иову: дай мне хлеба от рук твоих, чтобы я мог поесть.
\vs Tjb 2:8
И когда я услышал это, я дал ей подгорелого хлеба, чтобы дать его ему, и я известил его: Не жди есть моего хлеба, ибо это запрещено тебе.
\vs Tjb 2:9
Но привратница, устыдившись вручить ему подгорелый и сожженный хлеб (ибо она не знала, что это был Сатана), взяла своего хорошего хлеба и дала его ему.
\vs Tjb 2:10
Но он [не] взял его и, зная, что произошло, сказал деве: Иди прочь, плохая служанка, и принеси мне хлеб, который дали тебе, чтобы вручить мне.
\vs Tjb 2:11
И служанка воскликнула и сказала в печали: Ты говоришь истину, говоря, что я являюсь плохой служанкой, ибо я не сделала так, как была научена моим владыкой.
\vs Tjb 2:12
И она возвратилась и принесла ему горелого хлеба и сказала ему: Так говорит мой господин: тебе не есть от моего хлеба больше, ибо это запрещено тебе.
\vs Tjb 2:13
И это он дал мне, [говоря: это я дал] с тем, чтобы не могло быть навлечено против меня обвинение, что я не подал врагу, который просил.
\vs Tjb 2:14
И когда Сатана услышал это, он отослал служанку обратно ко мне, говоря: Как ты видишь этот хлеб весь сожженным, так буду я скоро жечь твоё тело, чтобы сделать его подобным тому.
\vs Tjb 2:15
И я ответил: Делай, что ты желаешь делать и исполни всё, что ты замыслил. Ибо я готов претерпеть всё, что бы ты ни навел на меня.
\vs Tjb 2:16
И когда диавол услышал это, он оставил меня, и, взойдя на [самое высокое] поднебесье, он взял от Господа клятву, что он сможет иметь власть над всем моим имением.
\vs Tjb 2:17
И после получения власти он пошел и тотчас взял всё мое богатство.

\vs Tjb 3:1
И я имел сто и тридцать тысяч овец, и из них я отделял семь тысяч на одежду сиротам и вдовам, и нуждающимся и больным.
\vs Tjb 3:2
Я имел загон из восьмисот псов, которые стерегли моих овец, и помимо них~--- двести, чтобы охраняли мой дом.
\vs Tjb 3:3
И я имел девять мельниц, работающих для всего города, и корабли, чтобы перевозить товары, и я доставлял их во всякий город и в селения немощному и больному и тем, которые были несчастны.
\vs Tjb 3:4
И я имел триста и сорок тысяч вьючных ослов, и из них я отбирал пятьсот, и потомство их я определял на продажу, а доходы отдавал нищим и нуждающимся.
\vs Tjb 3:5
Ибо со всех стран нищие приходили, чтобы встретиться со мною.
\vs Tjb 3:6
Ибо четыре двери моего дома были отверсты, каждая, будучи под ответственностью сторожа, который наблюдал, каждый ли из приходящих людей получал милостыню, и видел ли меня сидящим у одной двери, так чтобы они могли выходить через другую и получать всё, в чем они нуждались.
\vs Tjb 3:7
Также я имел тридцать столовых наборов, предназначенных на всякое время для одиноких странников, и я также имел двенадцать просторных столов для вдов.
\vs Tjb 3:8
И каждый, кто бы ни приходил, прося о милостыне, он находил пищу на моем столе, получая всё, в чем он нуждался, и я не позволял никому покинуть мою дверь с пустым животом.
\vs Tjb 3:9
Я также имел три тысячи пятьсот пар волов, и я отбирал из них пятьсот и отводил их пахарям.
\vs Tjb 3:10
И с ними я делал всю работу на всяком поле, у тех, кто хотел бы взять его на попечение, и доход их урожаев я откладывал для нищих на их столе.
\vs Tjb 3:11
Я также имел пятьдесят пекарен, от которых я посылал [хлеб] к столу для бедняков.
\vs Tjb 3:12
И я имел рабов, избранных для служения им.
\vs Tjb 3:13
Имелись также некоторые странники, которые видели мою добрую волю; они желали послужить как служители сами.
\vs Tjb 3:14
Другие, будучи в бедствии и неспособные приобрести средства к существованию, приходили с просьбой, говоря:
\vs Tjb 3:15
Мы просим тебя: поскольку мы тоже можем выполнять эту обязанность служителей и не имеем никакого стяжания, сжалься над нами и ссуди денег нам, с тем чтобы мы могли пойти в большие города и продавать товары.
\vs Tjb 3:16
И излишек от нашей прибыли мы можем отдавать в помощь нищим, и затем мы возвратим тебе твою собственность.
\vs Tjb 3:17
И когда я слышал это, я был доволен тем, что они возьмут это совместно со мною ради бережливости и милосердия к нищим.
\vs Tjb 3:18
И по желанию сердца я давал им что они хотели, и я принимал их расписки, но не брал никакого иного залога от них, кроме письменного свидетельства.
\vs Tjb 3:19
И они ходили повсюду и подавали вовремя бедным, насколько они преуспевали.
\vs Tjb 3:20
Часто, однако, некоторые из их товаров были теряемы на пути или на море, или их он отнимал у них.
\vs Tjb 3:21
Тогда они придут и скажут: Мы просим тебя: будь великодушен к нам, с тем чтобы мы могли подумать, как мы можем возвратить тебе твою собственность.
\vs Tjb 3:22
И когда я слышал это, я сочувствовал им и вручал им их расписку, и часто читавший её перед ними разрывал её, и прощал им из их долга, говоря им:
\vs Tjb 3:23
Что я посвятил для помощи бедным, я не буду брать от тебя.
\vs Tjb 3:24
И так я ничего не брал от моего должника.
\vs Tjb 3:25
И тогда муж с весёлым сердцем приходил ко мне, говоря: Мне не нужен трудовой заработок бедняка, необходимый ему;
\vs Tjb 3:26
но я желаю служить нуждающемуся за вашим столом. И он соглашался работать, и он ел свою долю.
\vs Tjb 3:27
Однако же я давал ему его плату, и я\fnote{я}{он(?)} уходил домой, радуясь.
\vs Tjb 3:28
А когда он не хотел брать это, я вынуждал его, чтобы делать так, говоря: Я знаю, что ты~--- муж труда, который надеется и ждет своей платы, и ты должен брать это.
\vs Tjb 3:29
Никогда я не отсрочиваю плату мзды наемнику или любому другому, ни задерживаю в моём доме в течение одного вечера из его найма, который был должен ему.
\vs Tjb 3:30
Те, которые доили коров и овец, сообщали проходящим, что они должны взять свою долю.
\vs Tjb 3:31
Ибо молоко текло в таком множестве, что оно свертывалось в масло на склонах и на краю дороги; и у камней и холмов скот ложился, чтобы родить своё потомство.
\vs Tjb 3:32
Ибо мои слуги утомлялись хранением мяса вдов и нищих и делили его [себе] на маленькие кусочки.
\vs Tjb 3:33
Ибо они, бывало, ругались и говорили: О, что мы имеем от его мяса, чем мы могли бы быть удовлетворены!, хотя я был очень любезен к ним.
\vs Tjb 3:34
Я также имел шесть арф [и шесть рабов, играющих на арфах], и также лиру десятиструнную, и я ударял по ним в течение дня.
\vs Tjb 3:35
И я брал лиру, и вдовы подпевали [мне] после их еды.
\vs Tjb 3:36
И с музыкальным орудием я напоминал им о Боге, что они должны воздавать хвалу Господу.
\vs Tjb 3:37
И когда мои рабыни, бывало, роптали, тогда я брал музыкальные орудия и играл столько, сколько они сделали бы за их плату, и давал им облегчение от их труда и воздыханий.

\vs Tjb 4:1
И мои дети, взяв ответственность за служение, брали своё ежедневное пропитание вместе с их тремя сестрами, начиная со старшего брата, и делали праздник.
\vs Tjb 4:2
И я вставал утром и предлагал, как искупительную жертву за них, пятьдесят овнов и девятнадцать овец, и что оставалось, как остаток было посвящаемо бедным.
\vs Tjb 4:3
И я говорил им: Берите это как остаток, и молитесь за моих детей.
\vs Tjb 4:4
Возможно, мои сыновья грешили перед Господом, говоря в надменности духа: Мы~--- дети этого богатого человека. Наше~--- всё это добро; почему мы должны быть слугами нищих?
\vs Tjb 4:5
И говоря так в надменности духа, они, возможно, вызывали гнев Бога, ибо гордое превозношение~--- мерзость пред Господом.
\vs Tjb 4:6
И так я приносил волов как жертву священнику на алтарь, говоря: Не хулили ли мои дети когда-либо Бога в своих сердцах?
\vs Tjb 4:7
Пока я жил таким образом, Соблазнитель не мог спокойно смотреть на добро [творимое мною], и он испросил у Бога войну против меня.
\vs Tjb 4:8
И он напал на меня безжалостно.
\vs Tjb 4:9
Сначала он сжег большое количество овец, потом верблюдов, затем он сжег волов и всё моё стадо; или же они были захвачены не только врагами, но также и теми, кто был облагодетельствован мною.
\vs Tjb 4:10
И пастухи пришли и возвестили это мне.
\vs Tjb 4:11
Но когда я услышал это, я воздал хвалу Богу и не богохульствовал.
\vs Tjb 4:12
И когда Соблазнитель познал моё терпение, он замыслил новое дело против меня.
\vs Tjb 4:13
Он вошел в царя Фираса и осадил мой город, и после того, как он увел всё, что было там, он сказал им в злобе, говоря хвастливой речью:
\vs Tjb 4:14
Этот муж Иов~--- тот, который получил все блага земли и не оставил ничего для других, он разрушил и низверг храм бога.
\vs Tjb 4:15
Поэтому я воздам ему тем же, что он сделал дому великого бога.
\vs Tjb 4:16
Ныне идите со мною, и мы будем грабить всё, что осталось в его доме.
\vs Tjb 4:17
И они ответили и сказали ему: Он имеет семь сыновей и трех дочерей.
\vs Tjb 4:18
Позаботься, чтобы они не убежали в другие страны, и не стали бы нашими мучителями, и тогда они превозмогут нас силою и убьют нас.
\vs Tjb 4:19
И он сказал: Ничего не бойтесь. Его стада и его богатство я уничтожил огнем, и остальное я расхитил, и вот, его детей я убью.
\vs Tjb 4:20
И говоря так, он пошел и обрушил дом на моих детей и убил их.
\vs Tjb 4:21
И мои сограждане, видя то, что сказанное им действительно совершилось, пришли и преследовали меня, и отняли у меня всё, что было в моем доме.
\vs Tjb 4:22
И я увидел моими глазами грабеж моего дома, и люди невоспитанные и безчестные сидели за моим столом и на моих ложах, и я не мог возражать против них.
\vs Tjb 4:23
Ибо я был истощен подобно женщине с её ложеснами, освободившимися от множества болей, помня главное~--- что эта война была предсказана мне Господом через Его ангела.
\vs Tjb 4:24
И я стал подобным тому, кто, видя бурное море и противные ветры, в то время как груз судна посреди океана слишком тяжел, сбрасывает тяжесть в море, говоря:
\vs Tjb 4:25
Я хочу уничтожить всё это для того, чтобы благополучно прибыть в город, так чтобы я мог взять как прибыль спасенное судно и лучшее из моих вещей.
\vs Tjb 4:26
Так я управлял моими делами.
\vs Tjb 4:27
Но вот пришел другой вестник и возвестил мне о гибели моих детей, и я был потрясен ужасом.
\vs Tjb 4:28
И я разодрал мою одежду и сказал: Господь дал, Господь и взял. Как это было угодно Господу, так это и сделалось. Да будет имя Господне благословенно.

\vs Tjb 5:1
И когда Сатана увидел, что он не смог возбудить во мне отчаяние, он пошел и выпросил моё тело у Господа, дабы причинить язву мне, ибо Велиал не мог вынести моего терпения.
\vs Tjb 5:2
Тогда Господь предал меня в его руки, чтобы использовать моё тело, как он хотел, но Он не дал ему власти над моей душой.
\vs Tjb 5:3
И он пришел ко мне, я же был сидящим на моём троне, всё еще печалясь по моим детям.
\vs Tjb 5:4
И он был подобен великому урагану и перевернул мой трон и бросил меня оземь.
\vs Tjb 5:5
И я долго лежал на полу в течение трех часов. И он поразил меня тяжкой проказой от темени головы моей до кончиков ног моих.
\vs Tjb 5:6
И я покинул город в великом ужасе и горе и сел на навозную кучу моим телом червоточивым.
\vs Tjb 5:7
И я орошал землю мокротою моего воспаленного тела, ибо гной стекал с моего тела, и множество червей покрывало его.
\vs Tjb 5:8
И когда один [какой-нибудь] червь сползал с моего тела, я клал его назад, говоря: Останься на том месте, где ты находился, пока Тот, Кто послал тебя, не направит тебя куда-нибудь еще.
\vs Tjb 5:9
Так я претерпевал в течение семи лет, сидя на навозной куче вне города, будучи поражен проказой.
\vs Tjb 5:10
И я увидел своими глазами моих томящихся детей
\vs Tjb 5:11
и мою унижающуюся жену, которая [некогда] была приведена в её свадебный чертог в такой великой роскоши и с копьеносцами как телохранителями. Я видел её выполняющею работу носильщика воды, подобно рабу, в доме простого человека, для того чтобы заработать немного хлеба и принести его мне.
\vs Tjb 5:12
И в моем лютом бедствии я сказал: О, что [значат] эти хвастливые правители города, которые будут теперь нанимать мою жену как служанку, которых душу я не подумаю сравнить [даже] с моими сторожевыми псами!
\vs Tjb 5:13
И после этого я обрел храбрость вновь.
\vs Tjb 5:14
Однако, впоследствии они отказывали [ей] даже в хлебе [для меня], чтобы она имела только её собственное пропитание.
\vs Tjb 5:15
Но она брала это и разделяла это между собою и мною, говоря скорбно: Горе мне! Отныне он больше не сможет питаться хлебом, и он не может пойти на торжище попросить хлеба у хлеботорговцев для того, чтобы принести его мне [и] чтобы он мог есть.
\vs Tjb 5:16
И когда Сатана узнал это, он принял облик хлеботорговца; и это было как будто случайным, что моя жена встретила его и спросила его о хлебе, думая, что это был его человеческий вид.
\vs Tjb 5:17
Но Сатана сказал ей: Дай мне цену, и потом бери, что ты пожелаешь.
\vs Tjb 5:18
Тогда она ответила, говоря: Где я возьму денег? Разве ты не знаешь, какая беда произошла со мною? Если ты имеешь жалость, яви её мне; если нет, ты смотри.
\vs Tjb 5:19
И он ответил, говоря: Если бы ты не заслуживала этой беды, ты бы не испытала всё это.
\vs Tjb 5:20
Ныне, если нет сребренника в руке твоей, дай мне волосы головы твоей и возьми три буханки хлеба за это, так что ты сможешь прожить на них три дня.
\vs Tjb 5:21
Тогда она сказала в себе: Что есть волосы головы моей по сравнению с моим голодающим мужем?
\vs Tjb 5:22
И так, подумав над вопросом, она сказала ему: Встань и отрежь мои волосы.
\vs Tjb 5:23
Тогда он взял ножницы и отнял волосы её головы в присутствии всех и дал ей три буханки хлеба.
\vs Tjb 5:24
Тогда она взяла их и принесла их мне. И Сатана последовал за ней по дороге, притаившись когда он шел и весьма безпокоя её сердце.

\vs Tjb 6:1
И тотчас же моя жена пришла ко мне и, вопия громко и плача, она сказала: Иов, Иов! Как долго ты сидишь на навозной куче вне города, размышляя уже на протяжении [столького] времени и ожидая получить твоё желанное спасение!
\vs Tjb 6:2
И я должна была блуждать с места на место, скитаясь повсюду как наемная служанка, [и слышать:] вот их память уже исчезла от земли.
\vs Tjb 6:3
И мои сыновья и дочери, которых я носила на моей груди, и труды и муки, которые я выдержала, были напрасны?
\vs Tjb 6:4
И ты сидишь в смраде болезни и червях, проводя ночи на холодном воздухе.
\vs Tjb 6:5
И я подвергалась всяким испытаниям и скорбям и мукам, днем и ночью, пока я не преуспевала в снабжении тебя хлебом.
\vs Tjb 6:6
Поскольку твоего излишка хлеба больше не позволили мне [брать]; и поскольку я едва могу брать мою собственную пищу и делить её между нами, я размышляла в моем сердце, что это несправедливо, что ты должен находиться в болезни и голоде из-за [отсутствия] хлеба.
\vs Tjb 6:7
И тогда я решилась идти на торжище без робости. И когда хлеботорговец сказал мне: Дай мне деньги и ты получишь хлеб, я открыла ему наше бедственное положение.
\vs Tjb 6:8
Тогда я услышала, как он сказал: Если ты не имеешь никаких денег, вручи мне волосы твоей головы, и возьми три буханки хлеба для того, чтобы ты могла жить на них три дня.
\vs Tjb 6:9
И я уступила несправедливости и сказала ему: Встань и отрежь мои волосы! И он встал, и публично отрезал ножницами волосы моей головы на рыночной площади, в то время как толпа стояла рядом и удивлялась.
\vs Tjb 6:10
Кто тогда не удивлялся, говоря: Это ли Сифь, жена Иова, которая имела четырнадцать занавесов, чтобы закрывать её сокровенный чертог, и двери за дверями, так что тот был весьма польщен, кто был приведен подле него; и ныне, вот, она обменивает свои волосы на хлеб!
\vs Tjb 6:11
\ldots кто имел верблюдов, нагруженных товарами, и они отводились в отдаленные страны к нищим; и ныне она продает свои волосы за хлеб!
\vs Tjb 6:12
Смотрите на неё, кто имела семь неподвижных столовых наборов в её доме, за которыми всякий бедный человек и всякий странник ел; и ныне она продает свои волосы за хлеб!
\vs Tjb 6:13
Смотрите на неё, кто имела купальню, чтобы омывать свои ноги, сделанную из золота и серебра; и ныне она ходит по земле [босая], и [продает свои волосы за хлеб!]
\vs Tjb 6:14
Смотрите на неё, кто имела одеяние, сделанное из виссона, вышитое золотом; и ныне она обменивает свои волосы на хлеб!
\vs Tjb 6:15
Смотрите на неё, кто имела ложа из золота и серебра; и ныне она продает свои волосы за хлеб!
\vs Tjb 6:16
Затем вкратце, Иов, после стольких вещей, которые были сказаны мне, я ныне скажу тебе одним словом:
\vs Tjb 6:17
Так как слабость моего сердца сокрушает мои кости, встань и возьми эти буханки хлеба и насладись ими, и потом прокляни Господа и умри!
\vs Tjb 6:18
Ибо я тоже заменила бы оковы смерти за хлеб насущный моему телу.
\vs Tjb 6:19
Но я ответил ей: Вот, я был в течение этих семи лет пораженным проказой, и я терпел червей в моем теле, и я не был отягощен в моей душе всеми этими мучениями.
\vs Tjb 6:20
И как [за] слово, которое ты говоришь: Прокляни Бога и умри, вместе с тобою я выдержу зло, которое ты видишь? И позволь нам перенести разорение всего, что мы имеем.
\vs Tjb 6:21
Всё же ты хочешь, чтобы мы прокляли Бога и чтобы Он был заменен на великого Плутона.
\vs Tjb 6:22
Почему ты не помнишь тех великих благ, которыми мы обладали? Если эти блага исходят из уделов Господних, не должны ли мы также претерпевать [от Него и] зло и быть премудрыми во всем, пока Господь не помилует [нас] снова и окажет жалость к нам?
\vs Tjb 6:23
Ты не видишь Соблазнителя, ставшего позади тебя и спутавшего твои мысли, чтобы ты обманывала меня.
\vs Tjb 6:24
И он обратился к Сатане и сказал: Почему же ты не приходишь ко мне явно, не перестанешь скрывать себя? Ты~--- жалкий
\vs Tjb 6:25
лев, показывающий свою силу в удобной клетке, или птица, летающая в корзине. Ныне я говорю тебе: выходи и веди твою войну против меня.
\vs Tjb 6:26
Тогда он вышел из-за спины моей жены и поставил себя предо мною, вопия; и он сказал: Вот, Иов, я сдаюсь и уступаю дорогу тебе, который искусен, но~--- плоть, тогда как я~--- дух.
\vs Tjb 6:27
Ты поражен проказой, но я в великой печали.
\vs Tjb 6:28
Ибо я подобен соревнующемуся с борцом борцу, который в бою одной рукою низверг своего соперника и скрыл его в прахе и сокрушил каждый член его, тогда как тот, который лежит внизу, являя свою храбрость, издает звуки торжества, свидетельствующие о его великом превосходстве.
\vs Tjb 6:29
Так и ты, о Иов, унижен и поражен проказой и мукою, и все же ты вынес победу в соревновании со мною, и вот, я уступаю тебе.
\vs Tjb 6:30
Тогда он покинул меня смущенный.
\vs Tjb 6:31
Ныне, мои дети, вы делайте также, являя твердость сердца во всяком зле, которое происходит с вами, ибо твердость сердца~--- больше всех дел.

\vs Tjb 7:1
В это время цари услышали о том, что произошло со мною, и они встали и пришли ко мне, каждый от его земли, чтобы посетить меня и утешить меня.
\vs Tjb 7:2
И когда они подошли ко мне, они возопили громким голосом, и каждый разодрал свою одежду.
\vs Tjb 7:3
И после того, как они поклонились, касаясь земли своими головами, они сидели рядом со мною семь дней и семь ночей, и ни один не сказал ни слова.
\vs Tjb 7:4
Их было числом четверо: Елифаз, царь Фемана, и Вилдад, и Софар, и Елиуй.
\vs Tjb 7:5
И когда они заняли своё место, они беседовали о том, что произошло со мною.
\vs Tjb 7:6
В то время, когда они в первый раз приходили ко мне и я показывал им мои драгоценные камни, они были удивлены и сказали:
\vs Tjb 7:7
Если бы от нас, трех царей, всё наше имущество было бы соединено в одно, оно не сравнилось бы с драгоценными камнями царства Иовава. Ибо твое превосходство больше, чем всех людей Востока.
\vs Tjb 7:8
И поэтому, когда они ныне пришли в землю Уц, чтобы посетить меня, они спросили в городе: Где~--- Иовав, правитель этой всей земли?
\vs Tjb 7:9
И они сказали им обо мне: Он сидит на навозной куче вне города, ибо он не входит в город в течение семи лет.
\vs Tjb 7:10
И тогда они снова спросили о моём имуществе, и вот было показано им всё, что произошло со мною.
\vs Tjb 7:11
И когда они узнали это, они вышли из города с жителями, и мой согражданин показал меня им.
\vs Tjb 7:12
Но они возражали и говорили: Конечно, это~--- не Иовав.
\vs Tjb 7:13
И пока они колебались, вот Елифаз, царь Фемана, говорит: Давайте, подойдем ближе и посмотрим.
\vs Tjb 7:14
И когда они подошли ближе, я вспомнил их, и я сильно плакал, когда я узнал о цели их путешествия.
\vs Tjb 7:15
И я посыпал прах на мою голову, и пока отряхивал свою голову, я открыл им, кто я был.
\vs Tjb 7:16
И когда они увидели меня, трясущего своей головою, они поверглись ниц до земли, все охваченные волнением.
\vs Tjb 7:17
И пока толпа стояла вокруг, я видел этих трех царей лежащими на земле в течение трех часов подобно мертвым.
\vs Tjb 7:18
Тогда они встали и сказали друг другу: Мы не можем поверить, что это~--- Иовав.
\vs Tjb 7:19
И, наконец, после того, как они на седьмой день узнали всё обо мне и искали [и не нашли] мои стада и другое имущество, они сказали:
\vs Tjb 7:20
Разве мы не знаем, сколько товаров посылал он городам и селениям, повсюду подавая нищим, кроме всего, что было отдано им внутри его собственного дома? Как же мог он впасть в таковое состояние погибели и горя!
\vs Tjb 7:21
И после семи дней Елиуй сказал царям: Давайте подойдем ближе и рассмотрим его тщательно, истинно ли он Иовав или нет?
\vs Tjb 7:22
И они, будучи на расстоянии стадии от его зловонного тела, встали и шагнули ближе, неся благовония в их руках, а их воины пошли с ними и бросали ароматные шарики ладана к ним так, чтобы они могли приблизиться ко мне.
\vs Tjb 7:23
И после того, как они так прошли три часа, покрывая путь ароматом, они почти достигли.
\vs Tjb 7:24
И Елифаз начал и сказал: Ты ли, воистину, Иов, соцарствующий нам? Ты ли тот, кто имел великую славу?
\vs Tjb 7:25
Ты ли тот, кто когда-то сиял подобно дневному солнцу на всю землю? Ты ли тот, кто когда-то походил на луну и звезды, сияющие всю ночь?
\vs Tjb 7:26
И я ответил ему и сказал: Это я. И затем все плакали и стенали, и они воспели царскую плачевную песнь, [и] всё их войско соединилось с ними в хоре.
\vs Tjb 7:27
И опять Елифаз сказал мне: Ты ли тот, кто приказал раздать семь тысяч овец для одежды нищим? Поблекла, значит, преходящая слава твоего престола!
\vs Tjb 7:28
Ты ли тот, кто повелел трем тысячам волов пахать поле для бедных? Поблекла, значит, твоя преходящая слава!
\vs Tjb 7:29
Ты ли тот, кто имел золотые ложа, и ныне ты сидишь на навозной куче? [Поблекла, значит, твоя преходящая слава!]
\vs Tjb 7:30
Ты ли тот, кто имел шестьдесят столовых набора для нищих? Ты ли тот, кто имел кадило для прекрасных благовоний, отделанное драгоценными камнями, и ныне ты в зловонии? Поблекла, значит, твоя преходящая слава!
\vs Tjb 7:31
Ты ли тот, кто имел золотой набор подсвечников на серебряных подставках; и ныне должен ты тосковать из-за естественного отражения луны? [Поблекла, значит, твоя преходящая слава!]
\vs Tjb 7:32
Ты ли тот самый, кто делал притирание из смеси ладана, и ныне ты в мерзости! [Поблекла, значит, твоя преходящая слава!]
\vs Tjb 7:33
Ты ли тот, кто высмеивал неправедных делателей и презирал грешников, и ныне ты стал посмешищем у всех! [Поблекла, значит, твоя преходящая слава!]
\vs Tjb 7:34
И пока Eлифаз много времени вопиял и стенал, а все остальные соединились с ним, так что смятение было весьма великим, я сказал им:
\vs Tjb 7:35
Умолкните и я покажу вам мой престол и славу его великолепия: моя слава будет вечной.
\vs Tjb 7:36
Весь мир погибнет и его слава исчезнет, и все те, кто прилепляются к нему, будут в преисподней, но мой престол пребывает в вышнем мире, и его слава и великолепие будут одесную от Искупителя [моего] на небесах.
\vs Tjb 7:37
Мой престол существует в обществе святых и их славы в нетленном мире.
\vs Tjb 7:38
Ибо реки высохнут, и их надменность будет унижена до глубины бездны, но потоки моей земли, в которой мой престол воздвигнут, не высохнут, но останутся нерушимыми в силе.
\vs Tjb 7:39
Цари погибают и князья исчезают, и их слава и гордость~--- как отражение в зеркале; но моё царство продлится всегда и вечно, и его слава и красота пребывают в колеснице моего Отца.

\vs Tjb 8:1
Когда я говорил им так, Елифаз разгневался и сказал другим друзьям: Для этой ли цели мы пришли сюда с нашим войском утешать его? Вот, он поносит нас. Поэтому давайте мы возвратимся в наши страны.
\vs Tjb 8:2
Этот человек сидит здесь в червоточивом страдании среди невыносимого гниения, и все же он испытывает своё спасение: Погибнут царства и их правители, но моё царство, говорит он, продлится вовек.
\vs Tjb 8:3
Eлифаз затем встал в большом смятении, и, отвернувшись от них в великой ярости, сказал: Я ухожу отсюда. Воистину мы пришли утешить его, но он объявляет войну нам ввиду наших войск.
\vs Tjb 8:4
Но тогда Вилдад схватил его за руку и сказал: Не так должно говорить со страдающим человеком, и особенно с пораженным таковыми многими бедствиями.
\vs Tjb 8:5
Вот, мы, будучи в добром здравии, не осмеливались приблизиться к нему по причине зловония, кроме как с помощью множества ароматных благовоний. Но ты, Елифаз, забываешь обо всем этом.
\vs Tjb 8:6
Скажи мне прямо: позволишь ли нам быть великодушными и узнать, какова причина [его бедствия]? Не мог же он при воспоминании его прежних дней счастья стать безумным в его разуме?
\vs Tjb 8:7
Кто не был бы в совершенном недоумении, видя себя впадшим в таковое несчастье и беду? Но позвольте мне приступить к нему, чтобы я смог выяснить, в чем суть его дела.
\vs Tjb 8:8
И Вилдад встал и приблизился ко мне, говоря: Ты ли Иов? И [еще] он сказал: Находится ли твоё сердце в добром расположении?
\vs Tjb 8:9
И я сказал: Я не прилепляюсь к земным делам, тогда как земля со всем, что обитает на ней~--- непостоянна. Но моё сердце прилепляется к небу, ибо там, в небесах, нет печали.
\vs Tjb 8:10
Тогда Вилдад возразил и сказал: Мы знаем, что земля непостоянна, ибо она изменяется по сезонам. По временам она в состоянии мира, и по временам она в состоянии войны. Но о небе мы слышим, что оно совершенно неизменно.
\vs Tjb 8:11
Но истинно ли ты в покое? Поэтому позволь мне спрашивать и говорить; и когда ты ответишь мне на моё первое слово, я задам второй вопрос, и если вновь ты ответишь словами хорошо подобранными, станет очевидно, что твоё сердце не пребывает неуравновешенным.
\vs Tjb 8:12
И я сказал\fnote{я сказал}{он сказал(?)}: На что ты полагаешь твою надежду? И я ответил: На Бога живого.
\vs Tjb 8:13
И он сказал мне: Кто лишил тебя всего, чем ты обладал? И кто причинил тебе эти несчастья? И я сказал: Бог.
\vs Tjb 8:14
И он сказал: Если ты всё еще полагаешь свою надежду на Бога, то как Он может творить неправедный суд, нанося тебе эти несчастья и беды, и отняв у тебя все твои владения?
\vs Tjb 8:15
И так как Он забирает их, то ясно, что Он не дает тебе ничего. Царь станет ли безчестить своего воина, который хорошо служит ему как телохранитель?
\vs Tjb 8:16
[И я ответил на притчу]: Кто уразумеет глубины Господни и Его мудрость, чтобы быть способным обвинить Бога в несправедливости?
\vs Tjb 8:17
[И Вилдад сказал]: Ответь мне, о Иов, на это. Снова я скажу тебе: если ты~--- в здравом разсудке, вразуми меня, если ты имеешь мудрость:
\vs Tjb 8:18
почему мы видим восход солнца на Востоке, а закат на Западе? И опять, когда встаем утром, [почему] мы находим его восходящим на Востоке? Сообщи мне, что ты думаешь об этом?
\vs Tjb 8:19
Тогда сказал я: Зачем я буду выдавать величайшие тайны Божии и мои уста должны преткнуться при раскрытии дел, принадлежащих Владыке? Никогда!
\vs Tjb 8:20
Кто мы, что мы будем вникать в дела вышнего мира, тогда как мы~--- только из плоти; нет~--- земля и прах!
\vs Tjb 8:21
Чтобы ты знал, что моё сердце здраво, послушай, что я спрошу тебя:
\vs Tjb 8:22
Через утробу проходит пища, и воду ты пьешь через уста, и затем это течет через одно и то же горло, и когда оба опускаются, становясь выделением, они снова разделяются; кто производит это разделение?
\vs Tjb 8:23
И Вилдад сказал: Я не знаю. И я возразил и сказал ему: Если ты не понимаешь даже испражнений тела, как можешь ты понимать небесные круговращения?
\vs Tjb 8:24
Тогда Софар возразил и сказал: Мы не спрашиваем о своих делах, но мы желаем знать,~--- в здравом ли ты состоянии, и вот, мы видим, что твой разсудок не поколеблен.
\vs Tjb 8:25
Что ныне ты хочешь, чтобы мы сделали для тебя? Вот, мы пришли сюда и привели лекарей трех царей, и если ты желаешь, ты можешь вылечиться у них.
\vs Tjb 8:26
Но я ответил и сказал: Моё лекарство и моё возстановление происходят от Бога, Творца врачей.

\vs Tjb 9:1
И когда я говорил им так, вот, туда прибежала моя жена Сифь, одетая в лохмотья, от служения тому хозяину, которым она была нанята как рабыня, хотя ей запретили покидать [его], чтобы цари, видя её, не могли взять её как пленницу.
\vs Tjb 9:2
И когда она пришла, она простерлась обезсиленная у их ног, рыдая и говоря: Вспомнили, Елифаз и вы, остальные друзья, сколь я одинока с вами, и как я изменилась, как ныне я одета, чтобы встретить вас!
\vs Tjb 9:3
Тогда цари упали навзничь в великом плаче и, будучи в крайнем недоумении, они хранили молчание. Но Елифаз взял свою пурпурную мантию и бросил её ей, чтобы оделась в это.
\vs Tjb 9:4
Но она попросила его, говоря: Я прошу как милость у вас, моих господ, чтобы вы приказали вашим воинам копать среди развалин нашего дома, который упал на моих детей, так чтобы их кости могли быть принесены целыми к могилам.
\vs Tjb 9:5
Это первое, в чем мы имеем нужду в нашем несчастье, обезсилев совсем, и таким образом мы сможем по крайней мере увидеть их кости.
\vs Tjb 9:6
Ибо я имела безотчетное материнское чувство как у диких зверей, что мои десять детей должны погибнуть в один день; и ни одному из них я не могла бы дать достойное погребение?
\vs Tjb 9:7
И цари дали повеление, чтобы руины моего дома были выкопаны. Но я запретил это, предостерегая:
\vs Tjb 9:8
Не ходите в напрасной скорби; ибо мои дети не будут найдены, потому что они соблюдаются их Творцом и Владыкой.
\vs Tjb 9:9
И цари ответили и сказали: Кто станет сему противоречить? Он [вышел] из своего ума и бредит.
\vs Tjb 9:10
Ибо в то время как мы хотим принести кости его детей обратно, он запрещает нам делать [это], говоря так: Они были взяты и помещены на хранение их Творцом. Поэтому докажи нам [твою] истину.
\vs Tjb 9:11
Но я сказал им: Поднимите меня, чтобы я мог встать; и они подняли меня, поддерживая мои руки с обоих сторон.
\vs Tjb 9:12
И я стал прямо и объявил сначала похвалу Богу, а после молитвы я сказал им: Посмотрите глазами вашими на Восток.
\vs Tjb 9:13
И они посмотрели и увидели моих детей с венцами подле Царя славы, Владыки небес.
\vs Tjb 9:14
И когда моя жена Сифь увидела это, она пала на землю и простерла[сь] пред Богом, говоря: Теперь я знаю, что моя память останется у Господа.
\vs Tjb 9:15
И после того, как она сказала [это] и настал вечер, она пошла в город, обратно к хозяину, которому она служила как рабыня, и легла в воловьих яслях и умерла там от истощения.
\vs Tjb 9:16
И когда её жестокий хозяин искал её и не находил её, он пришел к загону своего стада, и там он увидел её простершейся в яслях мертвой, в то время как все животные вокруг плакали о ней.
\vs Tjb 9:17
И все, кто видели её, плакали и стенали, и вопль распространился повсюду во всем городе.
\vs Tjb 9:18
И люди унесли её и обернули её и похоронили её в доме, который упал на её детей.
\vs Tjb 9:19
И городские нищие сотворили великий плачь по ней и говорили: Вот, это Сифь, подобной кому в благородстве и в славе не найдется среди жен. Увы,~--- она не обрела достойной могилы!
\vs Tjb 9:20
Погребальную песнь по ней вы найдете в записи.

\vs Tjb 10:1
Но Eлифаз и те, что были с ним, были удивлены этим вещам, и они сидели со мною и отвечали мне, говоря в хвастливых словах обо мне в течение двадцати семи дней.
\vs Tjb 10:2
Они повторяли это снова и снова: что я страдал так по заслугам, потому что совершил много грехов, и что, вот, надежды не осталось мне; но я опровергал этих мужей в пылу спора.
\vs Tjb 10:3
И они встали в раздражении, готовые разстаться в гневном духе. Но Елиуй заклинал их остаться еще немного до тех пор, пока он не покажет им, как это было.
\vs Tjb 10:4
Ибо,~--- сказал он,~--- так много дней вы проводите, позволяя Иову хвалиться, что он праведен. Но я больше не буду терпеть этого.
\vs Tjb 10:5
Ибо изначально я продолжаю плакать по нему, помня его прежнее счастье. Но теперь он говорит хвастливо, и в заносчивой гордыне он говорит, что он имеет свой престол на небесах.
\vs Tjb 10:6
Поэтому, послушайте меня, и я поведаю вам, какова причина [такой] его судьбы.
\vs Tjb 10:7
Тогда вдохновляемый духом Сатаны Елиуй сказал жестокие слова, которые записаны в записях, оставленных Елиуем.
\vs Tjb 10:8
И когда он закончил, Бог явился мне в буре и мраке, и говорил, осуждая Елиуя и показывая мне, что тот, кто говорил, был не человек, а бешеное животное.
\vs Tjb 10:9
И когда Бог закончил говорить со мною, Господь сказал Eлифазу: Ты и твои друзья согрешили в том, что вы не говорили истины о Моем рабе Иове.
\vs Tjb 10:10
Поэтому поднимитесь и побудите его принести искупительную жертву за вас, чтобы ваши грехи могли быть прощены; ибо если бы не он, Я уничтожил бы вас.
\vs Tjb 10:11
И так они принесли мне всё, что необходимо для жертвы, и я взял это и принес за них искупительную жертву, и Господь принял её благосклонно и простил им их неправду.
\vs Tjb 10:12
После того, как Елифаз, Вилдад и Софар увидели, что Бог милостиво простил их грех через Его раба Иова, но что Он не соизволил простить Елиуя, тогда Eлифаз начал петь гимн, в то время как остальные подпевали, [и] их воины также присоединились при водруженном алтаре.
\vs Tjb 10:13
И Eлифаз говорил так: Отпущен грех и наша несправедливость омыта;
\vs Tjb 10:14
но Елиуй, оный Велиал, не будет иметь памяти среди живущих; его светило гаснет и теряет свой свет.
\vs Tjb 10:15
Слава его светильника явится ему, ибо он~--- сын тьмы, а не света.
\vs Tjb 10:16
Привратники обиталища тьмы дадут ему их славу и красоту в удел; его царство исчезло, его престол разсыпался, и честь его стати~--- в Шеоле.
\vs Tjb 10:17
Ибо он возлюбил лесть змеи и кожу дракона, его желчь и его яд~--- аспида.
\vs Tjb 10:18
Ибо он не стремился к Господу, ни боялся он Его, но он ненавидел тех, кого Он избрал.
\vs Tjb 10:19
Оттого Бог забыл его, и святые оставили его, его гнев и раздражение будут ему запустением, и он не обретет ни милосердия в его сердце, ни мира, ибо он имел яд аспида на его языке.
\vs Tjb 10:20
Праведен Господь, и Его суды~--- истинны. У Него нет лицеприятия, ибо Он судит всех одинаково.
\vs Tjb 10:21
Вот, Господь грядет! Вот, святые приготовились: венцы и награды победителей предшествуют им!
\vs Tjb 10:22
Да возрадуются святые, и да возликуют сердца их в веселии; ибо они получат славу, которая соблюдается для них.
\vs Tjb 10:23
Хор: Наши грехи прощены, наша несправедливость очищена, но Елиую нет памяти среди живущих.
\vs Tjb 10:24
После того, как Елифаз окончил гимн, мы встали и возвратились в город, каждый к дому, где он жил.
\vs Tjb 10:25
И народ сотворил пир для меня в благодарность и восхищение Богу, и все мои друзья возвратились ко мне.
\vs Tjb 10:26
И все те, кто видели меня в моем прежнем счастье, спросили меня, говоря: Что это за три вещи здесь среди нас? \ldots

\vs Tjb 11:1
Но я, желая взяться снова за мой труд благотворительности для бедных, просил их, говоря:
\vs Tjb 11:2
Дайте мне каждый агнца для одежды нищим в их наготе, и четыре драхмы серебра или золота.
\vs Tjb 11:3
Тогда Господь благословил всё, что было отложено мне, и после немногих дней я снова стал богат имением, в стадах и всех делах, которые я потерял, и я вновь получил всё вдвойне.
\vs Tjb 11:4
Потом я также взял в жену вашу мать и стал отцом вам десятерым вместо десяти детей, которые умерли.
\vs Tjb 11:5
И ныне, дети мои, позвольте мне предупредить вас: Вот, я умираю [и] вы получите моё жилище,
\vs Tjb 11:6
только не оставляйте Господа. Будьте милосердны к нищим; не презирайте немощных; не берите себе жен из иноплеменников.
\vs Tjb 11:7
Вот, дети мои, я разделю среди вас, чем я обладаю, так чтобы каждый мог иметь власть над его собственностью и полную силу делать благое с его долей.
\vs Tjb 11:8
И после того, как он сказал так, он принес всё своё стяжание и разделил его между его семью сыновьями, но он не дал ничего из своего имущества его дочерям.
\vs Tjb 11:9
Тогда они сказали своему отцу: Наш господин и отец! Не являемся ли мы тоже твоими детьми? Почему тогда ты не даешь также и нам часть твоего имения?
\vs Tjb 11:10
Тогда сказал Иов своим дочерям: Не гневайтесь, мои дочери. Я не забыл вас. Вот, я приготовил для вас имение, лучшее чем то, которое взяли ваши братья.
\vs Tjb 11:11
И он призвал свою дочь, имя которой Йемима, и сказал ей: Возьми это витое кольцо, используемое как ключ, и иди к сокровищнице, и принеси мне золотой ковчежец, чтобы я дал вам ваше имение.
\vs Tjb 11:12
И она пошла и принесла его ему, и он открыл его и взял трехслойные опоясания, вид которых человек не может изъяснить.
\vs Tjb 11:13
Ибо они были не земной работы, но небесные искры света сверкали через них подобно лучам солнца.
\vs Tjb 11:14
И он дал по одному поясу каждой из его дочерей и сказал: Оденьте их как опоясания ваши, чтобы все дни жизни вашей они могли окутывать вас и наделять каждую из вас добродетелью.
\vs Tjb 11:15
И другая дочь, имя которой было Касия, сказала: Это достояние, о котором ты говоришь, разве оно лучше, чем таковое же у наших братьев? Что теперь? Сможем ли мы жить на это?
\vs Tjb 11:16
И их отец сказал им: Не только здесь вам хватит на жизнь, но они приведут вас в лучший мир жительства~--- на небеса.
\vs Tjb 11:17
Или вы не знаете, мои дети, значение этих вещей здесь? Тогда послушайте! Когда Господь посчитал меня достойным иметь сострадание ко мне и удалить от моего тела проказу и червей, Он призвал меня и вручил мне эти три пояса.
\vs Tjb 11:18
И Он сказал мне: Встань и препояшь чресла твои, как подобает мужу: Я взыщу тебя и провозглашу тебя Моим.
\vs Tjb 11:19
И я взял их и обвязал их вокруг моих чресл, и тотчас черви оставили моё тело, также и проказа, и всё мое тело обрело новую силу от Господа. И так я пошел, как если бы я никогда не страдал,
\vs Tjb 11:20
но также и в моем сердце я забыл муки. Тогда говорил Господь со мною в Его великом могуществе и показал мне всё, что было и будет.
\vs Tjb 11:21
Теперь, когда вы, дети мои, под охраной их,~--- не будете иметь замышляющего против вас врага, ни [злых] намерений в вашем разуме, потому что этот филактерий от Господа.
\vs Tjb 11:22
Так встаньте же и опояшьте их вокруг себя, прежде чем я умру, дабы вы могли увидеть ангелов, грядущих на моё разлучение, так чтобы вы могли видеть с удивлением силы Божии.
\vs Tjb 11:23
Тогда встала та, чье имя было Йемима, и опоясалась; и тотчас она отлучилась от своего тела, как сказал её отец, и она обрела иное сердце, как будто она никогда не заботилась о земных делах.
\vs Tjb 11:24
И она пела ангельские песни голосом ангелов, и она воспевала ангельскую похвалу Богу в танце.
\vs Tjb 11:25
Тогда другая дочь, по имени Касия, надела пояс, и её сердце преобразилось так, что она больше не желала мирских дел.
\vs Tjb 11:26
И её уста претворились в речь небесных Начал, и она пела благодарственные славословия творения вышней обители, и если кто-либо желал познать творение небес, он мог найти постижение в гимнах Касии.
\vs Tjb 11:27
Тогда другая дочь, по имени Керенгаппух, опоясалась и её уста заговорили на языке том высоком; ибо её сердце преобразилось, восхищаясь превыше мирских дел.
\vs Tjb 11:28
Она говорила речами Керубов, воспевающих похвалу Владыке вселенских сил и превознося их славу.
\vs Tjb 11:29
И тот, кто желает следовать остаткам славы отца, найдет их записанными в молитвах Керенгаппух.

\vs Tjb 12:1
После того, как сии три окончили петь гимны, я, Нерос, брат Иова, сел рядом с ним, когда он возлег.
\vs Tjb 12:2
И я слышал изумительные дела о трех дочерях моего брата: одно всегда сопутствовало другому среди благоговейного безмолвия.
\vs Tjb 12:3
И я написал эту книгу, содержащую гимны, помимо гимнов и знамений [святого] слова, ибо они были великими делами Бога.
\vs Tjb 12:4
И Иов возлег от болезни на его ложе, однако без боли и страдания, потому что его боль не одолела силу разума его по причине чудесного действия пояса, который он обернул вокруг себя.
\vs Tjb 12:5
Но по прошествии трех дней Иов увидел, что святые ангелы пришли за его душой, и тотчас он встал и взял лиру и дал её своей дочери Йемиме.
\vs Tjb 12:6
И Касии он дал касию, а Керенгаппух он дал тимпан, чтобы они могли благословлять святых ангелов, которые пришли за его душою.
\vs Tjb 12:7
И они взяли их, и пели, и играли на арфе и восхваляли и прославляли Бога в святой речи.
\vs Tjb 12:8
И после этого Он пришел~--- Тот, Который возседает на большой колеснице, и облобызал Иова, в то время как его три дочери смотрели, но другие не видели этого.
\vs Tjb 12:9
И Он взял душу Иова, и Он вознесся ввысь, взяв её рукою и перенеся её на колесницу, и Он пошел на Восток.
\vs Tjb 12:10
Его тело, однако, было принесено к могиле, в то время как три дочери шествовали впереди, надев свои пояса и воспевая гимны в похвалу Богу.
\vs Tjb 12:11
Тогда провели Нерос, его брат, и его семь сыновей с остальным народом и нищими, вдовами и немощными, великую скорбь по нему, говоря:
\vs Tjb 12:12
Горе нам, ибо сегодня была взята от нас сила немощных, свет слепых, отец сирот,
\vs Tjb 12:13
приют странников; уведен руководитель заблудших, покров нагих, защита вдов. Кто не будет сетовать по мужу Божию!
\vs Tjb 12:14
И поскольку они скорбели таким и таковым образом, они не хотели перенести его, чтобы положить в могилу.
\vs Tjb 12:15
После трех дней, однако, он был, наконец, положен в могилу, как бы в приятном сне, и он наследовал доброе имя, которое станет прославляться повсюду всеми поколениями мира.
\vs Tjb 12:16
Он оставил семь сыновей и трех дочерей, и вот не нашлось дочерей на земле, столь же прекрасных как дочери Иова.
\vs Tjb 12:17
Имя Иова было прежде Иовав, и он был назван Иовом у Господа.
\vs Tjb 12:18
Он жил прежде его бедствия восемьдесят пять лет, а после бедствия он взял двойную долю всего; следовательно его годы также удвоились, то есть~--- сто семьдесят лет. Таким образом он жил всего двести пятьдесят пять лет.
\vs Tjb 12:19
И он увидел сыновей его сыновей до четвертого поколения.
\vs Tjb 12:20
Так написано: он опять возстанет с теми, кого Господь пробудит.
\vs Tjb 12:21
Нашему Господу слава. Аминь.

\bibbookdescr{Ahh}{
  inline={\LARGE Книга\\\Huge Ахиахара премудрого},
  toc={Книга Ахиахара},
  bookmark={Книга Ахиахара},
  header={Книга Ахиахара премудрого},
  abbr={Ахх}
}
\vs Ahh 1:1
Сказал Ахиахар: Когда жил я во дни Сеннахериба, царя Ниневийского, и когда был я, Ахиахар, хранителем сокровищ его и писцом, и когда был я молод,
\vs Ahh 1:2
прорицатели, волхвы и мудрецы сказали мне: Не будет у тебя дитяти.
\vs Ahh 1:3
И стяжал я богатство великое, и блага имел я в избытке, и взял себе шестьдесят жен,
\vs Ahh 1:4
и построил им шестьдесят дворцов, пространных, чудных и удивительных, и дома многие;
\vs Ahh 1:5
и достиг я шестидесяти лет, и не родилось дитя у меня.
\vs Ahh 1:6
Тогда я, Ахиахар, стал приносить богам жертвы и приношения, возжигал пред ними курения и ароматы
\vs Ahh 1:7
и говорил к ним: о, боги, дайте мне сына, в котором будет благоволение мое до того дня, когда я умру, и он наследует мне, и закроет очи мои, и похоронит меня.
\vs Ahh 1:8
И от дня смерти моей до смерти его, если будет он брать на всякий день от золота моего вволю и расточать его непрестанно, богатство мое не кончится.
\vs Ahh 1:9
Идолы не отвечали ему, и он оставил их и преисполнился мукою и тоскою великою.
\vs Ahh 1:10
И изменил он речь свою, и помолился Богу, и уверовал, и призвал Его в горении сердца своего,
\vs Ahh 1:11
и сказал: Боже небес и земли, Создатель всех тварей, я молю Тебя даровать мне сына, в котором будет благоволение мое, который утешит меня в час мой смертный, и закроет очи мои, и предаст меня погребению.
\vs Ahh 1:12
И пришел голос, и сказал ему:
\vs Ahh 1:13
За то, что ты надеялся на богов, и возложил на них упование твое, и приносил им дары, ты умрешь, ни сынов не имея, ни дочерей;
\vs Ahh 1:14
однако же говорю тебе: вот, у тебя есть Надав, сын сестры твоей; возьми его, научи его всей науке твоей, и он примет наследие твое.

\vs Ahh 2:1
И взял я Надава, сына сестры моей, и пестовал его, и взращивал его, и приставил к нему восемь кормилиц, чтобы питать его.
\vs Ahh 2:2
Я давал ему елея и меда, облачал его в пурпур и багрянец, покоил его на ложах мягких и на коврах.
\vs Ahh 2:3
И преуспевал Надав, сын сестры моей, и возрастал, подобно благородному кедру.
\vs Ahh 2:4
И учил я его письму, и мудрости, и философии.
\vs Ahh 2:5
Когда же вернулся царь Ассур-Аддин от празднеств своих и от странствий своих, он однажды призвал меня, Ахиахара, писца своего и хилиарха своего,
\vs Ahh 2:6
и сказал мне: о, друг мой достославный, дорогой, почитаемый, мудрый и разумный, хранитель печати моей и поверенный тайн моих! ты состарился и одряхлел, и смерть твоя приблизилась; скажи, кто будет мне служить после смерти твоей и погребения твоего?
\vs Ahh 2:7
И сказал я ему: о, владыка мой и царь, вечно живи в роды родов! у меня есть сын сестры моей, который мне как сын.
\vs Ahh 2:8
Вот, я наставил его во всей мудрости моей, и он мудр и рассудителен.
\vs Ahh 2:9
И повелел мне владыка мой: ступай, приведи его, чтобы мне видеть его, и если мне будет угодно, он будет служить мне и ходить предо мною.
\vs Ahh 2:10
Что до тебя, продолжай путь твой; он упокоит тебя от трудов твоих и окружит старость твою почетом и славою.
\vs Ahh 2:11
И я, Ахиахар, взял Надава, сына сестры моей, и представил его пред лице царя Ассур-Аддина, и отдал его в руку цареву;
\vs Ahh 2:12
и когда увидел его царь, он имел в нем свое благоволение и возрадовался ему,
\vs Ahh 2:13
и сказал: да сохранит Господь сына твоего!
\vs Ahh 2:14
Как ты служил мне и отцу моему Сеннахерибу и как ты вел дела наши со тщанием, так будет делать и Надав, сын сестры твоей;
\vs Ahh 2:15
он послужит мне и устроит дела мои, а я воздам ему честь, и возвышу его ради тебя, и позабочусь о нем.
\vs Ahh 2:16
И склонился я пред царем, и сказал ему: владыка мой царь, вовеки живи!
\vs Ahh 2:17
Прошу тебя позаботиться о нем и помогать ему; пусть обитает он в доме твоем, как и я служил тебе и служил отцу твоему.
\vs Ahh 2:18
И подал царь ему руку, и поклялся держать его при себе в почете и чести.
\vs Ahh 2:19
И поднялся я, и сказал: да будет так, о царь!
\vs Ahh 2:20
И наставлял я сына моего Надава, и передавал ему премудрость мою, и обильно уделял ему поучение, пока он не стал писцом, как я.
\vs Ahh 2:21
Вот как наставлял я его, и вот как говорил я, Ахиахар Премудрый.

\vs Ahh 3:1
О, Надав, сын мой, послушай слова мои, последуй советам моим и помни о речах моих.
\vs Ahh 3:2
Ей, Надав, сын мой! Если будешь ты внимать словам моим, и замкнешь их в сердце твоем, и никому не откроешь их,
\vs Ahh 3:3
из страха, чтобы пещь огненная не попалила языка твоего, и чтобы ты не причинил муки телу твоему и урона разуму твоему, и чтобы не посрамиться тебе пред Богом и перед людьми.
\vs Ahh 3:4
О, сын мой, если услышишь слово, никому не открывай его и не говори ничего из того, что увидишь.
\vs Ahh 3:5
Сын мой, не развязывай узла сокровенного и не запечатывай узла развязанного.
\vs Ahh 3:6
Сын мой, направляй стопу свою и слово свое, слушай и не спеши давать ответ.
\vs Ahh 3:7
Сын мой, не желай красоты внешней, ибо красота проходит и минует, но добрая память и доброе имя пребывают вовеки.
\vs Ahh 3:8
Сын мой, не бери жену с речью сварливою, ибо в речах ее горечь, и в нити ее яд, и ты попадешь в западню ее.
\vs Ahh 3:9
Сын мой, если увидишь, что женщина украшена нарядами и умащена благовониями, но нрав ее дурной, сварливый и бесстыжий, пусть сердце твое не желает ее;
\vs Ahh 3:10
если отдашь ей все, что имеешь ты, найдешь, что это не обратится к славе твоей, но ты прогневишь Бога, и ярость Его постигнет тебя.
\vs Ahh 3:11
Сын мой, не спеши говорить и не влагай в ответы и речи твои похвальбы, словно миндальное дерево, пускающее листья свои и зелень свою прежде всех деревьев и дающее плоды свои после всех;
\vs Ahh 3:12
будь, как древо приятное, хвалимое, сладкое, полное утехи, как смоковница, которая склоняет ветви, зеленеет и пускает листья последней, но плод ее бывает вкушаем первым.
\vs Ahh 3:13
Сын мой, склони главу твою, устреми взор твой долу и приготовься быть внимателен.
\vs Ahh 3:14
Будь разумен, покорен, сдержан, невозмутим.
\vs Ahh 3:15
Не будь бесстыден и сварлив.
\vs Ahh 3:16
Не возвышай голоса твоего с похвальбою и буйством,
\vs Ahh 3:17
ибо если бы громкого голоса было довольно, чтобы воздвигнуть дом, осел строил бы по два дома в день;
\vs Ahh 3:18
и если бы плуг направлялся силою, верблюд направлял бы его лучше всех.
\vs Ahh 3:19
Сын мой, лучше таскать камни с мудрым, нежели пить вино с глупцом.
\vs Ahh 3:20
Сын мой, пролей вино твое и окропи им могилы праведных.
\vs Ahh 3:21
Сын мой, иди босыми ногами по терниям и по колючкам, чтобы проторить тропу к детям твоим и детям детей твоих.
\vs Ahh 3:22
Сын мой, когда дует ветер, а море еще не возмутилось, веди ладью твою и корабль твой к гавани, пока море не возмутилось, и не пришло в движение, и не умножило валов своих, и не потопило корабля.
\vs Ahh 3:23
Сын мой, не забывайся с глупцом и не имей общения с нецеломудренным.
\vs Ahh 3:24
Сын мой, не приближайся к женщине сварливой и говорящей заносчиво, не желай красоты женщины словоохотливой и нечистой,
\vs Ahh 3:25
ибо красота женщины есть позор ее, и блеском одежды своей и красотой внешней она пленит тебя и обманет тебя.
\vs Ahh 3:26
Сын мой, как нет пользы от колец в ушах дикого осла, так нет пользы от женщины с пышною осанкою, если она лукава в словах своих и делах своих, лишена мудрости, словоохотлива и многоречива.
\vs Ahh 3:27
Сын мой, если мудрый недужен, врач сможет уврачевать и вылечить его, но нет врачевания для недугов и ран неразумного.
\vs Ahh 3:28
Сын мой, прими того, кто ниже тебя и беднее тебя; если он не воздаст тебе, Бог воздаст тебе.
\vs Ahh 3:29
Сын мой, не уставай наказывать дитя твое; наказание дитяти как удобрение сада, как завязывание кошеля, как обуздание скотины и как затвор на воротах.
\vs Ahh 3:30
Сын мой, оторви сына твоего от зла, чтобы уготовать себе покой в старости твоей;
\vs Ahh 3:31
поучай его и наказывай его, пока он юн, понудь его слушать повелений твоих, чтобы немного после он не стал вопить и восставать на тебя,
\vs Ahh 3:32
чтобы он не навлек на тебя бесчестия пред товарищами твоими,
\vs Ahh 3:33
чтобы не пришлось тебе опустить голову в людных местах и на площадях,
\vs Ahh 3:34
чтобы не краснел ты по причине лукавства дел его и не был ты уничижен по причине порочного бесстыдства его.
\vs Ahh 3:35
Сын мой, не доводи детей твоих до крайности, чтобы они не прокляли тебя и Бог не прогневался на них,
\vs Ahh 3:36
ибо написано: кто злословит отца своего и матерь свою, смертию умрет; это грех, прогневляющий Бога;
\vs Ahh 3:37
и еще: кто чтит отца своего и матерь свою, будет долголетен и будут ему блага обильные.
\vs Ahh 3:38
Сын мой, не пускайся в путь без меча и не переставай помнить о Боге в сердце твоем,
\vs Ahh 3:39
ибо ты не знаешь, когда враги лютые встретятся тебе; будь готов на пути твоем, ибо враги твои многочисленны.
\vs Ahh 3:40
Сын мой, каково древо, изобилующее плодами, листами и ветвями, таков муж с женою доброю, и плоды их дети их и родственники.
\vs Ahh 3:41
У кого нет ни жены, ни детей, ни родственников, презрен и пренебрегаем от врагов своих, как древо, стоящее вдали от дороги, которое прохожие пинают ногами и едят от плодов его, и дикий зверь стряхивает листы его и ест их.
\vs Ahh 3:42
Сын мой, если есть у тебя слуги, не предпочитай одного и не отвергай другого, ибо ты не знаешь, какого выберешь в конце.
\vs Ahh 3:43
Сын мой, коза блуждающая и умножающая шаги свои станет добычею волка.
\vs Ahh 3:44
Сын мой, услади язык твой словами Божьими и усовершенствуй слова уст твоих;
\vs Ahh 3:45
говори к любому с добротою и изяществом, ибо пасть пса промышляет ему хлеб и гортань его навлекает на него удары и камни.
\vs Ahh 3:46
Сын мой, не давай ближнему твоему наступать на ногу твою, чтобы он не наступил на грудь твою.
\vs Ahh 3:47
Сын мой, если зовешь мудрого делать работу твою, не говори ему долгих поучений и вразумлений, ибо он сделает работу твою, как желает сердце твое;
\vs Ahh 3:48
но если зовешь неразумного, не говори с ним перед другим, но лучше ступай и не зови его, ибо не сделает он работу по сердцу твоему, сколь бы долгие советы ни давал ты ему.
\vs Ahh 3:49
Сын мой, поспешно уходи со свадеб и с празднеств, не дожидаясь, чтобы главу твою умастили елеем и благовониями, дабы не навлечь на главу твою ударов и рубцов.
\vs Ahh 3:50
Сын мой, того, чья рука полна, именуют мудрым и досточтимым, а того, чья рука пуста, именуют злым, убогим, бедным и неимущим, и никто не воздает ему чести.
\vs Ahh 3:51
Сын мой, я вкушал полынь и пробовал мирру, но не нашел ничего горше бедности и нужды.
\vs Ahh 3:52
Сын мой, я поднимал железо и свинец, но не нашел ничего тяжелее хулы и клеветы.
\vs Ahh 3:53
Сын мой, я ворочал камни, но не нашел ничего столь тяжкого, как зять, живущий в доме тестя своего.
\vs Ahh 3:54
Сын мой, если нога твоя оступится и ты упадешь, это лучше, чем если ты оступишься языком твоим.
\vs Ahh 3:55
Сын мой, друг близкий лучше, чем брат далекий,
\vs Ahh 3:56
и доброе имя лучше, чем богатства мира, ибо богатства прейдут и развеются, но доброе имя пребудет вечно.
\vs Ahh 3:57
Сын мой, красота гибнет, разрушается и пропадает, и мир преходит, и все престает и прекращается, но доброе имя не преходит, не престает и не разрушается.
\vs Ahh 3:58
Сын мой, шум плача и рыдания лучше, нежели шум веселия и празднества,
\vs Ahh 3:59
ибо внимать шуму плача учит человека постигнуть грех свой и дать за него удовлетворение.
\vs Ahh 3:60
Сын мой, не восставай в суждении твоем на мужей славных и превосходных величием и властью, ибо от шуток и слов глумливых происходят гнев и раздор.
\vs Ahh 3:61
Слово гневное пробуждает и возбуждает ярость, и от ярости этой происходит раздор, а после раздора приходит и убийство.
\vs Ahh 3:62
\ldots если ты окажешься в месте таком, тебя могут убить или тебя могут позвать в свидетели;
\vs Ahh 3:63
и когда от тебя будут требовать и вымогать свидетельство твое, ты претерпишь страдание и от стыда или страха дашь свидетельство ложное и будешь посрамлен.
\vs Ahh 3:64
И я повелеваю тебе: спеши бежать из того места, где спорят, и душа твоя будет умиротворена.
\vs Ahh 3:65
Сын мой, стяжи сердце чистое и неоскверненное, разумение ясное и непомраченное,
\vs Ahh 3:66
доставь себе дух смиренный и найди себе стезю прямую, и не будет на свете человека достойнее тебя, и жизнь твоя будет блаженна.
\vs Ahh 3:67
Сын мой, не входи в сад судей, страшись судилища и не бери в жены дочь судьи.
\vs Ahh 3:68
Сын мой, защищай друга твоего перед начальником словами добрыми и исторгай немощь его из пасти льва.
\vs Ahh 3:69
Сын мой, не радуйся смерти врага твоего.
\vs Ahh 3:70
Сын мой, когда увидишь, что вошел человек старше тебя, встань перед ним.
\vs Ahh 3:71
Сын мой, око человеческое подобно источнику: оно не насытится, пока не наполнится прахом.
\vs Ahh 3:72
Сын мой, если хочешь быть мудр, воспрети устам твоим ложь и руке твоей хищение, и будешь мудр.
\vs Ahh 3:73
Сын мой, не входи в устройство брака женщины, ибо, если она будет недовольна, она проклянет тебя, и, если она будет счастлива, она не вспомнит о тебе.
\vs Ahh 3:74
Сын мой, если ты украл, извести власть имущего и предложи ему долю, и тогда ты можешь получить прощение, в противном же случае приключится тебе зло.
\vs Ahh 3:75
Сын мой, пусть лучше мудрый побьет тебя многими ударами жезла, чем неразумный помажет тебя елеем благовонным.
\vs Ahh 3:76
Сын мой, пусть нога твоя не бежит к другу твоему, чтобы он не пресытился тобою и не возненавидел тебя.
\vs Ahh 3:77
Сын мой, не возлагай кольца золотого на руку твою, если ты небогат, чтобы неразумные не глумились над тобою.

\vs Ahh 4:1
И когда я, Ахиахар, преподал мудрость эту Надаву, сыну сестры моей, я полагал, что он сохранит ее в сердце своем, пребывая при дворе, и не ведал того, что он не слушал слов моих, но бросал их словно на ветер.
\vs Ahh 4:2
Он усвоил обыкновение говорить: Ахиахар, отец мой, стар и утратил дух свой.
\vs Ahh 4:3
И Надав, сын мой, присвоил стада мои, и расточил добро мое, и не пощадил лучших слуг моих, и бил их пред лицем моим, и не пожалел скотов моих и мулов моих, и умерщвлял их.
\vs Ahh 4:4
Когда увидел я, что творил он, я сказал ему:
\vs Ahh 4:5
Сын мой, не тронь добра моего, ибо сказано в изречениях: чего рука твоя не стяжала око твое не видело.
\vs Ahh 4:6
И я известил обо всем этом владыку моего царя,
\vs Ahh 4:7
и повелел царь: пусть никто не дерзает приближаться к добру Ахиахара, писца; пока живет Ахиахар, да не приближается никто ни к достоянию его, ни к дому его.

\vs Ahh 5:1
Когда увидел Надав, что я взял брата его младшего и стал его воспитывать, это было ему неприятно; и позавидовал он, и возымел в уме своем помыслы злые по этой причине,
\vs Ahh 5:2
и сказал: Ахиахар, отец мой, стар, и мудрость его пропала, и слова его достойны презрения; ужели он отдаст добро свое брату моему, а меня изгонит из дома своего?
\vs Ahh 5:3
И когда я, Ахиахар, услышал слова Надавовы, я сказал:
\vs Ahh 5:4
Увы тебе, премудрость моя! Надав, сын мой, лишил тебя вкуса твоего и презрел мудрые слова мои.
\vs Ahh 5:5
Когда сказал я это, сын мой весьма раздражился и приготовил в сердце своем зло мне.
\vs Ahh 5:6
И пошел он ко двору царскому, чтобы сотворить зло, которое было в сердце его, как будто написал Ахиахар от лица своего письма лукавые, а он отправился ко двору объявить о них.
\vs Ahh 5:7
И вот письма от имени моего к царям, враждебным царю Сеннахерибу.
\vs Ahh 5:8
Одно было к царю Персидскому и Еламитскому, и он написал его так:
\vs Ahh 5:9
От Ахиахара, писца и хранителя печати царя Сеннахериба, мир тебе! Когда получишь ты это письмо, выступай немедля, и приходи в Ассирию, и возьмешь ты всю землю сию без войны и без боя.
\vs Ahh 5:10
Другое было от имени моего к фараону, царю Мицрейскому, и он составил его так:
\vs Ahh 5:11
Когда придет к тебе письмо это, выходи ко мне на долину южную двадцать пятого числа месяца Ава; и приведу тебя к Ниневии, и овладеешь ты царством без боя.
\vs Ahh 5:12
Он переписал письма эти по подобию руки моей и запечатал их печатью моей, а после подбросил их в один из покоев царских.

\vs Ahh 6:1
И написал он еще другое письмо от лица владыки моего царя:
\vs Ahh 6:2
От Ассур-Аддина Ахиахару, писцу моему и хранителю печати моей, мир!
\vs Ahh 6:3
Когда получишь ты это письмо, собери все воинство у горы и выступай к долине Нешрин двадцать пятого числа месяца Ава;
\vs Ahh 6:4
и когда увидишь ты, что я приближаюсь к тебе, построй воинства твои пред лицем моим, как бы ты приготовлялся к сражению,
\vs Ahh 6:5
ибо посланцы фараона, царя Мицрейского, придут со мною, и они увидят, каковы мои силы.
\vs Ahh 6:6
И сын мой Надав передал мне письмо через двух письмоносцев.
\vs Ahh 6:7
И взял сын мой Надав одно из писем, так, словно бы нашел его, и прочитал его пред царем.
\vs Ahh 6:8
И тогда царь уязвился весьма, и прогневался на Ахиахара, и говорил:
\vs Ahh 6:9
Боже! В чем погрешил я против Тебя и против Ахиахара, что он решился так поступить со мною?

\vs Ahh 7:1
И тогда ответил Надав и сказал царю:
\vs Ahh 7:2
Не печалься, о, владыка мой царь, но пойдем на долину Нешрин, как написано в этом письме; мы узнаем истину, и все будет так, как ты повелишь.
\vs Ahh 7:3
И повелел царь приготовляться выступать на долину, чтобы узнать, какова истина в этом деле,
\vs Ahh 7:4
и Надав, сын мой, сопровождал царя, и пришли они, и обрели меня с воинством, сопутствовавшим мне, в долине Нешрин.
\vs Ahh 7:5
И когда увидел я, что царь приближается ко мне, я построил воинство в боевой порядок пред лицем его, словно для сражения, доверясь письму, которое послал ко мне сын мой.
\vs Ahh 7:6
И сказал сын мой царю: ступай к себе со всяким спокойствием, о, владыка мой!
\vs Ahh 7:7
Я же приведу пред очи твои Ахиахара, отца моего.
\vs Ahh 7:8
И царь отошел в место свое.

\vs Ahh 8:1
И пришел ко мне Надав, сын мой, и взял слово, и сказал:
\vs Ahh 8:2
Владыка царь послал меня к тебе, чтобы сказать тебе: все, что ты сделал, хорошо, царь весьма хвалит тебя.
\vs Ahh 8:3
Ныне же отпусти воинства; пусть все расходятся к себе, ты же иди к царю один.
\vs Ahh 8:4
И пошел я тогда к царю, и когда увидел он меня, то сказал мне:
\vs Ahh 8:5
Ты пришел, Ахиахар, писец мой, отец и кормилец Ассура и Ниневии!
\vs Ahh 8:6
Я всегда почитал тебя и покоил тебя, ты же отпал от меня и стал один из врагов моих.
\vs Ahh 8:7
После он дал мне письмо, написанное от имени моего и запечатленное печатью моей.
\vs Ahh 8:8
И сказал мне царь: прочти письмо это.
\vs Ahh 8:9
И когда я прочел его, сотряслись члены мои, и язык мой перестал повиноваться мне; искал я слово мудрое и не находил его.
\vs Ahh 8:10
И взял слово Надав, сын мой, и сказал:
\vs Ahh 8:11
Убирайся с глаз царя, старик неразумный, и простри руки твои в узы и ноги твои в железа!
\vs Ahh 8:12
Тогда царь Ассур-Аддин отвратил лице свое от меня и сказал Навусемаку, палачу, который был со мною в дружбе:
\vs Ahh 8:13
Пойди, убей Ахиахара и удали голову его на сто локтей от тела его.
\vs Ahh 8:14
Тогда пал я лицем на землю, поклонился царю и сказал ему:
\vs Ahh 8:15
Владыка царь, вовеки живи! Ты хочешь умертвить меня; да будет по воле твоей.
\vs Ahh 8:16
Я знаю, что не погрешил пред тобою; но повели, владыка царь, чтобы меня умертвили перед дверью дома моего и чтобы тело мое отдали для погребения.
\vs Ahh 8:17
И повелел царь, чтобы было так.

\vs Ahh 9:1
И я, Ахиахар, послал сказать жене моей:
\vs Ahh 9:2
Приходи ко мне и приведи с собою тысячу девиц, одетых в виссон, в пурпур и в шафран, которые будут плясать предо мною и оплакивать меня до самой смерти моей.
\vs Ahh 9:3
И приготовь еды палачу Навусемаку, другу моему, и Парфянам, которые придут с ним;
\vs Ahh 9:4
выйди к ним навстречу и пригласи их войти ко мне, дабы и я смог войти в дом мой, как гость и чужак.
\vs Ahh 9:5
И когда жена моя приняла вестников, исполнилась она премудрости великой и выполнила все, что я велел ей.
\vs Ahh 9:6
И вышла она навстречу Навусемаку и Парфянам, и пригласила их войти в дом ее.
\vs Ahh 9:7
И принесла Эшфагни еды Навусемаку и Парфянам хлеба; и достала она им также вина, и разлила для них;
\vs Ahh 9:8
и служила им Эшфагни на пире их, пока они не опьянели и не уснули.
\vs Ahh 9:9
Когда опьянели Парфяне от вина, уснули они сном глубоким, и каждый из них уснул на месте своем.
\vs Ahh 9:10
И восхвалил я Господа небес и земли за все, что произошло, и сказал я:
\vs Ahh 9:11
Боже, Спаситель мира, ведающий все, что было, и все, что будет, призри на меня оком милостивым пред Навусемаком.

\vs Ahh 10:1
И когда я, Ахиахар, увидел все это, я заговорил и сказал Навусемаку:
\vs Ahh 10:2
Подними глаза твои к небу, Навусемак, и помысли о Боге; вспомни о хлебе и соли, которые мы съели с тобою, и не замышляй моей смерти.
\vs Ahh 10:3
Вспомни, что отец владыки моего царя также предал мне тебя, чтобы я умертвил тебя, а я не умертвил тебя, ибо знал, что ты не согрешил, и я оставил тебе жизнь до того дня, когда царь пожелал видеть тебя и дал мне дары многие. И ты спаси меня ныне.
\vs Ahh 10:4
Чтобы не распространилась молва и не сказали, что он не предан смерти, вот, у меня есть в темнице моей человек, заслуживший смерть; возьми одежды мои, надень на него и после повели Парфянам умертвить его.
\vs Ahh 10:5
Когда я сказал это, Навусемак, палач, друг мой, исполнился печали обо мне;
\vs Ahh 10:6
и взял он одежды мои, и облачил в них раба, который был в темнице, а после разбудил Парфян, которые поднялись под действием вина, и умертвили раба и отдалили голову его на сто локтей от туловища его, и отдали тело его для погребения.
\vs Ahh 10:7
И распространилась молва по Ассирии и по Ниневии, что Ахиахар умерщвлен.

\vs Ahh 11:1
И тогда Навусемак совместно с женою моею Эшфагни устроил мне в земле укром в три локтя ширины, и четыре локтя длины, и пять локтей высоты;
\vs Ahh 11:2
они дали мне есть и пить и послали сказать владыке моему царю, что Ахиахар предан смерти.
\vs Ahh 11:3
И сказал царь: страдание Ахиахарово пало на главу мою; писец мой и мудрец, защищавший пролом в стене града, я отправил тебя на гибель по слову отрока!
\vs Ahh 11:4
И призвал царь Надава, сына моего и сказал ему: ступай, оплакивай отца твоего!
\vs Ahh 11:5
Надав, сын мой, пошел в дом мой, и он не оплакивал меня и не творил память обо мне, но собрал женщин блудных и посадил их есть и пить среди песен и веселия.
\vs Ahh 11:6
Он убивал, и обнажал, и избивал слуг моих и служанок моих, он не постыдился даже женщины, воспитавшей его, но велел ей совершить с ним блуд и распутство.
\vs Ahh 11:7
Я же в недрах рва темного слышал вопль поваров моих, и пирожников моих, и хлебников моих, которые творили плач и стенание.
\vs Ahh 11:8
И тотчас обратил я молитву мою ко Всевидящему.
\vs Ahh 11:9
Прошли дни, и пришел Навусемак, и отворил затвор мой, и дал мне есть и пить,
\vs Ahh 11:10
и я сказал ему: поминай меня пред лицем Бога, и после всего, что ты увидел, скажи Ему:
\vs Ahh 11:11
Яхве, Праведный и Благий на небесах и на земле, доныне Ахиахар был ограждаем Тобою, и он приносил Тебе в жертву тельцов тучных; и вот, лежит он во рву мрачном, и свет не доходит к нему.
\vs Ahh 11:12
Услышь, Яхве, вопль раба Твоего и смилуйся над ним!

\vs Ahh 12:1
И когда узнал царь Мицрейский, что я, Ахиахар, умерщвлен, он пришел в радость великую и послал Ассур-Аддину письмо:
\vs Ahh 12:2
Царь Мицрейский Ассур-Аддину, царю Ассирийскому и Ниневийскому, мир!
\vs Ahh 12:3
Мне нужно построить крепость между небом и землею; пошли мне зодчего мудрого, которому я мог бы поручить все дело, чтобы я его вопрошал, а он мне отвечал.
\vs Ahh 12:4
Если человек, которого ты пошлешь ко мне, сделает все, что я скажу, я соберу и пошлю к тебе через него подать с Мицры за три года.
\vs Ahh 12:5
Если же ты не пошлешь мне человека, который смог бы сделать то, о чем я говорю, собери и пошли ко мне через моего посланника подать с Ассирии и Ниневии за три года.
\vs Ahh 12:6
Когда письмо это было прочитано пред лицем царя, он повелел собрать всех своих вельмож, и мудрецов, и волшебников, и книжников царства своего и сказал им:
\vs Ahh 12:7
Кто из вас пойдет в Мицру и даст ответ фараону?
\vs Ahh 12:8
И ответили вельможи царю, и сказали ему все:
\vs Ahh 12:9
Ты знаешь, владыка царь, что во дни твои и во дни отца твоего все вопросы такого рода разрешал Ахиахар, писец,
\vs Ahh 12:10
ныне же Надав, сын его, наследовавший ремесло писца и наученный им мудрости его, должен заняться делом этим.

\vs Ahh 13:1
И когда услышал Надав слова эти, возопил он пред лицем царя воплем великим и сказал царю:
\vs Ahh 13:2
И боги не могут сделать таких дел, как же смогут это люди?
\vs Ahh 13:3
При словах этих царь опечалился и удручился, и сошел с престола своего, и облачился во вретище, и сел на землю, и возрыдал, и говорил с плачем:
\vs Ahh 13:4
Увы тебе, Ахиахар, писец мой, что я велел погубить тебя по слову отрока, и не осталось никого, кто был бы тебе подобен и равен.
\vs Ahh 13:5
А ныне кто вернет тебя мне? Я заплатил бы за тебя, оценив тебя на вес золота.
\vs Ahh 13:6
И когда услышал Навусемак от царя таковые слова, он простерся ниц, и поклонился царю, и сказал:
\vs Ahh 13:7
Царь, вовеки живи! Презирающий слово владыки своего повинен смерти; повели же распять меня на древе, ибо ослушался я слова твоего;
\vs Ahh 13:8
ведь Ахиахар, которого повелел ты мне умертвить, жив доселе.
\vs Ahh 13:9
И ответил царь Набусемаку: говори, о Навусемак, ибо ты человек добрый и справедливый и не можешь совершить зла.
\vs Ahh 13:10
Если дело и вправду так, как ты сказал, и если ты мне представишь Ахиахара живым, я дам тебе дары великие: серебра мириаду талантов и пурпура сотню риз.
\vs Ahh 13:11
Когда Навусемак услышал, что царь говорит это, он начал говорить:
\vs Ahh 13:12
Я молю владыку моего царя сказать мне одно лишь, что он забывает за мною эту вину и не держит гнева на меня.
\vs Ahh 13:13
И царь поклялся ему в этом с радостию.

\vs Ahh 14:1
И взошел тогда Навусемак на колесницу, и примчался так быстро, как ветер могучий.
\vs Ahh 14:2
И отворил он мне, и я вышел на свет.
\vs Ahh 14:3
И не была посрамлена надежда моя на Бога.
\vs Ahh 14:4
И отвел меня Навусемак к царю, и повергся я на землю.
\vs Ahh 14:5
Волосы мои спадали на плечи мои, и борода моя доходила до груди моей, и тело мое было засыпано землею, и ногти мои отросли, как когти орла.
\vs Ahh 14:6
Когда царь увидел меня, он много плакал и сказал мне:
\vs Ahh 14:7
О, Ахиахар, я не погрешил против тебя, но это сын твой, воспитанный тобою, погрешил против тебя.
\vs Ahh 14:8
И отвечал я, и сказал царю: владыка мой, ныне вижу я лице твое, и скорбь моя отнята у меня.
\vs Ahh 14:10
И отвечал царь, и сказал мне: иди в дом твой, и обрежь власы твои, и омой тело твое в водах, и давай себе покой сорок дней, а после приходи пред очи мои.
\vs Ahh 14:11
И пошел я в дом мой, и делал все, как повелел мне владыка мой царь;
\vs Ahh 14:12
но лишь двадцать дней оставался я в доме моем, а когда возвратились ко мне силы мои, пошел я пред очи царя.
\vs Ahh 14:13
И показал мне царь письмо, пришедшее от царя Мицрейского.
\vs Ahh 14:14
И заговорил царь, и сказал: ты только посмотри, Ахиахар, на Мицрейцев! Что написали они мне, и какую подать наложили они на Ассур и на Ниневию!
\vs Ahh 14:15
И отвечал я ему, и сказал: владыка мой царь, вовеки живи!
\vs Ahh 14:16
Об этом деле не пекись и не печалуйся; я пойду в Мицру, и я дам ответ, и я представлю всем недругам твоим загадку и ее решение, и я принесу подать с Мицры за три года.
\vs Ahh 14:17
И при словах таковых царь возрадовался весьма и устроил день веселия, и оставила печаль лице его, и принес он в жертву тельцов и овнов, и дал мне дары великие.
\vs Ahh 14:18
И поставил он Навусемака надо всеми, и дал ему чин высокий.

\vs Ahh 15:1
И написал я письмо Эшфагни, жене моей:
\vs Ahh 15:2
О, супруга моя, когда письмо это придет к тебе, прикажи, чтобы ловчие изловили для меня двух орлов юных;
\vs Ahh 15:3
и повели; чтобы слуги мои принесли для меня нити льняной и сделали из нее для меня две веревки в палец толщиною и в тысячу локтей длиною; и скажи, чтобы кузнецы сковали для меня две клетки.
\vs Ahh 15:4
И отдай Навухаила и Тевшалома, слуг моих, семи кормилицам первородившим, чтобы те питали их млеком своим, пока они не вырастут;
\vs Ahh 15:5
и помести с ними орлов юных, чтобы они возрастали вместе, и давай орлам в корм по два овна на каждый день.
\vs Ahh 15:6
И пусть отроки выучатся говорить: принесите глины и кирпичей! Зодчие, гости царя, не имеют себе дела.
\vs Ahh 15:7
Жена моя была весьма смышлена и сделала все, что я велел ей;
\vs Ahh 15:8
и получил я приказ от царя отправляться в Мицру.
\vs Ahh 15:9
При вести этой ассирияне и ниневитяне обрадовались весьма и удалились к себе.
\vs Ahh 15:10
И ответил я царю: владыка мой царь, дозволь мне отправляться в Мицру.
\vs Ahh 15:11
И когда повелел он мне отправляться, я взял с собою отряд многочисленный и выступил в путь.
\vs Ahh 15:12
И когда прибыл я к вечернему отдыху, я прежде всего отпустил войско,
\vs Ahh 15:13
после взял из клеток орлов юных, привязал веревки к ногам их и велел отрокам моим влезть на веревки, а после отпустил орлов,
\vs Ahh 15:14
и поднялись они в воздух; а отроки кричали, как были научены:
\vs Ahh 15:15
Принесите кирпичей, глины и строительного раствора! это нужно для гостей и зодчих царя!
\vs Ahh 15:16
И после этого я вернул их к себе на землю.

\vs Ahh 16:1
И когда прибыл я в Мицру, слуги царя доложили обо мне, и царь повелел, чтобы я пришел к нему.
\vs Ahh 16:2
Я вошел к царю и приветствовал его, и после он спросил меня:
\vs Ahh 16:3
Каково твое имя? И ответил я:
\vs Ahh 16:4
Абикам, один из муравьев царя Ниневийского.
\vs Ahh 16:5
И когда фараон услыхал это, он был недоволен и сказал:
\vs Ahh 16:6
Ужели владыка твой настолько презирает меня, что он отрядил ко мне муравья, чтобы отвечать мне?
\vs Ahh 16:7
И ответил я, и сказал ему: владыка, пчела есть малейшая среди птиц и насекомых, и посмотри, какое дивное дело творит она.
\vs Ahh 16:8
С почетом допускают ее к столу государей великих;
\vs Ahh 16:9
а пред Сеннахерибом и малые как великие, и он судит их по величию и по назначению, им определенному.
\vs Ahh 16:10
И после он сказал мне: ступай, Абикам, в отведенное тебе место, а поутру вставай и приходи ко мне.
\vs Ahh 16:11
И повелел царь, чтобы вельможи его заутра облачились в одежды цвета красного, а сам он облачился с утра в одежды из виссона и пурпура;
\vs Ahh 16:12
и воссел он на престоле своем, а вельможи его заняли места вокруг него и перед ним. И я был введен в присутствие царя, и затем он спросил меня:
\vs Ahh 16:13
С кем, можно сравнить меня, о Абикам, и с кем можно сравнить вельмож моих?
\vs Ahh 16:14
И ответил я ему: тебя можно сравнить с Вилом, о владыка мой царь, а вельмож твоих со жрецами его.
\vs Ahh 16:15
И сказал мне царь: ступай, Абикам, и поутру приходи.
\vs Ahh 16:16
И повелел царь вельможам своим сменить одежды свои и облечься в одежды из льна белого,
\vs Ahh 16:17
и сам он также облекся в белое, а после воссел на престоле своем, и вельможи его заняли места перед ним и вокруг него.
\vs Ahh 16:18
И ввели меня пред очи его, и спросил он меня:
\vs Ahh 16:19
С кем можно сравнить меня, о Абикам, и с кем можно сравнить вельмож моих?
\vs Ahh 16:20
И ответил я ему, и сказал ему: тебя можно сравнить с Солнцем, а вельмож твоих с лучами его.
\vs Ahh 16:21
И снова сказал мне царь: ступай, Абикам, а поутру возвращайся ко мне.
\vs Ahh 16:22
И повелел он вельможам своим заутра переоблачиться в одежды черные;
\vs Ahh 16:23
врата дворца были покрыты черным и алым; царь же облекся в одежды алые.
\vs Ahh 16:24
После фараон велел меня ввести; я вошел, и он спросил меня:
\vs Ahh 16:25
С кем можно сравнить меня, о Абикам, и с кем можно сравнить вельмож моих?
\vs Ahh 16:26
И сказал я ему: тебя можно сравнить с Месяцем, о царь, а вельмож твоих со звездами.
\vs Ahh 16:27
И сказал он мне: ступай, Абикам, а поутру приходи ко мне.
\vs Ahh 16:28
И повелел фараон вельможам своим облечься заутра в другие одежды разных цветов;
\vs Ahh 16:29
и врата дворца были затянуты красным разных оттенков; и царь облачился в одежды разноцветные.
\vs Ahh 16:30
После фараон велел меня ввести: я вошел, и он спросил меня:
\vs Ahh 16:31
С кем можно сравнить меня и с кем можно сравнить вельмож моих?
\vs Ahh 16:32
И ответил я ему: тебя можно сравнить с Нисаном, а вельмож твоих с цветами его.
\vs Ahh 16:33
Когда царь услыхал это, он возрадовался весьма и был исполнен веселия. Он сказал мне:
\vs Ahh 16:34
Абикам, ты сравнил меня один раз с Вилом, а вельмож моих со жрецами его,
\vs Ahh 16:35
другой раз ты сравнил меня с Солнцем, а вельмож моих с лучами его,
\vs Ahh 16:36
в третий раз ты сравнил меня с Месяцем, а вельмож моих со звездами,
\vs Ahh 16:37
в четвертый раз ты сравнил меня с Нисаном, а вельмож моих с цветами его;
\vs Ahh 16:38
с кем же ты сравнишь Ассур-Аддина, владыку твоего?
\vs Ahh 16:39
И ответил я, и сказал ему: сохрани меня Бог, о царь, говорить о владыке моем Ассур-Аддине, когда ты сидишь,
\vs Ahh 16:40
ибо владыка мой царь Ассур-Аддин подобен Вилсамину, а вельможи его подобны молниям;
\vs Ahh 16:41
стоит ему пожелать, и он обращает росу и дождь в твердый град, он заставляет дыму восходить к небесам владычества своего, он издает гром и рыкание и возбраняет Солнцу вставать и лучам его показываться;
\vs Ahh 16:42
он возбраняет Вилу и жрецам его уходить и приходить местами людными;
\vs Ahh 16:43
он возбраняет Месяцу восходить и звездам блистать.
\vs Ahh 16:44
И если он захочет повелеть ветру северному, ветер соделает дождь и град, и побьет Нисан, и погубит цветы его.
\vs Ahh 16:45
И возмутился царь, слыша это.
\vs Ahh 16:46
И спросил фараон: заклинаю тебя жизнью владыки твоего Ассур-Аддина, каково имя твое?
\vs Ahh 16:47
И ответил я: Ахиахар, писец! И печать царя Ассур-Аддина вручена мне.
\vs Ahh 16:48
И спросил Фараон: так ты жив?
\vs Ahh 16:49
И ответил я: так, я жив, о владыка мой царь, и я видел Ассур-Аддина, и он продлил дни мои, и Бог избавил меня от смерти, и от казни лютой, и от греха, которого не творили руки мои.
\vs Ahh 16:50
И сказал мне царь: ступай, писец, а поутру приходи к мне и скажи мне слово, которого никто не слыхал и которого не слыхали вельможи мои ни в едином из городов Мицрейских.

\vs Ahh 17:1
И тогда я, Ахиахар, отошел в уединение и написал письмо такое:
\vs Ahh 17:2
От фараона, царя Мицрейского, Ассур-Аддину, царю Ассирийскому, мир!
\vs Ahh 17:3
Цари имеют нужду в царях, и судьи имеют нужду в судьях, а в наше время они имеют нужду в дарах, ибо средства их умалены.
\vs Ahh 17:4
Сокровищнице моей недостает денег, но позволь мне занять в твоей сокровищнице девятьсот талантов серебра, и я вскоре верну их тебе.
\vs Ahh 17:5
Я свернул письмо это и взял его с собою, и я сказал царю:
\vs Ahh 17:6
Слова, написанного в письме этом, не слышал ни ты, ни другой человек.
\vs Ahh 17:7
Все возопили: мы слышали его, и нет в том никакого сомнения!
\vs Ahh 17:8
И тогда ответил я им: итак, вы слышали, что Мицра должна Ассирии девятьсот талантов! И все были исполнены изумления.
\vs Ahh 17:9
И сказал мне тогда царь: Ахиахар!
\vs Ahh 17:10
И я ответил: вот я!
\vs Ahh 17:11
И сказал он мне: построй мне дворец между небом и землею, превыше земли локтей на тысячу.
\vs Ahh 17:12
И тотчас взял я из клеток орлов моих юных, и привязал к ногам их веревку должной длины, и велел посадить на нее мальчиков, которые принялись кричать:
\vs Ahh 17:13
Глины сюда, кирпичей сюда! Вот идут зодчие! Дайте им, с чем работать, что нужно зодчим царевым, и смешайте для зодчих вина.
\vs Ahh 17:14
Вельможи увидели, услышали и были в изумлении.
\vs Ahh 17:15
Тогда я, Ахиахар, взял жезл и бил зодчих, пока они не побежали доставать потребное для стройки.
\vs Ahh 17:16
Тогда сказал царь: ты обезумел, Ахиахар! кто сможет подавать им наверх то, что они требуют?
\vs Ahh 17:17
Я сказал им: зачем же вы поминаете понапрасну имя Ассур-Аддина? Если бы он был здесь и если бы он пожелал построить два дворца в один день, он бы их построил.
\vs Ahh 17:18
Царь сказал мне: оставь ты этот дворец. Приходи ко мне поутру.
\vs Ahh 17:19
И когда поутру я вошел к нему, он посмотрел на меня, и увидел меня, и сказал:
\vs Ahh 17:20
Ахиахар, изъясни мне, что это у нас приключилось? Жеребец твоего владыки заржал в Ассуре, в Ниневии, а наши кобылицы услышали его и выкинули плод.
\vs Ahh 17:21
И тогда я, Ахиахар, вышел от царя; и я велел слугам взять кота, бога Мицрейцев, и бить его до тех пор, покуда Мицрейцы не услышали воплей его.
\vs Ahh 17:22
Мицрейцы пошли и донесли царю: этот Ахиахар взял кота, который есть бог, и бил его.
\vs Ahh 17:23
Царь внял им и спросил меня: о, Ахиахар, зачем ты учиняешь богам нашим бесчестие?
\vs Ahh 17:24
И я сказал ему: царь, вовеки живи! Этот кот учинил мне урон великий и отнюдь не малый;
\vs Ahh 17:25
ибо царь подарил мне петуха, имевшего голос весьма прекрасный, который пел тогда, когда мне надо было идти ко двору и когда царь меня требовал, и будил меня от сна моего.
\vs Ahh 17:26
И вот урон, учиненный мне котом этим: он побывал ночью этой в Ассуре, в Ниневии, и откусил голову петуху тому, и вернулся сюда.
\vs Ahh 17:27
Тогда царь сказал мне: будучи стар, ты заблуждаешься. Между Ассуром и Мицрой триста парасангов: как же кот мог за эту ночь дойти туда, откусить голову этому петуху и вернуться обратно?
\vs Ahh 17:28
Я сказал ему: пусть между Ассуром и Мицрой триста парасангов, разве не слышали мы, что кобылицы ваши услышали ржание жеребца нашего и выкинули плод? Так и с котом.
\vs Ahh 17:29
При этих словах царь смутился и в изумлении сказал мне: о, Ахиахар, изъясни мне то, что я скажу тебе:
\vs Ahh 17:30
есть у меня столп великий, сложенный из восьми тысяч семисот шестидесяти трех кирпичей, на верху которого насаждено двенадцать кедров;
\vs Ahh 17:31
на верху каждого из этих кедров по тридцати колес, и по каждому колесу бегут две нити, одна белая, а другая черная.
\vs Ahh 17:32
Я ответил царю о предмете, о котором он спрашивал меня: умы баранов и быков знают то, что ты спрашиваешь у меня, царь.
\vs Ahh 17:33
Столп, о котором говорил владыка мой царь, это год;
\vs Ahh 17:34
столп этот сложен из восьми тысяч семисот шестидесяти трех кирпичей, каковы суть восемь тысяч семьсот шестьдесят три часа;
\vs Ahh 17:35
двенадцать кедров суть двенадцать месяцев года;
\vs Ahh 17:36
тридцать колес суть тридцать дней месяца;
\vs Ahh 17:37
две нити, одна черная и другая белая, это ночь и день.
\vs Ahh 17:38
Царь сказал мне еще: перестань.
\vs Ahh 17:39
Однако я требую от тебя, о Ахиахар, чтобы ты свил две длинные веревки из песка, по пятидесяти локтей в длину и по пальцу в ширину.
\vs Ahh 17:40
Я отвечал ему: прикажи, владыка мой царь, чтобы мне принесли такую веревку из твоей сокровищницы, чтобы мне свить подобную ей.
\vs Ahh 17:41
Он сказал мне: ты не понял слов моих: если не совьешь ты мне веревки, как я сказал тебе, не получишь ты подати Мицрейской.
\vs Ahh 17:42
И тогда я, Ахиахар, покинул царя и провел ночь ту в размышлении великом, и поутру пришел мне помысл некий.
\vs Ahh 17:43
И стал я позади дворца, в котором обитал царь, и сделал в стене напротив солнца дыру, и прошло солнце сквозь стену дворца.
\vs Ahh 17:44
И сделал я другую дыру в той же стене; после взял я пригоршню пыли и вложил в дыры, и пыль явилась в луче и была увлечена.
\vs Ahh 17:45
И заговорил я, и сказал я царю: повели, владыка мой царь, чтобы эти лучи связали в пучок, и я сделаю подобный пучок, если ты пожелаешь.
\vs Ahh 17:46
Увидев это, царь и вельможи его были объяты изумлением и недоумением и были весьма унижены.
\vs Ahh 17:47
Тогда царь велел принести мне верхний камень от разбитого жернова, а после заговорил и сказал:
\vs Ahh 17:48
Прошей мне этот камень, Ахиахар!
\vs Ahh 17:49
И я тотчас взял пест из того же камня, что и жернов, бросил его и сказал царю:
\vs Ahh 17:50
Владыка мой царь, у меня нет с собой шильев сапожника, и я не нахожу того, что мне потребно;
\vs Ahh 17:51
вели, однако, сапожникам твоим продеть нить в этот пест, который одного естества с жерновом, и я тотчас прошью его.
\vs Ahh 17:52
На эти слова царь засмеялся и сказал: добро же, Ахиахар! День, в который ты рожден, да будет благословен пред лицем богов Мицрейских!
\vs Ahh 17:53
Поелику я вижу тебя живым и здравствующим, я сотворю этот день великим празднеством и временем веселия.
\vs Ahh 17:54
Когда царь фараон был побит во всем, когда я оказал отпор его хитростям, когда я разрешил и упразднил все измышления его и все загадки его, он отдал мне подать с Мицры за три года,
\vs Ahh 17:55
и сверх того я получил те девятьсот талантов, о которых шла речь в изготовленном мною письме, как о ссуженных моим государем, и о которых все будто бы слышали, по собственному их признанию.
\vs Ahh 17:56
Я был осыпан дарами от царя и почестями от вельмож его.

\vs Ahh 18:1
И тотчас царь Ассур-Аддин поспешил мне навстречу. И начал царь говорить мне слова мудрые:
\vs Ahh 18:2
Проси и требуй от меня, чего хочешь.
\vs Ahh 18:3
И сказал я: о, владыка мой царь, вовеки живи!
\vs Ahh 18:4
И царь сошел ко мне навстречу и радовался радостию великою.
\vs Ahh 18:5
Он почтил меня, и посадил подле себя на престоле своем и на твердыне своей, и сказал мне:
\vs Ahh 18:6
Проси у меня, Ахиахар, всего, чего пожелаешь. Если пожелаешь, отдам тебе все царство мое. И сказал ему:
\vs Ahh 18:7
О, владыка мой царь, вовеки живи, в роды и роды! Все, чего прошу я у величия твоего, если нашел я благоволение в очах твоих, это дать хорошее место Навусемаку, копьеносцу, ибо ему обязан я тем, что доселе живу.
\vs Ahh 18:8
И выказал мне тогда царь приязнь свою милостями многими, особенно же дарами и подарками, которые принял я от руки его.
\vs Ahh 18:9
И осыпал меня царь дарами многими, и дарил подарки Навусемаку.
\vs Ahh 18:10
И стал царь расспрашивать меня обо всем, что было со мною пред лицом фараоновым, и о загадках фараоновых;
\vs Ahh 18:11
и рассказывал я ему все от начала и до конца, по порядку и по отдельности; он же, слушая, дивился.
\vs Ahh 18:12
И затем вынул я сокровища, и сребро, и золото, и дары, и подарки, что дал мне царь Мицрейский, чтобы доставил я их ему из Мицры; и радовался он радостию несказанною.
\vs Ahh 18:13
И сказал он мне: сколько желаешь ты получить от меня?
\vs Ahh 18:14
Я же сказал ему: я не прошу ничего, кроме как видеть тебя счастливым и благоденственным.
\vs Ahh 18:15
Что бы делал я с этими богатствами и с прочим? Однако прошу у блаженства твоего, чтобы ты дал мне власть делать все, что я пожелаю, Надаву, дабы отомстить ему, и чтобы не взыскивал ты с меня кровь его.
\vs Ahh 18:16
И тотчас дозволил мне царь делать все, что я пожелаю.
\vs Ahh 18:17
Я взял Надава и пошел в дом мой; и связал я его узами и цепями железными, и возложил я оковы железные на руки его и на ноги его и железо на плечи его,
\vs Ahh 18:18
а после стал я бичевать его розгами и бить его ударами лютыми и припоминал ему поучение, которое преподал ему в премудрости, и в знании, и в философии.

\vs Ahh 19:1
И сказал я: сын мой, того, кто не слушал ушами своими, понуждают слушать спиною его.
\vs Ahh 19:2
Надав, сын мой, заговорил и сказал:
\vs Ahh 19:3
Зачем гневаешься ты на сына твоего? Я отвечал ему:
\vs Ahh 19:4
Я, сын мой, посадил тебя на престоле славы, ты же сбросил меня с престола моего; и правда моя спасла меня.
\vs Ahh 19:5
Ты был для меня, сын мой, как скорпион, который ужалил скалу, и та сказала ему: ужалил ты сердце неуязвимое.
\vs Ahh 19:6
Он ужалил иглу, и та сказал ему: ужалил ты жало, которое сильнее, чем твое.
\vs Ahh 19:7
Ты был для меня, сын мой, как тот, кто бросает камень в небо; до неба он не дометнет, однако, бросив, согрешит.
\vs Ahh 19:8
Ты был для меня, сын мой, как тот, кто увидел ближнего своего дрожащим от стужи, и взял сосуд с водою, и метнул в него.
\vs Ahh 19:9
Сын мой, отвечай мне! Ты напал на меня, как голодный лев на осла, блуждавшего поутру.
\vs Ahh 19:10
Сказал лев ослу: подойди в мире, брат мой и друг мой!
\vs Ahh 19:11
Осел ответил: такого мира пожелаю тому, кто не привязал меня и не помешал мне выйти навстречу тебе.
\vs Ahh 19:12
Сын мой, ты был для меня как западня, укрытая под навозом. И пришел воробей, и увидела его западня, и сказала: брат мой, что делаешь ты здесь?
\vs Ahh 19:13
И ответил воробей: смотрю на тебя.
\vs Ahh 19:14
И сказала западня: помолись Богу, слава Ему!
\vs Ahh 19:15
И спросил воробей: что у тебя за палка?
\vs Ahh 19:16
Ответила западня: это посох мой и опора моя, я подпираюсь им, когда стою на молитве.
\vs Ahh 19:17
Спросил воробей: что за зерна во устах твоих?
\vs Ahh 19:18
Ответила западня: это пища, и это хлеб, восстанавливающий силы тех, кто мучим голодом.
\vs Ahh 19:19
Я поместила его во рту моем, чтобы он служил для пропитания голодных, ищущих у меня прибежища своего.
\vs Ahh 19:20
Воробей сказал: вот, я весьма изнурен голодом, и я прихожу, чтобы есть зерна.
\vs Ahh 19:21
Западня ответила ему и сказала: приблизься, о брат мой, и не страшись!
\vs Ahh 19:22
Когда же приблизился воробей, чтобы взять зерен, она тотчас схватила его за голову; и сказал воробей западне:
\vs Ahh 19:23
Если таков пост твой, и такова молитва твоя, и для такой цели зерна те, Бог не примет ни поста твоего, ни молитвы твоей и не подаст тебе никакого блага.
\vs Ahh 19:24
Сын мой, ты был для меня как жук-долгоносик, обретающийся в хлебах, что не годен ни на что доброе, но губит хлеба.
\vs Ahh 19:25
Сын мой, ты был для меня как котел, к которому приладили золотые ручки, не отчистив его дна от черноты.
\vs Ahh 19:26
Сын мой, ты был для меня как птица, которая замкнута в западне и не может убежать от ловца;
\vs Ahh 19:27
и тогда она поднимает голос приятный и сладостный и собирает вокруг себя многих птиц, малых или больших, дабы они также были уловлены.
\vs Ahh 19:28
Сын мой, ты был для меня как козел, который ведет товарищей своих на живодерню и не может спасти себя же самого.
\vs Ahh 19:29
Сын мой, ты был для меня как пес, которого проняла стужа и который пошел греться к гончарам, а когда согрелся, норовил облаять их и искусать.
\vs Ahh 19:30
Они пытались ударить его, он же залаял, а они, страшась быть искусанными, убили его.
\vs Ahh 19:31
Сын мой, ты был для меня как та свинья, которая пошла в баню вместе с вельможами;
\vs Ahh 19:32
и прошла она в баню, и омылась, а как вышла, увидала грязь и принялась в ней валяться.
\vs Ahh 19:33
Сын мой, рука, которая не трудится, и не утомляется, и не совершает работ, будет отсечена по причине лености своей.
\vs Ahh 19:34
Сын мой, это я показал тебе лице царево, и привел тебя к милостям великим, и научил тебя, и воспитал тебя, и доставил тебе всякое благо; и чем ты воздал мне, и чем отплатил мне?
\vs Ahh 19:35
Увы, и ах, и горе! Если бы ты ничего не получил от меня и ничего не принял от меня, ты не имел бы никакой власти надо мною во все дни жизни твоей.
\vs Ahh 19:36
Сын мой, сказало дерево дровосекам: если бы в руках ваших не было части меня, вы не напали бы на меня.
\vs Ahh 19:37
Ты был для меня как кот, которому сказали: перестань воровать, а тогда входи и выходи, как захочешь.
\vs Ahh 19:38
Он же ответил им: это естество мое, и если бы имел я глаза из серебра, руки из золота и ноги из берилла, я отнюдь не отстал бы от воровских дел моих.
\vs Ahh 19:39
Ты был для меня, сын мой, как змея, что забралась на ветку терновника и плыла по реке;
\vs Ahh 19:40
волк увидел ее и сказал: зло забралось на зло, и зло злейшее несет их.
\vs Ahh 19:41
Ответила змея волку тому: а ты отводишь ли коз к хозяину их?
\vs Ahh 19:42
Сын мой, я видел козу, которую пригнали на живодерню, но еще не пришло время ее, и потому она вернулась к себе, и увидала она детей своих и отпрысков детей своих.
\vs Ahh 19:43
Сын мой, я давал тебе вкушать от всякой снеди доброй, а ты не насытил меня и хлебом, смешанным с прахом;
\vs Ahh 19:44
я помазывал тебя благовониями усладительными, а ты осквернил тело мое прахом;
\vs Ahh 19:45
я упоевал тебя вином старым, а ты не напоил меня даже водою.
\vs Ahh 19:46
Сын мой, я возрастил тебя высоким, как кедр, а ты согнул меня при жизни моей и упоил меня лукавством.
\vs Ahh 19:47
Сын мой, я возвысил тебя как башню, и я говорил:
\vs Ahh 19:48
Когда враг мой придет на меня, я поднимусь и найду прибежище.
\vs Ahh 19:49
Ты же увидел врага моего и склонился к нему.
\vs Ahh 19:50
Ты был для меня, сын мой, как крот, который выходит на лице земли, чтобы обвинять Бога, не давшего ему зрения; и прилетает орел, и уносит его.

\vs Ahh 20:1
И ответил Надав, сын мой, и сказал мне:
\vs Ahh 20:2
Далече да будет от тебя, владыка мой, обычай немилосердных, но поступи со мною по милости твоей!
\vs Ahh 20:3
Даже когда человек погрешает против Бога, Бог отпускает ему грехи; так и ты ныне прости меня, и я буду печься обо всех твоих скотах или буду пасти овец твоих и свиней твоих, и меня будут называть злым, а тебя добрым.
\vs Ahh 20:4
Я ответил ему и сказал ему: сын мой, ты был для меня как пальма, которая обреталась вдали от дороги и не давала плодов; и пришел хозяин ее, и хотел удалить ее.
\vs Ahh 20:5
И сказала ему пальма эта: дай мне год, и я принесу плод сафлора.
\vs Ahh 20:6
И сказал ей хозяин ее: злосчастная, тебя недостало на то, чтобы принести твой собственный плод, как достанет тебя на то, чтобы принести чужой?
\vs Ahh 20:7
Сын мой, старость орла лучше, чем юность грифа.
\vs Ahh 20:8
Сын мой, если бы волку велели держаться вдали от овец, он ответит, что ему мил прах, ими подымаемый.
\vs Ahh 20:9
Сказали волку: учись: буква элэп, буква бит.
\vs Ahh 20:10
Он ответил: баранина, козлятина.
\vs Ahh 20:11
Сын мой, ты оправдал пословицу, которая гласит: кого ты породил, называй сыном твоим, а кого ты воспитал, называй рабом твоим.
\vs Ahh 20:13
Сын мой, больше всякого иного слова оправдал ты это: возьми сына сестры твоей на руки твои и разбей его о камень.
\vs Ahh 20:14
Бог, сохранивший мне жизнь, знающий все и воздающий каждому по делам его, сын мой, да судит и да рассудит между мною и тобою.
\vs Ahh 20:15
Ничего больше не скажу тебе. Бог да воздаст тебе по делам твоим.

\vs Ahh 21:1
И когда юный Надав услышал слово это, тело его тотчас раздулось и стало как мех и бурдюк полный, и внутренности его вышли из чресл его.
\vs Ahh 21:2
Злое его деяние воспламенило его, палило его; иссушало его, обессиливало его, губило его, умертвило его.
\vs Ahh 21:3
Конец его привел его к погибели, и ниспал он в геенну вместе с завистливыми и горделивыми,
\vs Ahh 21:4
как сказано: Сын выроет ров, и согрешит, и падет в яму, которую сам выкопал;
\vs Ahh 21:5
и еще: Кто творит лукавое, впадет в погибель;
\vs Ahh 21:6
и еще: Кто строит кову брату своему, сам падет в нее.
\vs Ahh 21:7
Здесь кончается повесть об Ахиахаре, мудреце и философе достославном, который разумел тайны и толковал загадки.

\bibbookdescr{Tad}{
  inline={Завещание Адама},
  toc={Завещание Адама},
  bookmark={Завещание Адама},
  header={Завещание Адама},
  abbr={Адам}
}

\chhdr{Дневные часы}

\vs Tad 1:1
И более, уразумей о часах дневных и ночных,
и как подобает тебе умолять Бога
и молиться ему в каждое из его времён года.
\vs Tad 1:2
Ибо мой Творец научил меня всему этому,
и он сказал мне имена всех диких животных и зверей,
и птиц небесных;
и потом Бог дал мне уразуметь число дневных и ночных часов,
и он сказал мне, как ангелы славят Бога.
\vs Tad 1:3
Пойми же, о сын мой, что в 1-ый час дня
молитва
моих сыновей\fnote{моих сыновей}{\vsep\ небожителей.}
восходит к Богу.
\vs Tad 1:4
И во 2-ой час происходит молитва и прошение ангелов.
\vs Tad 1:5
В 3-ий час птицы небесные восхваляют его.
\vs Tad 1:6
И в 4-ый час
духи\fnote{духи}{\vsep\ животные.}
поклоняются Ему.
\vs Tad 1:7
И в 5-ый час все
дикие звери и животные\fnote{дикие звери и животные}{\vsep\
живущие выше небес.}
приветствуют его.
\vs Tad 1:8
В 6-ой час происходит прошение Керубов,
которые свидетельствуют против
беззаконий нашей человеческой природы\fnote{которые \ldots\ природы}{\vsep\
--- .}.
\vs Tad 1:9
И в 7-ой час все ангелы предстают Богу,
и отходят от него, ибо в этот час молитва
всякого живого существа восходит к Богу.

\vs Tad 1:10
В 8-ой час сияние
небожителей\fnote{небожителей}{\vsep\ огня и вод.}
восхваляет его.
\vs Tad 1:11
И в 9-ый час ангелы Божии, стоящие пред престолом Всевышнего,
воздают ему почести.
\vs Tad 1:12
И в 10-ый час Святой Дух осеняет воды,
и демоны бегут и удаляются от вод.
И если бы Святой Дух не осенял воды в этот час ежедневно,
никто бы не мог пить воду,
ибо плоть разрушалась бы злыми демонами. 
И все, кто повстречаются демонам в этот час будут ранены.
\vs Tad 1:13
И в 11-ый час происходит прославление праведных.
\vs Tad 1:14
И в 12-ый час Бог Всевышний принимает молитвы
и прошения сынов человеческих.
 
\chhdr{Ночные часы}

\vs Tad 2:1
И в 1-ый час ночи демоны воздают благодарение и хвалу Богу Всевышнему,
и в это время не причиняют они зла и вреда никому из сынов человеческих,
доколе они не кончат своё воздаяние почести.
\vs Tad 2:2
И во 2-ой час ночи
рыбы\fnote{рыбы}{\vsep\ голуби.}
и всякая тварь, существующая в воде,
восхваляет Бога, и дикие звери и чудовища морские.
\vs Tad 2:3
И в 3-ий час огонь восхваляет его
(в этот час он находится в глубочайшей бездне),
и в этот час никто не может обратиться к Нему.
\vs Tad 2:4
И в 4-ый час Серафы восхваляют его так: Святый, Святый, Святый.
\vs Tad 2:5
И в 5-ый час воды, которые превыше небес,
восхваляют его.
Ныне издавна я садился и слушал, вместе с ангелами, и восхищался тому,
как в\acc{о}ды восклицают;
подобно шуму мощных
колёс\fnote{колёс}{\vsep\ крыльев.},
и они восклицали подобно морским водам голосом,
восхваляющим Бога.
\vs Tad 2:6
И в 6-ой час облака восхваляют Бога в страхе и трепете.
\vs Tad 2:7
И в 7-ой час земля утихала в безмолвии,
и всякая тварь на ней, и воды засыпали. 

\vs Tad 2:8
В 8-ой час земля порождает траву и зелень,
и даёт деревьям производить листья и плоды.
\vs Tad 2:9
И в 9-ый час ангелы исполняют своё
служение почитания Бога,
и молитва сынов человеческих приходит перед Бога Всевышнего.
\vs Tad 2:10
И в 10-ый час врата небесные открываются,
и Бог слышит молитвы сынов верующих и прошения,
которые они испрашивают у Бога, исполняются для них;
и при звуке крыльев Серафов в это время
петухи поют и восхваляют Бога.
\vs Tad 2:11
И в 11-ый час радость и ликование стоит по всей земле,
ибо солнце входит в рай,
и его свет восходит по всем концам мира
и освещает всякую тварь.
\vs Tad 2:12
И в 12-ый час прилично моим детям вставать
перед Богом и воздавать ему почести,
ибо в этот час в великом безмолвии
покоятся все небесные духи.
 

\chhdr{Адам предсказывает пришествие Христа}

\vs Tad 3:1
И сказал Адам Сифу, сыну своему:
\vs Tad 3:2
Ныне же познай всё это, и внемли моему слову,
и уразумей, что слово Бога Всевышнего снизойдет на землю,
именно так, как он сказал мне в то время,
когда он изгнал меня из рая.
\vs Tad 3:3
Ибо он сказал мне, что его слово в последние дни
станет мужем от девы, имя которой Мариамь,
и сокроется в ней, и облечётся плотью,
и родится как муж великой силы
и деятельного искусства и познания.
\vs Tad 3:4
Никто не познает его, кроме него самог\acc{о} и того,
кому он откроется явно.
\vs Tad 3:5
И Бог сказал, что он будет ходить среди людей на земле,
и возрастать по дням и годам,
и явно сотворит знамения и чудеса,
и будет ходить по воде как по суше,
и явно запретит морю и ветрам, и они подчинятся ему,
и когда он воскликнет морским волнам,
они скоро поспешат ответить ему.
\vs Tad 3:6
И он даст слепому зрение, и очистит прокажённых,
и откроет слух глухим, и немым~--- язык,
и поднимет расслабленных, и сделает хромых ходящими,
и обратит многих от заблуждения к познанию Бога,
и изгонит демонов из людей.
\vs Tad 3:7
И кроме того Бог сказал мне, говоря:
Не печалься, Адам, ибо ты пожелал стать богом
и нарушил мою заповедь.
\vs Tad 3:8
Вот, я утвержу тебя, но не теперь,
но спустя много дней.
\vs Tad 3:9
И опять он сказал мне, говоря:
я Бог, изгнавший тебя из рая удовольствий на землю,
которая будет порождать тернии и шипы,
и ты будешь жить на ней.
\vs Tad 3:10
Согни спину, и пусть твои колени дрожат в старости,
и я дам твою плоть в пищу червям.
\vs Tad 3:11
И через 5 дней и половину дня я сжалюсь над тобою
и явлю тебе милость в изобилии моего сострадания и милости.
\vs Tad 3:12
И я приду в твой дом, и я буду жить в твоей плоти,
и ради тебя мне угодно будет родиться как младенцу.
\vs Tad 3:13
И ради тебя мне угодно будет ходить по торжищам.
\vs Tad 3:14
И ради тебя мне угодно будет поститься 40 дней.
\vs Tad 3:15
И ради тебя мне угодно будет принять омовение.
\vs Tad 3:16
И ради тебя мне угодно будет претерпеть страдание.
\vs Tad 3:17
И ради тебя мне угодно будет быть повешенным на крёстном древе.
\vs Tad 3:18
Всё это~--- ради тебя, о Адам.

\vs Tad 3:19
И по прошествии 3-ёх дней в могиле я воскрешу тело, которое я принял
от тебя.
\vs Tad 3:20
И поставлю тебя по правую руку мою и сделаю тебя богом, чего ты и хотел.
\vs Tad 3:21
И праведность небесная восстановится.

\vs Tad 3:22
И тогда, я, Сиф, спросил у отца: Как называется плод, который ты съел?
\vs Tad 3:23
И он ответил мне: Инжир, сын мой, стал вратами через которые зашла смерть
ко мне и к моему роду.
\vs Tad 3:24
И также жизнь войдёт в меня и моих сыновей когда Господь наш станет
человеком через деву и наденет тело святое в конце века.

\vs Tad 3:25
Кроме того, ты должен знать, о Сиф, сын мой,
вот придёт Потоп и омоет всю землю,
из-за сынов убийцы Каина, убившего своего брата из зависти,
из-за его сестры Луд.
\vs Tad 3:26
И после Потопа 6000 лет будет предоставлено миру,
а потом придут последние дни,
и всё будет исполнено, и придёт его время,
и огонь потребит всё, что найдено пред Богом,
и земля освятится,
и Господь господствующих будет ходить по ней.

\vs Tad 3:27
И Сиф записал эту заповедь,
и запечатал её своею печатью
и печатью своего отца Адама,
которую он унёс с собою из рая,
и печатью своей матери Евы.

\bibbookdescr{Tmo}{
  inline={Завещание Моисея},
  toc={Завещание Моисея},
  bookmark={Завещание Моисея},
  header={Завещание Моисея},
  abbr={Зав~Мо}
}
\vs Tmo 1:1
Завещание Моисея, данное им в сто двадцатый год жизни его, который есть четырехсотый по отправлении из Ханаана, когда вышел народ с Моисеем и дошел до Бен-Амми за Иорданом по пророчеству Моисееву.
\vs Tmo 1:2
Призвал Моисей к себе Иесуа, сына Нуна, человека, угодного Яхве, дабы стал он преемником народа и ковчега Завета со всеми святынями его и дабы ввел народ в землю, данную коленам его, дать им ее по завету и по клятве, которую произнес он в скинии, что даст ее через Иесуа;
\vs Tmo 1:3
И сказал Моисей к Иесуа такое слово: "Обещай, что все сотворишь, что поручено тебе, сотворишь со старанием, в точности и без ропота, ибо так говорит Владыка мира.
\vs Tmo 1:4
Создал Он мир ради народа Своего и не сделал начала творения ясно видимым от начала мира, дабы обличились тем народы и низкими речами своими обличили себя.
\vs Tmo 1:5
Так Он измыслил и изобрел меня, от начала мира готового стать судьею завета Его.
\vs Tmo 1:6
И ныне открою тебе, что совершилось время лет жизни моей и отхожу я в успение отцов моих.
\vs Tmo 1:7
Предо всем народом прими писание сие, дабы не забывал ты хранить книги, кои передам тебе,
\vs Tmo 1:8
Ты же их расположишь в порядке и запечатаешь и положишь в сосудах глиняных в месте, созданном от начала мира, дабы призывалось имя Его вплоть до дня покаяния с почитанием, коим почтил их Яхве на исходе дней.

\vs Tmo 2:1
Войдут они с тобою в землю, которую назначил и обещал Он дать отцам их.
\vs Tmo 2:2
В ней благословишь ты и дашь каждому и установишь им жребий мой и утвердишь им царство и управление на местах определишь им по тому, как угодно будет Яхве, Богу их, по суду и справедливости.
\vs Tmo 2:3
После того как войдут в землю свою, пройдет пять лет, и будет власть вождей и тираннов восемнадцать лет, и на девятнадцать лет отделятся десять колен, ибо отойдут два колена и перенесут ковчег Завета.
\vs Tmo 2:4
Тогда Бог небесный явит ковчег Свой и башню святилища Своего. И утвердятся два колена святости, десять же колен установят себе по законам своим царства.
\vs Tmo 2:5
И будут приносить жертвы двадцать лет: за семь лет соорудят стены, и ограждать их буду девять лет, и нападать будут на завет Яхве четыре года, и, наконец, осквернят договор, который сотворил с ними Яхве.
\vs Tmo 2:6
И принесут детей своих в жертву чужеземным богам, и установят в скинии идолов, служа им, и в доме Яхве будут вершить преступления, и многих идолов всех животных сделают.

\vs Tmo 3:1
В те времена придет к ним с востока царь, и покроет конница землю их, и сожжет он огнем поселения их со святым храмом Яхве, и все святые сосуды он истребит, и весь народ изгонит, и уведет их в землю отчизны своей, и два колена уведет с собой.
\vs Tmo 3:2
Тогда воззовут два колена к десяти коленам, и лягут словно львица, покрытые пылью в полях, алчущие и жаждущие с детьми нашими, и возопиют: "Праведен и свят Яхве! Отчего вы грешили, а мы так же уведены с вами?"
\vs Tmo 3:3
Тогда восплачут десять колен, слыша слова упрека от двух колен и скажут: "Что сделали мы вам, братья? Не во весь ли дом Израилев вошло горе это?"
\vs Tmo 3:4
И все колена восплачут, вопия к небу и говоря: "Бог Авраама, Бог Исаака, Бог Иакова, воспомни завет Твой, который заключил Ты с ними, и клятву, которою клялся Ты им, что никогда не упразднится семя их от земли, которую Ты дал им".
\vs Tmo 3:5
И в тот день воспомнят имя мое, говоря колено к колену, и всякий человек к ближнему своему: "Не то ли это, в чем удостоверял нас Моисей в пророчествах своих, он, претерпевший многое в Египте и в море Суф, и в пустыне в продолжение сорока лет, свидетельствуя и призывая в свидетели небо и землю, да не преступим мы заповедей Его, в коих был он нам судьею;
\vs Tmo 3:6
И так случилось с нами по словам Его, и по уверению Его, как свидетельствовал он нам в те времена, и вышло, что ведут нас, плененных, в Восточную землю, где и будем мы рабами около семидесяти семи лет".

\vs Tmo 4:1
Тогда войдет один, стоящий над ними, и прострет руки, и преклонит колени свои, и станет молиться за них, говоря:
\vs Tmo 4:2
"Яхве, Царь всех, на высоком престоле властвующий над миром, возжелавший, дабы сей народ был народом Твоим избранным, тогда хотел Ты называться их Богом по завету, который заключил Ты с отцами их.
\vs Tmo 4:3
И пошли они, плененные, в землю чуждую с женами и детьми своими, и пребывают у врат иноплеменных и там. Где же величие великое? Призри и смилуйся над ними, Господь небесный!"
\vs Tmo 4:4
Тогда воспомнит о них Бог по завету, что сотворил Он с отцами их, и явит Он милосердие Свое, и вложит в те времена в душу царя, дабы смиловался над ними, и отпустит их царь в землю и область их.
\vs Tmo 4:5
Тогда поднимутся некоторые части колен и пойдут в свое место установленное и обновят укрепления его.
\vs Tmo 4:6
Два же колена пребудут в вере своей, печальные и плачущие, ибо не смогут принести жертв Яхве, Богу отцов своих, десять же колен увеличатся и умножатся среди племен во времена пленения их.

\vs Tmo 5:1
Когда же приблизятся времена обличения, мщение наступит от царей, соучастников преступлений, кои разделятся воистину.
\vs Tmo 5:2
Потому и было сказано: "Уклонятся от праведности, и перейдут к неправедности, и осквернят нечестиями дом служения своего, и осквернятся служением чужим богам.
\vs Tmo 5:3
И не последуют истине Божией, но осквернят алтарь дарами, кои воздадут Яхве не жрецы, но рожденные рабами от рабов.
\vs Tmo 5:4
Ибо те, которые суть ученые учителя их, будут взирать в те времена на лица страстей, и, принимая дары, продадут праведность в наказаниях.
\vs Tmo 5:5
И настолько наполнится население и предел обитания их преступлениями и обидами Бога, что те, кто творил беззаконие пред лицем Яхве, судьями станут и судить будут, кто как пожелает".

\vs Tmo 6:1
Тогда возстанут у них цари властные. Назовутся они священниками великого Бога и удалят творящих нечестие от святая святых.
\vs Tmo 6:2
И придет вслед за ними царь дерзновенный, который не будет из рода священнического. Сей человек безрассудный и злой, и будет судить он их, как они того достойны.
\vs Tmo 6:3
Истребит он вождей их мечом, и в неизвестные места порознь положит тела их, дабы не ведал никто, где тела их. Погубит он старших возрастом и юношей не пощадит.
\vs Tmo 6:4
Тогда страх пред ним будет великий в земле их, и станет он вершить суд над ними, как вершили его Египтяне, тридцать четыре года, и покарает их.
\vs Tmo 6:5
И породит он сыновей, кои будут царствовать не столь долго, и придут в землю их когорты мощного царя Западного, и одолеет он их и уведет в плен и часть храма их огнем сожжет, некоторых же распнет вокруг поселения их.

\vs Tmo 7:1
После того совершатся времена и будут править ими люди погибельные и нечестивые, называющие себя праведными,
\vs Tmo 7:2
И возбудят они гнев душ своих, будучи людьми коварными, себе угождающими, лживыми во всех делах своих и во всякий час дня, любящими пиры, чревоугодниками, пожиратели имущества бедных,
\vs Tmo 7:3
Скажут, что творили это по милосердию и истребляя стяжателей.
\vs Tmo 7:4
Будут обманывать, скрываясь, дабы не уличили их, нечестивцев, в преступлении, исполненные неправедности, от восхода до заката говоря: "Будут у нас роскошные ложа, и станем есть и пить на них. И помыслили мы, что будем, словно князья".
\vs Tmo 7:5
И руки их, и умы творят нечистое, и уста их полны слов надутых. И скажут они: "Не касайся, да не осквернишь места моего"

\vs Tmo 8:1
Придет к ним мщение и гнев, какого не бывало у них от века до того времени.
\vs Tmo 8:2
Тогда возставит им Яхве царя из царей земли и мощь из мощи великой, что распнет на кресте исповедующих обрезание.
\vs Tmo 8:3
И предаст он пыткам тех, кто откажется, и повелит в оковах отвести в темницу. А жен их отдадут богам языческим; сыновья же их, мальчики, обрезанные врачами, принуждены будут принять необрезание.
\vs Tmo 8:4
Будут карать их пытками, огнем и железом, заставят их носить идолов своих оскверненных, как и те, кто им служит, и заставят их мучители войти в тайное место свое, и понудят их стрекалами произнести слова хульные, а потом и законы похулить, положенные на алтаре их.

\vs Tmo 9:1
Тогда возстанет муж из колена Левиева, имя коему будет Таксо. Имея семерых сыновей, обратится к ним с просьбою:
\vs Tmo 9:2
"Смотрите, сыны, вот, свершилось второе отмщение народу жестокое и нечестивое, и пленение безжалостное, и превосходят они бывшие доселе. Какое племя, какая земля, какой народ, нечестивых, творивших преступление в доме своем, столько бед претерпел, сколько нас обступило?
\vs Tmo 9:3
Ныне, послушайте меня, дети, ибо видите и знаете, что никогда не испытывали Бога ни родители наши, ни праотцы, преступая заповеди Его.
\vs Tmo 9:4
Ибо знаете: в этом сила наша. Сделаем так: будем поститься три дня, а на четвертый день войдем в пещеру, которая на поле, и лучше умрем, чем преступим заповеди Бога богов, Господа отцов наших.
\vs Tmo 9:5
Если так сотворим и умрем, кровь наша отомщена будет пред Яхве".

\vs Tmo 10:1
И тогда явится царствие Его во всяком творении Его. И тогда диавол обретет конец, и скорбь с ним отойдет.
\vs Tmo 10:2
Тогда наполнится рука ангела, утвержденного на небесах, и тотчас избавит он их от врагов их.
\vs Tmo 10:3
Ибо поднимется Небесный с престола царствия Своего и выйдет из святого жилища Своего с негодованием и гневом на сынов Своих.
\vs Tmo 10:4
И задрожит земля и до пределов своих сотрясется, и высокие горы понизятся и сотрясутся, и долины падут, солнце не даст света, и во мрак обратятся рога луны и сокрушатся, и все обратится в кровь, и круг звезд смешается, и море отступит до бездны, и источники вод изсякнут, и реки высохнут.
\vs Tmo 10:5
Ибо возстанет Великий Бог, Единый и Вечный, и явится всем и отомстит народам и уничтожит всех идолов их.
\vs Tmo 10:6
Тогда блажен будешь ты, Израиль, и поднимешься ты на головы и на крылья орлиные, и наполнятся они воздухом, и возвысит тебя Бог и утвердит тебя в небе звездном в месте пребывания звезд.
\vs Tmo 10:7
И воззришь с высоты, и увидишь врагов своих на земле, и узнаешь их, и возрадуешься, и возблагодаришь, и хвалу вознесешь Создателю твоему.
\vs Tmo 10:8
Ты же, Иесуа Нун, сбереги слова сии и книгу сию. Ибо от того дня, когда приму я смерть, пройдет до пришествия Его двести пятьдесят времен. Столько времени пройдет, пока не прейдут времена.
\vs Tmo 10:9
Я же отхожу к успению отцов моих. Итак, ты, Иесуа Нун, мужайся, тебя избрал Бог быть мне преемником в завете сем.

\vs Tmo 11:1
Когда услышал Иесуа слова Моисея, записанные в писании его, и все, что предрек он, разодрал одежды свои, пал к ногам его, и утешал его Моисей и плакал с ним.
\vs Tmo 11:2
И отвечал ему Иесуа и сказал: Утешишь ты меня, господин мой Моисей, и как утешить меня в том, что сказано голосом горьким, что вышел из уст твоих, и полон слез и рыданий, ибо уходишь ты от народа Израилева.
\vs Tmo 11:3
Какое место примет тебя, каков будет памятник могильный, кто осмелится перенести тело твое из одного места в другое?
\vs Tmo 11:4
Ибо у всех, кто умирает в свое время, есть могилы свои на земле, твоя же могила от восхода солнца до заката, и от юга до севера весь мир есть могила твоя, господин мой.
\vs Tmo 11:5
Уходишь ты, и кто будет питать народ сей, и кто сжалится над ними, и кто вождем будет им в пути, и кто молиться станет за них? Не смогу я и одного дня вести их в земле предков.
\vs Tmo 11:6
Как же буду я народу сему словно отец для единого сына или мать для дочери-девицы, что готовит ее для славного мужа, оберегает, боясь, тело ее от солнца и старается, дабы не поранила та ног своих, бегая по земле?
\vs Tmo 11:7
Как дам им пищу по желанию их и насыщу их? Ведь их шестьсот тысяч возросло их число молитвами твоими, господин мой Моисей.
\vs Tmo 11:8
Какая же у меня мудрость и какое разумение в доме Божием словами судить и давать ответы?
\vs Tmo 11:9
Но и цари Аморрейские, когда услышат об этом, помыслят, что одолеют нас,
\vs Tmo 11:10
Ибо нет больше с нами Духа Святого, достойного Яхве, слову многоликого и непонятного Яхве верного во всем, божественного пророка всего мира, ведь умер он и нет более в веке сем учителя, и скажут они тогда:
\vs Tmo 11:11
"Пойдем на них, если нечестивое совершили они единожды Господу Своему, нет у них заступника, который бы вознес молитвы Яхве, каков был Моисей, великий ангел.
\vs Tmo 11:12
Он по целым часам стоял днем и ночью коленами своими на земле, молясь и взирая на мир и всех людей с милосердием и праведностью, памятуя о завете предков своих и клятвами умилостивляя Яхве".
\vs Tmo 11:13
Итак, скажут они: "Не с ними Бог. Пойдем же и сотрем их с лица земли". Вот как будет с народом сим, господин мой Моисей".

\vs Tmo 12:1
И, закончив слова свои, вновь пал Иесуа к стопам Моисеевым. И взял Моисей его за руку и посадил пред собою на седалище.
\vs Tmo 12:2
И отвечал и сказал ему Моисей: Иесуа, не бойся за себя, но будь уверен и внемли словам моим.
\vs Tmo 12:3
Все народы, какие есть в мире, создал Бог, и предусмотрел Он о них и о нас от начала творения всего мира.
\vs Tmo 12:4
И до скончания века нет ничего, чего бы не усмотрел Он до самой малой вещи, но все Он предусмотрел и устроил.
\vs Tmo 12:5
Все в этом мире предусмотрел Он, и вот \ldots
\vs Tmo 12:6
Меня поставил Он молиться за них и за грехи их и заступником быть им не по добродетели моей и не по немощи, но в меру милосердия Его и терпения Его.
\vs Tmo 12:7
Говорю же тебе, Иесуа: не по благочестию народа сего истребишь ты язычников; все тверди небесные Богом созданы и одобрены, и лишь под Его десницею они.
\vs Tmo 12:8
Итак, творящие и совершающие заповеди Божии возрастают и по доброму пути продвигаются, а согрешающим и пренебрегающим, нет им добрых заповедей в том, что предсказано.
\vs Tmo 12:9
И будут они покараны язычниками и преданы многим пыткам. Но не может быть того, чтобы совершенно уничтожил их Он и оставил.
\vs Tmo 12:10
Ибо выйдет Бог, провидящий все вовеки, и тверд Завет Его и клятва в том, что \ldots

\bibbookdescr{Trb}{
  inline={Завещание Рувима,\\первородного сына Иакова и Лии\fns{В греч. тексте $+$ ``о мыслях''.}},
  toc={Завещание Рувима},
  bookmark={Завещание Рувима},
  header={Завещание Рувима},
  abbr={Рув}
}
\vs Trb 1:1
Список завета Рувима, которое он завещал сыновьям своим,
прежде чем умереть, в 125-ый год жизни своей.
\vs Trb 1:2
Два года спустя после кончины Иосифа, брата его, занемог Рувим,
и собрались проведать его дети и дети детей его.
\vs Trb 1:3
И сказал он им: дети мои, я умираю и отправляюсь дорогою отцов моих.
\vs Trb 1:4
Увидев же там Иуду, Гада и Асира,
братьев своих, сказал им:
поднимите меня, братья, дабы говорить мне к братьям моим
и детям моим о том, что сокрыто в сердце у меня.
Ибо я отхожу от вас отныне.
\vs Trb 1:5
И поднявшись, поцеловал их и, рыдая, сказал им:
слушайте, братья мои и дети мои, внемлите Рувиму,
отцу вашему, что завещаю вам.

\vs Trb 1:6
Вот, я заклинаю вас Богом небесным, да не совершите проступка
по незнанию юности, как я предался пороку и осквернил
ложе отца моего Иакова.
\vs Trb 1:7
Говорю же вам, что заполнила болезнь великая
поясницу мою на 7 месяцев, и, если бы не просил отец
наш Иаков обо мне Господа, желал убить меня Господь.
\vs Trb 1:8
Было мне 30 лет, когда совершил я злое пред лицом Господа,
и слабость смертная охватила меня на 7 месяцев.
\vs Trb 1:9
После же этого каялся я пред лицом Господа 7 лет,
ибо возжелала того душа моя.
\vs Trb 1:10
И не пил я вина и сикера, и мясо не входило в уста мои,
и никакого хлеба вожделенного не пробовал я,
но пребывал в печали о согрешении моём, ибо оно было велико,
и не было подобного ему в Израиле.

\vs Trb 2:1
А теперь, выслушайте меня, дети мои, что увидел я в раскаянии моём о
7-и духах соблазна.
\vs Trb 2:2
Ибо 7 духов поставлены против человека Велиаром, и они суть источники дел юношеских.
\vs Trb 2:3
И \bibemph{иные} 7 духов даны ему по сотворении его,
дабы в них было всякое дело человеческое.
\vs Trb 2:4
Первый~--- дух жизни, на коем зиждется его существование.
Второй~--- дух зрения, из коего происходит желание.
\vs Trb 2:5
Третий~--- дух слуха, из коего происходит научение.
Четвёртый~--- дух обоняния, из коего происходит вкус
от втягивания воздуха и вдыхания.
\vs Trb 2:6
Пятый~--- дух речи, из коей происходит знание.
\vs Trb 2:7
Шестой~--- дух вкуса, из коего происходит вкушение пищи и питья,
и сила на нём зиждется, ибо в пище~---  основание силы.
\vs Trb 2:8
Седьмой~--- дух деторождения и плотского сообщения,
из коего от любви к наслаждениям происходит собрание грехов.
\vs Trb 2:9
Оттого этот дух~--- последний в сотворении и первый в юности,
ибо исполнен неразумия, он же ведет юношу, словно слепого в яму
и словно стадо к пропасти.

\vs Trb 3:1
Ко всем же этим есть ещё восьмой дух~--- дух сна,
на коем основан экстаз естества и образ смерти.
\vs Trb 3:2
С этими духами соединяется дух обмана.]
\vs Trb 3:3
Первый~--- дух блуда, содержится он в естестве и чувствах.
Второй~--- дух ненасытности желудка.
\vs Trb 3:4
Третий~--- дух борьбы, что в печени и в желчи.
Четвёртый~--- дух угождения и магии, дабы казаться прекрасным,
в чём нет никакой пользы.
\vs Trb 3:5
Пятый~--- дух гордыни, дабы похваляться и кичиться.
Шестой~--- дух лжи губительной и пристрастной,
дабы измышлять речи и скрывать дела даже от родичей и ближних.
\vs Trb 3:6
Седьмой~--- дух несправедливости, от коего воровство и грабежи
для услаждения сердца своего.
Ибо несправедливость содействует прочим духам,
когда отнимается нечто у других людей.
\vs Trb 3:7
[Со всеми же этими соседствует дух сна, восьмой дух,
он же~--- дух обмана и фантазии.]
\vs Trb 3:8
И так гибнет всякий юноша, затемняющий ум свой от истины,
не входящий в закон Бога, не слушающий наставлений отцов своих,
каков и я был в юности моей.
\vs Trb 3:9
Ныне, дети мои, возлюбите истину, и она убережет вас.
Я учу вас, внемлите словам Рувима, отца вашего.
\vs Trb 3:10
Не взирайте на женщин, не сходитесь с женщиной иного мужа,
не имейте дел ненужных с женщинами.
\vs Trb 3:11
Ибо, если бы не увидел я Баллу, когда купалась она в скрытом месте,
не впал бы я в беззаконие великое.
\vs Trb 3:12
Захватила меня мысль о наготе женской и не давала мне уснуть,
пока не совершил я мерзость.
\vs Trb 3:13
Когда Иаков, отец мой, ушёл к Исааку~--- а были мы в Гадере
близ Ефрафы в Вифлееме опьянела Балла и лежала непокрытая
в спальне.
\vs Trb 3:14
И я, вошедши и увидевши наготу её, сотворил нечестивое,
[а она не чувствовала,] и я отошёл, оставив её спящей.
\vs Trb 3:15
И тотчас ангел Божий открыл нечестивое дело моё отцу моему.
Придя, сетовал он на меня, более не прикасаясь к ней.

\vs Trb 4:1
Так не смотрите же, дети мои, на красоту женскую
и не помышляйте о делах женщин,
но живите в простоте сердечной, в страхе Господнем,
трудитесь, творя добрые дела, и отвлекаясь грамматикой,
и на пастбищах ваших дотоле, пока не даст вам Господь супругу,
какую он пожелает, дабы не претерпеть вам того же, что мне.
\vs Trb 4:2
Ибо вплоть до кончины отца моего не хватало смелости мне посмотреть
в глаза ему или говорить с кем-либо из братьев моих, из-за укоризны.
\vs Trb 4:3
И доныне мучит меня совесть из-за греха моего.
\vs Trb 4:4
И много утешал меня отец мой, и просил за меня Господа,
да отойдёт от меня гнев его, и так поступил со мной Господь.
С тех пор и поныне остерегался я и не грешил.
\vs Trb 4:5
Потому говорю вам, дети мои, сохраните всё,
что внушаю вам, и не грешите.
\vs Trb 4:6
Ибо грех блуда есть пропасть душевная, отделяющая от Бога
и приближающая к идолам, ведь он помрачает ум и помыслы
и сводит юношей в Ад прежде времени.
\vs Trb 4:7
Блуд сгубил многих, ибо стар ли кто, знатен ли, богат или беден,
одинаково порицание обретает он у сынов человеческих и даёт повод
Велиару создать преткновение ему.

\vs Trb 4:8
Слышали же вы об Иосифе, как остерегался он всякой женщины
и хранил помыслы в чистоте ото всякого блуда и обрёл
благодать у Господа и у человеков.
\vs Trb 4:9
А ведь многое творила ему Египтянка, и колдунов призывая
и снадобье ему поднося, но не впал помысел души его
в вожделение злое.
\vs Trb 4:10
За это избавил его Бог отцов наших от всякого зла видимой
и таящейся смерти.
\vs Trb 4:11
Если же не овладеет блуд помыслами вашими, не сможет одолеть вас и Велиар.

\vs Trb 5:1
Злы женщины, дети мои, и, не имея власти и силы над мужами,
коварно действуют своими чарами, дабы привлечь их к себе.
\vs Trb 5:2
Кого же такими чарами не могут приворожить, того обманом покоряют.
\vs Trb 5:3
Говорил же мне ангел Божий и учил меня, что уступают женщины тому духу
блуда больше, нежели мужи.
И замышляют они в сердце своём против мужей,
и украшениями соблазняют помыслы их,
и через очи подсыпают им яд, и так порабощают их.
\vs Trb 5:4
Ибо не может женщина прямо принудить мужа,
но совершает это злодейство своими чарами блудными.
\vs Trb 5:5
Итак, убегайте блуда, дети мои,
и приказывайте женам вашим и дочерям вашим,
да не украшают голов и лиц своих для обмана здравых помыслов мужчин.
Ибо всякая женщина, прибегающая к этим ухищрениям,
обречена на муку вечную.
\vs Trb 5:6
Ибо так обольстили они Стражей, бывших до
Катаклизма\fnote{Катаклизма}{Потопа или расстворения предыдущего мира.}.
Те постоянно смотрели на них, и возжелали их,
и замыслили дело: приняв человеческое обличье, сошлись с
женщинами в образе мужей их.
\vs Trb 5:7
А те, вообразив в вожделении своем, породили Гигантов,
ведь показались им Стражи достигающими небес.

\vs Trb 6:1
Так остерегайтесь же блуда.
И если желаете очистить разум, то сдерживайте чувства
свои от женщин.
\vs Trb 6:2
А женщинам внушайте не иметь общения с мужами,
дабы и они очищали \bibemph{свой} разум.
\vs Trb 6:3
Ибо постоянное общение,
если и не совершится нечестивое,
для них есть болезнь неисцелимая,
для нас же погибель Велиарова и позор вечный.
\vs Trb 6:4
Нет ни совести, ни благочестия в блуде,
и всякая ревность живет в вожделении его.
\vs Trb 6:5
Потому и говорю вам:
будете вы ревновать и стремиться превзойти сыновей Левия,
но не сможете.
\vs Trb 6:6
Потому что Бог отомстит за них, вы же умрёте смертью злою.
\vs Trb 6:7
Ибо Левию дал Бог власть
[и с ним Иуде, и мне, и Дану, и Иосифу, дабы мы были вождями].
\vs Trb 6:8
Посему завещаю вам слушать Левия, ибо он познает закон Господа
и установит суд и будет приносить жертвы за Израиля вплоть
до конца времен~--- первосвященник помазанный,
коего призвал Господь.
\vs Trb 6:9
Хочу, чтобы поклялись вы Богом небесным, что будете творить правду,
каждый ближнему своему, и иметь любовь, каждый к брату своему.
\vs Trb 6:10
А к Левию подойдите в смирении сердца вашего,
да примете благословение из уст его.
\vs Trb 6:11
Ибо он благословит Израиля и Иуду,
ибо в нём избрал Господь царствовать над всеми народами.
\vs Trb 6:12
И поклонитесь семени его, ибо за вас будет умирать оно в
войнах зримых и незримых.
И будет он над вaми царём вечным.

\vs Trb 7:1
И умер Рувим, завещав это сыновьям своим.
\vs Trb 7:2
И положили его во гроб, а после вынесли его из Египта
и погребли в Хевроне в пещере, где погребён был и отец его.

\bibbookdescr{Tsm}{
  inline={Завещание Симеона,\\второго сына Иакова и Лии\fns{В греч. тексте $+$ ``о зависти''.}},
  toc={Завещание Симеона},
  bookmark={Завещание Симеона},
  header={Завещание Симеона},
  abbr={Сим}
}
\vs Tsm 1:1
Список слов Симеона, речённых им к сыновьям его перед тем,
как умер он в 120-ый год жизни своей,
в тот же год, что и брат его Иосиф.
\vs Tsm 1:2
Когда занемог Симеон, пришли проведать его дети его, и, сделав
усилие, сел он, поцеловал их и сказал:
\vs Tsm 2:1
послушайте, дети мои, Симеона, отца вашего; возвещу вам то,
что имею я в сердце моём.

\vs Tsm 2:2
Родился я от Иакова и был вторым сыном отца моего, и Лия, мать моя,
нарекла меня Симеоном, ибо услышал Господь мольбу ее.
\vs Tsm 2:3
Сделался я весьма сильным, не боялся труда и не страшился никакого дела.
\vs Tsm 2:4
Ибо сердце моё было сухим, печень моя недвижимой,
а внутренности мои нечувствительными.
\vs Tsm 2:5
Ведь и мужество даётся от Всевышнего людям в душах и телах.
\vs Tsm 2:6
Во время юности моей завидовал я сильно Иосифу,
ибо возлюбил его отец мой более всех.
\vs Tsm 2:7
И утвердился я против него в сердце моём, возжелав убить его,
так как Князь обмана и дух зависти ослепили мне ум, и забыл я,
что это брат мой, и не пощадил отца моего Иакова.
\vs Tsm 2:8
Но Бог его и Бог отцов наших послал ангела своего и избавил
Иосифа от рук моих.

\vs Tsm 2:9
Ибо, когда я отправился в Сиким, чтобы принести притирание для стада,
а Рувим  в Дофаим, где было необходимое нам и все хранилища наши,
Иуда, брат мой, продал Иосифа Измаильтянам.
\vs Tsm 2:10
Рувим, услышав об этом, опечалился, ибо он хотел отвести его к отцу.
\vs Tsm 2:11
Я же, услышав это, сильно разгневался на Иуду,
ибо он отпустил Иосифа живым,
и 5 месяцев пребывал я в гневе на него.
\vs Tsm 2:12
И сковал меня Господь и удалил от меня дело рук моих,
ибо правая рука моя стала наполовину сухой на 7 дней.
\vs Tsm 2:13
И познал я, дети, что из-за Иосифа случилось это со мною.
И, раскаявшись, заплакал я и молил Господа Бога,
чтобы восстановилась рука моя и удержался я от всякой скверны
и зависти и ото всякого безрассудства.
\vs Tsm 2:14
Ибо понял я, что злое дело замыслил перед лицом Господа и Иакова,
отца моего, против Иосифа, брата моего, позавидовав ему.

\vs Tsm 3:1
Ныне, дети мои, послушайте меня и остерегитесь духа обмана и зависти.
\vs Tsm 3:2
Ведь зависть властвует надо всем помыслом человека
и не дает ему ни есть, ни пить, ни делать ничего доброго.
\vs Tsm 3:3
Но всечасно подстрекает она убить того, кому человек завидует,
но тот всечасно процветает, а завистник чахнет.
\vs Tsm 3:4
И вот, 2 года сокрушал я в страхе Господнем душу мою постом.
И узнал я, что избавление от зависти происходит через страх Божий.
\vs Tsm 3:5
Если кто прибегает к Господу, оставляет его злой дух
и становится разум лёгким.
\vs Tsm 3:6
И наконец, начинает он сочувствовать тому, кому завидовал, и
примиряется с любящими его, и так избавляется от зависти.

\vs Tsm 4:1
Спросил отец мой, что со мною, ибо заметил меня скорбящим, и
сказал я ему, что переполняется печень моя.
\vs Tsm 4:2
Ибо печалился я чрезвычайно, что виновен в продаже Иосифа.
\vs Tsm 4:3
И когда пошли мы в Египет и связали меня как соглядатая,
познал я, что справедливо страдаю и не опечалился.
\vs Tsm 4:4
Иосиф же был добрый муж, дух Божий в себе имевший,
милостивый и сострадательный; не вспомнил мне зла, но
возлюбил меня с братьями моими.

\vs Tsm 4:5
Так остерегайтесь же, дети мои, всякой ревности и зависти и живите в
простоте сердечной, чтобы дал и вам Бог милость и славу и благословение на
головы ваши, как вы видите то на Иосифе.
\vs Tsm 4:6
Ни в какой день не стыдил он нас за дело это,
но возлюбил нас как душу свою, и более сыновей своих почтил нас,
и богатство, и скот, и плоды даровал нам.

\vs Tsm 4:7
И вы, дети мои, возлюбите каждый брата своего в доброте сердечной,
и отойдёт от вас дух зависти.
\vs Tsm 4:8
Ибо озлобляет он душу и губит тело, гнев и вражду вводит в
помышление и побуждает к крови и вводит разум в экстаз,
и смятение создает в душе и дрожь в теле.
\vs Tsm 4:9
Даже во сне злая зависть, соблазняя человека,
пожирает его и духами злыми возмущает душу его,
и заставляет тело его содрогаться,
и смятением лишает сна ум его,
и как дух злой и губительный является людям.
\vs Tsm 5:1
Оттого Иосиф был прекрасен лицом и приятен видом своим,
что не поселялось в нем ничто злое;
ибо смущение духа проступает явно на лице человека.

\vs Tsm 5:2
Ныне, дети мои, смягчите сердце ваше пред Господом
и выпрямите пути ваши пред людьми,
и стяжаете благодать пред лицом Господа и людей.
\vs Tsm 5:3
И остерегайтесь блуда, ибо блуд порождает всякое зло,
отдаляя от Бога и приближая к Велиару.
\vs Tsm 5:4
Видел я в книге Еноха, что сыновья ваши совратятся
от блуда и обиду нанесут мечом своим сыновьям Левия.
\vs Tsm 5:5
Но не смогут они противостоять Левию,
ибо поведёт он брань Господню и одолеет всякое войско ваше.
\vs Tsm 5:6
И будут они малочисленны, разделенные в Левин и в Иуде, и
не будет из вас никого, кто властвовал бы,
как и пророчествовал отец наш в благословениях своих.

\vs Tsm 6:1
И вот, сказал я вам всё, дабы оправдать себя от греха вашего.
\vs Tsm 6:2
И если удалите от себя зависть и всякое жестокосердие,
словно роза расцветут кости мои в Израиле,
и словно лилия плоть моя в Иакове,
и будет благоухание моё словно аромат Ливана,
и умножатся святые от меня во веки веков,
и взрастут отрасли их.
\vs Tsm 6:3
Тогда погибнет семя Ханаана,
и не будет остатка у Амалика,
и сгинут все Каппадокийцы,
и все Хетты истребятся.
\vs Tsm 6:4
Тогда угаснет земля Хама, и погибнет весь народ.
Тогда почиет вся земля от смуты, и всё, что под небесами, от войны.
\vs Tsm 6:5
Тогда прославится Сим,
ибо Господь Бог Израиля придет на землю [как человек] и тем
спасёт Адама.
\vs Tsm 6:6
Тогда предан будет всякий дух соблазна на поругание,
и люди обретут власть над злыми духами.
\vs Tsm 6:7
Тогда воскресну и я в радости и благословлю Всевышнего ради чудес его,
[ибо Господь, приняв тело и вкусив пищу с людьми, спас людей.]

\vs Tsm 7:1
Ныне, дети мои, слушайте Левия и Иуду,
и не восставайте на два эти колена,
ибо от них исполнится нам спасение Божие.
\vs Tsm 7:2
Ибо восстанет Господь из Левия как Первосвященник,
а из Иуды как Царь [Бог и человек].
Он спасёт [все народы и] род Израиля.
\vs Tsm 7:3
Для того внушаю вам это, дабы и вы внушили детям вашим,
да сохранят всё в поколениях своих.

\vs Tsm 8:1
Завершил Симеон наставление сыновей своих и почил с отцами
своими, будучи 120-и лет.
\vs Tsm 8:2
И положили его во гроб деревянный,
чтобы отнести кости его в Хеврон.
И отнесли их втайне, пока Египтяне вели войну.
\vs Tsm 8:3
Ибо кости Иосифа сохранили Египтяне в гробнице царей.
\vs Tsm 8:4
Сказали им прорицатели, что, если вынесут кости Иосифа,
тьма и мрак будут по всей земле и несчастье великое Египтянам,
так что и со светильником не узнает никто брата своего.

\vs Tsm 9:1
И оплакали сыновья Симеона, отца своего.
И пребывали в Египте вплоть до дней, когда Моисей вывел их рукою своею.

\bibbookdescr{Tlv}{
  inline={Завещание Левия,\\третьего сына Иакова и Лии\fns{В греч. тексте $+$ ``о священстве и о гордыне''.}},
  toc={Завещание Левия},
  bookmark={Завещание Левия},
  header={Завещание Левия},
  abbr={Лви}
}
\vs Tlv 1:1
Список слов Левия, которые говорил он сыновьям своим обо всём,
что они совершат и что случится с ними вплоть до дней Суда.
\vs Tlv 1:2
Он был здоров, когда призвал их к себе;
открылось же ему, что должен он вскоре умереть.
И когда собрались они, сказал к ним:

\vs Tlv 2:1
Я, Левий, родился в Хевроне и пришёл с отцом моим в Сиким.
\vs Tlv 2:2
Не было мне ещё двадцати лет,
когда сотворил я возмездие Еммору вместе с Симеоном за сестру нашу Дину.
\vs Tlv 2:3
И когда я пас стадо в Авелмехоле,
дух познания Господа сошел на меня,
и узрел я всех людей, уклонившихся от пути своего,
и грех выстроил себе дом на стенах, а неправедность восседала на башнях.
\vs Tlv 2:4
И был я в скорби о роде сынов человеческих и молил Господа,
чтобы спас меня.
\vs Tlv 2:5
Тогда снизошел на меня сон, и увидел я гору высокую и сам был на ней.

\vs Tlv 2:6
И вот, разверзлись небеса, и ангел Господень сказал мне:
Левий, Левий, войди!
\vs Tlv 2:7
И взошёл я на первое небо и узрел там великую воду висящую.
\vs Tlv 2:8
И ещё увидел я второе небо, много более светлое и сияющее,
высота же его была бесконечной.
\vs Tlv 2:9
И сказал я ангелу: что это такое?
И отвечал мне ангел: не удивляйся тому, ибо иное небо узришь,
более светлое и несравненное.
\vs Tlv 2:10
И поднявшись туда, встанешь ты рядом с Господом, и слугой ему будешь,
и тайны его возвестишь людям, и о грядущем избавлении Израиля возгласишь.
\vs Tlv 2:11
И через тебя и через Иуду явится Господь людям,
чтобы спасти собою весь род человеческий.
\vs Tlv 2:12
И жизнь твоя~--- удел Господа,
и будет он тебе полем и виноградником, и плодом, и золотом, и серебром.

\vs Tlv 3:1
Так услышь о показанных тебе небесах.
Нижнее оттого мрачно на вид, что зрит оно нечестия людские.
\vs Tlv 3:2
И имеет оно. огонь, снег и лед,
уготовленные на день Суда Божией справедливостью.
В нём~---  все духи воздаяний для возмездия людям.
\vs Tlv 3:3
На втором же небе~--- силы войск, построенных на день Суда,
дабы воздать духам соблазна и Велиара, а на них~--- святые.
\vs Tlv 3:4
В высшем же из всех пребывает великая слава,
превосходящая всякую святость.
\vs Tlv 3:5
В следующем же небе~--- архангелы,
служащие Господу и умилостивляющие его
ко всякому неведению праведных.
\vs Tlv 3:6
Подносят они Господу ароматы благоуханные,
жертву мысленную и незапятнанную кровью.
\vs Tlv 3:7
В том же небе, что за ним,~--- ангелы,
несущие молитвы ангелам о лице Божием.
\vs Tlv 3:8
В следующем за ним~--- престолы и власти,
коими хвалебная песнь Богу воспевается.

\vs Tlv 3:9
Когда же смотрит на нас Господь,
все мы дрожим, а небо и земля
и бездна от лица величия его сотрясаются.
\vs Tlv 3:10
Сыны же человеческие, не чувствующие того,
согрешают и гневят Всевышнего.
\vs Tlv 4:1
Познай же ныне, что сотворит Господь Суд над сынами человеческими.
Когда скалы рухнут, и солнце погаснет, и воды высохнут,
и огонь затаится, и всякое творение смутится,
и незримые духи истощатся, и Ад лишится защиты своей
[от страдания Всевышнего],
тогда люди утратят веру и будут упорствовать в неправедности своей,
и за то будут судимы и примут кару.

\vs Tlv 4:2
И услышал Всевышний молитву твою, да избавит тебя от неправедности
и сделает сыном своим, и рабом, и слугою пред лицом его.
\vs Tlv 4:3
Светом знания просияешь ты в Иакове,
и будешь как солнце всему семени Израиля.
\vs Tlv 4:4
И дастся тебе благословение и всему семени твоему дотоле,
пока не посетит Господь все народы по милосердию своему,
во веки веков.
[Но только сыновья твои возложат руки на него, дабы распять его.]

\vs Tlv 4:5
И для того даны тебе совет и знание,
чтобы наставил ты сыновей своих в этом.
\vs Tlv 4:6
Ибо благословляющие тебя благословенны будут,
а проклинающие тебя погибнут.

\vs Tlv 5:1
И вслед за тем открыл мне ангел врата небесные,
и увидел я Святого Всевышнего, восседающего на престоле.
\vs Tlv 5:2
И сказал он мне: Левий, тебе дал я благословение на священство,
доколе не приду и не поселюсь среди Израиля.
\vs Tlv 5:3
И тогда свёл меня ангел на землю и дал мне оружие и меч и сказал мне:
сотвори месть Сихему за Дину, сестру твою, и я буду с тобой,
ибо Господь послал меня.
\vs Tlv 5:4
И погубил я в то время сынов Еммора, как написано на скрижалях отцов.
\vs Tlv 5:5
И сказал ему: прошу тебя, господи, научи меня имени твоему,
дабы призывать мне его в день скорби.
\vs Tlv 5:6
И отвечал он: я ангел, просящий за народ Израилев,
да знает, что не сокрушится он.
\vs Tlv 5:7
И я, проснувшись, благословил Всевышнего.

\vs Tlv 6:1
И тогда пошел я к отцу моему, обрел бронзовый щит, отчего и имя
горы~--- Щит, что близ Гевала одесную Авимы.
\vs Tlv 6:2
И сохранил я слова те в сердце моём.
\vs Tlv 6:3
И совещался я с отцом моим и Рувимом, дабы сказать сынам Еммора,
чтобы приняли они обрезание, ибо пылал я рвением из-за мерзости,
которую сотворили они над сестрою моей.
\vs Tlv 6:4
И я убил первым Сихема, а Симеон~--- Еммора.
\vs Tlv 6:5
А вслед за тем пришли братья мои и перебили город тот остриём меча.
\vs Tlv 6:6
И услышал о том отец мой и, разгневавшись, огорчился,
что приняли они обрезание и умерли,
и в благословениях своих обошел нас.

\vs Tlv 6:7
В том согрешили мы, что сотворили это против воли его, а он болен
был в тот день.
\vs Tlv 6:8
Но я видел, что воля Божия была во зло Сикимам,
так как они хотели и Сарре и Ревекке сделать то,
что сделали Дине, сестре нашей,
но воспрепятствовал им Господь.
\vs Tlv 6:9
И преследовали они Авраама, отца нашего, бывшего чужеземцем,
и изнуряли скот беременный,
и Евлаю, родившуюся в доме Авраама, жестоко оскорбляли.
\vs Tlv 6:10
И так делали они всем чужеземцам,
силою похищая жен их и принуждая их.
\vs Tlv 6:11
И настиг их, наконец, гнев Божий.

\vs Tlv 7:1
И сказал я отцу моему Иакову:
тобою уничтожит Господь Хананеев
и даст землю их тебе и семени твоему после тебя.
\vs Tlv 7:2
Отныне назовутся Сикимы городом глупцов.
Ибо как смеются над глупцами, так посмеёмся и мы над ними.
\vs Tlv 7:3
Безумие сотворили они в Израиле, осквернив сестру мою.
И, встав, пошли мы в Вефиль.

\vs Tlv 8:1
И снова узрел я видение, подобное прежнему,
после того как были мы здесь 70 дней.
\vs Tlv 8:2
И узрел я семерых мужей в белых одеждах, говорящих мне:
восстав, облачись в одеяния священства,
и венец праведности,
и наперсник знания,
и подир правды,
и дощечку веры,
и митру главы,
и ефод пророчества.
\vs Tlv 8:3
И каждый из них нёс нечто, вручал мне и говорил мне:
отныне стань священником, и ты, и всё семя твое.
\vs Tlv 8:4
И первый помазал меня елеем святым и дал мне жезл.
\vs Tlv 8:5
А второй омыл меня водою чистой, и дал мне вкусить хлеба и вина,
и облачил меня в одеяние святое и славное.
\vs Tlv 8:6
Третий же облачил меня в виссон, подобный ефоду.
\vs Tlv 8:7
Четвёртый же надел на меня пояс, подобный порфире.
\vs Tlv 8:8
Пятый же дал мне ветвь тучной оливы.
\vs Tlv 8:9
Шестой надел мне на голову венец.
\vs Tlv 8:10
Седьмой надел мне диадему священства и наполнил руки мои фимиамом,
дабы служил я священником Господу Богу.
\vs Tlv 8:11
И говорят мне: Левий, разделится семя твоё на три чина в знак славы
Господа грядущего.
\vs Tlv 8:12
И будет первый жребий велик, и над ним не явится другого.
\vs Tlv 8:13
Второй будет жребий священства.
\vs Tlv 8:14
Третьему наречено будет новое имя, ибо восстанет
царь от Иуды и сотворит новое священство
по образу народов для всех народов.
\vs Tlv 8:15
И обретёт любовь явление его, ибо он будет
пророком Всевышнего от семени Авраама, отца вашего.
\vs Tlv 8:16
И всё желанное в Израиле твоё будет и семени твоего, и семя твоё
вкушать будет всё прекрасное видом, и трапезу Господа разделит.
\vs Tlv 8:17
И будут из них священники и судьи, и книжники, и на устах у них святое будет.
\vs Tlv 8:18
И очнувшись от сна, понял я, что этот сон подобен первому.
\vs Tlv 8:19
И скрыл это в сердце моём, и не возвестил о том ни одному человеку на земле.

\vs Tlv 9:1
Спустя же два дня пришли я, Иуда и отец наш Иаков к Исааку, праотцу нашему.
\vs Tlv 9:2
И благословил меня отец отца моего по видениям, которые видел я.
И не пожелал он отправиться с нами в Вефиль.
\vs Tlv 9:3
Когда же пришли мы в Вефиль, увидел отец мой Иаков видение обо мне,
что буду я у них священником.
\vs Tlv 9:4
И восстав наутро, принёс через меня Господу десятину от всего.
\vs Tlv 9:5

И так пришли мы в Хеврон, чтобы пребывать там.
\vs Tlv 9:6
И постоянно призывал меня Исаак, дабы наставлять меня в законе Господа, как и
явил мне ангел.
\vs Tlv 9:7
И учил он меня закону священства, жертвоприношений, всесожжении,
первенцев от плодов, жертв доброхотных и искупительных.
\vs Tlv 9:8
И каждый день наставлял он меня, и занят был со мною, и говорил мне:
\vs Tlv 9:9
удерживай себя от духа блуда, ибо он продолжителен и осквернит
святое через семя твое.
\vs Tlv 9:10
Потому возьми себе жену, ещё будучи молод,
чтобы не было на ней позора и скверны,
и не из рода иноплеменных народов.
\vs Tlv 9:11
И прежде чем войти в святое место, соверши омовение;
и когда приносишь жертву, омойся;
и закончив жертвоприношение, также омойся.
\vs Tlv 9:12
И 12 деревьев, имеющих листья, принеси Господу,
как учил и меня Авраам.
\vs Tlv 9:13
И от всякого животного чистого и пернатого принеси жертву Господу.
\vs Tlv 9:14
И от всех первых плодов и вина принеси первины в жертву Господу Богу.
И осоли всякую жертву солью.

\vs Tlv 10:1
Ныне, сохраните то, что завещаю вам, дети,
ибо услышанное мною от отцов наших возвестил вам.
\vs Tlv 10:2
И вот, неповинен я в нечестии вашем и в преступлениях,
которые совершите вы в конце веков [против Спасителя мира Христа],
соблазняя Израиль и навлекая на него беды всякие от Бога.
\vs Tlv 10:3
И сотворите вы беззакония в Израиле, так что не вынесет
Иерусалим злых дел ваших,
но порвётся завеса в Храме и не скроет непристойности вашей.
\vs Tlv 10:4
И рассеетесь вы пленниками среди народов и будет
там позор и проклятие на вас.
\vs Tlv 10:5
Ибо дом, который изберёт Господь,
Иерусалимом наречётся, как сказано в книге Еноха праведного.

\vs Tlv 11:1
Когда же я взял себе жену, было мне 28 лет;
ей было имя Мелха. 
\vs Tlv 11:2
И зачала она, и родила сына, и нарекли ему имя Гирсон,
ведь были мы в чужой земле.
\vs Tlv 11:3
И увидел я, что не быть ему среди первых.
\vs Tlv 11:4
Кааф же родился в 35-ый год жизни моей,
и было то при восходе солнца.
\vs Tlv 11:5
И узрел я в видении:
стоял он в вышних посреди собрания.
\vs Tlv 11:6
Оттого нарек я ему имя Кааф,
[что значит начало великих дел и наставление].
\vs Tlv 11:7
И 3-го сына родила мне на 40-ом году жизни моей, и оттого,
что страдала в родах его, нарек я его Мерари, что значит огорчение.
\vs Tlv 11:8
Иохаведа же родилась в Египте на 64-ом году моем:
ибо был я во славе между братьев моих.

\vs Tlv 12:1
И взял Гирсон жену и родил от нее Ливни и Шимеи.
\vs Tlv 12:2
Сыновья же Каафовы суть Амрам, Ицгар, Хеврон и Узиил.
\vs Tlv 12:3
А сыновья Мерари суть Махли и Муши.
\vs Tlv 12:4
На 94-ом же году моём взял Амрам Иохаведу,
дочь мою, себе в жёны, ибо в один день родились он и дочь моя.
\vs Tlv 12:5
8-и лет был я, когда вошли мы в землю Ханаанскую,
18-и лет, когда убил я Сихема;
с 19-и лет был я священником,
в 28 лет взял я жену,
и 48-и вошёл я в Египет.
\vs Tlv 12:6
И вот, дети мои, вы~--- третье поколение.
\vs Tlv 12:7
Иосиф умер, когда было мне 118 лет.

\vs Tlv 13:1
Ныне, дети мои, завещаю вам:
бойтесь Господа Бога вашего всем сердцем вашим,
и живите в простоте по всем законам его.
\vs Tlv 13:2
И учите детей ваших грамоте,
дабы имели они знание во всю жизнь свою,
читая постоянно закон Божий.
\vs Tlv 13:3
Ибо всякий, кто познает закон Господа,
почитаем будет, и не примут его как чужого,
куда бы ни пришел он.
\vs Tlv 13:4
И многих друзей, б\acc{о}льших, нежели родители, обретет он,
и возжелают многие из людей служить ему и слушать закон из уст его.

\vs Tlv 13:5
Творите же справедливость, дети мои, на земле, да обретёте её на небесах.
\vs Tlv 13:6
И сейте в душах ваших доброе, и обретёте его в жизни вашей;
если же посеете злое, всякую смуту и скорбь пожнёте.
\vs Tlv 13:7
Мудрость обретёте вы в страхе Божием, ибо если придёт пленение
и уничтожатся города, и земли, и золото, и серебро,
и всякое имущество погибнут,
то мудрости у мудрого никто не сможет отнять,
разве только ослепление нечестия и ожесточение греха.

\vs Tlv 13:8
Если кто убережет себя от злых этих дел,
то будет у него мудрость, и для неприятелей~--- сияющая,
и в чужой земле родина,
и среди врагов друга даст ему.
\vs Tlv 13:9
Всякий, кто учит добру и творит добро,
воссядет на престоле рядом с царями, как Иосиф, брат мой.

\vs Tlv 14:1
Познал я, дети мои, из писаний Еноха,
что в конце веков согрешите вы против Господа,
наложив руки [на Него] и у всех народов будете посмешищем.
\vs Tlv 14:2
А ведь отец наш Израиль чист от нечестия первосвященников
[которые возложат руки свои на Спасителя мира].
\vs Tlv 14:3
Как чисто солнце над землёй пред лицом Господа, так и вы
будьте светочами Израиля надо всеми народами.
\vs Tlv 14:4
И если вы помрачитесь нечестием, что тогда делать народам,
в слепоте пребывающим?
И навлечёте вы проклятие на род ваш за то,
что свет закона, данный вам для просвещения всякого человека,
его захотите вы убить, уча заповедям, которые противны законам Божиим.

\vs Tlv 14:5
Приношения Господу расхитите, и от частей его украдёте отборные,
и пожрёте их дерзко с блудницами.
\vs Tlv 14:6
И заповедям Господним учить станете из алчности,
и замужних женщин оскверните,
и с блудницами и с прелюбодейками осквернитесь,
дочерей же язычников возьмёте в жёны,
и будет смешение ваше подобно Содому и Гоморре.
\vs Tlv 14:7
И возгордитесь вы в священстве вашем, вознесясь над людьми,
и не только над ними, но и над заповедями Божиими.
\vs Tlv 14:8
Ибо презрите вы святое, ругаясь и насмехаясь.
 
\vs Tlv 15:1
Оттого Храм, избранный Господом, запустеет в нечистоте вашей,
а вы пленниками будете у всех народов.
\vs Tlv 15:2
И мерзостью будете для них, и срам стяжаете и позор вечный
от правосудия Божия.
\vs Tlv 15:3
И все ненавидящие вас возрадуются погибели вашей.
\vs Tlv 15:4
И если не обретёте милости через Авраама, Исаака и Иакова,
отцов ваших, ни единого из семени вашего не останется на земле.

\vs Tlv 16:1
И ныне познал я, что 70 седмин пребудете вы в заблуждении
и станете осквернять священство и жертвенники пятнать.
\vs Tlv 16:2
И закон отвергнете, и речи пророков уничтожите в совращении злом.
Преследовать будете вы мужей справедливых, и благочестивых возненавидите,
а словами правдивыми гнушаться станете.
\vs Tlv 16:3
А человека, обновляющего закон силою Всевышнего,
в обмане обвините, и затем и подниметесь, чтобы убить его, не зная,
что восстанет он, и во злобе вашей примете кровь его невинную на головы ваши.
\vs Tlv 16:4
Говорю же вам, что из-за того запустеют святыни ваши до основания.
\vs Tlv 16:5
И не будет чисто место ваше, но будете прокляты
и рассеяны среди народов дотоле, пока не явится он вновь,
и не смилуется, и не примет вас к себе.

\vs Tlv 17:1
И как услышали вы о семидесяти седминах, услышьте и о
священстве.
\vs Tlv 17:2
Ибо каждый юбилей будет священство.
И в первый юбилей первый помазанный
на священство велик будет и станет говорить с Богом как с отцом,
и священство его наполнится Господом, и во дни радости его для спасения мира
он воскреснет.
\vs Tlv 17:3
Во второй же юбилей помазанный взят будет в печали возлюбленного,
и будет священство его почтено и превыше всего прославится.
\vs Tlv 17:4
Третий же священник скорбью объят будет.
\vs Tlv 17:5
Четвёртый же в страданиях будет,
ибо множество несправедливости поднимется против него,
и во всём Израиле возненавидит каждый ближнего своего.
\vs Tlv 17:6
Пятый тьмою будет объят.
\vs Tlv 17:7
Так же~--- и шестой, и седьмой.
\vs Tlv 17:8
В седьмой же юбилей будет мерзость,
которой не могу высказать пред лицом людей,
ибо тогда узнают, как творить её.
\vs Tlv 17:9
Оттого пленены будут и ограблены, и исчезнет земля их и само бытие их.
\vs Tlv 17:10
В пятую же седмину вернутся они в землю опустошения
их и возобновят Дом Господень.
\vs Tlv 17:11
В седьмую же седмину обретут они священников,
которые будут идолопоклонники, любостяжатели, гордецы,
беззаконники, нечестивцы, растлители детей и скотоложцы.

\vs Tlv 18:1
И когда придёт отмщение им от Господа,
исчезнет священство.
\vs Tlv 18:2
Тогда восставит Господь священника нового,
которому все слова Господа откроются,
и сам будет вершить он суд правды на земле множество дней.
\vs Tlv 18:3
И взойдёт на небесах звезда его, словно царская,
свет знания несущая, словно свет солнца, и возвеличится во вселенной.
\vs Tlv 18:4
Озарит она землю, словно солнце, и истребит всякий мрак из
поднебесной, и настанет мир на всей земле.
\vs Tlv 18:5
Небеса возвеселятся во дни его,
и земля возрадуется, и облака возликуют,
[и знание Господне прольется на землю, как вода морская,]
и ангелы славы лика Господня возрадуются ему.
\vs Tlv 18:6
Небеса разверзнутся, и из Храма Славы сойдёт
на него святость с голосом Отцовым, словно голос Авраама к Исааку.
\vs Tlv 18:7
И прольётся на него слава Всевышнего,
и дух знания и святости почиет на нём [в воде].
\vs Tlv 18:8
Ибо он даст величие Господа сынам своим воистину навеки;
и не унаследует ему никто в поколениях и поколениях до века.
\vs Tlv 18:9
И в священство его народы исполнятся знанием на земле и освящены
будут благодатью Господней.
[Израиль же умалится в незнании и помрачится в скорби.]
В священство его исчезнет грех, и беззаконники перестанут творить зло.
\vs Tlv 18:10
И отверзнет он врата Рая и отвратит меч, угрожающий Адаму.
\vs Tlv 18:11
И даст он святым вкусить от Древа Жизни, и дух святости пребудет на них.
\vs Tlv 18:12
И Велиара он свяжет и даст власть детям своим попрать злых духов.
\vs Tlv 18:13
И возрадуется Господь детям своим, и благоволить будет возлюбленным
его до века.
\vs Tlv 18:14
Тогда возвеселятся Авраам, Исаак и Иаков, и я возрадуюсь,
и все святые облекутся радостью.

\vs Tlv 19:1
Ныне же, дети мои, всё вы слышали.
Изберите себе либо свет, либо тьму;
либо закон Господа, либо дела Велиара.
\vs Tlv 19:2
И отвечали ему сыновья его, говоря:
пред лицом Господа будем жить мы, и по закону его.
\vs Tlv 19:3
И сказал им отец их: свидетель Господь, и свидетели ангелы его,
и свидетели вы, и свидетель я речам уст ваших.
И сказали ему сыновья его: свидетели.

\vs Tlv 19:4
Так окончил Левий завещание сыновьям своим, и вытянул ноги свои на
ложе, и приложился к отцам своим, прожив 137 лет.
\vs Tlv 19:5
И положили его во гроб и после погребли его в Хевроне с
Авраамом, Исааком и Иаковом.

\bibbookdescr{Tju}{
  inline={Завещание Иуды,\\четвёртого сына Иакова и Лии},
  toc={Завещание Иуды},
  bookmark={Завещание Иуды},
  header={Завещание Иуды},
  abbr={Ида}
}
\vs Tju 1:1
Список слов Иуды, кои сказал он сыновьям своим, прежде чем умереть.
\vs Tju 1:2
Собравшись, пришли они к нему, и сказал он им:
\vs Tju 1:3
внемлите, дети мои, Иуде, отцу вашему.
Четвёртым сыном был я у отца моего Иакова, и Лия,
мать моя, нарекла меня Иудой, говоря: благодарю Господа за то,
что дал он мне и четвёртого сына.

\vs Tju 1:4
Смышлён был я в юности моей и слушался каждого слова отца моего.
\vs Tju 1:5
И чтил я мать мою и сестру матери моей.
\vs Tju 1:6
И когда настала пора зрелости моей,
благословил меня отец мой, говоря:
царем будешь ты, благим путем идущим во всем.

\vs Tju 2:1
И дал мне Господь милость во всех делах моих: в поле и в доме.
\vs Tju 2:2
Помню, что гнался я за оленем, и взял его,
и приготовил его в пищу отцу моему, и ел он.
\vs Tju 2:3
И серну одолел я в беге, и всё, что было на равнине, ловил я.
\vs Tju 2:4
Льва убил я и спас козленка из пасти его.
Медведицу поймал я за лапы, и бросил её в пропасть, и разбилась она.
\vs Tju 2:5
Дикую свинью нагнал я, и схватил на бегу, и растерзал её.
\vs Tju 2:6
Барс в Хевроне напал на пса моего, и я схватил барса за хвост,
и бросил его о скалу, и разбился он надвое.
\vs Tju 2:7
Быка дикого нашёл я, пасшегося в поле, и за рога поймал его,
и по кругу прогнав его, и помрачив зрение его, бросил и убил его.

\vs Tju 3:1
Когда же пришли два царя Ханаанских вооруженных к пастбищам нашим,
и народ многочисленный с ними, подбежал я один к царю одному,
и, ударив его по голеням, убил его.
\vs Tju 3:2
Другого же царя, Таппуаха, сидящего на коне
[убил я и тем весь народ его рассеял.
\vs Tju 3:3
И царя Ахора,] мужа огромного роста нашёл я,
стрелявшего из лука вперед и назад,
и поднял я камень в 60 фунтов, бросил его в коня и убил его.
\vs Tju 3:4
[И бился я с Ахором 2 часа, и убил его, и рассёк щит его на две части,
и отсёк ноги его.]
\vs Tju 3:5
Когда же снимал я панцирь его, вот, 8 мужей, бывших с ним,
сразились со мною.
\vs Tju 3:6
Намотал я одежду на руку мою, и метал в них камни, как из пращи,
и 4-х убил, а остальные бежали.

\vs Tju 3:7
Отец же мой Иаков убил Велисафа, царя всех царей,
мужа огромного роста, в 12 локтей.
\vs Tju 3:8
И трепет напал на них, и перестали воевать с нами.
\vs Tju 3:9
Оттого не знал беды отец мой в войнах, что с ним был я и братья мои.
\vs Tju 3:10
Ибо узрел он в видении обо мне, что ангел силы следует за мной повсюду,
да не буду побежден.

\vs Tju 4:1
После того произошла у нас война на юге, б\acc{о}льшая бывшей в Сикиме.
И встал я рядом с братьями моими, и преследовали мы 1000-у,
и убили из них 200.
\vs Tju 4:2
И взошёл я на стены и убил царя их.
\vs Tju 4:3
Так освободили мы Хеврон, и взяли всех врагов в плен.

\vs Tju 5:1
На другой день пошли мы в Арету, город могучий и сильный,
грозящий нам смертью.
\vs Tju 5:2
Я и Гад подошли к городу с востока, а Рувим и Левий~--- с запада.
\vs Tju 5:3
И помыслили те, что были на стенах, что мы одни, и пошли на нас.
\vs Tju 5:4
И тут тайно вошли братья наши с других сторон в город.
\vs Tju 5:5
И взяли мы его острием меча, а тех, кто бежал в башню,
огнём сожгли, и так захватили всех и всё имущество их.
\vs Tju 5:6
Когда же уходили мы, мужи из Таппуаха отняли у нас добычу нашу,
и, увидев то, вступили мы в битву с ними.
\vs Tju 5:7
И перебили всех, и обратно взяли добычу.

\vs Tju 6:1
И когда были мы у вод Хозевы, пошли на нас войной люди из Иовеля.
\vs Tju 6:2
И восстав на них, обратили мы их в бегство,
и союзников их из Силома убили,
и не дали им прохода, чтобы идти на нас.
\vs Tju 6:3
И вновь пошли на нас люди из Махира на 5-ый день,
и, восстав на них с мощным ножом,
победили мы их и убили также и их прежде,
нежели выступили они в поход.
\vs Tju 6:4
Когда же подошли мы к городу их,
покатили на нас камни женщины их с высоты горы, где был город.
\vs Tju 6:5
И, укрывшись, я и Симеон сзади взошли на гору и уничтожили и этот город.

\vs Tju 7:1
А на другой день сказали нам,
что царь города Гааш с народом многочисленным идёт на нас.
\vs Tju 7:2
Тогда я и Дан, сделав вид, что мы Амореяне,
как союзники вошли в город их.
\vs Tju 7:3
И глубокой ночью пришли братья наши, мы же открыли им ворота,
и всех жителей перебили и ограбили и 3 стены их разрушили.
\vs Tju 7:4
И подошли к Фамне, где было всё хранилище их.
\vs Tju 7:5
Тут разгневали меня насмешки их, и двинулся я к ним на вершину,
а они метали в меня камни и стреляли из лука.
\vs Tju 7:6
И если бы Дан, брат мой, не вступил в бой вместе со мною, убили бы они меня.
\vs Tju 7:7
И отважно наступили мы на них, и бежали они все, и,
отойдя иным путем к отцу нашему, они умолили его, и он заключил мир с ними.
\vs Tju 7:8
И не сделали мы им никакого зла,
а сделали их данниками нашими и отдали им добытое от них.
\vs Tju 7:9
И восстановили мы города их:
я пострил Фамну, а отец мой построил Рабаэл.
\vs Tju 7:10
Было же мне 20 лет, когда совершилась война эта.
\vs Tju 7:11
И страшились Хананеи меня и братьев моих.

\vs Tju 8:1
Было у меня много скота, и имел я начальником над пастухами
Хиру Одолламитянина.
\vs Tju 8:2
Придя к нему, увидел я Варсаву, царя Одоллама.
И говорил он с нами, и устроил нам пир.
И предложил он, и дал мне в жёны дочь свою, именем Вирсавию.
\vs Tju 8:3
Она родила мне Ира, Онана и Шелу.
И двоих погубил Господь, а Шела остался жить.

\vs Tju 9:1
18 лет был мой отец в мире с братом своим Исавом,
и дети Исава с нами, после того как пришли мы из Месопотамии, от Лавана.
\vs Tju 9:2
Когда же исполнились 18 лет, пришел к нам Исав, брат отца моего,
с народом сильно вооруженным и могучим.
\vs Tju 9:3
И поразил стрелою Иаков Исава,
и тот был унесен раненым на гору Сеир и умер.
\vs Tju 9:4
И мы преследовали сыновей Исава,
а был у них город с железными стенами и медными воротами,
и не могли мы войти в него.
Окружили мы и осадили его.
\vs Tju 9:5
И когда не отворяли они нам 20 дней,
приставил я лестницу на виду у всех и,
держа щит над головой моей и сдерживая удары камней,
взошёл наверх и убил четверых могучих мужей их.
\vs Tju 9:6
Рувим и Гад убили еще шестерых.
\vs Tju 9:7
Тогда просили они нас о мире, и, посоветовавшись с отцом нашим,
приняли мы их в данники.
\vs Tju 9:8
И давали они нам 50 гомеров пшеницы, и масла 50 батов,
и вина 50 мер вплоть до голода, когда пошли мы в Египет.

\vs Tju 10:1
После того взял в жены Ир, сын мой, Фамарь из Месопотамии,
бывшую дочерью Арама.
\vs Tju 10:2
Ир же был недобрым и смущался пред Фамарью,
ибо не была она из Ханаана, и умертвил его ангел Господень.
\vs Tju 10:3
И дал я её Онану, 2-му сыну моему, и его убил Господь.
\vs Tju 10:4
Ибо он не познавал её, хотя прожил с нею год, не желая иметь детей от неё.
\vs Tju 10:5
Когда же пригрозил я ему, сошёлся он с нею,
но излил семя на землю по совету матери своей.
И от греха этого умер и он.
\vs Tju 10:6
Я же хотел дать Фамари и Шелу, но мать его не дозволила.
Злые помыслы имела она,
ибо не была Фамарь из дочерей Хананеев, как она сама.

\vs Tju 11:1
Я же знал, что злой род Хананеи, но мысли юности ослепили разум мой.
\vs Tju 11:2
И, увидев, как она разливает вино, прельстился я
и взял её без воли на то отца моего.
\vs Tju 11:3
Она же в мое отсутствие пошла и взяла Шелу жену из Ханаана.
\vs Tju 11:4
А я, узнав, что сотворила она, проклял ее в скорби души моей.
\vs Tju 11:5
И умерла она от грехов своих вслед за детьми своими.

\vs Tju 12:1
Когда овдовела Фамарь и прошло 2 года, услышала она,
что иду я стричь овец и, нарядившись в наряд свадебный,
села в городе Енаиме у ворот.
\vs Tju 12:2
Был же закон у Амореев, чтобы вдова 7 дней сидела блудницей у ворот.
\vs Tju 12:3
И я, опьянённый вином, не узнал её,
и прельстила меня красота её из-за прекрасного наряда.
\vs Tju 12:4
И свернув с дороги к ней, сказал я: войду к тебе.
А она спросила: а что ты дашь мне?
И дал я ей посох мой, и пояс, и диадему царства моего в залог.
И когда вошёл к ней, зачала она.
\vs Tju 12:5
И не зная, что сам сотворил, хотел я убить Фамарь.
Она же, послав мне тайно всё данное ей, устыдила меня.
\vs Tju 12:6
И призвав её, услышал я те слова тайные, что говорил ей,
когда возлежал с нею в опьянении моём.
И не мог убить её, ибо то было дано Господом.
\vs Tju 12:7
И сказал я: не лукавила она, взяв у другой женщины этот знак.
\vs Tju 12:8
Но не сходился я с ней более до конца жизни моей,
ибо мерзость сотворил я во всём Израиле.
\vs Tju 12:9
А жители города того говорили, что не было блудницы у ворот,
ибо она пришла из другого места и недолго сидела там.
\vs Tju 12:10
И помыслил я, что не видел никто, как вошёл я к ней.

\vs Tju 12:11
После того пошли мы в Египет к Иосифу, так как был голод.
\vs Tju 12:12
Было мне 46 лет, и 73 года провел я в Египте.

\vs Tju 13:1
Ныне завещаю вам, дети мои, послушайте Иуду, отца вашего,
и сохраните слова мои, чтобы делать всё по велениям Господа
и подчиняться заповедям его.
\vs Tju 13:2
Не идите за вожделениями вашими в гордыне сердца своего,
и не похваляйтесь делами и силой молодости вашей,
ибо это злое дело пред лицом Господа.
\vs Tju 13:3
Когда я возгордился, что в войнах не прельстило меня лицо
женщины благообразной, и позорил брата моего Рувима из-за Баллы,
женщины отца моего, тогда стал приступать ко мне дух зависти и блуда,
пока не согрешил я с Вирсавией Хананеянкой и с Фамарью,
невесткой моей.
\vs Tju 13:4
Ибо сказал я тестю моему:
посоветуюсь с отцом моим и тогда возьму дочь твою.
Он же не захотел, но показал мне золота несметное количество,
что было за дочерью его, ибо он был царь.
\vs Tju 13:5
И нарядил он её в золото и жемчуги, велел ей разливать вино на пиру.
\vs Tju 13:6
И совратило вино очи мои, и помрачило мне сердце наслаждением.
\vs Tju 13:7
И возлюбив её, возлёг с нею, и пренебрег заповедью Господа
и заповедью отца моего, и взял её в жены.
\vs Tju 13:8
И воздал мне Господь за помысел души моей, ибо не был я счастлив в детях её.

\vs Tju 14:1
И ныне говорю, дети мои: не опьяняйтесь вином,
ибо вино отвращает разум от истины,
и производит страсть вожделения,
и вводит очи в соблазн.
\vs Tju 14:2
Ведь дух блуда словно слугою имеет вино, дабы услаждать ум,
так что совращают эти два помысла человека.
\vs Tju 14:3
Ибо, если некто пьёт вино до опьянения,
мыслями нечистыми возмущает он ум свой,
и для блуда разгорячает тело свое,
дабы насладиться, и грех совершает, и не стыдится.
\vs Tju 14:4
Таково вино, дети мои, ибо не стыдится опьяневший никого.
\vs Tju 14:5
Вот, и меня оно соблазнило,
так что не устыдился я множества жителей города,
ибо на глазах у всех возлёг с Фамарью,
и совершил грех великий, и раскрыл тайну своей нечистоты сыновьям моим.
\vs Tju 14:6
Пил я вино, и не устыдился заповеди Божией, и взял в жёны Хананеянку.
\vs Tju 14:7
Ибо великое умение нужно пьющему вино, дети мои; это умение винопития,
дабы пить до того времени, пока имеет человек стыд.
\vs Tju 14:8
Когда же перейдёт он предел, входит в ум его дух соблазна
и заставляет пьяного сквернословить,
и творить беззакония, и не стыдиться бесчестия своего,
но кичиться им и мнить себя прекрасным.

\vs Tju 15:1
Блудящий наказания не чувствует и бесчестия не стыдится.
\vs Tju 15:2
Если же царь блудит, лишается он царства,
порабощённый блудом, как и я то претерпел.
\vs Tju 15:3
Отдал я посох мой, который есть опора племени моего,
и пояс мой, который есть сила,
и диадему, которая есть слава царства моего.
\vs Tju 15:4
И раскаявшись в том, не пил я вина и не вкушал мяса до старости моей,
и никакого веселья не видел.
\vs Tju 15:5
И показал мне ангел Божий, что и царем, и нищим правят женщины.
Но не в них преуспеяние жизни.
\vs Tju 15:6
У царя отнимают они славу,
у мужественного~--- силу,
а у нищего~--- самую малую опору в его нищете.

\vs Tju 16:1
Остерегайтесь же, дети мои, преступить предел, положенный вину,
ибо в нём~--- 4 злые духа:
вожделения, жаркой страсти, распутства и алчности.
\vs Tju 16:2
Когда пьёте вино в радости, будьте умеренны, боясь Бога.
Ибо если в радости исчезнет страх Божий, наступит опьянение,
и придет бесстыдство.
\vs Tju 16:3
Если же хотите жить разумно, вовсе не прикасайтесь к вину,
дабы не согрешить в словах надменных, и побоищах, и клевете,
и нарушении заповедей Божиих, и не погибнете не в свой час.
\vs Tju 16:4
Также раскрывает вино тайны Божии и людские,
как и я раскрыл заповеди Божий и тайны Иакова, отца моего,
Хананеянке Вирсавии, чего не велел мне Бог раскрывать.

\vs Tju 17:1
И ныне завещаю вам, дети мои,
не любить серебра и не смотреть на красоту женщин,
ибо и я от серебра и золота, и от красоты соблазнился Вирсавией Хананеянкой.
\vs Tju 17:2
[И знаю, что из-за этих двух предан будет род мой на погибель блуда.
\vs Tju 17:3
Ибо и мудрых мужей из сынов моих собьют они с пути,
и умалят царство Иуды, данное мне Господом за послушание отцу моему.
\vs Tju 17:4
Ведь я никогда не огорчал отца моего Иакова, ибо делал всё,
что говорил он мне.
\vs Tju 17:5
И Авраам, отец деда моего, благословил меня царствовать в Израиле,
и так же благословил меня Иаков.
\vs Tju 17:6
И знаю я, что от меня восстановится царство.

\vs Tju 18:1
И познал я, и читал в книгах Еноха праведного,
какое зло сотворите вы в последние дни.]
\vs Tju 18:2
Остерегайтесь же, дети мои, блуда и сребролюбия, и послушайте Иуду,
отца вашего.
\vs Tju 18:3
Ибо они уводят от закона
Божия и помрачают помысел душевный,
и гордыне научают, и не дают мужу иметь сострадание к ближнему своему.
\vs Tju 18:4
Лишают они душу его всякой доброты
и утесняют его болями и страданием,
сон прогоняют от него и плоть его истребляют.
\vs Tju 18:5
Жертвам Богу он препятствует, о благословении Божием не помнит,
и когда пророк говорит, не слушает, и от слов благочестия отвращается.
\vs Tju 18:6
Ибо двум страстям, противным заповедям Божиим,
рабски служит он и Богу повиноваться не может.
Помрачили они душу его, и днем ходит он словно ночью.

\vs Tju 19:1
Дети мои, сребролюбие ведёт к идолопоклонству,
ибо в соблазне серебра называют богами тех,
кто не есть Бог, а тот, кто имеет серебро, в безумие впадает.
\vs Tju 19:2
От серебра погиб я, дети мои, и если бы не раскаяние моё,
и смирение, и мольбы отца моего, бездетным умер бы я.
\vs Tju 19:3
Но Бог отцов наших смиловался надо мною,
ибо по неведению сотворил я это.
\vs Tju 19:4
Ибо ослепил меня отец обмана и пребывал я в заблуждении
как человек и плоть, грехами сокрушенная, и познал я немощь мою,
когда думал, что непобедим я.

\vs Tju 20:1
Знайте же, дети мои, что 2 духа смотрят
за человеком~--- дух правды и дух лжи.
\vs Tju 20:2
Посредине же~--- дух познания, склоняющего ум туда, куда пожелает.
\vs Tju 20:3
А правдивое и лживое написаны на сердце человека, и всё это известно Господу.
\vs Tju 20:4
И нет часа, в который могли бы укрыться дела людские,
ибо на самом сердце написано пред лицом Господа.
\vs Tju 20:5
А дух правды обличает всё, и жжет грешника огнём в его же сердце,
и не может он поднять лица к Судье.

\vs Tju 21:1
Ныне, дети мои, возвещаю вам:
любите Левия, и пребывайте с ним, и не возноситесь над ним,
да не уничтожитесь вы.
\vs Tju 21:2
Ибо мне дал Бог царство, ему же~--- священство,
и подчинил царство священству.
\vs Tju 21:3
Мне дал он то, что на земле, ему~--- то, что на небесах.
\vs Tju 21:4
Как небеса выше земли,
так священство Божие выше стоит, нежели царство земное,
если согрешив, не отпадёт оно от Господа
и не станет править священством царство земное.
\vs Tju 21:5
Ибо сказал мне ангел Господень: избрал его Господь и поставил выше тебя,
чтобы приблизился ты к нему, и вкушал от трапезы его,
и первенцев от богатств сынов Израиля приносил ему.
Ты же будешь царем над Иаковом.
\vs Tju 21:6
И будешь ты подобен морю.
Ибо, как на море праведные и неправедные попадают в бурю,
и одни попадают в плен, другие же обогащаются,
так и в тебе со всяким родом людей так будет:
одни будут терпеть опасности и пленение,
другие же обогащаться, похищая чужое.
\vs Tju 21:7
Ибо цари китам уподобятся: пожирая людей, словно рыб,
станут они порабощать сыновей и дочерей свободных и грабить дома,
поля, пастбища и всякое добро.
\vs Tju 21:8
И неправедно тела многих отдадут в пищу воронам и цаплям,
и преуспеют во зле, и возвысятся в алчности.

\vs Tju 21:9
И будут лжепророки, словно вихри, и многих праведных будут преследовать.
\vs Tju 22:1
И наведёт на них Господь раздоры друг с другом,
и войны будут в Израиле непрерывные.
\vs Tju 22:2
И к иноплеменным перейдёт царство моё до прихода спасения к Израилю,
до явления Бога справедливого, когда почиет Иаков в мире и все народы.
\vs Tju 22:3
И сам Господь сохранит навеки силу царства моего,
ибо клялся он мне клятвою, что не угаснет царство семени моего до века.

\vs Tju 23:1
Великое горе для меня, дети мои, от нечестия и обмана,
которые сотворите вы в царстве моём,
когда последуете за чревовещателями, прорицателями и бесами соблазна.
\vs Tju 23:2
Дочерей ваших певицами и блудницами сделаете,
и смешаетесь с мерзостью языческой.
\vs Tju 23:3
За то наведёт на вас Господь голод и мор, смерть и меч,
осаду от врагов и позор от друзей, и воспаление очей,
и убийство детей, и похищение имущества, и сожжение Храма Божьего,
и порабощение вас самих язычниками.
\vs Tju 23:4
И оскопят сыновей ваших, чтобы стали они евнухами у жен их.
\vs Tju 23:5
И будет так дотоле, пока не посетит вас Господь,
когда раскаетесь вы и станете жить по всем заповедям его,
и выведет он вас из плена языческого.

\vs Tju 24:1
После того взойдет вам звезда из Иакова в знак мира,
и восстанет человек [от семени моего],
как солнце праведности, и будет жить с людьми в кротости и справедливости,
и не будет на нём никакого греха.
\vs Tju 24:2
И разверзнутся над ним небеса, дабы излить дух благословения
Отца Святого, и сам он изольёт дух милости на вас.
\vs Tju 24:3
И будете ему сыновьями истинными, и жить будете по заветам
его первым и последним.
\vs Tju 24:4
[Он есть отрасль Бога Всевышнего и источник, дающий всем жизнь.]
\vs Tju 24:5
Тогда воссияет скипетр царства моего, и из корня вашего выйдет ствол.
\vs Tju 24:6
А на нём взрастёт жезл праведности для народов,
дабы судить и спасти всех призывающих имя Господа.

\vs Tju 25:1
После того восстанут к жизни Авраам, Исаак и Иаков,
а я и братья мои станем вождями колен Израиля:
1-ый~--- Левий,
2-ой~--- я,
3-ий~--- Иосиф,
4-ый~--- Вениамин,
5-ый~--- Симеон,
6-ой~--- Иссахар,
и так все по порядку.
\vs Tju 25:2
И благословил Господь Левия;
ангел лика Господня~--- меня;
силы славы~--- Симеона;
небо~--- Рувима;
земля~--- Иссахара;
море~--- Завулона;
горы~--- Иосифа;
скиния~--- Вениамина;
светильники~--- Дана;
сад Едемский~--- Неффалима;
солнце~--- Гада;
луна~--- Асира.
\vs Tju 25:3
И будете вы один народ Господень и один язык,
и не будет там духа соблазна Велиарова,
ибо он будет ввержен в огонь навечно.
\vs Tju 25:4
И в скорби скончавшиеся восстанут в радости, а нищие Господом
обогащены будут, а умирающие Господом вдохновлены к жизни будут.
\vs Tju 25:5
И в веселии побегут олени Иакова,
и орлы Израиля полетят в радости, [а нечестивые
восскорбят, и грешники зарыдают], и все народы прославят Господа навеки.

\vs Tju 26:1
Храните же, дети мои, все законы Господни, ибо он есть надежда для всех,
соблюдающих пути его.

\vs Tju 26:2
И сказал Иуда: вот, 118-ти лет умираю я сегодня.
\vs Tju 26:3
Да не погребает меня никто в пышной одежде,
и да не разрезают мне чрево,
что угодно творить царствующим,
но отнесите меня в Хеврон, где и отцы мои.
\vs Tju 26:4
И сказав это, почил он, и сделали сыновья его во всём так, как
завещал он им, и погребли его с отцами его в Хевроне.

\bibbookdescr{Tis}{
  inline={Завещание Иссахара,\\пятого сына Иакова и Лии},
  toc={Завещание Иссахара},
  bookmark={Завещание Иссахара},
  header={Завещание Иссахара},
  abbr={Исс}
}
\vs Tis 1:1
Список слов Иссахара.
Ибо он призвал сыновей своих и сказал им:
выслушайте, дети, Иссахара, отца вашего;
внемлите словам возлюбленного Господом.

\vs Tis 1:2
Родился я пятым сыном Иакова, платою за мандрагоры.
\vs Tis 1:3
Ибо Рувим, брат мой, принёс с поля мандрагоры,
и Рахиль, встретив его, взяла их.
\vs Tis 1:4
И плакал Рувим, и на голос его вышла Лия, мать моя.
\vs Tis 1:5
А были то яблоки благовонные, которые рождаются в земле Харана
на дне ложбин водных.
\vs Tis 1:6
И сказала Рахиль: не дам я тебе их, но мне самой нужны они,
дабы иметь детей.
Ведь обошёл меня Господь, и не рождала я сыновей Иакову.
\vs Tis 1:7
А яблок было 2.
И сказала Лия Рахили: да будет довольно тебе,
что взяла ты мужа моего, так возьмёшь ещё и это у меня?
\vs Tis 1:8
Отвечала ей Рахиль:
да будет Иаков с тобою в ночь эту за мандрагоры сына твоего.
\vs Tis 1:9
Сказала же ей Лия: мой Иаков, ибо я~--- жена юности его.
\vs Tis 1:10
И сказала Рахиль:
не возносись и не похваляйся,
ибо ко мне первой прилепился он и ради меня служил отцу моему 14 лет.
\vs Tis 1:11
И если бы не возросла хитрость на земле,
и злоба человеческая не преуспевала бы,
не была бы ты тою, что узрела лицо Иакова.
\vs Tis 1:12
Ибо ты не жена его, но хитростью опередила меня.
\vs Tis 1:13
И обманул меня отец мой, и удалил в ту ночь,
и не позволил мне видеть Иакова, ибо, если бы я там была,
не случилось бы того.
\vs Tis 1:14
Но за мандрагоры уступлю тебе на одну ночь Иакова.
\vs Tis 1:15
И познал Иаков Лию, и, зачав, родила она меня,
и от этой платы наречен я был Иссахаром.

\vs Tis 2:1
Тогда явился Иакову ангел Господень, говоря:
родит двоих детей Рахиль,
ибо пренебрегла она сообщением с мужем своим и воздержание избрала.
\vs Tis 2:2
И если бы мать моя Лия за сообщение с Иаковом не отдала 2 яблока,
то родила бы 8-ых сыновей, но из-за того родила 6-ых,
а Рахиль двоих, ибо в мандрагорах призрел на неё Господь.
\vs Tis 2:3
Ибо видел он, что ради детей желала она сойтись с Иаковом,
а не ради любострастия.
\vs Tis 2:4
И на другой день отдала она Иакова, чтобы взять и другую мандрагору.
Ибо в мандрагорах услышал Господь Рахиль.
\vs Tis 2:5
А она, возжелав их, не вкусила, но отнесла их в дом Господень
и отдала священнику, бывшему в то время.

\vs Tis 3:1
Когда же возмужал я, дети мои, жил я в прямоте сердечной,
и стал земледельцем отцу моему и братьям моим,
и приносил плоды с полей.
\vs Tis 3:2
И благословил меня отец мой, видя, что в простоте живу я.
\vs Tis 3:3
И не был я суетным в делах моих, ни завистником, ни клеветником
ближнему моему.
\vs Tis 3:4
Не наговаривал я никогда ни на кого, и не хулил жизнь никакого человека.
\vs Tis 3:5
45-и лет взял я себе жену, ибо труд снедал силы мои,
и не помышлял я о наслаждении от женщины, но от усталости засыпал я.
\vs Tis 3:6
И радовался простоте моей отец мой, ибо всякое первородное через
священника приносил я Господу, а после и отцу моему.
\vs Tis 3:7
И Господь утысячерял добро моё в руках моих,
и знал Иаков, отец мой, что Бог помогает простоте моей.
\vs Tis 3:8
Ибо всем бедным и страждущим уделял я от благ земли в простоте сердца моего.

\vs Tis 4:1
И ныне, внемлите мне, дети мои, и живите в простоте сердец
ваших, ибо узрел я, что в этом всякое благоугождение Господу.
\vs Tis 4:2
Простосердечный не стремится к золоту, и не хочет превзойти ближнего
богатством, и не домогается многообразных яств, и не желает разных одежд.
\vs Tis 4:3
Не хочет он приписать многих лет к своей жизни, но приемлет одну лишь
волю Божию.
\vs Tis 4:4
И духи соблазна ничего не могут против него,
ибо не воззрел он на красоту женскую, дабы не осквернить порчею ума своего.
\vs Tis 4:5
Не завидует он в помыслах своих, и клевета не изнуряет души его,
ни желание ненасытное ума его.
\vs Tis 4:6
Живет он в простоте души, всё зрит в прямоте сердца,
но оберегает очи свои от соблазна мирского,
дабы не видеть уклонений от заповедей Господних.

\vs Tis 5:1
Храните же, дети мои, закон Божий, и простоту обретайте, и в
беззлобии живите, не заботясь излишне о делах ближнего.
\vs Tis 5:2
Но возлюбите Господа и ближнего, а бедного и слабого жалейте.
\vs Tis 5:3
Склоните спины ваши к земледелию и утруждайте себя всяким земледелием,
принося Господу плоды с благодарностью.
\vs Tis 5:4
Ибо в первенцах плодов земных благословит вас Господь,
как благословил он всех святых от Авеля и доныне.
\vs Tis 5:5
Ибо не дастся вам иной удел, кроме тучности земли в трудах плодородия.
\vs Tis 5:6
Так и отец мой Иаков благословениями земли
и первенцев плодов благословил меня.
\vs Tis 5:7
А Левий и Иуда прославлены у Господа и в сынах Иакова.
И дал им наследие Господь:
Левию дал он священство, а Иуде~--- царство.
\vs Tis 5:8
Вы же слушайтесь их и пребывайте в прямодушии отца вашего.

\vs Tis 6:1
Знайте же, дети мои, что в последние времена
оставят сыновья ваши простоту, и погрязнут в алчности,
и отринут беззлобие, и совершат злодеяния,
и оставят заповеди Господа,
и прилепятся к Велиару.
\vs Tis 6:2
И оставят они земледелие, и последуют злым помыслам своим,
и рассеются среди народов, и рабами будут врагам своим.
\vs Tis 6:3
И вы скажите это детям вашим, дабы, если согрешат, тотчас обращались
вновь к Господу.
\vs Tis 6:4
Ибо он милостив, и пожалеет их, и вернёт в землю их.

\vs Tis 7:1
Вот, как видите вы, живу я 126 лет и не знал греха смертного.
\vs Tis 7:2
И кроме жены моей, не познавал я другой и не совершал блуда,
взирая очами моими.
\vs Tis 7:3
Вина соблазняющего не пил, не желал ничего из имущества ближнего моего.
\vs Tis 7:4
Хитрость не рождалась в сердце моём, ложь не входила на уста мои.
\vs Tis 7:5
Сострадал я всякому человеку скорбящему, и с нищим делил хлеб мой.
Благочестие творил я во все дни мои и правду хранил.
\vs Tis 7:6
Господа любил я и всякого человека всем сердцем моим.
\vs Tis 7:7
Так и вы делайте, дети мои, и всякий дух Велиаров бежит от вас,
и никакое дело злых людей не возобладает над вами,
и всякого зверя дикого усмирите,
если с вами будет Бог небес и земли, помогающий людям простосердечным.

\vs Tis 7:8
И сказав это сыновьям своим, завещал им, дабы отнесли его в Хеврон и
погребли там с отцами его.
\vs Tis 7:9
И вытянув ноги свои, почил он в старости прекрасной сном вечным.

\bibbookdescr{Tzb}{
  inline={Завещание Завулона,\\шестого сына Иакова и Лии},
  toc={Завещание Завулона},
  bookmark={Завещание Завулона},
  header={Завещание Завулона},
  abbr={Зав}
}
\vs Tzb 1:1
Список слов Завулона, речённых им к сыновьям своим,
прежде чем умер он в 114-ый год жизни своей,
спустя 2 года после смерти Иосифа.
\vs Tzb 1:2
Сказал он им: слушайте меня, сыновья Завулона, внемлите речам отца вашего.

\vs Tzb 1:3
Я, Завулон, даром прекрасным родился у отца и матери моих.
Ибо когда родился я, премного возрос отец мой мелким и крупным
скотом, когда пеструю скотину получил он в удел.
\vs Tzb 1:4
Не знал я греха за собою во все дни мои, кроме только мысленного.
\vs Tzb 1:5
Не вспомню, чтобы совершил я несправедливость,
кроме греха неведения, сотворенного мною против Иосифа,
когда сговорился я с братьями моими не возвещать отцу моему о случившемся.
\vs Tzb 1:6
Но плакал я много дней из-за Иосифа, втайне, ибо страшился я братьев моих,
ибо положили они, что выдавший тайну будет убит.
\vs Tzb 1:7
Но когда желали убить Иосифа, молил я их со слезами не делать греха этого.

\vs Tzb 2:1
Ибо Симеон, Дан и Гад приступили к Иосифу, чтобы убить его,
и говорил он им со слезами, пав на лицо своё:
\vs Tzb 2:2
пощадите меня, братья мои; пожалейте сердце Иакова, отца нашего.
Не поднимайте рук ваших, дабы пролить кровь невинную,
ибо не согрешил я против вас.
\vs Tzb 2:3
Если же и согрешил, наказанием накажите меня,
но не поднимайте руки,
чтобы убить брата вашего ради отца нашего Иакова.
\vs Tzb 2:4
Когда же говорил он, скорбя, эти речи,
не вынес я стонов его и начал плакать,
и сотряслись внутренности мои, и всё во мне ослабло.
\vs Tzb 2:5
И заплакал я с Иосифом и забилось громко сердце моё,
и задрожали суставы тела моего, и не был я в силах стоять.
\vs Tzb 2:6
Когда же увидел Иосиф, что плачу вместе с ним,
а они подступили и хотят убить его, спрятался за спину мою,
умоляя помочь ему.
\vs Tzb 2:7
Тогда встал Рувим посредине и сказал:
братья мои, не будем убивать его,
но бросим его в один из сухих колодцев,
которые рыли отцы наши и не находили там воды.
\vs Tzb 2:8
Ибо для того не дал Господь подняться туда воде,
чтобы спасся Иосиф.
\vs Tzb 2:9
И сделали они так до той поры,
когда продали его Измаильтянам.

\vs Tzb 3:1
От платы за Иосифа не взял я своей части, дети мои.
\vs Tzb 3:2
Но Симеон, Дан, Гад и другие братья наши,
взяв плату за него, купили обувь себе и жёнам и детям своим,
говоря:
\vs Tzb 3:3
пропитания не купим, ибо это цена крови брата нашего,
но ногами потопчем её за то, что говорил он,
будто станет властвовать над нами;
и увидим, что будет из его снов.
\vs Tzb 3:4
Оттого записано в законе Моисеевом:
с нежелающего возместить семя брату своему
да снимут обувь его и плюнут в лицо ему.
\vs Tzb 3:5
А братья Иосифа не хотели, чтобы жил он,
и Господь снял с них обувь,
которую носили они за брата своего Иосифа.
\vs Tzb 3:6
И когда пришли они в Египет, за воротами сняли с них
обувь дети Иосифа,
и так преклонились они пред Иосифом, словно пред фараоном.
\vs Tzb 3:7
И не только преклонились пред ним,
но и оплёваны были в тот же час,
пав пред ним, и опозорены были пред Египтянами.
\vs Tzb 3:8
Ибо Египтяне слышали обо всех злых делах,
которые совершили они против Иосифа.

\vs Tzb 4:1
И сделав это, сели братья мои есть и пить.
\vs Tzb 4:2
Я же, терзаясь из-за Иосифа, не ел,
но смотрел на колодец, ибо опасался,
как бы Симеон, Дан и Гад не пошли и не убили Иосифа.
\vs Tzb 4:3
Увидев, что я не ем, они оставили меня стеречь его,
\vs Tzb 4:4
пока не продали Измаильтянам.

\vs Tzb 4:5
Затем пришел Рувим и, услышав,
что продали Иосифа в его отсутствие,
разодрал хитон свой и сказал плача:
как посмотрю я в лицо отцу моему Иакову?
\vs Tzb 4:6
И взяв серебро,
побежал вслед за купцами и, не найдя их,
вернулся опечаленный.
Купцы же, сойдя с широкой дороги,
пошли кратчайшим путём через землю Троглодитов.
\vs Tzb 4:7
И скорбел Рувим, и не ел хлеба в тот день.
И подойдя к нему, сказал Дан:
\vs Tzb 4:8
не плачь и не скорби; мы найдем, что сказать отцу нашему.
\vs Tzb 4:9
Зарежем козла, и вымараем хитон Иосифа,
и пошлём его Иакову, говоря:
узнай, сына ли твоего этот хитон?
Так они и сделали.
\vs Tzb 4:10
Ибо, продавая Иосифа, сняли с него хитон и одели на него плащ рабский.
\vs Tzb 4:11
Симеон же взял хитон и не хотел отдать его, ибо он желал убить Иосифа
и гневался, что не убили его.
\vs Tzb 4:12
И встав, сказали все мы ему:
если не отдашь хитон, скажем отцу нашему,
что ты один сотворил это зло в Израиле.
\vs Tzb 4:13
И отдал он им хитон. И сделали так, как сказал Дан.

\vs Tzb 5:1
Ныне, дети мои, завещаю вам хранить заповеди Господа,
и творить милость ближнему,
и добросердечными быть не только к людям,
но и к бессловесным животным.
\vs Tzb 5:2
За это и благословил меня Господь,
и когда занемогли все братья мои,
оставался я здоров; ибо знал Господь помыслы каждого.
\vs Tzb 5:3
Имейте же милость в сердцах ваших, ибо что сделает человек ближнему
своему, то сделает Господь с ним самим.
\vs Tzb 5:4
И болели, и умирали сыновья братьев моих из-за Иосифа,
ибо не имели милости в сердцах своих.
А мои сыновья сохранялись в здравии,
как вам то известно.
\vs Tzb 5:5
И когда были мы в земле Ханаанской,
ловил я рыб для отца моего Иакова,
и многие утонули в море,
а я невредим остался.

\vs Tzb 6:1
Первым я был, кто сделал лодку, чтобы плавать по морю,
ибо дал мне Господь для того знание и мудрость.
\vs Tzb 6:2
И приладил я весло деревянное сзади у неё,
а на другом прямом куске дерева натянул парус посредине.
\vs Tzb 6:3
И плавал я в лодке той по морским водам,
и ловил рыбу для дома отца моего,
пока не пошли мы в Египет.
\vs Tzb 6:4
[И из добычи моей всякому человеку чужому уделял я от доброты сердца.
\vs Tzb 6:5
Был ли кто чужестранец, или больной, или старый, готовил я рыбу,
делал её хорошо и давал всякому по надобности его,
соболезнуя и сострадая.
\vs Tzb 6:6
За это много рыб давал мне Господь, когда ловил я.
Ибо тот, кто делится с ближним,
получает многократно от Господа.]
\vs Tzb 6:7
5 лет ловил я рыбу [давая всякому человеку,
какого видел, и вдоволь отдавая дому отца моего].
\vs Tzb 6:8
Летом ловил я, а зимою пас стада вместе с братьями моими.

\vs Tzb 7:1
[Ныне возвещу вам, что сделал я.
Увидел я зимою страждущего от наготы,
и сжалился над ним, и, украв плащ из дома отца моего,
тайно дал страждущему.
\vs Tzb 7:2
Так и вы, дети мои, милостиво уделяйте из того, что даёт вам Господь,
всем без различия, и давайте всякому человеку в доброте сердечной.
\vs Tzb 7:3
Если же не имеете, что подать нуждающемуся,
сострадайте ему сердцем своим.
\vs Tzb 7:4
Помню, как не нашла рука моя, что подать нуждающемуся,
и прошёл я с ним семь стадиев и плакал с ним вместе,
и сердце моё сотрясалось от сострадания к нему.
\vs Tzb 8:1
Так и вы, дети мои, добросердечны будьте со всяким человеком в милости,
дабы и Господь, сжалившись, помиловал вас.
\vs Tzb 8:2
Ибо в последние дни пошлёт Бог сердце своё на землю,
и где найдет сердце милостивое, поселится в нём.
\vs Tzb 8:3
Ибо как человек жалеет ближнего своего, так сжалится и Господь над ним.]

\vs Tzb 8:4
Когда же пришли мы в Египет, не вспомнил нам зла Иосиф.
\vs Tzb 8:5
Воззрев на него, дети мои, возлюбите и вы друг друга, и не замышляйте
зла каждый на брата своего.
\vs Tzb 8:6
Ибо это разделяет единое, и всякое родство уничтожает,
и душу возмущает, и лицо искажает.

\vs Tzb 9:1
Посмотрите на воды и узрите, что когда текут они вместе,
то камни, деревья, землю и иное сметают они.
\vs Tzb 9:2
Если же на много частей разделятся, земля поглотит их,
и станут они ничтожными.
\vs Tzb 9:3
И вы, если разделитесь, будете таковыми.
\vs Tzb 9:4
Не разделяйтесь же на две головы, ибо всё, что сотворил Господь,
одну голову имеет, а два плеча, две руки, две ноги
и все другие члены слушаются одной головы.
\vs Tzb 9:5
Узнал же я из Писаний отцов наших,
что разделитесь вы в Израиле,
и за двумя царями последуете, и всякую мерзость сотворите.

\vs Tzb 9:6
И возьмут вас в плен враги ваши,
и зло будет вам от язычников среди многой скорби и бессилия.
\vs Tzb 9:7
После того, вспомнив о Господе, обратитесь вы, и помилует он вас,
ибо он милостив и добр сердцем,
и не мыслит зла против сынов человеческих,
ведь они~--- плоть и блуждают во злых делах своих.
\vs Tzb 9:8
И после того взойдёт для вас сам Господь,
свет справедливости, и возвратитесь вы в землю вашу,
и узрите его в Иерусалиме, избранном ради имени его святого.
\vs Tzb 9:9
И вновь злобою дел ваших прогневаете его,
и отвержены будете им вплоть до конца времен.

\vs Tzb 10:1
И ныне, дети мои, не скорбите, что умираю я,
и не унывайте, когда отойду я.
\vs Tzb 10:2
Ибо снова восстану я среди вас,
как предводитель среди сынов своих,
и возрадуюсь среди тех из рода моего,
кто сохранит закон Господень и заповеди Завулона, отца своего.
\vs Tzb 10:3
На нечестивых же наведёт Господь огонь вечный,
и погубит их до потомства потомков их.
\vs Tzb 10:4
Я же ныне отхожу к покою, как и отцы мои отошли.
\vs Tzb 10:5
Вы же бойтесь Господа Бога нашего всеми силами вашими во все дни жизни вашей.

\vs Tzb 10:6
И сказав это, почил он сном прекрасным,
и положили его сыновья во гроб деревянный.
\vs Tzb 10:7
После же отнесли его и погребли в Хевроне с отцами его.

\bibbookdescr{Tdn}{
  inline={Завещание Дана,\\седьмого сына Иакова и Баллы},
  toc={Завещание Дана},
  bookmark={Завещание Дана},
  header={Завещание Дана},
  abbr={Дна}
}
\vs Tdn 1:1
Список слов Дана, речённых им к сыновьям своим
в последние дни его в 125-ый год жизни его.
\vs Tdn 1:2
Ибо, призвав семейство свое, сказал он:
услышьте, сыновья Дановы, слова мои и внемлите речам отца вашего.

\vs Tdn 1:3
Испытал я сердцем моим и всею жизнью моей,
что прекрасны и угодны Богу правда и совершение справедливых дел,
и что злы ложь и гнев, научающий человека злому.
\vs Tdn 1:4
Исповедуюсь вам сегодня, дети мои,
что положил я в сердце моём о смерти Иосифа, брата моего,
мужа доброго и правдивого.
\vs Tdn 1:5
Радовался я тому, что продали его, ибо возлюбил его отец более нас.
\vs Tdn 1:6
Сказал же мне дух зависти и гордыни: ты тоже сын его.
\vs Tdn 1:7
И один из духов Велиаровых возбуждал меня: возьми меч и убей им Иосифа,
и, когда умрёт он, возлюбит тебя отец.
\vs Tdn 1:8
Дух гнева убеждал меня задушить Иосифа, как барс душит козла.
\vs Tdn 1:9
Но Бог отцов моих не предал его в руки мои, дабы убил я его,
найдя одного, и уничтожил тем два скипетра в Израиле.

\vs Tdn 2:1
И ныне, дети мои, вот, я умираю и воистину говорю вам,
что если не сохраните себя от духа лжи и гнева
и не возлюбите правду и долготерпение, погибнете вы.
\vs Tdn 2:2
Ибо гнев есть ослепление, и не дозволяет он видеть ничьего лица в правде.
\vs Tdn 2:3
Ибо, если и отец то или мать, как на врагов взирает на них;
если и брат, не ведает; если и пророк Господа, не слушает;
если и праведник, не видит; если и друг, не узнаёт.
\vs Tdn 2:4
Ибо одолевает его дух гнева сетью соблазна,
и ослепляет очи его, и ложью помрачает помысел его,
и своё зрение даёт ему.
\vs Tdn 2:5
Чем же ослепляет он очи его?
Ненавистью сердечной, дабы завидовал брату своему.
\vs Tdn 3:1
Ибо зол гнев, дети мои, который душу возмущает.
\vs Tdn 3:2
И овладевает он телом того, кто гневается,
и властвует над душою его, и даёт силу телу,
да творит оно всякие беззакония.
\vs Tdn 3:3
Когда же сотворит всё это тело, оправдывает содеянное и душа,
ибо неправильно видит она.
\vs Tdn 3:4
Из-за того гневный, если он силён телом,
во гневе тройную силу обретает:
одну~--- от помощи помогающих ему,
вторую~--- от богатства, которым убеждает и побеждает он неправедно,
третью же~--- телесную, которой и творит он зло.
\vs Tdn 3:5
Если же слаб гневный, двойная сила от ярости у него возникает,
ибо помогает ему гнев постоянно в беззаконии.
\vs Tdn 3:6
Этот дух всегда с ложью от десницы Сатаны исходит,
дабы в жестокости и лжи творились дела его.
\vs Tdn 4:1
Так познайте же, что тщетна сила гнева.
\vs Tdn 4:2
Ибо в речах обостряется он сперва, потом в делах силу дает тому,
кто гневается, и вредом горьким возмущает рассудок его,
и так великим гневом возбуждает душу его.
\vs Tdn 4:3
И потому, если кто говорит против вас, не предавайтесь гневу,
и если кто станет восхвалять вас как святых, не превозноситесь.
И не переменяйтесь ни от наслаждения, ни от неприязни.
\vs Tdn 4:4
Ибо вначале услаждается слух, а оттого обостряется ум
и внимает возбуждению, и прогневавшись, думает человек,
что справедлива ярость его.
\vs Tdn 4:5
Если же вред какой или погибель приступают к вам, дети, не тревожьтесь,
ибо тот дух хочет волновать гибнущего, дабы пал он от возбуждения ярости.
\vs Tdn 4:6
И если претерпеваете страдания вольно или невольно, не огорчайтесь, ибо
от горя пробуждаются и гнев с ложью.
\vs Tdn 4:7
Ведь это двуликое зло~--- гнев с ложью,
и помогают они друг другу, возмущая сердце.
А когда возмущается душа постоянно,
отступает Господь от неё, и властвует над нею Велиар.

\vs Tdn 5:1
Храните же, дети мои, заповеди Господа, и закон его блюдите.
А от гнева отступите, и ложь возненавидьте, дабы поселился в вас Господь
и бежал от вас Велиар.
\vs Tdn 5:2
Правду говорите каждый ближнему своему и не впадайте в ярость и смятение,
но пребывайте в мире, имея Бога мира,
и не будет иметь война власти над вами.

\vs Tdn 5:3
Возлюбите Господа во всю жизнь вашу,
так же и друг друга сердцем правдивым.

\vs Tdn 5:4
Знаю я, что в последние дни отступите вы от Господа,
и вознегодуете на Левия, и выступите против Иуды,
но не сможете ничего против них.
Поведёт их обоих ангел Господень,
ибо на них будет стоять Израиль.
\vs Tdn 5:5
И когда отступите вы от Господа, живя во всяком зле,
сотворите мерзости языческие,
блуду предаваясь с жёнами беззаконников,
и всякое зло будут творить через вас духи злые.
\vs Tdn 5:6
[Ибо читал я книгу Еноха праведного и узнал,
что владыка ваш~--- Сатана,
и что духи злобы и гордыни присоветуют вам стать друзьями сынов Левия,
дабы заставить их согрешить перед Господом.
\vs Tdn 5:7
И приблизились сыны мои к сынам Левия,
и грешили с ними во всем, а сыны Иуды пребудут в алчности,
похищая чужое, словно львы.]
\vs Tdn 5:8
Оттого уведены будете [ими] в плен,
и там примете все казни Египетские и всё зло языческое.
\vs Tdn 5:9
Тогда, обратившись к Господу, помилованы будете,
и приведёт вас к святыне своей и даст вам мир.
\vs Tdn 5:10
И приведёт вам Господь спасение от [Иуды и] Левия,
и поведёт он войну против Велиара и отмщение воздаст за отцов ваших.
\vs Tdn 5:11
И заберёт пленных у Велиара [--- души святых],
и обратит сердца непокорные к Господу,
и даст призывающим его мир вечный.
\vs Tdn 5:12
И почиют в Едеме святые, и возрадуются праведные новому Иерусалиму,
славе Бога вечного.
\vs Tdn 5:13
И не будет более Иерусалим в запустении,
и не будет пленён Израиль, ибо Господь будет посреди него
[живя между людей], и Святой Израилев, царствующий в нём
[в унижении и в нищете, и верующий в него царствовать
будет над людьми воистину].

\vs Tdn 6:1
И ныне, бойтесь Господа, дети мои, и берегите себя от Сатаны и духов его.
\vs Tdn 6:2
Придите к Богу и ангелу, утешающему вас,
ибо он~--- посредник между Богом и людьми,
и за мир у Израиля восстанет он против царства Врага.
\vs Tdn 6:3
Оттого строит козни Враг всем, призывающим Господа.
\vs Tdn 6:4
Ибо знает Враг, что в тот день, когда обратится Израиль,
кончится царство его.
\vs Tdn 6:5
Сам ангел мира даст силу Израилю не впасть в погибель зла.
\vs Tdn 6:6
И будет во времена беззакония Израиля:
не отступится от них Господь, но придёт и к народам,
ищущим воли его, ибо не равен ему ни один из ангелов.
\vs Tdn 6:7
Имя же его во всяком месте Израиля и в народах.
\vs Tdn 6:8
Оберегайте же себя, дети мои, от всякого злого дела,
и отриньте от себя гнев и ложь, и возлюбите правду и долготерпение.
\vs Tdn 6:9
И что услышали от отца вашего, передайте и вы детям вашим,
[да примет вас Спаситель народов, ибо он правдив и долготерпелив,
кроток и смирен, и научает делами своими закону Господа.]
\vs Tdn 6:10
Так отступите от всякой неправедности, и прилепитесь к справедливости
Божией, и будет род ваш спасён навеки.
\vs Tdn 6:11
А меня похороните рядом с отцами моими.

\vs Tdn 7:1
И сказав это, поцеловал он их и почил сном прекрасным.
\vs Tdn 7:2
И погребли его сыновья его.
А после того отнесли кости его туда,
где погребены Авраам, Исаак и Иаков.
\vs Tdn 7:3
[Также пророчествовал им Дан,
что забудут они Бога своего и лишатся земли удела своего,
и рода семени своего.]

\bibbookdescr{Tnf}{
  inline={Завещание Неффалима,\\восьмого сына Иакова и Баллы},
  toc={Завещание Неффалима},
  bookmark={Завещание Неффалима},
  header={Завещание Неффалима},
  abbr={Неф}
}
\vs Tnf 1:1
Список завещания Неффалима,
данного им в час кончины его в 130-ый год жизни его.
\vs Tnf 1:2
Когда собрались сыновья его в 7-ой месяц 1-го числа, устроил им пир.
\vs Tnf 1:3
И пробудившись наутро, сказал он им:
я умираю.
И они не поверили ему.
\vs Tnf 1:4
И восславив Господа, собрал он силы и сказал:
после пира, бывшего вчера, умерла плоть моя.

\vs Tnf 1:5
И начал говорить:
слушайте, дети мои, сыновья Неффалима, слушайте слова отца вашего.
\vs Tnf 1:6
Я родился от Баллы, ибо хитрость сотворила Рахиль,
и вместо себя дала Баллу Иакову,
и та зачала и родила меня на колени Рахили,
и оттого наречено мне было имя Неффалим.
\vs Tnf 1:7
Премного возлюбила меня Рахиль, ибо на колени её родился я,
и когда был я ещё мал, целовала меня, говоря:
да будет мне дан брат твой от чрева моего, такой, как ты.
\vs Tnf 1:8
Оттого сходен был со мною во всем Иосиф по мольбам Рахили.
\vs Tnf 1:9
Мать же моя Балла была дочерью Руфея, брата Деворы,
кормилицы Ревекки, и родилась в один день с Рахилью.
\vs Tnf 1:10
Руфей же был из рода Авраама, Халдей,
чтущий Бога, свободный и знатный.
\vs Tnf 1:11
И попав в плен, был он куплен Лаваном,
и тот дал ему в жёны Енан, служанку свою,
которая родила дочь и нарекла ей имя Зелфа по имени того города,
где Руфей был взят в плен.
\vs Tnf 1:12
После же родила она Баллу и сказала:
к новому торопится дочь моя,
ибо родилась быстро и, взяв грудь,
сразу принялась сосать.

\vs Tnf 2:1
Был я лёгок ногами, словно серна, и поручал мне всякую весть
отец мой Иаков, и как серну благословил меня.
\vs Tnf 2:2
Как знает гончар сосуд, сколько вмещает он,
и сообразно с этим берёт глину для него,
так же и Господь в согласии с духом творит тело,
а по силе телесной влагает дух.
\vs Tnf 2:3
И нет расхождения ни на треть волоса,
ибо всё творение весами, и мерою, и правилом совершается.
\vs Tnf 2:4
И как знает гончар пользу всякого сосуда, для чего он пригоден,
так же и Господь знает тело,
до какого предела пребывает оно в добре,
а когда ко злу переходит.
\vs Tnf 2:5
Ибо нет творения и никакого помысла нет,
которых не ведал бы Господь.
Ибо всякого человека сотворил он по образу своему.
\vs Tnf 2:6
И какова сила его, таково и дело его;
каково око его, таков и сон его;
какова душа его, таково и слово его~--- либо по закону Господа,
либо по закону Велиара.
\vs Tnf 2:7
И как различают между светом и тьмою,
между зрением и слухом, так и между мужем и мужем различают,
и между женщиной и женщиной, и нельзя сказать,
что один подобен другому лицом или помыслом.
\vs Tnf 2:8
Ибо всё сделал Бог прекрасно в порядке своём:
5 чувств поместил в голове,
и горло приладил к голове,
и волосы на ней взрастил для красоты и славы;
после сердце сотворил для рассуждения,
желудок~--- для пищеварения,
чрево~--- для очищения тела,
горло~--- для дыхания,
печень~--- для гнева,
желчь~--- для огорчения,
селезёнку~--- для веселья,
почки~--- для разумения,
бока~--- для сна,
бёдра~--- для мощи,
и так далее.
\vs Tnf 2:9
Так да будут, дети мои, все дела ваши в порядке своём,
и в добром помышлении,
и в страхе Божием,
и ничего безрассудного не делайте в небрежении,
или же не в свой час.
\vs Tnf 2:10
Ибо если скажешь оку: слушай,~--- не сможет оно.
Так и вы, пребывая во тьме, не сможете творить дела света.

\vs Tnf 3:1
Так не стремитесь в любостяжании погубить дела ваши,
и словами пустыми не обманывайте душ ваших,
ибо молча, в чистоте сердца узрите,
как поддержать волю Божию, а волю Велиара отвергнуть.
\vs Tnf 3:2
Солнце, луна и звёзды не меняют порядка своего;
так и вы не меняйте закона Божьего,
лишая порядка дела ваши.
\vs Tnf 3:3
Язычники, заблуждаясь и отвергая Господа,
изменили порядок свой,
стали слушаться они дерева и камня,
духов соблазна.
\vs Tnf 3:4
Вы же не делайте так, дети мои;
узнавайте Господа на небосводе,
на земле, на море и во всех творениях его,
создавшего всё,
дабы не уподобиться вам Содому,
изменившему строй естества своего.
\vs Tnf 3:5
Так же и Стражи изменили строй естества своего,
и низверг их Господь потопом,
из-за них сделав землю лишённой поселений и плодов.

\vs Tnf 4:1
То говорю я вам, дети мои, что узнал я из писаний Еноха,
что и вы отступитесь от Господа,
и станете жить во всём беззаконии языческом,
и сотворите всё зло Содомское.
\vs Tnf 4:2
И наведёт на вас Господь пленение,
и рабами будете врагам вашим,
и всякой беде и горю подвергнетесь,
пока не избавит Господь всех вас.
\vs Tnf 4:3
Когда уменьшитесь вы и умалитесь,
обратитесь вы и познаете Бога вашего,
и он возвратит вас в землю вашу по великому милосердию своему.
\vs Tnf 4:4
И будет: пришедшие в землю отцов своих вновь забудут Господа
и совершат нечестия.
\vs Tnf 4:5
И рассеет их Господь по лицу всей земли,
пока не придет милосердие Господне,~--- Человек,
справедливость творящий и милость всем дальним и ближним.
\vs Tnf 5:1
Ибо в 40-ой год жизни моей узрел я видение
на горе Елеонской к востоку от Иерусалима,
что солнце и луна остановились.
\vs Tnf 5:2
И вот, Исаак, отец отца моего, сказал нам:
бегите и возьмите каждый по силе своей,
и получит овладевший солнце и луну.
\vs Tnf 5:3
И побежали все разом, и Левий овладел солнцем,
а Иуда успел взять луну,
и возвысились оба с тем, что взяли они.
\vs Tnf 5:4
И когда Левий стал как солнце, вот,
некий юноша дал ему 12 ветвей пальмовых,
а Иуда стал светел как луна,
и было под ногами их 12 лучей.
\vs Tnf 5:5
[И побежали оба, Левий и Иуда, и взяли их себе.]
\vs Tnf 5:6
И вот, явился на земле бык, имеющий 2 рога великих
и крылья орла на спине своей;
и когда хотели мы взять его, не смогли.
\vs Tnf 5:7
Иосиф же, придя, схватил его и взошел с ним на высоту.
\vs Tnf 5:8
И увидел я, что был я там,
и вот, святое писание увидели мы,
говорящее:
Ассирийцы, Мидяне, Персы, Халдеи, Сирияне
унаследуют пленение 12-ти скипетров Израиля.

\vs Tnf 6:1
И опять, через 5 дней, узрел я,
что отец мой Иаков стоит на море Ямнийском, и мы с ним.
\vs Tnf 6:2
И вот, корабль подошёл, плывущий без моряков и рулевых,
и написано было на нём, что это корабль Иакова.
\vs Tnf 6:3
И сказал нам отец наш: взойдем на корабль наш.
\vs Tnf 6:4
Когда же взошли мы, сделалась сильная буря и вихрь великий,
и ушёл от нас отец наш, державший руль.
\vs Tnf 6:5
А мы, гонимые бурей, носились по морю,
и наполнился корабль водою, и заливали его валы огромные,
и от них рассыпался он.
\vs Tnf 6:6
И поплыл Иосиф в лодке, а мы разделились на 10 досок.
Левий же и Иуда были на одной доске.
\vs Tnf 6:7
И разметало нас всех по разным концам земли.
\vs Tnf 6:8
И Левий, облачившись во вретище, молился Господу.
\vs Tnf 6:9
Когда же утихла буря, прибыло судно к земле в мире.
\vs Tnf 6:10
И вот, пришёл отец наш, и все мы вместе возвеселились.

\vs Tnf 7:1
2 этих сна поведал я отцу моему, и сказал он мне:
должно тому исполниться в свои времена,
когда многое вынесет Израиль.
\vs Tnf 7:2
Потом сказал отец мой:
верю я Богу, что жив Иосиф, ибо всечасно вижу я,
что числит его с живыми Господь.
\vs Tnf 7:3
И сказал плача: увы, дитя мое Иосиф, ты жив,
a я не вижу тебя, и ты не видишь Иакова,
породившего тебя.
\vs Tnf 7:4
От этих слов его заплакал и я;
и воспылал я сердцем моим возвестить,
что продан был Иосиф, но побоялся я братьев моих.

\vs Tnf 8:1
И вот, дети мои, показал я вам последние времена,
когда всё совершится в Израиле.
\vs Tnf 8:2
А вы поведайте о том детям вашим,
дабы они едины были с Левием и с Иудою.
Ибо через них придёт спасение Израилю,
и в них благословен будет Иаков.
\vs Tnf 8:3
От скипетра их явится Бог [живущий в людях] на земле,
дабы спасти род Израиля и привести к нему праведных из язычников.
\vs Tnf 8:4
И если вы также будете делать добро,
благословят вас люди и ангелы,
и через вас прославлен будет Бог среди народов,
а дьявол бежит от вас, и звери убоятся вас,
и Господь возлюбит вас [и ангелы обнимут вас].
\vs Tnf 8:5
Вырастивший доброго сына стяжает память добрую,
так же и память добрая о благом деле у Бога пребывает.
\vs Tnf 8:6
Того же, кто сотворит недоброе, проклянут его и ангелы,
и люди, а Бога хулить станут язычники из-за него,
а дьявол поселится в нём, как в орудии своём,
а всякий зверь власть будет иметь над ним,
и возненавидит его Господь.
\vs Tnf 8:7
Заповеди же закона двояки, и нужно искусство для их исполнения.
\vs Tnf 8:8
Ибо время сообщению с женой, и время воздержанию для молитвы.
\vs Tnf 8:9
И обе заповеди~--- от Бога,
и если бы не исполнялись они в порядке своём,
грех великий учинялся бы людьми.
Так же и с остальными заповедями.
\vs Tnf 8:10
Будьте же мудры в Боге, дети мои, и благоразумны,
видя порядок заповедей его и законы всех дел,
дабы возлюбил вас Господь.

\vs Tnf 9:1
И много подобного завещав им, просил,
чтобы отнесли кости его в Хеврон и погребли там с отцами его.
\vs Tnf 9:2
И вкушал он, и пил в веселии души, после же закрыл лицо своё и умер.
\vs Tnf 9:3
И сделали сыновья его всё так, как завещал им Неффалим, отец их.

\bibbookdescr{Tgd}{
  inline={Завещание Гада,\\девятого сына Иакова и Зелфы},
  toc={Завещание Гада},
  bookmark={Завещание Гада},
  header={Завещание Гада},
  abbr={Гад}
}
\vs Tgd 1:1
Список завещания Гада,
речённого им к сыновьям своим в 125-ый год жизни его.
Сказал он им:
\vs Tgd 1:2
послушайте, дети мои:
родился я 9-ым сыном Иакова и смелым был на пастбищах.
\vs Tgd 1:3
Стерёг я по ночам стадо, и когда приходил лев,
или волк, или другой зверь на пастбище,
преследовал я его, и настигал, и ловил за ногу его рукою моей,
и бросал его словно камень, и убивал его.
\vs Tgd 1:4
Иосиф же, брат мой, пас стадо вместе с нами около 30-ти дней и,
будучи молод, занемог от зноя.
\vs Tgd 1:5
И возвратился он в Хеврон, к отцу нашему;
и положил его тот рядом с собой,
ибо весьма любил его.

\vs Tgd 1:6
И сказал Иосиф отцу нашему:
сыновья Зелфы и Баллы приносят жертвы
из добрых животных и поедают их, а Рувим и Иуда того не ведают.
\vs Tgd 1:7
Видел же он, как вытащил я барана из пасти медведицы,
и её убил, а барана принес в жертву:
горевали мы, что не может он жить, и съели его.
\vs Tgd 1:8
И оттого гневался я на Иосифа вплоть до дня, когда был он продан.
\vs Tgd 1:9
И был во мне дух ненависти, и не желал я ни слышать об Иосифе,
ни видеть его, ибо в лицо укорял он нас,
говоря, что без Иуды едим мы животных.
Ибо всему, что говорил он, верил отец.

\vs Tgd 2:1
Исповедуюсь ныне в грехе моём, дети,
ибо многократно желал я убить Иосифа, возненавидев его душою.
\vs Tgd 2:2
И за сон его обратил я на него ненависть,
и хотел его извести с земли (живого),
как телец изводит траву на поле.
\vs Tgd 2:3
Но Иуда и я продали его Измаильтянам за 30 золотых, спрятав 10 и показав
только 20 своим братьям.
\vs Tgd 2:4
И так из-за жадности наш замысел был приведён в исполнение.
\vs Tgd 2:5
Так Бог отцов наших избавил его от рук моих,
дабы не сотворил я беззакония великого в Израиле.

\vs Tgd 3:1
А ныне услышьте слово правды:
творите справедливость,
и делайте всё по закону Всевышнего, 
и не соблазняйтесь духом ненависти,
ибо зло это во всех делах человеческих.
\vs Tgd 3:2
Всё, что творит человек, мерзостно для ненавидящего:
если делает по закону Господа, не хвалит его,
если боится Господа и желает справедливости, не любит его.
\vs Tgd 3:3
Правду порицает, счастливому завидует,
злословию радуется, гордыню любит,
ибо ненависть ослепляет душу его, как и меня,
когда смотрел я на Иосифа.

\vs Tgd 4:1
Берегитесь же ненависти, дети мои,
ибо ненавидящий и против Господа беззаконие творит.
\vs Tgd 4:2
Ибо не хочет он слышать заповедей его о любви к ближнему,
и тем грешит против Бога.
\vs Tgd 4:3
Если падёт брат его,
стремится сразу возвестить всем и желает,
чтобы осуждённый и покаранный умер он.
\vs Tgd 4:4
Если же раб какой, клевещет на него перед господином его,
и радуется, если в мучениях умрёт он.
\vs Tgd 4:5
Ибо зависти содействует ненависть также и против счастливых:
видящий успех чей-то или слышащий о нём всегда изнемогает.
\vs Tgd 4:6
Как любовь мёртвых желает оживить и на смерть обречённых
воззвать к жизни хочет, так ненависть живых желает убить
и не хочет, чтобы лишь немного согрешившие живы были.
\vs Tgd 4:7
Ибо дух ненависти через малодушие содействует Сатане
во всём на погибель людей, а дух любви через
долготерпение закону Божию содействует во спасение людей.

\vs Tgd 5:1
И потому ненависть~--- зло,
что постоянно содействует она лжи, говоря против правды,
и малое великим делает, и свет тьмою представляет,
и о сладком говорит, что оно горько,
и клевете научает, и гнев возбуждает,
и войну поднимает, и гордыню, и всякую алчность,
а сердц\acc{а} злом и ядом дьявольским наполняет.
\vs Tgd 5:2
По опыту своему говорю вам это, дети мои, дабы изгнали вы ненависть
дьявольскую и Бога возлюбили.
\vs Tgd 5:3
Праведность изгоняет ненависть,
смирение убивает зависть,
ибо праведный и смиренный стыдится творить неправедное,
и не от того, что другой осудит его, а своё же сердце,
ибо видит Господь душу его.
\vs Tgd 5:4
Не станет говорить он против человека благочестивого,
ибо страх Божий живёт в нём.
\vs Tgd 5:5
Ибо страшась оскорбить Господа, вовек не пожелает
он обидеть и человека, даже и в мыслях своих.
\vs Tgd 5:6
Узнал в конце о том и я, когда раскаялся об Иосифе.
\vs Tgd 5:7
Ибо истинное обращение к Богу [убивает незнание и]
прогоняет тьму, и освещает очи, и знание дает душе,
и помыслы ведёт ко спасению.
\vs Tgd 5:8
И не от людей научился я этому, а в покаянии познал.
\vs Tgd 5:9
Навёл же на меня Бог болезнь печени,
и если бы не помогли мольбы отца моего,
испустил бы я, верно, дух мой.
\vs Tgd 5:10
Ибо чем человек грешит, тем он и карается.
\vs Tgd 5:11
Оттого, что была печень моя безжалостна к Иосифу,
был я осужден на страдание печени немилосердное
в течение 11 месяцев, по времени, что гневался я на Иосифа.

\vs Tgd 6:1
И ныне, дети мои, даю вам совет:
любите каждый ближнего своего, прогоняйте ненависть из сердец ваших.
Возлюбите друг друга делом, словом и помыслом душевным.
\vs Tgd 6:2
Ибо я пред лицом отца моего мирно говорил с Иосифом;
когда же вышел от него, дух ненависти помрачил мой разум
и смутил рассуждение мое, так что захотел я убить Иосифа.
\vs Tgd 6:3
Возлюбите друг друга от сердца, и если кто согрешит против тебя,
говори ему: мир тебе; и не затаи коварства в душе своей.
Если же, раскаявшись, признает он вину свою, отпусти ему.
\vs Tgd 6:4
Если же станет отрицать он, не вступай в спор с ним,
дабы не согрешить дважды, когда он начнёт ругаться.
\vs Tgd 6:5
Да не услышит во время тяжбы чужой человек тайн твоих,
дабы не возненавидел он тебя и не сделался врагом тебе,
и великий грех не сотворил тебе, ибо часто будет
он замышлять коварство и зло творить, вникая в дела твои.
\vs Tgd 6:6
Если же будет он отрицать и, уличённый во грехе,
устыдится, успокойся и не обличай его;
ибо раскается он, что согрешил против тебя,
и, устрашившись, пожелает жить в мире с тобой.
\vs Tgd 6:7
Если же нет в нём стыда и упорствует он во зле,
и тогда отпусти ему от сердца, а возмездие оставь Богу.

\vs Tgd 7:1
И если кто-либо счастливее вас, не огорчайтесь,
но молитесь за него, дабы и в конце был он счастлив,
ибо это полезно вам будет.
\vs Tgd 7:2
И если и далее он возвышается, не завидуйте ему,
помня, что всякая плоть умрет.
Хвалы же возносите Господу,
добро и счастье дающему людям.
\vs Tgd 7:3
Исследуйте суды Господа, и просветит он,
и успокоит помыслы ваши.
\vs Tgd 7:4
Если же кто злом богатеет, как Исав, брат отца моего,
не ревнуйте: ожидайте, что Господь положит предел.
\vs Tgd 7:5
Если отнимется злое богатство, и раскается человек,
простит Господь, а не раскается~--- предан будет на вечные муки.
\vs Tgd 7:6
А бедный, если без зависти радуется он всему,
что дает Господь, превыше всех богатеет,
ибо не ведает он суеты праздных людей.
\vs Tgd 7:7
Удалите же зависть от душ ваших и возлюбите друг друга в прямоте сердца.

\vs Tgd 8:1
Скажите это и вы детям вашим, дабы чтили они Левия и Иуду,
ибо от них восставит Господь спасение Израилю.
\vs Tgd 8:2
Ибо познал я, что отступятся дети ваши от них,
и во всяком зле, вреде и порче будут пред Богом.

\vs Tgd 8:3
И отдохнув немного, сказал ещё:
дети мои, послушайте отца вашего,
и похороните меня рядом с отцами моими.
\vs Tgd 8:4
И вытянув ноги, почил в мире.
\vs Tgd 8:5
И спустя 5 лет отнесли его в Хеврон и положили рядом с отцами его.

\bibbookdescr{Tas}{
  inline={Завещание Асира,\\десятого сына Иакова и Зелфы},
  toc={Завещание Асира},
  bookmark={Завещание Асира},
  header={Завещание Асира},
  abbr={Аср}
}
\vs Tas 1:1
Список завещания Асира, данного им сыновьям его в 125-ый год жизни его.
\vs Tas 1:2
Будучи здоров, говорил он им:
послушайте, дети Асира, отца вашего,
и всё прямое пред лицом Господа покажу вам.

\vs Tas 1:3
2 пути дал Бог сынам человеческим,
и 2 помысла, и 2 дела, и 2 способа, и 2 исхода.
\vs Tas 1:4
Оттого все по 2 одно против другого.
\vs Tas 1:5
Ибо есть 2 пути доброго и злого, и 2 помышления о них в груди нашей,
различающие их.
\vs Tas 1:6
Если желает душа быть доброй,
все дела свои творит она в справедливости,
а если и согрешит, тотчас же кается.
\vs Tas 1:7
Помышляя праведное и отвергая худое,
тотчас же истребляет она зло и с корнем вырывает грех.
\vs Tas 1:8
Если же к худому клонится помышление души,
всякое дело её во зле, и отвергает она добро,
и прилепляется ко злу, и властвует над нею Велиар;
а если и доброе творит, во зло его обращает.
\vs Tas 1:9
Когда начинает творить добро,
исход дела того злым бывает,
ибо сокровище помышления злым духом наполняется.

\vs Tas 2:1
Бывает, что душа на словах доброе выше злого ставит,
но исход дела ее злой.
\vs Tas 2:2
Бывает, что человек не щадит тех,
кто в недобром ему помогает,
и это двулико, но всё в целом~--- зло.
\vs Tas 2:3
Бывает, что человек возлюбит делающего зло,
так что и умереть во зле согласится ради него, и ясно,
что это двулико, но всё в целом~--- злое дело.
\vs Tas 2:4
И если и есть любовь, во зле тот,
кто скрывает злое под именем доброго;
исход же дела недобрый.

\vs Tas 2:5
Иной крадёт, обижает, грабит, корыстолюбив,
но бедных жалеет; и это двулико, но всё в целом~--- зло.
\vs Tas 2:6
Отнимающий у ближнего своего гневит Бога,
ложно клянётся Всевышним,
а нищего жалеет.
Наставляющего в законе Господнем гонит и хулит,
а бедняку подаёт помощь.
\vs Tas 2:7
Душу пятнает он, а тело украшает, многих убивает,
а немногих жалеет, и это двулико, а всё в целом~--- зло.

\vs Tas 2:8
Иной предаётся блуду и разврату, а от пищи воздерживается;
и в посте злые дела творит, и силою богатства многих притесняет,
а наставления даёт несмотря на великое зло своё;
и это двулико, всё же вместе~--- зло.
\vs Tas 2:9
Такие люди~--- как зайцы, ибо наполовину чисты они,
но по правде нечисты.
\vs Tas 2:10
Ибо так сказал Бог на скрижалях заповедей.

\vs Tas 3:1
Вы же, дети мои, сами не будьте двуликими~--- и добрыми,
и злыми вместе, но к одной доброте прилепитесь,
ибо в ней обитает Господь Бог,
и люди её желают.
\vs Tas 3:2
А зла убегайте, убивая помышление злое делами добрыми,
ибо двуликие служат не Богу, но страстям своим,
дабы угодить Велиару и людям, подобным себе.

\vs Tas 4:1
А люди добрые и одноликие праведны пред Богом,
если и говорят двуликие, что согрешают они.
\vs Tas 4:2
Многие убивающие злых 2 дела совершают~--- доброе и злое,
но всё в целом~--- добро, ибо гибнет вырванное с корнем зло.
\vs Tas 4:3
Ненавидящий того, кто и милостив и неправеден вместе,
и блудит и постится вместе, также двуликое совершает,
но всё дело его~--- доброе;
ибо он уподобляется Господу, не принимая за истинное добро то,
что добрым только кажется.
\vs Tas 4:4
Иной же не хочет видеть дня праздничного с распутными,
дабы не осрамить тела своего и не запятнать души своей,
и это двулико, но в целом~--- добро.
\vs Tas 4:5
Такие люди оленям и ланям подобны, ибо они,
имея обличье диких зверей, кажутся нечистыми,
но в целом~--- чисты.
Ведь в ревности Господней живут они, удаляясь от того,
что и Бог возненавидел и запретил заповедями своими,
отделяя доброе от злого.

\vs Tas 5:1
Смотрите, дети, что во всём есть
2 стороны~--- одна противоположна другой,
и одна за другой сокрыта:
в приобретении~--- любостяжательство,
в радости~--- опьянение,
в веселии~--- скорбь,
в браке~--- распутство.
\vs Tas 5:2
Жизни следует смерть,
славе~--- бесчестие,
дню~--- ночь,
свету~--- тьма
и всё под днём, под жизнью~--- праведное, а под смертью~--- неправедное.
Оттого и за смертью грядёт жизнь вечная.
\vs Tas 5:3
И нельзя назвать правду ложью,
или праведное~--- неправедным,
ибо всякая правда~--- в свете, как всё~--- под Богом.
\vs Tas 5:4
Всё это испытал я в жизни моей,
и не уклонялся от правды Господней,
и заповеди Всевышнего изучал,
и был одноликим, всею силою души моей стремясь к добру.
\vs Tas 6:1
Следуйте и вы, дети мои, заповедям Господа,
и будьте одноликими, следуя правде.
\vs Tas 6:2
Ибо двуликие двоякий грех совершают,
ибо и делают злое, и одобряют делающих,
подражая духам соблазна и борясь против людей.
\vs Tas 6:3
Вы же, дети мои, храните закон Господа,
и не внимайте злу, схожему с добром,
а взирайте на то, что сутью своей благо,
и его блюдите по всем заповедям Господним,
в нём пребывая и почивая.
\vs Tas 6:4
Ибо конец жизни человека являет праведность его,
и встречает он либо ангелов Господних, либо Велиаровых.
\vs Tas 6:5
Когда смятенная душа отходит,
обличается она злым духом,
ибо человек тот был рабом страстей и дурных дел.
\vs Tas 6:6
Если же спокойна душа,
в радости узнаёт она ангела мира,
и ведёт он её в жизнь вечную.

\vs Tas 7:1
Не уподобляйтесь Содому,
не узнавшему ангелов Господних
и погибшему навечно.
\vs Tas 7:2
Ибо знаю я, что согрешите вы и преданы
будете в руки врагов ваших,
и земля ваша запустеет, и святыни ваши разрушатся,
вы же рассеяны будете по 4-ём углам земли,
и будете в рассеянии презираемы как вода бесполезная.
\vs Tas 7:3
До той поры будет это, когда посмотрит Всевышний на землю,
и сам придёт как человек, с людьми вкушающий и пьющий,
и снесёт голову дракона в воде, и избавит он Израиля и все народы
Бог, в человека облёкшийся.
\vs Tas 7:4
Скажите же, дети мои, и вы детям вашим об этом,
дабы не ослушались его.
\vs Tas 7:5
Ибо узнал я, что ослушаетесь вы и пребудете в нечестии,
внимая не закону Божию, но советам людским,
совращаясь во зле.
\vs Tas 7:6
И за то разделены будете вы, подобно Гаду и Дану, братьям моим,
чьей земли, рода и языка не узн\acc{а}ете.
\vs Tas 7:7
Но вновь восставит он вас в вере милосердием своим
и ради Авраама, Исаака и Иакова.

\vs Tas 8:1
И сказав это, завещал им: похороните меня в Хевроне.
И умер, почив сном прекрасным.
\vs Tas 8:2
И сделали сыновья его, как завещал он им,
и отнесли его в Хеврон, и погребли там с отцами его.

\bibbookdescr{Tjs}{
  inline={Завещание Иосифа,\\одиннадцатого сына Иакова и Рахили},
  toc={Завещание Иосифа},
  bookmark={Завещание Иосифа},
  header={Завещание Иосифа},
  abbr={Исф}
}
\vs Tjs 1:1
Список завещания Иосифа.
Когда собрался он умирать, то,
призвав сыновей и братьев своих,
сказал им:
\vs Tjs 1:2
братья мои и дети мои,
послушайте Иосифа, возлюбленного Израиля, внемлите речам уст моих.
\vs Tjs 1:3
Видел я в жизни моей зависть и смерть.
И не соблазнился, но пребывал в правде Господней.
\vs Tjs 1:4
Братья мои возненавидели меня, Господь же возлюбил меня.
Они желали меня убить, но Бог отцов моих сохранил меня.
В колодец меня бросили, но Всевышний вывел меня оттуда.
\vs Tjs 1:5
Продан я был в рабство, но Владыка над всеми освободил меня.
Был я в плену, но могучая рука его помогла мне.
Голод мучил меня, но сам Господь накормил меня.
\vs Tjs 1:6
Одинок я был, и Бог утешил меня;
занемог, и Господь посетил меня,
в темнице был я, и Бог мой смиловался надо мною;
горькие слова слышал от Египтян, и избавил меня;
рабом был и возвысил меня.

\vs Tjs 2:1
И главный повар фараона доверил мне дом свой.
\vs Tjs 2:2
И боролся я с женщиной бесстыдной, склонявшей меня согрешить с нею,
но Бог отцов моих избавил меня от огня пылающего.
\vs Tjs 2:3
Заключили меня в темницу, били и насмехались надо мною,
но дал мне Господь благорасположение тюремщика.
\vs Tjs 2:4
Ибо не оставляет Господь боящихся его ни во тьме, ни в оковах,
ни в скорби, ни в нужде.
\vs Tjs 2:5
Ведь не стыдится Бог подобно человеку, и не робеет подобно сыну
человеческому, и не бежит в страхе подобно землеродному.
\vs Tjs 2:6
Но во всех этих печалях помогает он и различными способами утешает,
и лишь ненадолго отступает от человека,
дабы испытать помышление души его.
\vs Tjs 2:7
10-ти испытаниям подверг он меня,
и все их выдержал я терпеливо.
Ибо великое средство~--- долготерпение,
и много благого даёт стойкость.

\vs Tjs 3:1
Сколько раз угрожала мне смертью Египтянка!
Сколь часто, предав меня пыткам, звала к себе,
а когда не хотел я сойтись с нею, говорила мне:
\vs Tjs 3:2
будешь владыкою надо мною и надо всем, что есть в доме моём,
если предашь себя мне, и будешь ты как хозяин наш.
\vs Tjs 3:3
Я же памятовал о словах отца моего и,
войдя в комнату, плача молил Господа.
\vs Tjs 3:4
И постился я тогда 7 лет,
а Египтянам казалось, что живу я в роскоши.
Ибо постящиеся ради Господа радость на лице являют.
\vs Tjs 3:5
Когда же отсутствовал господин мой, не пил я вина
и раз в 3 дня принимал пищу,
а остальное отдавал бедным и слабым.
\vs Tjs 3:6
И на рассвете обращался я к Господу
и плакал о Египтянке из Мемфиса,
ибо непрестанно и премного беспокоила она меня.
Ибо и ночью подходила она ко мне под тем предлогом,
что желает проведать меня.
\vs Tjs 3:7
И поскольку не было у неё ребёнка мужского пола,
делала она так, будто я~--- сын её.
\vs Tjs 3:8
И до времени как сына меня обнимала, а я не знал того.
После же захотела она во блуд вовлечь меня.
\vs Tjs 3:9
И когда понял, опечалился я до смерти.
И когда удалилась она, пошёл я к себе
и горевал о ней многие дни,
ибо познал я хитрость её и соблазн.
\vs Tjs 3:10
И говорил я ей слова Всевышнего,
дабы отвратилась она от страсти злой.

\vs Tjs 4:1
И часто льстила она мне речами своими как святому мужу,
и хитрыми словами хвалила чистоту мою пред лицом мужа своего,
желая ввести меня в искушение,
когда будем мы одни.
\vs Tjs 4:2
Явно прославляла она целомудрие мое,
а втайне говорила мне: не бойся мужа моего, ибо он убеждён
в целомудрии твоём;
и если кто скажет ему о нас, не поверит он.
\vs Tjs 4:3
Тогда я, пав на землю, молил Бога,
чтобы избавил он меня от коварства её.
\vs Tjs 4:4
Когда же ничего не достигла она,
снова приходила ко мне как бы наставления ради,
дабы слушать слово Божие.
\vs Tjs 4:5
И говорила мне:
если хочешь, чтобы оставила я идолов,
сойдись со мною, а я сумею убедить мужа моего отречься от них,
и будем жить пред лицом Господа твоего.
\vs Tjs 4:6
Я же отвечал ей, что не хочет Господь,
чтобы в нечистоте почитали его,
и не развратникам благоволит он,
но только тем, кто с чистым сердцем 
и устами незапятнанными приходит к нему.
\vs Tjs 4:7
Она же была рассержена и желала исполнить желание своё.
\vs Tjs 4:8
А я предался посту и молитве, дабы избавил меня Господь от неё.

\vs Tjs 5:1
И вновь, в иное время сказала она мне:
если блудить не желаешь,
тогда убью я мужа моего ядом и возьму тебя в мужья.
\vs Tjs 5:2
Я же, услышав это, разодрал одежды мои и сказал ей:
женщина, постыдись Бога и не сотвори дела этого злого,
дабы не погибнуть тебе.
Ибо знай, что я разглашу всем этот твой умысел.
\vs Tjs 5:3
Она же, убоявшись, молила меня,
чтобы не разглашал я замысла того.
\vs Tjs 5:4
И удалилась она, ублажив меня дарами и услаждениями всяческими.

\vs Tjs 6:1
А после того послала мне кушанья, намешав в них колдовское зелье.
\vs Tjs 6:2
Но когда пришел евнух и принес кушанья, взглянул я и увидел
испуганного мужа, подающего мне блюдо и нож;
и понял я, что делается это, дабы соблазнить меня.
\vs Tjs 6:3
И когда вышел он, плакал я и не испробовал ни этого,
ни другого какого-либо из кушаний её.
\vs Tjs 6:4
Через день же пришла она ко мне и, увидев, сказала мне:
отчего не отведал ты кушанья?
\vs Tjs 6:5
И отвечал я ей:
оттого, что наполнила ты его зельем смертельным;
и как говорила ты, что, мол, не приближусь я к идолам,
а к одному только Господу?
\vs Tjs 6:6
Ныне же знай, что Бог отца моего открыл мне
через ангела своего зло твоё, и сохранил я кушанье это,
дабы обличить тебя, и, увидев то, быть может, покаешься ты.
\vs Tjs 6:7
Но дабы узнала ты,
что против чтящих Бога в целомудрии не имеет
силы зло нечестивцев,
вот, возьму я от кушанья и съем пред тобою.
И сказав это, помолился я так:
да будет со мною Бог отцов моих и ангел Авраама.
И вкусил я.
\vs Tjs 6:8
Она же, узрев это, пала с плачем на лицо своё к ногам моим,
и поднял я её и вразумлял.
\vs Tjs 6:9
Она же обещала мне не творить никогда нечестия такого.

\vs Tjs 7:1
Но сердце её лежало ещё во зле, и смотрела она,
каким бы способом поймать меня в западню.
И стеная непрестанно, чахла она,
хоть и не была больна.
\vs Tjs 7:2
Увидев же это, сказал ей муж её:
отчего исхудало лицо твоё?
Она же отвечала ему:
страдаю я болью сердечной, и стенание духа мучает меня.
И утешал он её словами своими.
\vs Tjs 7:3
Она же, улучив удобное время, вбежала ко мне,
когда уже ушёл муж её, и сказала мне:
терзаюсь я, и если не возляжешь со мною, брошусь я со скалы.
\vs Tjs 7:4
Я же, поняв, что дух Велиаров мучит её,
обратился с мольбою к Господу и сказал ей:
\vs Tjs 7:5
Что ты, несчастная женщина, терзаешься и мятёшься,
ослеплённая грехом?
помни, что если убьёшь ты себя,
то Астифо, наложница мужа твоего и соперница твоя,
перебъёт всех детей твоих,
и исчезнет память о тебе на земле.
\vs Tjs 7:6
И сказала она мне: вот, всё же ты любишь меня.
Да будет мне довольно этого.
Только вступись за жизнь мою и детей моих,
а я буду ожидать, пока не услажу страсти моей.
\vs Tjs 7:7
Ибо не знала она, что ради Господа моего сказал я так,
а не ради неё.
\vs Tjs 7:8
Но кто одержим страстью желания и рабски служит ей,
как эта женщина, тот, если и доброе что услышит
ко страданию своему, относит это к страсти злой.

\vs Tjs 8:1
И вот, говорю, дети мои, что было около 6-го часа,
когда вышла она от меня.
И преклонив колени к Господу,
стоял я так весь день и всю ночь,
а на рассвете восстал,
плача и моля избавить меня от Египтянки.
\vs Tjs 8:2
И тогда, наконец, схватила она меня за одежды,
силою желая принудить меня сойтись с нею.
\vs Tjs 8:3
И увидев, что в безумии схватила меня за хитон,
оставил его ей и убежал нагим.
\vs Tjs 8:4
Она же, взяв хитон, ложно донесла на меня.
И муж её, придя, заключил меня под стражу
в доме своём и, побив бичами, отослал в темницу фараонову.
\vs Tjs 8:5
И когда был я в оковах, терзалась Египтянка от горя.
И, приходя, внимала она тому, как благодарил
я Господа и пел хвалы ему в доме тьмы и ликовал,
радостным голосом славя Бога моего, ибо избавил
он меня от Египтянки.

\vs Tjs 9:1
Она же часто посылала ко мне, говоря:
благоволи исполнить желание моё,
и я освобожу тебя из оков и от тьмы избавлю.
\vs Tjs 9:2
А я даже мыслию не склонился к ней.
Ведь больше любит Бог целомудренного,
который терпит тьму во рву,
нежели распутника, который роскошествует в царских палатах.
\vs Tjs 9:3
Если же тот, кто живёт в целомудрии, желает и славы,
и знает Всевышний, что это полезно ему,
подаст он, как подал и мне.
\vs Tjs 9:4
Сколько раз она, и будучи больной,
сходила ко мне по вечерам и слушала голос мой,
когда молился я;
я же, слыша стенания её, молчал.
\vs Tjs 9:5
И когда был я в доме её, обнажала она руки и бёдра свои,
дабы возлёг я с нею;
ибо она была прекрасна весьма и украшалась премного,
чтобы соблазнить меня.
И уберёг меня Господь от злых умыслов её.

\vs Tjs 10:1
Зрите же, дети мои, что творит терпение и молитва с постом.
\vs Tjs 10:2
Так и вы, если к целомудрию и чистоте стремиться будете
в терпении и молитве с постом,
в смирении сердечном, поселится в вас Господь,
ибо он любит целомудрие.
\vs Tjs 10:3
А там, где живёт Всевышний, если и зависть приступит,
или рабство, или клевета, Господь, живущий в том человеке,
за целомудрие не только избавит его от бед,
но и возвысит, и прославит, как и меня.
\vs Tjs 10:4
Ибо всякий человек прельщается или в делах, или в словах,
или в помыслах своих.

\vs Tjs 10:5
Знают братья мои, как возлюбил меня отец мой,
но нисколько не возносился я в мыслях моих,
и хоть был ещё ребёнком, страх Божий имел
в сердце моём, ибо знал, что всё прейдет.
\vs Tjs 10:6
И не восстал я в злобе против них,
но почтил их; и уважая их,
даже когда продали меня, умолчал пред Измаильтянами,
что я сын Иакова, мужа великого и праведного.

\vs Tjs 11:1
Так и вы, дети мои, имейте во всяком деле вашем
страх Божий перед очами и чтите братьев ваших,
ибо всякий творящий закон Божий возлюблен им будет.
\vs Tjs 11:2
И когда шёл я с Измаильтянами,
вопрошали они меня: раб ли ты?
И говорил я, что раб домашний,
дабы не опозорить братьев моих.
\vs Tjs 11:3
Говорил же мне старший из них:
ты не раб, ибо видно это по тебе.
Я же сказал им: я ваш раб.
\vs Tjs 11:4
Когда же пришли в Египет, спорили из-за меня,
кто из них даст золото и возьмёт меня.
\vs Tjs 11:5
И решили все, что должен я остаться в Египте
с перекупщиком товаров их, пока не вернутся они с товарами своими.
\vs Tjs 11:6
Господь же даровал мне милость в очах перекупщика того,
и доверил он мне дом свой.
\vs Tjs 11:7
И благословил его Бог рукою моей и обогатил его
золотом, серебром и имуществом.
\vs Tjs 11:8
И был я у него 3 месяца.

\vs Tjs 12:1
А в то время прибыла в пышности великой Мемфиянка,
жена Пентефриса, ибо услышала она обо мне от евнухов своих.
\vs Tjs 12:2
И сказала она мужу своему, что разбогател тот купец
руками некоего юного Еврея, и говорят, что украли его
из земли Ханаанской.
\vs Tjs 12:3
Сотвори же ныне суд и забери юношу в дом наш,
и благословит тебя Бог Еврейский,
ибо благодать небесная на юноше том.

\vs Tjs 13:1
Послушался Пентефрис слов её, и призвал к себе купца,
и сказал ему: что это слышу я о тебе, что крадёшь
ты души из земли Ханаанской,
и в рабы перепродаёшь их?
\vs Tjs 13:2
А купец пал к ногам его и стал умолять:
прошу тебя, господин, не знаю я, что ты говоришь.
\vs Tjs 13:3
И сказал ему Пентефрис: откуда же этот Еврейский юноша?
И отвечал тот: Измаильтяне отдали мне его до той поры,
когда возвратятся они.
\vs Tjs 13:4
И не поверил ему Пентефрис, но приказал раздеть его донага и бить.
Когда же оставался тот при словах своих, сказал Пентефрис:
да будет приведён юноша.
\vs Tjs 13:5
И войдя, поклонился я Пентефрису,
ибо он был 3-им в ряду владык после фараона.
\vs Tjs 13:6
И отведя меня в сторону, вопросил он:
раб ты или свободный?
Я же ответил: раб.
\vs Tjs 13:7
И вопросил он: чей?
И сказал я: Измаильтян.
\vs Tjs 13:8
Он же вопросил: как сделался ты рабом их?
И отвечал я: в земле Ханаанской купили они меня.
\vs Tjs 13:9
И сказал он мне: ты лжёшь.
И тотчас приказал бить и меня нагого.

\vs Tjs 14:1
А Мемфиянка видела через окно,
как били меня, ибо рядом был дом её,
и послала к Пентефрису, говоря:
неправеден суд твой, ибо свободного
и украденного наказываешь ты как преступника.
\vs Tjs 14:2
А я не отказывался от слов моих, хотя и били меня,
и приказал он охранять меня, пока, сказал он,
не придут хозяева юноши.
\vs Tjs 14:3
И сказала ему жена его: за что мучаешь ты и держишь
в оковах юношу, попавшего в плен, коего лучше
бы было освободить, дабы служил он тебе?
\vs Tjs 14:4
Ибо желала она видеть меня, чтобы совершить грех,
а я не знал ничего об этом.
\vs Tjs 14:5
И сказал ей муж её: не отнимают чужого Египтяне,
пока не совершится разбирательство.
\vs Tjs 14:6
А затем сказал купцу: юноша должен быть заключен в тюрьму.

\vs Tjs 15:1
Спустя же 24 дня пришли Измаильтяне;
ибо услышали они, что Иаков, отец мой, премного печалится обо мне.
И придя, сказали они мне: 
\vs Tjs 15:2
что же это ты назвал себя рабом?
И вот, узнали мы, что ты сын человека великого в земле Ханаанской,
и печалится о тебе отец твой во вретище и пепле.
\vs Tjs 15:3
Когда услышал я это, размягчилось и растаяло сердце моё,
и хотел я заплакать громко, но сдержал себя,
дабы не опозорить братьев моих, и сказал им:
ничего не знаю, раб я.
\vs Tjs 15:4
Тогда решили они продать меня, дабы не был я найден
в руках у них.
\vs Tjs 15:5
Ибо они страшились отца моего, как бы не пришёл он,
дабы отомстить им ужасно.
Слышали они, что велик он пред Богом и людьми.
\vs Tjs 15:6
Тут сказал им купец: избавьте меня от суда Пентефриса.
\vs Tjs 15:7
И пошли они и просили меня:
скажи, что за серебро был продан ты нам,
и он освободит нас от ответственности.

\vs Tjs 16:1
А Мемфиянка сказала мужу своему:
купи этого юношу, ибо я слышу, говорят,
что продают его.
\vs Tjs 16:2
И послала она евнуха к Измаильтянам с просьбой купить меня.
\vs Tjs 16:3
А евнух не купил меня, но возвратился и сказал госпоже своей,
что большую цену просят они за юношу.
\vs Tjs 16:4
И послала она евнуха обратно, говоря:
если и 2 мины просят они, дай им,
не жалей золота;
только купи юношу и приведи его ко мне.
\vs Tjs 16:5
И пошёл евнух и, отдав им 80 золотых, взял меня;
Египтянке же сказал он, что отдал 100.
\vs Tjs 16:6
А я знал о том, но промолчал, дабы не опозорить евнуха.

\vs Tjs 17:1
Смотрите же, дети мои, сколько пришлось перенести мне,
дабы не опозорить братьев моих.
\vs Tjs 17:2
И вы любите друг друга, и в долготерпении скрывайте
прегрешения друг друга.
\vs Tjs 17:3
Ибо радуется Бог единомыслию братьев и помыслу сердца благого,
стремящегося к добру.
\vs Tjs 17:4
Когда же пришли братья мои в Египет, знают они,
что возвратил я им серебро, и не укорял их,
и утешил их.
\vs Tjs 17:5
А после смерти Иакова, отца моего, ещё более возлюбил их и всё,
чего желали они, в изобилии делал им.
\vs Tjs 17:6
И не допускал я, чтобы горевали они хотя бы из-за самого малого,
и всё, что было в руке моей, давал им.
\vs Tjs 17:7
И сыновья их~--- мои сыновья, а мои сыновья~--- как рабы их,
и душа их~--- моя душа, и всякая боль их~--- моя боль,
и всякая истома их~--- моя болезнь,
и воля их~--- моя воля.
\vs Tjs 17:8
И не превозносился я среди них, хвалясь славой моей в мире,
но был среди них как один из малейших.

\vs Tjs 18:1
Если и вы, дети мои, жить будете по заповедям Господним,
возвысит вас Бог вовеки.
\vs Tjs 18:2
И если кто-либо пожелает зло сделать вам,
сотворите доброе дело и помолитесь за него,
и ото всякого зла избавлены будете вы Господом.
\vs Tjs 18:3
Ибо вот, видите вы, что за смирение и долготерпение
моё взял я в жёны себе дочь жреца Гелиопольского,
и 100 талантов золота дали мне с нею,
и сделал их Господь мой рабами моими.
\vs Tjs 18:4
И обличье прекрасное дал он мне превыше прекрасных в Израиле,
и до старости во здравии и в красоте хранил меня.
Ибо во всём был я подобен Иакову.

\vs Tjs 19:1
Услышьте же, дети мои, также и о сне, который видел я.
\vs Tjs 19:2
Видел я 12 оленей, которые паслись,
и 9 из них были рассеяны по всей земле,
3 же спаслись, но на следующий день и они были рассеяны.
\vs Tjs 19:3
И узрел я, что 3 оленя сделались 3-мя агнцами и возопили к Господу,
и привёл их Господь на место цветущее и водою обильное
и вывел из тьмы на свет.
\vs Tjs 19:4
И тут возопили к Господу 9 оленей,
потом собрались они и стали как 12 овец
и в недолгом времени увеличились
и стали многими стадами.
\vs Tjs 19:5
После того взглянул я, и вот, явилось 12 быков,
сосущих одну телицу, которая море молока давала,
и пили от неё 12 стад и бесчисленные стада.
\vs Tjs 19:6
И у 4-го быка выросли рога до неба и стали
как стена для стад, а между двух рогов вырос иной рог.
\vs Tjs 19:7
И узрел я тельца, который двенадцатикратно окружил их,
и подал он помощь всем быкам.
\vs Tjs 19:8
И увидел я среди рогов некую деву,
имеющую пёструю одежду,
и от неё произошёл агнец,
и слева от него~--- лев,
и пошли против него все звери и все гады,
и победил их агнец, и погубил их.
\vs Tjs 19:9
И радовались ему быки, и телица, и ангелы, и вся земля.
\vs Tjs 19:10
И должно тому быть в последние дни.

\vs Tjs 19:11
Вы же, дети мои, храните заповеди Господа и чтите Левия и Иуду,
ибо от семени их придёт агнец Божий, дабы принять на себя грех мира,
Спаситель всех народов и Израиля.
\vs Tjs 19:12
Ибо царствие его будет вечным, и не прейдёт оно.
Моего же царства, которое в вас, не станет,
словно сторожки в саду, что уничтожается по прошествии лета.

\vs Tjs 20:1
Знаю я, что после кончины моей притеснять будут вас Египтяне,
и Бог отомстит за вас и приведёт вас к обещанному отцам моим.
\vs Tjs 20:2
Вы же возьмите с собою кости мои, ибо,
когда понесёте вы туда эти кости,
будет с вами Господь в свете, а Велиар во тьме будет с Египтянами.
\vs Tjs 20:3
А мать свою Асинефу отведите к Ипподрому и похороните её рядом с Рахилью,
матерью моей.

\vs Tjs 20:4
И сказав это, вытянул он ноги свои и почил сном прекрасным.
\vs Tjs 20:5
И оплакал его весь Израиль и весь Египет в скорби великой.
Ибо и для Египтян был он как соплеменник их,
и добро им творил, помогая во всем и советом, и делом своим.
\vs Tjs 20:6
А когда вышли сыны Израиля из Египта,
взяли они с собою кости Иосифа и погребли их
в Хевроне с отцами его.
И было лет жизни его 110.

\bibbookdescr{Tbn}{
  inline={Завещание Вениамина,\\двенадцатого сына Иакова и Рахили},
  toc={Завещание Вениамина},
  bookmark={Завещание Вениамина},
  header={Завещание Вениамина},
  abbr={Внм}
}
\vs Tbn 1:1
Список слов Вениамина, которые сказал он сыновьям своим,
прожив 125 лет.
\vs Tbn 1:2
Поцеловав их, молвил он:
как Исаак родился у Авраама в старости его,
так же и я родился у Иакова.
\vs Tbn 1:3
А Рахиль, мать моя, родив меня, умерла, и я не имел молока.
Потому кормила меня Балла, служанка её.
\vs Tbn 1:4
Рахиль, родив Иосифа, 12 лет была неплодна,
и молила Господа, и постилась, и зачав, родила меня.
\vs Tbn 1:5
Ибо премного любил отец мой Рахиль
и желал видеть двоих сыновей от неё.
\vs Tbn 1:6
Оттого назван был я Вениамин, то есть сын дней.

\vs Tbn 2:1
Когда же пришёл я в Египет, узнал меня брат мой Иосиф,
и спросил он меня: что сказали братья мои отцу,
когда продали меня?
\vs Tbn 2:2
И сказал я ему:
вымазали они хитон твой кровью и отослали его отцу, говоря:
узнай, сына ли твоего этот хитон.
\vs Tbn 2:3
И сказал он мне: да, брат, ибо взяли меня Измаильтяне,
и один из них снял с меня хитон, дал мне какую-то одежду,
ударил бичом и велел бежать.
\vs Tbn 2:4
И пошёл он спрятать одежду мою,
и встретился ему лев и убил того Измаильтянина.
\vs Tbn 2:5
И те, кто был с ним, устрашились и продали меня другим людям.
\vs Tbn 2:6
И не солгали братья мои в словах своих.
Ибо Иосиф желал скрыть от меня дела братьев наших,
и позвав их к себе, сказал им:
\vs Tbn 2:7
не говорите отцу моему, что сделали вы мне,
но так скажите, как рассказал я Вениамину.
\vs Tbn 2:8
И да будут мысли ваши такими же,
и да не дойдут слова эти до сердца отца моего.

\vs Tbn 3:1
И ныне, дети мои, возлюбите вы Господа Бога небес и земли,
и храните заповеди его, уподобляясь доброму и благочестивому
мужу Иосифу.
\vs Tbn 3:2
И да будут помыслы ваши добрыми,
как вы знаете то обо мне.
Ибо имеющий правильные помыслы всё правильно видит.
\vs Tbn 3:3
Бойтесь Господа и любите ближнего;
и если духи Велиаровы во всякую злую печаль ввергнут вас,
да не обретут власти над вами, как не смогли того над Иосифом,
братом моим.
\vs Tbn 3:4
Сколь многие люди желали убить его, и Бог защитил его.
Ибо тот, кто боится Бога и любит ближнего,
не будет сражен духом Велиаровым,
но защитит его страх Божий.
\vs Tbn 3:5
И кознями людей или зверей не может он быть порабощён,
но поможет ему любовь, которую имеет он к ближнему.
И до смерти Иакова не хотел Иосиф говорить о том,
но Иаков, узнав от Господа, сказал ему.
Но и тогда отрицал Иосиф, и едва убедился клятвами Израиля.
\vs Tbn 3:6
И просил Иосиф отца нашего помолиться за братьев его,
дабы не зачёл им Господь грех тот злой,
что совершили против него.
\vs Tbn 3:7
И воскликнул Иаков: О достойное дитя,
победил ты сердце Иакова, отца своего;
и обняв его, целовал 2 часа, говоря:
\vs Tbn 3:8
Исполнится на тебе пророчество небесное об агнце Божием
и Спасителе мира, что безупречный предан будет за беззаконников,
а безгрешный умрёт за нечестивцев в крови Завета во спасение народов
и Израиля, и уничтожит Велиара и слуг его.

\vs Tbn 4:1
Зрите же, дети мои, каков исход доброго мужа.
В доброте уподобляйтесь милосердию его, дабы и вам носить венцы славы.
\vs Tbn 4:2
Ибо у доброго человека око не омрачится, он ведь жалеет всех, если и
грешники это.
\vs Tbn 4:3
Если и недоброго желают ему, всё же творящий добро побеждает зло,
обороняемый Богом.
Праведных же любит он как душу свою.
\vs Tbn 4:4
Если кто славен, не завидует ему; если кто богат, не ревнует;
если мужествен кто, хвалит его; мудрого любит он, бедного жалеет;
слабому сострадает, Бога славит.
\vs Tbn 4:5
Имеющего страх Божий защищает он, любящему Господа помогает;
отвергающего Всевышнего наставляет он и обращает,
а имеющего благодать доброго духа любит как душу свою.

\vs Tbn 5:1
Если и вы будете иметь добрые помыслы,
то даже злые люди примирятся с вами,
и распутные устыдятся вас и обратятся ко благу,
и любостяжатели не только отступят от страсти своей,
но и то, что нажили они алчностью, отдадут страждущим.
\vs Tbn 5:2
Если будете творить добро, то и нечистые духи побегут от вас,
и звери устрашатся вас.
\vs Tbn 5:3
Ибо где свет добрых дел, там и тьма бежит от него.
\vs Tbn 5:4
И тот, кто надменно хулит благочестивого мужа, раскается,
ибо жалеет благочестивый хулителя и молчит.
\vs Tbn 5:5
И если кто предаст праведника, будет молиться тот.
И пусть ненадолго унижен будет, вскоре ещё светлее засияет,
как было то с Иосифом, братом моим.

\vs Tbn 6:1
Помышление доброго мужа~--- не в соблазняющей руке духа Велиарова.
Ибо ангел мира ведёт душу его.
\vs Tbn 6:2
И не взирает он с вожделением на тленное
и не собирает золота из любви к наслаждениям.
\vs Tbn 6:3
Не радуется он наслаждениям, не обижает ближнего,
не наполняется роскошью, не соблазняется взорами очей.
Ибо Господь~--- удел его.
\vs Tbn 6:4
Доброе помышление не внимает ни славе, ни хуле человеческой,
и ни лжи, ни спора, ни хулы не ведает.
Ибо Господь обитает в нём, и освещает душу его,
и радуется он за всех во всякий час.
\vs Tbn 6:5
Благой помысел не имеет двух языков~--- благословения и проклятия,
чести и поругания, покоя и смятения, лицемерия и правды,
бедности и богатства, но обо всех у него чистое и незамутненное суждение.
\vs Tbn 6:6
Нет у такого человека ни зрения двойного, ни слуха, ибо во всём,
что делает и что говорит, знает, что видит Господь душу его.
\vs Tbn 6:7
И очищает он помыслы свои, дабы не осудили его Бог и люди.
У Велиара же всякое дело двойное, и нет в нём простоты.
\vs Tbn 7:1
Потому, дети мои, говорю вам:
убегайте зла Велиарова, ибо нож дает он повинующимся ему.
\vs Tbn 7:2
А нож этот 7 зол порождает, сначала же зачинает мысль от Велиара.
И первое зло~--- убийство,
второе~--- разрушение,
третье~--- угнетение,
четвёртое~--- изгнание,
пятое~--- нужда,
шестое~--- смятение,
седьмое~--- опустошение.
\vs Tbn 7:3
Оттого и Каин 7-ми возмездиям подвергся от Господа,
ибо каждые 100 лет по одному удару наносил ему Господь.
\vs Tbn 7:4
Когда было Каину 200 лет, начал он получать их,
а в 900 был повержен за Авеля, праведного брата его.
7 зол было суждено Каину, а Ламеху~--- 70 раз 7.
\vs Tbn 7:5
Ибо до века будут караться таким судом подражающие Каину
в зависти и ненависти к братьям.
 
\vs Tbn 8:1
Вы же, дети мои, убегайте злобы, зависти и ненависти к братьям,
а прилепитесь к доброте и любви.
\vs Tbn 8:2
Ибо имеющий чистый помысел не взирает на женщину для блуда,
и незапятнано сердце его, ибо почиет на нем дух Божий.
\vs Tbn 8:3
Ибо как солнце не оскверняется, если и видит грязь и нечистоты,
но напротив, оно высушивает их и дурной запах изгоняет,
так же и чистый ум, в мерзостях земных пребывающий,
скорее очищает их, сам же не оскверняется.

\vs Tbn 9:1
Скажу вам, по словам Еноха праведного, и о недобрых делах ваших,
ибо блудить станете вы блудом Содомским, и не останется вас,
кроме немногих.
И вновь с женщинами предадитесь распутству,
и не будет в вас царства Божия,
ибо Господь тотчас заберёт его.
\vs Tbn 9:2
Только в одном уделе вашем возникнет храм Божий,
и будет последний славнее первого, и соберутся туда 12 колен
и все народы до той поры, когда пошлёт Всевышний спасение своё
посещением единородного Пророка.
\vs Tbn 9:3
И войдёт он в 1-ый храм, и там будет поруган Господь и поднят на древо.
\vs Tbn 9:4
И раздерётся завеса в храме, и перейдёт дух Божий к народам,
словно огонь прольётся.
\vs Tbn 9:5
И поднявшись из ада, взойдет он с земли на небо.
Познал я, сколь смирен будет он на земле и сколь прославлен на небе.

\vs Tbn 10:1
Когда же был Иосиф в Египте, желал я видеть лицо его и обличье его,
и по молитвам Иакова, отца моего, узрел я его, бодрствуя днём,
таким, каким был весь вид его.
\vs Tbn 10:2
И сказал им затем: Знайте, дети мои, что я умираю.
\vs Tbn 10:3
Творите же правду каждый ближнему своему, и закон Господа,
и заповеди его храните.
\vs Tbn 10:4
Ибо оставляю вам это вместо всякого наследства,
а вы передайте детям вашим на владение вечное,
ибо делали так Авраам, Исаак и Иаков.
\vs Tbn 10:5
И всё это оставили они нам в наследство, сказав:
Храните заповеди Бога до той поры, когда откроет Господь
спасение своё всем народам.
\vs Tbn 10:6
И тогда узрите вы Еноха, и Ноя, и Сима, и Авраама,
и Исаака, и Иакова восставшими одесную его в радости.
\vs Tbn 10:7
Тогда и мы воскреснем, каждый в уделе власти своей
и преклонимся пред Царём Небесным на землю явившимся
в обличье человеческом смиренно, и те, кто уверует
в него на земле, возрадуются с ним.
\vs Tbn 10:8
И все воскреснут: одни~--- для славы, другие~--- для бесславия,
и будет судить Господь первых Израиля за неправедность
их ибо не уверовали они в Бога, явившегося во плоти.
\vs Tbn 10:9
После же будет судить он все народы ибо не уверовали в него,
явившегося на землю.
\vs Tbn 10:10
И обличит он Израиля через народы избранные,
как обличил он Исава через Мадианитян,
возлюбивших братьев их.
Будьте же, дети мои, в уделе боящихся Господа.
\vs Tbn 10:11
Если пребудете вы в святости, дети мои,
и по заповедям Господа, то в твёрдой надежде будете вновь жить со мною,
и соберётся пред Господом весь Израиль.

\vs Tbn 11:1
И не назовусь я более волком хищным за хищность вашу,
но работником Господним, пищу раздающим тем, кто творит добро.
\vs Tbn 11:2
И восстанет в последние времена возлюбленный Господа от семени Иуды и Левия,
творящий благоволение уст его знанием новым освещая все народы.
Свет знания, придёт он к Израилю во спасение его,
и похитит у них как волк и отдаст собранию народов.
\vs Tbn 11:3
До скончания века пребудет он в собраниях народов и во властителях их,
словно песня сладкозвучная на устах всех.
\vs Tbn 11:4
И записан будет он в книги святые, и дело, и слово его,
и будет он избранником Божиим до века.
\vs Tbn 11:5
И будет ходить он среди них, подобно Иакову, отцу моему, говоря:
Сам восполнит он недостаток племени твоего.

\vs Tbn 12:1
И закончив речи свои, сказал он: Завещаю вам, дети мои,
вынесите кости мои из Египта и похороните меня в Хевроне рядом с отцами моими.
\vs Tbn 12:2
И умер Вениамин, будучи 125-и лет, в старости прекрасной,
и положили его во гроб.
\vs Tbn 12:3
И в 91-ый год прихода сынов Израиля в Египет,
отнесли кости отца своего тайно,
во время войны Ханаанской, в Хеврон и погребли там у ног отцов его.
\vs Tbn 12:4
А сами возвратились они из земли Ханаанской
и пребывали в Египте вплоть до дней исхода их из земли Египетской.

\bibbookdescr{Pss}{
  inline={Псалмы Соломона},
  toc={Псалмы Соломона},
  bookmark={Псалмы Соломона},
  header={Псалмы Соломона},
  abbr={Псс}
}
\vs Pss 1:1
Воззвал я к Господу в смертной тоске моей, к Богу воззвал в засильи
грешников.
\vs Pss 1:2
Раздался внезапно клич военный предо мною: я буду услышан, ибо
исполнился праведности.
\vs Pss 1:3
Помыслил я в сердце моем, что исполнился праведности, ибо
достиг процветания и умножился в потомках моих.
\vs Pss 1:4
Пусть богатство их разойдется по всей земле, а слава~--- до
края земли.
\vs Pss 1:5
Возвысились они до звезд, и сказал я: Нет, они не упадут!
\vs Pss 1:6
Но возгордились в богатстве своем и не удержались.
\vs Pss 1:7
Грехи их~--- в тайне, и я не ведал.
\vs Pss 1:8
Беззакониями своими превзошли они племена, бывшие до них,
осквернили святыни Господни скверною.

\vs Pss 2:1
Возгордился грешник, стенобитным орудием сокрушил стены
крепкие, и Ты не воспрепятствовал.
\vs Pss 2:2
Взошли к жертвеннику Твоему народы чуждые, попирали сандалиями
своими в надменности,
\vs Pss 2:3
за то, что сыны Иерусалимские осквернили святыни Господни,
принесли дары Богу в нечестии.
\vs Pss 2:4
Из-за таких их дел сказал Он: Отбросьте это далеко от Меня, нет
на том Моего благоволения.
\vs Pss 2:5
Блеск славы их обратился в ничто, пред лицем Божиим унижен
окончательно.
\vs Pss 2:6
Сыны и дочери - в плену позорном, печать на шеях их,
заклеймлены они между народами.
\vs Pss 2:7
По прегрешениям их сделал им: предал в руки превосходящих
силою,
\vs Pss 2:8
отвратил лице Свое, отказался миловать их~--- юношу и
старика, детей вместе с ними.
\vs Pss 2:9
Ибо недостойное совершили все вместе~--- не слушали.
\vs Pss 2:10
И небосвод омрачился, и земля отвратилась от них,
\vs Pss 2:11
ибо не сделал всякий человек на ней, сколько они сотворили.
\vs Pss 2:12
И узнает земля праведность судов Твоих, Боже.
\vs Pss 2:13
И вот, выставил Бог сынов Иерусалимских на глумление: вместо
притонов разврата прохожий всякий сюда
сворачивал, при свете дня смеялись беззаконию своему,
\vs Pss 2:14
тем же, что совершали при свете дня, выставили напоказ нечестие
свое.
И дочери Иерусалимские доступны стали все по суду Твоему,
\vs Pss 2:15
за то, что осквернили себя, сходясь с кем попало. Чревом и утробой
моей скорблю о них.
\vs Pss 2:16
Я оправдаю Тебя, Боже, в правоте сердца, ибо в судах
Твоих~--- правда Твоя, Боже,
\vs Pss 2:17
ибо воздал Ты грешникам по делам их, по грехам их, тяжким
весьма.
\vs Pss 2:18
Открыл Ты грехи их, дабы ясен стал приговор Твой,
\vs Pss 2:19
стер память о них с лица земли.
Бог~--- Судия праведный и не смотрит на лица.
\vs Pss 2:20
Совлек Он красу Иерусалима с престола славы, предали поруганию его
язычники, топча ногами.
\vs Pss 2:21
Препоясал он вретище вместо красивых одежд, бечеву вокруг головы
своей вместо венца,
\vs Pss 2:22
снял митру славы, которую возложил на него Бог:
\vs Pss 2:23
в бесчестии пала краса его на землю.
\vs Pss 2:24
И увидел я, и вознес мольбу к лицу Господа, и сказав: Довольно,
Господи, тяготела рука Твоя над Иерусалимом~--- положи конец нашествию
язычников!
\vs Pss 2:25
Ибо насмеялись и не знали пощады в гневе и ярости злобной.
\vs Pss 2:26
И довершат они дело свое, если Ты, Господи, не накажешь их в гневе
Твоем,
\vs Pss 2:27
ибо не из любви к Тебе сделали так, но из похоти душевной
\vs Pss 2:28
излили гнев свой на нас в расхищении.
Так не помедли, Боже, воздать им на головы их,
\vs Pss 2:29
дабы обречь высокомерие дракона на поругание!
\vs Pss 2:30
И не помедлил Бог, явив мне надменность его пронзенной на горах
Египетских, хуже самого последнего униженной на земле и на море,
\vs Pss 2:31
тело его~--- носимым на волнах в полном нечестии, и не было
никого, кто бы похоронил:
\vs Pss 2:32
ибо унизил его Господь и обрек на поругание. И не помыслил он, что
человек,
и о последнем не помыслил.
\vs Pss 2:33
Сказал: Господином земли и моря сделаюсь!~--- и не знал, что Бог
велик, могуч в крепости Своей великой.
\vs Pss 2:34
Он~--- Царь на небесах, судящий царей и владык,
\vs Pss 2:35
пробуждающий меня к славе и усыпляющий надменных к погибели вечной в
бесчестии, ибо не узнали Его.
\vs Pss 2:36
Теперь же узрите, владыки земные, суд Господень, ибо велик Царь и
праведен, судящий поднебесную.
\vs Pss 2:37
Благословляйте Бога, боящиеся Господа в мудрости, ибо милость
Господня в день суда~--- на боящихся Его,
\vs Pss 2:38
когда разведет Он праведного и грешника, воздаст грешникам навеки по
делам их
\vs Pss 2:39
и помилует праведного, освободит от унижения перед грешником,
воздаст грешнику за то, что он сделал праведному.
\vs Pss 2:40
Ибо благ Господь к уповающим на Него с твердостью. Да сделает Он по
милости Своей тем, кто с Ним, да пребудут непрестанно пред лицем Его в
силе.
\vs Pss 2:41
Славен Господь вовек пред лицем слуг Его!

\vs Pss 3:1
Что ты спишь, душа, и не славишь Господа?
\vs Pss 3:2
Пойте новую песнь Богу хвалимому! Пой и бодрствуй, дабы служить
Ему, ибо псалом Богу, идущий от всего сердца, благ.
\vs Pss 3:3
Праведные всегда помнят Господа, признают и хвалят суды
Господни.
\vs Pss 3:4
Не пренебрежет праведный вразумлением Господним, неизменно
благоволение его к Господу.
\vs Pss 3:5
Оступился праведный~--- и прославил Господа, упал~--- и
взирает, что сделает ему Бог,
\vs Pss 3:6
смотрит, откуда придет спасение к нему.
\vs Pss 3:7
Честны праведные пред Богом, Спасителем их, нет в доме
праведного прегрешения на прегрешении.
\vs Pss 3:8
Праведный всегда блюдет дом свой, и в падении своем изгоняя
неправедность.
\vs Pss 3:9
Постясь, молит он простить неведение его и смиряет душу
свою.
\vs Pss 3:10
И Господь очищает всякого, кто благочестив, и дом его.
\vs Pss 3:11
Оступился грешник~--- и проклинает жизнь свою, день, в который
родился, и материнское чрево.
\vs Pss 3:12
Громоздил он прегрешения на прегрешения в жизни своей.
\vs Pss 3:13
Он пал, ибо мерзостно тело его, и уже не восстанет: гибель
грешника~--- навеки,
\vs Pss 3:14
И когда воззрит на праведных, не опомнится он.
\vs Pss 3:15
Таков удел грешников навеки.
\vs Pss 3:16
Те же, кто боится Господа, восстанут для жизни вечной, и жизнь
их~--- в свете Господнем и не угаснет никогда.

\vs Pss 4:1
Что ты сидишь, нечестивый, в синедрионе? Сердце твое далеко
отступило от Господа, беззакониями гневишь ты Бога Израиля!
\vs Pss 4:2
Чрезмерен он речами, чрезмерен и подозрительностью более всех,
жесток в словах, когда осуждает грешников на суде.
\vs Pss 4:3
И рука его среди первых ляжет на грешника, словно бы в
ревности, сам же он отягощен пестротою грехов и неумеренностью.
\vs Pss 4:4
Глаза его~--- на всякой женщине без различия, язык его лжив
в договоре клятвенном.
\vs Pss 4:5
В ночи и в тайне согрешает он, словно никем не зрим, глазами
говорит он со всякою женщиной, замышляя недоброе,
\vs Pss 4:6
скоро входит он во всякое жилище, веселясь духом, словно
безгрешен.
\vs Pss 4:7
Истреби, Боже, живущих в лицемерии рядом с благочестивыми,
истреби разрушением плоти и бедностью жизнь их.
\vs Pss 4:8
Открой, Боже, дела людей-человекоугодников, да будут осмеяны и
поруганы дела их.
\vs Pss 4:9
И да признают благочестивые праведность кары Бога их, когда
истребляет Он грешников от лица праведника,
\vs Pss 4:10
человекоугодников, говорящих закон с хитростью.
\vs Pss 4:11
И словно змея глаза их в жилище человека твердого, дабы разрушить
мудрость невинных речами преступными.
\vs Pss 4:12
Речи его~--- обман, дабы потворствовать вожделениям
неправедного.
\vs Pss 4:13
Не отступал он, пока не рассеет как сирот и не опустошит дом ради
вожделения преступного.
\vs Pss 4:14
Лгал в речах своих, будто нет Видящего и Судящего.
\vs Pss 4:15
Исполнился он беззакония, и глаза его на доме чужом, дабы разрушить
его речами
соблазна. И не насыщается всем этим душа его, подобно преисподней.
\vs Pss 4:16
Да будет, Господи, удел его~--- бесчестие пред лицом Твоим,
выхождение его~--- в стенаниях, а вхождение его~--- в несчастии.
\vs Pss 4:17
В скорбях и в нищете и в смятении жизнь его, Господи: сон его~--- в
скорбях, а пробуждение его~--- в смятении.
\vs Pss 4:18
Да отнимется сон от висков его в ночи, да отпадет он бесславно от
всякого дела рук своих.
\vs Pss 4:19
С пустыми руками да войдет он в дом свой, и пусть не станет в доме
том ничего, чем насыщал он душу свою.
\vs Pss 4:20
В одиночестве бездетном старость его до самой смерти.
\vs Pss 4:21
Да будет разбросана плоть человекоугодников зверями, а кости
преступников~--- под солнцем в бесславии.
\vs Pss 4:22
Да выклюют вороны глаза людей льстивых,
\vs Pss 4:23
ибо опустошили они многие жилища людские в бесславии и разметали в
вожделении.
\vs Pss 4:24
И не вспомнили они о Боге, и не убоялись Бога во всем том,
\vs Pss 4:25
но прогневили они Бога и рассердили. Да истребит Он их от земли, ибо
души кротких совратили они обманом.
\vs Pss 4:26
Блаженны боящиеся Господа в кротости своей.
\vs Pss 4:27
Избавит их Господь от людей коварных и грешников и избавит нас от
всякого соблазна преступного.
\vs Pss 4:28
Да истребит Бог творящих в гордыне всякую неправедность, ибо
Он~--- великий Судия и могучий Господь Бог наш праведный.
\vs Pss 4:29
Да будет, Господи, милость Твоя на всех любящих Тебя.

\vs Pss 5:1
Господи Боже, прославлю имя Твое в ликовании средь видящих
праведность судов Твоих.
\vs Pss 5:2
Ибо Ты благ и милостив, Ты~--- прибежище бедного.
\vs Pss 5:3
Когда я взываю к Тебе, не отвратись в молчании от
меня;
\vs Pss 5:4
ибо не возьмет добычи человек от мужа могучего.
\vs Pss 5:5
Да и кто возьмет из всего, что сотворил Ты, если Ты не
дашь?
\vs Pss 5:6
Ибо человек и жребий его у Тебя на весах, и не прибавит он,
чтобы наполнить сверх суда Твоего, Боже!
\vs Pss 5:7
В печали нашей призовем на помощь Тебя, и Ты не отвергнешь
мольбу нашу, ибо Ты Единый Бог наш.
\vs Pss 5:8
Да не опустишь тяжко руки Твоей на нас, чтобы не пришлось
согрешить нам.
\vs Pss 5:9
И если не обратишься к нам, не покинем Тебя, но последуем за
Тобою.
\vs Pss 5:10
Ибо если взалчу, воззову к Тебе, Боже, и Ты дашь мне.
\vs Pss 5:11
Пернатых и рыб кормишь Ты, давая дождь в пустынях для взращения
зелени, дабы уготовить пропитание в пустыне всякому живущему.
\vs Pss 5:12
И если взалчут они, к Тебе поднимут лица свои.
\vs Pss 5:13
Царей и начальников и народы Ты, Боже, питаешь; нищего и убогого кто
надежда, если не Ты, Господи?
\vs Pss 5:14
И Ты услышишь~--- ибо кто благ и праведен, кроме Тебя?~--- услышишь,
как ликует душа смиренного, когда раскрываешь Ты милостиво руку
Твою.
\vs Pss 5:15
Доброта человеческая к ближнему скупа и холодна, и если повторит
даяние без ропота, и тому подивишься.
\vs Pss 5:16
Но Твое даяние великое благостно и обильно, и тот, чья надежда на
Тебя, Господи, не поскупится на даяние сам.
\vs Pss 5:17
На всей земле милость Твоя благостная, Господи!
\vs Pss 5:18
Блажен тот, о ком помнит Бог, давая ему достаток умеренный:
\vs Pss 5:19
если получает чрезмерно человек, согрешает.
\vs Pss 5:20
Довольно для него меры праведной, и потому благодарение Господу за
насыщение праведное.
\vs Pss 5:21
Да возвеселятся в благах боящиеся Господа, и благость Твоя на
Израиле в Царствии Твоем.
\vs Pss 5:22
Благословенна слава Господня, ибо Он Царь наш!

\vs Pss 6:1
Блажен муж, в ком сердце его готово призвать имя Господне:
\vs Pss 6:2
помня имя Господне, спасется он.
\vs Pss 6:3
Пути его направляются Господом, и хранимы дела рук его.
\vs Pss 6:4
Лукавыми ночными видениями его не возмутится душа его,
\vs Pss 6:5
на переправе рек, среди волнения вод морских не устрашится.
\vs Pss 6:6
Восстал он от сна своего и благословил имя Господне.
\vs Pss 6:7
В постоянстве сердца своего воспел имя Бога своего, молился, да
не отвратится лице Господне от всего дома его.
\vs Pss 6:8
И Господь услышал мольбу каждого, пребывающего в страхе Божием,
и всякую просьбу души, уповающей на Него, исполнит Господь.
\vs Pss 6:9
Благословен Господь, дающий милость воистину любящим Его!

\vs Pss 7:1
Не отступись от нас, Боже, дабы не потеснили нас ненавидящие
нас без причины.
\vs Pss 7:2
Ибо Ты отразил их, Боже, и да не потопчет нога их наследие
Святыни Твоей.
\vs Pss 7:3
Вразуми нас по воле Твоей и не предай язычникам.
\vs Pss 7:4
Если же смерть пошлешь, прикажи ей о нас, ибо Ты милостив и не
прогневаешься на дела наши.
\vs Pss 7:5
Имя Твое живет среди нас, и Ты пощадишь нас.
\vs Pss 7:6
И не осилит нас язычник, ибо Ты защитник наш.
\vs Pss 7:7
И мы призовем Тебя, и Ты услышишь нас.
\vs Pss 7:8
Ибо Ты сжалишься над родом Израиля вовеки и не отвергнешь.
И мы под игом Твоим вовеки и под бичом вразумления Твоего.
\vs Pss 7:9
Ты направишь нас на путь в час заступничества Твоего, помилуешь
дом Иакова в день, который Ты обещал им.

\vs Pss 8:1
Скорбь и голос войны услышало ухо мое, голос трубы, возвещающей
убийство и гибель,
\vs Pss 8:2
голос народа великого, словно ветра великого, словно вихрь огня
великого, несущегося по пустыне.
\vs Pss 8:3
И сказал я в сердце моем: где положит остановить его Бог?
\vs Pss 8:4
Голос услышал я: в Иерусалиме, граде Святыни.
\vs Pss 8:5
Сокрушились чресла мои от этих слов, ослабели колени мои.
\vs Pss 8:6
Убоялось сердце мое, сотряслись кости мои, как лен.
\vs Pss 8:7
Сказал я: направляют они пути свои в праведности. Помыслил тут
о судах Божиих от сотворения неба и земли, оправдал Бога в судах Его от
века.
\vs Pss 8:8
Открыл Бог прегрешения их при свете дня, узнала вся земля суды
Божий праведные.
\vs Pss 8:9
В местах потаенных беззакония их,
\vs Pss 8:10
гневя Господа, сын с матерью, отец с дочерью смешивались,
\vs Pss 8:11
прелюбодействовали каждый с женой ближнего своего, сговаривались
друг с другом о подобном, скрепляя сговор клятвами.
\vs Pss 8:12
Святыни Божий расхищали~--- и не было наследника-избавителя,
\vs Pss 8:13
топтали жертвенник Господень~--- только что от всякой
скверны, во время кровей поганили жертвы, словно мясо нечистое.
\vs Pss 8:14
Не оставили греха, какого не сотворили бы хуже язычников.
\vs Pss 8:15
Потому приготовил для них Бог дух заблуждения~--- напоил чашей
вина неразбавленного допьяна.
\vs Pss 8:16
Привел мужа от края земли~--- мужа, бьющего крепко,
\vs Pss 8:17
положил войну на Иерусалим и землю его.
\vs Pss 8:18
Вышли к мужу тому старейшины земли с радостью, сказали ему: Желанен
приход твой! сюда, войдите с миром!
\vs Pss 8:19
Сгладили пути неровные при входе их, открыли врата в Иерусалим,
увенчали стены его.
\vs Pss 8:20
Вошел он как отец в дом сыновей своих, с миром, утвердил стопы свои
прочно,
\vs Pss 8:21
занял башни города и стену Иерусалимскую:
\vs Pss 8:22
ибо Бог привел его, неколебимого, в заблуждении их.
\vs Pss 8:23
Погубил он старейшин их и всякого мудрого в совете, проливал кровь
жителей Иерусалимских, как воду нечистую,
\vs Pss 8:24
увел сынов и дочерей их, которых родили в скверне.
\vs Pss 8:25
И сделали они в меру нечистоты своей по примеру отцов своих,
\vs Pss 8:26
осквернили Иерусалим и посвященное имени Божию.
\vs Pss 8:27
Оправдан Бог в судах Его над народами земли,
\vs Pss 8:28
и чтящие Бога словно агнцы безгрешные среди них.
\vs Pss 8:29
Хвалим Господь, судящий всякую землю в праведности Его!
\vs Pss 8:30
Ведь вот, Боже, явил Ты нам суд Свой в праведности Твоей,
\vs Pss 8:31
увидели глаза их суды Твои, Боже, оправдали мы имя чтимое Твое
вовек.
\vs Pss 8:32
Ибо ты~--- Бог праведности, судящий Израиля во вразумлении.
\vs Pss 8:33
Обрати, Боже, милость Свою на нас и пожалей нас,
\vs Pss 8:34
собери рассеяние Израилево, будь благ и милостив,
\vs Pss 8:35
ибо вера Твоя~--- с нами, и мы ожесточим шею нашу, и Ты наставник
наш.
\vs Pss 8:36
Не оставь нас, Боже наш, чтобы не поглотили нас язычники в
отсутствие избавителя.
\vs Pss 8:37
И Ты~--- Бог наш от начала, и на Тебя понадеялись мы,
Господи.
\vs Pss 8:38
И мы не отступимся от Тебя, ибо благи суды Твои над нами.
\vs Pss 8:39
На нас и детях наших благоволение Твое вовек, Господи, Спаситель
наш, и не будем мы поколеблены впредь на вечное время.
\vs Pss 8:40
Хвалим Господь в судах Его устами праведников,
\vs Pss 8:41
ты же благословен, Израиль, Господом вовек.

\vs Pss 9:1
Уведен Израиль от дома своего в чужую землю, отступили от
Господа, избавившего их~---
\vs Pss 9:2
отказано им в наследстве, что даровал им Господь
пред всеми народами, рассеян Израиль по слову Божию.
\vs Pss 9:3
И да оправдаешься Ты, Боже, в праведности Твоей беззакониями
их,
\vs Pss 9:4
ибо Ты Судия праведный над всеми народами земли.
\vs Pss 9:5
И не скроется от ведения Твоего ни один, творящий неправое,
\vs Pss 9:6
так же, как праведность чтящих Тебя~--- пред лицем Твоим,
Господи, ибо где скроется человек от ведения Твоего?
\vs Pss 9:7
Боже, дела наши~--- выбор и власть души нашей, совершить
правое или неправое делами рук наших.
\vs Pss 9:8
Ты же в правде Своей будь милостив к сынам человеческим!
\vs Pss 9:9
Делающий правое готовит себе жизнь подле Господа, делающий же
неправое сам повинен в гибели души своей.
\vs Pss 9:10
Ибо праведен суд Господень над мужем и домом.
\vs Pss 9:11
Кому явишь доброту, Боже, если не призывающим Господа?
\vs Pss 9:12
Очистит Господь грешную душу, дав исповедаться, дав излить ей грехи
ее,
\vs Pss 9:13
ибо стыдно нам и лицам нашим всего содеянного.
\vs Pss 9:14
Да и кому отпустит Он грехи, как не тем, что согрешили?
\vs Pss 9:15
Праведных, Боже, благословишь, не накажешь, если где оступились, и
благость Твоя~--- на грешащих и кающихся.
\vs Pss 9:16
И ныне Ты Бог, и мы~--- народ, который Ты возлюбил. Взгляни и
сжалься, Бог Израилев: ибо мы~--- Твои, и не отврати милости Твоей от нас,
дабы не подступились к нам.
\vs Pss 9:17
Ибо избрал Ты семя Авраамово среди всех народов
\vs Pss 9:18
и положил имя Свое на нас, Господи, и не оставишь вовек.
\vs Pss 9:19
В Завете свидетельствовал Ты отцам нашим о нас, и мы станем уповать
на Тебя, обратив к Тебе душу нашу.
\vs Pss 9:20
Милость Господня на доме Израилевом во веки веков и впредь!

\vs Pss 10:1
Блажен муж, кого воспомнил Господь в обличении, кто отвращен
был бичом от пути злого и очистился от грехов, дабы не умножать числа их.
\vs Pss 10:2
Готовящий спину свою для бича очистится, ибо Господь благ к
терпящим наказание Его.
\vs Pss 10:3
Ибо Он прямыми сделает пути праведных и не искусит вразумлением
Своим.
\vs Pss 10:4
И милость Господня на воистину любящих Его, и воспомнит Господь
о рабах Своих в милости.
\vs Pss 10:5
Свидетельство~--- в законе Завета вечного, свидетельство
Господа~--- в посещении путей человеческих.
\vs Pss 10:6
Праведен и благ Господь наш в судах Его вовеки, и прославит
Израиль имя Господне в радости.
\vs Pss 10:7
И возблагодарят Его благочестивые в собрании народа, и над
нищими смилуется Господь в радости Израиля.
\vs Pss 10:8
Ибо Он милостив и благ вовеки, и собрания Израиля прославят имя
Господне.
\vs Pss 10:9
От Господа спасение на дом Израилев в радость вечную!

\vs Pss 11:1
Вострубите на Сионе трубою знамения священные!
\vs Pss 11:2
Возгласите в Иерусалиме глас благовествующего! Ибо
смилостивился Бог над Израилем и посетил их.
\vs Pss 11:3
Встань, Иерусалим, на возвышенности и узри детей твоих, от
востока и запада собранных воедино Господом.
\vs Pss 11:4
От севера идут, радуясь Богу своему, с островов отдаленных
собрал их Бог.
\vs Pss 11:5
Горы высокие принизились до равнины для них,
\vs Pss 11:6
холмы бежали прочь при входе их, дубравы дали тень им на пути
их:
\vs Pss 11:7
всякое древо благоуханное взрастил для них Бог, дабы прошел
Израиль в присутствии славы Бога их.
\vs Pss 11:8
Облекись, Иерусалим, в одеяние славы твоей, приготовь одежды
святости твоей, ибо возвестил Бог благо Израиля во веки веков и впредь.
\vs Pss 11:9
Да исполнит Господь, что возвестил над Израилем и Иерусалимом,
да поднимет Господь Израиля именем славы Своей! Милость Господня да пребудет
над Израилем во веки веков и впредь!

\vs Pss 12:1
Господи, избавь душу мою от мужа преступного и порочного, от
языка преступного и злоречивого, говорящего ложь и обман!
\vs Pss 12:2
Чиня разврат, речи с языка мужа порочного подобны огню,
поджигающему на току солому.
\vs Pss 12:3
Так пусть же приход его наполнит жилища словом лживым, пусть
вырубит деревья воспаляющей преступной радости,
\vs Pss 12:4
пусть посеет вражду среди преступных жилищ злоречивыми устами!
Да отдалит Бог от непорочных уста совершающих беззаконие в неведении, и да
будут разметаны кости злоречивцев знающими страх Божий!
\vs Pss 12:5
В пламени огненном язык лукавый да будет умерщвлен руками
праведных!
\vs Pss 12:6
Да охранит Господь покой души, ненавидящей делающих беззаконие,
и да направит Господь мужа, несущего мир в дом.
\vs Pss 12:7
Да пребудет спасение Господне на Израиле, чаде Его, вовек,
\vs Pss 12:8
и да сгинут грешники от лица Господа безвозвратно;
праведники же Господни да унаследуют обетование Господне!

\vs Pss 13:1
Десница Господня оборонила меня, десница Господня оберегла
нас.
\vs Pss 13:2
Рука Господня избавила нас от меча нависшего, от голода и
смерти грешников.
\vs Pss 13:3
Звери злобные набежали на них, зубами своими разорвали плоть
их
и челюстями своими раздробили кости их, и от всего этого избавил нас
Господь.
\vs Pss 13:4
Смутился благочестивый от согрешений своих, и да не будет он
взят вместе с грешниками.
\vs Pss 13:5
Ибо ужасно низвержение грешника, но ничто из этого не коснется
праведного:
\vs Pss 13:6
неравно вразумление праведных, согрешающих в неведении, и
низвержение грешников.
\vs Pss 13:7
В одеждах наказывается праведник, да не восторжествует грешник
над праведным.
\vs Pss 13:8
Ибо наставляет Он праведного как сына любимого и вразумляет его
как первенца.
\vs Pss 13:9
И милостив Господь к почитающим Его, и согрешения их загладит
вразумлением, ибо жизнь праведных навеки;
\vs Pss 13:10
но грешники будут изъяты и погублены, и не сыщется более памяти о
них.
\vs Pss 13:11
На благочестивых же милость Господня, и на боящихся Его милость
Его.

\vs Pss 14:1
Верен Господь воистину любящим Его, терпящим наказание Его,
совершающим путь в праведности предписаний Его, по закону, какой дал Он нам для
жизни нашей.
\vs Pss 14:2
Праведники Господни будут жить по нему вовек, рай Господень,
дерева жизни~--- праведники Господни.
\vs Pss 14:3
Росток их пустил корни вовек,
и не будут вырваны из земли во все дни, ибо жребий и наследие
Божие~--- Израиль.
\vs Pss 14:4
И не так грешники и беззаконники,
что возлюбили день в участи греха своего, в горечи гнили, в похоти своей,
\vs Pss 14:5
и не вспомнили о Боге,
ибо пути людские открыты пред лицем Его вполне, и хранилища сердец ведомы
ранее, нежели исполнится.
\vs Pss 14:6
За то удел их~--- ад, мрак и гибель, и не найти их будет в
день, когда милость коснется
праведных: чтящие же Господа унаследуют жизнь в радости.

\vs Pss 15:1
В беде моей призвал я имя Господне,
на помощь понадеялся Бога Иаковлева и был спасен,
\vs Pss 15:2
ибо надежда и прибежище несчастных Ты, Боже.
\vs Pss 15:3
Да и что имеет силу, Боже, как не воздать хвалу Тебе по
истине?
\vs Pss 15:4
И что может человек, как не воздать хвалу имени
Твоему~---
\vs Pss 15:5
псалмом и хвалебной песнью в радости сердца, плодом уст на
слаженном инструменте языка, первенцем уст от сердца благочестивого и
праведного?
\vs Pss 15:6
Делающий так не поколеблется вовек силами зла, пламя огненное и
гнев нечестивцев не коснутся его,
\vs Pss 15:7
когда выйдет он к грешникам от лица Божьего, чтобы сокрушить
всю твердость грешников,
\vs Pss 15:8
ибо знак Божий на праведных во спасение: голод, меч и смерть
далеко от праведных,
\vs Pss 15:9
побегут, как от войны, от благочестивых, кинутся в погоню за
грешниками и настигнут их. И не избегнут делающие беззаконие суда Божьего, но
словно врагами хитроумными будут захвачены,
\vs Pss 15:10
ибо знак гибели на челе их,
\vs Pss 15:11
и наследство грешников~--- гибель и мрак, и беззакония их
устремятся за ними до глубин ада.
\vs Pss 15:12
Наследство их не отойдет к детям их,
\vs Pss 15:13
но беззакония обратят в пустыню дома грешников, и погибнут грешники
в день суда Господня вовек,
\vs Pss 15:14
когда посетит Бог землю в день суда Своего, дабы воздать грешникам
во веки веков.
\vs Pss 15:15
Боящиеся же Господа узнают милость Его в тот день и продолжат жизнь
свою милостынью Бога их.

\vs Pss 16:1
Погрузилась в дрему душа моя, забыв о Господе, и едва не впал я
в забытье сонное,
\vs Pss 16:2
подобно тем, кто удалился от Бога.
Немного~--- и была бы предана смерти душа моя вблизи врат преисподней вместе
с грешником,
\vs Pss 16:3
унесена была бы душа моя от Господа Бога Израиля, если бы не
помог мне Господь милостью Своею вечной.
\vs Pss 16:4
Уколол Он меня, как шипом коня колют, чтобы служил я Ему,
Спаситель и Заступник мой во всякий час~--- Он спас меня.
\vs Pss 16:5
Славлю Тебя, Боже, ибо Ты помог мне во спасение мое и не
сопричислил меня к грешникам в погибели.
\vs Pss 16:6
Не удали от меня милость Твою, Боже, ниже память о Тебе от
сердца моего, доколе я не умру.
\vs Pss 16:7
Отврати меня, Боже, от согрешения злого и ото всякой злой жены,
соблазняющей неразумного.
\vs Pss 16:8
И да не совратит меня красота жены беззаконной и никто
одержимый согрешением негодным.
\vs Pss 16:9
Дела рук моих направь в страхе Твоем, и пути мои сохрани в
памяти о Тебе.
\vs Pss 16:10
Язык мой и уста мои одень словами правды, гнев и ярость безрассудную
далекими сделай от меня,
\vs Pss 16:11
ропот и малодушие в скорби удали от меня; если согрешу, вразуми
меня, дабы я обратился к Тебе.
\vs Pss 16:12
Благоволением и радостью утверди душу мою; укрепишь душу мою~--- и
довольно будет мне даяния Твоего,
\vs Pss 16:13
ибо, если не укрепишь Ты, кто вынесет наказание бедностью,
\vs Pss 16:14
когда изобличает душу человека рука мерзости его? Испытание
Твое~--- в плоти человеческой и в скорби бедности.
\vs Pss 16:15
Хранящий праведность во всех бедах помилован будет Господом.

\vs Pss 17:1
Господи, Ты Сам Царь наш во веки вечные, ибо Тобою, Боже,
похвалится душа наша.
\vs Pss 17:2
Каково время жизни человека на земле? По времени его~--- и
надежда на него.
\vs Pss 17:3
Но мы понадеемся на Бога, Спасителя нашего, ибо сила Бога
нашего вовеки с милостью,
\vs Pss 17:4
и Царствие Бога нашего вовеки в суде над народами.
\vs Pss 17:5
Ты, Господи, избрал Давида царем над Израилем, и Ты клялся ему
о семени его навеки, да не угаснет пред Тобою царство его.
\vs Pss 17:6
И в согрешениях наших восстали на нас грешники, и напали на нас
и притеснили нас.
Чего не обещал Ты им, взяли они силою,
\vs Pss 17:7
и не прославили честного имени Твоего. В славе воздвигли они
царство в знак величия своего.
\vs Pss 17:8
Пустым сделали они трон Давидов в многошумном высокомерии.
Но Ты, Боже, низвергнешь их и возьмешь семя их от земли,
\vs Pss 17:9
восставив на них человека, чуждого роду нашему,
\vs Pss 17:10
по грехам их воздашь Ты им, Боже, да случится с ними по делам
их.
\vs Pss 17:11
По делам их не помилует их Бог, испытал Он семя их и не отпустил
им.
\vs Pss 17:12
Верен Господь во всех судах Его, кои вершит Он на земле.
\vs Pss 17:13
Лишил беззаконный землю нашу живущих на ней, похитил юного и старого
и детей их вместе с ними.
\vs Pss 17:14
Во гневе своем отослал он их на Запад и насмеялся над правителями
земли и не пощадил их.
\vs Pss 17:15
Чуждый нам, возгордился враг, и сердце его чуждо Бога нашего.
\vs Pss 17:16
И все, что соделал в Иерусалиме,- как язычники в городах своих богам
своим.
\vs Pss 17:17
И превосходили их сыны Завета среди народов смешанных, и не было
среди них никого, кто творил бы в Иерусалиме милость и правду.
\vs Pss 17:18
Бежали от них любящие собрания благочестивых, словно воробьи
разлетелись от гнезда своего.
\vs Pss 17:19
Скитались в пустынях, дабы спасти души свои от зла, и драгоценна в
очах переселенца душа, спасшаяся из них.
\vs Pss 17:20
По всей земле соделалось рассеяние их беззаконными, ибо перестало
небо изливать дождь на землю,
\vs Pss 17:21
сомкнулись источники вечные в безднах средь гор высоких, ибо не было
среди людей тех никого, кто творил бы справедливость и суд. От правителя их до
малейшего из народа~--- во грехе всяческом.
\vs Pss 17:22
Царь в беззаконии, а судья в неправде, а народ во грехе.
\vs Pss 17:23
Призри на них, Господи, и восставь им царя их, сына Давидова, в тот
час, который Ты знаешь, Боже, да царит он над Израилем, отроком Твоим.
\vs Pss 17:24
И препояшь его силою поражать правителей неправедных.
\vs Pss 17:25
Да очистит он Иерусалим от язычников, топчущих город на погибель. В
премудрости и праведливости
\vs Pss 17:26
да изгонит он грешников от наследия Твоего, да искоренит гордыню
грешников, подобно сосудам глиняным сокрушит жезлом железным всякое упорство
их.
\vs Pss 17:27
Да погубит он язычников беззаконных словами уст своих, угрозою его
побегут язычники от лица его, и обличит он грешников словом сердца их.
\vs Pss 17:28
И соберет он народ святой, и возглавит его в справедливости, и будет
судить колена народа, освященного Господом Богом его.
\vs Pss 17:29
И не позволит он поселиться среди них неправедности, и не будет с
ними никакой человек, ведающий зло.
\vs Pss 17:30
Ибо он будет знать, что все они~--- сыны Бога их, и разделит он
их по коленам их на земле.
\vs Pss 17:31
Ни переселенец, ни чужеродный не поселятся с ними более. Будет
судить он народы и племена в премудрости и справедливости его.
\vs Pss 17:32
И возьмет он народы язычников служить ему под игом его, и прославит
он Господа в очах всей земли,
\vs Pss 17:33
и очистит он Иерусалим, освятив его, как был он в начале.
\vs Pss 17:34
Придут племена от края земли, дабы видеть славу его, неся в дар
истомленных сынов Иерусалима,
\vs Pss 17:35
и дабы видеть славу Господа, коею прославил Он эту землю; и сам
справедливый царь научен будет Богом о них.
\vs Pss 17:36
И нет неправедности во дни его среди них, ибо все святы, а царь
их~--- помазанник Господень.
\vs Pss 17:37
Не понадеется он на коня, всадника и лук, не станет собирать себе в
изобилии золота и серебра для войны, не станет оружием стяжать надежд на день
войны.
\vs Pss 17:38
Сам Господь~--- Царь его, надежда сильного надеждою на Бога, и
поставит он все племена пред собою в страхе.
\vs Pss 17:39
Ударит он по земле словом уст своих навеки,
\vs Pss 17:40
благословит он народ Господа в премудрости с радостию.
\vs Pss 17:41
И сам он чист от согрешения, дабы править народом великим;
обличит он правителей и уничтожит грешников влаетию слова своего.
\vs Pss 17:42
И не ослабеет во дни те, уповая на Бога своего, ибо сделал его Бог
сильным духом святым и премудрым в рассуждении, с мощью и справедливостью.
\vs Pss 17:43
И благословение Господа с ним в силе его, и не ослабеет он.
Упование его на Господа,
\vs Pss 17:44
и кто может против него?
Мощный в делах своих и сильный в страхе Божием,
\vs Pss 17:45
пасущий стадо Господне в вере и справедливости, не даст он ослабеть
никому на пастбище их.
\vs Pss 17:46
В благочестии поведет он их всех, и не будет среди них гордыни для
угнетения среди них.
\vs Pss 17:47
Такова краса царя Израиля, которую познал Бог, восставив его над
домом Израиля для вразумления его.
\vs Pss 17:48
Речи его, пламенем очищенные, драгоценнее золота, в собраниях будет
судить он людей~--- колена освященных.
\vs Pss 17:49
Слова его~--- как слова святых среди людей освященных.
\vs Pss 17:50
Блаженны, кто будет жить в те дни, ибо узрят они сотворенное Богом
счастие Израиля в собрании колен его.
\vs Pss 17:51
Да ускорит Бог милость Свою над Израилем, да избавит нас от
нечистоты врагов нечестивых. Сам Господь~--- Царь наш во веки вечные.
с того дня, как утвердил их Бог, и от века.
И не отклонились с того дня, как утвердил Он их; с давних времен не отступили
они от пути своего, если Сам Бог не приказал им через слуг Своих.

\vs Pss 18:1
Господи! милость Твоя~--- на творениях рук Твоих вовек,
\vs Pss 18:2
благодать Твоя, дар изобилия - на Израиле.
Глаза Твои взирают на них, и не узнает нужды ни одно из них,
\vs Pss 18:3
уши Твои услышат мольбу несчастного, молящегося с упованием.
Суды Твои надо всею землей исполнены милосердия,
\vs Pss 18:4
любовь же Твоя~--- на семени Авраамовом, сынах
Израилевых.
Наказание Твое на нас~--- как на сыне первородном и единственном,
\vs Pss 18:5
дабы отвратить душу послушную от греха по неведению.
\vs Pss 18:6
Да очистит Бог Израиля в день милости благословением, в день
избрания~--- возвращением помазанника Его.
\vs Pss 18:7
Блаженны, кому случится во дни те увидеть блага Господни, какие
явит Он роду, грядущему
\vs Pss 18:8
под жезл вразумления помазанника Господня~--- в страхе пред
Богом своим, в мудрости духа, праведности и силы,
\vs Pss 18:9
дабы направил Он человека в делах праведности страхом Божиим,
восстановил их всех в страхе Господнем~---
\vs Pss 18:10
благой род, боящийся Бога, во дни милости.
\vs Pss 18:11
Велик Бог наш и славен, живущий в вышних,
\vs Pss 18:12
определивший ход светил по времени от дней ко дням, и не сошли с
пути, какой Он заповедал им.
\vs Pss 18:13
В страхе Божием путь их каждый день~--- с того дня, как утвердил
их Бог, и от века.
\vs Pss 18:14
И не отклонились с того дня, как утвердил Он их; с давних времен не
отступили они от пути своего, если Сам Бог не приказал им через слуг Своих.

\bibbookdescr{Ode}{
  inline={Оды Соломона},
  toc={Оды Соломона},
  bookmark={Оды Соломона},
  header={Оды Соломона},
  abbr={Оды}
}
\vs Ode 1:1
Яхве на
главе моей подобен венцу, и да не пребуду никогда без Него.
\vs Ode 1:2
Сплетенный
для меня~--- истинный венец, и он побудил ветви Твои прорасти во мне.
\vs Ode 1:3
Ибо не похож
он на увядший венец, который не цветет;
\vs Ode 1:4
ибо Ты
обитаешь над моею главой и расцвел на мне.
\vs Ode 1:5
Полны и
спелы плоды Твои; они исполнены спасения Твоего \ldots

\vs Ode 2:1
\bibemph{не сохранилась.}

\vs Ode 3:1
\ldots\ полагаюсь я на любовь Яхве.
\vs Ode 3:2
И чресла Его
пребывают с Ним, и зависим я от них; и Он любит меня.
\vs Ode 3:3
Ибо мне бы
не стоило узнавать о том, как любит Яхве, если бы постоянно не любил Он меня.
\vs Ode 3:4
Кто же
способен различать любовь, как не тот, кто любим?
\vs Ode 3:5
Люблю я
Возлюбленного и сам я люблю Его, ибо, где бы ни был покой Его, также и я там.
\vs Ode 3:6
И не
сделаюсь чужим я, ибо нет ревности между Яхве Всевышним и Милостивым.
\vs Ode 3:7
Я соединился
с Ним, ибо любящий отыскал Возлюбленного; поскольку же люблю я того, кто есть
Сын, я сделаюсь Сыном.
\vs Ode 3:8
Истинно,
тот, кто соединится с бессмертным, воистину станет бессмертен.
\vs Ode 3:9
И тот, кто
восторгается Жизни, оживет.
\vs Ode 3:10
Вот Дух Яхве, не обманчивый, учащий сынов человеческих познавать пути Его.
\vs Ode 3:11
Будь же
мудрым, и понимающим, и пробужденным.
Аллилуйя.

\vs Ode 4:1
Ни один
человек не может осквернить святое место Твоё, о, Боже мой, как не сможет он и
изменить его и поместить его в иное место,
\vs Ode 4:2
ибо нет у
него власти над ним; ибо Святилище Свое создал Ты прежде, чем создал Ты особые
места.
\vs Ode 4:3
Древний же
не извратится тем, кто ниже него. Дал ты сердце Своё, о, Яхве, верующим в
Тебя.
\vs Ode 4:4
Не будешь ты
ни праздным, ни бесплодным;
\vs Ode 4:5
ибо один
день веры Твоей дивнее всех дней и лет.
\vs Ode 4:6
Ибо кто
положится на милость Твою и отвергнут будет?
\vs Ode 4:7
Ибо известна
печать Твоя; и творения Твои известны ей,
\vs Ode 4:8
и воинство
Твоё одержимо ею, и архангелы избранные облачены ею.
\vs Ode 4:9
Ты воздал
нам сопричастностью Твоею, не оттого, что Ты нуждался в нас, но чтобы мы
всегда нуждались в Тебе.
\vs Ode 4:10
Излей же на
нас живительный дождь Свой, и раскрой обильные источники Свои, щедро дающие
нам молоко и мёд.
\vs Ode 4:11
Ибо не
пребывает с Тобою печаль; чтобы не сожалел Ты ни о чем обещанном Тобою,
\vs Ode 4:12
ибо
результат был явлен Тебе.
\vs Ode 4:13
Ибо
отданное Тобой Ты отдал свободно, так что не передумаешь и снова не заберешь,
\vs Ode 4:14
ибо всё
было явлено Тебе как Богу и восставлено пред Тобою от начала.
\vs Ode 4:15
И Ты, о,
Яхве, создал всё.
Аллилуйя.

\vs Ode 5:1
Я молю Тебя,
о Яхве, ибо я люблю Тебя.
\vs Ode 5:2
О,
Всевышний, не отрекись от меня, ибо Ты~--- надежда моя.
\vs Ode 5:3
Да получу я
свободно милость Твою, и да буду жить ею.
4.
Преследователи мои придут, но да не увидят они меня.
\vs Ode 5:5
Да ниспадет
на очи их облако тьмы; да окутает их воздух тьмы густой.
\vs Ode 5:6
И да не
будет у них света, чтобы видеть, дабы им было не схватить меня.
\vs Ode 5:7
Да замрут
все их поползновения, дабы, где бы ни спрятались они, пасть на их собственные
головы.
\vs Ode 5:8
Ибо
замыслили они нечто, что не для них.
\vs Ode 5:9
Приуготовили
себя они злонамеренно, но найдут их бессильными.
\vs Ode 5:10
Истинно
надеюсь на Яхве, да не убоюсь.
\vs Ode 5:11
А раз Яхве спасение моё, да не убоюсь.
\vs Ode 5:12
И подобен
Ты сотканному венцу на главе моей, и да не дрогну я.
\vs Ode 5:13
Даже если
всё дрогнет, я устою неколебимо.
\vs Ode 5:14
И даже если
погибнет всё видимое, я не умру;
\vs Ode 5:15
Ибо со мною
Яхве, а я~--- с Ним.
Аллилуйя.

\vs Ode 6:1
Как ветер
проскальзывает сквозь арфу и струны говорят,
\vs Ode 6:2
так и Дух Яхве
говорит через чресла мои, а я глаголю через любовь Его,
\vs Ode 6:3
ибо сокрушает
Он всё, что чуждо, и всё сущее~--- от Яхве,
\vs Ode 6:4
ибо так было
от начала и пребудет до самого конца,
\vs Ode 6:5
чтобы не было
ничего вопреки и ничто не восстало бы на Него.
\vs Ode 6:6
Умножил Яхве
знание Своё, и был Он усерден, дабы должное стать известным по милости Его,
воздалось бы нам.
\vs Ode 6:7
И похвалу свою
воздал Он нам именем Своим, духи же наши восхваляли Его Святой Дух.
\vs Ode 6:8
И изошел
Поток, и стал рекой~--- великой и широкой; истинно, смыла она всё, и разрушила
всё, и вынесла к Храму.
\vs Ode 6:9
И преграды,
возведенные людьми, не смогли сдержать её, как не смогли даже умения тех, кто
обыкновенно сдерживает воду.
\vs Ode 6:10
Ибо разлилась
она по поверхности всей земли и заполонила всё.
\vs Ode 6:11
Когда пьют
все жаждущие на земле, и жажда облегчается и утоляется;
\vs Ode 6:12
ибо питие
дано от Всевышнего.
\vs Ode 6:13
Потому
блаженны служащие этого пития, которым вверили воду Его.
\vs Ode 6:14
Освежили они
уста пересохшие и воспрянули увядшей было волей.
\vs Ode 6:15
Даже живые,
близкие к угасанию, восстали из смерти.
\vs Ode 6:16
И омертвевшие
члены воспрянули и восстановились.
\vs Ode 6:17
Они дали силу
идти и свет глазам их.
\vs Ode 6:18
Ибо всякий
узнал их как (принадлежащих) Яхве и ожил живой водою вечности.
Аллилуйя.

\vs Ode 7:1
Как гневаются
над нечестивостью, так же и радуются Возлюбленному, и вкушают свободно от плодов
этих.
\vs Ode 7:2
Радость же моя
Яхве, и путь мой~--- к Нему, и этот путь мой превосходен.
\vs Ode 7:3
Ибо есть у
меня Помощник~--- Яхве. Он щедро явил Себя мне в простоте Своей, ибо благость Его
умалила суровость Его.
\vs Ode 7:4
Сделался Он
подобным мне, чтобы я заполучил Его. Он решил пребывать в облике, подобном
моему, чтобы я положился на Него.
\vs Ode 7:5
И не трепетал
я, когда увидел Его, ибо Он был милостив ко мне.
\vs Ode 7:6
Подобным
природе моей сделался Он, чтобы мог я понять Его. И подобным облику моему, дабы
не отвратился я от Него.
\vs Ode 7:7
Отец же знания
Слово знания.
\vs Ode 7:8
Он,
сотворивший мудрость, мудрее трудов Своих.
\vs Ode 7:9
И Он,
сотворивший меня, когда я еще не знал, что мне следовало делать, когда я начал
быть.
\vs Ode 7:10
Оттого был Он
милостив ко мне Своей щедрой милостью и позволил мне просить у Него и извлечь
пользу из жертвы Его.
\vs Ode 7:11
Ибо именно Он
нетленный, совершенство миров и их Отца.
\vs Ode 7:12
Он позволил
им явиться тем, которые Его; для того, чтобы они опознали Его, сотворившего их и
не думали, что они родились сами собой.
\vs Ode 7:13
Ибо к знанию
направил Он путь Свой, Он расширил его, и удлинил его, и привел его к полному
совершенству.
\vs Ode 7:14
И наставил
над ним следы Света Своего, и продолжалось это от начала до конца.
\vs Ode 7:15
Ибо Сам по
Себе служил Он, и Сыном наслаждался Он.
\vs Ode 7:16
И в силу
спасения Своего всем овладеет Он. И узнают Всевышнего святые Его:
\vs Ode 7:17
Возгласит
тем, кто поет песни явления Яхве, чтобы вышли они встречать Его и пели Ему,
радостно и с арфой, берущей многие ноты.
\vs Ode 7:18
Грядут
пророки прежде Него, и узрят их пред Ним.
\vs Ode 7:19
И восхвалят
они Яхве в любви Его, ибо близок Он и видит.
\vs Ode 7:20
И ненависть
исчезнет с земли, и вместе с ревностью утонет она.
\vs Ode 7:21
Ибо
невежество рушилось на ней, ибо знание Яхве пребывало на ней.
\vs Ode 7:22
Да воспоют
певцы милость Всевышнего Яхве, и да привнесут песнопения свои.
\vs Ode 7:23
И да пребудет
сердце их подобным дню, а бархатные голоса их подобными волшебной красе Яхве.
\vs Ode 7:24
Да не
пребудет там никто дышащий, в ком нет знания или гласа.
\vs Ode 7:25
Ибо дал Он
уста тварям Своим, чтобы открыли голос уст навстречу Ему и чтобы воспеть Его.
\vs Ode 7:26
Веруйте же в
силу Его и возгласите милость Его.
Аллилуйя.

\vs Ode 8:1
Откройте же,
откройте сердца ваши ликованию Яхве, и да пребудет ваша любовь обильной от
сердца к устам,
\vs Ode 8:2
чтобы принести
плоды Яхве, святую жизнь, и чтобы говорить со смирением в свете Его.
\vs Ode 8:3
Восстаньте же
и стойте неподвижно, вы, кого порой ниспосылают вниз.
\vs Ode 8:4
Вы,
пребывавшие в безмолвии, говорите же, ибо отверзлись уста ваши.
\vs Ode 8:5
Вы, бывшие
презираемыми, отныне возвыситесь, ибо возвеличена была Правда ваша.
\vs Ode 8:6
Ибо с вами
десница Яхве, и будет Он помощником вам.
\vs Ode 8:7
И уготован вам
мир прежде того, что может быть войной вашей.
\vs Ode 8:8
Слушайте же
слово истины, и получайте знание Всевышнего.
\vs Ode 8:9
Ни плоть вашу
не следует понимать так, как возвещу я вам, ни одежду вашу, что явлю я вам.
\vs Ode 8:10
Храните же
тайну мою, вы, хранимые ею, храните веру мою, вы, хранимые ею.
\vs Ode 8:11
И понимайте
знание моё, вы, понимающие меня в истине, любите меня с нежностью, вы, любящие;
\vs Ode 8:12
ибо не
отвращу лица моего от тех, кто мои, ибо я знаю их.
\vs Ode 8:13
И прежде, чем
появились они, распознал я их и наложил печать на лица их.
\vs Ode 8:14
Я создал
чресла их, и Мои собственные груди приуготовил Я для них, чтобы могли они испить
Моё святое молоко и жить им.
\vs Ode 8:15
Я любезен им,
и не пристыжён Я ими.
\vs Ode 8:16
Ибо
мастерство Моё~--- они, и сила мыслей Моих.
\vs Ode 8:17
Затем кто
восстанет против труда Моего? И кто не подвержен им?
\vs Ode 8:18
Я возжелал и
создал разум и сердце, и они~--- Мои. И одесную посадил Я избранных Моих.
\vs Ode 8:19
И правда Моя
шествует перед ними, и не лишатся они имени Моего, ибо с ними оно.
\vs Ode 8:20
Молитесь же и
возрастайте, и блюдите себя в любви Яхве.
\vs Ode 8:21
И вы, бывшие
возлюбленными в Возлюбленном, и вы, хранимые в Том, Кто жив, и вы, спасенные в
Том, Кто спасен.
\vs Ode 8:22
И найдут вас
непорочными в любые времена, во имя Отца вашего.
Аллилуйя.

\vs Ode 9:1
Навострите же
уши ваши, и скажу я вам.
\vs Ode 9:2
Отдайтесь же
мне, чтобы я также отдался вам.
\vs Ode 9:3
(Чтобы отдал я
вам) Слово Яхве и страсти Его, святую мысль, которую думал Он о Помазаннике
Своем.
\vs Ode 9:4
Ибо в воле
Яхве~--- жизнь ваша, и цель Его~--- вечная жизнь, и совершенство ваше~--- непорочно.
\vs Ode 9:5
Обогатитесь же
в Боге-Отце, и стяжайте цель Всевышнего. Станьте же сильными и искупленными
милостью Его.
\vs Ode 9:6
Ибо возвещаю я
мир вам, святым Его, дабы никто из слышащих не впал в войну.
\vs Ode 9:7
А также дабы
познавшие Его не погибли, и дабы стяжавшие Его не устыдились.
\vs Ode 9:8
Вечный же
венец суть Истина; блаженны носящие ее на главах своих.
\vs Ode 9:9
Это~--- камень
драгоценный, ибо из-за венца этого велись войны.
\vs Ode 9:10
Но взяла его
Правда и отдала вам.
\vs Ode 9:11
Возложите же
венец этот в истинном согласии с Яхве, и всех побежденных впишут в книгу Его.
\vs Ode 9:12
Ибо книга их
награда победы вашей, и видит она вас пред собою и желает, чтобы вы спасены
были.
Аллилуйя.

\vs Ode 10:1
Яхве наставил
уста мои Словом Своим и открыл сердце моё Светом Своим.
\vs Ode 10:2
И велел Он мне
остаться в бессмертной жизни Его и позволил мне возвестить о плоде покоя Его,
\vs Ode 10:3
преобразить
жизни жаждущих прийти к Нему и вести пленных к свободе.
\vs Ode 10:4
Я же осмелел и
стал сильным, и захватил мир этот, и стала неволя Моей во славу Всевышнего и
Бога, Отца моего.
\vs Ode 10:5
И кротких,
бывших рассеянными, собрали вместе, но не осквернился я любовью своей к ним, ибо
они благодарили меня в вышних местах.
\vs Ode 10:6
И следы света
легли на сердце их, и шли они как по жизни моей, и спасены были, и сделались они
народом моим вовеки веков.
Аллилуйя.

\vs Ode 11:1
Моё сердце
было обрезано и появился цветок у него, затем же милость проросла в нем, и дало
плоды сердце моё ради Яхве.
\vs Ode 11:2
Ибо Всевышний
обрезал его своим Духом Святым, и он открыл мою внутреннюю жизнь навстречу Ему и
наполнил меня любовью Своей.
\vs Ode 11:3
И обрезание
Его сделалось спасением моим, и взошел я на путь, на покой Его, на путь истины.
\vs Ode 11:4
От начала до
конца стяжал я знание Его.
\vs Ode 11:5
И встал я на
скале истины, где Он оставил меня.
\vs Ode 11:6
И говорящие
воды щедро коснулись уст моих из фонтана Яхве.
\vs Ode 11:7
И так пил я и
пьянел от живой воды бессмертной.
\vs Ode 11:8
И опьянение
моё не привело к невежеству, но отрекся я от спеси,
\vs Ode 11:9
и обратился ко
Всевышнему, Богу моему, и обогатился пользой Его.
\vs Ode 11:10
И отверг я
глупость, павшую на землю, и разоблачил её и отбросил прочь от себя.
\vs Ode 11:11
И Яхве
обновил меня одеянием Своим и овладел мною светом Своим.
\vs Ode 11:12
И воздал Он
мне свыше бессмертным покоем, и сделался я подобным земле цветущей и радостной
плодами своими.
\vs Ode 11:13
И Яхве
подобен солнцу над лицом земли.
\vs Ode 11:14
Мне
просветлили очи, и лицо моё окропилось росой;
\vs Ode 11:15
и освежилось
дыхание моё благоуханным ароматом Яхве.
\vs Ode 11:16
И взял Он
меня в Рай Свой, где богатство удовольствия Яхве. Я узрел цветущие и
плодоносящие деревья, и самовозросшей была крона их. Прорастали ветви их и сияли
плоды их. Из бессмертной земли были корни их. И река радости орошала их и
окружала их в земле вечной жизни.
\vs Ode 11:17
Затем
поклонился я Яхве за великолепие Его.
\vs Ode 11:18
И сказал я:
Блаженны, о Яхве, возросшие в земле Твоей, и те, кому есть место в Раю Твоем,
\vs Ode 11:19
и кто растет
ростом деревьев Твоих и перебрался из тьмы в свет.
\vs Ode 11:20
Узри же: все
работники Твои чисты, они, делающие добрые дела, и обращающиеся из дикости в
приязнь Твою.
\vs Ode 11:21
Ибо резкий
запах деревьев сих изменился в земле Твоей,
\vs Ode 11:22
и всё
делается частичкой Тебя. Блаженны же труженики вод Твоих и вечные памятники
набожных слуг Твоих.
\vs Ode 11:23
Истинно, в
Раю Твоем обителей много. И нет там ничего пустого, но всё исполнено плодами.
\vs Ode 11:24
Славься же
Ты, Боже, (и да пребудет) райское ликование вовеки.
Аллилуйя.

\vs Ode 12:1
Он наполнил
меня словами истины, дабы я проповедовал Его.
\vs Ode 12:2
И подобно
течению вод, истина вытекает из уст моих, и уста мои возвещают плоды Его.
\vs Ode 12:3
И побудил Он
знание Своё изобиловать во мне, ибо уста Яхве суть Слово истинное и врата Света
Его.
\vs Ode 12:4
И Всевышний
воздал Его коленам Его, которые суть толковники красы Его, и толковники славы
Его, и исповедники цели Его, и проповедники разума Его, и учителя дел Его.
\vs Ode 12:5
Ибо невыразима
тонкость Слова этого; и каково изречение Его, таковы и быстрота Его, и острота
Его, ибо беспредельность Его суть развитие Его.
\vs Ode 12:6
Никогда не
падает Он, но выстаивает, и никто не может понять нисхождение Его или же путь
Его.
\vs Ode 12:7
Ибо каково
дело Его, такова же надежда Его, ибо Он суть свет и заря мысли.
\vs Ode 12:8
И через Него
колена говорят друг с другом, и безмолвные речь обрели.
\vs Ode 12:9
И из Него
изошли любовь и равенство, и друг другу говорили они, что это принадлежало им.
\vs Ode 12:10
И подтолкнуло
их Слово, и познали Того, Кто создал их, ибо они пребывали в гармонии.
\vs Ode 12:11
Ибо уста
Всевышнего говорили им, и объяснение Его расцвело через Него.
\vs Ode 12:12
Ибо жилище
Слова~--- человек, а истина Его~--- любовь.
\vs Ode 12:13
Блаженны же
воспринявшие всё через Него и познавшие Яхве в истине Его.
Аллилуйя.

\vs Ode 13:1
Узрите же,
Яхве~--- зеркало наше. Откройте очи ваши и узрите их в Нем.
\vs Ode 13:2
И изучите вид
лица вашего, вознося хвалы Духу Святому,
\vs Ode 13:3
и сотрите
краску с лица вашего, и возлюбите святость Его и положитесь на неё.
\vs Ode 13:4
Тогда будете
вы безупречными с Ним во все времена.
Аллилуйя.

\vs Ode 14:1
Как очи сына
на отца его, также и мои очи, о, Яхве, к Тебе во все времена.
\vs Ode 14:2
Ибо сердце моё
и радость моя~--- с Тобой.
\vs Ode 14:3
Не отврати же
милостей Своих от меня, о Яхве, и не отними доброты Своей у меня.
\vs Ode 14:4
Протяни же ко
мне, мой Господь, на все времена, десницу Свою, и веди меня до самого конца,
согласно воле Твоей.
\vs Ode 14:5
Позволь же мне
быть угодным Тебе, о Яхве, во славу Твою и во имя Твоё позволь мне спастись от
Нечистого.
\vs Ode 14:6
И снисхождение
Твое, о, Яхве, да снизойдет на меня, и плоды любви Твоей.
\vs Ode 14:7
Обучи же меня
одам истины Твоей, дабы плодоносил я в Тебе.
\vs Ode 14:8
И открой мне
арфу Твоего Святого Духа, чтобы с каждой нотой восхвалял я Тебя, о Яхве.
\vs Ode 14:9
И по многим
милостям Твоим дари мне и спеши дарить по просьбам нашим.
\vs Ode 14:10
Ибо хватит
Тебя на все нужды наши.
Аллилуйя.

\vs Ode 15:1
Как солнце~---
радость алчущих восхода его, так же моя радость~--- Яхве.
\vs Ode 15:2
Ибо Он~---
Солнце моё, и лучи Его вознесли меня, и свет Его рассеял всю тьму с лица моего.
\vs Ode 15:3
Глаза обрел я
в Нем и увидел святой день Его.
\vs Ode 15:4
Уши обрел я и
услышал истину Его.
\vs Ode 15:5
Мысль знания
обрел я и преисполнился восторгом всецело через Него.
\vs Ode 15:6
Отрекся я от
пути ложного и пришел к Нему и премного стяжал спасения у Него.
\vs Ode 15:7
И по щедрости
Своей воздал Он мне, и по образу превосходной красоты Своей создал Он меня.
\vs Ode 15:8
Облекся я
бессмертием именем Его и очистился от тлена милостью Его.
\vs Ode 15:9
Повержена
смерть перед лицом моим, и повержен Шеол словом моим.
\vs Ode 15:10
И взошла в
земле Яхве жизнь вечная, и возвестили её верным Его и беспредельно дана была
всем верующим в Него.
Аллилуйя.

\vs Ode 16:1
Как дело
пахаря~--- пахота, а дело рулевого~--- править кораблем, так и моё дело~--- псалом
Яхве в гимнах Его.
\vs Ode 16:2
Искусство моё
и служение моё~--- в гимнах Его, ибо любовь Его питала сердце мо, а плоды Свои
излил Он на уста мои.
\vs Ode 16:3
Ибо любовь моя
Сам Яхве, затем же стану я петь о Нем.
\vs Ode 16:4
Ибо силен я
похвалами Его и веру имею в Нем.
\vs Ode 16:5
Открою я уста
мои, а Дух Его возвестит через меня славу Яхве и красу Его,
\vs Ode 16:6
работу рук Его
и труд перст Его
\vs Ode 16:7
ради многих
милостей Его и силы Слова Его.
\vs Ode 16:8
Ибо Слово Яхве
изучает невидимое и открывает мысль Его.
\vs Ode 16:9
Ибо видит око
труды Его, а ухо слышит мысль Его.
\vs Ode 16:10
Ибо именно Он
создал землю широкой и налил воды в море,
\vs Ode 16:11
Он расширил
небо и зажег звезды,
\vs Ode 16:12
и Он создал
творение и наставил его, а затем почил Он от трудов Своих.
\vs Ode 16:13
И сотворил
все вещи бегущими путями своими и делающими дела свои, ибо никогда не могут они
ни перестать быть, ни потерпеть неудачу.
\vs Ode 16:14
И светила~---
подданные Слова Его.
\vs Ode 16:15
Сосуд же
света~--- солнце, а сосуд тьмы~--- ночь.
\vs Ode 16:16
Ибо создал Он
солнце во имя дня, дабы был свет, ночь же приносит тьму на лицо земли,
\vs Ode 16:17
и по частичке
друг от друга составляют они красоту Божью.
\vs Ode 16:18
И нет ничего
вне Яхве, ибо Он был прежде, чем что-либо начало быть.
\vs Ode 16:19
И эти миры~---
по слову Его и по мысли сердца Его.
\vs Ode 16:20
Восхваляйте
же и чтите имя Его.
Аллилуйя.

\vs Ode 17:1
Затем
увенчался я Богом моим, и венец мой был живым.
\vs Ode 17:2
И оправдался я
Господом моим, ибо спасение моё нетленно.
\vs Ode 17:3
Освободился я
от гордыни, и не осужден.
\vs Ode 17:4
Узы мои были
разрублены руками Его, стяжал я образ и подобие новой личности, и я ходил под
Ним и был спасен.
\vs Ode 17:5
И водила мной
мысль истины, и я следовал за ней и не блуждал я.
\vs Ode 17:6
И все видевшие
меня были изумлены, и незнакомцем казался я им.
\vs Ode 17:7
А Тот, Кто
знал и возвысил меня, суть Всевышний во всем совершенстве Своем.
\vs Ode 17:8
И прославил Он
меня добротой Своей и вознес понимание моё до вершин истины.
\vs Ode 17:9
И оттуда
указал мне путь шагов Своих, и отверз я двери закрытые.
\vs Ode 17:10
И сокрушил я
засовы железные, ибо собственные кандалы мои сделались горячи и расплавились
предо мною.
\vs Ode 17:11
И ничто не
являлось мне закрытым, ибо всё открывал я.
\vs Ode 17:12
И шел я ко
всем узам моим, чтобы избавиться от них, дабы не оставить никого скованным или
же связанным.
\vs Ode 17:13
И щедро
раздавал я знание своё и восстание своё через любовь свою.
\vs Ode 17:14
И посеял я
плоды свои в сердцах и преобразил их собою.
\vs Ode 17:15
Затем стяжали
они благодать мою и жили, и собирались подле меня и спасались.
\vs Ode 17:16
Ибо сделались
они чреслами моими, а я был главой их.
\vs Ode 17:17
Слава Тебе,
Глава наша, о Яхве, Помазанник.
Аллилуйя.

\vs Ode 18:1
Сердце моё
воспрянуло и обогатилось в любви Всевышнего, дабы под именем моим восхвалял я
Его.
\vs Ode 18:2
Чресла же мои
усилены были, дабы не выпасть из-под власти Его.
\vs Ode 18:3
Немощи вышли
из тела моего, и стояло оно твердо, ради Яхве, по воле Его, ибо твердо Царство
Его.
\vs Ode 18:4
О, Яхве, ради
нуждающихся, не отпускай Слово Своё от меня.
\vs Ode 18:5
И ради трудов
их, удержи при мне совершенство Своё.
\vs Ode 18:6
Да не
низвергнется свет тьмою, а истина да отделится от лжи.
\vs Ode 18:7
Да приведет к
победе десница Твоя спасение наше, и да придет она из всякой области и да
утвердится на берегу всякого, снедаемого горестями.
\vs Ode 18:8
Ты~--- Бог мой,
не в Твоих устах ложь и смерть, только совершенство~--- воля Твоя.
\vs Ode 18:9
И не ведаешь
Ты гордыни, ибо никто из творящих её не ведает Тебя.
\vs Ode 18:10
И не ведаешь
Ты ошибки, ибо никто из творящих её не ведает Тебя.
\vs Ode 18:11
И явилось
невежество словно пыль и словно пена морская.
\vs Ode 18:12
И пустые люди
думали, что величественно оно, и сделались они подобными типу его и были они
истощены.
\vs Ode 18:13
Но те, кто
ведали, поняли и осмыслили и не осквернились мыслями своими,
\vs Ode 18:14
ибо пребывали
они в разуме Всевышнего и высмеяли ходивших ложными путями.
\vs Ode 18:15
Затем же
изрекали они истину от дыхания, которое вдохнул в них Всевышний.
\vs Ode 18:16
Хвала и честь
великая имени Его.
Аллилуйя.

\vs Ode 19:1
Чашку молока
предложили мне, и пил я его во сладости доброты Яхве.
\vs Ode 19:2
Сын~--- чаша
эта, а Отец~--- Тот, кто доил, а Дух Святой~--- Та, которая доила Его;
\vs Ode 19:3
ибо груди Его
были полны, и не хотелось бы, чтобы млеко Его пропало без толку.
\vs Ode 19:4
Дух Святой
обнажил грудь Её, и смешал молоко из двух грудей Отца.
\vs Ode 19:5
Затем дала Она
смесь колену без ведома их, и получившие её пребывают в совершенстве одесную.
\vs Ode 19:6
Чрево девы
поглотило её, и обрела она идею и порождала.
\vs Ode 19:7
Так дева стала
матерью с милосердием превеликим.
\vs Ode 19:8
И трудилась
она и родила Сына, но без боли, ибо не было это бесцельно.
\vs Ode 19:9
И не надо было
ей повитухи, ибо Он побудил её к порождению жизни.
\vs Ode 19:10
Родила же
она, подобно сильному человеку, со страстью, и родила она согласно проявлению, и
приобрела она согласно Великой Силе.
\vs Ode 19:11
И любила она
искупительной (любовью), и оберегала с добротой, и вещала великолепно.
Аллилуйя.

\vs Ode 20:1
Я~--- священник
Яхве, и ему служу я священником;
\vs Ode 20:2
и Ему
предлагаю я подношение мысли Его.
\vs Ode 20:3
Ибо мысль Его
не подобна ни миру сему, ни плоти, ни поклоняющимся по законам плоти.
\vs Ode 20:4
Приношение же
Яхве~--- правда и чистота сердца и уст.
\vs Ode 20:5
Жертвуйте же
безупречно внутренней жизнью вашей, и да не погасится сострадание состраданием
вашим, и да не станете вы угнетать себя.
\vs Ode 20:6
Не подкупайте
иноземца, ибо он не похож на вас, и не следует также пытаться обмануть ближнего
вашего или же лишить его одежды, дабы обнажить его.
\vs Ode 20:7
Но щедро
положитесь на милость Яхве, и придите в Рай Его, и сделайте себе гирлянду из
древа Его.
\vs Ode 20:8
Затем положите
её на главу вашу, и будьте радостны, и положитесь на покой Его.
\vs Ode 20:9
ибо слава Его
проследует пред вами, и стяжаете вы от доброты Его и от милости Его, и помажут
вас в истине, с хвалою святости Его.
\vs Ode 20:10
Хвала и честь
имени Его.
Аллилуйя.

\vs Ode 21:1
Воздел я руки
ввысь во имя сострадания Яхве.
\vs Ode 21:2
Ибо совлек Он
с меня узы мои, а Помощник мой вознес меня соответственно состраданию Его и
спасению Его.
\vs Ode 21:3
И совлек я
тьму и облекся светом
\vs Ode 21:4
и даже сам
обрел чресла. В них не было болезни, или несчастья, или страдания.
\vs Ode 21:5
И щедро
помогала мне мысль Яхве, и Его вечное братство.
\vs Ode 21:6
И был я поднят
в свет, и я проследовал перед Ним.
\vs Ode 21:7
И постоянно
пребывал я подле Него, хваля и исповедуя Его.
\vs Ode 21:8
Подвиг Он
сердце моё переполниться, и нашли его в устах моих; и вросло оно в уста мои.
\vs Ode 21:9
Затем же
сделалось чертой лица моего ликование Яхве и похвала Его.
Аллилуйя.

\vs Ode 22:1
Он, подвигший
меня снизойти свыше и вознестись из мест нижних,
\vs Ode 22:2
и Он,
собирающий тех, что в середине, и сбрасывающий их ко мне,
\vs Ode 22:3
Он,
раскидавший врагов моих и соперников моих,
\vs Ode 22:4
Он, давший мне
власть над узами, дабы мог я развязать их,
\vs Ode 22:5
Он, моими
руками свергнувший дракона семиглавого и поставивший меня на корни его, дабы
сокрушил я семя его,
\vs Ode 22:6
Ты был там и
помогал мне, и в каждом месте имя Твоё окружало меня.
\vs Ode 22:7
Десница Твоя
сокрушила едкий яд его, и рука Твоя указала путь верующим в Тебя.
\vs Ode 22:8
И вызволила
она их из могил и отделила от мертвых.
\vs Ode 22:9
Она взяла
мертвые кости и покрыла их плотью.
\vs Ode 22:10
Но были они
неподвижны, поэтому дала она им жизненную силу.
\vs Ode 22:11
Непорочен был
путь Твой и лицо Твоё; привел Ты мир Свой к тлену, чтобы всё распалось и
обновилось.
\vs Ode 22:12
И основание
всего~--- скала Твоя. И на ней воздвиг Ты Царство Своё, и сделалось оно местом
обитания святых.
Аллилуйя.

\vs Ode 23:1
Радость~---
святым. И кто же облечется в неё, как не сами они?
\vs Ode 23:2
Милость~---
избранным. И кому же стяжать её, как не верующим в нее от начала?
\vs Ode 23:3
Любовь~---
избранным. И кто же облечется в неё, как не одержимые ею от начала?
\vs Ode 23:4
Ходите в
знании Яхве, и щедро познаете вы милость Яхве; как ради ликования Его, так и во
имя совершенства знания Его.
\vs Ode 23:5
И мысль Его
уподобилась письменам, и воля Его снизошла свыше
\vs Ode 23:6
и послана была
она подобно стреле из лука, выстрелившей с усилием.
\vs Ode 23:7
И много рук
поспешили к письму, дабы похитить его, а затем взять и прочесть его.
\vs Ode 23:8
Но ускользнуло
оно из пальцев их, и испугались они его, и печати, бывшей на нем.
\vs Ode 23:9
Ибо не
дозволялось им терять печать его, ибо власть печати этой была величественнее их.
\vs Ode 23:10
Но видевшие
письмо пришли за ним, чтобы изучить, где его должно бросить, и кто должен
прочесть его, и кто должен услышать его.
\vs Ode 23:11
Но колесу
досталось оно, и вошло оно чрез него.
\vs Ode 23:12
И знак был с
ним~--- Царства и Провидения.
\vs Ode 23:13
И всё,
мешавшее колесу, скосило оно и обрубило.
\vs Ode 23:14
И сдержало
оно множество недругов, и вымостило реки.
\vs Ode 23:15
И испещрило и
выкорчевало оно многие леса и расчистило путь.
\vs Ode 23:16
Пала глава к
ногам, ибо к ногам прикатилось колесо, и всё пришедшее с ним.
\vs Ode 23:17
Письмо же
было одним из повелений, и поэтому все области собрались воедино.
\vs Ode 23:18
И виден был
на главе его, на открывшейся главе, даже Сын Истины от Всевышнего Отца.
\vs Ode 23:19
И Он
наследовал всё и обладал всем, и прекратились затем происки многих.
\vs Ode 23:20
Затем же все
соблазнители заупрямились и сбежали, а преследователи увяли и были уничтожены.
\vs Ode 23:21
А письмо
сделалось большим томом, полностью начертанным перстом Божьим.
\vs Ode 23:22
И имя Отца
было на нем, а также Сына и Святого Духа, дабы властвовать вовеки веков.
Аллилуйя.

\vs Ode 24:1
Парила голубка
над главой нашего Яхве Помазанника, ибо был Он главой её,
\vs Ode 24:2
и пела она над
Ним, и услышан был глас её.
\vs Ode 24:3
Затем убоялись
живущие, и обеспокоились чужестранцы.
\vs Ode 24:4
Птица же стала
взлетать, и всякая тварь ползучая подохла в норе своей.
\vs Ode 24:5
И отверзались
и захлопывались бездны, и искали Яхве они, подобно готовым родить.
\vs Ode 24:6
Но не был Он
отдан им на съедение, ведь Он не принадлежал им.
\vs Ode 24:7
Но пропасти
погрузились в печать Яхве, и погибли они в той мысли, с которой оставались от
начала.
\vs Ode 24:8
Ибо от начала
пребывали в трудах они, и концом их тяжкого труда была жизнь.
\vs Ode 24:9
И все они,
пребывавшие в лишениях, погибли, ибо неспособны были они сказать такое слово,
чтобы суметь остаться.
\vs Ode 24:10
И сокрушил
Яхве символы всех не нашедших правды в них.
\vs Ode 24:11
Ибо их
обделили мудростью, их, занимавшихся самовосхвалением в разуме своем.
\vs Ode 24:12
Так их
отвергли, ибо правда не была с ними.
\vs Ode 24:13
Ибо Яхве
открыл путь Свой и широко распространил милость Свою.
\vs Ode 24:14
И те, кто
поняли это, познали святость Его.
Аллилуйя.

\vs Ode 25:1
Спасен я был
от уз моих и бежал к Тебе, о мой Господь.
\vs Ode 25:2
Ибо Ты~---
десница спасения и Спаситель мой.
\vs Ode 25:3
Ты сдержал
восставших против меня, и больше их не видели.
\vs Ode 25:4
Ибо лицо Твоё
пребывало со мной, спасая меня милостью Твоей.
\vs Ode 25:5
Я же был
презираемым и отверженным в глазах многих, и был я в их глазах свинцу подобен.
\vs Ode 25:6
И обрел я силу
от Тебя, и помощь.
\vs Ode 25:7
Светильник
водрузил ты ради меня и справа, и слева, чтобы не было во мне ничего
неосвещенного.
\vs Ode 25:8
И облекся я
покровом Духа Твоего и отбросил прочь от себя одежды кожаные.
\vs Ode 25:9
Ибо десница
Твоя вознесла меня и принудила болезнь оставить меня.
\vs Ode 25:10
И стал я
могучим истиной Твоей и святым правдой Твоей.
\vs Ode 25:11
И все недруги
мои убоялись меня, и сделался я (человеком) Яхве во имя Яхве.
\vs Ode 25:12
И оправдался
я добротой Его и покоем Его во веки веков.
Аллилуйя.

\vs Ode 26:1
Излил я хвалу
Яхве, ибо я~--- Он Сам.
\vs Ode 26:2
И зачитаю я
святую оду Его, ибо сердце моё~--- с Ним.
\vs Ode 26:3
Ибо арфа его в
руке моей, и не утихнут оды покоя Его.
\vs Ode 26:4
Желаю я
воззвать к Нему всем сердцем моим, желаю я восхвалять и возвышать Его всеми
чреслами моими.
\vs Ode 26:5
Ибо от востока
до запада пребывает хвала Его,
\vs Ode 26:6
а также с юга
на север простирается благодарение Его,
\vs Ode 26:7
даже с пиков
вершин и до края их в совершенстве Его.
\vs Ode 26:8
Кто же сумеет
записать оды Яхве и кто сумеет прочесть их?
\vs Ode 26:9
И кто сумеет
приуготовить себя к жизни, чтобы спастись самому?
\vs Ode 26:10
И кто сумеет
вынудить Всевышнего, чтобы Собственными устами зачитал Он?
\vs Ode 26:11
Кто же сумеет
истолковать чудеса Яхве? Убьют толкующего~--- истолкованное еще останется.
\vs Ode 26:12
Ибо
достаточно воспринять и удовольствоваться, ведь сочинители од стоят спокойные;
\vs Ode 26:13
подобно реке
с усиленно бьющим истоком и текущей на смену им, дабы отыскать это.
Аллилуйя.

\vs Ode 27:1
Простер я руки
свои и освятил Господа Моего,
\vs Ode 27:2
ибо
простирание рук моих~--- знак Его,
\vs Ode 27:3
а простирание
моё~--- вертикальный крест.
Аллилуйя.

\vs Ode 28:1
Как крылья
голубей распростерты над птенцами их, а клювики птенцов смотрят в клювы их, так
же и крылья Духа распростерты над сердцем моим.
\vs Ode 28:2
Сердце моё
непрерывно освежается и прыгает от радости, как малыш, прыгающий от радости во
чреве матери своей.
\vs Ode 28:3
Я веровал, а
значит пребывал я в покое, ибо верующий~--- Тот, в кого я уверовал.
\vs Ode 28:4
Он с чувством
благословил меня, и глава моя пребывает с Ним.
\vs Ode 28:5
Ни кинжал не
разделит меня с Ним, ни меч,
\vs Ode 28:6
ибо готов я
прежде, чем настанет погибель, и восстал в Его бессмертном уделе.
\vs Ode 28:7
И объяла меня
жизнь бессмертная, и облобызала меня.
\vs Ode 28:8
И от жизни
этой исходит Дух, Который во мне. И не может Он умереть, ибо Он~--- Жизнь.
\vs Ode 28:9
Видевшие же
меня изумились; ибо притесняли меня.
\vs Ode 28:10
И думали они,
что поглощали меня, ибо казался я им одним из пропавших.
\vs Ode 28:11
Но
несправедливость моя сделалась спасением моим.
\vs Ode 28:12
И сделался я
отвращением их, ибо не было во мне ревности.
\vs Ode 28:13
Ибо долго
делал я добро всякому, ненавидящему меня.
\vs Ode 28:14
И окружили
они меня словно собаки, что по глупости нападают на хозяев своих.
\vs Ode 28:15
Ибо мысль их
развращена и разум их извращен.
\vs Ode 28:16
Но несу воду
я в деснице моей, а горечь их продлил я приязнью своей.
\vs Ode 28:17
И не погиб я,
ибо ни братом их не был я, ни рождение моё не было подобным их.
\vs Ode 28:18
И искали они
смерти моей, но не нашли ее возможной, ибо был я старше, чем память их; и во
тщете массами набросились они на меня.
\vs Ode 28:19
И бывшие
после меня тщетно пытались сокрушить памятник Тому, Кто был пред ними.
\vs Ode 28:20
Ибо мысль
Всевышнего не могла быть предрассудком, а сердце Его превыше всякой мудрости.
Аллилуйя.

\vs Ode 29:1
Яхве~--- надежда
моя, и да не устыжусь Его.
\vs Ode 29:2
Ибо во хвале
Своей создал Он меня, и милостью Своей даже её воздал Он мне.
\vs Ode 29:3
И милостями
Своими возвеличил Он меня, и великой честью Своей возвысил Он меня.
\vs Ode 29:4
И побудил Он
меня восстать из глубин Шеола, и из уст смерти вызволил Он меня.
\vs Ode 29:5
И умалил я
врагов своих, и оправдал меня Он милостью Своей.
\vs Ode 29:6
Ибо уверовал я
в Помазанника Яхве и счел, что Он и есть Бог.
\vs Ode 29:7
И явил Он мне
знак, и водил меня светом Своим.
\vs Ode 29:8
И дал Он мне
скипетр власти Своей, чтобы подчинил я машины людские и умалил силу могущества,
\vs Ode 29:9
воевал словом
Его и одержал победу силой Его.
\vs Ode 29:10
И низверг
Яхве врага моего словом Своим, и тот уподобился пыли, сдуваемой ветром.
\vs Ode 29:11
И воздал я
хвалу Всевышнему, ибо возвеличил Он слугу Своего и Сына служанки Своей.
Аллилуйя.

\vs Ode 30:1
Наполнитесь же
водою из живого источника Яхве, ибо он открылся вам,
\vs Ode 30:2
и придите все
алчущие и напейтесь, и отдохните подле источника Яхве,
\vs Ode 30:3
ибо приятен и
сверкающ он и вечно освежает.
\vs Ode 30:4
Ибо много
слаще вода Его, нежели мёд, и соты пчелиные не сравнятся с ним;
\vs Ode 30:5
ибо исходил он
из уст Яхве, а вызван был из сердца Яхве.
\vs Ode 30:6
И пришла она
бесконечной и невидимой, и пока не налилась в середине, они не узнали её.
\vs Ode 30:7
Блаженны же
пившие оттуда и освежившиеся ею.
Аллилуйя.

\vs Ode 31:1
Пред Яхве
исчезали пропасти, а Тьма рассеивалась пред появлением Его.
\vs Ode 31:2
Ошибка же
ошиблась и погибла из-за Него, а неуважению не дали тропу, ибо была она
затоплена правдой Яхве.
\vs Ode 31:3
Отверз Он уста
Свои и изрек милость и ликование, и зачитал новый гимн имени Своему.
\vs Ode 31:4
Затем же
возвысил Он глас Свой ко Всевышнему и вверил ему ставших Сыновьями из-за Него.
\vs Ode 31:5
И лицо Его
признано было, ибо так воздал Ему Святой Отец Его.
\vs Ode 31:6
Придите,
обиженные, и возрадуйтесь.
\vs Ode 31:7
И владейте
собою милостиво и вберите в себя жизнь бессмертную.
\vs Ode 31:8
И осудили меня
они, когда восстал я,~--- меня, не осужденного.
\vs Ode 31:9
Затем же
разделили они добро моё, хотя ничего не причиталось им.
\vs Ode 31:10
Но вытерпел
я, и держал себя в руках, и безмолвен был, дабы не быть умерщвленным ими.
\vs Ode 31:11
Но
непоколебимо стоял я, словно твердая скала, долгое время побиваемая накатами
волн, и терпел.
\vs Ode 31:12
И желчность
их смиренно вынес я, дабы искупить народ мой и наставить его,
\vs Ode 31:13
и дабы не
отречься от обетов патриархам, которым был обещан я во спасение потомков их.
Аллилуйя.

\vs Ode 32:1
Блаженным~---
радость сердец их и свет Того, Кто пребывает в них,
\vs Ode 32:2
и Слово правды
само родившееся,
\vs Ode 32:3
ибо усилен был
Он Святою Силою Всевышнего, и не поколеблен Он вовеки веков.
Аллилуйя.

\vs Ode 33:1
Но вновь
поспешила милость и отвергла Искусителя и снизошла в него, дабы низвергнуть его.
\vs Ode 33:2
И вызвал он
сплошное разрушение перед собой и испортил весь труд свой.
\vs Ode 33:3
И стоял он на
пике горной вершины и громко вопил из конца в конец земли.
\vs Ode 33:4
Затем приволок
он к нему всех подчиненных ему, ибо не как грешник явился он.
\vs Ode 33:5
Однако именно
совершенная дева стояла, проповедуя, и взывая, и говоря:
\vs Ode 33:6
О, вы, сыновья
человеческие, вернитесь, и вы, дщери их, придите,
\vs Ode 33:7
и оставьте
пути Искусителя этого, и приблизьтесь ко мне.
\vs Ode 33:8
И войду я в
вас, и выведу из разрушения вас, и сделаю мудрыми вас на путях правды.
\vs Ode 33:9
Не будьте же
ни грешными, ни гибнущими.
\vs Ode 33:10
Повинуйтесь
мне, и спасены будете, ибо возглашу я вам милость Божью.
\vs Ode 33:11
И чрез меня
спасетесь вы и сделаетесь блаженными. Я~--- суд ваш;
\vs Ode 33:12
и
положившихся на меня не обвинят неправедно, но нетленность стяжают они в новом
мире.
\vs Ode 33:13
Избранные мои
пошли за мною, и пути мои сделаю я известными тем, кто взыскивает меня; и
завещаю я им имя своё.
Аллилуйя.

\vs Ode 34:1
Нет ни
трудного пути там, где есть простодушие, ни преграды для прямодушия,
\vs Ode 34:2
ни урагана в
глубине просветленной мысли.
\vs Ode 34:3
Там, где
окружен некто со всех сторон приятной страной, там ничто не разделено в нем.
\vs Ode 34:4
То, что внизу,
подобно тому, что вверху.
\vs Ode 34:5
Ибо всё~---
свыше, а снизу~--- ничего, но спорят с этим те, в ком нет понимания.
\vs Ode 34:6
Милость же
явлена во спасение ваше.
Аллилуйя.

\vs Ode 35:1
Живительный
ливень Яхве преспокойно накрыл меня и облако покоя: побудили они взойти над
головой моей,
\vs Ode 35:2
чтобы могло
оно оберегать меня во все времена. И стало оно спасением мне.
\vs Ode 35:3
Всякий
обеспокоился и убоялся, и изошли из них дым и судилище.
\vs Ode 35:4
Я же спокоен
был в воинстве Яхве; больше, нежели тенью был он для меня, и больше, нежели
основанием.
\vs Ode 35:5
И носили меня,
как мать дитя своё, Он же дал мне молоко, росу Яхве.
\vs Ode 35:6
И обогатился я
пользою Его, и покоился в совершенстве Его.
\vs Ode 35:7
И распростер я
в вознесении руки свои, и направился ко Всевышнему, и искуплен был пред Ним.
Аллилуйя.

\vs Ode 36:1
Покоился я в
Духе Яхве, и Он вознес меня к небесам;
\vs Ode 36:2
и велел мне
стать на ноги в вышнем месте Яхве, перед совершенством Его и славой Его, где и
продолжил я славить Его сочинением од Его.
\vs Ode 36:3
Дух же принес
меня к лицу Яхве, а поскольку был я Сыном человека, меня назвали Светом, Сыном
Божьим;
\vs Ode 36:4
ибо был я
достославным среди славных и величайшим среди великих.
\vs Ode 36:5
Ибо к величию
Всевышнего сделал меня Он, и по новизне Своей обновил Он меня.
\vs Ode 36:6
И помазал Он
меня совершенством Своим, и сделался я одним из тех, кто подле Него.
\vs Ode 36:7
И открылись
уста мои, словно облако росы, и хлынуло сердце моё как скважина праведности.
\vs Ode 36:8
И приближение
моё было мирным, и поставили меня в Духе Провидения.
Аллилуйя.

\vs Ode 37:1
Воздел я руки
свои к Яхве, и ко Всевышнему возвысил свой глас я.
\vs Ode 37:2
И говорил я
устами сердца своего, и слышал Он меня, когда глас мой достигал Его.
\vs Ode 37:3
Его же Слово
изошло ко мне, дабы воздались мне плоды трудов моих,
\vs Ode 37:4
и воздался бы
покой мне милостью Яхве.
Аллилуйя.

\vs Ode 38:1
Взошел я во
Свет Истины словно в колесницу, и Истина вела меня и велела прийти,
\vs Ode 38:2
и велела мне
пройти через пропасти и заливы и спасала меня от скал и долин,
\vs Ode 38:3
и стала для
меня небесами спасения, и водрузила меня в месте бессмертной жизни.
\vs Ode 38:4
И шел Он со
мною и велел мне отдыхать и не позволял мне оступаться, ибо был Он и является
Истиной.
\vs Ode 38:5
Мне было
безопасно, ибо я всегда шел с Ним, и не оступался ни в чем, ибо я слушался Его;
\vs Ode 38:6
ибо ошибка
сбежало от Него и никогда не сталкивалась с Ним.
\vs Ode 38:7
Но Истина шла
прямым путем, и всё, чего не смыслил я, являл Он мне:
\vs Ode 38:8
все яды Ошибки
и смертельные болезни, считавшиеся сладостными.
\vs Ode 38:9
И поражая
Искусителя, я видел, как украшена была порочная невеста, и жениха, совращающего
и совращаемого.
\vs Ode 38:10
И спросил я
Истину: Кто они? И сказала Она мне: Это обманщик и ошибка,
\vs Ode 38:11
и подражают
они Возлюбленному и Невесте Его, и понуждают мир сей оступиться и совращают его.
\vs Ode 38:12
И зовут они
многих на пир брачный, и позволяют им пить вино отравы своей,
\vs Ode 38:13
дабы вынудить
их вытошнить мудростью и знанием их, и готовят для них бессмыслицу.
\vs Ode 38:14
Затем же они
покидают их, и так они спотыкаются, словно безумные и растленные люди,
\vs Ode 38:15
ведь в них
нет ведения, да и не ищут они его.
\vs Ode 38:16
Но я сделался
мудрым, дабы не пасть в руки обманщиков, и возрадовался внутри себя, ибо истина
шла со мной.
\vs Ode 38:17
Ибо я был
создан, и жил, и был искуплен, и начала мои легли из-за руки Яхве, ибо посадил
Он меня.
\vs Ode 38:18
Ибо посадил
Он корень, и полил его, и ухаживал за ним, и благословлял его, и плоды его
пребудут вовеки.
\vs Ode 38:19
Он глубоко
врос, и пророс, и развился, и полнился, и ширился,
\vs Ode 38:20
и Яхве одним
славился, посадкой Его и выращиванием Его,
\vs Ode 38:21
и заботой
Его, и благословением уст Его, в чудесном саду одесную Его,
\vs Ode 38:22
и в знаниях
сада Его, и в понимании разума Его.
Аллилуйя.

\vs Ode 39:1
Свирепые реки
сила Яхве, они бросают головой вниз презирающих Его,
\vs Ode 39:2
и путают тропы
их, и разрушают переправы их,
\vs Ode 39:3
и хватают тела
их, и растлевают естества их.
\vs Ode 39:4
Ибо они~---
быстрее молний, даже скорее.
\vs Ode 39:5
Но не помешают
тем, кто переправляется через них с верою
\vs Ode 39:6
и не выбросят
тех, кто безупречно сплавляется по ним.
\vs Ode 39:7
Ибо знак на
них~--- Сам Яхве, и знак этот~--- путь для переправляющихся во имя Яхве.
\vs Ode 39:8
Затем же
положитесь на имя Всевышнего и познайте Его, и вы переправитесь безопасно, ибо
реки станут послушны вам.
\vs Ode 39:9
Яхве же
вымостил их Словом Своим, и он ходил и пересекал их как посуху.
\vs Ode 39:10
И твердо
отпечатывались на водах следы Его, и не стирались, но подобны были бруску древа,
на Истине выстроенного.
\vs Ode 39:11
С обеих
сторон вздымались волны, но следы нашего Яхве Помазанника оставались тверды,
\vs Ode 39:12
и ни намокали
они, ни разрушались.
\vs Ode 39:13
И путь этот
был предначертан переправляющимся следом за Ним, и строго следующим стезею веры
Его, и чтящим имя Его.
Аллилуйя.

\vs Ode 40:1
Как истекает
мёд из сот пчелиных, а молоко~--- из жены, любящей детей своих, так и надежда на
Тебя, Боже мой.
\vs Ode 40:2
Как хлынет
вода из фонтана, так и сердце моё хлынет восхвалением Яхве, и вознесут хвалу Ему
уста мои.
\vs Ode 40:3
И усладится
язык мой гимнами Его, и чресла мои помажутся одами Его.
\vs Ode 40:4
Лицо же моё
веселится ликованием Его, и дух мой ликует в любви Его, и естество моё светится
в Нем.
\vs Ode 40:5
И уверует в
Него убоявшийся, и искупления достигнет в Нем.
\vs Ode 40:6
И владения Его
жизнь бессмертная, и стяжавшие её непорочны.
Аллилуйя.

\vs Ode 41:1
Да восхвалят
Яхве все чада Яхве, да стяжаем мы истину веры Его.
\vs Ode 41:2
И дети его да
утвердятся в Нем, а поэтому воспоем же любви Его.
\vs Ode 41:3
Живем мы в
Яхве милостью Его, а жизнь стяжали мы у Помазанника Его.
\vs Ode 41:4
Ибо великий
день засветил нам, и чудесен Тот, кто воздал нам славою Своей.
\vs Ode 41:5
Пребудем же
поэтому все мы в согласии во имя Яхве и воздадим почести в доброте Его.
\vs Ode 41:6
И да воссияют
лица наши во свете Его, и да сосредоточатся сердца наши на любви Его и днем, и
ночью.
\vs Ode 41:7
Так возликуем
же ликованием Яхве!
\vs Ode 41:8
Все видящие
меня да изумятся, ибо я~--- из иного рода.
\vs Ode 41:9
Ибо Отец
Истины вспомнил обо мне, Он, от начала владеющий мною.
\vs Ode 41:10
Ибо богатства
Его породили меня, и мысль сердца Его.
\vs Ode 41:12
И пребывает с
нами Слово Его на всем пути нашем~--- Спаситель, дающий жизнь и не отвергающий
нас.
\vs Ode 41:13
Сын же
Всевышнего явился в совершенстве Отца Своего.
\vs Ode 41:14
И забрезжил
из Слова свет, пребывавший в Нем прежде времени.
\vs Ode 41:15
Помазанник же
один в истине. И знали Его прежде основ мира сего, чтобы истиной имени Своего
вовеки наделял он жизнью людей.
\vs Ode 41:16
Новая же
песнь~--- Яхве, от любящих Его.
Аллилуйя.

\vs Ode 42:1
Распростер я
руки и приблизился к Яхве, ибо распростертые руки мои~--- знак Его.
\vs Ode 42:2
И
распростертые мои~--- вертикальный крест, поднявшийся на стезе Праведного.
\vs Ode 42:3
И сделался я
бесполезным для не знавших меня, ибо скроюсь я от тех, кто не одержим мною,
\vs Ode 42:4
а пребуду с
любящими меня.
\vs Ode 42:5
Все
преследовавшие меня мертвы, искавшие меня, злословившие против меня,~--- ибо жив
я.
\vs Ode 42:6
К тому же
воскрес я, и я с ними, и буду говорят устами их.
\vs Ode 42:7
Ибо отвергли
они притеснявших их, и запечатлел их клеймом любви моей.
\vs Ode 42:8
Подобно клейму
жениха на невесте клеймо моё на знающих меня.
\vs Ode 42:9
И как чертог
брачный выстраивается родными брачной пары, так и любовь моя (живет) верующими в
меня.
\vs Ode 42:10
Меня не
отвергли, хотя и хотели, и не погиб я, хотя они и считали меня (погибшим).
\vs Ode 42:11
Шеол же
увидел меня и поколебался, и Смерть низвергла меня и многих вместе со мною.
\vs Ode 42:12
Я был уксусом
и горечью его, и спустился я с нею вниз на глубину его.
\vs Ode 42:13
Затем же
отпустил он ноги и голову, ибо был он неспособен выносить лицо моё.
\vs Ode 42:14
И создал я
собрание живых среди его мертвых и говорил с ними устами живыми, дабы слово моё
не было бесполезным.
\vs Ode 42:15
И ринулись ко
мне умершие, и возопили, и заголосили:
<<Сын Божий, смилуйся над нами,
\vs Ode 42:16
и поступи с
нами по доброте Твоей, и вызволи нас из оков тьмы,
\vs Ode 42:17
и открой нам
дверь, в которую мы сможем выйти к Тебе, ибо осознали мы, что смерть наша не
коснулась Тебя.
\vs Ode 42:18
А еще да
спасемся с Тобою, ибо Ты~--- Спаситель наш.>>
\vs Ode 42:19
Затем услышал
я голос их, и вложил веру их в сердце моё.
\vs Ode 42:20
И вложил я
имя своё в головы их, ибо свободны они и принадлежат мне.
Аллилуйя.

\bibbookdescr{Tsm}{
  inline={Завещание Симеона,\\второго сына Иакова и Лии\fns{В греч. тексте $+$ ``о зависти''.}},
  toc={Завещание Симеона},
  bookmark={Завещание Симеона},
  header={Завещание Симеона},
  abbr={Сим}
}
\vs Tsm 1:1
Список слов Симеона, речённых им к сыновьям его перед тем,
как умер он в 120-ый год жизни своей,
в тот же год, что и брат его Иосиф.
\vs Tsm 1:2
Когда занемог Симеон, пришли проведать его дети его, и, сделав
усилие, сел он, поцеловал их и сказал:
\vs Tsm 2:1
послушайте, дети мои, Симеона, отца вашего; возвещу вам то,
что имею я в сердце моём.

\vs Tsm 2:2
Родился я от Иакова и был вторым сыном отца моего, и Лия, мать моя,
нарекла меня Симеоном, ибо услышал Господь мольбу ее.
\vs Tsm 2:3
Сделался я весьма сильным, не боялся труда и не страшился никакого дела.
\vs Tsm 2:4
Ибо сердце моё было сухим, печень моя недвижимой,
а внутренности мои нечувствительными.
\vs Tsm 2:5
Ведь и мужество даётся от Всевышнего людям в душах и телах.
\vs Tsm 2:6
Во время юности моей завидовал я сильно Иосифу,
ибо возлюбил его отец мой более всех.
\vs Tsm 2:7
И утвердился я против него в сердце моём, возжелав убить его,
так как Князь обмана и дух зависти ослепили мне ум, и забыл я,
что это брат мой, и не пощадил отца моего Иакова.
\vs Tsm 2:8
Но Бог его и Бог отцов наших послал ангела своего и избавил
Иосифа от рук моих.

\vs Tsm 2:9
Ибо, когда я отправился в Сиким, чтобы принести притирание для стада,
а Рувим  в Дофаим, где было необходимое нам и все хранилища наши,
Иуда, брат мой, продал Иосифа Измаильтянам.
\vs Tsm 2:10
Рувим, услышав об этом, опечалился, ибо он хотел отвести его к отцу.
\vs Tsm 2:11
Я же, услышав это, сильно разгневался на Иуду,
ибо он отпустил Иосифа живым,
и 5 месяцев пребывал я в гневе на него.
\vs Tsm 2:12
И сковал меня Господь и удалил от меня дело рук моих,
ибо правая рука моя стала наполовину сухой на 7 дней.
\vs Tsm 2:13
И познал я, дети, что из-за Иосифа случилось это со мною.
И, раскаявшись, заплакал я и молил Господа Бога,
чтобы восстановилась рука моя и удержался я от всякой скверны
и зависти и ото всякого безрассудства.
\vs Tsm 2:14
Ибо понял я, что злое дело замыслил перед лицом Господа и Иакова,
отца моего, против Иосифа, брата моего, позавидовав ему.

\vs Tsm 3:1
Ныне, дети мои, послушайте меня и остерегитесь духа обмана и зависти.
\vs Tsm 3:2
Ведь зависть властвует надо всем помыслом человека
и не дает ему ни есть, ни пить, ни делать ничего доброго.
\vs Tsm 3:3
Но всечасно подстрекает она убить того, кому человек завидует,
но тот всечасно процветает, а завистник чахнет.
\vs Tsm 3:4
И вот, 2 года сокрушал я в страхе Господнем душу мою постом.
И узнал я, что избавление от зависти происходит через страх Божий.
\vs Tsm 3:5
Если кто прибегает к Господу, оставляет его злой дух
и становится разум лёгким.
\vs Tsm 3:6
И наконец, начинает он сочувствовать тому, кому завидовал, и
примиряется с любящими его, и так избавляется от зависти.

\vs Tsm 4:1
Спросил отец мой, что со мною, ибо заметил меня скорбящим, и
сказал я ему, что переполняется печень моя.
\vs Tsm 4:2
Ибо печалился я чрезвычайно, что виновен в продаже Иосифа.
\vs Tsm 4:3
И когда пошли мы в Египет и связали меня как соглядатая,
познал я, что справедливо страдаю и не опечалился.
\vs Tsm 4:4
Иосиф же был добрый муж, дух Божий в себе имевший,
милостивый и сострадательный; не вспомнил мне зла, но
возлюбил меня с братьями моими.

\vs Tsm 4:5
Так остерегайтесь же, дети мои, всякой ревности и зависти и живите в
простоте сердечной, чтобы дал и вам Бог милость и славу и благословение на
головы ваши, как вы видите то на Иосифе.
\vs Tsm 4:6
Ни в какой день не стыдил он нас за дело это,
но возлюбил нас как душу свою, и более сыновей своих почтил нас,
и богатство, и скот, и плоды даровал нам.

\vs Tsm 4:7
И вы, дети мои, возлюбите каждый брата своего в доброте сердечной,
и отойдёт от вас дух зависти.
\vs Tsm 4:8
Ибо озлобляет он душу и губит тело, гнев и вражду вводит в
помышление и побуждает к крови и вводит разум в экстаз,
и смятение создает в душе и дрожь в теле.
\vs Tsm 4:9
Даже во сне злая зависть, соблазняя человека,
пожирает его и духами злыми возмущает душу его,
и заставляет тело его содрогаться,
и смятением лишает сна ум его,
и как дух злой и губительный является людям.
\vs Tsm 5:1
Оттого Иосиф был прекрасен лицом и приятен видом своим,
что не поселялось в нем ничто злое;
ибо смущение духа проступает явно на лице человека.

\vs Tsm 5:2
Ныне, дети мои, смягчите сердце ваше пред Господом
и выпрямите пути ваши пред людьми,
и стяжаете благодать пред лицом Господа и людей.
\vs Tsm 5:3
И остерегайтесь блуда, ибо блуд порождает всякое зло,
отдаляя от Бога и приближая к Велиару.
\vs Tsm 5:4
Видел я в книге Еноха, что сыновья ваши совратятся
от блуда и обиду нанесут мечом своим сыновьям Левия.
\vs Tsm 5:5
Но не смогут они противостоять Левию,
ибо поведёт он брань Господню и одолеет всякое войско ваше.
\vs Tsm 5:6
И будут они малочисленны, разделенные в Левин и в Иуде, и
не будет из вас никого, кто властвовал бы,
как и пророчествовал отец наш в благословениях своих.

\vs Tsm 6:1
И вот, сказал я вам всё, дабы оправдать себя от греха вашего.
\vs Tsm 6:2
И если удалите от себя зависть и всякое жестокосердие,
словно роза расцветут кости мои в Израиле,
и словно лилия плоть моя в Иакове,
и будет благоухание моё словно аромат Ливана,
и умножатся святые от меня во веки веков,
и взрастут отрасли их.
\vs Tsm 6:3
Тогда погибнет семя Ханаана,
и не будет остатка у Амалика,
и сгинут все Каппадокийцы,
и все Хетты истребятся.
\vs Tsm 6:4
Тогда угаснет земля Хама, и погибнет весь народ.
Тогда почиет вся земля от смуты, и всё, что под небесами, от войны.
\vs Tsm 6:5
Тогда прославится Сим,
ибо Господь Бог Израиля придет на землю [как человек] и тем
спасёт Адама.
\vs Tsm 6:6
Тогда предан будет всякий дух соблазна на поругание,
и люди обретут власть над злыми духами.
\vs Tsm 6:7
Тогда воскресну и я в радости и благословлю Всевышнего ради чудес его,
[ибо Господь, приняв тело и вкусив пищу с людьми, спас людей.]

\vs Tsm 7:1
Ныне, дети мои, слушайте Левия и Иуду,
и не восставайте на два эти колена,
ибо от них исполнится нам спасение Божие.
\vs Tsm 7:2
Ибо восстанет Господь из Левия как Первосвященник,
а из Иуды как Царь [Бог и человек].
Он спасёт [все народы и] род Израиля.
\vs Tsm 7:3
Для того внушаю вам это, дабы и вы внушили детям вашим,
да сохранят всё в поколениях своих.

\vs Tsm 8:1
Завершил Симеон наставление сыновей своих и почил с отцами
своими, будучи 120-и лет.
\vs Tsm 8:2
И положили его во гроб деревянный,
чтобы отнести кости его в Хеврон.
И отнесли их втайне, пока Египтяне вели войну.
\vs Tsm 8:3
Ибо кости Иосифа сохранили Египтяне в гробнице царей.
\vs Tsm 8:4
Сказали им прорицатели, что, если вынесут кости Иосифа,
тьма и мрак будут по всей земле и несчастье великое Египтянам,
так что и со светильником не узнает никто брата своего.

\vs Tsm 9:1
И оплакали сыновья Симеона, отца своего.
И пребывали в Египте вплоть до дней, когда Моисей вывел их рукою своею.

\bibbookdescr{1Sb}{
  inline={Первая книга Сивилл},
  toc={1-я Сивилл},
  bookmark={1-я Сивилл},
  header={1-я Сивилл},
  abbr={1~Сив}
}
\vs 1Sb 1:1 С самых истоков начав, возвещу я судьбу поколений,

\vs 1Sb 1:2 Все по порядку скажу от первого века и дальше,

\vs 1Sb 1:3 То, что случилось уже, что есть и что впредь ожидает

\vs 1Sb 1:4 Смертных людей, преступивших священный закон благочестья.

\vs 1Sb 1:5 Первым мне Бог повелел рассказать правдиво о том, как

\vs 1Sb 1:6 Мир порожден был,  а ты внимай моим песням прилежно,

\vs 1Sb 1:7 Смертный, чтобы из них ни слова зря не пропало.

\vs 1Sb 1:8 Царь, всех превыше стоящий, создал и небо и землю,

\vs 1Sb 1:9 Да зародится,  сказав, и тут же все зародилось.

\vs 1Sb 1:10 Тартаром твердь окружив, Он свет дал миру сладчайший,

\vs 1Sb 1:11 Сверху воздвиг небосвод, простер воды светлого моря,

\vs 1Sb 1:12 Множество ярких созвездий обвил вкруг полюса, землю

\vs 1Sb 1:13 Всю цветами украсил, смешал с потоками море,

\vs 1Sb 1:14 Воздух ветрами смутил и влажные дал ему тучи.

\vs 1Sb 1:15 К тварям живым перейдя, Он рыб глубинам доверил,

\vs 1Sb 1:16 Птиц  воздушным потокам, чащобам  зверей густошерстых,

\vs 1Sb 1:17 Гадов пустил по земле, и все, что ныне мы видим,

\vs 1Sb 1:18 Словом единым создал, и по слову все появилось

\vs 1Sb 1:19 Быстро и точно: сие созерцает теперь Нерожденный,

\vs 1Sb 1:20 С неба на землю смотря,  на том Его труд завершен был.

\vs 1Sb 1:21 И уже после того слепил Он живое творенье 

\vs 1Sb 1:22 Образ Свой запечатлев, человека, прекрасного видом,

\vs 1Sb 1:23 Сходного с Богом. Ему повелел в раю поселиться,

\vs 1Sb 1:24 Чтобы благие дела предметом забот его были.

\vs 1Sb 1:25 Тот, оказавшись один средь цветущего райского сада,

\vs 1Sb 1:26 Стал по беседе скучать и вседневно желаньем томился

\vs 1Sb 1:27 Облик увидеть такой же, как свой. Тут, кость его вынув,

\vs 1Sb 1:28 Бог из нее сотворил супругу законную  Еву,

\vs 1Sb 1:29 Женщину дивной красы, и велел ей в раю с человеком

\vs 1Sb 1:30 Жить совместно. Адам, ее оглядев, удивлен был.

\vs 1Sb 1:31 Радуясь сердцем, смотрел на себе подобную. С речью

\vs 1Sb 1:32 К ней обратился разумной, и сами собой получались

\vs 1Sb 1:33 Те слова у него  предусмотрено все было Богом.

\vs 1Sb 1:34 Похоть не застила ум их, стыда они не знавали,

\vs 1Sb 1:35 Были сердца далеки от всякого зла. Словно звери,

\vs 1Sb 1:36 Тело свое напоказ безпечно они выставляли.

\vs 1Sb 1:37 Сразу же после того, как создал, им указанье

\vs 1Sb 1:38 Дал Господь, чтоб они не трогали древа: на это

\vs 1Sb 1:39 Змей их ужасный подбил, на горе обманом заставив

\vs 1Sb 1:40 Смертную долю принять, а с нею вместе  познанье

\vs 1Sb 1:41 Зла и Добра. Между тем, предательство первой свершила

\vs 1Sb 1:42 Женщина: мужу дала, убедив неразумного словом.

\vs 1Sb 1:43 Он же, речами жены увлечен, позабыл о безсмертном

\vs 1Sb 1:44 Мира Творце и совет без внимания мудрый оставил.

\vs 1Sb 1:45 Так получили они по заслугам, когда им досталось

\vs 1Sb 1:46 Зло вместо блага в удел. Проткнув смоковницы листья,

\vs 1Sb 1:47 Сшили одежды себе и друг на друга надели,

\vs 1Sb 1:48 Чресла листвою прикрыв, ибо стыд у них появился.

\vs 1Sb 1:49 Бог же безсмертный обрушил Свой гнев, их выгнал из Рая

\vs 1Sb 1:50 В смертной юдоли свой век коротать, когда не хранили,

\vs 1Sb 1:51 Раз услыхав, они в памяти слово великого Бога.

\vs 1Sb 1:52 Те, очутившись внезапно среди плодородной равнины,

\vs 1Sb 1:53 Стали слезами ее поливать, непрерывно стеная.

\vs 1Sb 1:54 К ним смягчился Безсмертный тогда и сказал в утешенье:

\vs 1Sb 1:55 Род продолжайте, плодитесь, с землей обращайтесь умело,

\vs 1Sb 1:56 Так, чтобы в поте лица добывали, чем голод насытить.

\vs 1Sb 1:57 Слово такое изрек. В обмане виновного змея

\vs 1Sb 1:58 Землю заставил тереть животом и хвостом, без пощады

\vs 1Sb 1:59 Выгнав из Рая. Вражду тогда между ним поселил Он

\vs 1Sb 1:60 И человеком. Один уберечь свою голову тщится,

\vs 1Sb 1:61 Пятку спасает другой: близка ведь стала отныне

\vs 1Sb 1:62 Смерть и к людям, и к тем, кто злом отравляет советы.

\vs 1Sb 1:63 Начал тут род пополняться людской, как велено было

\vs 1Sb 1:64 Им, всемогущим Владыкой. Одно за другим приходили,

\vs 1Sb 1:65 Множа число, поколенья. Дома они начали строить,

\vs 1Sb 1:66 Стены и города возводить с немалым искусством.

\vs 1Sb 1:67 Долгий и радостный день им сопутствовал в жизни. Не зная

\vs 1Sb 1:68 Горя, смерть принимали они, погружаясь как будто

\vs 1Sb 1:69 В сон. Счастливыми были те люди; могучих героев,

\vs 1Sb 1:70 Бог возлюбил их, безсмертный Спаситель и Царь. Но однако

\vs 1Sb 1:71 Стали и эти в безумье грешить, безстыдно принявшись

\vs 1Sb 1:72 На смех отцов выставлять и над матерями глумиться.

\vs 1Sb 1:73 С близкими начали тут обращаться они, как с чужими,

\vs 1Sb 1:74 Брат поднял руку на брата. Пресытились кровью убитых,

\vs 1Sb 1:75 Ею себя запятнав, вели безразсудные войны.

\vs 1Sb 1:76 Пала за это на них с небес наивысшая кара:

\vs 1Sb 1:77 Люди из жизни теснимы быть начали. Всех их, преступных,

\vs 1Sb 1:78 Принял Аид. Называют его Аидом с тех пор, как

\vs 1Sb 1:79 Первым в нем очутился Адам, чашу смерти пригубив.

\vs 1Sb 1:80 Всюду его обступила земля. И это причиной

\vs 1Sb 1:81 Стало того, что о тех, кто живет на земле, говорится:

\vs 1Sb 1:82 В царство Аида уходят. Однако, и сгинув в Аиде,

\vs 1Sb 1:83 Первые люди почет заслужили, что первым дается.

\vs 1Sb 1:84 Сразу же после того, как их земля поглотила,

\vs 1Sb 1:85 Род сотворил Он другой  из тех, кто еще оставался

\vs 1Sb 1:86 Праведной жизни. Они трудились усердно, прекрасны

\vs 1Sb 1:87 Были дела их, стыдом превзошли остальных и имели

\vs 1Sb 1:88 Разум надежный. Искусства им были знакомы. Искали

\vs 1Sb 1:89 Выход в любом затрудненье и быстро его находили.

\vs 1Sb 1:90 Способ один изобрел, как надо вспахивать плугом

\vs 1Sb 1:91 Землю. Другой размышлял над тем, как строить прочнее,

\vs 1Sb 1:92 Третий  как по морю плавать, по птицам гадать, и на небе

\vs 1Sb 1:93 Звезды четвертый умел наблюдать. Про яды знал пятый.

\vs 1Sb 1:94 Магия делом была еще одного. Все ремесла

\vs 1Sb 1:95 Разным поручены были умельцам. Безсонными звали

\vs 1Sb 1:96 Хлебоедами их, поскольку они отличались

\vs 1Sb 1:97 Вечно ясным умом и ненаполнимым желудком.

\vs 1Sb 1:98 Телом могучие, все отошли, однако, под своды

\vs 1Sb 1:99 Страшного царства Аида. Там, скованны прочно цепями,

\vs 1Sb 1:100 Грех свой должны искупать, пребывая в геенне, где пламя

\vs 1Sb 1:101 Неугасимое жжет и огонь жестокий пылает.

\vs 1Sb 1:102 Вслед за ушедшими племя явилось, мощное духом.

\vs 1Sb 1:103 Третьим было по счету оно. Надменных и дерзких

\vs 1Sb 1:104 Объединяло людей, которые многие беды

\vs 1Sb 1:105 В мир принесли. Очень скоро сражения, войны, убийства

\vs 1Sb 1:106 Их истребили, носивших в груди жестокое сердце.

\vs 1Sb 1:107 Та же причина была, что еще один род прекратился.

\vs 1Sb 1:108 Младшее из четырех людских поколений, и это

\vs 1Sb 1:109 Кровью себя осквернило, повсюду ее проливая.

\vs 1Sb 1:110 Был им страх перед Богом неведом, как друг перед другом 

\vs 1Sb 1:111 Чувство стыда. Наконец, против них же самих обратились

\vs 1Sb 1:112 Гнев, сводящий с ума, совместно с буйным нечестьем.

\vs 1Sb 1:113 Так повергли несчастных убийства, сражения, войны

\vs 1Sb 1:114 В мрак преисподней  мужей, преступивших закон. Их небесный,

\vs 1Sb 1:115 Гневаясь, Бог перенес потом за пределы вселенной,

\vs 1Sb 1:116 Тартаром отгородив под самой земли сердцевиной.

\vs 1Sb 1:117 Был за этим еще один род человеческий создан,

\vs 1Sb 1:118 Много хуже других. Ему злую участь безсмертный

\vs 1Sb 1:119 Бог уготовил, когда творить беззаконие стали.

\vs 1Sb 1:120 Нравом надменнее были они, чем прежние люди, 

\vs 1Sb 1:121 Племя Гигантов, в речах нечестиво хулившее Бога.

\vs 1Sb 1:122 Только один среди всех человек был правдивый и верный 

\vs 1Sb 1:123 Ной, закон почитавший и думавший лишь о хорошем.

\vs 1Sb 1:124 Вот с такими словами с небес к нему Бог обратился:

\vs 1Sb 1:125 Мужество, Ной, собери, тотчас призови к покаянью

\vs 1Sb 1:126 Всех людей на земле, чтоб свои они жизни спасали.

\vs 1Sb 1:127 Дела безстыжим коль нет до того, что повсюду творится,

\vs 1Sb 1:128 Род Я их весь погублю невиданным прежде потопом.

\vs 1Sb 1:129 Ты же на прочной основе, воде не дающей прохода,

\vs 1Sb 1:130 Дом себе быстро построй деревянный, надежно стоящий.

\vs 1Sb 1:131 Знание дам Я тебе для того и умение строить,

\vs 1Sb 1:132 Дам укромное место, размер  обо всем позабочусь,

\vs 1Sb 1:133 Так что спасешься ты сам и все, кто живут с тобой вместе.

\vs 1Sb 1:134 Я же есть Сущий, и ты в своем сердце обдумай такое:

\vs 1Sb 1:135 Небо навлек на Себя, вокруг Себя море раскинул,

\vs 1Sb 1:136 Мне опора для ног  земля, вкруг тела разлился

\vs 1Sb 1:137 Воздух, и звезд хоровод Меня кругом обегает.

\vs 1Sb 1:138 Девять имею Я букв, Меня составляют четыре

\vs 1Sb 1:139 Слога, кто Я  ты пойми: три первых слога содержат

\vs 1Sb 1:140 Каждый две буквы, последний же слог  остальные. Согласных

\vs 1Sb 1:141 Пять. Всего же числа  девятнадцать сотен, десятков

\vs 1Sb 1:142 Три и вдобавок семерка. Узнай, кто Я есть, и ты станешь

\vs 1Sb 1:143 Мудрости высшей Моей чуждым уже не совсем.

\vs 1Sb 1:144 Так сказал. И того, кто все это слышал, великий

\vs 1Sb 1:145 Страх охватил. В уме остальное предвидя, он начал

\vs 1Sb 1:146 Тут людей умолять и такие слова говорил им:

\vs 1Sb 1:147 Веры в вас нет, безумья гонимые жалом! Не спустит

\vs 1Sb 1:148 Бог ничего из того, что вы сделали. Знает Безсмертный

\vs 1Sb 1:149 Все, Спаситель всезрящий, и вам об этом поведать

\vs 1Sb 1:150 Он направил меня, чтоб вы души свои не сгубили.

\vs 1Sb 1:151 Трезво на мир посмотрите, от зла отрекитесь и войны

\vs 1Sb 1:152 Между собой перестаньте вести в исступленье жестоком,

\vs 1Sb 1:153 Щедро землю кругом человеческой кровью питая.

\vs 1Sb 1:154 Люди, побойтесь Того, Кто Сам нерушим и огромен,

\vs 1Sb 1:155 На небе сущего Бога, создавшего все во вселенной.

\vs 1Sb 1:156 Все к Нему обратитесь с мольбами  Он милосердный! 

\vs 1Sb 1:157 Жизнь сохранить городов и всего великого мира,

\vs 1Sb 1:158 Четвероногих и птиц  пусть милостив будет ко всем Он.

\vs 1Sb 1:159 Время наступит, когда безкрайний, людьми населенный

\vs 1Sb 1:160 Мир, от вод погибая, провоет жуткую песню.

\vs 1Sb 1:161 Время наступит, и воздух над вами вдруг всколыхнется,

\vs 1Sb 1:162 Бога великого гнев устремится с неба на землю.

\vs 1Sb 1:163 Истинно время придет, когда на людей опрокинет

\vs 1Sb 1:165 Вечно живущий Спаситель, снискать если вам не удастся

\vs 1Sb 1:166 Милость Его и отныне совсем жить иначе, чем прежде,

\vs 1Sb 1:167 Так, чтоб ни зла, ни обид друг другу преступно не строя,

\vs 1Sb 1:168 Каждый праведной жизнью прикрыт был от Божьего гнева.

\vs 1Sb 1:169 Слыша такие слова, его на смех все поднимали,

\vs 1Sb 1:170 Звали несчастным безумцем, которого разум покинул.

\vs 1Sb 1:171 Ной же к ним вновь и опять обращался с докучливой речью:

\vs 1Sb 1:172 Жалость внушаете вы, постоянства лишенные, сердцем

\vs 1Sb 1:173 Злобные, стыд кто отринул, кого влечет лишь безстыдство,

\vs 1Sb 1:174 Жадные мира владыки, насильники и нечестивцы,

\vs 1Sb 1:175 Те, что неверья полны, злодеи, лжецы, кто ни слова

\vs 1Sb 1:176 Правды вовек не сказал, богохульники, прелюбодеи,

\vs 1Sb 1:177 Бога Всевышнего гнев кому не страшен,  расплата

\vs 1Sb 1:178 Всех вас теперь ожидает до родичей в пятом колене.

\vs 1Sb 1:179 С криком не мечетесь вы, жестокие, только смеетесь:

\vs 1Sb 1:180 Будет язвительный смех на губах, когда вдруг наступит

\vs 1Sb 1:181 То, о чем говорю: невиданный прежде, ужасный

\vs 1Sb 1:182 Хлынет на землю потоп, самим низпосланный Богом.

\vs 1Sb 1:183 Новый род на земле, священный, тут создан водою

\vs 1Sb 1:185 Будет  продолжится он, на корне сухом произросший.

\vs 1Sb 1:186 Сам собою поток в одну ночь исчезнет. Тогда же

\vs 1Sb 1:187 Вместе с людьми города разметает земли Колебатель,

\vs 1Sb 1:188 Их в укромных ущельях достав, и стены разрушит.

\vs 1Sb 1:189 Так погибнет весь мир, и люди исчезнут без счета,

\vs 1Sb 1:190 Те, что его населяют. А мне еще сколько придется

\vs 1Sb 1:191 Горя изведать и скольких еще погибших оплакать

\vs 1Sb 1:192 В доме своем деревянном? С волнами сколько смешаю

\vs 1Sb 1:193 Слез? Ведь только нахлынут по слову Божьему воды,

\vs 1Sb 1:194 Все поплывет  и земля, и горы, и небо над ними.

\vs 1Sb 1:195 Мир весь станет водой и водами будет погублен.

\vs 1Sb 1:195 Ветры дуть прекратят, наступит другая эпоха.

\vs 1Sb 1:196 Фригия! первою ты из воды приподнимешь вершину,

\vs 1Sb 1:197 Первая будешь кормить ты новое племя людское,

\vs 1Sb 1:199 Кончил когда он впустую слова расточать нечестивцам,

\vs 1Sb 1:200 Сам Всевышний явился, и вновь прозвучал Его голос:

\vs 1Sb 1:201 Время настало, о Ной, объявить обо всем по порядку,

\vs 1Sb 1:202 Что Я в тот день обещал тебе привести в исполненье:

\vs 1Sb 1:203 Неисчислимое зло, которое люди свершили,

\vs 1Sb 1:204 Миру без края вернуть за непослушание смертных.

\vs 1Sb 1:205 Ты же прийти поспеши с женой своей и с сыновьями,

\vs 1Sb 1:206 Также их жен позови и тех, кому повелел Я

\vs 1Sb 1:207 Волю Мою объявить: животных, змей и пернатых.

\vs 1Sb 1:208 Этим Сам зароню Я в сердце желанье явиться 

\vs 1Sb 1:209 Всем, кому Я предназначил продолжить дни свои дальше.

\vs 1Sb 1:210 Так было сказано. Ной пошел и громко об этом

\vs 1Sb 1:211 Им возвестил. Тогда жена, сыновья и невестки

\vs 1Sb 1:212 В дом деревянный взошли, и сразу за ними туда же

\vs 1Sb 1:213 Прочие твари, кому Господь повелел это сделать.

\vs 1Sb 1:214 Тут же засов закрепили, надежно дверь закрывавший.

\vs 1Sb 1:215 Косо он приходился в борту, что был гладко оструган.

\vs 1Sb 1:216 Воля небесного Бога тем самым вполне совершилась.

\vs 1Sb 1:217 Тучи собрал Он и скрыл сверкавший ярко диск солнца,

\vs 1Sb 1:218 Звезды вместе с луной и корону, венчавшую небо.

\vs 1Sb 1:219 Тьмою тут все окружив, загремел, людей повергая

\vs 1Sb 1:220 В ужас, наслал ураган  и ветры разом проснулись,

\vs 1Sb 1:221 Вздулись водные жилы и русла покинули, с неба

\vs 1Sb 1:222 Хлынули, вдруг открывшись, огромные водопады,

\vs 1Sb 1:223 Массы воды из трещин, глубоких провалов внезапно

\vs 1Sb 1:224 Вышли на свет, и под ними земля вся безкрайняя скрылась.

\vs 1Sb 1:225 Плавал тогда под дождем ковчег, что по слову был создан

\vs 1Sb 1:226 Бога: удары терпя от волн, подчиняясь порывам

\vs 1Sb 1:227 Ветра, вдруг поднимался он вверх, и множество пены

\vs 1Sb 1:228 Киль разсекал под журчанье воды, что двигалась всюду.

\vs 1Sb 1:229 Тут, когда весь уже мир затопил дождями Всевышний,

\vs 1Sb 1:230 В голову Ною пришло посмотреть, как исполнилась воля

\vs 1Sb 1:231 Господня, и заглянуть самому в морскую пучину.

\vs 1Sb 1:232 Быстро он дверь распахнул в борту, что был гладко оструган,

\vs 1Sb 1:233 Плотно створки которой одна к другой прилегали.

\vs 1Sb 1:234 Только ее он открыл  представилось взору пространство,

\vs 1Sb 1:235 Сплошь покрыто водой, везде, без конца и без края.

\vs 1Sb 1:236 Страх тут и трепет его охватили. В это мгновенье

\vs 1Sb 1:237 Стал редеть понемногу туман, в течение многих

\vs 1Sb 1:238 Дней уставший окутывать мир. Он бледно-кровавый

\vs 1Sb 1:239 Неба вечернего свод показал и усталого солнца

\vs 1Sb 1:240 Огненный диск. Насилу вернулось мужество к Ною.

\vs 1Sb 1:241 Вдаль направив полет, он сизую выпустил птицу,

\vs 1Sb 1:242 Чтобы узнала она, вдруг где-то еще сохранилась

\vs 1Sb 1:243 Твердая почва. Устав бить крыльями воздух, вернулась

\vs 1Sb 1:244 Птица, кругом облетев: вода нигде не спадала,

\vs 1Sb 1:245 Все было ею полно. Через несколько дней он отправил

\vs 1Sb 1:246 Снова голубку узнать, отступили ли воды. Она же,

\vs 1Sb 1:247 Легкая, в дальний опять отправилась путь и достигла

\vs 1Sb 1:248 Влажной земли. Проведя там какое-то время, обратно

\vs 1Sb 1:249 К Ною вернулась, неся засохшую ветку оливы 

\vs 1Sb 1:250 Знак удачи посольства. В сердцах пробудилась отвага,

\vs 1Sb 1:251 Землю увидеть надежда вселила великую радость.

\vs 1Sb 1:252 Сразу же после того еще чернокрылую птицу

\vs 1Sb 1:253 Ной поспешил отпустить. Она, доверившись крыльям,

\vs 1Sb 1:254 Вдаль устремилась охотно  достигнув земли, там осталась.

\vs 1Sb 1:255 Стало тогда очевидно, что ближе придвинулась суша.

\vs 1Sb 1:256 Скоро, плывя среди волн наугад но шумящему понту,

\vs 1Sb 1:257 Горы встречая воды повсюду, нетленное судно

\vs 1Sb 1:258 Дном увязнув, на узкой полоске земли утвердилось.

\vs 1Sb 1:259 Есть во Фригии черной, что без конца и без края,

\vs 1Sb 1:260 Горный обрывистый кряж, называется он Араратом:

\vs 1Sb 1:261 Здесь предстояло спастись всем тем, кто был с Ноем,  и жажду

\vs 1Sb 1:262 В душу вложил им Господь, едва в это место попали,

\vs 1Sb 1:263 Было тут много ключей, от которых питается Марсий.

\vs 1Sb 1:264 После, как спала вода, ковчег на высокой вершине

\vs 1Sb 1:265 Так и остался лежать, и вновь прозвучал тогда с неба

\vs 1Sb 1:266 Голос Великого Бога нетленный. Он слово такое

\vs 1Sb 1:267 Молвил: Ной, избранник судьбы, справедливый и верный!

\vs 1Sb 1:268 Смело покинь свой ковчег с сыновьями вместе, с женою,

\vs 1Sb 1:269 Три пусть выходят невестки: собой наполнить отныне

\vs 1Sb 1:270 Землю должны вы, плодясь и множа свой род, по закону

\vs 1Sb 1:271 Каждому часть уделив, из колена в колено, доколе

\vs 1Sb 1:272 Время суда не придет, который вас всех ожидает.

\vs 1Sb 1:273 Так произнес вечный голос, и Ной, осмелев, из ковчега

\vs 1Sb 1:274 Спрыгнул на землю, а следом  жена, сыновья и невестки,

\vs 1Sb 1:275 Племя пернатых, ползучие гады, и четвероногих

\vs 1Sb 1:276 Разные виды. Все вместе оставили дом деревянный,

\vs 1Sb 1:277 Вместе на землю сошли, и стала она общим домом.

\vs 1Sb 1:278 Ной тогда, всех людей превзошедший праведной жизнью,

\vs 1Sb 1:279 После Адама восьмой, спустился на твердую землю,

\vs 1Sb 1:280 Сорок дней и один проплавав по воле Господней.

\vs 1Sb 1:281 Так поднялся тогда новый род и жизнь свою начал,

\vs 1Sb 1:282 Первый и золотой, шестым был он и наилучшим

\vs 1Sb 1:283 С тех самых пор, как Господь впервые создал человека.

\vs 1Sb 1:284 Буду его называть я небесным, поскольку заботу

\vs 1Sb 1:285 Бог возложил на Себя обо всем, в чем нужда возникала.

\vs 1Sb 1:286 О поколение первое рода шестого! О радость,

\vs 1Sb 1:287 Что ты доставило мне, когда неминуемой смерти

\vs 1Sb 1:288 Я избежала, устав на волнах качаться и страха

\vs 1Sb 1:289 Много перетерпев вместе с мужем и деверьями,

\vs 1Sb 1:290 С женами их, со свекровью и свекром! Достойную славу

\vs 1Sb 1:291 Я тебе пропою: цветок на смоковнице будет

\vs 1Sb 1:292 Пестрый, до середины дойдут века и положат

\vs 1Sb 1:293 Царской власти начало, что носит скипетр, и трое

\vs 1Sb 1:294 Духом могучих царей, справедливейших, земли поделят.

\vs 1Sb 1:295 Многие годы продлится их власть. Делить по закону

\vs 1Sb 1:296 Между людьми они станут заботы и радость. Земля же

\vs 1Sb 1:297 Будет гордиться плодами, что сами собой вызревают,

\vs 1Sb 1:298 Вся расцветет и зерном осыпет счастливое племя.

\vs 1Sb 1:299 Старость с годами к отцам не придет, не зная болезней,

\vs 1Sb 1:300 Смерть сразу многим несущих, и даже озноба, как будто

\vs 1Sb 1:301 В сон погружаясь, умрут, отойдут к берегам Ахеронта,

\vs 1Sb 1:302 В царство Аида, где им будут возданы почести. Ибо

\vs 1Sb 1:303 Род их был родом блаженных и те изведали счастья,

\vs 1Sb 1:304 В головы чьи заложил Саваоф глубокую мудрость 

\vs 1Sb 1:305 С ними всегда обсуждал Он Свою безсмертную волю.

\vs 1Sb 1:306 Но даже этих счастливцев Аид впереди ожидает.

\vs 1Sb 1:307 После на смену придет тяжелое, крепкое племя

\vs 1Sb 1:308 Земнородных людей и будет по счету второе.

\vs 1Sb 1:309 Имя тем людям Титаны, один на другого похожи,

\vs 1Sb 1:310 Каждый ростом, лицом остальных напомнит. Осанка,

\vs 1Sb 1:311 Голос будет один, какой был Богом заложен

\vs 1Sb 1:312 Некогда предкам их в грудь. Однако и эти, имея

\vs 1Sb 1:313 Дерзкий нрав, замахнутся на то, что им не по силам;

\vs 1Sb 1:314 Смерть приближая свою, захотят сразиться со звездным

\vs 1Sb 1:315 Небом. За это на них океана великого воды

\vs 1Sb 1:316 Хлынут бурным потоком  и сам Саваоф, разсердившись,

\vs 1Sb 1:317 Их удерживать будет, мешая тому, чтобы снова

\vs 1Sb 1:318 Из-за злонравия смертных весь мир под водой оказался.

\vs 1Sb 1:319 Но когда Он заставит всех вод безпредельных волненье

\vs 1Sb 1:320 Гнев усмирить свой, сшибая валы и лишая их силы,

\vs 1Sb 1:321 На неглубоких местах, напротив, волну уменьшая

\vs 1Sb 1:322 Тем, что море землей окружит и о берег неровный

\vs 1Sb 1:323 Биться принудит его великий Бог-громовержец

\vs 1Sb 1:324 Сын Его к людям придет, уподобившись обликом смертным,

\vs 1Sb 1:325 В плоть облечен, как и все на земле. Он гласных четыре

\vs 1Sb 1:326 Будет иметь и двойной согласный. Тебе назову я

\vs 1Sb 1:327 Все число целиком: единиц в нем содержится восемь,

\vs 1Sb 1:328 Столько же, сколько десятков; вдобавок к этому сотен

\vs 1Sb 1:329 Тоже восемь предъявит неверящим людям то имя.

\vs 1Sb 1:330 Должен умом ты постичь, что Сын Безсмертного Бога,

\vs 1Sb 1:331 Выше Которого нет,  Христос, Помазанник Божий.

\vs 1Sb 1:332 Он исполнит закон Отца своего, не разрушит;

\vs 1Sb 1:333 Образ Его воплотив, передаст в полноте и ученье.

\vs 1Sb 1:334 Золото в дар принесут волхвы ему, ладан и смирну,

\vs 1Sb 1:335 Ибо он все совершит, что рожденье его предвещало.

\vs 1Sb 1:336 Голос тогда донесется неслыханный через пустыню,

\vs 1Sb 1:337 Чтобы людей известить, и всем повелит приготовить

\vs 1Sb 1:338 Тропы прямые, изгнать пороки с корнем из сердца.

\vs 1Sb 1:339 Также водою омыть велит он каждому тело,

\vs 1Sb 1:340 Свет чтоб оно обрело и чтобы, рожденные свыше,

\vs 1Sb 1:341 Люди больше нигде с благого пути не свернули.

\vs 1Sb 1:342 Этот Божественный голос опутанный пляскою варвар

\vs 1Sb 1:343 Разом отделит от тела, за что понесет наказанье.

\vs 1Sb 1:344 Будет тут знаменье смертным, когда из Египта нежданно

\vs 1Sb 1:345 Камень придет драгоценный, хранимый Богом. Споткнется

\vs 1Sb 1:346 Племя Евреев на Нем, Другие народы, напротив,

\vs 1Sb 1:347 Вместе Его руководству доверятся, ибо познают

\vs 1Sb 1:348 Бога Всевышнего так и дорогу увидят при свете,

\vs 1Sb 1:349 Что возсияет для всех. Ведь вечную жизнь Он укажет

\vs 1Sb 1:350 Избранным и принесет огонь на века нечестивым.

\vs 1Sb 1:351 Станет тогда же лечить больных Он и немощных телом 

\vs 1Sb 1:352 Всех, кто поверил в Него и свои возложил упованья.

\vs 1Sb 1:353 Видеть слепые начнут, хромые пойдут без поддержки,

\vs 1Sb 1:354 Те, кто не слышал, услышат, и вновь залепечут немые.

\vs 1Sb 1:355 Демонов выгонит Он, возстанут из гроба, кто умер.

\vs 1Sb 1:356 Будет ходить по волнам, пять тысяч в пустыне накормит

\vs 1Sb 1:357 Он от пяти хлебов и единой рыбы. Двенадцать

\vs 1Sb 1:358 Трапезы этой остатки корзин собою наполнят.

\vs 1Sb 1:360 Пьяный Израиль тогда ни во что не сможет проникнуть,

\vs 1Sb 1:361 На ухо туг, он никак не ответит, от хмеля тяжелый.

\vs 1Sb 1:362 Но когда на Евреев Всевышний гнев свой обрушит

\vs 1Sb 1:363 Меткоразящий и веру у их народа отнимет,

\vs 1Sb 1:364 Из-за того, что они распяли Божьего Сына,

\vs 1Sb 1:365 Будет Израиль плевать в Него из уст нечестивых

\vs 1Sb 1:366 Яда полной слюной и бить по щекам Его станет.

\vs 1Sb 1:367 Желчь Ему вместо еды и уксус вместо напитка

\vs 1Sb 1:368 Тут нечестиво дадут, побуждаемы тяжким безумьем,

\vs 1Sb 1:369 Ум поразившим и сердце, глазами смотря и не видя 

\vs 1Sb 1:370 Слепы хуже кротов, ужаснее змей ядовитых,

\vs 1Sb 1:371 Ползают что по земле, опутаны сонным дурманом.

\vs 1Sb 1:372 Он же как руки раскинет и все до конца перетерпит,

\vs 1Sb 1:373 На голове понесет венец терновый, и в ребра

\vs 1Sb 1:374 Ткнут Ему острый тростник  среди белого дня воцарится

\vs 1Sb 1:375 Ночь тогда на три часа и тьмою кругом все покроет.

\vs 1Sb 1:376 Знак тут храм Соломонов народам подаст величайший,

\vs 1Sb 1:377 В домы Аида когда отправится Он, возвещая

\vs 1Sb 1:378 Тем, кто умер, что день придет  и из гроба возстанут.

\vs 1Sb 1:379 Через три дня же обратно на свет из Аида вернется,

\vs 1Sb 1:380 Смертным дабы явить Свой образ и научить их.

\vs 1Sb 1:381 После по облакам пройдет Он к жилищу на небе,

\vs 1Sb 1:382 Миру вместо Себя завет Благовестья оставив.

\vs 1Sb 1:383 Здесь во имя Его росток появится новый

\vs 1Sb 1:384 Из народов, что чтут Закон великого Бога.

\vs 1Sb 1:385 Будут тогда на земле мудрецы, что дорогу покажут,

\vs 1Sb 1:386 Всяким пророкам конец после этого в мире настанет.

\vs 1Sb 1:387 С той поры, как Евреи пожнут недобрую жатву,

\vs 1Sb 1:388 Много у них серебра и золота много отнимет

\vs 1Sb 1:389 Римский кесарь. А после другие царства сменяться

\vs 1Sb 1:390 Станут одно за другим со смертью владык и обиды

\vs 1Sb 1:391 Людям чинить. Тогда великие беды придется

\vs 1Sb 1:392 Вынести смертным за то, что гордыми будут не в меру.

\vs 1Sb 1:393 Храм же когда Соломонов в Священной Земле под ударом

\vs 1Sb 1:394 Варварских полчищ падет, одетых в доспехи из меди,

\vs 1Sb 1:395 Изгнаны будут Евреи с Земли, и по миру скитаться

\vs 1Sb 1:396 Им предстоит, претерпев разорение полное, плевел

\vs 1Sb 1:397 В хлеб добавлять. Незавидный удел их всех ожидает.

\vs 1Sb 1:398 Что же до городов, то одних оплачут другие,

\vs 1Sb 1:399 Сами изведав позор  ведь некогда все согрешили,

\vs 1Sb 1:400 Гнев Великого Бога за это приняв в наказанье.

\bibbookdescr{2Sb}{
  inline={Вторая книга Сивилл},
  toc={2-я Сивилл},
  bookmark={2-я Сивилл},
  header={2-я Сивилл},
  abbr={2~Сив}
}
\vs 2Sb 1:1 Только дал смолкнуть Господь, мольбам моим частым внимая,

\vs 2Sb 1:2 Мудрой песне, как снова вложил Он мне радостный голос 

\vs 2Sb 1:3 В сердце, дабы могла я реченное Богом поведать. 

\vs 2Sb 1:4 Телом всем содрогаясь, начну говорить  ведь не знаю, 

\vs 2Sb 1:5 Что говорю, но от Бога исходят мои прорицанья.

\vs 2Sb 1:6 Время настанет, и в мир придут сотрясенья земные, 

\vs 2Sb 1:7 Молнии жгучие, громы и ржавый налет на растеньях, 

\vs 2Sb 1:8 Бешенство быстрых волков, убийства кровавые, гибель 

\vs 2Sb 1:9 Жизней людских, за людьми же быки мычащие сгинут, 

\vs 2Sb 1:10 Множество коз и овец, ослов терпеливых и прочий 

\vs 2Sb 1:11 Четвероногий скот, а пашни обширные будут 

\vs 2Sb 1:12 Брошены и в запустенье придут, а плоды не родятся, 

\vs 2Sb 1:13 Всюду в обычай войдет продажа и купля свободных, 

\vs 2Sb 1:14 Словно рабов, и везде разграбят священные храмы.

\vs 2Sb 1:15 Явится после того поколенье десятое смертных, 

\vs 2Sb 1:16 И вот тогда сокрушит Колебатель и Молниевержец 

\vs 2Sb 1:17 Идолов, бывших в почете, и мощь семихолмного Рима 

\vs 2Sb 1:18 Он потрясет, уничтожив богатства несметные разом: 

\vs 2Sb 1:19 Мощный пожрет их огонь, великое пламя Гефеста.

\vs 2Sb 1:20 Наземь с высоких небес поток польется кровавый

\vs 2Sb 1:21 Люди в ту пору начнут по всему безконечному миру 

\vs 2Sb 1:22 Гибель друг другу нести, и к этой смуте ужасной 

\vs 2Sb 1:23 Бог пошлет им еще чуму, перуны и голод, 

\vs 2Sb 1:24 Так за неправедный суд карая людей нечестивых. 

\vs 2Sb 1:25 В мире число людей тогда сократится настолько,

\vs 2Sb 1:26 Что если кто-то увидит ноги только след человечьей, 

\vs 2Sb 1:27 То подивится немало. Но Бог, в эфире живущий, 

\vs 2Sb 1:28 Всем справедливым мужам опять избавителем станет. 

\vs 2Sb 1:29 И на земле воцарятся надежный мир и согласье, 

\vs 2Sb 1:30 Вновь будет почва рождать и плод принесет изобильный, 

\vs 2Sb 1:31 Ибо делить перестанут и мучить ее как рабыню. 

\vs 2Sb 1:32 Всякая пристань и порт откроются людям свободно, 

\vs 2Sb 1:33 Как это было и прежде, безстыдство же вовсе исчезнет.

\vs 2Sb 1:34 После на небе Господь великое знаменье явит:

\vs 2Sb 1:35 Люди созвездие узрят  венку оно будет подобно, 

\vs 2Sb 1:36 Ярким сияньем своим небеса озарит и надолго 

\vs 2Sb 1:37 Так сохранится. И люди поймут, что грядет состязанье, 

\vs 2Sb 1:38 И за вот этот венок зовет их бороться Безсмертный. 

\vs 2Sb 1:39 Ибо наступит затем триумфа великого время

\vs 2Sb 1:40 В граде небесном: сюда весь мир сойдется обширный, 

\vs 2Sb 1:41 Стать можно каждому будет причастным славе нетленной. 

\vs 2Sb 1:42 Все народы тогда в безсмертных ристаньях к победе, 

\vs 2Sb 1:43 Коей прекраснее нет, устремятся; и грешник не сможет 

\vs 2Sb 1:44 Там победный венок купить за деньги безстыдно.

\vs 2Sb 1:45 И справедливо раздаст награжденья Спаситель блаженный. 

\vs 2Sb 1:46 Верных Он увенчает, а тем, кто мучения принял, 

\vs 2Sb 1:47 Смертью окончив борьбу, безсмертная будет награда. 

\vs 2Sb 1:48 Тем, кто девство храня, к победе нетленной стремился, 

\vs 2Sb 1:49 Он по заслугам воздаст, и тем, кто берег справедливость,

\vs 2Sb 1:50 И никого не забудет Он даже из дальних народов,

\vs 2Sb 1:51 Если праведно жили и знали единого Бога.

\vs 2Sb 1:52 Те, кто брак почитал, позорный блуд отвергая,

\vs 2Sb 1:53 Дар получат богатый, вовек не умрет в них надежда. 

\vs 2Sb 1:54 Ибо любая душа человечья  даяние Бога.

\vs 2Sb 1:55 Смертный не вправе пятнать ее никакими грехами.

\vs 2Sb 1:56 Следует честным трудом пропитанье стяжать, не пытаясь,

\vs 2Sb 1:57 Делая зло, богатеть, не нужно трогать чужого, 

\vs 2Sb 1:58 Хватит тебе своего; не лги, будь истине верен, 

\vs 2Sb 1:59 Идолам не поклоняйся, но Вечносущего Бога 

\vs 2Sb 1:60 В первую очередь чти, уважай и родителей также. 

\vs 2Sb 1:61 Праведен будь, чтобы суд над тобою неправедным не был. 

\vs 2Sb 1:62 Не обижай бедняка, чуждайся лицеприятья, 

\vs 2Sb 1:63 Коль будешь плохо судить, то Бог тебя же осудит.

\vs 2Sb 1:64 Ложных свидетельств беги, говори только чистую правду; 

\vs 2Sb 1:65 Чист оставайся и сам, подходи ко всем людям с любовью;

\vs 2Sb 1:66 Верную меру блюди и лучше дай больше, чем меньше.

\vs 2Sb 1:67 Ровными чаши весов должны быть, не наклоняй их.

\vs 2Sb 1:68 В клятвах своих не лги  случайно, иль с умыслом вредным 

\vs 2Sb 1:69 Страшен Бог для того, кто хоть в чем-либо клятву нарушил. 

\vs 2Sb 1:70 Дара не принимай, если он добыт преступленьем.

\vs 2Sb 1:71 И семена не кради: кто отнимет их, будет навеки

\vs 2Sb 1:72 Проклят, ибо украл он то, что дало бы пищу.

\vs 2Sb 1:73 Пусть клеветы и разврата ты будешь чужд, и убийства;

\vs 2Sb 1:74 Бедного не обижай, плати за работу исправно. 

\vs 2Sb 1:75 Речи разумно веди, а тайны храни в своем сердце.

\vs 2Sb 1:76 Помощь вдовам подай, сиротам и всем, кто несчастен.

\vs 2Sb 1:77 Сам не твори беззаконий и злу не позволь совершиться;

\vs 2Sb 1:78 Нищему сразу давай, не откладывай это на завтра;

\vs 2Sb 1:79 Щедрой рукой удели неимущему часть урожая  

\vs 2Sb 1:80 Кто помогает другому, ссужает даримое Богу.

\vs 2Sb 1:81 Милость во дни Суда от смерти даст избавленье,

\vs 2Sb 1:82 Милости хочет Господь от людей, а вовсе не жертвы.

\vs 2Sb 1:83 Дай одежду нагому, тому, кто голоден, хлеба,

\vs 2Sb 1:84 В дом свой бездомных прими, слепых проводи на дорогу. 

\vs 2Sb 1:85 Тех пожалей, с кем беда в коварном море случилась.

\vs 2Sb 1:86 Падает кто  поддержи, спаси, коль нависла угроза;

\vs 2Sb 1:87 Все страдают, а жизнь  колесо, и счастье неверно.

\vs 2Sb 1:88 Если богат, протяни несчастному помощи руку

\vs 2Sb 1:89 И удели из того, что сам получил ты от Бога. 

\vs 2Sb 1:90 Жизни схожи людские, и только жребий неравен.

\vs 2Sb 1:91 Над бедняком никогда не должен ты насмехаться.

\vs 2Sb 1:92 Злобно ты не ругай и того, кто упрека достоин.

\vs 2Sb 1:93 Ясным делает смерть, как жизнь прожита человеком 

\vs 2Sb 1:94 Был справедлив он, иль нет, на Суде великом решится. 

\vs 2Sb 1:95 Пей умеренно, разум вином повреждаться не должен;

\vs 2Sb 1:96 Крови не ешь и того, что идолам в жертву приносят.

\vs 2Sb 1:97 Меч можешь взять для защиты  мечом против друга не действуй,

\vs 2Sb 1:98 Лучше, впрочем, совсем никогда не брать его в руки:

\vs 2Sb 1:99 Если убьешь и врага, рука все ж запятнана будет. 

\vs 2Sb 1:100 Землю соседа не тронь, не ступай на нее ни ногою:

\vs 2Sb 1:101 Должно блюсти рубежи, неправедно их нарушенье.

\vs 2Sb 1:102 Польза и приобретенье, коль честно, и вред, коль нечестно. 

\vs 2Sb 1:103 Злак, на поле растущий, не смей губить никогда ты, 

\vs 2Sb 1:104 Пусть уваженье пришельцам не меньше, чем гражданам, будет.

\vs 2Sb 1:105 Люди за тягостный труд почитают гостеприимство, 

\vs 2Sb 1:106 Словно все чужды они друг другу, но так не должно быть: 

\vs 2Sb 1:107 Ибо смертные все от крови одной происходят, 

\vs 2Sb 1:108 А на земле для людей не назначено мест постоянных. 

\vs 2Sb 1:109 В мыслях своих не стремись к богатству, желай одного лишь:

\vs 2Sb 1:110 Малым довольствуясь, жить, ничего не стяжав не по праву. 

\vs 2Sb 1:111 Алчность, пристрастье к деньгам все пороки ведут за собою. 

\vs 2Sb 1:112 К золоту и серебру опасно влеченье  сокрыто 

\vs 2Sb 1:113 В этих металлах железо, несущее верную гибель; 

\vs 2Sb 1:114 В золоте и серебре обман для смертных таится,

\vs 2Sb 1:115 Золото, зол предводитель, ты смертью всему угрожаешь, 

\vs 2Sb 1:116 Не становись никогда для людей несчастьем желанным, 

\vs 2Sb 1:117 Из-за тебя и война, и все грабежи, и убийства, 

\vs 2Sb 1:118 Ты причина вражды детей с отцами и братьев.

\vs 2Sb 1:119 Козней не замышляй, против друга не вооружайся; 

\vs 2Sb 1:120 Не говори одного, коли в сердце держишь иное;

\vs 2Sb 1:121 Если же место меняешь, то сам как полип не меняйся.

\vs 2Sb 1:122 Честен будь, говори только то, что чувствуешь сердцем.

\vs 2Sb 1:123 Грех добровольный  зло, но если по принужденью 

\vs 2Sb 1:124 Точно судить не могу: вина в человеческой воле. 

\vs 2Sb 1:125 Ты не гордись ни умом, ни силой своей, ни богатством:

\vs 2Sb 1:126 Мудрость только у Бога, и мощь, и полное счастье.

\vs 2Sb 1:127 Прошлые злые дела твой дух пускай не смущают,

\vs 2Sb 1:128 Ведь невозможно никак небывшим бывшее сделать.

\vs 2Sb 1:129 Силу не применяй опрометчиво, сдерживай чувства: 

\vs 2Sb 1:130 Часто нанесший удар совершает убийство невольно.

\vs 2Sb 1:131 Жить без страданий нельзя, но боль пусть не будит гордыню;

\vs 2Sb 1:132 И к изобилью во всем не нужно людям стремиться,

\vs 2Sb 1:133 Роскошь большая влечет любовь к наслажденьям чрезмерным,

\vs 2Sb 1:134 Тот, кто богат, легко впадает в безстыдную дерзость. 

\vs 2Sb 1:135 Гнев, закипевший в душе, губительным сделаться может,

\vs 2Sb 1:136 Легок гнев небольшой, но, выросши, станет безумьем.

\vs 2Sb 1:137 Рвение в добром похвально, но пагубна ревность дурная,

\vs 2Sb 1:138 Злая дерзость  позор, благому дерзанию  слава,

\vs 2Sb 1:139 Слава любви к добру  позор влеченью Киприды. 

\vs 2Sb 1:140 Мил согражданам муж, приветливый и дружелюбный.

\vs 2Sb 1:141 Мера важна в еде, питье, но также и в слове.

\vs 2Sb 1:142 Мера  лучше всего, и вред в ее нарушеньи.

\vs 2Sb 1:143 Бранных речей не веди, не завидуй, не будь вероломен,

\vs 2Sb 1:144 Мыслей дурных избегай и не смей обманывать злостно. 

\vs 2Sb 1:145 Благоразумью учись и от постыдных дел воздержанью.

\vs 2Sb 1:146 Нравам не следуй дурным, за зло воздавай справедливо;

\vs 2Sb 1:147 Пользу несут уговоры, а гнев только гнев порождает.

\vs 2Sb 1:148 Слишком быстро не верь, убедись сначала надежно.

\vs 2Sb 1:149 Вот каково состязанье, и вот какие награды! 

\vs 2Sb 1:150 Это к безсмертию путь и жизни вечной ворота  

\vs 2Sb 1:151 Бог небесный открыл их самым праведным людям, 

\vs 2Sb 1:152 Здесь одержавшим победу; они увенчаны будут 

\vs 2Sb 1:153 И сквозь безсмертья врата пройдут с великою славой.

\vs 2Sb 1:154 Но когда миру всему вдруг знаменье будет такое:

\vs 2Sb 1:155 Дети с седыми висками начнут на свет появляться, 

\vs 2Sb 1:156 Беды к людям придут, и мор, и голод, и войны,

\vs 2Sb 1:157 Всем изменит удача, и горькие слезы польются.

\vs 2Sb 1:158 О, скольким детям придется тут справить пир поминальный,

\vs 2Sb 1:159 Жалко оплакав своих матерей и отцов; в покрывала 

\vs 2Sb 1:160 Трупы их завернут и зароют в землю сырую,

\vs 2Sb 1:161 Сами в пыли и крови. Увы, несчастные люди

\vs 2Sb 1:162 Рода последнего в мире, злодеи ужасные, как же

\vs 2Sb 1:163 Не понимают, глупцы, что, если жены не станут

\vs 2Sb 1:164 Больше рождать детей, людское племя угаснет? 

\vs 2Sb 1:165 Время жатвы приспело, коль некие, словно пророки,

\vs 2Sb 1:166 Будут вещать по земле и много обмана измыслят.

\vs 2Sb 1:167 Тут придет Велиал и немало знамений явит.

\vs 2Sb 1:168 Избранных, праведных самых в то время великие беды

\vs 2Sb 1:169 Ждут и смятенье, они подвергнутся все ограбленьям 

\vs 2Sb 1:170 Также, как и Евреи,  грозит им гневом ужасным

\vs 2Sb 1:171 Некий с Востока народ, из колен десяти состоящий.

\vs 2Sb 1:172 Станут искать они тех, кто погиб от руки Ассирийца;

\vs 2Sb 1:173 Кровью Евреям близки, язычникам смерть уготовят.

\vs 2Sb 1:174 После же власть обретут они и над людом могучим

\vs 2Sb 1:175 Избранных верных Евреев, в рабов их всех обращая, 

\vs 2Sb 1:176 Так же, как было и в прошлом; и сила еще не покинет 

\vs 2Sb 1:177 Этих мужей. А Всевышний, Всевидящий, в небе Живущий

\vs 2Sb 1:178 Сон нашлет на людей, глаза их тьмой покрывая. 

\vs 2Sb 1:179 Счастливы Божии слуги, которые бодрствовать будут

\vs 2Sb 1:180 В час, как придет Господь, их сонными Он не застанет, 

\vs 2Sb 1:181 Ибо все время глаза у них открытыми были. 

\vs 2Sb 1:182 То на рассвете случится, иль вечером, или же в полдень, 

\vs 2Sb 1:183 Но обязательно Он грядет  пророчество верно. 

\vs 2Sb 1:184 Сном будут люди объяты  и тут все звезды на небе

\vs 2Sb 1:185 Вдруг среди дня засияют и с ними оба светила;

\vs 2Sb 1:186 Быстрое время свой круг пройдет  и все совершится. 

\vs 2Sb 1:187 И в колеснице небесной тогда сойдет Фесвитянин, 

\vs 2Sb 1:188 Чтобы явить на земле тройное знаменье скорой 

\vs 2Sb 1:189 Мира кончины, и всем должны быть ясны эти знаки.

\vs 2Sb 1:190 Горе женам, в те дни имеющим плод в своем чреве,

\vs 2Sb 1:191 Иль неразумных детей молоком кормящих, а также 

\vs 2Sb 1:192 Тем из людей, кто тогда окажется в море плывущим. 

\vs 2Sb 1:193 Горе тому человеку, что день этот страшный увидит: 

\vs 2Sb 1:194 Ночь безпросветная мир до края окутает мглою,

\vs 2Sb 1:195 Сразу и Север, и Юг, и Восход, и Заход затмевая. 

\vs 2Sb 1:196 Тут величайший поток огня и пламени хлынет 

\vs 2Sb 1:197 С неба на землю, и все, что есть на свете, погубит: 

\vs 2Sb 1:198 Сушу, и Океан огромный, и синее море, 

\vs 2Sb 1:199 Реки, озера, ручьи и даже безжалостный Тартар,

\vs 2Sb 1:200 Даже небесную ось. А горящие в небе светила 

\vs 2Sb 1:201 Все воедино сольются и полностью форму утратят. 

\vs 2Sb 1:202 Звезды тогда упадут с небосвода в пучину морскую, 

\vs 2Sb 1:203 Души людей, умирая, зубами тогда заскрежещут, 

\vs 2Sb 1:204 Пламя будет их жечь и ливень серы ужасный

\vs 2Sb 1:205 Вплоть до часа, когда всю землю пепел покроет. 

\vs 2Sb 1:206 Все элементы вселенной тогда одинокими станут  

\vs 2Sb 1:207 Воздух, свет, небеса, земля и моря, дни и ночи,  

\vs 2Sb 1:208 Стаи безчисленных птиц не будут в небе метаться,

\vs 2Sb 1:209 По морю не проплывут водяные животные больше,

\vs 2Sb 1:210 Судно груженое также волны уже не прорежет. 

\vs 2Sb 1:211 И не оставят волы борозды ни единой на пашнях, 

\vs 2Sb 1:212 Не зашумят уж деревья от ветра. [Но все воедино 

\vs 2Sb 1:213 Сплавит Господь, а потом разнимет для очищенья.]

\vs 2Sb 1:214 После же явятся в мир посланники вечные Бога:

\vs 2Sb 1:215 Он Михаила пошлет, Гавриила с ним, Уриила

\vs 2Sb 1:216 И Рафаила  известно им зло, совершенное всяким. 

\vs 2Sb 1:217 Выведут души людские на свет из тумана и мрака, 

\vs 2Sb 1:218 Чтобы судил их Господь, на троне сидящий небесном, 

\vs 2Sb 1:219 Ибо только лишь Он великий Владыка нетленный,

\vs 2Sb 1:220 Он Вседержтггелъ, Который Судьею станет для смертных. 

\vs 2Sb 1:221 Тем, кто землей погребен, отдаст их жизни небесный 

\vs 2Sb 1:222 Бог, и дыхание вложит, и голос вернет им, а кости 

\vs 2Sb 1:223 Вместе соединит и плотью затем их оденет, 

\vs 2Sb 1:224 Жилы приладит Он к жилам и вены кровью наполнит, 

\vs 2Sb 1:225 Кожею тело покроет и вырастит волосы снова 

\vs 2Sb 1:225 Части сложит Он все, придаст им дух и движенье; 

\vs 2Sb 1:226 Так людские тела за один лишь день воскресит Он. 

\vs 2Sb 1:227 Тут стальные засовы Аида, что чужд милосердья 

\vs 2Sb 1:228 И нерушим в своей силе всегда был прежде, сломает 

\vs 2Sb 1:229 Ангел тот, Уриил, что послан Богом, великий.

\vs 2Sb 1:230 Тартара мощь низложив, на суд печальные тени 

\vs 2Sb 1:231 Он поведет, и всех раньше Титанов, в давнее время 

\vs 2Sb 1:232 Живших, а с ними Гигантов, и сгинувших в водах Потопа, 

\vs 2Sb 1:233 Также и в бурной волне морской свою смерть повстречавших, 

\vs 2Sb 1:234 Тех, кого дикие звери, и змеи, и птицы пожрали, 

\vs 2Sb 1:235 Всех приведет Уриил к подножью Господнего трона; 

\vs 2Sb 1:236 С ними и тех, кто в огне, что плоть пожирает, сгорели, 

\vs 2Sb 1:237 Он соберет для Суда и пред Божьим престолом поставит.

\vs 2Sb 1:238 Мертвых когда воскресит он и судьбы земные разрушит, 

\vs 2Sb 1:239 В небе высоко гремящий Господь Саваоф Адонаи,

\vs 2Sb 1:240 Сев на престоле Своем, могучий столп установит, 

\vs 2Sb 1:241 На облаках придет к Безсмертному также Безсмертный 

\vs 2Sb 1:242 В славе Христос и с ним безупречные ангелы вместе. 

\vs 2Sb 1:243 Он одесную возсядет Великого Бога и станет 

\vs 2Sb 1:244 Праведно живших судить и людей, прозябавших в нечестье.

\vs 2Sb 1:245 Дружный с Самим Всевышним, придет Моисей, облеченный

\vs 2Sb 1:246 Плотью, как в жизни, а с ним Авраам предстанет великий, 

\vs 2Sb 1:247 Тут Исаак и Иаков, затем Иисус с Илиею,

\vs 2Sb 1:248 Аввакум, Даниил, Иона и кто от Евреев 

\vs 2Sb 1:249 Приняли смерть, будут здесь. И все Евреи погибнут,

\vs 2Sb 1:250 Чтобы за зло отплатить по слову Иеремии

\vs 2Sb 1:251 И получить по заслугам за все, что содеяли в жизни. 

\vs 2Sb 1:252 Всем тут придется пройти сквозь пламени жгучую реку 

\vs 2Sb 1:253 И негасимый огонь, в котором праведник всякий 

\vs 2Sb 1:254 Жизнь свою сохранит, но сгинут все нечестивцы;

\vs 2Sb 1:255 Тем исчезнуть навек, кто раньше зло сотворили: 

\vs 2Sb 1:256 Кто или сам убивал, или рядом стоял, не мешая, 

\vs 2Sb 1:257 Воры, обманщики все, домов разорители злые, 

\vs 2Sb 1:258 Клеветники, попрошайки и гадкие прелюбодеи, 

\vs 2Sb 1:259 Дерзко закон преступавшие, чтящие идолов разных,

\vs 2Sb 1:260 Те, кто отрекся от веры в Великого, Вечного Бога, 

\vs 2Sb 1:261 Кто притеснял и бранил людей справедливых и честных, 

\vs 2Sb 1:262 Кто убивал святых, гонитель истинной веры; 

\vs 2Sb 1:263 Также кто хитростью злой полны и безстыдством двуличным, 

\vs 2Sb 1:264 [Будучи старцами даже почтенными, правду боялись

\vs 2Sb 1:265 Ясно сказать на Суде и обиду другим учиняли, 

\vs 2Sb 1:266 Веря обманчивым слухам \ldots

\vs 2Sb 1:267 Люди, несущие гибель, страшнее, чем волки и барсы, 

\vs 2Sb 1:268 Те, что гордятся безмерно, и те, что деньги ссужают, 

\vs 2Sb 1:269 Дабы потом по домам лихву собирать за лихвою,

\vs 2Sb 1:270 Вдов несчастных, сирот обирая позорно до нитки; 

\vs 2Sb 1:271 С ними и все, кто сиротам и вдовам то уделяет, 

\vs 2Sb 1:272 Что нечестно нажили, и все, кто, делясь с неимущим, 

\vs 2Sb 1:273 Станут его попрекать; и дети, что бросили старых 

\vs 2Sb 1:274 Мать и отца, не отдав им доли сыновнего долга,

\vs 2Sb 1:275 Также и дети такие, которые не подчинялись

\vs 2Sb 1:276 Воле родителей, им отвечая лишь руганью злобной; 

\vs 2Sb 1:277 Те, кто поклялся, а после держать не хотел свое слово; 

\vs 2Sb 1:278 Слуги, что против власти господ мятеж затевали; 

\vs 2Sb 1:279 Все запятнавшие тело свое развратом безстыдным

\vs 2Sb 1:280 И вступавшие в связь нечестивую, пояс девичий 

\vs 2Sb 1:281 Развязавшие тайно; и женщины, что из утробы 

\vs 2Sb 1:282 Силой изгнали свой плод, и те, что рожденных сгубили; 

\vs 2Sb 1:283 И колдуны, и колдуньи. Всех этих грешников вместе 

\vs 2Sb 1:284 Гнев Живущего в небе Безсмертного Бога погонит

\vs 2Sb 1:285 Вплоть до того столпа, который кругом обегает 

\vs 2Sb 1:286 Пламени неугасимый поток. И ангелы Божьи, 

\vs 2Sb 1:287 Вестники Сущего вечно, их всех наказаньям подвергнут: 

\vs 2Sb 1:288 Ждет их пламени бич и жуткая цепь огневая,

\vs 2Sb 1:289 Не разорвать им оков, что тесно опутывать будут

\vs 2Sb 1:290 Их тела; а потом, во мраке ночи кромешной

\vs 2Sb 1:291 Адским зверям на съеденье в геенну их всех побросают  

\vs 2Sb 1:292 Множество страшных чудовищ таится в той тьме безграничной.

\vs 2Sb 1:293 Но когда казней различных от ангелов вдоволь претерпят, 

\vs 2Sb 1:294 Все, кто сердцем был зол, грядет им последняя кара:

\vs 2Sb 1:295 Огненный круг колеса из потока великого выйдет, 

\vs 2Sb 1:296 Тяжко давить оно будет вершителей дел беззаконных; 

\vs 2Sb 1:297 И раздадутся тогда отовсюду плач и стенанья, 

\vs 2Sb 1:298 Горький удел ужаснет и отцов, и детей неразумных, 

\vs 2Sb 1:299 И матерей, и младенцев, еще кормящихся грудью.

\vs 2Sb 1:300 Слез не выплакать им никогда, и жалкие крики 

\vs 2Sb 1:301 Уж не услышит никто, хоть будут звучать отовсюду: 

\vs 2Sb 1:302 Так что, терзаясь во тьме глубокого Тартара, станут 

\vs 2Sb 1:303 Вопли они испускать напрасно, и в скорбных угодьях 

\vs 2Sb 1:304 Трижды заплатят за все совершенные ими злодейства,

\vs 2Sb 1:305 Пламенем жарким палимы, зубами они заскрежещут, 

\vs 2Sb 1:306 Жажда сильнейшая им причинит мучения злые 

\vs 2Sb 1:307 И пожелают тогда умереть, но больше не смогут: 

\vs 2Sb 1:308 Не успокоит их смерть, и ночь не даст передышки. 

\vs 2Sb 1:309 Долго Всевышнего Бога молить они будут напрасно 

\vs 2Sb 1:310 И отвратит Господь Свой лик, чтоб их больше не видеть: 

\vs 2Sb 1:311 Ибо ведь людям заблудшим Он семь веков предоставил 

\vs 2Sb 1:312 Для покаянья  за них просила Дева святая. 

\vs 2Sb 1:313 Тех же, кто делал добро и был всегда справедливым, 

\vs 2Sb 1:314 Славился кто благочестьем и верным ума разсужденьем 

\vs 2Sb 1:315 Ангелы этих людей поднимут над страшным потоком 

\vs 2Sb 1:316 Пламени и поведут их к свету и к жизни безпечной 

\vs 2Sb 1:317 Тем нетленным путем, что Богом проложен Великим, 

\vs 2Sb 1:318 Где три источника бьют  медовый, винный и млечный. 

\vs 2Sb 1:319 Общею станет земля; перестав уже быть разделенной

\vs 2Sb 1:320 Стенами и рубежами, сама даст плод изобильный; 

\vs 2Sb 1:321 Вместе все заживут, нужды не имея в богатстве. 

\vs 2Sb 1:322 Тут не будет уже никто ни богатым, ни бедным, 

\vs 2Sb 1:323 Ни рабом, ни тираном, ни малым и ни великим; 

\vs 2Sb 1:324 Нет ни царей, ни вождей  все люди равны меж собою.

\vs 2Sb 1:325 Больше не скажет никто: наступила ночь, или завтра, 

\vs 2Sb 1:326 Или вчера это было, и дней, заботами полных, 

\vs 2Sb 1:327 Также не станет; исчезнут четыре времени года,

\vs 2Sb 1:328 Смерть и брачный союз; покупка вещей и продажа; 

\vs 2Sb 1:329 Даже Запад с Востоком  все в долгий день превратится,

\vs 2Sb 1:330 Тут Вседержитель Нетленный еще одно людям дарует: 

\vs 2Sb 1:331 Те, кто был праведной жизни, к Нему с мольбой обратятся, 

\vs 2Sb 1:332 Чтобы грешных Он спас от огня и от мук непрерывных,  

\vs 2Sb 1:333 Просьбы услышит Господь, и все это так совершится: 

\vs 2Sb 1:334 Он невредимыми всех из огня неусыпного вынет

\vs 2Sb 1:335 И через Свой народ в иные пошлет их угодья, 

\vs 2Sb 1:336 И для жизни иной, нетленной в полях Елисейских, 

\vs 2Sb 1:337 Там, где широко простер свои воды поток Ахеронта, 

\vs 2Sb 1:338 Озером став глубочайшим, которое вечно пребудет. 

\vs 2Sb 1:339 Горе мне, горе, несчастной! В тот день что будет со мною?!

\vs 2Sb 1:340 Я ведь стремилась в грехе превзойти всех людей безрассудно.

\vs 2Sb 1:341 И о супруге своем, и о здравом уме позабывши. 

\vs 2Sb 1:342 Но во дворце я жила моего богатого мужа, 

\vs 2Sb 1:343 Бедных туда не пускала; и зная, что зло совершаю, 

\vs 2Sb 1:344 На беззакония шла. Спаситель, меня от мучений,

\vs 2Sb 1:345 Наглую псицу, избавь, и все безстыдства прости мне. 

\vs 2Sb 1:346 Также молю Тебя: дай этой песне немного покоя, 

\vs 2Sb 1:347 Манны Податель благой, Владыка великого Царства!

\bibbookdescr{3Sb}{
  inline={Третья книга Сивилл},
  toc={3-я Сивилл},
  bookmark={3-я Сивилл},
  header={3-я Сивилл},
  abbr={3~Сив}
}
\vs 3Sb 1:1 В небе на троне Cидящий превыше самих херувимов, 

\vs 3Sb 1:2 О Громовержец Блаженный, молю Тебя  дай мне покоя! 

\vs 3Sb 1:3 Вестница истины всей, я устала вещать непрестанно. 

\vs 3Sb 1:4 Но отчего мое сердце трепещет все снова и снова? 

\vs 3Sb 1:5 Бич меня нудит какой устами правдивое пенье

\vs 3Sb 1:6 Смертным открыто излить? Опять обо всем расскажу я, 

\vs 3Sb 1:7 Что бы Господь ни велел мне людям ясно поведать.

\vs 3Sb 1:8 Люди, в облике вашем творение Божие зримо, 

\vs 3Sb 1:9 Что ж вы блуждаете зря, отнюдь не желая тропою

\vs 3Sb 1:10 В жизни прямою идти, о Безсмертном Создателе помня? 

\vs 3Sb 1:11 Только Единый есть Бог  в небесах, никем не рожденный,

\vs 3Sb 1:12 Неизречен и невидим, Он видит все, что есть в мире. 

\vs 3Sb 1:13 Бог не был создан ничьею рукой никогда, ни из камня 

\vs 3Sb 1:14 И ни из золота и ни из кости слоновьей твореньем

\vs 3Sb 1:15 Не был. Извечность Свою Он сам доказал непреложно  

\vs 3Sb 1:16 Сущий ныне, был раньше и впредь всегда Он пребудет. 

\vs 3Sb 1:17 Смертным дано ли очам Всевышнего Бога увидеть? 

\vs 3Sb 1:18 Разве вместит кто-нибудь одно только имя услышать 

\vs 3Sb 1:19 Бога Великого, в небе Живущего, мира Владыки?

\vs 3Sb 1:20 Сущее все Он создал Своим словом  и небо, и море, 

\vs 3Sb 1:21 Неутомимое солнце, луну, что растет постепенно, 

\vs 3Sb 1:22 Множество звезд светоносных и матерь могучую Тефис, 

\vs 3Sb 1:23 Дни с ночами, источники рек, огонь негасимый. 

\vs 3Sb 1:24 Бог сотворил человека, который был первым из смертных,

\vs 3Sb 1:25 Имя ему  Адам, и эти буквы четыре

\vs 3Sb 1:26 Север, и Юг, и Восток, и Запад собой заполняют. 

\vs 3Sb 1:27 Сам Господь утвердил людские вид и обличье, 

\vs 3Sb 1:28 Сделал зверей Он, и гадов, и птиц, летающих в небе.

\vs 3Sb 1:29 Бога не чтите вы и не боитесь в своем заблужденье,

\vs 3Sb 1:30 Вы поклоняетесь змеям и жертвы приносите кошкам, 

\vs 3Sb 1:31 Также и всяким кумирам, людей изваяньям из камня, 

\vs 3Sb 1:32 И перед входами в храмы безбожные вечно сидите. 

\vs 3Sb 1:33 Сущего Бога побойтесь, Который все наблюдает, 

\vs 3Sb 1:34 О почитатели мерзости каменной, как позабыли

\vs 3Sb 1:35 Вы о Мессии, что создал, Безсмертный, и небо и землю? 

\vs 3Sb 1:36 Род людей кровожадных, лукавых, дурных, нечестивых, 

\vs 3Sb 1:37 Племя злонравное с полными лжи языками двойными, 

\vs 3Sb 1:38 Идолов чтите, прелюбы творите и злое коварство 

\vs 3Sb 1:39 В сердце своем замышляете вы, друг у друга крадете \ldots

\vs 3Sb 1:40 В мыслях безстыдство у вас, в груди  свирепое жало! 

\vs 3Sb 1:41 Тот, кто владеет богатством, ничем не поделится с бедным; 

\vs 3Sb 1:42 Злобы ужасной полны, все люди про верность забудут; 

\vs 3Sb 1:43 Многие вдовы и с ними замужние женщины даже 

\vs 3Sb 1:44 Тайно станут любить других, желая наживы,

\vs 3Sb 1:45 После ж и вовсе в открытую будут греху предаваться.

\vs 3Sb 1:46 Рим пока еще медлит, но время настанет  Египтом 

\vs 3Sb 1:47 Он овладеет. Тогда величайшее царство на землю 

\vs 3Sb 1:48 Скоро к людям сойдет, им Царь будет править Безсмертный, 

\vs 3Sb 1:49 Вождь священный придет, держащий скиптры земные,

\vs 3Sb 1:50 Сколько б времен ни прошло, все ж нет конца Его Царству. 

\vs 3Sb 1:51 Вспыхнет тогда у латинских мужей великая ярость; 

\vs 3Sb 1:52 Трое разрушат Рим, когда бросят жребий несчастный. 

\vs 3Sb 1:53 Люди погибнут все под гнетом собственных кровель, 

\vs 3Sb 1:54 Ибо огненный ливень с небес на землю прольется.

\vs 3Sb 1:55 О, я несчастная, страшный тот день  когда ж он настанет, 

\vs 3Sb 1:56 День, когда призовет на суд Властитель Небесный? 

\vs 3Sb 1:57 Стройтесь до времени, о города, и еще украшайтесь 

\vs 3Sb 1:58 Пышностью храмов, рынков, ристалищ, кумиров из камня, 

\vs 3Sb 1:59 Золота и серебра, и все это так сохранится

\vs 3Sb 1:60 Вплоть до горького дня  в парах удушливой серы 

\vs 3Sb 1:61 Люди тогда задохнутся \ldots\ Но лучше все по порядку 

\vs 3Sb 1:62 Я о несчастьях скажу, в каких городах они будут \ldots

\vs 3Sb 1:63 Явится вслед за тем Велиал, он придет из Себасты, 

\vs 3Sb 1:64 Станет горы сдвигать, усмирит и бурное море, 

\vs 3Sb 1:65 Солнце с луной светоносные он в небесах остановит,

\vs 3Sb 1:66 Тех, кто усоп, воскресит и много знамений чудных 

\vs 3Sb 1:67 Людям он явит, но мира конец еще не наступит  

\vs 3Sb 1:68 Будет все только соблазн, хоть, конечно, немало обманет 

\vs 3Sb 1:69 Верных сей Велиал Евреев и множество прочих

\vs 3Sb 1:70 Смертных мужей, что Закона и Божьего Слова не знают. 

\vs 3Sb 1:71 Но лишь начнут исполняться угрозы великого Бога, 

\vs 3Sb 1:72 Пламень, сжигающий все, потоками хлынет на землю; 

\vs 3Sb 1:73 Сгинут в пламени том Велиал и надменные люди  

\vs 3Sb 1:74 Все, кто веру речам и делам его даровали.

\vs 3Sb 1:75 Женщине миром всецело тогда завладеет, и станет 

\vs 3Sb 1:76 Он ей во всем подчиняться и слушаться безпрекословно. 

\vs 3Sb 1:77 После того вдова окажется мира царицей; 

\vs 3Sb 1:78 Бросит в море она серебро и злато людское, 

\vs 3Sb 1:79 Также всю медь и железо утопит о соленой пучине;

\vs 3Sb 1:80 Все элементы тогда с лица земного исчезнут. 

\vs 3Sb 1:81 Руки могучие Бог из чертогов эфирных протянет, 

\vs 3Sb 1:82 Свод небесный свернет, как будто свиток прочтенный; 

\vs 3Sb 1:83 Весь небосвод многовидный обрушится наземь и в море, 

\vs 3Sb 1:84 Огненный дождь будет лить и все сжигать непрестанно 

\vs 3Sb 1:85 Землю и воду спалит и небесную ось уничтожит. 

\vs 3Sb 1:86 Так творение Божье окажется сплавом единым, 

\vs 3Sb 1:87 После же снова на части разнимется для очищенья. 

\vs 3Sb 1:88 Больше не будут с небес никогда смеяться светила, 

\vs 3Sb 1:89 Ночь и заря упразднятся, и дней, заботами полных,

\vs 3Sb 1:90 Также не станет, исчезнут четыре времени года. 

\vs 3Sb 1:91 Век начнется великий, и Суд Всемощного Бога 

\vs 3Sb 1:92 Будет над миром, когда реченное все совершится.

\vs 3Sb 1:93 О судоходные воды, о суша вся от Востока 

\vs 3Sb 1:94 И до Заката  хоть больше уже не закатится солнце  

\vs 3Sb 1:95 Все подчинится Ему, в этот мир пришедшему снова, 

\vs 3Sb 1:96 Ибо сам Он познал Свою силу могучую первым.

\vs 3Sb 1:97 Все угрозы привел Безсмертный Бог в исполненье, 

\vs 3Sb 1:98 Коими людям грозил  они в земле Ассирийской 

\vs 3Sb 1:99 Башню построили  все меж собою согласными были  

\vs 3Sb 1:100 Страстно желали до звезд по этой башне добраться. 

\vs 3Sb 1:101 Тут Безсмертный ветрам повелел лететь что есть силы 

\vs 3Sb 1:102 К месту тому  ветра повергли огромное зданье,

\vs 3Sb 1:103 Ссору тогда меж собой учинили строители башни; 

\vs 3Sb 1:104 Вот почему с тех пор это место зовут Вавилоном.

\vs 3Sb 1:105 После крушения башни язык людской разделился 

\vs 3Sb 1:106 И превратился в обилье наречий разных, а дальше 

\vs 3Sb 1:107 Смертные, землю заполнив, ее поделили на царства. 

\vs 3Sb 1:108 То поколение было десятым с тех пор, как всемирный 

\vs 3Sb 1:109 Залил землю Потоп и первых людей уничтожил.

\vs 3Sb 1:110 Кронос, Титан и Япет над миром стали царями, 

\vs 3Sb 1:111 Люди их называли сынами Урана и Геи, 

\vs 3Sb 1:112 Имя земли и небес потому к царям прилагая, 

\vs 3Sb 1:113 Что наилучшими были они средь того поколенья. 

\vs 3Sb 1:114 Землю натрое всю разделили и бросили жребий,

\vs 3Sb 1:115 Каждый стал управлять в удел полученной частью; 

\vs 3Sb 1:116 Все отцу дали клятвы, и правильным было деленье, 

\vs 3Sb 1:117 Так что меж ними вражды не возникло. Но время настало 

\vs 3Sb 1:118 Умер старый отец. И, клятвы нарушив преступно, 

\vs 3Sb 1:119 Дети тогда меж собой учинили раздор величайший:

\vs 3Sb 1:120 Стали спорить, кому быть царем над всею землею. 

\vs 3Sb 1:121 Тут вражда началась у Титана и Кроноса злая. 

\vs 3Sb 1:122 Рея  сестра, мать Гея, Деметра и Афродита, 

\vs 3Sb 1:123 Та, что венки сплетать мастерица, и Гестия с ними, 

\vs 3Sb 1:124 Также Диона прекрасноволосая их помирили;

\vs 3Sb 1:125 Вместе собрали царей, их братьев и родичей разных, 

\vs 3Sb 1:126 Всех отцов и потомков, кто кровью был близок, созвали. 

\vs 3Sb 1:127 Те же держали совет и решили, что Кронос над всеми 

\vs 3Sb 1:128 Царствовать должен, поскольку он старше, благообразней. 

\vs 3Sb 1:129 Должен был Кронос Титану поклясться страшною клятвой

\vs 3Sb 1:130 В том, что мужского потомства иметь никогда он не будет, 

\vs 3Sb 1:131 Чтоб после смерти отца не мог его сын воцариться. 

\vs 3Sb 1:132 И когда срок наступал разрешиться от бремени Рее, 

\vs 3Sb 1:133 Подле Титаны садились и мальчиков всех разрывали 

\vs 3Sb 1:134 В клочья, а девочек всех у сосцов оставляли кормиться.

\vs 3Sb 1:135 Третьи роды настали у Реи, и первая Гера

\vs 3Sb 1:136 Вышла на свет, и, увидев младенца своими глазами, 

\vs 3Sb 1:137 Злые встали Титаны и все ушли восвояси. 

\vs 3Sb 1:138 Только затем появился ребенок пола мужского, 

\vs 3Sb 1:139 Рея тайком отослала его, чтобы спасся и вырос,

\vs 3Sb 1:140 Через трех жителей Крита во Фригию, клятвой связав их; 

\vs 3Sb 1:141 То, что сына вот так переслала, дало ему имя.

\vs 3Sb 1:142 Позже Рея спасла и другое дитя  Посейдона. 

\vs 3Sb 1:143 Третьим сыном Плутон был у этой женщины чудной  

\vs 3Sb 1:144 Недалеко от Додоны от бремени им разрешилась,

\vs 3Sb 1:145 Там, где несет свои воды Эвроп, который, с Пенеем 

\vs 3Sb 1:146 Слившись, в море течет и зовется Стигийской рекою. 

\vs 3Sb 1:147 Стало известно Титанам, что тайно смерти избегли 

\vs 3Sb 1:148 Дети, рожденные Реей от Кроноса. Тут возмутился 

\vs 3Sb 1:149 Сам Титан. Шестьдесят сыновей призвавши на помощь,

\vs 3Sb 1:150 Брата цепями сковал, а с ним и жену его Рею, 

\vs 3Sb 1:151 В землю упрятал обоих и там в оковах держал их. 

\vs 3Sb 1:152 Только узнали о том могучего Кроноса дети, 

\vs 3Sb 1:153 Подняли шум боевой и затеяли жаркую битву, 

\vs 3Sb 1:154 Битва великая та положила всем войнам начало,

\vs 3Sb 1:155 Первоначало всех войн среди смертных собою явила. 

\vs 3Sb 1:156 Вот за это наслал Господь на Титанов несчастье, 

\vs 3Sb 1:157 Сгинули все их потомки, но племя Кроноса  тоже. 

\vs 3Sb 1:158 Время затем совершило свой круг, и царство Египта , 

\vs 3Sb 1:159 Было воздвигнуто, вслед появились новые царства 

\vs 3Sb 1:160 Персов, Мидян, Эфиопов, в Ассирии вкруг Вавилона,

\vs 3Sb 1:161 У Македонцев и снова в Египте, в конце же  у Римлян. 

\vs 3Sb 1:162 Бог Всемогущий тогда вложил мне пророчество в душу, 

\vs 3Sb 1:163 И возвестить повелел по всей земле это слово, 

\vs 3Sb 1:164 Дабы властителям стало известно, что будет в грядущем.

\vs 3Sb 1:165 Первое то мне открыл Господь Единый, какие

\vs 3Sb 1:166 Царства людские возникнут и сколько их будет на свете. 

\vs 3Sb 1:167 Первым дом Соломонов над Азией всей воцарится, 

\vs 3Sb 1:168 Персии, Фригии станет владыкою и Финикии, 

\vs 3Sb 1:169 Островитяне, Карийцы, Мизийцы ему подчинятся,

\vs 3Sb 1:170 Он покорит и Лидийцев, богатое золотом племя... 

\vs 3Sb 1:171 Эллинов род, злодеев надменных, господствовать будет 

\vs 3Sb 1:172 Дальше, а после него  великое пестрое племя 

\vs 3Sb 1:173 Тех Македонцев, что тучи войны надвинут на смертных; 

\vs 3Sb 1:174 Бог небесный, однако, их всех уничтожит под корень.

\vs 3Sb 1:175 Но вслед за ними грядет другого царства начало: 

\vs 3Sb 1:176 Белый, могучий народ с берегом Гесперийского моря 

\vs 3Sb 1:177 Выйдет, разные страны захватит и в ужас повергнет 

\vs 3Sb 1:178 Многих, а в душах царей он страх надолго поселит. 

\vs 3Sb 1:179 Золота и серебра в городах награблено будет

\vs 3Sb 1:180 Тут немало, но вновь появится золото в мире

\vs 3Sb 1:181 И серебро, а потом и других украшений в достатке. 

\vs 3Sb 1:182 Смертные много тогда претерпят, но в наказанье

\vs 3Sb 1:183 Низко падут нечестивцы надменные, мерзостью жуткой 

\vs 3Sb 1:184 Жизнь их наполнится вся, мужчина с мужчиною станут

\vs 3Sb 1:185 Здесь предаваться разврату, а малых детей на продажу 

\vs 3Sb 1:186 Будут в позорных домах выставлять. Великое горе 

\vs 3Sb 1:187 К людям придет в те дни и посеет страшную смуту, 

\vs 3Sb 1:188 Все устои разрушит и злом это царство наполнит; 

\vs 3Sb 1:189 Страсть к наживе лихой, позорная алчность охватят

\vs 3Sb 1:190 Многие страны, а больше других  Македонскую землю. 

\vs 3Sb 1:191 Долго у них в чести коварство и ненависть будут, 

\vs 3Sb 1:192 Это продлится до царства седьмого по счету в Египте  

\vs 3Sb 1:193 Родом должен быть Эллин в то время Египта владыка  

\vs 3Sb 1:194 Сила появится вновь у народа великого Бога:

\vs 3Sb 1:195 Праведной жизни пути он смертным указывать станет...

\vs 3Sb 1:196 Но отчего же Господь вещать меня заставляет,

\vs 3Sb 1:197 Что будет первым несчастьем для всех человеков, что дальше

\vs 3Sb 1:198 С ними случится, где бедствий конец и где их источник?

\vs 3Sb 1:199 Первыми примут Титаны от Бога жестокую кару: 

\vs 3Sb 1:200 Мощного Кроноса им сыновья отомстят по заслугам, 

\vs 3Sb 1:201 Ибо сковали Титаны отца их и мать вероломно. 

\vs 3Sb 1:202 Позже у Эллинов власть захватят злые тираны, 

\vs 3Sb 1:203 И воцарятся у них надменные прелюбодеи 

\vs 3Sb 1:204 И нечестивцы, которым все доброе чуждо, а войнам 

\vs 3Sb 1:205 Впредь не будет конца. Фригийцы грозные сгинут, 

\vs 3Sb 1:206 Бедствий черные дни для Троянского града настанут. 

\vs 3Sb 1:207 Горе придет не замедлив и к Персам и к Ассирийцам, 

\vs 3Sb 1:208 В Ливию и к Эфиопам прошествует через Египет, 

\vs 3Sb 1:209 Быть ему и у Карийцев, в Памфилии также... да что я 

\vs 3Sb 1:210 Перечисляю народы?  у всех людей будет горе. 

\vs 3Sb 1:211 Только конец одному, как вскоре второе нагрянет 

\vs 3Sb 1:212 К людям несчастье, но я вначале скажу о первейшем.

\vs 3Sb 1:213 Горе постигнет и тех, кто возле великого храма, 

\vs 3Sb 1:214 Что Соломон воздвиг, живут в благочестье  и предки 

\vs 3Sb 1:215 Праведны были у них, и вот теперь поведу я 

\vs 3Sb 1:216 Речь об этом народе, земле его, предках и ясно 

\vs 3Sb 1:217 Все опишу для тебя, коварный и суетный смертный!

\vs 3Sb 1:218 Город есть на Востоке, зовется он Уром Халдейским, 

\vs 3Sb 1:219 Праведной жизни народ происходит оттуда, те люди

\vs 3Sb 1:220 Мыслили здраво всегда и много благого творили. 

\vs 3Sb 1:221 Их не заботит ничуть светил небесных вращенье, 

\vs 3Sb 1:222 Не помышляют они о земных чудовищах жутких. 

\vs 3Sb 1:223 Ни о манящих глубинах соленых вод Океана. 

\vs 3Sb 1:224 То, как птицы клюют, иль то, как люди чихают,

\vs 3Sb 1:225 Их не волнует, они чародеям и магам не верят,

\vs 3Sb 1:226 Чревовещатели ложью своей соблазнить их не в силах. 

\vs 3Sb 1:227 Не признают они там ни халдейских гаданий по звездам, 

\vs 3Sb 1:228 Ни астрономии. Нет, все то почитают обманом, 

\vs 3Sb 1:229 Чем занимаются изо дня в день неразумные люди,

\vs 3Sb 1:230 Души свои упражняя в вещах совершенно ненужных. 

\vs 3Sb 1:231 Этим своим заблужденьям они еще обучают 

\vs 3Sb 1:232 Разных глупцов, оттого много бед бывает, ведь люди, 

\vs 3Sb 1:233 Сбившись с благого пути, забывают о праведной жизни. 

\vs 3Sb 1:234 Те же, о ком говорю, почитают все справедливость

\vs 3Sb 1:235 И добродетель, не думают, как бы им стать побогаче 

\vs 3Sb 1:236 (Смертным нажива несет лишь зло, и голод, и войны), 

\vs 3Sb 1:237 Верная мера у них во всем в городах и в селеньях. 

\vs 3Sb 1:238 Здесь никто по ночам ничего у других не ворует, 

\vs 3Sb 1:239 Коз, овец и волов не бывает, чтоб тут угоняли,

\vs 3Sb 1:240 В поле земли никогда не отнимет сосед у соседа, 

\vs 3Sb 1:241 Самый богатый у них не обидит того, кто беднее, 

\vs 3Sb 1:242 Горя не причинит вдове, а напротив, поможет 

\vs 3Sb 1:243 Хлебом в нужде, вином и оливками  не поскупится. 

\vs 3Sb 1:244 Есть тут немало счастливцев, но, если кто-то несчастен,

\vs 3Sb 1:245 С бедным своим урожаем поделится летом имущий. 

\vs 3Sb 1:246 Ибо послушны они реченью великого Бога  

\vs 3Sb 1:247 Общей создал для всех небесный Царь эту землю.

\vs 3Sb 1:248 В дни, как Египет покинут и двинутся в путь по пустыне 

\vs 3Sb 1:249 Эти двенадцать колен, от Господа сопровождены;

\vs 3Sb 1:250 Будет дано им: в ночи столп огня озарит их дорогу, 

\vs 3Sb 1:251 Скрытые обликом, днем пойдут они безопасно. 

\vs 3Sb 1:252 Посланный Богом народу, его предводителем станет 

\vs 3Sb 1:253 Славный муж Моисей, который ребенком в болоте 

\vs 3Sb 1:254 Был царицею найден  она его воспитала,

\vs 3Sb 1:255 Сыном назван; и вот, с ним вышел народ из Египта. 

\vs 3Sb 1:256 Бог к Синайской горе привел их и с неба народу 

\vs 3Sb 1:257 Дал закон благочестья, на двух записав его досках. 

\vs 3Sb 1:258 И повелел: того, кто не станет блюсти предписаний,

\vs 3Sb 1:259 Или закон покарает, иль руки накажут людские,

\vs 3Sb 1:260 Если ж и скрыться сумеет, расплата его не минует. 

\vs 3Sb 1:261 [Общей создал для всех небесный Царь эту землю, 

\vs 3Sb 1:262 В сердце им всем Господь благое вложил помышленье.] 

\vs 3Sb 1:263 Только добрым сторицей воздаст хлебодарная пашня, 

\vs 3Sb 1:264 Так отмерил сам Бог. Но и добрых людей ожидают

\vs 3Sb 1:265 Беды, им не избегнуть никак ужасного мора.

\vs 3Sb 1:266 И побежишь ты тогда, покинув храм свой чудесный, 

\vs 3Sb 1:267 Ибо священную землю оставить велят тебе судьбы. 

\vs 3Sb 1:268 Жить придется тебе в земле Ассирийской, увидишь 

\vs 3Sb 1:269 Жен и малых детей рабами, людям враждебным.

\vs 3Sb 1:270 Все тут богатство погибнет, не сможешь добыть пропитанья; 

\vs 3Sb 1:271 Будут тобою полны все земли и воды морские, 

\vs 3Sb 1:272 Но не полюбит никто обычай твой и законы. 

\vs 3Sb 1:273 Вся же твоя страна опустеет: холм укрепленный, 

\vs 3Sb 1:274 Храм великого Бога и длинные мощные стены

\vs 3Sb 1:275 Рухнут тогда во прах, а причиною  то, что не чтил ты 

\vs 3Sb 1:276 Господом данный закон священный, но в заблужденье 

\vs 3Sb 1:277 Идолам мерзким служил и ничуть Того не боялся, 

\vs 3Sb 1:278 Кто породил всех богов и людей  Безсмертного Бога; 

\vs 3Sb 1:279 Чтить ты Его не желал, почитал изваяния смертных.

\vs 3Sb 1:280 Вот за это земля плодородная будет пустыней 

\vs 3Sb 1:281 Семь десятков времен, и во храме чудес не увидят. 

\vs 3Sb 1:282 Но в конце тебя ждут великая радость и слава: 

\vs 3Sb 1:283 Все исполнят Господь и смертный, когда не предашь ты 

\vs 3Sb 1:284 Веры в священный закон, полученный некогда свыше, 

\vs 3Sb 1:285 Ноги устанут твои, но светлого дня ты достигнешь.

\vs 3Sb 1:286 Царь будет послан от Бога с высокого неба на землю,

\vs 3Sb 1:287 Каждого станет судить в крови и в пламенном свете. 

\vs 3Sb 1:288 Только один парод, одно лишь царское племя 

\vs 3Sb 1:289 Не поколеблется тут. Ему предназначено править 

\vs 3Sb 1:290 В смене времен и начать строительство нового храма. 

\vs 3Sb 1:291 Всякий владыка Персидский тут помощь оказывать станет 

\vs 3Sb 1:292 Бронзою, кованым прочным железом и золотом даже, 

\vs 3Sb 1:293 Ибо Господь сновиденья священные ночью пошлет им. 

\vs 3Sb 1:294 Так, воздвигнувшись вновь, святыня пребудет, как прежде.

\vs 3Sb 1:295 В сердце утихло моем звучанье божественной песни, 

\vs 3Sb 1:296 И обратила мольбы я к Творцу, чтоб Он дал мне покоя.

\vs 3Sb 1:297 Но Всемогущий опять вложил пророчество в душу

\vs 3Sb 1:298 И возвестить повелел по всей земле это слово,

\vs 3Sb 1:299 Дабы властителям стало известно, что будет в грядущем.

\vs 3Sb 1:300 Первое, что наказал мне Господь Единый поведать,  

\vs 3Sb 1:301 Сколько горьких напастей отмерил Он Вавилону 

\vs 3Sb 1:302 Карою за разграбленье великого Божьего храма. 

\vs 3Sb 1:303 Горе тебе, Вавилон и племя мужей Ассирийских! 

\vs 3Sb 1:304 Шум ужасный услышат родившие грешников земли,

\vs 3Sb 1:305 Клич боевой принесет погибель внезапную людям, 

\vs 3Sb 1:306 Бог, моих песен владыка, сразит их могучим ударом. 

\vs 3Sb 1:307 Бог к тебе, Вавилон, сойдет из высей воздушных, 

\vs 3Sb 1:308 Спустится Он со святых небес на грешную землю.

\vs 3Sb 1:309 Гнев Господень сулит сынам твоим вечную гибель.

\vs 3Sb 1:310 Станешь тогда ты, как если б и не было вовсе на свете 

\vs 3Sb 1:311 Города никогда такого, наполнишься кровью, 

\vs 3Sb 1:312 Вспомнишь, как сам проливал кровь добрых и справедливых, 

\vs 3Sb 1:313 Что и доселе еще вопиет к высокому небу.

\vs 3Sb 1:314 Страшный удар потрясет твои жилища, Египет, 

\vs 3Sb 1:315 Ты никогда и представить не мог, что случится такое! 

\vs 3Sb 1:316 Меч тяжелый тебя пронзит посредине, а следом 

\vs 3Sb 1:317 Голод и мор и рассеянье будут, идя чередою, 

\vs 3Sb 1:318 В царство седьмое губить страну, и так ты исчезнешь.

\vs 3Sb 1:319 Гог и Магог, увы, увы тебе, край Эфиопский! 

\vs 3Sb 1:320 Реки текут по тебе сейчас, но в будущем хлынет 

\vs 3Sb 1:321 Кровь потоками здесь, и, затоплена черною кровью, 

\vs 3Sb 1:322 Судным местом тогда среди смертных ты прозовешься.

\vs 3Sb 1:323 Ливия, горе тебе, и землям горе и водам!

\vs 3Sb 1:324 Дочери Запада, вас настигнет день несчастливый, 

\vs 3Sb 1:325 Вам не уйти от борьбы тяжелой, она неотступно

\vs 3Sb 1:326 Будет преследовать вас, и суд наступит ужасный.

\vs 3Sb 1:327 И поневоле придется погибнуть вам всем в это время.

\vs 3Sb 1:328 Ибо бессмертного Бога жилища вы источили

\vs 3Sb 1:329 И растерзали его вконец зубами стальными, 

\vs 3Sb 1:330 Край свой увидишь тогда в страну мертвецов превращенным:

\vs 3Sb 1:331 Сгинут одни от войны и по воле несчастного рока, 

\vs 3Sb 1:332 Голод иных уничтожит, чума и бешенство вражье, 

\vs 3Sb 1:333 Все твои города, все земли пустынею станут. 

\vs 3Sb 1:334 Вспыхнет звезда на Заходе  она наречется кометой  

\vs 3Sb 1:335 Вестницей станет она сражений, голода, смерти, 

\vs 3Sb 1:336 Гибели славных вождей и прочих людей знаменитых.

\vs 3Sb 1:337 Знаменья будут тогда даны величайшие смертным: 

\vs 3Sb 1:338 Течь прекратит Танаис в Меотиду струей многоводной,

\vs 3Sb 1:339 Высохнув, русло его плодородною пашнею станет,

\vs 3Sb 1:340 В озеро воды польются по множеству малых протоков. 

\vs 3Sb 1:341 Много в почве возникнет провалов и пропастей, рухнут 

\vs 3Sb 1:342 В них города со всеми людьми. Эта страшная участь 

\vs 3Sb 1:343 В Азии Смирну постигнет, Иас, Кебрен, Пандонию, 

\vs 3Sb 1:344 Антиохию, Эфес, Колофон, Никею, Скиагру,

\vs 3Sb 1:345 Астипалею, Синоп, Иераполь, счастливую Газу. 

\vs 3Sb 1:346 Так же погибнут в Европе Танагра и Меропея, 

\vs 3Sb 1:347 Так пропадут Микены, Магнезия и Антигона. 

\vs 3Sb 1:348 Знайте тогда, что вскоре конец настанет Египту, 

\vs 3Sb 1:349 Прежнее лето всегда будет лучшим для Александрийцев.

\vs 3Sb 1:350 Сколько бы Рим ни взял с покоренной Азии дани, 

\vs 3Sb 1:351 Втрое больше ему возвратить сокровищ придется 

\vs 3Sb 1:352 Азии, ибо надменным она победителем станет. 

\vs 3Sb 1:353 Много богатств возьмет с Азиатов народ Италийский,

\vs 3Sb 1:354 Двадцатикратно, однако, он собственной рабскою службой

\vs 3Sb 1:355 Должен будет вернуть, в нищете пребывая великой. 

\vs 3Sb 1:356 В золоте, в роскоши ты, о дочь Латинского Рима, 

\vs 3Sb 1:357 С множеством женихов сколь часто вином упивалась!  

\vs 3Sb 1:358 В жены тебя отдадут не в пышном наряде  служанкой, 

\vs 3Sb 1:359 Срежет тебе госпожа копну волос твоих пышных.

\vs 3Sb 1:360 Восторжествует тогда справедливость, и с неба на землю

\vs 3Sb 1:361 Сброшено будет одно, из праха восстанет другое  

\vs 3Sb 1:362 Слишком уж люди погрязли в пороке и жизни нечестной.

\vs 3Sb 1:363 Делос невидимым станет, а Самос в песок превратится, 

\vs 3Sb 1:364 Рим руинами будет  исполнятся все предсказанья. 

\vs 3Sb 1:365 Больше ни слова о Смирне  пускай себе погибает 

\vs 3Sb 1:366 От преступлений вождей, неразумных и несправедливых.

\vs 3Sb 1:367 В Азии тихий покой воцарится, счастливою станет 

\vs 3Sb 1:368 В те времена и Европа: блаженную жизнь и здоровье 

\vs 3Sb 1:369 Небо людям пошлет вместо злого снега и града,

\vs 3Sb 1:370 Даст оно много зверей и птиц и ползучих в достатке.

\vs 3Sb 1:371 О, сколь счастливы те мужи и жены, которым 

\vs 3Sb 1:372 Жить доведется в тот век, похожий на дивную сказку.

\vs 3Sb 1:373 Благозаконие и справедливость со звездного неба 

\vs 3Sb 1:374 К людям придут, и тогда воцарится всем смертным на пользу

\vs 3Sb 1:375 Мудрое мыслей единство, а с ним  любовь и доверье,

\vs 3Sb 1:376 Гостеприимства законы блюсти станут люди; при этом 

\vs 3Sb 1:377 Вовсе исчезнут нужда и насилие, больше не будет 

\vs 3Sb 1:378 Зависти, гнева, насмешек, безумства и преступлений; 

\vs 3Sb 1:379 Ссоры, жестокая брань, грабеж по ночам и убийства 

\vs 3Sb 1:380 В общем, всякое зло в те дни на земле прекратится. 

\vs 3Sb 1:381 Но Македонцы сулят всей Азии тяжкие беды; 

\vs 3Sb 1:382 Вырастет и для Европы еще великое горе  

\vs 3Sb 1:383 Горе от племени мнимых Кронидов и рабского рода; 

\vs 3Sb 1:384 И Вавилон, хорошо укрепленный, захватит их войско.

\vs 3Sb 1:385 Этих людей назовут владыками целого света,

\vs 3Sb 1:386 Но погибнет их царство от страшных бед, не оставив 

\vs 3Sb 1:387 Даже законов потомкам, по разным разбредшимся странам.

\vs 3Sb 1:388 Муж коварный в то время в счастливую Азию вступит, 

\vs 3Sb 1:389 Будет носить на плечах порфирное он одеянье.

\vs 3Sb 1:390 Молния в мир его принесет. Потому-то, свирепый, 

\vs 3Sb 1:391 Дикий и непостоянный, ярмо для Азии злое 

\vs 3Sb 1:392 Этот муж приготовит, и кровью убийства сырая 

\vs 3Sb 1:393 Здесь упьется земля, но Аид усмирит кровопийцу. 

\vs 3Sb 1:394 Род, который под корень хотелось ему уничтожить,

\vs 3Sb 1:395 Сам погубит потом его, а единственный корень  

\vs 3Sb 1:396 Срубит после один из десятка рогов кровожадный, 

\vs 3Sb 1:397 После же новый побег он с прежними рядом насадит. 

\vs 3Sb 1:398 Но, погубив отца порфироносного рода, 

\vs 3Sb 1:399 Тоже погибнет от рук детей, заговорщиков дерзких,

\vs 3Sb 1:400 И воцарится затем тот рог, что посажен был рядом.

\vs 3Sb 1:401 Знаменье явится вскоре Фригийской земле плодоносной: 

\vs 3Sb 1:402 Род огромный и злой  потомки матери Реи  

\vs 3Sb 1:403 Вечным мнивший себя, ибо вырос из корня сухого, 

\vs 3Sb 1:404 В ночь исчезнет одну. В эту ночь земли Колебатель

\vs 3Sb 1:405 Почву разверзнет в том граде, которому люди позднее 

\vs 3Sb 1:406 Имя дадут Дорилейон. Все это в древней случится 

\vs 3Sb 1:407 Черной Фригийской земле, не раз слезами политой. 

\vs 3Sb 1:408 Времени этому люди дадут Колебателя имя, 

\vs 3Sb 1:409 Ибо Он щели разверзнет земные и стены разрушит.

\vs 3Sb 1:410 Знаки все это дурные  за ними беды начнутся.

\vs 3Sb 1:411 Множество разных народов придет со своими вождями, 

\vs 3Sb 1:412 Чтобы в земле воевать, где предки живут Энеадов. 

\vs 3Sb 1:413 Станут добычей они снедаемым жадностью людям.

\vs 3Sb 1:414 Горе тебе, Илион! Эриния вырастит в Спарте

\vs 3Sb 1:415 Ветвь чудесную, чья красота всем известною станет. 

\vs 3Sb 1:416 Но породит она бурю над Азией и над Европой, 

\vs 3Sb 1:417 Ты, Илион, больше всех услышишь тут плача и стонов, 

\vs 3Sb 1:418 Вечно будут, однако, виновницу помнить потомки.

\vs 3Sb 1:419 Явится старец затем и напишет много неправды, 

\vs 3Sb 1:420 Ложно и город родной назовет. Хоть света не узрят 

\vs 3Sb 1:421 Очи его, но с великим умом и, мысль облекая 

\vs 3Sb 1:422 Ясно в слова, он напишет  но, Хиос своим называя, 

\vs 3Sb 1:423 Речь о том поведет, что у стен Илиона свершилось. 

\vs 3Sb 1:424 Ложь его будет правдивой казаться: слова и размеры 

\vs 3Sb 1:425 Он ведь из книг моих почерпнет, сперва прочитав их.

\vs 3Sb 1:426 Очень красиво опишет воителей подвиги старец,

\vs 3Sb 1:427 Гектора, сына Приама, и сына Пелея, Ахилла,

\vs 3Sb 1:428 Также и прочих, кто в этой войне подвизался отважно.

\vs 3Sb 1:429 Изобразит он, что боги сражавшимся там помогали, 

\vs 3Sb 1:430 Самую разную ложь сочинит для людей скудоумных.

\vs 3Sb 1:431 Павшим у стен Илиона причтется великая слава:

\vs 3Sb 1:432 Поочередно певец про оба войска расскажет.

\vs 3Sb 1:433 Много зла причинит Ликийцам Локра потомство;

\vs 3Sb 1:434 Ты, Халкидон, у морской теснины лежащий, погибнешь 

\vs 3Sb 1:435 В час, как придет к тебе дитя земли Этолийской.

\vs 3Sb 1:436 Кизик, море отнимет твое богатство и счастье;

\vs 3Sb 1:437 Плохо придется тебе, Византий, стоящий напротив

\vs 3Sb 1:438 Азии; стоном и кровью до края ты будешь наполнен.

\vs 3Sb 1:439 С той вершины горы, что над Ликией высится, воды 

\vs 3Sb 1:440 Хлынут потоками вниз, из твердого выбившись камня.

\vs 3Sb 1:441 Чтобы утихли они, надо сбыться отцов предсказаньям.

\vs 3Sb 1:442 Город обильной вином Пропонтиды, увы тебе, Кизик! 

\vs 3Sb 1:443 Бурной Риндакской волною ты будешь залит и потоплен.

\vs 3Sb 1:444 Родос, и твой век недолог, хотя и немалое время 

\vs 3Sb 1:445 Рабства ты не познаешь и славиться будешь богатством, 

\vs 3Sb 1:446 И не оспорит никто твоего господства над морем. 

\vs 3Sb 1:447 Все же станешь добычей снедаемым жадностью людям, 

\vs 3Sb 1:448 За красоту и богатство  ужасное иго претерпишь.

\vs 3Sb 1:449 В Лидии вздрогнет земля, и Персия вся сокрушится;

\vs 3Sb 1:450 Сколько несчастий Европу и Азию тут ожидают!

\vs 3Sb 1:451 Царь Сидонский и много других владык кровожадных 

\vs 3Sb 1:452 За море смерть понесут с собою  на Самос и дальше. 

\vs 3Sb 1:453 В море много земли потоки кровавые смоют, 

\vs 3Sb 1:454 Жены и девы в красивых нарядах горько заплачут;

\vs 3Sb 1:455 Жалкую долю свою проклянут они, ибо навеки 

\vs 3Sb 1:456 Эти любимых отцов, а те  сыновей потеряют.

\vs 3Sb 1:457 Будет для Кипра знак  ужасное землетрясенье, 

\vs 3Sb 1:458 Множество душ оно Аиду отдаст в одночасье!

\vs 3Sb 1:459 Рухнут и мощные стены с Эфесом соседнего Тралла 

\vs 3Sb 1:460 За преступления их обитателей, злых и жестоких. 

\vs 3Sb 1:461 Воды горячие с неба на землю польются, и станет 

\vs 3Sb 1:462 Впитывать почва ту влагу и запах удушливой серы.

\vs 3Sb 1:463 Самос царский дворец построит, как сроки настанут.

\vs 3Sb 1:464 Не чужеземный Арей твоим, Италия, бедам 

\vs 3Sb 1:465 Будет причиной, но кровь, с которою тяжко бороться, 

\vs 3Sb 1:466 Кровь родных сыновей разорит тебя без пощады! 

\vs 3Sb 1:467 Все об этом позоре узнают, и, в пепле простершись, 

\vs 3Sb 1:468 Ты погибнешь  и раньше могла все это предвидеть: 

\vs 3Sb 1:469 Матерь добрых людей, зверенышей диких вскормила.

\vs 3Sb 1:470 Муж-paзоритель когда придет из Италии новый, 

\vs 3Sb 1:471 Лаодикия, во прах преклонишь тогда ты колени. 

\vs 3Sb 1:472 Славный город Карийский, у струй прекрасного Лика, 

\vs 3Sb 1:473 Горько оплакав надменного предка, навеки умолкнешь.

\vs 3Sb 1:474 Племя Фракийцев взойдет на вершины высокого Гема.

\vs 3Sb 1:475 Лихо придет и к Кампанцам  опустошительный голод; 

\vs 3Sb 1:476 Древний день своего основания также оплачут 

\vs 3Sb 1:477 Кирн и Сардиния, их удары холодного ветра, 

\vs 3Sb 1:478 Посланы Богом святым, в пучину соленую сбросят, 

\vs 3Sb 1:479 Станут они в волнах морским обитальцам добычей.

\vs 3Sb 1:480 Горе! сколько Аид невест прекрасных добудет,

\vs 3Sb 1:481 Сколько непогребенных юнцов не отпустят глубины! 

\vs 3Sb 1:482 О, невинные дети! О, тяжкое злато в пучине!

\vs 3Sb 1:483 Царский возникнет род в земле Мизийцев счастливой.

\vs 3Sb 1:484 Жизнь Кархедона, однако, не долго вовсе продлится. 

\vs 3Sb 1:485 Жалобный стон разнесется среди Галатов, и будет 

\vs 3Sb 1:486 На Тенедосе несчастье последнее самым ужасным. 

\vs 3Sb 1:487 И Сикион, и Коринф возгордятся лаем доспехов, 

\vs 3Sb 1:488 Но не минует их участь вести бесславные войны.

\vs 3Sb 1:489 В сердце утихло моем звучанье божественной песни,

\vs 3Sb 1:490 Но Всемогущий опять вложил мне пророчество в душу 

\vs 3Sb 1:491 И возвестить повелел по всей земле это слово.

\vs 3Sb 1:492 Горе вам всем, Финикийцы, мужчинам и женщинам горе! 

\vs 3Sb 1:493 Также и всем городам Побережья морского  не сможет 

\vs 3Sb 1:494 Светлой дорогой никто из вас до солнца достигнуть.

\vs 3Sb 1:495 Больше семей не возникнет и новых детей не родится  

\vs 3Sb 1:496 За дерзновенный язык и жизнь порочную смертных, 

\vs 3Sb 1:497 Наглых и беззаконных хулителей и святотатцев. 

\vs 3Sb 1:498 Страшные лживые речи вели преступники эти, 

\vs 3Sb 1:499 И против Господа мерзкий мятеж они учиняли;

\vs 3Sb 1:500 Карой за все злодеянья Господь бичевые удары

\vs 3Sb 1:501 Страшно обрушит на смертных от края земли и до края. 

\vs 3Sb 1:502 Горькая ждет их судьба, когда города и жилища 

\vs 3Sb 1:503 До основанья огнем небесным выжжены будут.

\vs 3Sb 1:504 Горе, о горе тебе, печали и скорби обитель, 

\vs 3Sb 1:505 Крит, от удара ты рухнешь ужасного, сгинешь навеки. 

\vs 3Sb 1:506 Ты задымишься тогда на глазах у целого света, 

\vs 3Sb 1:507 И не оставит уж пламя тебя, пока не исчезнешь.

\vs 3Sb 1:508 Фракия, горе тебе  рабынею жалкою станешь: 

\vs 3Sb 1:509 Время настанет, Галаты набег совершат на Элладу 

\vs 3Sb 1:510 Вместе с Дарданцами, тут-то несчастье тебя ожидает  

\vs 3Sb 1:511 Беды несла ты другим, теперь же возмездие примешь.

\vs 3Sb 1:512 Горе вам, Гог и Магог и все племена по соседству  

\vs 3Sb 1:513 Марсы, Анги, иные  вас ждет ужасная участь.

\vs 3Sb 1:514 Много крушений грядут к Ликийцам, Мизийцам, Фригийцам,

\vs 3Sb 1:515 К жителям Лидии и Памфилийцам, а также и к людям 

\vs 3Sb 1:516 Варварской речи  ко всем Эфиопам, Маврам, Арабам, 

\vs 3Sb 1:517 Каппадокийцам. Но что я пророчить каждому стану 

\vs 3Sb 1:518 Жребий его? Ибо всем племенам, населяющим землю, 

\vs 3Sb 1:519 Страшный удар ниспошлет и тяжкую кару Всевышний.

\vs 3Sb 1:520 Варварский, чуждый народ появится в Эллинском крае  

\vs 3Sb 1:521 Многим голов не сносить в то время мужам знаменитым, 

\vs 3Sb 1:522 Много жирных овец у смертных угнано будет, 

\vs 3Sb 1:523 Зычно ревущих быков, коней и мулов без счета. 

\vs 3Sb 1:524 Крепкой постройки дома предав огню беззаконно,

\vs 3Sb 1:525 Жителей в рабство насильно угонят враги на чужбину. 

\vs 3Sb 1:526 Эллин, увидишь картины ужасные: жен беззащитных 

\vs 3Sb 1:527 Вместе с детьми из покоев выбрасывать станут на землю 

\vs 3Sb 1:528 Нежным коленом, и жены в оковах вражьих претерпят 

\vs 3Sb 1:529 Весь позор униженья у варваров. Нет им защиты

\vs 3Sb 1:530 Здесь от расправы, никто им в страшной беде не поможет! 

\vs 3Sb 1:531 Враг все богатство твое и все достоянье присвоит, 

\vs 3Sb 1:532 И затрясутся колени твои, если это увидишь. 

\vs 3Sb 1:533 Сто человек побегут, чтоб спастись, но один всех погубит. 

\vs 3Sb 1:534 Пятеро гневом ужасным тогда вскипят, но позорно

\vs 3Sb 1:535 Между собой препираться начнут и оружье подымут 

\vs 3Sb 1:536 Друг против друга на радость врагам, а Элладе  на горе. 

\vs 3Sb 1:537 В рабстве придется Элладе сносить тяжелое иго; 

\vs 3Sb 1:538 Всех будут мучить война и с нею губительный голод. 

\vs 3Sb 1:539 Небо высокое медью Господь покроет, а землю

\vs 3Sb 1:540 Высушит всю, и в железо поверхность ее превратится. 

\vs 3Sb 1:541 И, убедившись, что больше нельзя ни вспахать, ни посеять, 

\vs 3Sb 1:542 Горько люди заплачут. Но Бог великий, что создал 

\vs 3Sb 1:543 Все на свете, теперь обрушит жестокое пламя 

\vs 3Sb 1:544 Вниз, и тогда лишь треть людей на земле уцелеет.

\vs 3Sb 1:545 О, для чего ты, Эллада, на смертных вождей полагалась? 

\vs 3Sb 1:546 Им не дано ведь никак избегнуть конца рокового; 

\vs 3Sb 1:547 Что ж ублажаешь дарами никчемными тех, кто погибнет, 

\vs 3Sb 1:548 А изваяниям жертвы приносишь? Отколь научилась 

\vs 3Sb 1:549 Делать такое, презрев Лицо всемогущего Бога?

\vs 3Sb 1:550 Имя Родителя общего чти, не оставь в небреженье! 

\vs 3Sb 1:551 Правили тысячу лет и еще пять сотен вдобавок 

\vs 3Sb 1:552 Много надменных царей в Элладе, и вот от них-то 

\vs 3Sb 1:553 Первых учиться злу неразумные смертные стали: 

\vs 3Sb 1:554 Идолов мертвых воздвигли для тех, кто сами невечны;

\vs 3Sb 1:555 Этим в умы вам вложили пустые ложные мысли. 

\vs 3Sb 1:556 Но когда Божий гнев разразится над вами внезапно, 

\vs 3Sb 1:557 Сразу узнаете тут Лицо всемогущего Бога. 

\vs 3Sb 1:558 Тут же все души людские, наполнив воздух стенаньем, 

\vs 3Sb 1:559 Руки к широкому небу с мольбою протягивать станут,

\vs 3Sb 1:560 И о защите молить Царя великого в небе,

\vs 3Sb 1:561 И вопрошать: кто же их от страшного гнева избавит?

\vs 3Sb 1:562 Должен еще ты узнать, и в уме твоем пусть сохранится, 

\vs 3Sb 1:563 Сколько несчастий несут с собою бегущие годы

\vs 3Sb 1:564 Зычно ревущих быков и коров соберет в изобилье 

\vs 3Sb 1:565 К храму великого Бога Эллада, и больше не станет 

\vs 3Sb 1:566 Злобных побоищ на землях ее и гнетущего страха, 

\vs 3Sb 1:567 Голод и рабское иго тогда же вскоре исчезнут. 

\vs 3Sb 1:568 Род нечестивцев, однако, дотоле продлится, покуда 

\vs 3Sb 1:569 Срок судеб истечет и день настанет реченный. 

\vs 3Sb 1:570 Жертвуйте Богу тогда лишь, когда все исполнится, ибо 

\vs 3Sb 1:571 То, чего Он желает, небывшим остаться не может, 

\vs 3Sb 1:572 Все заставит Господь по воле Его совершиться.

\vs 3Sb 1:573 Явится племя святое людей, благочестия полных, 

\vs 3Sb 1:574 Господу истинно верных и волей и помыслом всяким.

\vs 3Sb 1:575 Храм великого Бога почтят кроплением влаги 

\vs 3Sb 1:576 И возжиганием дыма от тучных жертв, гекатомбой 

\vs 3Sb 1:577 Тою священной, когда закалают быков превосходных, 

\vs 3Sb 1:578 Жирных баранов, овец и едва родившихся агнцев; 

\vs 3Sb 1:579 Многое с мыслью благой на алтарь великий возложат.

\vs 3Sb 1:580 И по законам великого Бога, храня справедливость,

\vs 3Sb 1:581 Будут счастливую жизнь проводить в городах и на пашнях. 

\vs 3Sb 1:582 Их Бессмертный возвысит, пророками сделав, и станут 

\vs 3Sb 1:583 Радость большую нести всем смертным, на свете живущим. 

\vs 3Sb 1:584 Только им даровал разуменье благое Всевышний,

\vs 3Sb 1:585 Веру и лучшие чувства людские вложил Он в их души. 

\vs 3Sb 1:586 Делу рук человечьих они молиться не станут, 

\vs 3Sb 1:587 Золото, медь, серебро, слоновья кость не прельстят их, 

\vs 3Sb 1:588 Крашеным идолам шатким из дерева, камня и глины, 

\vs 3Sb 1:589 Изображеньям зверей и всему, что смертных бездумных

\vs 3Sb 1:590 Вводит легко в соблазн, не будут они поклоняться. 

\vs 3Sb 1:591 Но поутру ото сна пробудясь, омывают водою 

\vs 3Sb 1:592 Руки и чистыми их всегда к небесам воздевают. 

\vs 3Sb 1:593 Чтут одного лишь Владыку  Безсмертного, Вечного Бога, 

\vs 3Sb 1:594 Мать и отца вслед за Ним; и больше, чем прочие люди,

\vs 3Sb 1:595 В мысли имеют они сохранять целомудрие ложа, 

\vs 3Sb 1:596 С детями пола мужского не водят позорную дружбу, 

\vs 3Sb 1:597 Как Египтяне и как Финикийцы, а также Латины, 

\vs 3Sb 1:598 Эллины разных племен и множество прочих народов: 

\vs 3Sb 1:599 Персы, Галаты и целая Азия  все, кто забыли

\vs 3Sb 1:600 И преступили священный закон Безсмертного Бога. 

\vs 3Sb 1:601 Людям пошлет Господь за эти грехи наказанье: 

\vs 3Sb 1:602 Пагубу разную, голод, страдания, жалкие стоны, 

\vs 3Sb 1:603 Войны жестокие, мор и слезы от боли ужасной. 

\vs 3Sb 1:604 К вечному ибо Отцу всех тех, кто мир населяет,

\vs 3Sb 1:605 Не пожелали почтенья иметь, а идолов чтили; 

\vs 3Sb 1:606 Будет, однако, пора, и дело рук своих сами 

\vs 3Sb 1:607 Сбросят в расселины гор, дабы скрыть позор величайший. 

\vs 3Sb 1:608 Новый владыка тогда воцарится в Египте, и станет 

\vs 3Sb 1:609 Он седьмым от начала правления Эллинов, то есть

\vs 3Sb 1:610 С той поры, как начнется здесь власть мужей Македонских. 

\vs 3Sb 1:611 Тут горящим орлом великий царь Азиатский 

\vs 3Sb 1:612 Явится, землю покрыв и пешим войском и конным; 

\vs 3Sb 1:613 Все на пути своем уничтожит и злом переполнит. 

\vs 3Sb 1:614 Царская власть сокрушится и Египте тогда, а захватчик,

\vs 3Sb 1:615 Всю добычу забрав, уплывет за широкое море.

\vs 3Sb 1:616 Люди пред Господом Богом, великим и вечным, колена 

\vs 3Sb 1:617 Белые тут преклонят, опустясь на кормилицу-землю. 

\vs 3Sb 1:618 Рухнут и сгинут в пожаре творения рук человечьих; 

\vs 3Sb 1:619 Но получат взамен от Бога великую радость

\vs 3Sb 1:620 Смертные, ибо земля, деревья и пастбища будут

\vs 3Sb 1:621 Истинный плод приносить, и тогда появится вдоволь 

\vs 3Sb 1:622 Сладкого меда, вина, молока белоснежного, хлеба  

\vs 3Sb 1:623 Главное, хлеба, ведь он  наивысшее благо для смертных.

\vs 3Sb 1:624 Только медлить не смей, злонравный и суетный смертный,

\vs 3Sb 1:625 Но обратись покаянно, моли прощенья у Бога! 

\vs 3Sb 1:626 В жертву Ему приноси козлят и ягнят первородных, 

\vs 3Sb 1:627 Сотни быков приноси, пока сменяются годы. 

\vs 3Sb 1:628 Господа ты умоляй низойти и явить Свою милость, 

\vs 3Sb 1:629 Ибо единый Он Бог, и быть другого не может.

\vs 3Sb 1:630 Чти справедливость всегда, никому не делай обиды  

\vs 3Sb 1:631 Это  Безсмертного Бога веление людям несчастным. 

\vs 3Sb 1:632 Но берегись пробужденья всевышнего Божьего гнева 

\vs 3Sb 1:633 В час, как на смертных чума нагрянет, несущая гибель, 

\vs 3Sb 1:634 И не уйдет человек тогда от расплаты ужасной.

\vs 3Sb 1:635 Встанет царь на царя, победит и землю отнимет, 

\vs 3Sb 1:636 Племя на племя пойдет, правители многих погубят, 

\vs 3Sb 1:637 Все вожди разбегутся по разным странам в то время. 

\vs 3Sb 1:638 Облик изменит земля, засилье варваров диких 

\vs 3Sb 1:639 Опустошит всю Элладу, ее плодородная почва

\vs 3Sb 1:640 Всяких лишится богатств; но вражды причиною будут 

\vs 3Sb 1:641 Золото и серебро  готовит многие беды 

\vs 3Sb 1:642 Любостяжательство людям, нет пастыря хуже, чем алчность.

\vs 3Sb 1:643 \ldots\ и на чужбине они останутся без погребенья, 

\vs 3Sb 1:644 Дикие звери и злые стервятники здесь растерзают

\vs 3Sb 1:645 Их тела, а когда реченное все совершится, 

\vs 3Sb 1:646 Этих усопших останки земля широкая скроет. 

\vs 3Sb 1:647 Но уж не вспашет ту землю никто, и никто не засеет; 

\vs 3Sb 1:648 Скажет несчастьем своим о позоре множества смертных \ldots

\vs 3Sb 1:649 В долгой смене времен и годов обращенья не станет 

\vs 3Sb 1:650 Копий, щитов и другого оружья, привычного людям; 

\vs 3Sb 1:651 Чтобы разжечь костры, железом древес не коснутся.

\vs 3Sb 1:652 Бог на землю пошлет царя, что придет от Восхода, 

\vs 3Sb 1:653 Злую войну прекратит этот царь по всей поднебесной, 

\vs 3Sb 1:654 Жизни лишая одних и клятвы другим выполняя. 

\vs 3Sb 1:655 Но не по воле своей совершит он деяния эти, 

\vs 3Sb 1:656 А подчиняясь благим веленьям великого Бога \ldots\

\vs 3Sb 1:657 Свой же народ Господь одарит чудесным богатством  

\vs 3Sb 1:658 Золотом и серебром и прекрасной одеждой пурпурной; 

\vs 3Sb 1:659 И плодородная почва и даже соленое море

\vs 3Sb 1:660 Много благ принесут. Но снова цари друг на друга 

\vs 3Sb 1:661 Примутся зло замышлять и творить его в гневе великом; 

\vs 3Sb 1:662 Будет недобрая зависть в обычае смертных несчастных. 

\vs 3Sb 1:663 Против все той же земли цари поднимут народы, 

\vs 3Sb 1:664 Только в поход соберутся они себе на погибель.

\vs 3Sb 1:665 Храм великого Бога святой и мужей наилучших 

\vs 3Sb 1:666 Всех истребить захотят; и, в землю эту явившись, 

\vs 3Sb 1:667 Город цари окружат и средь войск своих непокорных 

\vs 3Sb 1:668 Сядут на трон и начнут приносить нечестивые жертвы. 

\vs 3Sb 1:669 В этот-то час и раздастся с небес голос Бога могучий

\vs 3Sb 1:670 К диким и глупым народам, и суд начнется над ними, 

\vs 3Sb 1:671 Суд великого Бога, Который бессмертной рукою 

\vs 3Sb 1:672 Их умертвит. С высоты мечи огневые на землю 

\vs 3Sb 1:673 Он обрушит тогда; и огромные факелы будут 

\vs 3Sb 1:674 Всех людей освещать, внезапно средь них появившись.

\vs 3Sb 1:675 И от Господней руки земля, что все порождает, 

\vs 3Sb 1:676 Тут сотрясется, и все затрепещут  рыбы морские, 

\vs 3Sb 1:677 Звери земные и птиц несчетные стаи и воды, 

\vs 3Sb 1:678 Также и души людей содрогнутся, как только увидят 

\vs 3Sb 1:679 Лик Безсмертного Бога  и ужас будет великий.

\vs 3Sb 1:680 Гор высоких вершины, холмы и крутые обрывы 

\vs 3Sb 1:681 Он сокрушит, и черный всем взорам явится Тартар. 

\vs 3Sb 1:682 Полными трупов предстанут ущелья туманные в скалах, 

\vs 3Sb 1:683 Брызнет кровь из камней и вниз потоками хлынет 

\vs 3Sb 1:684 С гор по теснинам и быстро долины собою затопит.

\vs 3Sb 1:685 Рухнут крепкие стены, ведь их неразумные люди 

\vs 3Sb 1:686 Строили, вовсе не зная закона великого Бога, 

\vs 3Sb 1:687 Ни суда, что их ждет. Потому-то резню учинили 

\vs 3Sb 1:688 И в безумии вы на Святыню подняли копья. 

\vs 3Sb 1:689 Будет судить Господь вас всех мечом и войною,

\vs 3Sb 1:690 Пламенем и дождем, затопляющим землю, и серой, 

\vs 3Sb 1:691 С неба летящей, камнями огромными, градом ужасным 

\vs 3Sb 1:692 И умерщвленьем повсюду животных четвероногих. 

\vs 3Sb 1:693 Ясно люди поймут, что суд Безсмертного Бога 

\vs 3Sb 1:694 К ним пришел, и тогда умирающих стоны и вопли

\vs 3Sb 1:695 Землю всю огласят, и, лишаясь речи от страха,

\vs 3Sb 1:696 Кровью умывшись своей, погибнут. И почва впитает 

\vs 3Sb 1:697 Кровь, а тела мертвецов разорвут ненасытные звери.

\vs 3Sb 1:698 Все это мне повелел Господь великий и вечный 

\vs 3Sb 1:699 Так предсказать. И не может реченное мною не сбыться 

\vs 3Sb 1:700 Иль не исполниться в чем-то, ведь все задумано Богом  

\vs 3Sb 1:701 Чуждый обмана, витает Господний Дух в поднебесной.

\vs 3Sb 1:702 Дети великого Бога вокруг Святыни в покое 

\vs 3Sb 1:703 Будут жизнь проводить, Господним деяниям рады, 

\vs 3Sb 1:704 Ибо Творец, справедливый Судья и Владыка, дарует

\vs 3Sb 1:705 Многое: Он на защиту народа встанет, могучий, 

\vs 3Sb 1:706 Словно высокую стену, огонь кругом воздвигая. 

\vs 3Sb 1:707 Ни в городах, ни в селеньях война грозить им не будет; 

\vs 3Sb 1:708 Злую руку вражды отразит святая десница  

\vs 3Sb 1:709 Ведь Безсмертный Боец обороной им станет надежной.

\vs 3Sb 1:710 Скажут все города и все острова, что Всевышний 

\vs 3Sb 1:711 Этих мужей возлюбил великой любовью, и будут 

\vs 3Sb 1:712 И небосвод, и луна, и солнце, водимое Богом, 

\vs 3Sb 1:713 Им помогать во всем и печься о них неустанно. 

\vs 3Sb 1:714 И сотрясется в те дни земля, что все порождает.

\vs 3Sb 1:715 И людские уста воспоют сладкогласные гимны:

\vs 3Sb 1:716 Все к земле припадем, обратимся с молитвою к Богу! 

\vs 3Sb 1:717 Он  Безсмертный Царь, Владыка великий и вечный, 

\vs 3Sb 1:718 Он  наш единый Господь, пошлем же к Господнему храму, 

\vs 3Sb 1:719 Будем все вместе внимать законам Всевышнего Бога,

\vs 3Sb 1:720 Ибо нет ничего справедливей, чем эти законы. 

\vs 3Sb 1:721 Мы заблудились, свернув с дороги, указанной Богом, 

\vs 3Sb 1:722 Стали творения рук человеческих чтить неразумно, 

\vs 3Sb 1:723 Статуям смертных людей деревянным кланяться стали. 

\vs 3Sb 1:724 Веру обретшие души такие крики исторгнут.

\vs 3Sb 1:725 Люди Господа, все падем устами на землю,

\vs 3Sb 1:726 В каждом доме Творцу воспоем прекрасные гимны! 

\vs 3Sb 1:727 Все оружие вражье по миру всему соберем мы. 

\vs 3Sb 1:728 Смогут семь лет совершить свое обращенье по кругу 

\vs 3Sb 1:729 Копья, шлемы, щиты и множество разных доспехов,

\vs 3Sb 1:730 Луки и стрелы. Все это уйдет из рук нечестивцев, 

\vs 3Sb 1:731 Чтобы разжечь костры, железом древес не коснутся.

\vs 3Sb 1:732 Ты же в гордыне своей перестань возноситься, Эллада! 

\vs 3Sb 1:733 Жалкая, остерегись, моли милосердного Бога,

\vs 3Sb 1:734 Свой неразумный народ войной не веди в этот город  

\vs 3Sb 1:735 Пусть спокойно живут и земле великого Бога.

\vs 3Sb 1:736 А Камарину не трогай  ей лучше быть неподвижной, 

\vs 3Sb 1:737 И не буди леопарда, не то может зло приключиться. 

\vs 3Sb 1:738 Будь же воздержна, и пусть в груди твоей не проснется 

\vs 3Sb 1:739 Гордый дух и надменный, стремящийся в жаркую битву. 

\vs 3Sb 1:740 Господу верно служи и радости станешь причастна: 

\vs 3Sb 1:741 Ибо, как срок совершится, отмеренный точно судьбою,

\vs 3Sb 1:742 Суд Безсмертного Бога в тот день настанет для смертных.

\vs 3Sb 1:743 Власть Господня в тот день на добрых людей обратится. 

\vs 3Sb 1:744 Плод наилучший земля, которая все порождает, 

\vs 3Sb 1:745 Смертным даст  изобилье пшеницы, вина и оливок. 

\vs 3Sb 1:746 Множество сладкого меда пошлют небеса человеку, 

\vs 3Sb 1:747 Будет древесных плодов в достатке и тучной скотины: 

\vs 3Sb 1:748 Коз, и коров, и овец с ягнятами малыми вместе. 

\vs 3Sb 1:749 Вырвутся из-под земли молока белоснежного струи.

\vs 3Sb 1:750 Будут полны богатств города, а поля плодоносны;

\vs 3Sb 1:751 Шум боевой и резня ужасная вовсе исчезнут,

\vs 3Sb 1:752 С тяжким стоном земля уж больше не содрогнется,

\vs 3Sb 1:753 Войны и засуха миру угрозою быть перестанут,

\vs 3Sb 1:754 С ними же голод и град, что бьет урожай, упразднятся.

\vs 3Sb 1:755 Мир на землю сойдет великий, неведомый прежде, 

\vs 3Sb 1:756 Станут друзьями теперь цари до скончания века, 

\vs 3Sb 1:757 Люди по всей земле одним жить будут законом, 

\vs 3Sb 1:758 Что установит Господь, на небе правящий звездном; 

\vs 3Sb 1:759 Этим законом Безсмертный дела людские измерит,

\vs 3Sb 1:760 Ибо единый Он Бог, и быть другого не может, 

\vs 3Sb 1:761 И сожжет Он огнем человеческий род нечестивый.

\vs 3Sb 1:762 Так поспешите же, люди, слова мои сердцем усвоить: 

\vs 3Sb 1:763 Идолов мерзких оставьте, живому Богу служите; 

\vs 3Sb 1:764 Остерегайтеся блуда и грязного ложа мужского, 

\vs 3Sb 1:765 Если дети родятся, растите их, не убивайте  

\vs 3Sb 1:766 Все прегрешения эти влекут Божий гнев за собою.

\vs 3Sb 1:767 Бог ниспошлет наконец всем людям вечное царство: 

\vs 3Sb 1:768 Дав священный закон Его почитавшим как должно, 

\vs 3Sb 1:769 Пообещал Он, что мир и землю для них Он откроет 

\vs 3Sb 1:770 И распахнет им врата блаженства  великая радость, 

\vs 3Sb 1:771 Вечно здравый рассудок и мысли светлые станут 

\vs 3Sb 1:772 Их достояньем. Тогда к жилищу великого Бога

\vs 3Sb 1:773 С целого света дары принесут и ладан воскурят. 

\vs 3Sb 1:774 Люди не спросят уже к другому дому дороги,

\vs 3Sb 1:775 Кроме того, что велел Господь почитать Своим верным. 

\vs 3Sb 1:776 Сыном великого Бога те люди его называют.

\vs 3Sb 1:777 Все пути по равнинам и все обрывы крутые, 

\vs 3Sb 1:778 Горные выси и волны, что на море дико бушуют,  

\vs 3Sb 1:779 Станет все в эти дни легко человеку доступным.

\vs 3Sb 1:780 Ибо у добрых людей покой и мир воцарятся, 

\vs 3Sb 1:781 Меч упразднят пророки великого Бога, и сами 

\vs 3Sb 1:782 Смертных станут они судить и царить справедливо. 

\vs 3Sb 1:783 Праведным станет тогда и все богатство людское. 

\vs 3Sb 1:784 Вот каковы будут суд и власть Всевышнего Бога.

\vs 3Sb 1:785 Возвеселись и ликуй, о дева! Вечную радость 

\vs 3Sb 1:786 Тот даровал тебе, Кто создал небо и землю. 

\vs 3Sb 1:787 Он, в тебе поселившись, твоим станет светом безсмертным. 

\vs 3Sb 1:788 Овцы вместе с волками в горах травою питаться 

\vs 3Sb 1:789 Будут, а дикие барсы  пастись с козлятами вместе.

\vs 3Sb 1:790 Сможет теленок с медведем в загоне быть безопасно, 

\vs 3Sb 1:791 Лев плотоядный мякиной, как вол, насытится в яслях; 

\vs 3Sb 1:792 Малые дети его, связав, поведут за собою, 

\vs 3Sb 1:793 Зверь этот станет ручным по воле великого Бога. 

\vs 3Sb 1:794 Змей с младенцем уснет в одной постели спокойно

\vs 3Sb 1:795 И не сделает зла  Господня рука не позволит.

\vs 3Sb 1:796 Ясное знаменье я укажу тебе, чтобы узнал ты

\vs 3Sb 1:797 Время, когда конец всему земному настанет.

\vs 3Sb 1:798 Звездный свод озарять мечи огромные будут 

\vs 3Sb 1:799 В небе встанут они с Востока и с Запада ночью.

\vs 3Sb 1:800 Пепел внезапно и пыль посыплются сверху на землю 

\vs 3Sb 1:801 В целом мире, и днем угаснет солнца сиянье, 

\vs 3Sb 1:802 Вместо того луна в небесах появится тотчас, 

\vs 3Sb 1:803 Бледным лучом своим осветив земную поверхность. 

\vs 3Sb 1:804 Кровь, просочившись из камня, вам тоже даст несомненный

\vs 3Sb 1:805 Знак, а в тумане предстанет сражение пеших и конных, 

\vs 3Sb 1:806 Схожее с травлей зверей и во мгле виденью подобно. 

\vs 3Sb 1:807 Значит, вскоре Господь, живущий в небе, положит 

\vs 3Sb 1:808 Войнам конец. Но каждый пусть жертвы Богу приносит.

\vs 3Sb 1:809 Из Ассирийской земли, от мощных стен Вавилона 

\vs 3Sb 1:810 Я в Элладу явилась, как некий огонь, в исступленье,

\vs 3Sb 1:811 Чтобы всем смертным изречь, чем Бог угрожает во гневе \ldots\

\vs 3Sb 1:812 Смертным все предскажу в загадках, внушенных мне Богом.

\vs 3Sb 1:813 Станут тогда говорить в Элладе, что я  чужеземка, 

\vs 3Sb 1:814 Что родилась я в Эритрах, безстыдная; и нарекут мне

\vs 3Sb 1:815 Имя Сивиллы и скажут, как будто Гностом и Киркой 

\vs 3Sb 1:816 Мать и отца моих звали, а я  безумная лгунья. 

\vs 3Sb 1:817 Но когда все свершится, слова мои вспомните сразу 

\vs 3Sb 1:818 И не безумной сочтете  великой пророчицей Бога. 

\vs 3Sb 1:819 Мне Господь не открыл того, что родителям прежде

\vs 3Sb 1:820 Он поведал моим, но то, что было вначале, 

\vs 3Sb 1:821 И грядущее все вложил мне в душу Всевышний, 

\vs 3Sb 1:822 Чтобы пророчить могла я о бывшем и будущем людям. 

\vs 3Sb 1:823 Ибо, покрыли когда всю землю воды Потопа, 

\vs 3Sb 1:824 Славный муж лишь один в то время спасся от смерти:

\vs 3Sb 1:825 Дом деревянный построив, по водам проплыл и с собою 

\vs 3Sb 1:826 Взял он птиц и зверей, чтобы мир наполнился снова. 

\vs 3Sb 1:827 Мне же стать довелось невесткой этого мужа; 

\vs 3Sb 1:828 Первое с ним совершилось, ему же последнее ясным 

\vs 3Sb 1:829 Сделал Господь  и во всем уста мои будут правдивы.

\bibbookdescr{4Sb}{
  inline={Четвёртая книга Сивилл},
  toc={4-я Сивилл},
  bookmark={4-я Сивилл},
  header={4-я Сивилл},
  abbr={4~Сив}
}
\vs 4Sb 1:1 Слушай, Азийский народ надменный и Европейцы, 

\vs 4Sb 1:2 Все, что намерена я правдиво вам напророчить, 

\vs 4Sb 1:3 Мощные звуки издав из широкоотверстого горла! 

\vs 4Sb 1:4 И не от лживого Феба, которого глупые люди

\vs 4Sb 1:5 Богом назвали, ему приписав, что будто пророк он, 

\vs 4Sb 1:6 Стану вещать, но послушна желанию вечного Бога, 

\vs 4Sb 1:7 Руки кого не слепили людские, подобно тому как 

\vs 4Sb 1:8 Идолов лепят немых и из камня их высекают. 

\vs 4Sb 1:9 В камень не заключен, безмолвно в храме стоящий  

\vs 4Sb 1:10 Глух ко всему, позор жалчайший для рода людского, 

\vs 4Sb 1:11 Бог не видим с земли, глазами Его не измерить,

\vs 4Sb 1:12 Теми, что смертным даны, рукою смертной не создан, 

\vs 4Sb 1:13 Разом всех видя с небес, никем быть увиден не может. 

\vs 4Sb 1:14 Ночь, несущую мрак, светлый день и яркое солнце, 

\vs 4Sb 1:15 Звезды вместе с луной, кишащее рыбами море,

\vs 4Sb 1:16 Землю, реки и устья источников вечнотекущих  

\vs 4Sb 1:17 Все сотворил Он для жизни; дожди, что прольются над пашней,

\vs 4Sb 1:18 Ей обещав урожай, дав деревья, лозу и оливу. 

\vs 4Sb 1:19 Он меня в грудь поразил, бичом полоснул мне по сердцу, 

\vs 4Sb 1:20 Чтобы я людям про то, что есть теперь и что будет

\vs 4Sb 1:21 С ними, от первого рода начав и окончив десятым, 

\vs 4Sb 1:22 Все достоверно сказала. Ведь Тот мне это доверил, 

\vs 4Sb 1:23 Сам Кто причина всему. Народ, послушай Сивиллу, 

\vs 4Sb 1:24 Льющую голос правдивый из уст, благочестия полных!

\vs 4Sb 1:25 Те из людей на земле изведают счастье, что Бога 

\vs 4Sb 1:26 Будут великого чтить и Его прославлять непрестанно, 

\vs 4Sb 1:27 Прежде еды и питья стремясь к благочестию в жизни. 

\vs 4Sb 1:28 Храмы отвергнут они все сразу, лишь только увидят;

\vs 4Sb 1:29 То же и алтари  постройки из мертвого камня 

\vs 4Sb 1:30 Лживых во славу богов, оскверненные кровью животных, 

\vs 4Sb 1:31 Жертвенным дымом. Их люди во имя единого Бога 

\vs 4Sb 1:32 Все забросают камнями, запрет положив на убийство, 

\vs 4Sb 1:33 Тайную мзду не приняв, что стало бы худшим началом. 

\vs 4Sb 1:34 Также не будут искать утех на чужом они ложе, 

\vs 4Sb 1:35 Дерзость мужей им чужда и всегда ненавистна пребудет.

\vs 4Sb 1:36 Не переймут никогда тот характер, нрав и обычай 

\vs 4Sb 1:37 Прочие люди. Они, во всем тяготея к безстыдству, 

\vs 4Sb 1:38 Горло в насмешках сорвав, над этими станут смеяться  

\vs 4Sb 1:39 Глупые дети!  и первым наивно приписывать станут 

\vs 4Sb 1:40 То беззаконье и зло, которое сами свершили.

\vs 4Sb 1:41 Полон неверия род людской. Когда же наступит

\vs 4Sb 1:42 Суд над людьми и всем миром, что Сам вселенной Создатель 

\vs 4Sb 1:43 Будет вершить, на него нечестивых и праведных вместе 

\vs 4Sb 1:44 Вызвав,  Он их разведет: порочных в пламень отправит, 

\vs 4Sb 1:45 В мрак преисподней, чтоб там осознали они, что творили;

\vs 4Sb 1:46 Праведным выпадет жить на равнине, обильной плодами, 

\vs 4Sb 1:47 Вместе с дыханьем Господь им жизнь и радость дарует. 

\vs 4Sb 1:48 Это случиться должно при людях в десятом колене, 

\vs 4Sb 1:49 Что же в первом их ждет и дальше  о том расскажу я.

\vs 4Sb 1:50 Править всеми людьми Ассирийцы будут вначале, 

\vs 4Sb 1:51 Власть над миром держа в пределах шести поколений,  

\vs 4Sb 1:52 После чего Божий гнев на них обрушится с неба, 

\vs 4Sb 1:53 На города и людей, живущих под тем небосводом. 

\vs 4Sb 1:54 Морем тут станет земля из-за вод, что внезапно нахлынут.

\vs 4Sb 1:55 Свергнут Мидийцы их власть и сами возсядут на троны  

\vs 4Sb 1:56 Два поколенья всего будут править. При них совершится 

\vs 4Sb 1:57 Вот что: наступит вдруг ночь среди дня и землю накроет, 

\vs 4Sb 1:58 Звезды с небес пропадут, и Луны круг тоже исчезнет; 

\vs 4Sb 1:59 Почва, вся сотрясаясь от мощных подземных ударов, 

\vs 4Sb 1:60 Много сметет городов и того, что построили люди,  

\vs 4Sb 1:61 Из глубины же морской острова всплывут на поверхность.

\vs 4Sb 1:62 Но когда разольется Евфрат великий от крови, 

\vs 4Sb 1:63 Страшная битва случится тут между Мидийцев и Персов 

\vs 4Sb 1:64 В их друг с другом войне. Под копьями Персов Мидийцы,

\vs 4Sb 1:65 Падая, прочь побегут через воды великого Тигра. 

\vs 4Sb 1:66 Сила же Персов пускай величайшею в мире пребудет, 

\vs 4Sb 1:67 Им предстоит лишь одно поколение счастливо править.

\vs 4Sb 1:68 Многие беды ждут мир, их осыплют проклятьями люди: 

\vs 4Sb 1:69 Кровопролитные войны, убийства, изгнания, распри, 

\vs 4Sb 1:70 Гибель больших городов, падение башен высоких  

\vs 4Sb 1:71 Эллин надменный когда по соленой волне Геллеспонта, 

\vs 4Sb 1:72 Смерть неся Финикийцам и Азии, путь свой направит.

\vs 4Sb 1:73 В хлебном Египте, где вся земля распахана плугом, 

\vs 4Sb 1:74 Голод и неурожай на двадцать лет воцарятся. 

\vs 4Sb 1:75 Нил тому будет причиной, колосьям жизнь приносящий,  

\vs 4Sb 1:76 Темные воды свои упрячет он где-то под землю.

\vs 4Sb 1:77 Будет из Азии царь, что копье большое поднимет, 

\vs 4Sb 1:78 На несметных судах. По влажным дорогам пучины 

\vs 4Sb 1:79 Шагом пройдет, проплывет, разсекши высокую гору. 

\vs 4Sb 1:80 После же бегства с войны его грозная Азия примет.

\vs 4Sb 1:81 Остров Сицилия весь сожжен будет мощным потоком 

\vs 4Sb 1:82 Лавы кипящей, из недр что извергнет с пламенем Этна. 

\vs 4Sb 1:83 Город же славный Кротон погрузится в глубокое море.

\vs 4Sb 1:84 Вспыхнет в Элладе вражда. Совсем обезумев от гнева, 

\vs 4Sb 1:85 Много с землей городов сровняют и многих погубят 

\vs 4Sb 1:86 В битве жестокой. Война принесет всем поровну горя.

\vs 4Sb 1:87 В роде когда же людском поколений сменится десять, 

\vs 4Sb 1:88 Персов рабский ярем тогда ожидает и ужас.

\vs 4Sb 1:89 Слава правителей мира когда отойдет к Македонцам,

\vs 4Sb 1:90 Фивам не избежать позорного будет захвата, 

\vs 4Sb 1:91 Тир населят Карийцы, а жители Тира погибнут.

\vs 4Sb 1:92 Самос засыплет песком, сровняет его с берегами.

\vs 4Sb 1:93 Делос исчезнет из глаз, и все, что на Делосе, тоже.

\vs 4Sb 1:94 Грозный на вид Вавилон, однако слабый в сраженье,

\vs 4Sb 1:95 Будет стоять, возведен на надеждах, не могущих сбыться. 

\vs 4Sb 1:96 Бактры займут Македонцы; их жители, город оставив,

\vs 4Sb 1:97 Так же, как жители Суз, все ринутся в землю Эллады.

\vs 4Sb 1:98 Все это в будущем ждет, когда Пирам среброструйный, 

\vs 4Sb 1:99 Воду в залив вынося, священный остров омоет. 

\vs 4Sb 1:100 В море сползут Сибарис и Кизик, от колебаний 

\vs 4Sb 1:101 Почвы; оба падут под напором подземных ударов. 

\vs 4Sb 1:102 Родос последним постигнет несчастье, но будет сильнейшим.

\vs 4Sb 1:103 Власть Македонцев продлится недолго: там, где заходит 

\vs 4Sb 1:104 Солнце, начнется война Италийская, ей подчинятся 

\vs 4Sb 1:105 Все и под рабским ярмом Италийцам прислуживать станут. 

\vs 4Sb 1:106 Ты же, несчастный Коринф, свое разоренье увидишь. 

\vs 4Sb 1:107 Башни твои, Кархедон, к земле преклонят колено.

\vs 4Sb 1:108 Стойкая Лаодикия, тебя опрокинет однажды 

\vs 4Sb 1:109 Землетрясенье, но ты поднимешься вновь из развалин. 

\vs 4Sb 1:110 О Ликийские Миры, краса городов! Никогда вас 

\vs 4Sb 1:111 Прочно земля не удержит, сама сотрясаясь. В паденье 

\vs 4Sb 1:112 Ниже и ниже клонясь, вы другую страну изберете, 

\vs 4Sb 1:113 Чтобы в ней жизнь продолжать  настоящий метек, а не город.

\vs 4Sb 1:114 Из-за нечестья тогда же затихнет и город Патары, 

\vs 4Sb 1:115 Море его поглотит при землетрясенье и буре.

\vs 4Sb 1:116 Также тебе предстоит, Армения, рабская участь.

\vs 4Sb 1:117 Грозной войны ураган домчит до Иерусалима,

\vs 4Sb 1:118 Путь свой начав с Апеннин, и Храм великий разрушит. 

\vs 4Sb 1:119 Тут, безрассудству отдавшись, когда благочестье отринут 

\vs 4Sb 1:120 И в преддверии храма творить будут жуткую бойню,  

\vs 4Sb 1:121 Царь великий тогда из Италии, словно разбойник,

\vs 4Sb 1:122 Пустится в бегство, невидим, неслышим, за воды Евфрата. 

\vs 4Sb 1:123 После того, как он грех величайший  матери гибель  

\vs 4Sb 1:124 Не побоится принять, и другие свершит преступленья. 

\vs 4Sb 1:125 Многие кровью зальют подножие Римского трона 

\vs 4Sb 1:126 Сразу, как тот убежит через земли Парфянского царства.

\vs 4Sb 1:127 В Сирию воин из Рима придет. Он, Иерусалимский 

\vs 4Sb 1:128 Храм предоставив огню и многих убив Иудеев, 

\vs 4Sb 1:129 Их великую землю, дорогами славную, сгубит.

\vs 4Sb 1:130 Пафос и Саламин уничтожит землетрясенье,

\vs 4Sb 1:131 Кипр, омываем полной, когда черная скроет пучина.

\vs 4Sb 1:132 В час, когда, из глубин разверстой земли Италийской 

\vs 4Sb 1:133 Вырвавшись, огненный столб до широкого неба достанет, 

\vs 4Sb 1:134 Много тут городов он сожжет и многих погубит. 

\vs 4Sb 1:135 Тучи горящего пепла весь воздух собою заполнят, 

\vs 4Sb 1:136 С неба частички его будут падать как красная краска.

\vs 4Sb 1:137 В этом увидеть должны явление Божьего гнева, 

\vs 4Sb 1:138 Благочестивое племя поскольку безвинно страдает. 

\vs 4Sb 1:139 Повод для новой войны появится скоро: на Запад 

\vs 4Sb 1:140 Явится тот, кто бежал из Рима; копье он поднимет, 

\vs 4Sb 1:141 Снова Евфрат перейдя, приведет несметное войско.

\vs 4Sb 1:142 Бедная Антиохия! Ты городом зваться не станешь 

\vs 4Sb 1:143 После того, как падешь под копьями по безразсудству. 

\vs 4Sb 1:144 Голод погубит тогда Киприотов и страшная битва.

\vs 4Sb 1:145 Остров несчастный, о Кипр! Увы тебе! Волны морские 

\vs 4Sb 1:146 Скроют тебя под собой  добычу неистовой бури.

\vs 4Sb 1:147 В Азию груз драгоценный прибудет, что некогда Римом 

\vs 4Sb 1:148 Был добыт на войне и в городе этом хранился. 

\vs 4Sb 1:149 Дважды по столько затем придется еще им отправить 

\vs 4Sb 1:150 Азии в качестве платы за все неудачи в сраженьях.

\vs 4Sb 1:151 Карии все города, что лежат по теченью Меандра  

\vs 4Sb 1:152 Стенами окружены, прекрасны,  свирепый погубит 

\vs 4Sb 1:153 Голод, сокроет когда Меандр свою черную воду.

\vs 4Sb 1:154 Стоит в сердцах человечьих изсякнуть почтению к Богу,

\vs 4Sb 1:155 Вере и праву навеки из мира стоит исчезнуть,

\vs 4Sb 1:156 Как, нетвердые духом в дерзаньях своих нечестивых,

\vs 4Sb 1:157 Люди станут вершить произвол и творить злодеянья. 

\vs 4Sb 1:158 С благочестивым никто к беседе стремиться не будет, 

\vs 4Sb 1:159 Их же, напротив, самих истребят глупцы и безумцы, 

\vs 4Sb 1:160 Наглости собственной рады, с руками, покрытыми кровью. 

\vs 4Sb 1:161 Тут им придется узнать, что милостив дольше не будет

\vs 4Sb 1:162 Бог, но, безудержный в гневе, намерен род погубить их  

\vs 4Sb 1:163 Так, чтобы весь он сгорел во время большого пожара.

\vs 4Sb 1:164 Образ мыслей смените, пустые люди, и кару

\vs 4Sb 1:165 Не вынуждайте Его для вас выбирать. Отказавшись

\vs 4Sb 1:166 От мечей и убийств, от стонов и беззаконья, 

\vs 4Sb 1:167 В реках вечнотекущих омойте все свое тело.

\vs 4Sb 1:168 Руки воздев к небесам, к тому, что прежде свершили, 

\vs 4Sb 1:169 О снисхожденье просите и, Богу хвалу воздавая, 

\vs 4Sb 1:170 Милость Его призывайте к себе, нечестивым. Дарует 

\vs 4Sb 1:171 Он прощение всем, не погубит  снова утихнет 

\vs 4Sb 1:172 Гнев, если в душах своих воспитаете вы благочестье. 

\vs 4Sb 1:173 Если ж не верите мне и нечестие вашему сердцу, 

\vs 4Sb 1:174 Глупые люди, дороже всего, а речи  впустую, 

\vs 4Sb 1:175 Пламя охватит тогда весь мир и знак величайший 

\vs 4Sb 1:176 Меч подаст и труба на восходе дневного светила. 

\vs 4Sb 1:177 Глас тот мощный и рев услышат во всей поднебесной. 

\vs 4Sb 1:178 Выжжена будет земля, человеческий род уничтожен, 

\vs 4Sb 1:179 Вместе же с ним города, пресноводные реки и море. 

\vs 4Sb 1:180 Пеплом все станет, и прах раскаленный ляжет повсюду.

\vs 4Sb 1:181 Но когда, кроме золы, ничего уже в мире не будет, 

\vs 4Sb 1:182 Бог успокоит огонь несказанный, как некогда вызвал.

\vs 4Sb 1:183 Пепел и кости людские вновь Сам соберет и придаст им

\vs 4Sb 1:184 Прежнюю форму. Так род Он смертных людей возстановит.

\vs 4Sb 1:185 После того будет Суд, и Сам Он вершить его станет,

\vs 4Sb 1:186 Мир к ответу призвав: тут тех, кто, живя нечестиво, 

\vs 4Sb 1:187 Истинной веры не знал, земляная толща накроет

\vs 4Sb 1:188 Душного Тартара, пропасть поглотит ужасной геенны.

\vs 4Sb 1:189 Людям же праведным вновь разрешит на земле поселиться,

\vs 4Sb 1:190 Вместе с дыханием жизнь Господь им и радость дарует. 

\vs 4Sb 1:191 Все они тотчас себя увидят при благостном свете

\vs 4Sb 1:192 Солнца, которое впредь уходить с небосвода не будет.

\vs 4Sb 1:193 Счастлив тот человек, кому жить в это время придется.

\bibbookdescr{5Sb}{
  inline={Пятая книга Сивилл},
  toc={5-я Сивилл},
  bookmark={5-я Сивилл},
  header={5-я Сивилл},
  abbr={5~Сив}
}
\vs 5Sb 1:1 Слушай, что я расскажу о горестном веке Латинян: 

\vs 5Sb 1:2 Прежде всего, как умрут владыки Египта, которых 

\vs 5Sb 1:3 Всех, одного за другим, земля забрала равнодушно; 

\vs 5Sb 1:4 После рожденного в Пелле, которому под ноги пали

\vs 5Sb 1:5 Все Восточные страны и Запад, безмерно богатый, 

\vs 5Sb 1:6 Кто, посрамлен Вавилоном, был мертвым отправлен к Филиппу,

\vs 5Sb 1:7 Сыном Аммона и Зевса напрасно кого называли; 

\vs 5Sb 1:8 Также и после того, кто плоть и кровь Ассарака, 

\vs 5Sb 1:9 Кто из-под Трои бежал, пройдя сквозь пламени стены;

\vs 5Sb 1:10 После ряда царей  мужей, возлюбивших сраженья, 

\vs 5Sb 1:11 После младенцев, рожденных от зверя, губителя стада, 

\vs 5Sb 1:12 Будет первый властитель, который буквой начальной 

\vs 5Sb 1:13 Увенчает двадцатку. Не знать ему равного в битвах, 

\vs 5Sb 1:14 Имя его начинаться с десятки будет. За этим

\vs 5Sb 1:15 Тот станет править, чей знак  начало всего алфавита. 

\vs 5Sb 1:16 Робость пред ним ощутят Сицилия, Фракия, Мемфис  

\vs 5Sb 1:17 Мемфис, поверженный в прах виною своих полководцев, 

\vs 5Sb 1:18 Из-за упрямства жены, что бросится в волны морские,  

\vs 5Sb 1:19 Он установит народам законы и всех подчинит их. 

\vs 5Sb 1:20 Времени много пройдет  другому власть он оставит, 

\vs 5Sb 1:21 Будет который иметь знак триста на первую букву. 

\vs 5Sb 1:22 Даст ему имя река. Владыкой Персов считаться 

\vs 5Sb 1:23 Станет он и Вавилона, на копья насадит Мидийцев. 

\vs 5Sb 1:24 После же тот примет власть, чьим знаком выпала тройка.

\vs 5Sb 1:25 Царь будет править за ним, которому знак дважды десять. 

\vs 5Sb 1:26 Он доберется до самых окраинных вод Океана 

\vs 5Sb 1:27 И к берегам Авзонийским едва до отлива поспеет. 

\vs 5Sb 1:28 Тот, кому знак пятьдесят назначен судьбой, государем

\vs 5Sb 1:29 Станет, чудовищный змей, войною дышащий тяжкой, 

\vs 5Sb 1:30 Руку который па мать поднимет и смуту посеет,

\vs 5Sb 1:31 Сам нападая, гоня, убивая, творя беззаконье.

\vs 5Sb 1:32 Гору, что между двух волн, разсечет и забрызгает грязью.

\vs 5Sb 1:33 После же смерти исчезнет. Затем вернется обратно,

\vs 5Sb 1:34 С Богом равняясь. Но скоро покажет, что вовсе не Бог он. 

\vs 5Sb 1:35 Трое царей вслед за ним один другого погубят.

\vs 5Sb 1:36 Благочестивых убийца тогда станет править, могучий.

\vs 5Sb 1:37 Семь раз по десять  свой знак  он явит миру. Отнимет

\vs 5Sb 1:38 Власть у него его сын, чьим знаком будут три сотни.

\vs 5Sb 1:39 После судьбой решено быть царю  по знаку четверке. 

\vs 5Sb 1:40 Вслед за этим придет старик, чье число пять десятков.

\vs 5Sb 1:41 Тот же, кто после него воцарится на троне, от века

\vs 5Sb 1:42 Имени первую букву значением триста имеет.

\vs 5Sb 1:43 Горы ногою поправ, спеша на Восточную битву,

\vs 5Sb 1:44 Кельт, он безславную смерть найдет в пути от болезни. 

\vs 5Sb 1:45 Примет мертвого пыль чужеземная; имя Немейский

\vs 5Sb 1:46 Дал ей цветок. А затем  другой, в серебряном шлеме,

\vs 5Sb 1:47 Станет у власти. Ему свое имя море подарит.

\vs 5Sb 1:48 Будет он доблестный муж, с умом, проникающим всюду.

\vs 5Sb 1:49 Так, при тебе, наилучший, что всех превзошел, темнокудрый, 

\vs 5Sb 1:50 И при потомках твоих придет, наконец, это время.

\vs 5Sb 1:51 После него  три царя, из которых третий  не скоро.

\vs 5Sb 1:52 Трижды несчастна, терзаюсь: в груди недоброе слово 

\vs 5Sb 1:53 Давит, Изиды сестра томится пророческой песнью.

\vs 5Sb 1:54 Первым вокруг твоего многослезного храма, Египет,

\vs 5Sb 1:55 Вихрем помчатся менады, и ты попадешь в злые руки

\vs 5Sb 1:56 В тот самый день, когда Нил понесет свои воды однажды

\vs 5Sb 1:57 Через страну Египтян на шестнадцать локтей полноводней,

\vs 5Sb 1:58 Так что омоет всю землю, пустив по ней литься потоки.

\vs 5Sb 1:59 Смолкнет тут радостный смех, чело земли омрачится.

\vs 5Sb 1:60 Мемфис! Ты горше других Египта беды оплачешь,

\vs 5Sb 1:61 Ибо, над всею землей до сих пор величаво царивший,

\vs 5Sb 1:62 Станешь чертогом печали. Тогда призовет тебя с неба

\vs 5Sb 1:63 Зычно сам Повелитель перунов: О, Мемфис могучий,

\vs 5Sb 1:64 Раньше пред жалким народом кичился своею ты славой,

\vs 5Sb 1:65 Ныне в несчастье и скорби заплачешь: придет постиженье

\vs 5Sb 1:66 Вечного Бога к тебе, Безсмертного, сущего в небе. 

\vs 5Sb 1:67 Где теперь воля твоя, что судьбы людские вершила? 

\vs 5Sb 1:68 В диком безумстве детей ты моих, помазанных Богом, 

\vs 5Sb 1:69 Тяжким гоненьям подверг и праведным зло уготовил.

\vs 5Sb 1:70 Будешь за эти дела наказан. Мачеха злая

\vs 5Sb 1:71 Станет уделом твоим, и вовек тебе счастья не видеть: 

\vs 5Sb 1:72 С неба скатившись звездой, обратно не сможешь подняться.

\vs 5Sb 1:73 Вот что Господь мне внушил правдиво Египту поведать 

\vs 5Sb 1:74 В самом исходе времен, когда люди погрязнут в пороках.

\vs 5Sb 1:75 Но продолжают страдать, нечестивцы, в преддверии кары  

\vs 5Sb 1:76 Гнева безсмертного Бога, что тяжко гремит, Небожитель. 

\vs 5Sb 1:77 Вместо Него почитают чудовищ и камни, повсюду 

\vs 5Sb 1:78 Видят священного страха предметы, в которых ни смысла 

\vs 5Sb 1:79 Нет, ни рассудка, ни слуха  о них говорить не пристало

\vs 5Sb 1:80 Мне, называя божков  творения рук человека. 

\vs 5Sb 1:81 Взявшись сами за труд воплотить нечестивые мысли, 

\vs 5Sb 1:82 Люди богов сотворили и каменных, и деревянных, 

\vs 5Sb 1:83 Медных и золотых, серебряных  в коих ни пользы 

\vs 5Sb 1:84 Нет, ни души, глухих, на огне из металла отлитых.

\vs 5Sb 1:85 Сделав же их для себя, впустую на них уповали: 

\vs 5Sb 1:86 Тмуис и Ксуис в беде, конец приходит засилью 

\vs 5Sb 1:87 Зевса, Геракла, Гермеса \ldots

\vs 5Sb 1:88 Славная меть городов, ты тоже, Александрия, 

\vs 5Sb 1:89 Жертвой войны упадешь и то, чем прежде владела,

\vs 5Sb 1:90 Все до остатка отдашь в наказанье за дерзкий характер. 

\vs 5Sb 1:91 Долгим молчание будет, но радостный день возвращенья 

\vs 5Sb 1:92 Больше тебе не нальет напиток нежный \ldots

\vs 5Sb 1:93 Перс наводнит твою землю, подобен жестокому граду, 

\vs 5Sb 1:94 Смерть и разруху неся, людей злонравных погубит.

\vs 5Sb 1:95 Кровью зальет алтари, завалит телами убитых 

\vs 5Sb 1:96 Варвар могучий, свершит он другие безумства, как эти, 

\vs 5Sb 1:97 Словно песчаная буря, замыслив конец твой ускорить. 

\vs 5Sb 1:98 Город счастливый, тогда претерпишь ты многие беды! 

\vs 5Sb 1:99 Вся будет Азия плакать, дары вспоминая, какими

\vs 5Sb 1:100 Голову ты ей венчала  теперь она тоже погибнет. 

\vs 5Sb 1:101 Новый Персидский владыка подвергнет страну разоренью, 

\vs 5Sb 1:102 Всякий им будет убит, и жизнь в тех местах прекратится. 

\vs 5Sb 1:103 Третья лишь часть уцелеет от жалкого племени смертных.

\vs 5Sb 1:104 Он же тут легким прыжком помчится на крыльях к Востоку, 

\vs 5Sb 1:105 Мучая землю войной, в пустыню ее превращая. 

\vs 5Sb 1:106 Власти на гребне своей, хотя и терзаемый страхом, 

\vs 5Sb 1:107 К городу праведных он подойдет, желая разрушить. 

\vs 5Sb 1:108 Посланный Богом тогда некий царь на него ополчится, 

\vs 5Sb 1:109 Что всех великих царей погубит и воинов лучших. 

\vs 5Sb 1:110 Так над людьми приговор исполнит Безсмертный Владыка.

\vs 5Sb 1:111 Гадкое сердце! Зачем ты меня подстрекаешь на это  

\vs 5Sb 1:112 Многих царей предсказать Египту ужасное царство? 

\vs 5Sb 1:113 Лучше вернись на Восток, к потомкам несмысленных

\vs 5Sb 1:114 Персов, Им покажи все как есть, и то, что еще ожидает.

\vs 5Sb 1:115 Воды Евфрата, разлившись, затопят окрестные земли, 

\vs 5Sb 1:116 Персов погубят они, Иберов и Вавилонян, 

\vs 5Sb 1:117 И Массагетов, войну ведущих при помощи луков. 

\vs 5Sb 1:118 Азия до Островов все блеском пожаров осветит. 

\vs 5Sb 1:119 Некогда великолепный, Пергам совсем опустеет.

\vs 5Sb 1:120 Так же, как он, и Питана предстанет безлюдной пустыней. 

\vs 5Sb 1:121 Лесбос опустится весь в пучину бездонную моря. 

\vs 5Sb 1:122 Смирна, с крутых берегов скользнув, заплачет однажды  

\vs 5Sb 1:123 Та, что была столь горда и известна, безславно погибнет. 

\vs 5Sb 1:124 Землю, что стала золой, слезами Вифинцы омоют,

\vs 5Sb 1:125 Сирию всю целиком, многолюдную с ней Финикию. 

\vs 5Sb 1:126 Ликия, горе тебе  столько бед для тебя замышляет 

\vs 5Sb 1:127 Море: однажды само на несчастную землю нахлынув, 

\vs 5Sb 1:128 Скроет в соленых волнах, при страшных подземных ударах 

\vs 5Sb 1:129 Берег Ликийский, где миро растет и где нет его вовсе.

\vs 5Sb 1:130 Гнев на Фригийцев падет ужасный из-за печали, 

\vs 5Sb 1:131 Ради которой пришла сюда Рея и здесь поселилась. 

\vs 5Sb 1:132 Море Таврский народ уничтожит и варваров племя, 

\vs 5Sb 1:133 А Эпидана поток по земле разметает Лапифов, 

\vs 5Sb 1:134 Водовороты крутя, Фессалийскую область погубит.

\vs 5Sb 1:135 Глубоководный Пеней увлечет за собою животных, 

\vs 5Sb 1:136 Тех, что когда-то родил Эпидан, как люди считают.

\vs 5Sb 1:137 Трижды несчастной Эллады поэты участь оплачут, 

\vs 5Sb 1:138 Царь Италийский когда перебьет сухожилие Истма, 

\vs 5Sb 1:139 Богу подобный, могучий, великого Рима властитель.

\vs 5Sb 1:140 Сам его Зевс, говорят, породил и владычица Гера. 

\vs 5Sb 1:141 Кто при стеченье народа поет сладкозвучные гимны

\vs 5Sb 1:142 Голосом нежным, убьет свою мать и многих несчастных.

\vs 5Sb 1:143 Вождь трусливый и наглый, бежит от стен Вавилона 

\vs 5Sb 1:144 Тот, кого среди смертных сильнейшие даже боятся; 

\vs 5Sb 1:145 Многих он жизни лишил, не щадил и матери чрева,

\vs 5Sb 1:146 Грязной любви предавался, вместилище всяких пороков.

\vs 5Sb 1:147 Путь свой к Мидийцам направит и грозным правителям Персов 

\vs 5Sb 1:148 Их он всех раньше призвал и славу им уготовил,

\vs 5Sb 1:149 На неугодный народ замышляя с толпой нечестивцев. 

\vs 5Sb 1:150 Богом поставленный храм захватил он, сжег безпощадно

\vs 5Sb 1:151 Тех, что входили в него, кого я по заслугам воспела.

\vs 5Sb 1:152 В храм он лишь только вступил, как здание все содрогнулось,

\vs 5Sb 1:153 Гибли повсюду цари, а те, кто остался у власти,

\vs 5Sb 1:154 Город великий сгубили с народом праведным вместе.

\vs 5Sb 1:155 В год же четвертый, когда звезда засияет большая, 

\vs 5Sb 1:156 Землю которая всю уничтожит одна ради мести,

\vs 5Sb 1:157 \ldots

\vs 5Sb 1:158 С неба большая звезда упадет в соленые воды, 

\vs 5Sb 1:159 Море она подожжет и с ним Вавилона твердыни, 

\vs 5Sb 1:160 Землю Италии, много виною которой погибло 

\vs 5Sb 1:161 Благочестивых Евреев, угодного Богу народа.

\vs 5Sb 1:162 Между порочных мужей ты себя запятнаешь пороком, 

\vs 5Sb 1:163 Целую вечность потом простоишь совсем опустелым,

\vs 5Sb 1:164 \ldots

\vs 5Sb 1:165 День основанья прокляв, за то, что требовал яда: 

\vs 5Sb 1:166 Ложу измены в тебе, малолетних детей совращенье, 

\vs 5Sb 1:167 Женственный город, дурной, нечестивый и самый несчастный,

\vs 5Sb 1:168 Самый порочный из всех городов земли Италийской, 

\vs 5Sb 1:169 Помесь менады с ехидной, вдовой на холмах ты возляжешь,

\vs 5Sb 1:170 Тибра поток по тебе будет плакать, по милой подруге, 

\vs 5Sb 1:171 В сердце чьем мерзость убийства, а дух отягчен преступленьем,

\vs 5Sb 1:172 Разве не знал ты, что может Господь и что замышляет? 

\vs 5Sb 1:173 Ты говорил: Я один, и никто меня не разрушит!

\vs 5Sb 1:174 Ныне же граждан твоих и тебя вечный Бог уничтожит, 

\vs 5Sb 1:175 Впредь никакое жилье не укажет на то, что здесь было, 

\vs 5Sb 1:176 Как в то время, когда твою славу Господь лишь задумал. 

\vs 5Sb 1:177 Будь же один, безрассудный, и, пламенем жарким охвачен, 

\vs 5Sb 1:178 Рухни в безжалостный мрак забытого Богом Аида.

\vs 5Sb 1:179 Снова теперь о твоем я горюю несчастье, Египет! 

\vs 5Sb 1:180 Мемфис, под гнетом страданий ты первым падешь на колени,

\vs 5Sb 1:181 Даже твои пирамиды ужасные вопли исторгнут.

\vs 5Sb 1:182 Пифон, что некогда прежде Диполисом звался по праву,

\vs 5Sb 1:183 Ты замолчишь навсегда, чтобы впредь не творить злодеяний,

\vs 5Sb 1:184 Город надменный, ларец всевозможных пороков, менадой 

\vs 5Sb 1:185 Жалкой, несчастной вдовой навеки отныне пребудешь 

\vs 5Sb 1:186 Ты, что была рождена править миром долгие годы.

\vs 5Sb 1:187 Но когда на себя кипассий Барка набросит

\vs 5Sb 1:188 Грязного белый поверх, то лучше бы ей не родиться.

\vs 5Sb 1:189 Фивы, великая сила куда ваша делась? Разбойник 

\vs 5Sb 1:190 Сгубит народ ваш, а вы, надев одежды печали,

\vs 5Sb 1:191 В плаче зайдетесь, одни, вину искупая несчастьем 

\vs 5Sb 1:192 Те прегрешенья, что прежде свершили, о, город надменный,

\vs 5Sb 1:193 Мир будет видеть ваш плач  за то, что не чтили закона.

\vs 5Sb 1:194 Царь Эфиопов могучий разрушит город Сиену, 

\vs 5Sb 1:195 Силой Тевхиру населит народ темнокожий Индийцев.

\vs 5Sb 1:196 Слезы, Пентаполь, прольешь: тебя муж многомощный погубит.

\vs 5Sb 1:197 Скорбная Ливия, кто твои беды возьмется исчислить?

\vs 5Sb 1:198 Кто, Кирена, тебя среди смертных достойно оплачет?

\vs 5Sb 1:199 Смолкнут стенанья твои только в час ненавистной кончины.

\vs 5Sb 1:200 В земли Британцев и Галлов, богатых золотом, хлынет 

\vs 5Sb 1:201 Вод Океанских поток, от крови все полноводней. 

\vs 5Sb 1:202 Много ведь горя они доставили детям Господним, 

\vs 5Sb 1:203 В год, когда царь Финикийский в Сидон огромное войско 

\vs 5Sb 1:204 Галлов из Сирии вел. Саму тебя тоже погубит

\vs 5Sb 1:205 Он, Равенна, с собой твоих граждан ведя на убийство.

\vs 5Sb 1:206 Не заноситесь, Индийцы и храбрый народ Эфиопов! 

\vs 5Sb 1:207 Ибо когда колесо небесной оси, Козерога

\vs 5Sb 1:208 Звезды, Телец побегут вкруг центра в созвездии Братьев  

\vs 5Sb 1:209 Дева, на небо взойдя, и Солнце, крутясь непрерывно, 

\vs 5Sb 1:210 Их хоровод поведут по всему небесному своду 

\vs 5Sb 1:211 Будет тут страшный пожар, который охватит всю землю, 

\vs 5Sb 1:212 В битве небесных светил обновится природа, погибнет, 

\vs 5Sb 1:213 Плачем мир огласив, в огне страна Эфиопов!

\vs 5Sb 1:214 Плачь ты тоже, Коринф, над своею судьбою несчастной!

\vs 5Sb 1:215 Мойры когда, три сестры, прядущие нити витые, 

\vs 5Sb 1:216 Вспять беглеца поведут, который тайком с перешейка, 

\vs 5Sb 1:217 Горы минуя, бежал, чтобы снова явить его людям. 

\vs 5Sb 1:218 Кто однажды скалу разсек безудержной медью, 

\vs 5Sb 1:219 Тот сгубит землю твою, разорит, как назначено было,

\vs 5Sb 1:220 Ибо от Бога дана ему сила дерзнуть на такое,

\vs 5Sb 1:221 Что ни один из царей до него не отважился сделать. 

\vs 5Sb 1:222 Прежде всего, отделив от трех голов основанья, 

\vs 5Sb 1:223 Щедро позволит другим голов этих мяса отведать, 

\vs 5Sb 1:224 Так что пожрут они плоть родную царя-нечестивца.

\vs 5Sb 1:225 Прочих людей на земле ожидают убийство и ужас 

\vs 5Sb 1:226 Из-за великого Града и верного Богу народа, 

\vs 5Sb 1:227 Что был спасаем всегда, кого Провиденье избрало.

\vs 5Sb 1:228 Ветреный и безрассудный, в себя все несчастья вобравший, 

\vs 5Sb 1:229 Тяжких страданий исток и их наивысшая степень,

\vs 5Sb 1:230 Город, задержанный в росте, но Мойрами все же спасенный, 

\vs 5Sb 1:231 Дерзкий, зачинщик всех бед, великое горе народам  

\vs 5Sb 1:232 Кто пожелал в тебе жить? Живя в тебе, кто не страдал бы? 

\vs 5Sb 1:233 Кто из царей твоих пал, достойную жизнь завершая? 

\vs 5Sb 1:234 Все ты испортил, что мог, залив всякой мерзостью землю, 

\vs 5Sb 1:235 Мира прекрасные складки тобой изменили свой облик. 

\vs 5Sb 1:236 Может быть,  думаешь ты,  она ищет ссоры со мною? 

\vs 5Sb 1:237 Что за нелепость! Хочу вразумить и вот  упрекаю: 

\vs 5Sb 1:238 Некогда свет возсиял средь людей благодатного Солнца, 

\vs 5Sb 1:239 Лившего те же лучи, что и солнце древних пророков. 

\vs 5Sb 1:240 Мед стекал с языка  напиток сладчайший для смертных; 

\vs 5Sb 1:241 Он обвинял, объяснял  и день на земле продолжался. 

\vs 5Sb 1:242 Из-за того, что был Он  о источник тягчайших пороков!  

\vs 5Sb 1:243 Горе и войны с земли однажды навеки исчезнут. 

\vs 5Sb 1:244 Ты же, начало всех зол и их наивысшая степень, 

\vs 5Sb 1:245 Город, задержанный в росте, но Мойрами все же спасенный, 

\vs 5Sb 1:246 Горькому слову внемли, неприятному, смертных несчастье!

\vs 5Sb 1:247 Люди когда воевать на Персидской земле перестанут,

\vs 5Sb 1:248 Стоны покинут ее и голод, тогда появиться

\vs 5Sb 1:249 Должен в ней будет народ Иудеев блаженных, небесный.

\vs 5Sb 1:250 Он ее среднюю часть вкруг Божьего града заселит, 

\vs 5Sb 1:251 Стену соорудив великую вплоть до Иоппы, 

\vs 5Sb 1:252 Ту, что поднимется ввысь под самые темные тучи. 

\vs 5Sb 1:253 Больше труба никогда не издаст воинственный голос, 

\vs 5Sb 1:254 Люди от вражьей руки перестанут гибнуть в сраженьях

\vs 5Sb 1:255 И установят трофей победе над злом в этом мире.

\vs 5Sb 1:256 Муж на землю с небес сойдет, Кому равных не будет, 

\vs 5Sb 1:257 Руки раскинет Свои на древе, обильном плодами. 

\vs 5Sb 1:258 Лучший среди Иудеев, Он солнца бег остановит 

\vs 5Sb 1:259 Речью прекрасной, что с губ Его безупречных польется.

\vs 5Sb 1:260 Больше не нужно тебе скорбеть душою, блаженный, 

\vs 5Sb 1:261 Богом рожденный цветок, желанный для всех и богатый, 

\vs 5Sb 1:262 Свет благодатный, достойный исход вожделенный, святыня, 

\vs 5Sb 1:263 Город земли Иудейской прекрасный, возвышенный в гимнах! 

\vs 5Sb 1:264 В пляске безумной тебя попирать нечестивой стопою

\vs 5Sb 1:265 Эллины больше не будут, душой исзступленью отдавшись  

\vs 5Sb 1:266 Вместо того окружат почитанием дети Господни  

\vs 5Sb 1:267 Те, что воздвигнут алтарь при звуках священных напевов, 

\vs 5Sb 1:268 Богу многие жертвы неся и молясь непрерывно. 

\vs 5Sb 1:269 Все, кто прежде терпел мучения из-за гонений,

\vs 5Sb 1:270 Радостных дней череду теперь в утешенье получат  

\vs 5Sb 1:271 Те же, кто в небеса нечестиво хулу возносили, 

\vs 5Sb 1:272 Вдруг умолкнут, осыпав друг друга безсмысленной бранью. 

\vs 5Sb 1:273 Скроет в себе их земля, до тех пор пока мир существует. 

\vs 5Sb 1:274 Тут прольется из туч пылающий огненный ливень,

\vs 5Sb 1:275 С пашен отныне собрать не придется блестящих колосьев  

\vs 5Sb 1:276 Все незасеянным впредь и невспаханным будет, доколе 

\vs 5Sb 1:277 Власть не признают над миром Безсмертного, вечного Бога 

\vs 5Sb 1:278 Смертные люди и чтить не забудут земли порожденья  

\vs 5Sb 1:279 Коршунов, также собак, которых дал миру Египет,

\vs 5Sb 1:280 Суетно превозносить, утруждая глупые губы. 

\vs 5Sb 1:281 Родина благочестивых, святая земля принесет им 

\vs 5Sb 1:282 Струи медовые, что из скал и источников льются.

\vs 5Sb 1:283 К чистым душою тогда притечет молоко неземное  

\vs 5Sb 1:284 К тем, что надежды свои на Творца одного возложили, 

\vs 5Sb 1:285 Вышнего Бога, Ему принеся почитанье и веру.

\vs 5Sb 1:286 Ясный мне ум для чего велит поведать такое? 

\vs 5Sb 1:287 Бедная Азия, жалость к тебе мою душу терзает, 

\vs 5Sb 1:288 Скорбь о народе Карийцев, богатых Лидийцев, Ионян. 

\vs 5Sb 1:289 Сарды, увы вам! И вам увы, сердцу милые Траллы! 

\vs 5Sb 1:290 Лаодикия, увы! прекраснейший город  погубит 

\vs 5Sb 1:291 Землетрясение вас и в прах обратит ваши стены.

\vs 5Sb 1:292 В скорбной Азийской земле, в стране богатых Лидийцев 

\vs 5Sb 1:293 Храм Артемиды Эфесской падет под ударами бури; 

\vs 5Sb 1:294 Трещины в почве, толчки  и с берега в море сползет он. 

\vs 5Sb 1:295 Так заливают корабль в непогоду свирепые волны. 

\vs 5Sb 1:296 Навзничь упав, тут Эфес испустит вопль, орошая 

\vs 5Sb 1:297 Берег слезами. Искать будет храм он, что высился прежде.

\vs 5Sb 1:298 Гневом тогда распален, нерушимый небесный Владыка 

\vs 5Sb 1:299 Молнию с силой метнет в преступника из поднебесья 

\vs 5Sb 1:300 Вместо зимы в этот день наступит пора урожая. 

\vs 5Sb 1:301 После того на земле людей ожидают несчастья: 

\vs 5Sb 1:302 В высях Гремящий убьет до единого всех нечестивцев, 

\vs 5Sb 1:303 Громы и молнии в ход пустив, горящие стрелы, 

\vs 5Sb 1:304 Целые тучи врагов  и род истребит их настолько,

\vs 5Sb 1:305 Что мертвых тел на земле будет больше, чем мелких песчинок.

\vs 5Sb 1:306 Смирна тогда же придет своего Ликурга оплакать 

\vs 5Sb 1:307 Под стенами Эфеса и здесь сама же погибнет.

\vs 5Sb 1:308 Глупая Кима с ее священной божественной влагой

\vs 5Sb 1:309 Брошена в руки людей безбожных, неправедных, диких, 

\vs 5Sb 1:310 Впредь возносить в небеса не будет радостных песен,

\vs 5Sb 1:311 Но безжизненным телом в волнах прибрежных качаться.

\vs 5Sb 1:312 Те, кто останутся жить, заплачут в голос от горя.

\vs 5Sb 1:313 Будет им знак  по нему поймут, за что претерпели 

\vs 5Sb 1:314 Кимский злосчастный народ, стыда лишенное племя. 

\vs 5Sb 1:315 После, лишь только они сожженную землю оплачут,

\vs 5Sb 1:316 Лесбос навеки уйдет под воды реки Эридана.

\vs 5Sb 1:317 Горе тебе, Керкира прекрасная! Пляски прервешь ты, 

\vs 5Sb 1:318 И Иераполь, живущий в позорном союзе с богатством! 

\vs 5Sb 1:319 Что пожелал, обретешь, оплаканный многими город  

\vs 5Sb 1:320 Там, где течет Термодонт, засыпан землею ты будешь. 

\vs 5Sb 1:321 Триполь, возросший на скалах близ вод Меандра, который 

\vs 5Sb 1:322 Волнами ночью под берег быть смытым судьбою назначен! 

\vs 5Sb 1:323 До основанья тебя разрушит промысел Божий.

\vs 5Sb 1:324 Пусть не желаю я зла земле, что соседняя Фебу: 

\vs 5Sb 1:325 Пущенный с неба перун роскошный Милет уничтожит 

\vs 5Sb 1:326 Из-за того, что коварным он Феба песням поверил \ldots

\vs 5Sb 1:327 Благоразумный совет и о смертных людях забота.

\vs 5Sb 1:328 Смилуйся, мира Создатель, над тучной землей Иудейской, 

\vs 5Sb 1:329 Щедро несущей плоды, чтоб мы Твои помыслы знали! 

\vs 5Sb 1:330 Ибо Ты первой ее сотворил в Своей милости, Боже, 

\vs 5Sb 1:331 С тем, чтобы даром Твоим она для смертных явилась 

\vs 5Sb 1:332 И могла бы внимать всему, что ей Бог доверяет.

\vs 5Sb 1:333 Трижды несчастная, жажду я видеть творенья Фракийцев,

\vs 5Sb 1:334 Стену промеж двух морей, что вихрем несущейся пыли

\vs 5Sb 1:335 Совлечена, как поток в глубину устремится, где рыбы.

\vs 5Sb 1:336 О Геллеспонт разнесчастный! Тебя запряжет Ассириец, 

\vs 5Sb 1:337 Битва Фракийцев великую силу разрушит. 

\vs 5Sb 1:338 С войском Египетский царь Македонии земли захватит, 

\vs 5Sb 1:339 Варваров область низложит могущество власть предержащих. 

\vs 5Sb 1:340 Там Памфилийцы, Галаты, Лидийцы и Писидийцы 

\vs 5Sb 1:341 Вместе одержат победу, на грозную битву собравшись.

\vs 5Sb 1:342 Трижды несчастная, ляжешь, Италия, мертвой пустыней, 

\vs 5Sb 1:343 Змей доколе в твоей цветущей земле не издохнет.

\vs 5Sb 1:344 В высях заоблачных, в небе широком однажды раздастся 

\vs 5Sb 1:345 Грома раскат, призывая прислушаться к голосу Бога. 

\vs 5Sb 1:346 Больше не явятся миру лучи нетленные солнца, 

\vs 5Sb 1:347 Также сияющий свет луны навеки угаснет  

\vs 5Sb 1:348 В самом исходе времен, когда Божья исполнится воля.

\vs 5Sb 1:349 Тьма тут окутает мир, и мрак по земле расползется, 

\vs 5Sb 1:350 Страшные звери на ней, ослепшие люди и горе.

\vs 5Sb 1:351 Долго продлится тот день, и смертные Бога узнают 

\vs 5Sb 1:352 Сущего на небесах Владыку, чье око всезряще.

\vs 5Sb 1:353 Не пожалеет тогда Он врагов Своих, но уничтожит 

\vs 5Sb 1:354 Тех, что баранов, овец, быков стада и мычащих 

\vs 5Sb 1:355 Телок золоторогих и тучных в жертву приносят

\vs 5Sb 1:356 Гермам бездушным, камням, из которых сделаны боги.

\vs 5Sb 1:357 Мудрый пусть торжествует закон и праведных слава!

\vs 5Sb 1:358 Чтобы нетленный Господь, разгневавшись, смерти не предал

\vs 5Sb 1:359 Весь человеческий род нечестивый, безстыдное племя, 

\vs 5Sb 1:360 Нужно Создателя чтить  Безсмертного Вечного Бога.

\vs 5Sb 1:361 В самом исходе времен, лунный свет когда потускнеет, 

\vs 5Sb 1:362 Мир безумство войны охватит, коварной и подлой.

\vs 5Sb 1:363 С края земли человек придет, что на мать покусился, 

\vs 5Sb 1:364 Бегством спасаясь и в сердце своем замышляя дурное. 

\vs 5Sb 1:365 Он всю землю захватит, и все ему станет подвластно,

\vs 5Sb 1:366 В самые тайные мысли людей он свободно проникнет,

\vs 5Sb 1:367 Из-за которой умрет, саму он, вернувшись, погубит.

\vs 5Sb 1:368 Многих мужей истребит, в том числе и великих тиранов, 

\vs 5Sb 1:369 Всех он огнем будет жечь, что никто доселе не делал, 

\vs 5Sb 1:370 Павших снова подняться, ревнуя к Богу, заставит.

\vs 5Sb 1:371 С Запада будет война грозить великая людям, 

\vs 5Sb 1:372 Крови потоки стекут с берегов в полноводные реки, 

\vs 5Sb 1:373 Желчь будет капать по капле в долинах земли Македонской \ldots

\vs 5Sb 1:374 Помощь с Заката придет, придет и смерть властелину. 

\vs 5Sb 1:375 И вот тогда по земле подует ветер холодный,

\vs 5Sb 1:376 Снова жестокой войной наполнится поле сражений.

\vs 5Sb 1:377 С неба на смертных людей прольется огненный ливень,

\vs 5Sb 1:378 Пламя, кровь и вода, блеск молний, тьма и мрак ночи.

\vs 5Sb 1:379 В битве настигшая смерть, резня под покровом тумана 

\vs 5Sb 1:380 Всех уничтожат царей  а с ними воинов лучших.

\vs 5Sb 1:381 Так прекратится война, и стихнет жуткая бойня.

\vs 5Sb 1:382 Больше никто за мечи и железо рукой не возьмется,

\vs 5Sb 1:383 Копий не тронет никто, что будут теперь под запретом.

\vs 5Sb 1:384 Мир тут получит народ разумный, в живых кто остался, 

\vs 5Sb 1:385 Выдержав пробу войной, чтоб радость вкушать беззаботно.

\vs 5Sb 1:386 Мать кто убил, откажитесь от дерзкой преступной отваги! 

\vs 5Sb 1:387 Те, кто на ложе свое нечестиво детей возводили, 

\vs 5Sb 1:388 И превращали в блудниц под кровом своим непорочных 

\vs 5Sb 1:389 Силой и страхом расправы, разнузданным, наглым безстыдством \ldots

\vs 5Sb 1:390 Мать с порожденьем своим смешалась в тебе беззаконно, 

\vs 5Sb 1:391 Дочь с породившим ее позорный союз заключала, 

\vs 5Sb 1:392 Пачкали в стенах твоих цари покорные губы, 

\vs 5Sb 1:393 Ложе делить со скотом искали в тебе нечестивцы. 

\vs 5Sb 1:394 Смолкни же, мерзостный город, жалчайший, средь праздников шумных,

\vs 5Sb 1:395 Ибо уже никогда горящей легко древесины 

\vs 5Sb 1:396 Чистые девы огонь священный в тебе не увидят. 

\vs 5Sb 1:397 Дом, извечно желанный, с тобою погас, когда снова 

\vs 5Sb 1:398 Видеть мне довелось, как падает он под ударом, 

\vs 5Sb 1:399 Весь охвачен огнем, сраженный рукой нечестивой 

\vs 5Sb 1:400 Вечно цветущий предел, хранящее Бога жилище. 

\vs 5Sb 1:401 Храм, что святыми построен и будет стоять нерушимо, 

\vs 5Sb 1:402 Тот, кому телом и духом поверили смертные люди.

\vs 5Sb 1:403 Он не начал бездумно заморскому богу молиться 

\vs 5Sb 1:404 И его из камней высекать, премудрый строитель.

\vs 5Sb 1:405 Также и золота блеск не чтил он  для душ обольщенье: 

\vs 5Sb 1:406 Богу, вдохнувшему жизнь в тела, Создателю мира 

\vs 5Sb 1:407 Издавна в жертву они овец и быков приносили. 

\vs 5Sb 1:408 Ныне же царь, что пришел невидимым, страшный преступник, 

\vs 5Sb 1:409 Всю их страну разорил и лежать в запустенье оставил,

\vs 5Sb 1:410 С войском явившись большим, с мужами, отважными духом.

\vs 5Sb 1:411 Сам он, на землю вступив безсмертную, жизни лишился. 

\vs 5Sb 1:412 Больше явлено людям такого не было знака, 

\vs 5Sb 1:413 Что и другие придут великий город разрушить.

\vs 5Sb 1:414 Муж с высоких небес сошел блаженный на землю, 

\vs 5Sb 1:415 Руки скиптр держали, что Бог ему вечный доверил. 

\vs 5Sb 1:416 Мощью он всех превзошел и тем справедливо богатство 

\vs 5Sb 1:417 Роздал, кто праведно жил  а прежние лишь отбирали. 

\vs 5Sb 1:418 Все он сжигал города и до основания рушил,

\vs 5Sb 1:419 Жег жилища людей, творивших когда-то злодейства.

\vs 5Sb 1:420 Город же, избранный Богом, блестеть заставил он ярко  

\vs 5Sb 1:421 Ярче сияющих звезд на небе и солнца с луною  

\vs 5Sb 1:422 Пышно украсил, и храм в нем Богу священный поставил, 

\vs 5Sb 1:423 В камень одетый, прекрасный, каких не бывало доселе. 

\vs 5Sb 1:424 Стену построил вокруг на много стадиев, в небо

\vs 5Sb 1:425 Что уходила и туч касалась, видна отовсюду  

\vs 5Sb 1:426 Так что могли созерцать все люди праведной веры 

\vs 5Sb 1:427 Славу Безсмертного Бога, давно желанное чудо. 

\vs 5Sb 1:428 Солнца восход и закат пропели Ему свои гимны. 

\vs 5Sb 1:429 Злу среди рода людского отныне места не будет:

\vs 5Sb 1:430 В браке изменам, с детьми не дозволенным Богом сношеньям,

\vs 5Sb 1:431 Смертоубийству, вражде  лишь законному единоборству. 

\vs 5Sb 1:432 Праведных время придет в конце, тогда и исполнит 

\vs 5Sb 1:433 Все это Бог-Громовержец, Строитель великого храма.

\vs 5Sb 1:434 Горе тебе, Вавилон, златотронный и златообутый, 

\vs 5Sb 1:435 Древний царский чертог, один управляющий миром!

\vs 5Sb 1:436 Тот, что некогда был великим и властным  ты больше,

\vs 5Sb 1:437 Город, в горах золотых у вод Евфрата не ляжешь.

\vs 5Sb 1:438 Но по земле распростершись в смятенье подземных ударов,

\vs 5Sb 1:439 Лишь под властью Парфян всем миром потом овладеешь. 

\vs 5Sb 1:440 Попридержи свой язык, нечестивый потомок Халдеев!

\vs 5Sb 1:441 Слов понапрасну не трать на то, как Персами править

\vs 5Sb 1:442 Станешь, Мидийцами как: ведь и власть, что имел, получил ты,

\vs 5Sb 1:443 Риму заложника дав и Азийских наемников выслав.

\vs 5Sb 1:444 Вот потому-то пойдешь, расчетливый царь, ты в Афины 

\vs 5Sb 1:445 Для выяснения цели: зачем посылал, дескать, выкуп.

\vs 5Sb 1:446 Вместо неискренних слов врагам свой гнев ты покажешь.

\vs 5Sb 1:447 В самом исходе времен однажды высохнет море, 

\vs 5Sb 1:448 Так что не смогут приплыть корабли к берегам Италийским. 

\vs 5Sb 1:449 Азия станет тогда, напротив, водой животворной, 

\vs 5Sb 1:450 То же и Крит. Много бед испытать тут придется и Кипру: 

\vs 5Sb 1:451 Пафос оплакивать будет печальный свой жребий, узнают 

\vs 5Sb 1:452 Все о судьбе Саламина, который постигло несчастье. 

\vs 5Sb 1:453 Больше плодов приносить не станет земля побережья, 

\vs 5Sb 1:454 Мощный набег саранчи погубит страну Киприотов.

\vs 5Sb 1:455 Будете плакать вы, глядя ни Тир, злополучные люди! 

\vs 5Sb 1:456 Гнев тебя ждет, Финикия, ужасный, доколе не рухнешь 

\vs 5Sb 1:457 Тяжко на землю  могли чтобы искренне плакать Сирены.

\vs 5Sb 1:458 В пятом колене людском, когда беды Египта отступят, 

\vs 5Sb 1:459 И цари Египтян друг с другом безстыдно мешаться 

\vs 5Sb 1:460 Станут, в Египте взойдут на трон Памфилийцев потомки. 

\vs 5Sb 1:461 У Македонцев тогда, и в Азии, и у Ликийцев 

\vs 5Sb 1:462 Ужас кровавой войны покроет все пылью и прахом. 

\vs 5Sb 1:463 Римский прервет его царь с владыками Запада вместе.

\vs 5Sb 1:464 Только лишь ветер холодный и снег приносящий подует,

\vs 5Sb 1:465 Только покроются льдом большая река и озера  

\vs 5Sb 1:466 Варварский тотчас народ устремится в Азийскую землю, 

\vs 5Sb 1:467 Словно безсильных, погубит он грозное племя Фракийцев. 

\vs 5Sb 1:468 Будут отцов тут своих поедать несчастные люди, 

\vs 5Sb 1:469 Мучимы голодом, в пищу себе их мясо готовить. 

\vs 5Sb 1:470 Звери же будут кормиться, беря из людского жилища  

\vs 5Sb 1:471 С птицами вместе, они всех смертных людей уничтожат. 

\vs 5Sb 1:472 В ходе жестокой войны прибудет воды в Океане, 

\vs 5Sb 1:473 Примет кровавый он цвет от тел и крови безумцев. 

\vs 5Sb 1:474 В это же время земля настолько уже истощится, 

\vs 5Sb 1:475 Что можно будет в уме мужчин перечислить и женщин.

\vs 5Sb 1:476 Жалкое племя в конце испустит страшные вопли, 

\vs 5Sb 1:477 В час, когда солнце зайдет, чтобы больше уже не подняться, 

\vs 5Sb 1:478 Но, в океанской воде оставаясь, очиститься ею  

\vs 5Sb 1:479 Ибо многих людей пришлось ему видеть нечестье. 

\vs 5Sb 1:480 Темная ночь без луны по небу тогда разольется,

\vs 5Sb 1:481 Мгла, какой прежде не знали, окутает складки земные. 

\vs 5Sb 1:482 Снова, однако, дорогу потом свет Божий укажет 

\vs 5Sb 1:483 Праведным людям, что в гимнах поспели Великого Бога.

\vs 5Sb 1:484 Ты, несчастная трижды Изида! У Нильских потоков 

\vs 5Sb 1:485 Сядешь одна, как менада немая у вод Ахеронта, 

\vs 5Sb 1:486 Память сама о тебе скоро жить на земле перестанет. 

\vs 5Sb 1:487 Ты же, Серапис, мученья претерпишь на каменном ложе, 

\vs 5Sb 1:488 В трижды несчастном Египте руиной падешь величайшей. 

\vs 5Sb 1:489 Все, что тянулись к тебе в стране Египтян, будут скоро

\vs 5Sb 1:490 Плакать, а те, кто вложил в свое сердце разум нетленный, 

\vs 5Sb 1:491 Бога кто в гимнах воспел, поймут, что ты вовсе ничтожен.

\vs 5Sb 1:492 Скажет один из жрецов, одетый в льняные одежды: 

\vs 5Sb 1:493 Люди, построим святыню в честь истинно Сущего Бога! 

\vs 5Sb 1:494 Люди, ужасный обычай, от предков идущий, изменим 

\vs 5Sb 1:495 Тот, по которому деды богам из глины и камня, 

\vs 5Sb 1:496 Шествия, жертвы, обряды творя, потеряли разсудок. 

\vs 5Sb 1:497 Несокрушимого Бога возславив, душой обратимся, 

\vs 5Sb 1:498 Люди, к Нему Самому  Создателю, Сущему вечно, 

\vs 5Sb 1:499 Кто всеми правит, Царю, справедливому мира Владыке,

\vs 5Sb 1:500 Душ Кормильцу, Отцу, Великому, Вечно Живому! 

\vs 5Sb 1:501 Так возведен будет храм в Египте, великий, священный; 

\vs 5Sb 1:502 Жертвы к нему понесет народ, наставленный Богом,  

\vs 5Sb 1:503 Те, кому вечную жизнь Господь на земле уготовил.

\vs 5Sb 1:504 Но лишь только уйдут Эфиопы от дерзких Трибаллов 

\vs 5Sb 1:505 И вознамерятся сами в Египте распахивать земли, 

\vs 5Sb 1:506 Зло они станут творить, чтобы гибель вселенной ускорить, 

\vs 5Sb 1:507 Наземь повергнут и храм великий в Египетском царстве. 

\vs 5Sb 1:508 Бог же за это на них Свой гнев ужасный обрушит, 

\vs 5Sb 1:509 Гибель тем самым неся преступникам и нечестивцам. 

\vs 5Sb 1:510 Больше никто в той земле уже не получит пощады, 

\vs 5Sb 1:511 Ибо сберечь не смогли того, что Господь им доверил.

\vs 5Sb 1:512 Видела я среди звезд сверкавшего Солнца угрозу,

\vs 5Sb 1:513 Гнев Луны величайший при свете блещущих молний.

\vs 5Sb 1:514 Звезды родили войну  Господь повелел им сражаться. 

\vs 5Sb 1:515 Вместо Солнца вовсю бушевало огромное пламя,

\vs 5Sb 1:516 Лунный двурогий изгиб потерял свою прежнюю форму.

\vs 5Sb 1:517 В битву вступила Венера, ко Льву на спину взобравшись;

\vs 5Sb 1:518 Прямо в загривок Тельца Козерог молодого ударил,

\vs 5Sb 1:519 Тот же за это лишил Козерога надежд на спасенье; 

\vs 5Sb 1:520 Дольше на небе сиять Орион Весам не позволил;

\vs 5Sb 1:521 Дева судьбу Близнецов в созвездье Овна изменила;

\vs 5Sb 1:522 Звезды Плеяд не взошли  их пояс Дракон уничтожил;

\vs 5Sb 1:523 В панцирь созвездия Льва наносить стали Рыбы удары;

\vs 5Sb 1:524 Рак не сумел устоять, боясь больше всех Ориона; 

\vs 5Sb 1:525 Встал на свой хвост Скорпион, перед Львом робея ужасным;

\vs 5Sb 1:526 Пес помчался стремглав от огня палящего Солнца;

\vs 5Sb 1:527 Гнев большого Светила заставил пылать Водолея. 

\vs 5Sb 1:528 Начал трястись Небосвод, пока не стряхнул воевавших. 

\vs 5Sb 1:529 Сильно разгневавшись, он с высоты на землю их бросил, 

\vs 5Sb 1:530 Так что, стремительно вниз в океанские воды сорвавшись, 

\vs 5Sb 1:531 Землю спалили огнем, а небо лишилось созвездий.

\bibbookdescr{6Sb}{
  inline={Шестая книга Сивилл},
  toc={6-я Сивилл},
  bookmark={6-я Сивилл},
  header={6-я Сивилл},
  abbr={6~Сив}
}
\vs 6Sb 1:1 Сына Безсмертного Бога пою я, Великого в славе, 

\vs 6Sb 1:2 Кто еще не был рожден, когда Всевышний Родитель 

\vs 6Sb 1:3 Трон Ему уготовил, и Кто родился вторично, 

\vs 6Sb 1:4 В плоть и кровь облечен, омытый водой Иордана,

\vs 6Sb 1:5 Что блестящей стопой, катя свои воды, несется. 

\vs 6Sb 1:6 Я воспеваю Того, Кто огня избежит и увидит 

\vs 6Sb 1:7 Первым Духа Господня, слетевшего в белой голубке. 

\vs 6Sb 1:8 Чистый цветок расцветет, источники вод заструятся  

\vs 6Sb 1:9 Людям укажет пути, укажет и торные тропы,

\vs 6Sb 1:10 К небу ведущие: всех Он умным словом научит.

\vs 6Sb 1:11 Будет к суду призывать, убеждать народ непослушный, 

\vs 6Sb 1:12 Смело славный Свой род от Отца Небесного выдав: 

\vs 6Sb 1:13 Будет ходить по волнам, людей избавлять от болезней, 

\vs 6Sb 1:14 Мертвых поднимет и прочь отгонит тяжкие муки,

\vs 6Sb 1:15 Всех из котомки одной Он досыта хлебом накормит. 

\vs 6Sb 1:16 Семя Давыдово даст росток  в руке Его будет 

\vs 6Sb 1:17 Целый мир и земля и огромное небо и море. 

\vs 6Sb 1:18 Молнией землю осветит, подобно тому как явился 

\vs 6Sb 1:19 Он впервые двоим, от общей плоти рожденным.

\vs 6Sb 1:20 Будет все так, когда миру Ребенок надежду подарит.

\vs 6Sb 1:21 Только тебе одному злые беды, Содом, угрожают:

\vs 6Sb 1:22 Ибо в безумии ты своего не узнал Господина,

\vs 6Sb 1:23 К смертным пришедшего людям, Которого тернием колким

\vs 6Sb 1:24 Ты увенчал, примешав к дерзновению черную злобу 

\vs 6Sb 1:25 В сердце своем, и сулит тебе это тяжкие муки. 

\vs 6Sb 1:26 О блаженное древо, на коем Бога распяли! 

\vs 6Sb 1:27 Не на земле пребывать тебе предстоит, а на небе, 

\vs 6Sb 1:28 Бог когда огненный взгляд обновленный как молнию кинет.

\bibbookdescr{7Sb}{
  inline={Седьмая книга Сивилл},
  toc={7-я Сивилл},
  bookmark={7-я Сивилл},
  header={7-я Сивилл},
  abbr={7~Сив}
}
\vs 7Sb 1:1 Родос злосчастный, тебя, тебя я первым оплачу! 

\vs 7Sb 1:2 Первым среди городов ты будешь и первым погибнешь, 

\vs 7Sb 1:3 Жизни лишен, без людей, одинокий и очень несчастный.

\vs 7Sb 1:4 Делос, ты поплывешь и на волнах будешь качаться. 

\vs 7Sb 1:5 Кипр, однажды тебя затопят воды морские.

\vs 7Sb 1:6 Остров Сицилия, ты погибнешь, охвачен пожаром.

\vs 7Sb 1:7 То, о чем говорю: ужасный, невиданный прежде 

\vs 7Sb 1:8 Хлынет на землю потоп, Самим низпосланный Богом.

\vs 7Sb 1:9 Ной лишь один уцелел, от всех людей убежавший.

\vs 7Sb 1:10 Все поплывет  и земля, и горы, и небо над ними; 

\vs 7Sb 1:11 Мир весь станет водой и водами будет погублен. 

\vs 7Sb 1:12 Ветры дуть прекратят, наступит другая эпоха.

\vs 7Sb 1:13 Фригия! Первой на свет суждено тебе снова подняться, 

\vs 7Sb 1:14 Первой, в нечестие впав, сама отречешься от Бога 

\vs 7Sb 1:15 И, предпочтенье отдав немым изваяньям, за это, 

\vs 7Sb 1:16 Жалкая, годы спустя ужасною смертью погибнешь.

\vs 7Sb 1:17 Много бед претерпев, Эфиопы несчастные, в страхе 

\vs 7Sb 1:18 Телом дрожа, под мечи себя безрассудно поставят.

\vs 7Sb 1:19 Трудолюбивый Египет, от века растящий колосья, 

\vs 7Sb 1:20 Тот, которого Нил питает семью рукавами, 

\vs 7Sb 1:21 Междуусобная рознь погубит. Тогда же, нежданно, 

\vs 7Sb 1:22 Аписа люди изгонят за то, что вовсе не бог он.

\vs 7Sb 1:23 Лаодикия, увы! Ты Бога впредь не увидишь,

\vs 7Sb 1:24 Но, погрязнув во лжи, будешь смыта Ликской волною.

\vs 7Sb 1:25 Сам рожденный Господь, великий, Который без счета 

\vs 7Sb 1:26 Звезд сотворит и ось проденет сквозь неба средину, 

\vs 7Sb 1:27 Людям на страх, в вышине, чтобы видели все, установит 

\vs 7Sb 1:28 Столп, измерив его огнем великим, чьи капли 

\vs 7Sb 1:29 Жизнь отнимут у тех, кто себя запятнал преступленьем. 

\vs 7Sb 1:30 Время такое наступит однажды, и смертные люди 

\vs 7Sb 1:31 Бога тут станут молить, но не будет предела страданьям 

\vs 7Sb 1:32 Их безконечным. Тогда чрез дом все свершится Давидов, 

\vs 7Sb 1:33 Ибо сам Бог удостоил его небесного трона. 

\vs 7Sb 1:34 Ангелы лягут у ног, которые именем Божьим 

\vs 7Sb 1:35 Свет огням алтарей дают и воды  потокам: 

\vs 7Sb 1:36 Те хранят города, другие  ветра посылают.

\vs 7Sb 1:37 Многих людей ожидают невзгоды, что, в души несчастных 

\vs 7Sb 1:38 Путь пролагая, сердца их всех измениться заставят.

\vs 7Sb 1:39 В пору, как юный росток, на корне возросший, прозреет, 

\vs 7Sb 1:40 Власть он, что некогда всех в избытке пищей снабжала.

\vs 7Sb 1:41 Это случиться должно с исполнением срока. Но стоит 

\vs 7Sb 1:42 Править воинственным Персам начать, как тогда же покои 

\vs 7Sb 1:43 Брачные чистых невест омрачатся всеобщим нечестьем. 

\vs 7Sb 1:44 Сына мать своего как супруга на ложе допустит, 

\vs 7Sb 1:45 Мать соблазнит ее сын. Уснет, к отцу прижимаясь,

\vs 7Sb 1:46 Дочь, исполняя обычай их варварский. Позже над ними 

\vs 7Sb 1:47 Римский Арей заблестит оружьем несметного войска. 

\vs 7Sb 1:48 Много смешают земли тут с кровью убитых в сраженье, 

\vs 7Sb 1:49 Но от твердости копий бежит Италийский воитель. 

\vs 7Sb 1:50 Бросят они в той стране из золота сделанный символ 

\vs 7Sb 1:51 Тот, что вперед выходя, всегда означал неизбежность.

\vs 7Sb 1:52 Время настанет, и весь погрязший в пороке, несчастный 

\vs 7Sb 1:53 Смерть обретет Илион вместо свадеб, когда зарыдают 

\vs 7Sb 1:54 Горько юные жены о том, что не ведали Бога, 

\vs 7Sb 1:55 Но ударяли в тимпаны и били ногами о землю.

\vs 7Sb 1:56 Бога спроси, Колофон: тебя страшный пожар ожидает.

\vs 7Sb 1:57 Ты, несчастная в браке, Фессалия! Снова увидеть 

\vs 7Sb 1:58 Лик твой земле не дано, как и пепел. Отсюда по морю, 

\vs 7Sb 1:59 Храбрая, вдаль отплывешь и войны испражнением станешь, 

\vs 7Sb 1:60 Пав под ударом мечей и сгинув в стремительных реках. 

\vs 7Sb 1:61 Стойкий Коринф! Ты у стен Арея грозного примешь: 

\vs 7Sb 1:62 Горе тебе, ибо вы падете сраженные оба.

\vs 7Sb 1:63 Тир, тебе одному пережить уготовано столько: 

\vs 7Sb 1:64 Набожных граждан твоих безсилье тебя же разрушит.

\vs 7Sb 1:65 Горная Сирия, ты поднялась над землей Финикийской, 

\vs 7Sb 1:66 Где к берегам приливают валы Беритского моря. 

\vs 7Sb 1:67 Бога узнать своего не смогла, несчастная  влагой 

\vs 7Sb 1:68 Кто Иорданской омыт, на Кого Божий Дух опустился. 

\vs 7Sb 1:69 Кто, до того, как земля и звездное небо возникли, 

\vs 7Sb 1:70 Словом Отца был рожден, Властелин, и, плотью облекшись 

\vs 7Sb 1:71 Через Духа Святого, к Отцу вскоре в домы вознесся. 

\vs 7Sb 1:72 Три Ему башни Уран великий поставил, в которых 

\vs 7Sb 1:73 Матери Бога теперь живут благородные. Имя 

\vs 7Sb 1:74 Первой  Надежда, второй  Благочестье и Набожность  третьей.

\vs 7Sb 1:75 Ни серебра не хотят, ни золота  радость приносят 

\vs 7Sb 1:76 Им поклоненье людей, их жертвы и чистые мысли.

\vs 7Sb 1:77 Жертвовать вечному Богу, великому, славному станешь, 

\vs 7Sb 1:78 Не растопив на огне крупицу ладана, нож свой 

\vs 7Sb 1:79 Не занеся над бараном с волнистым руном, но со всеми, 

\vs 7Sb 1:80 В ком течет твоя кровь, взяв птицу дикую в руки, 

\vs 7Sb 1:81 Вверх направишь ее, с молитвою глядя на небо.

\vs 7Sb 1:82 Воду на чистый огонь прольешь и скажешь при этом: 

\vs 7Sb 1:83 Твой Отец Тебя создал как Слово, Отче. Я птицу 

\vs 7Sb 1:84 Быструю выпустил с вестью  о Слове Слово, крещенье 

\vs 7Sb 1:85 Влагой Твое окропив  огонь, из какого Ты вышел.

\vs 7Sb 1:86 Ты не закроешь дверей, когда чужестранец безвестный 

\vs 7Sb 1:87 К дому придет твоему, нуждаясь в пище и крове. 

\vs 7Sb 1:88 Но, его голову взяв в ладони, обрызгав водою, 

\vs 7Sb 1:89 Трижды мольбу вознеси, обратись к своему Господину: 

\vs 7Sb 1:90 Я не жажду богатства, простой  простого я принял.

\vs 7Sb 1:91 Вдвое подай нам, Отец, склони Свой слух, Покровитель! 

\vs 7Sb 1:92 Даст Он, мольбе твоей вняв, когда же уйдет чужеземец: 

\vs 7Sb 1:93 Мукам меня не предай, о праведной веры Святыня, 

\vs 7Sb 1:94 Чистый, свободный, прошедший сквозь пламя \ldots

\vs 7Sb 1:95 Слабый мой дух укрепи. Отец. На Тебя я взираю, 

\vs 7Sb 1:96 Кто всякой скверны далек, руками не создан людскими \ldots

\vs 7Sb 1:97 Жребий, Сардиния, твой несчастен  в золу превратишься, 

\vs 7Sb 1:98 Сменится десять эпох  и островом быть перестанешь. 

\vs 7Sb 1:99 Тщетно тебя среди волн искать мореходам придется, 

\vs 7Sb 1:100 Птицы свой жалобный плач по тебе над морем поднимут.

\vs 7Sb 1:101 Камнем покрыта сплошным, Мигдония, крепость на море, 

\vs 7Sb 1:102 Славиться будешь века, чтобы после навеки погибнуть

\vs 7Sb 1:103 Всей под горячим дыханьем, от боли придя в изступленье.

\vs 7Sb 1:104 Кельтов земля! По горам, у подножия Альп недоступных 

\vs 7Sb 1:105 Скроет глубокий песок тебя; не выплатишь дани, 

\vs 7Sb 1:106 Колос не дашь и траву  безлюдною ляжешь пустыней, 

\vs 7Sb 1:107 Вечно покрытая льдом, под слоем холодных кристаллов, 

\vs 7Sb 1:108 Будешь страдать за вину, которой преступно не помнишь.

\vs 7Sb 1:109 Рим, чей дух непреклонен! Вослед Македонскому царству 

\vs 7Sb 1:110 Дротик метнешь ты в Олимп  за это немым и печальным 

\vs 7Sb 1:111 Бог тебя сделает, пусть казаться ты будешь в то время 

\vs 7Sb 1:112 Сильным как никогда  тут я обращусь к тебе с речью. 

\vs 7Sb 1:113 Чувствуя гибель, оплачешь ты блеск и славу былую  

\vs 7Sb 1:114 Я во второй раз, о Рим, возьмусь тебе это напомнить.

\vs 7Sb 1:115 Ныне же я по тебе, несчастная Сирия, плачу.

\vs 7Sb 1:116 Разум оставил вас, Фивы; нависли ужасные звуки 

\vs 7Sb 1:117 Громко взывающих флейт, труба им грозная вторит  

\vs 7Sb 1:118 Так что увидите вы поверженной в прах вашу землю.

\vs 7Sb 1:119 Горе, о горе тебе, несчастное злобное море?

\vs 7Sb 1:120 Все тебя пламя пожрет, людей ты погубишь волнами 

\vs 7Sb 1:121 Ибо такой на земле пожар забушует, что воды 

\vs 7Sb 1:122 Станут огнем, потекут и землю безкрайнюю сгубят, 

\vs 7Sb 1:123 Горы заставят они пылать, ключи, и потоки. 

\vs 7Sb 1:124 Мир же со смертью людей прекрасный свой облик утратит, 

\vs 7Sb 1:125 В муках сгорая, тогда не увидят несчастные неба,

\vs 7Sb 1:126 Полного звезд, но огнем оно все выжжено будет. 

\vs 7Sb 1:127 Быстро они не умрут: под гибнущей в пламени плотью 

\vs 7Sb 1:128 Души их будут пылать в продолжение многих столетий. 

\vs 7Sb 1:129 Так, злые муки терпя, Закон познают Господень  

\vs 7Sb 1:130 Тот, что всегда справедлив. Земля же под гнетом несчастья

\vs 7Sb 1:131 Всяких богов приняла на своих алтарях без разбора 

\vs 7Sb 1:132 И обманулась, понять не сумев зловещего дыма. 

\vs 7Sb 1:133 Тем страдать суждено сверх меры, кто ради корысти 

\vs 7Sb 1:134 Станет предсказывать зло, продляя тяжелое время. 

\vs 7Sb 1:135 Эти, надев на себя овец густорунные шкуры,

\vs 7Sb 1:136 Будут себя выдавать за Евреев, хоть рода иного, 

\vs 7Sb 1:137 Хитрые речи плести, наживаясь на общем несчастье. 

\vs 7Sb 1:138 Жизнь поменяют свою, но праведных не убедить им  

\vs 7Sb 1:139 Тех, что, от чистого сердца уверовав, молятся Богу.

\vs 7Sb 1:140 В третьем жребии лет, что пройдут, друг друга сменяя,

\vs 7Sb 1:141 В первой восьмерке опять грядет обновление мира. 

\vs 7Sb 1:142 Долгая ночь на земле тогда неподвижная ляжет, 

\vs 7Sb 1:143 Запах серы зловещий начнет повсюду носиться  

\vs 7Sb 1:144 Вестник насильственной смерти, другие же люди в то время 

\vs 7Sb 1:145 Будут под пологом ночи от голода гибнуть. Тут явит

\vs 7Sb 1:146 Бог чистый ум средь людей и род возстановит, который 

\vs 7Sb 1:147 Некогда жил на земле. Никто больше пашню не взрежет 

\vs 7Sb 1:148 Выгнутым плугом, быки железо вглубь не опустят, 

\vs 7Sb 1:149 Больше ростки не взойдут, не будет колосьев. Все вместе 

\vs 7Sb 1:150 Манну росистую есть белоснежными станут зубами.

\vs 7Sb 1:151 С ними пребудет тогда Господь, и Он их научит 

\vs 7Sb 1:152 Так же, как и меня, несчастную  столько свершила 

\vs 7Sb 1:153 Прежде недобрых я дел и знала об этом. Другое 

\vs 7Sb 1:154 Было содеяно мною невольно. Со многими ложе 

\vs 7Sb 1:155 Я разделила без мысли о браке; ужасную клятву

\vs 7Sb 1:156 Всем вероломно дала. На порог не пустила я бедных. 

\vs 7Sb 1:157 И, среди первых спускаясь в долину смерти, Господних

\vs 7Sb 1:158 Слов не сумела понять  за то и пожрет меня пламень. 

\vs 7Sb 1:159 Вечно мне жить не дано  погубит жестокое время, 

\vs 7Sb 1:160 Люди меня погребут, проплывая по морю мимо. 

\vs 7Sb 1:161 Буду побита камнями вещав, что велел мне родитель, 

\vs 7Sb 1:162 Сына я предала! Все киньте по камню, кидайте  

\vs 7Sb 1:163 Так сохраню себе жизнь и к небу очи воздену.

\bibbookdescr{8Sb}{
  inline={Восьмая книга Сивилл},
  toc={8-я Сивилл},
  bookmark={8-я Сивилл},
  header={8-я Сивилл},
  abbr={8~Сив}
}
\vs 8Sb 1:1 Гнев Господень ужасный грядет непокорному свету! 

\vs 8Sb 1:2 Все, чем Бог угрожает последнему веку, скажу я 

\vs 8Sb 1:3 Жителям каждого града, всем будет пророчество ясным. 

\vs 8Sb 1:4 С той поры, когда башня упала, а вслед человечий 

\vs 8Sb 1:5 Множеством говоров стал язык, внезапно распавшись, 

\vs 8Sb 1:6 Царство Египта вначале возникло, потом государства 

\vs 8Sb 1:7 Персов, Мидян, Эфиопов, в Ассирии вкруг Вавилона, 

\vs 8Sb 1:8 И в Македонии гордой  про всех уже сказано мною; 

\vs 8Sb 1:9 Ныне же я обращусь к пресловутой земле Италийской.

\vs 8Sb 1:10 Множество зол в конце времен причинит она смертным: 

\vs 8Sb 1:11 Всюду сведет на нет старания разных народов, 

\vs 8Sb 1:12 Многих отважных мужей она в плен угонит на Запад, 

\vs 8Sb 1:13 Все подчинит и народам свои предпишет законы. 

\vs 8Sb 1:14 Долго пусть мелют зерно жернова Господни, но мелко.

\vs 8Sb 1:15 Сгинет все от огня, и в тонкий пух превратятся 

\vs 8Sb 1:16 Гор высоких вершины, а всякая плоть станет пылью. 

\vs 8Sb 1:17 Алчность и безразсудство  всех бед и несчастий начало. 

\vs 8Sb 1:18 К золоту и серебру, к обманчивым, люди стремятся, 

\vs 8Sb 1:19 Лучшим, что есть на земле, металлы эти считая:

\vs 8Sb 1:20 Лучше, чем солнца сиянье, и лучше, чем небо иль море, 

\vs 8Sb 1:21 Или земля, что, простершись широко, все порождает, 

\vs 8Sb 1:22 Иль даже Бог, сотворивший весь мир и все подающий; 

\vs 8Sb 1:23 Веру и благочестье поставили ниже металлов. 

\vs 8Sb 1:24 Это безумье  источник неправды и смуты зачинщик,

\vs 8Sb 1:25 Мирной жизни оно враждебно, а войнам  причина, 

\vs 8Sb 1:26 Ведь от него и отцы с сыновьями своими враждуют, 

\vs 8Sb 1:27 Также и брак не в чести у тех, кто золото любит.

\vs 8Sb 1:28 Всюду на землях  границы, и всюду стражи на море, 

\vs 8Sb 1:29 Делится все хитроумно меж теми, кто златом владеет,

\vs 8Sb 1:30 Будто навечно хотят забрать плодоносную землю. 

\vs 8Sb 1:31 Грабят они бедняков, лишь бы только именье расширить, 

\vs 8Sb 1:32 Тех, кто им отдал свое, в рабов обращая хвастливо. 

\vs 8Sb 1:33 Если б земля не была далека от звездного неба, 

\vs 8Sb 1:34 Не был бы также и свет одинаково людям доступным,

\vs 8Sb 1:35 Но только тот, кто богат, покупать его мог бы за деньги, 

\vs 8Sb 1:36 Новый же мир сотворить для бедных Богу пришлось бы. 

\vs 8Sb 1:37 Рим надменный, тебе испытать когда-то придется 

\vs 8Sb 1:38 Неба удар справедливый, ты первый шею преклонишь, 

\vs 8Sb 1:39 Рухнешь наземь, огнем истребишься до основанья,

\vs 8Sb 1:40 Лежа на собственных землях, и все богатство погибнет, 

\vs 8Sb 1:41 А во дворцах будут жить лишь дикие волки и лисы; 

\vs 8Sb 1:42 Станешь пустынею, словно и не было города вовсе. 

\vs 8Sb 1:43 Где твой палладий? И где тот бог, что пришел бы на помощь, 

\vs 8Sb 1:44 Медный, иль золотой, иль каменный? Где же сената

\vs 8Sb 1:45 Постановления? Где потомки Кроноса, Реи

\vs 8Sb 1:46 Или же Зевса и всех, кто так были чтимы тобою? 

\vs 8Sb 1:47 То  божества без души, подобия трупов безсильных, 

\vs 8Sb 1:48 Скроет которых земля несчастного Крита, и станут 

\vs 8Sb 1:49 С гордостью там почитать мертвецов, что не чувствуют больше.

\vs 8Sb 1:50 После трижды пяти царей, о изнеженный город, 

\vs 8Sb 1:51 Что покорят весь мир от Восхода и вплоть до Заката, 

\vs 8Sb 1:52 Вождь воцарится седой, соименник ближнего моря. 

\vs 8Sb 1:53 Грозной ногою пройдет он по миру, дары добывая, 

\vs 8Sb 1:54 Многое множество злата, а с ним серебра еще больше

\vs 8Sb 1:55 Он у врагов заберет и, награбив, домой возвратится. 

\vs 8Sb 1:56 Царь тот в святилищах магов участником станет мистерий, 

\vs 8Sb 1:57 Мальчика сделает богом, но все богов почитанье 

\vs 8Sb 1:58 Сам низпровергнет, открыв всю лживость мистерий для смертных. 

\vs 8Sb 1:59 Будет ужасное время, когда сам Ужасный погибнет.

\vs 8Sb 1:60 Скажет однажды народ: О город, падет твоя сила!  

\vs 8Sb 1:61 Ибо почувствует вдруг дурного дня приближенье. 

\vs 8Sb 1:62 Горько заплачут тогда, предвидя удел твой несчастный, 

\vs 8Sb 1:63 Вместе родители все и все неразумные дети, 

\vs 8Sb 1:64 Скорбным рыданием их огласятся два берега Тибра.

\vs 8Sb 1:65 Трое за тем царем в последние дни будут править, 

\vs 8Sb 1:66 Имя собою исполнив Небесного Бога, чья сила 

\vs 8Sb 1:67 Вплоть до скончания всех времен пребудет, как ныне. 

\vs 8Sb 1:68 Скипетр удержит надолго один из них, муж престарелый, 

\vs 8Sb 1:69 Царь, сожаленья достойный, который сокровища мира

\vs 8Sb 1:70 Все в чертогах своих укроет, чтобы раздать их

\vs 8Sb 1:71 Людям, когда от границ земных беглец возвратится, 

\vs 8Sb 1:72 Мать погубивший; тогда богатой Азия станет. 

\vs 8Sb 1:73 Снимешь в те дни ты наряд, с широкою красной каймою 

\vs 8Sb 1:74 И облечешься, печалясь, в одежду скорби глубокой,

\vs 8Sb 1:75 О надменное царство, о дочь Латинского Рима! 

\vs 8Sb 1:76 Славною гордость твоя надменная быть перестанет, 

\vs 8Sb 1:77 Будешь лежать распростершись и больше тебе не подняться. 

\vs 8Sb 1:78 Честь легионов твоих падет со всеми орлами, 

\vs 8Sb 1:79 Где ж твоя сила? И кто союзником быть согласится,

\vs 8Sb 1:80 Пред неразумьем твоим безбожным главу преклоняя? 

\vs 8Sb 1:81 Тут среди смертных по всей земле начнется смятенье, 

\vs 8Sb 1:82 В день, как придет Вседержитель, возсядет на троне и будет 

\vs 8Sb 1:83 Души судить живых и мертвых  суд над вселенной. 

\vs 8Sb 1:84 Станут тогда немилы родители детям, а дети

\vs 8Sb 1:85 Тем, кто родил их, от горя нежданного и от нечестья. 

\vs 8Sb 1:86 Скрежет зубов тебя ждет, раскаянье и покоренье, 

\vs 8Sb 1:87 Грады рухнут когда и земля провалы разверзнет. 

\vs 8Sb 1:88 Тут пурпурный дракон по морским приплывет к тебе волнам, 

\vs 8Sb 1:89 Чревом наполненным он детей твоих вскармливать будет

\vs 8Sb 1:90 В дни, когда голод придет и войны гражданские грянут; 

\vs 8Sb 1:91 Все это значит, что день последний этого мира 

\vs 8Sb 1:92 Близок, и вскоре все люди на Суд будут призваны Богом. 

\vs 8Sb 1:93 Римлянам первым придется познать Его гнев безпощадный, 

\vs 8Sb 1:94 Ждет их кровавое время и в жизни сплошные несчастья.

\vs 8Sb 1:95 Горе тебе, о земля Италийская  варваров племя!  

\vs 8Sb 1:96 Ты позабыла о том, что, на свет появившись нагою, 

\vs 8Sb 1:97 И недостойной, назад уйдешь ты опять без одежды 

\vs 8Sb 1:98 В то же самое место, где суд над тобою свершится, 

\vs 8Sb 1:99 Ибо неправедно ты сама осуждала \ldots

\vs 8Sb 1:100 Руки твои словно руки Гигантов были, когда ты

\vs 8Sb 1:101 Шла в этот мир с высоты  теперь под землей твое место. 

\vs 8Sb 1:102 Нефтью, серой, асфальтом, великим огнем истребишься 

\vs 8Sb 1:103 И превратишься ты в груду горячего пепла навеки. 

\vs 8Sb 1:104 Каждый, кто ни посмотрит, услышит, из Тартара вопли,

\vs 8Sb 1:105 Громкие, полные скорби, и скрежет зубов, и удары 

\vs 8Sb 1:106 Жалких ладоней твоих, что в грудь безбожную бьются.

\vs 8Sb 1:107 Ночь одинаково всех ожидает  богатых и бедных, 

\vs 8Sb 1:108 Мы из земли нагими выходим, нагими же в землю 

\vs 8Sb 1:109 Снова ложимся, как только исполнится срок нашей жизни.

\vs 8Sb 1:110 Там уже нет никаких рабов, ни господ, ни тиранов, 

\vs 8Sb 1:111 Нет ни царей, ни вождей надменных и полных гордыни, 

\vs 8Sb 1:112 Ни хитроумных витий, ни начальников нет лихоимцев, 

\vs 8Sb 1:113 Больше уж на алтарях не прольется жертвенной крови, 

\vs 8Sb 1:114 Не зазвучат ни тимпан, ни кимвал \ldots

\vs 8Sb 1:115 Флейты многоотверстной безумных звуков не станет, 

\vs 8Sb 1:116 Нет и песни свирели, подобной извивам дракона, 

\vs 8Sb 1:117 Нет и варварских труб, что людям войну возвещают, 

\vs 8Sb 1:118 И не упьется вином уж никто на пирах нечестивых, ъ

\vs 8Sb 1:119 Нет ни плясок, ни пенья кифары; исчезнет коварство,

\vs 8Sb 1:120 Ссоры и гнев многовидный, и острых ножей не имеют 

\vs 8Sb 1:121 Те, кто жизнь завершил; единый лишь век остается. 

\vs 8Sb 1:122 Ключник великой ограды, привратник Божьего трона \ldots

\vs 8Sb 1:123 Идолы пусть вас украсят из золота, дерева, камня, 

\vs 8Sb 1:124 Пусть простоят до тех пор, когда день самый горький настанет,

\vs 8Sb 1:125 Чтоб твою кару узнать, о Рим, и вопли услышать. 

\vs 8Sb 1:126 Шею больше как раб под ярмо твое не преклонят 

\vs 8Sb 1:127 Ни Сириец, ни Эллин, ни варвар, ни племя иное. 

\vs 8Sb 1:128 Ждет разграбленье тебя, воздаcтся за то, что творил ты, 

\vs 8Sb 1:129 Будешь от страха стенать, пока за все не отплатишь.

\vs 8Sb 1:130 Целый мир над тобой, опозоренным, справит победу \ldots

\vs 8Sb 1:131 \ldots\ И в поколенье шестом царей Латинских угаснут 

\vs 8Sb 1:132 Жизни остатки, и руки удерживать скиптры не смогут. 

\vs 8Sb 1:133 Будет властвовать царь другой из того поколенья, 

\vs 8Sb 1:134 Скиптры все подчинит и землю всю покорит он, 

\vs 8Sb 1:135 Править будет один, но по воле Всевышнего Бога; 

\vs 8Sb 1:136 Дети и внуки его составят род нерушимый. 

\vs 8Sb 1:137 Так назначено Богом, когда круг времен совершится 

\vs 8Sb 1:138 И трижды пять царей над Египтом правленье закончат.

\vs 8Sb 1:139 Как подойдет к концу век птицы Феникса пятый \ldots

\vs 8Sb 1:140 \ldots\ Явится тут погубитель родов и племен без разбора, 

\vs 8Sb 1:141 И между ними  Евреев, Арес уничтожит Ареса, 

\vs 8Sb 1:142 Сам ведь на смерть обречет он угрозы надменные Рима. 

\vs 8Sb 1:143 Прежде цветущая, пала могучая Римлян держава, 

\vs 8Sb 1:144 Издавна всех городов, окрест лежащих, царица.

\vs 8Sb 1:145 Больше не будет победы для тучной Римской равнины, 

\vs 8Sb 1:146 После того как придет из Азии войск предводитель. 

\vs 8Sb 1:147 Все совершив, он захватит великий град; и в то время, 

\vs 8Sb 1:148 Как трижды триста исполнишь годов и еще сорок восемь, 

\vs 8Sb 1:149 Участь несчастная ждет тебя, от насилья погибнешь 

\vs 8Sb 1:150 И даже имя твое навеки будет забыто.

\vs 8Sb 1:151 Трижды несчастной, увы мне! когда же я день тот увижу, 

\vs 8Sb 1:152 Гибель несущий тебе, о Рим, и роду Латинян? 

\vs 8Sb 1:153 Радуйся, если желаешь, тому, кто, тайно рожденный 

\vs 8Sb 1:154 В Азии где-то, взошел на Троянскую колесницу, 

\vs 8Sb 1:155 Гневом кипя. Но когда перешеек Истмийский пробьет он, 

\vs 8Sb 1:156 Все озирая вокруг, враждебный ко всем, через море 

\vs 8Sb 1:157 Переплывет  тогда вслед за зверем великим польется 

\vs 8Sb 1:158 Черная кровь, но собака догонит губителя стада, 

\vs 8Sb 1:159 Льва; и, скиптра лишенный, пойдет он в царство Аида.

\vs 8Sb 1:160 Будет Родосцев несчастье последнее самым ужасным, 

\vs 8Sb 1:161 Фивы позорный захват ожидает затем, а Египет 

\vs 8Sb 1:162 От преступлений вождей своих негодных погибнет. 

\vs 8Sb 1:163 Тех же из смертных, кто смог избежать погибели страшной, 

\vs 8Sb 1:164 Трижды счастливыми нужно считать, и четырежды даже.

\vs 8Sb 1:165 Рим будет улочкой жалкой, а Делос невидимым станет, 

\vs 8Sb 1:166 Самос в песок превратится \ldots

\vs 8Sb 1:167 После придут, наконец, и к Персам великие беды 

\vs 8Sb 1:168 Карой за нрав их надменный  и вся гордыня исчезнет.

\vs 8Sb 1:169 Вождь святой подчинит себе все скиптры земные 

\vs 8Sb 1:170 С этой поры навсегда, и умерших от сна он пробудит.

\vs 8Sb 1:171 Волей Всевышнего жребий несчастный выпадет Риму:

\vs 8Sb 1:172 Всех, кто в городе этом живет, обрек Он на гибель.

\vs 8Sb 1:173 Но не хотят покориться, хоть лучше бы им это было.

\vs 8Sb 1:174 В пору, как день тот созреет, великой бедою чреватый  

\vs 8Sb 1:175 Голодом, мором и шумом войны, для людей нестерпимым,

\vs 8Sb 1:176 Снова тогда созовет несчастный, что раньше владыкой

\vs 8Sb 1:177 Был над Римом, совет, чтобы гибель тому уготовить \ldots

\vs 8Sb 1:178 \ldots\ Листья, едва распустившись, тотчас же станут сухими,

\vs 8Sb 1:179 Только на твердые скалы дожди с небосвода польются, 

\vs 8Sb 1:180 Почве же только ветра и огонь достанется жаркий, 

\vs 8Sb 1:181 Много семян оттого зазря пропадет в целом мире \ldots

\vs 8Sb 1:182 \ldots\ Зло продолжают творить, ибо всякий стыд потеряли 

\vs 8Sb 1:183 И не боятся уже ни людского, ни Божьего гнева, 

\vs 8Sb 1:184 Срам утратили все, променяли его на безстыдство; 

\vs 8Sb 1:185 Много насилий творят, закон попирают, тираны, 

\vs 8Sb 1:186 Лгут, обещаний не держат, ни слова правды не молвят, 

\vs 8Sb 1:187 Любят надутые речи, а веру поносят и гонят; 

\vs 8Sb 1:188 Нет для них насыщенья в богатстве, стремятся безстыдно 

\vs 8Sb 1:189 Больше и больше собрать  и сгинут под гнетом тиранов.

\vs 8Sb 1:190 Звезды все упадут прямо с неба в пучину морскую, 

\vs 8Sb 1:191 Но вместо них взойдут другие, одну из которых, 

\vs 8Sb 1:192 Что ярче всех заблестит, назовут кометою люди, 

\vs 8Sb 1:193 Знаменьем станет она войны и множества бедствий.

\vs 8Sb 1:194 Нет, не хотела б я жить во дни правленья нечистой, 

\vs 8Sb 1:195 Страстно желала б, напротив, когда небесная милость 

\vs 8Sb 1:196 В мире царицею станет, а всех коварных злодеев 

\vs 8Sb 1:197 Сын святой закует в оковы и в страшную бездну 

\vs 8Sb 1:198 Бросит  и смертных внезапно обнимет дом деревянный.

\vs 8Sb 1:199 После того как сойдет поколенье десятое в Тартар, 

\vs 8Sb 1:200 Женщина властью великой тогда завладеет, и много 

\vs 8Sb 1:201 Бед Господь низпошлет, когда она увенчает 

\vs 8Sb 1:202 Царским венцом главу  все в мире изменится сразу. 

\vs 8Sb 1:203 Жаркое солнце свой бег являть будет людям и ночью, 

\vs 8Sb 1:204 Звезды с неба исчезнут, и бешеный вихрь пронесется, 

\vs 8Sb 1:205 Опустошая весь мир, и мертвые всюду воскреснут, 

\vs 8Sb 1:206 Быстро хромые пойдут, и слышать смогут глухие, 

\vs 8Sb 1:207 Зренье получат слепцы, обретут дар речи немые. 

\vs 8Sb 1:208 Жизнь и богатство тогда для смертных общими будут, 

\vs 8Sb 1:209 Общей и вся земля; и, быть перестав разделенной 

\vs 8Sb 1:210 Стенами и рубежами, сама даст плод изобильный 

\vs 8Sb 1:211 И родники молока белоснежного, сладкого меда

\vs 8Sb 1:212 Даст и вино источит \ldots

\vs 8Sb 1:213 Суд бессмертного Бога \ldots

\vs 8Sb 1:214 Бог переменит все времена \ldots

\vs 8Sb 1:215 Зиму сделает летом; да сбудутся все предсказанья. 

\vs 8Sb 1:216 Но после гибели мира \ldots

\vs 8Sb 1:217 Из земли источит близость Судного дня капли пота,

\vs 8Sb 1:218 И снизойдет с небес тот Царь, что вечно пребудет.

\vs 8Sb 1:219 Суд учинит Он великий над миром и всякою плотью, 

\vs 8Sb 1:220 Узрят и верные Бога, и все неверные тоже,

\vs 8Sb 1:221 С высей как спустится Он в конце времен со святыми.

\vs 8Sb 1:222 Души плотских людей придут для суда к Его трону;

\vs 8Sb 1:223 Худо придется земле от засухи страшной и терний;

\vs 8Sb 1:224 Разных кумиров и все богатство смертные бросят, 

\vs 8Sb 1:225 И проникающий всюду огонь и сушу, и море

\vs 8Sb 1:226 Сгубит, и небосвод, и крепкие двери Аида.

\vs 8Sb 1:227 Тем, кто был жизни святой, свободы свет засияет;

\vs 8Sb 1:228 Огненной каре навек предаст нечестивцев Всевышний,

\vs 8Sb 1:229 Скажет всякий из них о зле, что тайно творил он, 

\vs 8Sb 1:230 Ибо сердечная тьма озарится светом Господним.

\vs 8Sb 1:231 Скрежет зубовный тогда и плач всеобщий раздастся,

\vs 8Sb 1:232 Солнца померкнет сиянье, и скроются звезд хороводы,

\vs 8Sb 1:233 Небо свернется как свиток, луны мерцанье угаснет,

\vs 8Sb 1:234 Высями станут долины, в низины холмы обратятся. 

\vs 8Sb 1:235 Больше высот на земле губительных вовсе не будет,

\vs 8Sb 1:236 Общее примут обличье равнины и горные кряжи;

\vs 8Sb 1:237 Глади морской не коснутся суда; земля запылает,

\vs 8Sb 1:238 А источники рек и бурные воды изсякнут.

\vs 8Sb 1:239 С неба труба пропоет печальным голосом песню, 

\vs 8Sb 1:240 Страшный позор несчастных оплачет и мира мученья,

\vs 8Sb 1:241 Пропасти, в почве разверзшись, покажут Тартара хаос,

\vs 8Sb 1:242 А пред небесным престолом Господним все люди сойдутся.

\vs 8Sb 1:243 С неба потоки огня и серы хлынут на землю.

\vs 8Sb 1:244 Будет для смертных тогда непреложным знаком, печатью, 

\vs 8Sb 1:245 Древом для верных тот рог, о котором так долго мечтали,

\vs 8Sb 1:246 Камень соблазна для мира, но жизнь для мужей справедливых,

\vs 8Sb 1:247 Радость света для званых двенадцатью водами давший,

\vs 8Sb 1:248 Есть у него и жезл железный, чтоб смертными править.

\vs 8Sb 1:249 Сам Господь наш Небесный записан тут акростихами  

\vs 8Sb 1:250 То Спаситель безсмертный и Царь, за нас пострадавший.

\vs 8Sb 1:251 Запечатлел же Его еще Моисей, распростерши 

\vs 8Sb 1:252 Руки святые, когда победил Амалика он верой; 

\vs 8Sb 1:253 Понял тогда народ, что избран Богом и славен 

\vs 8Sb 1:254 Жезл Давыда и Камень, что был заранее предсказан: 

\vs 8Sb 1:255 Тот, кто поверит в Него, сподобится жизни безсмертной.

\vs 8Sb 1:256 Он не во славе придет на суд, но как смертный несчастный, 

\vs 8Sb 1:257 Без красоты, без почета, чтоб дать несчастным надежду; 

\vs 8Sb 1:258 Тленному телу придаст Он форму, неверным дарует 

\vs 8Sb 1:259 Веру небесную Он, и вылепит вновь человека,

\vs 8Sb 1:260 Коего Сам Господь творил Своими руками. 

\vs 8Sb 1:261 Но обманул человека коварный змей и заставил 

\vs 8Sb 1:262 Смертную участь принять и познать благое и злое; 

\vs 8Sb 1:263 Так люди бросили Бога и тленному кланяться стали. 

\vs 8Sb 1:264 Сына в советники взяв изначально, рек Вседержитель:

\vs 8Sb 1:265 Сделаем вместе с Тобою, о Чадо, смертное племя, 

\vs 8Sb 1:266 Слепим его, отразив в нем Наше с Тобою обличье. 

\vs 8Sb 1:267 Я руками теперь, Ты позже словом послужишь 

\vs 8Sb 1:268 Делу Нашему, чтобы оно стало общим твореньем. 

\vs 8Sb 1:269 Помня об этом решенье, сойдет для суда Он на землю,

\vs 8Sb 1:270 В Деву святую вселившись подобным отображеньем, 

\vs 8Sb 1:271 Свет водой даровав через руки того, кто был старше, 

\vs 8Sb 1:272 Все Своим словом творя и любую болезнь исцеляя. 

\vs 8Sb 1:273 Словом же Он успокоит ветра и море разгладит, 

\vs 8Sb 1:274 Всюду покой принесет и веру, по свету скитаясь.

\vs 8Sb 1:275 Он хлебами пятью и рыбой из моря одною

\vs 8Sb 1:276 Целых пять тысяч мужей легко насытит в пустыне; 

\vs 8Sb 1:277 После, собрав те куски, что остались от пищи, наполнит 

\vs 8Sb 1:278 Ими двенадцать корзин, да придет надежда к народам. 

\vs 8Sb 1:279 Вызовет души блаженных и тех несчастных возлюбит,

\vs 8Sb 1:280 Кто, подвергаясь насмешкам, отплатит за зло только благом, 

\vs 8Sb 1:281 Кто нищету возлюбил, кто гоним и бичами терзаем. 

\vs 8Sb 1:282 Все доступно Его уму, Он все видит и слышит, 

\vs 8Sb 1:283 Узрит и то, что таится внутри, обнажит и очистит, 

\vs 8Sb 1:284 Ибо Он сам разуменье, и слух, и зренье всех сущих,

\vs 8Sb 1:285 Он же и Слово-Творец всех форм, и Ему все покорно. 

\vs 8Sb 1:286 Мертвых Он воскресит, исцелит любые болезни. 

\vs 8Sb 1:287 Он попадет, наконец, в безбожные руки неверных, 

\vs 8Sb 1:288 Станут грешной рукой наносить пощечины Богу, 

\vs 8Sb 1:289 Полную яда слюну из грязных уст извергая.

\vs 8Sb 1:290 Спину святую Свою ударам кнута Он подставит, 

\vs 8Sb 1:291 Ибо Он миру придет отдать непорочную Деву. 

\vs 8Sb 1:292 Будет молчанье хранить под ударами, чтоб не узнали, 

\vs 8Sb 1:293 Кто Он, чей и откуда; но к падшим речет Свое слово. 

\vs 8Sb 1:294 И увенчают Его венцом терновым, и станет

\vs 8Sb 1:295 Шип колючий наградой святым избранникам вечной. 

\vs 8Sb 1:296 Во исполненье закона пронзят тростником Его ребра, 

\vs 8Sb 1:297 Ведь подготовил тростник, не простым колеблемый ветром, 

\vs 8Sb 1:298 Душу Его к осужденью, обидам и наказанью. 

\vs 8Sb 1:299 Как совершится все то, что ныне предсказано мною,

\vs 8Sb 1:300 В Нем растворится всецело закон, что прежде народу 

\vs 8Sb 1:301 Дан непокорному был, в словах человечьих записан. 

\vs 8Sb 1:302 Руки раскинет и все, что есть в мире, Он ими обнимет; 

\vs 8Sb 1:303 Желчью кормили Его и уксусом горьким поили, 

\vs 8Sb 1:304 Кару получат они за враждебную трапезу эту.

\vs 8Sb 1:305 В храме порвется завеса, и день превратится внезапно 

\vs 8Sb 1:306 В ночь, что на три часа всю землю мраком покроет. 

\vs 8Sb 1:307 Скрытое бреднями мира, теперь станет ясно: не нужно 

\vs 8Sb 1:308 Больше храм почитать и закон, для людей непонятный  

\vs 8Sb 1:309 Сам ведь на землю сошел Безсмертный Владыка Небесный.

\vs 8Sb 1:310 Спустится после в Аид и для праведных вестником будет 

\vs 8Sb 1:311 Доброй надежды и дня, которым века прекратятся. 

\vs 8Sb 1:312 Смертный удел победит Он, на третий день пробудившись;

\vs 8Sb 1:313 Мертвых покинет тогда и вновь появится в мире, 

\vs 8Sb 1:314 Тем лишь, кто избран, сперва открыв воскресенья начало;

\vs 8Sb 1:315 Смоет водой родника, что дает безсмертия влагу, 

\vs 8Sb 1:316 Всякое прежнее зло, и, заново свыше родившись, 

\vs 8Sb 1:317 Быть перестанут рабами обычаев мира безбожных. 

\vs 8Sb 1:318 Явится прежде Господь своим, чтоб увидели ясно: 

\vs 8Sb 1:319 Вновь Он пришел во плоти, как прежде был, и покажет

\vs 8Sb 1:320 Им на руках и ногах четыре следа кровавых, 

\vs 8Sb 1:321 То будут Севера, Юга, Востока и Запада знаки, 

\vs 8Sb 1:322 Или число тех царств, что в мире свершат нечестиво 

\vs 8Sb 1:323 Дело позорное, кое нам всем в осужденье послужит.

\vs 8Sb 1:324 Дочь святая Сиона, ты много бед претерпела, 

\vs 8Sb 1:325 Радуйся ныне! Твой царь въезжает на ослике в Город, 

\vs 8Sb 1:326 Кротко, смиренно грядет, чтобы наше рабское иго, 

\vs 8Sb 1:327 Тяжко давившее спины, теперь упразднилось навеки, 

\vs 8Sb 1:328 Чтобы неправый закон исчез и гнетущие цепи.

\vs 8Sb 1:329 Так почти же Его как Бога и Божьего Сына, 

\vs 8Sb 1:330 В сердце свое прими и славь Его в радостных гимнах,

\vs 8Sb 1:331 Всею душой возлюби и себе возьми Его имя.

\vs 8Sb 1:332 Прежних всех удали и кровь, Им пролитую, смой ты;

\vs 8Sb 1:333 Жалких воплей твоих, и тленных жертв, и молений

\vs 8Sb 1:334 Вовсе не нужно Тому, Кто Сам безсмертен и вечен, 

\vs 8Sb 1:335 Но песнопений Он хочет из уст и сердца святого.

\vs 8Sb 1:336 Знай же, Кто Он таков  тогда и Отца Его узришь.

\vs 8Sb 1:337 Мира все элементы в то время придут в запустенье: 

\vs 8Sb 1:338 Воздух, море, земля и свет, от огня исходящий, 

\vs 8Sb 1:339 Ось небесная, ночь и все дни воедино сольются,

\vs 8Sb 1:340 В пламени формы свои они совершенно утратят. 

\vs 8Sb 1:341 Все светоносные звезды с небес упадут и исчезнут; 

\vs 8Sb 1:342 В воздухе больше не станут летать крылатые птицы, 

\vs 8Sb 1:343 Землю не тронет нога, ибо твари живые погибнут; 

\vs 8Sb 1:344 Смолкнут все голоса  людей, зверей и пернатых,

\vs 8Sb 1:345 Миру в его неустройстве не будет звука на пользу, 

\vs 8Sb 1:346 Но угрожающий шум издаст глубокое море, 

\vs 8Sb 1:347 Жители вод содрогнутся, и тут же конец им настанет; 

\vs 8Sb 1:348 Не поплывет по волнам и судно, везущее грузы. 

\vs 8Sb 1:349 Тяжко застонет земля, в сраженьях политая кровью;

\vs 8Sb 1:350 И заскрежещут зубами тогда все души людские,

\vs 8Sb 1:351 Грешные души погибель в стенаньях и ужасе встретят. 

\vs 8Sb 1:352 Жажда будет их жечь, убийства, болезни и голод, 

\vs 8Sb 1:353 И пожелают они умереть, но больше не смогут: 

\vs 8Sb 1:354 Не успокоит их смерть, и ночь не даст передышки.

\vs 8Sb 1:355 Долго Всевышнего Бога молить они будут напрасно  

\vs 8Sb 1:356 И отвратит Господь Свой лик, чтоб их больше не видеть: 

\vs 8Sb 1:357 Ибо ведь людям заблудшим Он семь веков предоставил 

\vs 8Sb 1:358 Для покаянья  за них просила Дева святая.

\vs 8Sb 1:359 Все эти речи Сам Бог вложил мне в сердце, и нужно, 

\vs 8Sb 1:360 Чтобы сбылось непременно все то, что уста мои молвят. 

\vs 8Sb 1:361 Мне известно число песчинок и вод в Океане, 

\vs 8Sb 1:362 Знаю земли тайники и Тартара мрачное царство, 

\vs 8Sb 1:363 Знаю все звезды на небе, и все деревья, и сколько 

\vs 8Sb 1:364 В мире четвероногих, плавучих и птиц оперенных, 

\vs 8Sb 1:365 Сколько людей живет, жило раньше и позже родится.

\vs 8Sb 1:366 В смертных запечатлел Я Сам и облик, и разум, 

\vs 8Sb 1:367 Дал им здравую мысль, научил их знаниям всяким,

\vs 8Sb 1:368 Создал Я ухо и глаз, и Сам все вижу и слышу, 

\vs 8Sb 1:369 Мысли чувствую все и всех событий Свидетель,

\vs 8Sb 1:370 Тот, который в начале молчит, а потом обличает, 

\vs 8Sb 1:371 Тяжко за тайное зло карая любого из смертных, 

\vs 8Sb 1:372 И у Господнего трона людей собеседником будет. 

\vs 8Sb 1:373 Я и глухого пойму и даже немого услышу. 

\vs 8Sb 1:374 Знаю и то, каково от земли до небес расстоянье,

\vs 8Sb 1:375 Знаю конец и начало, ведь Я создал небо и землю, 

\vs 8Sb 1:376 Создал Он все  от истока Он ведает все до предела. 

\vs 8Sb 1:377 Я  единственный Бог, иного не узрите Бога. 

\vs 8Sb 1:378 Люди подобье Мое деревянное обожествили, 

\vs 8Sb 1:379 Сделав своей же рукой изваянья безгласные, стали

\vs 8Sb 1:380 Кланяться им и молиться, служа нечестивые службы. 

\vs 8Sb 1:381 Чтили то, что несвято, Творца же совсем позабыли, 

\vs 8Sb 1:382 Все, рожденные Мной, приносят ненужные жертвы, 

\vs 8Sb 1:383 Точно во славу Мою, и достойным это считают, 

\vs 8Sb 1:384 Дым воскуряют, как будто заботясь о собственных мертвых.

\vs 8Sb 1:385 Мясо спаляют они и кости, полные мозга, 

\vs 8Sb 1:386 Жертвуя на алтарях и демонам кровь возливая. 

\vs 8Sb 1:387 Свечи Мне возжигают, хоть Я же свет даровал им, 

\vs 8Sb 1:388 Смертные Богу вино возливают, словно Он жаждет, 

\vs 8Sb 1:389 И напиваются сами у ног изваяний никчемных.

\vs 8Sb 1:390 Ваших не нужно Мне жертв, не нужно и возлияний; 

\vs 8Sb 1:391 Мне грязный дым неугоден и мерзостной крови потоки. 

\vs 8Sb 1:392 В память царей и тиранов свершают службы такие, 

\vs 8Sb 1:393 Кланяясь демонам мертвым, как будто в небе живущим,  

\vs 8Sb 1:394 Так почитаньем безбожным они себе гибель готовят.

\vs 8Sb 1:395 Бога лишенные люди зовут изваянья богами, 

\vs 8Sb 1:396 Но позабыли Творца и видят жизнь и надежду 

\vs 8Sb 1:397 В идолах, что не лишены и слуха, и голоса вовсе. 

\vs 8Sb 1:398 В злых лишь надежны делах, а добра не имеют и в мыслях.

\vs 8Sb 1:399 Мною даны два пути: есть к жизни дорога и к смерти,

\vs 8Sb 1:400 Я же благой дал совет  держаться праведной жизни, 

\vs 8Sb 1:401 Но предпочли эти люди погибель и вечное пламя. 

\vs 8Sb 1:402 Образ Мой  человек, наделенный здравою мыслью: 

\vs 8Sb 1:403 Вот ему и подай не кровавую  чистую пищу, 

\vs 8Sb 1:404 Сделай добро: удели тому, кто голоден, хлеба,

\vs 8Sb 1:405 Если жаждет  воды, если наг  одежды для тела;

\vs 8Sb 1:406 Милость твори от своих же трудов и ладонью святою. 

\vs 8Sb 1:407 Кто-то в унынье  утешь, устал кто  приди на подмогу: 

\vs 8Sb 1:408 Эту жертву живую воздай ты Богу Живому. 

\vs 8Sb 1:409 Сей зерно благочестья, тогда в награду получишь

\vs 8Sb 1:410 Плод безсмертный и свет негасимый, и жизни нетленной 

\vs 8Sb 1:411 Будешь достоин, когда всех людей огонь испытает. 

\vs 8Sb 1:412 Сплавлю Я все воедино и вновь разниму и очищу, 

\vs 8Sb 1:413 Небо сверну Я, как свиток, открою недра земные, 

\vs 8Sb 1:414 Мертвых в тот день воскрешу и судьбы людские разрушу,

\vs 8Sb 1:415 Жало смерти Я вырву; и суд последний устрою: 

\vs 8Sb 1:416 Стану жизни судить и праведных, и нечестивых, 

\vs 8Sb 1:417 Рядом встанет баран с бараном, и с пастырем пастырь, 

\vs 8Sb 1:418 Рядом с тельцом телец, чтоб друг друга они обличили. 

\vs 8Sb 1:419 Здесь обличатся все те, кто в жизни своей возвышался,

\vs 8Sb 1:420 Рты затыкая другим, и к зависти их побуждали, 

\vs 8Sb 1:421 Дабы и те справедливых людей в рабов обращали, 

\vs 8Sb 1:422 Всех заставляли молчать, влекомые только наживой. 

\vs 8Sb 1:423 Те же, кто праведно жил, все встанут рядом со Мною. 

\vs 8Sb 1:424 Больше в печали никто не скажет: То будет завтра,

\vs 8Sb 1:425 Или: То было вчера; и дней, заботами полных, 

\vs 8Sb 1:426 Также не станет, исчезнут четыре времени года, 

\vs 8Sb 1:427 С ними Восход и Закат, и все в долгий день обратится. 

\vs 8Sb 1:428 Свет пребудет вовеки, желанный и долгожданный  

\vs 8Sb 1:429 Навсегда Иисус Христос \ldots

\vs 8Sb 1:430 Боже, никем не рожденный, Чистейший, Вечный, Безсмертный!

\vs 8Sb 1:431 В небе живешь Ты и властью Своей усмиряешь дыханье 

\vs 8Sb 1:432 Пламени и поражаешь огнем могущество скиптров, 

\vs 8Sb 1:433 Гулких грома раскатов легко укрощаешь Ты силу, 

\vs 8Sb 1:434 Движешь Ты землю и держишь в узде волненье морское, 

\vs 8Sb 1:435 И ослабляешь бичи сверкающих огненных молний,

\vs 8Sb 1:436 Мощные ливней потоки, падение града и снега, 

\vs 8Sb 1:437 Что из тучи морозной летит, и бури порывы \ldots

\vs 8Sb 1:438 Ангелы, слуги Твои, заботливо распределяют

\vs 8Sb 1:439 Все, что Тобой решено и чему повелел Ты свершиться.

\vs 8Sb 1:440 Прежде творенья всего на совет призвав Свое Чадо, 

\vs 8Sb 1:441 Создал Ты смертных людей и жизни дал зародиться.

\vs 8Sb 1:442 Первым к Нему обратился Ты сладостной речью Твоею: 

\vs 8Sb 1:443 Сделаем ныне с Тобою подобного Нам человека 

\vs 8Sb 1:444 И дающее жизнь дыхание в грудь его вложим. 

\vs 8Sb 1:445 Смертен он будет, но все пускай на земле ему служит,

\vs 8Sb 1:446 Все ему подчиним, хоть его мы слепим из глины. 

\vs 8Sb 1:447 Это Ты Слову сказал, и стало так, как решил Ты. 

\vs 8Sb 1:448 Тотчас же все элементы велениям повиновались, 

\vs 8Sb 1:449 И со смертным твореньем все вечное соединилось: 

\vs 8Sb 1:450 Небо, воздух, огонь, земля и волны морские,

\vs 8Sb 1:451 Солнце, луна и созвездья \ldots

\vs 8Sb 1:452 Дни и ночи, и сон, пробуждение, дух и движенье, 

\vs 8Sb 1:453 И душа, и разсудок, искусство, сила и голос, 

\vs 8Sb 1:454 Стаи диких зверей, пернатые птицы и рыбы, 

\vs 8Sb 1:455 Те, кто ходит, и те, кто ползут, и амфибии также 

\vs 8Sb 1:456 Все человеку Он дал, Твоим решеньем наставлен.

\vs 8Sb 1:457 А у конца времен изменил Он землю: 

\vs 8Sb 1:458 Младенец Тело девы Марии покинул, и новый зажегся 

\vs 8Sb 1:459 Свет: пришел Он с небес и смертное принял обличье. 

\vs 8Sb 1:460 Мощный телом своим, сперва Гавриил появился,

\vs 8Sb 1:461 После же голос возвысил архангел и деве сказал он: 

\vs 8Sb 1:462 В чистое лоно свое прими ныне Бога, о дева! 

\vs 8Sb 1:463 Рек, и вдохнул благодать ей Господь; она же, услышав, 

\vs 8Sb 1:464 В страхе была и в смущенье, поскольку мужа не знала. 

\vs 8Sb 1:465 С места сойти не могла, в испуге дрожала, и сердце

\vs 8Sb 1:466 Быстро билось в груди от вести, еще непонятной. 

\vs 8Sb 1:467 Вскоре, однако, слова проникли в сердце, и тут же 

\vs 8Sb 1:468 Смех ее охватил, разрумянились юные щеки; 

\vs 8Sb 1:469 Перемешались в душе у девицы смущенье и радость, 

\vs 8Sb 1:470 И осмелела она. А слово, влетевшее в чрево,

\vs 8Sb 1:471 Плотью со временем стало и, в теле ожив материнском, 

\vs 8Sb 1:472 Облик людской обрело; и Мальчик на свет появился, 

\vs 8Sb 1:473 Девой рожден. Без сомненья  для смертных великое чудо, 

\vs 8Sb 1:474 Но не бывает великих чудес для Отца и для Сына. 

\vs 8Sb 1:475 Чада рожденье земле принесло великую радость,

\vs 8Sb 1:476 Возвеселился и в небе Престол, и мир  в ликованье. 

\vs 8Sb 1:477 Маги воздали честь звезде, невиданной прежде, 

\vs 8Sb 1:478 И, уверовав в Бога, лежащего в яслях узрели; 

\vs 8Sb 1:479 Пасшие коз и овец Дитя увидели также. 

\vs 8Sb 1:480 И наречен Вифлеем богоизбранный родиной Слова.

\vs 8Sb 1:481 \ldots\ Нужно в сердце иметь смиренье и зло ненавидеть, 

\vs 8Sb 1:482 К ближним питать любовь такую, как к собственной жизни, 

\vs 8Sb 1:483 Бога душою любить и Ему воздавать почитанье. 

\vs 8Sb 1:484 Ибо ведь мы от Него и святого Рожденья Христова 

\vs 8Sb 1:485 Небом произведены и единую кровь получили.

\vs 8Sb 1:486 Богу служа, мы всегда о блаженстве будущем помним, 

\vs 8Sb 1:487 Правды дорогой идя, прямым путем благочестья. 

\vs 8Sb 1:488 В храмы язычников мы никогда входить не дерзаем, 

\vs 8Sb 1:489 Нет для кумиров у нас ни молитвы, ни возлияний, 

\vs 8Sb 1:490 Ни аромата цветов, ни огня светильников ярких,

\vs 8Sb 1:491 Ни приношеньем даров богатых мы не почтим их 

\vs 8Sb 1:492 И благовонья на их алтарях никогда не воскурим; 

\vs 8Sb 1:493 В жертву быков и овец приносить им не станем, пытаясь 

\vs 8Sb 1:494 От наказания их таким путем откупиться; 

\vs 8Sb 1:495 Жирным дымом костра, который плоть пожирает,

\vs 8Sb 1:496 Мы осквернять не хотим сияния ясного неба.

\vs 8Sb 1:497 Но в помышлениях чистых, ликуя радостным сердцем, 

\vs 8Sb 1:498 Щедро дающей рукой и богатством любви безконечным, 

\vs 8Sb 1:499 Сладостной песней и в гимнах, достойных великого Бога, 

\vs 8Sb 1:500 Должно Тебя воспевать нам, неложно и непрестанно 

\vs 8Sb 1:501 Мудрый мира Создатель \ldots

\bibbookdescr{9Sb}{
  inline={Девятая книга Сивилл},
  toc={9-я Сивилл},
  bookmark={9-я Сивилл},
  header={9-я Сивилл},
  abbr={9~Сив}
}
\vs 9Sb 1:1 Смертные плотские люди, ведь вы совершенно ничтожны.

\vs 9Sb 1:2 Как же столь быстро в гордыне возноситесь вы, что забыли

\vs 9Sb 1:3 Вовсе о смерти и страхе пред Богом, Который все видит?

\vs 9Sb 1:4 Он, Высочайший, всеведущ, всех дел Он зоркий свидетель,

\vs 9Sb 1:5 Он, питающий все,  Создатель, вложивший дыханье

\vs 9Sb 1:6 Сладкое всем, и Вождя для смертных всех сотворивший.

\vs 9Sb 1:7 Бог  единый Владыка, безмерный и нерожденный,

\vs 9Sb 1:8 Правит всем и невидим, а Сам все сущее видит.

\vs 9Sb 1:9 Тленна плоть, для нее Господь остается незримым,

\vs 9Sb 1:10 Ибо ведь истинный Бог, Безсмертный, в небе живущий,

\vs 9Sb 1:11 Разве же плотским очам доступным сделаться может?

\vs 9Sb 1:12 Яркого солнца лучей не могут вынести люди,

\vs 9Sb 1:13 Смертны они, нелегко им стоять против света такого,

\vs 9Sb 1:14 Кости и жилы одни их плоть связуют собою.

\vs 9Sb 1:15 Чтите только Его, единого мира Владыку,

\vs 9Sb 1:16 Он лишь сущий от века и впредь вовеки пребудет,

\vs 9Sb 1:17 Сам он возник, нерожденный, и правит всем во вселенной,

\vs 9Sb 1:18 В каждом смертном живет и к свету путь указует.

\vs 9Sb 1:19 За неразумье свое обретете достойную плату 

\vs 9Sb 1:20 Истинно сущего Бога Безсмертного чтить перестали,

\vs 9Sb 1:21 Нет у вас для Него теперь гекатомбы священной,

\vs 9Sb 1:22 Ныне приносите жертвы Аида мрачного духам.

\vs 9Sb 1:23 Вы в слепоте и безумье бредете. Дорогу прямую,

\vs 9Sb 1:24 Ту что, верно ведет, покинули и заблудились

\vs 9Sb 1:25 Все средь шипов и колючек. О смертные, остановитесь!

\vs 9Sb 1:26 Бросьте скитаться во тьме и в черной ночи безпросветной 

\vs 9Sb 1:27 Мрак оставьте ночной, примите свет благодатный!

\vs 9Sb 1:28 Вот, он  видим для всех, и в нем нельзя ошибиться.

\vs 9Sb 1:29 Так подойдите сюда, не гонитесь за мраком и мглою  

\vs 9Sb 1:30 Солнца сладостный свет, смотрите, как ярко сияет!

\vs 9Sb 1:31 Знайте это и ныне в сердца свои мудрость вложите:

\vs 9Sb 1:32 Бог един  и дожди, и ветра, и мор, и голод, и беды,

\vs 9Sb 1:33 И снегопады, и лед. Смогу ли все перечислить?

\vs 9Sb 1:34 Он  и Небесный вождь, и Земной Владыка, и Сущий \ldots

\vs 9Sb 1:35 Если б рождалися боги и смерти при этом не знали,

\vs 9Sb 1:36 Смертных людей числом они превзошли бы намного,

\vs 9Sb 1:37 И не осталось бы места, где люди селиться могли бы \ldots

\vs 9Sb 1:38 Если рожденное все должно и погибнуть, то разве

\vs 9Sb 1:39 Могут богов создавать мужские и женские чресла?

\vs 9Sb 1:40 Только один есть Бог, Высочайший и все сотворивший:

\vs 9Sb 1:41 Небо, яркое солнце, луну и горящие звезды,

\vs 9Sb 1:42 И плодородную землю, и воды глубокого моря,

\vs 9Sb 1:43 Гор могучие выси, источники, бьющие вечно.

\vs 9Sb 1:44 Создал Он без числа и тех, кто в воде обитает,

\vs 9Sb 1:45 Пищу и тем подает, кому предназначено ползать,

\vs 9Sb 1:46 Разных питает он птиц, певучих и самых крикливых

\vs 9Sb 1:47 И рассекающих громко своими крыльями воздух;

\vs 9Sb 1:48 Дикое племя зверей породил Он в горных ущельях,

\vs 9Sb 1:49 Всякий скот Он заставил покорно служить человеку.

\vs 9Sb 1:50 Он надо всеми вождями поставил творение Божье,

\vs 9Sb 1:51 Сделав слугами людям обилье вещей непонятных  

\vs 9Sb 1:52 Как же смертная плоть постичь столь многое может?

\vs 9Sb 1:53 Мог только Тот познать, Кто все сотворил изначально,

\vs 9Sb 1:54 В небе живущий Создатель, Господь Безсмертный и вечный.

\vs 9Sb 1:55 Он возвращает сторицей тому, чьи деяния благи,

\vs 9Sb 1:56 Но против злых нечестивцев ужасный гнев возбуждает,

\vs 9Sb 1:57 Войны и голод несет им, страданья и многие слезы.

\vs 9Sb 1:58 Что же вы тщетной гордыней себя вырываете с корнем?

\vs 9Sb 1:59 Чтить как богов постыдитесь куниц и разных чудовищ!

\vs 9Sb 1:60 Это безумье и бред отнимают всякое чувство,

\vs 9Sb 1:61 Если боги у вас горшки похищают и чаши.

\vs 9Sb 1:62 Вместо того, чтобы жить счастливо в небе чудесном,

\vs 9Sb 1:63 Здесь, на земле пауков боятся, и черви их точат.

\vs 9Sb 1:64 Вы поклоняетесь змеям, собакам и кошкам, невежды,

\vs 9Sb 1:65 Тех, кто по небу летает, и тех, кто ползает, чтите,

\vs 9Sb 1:66 Также кумиров из камня и рук человечьих творенье,

\vs 9Sb 1:67 Статуи и даже кучи камней, что лежат при дорогах, 

\vs 9Sb 1:68 Много нелепых вещей, о которых и молвить позорно.

\vs 9Sb 1:69 Боги также ведут неразумных смертных коварно,

\vs 9Sb 1:70 Лживые их уста смертоносный яд источают.

\vs 9Sb 1:71 Но перед Тем, Кто есть жизнь и вечный свет негасимый,

\vs 9Sb 1:72 Радость роду людскому, сладчайшего сладостней меда,

\vs 9Sb 1:73 Щедро дает  перед Ним лишь одним главы преклоняйте,

\vs 9Sb 1:74 Следуйте той тропой, что в праведный век проторили.

\vs 9Sb 1:75 Все это бросили вы, испили отмщения кубок 

\vs 9Sb 1:76 Крепок напиток и пьян, водою ничуть не разбавлен 

\vs 9Sb 1:77 И помутился тотчас ваш дух в безумии тяжком.

\vs 9Sb 1:78 Вы протрезветь не хотите, вернуться к мыли разумной,

\vs 9Sb 1:79 Сущего Бога узнать, Царя, Который все видит.

\vs 9Sb 1:80 Вот за такие дела придет к вам жгучее пламя,

\vs 9Sb 1:81 Будет вас вечно палить огонь, который не гаснет,

\vs 9Sb 1:82 И постыдитесь тогда обмана негодных кумиров.

\vs 9Sb 1:83 Те же, кто Господа чтут, воистину сущего Бога,

\vs 9Sb 1:84 Вечную жизнь обретут, и в жизни той бесконечной

\vs 9Sb 1:85 Все, поселившись в Раю, в саду, прекрасно цветущем,

\vs 9Sb 1:86 Сладостный хлеб вкусят, со звездного посланный неба \ldots

\bibbookdescr{10Sb}{
  inline={\LARGE Десятая книга\\\Huge Сивилл},
  toc={10-я Сивилл},
  bookmark={10 Sybille},
  header={10-я Сивилл},
  abbr={10~Сив}
}
\vs 10Sb 1:1 Тогда, как прахом всё уляжется в земле,

\vs 10Sb 1:2 И Бог, оставив мир без света в мрачной мгле,

\vs 10Sb 1:3 Из праха воззовет и кости человека,

\vs 10Sb 1:4 И смертных оживит для будущего века,

\vs 10Sb 1:5 Тогда последует правдивый Божий суд,

\vs 10Sb 1:6 Где по заслугам все достойное найдут:

\vs 10Sb 1:7 Нечестие, порок закроются землею,

\vs 10Sb 1:8 Но святость и любовь возвысятся над нею,

\vs 10Sb 1:9 Когда Превечный Судия изволит дать

\vs 10Sb 1:10 Благочестивым дух и жизнь и благодать.

\vs 10Sb 1:11 Тогда узнает верно всякий сам себя,

\vs 10Sb 1:12 И узрит же тогда преясно всяк себя \ldots

\bibbookdescr{Tho}{
  inline={Евангелие от Фомы},
  toc={От Фомы},
  bookmark={От Фомы},
  header={От Фомы},
  abbr={Фм}
}

\vs Tho 1:0
Это тайные слова,
которые сказал Иисус живой
и которые записал Дидим Иуда Фома.
И он сказал:
тот, кто обретает истолкование этих слов, не вкусит смерти.

\vs Tho 1:1
Иисус сказал: пусть тот, кто ищет,
не перестаёт искать до тех пор, пока не найдёт,
и, когда он найдёт, он будет потрясён,
и, если он потрясён, он будет удивлён,
и он будет царствовать над всем.

\vs Tho 1:2
Иисус сказал: если те, которые ведут вас, говорят вам:
смотрите, царствие в небе!~--- тогда птицы небесные
опередят вас.
Если они говорят вам, что оно~--- в море,
тогда рыбы опередят вас.
Но царствие внутри вас и вне вас.

\vs Tho 1:3
Когда вы познаете себя,
тогда вы будете познаны и вы узнаете,
что вы~--- дети Отца живого.
Если же вы не познаете себя,
тогда вы в бедности и вы~--- бедность.

\vs Tho 1:4
Иисус сказал: старый человек в его дни
не замедлит спросить малого ребёнка 7-ми дней о месте жизни,
и он будет жить.
Ибо много первых будут последними, и они станут одним.

\vs Tho 1:5
Иисус сказал: познай того, кто перед лицом твоим,
и тот, кто скрыт от тебя,~--- откроется тебе.
Ибо нет ничего тайного, что не будет явным.

\vs Tho 1:6
Ученики его спросили его;
они сказали ему: хочешь ли ты, чтобы мы постились,
и как нам молиться, давать милостыню и воздерживаться в пище?
Иисус сказал: не лгите, и то, что вы ненавидите, не делайте этого.
Ибо всё открыто перед небом.
Ибо нет ничего тайного, что не будет явным,
и нет ничего сокровенного, что осталось бы нераскрытым.

\vs Tho 1:7
Иисус сказал: блажен тот лев, которого съест человек,
и лев станет человеком.
И проклят тот человек, которого съест лев, и лев станет человеком.

\vs Tho 1:8
И он сказал: человек подобен мудрому рыбаку,
который бросил свою сеть в море.
Он вытащил её из моря, полную малых рыб;
среди них этот мудрый рыбак нашёл большую хорошую рыбу.
Он выбросил всех малых рыб в море, он без труда выбрал большую рыбу.
Тот, кто имеет уши слышать, да слышит!

\vs Tho 1:9
Иисус сказал:
вот, сеятель вышел, он наполнил свою руку, он бросил семена.
Но иные упали на дорогу, прилетели птицы, поклевали их.
Иные упали на камень, и не пустили корня в землю,
и не послали колоса в небо.
И иные упали в терния, они заглушили семя, и червь съел их.
И иные упали на добрую землю и дали добрый плод в небо.
Это принесло 60 мер на одну и 120 мер на одну.

\vs Tho 1:10
Иисус сказал:
я бросил огонь в мир, и вот я охраняю его, пока он не запылает.

\vs Tho 1:11
Иисус сказал:
это небо прейдёт, и то, что над ним, прейдёт,
и те, которые мертвы, не живы, и те, которые живы, не умрут.

\vs Tho 1:12
В те дни вы ели мёртвое, вы делали его живым.
Когда вы окажетесь в свете, что вы будете делать?
В этот день вы~--- одно, вы стали двое.
Когда же вы станете двое, что вы будете делать?

\vs Tho 1:13
Ученики сказали Иисусу:
мы знаем, что ты уйдешь от нас.
Кто тот, который будет б\acc{о}льшим над нами?
Иисус сказал им:
в том месте, куда вы пришли, вы пойдёте к Иакову праведному,
из-за которого возникли небо и земля.

\vs Tho 1:14
Иисус сказал ученикам своим:
уподобьте меня, скажите мне, на кого я похож.
Симон Пётр сказал ему:
ты похож на ангела праведного.
Матфей сказал ему: ты похож на философа мудрого.
Фома сказал ему:
Господи, мои уста никак не примут сказать, на кого ты похож.
Иисус сказал:
я не твой господин, ибо ты выпил,
ты напился из источника кипящего, который я измерил.
И он взял его, отвёл его и сказал ему 3 слова.
Когда же Фома пришёл к своим товарищам, они спросили его:
что сказал тебе Иисус?
Фома сказал им: если я скажу вам одно из слов,
которые он сказал мне, вы возьмёте камни, бросите в меня,
огонь выйдет из камней и сожжёт вас.

\vs Tho 1:15
Иисус сказал:
если вы поститесь, вы зародите в себе грех,
и, если вы молитесь, вы будете осуждены,
и, если вы подаете милостыню, вы причините зло вашему духу.
И если вы приходите в какую-то землю и идёте в селения,
если вас примут, ешьте то, что вам поставят.
Тех, которые среди них больны, лечите.
Ибо то, что войдёт в ваши уста, не осквернит вас,
но то, что выходит из ваших уст, это вас осквернит.

\vs Tho 1:16
Иисус сказал:
когда вы увидите того, который не рождён женщиной,
падите ниц и почитайте его; он~--- ваш Отец.

\vs Tho 1:17
Иисус сказал:
может быть, люди думают, что я пришёл бросить мир в космос,
и они не знают, что я пришёл бросить на землю разделения,
огонь, меч, войну.
Ибо 5-ро будут в доме: 3-ое будут против 2-их и 2-ое против 3-их.
Отец против сына и сын против отца; и они будут стоять как единственные.

\vs Tho 1:18
Иисус сказал: я дам вам то, чего не видел глаз,
и то, чего не слышало ухо,
и то, чего не коснулась рука,
и то, что не вошло в сердце человека.

\vs Tho 1:19
Ученики сказали Иисусу:
скажи нам, каким будет наш конец.
Иисус сказал: открыли ли вы начало, чтобы искать конец?
Ибо в месте, где начало, там будет конец.
Блажен тот, кто будет стоять в начале:
и он познает конец, и он не вкусит смерти.

\vs Tho 1:20
Иисус сказал: блажен тот, кто был до того, как возник.

\vs Tho 1:21
Если вы у меня ученики и если слушаете мои слова,
эти камни будут служить вам.

\vs Tho 1:22
Ибо есть у вас 5 деревьев в раю,
которые неподвижны и летом и зимой,
и их листья не опадают.
Тот, кто познает их, не вкусит смерти.

\vs Tho 1:23
Ученики сказали Иисусу:
скажи нам, чему подобно царствие небесное.
Он сказал им: оно подобно зерну горчичному,
самому малому среди всех семян.
Когда же оно падает на возделанную землю,
оно даёт большую ветвь и становится укрытием для птиц небесных.

\vs Tho 1:24
Мария сказала Иисусу:
на кого похожи твои ученики?
Он сказал: они похожи на детей малых,
которые расположились на поле, им не принадлежащем.
Когда придут хозяева поля, они скажут: оставьте нам наше поле.
Они обнажаются перед ними, чтобы оставить это им и дать им их поле.

\vs Tho 1:25
Поэтому я говорю:
если хозяин дома знает, что приходит вор,
он будет бодрствовать до тех пор, пока он не придёт,
и он не позволит ему проникнуть в его дом царствия его,
чтобы унести его вещи.
Вы же бодрствуйте перед миром, препояшьте ваши чресла
с большой силой, чтобы разбойники не нашли пути пройти к вам.
Ибо нужное, что вы ожидаете, будет найдено.

\vs Tho 1:26
Да был бы среди вас знающий человек!
Когда плод созрел, он пришёл поспешно,~--- его
серп в руке его,~--- и он убрал его.
Тот, кто имеет уши слышать, да слышит!

\vs Tho 1:27
Иисус увидел младенцев, которые сосали молоко.
Он сказал ученикам своим:
эти младенцы, которые сосут молоко, подобны тем,
которые входят в царствие.
Они сказали ему:
что же, если мы~--- младенцы, мы войдём в царствие?
Иисус сказал им: когда вы сделаете воих одним,
и когда вы сделаете внутреннюю сторону как внешнюю сторону,
и внешнюю сторону как внутреннюю сторону,
и верхнюю сторону как нижнюю сторону,
и когда вы сделаете мужчину и женщину одним,
чтобы мужчина не был мужчиной и женщина не была женщиной,
когда вы сделаете глаз\acc{а} вместо гл\acc{а}за,
и руку вместо руки,
и ногу вместо ноги,
образ вместо образа,~--- тогда вы войдёте.

\vs Tho 1:28
Иисус сказал:
я выберу вас 1-го на 1000-чу и 2-их на 10000, и они будут стоять как 1.

\vs Tho 1:29
Ученики его сказали:
покажи нам место, где ты, ибо нам необходимо найти его.
Он сказал им: тот, кто имеет уши, да слышит!
Есть свет внутри человека света, и он освещает весь мир.
Если он не освещает, то~--- тьма.

\vs Tho 1:30
Иисус сказал:
люби брата твоего, как душу твою.
Охраняй его как зеницу ока твоего.

\vs Tho 1:31
Иисус сказал:
сучок в глазе брата твоего ты видишь,
бревна же в твоём глазе ты не видишь.
Когда ты вынешь бревно из твоего глаза,
тогда ты увидишь, как вынуть сучок из глаза брата твоего.

\vs Tho 1:32
Если вы не поститесь от мира, вы не найдёте царствия.
Если не делаете субботу субботой, вы не увидите Отца.

\vs Tho 1:33
Иисус сказал:
я встал посреди мира, и я явился им во плоти.
Я нашёл всех их пьяными, я не нашёл никого из них жаждущим,
и душа моя опечалилась за детей человеческих.
Ибо они слепы в сердце своём и они не видят,
что они приходят в мир пустыми;
они ищут снова уйти из мира пустыми.
Но теперь они пьяны.
Когда они отвергнут своё вино, тогда они покаются.

\vs Tho 1:34
Иисус сказал:
если плоть произошла ради духа, это~--- чудо.
Если же дух ради тела, это~--- чудо из чудес.
Но я, я удивляюсь тому,
как такое большое богатство заключено в такой бедности.

\vs Tho 1:35
Иисус сказал:
там, где 3 бога, там боги.
Там, где 2 или 1, я с ним.

\vs Tho 1:36
Иисус сказал:
нет пророка, принятого в своём селении.
Не лечит врач тех, которые знают его.

\vs Tho 1:37
Иисус сказал:
город, построенный на высокой горе, укреплённый,
не может пасть, и он не может быть тайным.

\vs Tho 1:38
Иисус сказал:
то, что ты услышишь твоим ухом,
возвещай это другому уху с ваших кровель.
Ибо никто не зажигает светильника
и не ставит его под сосуд
и никто не ставит его в тайное место,
но ставит его на подставку для светильника,
чтобы все, кто входит и выходит, видели его свет.

\vs Tho 1:39
Иисус сказал: если слепой ведет слепого, оба падают в яму.

\vs Tho 1:40
Иисус сказал: невозможно, чтобы кто-то вошёл в дом сильного и
взял его силой, если он не свяжет его руки.
Тогда он разграбит дом его.

\vs Tho 1:41
Иисус сказал:
не заботьтесь с утра до вечера и с вечера до утра о том,
что вы наденете на себя.

\vs Tho 1:42
Ученики его сказали:
в какой день ты явишься нам и в какой день мы увидим тебя?
Иисус сказал:
когда вы обнажитесь и не застыдитесь и возьмёте ваши одежды,
пол\acc{о}жите их у ваших ног, подобно малым детям,
растопчете их, тогда вы увидите сына того, кто жив,
и вы не будете бояться.

\vs Tho 1:43
Иисус сказал:
много раз вы желали слышать эти слова, которые я вам говорю,
и у вас нет другого, от кого слышать их.
Наступят дни~---- вы будете искать меня, но не найдёте меня.

\vs Tho 1:44
Иисус сказал:
фарисеи и книжники взяли ключи от знания.
Они спрятали их и не вошли и не позволили тем,
которые хотят войти.
Вы же будьте мудры, как змии, и чисты, как голуби.

\vs Tho 1:45
Иисус сказал:
виноградная лоза была посажена без Отца, и она не укрепилась.
Её выкорчуют, она погибнет.

\vs Tho 1:46
Иисус сказал:
тот, кто имеет в своей руке,~--- ему дадут;
и тот, у кого нет, то малое, что имеет,~--- у него возьмут.

\vs Tho 1:47
Иисус сказал: будьте странниками.

\vs Tho 1:48
Ученики его сказали ему:
кто ты, который говоришь нам это?
Иисус сказал им:
из того, что я вам говорю, вы не узнаёте, кто я?
Но вы стали как иудеи,
ибо они любят дерево и ненавидят его плод,
они любят плод и ненавидят дерево.

\vs Tho 1:49
Иисус сказал:
тот, кто высказал хулу на Отца,~--- ему простится,
и тот, кто высказал хулу на Сына,~--- ему простится.
Но тот, кто высказал хулу на Духа святого,~--- ему
не простится ни на земле, ни на небе.

\vs Tho 1:50
Иисус сказал:
не собирают винограда с терновника
и не пожинают смокв с верблюжьих колючек.
Они не дают плода.
Добрый человек выносит доброе из своего сокровища.
Злой человек выносит злое из своего злого сокровища,
которое в его сердце, и он говорит злое,
ибо из избытка сердца он выносит злое.

\vs Tho 1:51
Иисус сказал:
от Адама до Иоанна Крестителя из рождённых жёнами
нет выше Иоанна Крестителя \ldots\ Но я сказал:
тот из вас, кто станет малым, познает царствие и будет выше Иоанна.

\vs Tho 1:52
Иисус сказал:
невозможно человеку сесть на двух коней,
натянуть два лука,
и невозможно рабу служить двум господам:
или он будет почитать одного и другому он будет грубить.
Ни один человек, который пьёт старое вино,
тотчас не стремится выпить вино молодое.
И не наливают молодое вино в старые мехи,
дабы они не разорвались,
и не наливают старое вино в новые мехи,
дабы они не испортили его.
Не накладывают старую заплату на новую одежду,
ибо произойдёт разрыв.

\vs Tho 1:53
Иисус сказал:
если двое в мире друг с другом в одном и том же доме,
они скажут горе: переместись!~--- и она переместится.

\vs Tho 1:54
Иисус сказал:
блаженны единственные и избранные, ибо вы найдёте царствие,
ибо вы от него, вы снова туда возвратитесь.

\vs Tho 1:55
Иисус сказал:
если вам говорят: откуда вы произошли?~--- скажите им:
мы пришли от света, от места,
где свет произошёл от самого себя.
Он \ldots\ в их образ.
Если вам говорят: кто вы?~--- скажите: мы его дети,
и мы избранные Отца живого.
Если вас спрашивают:
каков знак вашего Отца, который в вас?~--- скажите им:
это движение и покой.

\vs Tho 1:56
Ученики его сказали ему:
в какой день наступит покой тех, которые мертвы?
И в какой день новый мир приходит?
Он сказал им:
тот покой, который вы ожидаете, пришёл,
но вы не познали его.

\vs Tho 1:57
Ученики его сказали ему:
24 пророка высказались в Израиле, и все они сказали о тебе.
Он сказал им:
вы оставили того, кто жив перед вами, и вы сказали о тех, кто мёртв.

\vs Tho 1:58
Ученики его сказали ему:
обрезание полезно или нет?
Он сказал им:
если бы оно было полезно, их отец зачал бы их в их матери обрезанными.
Но истинное обрезание в духе обнаружило полную пользу.

\vs Tho 1:59
Иисус сказал: блаженны бедные, ибо ваше~--- царствие небесное.

\vs Tho 1:60
Иисус сказал:
тот, кто не возненавидел своего отца и свою мать,
не сможет быть моим учеником,
и тот, кто не возненавидел своих братьев и своих сестёр
и не понес свой крест, как я, не станет достойным меня.

\vs Tho 1:61
Иисус сказал:
тот, кто познал мир, нашёл труп,
и тот, кто нашёл труп~--- мир недостоин его.

\vs Tho 1:62
Иисус сказал:
царствие Отца подобно человеку, у которого хорошие семена.
Его враг пришёл ночью, высеял плевел вместе с хорошими семенами.
Человек не позволил им вырвать плевел.
Он сказал им:
не приходите, чтобы, вырывая плевел,
вы не вырвали пшеницу вместе с ним!
Ибо в день жатвы плевелы появятся, их вырвут и их сожгут.

\vs Tho 1:63
Иисус сказал: блажен человек, который потрудился: он нашёл жизнь.

\vs Tho 1:64
Иисус сказал:
посмотрите на того, кто жив, пока вы живёте,
дабы вы не умерли,~--- ищите увидеть его!
И вы не сможете увидеть самаритянина,
который несёт ягнёнка и входит в Иудею.
Он сказал ученикам своим: почему он с ягнёнком?
Они сказали ему: чтобы убить его и съесть его.
Он сказал им: пока он жив, он его не съест,
но если он убивает его, и ягнёнок становится трупом.
Они сказали: иначе он не сможет ударить.
Он сказал им: вы также ищите себе место в покое,
дабы вы не стали трупом и вас не съели.

\vs Tho 1:65
Иисус сказал:
двое будут отдыхать на ложе: один умрёт, другой будет жить.
Саломея сказала: кто ты, человек, и чей ты сын?
Ты взошёл на моё ложе, и ты поел за моим столом.
Иисус сказал ей:
я тот, который произошёл от того, который равен;
мне дано принадлежащее моему Отцу.
Саломея сказала: я твоя ученица.
Иисус сказал ей: поэтому я говорю следующее:
когда он станет пустым, он наполнится светом,
но, когда он станет разделённым, он наполнится тьмой.

\vs Tho 1:66
Иисус сказал:
я говорю мои тайны \ldots\ тайна.
То, что твоя правая рука будет делать,~--- пусть
твоя левая рука не знает того, что она делает.

\vs Tho 1:67
Иисус сказал:
был человек богатый, у которого было много добра.
Он сказал: я использую моё добро,
чтобы засеять, собрать, насадить,
наполнить мои амбары плодами,
дабы мне не нуждаться ни в чём.
Вот о чём он думал в сердце своём.
И в ту же ночь он умер.
Тот, кто имеет уши, да слышит!

\vs Tho 1:68
Иисус сказал:
у человека были гости, и, когда он приготовил ужин,
он послал своего раба, чтобы он пригласил гостей.
Он пошёл к первому, он сказал ему:
мой господин приглашает тебя.
Он сказал: у меня деньги для торговцев, они придут ко мне вечером,
я пойду и дам им распоряжение: я отказываюсь от ужина.
Он пошёл к другому, он сказал ему: мой господин пригласил тебя.
Он сказал ему: я купил дом, и меня просят днём.
У меня не будет времени.
Он пошёл к другому, он сказал ему: мой господин приглашает тебя.
Он сказал ему: мой друг будет праздновать свадьбу,
и я буду устраивать ужин. Я не смогу прийти. Я отказываюсь от ужина.
Он пошёл к другому, он сказал ему: мой господин приглашает тебя.
Он сказал ему: я купил деревню, я пойду собирать доход.
Я не смогу прийти. Я отказываюсь.
Раб пришёл, он сказал своему господину:
те, кого ты пригласил на ужин, отказались.
Господин сказал своему рабу:
пойди на дороги, кого найдёшь, приведи их, чтобы они поужинали.
Покупатели и торговцы не войдут в места моего отца.

\vs Tho 1:69
Он сказал: У доброго человека был виноградник; он отдал его
работникам, чтобы они обработали его и чтобы он получил его плод от них. Он
послал своего раба, чтобы работники дали ему плод виноградника. Те схватили его
раба, они избили его, еще немного~--- и они убили бы его. Раб пришёл, он
рассказал своему господину. Его господин сказал: Может быть, они его не узнали
(в оригинале: Может быть, он их не узнал). Он послал другого раба. Работники
побили этого. Тогда хозяин послал своего сына. Он сказал: Может быть, они
постыдятся моего сына. Эти работники, когда узнали, что он наследник
виноградника, схватили его, они убили его. Тот, кто имеет уши, да слышит!

\vs Tho 1:70
Иисус сказал:
покажи мне камень, который строители отбросили!
Он~--- краеугольный камень.

\vs Tho 1:71
Иисус сказал:
тот, кто знает всё, нуждаясь в самом себе, нуждается во всём.

\vs Tho 1:72
Иисус сказал:
блаженны вы, когда вас ненавидят и вас преследуют.
И не найдут места там, где вас преследовали.

\vs Tho 1:73
Иисус сказал:
блаженны те, которых преследовали в их сердце;
это те, которые познали Отца в истине.
Блаженны голодные, потому что чрево того, кто желает, будет насыщено.

\vs Tho 1:74
Иисус сказал:
когда вы рождаете это в себе, то, что вы имеете, спасёт вас.
Если вы не имеете этого в себе, то, чего вы не имеете в себе,
умертвит вас.

\vs Tho 1:75
Иисус сказал:
я разрушу этот дом, и нет никого, кто сможет построить его ещё раз.

\vs Tho 1:76
Некий человек сказал ему:
скажи моим братьям, чтобы они разделили вещи моего отца со мной.
Он сказал ему: о человек, кто сделал меня тем, кто делит?
Он повернулся к своим ученикам, сказал им:
да не стану я тем, кто делит!

\vs Tho 1:77
Иисус сказал:
жатва обильна, работников же мало.
Просите же господина, чтобы он послал работников на жатву.

\vs Tho 1:78
Он сказал:
Господи, много вокруг источника, но никого нет в источнике.

\vs Tho 1:79
Иисус сказал:
многие стоят перед дверью, но единственные те,
которые войдут в брачный чертог.

\vs Tho 1:80
Иисус сказал:
царствие Отца подобно торговцу, имеющему товары,
который нашёл жемчужину.
Этот торговец~--- мудрый:
он продал товары и купил себе одну жемчужину.
Вы также~--- ищите его сокровище, которое не гибнет,
которое остаётся там, куда не проникает моль, чтобы съесть,
и не губит червь.

\vs Tho 1:81
Иисус сказал:
я~--- свет, который на всех.
Я~--- всё: всё вышло из меня и всё вернулось ко мне.
Разруби дерево, я~--- там;
подними камень, и ты найдёшь меня там.

\vs Tho 1:82
Иисус сказал:
почему вы пошли в поле?
чтобы видеть тростник, колеблемый ветром, и видеть человека,
носящего на себе мягкие одежды?
Смотрите, ваши цари и ваши знатные люди~--- это они носят на себе мягкие
одежды и они не смогут познать истину!

\vs Tho 1:83
Женщина в толпе сказала ему:
блаженно чрево, которое выносило тебя, и груди, которые вскормили тебя.
Он сказал ей:
блаженны те, которые услышали слово Отца и сохранили его в истине.
Ибо придут дни, вы скажете:
блаженно чрево, которое не зачало, и груди, которые не дали молока.

\vs Tho 1:84
Иисус сказал:
тот, кто познал мир, нашёл тело,
но тот, кто нашёл тело,~--- мир недостоин его.

\vs Tho 1:85
Иисус сказал:
тот, кто сделался богатым, пусть царствует,
и тот, у кого сила, пусть откажется.

\vs Tho 1:86
Иисус сказал:
тот, кто вблизи меня, вблизи огня,
и кто вдали от меня, вдали от царствия.

\vs Tho 1:87
Иисус сказал:
образы являются человеку, и свет, который в них, скрыт.
В образе света Отца он откроется,
и его образ скрыт из-за его света.

\vs Tho 1:88
Иисус сказал:
когда вы видите ваше подобие, вы радуетесь.
Но когда вы видите ваши образы,
которые произошли до вас,~--- они не умирают
и не являются~--- сколь великое вы перенесёте?

\vs Tho 1:89
Иисус сказал:
Адам произошёл от большой силы и большого богатства,
и он недостоин вас.
Ибо \ldots\ смерти.

\vs Tho 1:90
Иисус сказал:
лисицы имеют свои норы, и птицы имеют свои гнезда,
а Сын человеческий не имеет места,
чтобы преклонить свою голову и отдохнуть.

\vs Tho 1:91
Иисус сказал:
несчастно тело, которое зависит от тела,
и несчастна душа, которая зависит от них обоих.

\vs Tho 1:92
Иисус сказал:
ангелы приходят к вам и пророки, и они дадут вам то,
что ваше, и вы также дайте им то, что в ваших руках,
и скажите себе:
в какой день они приходят и берут то, что принадлежит им?

\vs Tho 1:93
Иисус сказал:
почему вы моете внутри чаши и не понимаете того,
что тот, кто сделал внутреннюю часть,
сделал также внешнюю часть?

\vs Tho 1:94
Иисус сказал:
придите ко мне, ибо иго моё~--- благо и власть моя кротка,
и вы найдёте покой себе.

\vs Tho 1:95
Они сказали ему:
скажи нам, кто ты, чтобы мы поверили в тебя.
Он сказал им: вы испытываете лицо неба и земли;
и того, кто перед вами,~--- вы не познали его;
и это время~--- вы не знаете, как испытать его.

\vs Tho 1:96
Иисус сказал:
ищите и вы найдете, но те вещи,
о которых вы спросили меня в те дни,~--- я не сказал вам тогда.
Теперь я хочу сказать их, но вы не ищете их.

\vs Tho 1:97
Не давайте того, что свято, собакам,
чтобы они не бросили это в навоз.
Не бросайте жемчуга свиньям, чтобы они не сделали это \ldots

\vs Tho 1:98
Иисус сказал:
тот, кто ищет, найдёт, и тот, кто стучит, ему откроют.

\vs Tho 1:99
Иисус сказал:
если у вас есть деньги, не давайте в рост,
но дайте \ldots\ от кого вы не возьмёте их.

\vs Tho 1:100
Иисус сказал:
царствие Отца подобно женщине, которая взяла немного закваски,
положила это в тесто и разделила это в большие хлебы.
Кто имеет уши, да слышит!

\vs Tho 1:101
Иисус сказал:
царствие подобно женщине, которая несёт сосуд,
полный муки, идёт удаляющейся дорогой.
Ручка сосуда разбилась, мука рассыпалась позади неё на дороге.
Она не знала, она не поняла, как действовать.
Когда она достигла своего дома,
она поставила сосуд на землю и нашла его пустым.

\vs Tho 1:102
Иисус сказал:
царствие Отца подобно человеку,
который хочет убить сильного человека.
Он извлёк меч в своём доме, он вонзил его в стену,
дабы узнать, будет ли рука его крепка.
Тогда он убил сильного.

\vs Tho 1:103
Ученики сказали ему:
твои братья и твоя мать стоят снаружи.
Он сказал им:
те, которые здесь,
которые исполняют волю моего Отца,~--- мои братья и моя мать.
Они те, которые войдут в царствие моего Отца.

\vs Tho 1:104
Иисусу показали золотой и сказали ему:
те, кто принадлежит Цезарю, требуют от нас подати.
Он сказал им: Дайте Цезарю то, что принадлежит Цезарю,
дайте Богу то, что принадлежит Богу,
и то, что моё, дайте это мне!

\vs Tho 1:105
Тот, кто не возненавидел своего отца и свою мать,
как я, не может быть моим учеником,
и тот, кто не возлюбил своего отца и свою мать,
как я, не может быть моим учеником.
Ибо моя мать \ldots\ но поистине она дала мне жизнь.

\vs Tho 1:106
Иисус сказал: Горе им, фарисеям! Ибо они похожи на собаку,
которая спит на кормушке быков. Ибо она и не ест и не дает есть быкам.

\vs Tho 1:107
Иисус сказал:
блажен человек, который знает, в какую пору приходят разбойники,
так что он встанет, соберёт \ldots\ и препояшет свои чресла,
прежде чем они придут.

\vs Tho 1:108
Они сказали:
пойдём помолимся сегодня и попостимся.
Иисус сказал:
каков же грех, который я совершил или которому я поддался?
Но когда жених выйдет из чертога брачного,
тогда пусть они постятся и пусть молятся!

\vs Tho 1:109
Иисус сказал:
тот, кто познает отца и мать,~--- его назовут сыном блудницы.

\vs Tho 1:110
Иисус сказал:
когда вы сделаете 2-ух 1-им, вы станете Сыном человеческим,
и, если вы скажете гор\acc{е}: сдвинься, она переместится.

\vs Tho 1:111
Иисус сказал: царствие подобно пастуху, у которого 100 овец.
Одна из них, самая большая, заблудилась.
Он оставил 99 и стал искать одну, пока не нашёл её.
После того как он потрудился, он сказал овце:
я люблю тебя больше, чем 99.

\vs Tho 1:112
Иисус сказал:
тот, кто напился из моих уст, станет как я.
Я также, я стану им, и тайное откроется ему.

\vs Tho 1:113
Иисус сказал:
царствие подобно человеку,
который имеет на своём поле тайное сокровище,
не зная о нём.
И он не нашёл до того, как умер, он оставил его своему сыну.
Сын не знал; он получил это поле и продал его.
И тот, кто купил его, пришёл, раскопал и нашёл сокровище.
Он начал давать деньги под проценты тем, кому он хотел.

\vs Tho 1:114
Иисус сказал:
тот, кто нашёл мир и стал богатым, пусть откажется от мира!

\vs Tho 1:115
Иисус сказал:
небеса, как и земля, свернутся перед вами,
и тот, кто живой от живого, не увидит смерти.
Ибо Иисус сказал: тот, кто нашёл самого себя,~--- мир недостоин его.

\vs Tho 1:116
Иисус сказал:
горе той плоти, которая зависит от души;
горе той душе, которая зависит от плоти.

\vs Tho 1:117
Ученики его сказали ему:
в какой день царствие приходит?
Оно не приходит, когда ожидают.
Не скажут: Вот, здесь!~--- или:
Вот, там!~--- Но царствие Отца распространяется по земле,
и люди не видят его.

\vs Tho 1:118
Симон Пётр сказал им:
пусть Мария уйдёт от нас, ибо женщины недостойны жизни.
Иисус сказал:
смотрите, я направлю её, дабы сделать её мужчиной,
чтобы она также стала духом живым, подобным вам, мужчинам.
Ибо всякая женщина, которая станет мужчиной, войдёт в царствие небесное.

\bibbookdescr{Did}{
  inline={Учение двенадцати апостолов},
  toc={Учение 12-и апостолов},
  bookmark={Учение 12-и апостолов},
  header={Учение 12-и апостолов},
  abbr={Дид}
}
\chhdr{Учение Господа народам через двенадцать апостолов}
\vs Did 1:1
Есть два пути: один~--- жизни и один~--- смерти,
но между обоими путями большое различие.
\vs Did 1:2
Путь жизни таков.
Во-первых, ты должен любить Бога, создавшего тебя;
во-вторых~--- ближнего своего, как себя самого;
и всего того, чего не хочешь, чтобы было с тобою,
и ты не делай другому.

\vs Did 1:3
Слов же сих учение таково: благословляйте проклинающих вас и
молитесь за врагов ваших, поститесь за гонящих вас, ибо какая
вам за то благодарность, если вы любите любящих вас?
Не то же ли делают и язычники?
Вы же любите ненавидящих вас и не будете иметь врага.
\vs Did 1:4
Удаляйся от плотских и мирских похотей.
Если кто ударит тебя в правую щёку, обрати к нему и другую и будешь совершен.
Если кто наймёт тебя на одну милю, иди с ним две.
Если кто отнимет у тебя верхнюю одежду, отдай и хитон.
Если кто возьмет у тебя твоё, не требуй назад, да и не сможешь.
\vs Did 1:5
Всякому, просящему у тебя, давай и не требуй назад, ибо Отец
хочет чтобы всё подаваемо было из его даров.
Блажен дающий по заповеди, ибо он неповинен.
Горе принимающему, ибо если кто, имея нужду,
принимает, тот будет неповинен, если же кто принимает,
не имея нужды, тот даст отчёт, почему принял и на что:
подвергшись же заключению, испытан будет относительно того,
что сделал, и не выйдет оттуда, пока не отдаст последнего кодранта.
\vs Did 1:6
Но и о сём также сказано: пусть милостыня твоя запотеет
в руках твоих, пока ты не узнаешь, кому дать.

\vs Did 2:1
Вторая же заповедь учения:
\vs Did 2:2
Не убивай,
не прелюбодействуй,
не совершай деторастления,
не будь блудником,
не кради,
не занимайся магией,
не изготавливай волшебных снадобий,
не умерщвляй дитя в зародыше и рожденного не убивай,
не пожелай достояния ближнего твоего.
\vs Did 2:3
Не клянись,
не лжесвидетельствуй,
не злословь,
не злопамятствуй.
\vs Did 2:4
Не будь двоедушным и двуязычным,
ибо двуязычие есть сеть смерти.
\vs Did 2:5
Да не будет слово твоё лживым и пустым, но преисполненным дела.
\vs Did 2:6
Не будь
ни корыстолюбивым,
ни хищником,
ни лицемером,
ни злобным,
ни надменным,
не принимай лукавого умысла на ближнего своего.
\vs Did 2:7
Не имей ненависти ни к одному человеку, но одних обличай, за
других молись, а иных люби более души своей.

\vs Did 3:1
Чадо моё!
Беги от всякого зла и от
всего\fnote{всего}{\vsep\ дела.}
подобного ему.
\vs Did 3:2
Не будь
ни гневливым, ибо гнев ведет к убийству,
ни ревнивым,
ни сварливым,
ни запальчивым, ибо от всего этого рождаются убийства.

\vs Did 3:3
Чадо моё!
Не будь ни похотником, ибо похоть ведет к блуду,
ни срамословом,
ни бесстыжеглазым, ибо от всего этого рождаются прелюбодеяния.

\vs Did 3:4
Чадо моё!
Не гадай по полёту птиц, ибо птицегадание ведет к идолослужению,
не заговаривай,
не занимайся математикой,
ни очищениями,
не желай смотреть на это, ибо от всего этого рождается идолослужение.

\vs Did 3:5
Чадо моё!
Не будь ни лживым, поелику ложь доводит до воровства,
ни сребролюбцем,
ни тщеславным, ибо от всего этого рождается воровство.

\vs Did 3:6
Чадо моё!
Не будь ни ропотником, ибо ропот доводит до богохульства,
ни своенравным,
ни лукавомыслящим, ибо от всего этого рождаются богохульства.
\vs Did 3:7
Но будь кротким, ибо кроткие наследуют землю.
\vs Did 3:8
Будь долготерпеливым, и милостивым, и незлобивым, и смиренным,
и благим, и всегда трепещущим от слов, которые услышал.
\vs Did 3:9
Не превозносись и не допускай в душе своей дерзости.
Да не прилепится душа твоя к гордым,
но обращайся с праведными и смиренными.
\vs Did 3:10
То, что случается с тобой, принимай как благо,
зная, что без Бога ничего не пргоисходит.

\vs Did 4:1
Чадо моё!
Возвещающего тебе Слово Божие помни день и ночь,
почитай же его, как Господа, ибо где возвещается господство,
там Господь есть.
\vs Did 4:2
Даже ищи каждый день иметь личное общение со святыми,
чтобы ты почивал на словах учения их.
\vs Did 4:3
Не производи разделения, а примиряй спорящих; суди праведно, не
будь лицеприятен при обличении преступлений.
\vs Did 4:4
Не думай двоедушно, так или нет.

\vs Did 4:5
Не будь протягивающим руки для принятия подаяний,
но сжимающим её для подаяния.
\vs Did 4:6
Если ты имеешь, что подать от труда рук твоих,
то дай выкуп за грехи твои.
\vs Did 4:7
Не колеблись подать и, подавая, не ропщи,
ибо ты должен знать, кто добрый Мздовоздаятель.
\vs Did 4:8
Не отвращайся от нуждающегося (ср. Сир.4:5), но во всем будь
общником с братом твоим и \bibemph{ничего} не называй своим,
ибо если вы соучастники в нетленном, то насколько более в тленном!
\vs Did 4:9
Не отнимай руки твоей от сына твоего или от дочери твоей,
но от юности учи их страху Божию.

\vs Did 4:10
В гневе твоём не отдавай приказаний рабу твоему или служанке твоей,
надеющимся на того же Бога, дабы они никогда не перестали бояться Бога,
сущего над обоими вами, ибо он пришел призвать ко спасению,
не по лицу судя, а тех коих уготовал дух.
\vs Did 4:11
Вы же, рабы, подчиняйтесь господам своим, как образу Божию,
по совести и со страхом.

\vs Did 4:12
Ненавидь всякое лицемерие и всё, что неугодно Господу.
\vs Did 4:13
Не оставляй заповедей Господа, но храни то, что принял,
не прибавляя и не убавляя.
\vs Did 4:14
В церкви исповедуй преступления свои и не приступай к молитве
своей в лукавой совести.
Этот путь есть путь жизни.

\vs Did 5:1
Путь же смерти таков.
Прежде всего он лукав и исполнен проклятия.
Убийства, прелюбодеяния, вожделения, блуд, кражи, идолослужение,
магия, изготовления снадобий, ограбления, лжесвидетельства, лицемерия,
двоедушие, хитрость, гордыня, злоба, самодовольство, любостяжание,
сквернословие, ревнование, дерзость, высокомерие, бахвальство, бесстрашие.
\vs Did 5:2
гонители благих, ненавистники истины, любители лжи,
не признающие воздаяния за праведность,
не привязывающиеся к благому, ни к суду праведному,
бдящие не во благо, но в зло; от которых далеки кротость и
терпение, любящие суетное, гоняющиеся за мздовоздаянием,
не милующие нищего, не болеззнующие об удрученном,
не вещающие Создавшего их, убийцы детей, губители создания Божия,
отвращающиеся от нуждающегося, обременяющие угнетенного,
заступники богатых, беззаконные судьи бедных, всегрешные.
О если бы вы, чада, избавились от всех таких!

\vs Did 6:1
Смотри, чтобы кто не совратил тебя с этого пути учения, ибо
таковой учит тебя вне Бога.
\vs Did 6:2
Ибо если ты сможешь понести всё иго Господне, то будешь
совершен, а если не можешь, то делай то, что можешь.
\vs Did 6:3
Относительно пищи понеси то, что можешь, но крепко
воздерживайся от идоложертвенного, ибо это есть служение
мёртвым богам.

\vs Did 7:1
Что же \bibemph{касается} крещения,
крестите так: преподав наперед всё это,
крестите во имя Отца и Сына и Святого Духа
в проточной воде.
\vs Did 7:2
Если же нет проточной воды, окрести в иной воде,
а если не можешь в холодной~--- в теплой.
\vs Did 7:3
Если же нет ни той, ни другой, то возлей воду на голову трижды
во имя Отца и Сына и Святого Духа.
\vs Did 7:4
А пред крещением пусть постятся крещающий и крещаемый и, если
могут, некоторые другие; вели же \bibemph{обязательно} поститься
крещаемому день или два до \bibemph{крещения}.

\vs Did 8:1
Посты же ваши да не будут с лицемерами:
они постятся во второй и пятый день недели,
вы же поститесь в четвертый и шестой.
\vs Did 8:2
И не молитесь, как лицемеры, но как повелел Господь в Евангелии своём,
так молитесь:
Отче наш, Cущий на Небе!
Да святится Имя твоё;
да приидет Царствие Твоё;
да будет Воля твоя и на земле, как на Небе;
хлеб наш насущный дай нам на сей день,
и оставь нам долг наш, как и мы оставляем должникам нашим,
и не введи нас в искушение,
но избавь нас от лукавого,
потому что твоя есть сила и слава во веки.
\vs Did 8:3
Трижды в день молитесь так.

\vs Did 9:1
Что же касается евхаристии, совершайте ее так.
\vs Did 9:2
Сперва о чаше:
Благодарим тебя, Отче наш, за святой виноград Давида,
отрока твоего, который виноград ты открыл нам чрез Иисуса, отрока твоего.
Тебе слава во веки!
\vs Did 9:3
О хлебе же ломимом: Благодарим Тебя, Отче наш, за жизнь и знание,
которые ты открыл нам чрез Иисуса, отрока твоего.
Тебе слава во веки.
\vs Did 9:4
Как сей преломляемый хлеб был рассеян по холмам и собранный
вместе стал единым, так и экклесия твоя от концов земли
да соберется в Царствие твоё, ибо твоя есть слава и сила
чрез Иисуса Христа во веки.
\vs Did 9:5
И от евхаристии вашей никто да не вкушает и не пьет, кроме
крещенных во имя Господне, ибо и о сем сказал Господь:
не давайте святыни псам.

\vs Did 10:1
По исполнении же вкушения так благодарите:
\vs Did 10:2
Благодарим тебя, Отче святый, за имя твоё святое,
которое ты вселил в сердцах наших,
и за знание, и веру, и бессмертие,
которые ты открыл нам чрез Иисуса, отрока твоего.
Тебе слава во веки!
\vs Did 10:3
Ты, Владыко Вседержитель, сотворил всё ради имени твоего,
пищу же и питие дал людям в наслаждение,
чтобы они благодарили тебя, а нам даровал духовную пищу и питие,
и жизнь вечную чрез отрока твоего.
\vs Did 10:4
Более всего благодарим тебя потому, что ты всемогущ.
Тебе слава во веки!
\vs Did 10:5
Помяни, Господи, экклесию твою, да избавишь её от всякого зла
и усовершишь её в любви твоей, и от четырёх ветров собери её,
освящённую, в царство твоё, которое ты уготовал ей,
потому что Твоя есть сила и слава во веки.
\vs Did 10:6
Да приидет благодать и да прейдёт мир сей.
Осанна Богу Давидову!
Кто свят, да приступает, кто нет, пусть покается.
Маранат. Аминь.
\vs Did 10:7
Пророкам же позволяйте совершать евхаристию когда захотят.

\vs Did 11:1
Кто, пришедши, будет учить вас всему этому,
пред сим сказанному, примите его.
\vs Did 11:2
Если же сам учащий, совратившись, будет преподавать иное
учение, к ниспровержению, не слушайте его; но если для
преумножения правды и познания Господа, примите его, как Господа.
\vs Did 11:3
Относительно же апостолов и пророков поступайте
согласно учению евангельскому.
\vs Did 11:4
Всякий апостол, приходящий к вам, пусть будет принят, как Господь.
\vs Did 11:5
Но пусть он не остаётся более одного дня, а если будет нужда,
то и другой \bibemph{день}, но если он пробудет три дня, то лжепророк.
\vs Did 11:6
Уходя же, апостол пусть ничего не принимает, кроме хлеба,
до \bibemph{следуюшего} места ночлега;
а если он будет требовать серебра, то он лжепророк.
\vs Did 11:7
И всякого пророка, говорящего в духе, не испытывайте и не
судите, ибо всякий грех отпустится,
а этот грех не отпустится.
\vs Did 11:8
Но не всякий, говорящий в духе,~--- пророк,
но \bibemph{только} тот, кто хранит пути Господни;
так что по путям распознаётся и лжепророк и пророк.
\vs Did 11:9
И никакой пророк, в духе определяющий быть трапезе,
не вкушает от неё, а если не так, то он лжепророк.
\vs Did 11:10
Всякий пророк, учащий истине, если он сам не делает того, чему
учит, есть лжепророк.
\vs Did 11:11
Но всякий пророк, признанный истинным, вступающий в мирское
таинство экклесии, но не учащий делать то, что сам делает,
не должен быть судим вами, ибо он суд имеет у Бога,
ибо так поступали и древние пророки.
\vs Did 11:12
Если же кто в духе скажет: дай мне серебра или чего другого,
вы не должны слушать того; но если он назначит подаяние для других, неимущих,
то никто да не осуждает его.

\vs Did 12:1
Всякий, приходящий во имя Господне, да будет принят, а потом,
уже испытав его,
вы узнаете,~--- ибо вы будете иметь разумение,~--- правого и ложного.
\vs Did 12:2
Если приходящий~--- странник, помогите ему, сколько можете,
но он не должен оставаться у вас более двух или трёх дней,
и то если бы нужда оказалась.
\vs Did 12:3
Если же он желает поселиться у вас, то, если он ремесленник,
пусть трудится и ест.
\vs Did 12:4
А если он не знает ремесла, то вы по своему усмотрению
позаботьтесь о нём, но так, чтобы христианин не жил среди вас праздным.
\vs Did 12:5
Если же он не желает так поступать, то он христоторговец.
Остерегайтесь таковых!

\vs Did 13:1
А всякий истинный пророк, желающий поселиться у вас,
достоин своего пропитания.
\vs Did 13:2
Точно так же и истинный учитель, и он достоин, как трудящийся,
своего пропитания.
\vs Did 13:3
Поэтому всякий начаток от произведений точила и гумна, от волов
и овец возьми и отдай пророкам, ибо они ваши архиереи.
\vs Did 13:4
Если же вы не имеете пророка, то отдайте начаток нищим.
\vs Did 13:5
Если ты приготовишь пищу, то, взявши начаток, отдай его по
заповеди.
\vs Did 13:6
Точно так же если ты открыл сосуд вина или елея, то возьми
начаток и отдай пророкам.
\vs Did 13:7
И от серебра, и от одежды, и от всякого имения возьми начаток,
сколько тебе угодно, и отдай его по заповеди.

\vs Did 14:1
Каждый день\fnote{Каждый день}{\vsep\
В день Господень, \bibemph{т.е. в субботу}.},
собравшись, преломите хлеб и благодарите,
исповедав прежде согрешения ваши дабы жертва ваша была чиста.
\vs Did 14:2
Всякий же, имеющий распрю с другом своим, да не приходит вместе
с вами, пока они не примирятся, чтобы не осквернилась жертва ваша.
\vs Did 14:3
Ибо о ней сказал Господь: на всяком месте и во всякое время
должно приносить мне жертву чистую, потому что я царь великий,
говорит Господь, и имя мое чудно в народах.

\vs Did 15:1
Рукополагайте себе старейшин и прислужников, достойных Господа,
мужей кротких и несребролюбивых, и истинных, и испытанных,
ибо и они исполняют для вас служение пророков и учителей.
\vs Did 15:2
Поэтому не презирайте их, ибо они~--- почтенные ваши
наравне с пророками и учителями.

\vs Did 15:3
Обличайте друг друга, но не во гневе, а в мире, как имеете в
евангелии, и со всяким, поступающим оскорбительно по отношению к другому,
пусть никто не говорит и никто у вас не слушает его, пока не покается.
\vs Did 15:4
Молитва же ваша и милостыня, и все вообще добрые дела творите
так, как имеете в евангелии Господа нашего.

\vs Did 16:1
Бодрствуйте относительно жизни вашей;
светильники ваши да не будут погашены,
и чресла ваши не препоясаны,
но будьте готовыми, ибо вы не знаете часа,
в который Господь ваш приидет.
\vs Did 16:2
Вы должны часто собираться вместе, исследуя, что потребно душам
вашим, ибо не принесёт вам пользы всё время вашей веры,
если не сделаетесь совершенными в последний час.
\vs Did 16:3
Ибо в последние дни умножатся лжепророки и губители, и овцы
превратятся в волков, и любовь превратится в ненависть.
\vs Did 16:4
Ибо, когда возрастёт беззаконие, люди будут ненавидеть друг
друга и преследовать, и тогда явится мирообольститель,
как бы Сын Божий, и совершит знамения и чудеса, и земля
предана будет в руки его, и сотворит беззакония,
каких никогда не было от века.
\vs Did 16:5
Тогда тварь человеческая пойдет в огонь испытания и многие
соблазнятся и погибнут, а устоявшие в вере своей спасутся
от проклятия его\fnote{от проклятия его}{\vsep\ этим самим проклятием.}.
\vs Did 16:6
И тогда явится знамение истины:
во-первых, знамение отверстия на небе,
потом знамение звука трубного
и третье~--- воскресение мертвых.
\vs Did 16:7
Но не всех вместе, а как сказано: приидет Господь и все святые с ним.
\vs Did 16:8
Тогда увидит мир Господа, грядущего на облаках небесных.

\bibbookdescr{1Cl}{
  inline={Послание Климента к Коринфянам},
  toc={1-е Климента},
  bookmark={1-е Климента},
  header={1-е Климента},
  abbr={1~Кли}
}
\vs 1Cl 1:1
Церковь Божья,
находящаяся в Риме, Церкви Божьей, находящейся в Коринфе, званным, освященным
по воле Божьей через Господа нашего Иисуса Христа.
\vs 1Cl 1:2
Благодать вам и мир от
Всемогущего Бога чрез Иисуса Христа да умножится.
\vs 1Cl 1:3
Внезапные и одно за другим случившиеся с нами
несчастия и бедствия были причиною того, братья, что поздно, как думается нам,
обратили мы внимание на спорные у вас дела, возлюбленные, и на неприличный и
чуждый избранникам Божьим, преступный и нечестивый мятеж,
\vs 1Cl 1:4
который немногие дерзкие и высокомерные люди
разожгли до такого безумия, что почтенное, славное и для всех достолюбезное
имя ваше подверглось великому поруганию.
\vs 1Cl 1:5
Ибо кто, побывавший у вас, не хвалил вашей,
всеми добродетелями исполненной и твердой, веры, не удивлялся вашему
трезвенному и кроткому во Христе благочестию, не превозносил вашей великой
щедрости в гостеприимстве, не прославлял вашего совершенного и верного знания?
\vs 1Cl 1:6
Во всем вы поступали нелицеприятно, ходили в
заповедях Божьих, повинуясь предводителям вашим и воздавая должную честь
старшим между вами.
\vs 1Cl 1:7
Юношам внушили скромность и благопристойность;
жен наставляли, чтобы они все делали с неукоризненной, честной и чистой
совестью, любя, как должно, своих мужей,
\vs 1Cl 1:8
и учили их, чтобы они, не выступая из правила
повиновения, пристойно распоряжались домашними делами, и вели себя вполне
целомудренно.

\vs 1Cl 2:1
Все вы были смиренны и
чужды тщеславия, любили более подчиняться, нежели повелевать, и давать, нежели
принимать.
\vs 1Cl 2:2
Довольствуясь тем, что Бог
дал вам на путь, и тщательно внимая словам Его, вы хранили их в глубине
сердца, и страдания Его были пред очами вашими.
\vs 1Cl 2:3
Таким образом всем был
дарован глубокий и прекрасный мир и ненасытное стремление делать добро: и на
всех было полное излияние Святого Духа.
\vs 1Cl 2:4
Полные святых желаний, с
искренним усердием и благочестивым упованием, вы простирали руки свои ко
Всемогущему Богу и умоляли Его быть милосердым, если вы в чем невольно
погрешили.
\vs 1Cl 2:5
День и ночь подвигом вашим
было попечение о всем братстве, чтобы число избранных Его спасалось в
добродушии и единомыслии.
\vs 1Cl 2:6
Вы были искренни,
чистосердечны, и не помнили зла друг на друге.
\vs 1Cl 2:7
Всякий мятеж и всякое
разделение было вам противно.
\vs 1Cl 2:8
Вы плакали о проступках
ближних; их недостатки считали собственными.
\vs 1Cl 2:9
Не скучали делать добро,
готовые на всякое дело доброе.
\vs 1Cl 2:10
Будучи украшены такою
добродетельною и почтенною жизнью, вы все совершали в страхе Господа: Его
повеления и заповеди были написаны на скрижалях сердца вашего.

\vs 1Cl 3:1
Вся слава и широта дана
была вам, и исполнилось, что написано: он ел и пил, разжирел и растолстел, и
сделался непокорен возлюбленный.
\vs 1Cl 3:2
А отсюда ревность и
зависть, вражда и раздор, гонение и возмущение, война и плен.
\vs 1Cl 3:3
Таким образом, люди
бесчестные восстали против почтенных, бесславные против славных, глупые против
разумных, молодые против старших.
\vs 1Cl 3:4
Поэтому удалились правда и
мир,~--- так как всякий оставил страх Божий, сделался туп в вере Его, не ходит
по правилам заповедей Его, и не ведет жизни, достойной Христа,
\vs 1Cl 3:5
но каждый последовал злым
своим пожеланиям, допустив снова беззаконную и нечестивую зависть, чрез
которую и смерть вошла в мир.

\vs 1Cl 4:1
Ибо так написано: и было
спустя несколько дней, приносил Каин от плодов земли жертву Богу: и также
Авель приносил от первородных овец и от туков их;
\vs 1Cl 4:2
и призрел Бог на Авеля и
на дары Его; на Каина же и на жертвы его не посмотрел.
\vs 1Cl 4:3
И весьма опечалился Каин,
и поникло лицо его.
\vs 1Cl 4:4
И сказал Бог Каину: что ты
стал печален, и от чего поникло лицо твое? Не согрешил ли ты, если ты
правильно принес, но неправильно разделил?
\vs 1Cl 4:5
Успокойся. К тебе
обращение его и ты будешь обладать тем.
\vs 1Cl 4:6
И сказал Каин Авелю, брату
своему: пойдем в поле;
\vs 1Cl 4:7
и было в то время, как они
находились в поле, возстал Каин на Авеля, брата своего, и убил его.
\vs 1Cl 4:8
Видите, братья, ревность и
зависть произвели братоубийство.
\vs 1Cl 4:9
По причине зависти отец
наш Иаков убежал от лица Исава, брата своего.
\vs 1Cl 4:10
Зависть была причиною,
что Иосиф гоним был на смерть и подвергся рабству.
\vs 1Cl 4:11
Зависть принудила Моисея
бежать от лица фараона, царя Египетского, когда услышал он от единоплеменника
своего:
\vs 1Cl 4:12
кто поставил тебя
решителем или судьею над нами? Не хочешь ли убить меня, как убил вчера
египтянина?
\vs 1Cl 4:13
За зависть Аарон и
Мариамь жили вне стана.
\vs 1Cl 4:14
Зависть Дафана и Авирона
живых низвела в ад за то, что они возмутились против Моисея, служителя
Божьего.
\vs 1Cl 4:15
По причине зависти Давид
не только подвергся ненависти иноплеменных, но был гоним и от Саула, царя
Израильского.

\vs 1Cl 5:1
Но оставив древние
примеры, перейдем к ближайшим подвижникам: возьмем достойные примеры нашего
поколения.
\vs 1Cl 5:2
По ревности и зависти
величайшие и праведные столпы подверглись гонению и смерти. Представим пред
глазами нашими блаженных апостолов.
\vs 1Cl 5:3
Петр от беззаконной
зависти понес не одно, не два, но многие страдания, и таким образом
претерпевши мученичество, отошел в подобающее место славы.
\vs 1Cl 5:4
Павел, по причине зависти,
получил награду за терпение: он был в узах семь раз, был изгоняем, побиваем
камнями.
\vs 1Cl 5:5
Будучи проповедником на
Востоке и Западе, он приобрел благородную славу за свою веру, так как научил
весь мир правде,
\vs 1Cl 5:6
и доходил до границы
Запада, и мученически засвидетельствовал истину перед правителями.
\vs 1Cl 5:7
Так он переселился из
мира, и перешел в место святое, сделавшись величайшим образцом терпения.

\vs 1Cl 6:1
К этим мужам, свято
провождавшим жизнь, присовокупилось великое множество избранных,
\vs 1Cl 6:2
которые по причине зависти
претерпели многие поругания и мучения, и оставили среди нас прекрасный пример.
\vs 1Cl 6:3
Завистью гонимы были
женщины Данаида и Дирка;
\vs 1Cl 6:4
претерпевши тяжкие и
ужасные мучения, они прошли твердым путем веры, и, немощные телом, получили
славную награду.
\vs 1Cl 6:5
Зависть отлучала жен от
мужей и извращала слова праотца нашего Адама: вот ныне кость от костей моих,
и плоть от плоти моей.
\vs 1Cl 6:6
Зависть и раздор
ниспровергли великие города и совершенно истребили великие народы.

\vs 1Cl 7:1
Это, возлюбленные, пишем
мы не только для вашего наставления, но и для собственного напоминания;
\vs 1Cl 7:2
потому что мы находимся на
том же поприще, и тот же подвиг предлежит нам.
\vs 1Cl 7:3
Итак, оставим пустые и
суетные заботы, и обратимся к славному и досточтимому правилу святого звания
нашего.
\vs 1Cl 7:4
Будем смотреть на то, что
добро, что угодно и приятно Создателю нашему.
\vs 1Cl 7:5
Обратим внимание на кровь
Христа,~--- и увидим, как драгоценна пред Богом кровь Его, которая была пролита
для нашего спасения, и всему миру принесла благодать покаяния.
\vs 1Cl 7:6
Пройдем все поколения и
узнаем, что Господь в каждом поколении милостиво принимал покаяние желавших
обратиться к Нему.
\vs 1Cl 7:7
Ной проповедовал покаяние,
и послушавшиеся его спаслись.
\vs 1Cl 7:8
Иона возвестил ниневитянам
погибель, но они, раскаявшись в своих грехах, умилостивили Бога своими
молитвами и получили спасение, хотя были далеки от Бога.

\vs 1Cl 8:1
Служители благодати
Божьей по вдохновению Духа Святого говорили о покаянии;
\vs 1Cl 8:2
и Сам Владыка всего
говорил о покаянии с клятвою: жив Я, говорит ЯХВЕ, не хочу смерти грешника,
но покаяния;
\vs 1Cl 8:3
и присовокупил еще
следующую прекрасную мысль: дом Израилев, обратитесь от нечестия вашего.
\vs 1Cl 8:4
Скажи сынам народа Моего:
хотя грехи ваши будут простираться от земли до неба, и хотя будут краснее
червленицы и чернее власяницы,
\vs 1Cl 8:5
но если вы обратитесь ко
Мне от всего сердца, и скажете: Отец! то Я услышу вас как народ святой.
\vs 1Cl 8:6
И в другом месте так
говорит: омойтесь, и очиститесь, удалите лукавство из душ ваших пред очами
Моими, отстаньте от злодейств ваших;
\vs 1Cl 8:7
научитесь делать добро,
ищите правды, избавьте обиженного, рассудите о сироте, оправдайте вдовицу, и
придите и будем судиться, говорит ЯХВЕ:
\vs 1Cl 8:8
и если будут грехи ваши,
как пурпур, то убелю их как снег; и если будут как червленица, то убелю их как
волну; и если хотите и послушаете Меня, то будете наслаждаться благами земли;
\vs 1Cl 8:9
если же не хотите и не
послушаете Меня, то меч истребит вас: ибо уста ЯХВЕ сказали это.
\vs 1Cl 8:10
Итак, Он всех Своих
возлюбленных хочет сделать участниками покаяния, и утвердил это всемогущею
Своею волею!

\vs 1Cl 9:1
Поэтому покоримся
величественной и славной воле Его, и, оставив суетные дела, раздор и зависть,
ведущую к смерти, припадем и обратимся к Его милосердию, умоляя Его милость и
благость.
\vs 1Cl 9:2
Будем постоянно взирать на
тех, которые совершенно послужили величественной Его славе.
\vs 1Cl 9:3
Возьмем Еноха, который по
своему послушанию был найден праведным, и преставился, и не видели его смерти.
\vs 1Cl 9:4
Ной был найден верным, и
по своему служению проповедал миру обновление, и через него спас Господь
животных, согласно вошедших в ковчег.

\vs 1Cl 10:1
Авраам, названный другом,
найден верным по своему послушанию словам Божьим.
\vs 1Cl 10:2
Он из послушания вышел из
земли своей, и от родства своего, и из дома отца своего, чтобы оставить землю
малую, родство малосильное и небольшой дом, наследовал обетованию Божьему.
\vs 1Cl 10:3
Ибо так сказал ему:
удались из земли твоей, и от родства твоего, и из дома отца твоего в землю,
которую покажут тебе.
\vs 1Cl 10:4
И сделаю тебя народом
великим; и благословлю тебя, и возвеличу имя твое, и будешь благословен.
\vs 1Cl 10:5
И благословлю
благословляющих тебя, и проклинающих тебя прокляну, и благословятся в тебе все
племена земные.
\vs 1Cl 10:6
И опять по разделении его
с Лотом сказал ему Бог: подними глаза твои, и взгляни с места, где ты теперь,
к северу и югу, и к востоку и к морю: ибо всю землю, которую ты видишь, отдам
тебе и семени твоему на век.
\vs 1Cl 10:7
И сделаю семя твое как
песок земной: если кто может сосчитать песок земной, то и семя твое сочтется.
\vs 1Cl 10:8
И еще сказано: вывел Бог
Авраама и сказал ему: взгляни на небо и сосчитай звезды, если можешь счесть
их: так будет семя твое.
\vs 1Cl 10:9
И поверил Авраам Богу, и
это вменилось ему в праведность.
\vs 1Cl 10:10
За веру и гостеприимство
был дан ему в старости сын, но он из послушания принес его в жертву Богу на
одной из показанных от Него гор.

\vs 1Cl 11:1
За гостеприимство и
благочестие Лот вышел невредимым из Содома, тогда как вся окрестная страна
была наказана огнем и серою:
\vs 1Cl 11:2
и тем ясно показал
Господь, что Он не оставляет уповающих на Него; а уклоняющихся от Него
подвергает мучениям и казни.
\vs 1Cl 11:3
Ибо вышедшая с ним жена
его, так как была других мыслей и не согласна с ним, поставлена в знамение:
\vs 1Cl 11:4
она сделалась соляным
столбом, и даже до сего дня, чтобы все знали, что двоедушные и сомневающиеся о
могуществе Божьем служат примером суда и знамением для всех родов.

\vs 1Cl 12:1
За веру и гостеприимство
была спасена Раав блудница.
\vs 1Cl 12:2
Когда Иисусом Нуном были
посланы соглядатаи в Иерихон, и царь земли той узнал, что они пришли
осматривать его землю, то послал людей схватить их, чтобы схватив, предать их
смерти.
\vs 1Cl 12:3
Но гостеприимная Раав,
приняв их к себе, скрыла на верху своего дома в снопах льна.
\vs 1Cl 12:4
И когда от царя явились к
ней и говорили: люди пришли к тебе, соглядатаи земли нашей, выведи их, так
повелевает царь,
\vs 1Cl 12:5
то она отвечала: приходили
ко мне два человека, которых вы ищете, но они скоро ушли, и теперь в пути;
таким образом, она не показала их посланным.
\vs 1Cl 12:6
А мужам тем сказала: знаю
верно, что ЯХВЕ, Бог ваш, предаст вам этот город; потому что страх и трепет от
вас напал на живущих в нем. Итак, когда удастся вам взять его, сохраните меня
и дом отца моего.
\vs 1Cl 12:7
А они сказали ей: будет
так, как ты сказала нам. Как скоро узнаешь о приближении нашем, собери всех
своих под кровлю твою и будут целы, а кто будет найден вне дома, погибнет.
\vs 1Cl 12:8
Притом дали ей знак, чтобы
она свесила из дома своего красную вервь,~--- и тем показали, что всем верующим
и уповающим на Бога будет искупление кровью Господа.
\vs 1Cl 12:9
Видите, возлюбленные, в
этой жене была не только вера, но и пророчество.

\vs 1Cl 13:1
Итак, будем смиренны,
братья, отложив всякое надмение, гордость, неразумие и гнев, и будем
поступать, как написано.
\vs 1Cl 13:2
Ибо говорит Дух Святой:
да не похвалится мудрый мудростью своей, ни сильный силой своей, ни богатый
богатством своим,
\vs 1Cl 13:3
но хвалящийся пусть
хвалится ЯХВЕ, ища Его, и творя суд и правду.
\vs 1Cl 13:4
Особенно будем помнить
слова Господа Иисуса, которые изрек Он, научая кротости и великодушию.
\vs 1Cl 13:5
Он так сказал: милуйте,
чтобы быть помилованными, отпускайте, дабы вам было отпущено;
\vs 1Cl 13:6
как вы делаете, так вам
будут делать; как даете, так вам дано будет;
\vs 1Cl 13:7
как судите, так сами
судимы будете; как будете снисходить, так к вам будут снисходить; какою мерою
мерите, такою отмерится вам.
\vs 1Cl 13:8
Этой заповедью и этими
внушениями утвердим себя, чтобы ходить со смирением, повинуясь святым
повелениям Его.
\vs 1Cl 13:9
Ибо святое слово говорит:
на кого воззрю,~--- только на кроткого и тихого, и трепещущего слов Моих.

\vs 1Cl 14:1
Итак, праведное и святое
дело, братья, более повиноваться Богу, нежели последовать тем, которые в
надменности и кичливости стали предводителями презренной зависти.
\vs 1Cl 14:2
Ибо не малому вреду, а
напротив, подвергнемся великой опасности, если опрометчиво отдадим себя на
волю тех людей, которые подстрекают нас к раздору и мятежам, чтобы отвести нас
от добродетели.
\vs 1Cl 14:3
Будем снисходительны друг
к другу, как милосерд и благ Сотворивший нас;
\vs 1Cl 14:4
ибо написано: добрые
будут обитателями земли, и невинные останутся на ней; а беззаконные истребятся
с нее.
\vs 1Cl 14:5
И опять говорит Писание:
я видел нечестивого превозносящегося и возвышающегося, как кедры ливанские; и
прошел я мимо, и вот его уже не стало, и искал я места его, и не нашел.
\vs 1Cl 14:6
Храни невинность и
соблюдай правоту, потому что мирного человека ожидают добрые последствия.

\vs 1Cl 15:1
Итак, присоединимся к
тем, которые с благочестием хранят мир, а не к тем, которые с лицемерием
желают мира;
\vs 1Cl 15:2
ибо сказано где-то: эти
люди почитают Меня устами, сердце же их далеко отстоит от Меня.
\vs 1Cl 15:3
И в другом месте: устами
своими они благословляли, а сердцем своим проклинали.
\vs 1Cl 15:4
И еще сказано: возлюбили
Его устами своими, и языком своим солгали Ему; сердце же их не было право с
Ним; и они не были верны в завете Его.
\vs 1Cl 15:5
Да будут немы уста
льстивые, и да истребит Господь уста льстивых и язык велеречивый,~--- тех,
которые говорят: язык наш возвеличим, уста наши при нас: кто нам Господь?
\vs 1Cl 15:6
Ради бедствий нищих, и
воздыхания убогих, ныне Я восстану, говорит ЯХВЕ: послужу им спасением, и буду
поступать с ними честно.

\vs 1Cl 16:1
Ибо Христос принадлежит
смиренным, а не тем, которые возносятся над стадом Его.
\vs 1Cl 16:2
Жезл величия Божьего,
Господь наш Иисус Христос, не пришел в блеске великолепия и надменности, хотя
и мог бы, но смиренно, как сказал о Нем Дух Святой.
\vs 1Cl 16:3
Ибо говорит Он: ЯХВЕ, кто
верил слуху нашему? и рука ЯХВЕ кому открылась?
\vs 1Cl 16:4
Мы возвестили пред Ним; Он
как малый отрок, как корень в земле жаждущей,~--- не имеет ни вида, ни славы.
\vs 1Cl 16:5
И мы видели Его, и не имел
Он ни вида, ни красоты; но вид Его бесчестен, унижен более вида человеков: Он
человек в язве и страдании, умеющий переносить болезнь;
\vs 1Cl 16:6
потому что отвратилось
лицо Его,~--- было поругано и презрено. Он грехи наши носит и за нас страдает;
\vs 1Cl 16:7
а мы думали, что Он
праведно подвержен страданию, и язве, и мучению;
\vs 1Cl 16:8
но Он уязвлен был за грехи
наши и мучен был за беззакония наши; наказание мира нашего на Нем, чрез рану
Его мы исцелились.
\vs 1Cl 16:9
Все мы, как овцы,
заблудились; человек блуждал на пути своем, и ЯХВЕ предал Его за грехи наши.
\vs 1Cl 16:10
И Он, будучи мучим, не
отверзает уст: как овца был веден на заклание, и как агнец безгласный пред
стригущим его, так Он не отверзает уст Своих. За смирение Его с Него снят был
суд.
\vs 1Cl 16:11
Кто расскажет Его род,
когда жизнь Его берется от земли? За беззакония людей Моих Он идет на смерть.
\vs 1Cl 16:12
И потому помилую злых за
гроб Его, и богатых за смерть Его; ибо Он не сделал беззакония и обмана не
нашлось в устах Его.
\vs 1Cl 16:13
И ЯХВЕ угодно очистить
Его от язвы; если дадите жертву о грехе, то душа ваша узрит семя долговечное.
\vs 1Cl 16:14
И Господь хочет спасти
Его от страдания души Его, показать Ему свет и образовать разумом, и оправдать
праведного, который благодетельно послужил многим; и грехи их Он понесет.
\vs 1Cl 16:15
Поэтому Он будет обладать
многими и разделит добычи сильных,~--- за то, что предана была на смерть душа
Его и был причтен к злодеям; и Он уничтожил грехи многих и за беззакония их
был предан.
\vs 1Cl 16:16
И опять Он же говорит: Я
червь, а не человек, поношение человеков и уничижение людей.
\vs 1Cl 16:17
Все видящие Меня
издевались надо Мною, говорили устами и кивали головою, говоря: Он уповал на
Господа, пусть избавит Его и сохранит Его, так как благоволит к Нему.
\vs 1Cl 16:18
Видите возлюбленные,
какой дан нам образец: ибо если Господь так смирил Себя, то что должны делать
мы, которые чрез Него пришли под иго благодати Его?

\vs 1Cl 17:1
Будем подражать и тем,
которые скитались в козьих и овечьих кожах, проповедуя о пришествии Христовом:
\vs 1Cl 17:2
разумеем пророков Илию,
Елисея и Иезекииля, также и тех, которые получили прекрасное свидетельство.
\vs 1Cl 17:3
Авраам получил великое
свидетельство, и назван другом Божьим: но, взирая на славу Божью, со смирением
говорит: я земля и пепел.
\vs 1Cl 17:4
Далее и об Иове так
написано: Иов был праведен и непорочен, истинен и благочестив, и удалялся от
всякого зла.
\vs 1Cl 17:5
Но он, сам себя осуждая,
сказал: никто не чист от скверны, хотя бы и один день была жизнь Его.
\vs 1Cl 17:6
Моисей назван верным во
всем доме его, и Бог через его служение совершил суд над Египтом посредством
мучений и казней:
\vs 1Cl 17:7
но и он, столько
прославленный, не величался, но, когда из купины было к нему Божественное
слово, сказал: кто я, что Ты меня посылаешь? Я заика и косноязычен. И опять
говорит: я пар из котла.

\vs 1Cl 18:1
Что же скажем о
прославленном Давиде, о котором сказал Бог: Я нашел человека по сердцу Моему,
Давида сына Иессеева, милостью вечной Я помазал его?
\vs 1Cl 18:2
Но и он говорит Богу:
помилуй меня, Боже, по великой милости Твоей, и по множеству щедрот Твоих
очисти беззаконие мое.
\vs 1Cl 18:3
Еще более~--- омой меня от
неправды моей, и очисти меня от греха моего, ибо я знаю неправду свою и грех
мой всегда предо мною.
\vs 1Cl 18:4
Тебе одному согрешил я и
пред Тобою сделал зло, чтобы Ты оправдался в словах Твоих и победил, когда
станут судить Тебя.
\vs 1Cl 18:5
Ибо в беззакониях зачат я
и в грехах родила меня мать моя.
\vs 1Cl 18:6
Ты возлюбил истину:
сокровенные тайны премудрости Твоей Ты открыл мне.
\vs 1Cl 18:7
Окропи меня иссопом, и
буду чист, омой меня, и буду белее снега.
\vs 1Cl 18:8
Слуху моему дай радость и
веселье: и сокрушенные кости мои возрадуются.
\vs 1Cl 18:9
Отврати лицо от грехов
моих, и изгладь все беззакония мои.
\vs 1Cl 18:10
Создай во мне сердце
чистое, Боже, и дух правый обнови в утробе моей.
\vs 1Cl 18:11
Не отвергни меня от лица
Твоего, и Духа Твоего Святого не отними от меня.
\vs 1Cl 18:12
Воздай мне радость
спасения Твоего, и укрепи меня Духом Адонаи.
\vs 1Cl 18:13
Научу грешников путям
Твоим и нечестивые обратятся к Тебе.
\vs 1Cl 18:14
Избавь меня от пролития
крови, Боже, Бог спасения моего. Язык мой воспоет правду Твою.
\vs 1Cl 18:15
Адонай, открой уста мои,
и уста мои возвестят хвалу Твою.
\vs 1Cl 18:16
Если бы Ты восхотел иной
жертвы, я принес бы; но всесожжения Тебе неугодны.
\vs 1Cl 18:17
Жертва Богу~--- дух
сокрушенный; сердце сокрушенное и смиренное Бог не презрит.

\vs 1Cl 19:1
Смирение и послушливая
покорность этих мужей, получивших столь славное свидетельство от Самого Бога,
сделали лучшими не только нас, но и прежде бывшие поколения,
\vs 1Cl 19:2
именно тех, которые со
страхом и искренностью принимали глаголы Его.
\vs 1Cl 19:3
Итак, имея пред собою
столь многие великие и славные деяния, обратимся к цели мира, указанной нам
изначала,
\vs 1Cl 19:4
и взирая к Отцу и
Создателю всего мира, вникнем в Его величественные и превосходные дары мира и
в Его благодеяния.
\vs 1Cl 19:5
Воззрим на Него умом и
душевными очами, посмотрим на долготерпение Его воли, и помыслим, как Он
кроток ко всему творению Своему.

\vs 1Cl 20:1
Небеса, по Его
распоряжению движущиеся, в мире повинуются Ему: и день и ночь совершают
определенное им течение, не препятствуя друг другу.
\vs 1Cl 20:2
Солнце и лики звезд, по
Его велению, согласно, без малейшего уклонения проникают на назначенные им
пути.
\vs 1Cl 20:3
Плодоносящая земля, по Его
воле, в определенные времена производит изобильную пищу человекам, зверям и
всем находящимся на ней животным, не замедляя и не изменяя ничего из
предписанного им.
\vs 1Cl 20:4
Неисследуемые и
непостижимые области бездны и преисподней держатся теми же велениями.
\vs 1Cl 20:5
Беспредельное море, по Его
устроению совокупленное в большие водные массы, не выступает за положенные ему
преграды, но делает так, как Он повелел.
\vs 1Cl 20:6
Ибо Он сказал: доселе
дойдешь, и волны твои в тебе сокрушатся.
\vs 1Cl 20:7
Непроходимый для людей
океан, и миры за ним находящиеся, управляются теми же повелениями Господа.
\vs 1Cl 20:8
Времена года~--- весна,
лето, осень и зима мирно сменяются одни другими.
\vs 1Cl 20:9
Определенные ветры, каждый
в свое время, беспрепятственно совершают свое служение.
\vs 1Cl 20:10
Неиссякающие источники,
созданные для наслаждения и здравия, непрестанно доставляют людям свою влагу,
необходимую для их жизни.
\vs 1Cl 20:11
Наконец, малейшие
животные мирно и согласно составляют сожительства между собою.
\vs 1Cl 20:12
Всему этому повелел быть
в согласии и мире великий Создатель и Владыка всего,
\vs 1Cl 20:13
Который благотворит всем,
а преимущественно нам, которые прибегли к милосердию Его чрез Господа нашего
Иисуса Христа, Которому слава и величие во веки веков. Аминь.

\vs 1Cl 21:1
Смотрите, возлюбленные,
чтобы столь многие благодеяния Его не обратились всем нам в осуждение, если
мы, живя достойно Его, не будем единодушно совершать благое и угодное Ему.
\vs 1Cl 21:2
Ибо сказано где-то: Дух
Господа есть светильник, испытующий тайны утробы.
\vs 1Cl 21:3
Помыслим, как Он близок к
нам, и что ни одна из наших мыслей или совещаний, какие мы делаем, не закрыты
от Него.
\vs 1Cl 21:4
Итак, надлежит нам не
отступать от воли Его: лучше воспротивимся глупым и несмысленным,
превозносящимся и хвалящимся пышностью слова своего людям, нежели Богу.
\vs 1Cl 21:5
Будем благоговеть перед
Господом Иисусом Христом, кровь Которого предана за нас, будем почитать
предстоятелей наших, уважать пресвитеров, юношей воспитывать в страхе Божьем,
\vs 1Cl 21:6
жен своих направлять к
добру, чтобы они отличались достолюбезным нравом целомудрия, показывали чистое
свое расположение к кротости, скромность языка своего обнаруживали молчанием,
любовь свою оказывали не по склонностям, но равную ко всем, свято боящимся
Бога.
\vs 1Cl 21:7
Дети ваши пусть получают
воспитание христианина; пусть научаются, как сильно пред Богом смирение, что
значит пред Богом чистая любовь, как прекрасен и велик страх Божий и
спасителен для всех, свято ходящих в нем с чистым умом.
\vs 1Cl 21:8
Ибо Он есть испытатель
мыслей и желаний наших: Его дыхание в нас, и когда захочет, возьмет его.

\vs 1Cl 22:1
Все сие подтверждает вера
христианская. Ибо Сам Христос чрез Духа Святого так взывает к нам:
\vs 1Cl 22:2
приходите, дети,
послушайте Меня; страху ЯХВЕ научу вас.
\vs 1Cl 22:3
Кто есть человек, хотящий
жизни, любящий видеть дни благие?
\vs 1Cl 22:4
Удержи язык твой от зла, и
уста твои, чтобы не говорить коварства.
\vs 1Cl 22:5
Уклонись от зла и сотвори
доброе; ищи мира, и гонись за ним.
\vs 1Cl 22:6
Очи ЯХВЕ~--- на праведных, и
уши Его~--- на молитву их:
\vs 1Cl 22:7
а на делающих злое лице
ЯХВЕ для того, чтобы истребить с земли память их.
\vs 1Cl 22:8
Воззвал праведник, и ЯХВЕ
услышал его, и избавил его от всех скорбей его.
\vs 1Cl 22:9
Много бичей грешному:
уповающих же на ЯХВЕ будет окружать милость.

\vs 1Cl 23:1
Милосердый во всем и
благодетельный Отец милостив к боящимся Его, и дары Свои охотно и ласково
раздает приступающим к Нему с чистым расположением.
\vs 1Cl 23:2
Посему не будем
сомневаться, и душа наша да не отчаивается о превосходных и славных дарах Его:
\vs 1Cl 23:3
да будет далеко от нас
сказанное в Писании, где оно говорит: несчастны двоедушные, колеблющиеся
душою и говорящие:
\vs 1Cl 23:4
это мы слышали и во время
отцов наших, и вот мы состарились, но ничего такого с нами не случилось.
\vs 1Cl 23:5
Неразумные! Сравните себя
с деревом, возьмите виноградную лозу:
\vs 1Cl 23:6
сперва она теряет лист,
потом образуется отпрыск, потом лист, потом цвет, и после этого незрелый,
наконец, спелый виноград.
\vs 1Cl 23:7
Видите, как в короткое
время древесный плод достигает зрелости.
\vs 1Cl 23:8
Скоро поистине и внезапно
совершится воля Господа по свидетельству самого Писания: скоро придет, и не
замедлит, и внезапно придет в храм Свой ЯХВЕ и Святой, Которого вы ожидаете.

\vs 1Cl 24:1
Рассмотрим, возлюбленные,
как Господь постоянно показывает нам будущее воскресение, которого начатком
сделал Господа Иисуса Христа, воскресив Его из мертвых.
\vs 1Cl 24:2
Посмотрим, возлюбленные,
на воскресение, совершающееся во всякое время.
\vs 1Cl 24:3
День и ночь представляют
нам воскресение: ночь отходит ко сну,~--- встает день; проходит день,~--- настает
ночь.
\vs 1Cl 24:4
Посмотрим на плоды, каким
образом происходит сеяние зерен.
\vs 1Cl 24:5
Вышел сеятель, бросил их в
землю, и брошенные семена, которые упали на землю сухие и голые, сгнивают;
\vs 1Cl 24:6
но после этого разрушения
великая сила Промысла Господня воскрешает их, и из одного возвращает многие и
производит плод.

\vs 1Cl 25:1
Взглянем на необычайное
знамение, бывающее в восточных странах, то есть около Аравии.
\vs 1Cl 25:2
Есть там птица, которая
называется Феникс. Она рождается только одна и живет по пяти сот лет.
\vs 1Cl 25:3
Приближаясь к своему
разрушению смертному, она из ливана, смирны и других ароматов делает себе
гнездо, в которое, по исполнении своего времени, входит и умирает.
\vs 1Cl 25:4
Из гниющего же тела
рождается червь, который, питаясь влагою умершего животного, оперяется;
\vs 1Cl 25:5
потом, пришедши в
крепость, берет то гнездо, в котором лежат кости его предка, и с этою ношею
совершает путь из Аравии в Египет, в город, называемый Илиополь,
\vs 1Cl 25:6
и прилетая днем, в виду
всех кладет это на жертвенник солнца, и таким образом назад удаляется.
\vs 1Cl 25:7
Жрецы рассматривают
летописи, и находят, что она являлась по исполнении пятисот лет.

\vs 1Cl 26:1
Итак, почтем ли мы
великим и удивительным, если Творец всего воскресит тех, которые в уповании
благой веры свято служили Ему,
\vs 1Cl 26:2
когда Он и посредством
птицы открывает нам Свое великое обещания Своего?
\vs 1Cl 26:3
Ибо говорится где-то: и
Ты воскресишь меня и восхвалю Тебя.
\vs 1Cl 26:4
И еще: я уснул, и спал,
восстал, потому что Ты со мной.
\vs 1Cl 26:5
Так же Иов говорит: и Ты
воскресишь эту плоть мою, которая терпит все это.

\vs 1Cl 27:1
В этой надежде да
прилепятся души наши к Тому, Кто верен в обещаниях и праведен в судах.
\vs 1Cl 27:2
Заповедавший не лгать, тем
более Сам не солжет; ибо для Бога ничего нет невозможного: невозможно только
солгать.
\vs 1Cl 27:3
Итак, да воспламенится в
нас вера Его, и будем помышлять, что все близко к Нему.
\vs 1Cl 27:4
Словом величества Своего
Он все создал, словом же может и разрушить это.
\vs 1Cl 27:5
Кто скажет Ему: зачем
сделал? или кто воспротивится могуществу силы Его.
\vs 1Cl 27:6
Когда Ему угодно, Он все
сделает, и ничего из определенного Им не останется без исполнения.
\vs 1Cl 27:7
Все пред Ним, и ничто не
скрыто от совета Его.
\vs 1Cl 27:8
Если небеса поведают
славу Божью, то твердь возвещает о творении рук Его;
\vs 1Cl 27:9
день дню отрыгает слово, и
ночь ночи возвращает ведение.
\vs 1Cl 27:10
И нет слов, ни речей,
звуки которых не были бы слышимы.

\vs 1Cl 28:1
Итак, если Бог все видит
и слышит, то убоимся Его, и оставим нечистые стремления к худым делам, чтобы
милосердием Его покрыться от будущих судов.
\vs 1Cl 28:2
Ибо куда может кто-либо из
нас убежать от крепкой руки Его? Какой мир примет убежавшего от Него?
\vs 1Cl 28:3
Ибо говорит негде Писание:
куда пойду и где скроюсь от лица Твоего?
\vs 1Cl 28:4
Если взойду на небо, Ты
там; если пойду на конец земли, и там десница Твоя; если расположусь в
безднах, и там Дух Твой.
\vs 1Cl 28:5
Итак, куда мог бы кто
удалиться, или куда убежать от Того, Кто все объемлет?

\vs 1Cl 29:1
Итак, приступим к Нему в
святости души, поднимая к Нему чистые и нескверные руки,
\vs 1Cl 29:2
и любя кроткого
милосердого Отца нашего, Который избрал нас в достояние Себе;
\vs 1Cl 29:3
ибо так написано: когда
Вышний разделял народы, когда расселял сынов Адамовых, то Он поставил пределы
народов по числу ангелов Божьих:
\vs 1Cl 29:4
и уделом ЯХВЕ стал народ
Его Иаков, межею наследия Его~--- Израиль.
\vs 1Cl 29:5
И в другом месте говорится
вот ЯХВЕ избирает Себе народ из среды народов, как человек берет начатки с
гумна своего и произойдет из того народа святое святых.

\vs 1Cl 30:1
Итак, будучи уделом
Святого, будем делать все относящееся к святости,
\vs 1Cl 30:2
убегая злословия, нечистых
и порочных связей, пьянства, страсти к нововведениям,
\vs 1Cl 30:3
низких пожеланий,
скверного расового смешения и гнусной гордости.
\vs 1Cl 30:4
Ибо говорится: Бог гордым
противится, смиренным же дает благодать.
\vs 1Cl 30:5
Итак, присоединимся к тем,
которым дана от Бога благодать.
\vs 1Cl 30:6
Облечемся в единомыслие,
будем смиренны, воздержны, далеки от всякой клеветы и злоречия, оправдывая
себя делами, а не словами.
\vs 1Cl 30:7
Ибо сказано: кто говорит
много, тот должен и слушать в свою очередь; или многоречивый будет праведен?
Благословен рожденный от жены, малолетний. Не будь многоречив.
\vs 1Cl 30:8
Хвала наша да будет у
Бога, а не от нас самих; Бог ненавидит тех, которые сами хвалят себя.
\vs 1Cl 30:9
Пусть свидетельство о
добром поведении нашем будет дано от других, так как дано было оно отцам нашим
праведным.
\vs 1Cl 30:10
Наглость, надменность и
дерзость свойственны проклятым от Бога; умеренность, смиренномудрие и кротость
у благословенных от Бога.

\vs 1Cl 31:1
Итак, взыщем
благословения Его, и посмотрим, какие пути приводят к благословению. Вспомним,
что было от начала.
\vs 1Cl 31:2
За что был благословен
отец наш Авраам? Не за то ли, что по вере своей творил правду и истину?
\vs 1Cl 31:3
Исаак, с уверенностью зная
будущее, охотно стал жертвою.
\vs 1Cl 31:4
Иаков со смирением оставил
из-за брата землю свою, пошел к Лавану и служил; и даны ему двенадцать колен
Израилевых.

\vs 1Cl 32:1
Если кто рассмотрит все в
подробности, то познает величие даров, данных от Бога.
\vs 1Cl 32:2
От Иакова все священники и
левиты, служащие при жертвеннике Божьем.
\vs 1Cl 32:3
От него Господь Иисус по
плоти: от него цари, начальники, вожди чрез Иуду;
\vs 1Cl 32:4
и прочие его колена в
немалой славе, так как обещал Бог: будет семя твое, как звезды небесные.
\vs 1Cl 32:5
И все они прославились и
возвеличились не сами собой, и не делами своими, и не правотой действий,
совершенных ими, но волей Божьей.
\vs 1Cl 32:6
Так и мы, будучи призваны
по воле Его во Христе Иисусе, оправдываемся не сами собою, и не своею
мудростью, или разумом, или благочестием, или делами, в святости сердца нами
совершаемыми,
\vs 1Cl 32:7
но посредством веры,
которую Вседержитель Бог от века всех оправдывал. Ему да будет слава во веки
веков. Аминь.

\vs 1Cl 33:1
Итак, что нам делать,
братья? Отстать ли от добродетели и любви?~--- Отнюдь нет, не дай Господь, чтоб
это сталось с нами;
\vs 1Cl 33:2
напротив, со всем усилием
и готовностью поспешим совершать доброе дело.
\vs 1Cl 33:3
Ибо Сам Творец и Владыка
всего веселится о делах Своих.
\vs 1Cl 33:4
Он высочайшею Своею силою
утвердил небеса и непостижимою Своею мудростью украсил их;
\vs 1Cl 33:5
Он отделил землю от
окружающей ее воды, и утвердил на прочном основании Своего хотения,
\vs 1Cl 33:6
и Своею властью повелел
быть ходящим на ней животным.
\vs 1Cl 33:7
Он также сотворил море и в
нем животных, и оградил Своим могуществом.
\vs 1Cl 33:8
Сверх всего этого Он
святыми и чистыми руками образовал отличнейшее и по разуму превосходнейшее
существо, человека, начертание Своего образа.
\vs 1Cl 33:9
Ибо так говорит Бог:
сотворим человека по образу и подобию Нашему. И сотворил Бог человека, мужа и
жену сотворил их.
\vs 1Cl 33:10
Совершив все это, Он
одобрил и благословил и сказал: раститесь и умножайтесь.
\vs 1Cl 33:11
Познаем также, что все
праведные украсились добрыми делами; и Сам Господь радовался, украсив Себя
делами.
\vs 1Cl 33:12
Имея такой пример,
неленостно последуем воле Его, и всею силою будем творить дело правды.

\vs 1Cl 34:1
Добрый работник смело
получает хлеб за труд свой; ленивый же и беспечный не смеет и взглянуть на
того, кто дал ему работу.
\vs 1Cl 34:2
И нам надлежит быть
ревностными в делании добра, ибо все от Него.
\vs 1Cl 34:3
Ибо предсказывает нам:
вот ЯХВЕ, и награда Его перед лицом Его, чтобы воздать каждому по делу его.
\vs 1Cl 34:4
Так увещевает Он нас всем
сердцем обратиться к Нему, и ни в каком добром деле не быть беспечными и
нерадивыми;
\vs 1Cl 34:5
в Нем да будет похвала и
надежда наша; покоримся воле Его.
\vs 1Cl 34:6
Помыслим о всем множестве
ангелов Его, как они, предстоя, исполняют волю Его.
\vs 1Cl 34:7
Ибо говорит Писание: тьмы
тем предстояли пред Ним и тысячи тысяч служили Ему, и взывали: свят, свят,
свят ЯХВЕ Цебаот; полно все творение славы Его.
\vs 1Cl 34:8
Так и мы, в единомысленном
собрании, единым духом, как бы из одних уст, будем взывать к Нему непрестанно,
чтобы сделаться нам участниками великих и славных обетований Его.
\vs 1Cl 34:9
Ибо говорит: око не
видело, и ухо не слышало, и на сердце человеку не приходило то, что Он
уготовал уповающим на Него.

\vs 1Cl 35:1
Как блаженны и чудны дары
Божьи, возлюбленные~--- жизнь в бессмертии, сияние в правде, истина в свободе,
вера в уповании, воздержание в святости: все это доступно нашему разумению.
\vs 1Cl 35:2
Какие же еще уготовляются
ждущим? Творец и Отец веков, Всесвятой, Он Сам знает их величие и красоту.
\vs 1Cl 35:3
Итак, употребим все усилия
быть в числе уповающих на Него, чтобы участвовать в обетованных дарах.
\vs 1Cl 35:4
Каким же образом это
будет, возлюбленные? Если ум наш будет утвержден в вере в Бога; если будем
искать того, что Ему угодно и приятно;
\vs 1Cl 35:5
если будем исполнять то,
что согласно с Его святою волею, и ходить путем истины, отвергнув от себя
всякую неправду и беззаконие,
\vs 1Cl 35:6
любостяжание, распри,
злонравие и коварство, клеветы и злословие, нечестие, гордость и величавость,
тщеславие и негостеприимность.
\vs 1Cl 35:7
Ибо делающие это
ненавистны Богу, и не только делающие, но и одобряющие это.
\vs 1Cl 35:8
Писание говорит: грешнику
сказал Бог: зачем ты познаешь заповеди Мои и принимаешь завет Мой устами
твоими, а возненавидел вразумление и отверг слова Мои?
\vs 1Cl 35:9
Если ты видел вора, то
бежал с ним, и с прелюбодеем принимал участие.
\vs 1Cl 35:10
Уста твои были исполнены
злобы, и язык твой сплетал обманы.
\vs 1Cl 35:11
Сидя на суде, ты клеветал
на брата твоего и сыну матери твоей строил ковы.
\vs 1Cl 35:12
Ты это делал, и Я молчал;
ты, беззаконный, подумал, что буду тебе подобен.
\vs 1Cl 35:13
Но обличу тебя и
представлю тебя перед лицом твоим.
\vs 1Cl 35:14
Разумейте же это вы,
забывающие Бога, чтобы вам не быть похищенными как бы львом, и некому будет
избавить вас.
\vs 1Cl 35:15
Жертва хвалы прославит
Меня, и там путь, на котором явлю ему спасение Божье.

\vs 1Cl 36:1
Таков путь, возлюбленные,
которым мы обретаем наше спасение, Иисуса Христа, Первосвященника наших
приношений, заступника и помощника в немощи нашей.
\vs 1Cl 36:2
Посредством Него взираем
мы на высоту небес; чрез Него, как бы в зеркале видим чистое и пресветлое лицо
Бога;
\vs 1Cl 36:3
чрез Него отверзлись очи
сердца нашего; чрез Него несмысленный и омраченный ум наш возникает в чудный
Его свет;
\vs 1Cl 36:4
чрез Него восхотел
Господь, чтобы мы вкусили бессмертного знания.
\vs 1Cl 36:5
Он, будучи сиянием величия
Его, столько превосходнее ангелов, сколько славнейшее пред ними наследовал
имя.
\vs 1Cl 36:6
Ибо так написано: Он
творит ангелов Своих духами и служителей Своих пламенем огненным.
\vs 1Cl 36:7
О Сыне же Своем так сказал
Господь: Сын Мой Ты, Я ныне родил Тебя, проси от Меня и дам Тебе народы в
достояние Твое, и пределы земли~--- в обладание Твое.
\vs 1Cl 36:8
И еще говорит к Нему:
сиди одесную Меня, доколе положу врагов Твоих в подножие ног Твоих.
\vs 1Cl 36:9
Кто же враги?~--- Порочные,
противящиеся воле Божьей.

\vs 1Cl 37:1
Итак, братья! будем всеми
силами воинствовать под святыми Его повелениями.
\vs 1Cl 37:2
Представим себе
воинствующих под начальством вождей наших; как стройно, как усердно, как
покорно исполняют они приказания.
\vs 1Cl 37:3
Не все епархи, не все
тысяченачальники, или стоначальники или пятидесятиначальники и так далее, но
каждый в своем чине исполняет приказания царя и полководцев.
\vs 1Cl 37:4
Ни великие без малых, ни
малые без великих не могут существовать.
\vs 1Cl 37:5
Все они как бы связаны
вместе, и это доставляет пользу.
\vs 1Cl 37:6
Возьмем тело наше: голова
без ног ничего не значит, равно и ноги без головы, и малейшие члены в теле
нашем нужны и полезны для целого тела;
\vs 1Cl 37:7
все они согласны и
стройным подчинением служат для целого тела.

\vs 1Cl 38:1
Так пусть будет здраво и
все тело наше в Иисусе Христе, и каждый повинуется ближнему своему сообразно
со степенью, на которой он поставлен дарованием Его.
\vs 1Cl 38:2
Сильный не пренебрегай
слабого, и слабый почитай сильного; богатый подавай бедному, и бедный
благодари Бога, что Он даровал ему, чрез кого может быть восполнена его
скудость.
\vs 1Cl 38:3
Мудрый показывай мудрость
свою не в словах, но в добрых делах.
\vs 1Cl 38:4
Смиренный не сам о себе
свидетельствуй, но предоставляй другому дать о тебе свидетельство.
\vs 1Cl 38:5
Чистый по плоти молчи и не
превозносись, зная, что есть другой, дарующий ему воздержание.
\vs 1Cl 38:6
Помыслим, братья, из
какого вещества мы произошли, и какими вошли в мир, как бы из гроба и мрака.
\vs 1Cl 38:7
Творец и Создатель наш
ввел нас в мир Свой, наперед приготовил нам Свои благодеяния прежде рождения
нашего.
\vs 1Cl 38:8
Итак, все имея от Него, мы
должны за все благодарить Его. Ему слава во веки веков. Аминь.

\vs 1Cl 39:1
Безумные, несмысленные,
глупые и невежды смеются и ругаются над нами, желая самих себя возвысить в
собственных мыслях своих.
\vs 1Cl 39:2
Но что может смертный, или
какая крепость в земнородном?
\vs 1Cl 39:3
Ибо написано: не было
образа пред глазами моими; но я слышал тихое веяние и голос:
\vs 1Cl 39:4
что же? будет ли человек
чист пред ЯХВЕ, или в делах своих непорочен, если Он на служителей Своих не
полагается и в ангелах Своих усматривает недостатки?
\vs 1Cl 39:5
Небо не чисто пред Ним;
тем менее живущие в бренных храминах, из числа которых и мы сами из того же
брения.
\vs 1Cl 39:6
Как бы моль поела их, и от
утра до вечера их уже нет: от того, что не могут помочь самим себе, они
погибли.
\vs 1Cl 39:7
Дунул на них и погибли,
потому что не имеют мудрости.
\vs 1Cl 39:8
Призови же, услышит ли
тебя кто-нибудь, или увидишь ли кого из святых ангелов?
\vs 1Cl 39:9
Безумного губит гнев и
глупого умерщвляет рвение.
\vs 1Cl 39:10
Я видел безумных
укореняющихся, но тотчас истреблено было их жилище.
\vs 1Cl 39:11
Да будут сыны их далеко
от спасения и да будут презрены при дверях меньших, и некому будет спасти их.
\vs 1Cl 39:12
Ибо что они собрали,
поедят праведные, сами же от зол не будут изъяты.

\vs 1Cl 40:1
Будучи убеждены в этом и
проникая в глубины божественного ведения, мы должны в порядке совершать все,
что Господь повелел совершать в определенные времена.
\vs 1Cl 40:2
Он повелел, чтобы жертвы и
священные действия совершались не случайно и не без порядка, но в определенные
времена и часы.
\vs 1Cl 40:3
Также где и через кого
должно быть это совершаемо, Сам Он определил высочайшим Своим изволением,
чтобы все совершалось свято и богоугодно, и было приятно воле Его.
\vs 1Cl 40:4
Итак, приятны Ему и
блаженны те, которые в установленные времена приносят жертвы свои;
\vs 1Cl 40:5
ибо, следуя заповедям
Господним, они не погрешают.
\vs 1Cl 40:6
Первосвященнику дано свое
служение, священникам назначено свое дело, и на левитов возложены свои
должности;
\vs 1Cl 40:7
из народа человек связан
постановлениями для народа.

\vs 1Cl 41:1
Каждый из вас, братья,
благодари Бога за свое собственное положение, храня добрую совесть и с
благоговением не преступая определенного правила служения своего.
\vs 1Cl 41:2
Не повсюду, братья,
приносятся жертвы непрерывные, или обетные или жертвы за грех, и жертвы
повинности, но только в Иерусалиме,
\vs 1Cl 41:3
и там не на всяком месте
совершается приношение, а пред храмом на жертвеннике, после того как жертва
будет осмотрена первосвященником и вышеназванными служителями.
\vs 1Cl 41:4
Те же, которые делают
что-либо вопреки Его воле, наказываются смертью.
\vs 1Cl 41:5
Видите, братья, чем
большего сподобились мы видения, тем большей подлежим опасности.

\vs 1Cl 42:1
Апостолы были посланы
проповедовать благовестие нам от Господа Иисуса Христа, Иисус Христос~--- от
Бога.
\vs 1Cl 42:2
Христос был послан от
Бога, а апостолы~--- от Христа; то и другое было в порядке по воле Божьей.
\vs 1Cl 42:3
Итак принявши повеление,
совершенно убежденные чрез воскресение Господа нашего Иисуса Христа и
утвержденные в вере словом Божьим, с полнотой Духа Святого пошли
благовествовать наступающее царство Божье.
\vs 1Cl 42:4
Проповедуя по странам и
городам, они первенцев, по духовном испытании поставляли в епископы и диаконы
для будущих верующих.
\vs 1Cl 42:5
И это не новое
установление; ибо много веков прежде было писано о епископах и диаконах.
\vs 1Cl 42:6
Так говорит Писание:
поставлю епископов их в правде и диаконов в вере.

\vs 1Cl 43:1
И чему дивиться, если те,
которым во Христе вверено было бы от Бога это дело, поставляли вышеупомянутых?
\vs 1Cl 43:2
Блаженный Моисей,
вверенный служитель во всем доме Божьем, все заповеданное Ему изобразил в
священных книгах;
\vs 1Cl 43:3
ему последовали и прочие
пророки, утверждая своим свидетельством его узаконения.
\vs 1Cl 43:4
Когда возникла распря о
священстве, и колена разногласили о том, какое из них должно быть украшено
этим славным именем,
\vs 1Cl 43:5
то повелел двенадцати
начальникам колен принести к нему жезлы, на которых было написано имя каждого
колена;
\vs 1Cl 43:6
и взявши их, связал,
запечатал перстнями начальников колен, положил их в скинии свидетельства на
трапезе Господней.
\vs 1Cl 43:7
И, заключив скинию,
запечатал замки также, как и жезлы, и сказал им: братья, которого колена жезл
расцветет, то избрал Бог к священству и служению Себе.
\vs 1Cl 43:8
На другой день утром
созвал он всего Израиля, шесть сот тысяч человек, и показал начальникам колен
печати их, и отворил скинию свидетельства и вынес жезлы:
\vs 1Cl 43:9
и оказалось, что жезл
Аарона не только расцвел но даже имел на себе плод.
\vs 1Cl 43:10
Как вы думаете,
возлюбленные, не знал ли Моисей прежде, что это будет?
\vs 1Cl 43:11
Конечно знал, но так
поступил он для того, чтобы не было возмущения в Израиле, для прославления
имени истинного и единого Бога. Ему слава во веки веков. Аминь.

\vs 1Cl 44:1
И апостолы наши знали
чрез Господа нашего Иисуса Христа, что будет раздор о епископском достоинстве.
\vs 1Cl 44:2
По этой самой причине они,
получивши совершенное предведение, поставили вышеозначенных, и потом
присовокупили закон, чтобы когда они почиют, другие испытанные мужи принимали
на себя их служение.
\vs 1Cl 44:3
Итак, почитаем
несправедливым лишить служения тех, которые поставлены самими апостолами или
после них другими достоуважаемыми мужами, с согласия всей Церкви, и служили
стаду Христову неукоризненно, со смирением, кротко и беспорочно, и притом в
течение долгого времени от всех получили одобрение.
\vs 1Cl 44:4
И не малый будет на нас
грех, если неукоризненно и свято приносящих дары будем лишать епископства.
\vs 1Cl 44:5
Блаженны предшествовавшие
нам пресвитеры, которые разрешились от тела после многоплодной и совершенной
жизни:
\vs 1Cl 44:6
им нечего опасаться, чтобы
кто мог свергнуть их с занимаемого ими места.
\vs 1Cl 44:7
Ибо мы видим, что вы
некоторых, похвально провождающих жизнь, лишили служения безукоризненно ими
проходимого.

\vs 1Cl 45:1
Вы, братья, спорливы и
ревностны в том, что ни мало не относится к спасению.
\vs 1Cl 45:2
Загляните в Писания, эти
истинные глаголы Духа Святого. Заметьте, что в них ничего несправедливого и
превратного не написано.
\vs 1Cl 45:3
Вы не найдете чтобы люди
праведные были низвергаемы людьми святыми.
\vs 1Cl 45:4
Были гонимы праведные, но
от беззаконных; были заключаемы в темницу, но от нечестивых; были побиваемы
камнями от злодеев; были убиваемы от порочных, увлекавшихся преступною
завистью. Все эти страдания они перенесли со славою.
\vs 1Cl 45:5
Ибо что скажем, братья?
Даниил от богобоязненных ли людей был брошен в ров львиный? Анания, Азария и
Мисаил от чтителей ли благолепного и славного служения Всевышнему были
ввержены в пещь огненную?~--- Отнюдь нет.
\vs 1Cl 45:6
Кто же сделал это?~--- Люди
порочные, полные всякого зла, дошли до такого неистовства, что святою и
непорочною волею служащих Богу подвергли мучениям:
\vs 1Cl 45:7
они не знали того, что
Вышний есть заступник и защитник тех, которые с чистой совестью чтут
всесовершенное имя Его. Ему слава во веки веков. Аминь.
\vs 1Cl 45:8
А они, терпя в уповании, и
были превознесены Богом, и сделались достолюбезными в памяти их во веки веков.
Аминь.

\vs 1Cl 46:1
Таким примерам и мы
должны подражать, братья.
\vs 1Cl 46:2
Ибо написано: прилепитесь
к святым; ибо прилепляющиеся к ним освятятся.
\vs 1Cl 46:3
И опять в другом месте
сказано: с мужем невинным будешь невинен, и с избранным будешь избран, а с
развращенным развратишься.
\vs 1Cl 46:4
Итак, присоединимся к
невинным и праведным, они-то суть избранные Божьи.
\vs 1Cl 46:5
К чему у вас распри; гнев
несогласия, разделения, война?
\vs 1Cl 46:6
Не одного ли Бога и одного
Христа имеем мы? Не один ли Дух благодати излит на нас, не одно ли призвание
во Христе?
\vs 1Cl 46:7
Для чего раздираем и
расторгаем члены Христовы, восстаем против собственного тела, и до такого
доходим безумия, что забываем, что мы друг другу члены?
\vs 1Cl 46:8
Вспомните слова Иисуса,
Господа нашего. Он сказал: горе тому человеку; хорошо было бы ему не
родиться, нежели соблазнить одного из избранных Моих;
\vs 1Cl 46:9
было бы лучше для него,
если бы он повесил камень жерновный и ввергнулся в море, нежели соблазнить
одного из малых Моих.
\vs 1Cl 46:10
Ваше разделение многих
развратило, многих повергло в уныние, многих в сомнение, и всех нас в печаль,
а смятение ваше все еще продолжается.

\vs 1Cl 47:1
Возьмите послание
блаженного апостола Павла. О чем он прежде всего писал вам в начале ангельской
проповеди?
\vs 1Cl 47:2
Истинно он по вдохновению
написал вам как о себе самом, так и о Кифе и Аполлосе, потому что и тогда
произошло у вас разделение на различные стороны.
\vs 1Cl 47:3
Но тогдашнее разделение
подвергло вас меньшему греху; ибо вы преклонялись на стороны прославленных
апостолов, и на сторону мужа, им одобренного.
\vs 1Cl 47:4
А теперь подумайте, какие
люди развратили вас и уменьшили красоту знаменитой братской любви вашей.
\vs 1Cl 47:5
Постыдное, возлюбленные, и
чрезвычайно постыдное и христианской жизни недостойное слышится дело:
твердейшая и древняя церковь Коринфская из-за одного или двух человек
возмутилась против пресвитеров.
\vs 1Cl 47:6
И этот слух дошел не
только до нас, но и до самых врагов наших, так что чрез ваше безумие имя
Господне подвергается поруганию, и вам самим готовится опасность.

\vs 1Cl 48:1
Итак, прекратим это как
можно скорее, и припадем к Господу, и слезно будем умолять Его, чтобы Он,
умилосердившись, примирился с нами, и восстановил в нас прежнюю прекрасную и
чистую жизнь братской любви.
\vs 1Cl 48:2
Это~--- врата правды,
отверстые к жизни как написано: откройте Мне врата правды; Я войду с вами и
восхвалю ЯХВЕ. Это~--- врата ЯХВЕ, праведные войдут ими.
\vs 1Cl 48:3
Из многих открытых врат
врата правды суть врата Христовы, и блаженны те, которые входят ими и
направляют шествие свое в святости и правде, все совершая без возмущения.
\vs 1Cl 48:4
Если кто тверд в вере, или
способен предлагать ведение, или мудр в обсуждении речей, или чист по своим
делам; тем более он должен смиряться, чем более кажется великим, и должен
искать общей пользы, а не своей.

\vs 1Cl 49:1
Кто имеет любовь во
Христе, тот должен соблюдать заповеди Христовы.
\vs 1Cl 49:2
Кто может изъяснить союз
любви Божьей? Кто способен, как должно, высказать величие благости Его?
\vs 1Cl 49:3
Несказанна высота, на
которую возводит любовь. Любовь соединяет нас с Богом; любовь покрывает
множество грехов, любовь все принимает, все терпит великодушно.
\vs 1Cl 49:4
В любви нет ничего
низкого, ничего надменного, любовь не допускает разделения, любовь не заводит
возмущения,
\vs 1Cl 49:5
любовь все делает в
согласии, любовью все избранные Божьи достигли совершенства, без любви нет
ничего благоугодного Богу.
\vs 1Cl 49:6
По любви воспринял нас
Господь; по любви, которую имел к нам Иисус Христос, Господь наш, по воле
Божьей дал кровь за нас, и плоть за плоть нашу, и душу за души наши.

\vs 1Cl 50:1
Видите ли, возлюбленные,
как велика и дивна любовь, и невыразимо ее совершенство.
\vs 1Cl 50:2
Кто может иметь ее, если
кого Сам Бог не удостоит?
\vs 1Cl 50:3
Итак будем просить и
умолять Его милосердие, чтобы жить нам в любви непорочно, без человеческого
разделения.
\vs 1Cl 50:4
Все роды от Адама до сего
дня миновали; но усовершившиеся в любви по благодати Божьей находятся на месте
благочестивых: они откроются с пришествием царства Христова.
\vs 1Cl 50:5
Ибо написано: войди на
некоторое время в храмины, пока пройдет гнев и негодование Мое, и вспомню о
дне добром, и воскрешу вас от гробов ваших.
\vs 1Cl 50:6
Блаженны мы, возлюбленные,
если исполняем заповеди Божьи в единомыслии любви, дабы чрез любовь были
прощены нам грехи наши.
\vs 1Cl 50:7
Ибо написано: блаженны
те, которых беззакония отпущены и которых покрылись грехи. Блажен человек,
которому ЯХВЕ не вменит греха, и в устах его нет обмана.
\vs 1Cl 50:8
Это обещание блаженства
относится к тем, которые избраны Богом чрез Иисуса Христа, Господа нашего. Ему
слава во веки веков. Аминь.

\vs 1Cl 51:1
Итак, в чем мы согрешили
по каким-либо наветам врага, должны мы просить прощения.
\vs 1Cl 51:2
И те, которые были
предводителями возмущения и раздора, должны иметь в виду общую надежду.
\vs 1Cl 51:3
Ибо провождающие жизнь со
страхом и любовью лучше хотят сами подвергнуться неприятностям, нежели ближних
своих,
\vs 1Cl 51:4
и охотнее на себя примут
осуждение, нежели на преданное нам доброе и святое согласие.
\vs 1Cl 51:5
И лучше человеку
признаться в своих грехах, нежели ожесточать сердце свое,
\vs 1Cl 51:6
как ожесточилось сердце
возмутившихся против раба Божьего Моисея:
\vs 1Cl 51:7
суд над ними совершился
явно, ибо они живые снизошли во ад и поглотила их смерть.
\vs 1Cl 51:8
Фараон, войско его, все
начальники Египетские, и колесницы и всадники их не по другой какой причине
потонули в море Суф и погибли, но потому, что ожесточились их несмысленные
сердца, после стольких знамений и чудес, совершенных в земле Египетской через
раба Божьего Моисея.

\vs 1Cl 52:1
Братья! Господь ни в чем
не имел нужды, и ничего ни от кого не требует, кроме исповедания Ему.
\vs 1Cl 52:2
Ибо говорит избранный
Давид: исповедуюсь ЯХВЕ, и это будет Ему приятнее, нежели молодой телец, у
которого растут рога и копыта. Пусть видят это бедные и возрадуются.
\vs 1Cl 52:3
И опять говорит: принеси
Богу жертву хвалы и воздай Вышнему молитвы твои.
\vs 1Cl 52:4
И призови Меня в день
скорби твоей, и избавлю тебя, и ты прославишь Меня.
\vs 1Cl 52:5
Ибо жертва Богу~--- дух
сокрушенный.

\vs 1Cl 53:1
Вы знаете, возлюбленные,
и хорошо знаете священные Писания, и разумеете слова Божьи.
\vs 1Cl 53:2
Итак, приведите себе на
память: когда Моисей взошел на гору и провел сорок дней и сорок ночей в посте
и смирении,
\vs 1Cl 53:3
тогда сказал ему ЯХВЕ:
Моисей, Моисей, сойди поскорей отсюда, потому что совершил преступление народ
твой, который ты вывел из земли Египетской;
\vs 1Cl 53:4
скоро они совратились с
пути, который ты заповедал им,~--- сделали себе слияния.
\vs 1Cl 53:5
И сказал ему ЯХВЕ: говорил
Я тебе раз и два, говоря: видел Я народ этот, и вот он~--- народ жестоковыйный.
\vs 1Cl 53:6
Дай Мне истребить его, и
погублю имя его под небом, а тебя сделаю народом великим и дивным и
многочисленнее этого.
\vs 1Cl 53:7
Моисей же сказал: нет,
ЯХВЕ, прости грех народу этому, или и меня истреби из книги живых.
\vs 1Cl 53:8
О, великая любовь! О,
несравненное совершенство! Раб смело говорит Господу, просит прощения народу;
в противном случае хочет и сам быть истребленным вместе с ними.

\vs 1Cl 54:1
Итак, кто из вас
благороден, кто добродушен, кто исполнен любви, тот пусть скажет:
\vs 1Cl 54:2
если из-за меня мятеж
раздор и разделение, я отхожу, иду, куда вам угодно, и исполню все, что велит
народ, только бы стадо Христово было в мире с поставленными пресвитерами.
\vs 1Cl 54:3
Кто поступит таким
образом, тот приобретет себе великую славу в Господе, и всякое место примет
его:
\vs 1Cl 54:4
ибо ЯХВЕ земля и
исполнение ее.
\vs 1Cl 54:5
Так поступали и будут
поступать все, провождающие похвальную божественную жизнь.

\vs 1Cl 55:1
Но представим примеры
народов. Многие цари и вожди во время моровой язвы, по внушениям прорицалища,
предавали себя на смерть, чтобы своею кровью спасти граждан.
\vs 1Cl 55:2
Многие удалялись из своих
городов, чтобы прекратилось возмущение в них.
\vs 1Cl 55:3
И из своих мы знаем
многих, которые предали себя в узы, дабы других освободить.
\vs 1Cl 55:4
Многие предали себя в
рабство, и, взявши за себя цену, питали других.
\vs 1Cl 55:5
Многие женщины,
укрепленные благодатью Божьей, совершили много дел мужественных.
\vs 1Cl 55:6
Блаженная Иудифь во время
осады города испросила позволения у старейшин пойти в стан иноплеменников.
\vs 1Cl 55:7
И пошла она, подвергая
себя опасности из любви к своему отечеству и народу осажденному, и Господь
предал Олоферна в руки женщины.
\vs 1Cl 55:8
Не меньшей опасности
подвергла себя совершенная по вере Есфирь, дабы избавить от предстоявшей
погибели двенадцать колен Израилевых.
\vs 1Cl 55:9
В посте и смирении она
умоляла всевидящего ЯХВЕ, Бога веков, Который, видя смирение души ее, избавил
народ, для блага которого она подвергла себя опасности.

\vs 1Cl 56:1
Будем и мы молиться о
тех, которые находятся во грехе, чтобы дарована им была кротость и смирение,
чтобы они послушались не нас, но воли Божьей.
\vs 1Cl 56:2
Ибо таким образом будет
для них плодотворно и совершенно милосердное воспоминание их пред Богом и
святыми.
\vs 1Cl 56:3
Примем наказание, на
которое никто не должен досадовать, возлюбленные!
\vs 1Cl 56:4
Взаимно делаемое нами друг
другу вразумление хорошо и весьма полезно, ибо оно не прилепляет нас к воле
Божьей.
\vs 1Cl 56:5
Ибо так говорит Святое
Слово: тяжко наказал меня ЯХВЕ, но смерти не предал меня;
\vs 1Cl 56:6
ибо кого любит ЯХВЕ, того
наказывает, и бьет всякого сына, которого принимает.
\vs 1Cl 56:7
Праведник накажет меня
милостиво и обличит меня; елей же грешного да не намастит головы моей.
\vs 1Cl 56:8
И еще говорит: блажен
человек, которого обличил ЯХВЕ; и вразумления Вседержителя не отвращайся,
\vs 1Cl 56:9
ибо Он производит скорбь и
опять восстановляет, поражает и руки Его исцеляют.
\vs 1Cl 56:10
Шесть раз избавит тебя от
бед, в седьмой же не коснется тебя зло.
\vs 1Cl 56:11
Во время голода избавит
тебя от смерти, во время войны спасет тебя от руки железа;
\vs 1Cl 56:12
от бича языка защитит
тебя, и не убоишься пред наступающими бедствиями.
\vs 1Cl 56:13
Ты посмеешься над
неправедными и беззаконными, и диких зверей не устрашишься; ибо звери дикие
будут мирны с тобою.
\vs 1Cl 56:14
Потом ты узнаешь, что дом
твой будет наслаждаться миром и не будет недостатка в помещении твоего шатра.
\vs 1Cl 56:15
Узнаешь также, что велико
семя твое и дети твои будут, как различные злаки полевые.
\vs 1Cl 56:16
Во гроб же сойдешь, как
пшеница созрелая, вовремя пожатая, или как стог гумна, вовремя свезенный.
\vs 1Cl 56:17
Видите, возлюбленные, что
наказуемые Господом~--- под Его защитою,
\vs 1Cl 56:18
ибо, как благой, Бог
наказывает для того, чтобы мы вразумились святым Его наказанием.

\vs 1Cl 57:1
Итак, вы, положившие
начало возмущению, покоритесь пресвитерам, и примите вразумление к покаянию,
преклонив колена сердца своего.
\vs 1Cl 57:2
Научитесь покорности,
отложивши тщеславную и надменную дерзость языка.
\vs 1Cl 57:3
Ибо лучше вам быть в стаде
Христа малыми и уважаемыми, нежели казаться чрезмерно высокими и лишиться
упования Его.
\vs 1Cl 57:4
Ибо так говорит
всесовершенная Премудрость: вот предложу вам слово Моего дыхания и научу вас
Моему разуму.
\vs 1Cl 57:5
Поскольку Я звала, и вы не
послушали, Я простирала слова, и вы не внимали, но отвергали Мои советы, и не
покорялись Моим обличениям:
\vs 1Cl 57:6
то Я посмеюсь вашей
погибели, и порадуюсь, когда придет вам пагуба, и когда внезапно настигнет вас
смятение, явится переворот подобно буре, или когда придет вам скорбь и
бедствие.
\vs 1Cl 57:7
Будет тогда, что призовете
Меня, а Я не послушаю вас; будут искать Меня злые и не найдут.
\vs 1Cl 57:8
Ибо они возненавидели
премудрость, страха ЯХВЕ не приняли, и не хотели внимать Моим советам, но
смеялись Моим обличениям.
\vs 1Cl 57:9
И потому они вкусят плоды
своих путей и насытятся своего нечестия.
\vs 1Cl 57:10
Ибо за то, что обидели
младенцев, они убиты будут, и суд нечестивых погубит.
\vs 1Cl 57:11
Меня же слушающий будет
обитать уверенно в надежде и упокоится без страха от всякого зла.

\vs 1Cl 58:1
Итак, будем повиноваться
всесвятому и славному имени Его, избегая прореченных Премудростью угроз
непокорным, дабы обитать уверенно в пресвятом имени величия Его.
\vs 1Cl 58:2
Примите совет наш, и не
раскаетесь. Ибо жив Бог и жив Господь Иисус Христос и Дух Святой, вера и
надежда избранных,
\vs 1Cl 58:3
так что выполнивший в
смиренномудрии, с непрестанной кротостью, Богом данные заповеди и повеления,
не раскаиваясь,~---
\vs 1Cl 58:4
сей поставится и изберется
в число спасающихся чрез Иисуса Христа, чрез Которого Ему слава во веки веков.
Аминь.

\vs 1Cl 59:1
Если же некоторые не
покорятся сказанному Им через нас,~--- пусть знают, что свяжут себя падением и
немалою опасностью.
\vs 1Cl 59:2
Мы же неповинны будем во
грехе сем и будем непрестанно молиться, прося и умоляя:
\vs 1Cl 59:3
да сохранит Творец всех
нерушимо исчисленное число избранных Своих во всем мире чрез возлюбленного
Отрока Своего, Иисуса Христа,~---
\vs 1Cl 59:4
чрез Которого Ты призвал
нас из тьмы в свет, из неведения~--- в познание славы Имени Его,
\vs 1Cl 59:5
надеяться на прежде всего
творения Имя Твое~--- ЯХВЕ.
\vs 1Cl 59:6
Отверзший очи сердца
нашего, чтобы познать Тебя, Единого Вышнего в вышних,
\vs 1Cl 59:7
Святого во святых
почивающего, смиряющего надмение гордых, разрушающего замыслы народов,
\vs 1Cl 59:8
смиренных возносящего и
смиряющего вознесенных,
\vs 1Cl 59:9
обогащающего и
разоряющего, убивающего и животворящего, Единого Благодетеля духов и Бога
всякой плоти,
\vs 1Cl 59:10
видящего бездны, Всевидца
человеческих дел, в опасности пребывающих Помощника,
\vs 1Cl 59:11
отчаявшихся Спасителя,
всякого духа Творца и Надзирателя, умножающего на земле народы и из всех
избравшего любящих Тебя чрез возлюбленного Отрока Твоего Иисуса Христа, чрез
Которого Ты нас научил, освятил, почтил.
\vs 1Cl 59:12
Просим, Владыка,
Помощником и Заступником нашим быть, пребывающих из нас в скорби спаси,
смиренных помилуй,
\vs 1Cl 59:13
падших воздвигни,
просящим явись, немощных исцели,
\vs 1Cl 59:14
заблуждающихся от народа
Твоего обрати, напитай алчущих, плененных из нас освободи, восставь немощных,
утешь малодушных:
\vs 1Cl 59:15
да познают Тебя все
народы, ибо Ты~--- Един Бог, и Иисус Христос~--- Отрок Твой, и мы люди Твои и
овцы пажити Твоей.

\vs 1Cl 60:1
Ибо Ты через совершаемое
Тобой сделал зримым вечный состав мира;
\vs 1Cl 60:2
Ты, ЯХВЕ, сотворил
вселенную, верный во всех родах и праведный в судах, чудный в силе и
великолепии,
\vs 1Cl 60:3
мудрый в творении и
разумный в основании сотворенного, благой в видимом и верный в надеющихся на
Тебя, милостивый и щедрый, оставь нам беззакония наши и неправды и грехи и
прегрешения.
\vs 1Cl 60:4
Не вмени всякого греха
рабов Твоих и рабынь, но очисти нас очищением истины Твоей и исправь стопы
наши, чтобы ходить в святости, правде и простоте сердца и творить благое и
угодное пред Тобою и пред князьями нашими.
\vs 1Cl 60:5
О, ЯХВЕ, яви лице Твое нам
во благо в мире, чтобы осениться нам рукою Твоею сильною и избавиться от
всякого зла Твоею мышцею высокою, и избавь нас от ненавидящих нас неправедно.
\vs 1Cl 60:6
Подай единомыслие и мир
нам и всем населяющим землю также, как Ты дал отцам нашим, призывающим им Тебя
свято в вере и истине,
\vs 1Cl 60:7
чтобы покорными быть
всемогущему и всесовершенному Имени Твоему, и князьям и вождям нашим на земле.

\vs 1Cl 61:1
Ты, Владыка, дал власть
царства им ради великолепия и неизреченной Твоей державы, чтобы познать нам
данную Тобою им славу и честь покоряться им, ни в чем не противиться воле
Твоей;
\vs 1Cl 61:2
подай им, ЯХВЕ, здравие,
мир, единомыслие, благостояние, дабы исполнять им Тобою данное им водительство
без соблазна.
\vs 1Cl 61:3
Ибо Ты, Владыка
пренебесный, Царь веков, дающий сынам человеческим славу и честь и власть над
сущими на земле;
\vs 1Cl 61:4
Ты, ЯХВЕ, исправь совет их
ко благу и угодному пред Тобою, да совершая в мире и кротости благочестно
данную им Тобою власть, обретут Тебя милостива.
\vs 1Cl 61:5
Единый могущий творить сие
и великое благо с нами, Тебе исповедуемся чрез Первосвященника и Ходатая душ
наших Иисуса Христа, чрез Которого Тебе слава и величие и ныне и в род родов
и во веки веков. Аминь.

\vs 1Cl 62:1
Итак, о делах приличных
богопочтению нашему и полезнейших для жизни добродетельной, желающим вести ее
благочестно и праведно, мы достаточно написали вам, мужи братия.
\vs 1Cl 62:2
Ибо мы всюду касались
того, что относится к вере, покаянию, искренней любви, воздержанию,
целомудрию и терпению,
\vs 1Cl 62:3
напоминая, что должно вам
в справедливости, истине и великодушии свято благоугождать Вседержителю Богу,
в единомыслии, незлопамятно, в любви и мире, с непрестанною кротостью,
\vs 1Cl 62:4
как и названные выше отцы
наши благоугождали, смиренномудрствуя по отношению к Отцу, Богу и Творцу, и ко
всем людям.
\vs 1Cl 62:5
И тем приятнее нам было
напомнить об этом, что мы пишем~--- как мы ясно знаем~--- мужам верным и славным,
вникающим в изречения учения Божьего.
1
Итак, справедливо,~--- следуя столь великим и многим примерам,~--- склонить выю и
занять место послушания,
\vs 1Cl 63:2
дабы успокоившись от
суетного волнения, достигли мы в истине предлежащей нам цели без всякого
позора.
\vs 1Cl 63:3
Ибо вы доставите нам
радость и веселье, если послушаетесь написанного нами чрез Святого Духа и
пресечете несправедливый гнев ревности вашей, сообразно увещанию к миру и
согласию, нами обращенному к вам в этом послании.
\vs 1Cl 63:4
Послали же мы мужей верных
и мудрых, от юности до старости обращавшихся непорочно среди нас, которые и
будут свидетелями между нами и вами.
\vs 1Cl 63:5
А поступили мы так, дабы
знали вы, что вся забота наша и была и есть~--- чтобы в скорости достигли вы
мира.

\vs 1Cl 64:1
Всевидящий Бог и Владыка
духов и Господь всякой плоти, избравший Господа Иисуса Христа и чрез Него~---
нас в народ избранный,
\vs 1Cl 64:2
да даст всякой душе,
призывающей великое и святое имя Его, веру, страх, мир, терпение, великодушие,
воздержание, чистоту и целомудрие
\vs 1Cl 64:3
в благоугождение имени его чрез первосвященника и ходатая
нашего Иисуса Христа, чрез которого ему слава,
величие, держава и честь ныне и во веки веков.
Аминь.

\vs 1Cl 65:1
Посланных от нас, Клавдия Эфеба и Валерия Витона с Фортунатом,
немедленно отпустите к нам в мире с радостью,
\vs 1Cl 65:2
чтобы они скорее известили нас о желаемом
и вожделенном для нас мире и согласии вашем,
\vs 1Cl 65:3
дабы и мы скорее могли порадоваться о вашем благоустройстве.
\vs 1Cl 65:4
Благодать Господа нашего Иисуса Христа да будет
с вами и со всеми, которые повсюду призваны Богом и чрез него:
\vs 1Cl 65:5
чрез которого ему слава,
честь, держава и величие, престол вечный,
от веков во веки веков. Аминь.

\bibbookdescr{3Co}{
  inline={Третье Послание к Коринфянам Святого Апостола Павла},
  toc={3-е Коринфянам},
  bookmark={3-е Коринфянам},
  header={3-е Коринфянам},
  abbr={3~Кор}
}
\vs 3Co 1:1
Павел, узник Иисуса Христа, братьям в Коринфе~--- радоваться!
\vs 3Co 1:2
Так как я пребываю во многих бедах,
не удивляюсь тому, что учение лукавого столь быстро множится.
\vs 3Co 1:3
Потому и придёт вскоре Господь Иисус Христос,
что отвергнут он теми, кто извращает слова его.
\vs 3Co 1:4
Передавал же я изначально вам то,
что получил от апостолов,
которые прежде меня были и всё время с
Господом Иисусом Христом пребывали:
\vs 3Co 1:5
что был Господь наш Иисус Христос Марией рождён
от семени Давидова, когда ниспослан был в неё Отцом
с небес Дух Святой,
\vs 3Co 1:6
чтобы мог он прийти в сей мир и всякую плоть
искупить плотью своею и во плоти нас из мертвых
поднять, и явил он собой пример нам в том.
\vs 3Co 1:7
И поскольку человек был сотворён его отцом,
\vs 3Co 1:8
будучи пропавшим, он был найден,
дабы воскреснуть через усыновление.
\vs 3Co 1:9
И потому послал сперва Всемогущий Бог,
сотворивший небо и землю,
пророков евреям, чтобы избавились те от грехов своих;
\vs 3Co 1:10
и потому положил он спасти дом Израилев,
что посылал он частицу Духа Христова пророкам,
которые во многие времена возвещали
безупречное почитание Бога.
\vs 3Co 1:11
Но поскольку возжелал князь неправедный сам
Богом быть, налагал он руки на них и истреблял пророков,
и потому страстями опутана всякая плоть человеческая.
\vs 3Co 1:12
Но справедлив Бог Всемогущий,
не отрекается он от творений своих,
\vs 3Co 1:13
и он послал в огне духа в Марию Галилеянку,
\vs 3Co 1:14
веровавшую всем сердцем своим, и приняла
она Духа Святого во чреве своём,
дабы Иисусу в сей мир явиться
\vs 3Co 1:15
с тем, чтобы сокрушён был лукавый тою же плотью,
через которую он приобрёл власть,
и убедился, что не Бог он вовсе.
\vs 3Co 1:16
И потому собственным телом своим спас
Иисус Христос всякую плоть и привёл
её через веру в жизнь вечную,
\vs 3Co 1:17
чтобы храм праведности мог явить он телом своим,
\vs 3Co 1:18
которым мы искуплены.

\vs 3Co 1:19
Так что они дети не праведности, но сыны зла,
отрицающие истину вопреки промыслу Божьему,
говорящие, будто земля и небо и всё,
что в них, не отцом созданы.
\vs 3Co 1:20
Самые что ни есть они сыны зла,
ибо исповедуют они проклятую веру змея.
\vs 3Co 1:21
Отвернитесь от них и учения их бегите!
\vs 3Co 1:22
Ибо вы сыны не строптивости,
но сыны церкви возлюбленной.
\vs 3Co 1:23
И сего-то ради возвещаются воскресения сроки.
\vs 3Co 1:24
А что до тех, кто говорит вам,
будто нет воскресения во плоти,
\vs 3Co 1:25
то для них-то и нет воскресения,
ибо в того не верят, кто воскрес уже.
\vs 3Co 1:26
И воистину неведомо им,
о мужи коринфские, как пшеницу сеют или семена иные,
что бросают их голыми в землю, и когда уже истлеют в ней,
поднимаются вновь они волей Божьей, обретая тело и одежду.
\vs 3Co 1:27
И не только в том поднимаются теле,
которое в землю брошено было,
но приумноженные изобильно.
\vs 3Co 1:28
А если нас не должно сравнивать с одним лишь зерном, то
\vs 3Co 1:29
ведомо вам, что Иона,
сын Амафии, не пожелавший учить в Ниневии,
был китом проглочен,
\vs 3Co 1:30
а через 3 дня и 3 ночи Бог услышал из глубин
преисподней молитву Ионы, и не повредился ни единый
член его, даже волос или ресница.
\vs 3Co 1:31
Тем более воскресит он вас,
уверовавших во Христа Иисуса,
подобно тому, как и сам он воскрес.
\vs 3Co 1:32
И если труп ожил, сброшенный сынами Израилевыми
на кости пророка Елисея, то вы тем более воскреснете,
ибо брошены вы на тело и кости и дух Господни,
и восстанете в сей же день целыми во плоти своей.
\vs 3Co 1:33
Подобно и об Илии пророке известно,
что он воскресил из мёртвых сына вдовицы;
тем более вас Господь Иисус во гласе трубы,
во мгновение ока, воскресит,
ибо он показал нам образ в своём теле.

\vs 3Co 1:34
А ежели принимаете вы и иное что-то,
то уж не отягощайте меня.
\vs 3Co 1:35
Ведь для того на руках моих оковы,
чтобы мог я Христа приобрести,
и для того язвы его на теле моем,
дабы я мог достигнуть воскресения из мёртвых.

\vs 3Co 1:36
И всякий, кто живет по заповедям,
которые он получил от блаженных пророков
и святого благовествования, получит награду и,
возстав из мёртвых, обретёт жизнь вечную.
\vs 3Co 1:37
Тот же, кто отступает от них,
пусть горит в огне, с теми вместе,
кто ведёт его такой дорогой,
\vs 3Co 1:38
потому что они безбожные люди и ехиднино порождение.
\vs 3Co 1:39
Отвергнитесь от них силой Господней,
\vs 3Co 1:40
и да пребудут с вами мир и любовь, и милость. Аминь.

\bibbookdescr{Lao}{
  inline={Послание к Лаодикийцам Святого Апостола Павла},
  toc={к Лаодикийцам},
  bookmark={к Лаодикийцам},
  header={к Лаодикийцам},
  abbr={Лао}
}
\vs Lao 1:1
Павел, апостол не от человеков и не через человека,
но Иисусом Христом,~--- братьям, которые в Лаодикии.
\vs Lao 1:2
Благодать вам и мир от Бога Отца и Господа Иисуса Христа.

\vs Lao 1:3
Благодарю Христа всякою молитвою моею,
что пребываете в нём и твёрдо стоите в делах его,
ожидая обетования в день суда.
\vs Lao 1:4
И да не погубят вас пустословия некоторых
вкравшихся в доверие, чтобы отвратить вас
от истины Евангелия, которое мною проповедуется.
\vs Lao 1:5
И ныне устраивает Бог, что те, кто без меня~--- служат
к успеху истины благовествования и творят благость
дел спасения, жизни вечной;
\vs Lao 1:6
и ныне явны узы мои, которые терплю во Христе,
которыми веселюсь и радуюсь.
\vs Lao 1:7
И сие мне есть ко спасению вечному,
которое соделывается и вашими молитвами,
и служению Святого Духа, жизнью ли, или смертью.
\vs Lao 1:8
Ибо мне воистину жизнь во Христе и умереть~--- радость.
\vs Lao 1:9
И сам он в вас сотворит милость свою,
чтобы сию любовь имели вы, и пребывали единодушны.
\vs Lao 1:10
Посему, возлюбленные, что слышали в моём присутствии,
то сохраняйте и творите в страхе Божьем,
и да будет вам жизнь вовеки;
\vs Lao 1:11
ибо действующий в вас Бог~--- Господь.
\vs Lao 1:12
И что бы вы ни делали~--- делайте неотступно.
\vs Lao 1:13
То есть, возлюбленные, радуйтесь о Христе.
И остерегайтесь нечистых в корысти.
\vs Lao 1:14
Да будут все прошения ваши явны пред Богом.
И будьте непоколебимы в духе Христовом.
\vs Lao 1:15
И что непорочно, и истинно, и честно, и праведно,
и любезно~--- творите.
\vs Lao 1:16
И что слышали и приняли,
в сердце сохраните, да будет вам мир.

\vs Lao 1:17
Приветствуйте всех братьев целованием святым.
\vs Lao 1:18
Приветствуют вас святые.
\vs Lao 1:19
Благодать Господа Иисуса с духом вашим.
\vs Lao 1:20
Распорядитесь, чтобы послание Колоссянам было прочитано вам.

\include{tex/Brn}
\bibbookdescr{1Er}{
  inline={Пастырь Ермы. Книга 1. Видения},
  toc={1-я Ермы},
  bookmark={1-я Ермы},
  header={1-я Ермы},
  abbr={1~Ермы}
}
\chhdr{Видение 1-е.}
\vs 1Er 1:1
Воспитатель мой продал в Риме одну отроковицу.
По прошествии многих лет я увидел её, узнал и полюбил как сестру.
\vs 1Er 1:2
Через некоторое время, увидев, что она купается в реке Тибр,
я подал ей руку и вывел из реки.
\vs 1Er 1:3
Глядя на ее красоту, я думал:
<<Счастлив бы я был, если бы имел жену такую же и лицом и нравом.>>
Только это, и ничего более я не подумал.
\vs 1Er 1:4
Позже шёл я с такими мыслями и прославлял творение Божье,
раздумывая, сколь величественно оно и прекрасно.
\vs 1Er 1:5
Во время прогулки я заснул, и дух подхватил меня
и понёс куда-то, через местность,
по которой человек не мог пройти.
Была она скалиста, крута и непроходима из-за вод.
\vs 1Er 1:6
Миновав её, я достиг равнины и, преклонив колена,
начал молиться Господу и исповедовать грехи свои.
\vs 1Er 1:7
И во время моей молитвы отверзлось небо
и увидел я ту женщину, которую пожелал себе.
\vs 1Er 1:8
Она приветствовала меня с неба:
<<Здравствуй, Ерма.>>
\vs 1Er 1:9
Взглянув на неё, я спросил:
<<Госпожа, что ты здесь делаешь?>>
\vs 1Er 1:10
Я взята сюда, чтобы обличить пред Господом грехи твои,~--- она ответила.
\vs 1Er 1:11
Госпожа, ужели ты меня будешь обвинять?
\vs 1Er 1:12
Нет, но выслушай слова, которые хочу сказать тебе.
Бог, живущий на небесах, сотворивший из ничего всё
сущее и умноживший ради святой Церкви своей,
гневается на тебя за то, что ты согрешил против меня.
\vs 1Er 1:13
Госпожа, если я согрешил против тебя,
то каким образом?~--- спросил я.~--- Где или когда я сказал тебе
какое-нибудь дурное слово?
\vs 1Er 1:14
Не всегда ли я уважал тебя как госпожу;
не всегда ли я почитал тебя как сестру?
Что же наговариваешь на меня столь дурное?
\vs 1Er 1:15
Тогда она, улыбаясь, ответила мне:
<<В сердце твоём возникло нечистое пожелание.
Ужели не думаешь, что для человека праведного и то порочно,
если в сердце его возникает худое пожелание?
Это~--- грех для него, и притом тяжкий.
\vs 1Er 1:16
Ибо человек праведный и помышляет праведное.
И когда он помышляет праведное и неуклонно к тому стремится,
то имеет на небесах благоволение Господа во всяком деле.
\vs 1Er 1:17
Те же, которые затаили нечистое в сердцах своих,
навлекают на себя смерть и тлен; особенно те,
которые любят настоящий век, роскошествуют
в богатстве своём и не ожидают благ будущих,~--- гибнут души их.
\vs 1Er 1:18
А это делают двоедушные, которые не имеют надежды
в Господе, не радеют о своей жизни.
\vs 1Er 1:19
Но ты молись Господу, и исцелит он грехи твои,
и всего дома твоего, и всех святых.>>

\vs 1Er 2:1
После того как произнесла она эти слова, небеса заключились.
\vs 1Er 2:2
И я, весь в скорби и страхе, сказал себе:
<<Если это вменяется мне в грех, то как могу спастись или
каким образом умолю Господа о бесчисленных грехах моих?
Какими словами упрошу Господа быть ко мне милостивым?>>
\vs 1Er 2:3
Размышляя так, увидел я вдруг перед собой большую кафедру,
словно сотворённую из в\acc{о}лны, белой как снег.
\vs 1Er 2:4
И пришла старая женщина в блестящей одежде с книгою в руке,
села одна и приветствовала меня:
<<Здравствуй, Ерма.>>
\vs 1Er 2:5
И я, в печали и слезах, ответил:
<<Здравствуй, госпожа.>>
\vs 1Er 2:6
Она спросила:
<<Что печален, Ерма, ты, который был терпелив, умерен и всегда весел?>>
\vs 1Er 2:7
Госпожа, одна прекрасная женщина, укорила меня,
будто я согрешил против неё,~--- ответил я.
\vs 1Er 2:8
И она сказала мне:
<<В сердце твоё м возникло вожделение к ней.
Это должно быть чуждо рабу Господню,
ведь для рабов Божьих даже и такой помысел составляет грех.
\vs 1Er 2:9
И сердце чистое не должно желать дурного~--- особенно твоё, Ерма;
ты избегаешь всякого преступного пожелания
и исполнен простоты и великого незлобия.

\vs 1Er 3:1
Впрочем, не ради тебя гневается на тебя Господь,
но за дом твой, который впал в нечестие перед
Господом и своими родителями.
\vs 1Er 3:2
И ты, любя детей, не вразумлял своего семейства,
но позволил им сильно развратиться.
\vs 1Er 3:3
За это и гневается на тебя Господь,
но он исправит всё, что прежде сделано худого в доме твоём.
\vs 1Er 3:4
За их грехи и беззакония ты подавлен мирскими делами.
\vs 1Er 3:5
Но милосердие Божье сжалилось над тобою и семейством твоим
и сохранило тебя в славе.
\vs 1Er 3:6
Ты только не колеблись, но будь благодушен и укрепляй свое семейство.
\vs 1Er 3:7
Как кузнец, усердно работая молотом,
совершает свой труд, так и праведное слово
ежедневное победит всякое зло.
\vs 1Er 3:8
Поэтому не переставай вразумлять детей своих,
ибо Господь знает, что они покаются от всего сердца
своего и будут написаны в Книге жизни.>>
\vs 1Er 3:9
Сказав это, она спросила меня:
<<Хочешь послушать, что я буду читать?>>
\vs 1Er 3:10
Хочу, госпожа,~--- ответил я.
\vs 1Er 3:11
Итак, слушай.
И, раскрыв книгу, она читала величественные и дивные слова,
которых не мог я удержать в памяти, ибо были они страшны,
человек не мог вынести их.
\vs 1Er 3:12
Впрочем, самые последние слова я запомнил,
так как были они краткими и отрадными для нас:
\vs 1Er 3:13
<<Вот Бог Саваоф, который невидимою силою
и великим своим разумом сотворил мир,
и славным светом своим благоукрасил тварь,
\vs 1Er 3:14
и всесильным словом своим утвердил небо,
и землю основал на водах, и всемощной силой своею создал свою
святую Церковь, которую и благословил.
\vs 1Er 3:15
Вот, он изменит небеса и горы, холмы и моря,
и всё уравняется для избранных его,
\vs 1Er 3:16
чтобы исполнить обещание, которое он дал,
с великою славою и торжеством, если они соблюдут заповеди
Божьи, полученные ими с великою верою.>>

\vs 1Er 4:1
Окончив чтение, она встала с кафедры;
и пришли четверо юношей и понесли кафедру на восток.
\vs 1Er 4:2
А она подозвала меня к себе и, коснувшись груди моей, спросила:
<<Понравилось ли тебе мое чтение?>>
\vs 1Er 4:3
Госпожа, самое последнее мне нравится,
но предыдущее страшно и жестоко.
\vs 1Er 4:4
И она сказала:
<<Эти последние слова относятся к праведным,
а первые~--- к отступникам и народам.>>
\vs 1Er 4:5
В это время явились 2 каких-то мужа,
подняли её на плечи и отправились вслед за кафедрой, на восток.
\vs 1Er 4:6
Она удалилась весёлая и на прощание произнесла:
<<Мужайся, Ерма!>>

\chhdr{Видение 2-е.}
\vs 1Er 5:1
Гуляя в окрестностях Кумских в то же примерно время,
что и в прошлом году, вспомнил я о прежнем видении,
и снова вознёс меня дух туда же, где прежде.
\vs 1Er 5:2
Достигнув того места,
я преклонил колена и начал молиться Господу
и прославлять имя его за то, что он удостоил меня
и открыл мне прежние грехи мои.
\vs 1Er 5:3
И когда восстал я от молитвы,
увидел пред собою ту старицу,
которую видел прежде: она гуляла и читала какую-то книгу.
\vs 1Er 5:4
Можешь ли возвестить это избранникам Божьим?~--- спросила она меня.
\vs 1Er 5:5
Я ответил:
<<Госпожа, так много я не могу запомнить, но дай мне книгу; я перепишу.>>
\vs 1Er 5:6
Возьми,~--- сказала она,~--- а потом возврати её мне.
\vs 1Er 5:7
Взяв книгу, я удалился в поле и списал всё буква в букву,
не понимая смысла.
\vs 1Er 5:8
И когда окончил я списывание книги,
вдруг забрали её из рук моих, но кто это был~--- не увидел я.

\vs 1Er 6:1
Спустя 15 дней, в которые я постился
и много молился Господу открылся мне смысл написанного.
\vs 1Er 6:2
Написано было следующее:
<<Дети твои, Ерма, отступили от Господа, хулили его и в великом нечестии
предали своих родителей; и прослыли они предателями родителей;
\vs 1Er 6:3
предавши их, они не исправились,
но присоединили к грехам своим распутство и нечестие скверны и
таким образом исполнили неправды свои.
\vs 1Er 6:4
Объяви эти слова всем детям своим и жене своей,
так как и она не воздержана в речах своих и тем согрешает.
\vs 1Er 6:5
Услышав же эти слова, она обуздает свой язык
и заслужит помилование.
\vs 1Er 6:6
Она образумится после того, как передашь ей слова,
которые Господь повелел открыть тебе.
\vs 1Er 6:7
Тогда отпустятся грехи, совершённые прежде,
как им, так и всем святым, если от всего сердца покаются
они и удалят сомнения из сердец своих.
\vs 1Er 6:8
Ибо славою своею поклялся Господь,
что тот из избранных его, кто и в этот предопределённый день будет
продолжать грешить, не получит спасения.
\vs 1Er 6:9
Ибо покаянию праведных положены сроки,
и определены дни покаяния для всех святых,
но народам позволено каяться до самого последнего дня.
\vs 1Er 6:10
Поэтому скажи настоятелям Церкви,
чтобы они совершали пути свои в истине,
дабы могли получить обетования со многою славою.
\vs 1Er 6:11
И вы, праведники, стойте твердо и не будьте двоедушны,
чтобы переселение ваше было со святыми ангелами.
\vs 1Er 6:12
Блаженны те, кто претерпит наступающее великое гонение
и не отречётся от своей жизни,
\vs 1Er 6:13
ибо сыном своим поклялся Господь,
что отрекающиеся от Господа губят свою жизнь.
\vs 1Er 6:14
Это относится к тем, которые отрекутся в предстоящие дни;
\vs 1Er 6:15
к тем же, которые прежде отрекались,
по великому милосердию он сделался милостивым.

\vs 1Er 7:1
А ты, Ерма, не помни неправды детей своих
и не оставляй жены своей, но позаботься о том, чтобы они
освободились от прежних грехов.
\vs 1Er 7:2
Они образумятся правым
учением, если ты не будешь держать зла на них.
\vs 1Er 7:3
Ибо злопамятство приводит
к смерти, забвение зла~--- к жизни вечной.
\vs 1Er 7:4
А ты, Ерма, потерпел большие мирские бедствия
за преступления дома твоего, поскольку не обращал на
них внимания как на не касающиеся тебя нисколько
и предался неправедным своим занятиям.
\vs 1Er 7:5
Но то, что не отступил ты от живого Бога, спасёт тебя;
простота твоя и великое воздержание спасут тебя,
если ты пребудешь в них; и всех спасут они,
кто поступает так же.
\vs 1Er 7:6
Пребывающие в невинности и простоте будут
сильны против всякого зла и обретут жизнь вечную.
\vs 1Er 7:7
Блаженны все делающие правду: они не погибнут вовек.
\vs 1Er 7:8
Но скажешь: вот приходит великое гонение.
Если тебе кажется, то опять отрекись.
\vs 1Er 7:9
Господь близок к обращающимся,
как написали в книгах Елдада и Модада,
которые в пустыне пророчествовали народу.>>

\vs 1Er 8:1
Во время сна моего,
братия, один красивый юноша явился мне и спросил:
<<Кто, ты думаешь, та старица, от которой получил ты книгу?>>
\vs 1Er 8:2
Сивилла,~--- ответил я.
\vs 1Er 8:3
Ошибаешься,~--- сказал он,~--- она не сивилла.
\vs 1Er 8:4
Кто же она, господин?
\vs 1Er 8:5
Она есть Церковь Божья.
\vs 1Er 8:6
Я спросил его, почему же она стара.
\vs 1Er 8:7
Так как,~--- объяснил он,~--- сотворена она прежде всего,
и для неё сотворён мир.
\vs 1Er 8:8
После того было мне видение в доме моём,
и пришла та старица и спросила меня,
отдал ли я уже книгу предстоятелям Церкви.
\vs 1Er 8:9
Я отвечал, что нет ещё, и она сказала:
<<Хорошо, потому что я добавлю ещё несколько слов.
\vs 1Er 8:10
Когда же исчерпаю все слова,
тогда пусть через тебя они дойдут до избранных.
\vs 1Er 8:11
Для этого ты напишешь 2 книги
и одну отдашь Клименту; а другую~--- Гранте.>>
\vs 1Er 8:12
Климент отошлёт во внешние города, ибо ему это предоставлено;
Гранта же будет назидать вдов и сирот.
\vs 1Er 8:13
А ты прочтёшь её в этом городе вместе с пресвитерами,
предстоятелями Церкви.

\chhdr{Видение 3-е.}
\vs 1Er 9:1
Было мне, братья, следующее видение.
После того как я много раз постился и молил
Господа об откровении, которое было обещано мне чрез ту старицу,
\vs 1Er 9:2
ночью явилась старица и сказала:
<<Так как ты очень просишь и желаешь знать всё,
то приходи в поле и около 6-и часов я явлюсь тебе и покажу то,
что нужно тебе видеть.>>
\vs 1Er 9:3
Я спросил её:
<<На каком месте поля?>>
\vs 1Er 9:4
Она говорит:
<<Где хочешь; место же выбери сам.>>
\vs 1Er 9:5
И я избрал место прекрасное, уединенное.
Но прежде, нежели начал я говорить и сказал ей о месте,
она говорит мне:
<<Приду, куда пожелаешь.>>
\vs 1Er 9:6
Итак, братья, заметил я часы и явился на поле,
к месту куда назначил ей прийти.
\vs 1Er 9:7
И вижу я поставленную скамью, на ней льняная подушка,
а над скамьей простёрта парусина.
\vs 1Er 9:8
Видя такие приготовления,
между тем как никого нет на месте,
я изумился, волосы у меня поднялись,
и ужас объял меня оттого, что я был один.
\vs 1Er 9:9
Но придя в себя и вспомнив славу Божью,
я ободрился и, преклонив колена,
исповедал Богу свои грехи, как всегда.
\vs 1Er 9:10
Вот, пришла старица с 6-ю юношами,
которых я прежде видел, и, ставши позади меня,
слушала, как я молился и исповедовался перед Богом.
\vs 1Er 9:11
Коснувшись меня, она сказала:
<<Перестань молиться только о грехах своих,
молись и о правде, чтобы часть из неё получил ты для дома своего.>>
\vs 1Er 9:12
Взяв меня за руку, она привела меня к скамейке
и велела тем юношам:
<<Идите и стройте.>>
\vs 1Er 9:13
Когда мы остались одни, она сказала мне:
<<Садись здесь.>>
\vs 1Er 9:14
Госпожа, пусть прежде сядут пресвитеры.
\vs 1Er 9:15
Я тебе говорю, настаивала она,~--- садись.
\vs 1Er 9:16
Я хотел было сесть по правую сторону,
но она рукою показала,
чтобы садился я по левую сторону.
\vs 1Er 9:17
Когда опечалился я, что не позволила сесть мне
по правую сторону, она проговорила:
<<Не печалься, Ерма.
Место по правую сторону принадлежит тому
кто уже угодил Богу и пострадал за имя его.>>
\vs 1Er 9:18
У тебя много недостает для того, чтобы сидеть с ними.
Но оставайся в простоте своей, как прежде, и будешь сидеть с ними,
\vs 1Er 9:19
равно как и все, кто будет творить дела их и претерпит то,
что они претерпели.>>

\vs 1Er 10:1
Я сказал ей:
<<Госпожа, я желал бы узнать, что они претерпели.>>
\vs 1Er 10:2
Слушай:
<<Лютых зверей, бичевание, темницы, кресты ради имени его.
За это принадлежит правая сторона святыни им и всякому,
кто пострадает за имя Божье, а остальным~--- левая сторона.
\vs 1Er 10:3
Но для тех и других, и для сидящих по правую сторону
и для сидящих по левую,~--- одни и те же дары обетования;
только сидящие по правую сторону имеют некоторую честь.
\vs 1Er 10:4
Ты желаешь сидеть по правую сторону с ними,
но у тебя много слабостей.
Очисти себя от своих слабостей,
и все недвоедушные должны очиститься к тому дню от своих слабостей.>>
\vs 1Er 10:5
Сказав это, она хотела удалиться, но я бросился к ногам её
и умолял её Господом, чтобы явила мне обещанное видение.
\vs 1Er 10:6
И она опять взяла меня за руку,
подняла и посадила на скамейку по левую сторону и,
поднимая какой-то блестящий жезл, спросила:
<<Видишь ли большую работу?>>
\vs 1Er 10:7
Госпожа, ничего не вижу.
\vs 1Er 10:8
Неужели не видишь против себя великой башни,
которая на водах строится из блестящих квадратных камней?
\vs 1Er 10:9
Действительно, строилась квадратная башня теми 6-ю юношами,
которые пришли с нею.
Многие тысячи других мужей приносили камни.
\vs 1Er 10:10
Некоторые доставали камни со дна,
другие из земли и подавали тем 6-и юношам,
они же принимали их и строили.
\vs 1Er 10:11
Камни, извлечённые со дна, сразу клали в здание,
потому что они были гладкие и ровные и так примыкали один к другому,
что соединения их нельзя было заметить,
и башня казалась возведенной из одного камня.
\vs 1Er 10:12
Камни же, принесённые из земли,
не все использовались для строительства.
Некоторые из них строители откладывали,
потому что были они шероховаты,
или с трещинами, или светлы и круглы
и не годились для здания башни.
\vs 1Er 10:13
А некоторые камни они раскалывали и отбрасывали далеко в сторону.
И отброшенные камни, видел я, падали на дорогу и,
не оставаясь на ней, скатывались:
\vs 1Er 10:14
одни~--- в место пустынное,
другие попадали в огонь и горели,
иные падали близ воды и не могли скатиться в воду;
хотя и стремились попасть в неё.

\vs 1Er 11:1
Показав мне это, старица хотела удалиться, но я сказал:
<<Госпожа, какая польза мне видеть, но не понимать,
что значит это строение?>>
\vs 1Er 11:2
Она отвечала мне:
<<Любопытный ты человек, если желаешь понять значение башни.>>
\vs 1Er 11:3
Действительно, госпожа,
говорю я,~--- желаю узнать и возвестить братьям, чтобы и они возрадовались,
услышав это, и прославили Господа.
\vs 1Er 11:4
Услышат многие.
И, услышавши, некоторые возрадуются, а другие восплачут;
\vs 1Er 11:5
впрочем, и последние, если, услышавши, принесут покаяние,
также будут радоваться.
\vs 1Er 11:6
Выслушай теперь объяснение башни, я открою всё,
и не докучай мне более об откровении.
\vs 1Er 11:7
Откровения эти закончились, ибо имеют предел.
А ты не перестаёшь требовать откровений, потому что настойчив.
\vs 1Er 11:8
Итак, башня, которую видишь строящейся,~--- это я, Церковь,
которая явилась тебе теперь и прежде.
\vs 1Er 11:9
Спрашивай же что хочешь о башне, и я открою тебе,
чтобы возрадовался ты со святыми.
\vs 1Er 11:10
Госпожа, если однажды сочла ты меня достойным того,
чтобы всё открыть мне, то открой,~--- просил я старицу.
\vs 1Er 11:11
Всё, что следует открыть тебе, откроется,
только бы сердце твоё было с Господом
и ты не сомневался, что бы ни увидел.
\vs 1Er 11:12
Госпожа,~--- спросил я её, почему башня построена на водах?
\vs 1Er 11:13
И прежде я говорила тебе, отвечала она,~--- что ты любопытен
и усердно изыскиваешь; ища~--- найдёшь истину:
\vs 1Er 11:14
слушай же, почему башня строится на водах:
жизнь ваша через воду спасена и спасётся.
А башня основана словом всемогущего и преславного имени
и держится невидимою силою Господа.

\vs 1Er 12:1
Я на это сказал ей:
<<Величественное и дивное дело!
А кто, госпожа, те 6 юношей, которые строят?>>
\vs 1Er 12:2
Это~--- первозданные ангелы Божии,
которым Господь вверил всё своё творение для того,
чтобы они умножали, благоустраивали и управляли его творением:
их силами и будет окончено строительство башни.>>
\vs 1Er 12:3
А кто те остальные, которые приносят камни?
\vs 1Er 12:4
И это~--- святые ангелы Господа, но первые выше.
Когда окончится строительство башни, они все вместе
будут ликовать около башни и прославлять Господа за то,
что совершилось строительство башни.
\vs 1Er 12:5
Желал бы я знать,~--- сказал я,~--- какое значение
и в чём различие камней.
\vs 1Er 12:6
И она отвечала мне:
<<Разве ты лучше всех, чтобы тебе это было открыто?
Есть более достойные, которым следовало бы открыть эти видения.
\vs 1Er 12:7
Но, чтобы прославлялось имя Божье,
тебе это открыто и ещё откроется ради тех, кто имеет сомнение в
сердце своем, будет ли это или нет.
\vs 1Er 12:8
Скажи им, что всё это истинно и что ничего нет ложного,
но всё твердо и крепко основано.

\vs 1Er 13:1
Выслушай теперь и о камнях, на которых возведено здание.
\vs 1Er 13:2
Камни квадратные и белые,
хорошо приходящиеся один к другому своими соединениями,
это суть апостолы, епископы, учителя и дьяконы,
\vs 1Er 13:3
которые ходили в святом учении Божьем,
надзирали и свято и непорочно служили
избранникам Божьим как почившие, так и живущие еще доселе,
\vs 1Er 13:4
которые всегда пребывали в мире и согласии
и слушали взаимно друг друга: потому-то они и в здании башни
хорошо примыкают один к другому.
\vs 1Er 13:5
А камни, извлекаемые из
глубины и закладываемые в здание и соприкасающиеся с прочими камнями,
вошедшими в здание, это суть те,
которые уже умерли и пострадали за имя Господа.>>
\vs 1Er 13:6
Госпожа, я желаю знать, кого означают другие камни,
которые достали из земли.
\vs 1Er 13:7
Те, которые неотделанными
кладутся в основание, означают людей, которых Бог одобрил за то, что они жили
праведно пред Господом и исполняли его заповеди.
\vs 1Er 13:8
А которые приносятся и
кладутся в само здание башни, это суть новообращенные к вере и верные.
\vs 1Er 13:9
Ангелами призываются они к
совершению добра, и потому не нашлось в них зла.
\vs 1Er 13:10
А те камни, которые откладываются в сторону возле башни?
\vs 1Er 13:11
Она ответила:
<<Это те, которые согрешили и желают покаяться;
потому они брошены невдалеке от башни,
что будут пригодны, если покаются.
\vs 1Er 13:12
Посему желающие покаяться
будут тверды в вере, если только принесут покаяние теперь,
пока строится башня.
\vs 1Er 13:13
Ибо когда строительство окончится,
то им уже не найдётся места, и они, отверженные,
только останутся лежать при башне.

\vs 1Er 14:1
Желаешь знать, кто те,
которые раскалывают и отбрасывают далеко от башни?
\vs 1Er 14:2
Желаю, госпожа.
\vs 1Er 14:3
Это суть сыны беззакония,
которые уверовали притворно и от которых не отступила неправда всякого рода;
\vs 1Er 14:4
потому они не имеют спасения,
что не годны в здание по неправедности своей,~--- они расколоты и
отброшены далеко по гневу Господа за то, что оскорбили его.
\vs 1Er 14:5
А значение прошлых камней,
которые во множестве видел ты сложенными
и не использованными в строительстве, таково.
\vs 1Er 14:6
Шероховатые суть те,
которые познали истину; но не остались в ней и не находятся в общении со
святыми, потому они и не годны.
\vs 1Er 14:7
Камни с трещинами~--- это
суть те, которые держат в сердцах вражду друг к другу; будучи вместе, они
миролюбивы, но, разойдясь, обретают в сердцах злобу.
И эта злоба~--- трещины в камнях.
\vs 1Er 14:8
Камни меньшего размера
это те люди, которые, хоть и уверовали, но имеют еще много неправды, поэтому
они коротки.
\vs 1Er 14:9
Кто же, госпожа, белые и
круглые, что тоже не идут в здание башни?
\vs 1Er 14:10
Она отвечала мне:
<<Доколе ты будешь глуп и неразумен?
Ты обо всём спрашиваешь и ничего не понимаешь.>>
\vs 1Er 14:11
Белые и круглые камни
это те, которые имеют веру, но имеют и богатства века сего; и когда придёт
гонение, то ради богатств своих и попечений они отрекутся от Господа.
\vs 1Er 14:12
Когда же будут они угодны Господу?
\vs 1Er 14:13
Когда отсечены будут богатства их, которые их утешают,
тогда они будут полезны Господу для здания.
\vs 1Er 14:14
Ибо как круглый камень,
пока не будет обсечен и не лишится некоторых своих частей, не сможет стать
квадратным, так и богатые в нынешнем веке, если не лишатся своих богатств, не
смогут быть угодными Господу.
\vs 1Er 14:15
Прежде всего ты должен знать это по себе самому:
когда ты был богат, был бесполезен; а теперь ты
полезен и годен для жизни; ты и сам был из тех камней.

\vs 1Er 15:1
Прочие же камни, которые
ты видел, были отброшены далеко от башни, катились по дороге и с дороги
скатывались в места пустынные, означают тех, которые, хотя уверовали, но, по
сомнению своему, оставили истинный путь, думая, что они могут найти лучший.
\vs 1Er 15:2
Но они обольщаются и бедствуют, ходя по путям пустынным.
\vs 1Er 15:3
Камни, упавшие в огонь и
горевшие, означают тех, которые навсегда отказались от живого Бога и которым,
по причине преступных похотей, ими творимых, уже не приходит мысль покаяться.
\vs 1Er 15:4
Кого же означают камни,
которые падали близ воды и не могли скатиться в неё?
\vs 1Er 15:5
Тех, которые слышали Слово
и желают креститься во имя Господа, когда приходит им на память святость
истины, но потом они уклоняются и опять предаются своим порочным пожеланиям.
\vs 1Er 15:6
Так она окончила объяснение башни.
\vs 1Er 15:7
Но я, будучи настойчив, спросил её:
<<Есть ли покаяние для тех камней, которые отброшены,
и будет ли им место в этой башне?>>
\vs 1Er 15:8
Она сказала:
<<Есть для них покаяние; но в этой башне не найдут они места,
а попадут в иное, низшее место,
причем когда они пострадают и исполнятся дни грехов их.
\vs 1Er 15:9
И за то они будут
переведены, что приняли Слово истинное.
\vs 1Er 15:10
И тогда избавятся они от
наказаний своих, когда содрогнутся сердцем от порочных дел,
ими сотворенных, и они покаются.
\vs 1Er 15:11
Если же они не опомнятся,
то не спасутся из-за упорства своего сердца.>>

\vs 1Er 16:1
Когда я перестал спрашивать старицу обо всём этом,
она предложила:
<<Хочешь увидеть ещё что-то?>>
\vs 1Er 16:2
И так как я очень желал
увидеть, то радость отразилась на лице моём.
\vs 1Er 16:3
Взглянув на меня, она улыбнулась
и спросила:
<<Видишь 7 женщин вокруг башни?>>
\vs 1Er 16:4
Вижу, госпожа.
\vs 1Er 16:5
Башня эта по распоряжению Господа ими поддерживается.
\vs 1Er 16:6
Слушай теперь об их действиях.
Первая из них, которая держит башню руками, называется Верою;
посредством неё спасаются избранники Божьи.
\vs 1Er 16:7
Другая же, которая препоясана и ведет себя мужественно,
называется Воздержанием, она~--- дочь Веры.
\vs 1Er 16:8
Кто последует за нею, будет блажен в своей жизни,
ибо удержится от всех худых дел и всякой злой
похоти и станет наследником вечной жизни.
\vs 1Er 16:9
Кто же другие 5, госпожа?
\vs 1Er 16:10
Дочери одна другой.
Одна называется Простотою,
другая~--- Невинностью,
3-я~--- Скромностью,
4-я~--- Знанием,
5-я~--- Любовью.
\vs 1Er 16:11
Поэтому, когда исполнишь дела матери их,
тогда сможешь и всё соблюсти.
\vs 1Er 16:12
Хотел бы я знать, госпожа, какую каждая из них имеет силу?
\vs 1Er 16:13
Слушай,~--- отвечала она, силы их одинаковы:
они связаны между собою и следуют одна за другою,
как и рождены.
\vs 1Er 16:14
От Веры рождается Воздержание,
от Воздержания Простота,
от Простоты Невинность,
от Невинности Скромность,
от Скромности Знание,
от Знания Любовь.
\vs 1Er 16:15
Действия их чисты, целомудренны и святы,
и кто послужит им и будет в силе исполнять дела их, тот
будет иметь обитель в башне со святыми Божьими.
\vs 1Er 16:16
Я спросил её о времени, не конец ли уж теперь.
\vs 1Er 16:17
Но она громко воскликнула:
<<Неразумный человек!
Неужели не видишь ты, что башня всё ещё строится?
Когда башня будет построена, тогда и будет конец.
\vs 1Er 16:18
Не спрашивай у меня ничего более.
И этого напоминания и обновления душ ваших
достаточно для тебя и для всех святых.
\vs 1Er 16:19
Не для тебя одного это открыто, но чтобы ты возвестил всем.
\vs 1Er 16:20
Итак, по прошествии 3-х дней ты, Ерма,
должен уразуметь следующие слова, которые имею сказать тебе,
чтобы ты довел их до ушей святых, чтобы, слушая и исполняя их,
очистились от своих неправд~--- и ты вместе с ними.>>

\vs 1Er 17:1
Послушайте меня, дети.
Я воспитала вас в великой простоте,
невинности и целомудрии, по милосердию Господа,
\vs 1Er 17:2
Который излил на вас
правду, чтобы вы очистились от всякого беззакония и лжи, а вы не хотите
отступиться от неправд ваших. Итак, теперь послушайте меня.
\vs 1Er 17:3
Живите в мире, заботьтесь
друг о друге, поддерживайте себя взаимно и не пользуйтесь одни творениями
Божьими, но щедро раздавайте нуждающимся.
\vs 1Er 17:4
Некоторые от многих яств
наносят вред своей плоти и истощают её.
А у других, не имеющих пропитания,
также истощается плоть оттого,
что нет в достатке пищи и гибнут тела их.
\vs 1Er 17:5
Такое невоздержание пагубно для тех,
кто имеет и не делится с нуждающимися.
Подумайте о грядущем суде.
\vs 1Er 17:6
Вы, кто превосходит других, отыскивайте алчущих,
пока ещё не окончена башня.
\vs 1Er 17:7
Ибо после, когда завершится строительство,
пожелаете благотворить, но не будет вам места.
\vs 1Er 17:8
Итак, смотрите вы, гордящиеся своими богатствами,
чтобы не восстенали терпящие нужду,
\vs 1Er 17:9
стон их взойдет к Господу
и удалены вы будете со своими сокровищами за двери башни.
\vs 1Er 17:10
Тем теперь говорю, кто
начальствует в Церкви и главенствует: не будьте подобны злодеям.
\vs 1Er 17:11
Злодеи, по крайней мере, яд свой носят в сосудах,
а вы отраву свою и яд держите в сердце;
\vs 1Er 17:12
не хотите очистить сердец
ваших и чистым сердцем сойтись в единомыслие,
чтобы иметь милость от Великого Царя.
\vs 1Er 17:13
Смотрите, дети, чтобы такие разделения ваши не лишили вас жизни.
\vs 1Er 17:14
Как хотите вы воспитывать избранников Божьих,
когда сами не имеете научения?
\vs 1Er 17:15
Поэтому вразумляйте себя взаимно и будьте в мире между собою,
чтобы и я, радостно представ пред Отцом вашим,
могла дать отчёт за вас Господу.

\vs 1Er 18:1
Когда она перестала говорить со мною,
пришли те 6 юношей, которые строили,
и понесли её к башне,
а другие 4 взяли скамью и также отнесли её в башню.
\vs 1Er 18:2
Лица сих последних я не видел,
потому что они были обращены в другую сторону.
\vs 1Er 18:3
Когда она удалялась, я просил её объяснить различные облики,
в которых являлась она мне.
\vs 1Er 18:4
Но она сказала в ответ:
<<Это пусть другой объяснит тебе.>>
\vs 1Er 18:5
А явилась она мне, братья, в 1-м видении, в прошлом году,
очень старою, сидящею на кафедре.
\vs 1Er 18:6
Во 2-м видении она имела лицо юное,
но тело и волосы старческие, и беседовала со мною стоя;
впрочем, была веселее, нежели прежде.
\vs 1Er 18:7
В 3-м же видении она вся была гораздо моложе,
с прекрасным лицом, но со старческими волосами;
она была вполне весела и сидела на скамье.
\vs 1Er 18:8
И очень я печалился, что не понятны мне такие различия,
пока не увидел во сне ночном ту старицу,
\vs 1Er 18:9
и она сказала мне:
<<Всякая молитва нуждается в смирении,
поэтому постись и получишь от Господа, чего просишь.>>
\vs 1Er 18:10
Итак, я постился 1 день,
и в ту же ночь явился мне юноша и сказал:
<<Почему ты так часто в молитве просишь откровений?
\vs 1Er 18:11
Смотри, чтобы, прося многого, не повредить тебе своей плоти.
Достаточно для тебя и этих откровений.
\vs 1Er 18:12
Сможешь ли видеть откровения ещё больше тех, которые видел?>>
\vs 1Er 18:13
Господин, я об одном только прошу,
чтобы мне было дано полное объяснение насчёт 3-х обликов той старицы.
\vs 1Er 18:14
Доколе будете вы неразумны?~--- укорил он.~--- Сомнения ваши
делают вас неразумными, потому что не
имеете в сердцах ваших устремления к Господу.
\vs 1Er 18:15
Я отвечал ему:
<<От тебя мы узнаем об этом вернее.>>

\vs 1Er 19:1
Слушай,~--- сказал он,~--- об обликах, которые тебя интересуют.
Почему в 1-м видении явилась тебе старица, сидящая на кафедре?
\vs 1Er 19:2
Потому что дух ваш обветшал и ослабел
и не имеет силы от грехов ваших и сомнений сердца.
\vs 1Er 19:3
Ибо как старцы, не имеющие
надежды на обновление, ничего другого не желают,
кроме успокоения на ложе,
\vs 1Er 19:4
так и вы, обременённые житейскими делами,
впали в беспечность и не возложили попечений своих на
Господа; одряхлел ваш разум и состарились вы в печалях ваших.
\vs 1Er 19:5
Я желаю узнать, господин, почему она сидела на кафедре?
\vs 1Er 19:6
Потому,~--- отвечал он мне,
что всякий немощный сидит на седалище по причине своей слабости,
чтобы имело поддержку немощное тело его.
Вот тебе смысл 1-го явления.

\vs 1Er 20:1
Во 2-м видении ты видел её стоящей,
с помолодевшим лицом и более веселою, нежели прежде;
а тело и волосы были у неё старческие.
\vs 1Er 20:2
Выслушай и эту притчу.
Когда кто сильно состарится и отчается в самом себе
из-за своей слабости и бедности,
то ничего другого не ожидает,
только последнего дня своей жизни.
\vs 1Er 20:3
Но вдруг получает он наследство.
Узнав об этом, он вскакивает повеселевший, к нему возвращаются
силы, обновляется дух его, который одряхлел от прежних дел;
он уже не лежит, но, восставши, мужественно действует.
\vs 1Er 20:4
То же произошло и с вами, когда услышали вы об откровении,
которое Бог сообщил вам.
\vs 1Er 20:5
Господь сжалился над вами и обновил дух ваш~--- и вы отложили
свои немощи, пришло к вам мужество, вы укрепились в вере,
и Господь, видя вашу верность, возрадовался.
\vs 1Er 20:6
Поэтому показал он вам строение башни~--- и иное покажет,
если будет между вами чистосердечный мир.

\vs 1Er 21:1
В 3-м видении ты видел, что она ещё моложе, прекрасна,
весела и лицо её светло.
\vs 1Er 21:2
Сравнить это с тем можно, как если бы к печалящемуся
человеку пришёл добрый вестник
\vs 1Er 21:3
тотчас он забыл бы прежнюю скорбь,
ни о чём другом не думал, как об услышанной им вести;
ободряется и обновляется дух его от радости.
\vs 1Er 21:4
Так точно и вы получили обновление душ ваших,
узнав такие блага.
\vs 1Er 21:5
А что ты видел её сидящею на скамье~--- это означает
твёрдое положение, так как скамейка имеет 4 ножки и стоит твёрдо,
да и мир поддерживается 4-мя стихиями.
\vs 1Er 21:6
Поэтому и те, которые
всецело, от всего сердца покаются, помолодеют и окрепнут.
\vs 1Er 21:7
Теперь имеешь ты полное объяснение.
Не проси более никаких откровений.
Если же нужно будет, то откроется тебе.

\chhdr{Видение 4-е.}
\vs 1Er 22:1
Спустя 20 дней было мне, братья,
видение гонения, которое должно случиться.
\vs 1Er 22:2
Шел я по полю при дороге Шампанской,
от большой дороги до поля почти 10 стадиев:
через это место путь бывает редко.
\vs 1Er 22:3
Гуляя один, я молил
Господа, чтобы он подтвердил те откровения,
которые явил мне чрез святую свою Церковь,
укрепил меня и дал покаяние всем рабам своим,
которые соблазнились;
\vs 1Er 22:4
дабы прославлялось великое и досточтимое имя его за то,
что удостоил показать мне чудеса свои.
\vs 1Er 22:5
И в то время когда я прославлял и благодарил его, голос был мне:
<<Не сомневайся, Ерма!>>
\vs 1Er 22:6
Стал я думать:
<<Что мне сомневаться, когда я так укреплён Господом и видел дивные дела?>>
\vs 1Er 22:7
Пройдя немного, братья, вдруг увидел я пыль,
поднимающуюся до неба, и подумал, что это идёт скот,
пыль поднимая.
\vs 1Er 22:8
Расстояние между тучей пыли и мною было около стадия.
\vs 1Er 22:9
Между тем пыль поднималась гуще и гуще,
так что мне стало это казаться чем-то сверхъестественным.
\vs 1Er 22:10
Несколько проглянуло солнце, и увидел я огромного зверя
наподобие дракона, из уст которого выходила огненная саранча.
\vs 1Er 22:11
В длину это животное имело около 100 футов,
а голова была подобна глиняному сосуду.
\vs 1Er 22:12
И начал я плакать и молить Господа, чтобы спас меня от него.
\vs 1Er 22:13
Потом вспомнил я слова, которые слышал:
<<Не сомневайся, Ерма.>>
\vs 1Er 22:14
Итак, братья, облёкшись верою в Бога и вспомнив явленные
мне им великие дела, я смело пошёл к зверю.
\vs 1Er 22:15
Зверь же метался с такою яростью и был так свиреп и силён,
что, казалось, при нападении мог бы разрушить город.
\vs 1Er 22:16
Я приблизился к нему,
и это огромное устрашающее животное мирно
растянулось на земле, высунув язык.
\vs 1Er 22:17
Я прошёл мимо него, и оно не пошевелилось.
\vs 1Er 22:18
Голова этого зверя была 4-х цветов:
чёрного, потом красного, или кровавого,
далее золотистого и, наконец, белого.

\vs 1Er 23:1
После того как я прошёл мимо зверя и удалился почти на 30 футов,
встречается мне разукрашенная дева,
словно выходящая из брачного чертога,
\vs 1Er 23:2
в белых башмаках, покрытая белыми одеждами до самого чела;
митра была ее покрывалом, волосы у ней были белые.
\vs 1Er 23:3
По прежним видениям я догадался,
что это Церковь, и обрадовался.
\vs 1Er 23:4
Она приветствовала меня:
<<Здравствуй, человек.>>
И я ответил ей также приветствием.
\vs 1Er 23:5
Она спросила:
<<Ничто не встретилось тебе, человек?>>
\vs 1Er 23:6
Госпожа, мне встретилось такое животное,
которое могло бы истребить народы, но силою Бога
и по великому его милосердию я спасся от него.
\vs 1Er 23:7
Счастливо спасся ты, сказала она,~--- потому,
что заботу свою ты возложил на Господа и ему открыл своё сердце,
\vs 1Er 23:8
веруя, что никем другим не можешь быть спасён,
кроме его великого и преславного имени.
\vs 1Er 23:9
За это Господь послал ангела своего,
поставленного над зверями, которому имя Егрин,
и он заградил пасть его, чтобы не пожрал тебя.
\vs 1Er 23:10
Ты избежал великого бедствия по вере твоей,
так как ты не усомнился при виде такого зверя.
\vs 1Er 23:11
Итак, пойди и возвести избранникам Бога
о великих делах его и скажи им,
что зверь этот есть образ грядущей великой напасти.
\vs 1Er 23:12
Поэтому, если приготовите себя и от всего сердца
покаетесь перед Господом, то можете избежать её,
\vs 1Er 23:13
если сердце ваше будет чисто и непорочно
и в остальные дни жизни вашей
будете безукоризненно служить Богу.
\vs 1Er 23:14
Возложите на Господа печали ваши, и он сам уврачует их.
\vs 1Er 23:15
Вы, двоедушные, веруйте в Бога, что он всё может~--- и
отвратить от вас гнев свой, и послать бичи на двоедушных.
\vs 1Er 23:16
Горе тем, которые услышат эти слова и презрят их,
лучше было им не родиться.

\vs 1Er 24:1
Я спросил её о 4-х цветах, которые были у зверя на голове.
\vs 1Er 24:2
Она сказала на это:
<<Опять ты любопытствуешь, спрашивая о вещах такого рода.>>
\vs 1Er 24:3
Да, госпожа, объясни мне, что они означают.
\vs 1Er 24:4
Слушай же,~--- отвечала она.
Чёрный цвет означает мир, в котором вы живете.
\vs 1Er 24:5
Огненный и кровавый~--- то,
что этому миру должно погибнуть посредством крови и огня.
\vs 1Er 24:6
А золотистая часть~--- это все вы, избегающие этого мира.
\vs 1Er 24:7
Ибо как золото испытывается посредством огня
и становится годным, так испытываетесь и вы,
живущие среди них.
\vs 1Er 24:8
Те, которые сохранят твёрдость и будут искушены ими, очистятся.
\vs 1Er 24:9
И как золото оставляет в огне примеси свои,
так и вы оставите всякую скорбь и печаль, очиститесь и
будете годны для здания башни.
\vs 1Er 24:10
Белая же часть означает будущий век,
в котором станут жить избранники Божьи,
\vs 1Er 24:11
потому что непорочны и чисты будут те,
которые избраны Богом в жизнь вечную.
\vs 1Er 24:12
Итак, не переставай доносить это до слуха святых.
\vs 1Er 24:13
Имеете вы и образ грядущего великого бедствия.
Если захотите вы, оно будет не страшно для вас:
помните заповеданное вам.
\vs 1Er 24:14
Сказав это, она удалилась, и не видел я, куда она ушла.
\vs 1Er 24:15
Раздался шум, и я в страхе бросился назад,
думая, что приближается тот зверь \ldots

\bibbookdescr{2Er}{
  inline={Пастырь Ермы. Книга 2. Заповеди},
  toc={2-я Ермы},
  bookmark={2-я Ермы},
  header={2-я Ермы},
  abbr={2~Ермы}
}
\chhdr{Видение 5-е.}
\vs 2Er 1:1
Когда я
помолился дома и сидел на ложе, вошел ко мне человек почтенного вида, в
пастушеской одежде: на нем был белый плащ, сума за плечами и посох в руке.
\vs 2Er 1:2
Он приветствовал меня, и я
ответил ему также приветствием. Тотчас же он сел подле меня и говорит:
\vs 2Er 1:3
Я послан от
достопоклоняемого ангела, чтобы жить с тобою остальные дни твоей жизни.
\vs 2Er 1:4
Мне показалось, что он
искушает меня, и сказал я ему: кто же ты? Я знаю того, кому препоручен я.
\vs 2Er 1:5
Не узнаешь меня? Нет.
\vs 2Er 1:6
Я тот самый пастырь,
которому препоручен ты.
\vs 2Er 1:7
Пока он говорил, вид его
изменился, и я узнал, что это тот, которому я препоручен.
\vs 2Er 1:8
Тотчас я смутился, объял
меня страх, и весь я разрывался от скорби, что отвечал ему так лукаво и
неразумно.
\vs 2Er 1:9
Он же сказал мне: не
смущайся, но укрепись заповедями, которые услышишь от меня.
\vs 2Er 1:10
Ибо я послан для того,
чтобы снова показать тебе все, что видел ты прежде, и особенно то, что полезно
для вас.
\vs 2Er 1:11
Итак, я приказываю тебе
сперва записать заповеди мои и притчи, чтобы перечитывать их время от времени,
так удобнее будет тебе выполнять их.
\vs 2Er 1:12
Поэтому я записал
заповеди и притчи так, как повелел он мне.
\vs 2Er 1:13
Если, услышав их, вы
будете поступать по ним и исполните их с чистым сердцем, то получите от
Господа то, что обещал Он вам.
\vs 2Er 1:14
Если же, услышав их, не
покаетесь, но обратитесь к грехам вашим, то воспримите от Господа наказание.
\vs 2Er 1:15
Все это повелел мне
записать этот пастырь, ангел покаяния.
 
\chhdr{Заповедь 1-я.}
\vs 2Er 2:1
Прежде
всего веруй, что един есть Бог, все сотворивший и совершивший, приведший все
не из чего в бытие.
\vs 2Er 2:2
Он все объемлет, Сам
будучи необъятен, и не может быть ни словом определен, ни умом постигнут.
\vs 2Er 2:3
Итак, веруй в Него, бойся
Его и, боясь, соблюдай воздержание.
\vs 2Er 2:4
Храни это и отринешь от
себя всякую похоть и беззаконие, и облечешься во всякую добродетель и правду и
будешь жить с Богом, если сохранишь эту заповедь.

\chhdr{Заповедь 2-я.}
\vs 2Er 3:1
Пастырь
сказал мне: имей простоту и будь незлобив, будь как дитя, которое не знает
лукавства, губящего жизнь людей.
\vs 2Er 3:2
Ни о ком не говори худо и
не стремись слушать того, кто говорит худо.
\vs 2Er 3:3
Если же будешь слушать, то
будешь причастен греху злословящего; веря ему, ты будешь подобен ему, потому
что поверил злословящему на брата твоего.
\vs 2Er 3:4
Гибельно злословие: это~--- дух беспокойный,
который никогда не находится в мире, но всегда живёт в несогласии.
\vs 2Er 3:5
Удерживайся от него и
всегда имей мир с братом твоим.
\vs 2Er 3:6
Облекись
благопристойностью, в которой нет ничего оскорбительного, но все ровно и
приятно.
\vs 2Er 3:7
Делай добро и от плода
трудов твоих, который дарует тебе Бог, давай всем бедным просто, нимало не
сомневаясь, кому даешь.
\vs 2Er 3:8
Давай всем, ибо Бог хочет,
чтобы всем досталось от Его даров.
\vs 2Er 3:9
Берущие дадут отчет Богу
почему и на что брали. Берущие по нужде не будут осуждены, а берущие притворно
подвергнутся суду.
\vs 2Er 3:10
Дающий же не будет
виноват: ибо он исполнил служение, назначенное Богом, не разбирая, кому дать и
кому не давать, и исполнил с похвалою пред Богом.
\vs 2Er 3:11
Итак, соблюдай эту
заповедь, как я сказал тебе, чтобы покаяние твоё и семейства твоего было в
простоте и сердце твое явилось чистым и непорочным пред Богом.

\chhdr{Заповедь 3-я.}
\vs 2Er 4:1
Также
сказал он мне: люби истину, и пусть исходит из уст твоих всякая истина,
\vs 2Er 4:2
чтобы дух, который Господь
поселил в этом теле, предстал истинным пред всеми людьми, и чтобы прославлялся
Господь, который дал тебе дух,
\vs 2Er 4:3
потому что Бог истинен во
всяком слове и никакой лжи нет в Нем.
\vs 2Er 4:4
И те, которые лгут,
отвергают Бога и не возвращают Ему залога, который получили; а они получили от
Него дух нелживый.
\vs 2Er 4:5
Если они возвращают его
лживым, то бесчестят заповедь Господа и становятся похитителями.
\vs 2Er 4:6
Услышав это, я горько
заплакал. Видя скорбь мою, он спросил: о чем ты плачешь?
\vs 2Er 4:7
Не знаю, господин, могу ли
спастись я. Почему?
\vs 2Er 4:8
Потому, что никогда в
жизни, господин, не произнес я слова правдивого, но всегда говорил коварно и
выдавал пред всеми ложь за истину; и никто не прекословил мне, потому что
доверяли моему слову.
\vs 2Er 4:9
Как же я могу жить, когда
поступал таким образом?
\vs 2Er 4:10
Он отвечал: ты
рассуждаешь хорошо и верно, ибо следовало тебе, как рабу Божьему, ходить в
истине, и не соединять лукавой совести с духом истины, и не оскорблять Духа
Божьего Святого и истинного.
\vs 2Er 4:11
И я сказал ему: никогда,
господин, я не слышал таких слов.
\vs 2Er 4:12
Услышав сейчас, впредь
соблюдай их и старайся, чтобы и те лживые слова, которые прежде говорил ты,
стали верными от последующих речей твоих, если они окажутся правдивыми.
\vs 2Er 4:13
Ибо и те могут сделаться
верными, если отныне будешь говорить правду; и если будешь соблюдать истину,
можешь получить себе жизнь.
\vs 2Er 4:14
И всякий, кто только
послушается этой заповеди, будет исполнять ее и удаляться от лжи, будет жить
с Богом.

\chhdr{Заповедь 4-я.}
\vs 2Er 5:1
Заповедую тебе, говорил пастырь, соблюдать целомудрие.
\vs 2Er 5:2
И да не войдет тебе в
сердце помысел о чужой жене, или о любодеянии, или о каком-либо подобном
дурном деле, ибо все это великий грех.
\vs 2Er 5:3
А ты помни о Господе во
все часы и никогда не согрешишь.
\vs 2Er 5:4
Если какой низкий помысел
взойдет на твое сердце, то совершишь великий грех; и кто творит такое
преступное дело, обрекает себя на смерть.
\vs 2Er 5:5
Итак, смотри ты,
воздерживайся от таких помыслов.
\vs 2Er 5:6
Ибо где обитает
целомудрие, там никогда не должен возникать худой помысел в сердце человека
праведного.
\vs 2Er 5:7
И попросил я: позволь мне,
господин, спросить тебя немного. Спрашивай.
\vs 2Er 5:8
Если, господин, сказал
я, муж имеет жену верную в Господе и заметит ее в прелюбодеянии, то будет ли
он грешен, живя с нею?
\vs 2Er 5:9
И ответил он мне: доколе
муж не знает греха своей жены, он не грешит, если живет с нею.
\vs 2Er 5:10
Если же узнает муж о
грехе ее, а она не покается в своем прелюбодеянии, то муж согрешит, живя с
нею, и сделается участником в ее прелюбодеянии.
\vs 2Er 5:11
Что же делать, спросил
я, если жена будет оставаться в своем пороке?
\vs 2Er 5:12
Пусть муж отпустит ее и
останется один. Если же, отпустивши свою жену, возьмет другую, то и сам примет
грех прелюбодеяния.
\vs 2Er 5:13
Что же, господин, если
жена отпущенная покается и пожелает возвратиться к мужу своему, то должна ли
она быть принята мужем?
\vs 2Er 5:14
Если не примет ее муж, он
совершит грех великий, он ответил мне.
\vs 2Er 5:15
Должно принимать
грешницу, которая раскаивается, но не много раз. Ибо для рабов Божьих покаяние
положено одно.
\vs 2Er 5:16
Поэтому ради раскаяния не
должен муж, отпустив жену свою, брать себе другую. Так же делать надлежит и
жене.
\vs 2Er 5:17
Но прелюбодейство не
только в осквернении плоти своей: прелюбодействует и тот, кто поступает
подобно народам.
\vs 2Er 5:18
Избегай общения с тем,
кто совершает такие дела и не кается, иначе и ты будешь причастен греху его.
\vs 2Er 5:19
Итак, заповедуется вам,
чтобы вы оставались одинокими как муж, так и жена, ибо в этом случае может
иметь место покаяние.
\vs 2Er 5:20
Но я не даю повода к тому
чтобы так делалось: пусть не грешит более тот, кто уже согрешил.
\vs 2Er 5:21
Что касается прежних
грехов его, то есть Бог, который может дать исцеление, ибо Он имеет власть над
всем.

\vs 2Er 6:1
И я опять просил его:
поскольку Господь удостоил меня того, чтобы ты всегда жил со мною, то дозволь
сказать мне еще несколько слов,
\vs 2Er 6:2
потому что я ничего не
понимаю и сердце мое омрачено прежними делами моими. Вразуми меня, так как я
совершенно ничего не смыслю.
\vs 2Er 6:3
И он в ответ сказал мне: я
приставник покаяния и всем кающимся даю разум. Или самое покаяние, ты думаешь,
не есть великий разум?
\vs 2Er 6:4
Грешник кающийся уразумел,
что он согрешил пред Господом, он осудил всем сердцем содеянные им дела и,
раскаявшись, более уже не делает зла, но совершает добро, и смиряет душу и
мучит ее за то, что согрешила. Итак, понимаешь, что покаяние есть великий
разум?
\vs 2Er 6:5
Потому-то, господин, я и
расспрашиваю тебя подробно обо всем, что я грешник и желаю узнать, что мне
делать, чтобы жить, ибо грехи мои многочисленны и разнообразны.
\vs 2Er 6:6
Ты будешь жив, сказал
он, если сохранишь мои заповеди и будешь поступать по ним; и всякий, кто
только услышит и исполнит эти заповеди, будет жить с Богом.

\vs 2Er 7:1
Я сказал ему: господин, я
слышал от некоторых учителей, что нет иного покаяния, кроме того, когда сходим
в воду и получаем отпущение прежних грехов наших.
\vs 2Er 7:2
Справедливо ты слышал. Ибо
получившему отпущение грехов не должно более грешить, но жить в чистоте.
\vs 2Er 7:3
И так как ты обо всем
расспрашиваешь, объясню тебе это, не давая повода к заблуждению тем, которые
собираются уверовать или только что уверовали в Господа.
\vs 2Er 7:4
Они не имеют покаяния во
грехах, но имеют отпущение прежних грехов своих.
\vs 2Er 7:5
Тем же, которые призваны
прежде, положил Господь покаяние, ибо Он сердцеведец, провидящий все, знал
слабость людей и великое коварство дьявола, который будет сеять вред и злобу
среди рабов Божьих.
\vs 2Er 7:6
Поэтому милосердный
Господь сжалился над своим созданием и положил покаяние, над которым и дана
мне власть.
\vs 2Er 7:7
Итак, я говорю тебе, после
этого великого и святого призвания, если кто, будучи искушен дьяволом,
согрешил, пусть покается.
\vs 2Er 7:8
Если же часто он будет
грешить и творить покаяние, не принесет ему покаяние пользы, ибо с трудом он
будет жить с Богом.
\vs 2Er 7:9
И я сказал: господин, я
обновился, когда услышал об этом так обстоятельно. Ибо я знаю, что спасусь,
если еще не присовокуплю ничего к грехам своим.
\vs 2Er 7:10
Спасешься, ответил он,
ты и все, которые сделают то же.

\vs 2Er 8:1
И опять я попросил его:
господин, так как ты терпеливо меня выслушиваешь, объясни мне еще вот что.
\vs 2Er 8:2
Если муж или жена умрет и
один из них вступит в брак согрешает ли вступающий в брак?
\vs 2Er 8:3
Не согрешает, но если
останется сам по себе, то приобретет себе большую славу у Господа.
\vs 2Er 8:4
Поэтому храни чистоту и
целомудрие и будешь жить с Богом.
\vs 2Er 8:5
То, что я говорю и
собираюсь сказать тебе после, соблюдай с этого самого дня, ибо ты поручен мне
и живу в твоем доме.
\vs 2Er 8:6
И прежним грехам твоим
будет отпущение, если сохранишь мои заповеди; и все, кто сохранит их и будет
ходить в чистоте, получит отпущение.

\chhdr{Заповедь 5-я.}
\vs 2Er 9:1
Будь
великодушен и терпелив, сказал пастырь, и будешь господствовать над всеми
злыми делами и сотворишь всякую правду.
\vs 2Er 9:2
Если будешь великодушен,
то Дух Святой, в тебе обитающий, останется чист и не омрачится от какого-либо
злого духа, но, ликуя, расширится, и вместе с сосудом, в котором обитает,
будет радостно служить Господу.
\vs 2Er 9:3
Если же найдет какой-либо
гнев, то Дух Святой, сущий в тебе, тотчас же будет стеснен и постарается
удалиться,
\vs 2Er 9:4
ибо подавляется злым духом
и, оскорбляемый гневом, не имеет возможности служить Господу, как желает.
\vs 2Er 9:5
Поэтому, когда оба духа
обитают вместе, плохо бывает человеку.
\vs 2Er 9:6
Так, если взять немножко
полыни и положить в сосуд с медом, не весь ли мед испортится?
\vs 2Er 9:7
И столько меда пропадает
от незначительного количества полыни, теряет сладость и уже не имеет
приятности для своего владельца, потому что делается горьким и негодным к
употреблению. Но если в мед не класть полынь, он останется сладок.
\vs 2Er 9:8
Сам видишь, великодушие
слаще меда, и оно угодно Богу и Господь обитает в нем, а гнев горек и
неугоден.
\vs 2Er 9:9
Итак, если к великодушию
примешивается гнев, то дух возмущается, и неприятна Богу молитва его.
\vs 2Er 9:10
И я сказал ему: желал бы
я узнать, господин, действие гнева, чтобы уберечь себя от него.
\vs 2Er 9:11
Если ты и твои домочадцы
не будете удерживаться от него, то потеряете всякую надежду спасения.
\vs 2Er 9:12
Но воздерживайся от
гнева, ибо я с тобою; и от него воздержатся все, которые покаются от всего
сердца своего, ибо я буду с ними и сохраню их.
\vs 2Er 9:13
Все такие принимаются
святейшим ангелом в число праведных.

\vs 2Er 10:1
Послушай теперь и о
действии гнева, как он вреден и как губит рабов Божьих и отвращает их от
правды.
\vs 2Er 10:2
Он не может вредить людям,
исполненным веры, потому что с ними пребывает сила Божья; совращает же
сомневающихся и не имеющих ее.
\vs 2Er 10:3
Как скоро он увидит таких
людей спокойными проникает в сердце их, и муж или жена сердятся друг на
друга по каким-нибудь житейским делам:
\vs 2Er 10:4
или из-за пищи, или
пустого слова, или какого приятеля, или долга, или из-за подобных мелочных
вещей. Все это глупо, пусто и неприлично рабам Божьим.
\vs 2Er 10:5
Но великодушие твердо и
мужественно, имеет крепкую силу и пребывает в великой широте, весело и
беззаботно радуясь, и прославляет Господа во всякое время чуждое всякой
горечи, всегда мирное и кроткое.
\vs 2Er 10:6
Это великодушие живет с
имеющими полную веру. А гнев безрассуден, пуст и легкомыслен.
\vs 2Er 10:7
От безрассудства рождается
огорчение, от огорчения раздражение, от раздражения гнев, от гнева же
неистовство.
\vs 2Er 10:8
Неистовство, происшедшее
от стольких зол, есть великий и неискупимый грех.
\vs 2Er 10:9
И когда все это находится
в одном сосуде, где обитает и Дух Святой, то сосуд не вмещает их в себе, но
переполняется:
\vs 2Er 10:10
добрый дух не может жить
вместе со злым духом, а удаляется от такого человека и ищет себе пристанища в
кротости и тишине.
\vs 2Er 10:11
Когда он отступит от
человека, в котором обитал, человек, исполненный духами злыми, делается чужд
Святого Духа и закрыт для благой мысли. Так бывает со всеми гневливыми.
\vs 2Er 10:12
Итак, ты удаляйся гнева,
но облекись в великодушие и противься всякому огорчению и будешь в чистоте и
святости, любезной Богу.
\vs 2Er 10:13
Смотри поэтому, чтобы
как-нибудь не пренебречь тебе этой заповедью, ибо если соблюдешь эту заповедь,
то можешь исполнить и прочие мои заповеди, которые хочу тебе преподать.
\vs 2Er 10:14
Итак, теперь утверждайся
в этих заповедях, чтобы тебе жить с Богом, равно и все, кто соблюдет их,
будут жить с Богом.

\chhdr{Заповедь 6-я.}
\vs 2Er 11:1
Я повелел тебе, сказал пастырь, в первой заповеди, чтобы хранил ты веру,
страх и воздержание.
\vs 2Er 11:2
Да, господин, подтвердил
я.
\vs 2Er 11:3
А теперь я хочу объяснить
тебе силу этих добродетелей, чтобы знал ты, как каждая из них действует и
какую имеет власть.
\vs 2Er 11:4
Двоякого рода их действия
и состоят в праведном и неправедном.
\vs 2Er 11:5
Ты веруй праведному,
неправедному нисколько не веруй.
\vs 2Er 11:6
Ибо правда имеет путь
прямой, а неправда кривой.
\vs 2Er 11:7
Но ты иди путем прямым, а
кривой оставь.
\vs 2Er 11:8
Кривой путь неровен, имеет
множество преткновений, скалист и тернист и ведет к погибели идущих по нему.
\vs 2Er 11:9
А те, которые следуют
прямому пути, идут ровно и без препятствий, потому что он не скалист и не
тернист. Итак, видишь, что лучше идти этим путем.
\vs 2Er 11:10
Я сказал: мне нравится
идти этим путем.
\vs 2Er 11:11
И пойдешь ты, равно как
пойдут по нему и все, которые от всего сердца обратятся к Господу.

\vs 2Er 12:1
Послушай теперь,
продолжал он, о вере. Два ангела с человеком: один добрый, а другой злой.
\vs 2Er 12:2
Я спросил его: каким
образом, господин, я могу распознать их, если оба ангела живут со мною?
\vs 2Er 12:3
Слушай и разумей. Добрый
ангел тих и скромен, кроток и мирен.
\vs 2Er 12:4
Поэтому войдя в твое
сердце, постоянно будет внушать он тебе справедливость, целомудрие, чистоту
ласковость, снисходительность, любовь и благочестие.
\vs 2Er 12:5
Когда все это вселится в
твое сердце, знай, что добрый ангел с тобою: верь этому ангелу и следуй делам
его.
\vs 2Er 12:6
Послушай и о действиях
ангела злого. Прежде всего он злобен, гневлив и безрассуден, и действия его
злы и развращают рабов Божьих.
\vs 2Er 12:7
Поэтому когда войдет он в
твое сердце, из действий его разумей, что это ангел злой.
\vs 2Er 12:8
Каким образом, спросил
я, узнаю его, господин?
\vs 2Er 12:9
Слушай. Когда овладеют
тобой гнев или досада, знай, что он в тебе;
\vs 2Er 12:10
также когда возникнет в
сердце твоем пожелание разных и роскошных яств, и напитков, и чужих жен, то
вселяются в него гордость, хвастовство, надменность и тому подобное тогда
знай, что с тобою злой ангел.
\vs 2Er 12:11
Поэтому ты, зная его
дела, избегай и не верь ему: дела его злы и не свойственны рабам Божьим.
\vs 2Er 12:12
Таковы действия того и
другого ангела. Разумей их, верь ангелу доброму и удаляйся от ангела злого,
потому что внушение его во всяком деле злое.
\vs 2Er 12:13
Даже если в сердце
человека верующего войдет помысел злого ангела, то он непременно согрешит.
\vs 2Er 12:14
Если же злые люди откроют
сердце свое делам ангела доброго, то обязательно он сделает что-нибудь доброе.
\vs 2Er 12:15
Итак, видишь, что хорошо
следовать ангелу доброму. Если станешь повиноваться ему и творить его дела, то
будешь жить с Богом;
\vs 2Er 12:16
равно как и все, которые
будут следовать его делам, будут жить с Богом.

\chhdr{Заповедь 7-я.}
\vs 2Er 13:1
Бойся, говорил пастырь, Господа и соблюдай заповеди его,
\vs 2Er 13:2
ибо, соблюдая заповеди
Божьи, будешь тверд в любом деле и преуспеешь в нем.
\vs 2Er 13:3
Боясь Господа, будешь все
делать хорошо. Вот страх, которым должно страшиться, чтобы спастись.
\vs 2Er 13:4
Дьявола же не бойся: боясь
Господа, ты будешь господствовать над дьяволом, потому что в нем нет никакой
силы.
\vs 2Er 13:5
А в ком нет силы, того не
должно бояться.
\vs 2Er 13:6
В ком есть превосходная
сила, того и должно бояться.
\vs 2Er 13:7
Ибо всякий, имеющий силу,
внушает страх; а кто не имеет силы, всеми презирается.
\vs 2Er 13:8
Бойся, впрочем, дел
дьявола, потому что они злы; боясь Господа, ты не совершишь дел дьявола, но
удержишься от них.
\vs 2Er 13:9
Двоякий есть страх. Если
ты захотел сделать злое, то бойся Бога и не сделаешь этого.
\vs 2Er 13:10
Равно если бы захотел ты
сделать доброе, то опять бойся Бога и сделаешь его.
\vs 2Er 13:11
Подлинно, страх Божий
велик, силен и славен.
\vs 2Er 13:12
Итак, бойся Бога, и
будешь жить. И все те, которые будут бояться Его, соблюдая Его заповеди, будут
жить с Богом;
\vs 2Er 13:13
а которые не соблюдают
Его заповедей, в тех нет жизни.

\chhdr{Заповедь 8-я.}
\vs 2Er 14:1
Я сказал тебе, продолжал поучения пастырь, что творения Божьи двояки, двояко
и воздержание.
\vs 2Er 14:2
Поэтому от некоторых
следует воздерживаться, а от иных не следует.
\vs 2Er 14:3
Открой мне, господин,
попросил я, от чего следует воздерживаться и от чего не следует.
\vs 2Er 14:4
Воздерживайся, отвечал
он, от зла и не делай его, а от доброго не воздерживайся, но делай его.
\vs 2Er 14:5
Ибо если будешь
удерживаться от доброго и не будешь его делать, согрешишь.
\vs 2Er 14:6
Итак, удерживайся от
всякого зла и делай всякое добро.
\vs 2Er 14:7
От какого зла, спросил
я, должно удерживаться?
\vs 2Er 14:8
От прелюбодеяния, пьянства
и чрезмерных пиршеств, от излишеств в яствах, от роскоши и тщеславия,
\vs 2Er 14:9
от гордости, от лжи и
клеветы, от лицемерия, злопамятства и всякого оскорбления чести другого.
\vs 2Er 14:10
Таковы дела злые, от
которых должно воздерживаться рабу Божьему. Кто не воздерживается от них, тот
не может жить с Богом.
\vs 2Er 14:11
Послушай теперь и о
делах, следующих за ними.
\vs 2Er 14:12
Разве и еще есть,
господин, дела злые?
\vs 2Er 14:13
И подлинно есть еще много
такого, от чего должен воздерживаться раб Божий. Это воровство,
лжесвидетельство, пожелание чужого, надменность и тому подобное.
\vs 2Er 14:14
Не почитаешь ли всего
этого злым? Подлинно, это есть зло рабов Божьих и от всего этого должен
воздерживаться раб Божий, чтобы жить с Богом и быть вместе с теми, которые
воздерживаются от злых дел.
\vs 2Er 14:15
А теперь слушай о тех
добрых делах, которые положено творить, чтобы спастись.
\vs 2Er 14:16
Прежде всего это вера,
страх Божий, любовь, согласие, справедливость, истина, терпение лучше их
ничего нет в жизни человеческой:
\vs 2Er 14:17
кто соблюдает их и во все
дни не станет избегать, тот блажен в своей жизни.
\vs 2Er 14:18
Затем следуют добрые
дела, состоящие в том, чтобы служить вдовам, печься о сиротах и бедных,
избавлять от нужды рабов Божьих,
\vs 2Er 14:19
быть гостеприимным, не
прекословить, быть уравновешенным, считать себя ниже всех людей,
\vs 2Er 14:20
почитать старших
возрастом, соблюдать правду, хранить братство, переносить обиды, быть
великодушным,
\vs 2Er 14:21
не отвергать отпадших от
веры, но обращать и успокаивать их, вразумлять согрешающих, не притеснять
должников и тому подобное.
\vs 2Er 14:22
Не почитаешь ли это
добром?
\vs 2Er 14:23
Нет ничего лучше и
достойнее этого! воскликнул я.
\vs 2Er 14:24
Вот и твори эти дела и не
воздерживайся жить с Богом, равно как и все, которые соблюдут эту заповедь,
будут жить с Богом.

\chhdr{Заповедь 9-я.}
\vs 2Er 15:1
Далее говорил мне пастырь:
отринь от себя сомнения и нисколько не колеблись просить
чего-либо у Господа,
\vs 2Er 15:2
говоря себе: каким образом
могу я просить у Господа и получить, столько согрешив пред Ним?
\vs 2Er 15:3
Не помышляй этого, но от
всего сердца обращайся к Господу и проси без сомнения и познаешь великую
благость Его,
\vs 2Er 15:4
потому что Он не презрит
тебя, но исполнит прошение души твоей.
\vs 2Er 15:5
Ибо Бог не как люди,
которые помнят обиды, Он не помнит зла и милосерден к своему созданию.
\vs 2Er 15:6
Итак, очисти сердце свое
от всех сует настоящего века и прежде всего выполняй данные тебе от Бога
наказы
\vs 2Er 15:7
и получишь все блага,
которых просишь, и все прошения твои не будут оставлены, если будешь просить у
Господа без сомнения.
\vs 2Er 15:8
Те же, которые
сомневаются, совсем ничего не получают из того, о чем просят.
\vs 2Er 15:9
Исполненные веры всего
просят с упованием и получают от Господа, ибо просят без сомнения.
\vs 2Er 15:10
Всякий колеблющийся
человек с трудом спасется, если только не покается.
\vs 2Er 15:11
Поэтому очисти сердце
свое от сомнения, облекись в веру и, веруя Господу, получишь все, о чем
просишь.
\vs 2Er 15:12
Но если иногда, прося о
чем-либо Господа, долго не получаешь, не колеблись оттого, что сразу не
выполняются прошения души твоей.
\vs 2Er 15:13
Ибо, может быть, для
испытания или за грех твой, которого не знаешь, позднее получишь то, что
просишь.
\vs 2Er 15:14
Но ты не переставай
высказывать желание души своей и будешь вознагражден.
\vs 2Er 15:15
Если же придешь в уныние
и перестанешь просить, то жалуйся на себя, а не на Бога, что Он не дает тебе.
\vs 2Er 15:16
Итак, видишь, как
гибельно и ужасно сомнение, и многих даже твердых в вере совсем отторгает от
веры.
\vs 2Er 15:17
Ибо сомнение это дочь
дьявола и сильно злоумышляет на рабов Божьих.
\vs 2Er 15:18
Итак, отвергни сомнение и
одолей его во всяком деле, вооружившись сильной и могущественной верой.
\vs 2Er 15:19
Ибо вера все обещает и
все совершает, сомнение же ни в чем не доверяет себе и оттого не имеет успеха
в делах своих.
\vs 2Er 15:20
Итак, видишь, что вера
исходит свыше от Бога и имеет великую силу.
\vs 2Er 15:21
Сомнение же есть земной
дух, от дьявола, и силы не имеет.
\vs 2Er 15:22
Поэтому служи вере,
имеющей силу, и удаляйся от сомнения, которое бессильно,
\vs 2Er 15:23
и будете жить с Богом
ты и все люди, поступающие так же.

\chhdr{Заповедь 10-я.}
\vs 2Er 16:1
Удаляй от себя всякую печаль, потому что она сестра сомнения и гнева.
\vs 2Er 16:2
Каким образом, господин,
удивился я, она сестра их? Мне кажется, печаль это одно, другое гнев, и
сомнение само по себе.
\vs 2Er 16:3
И он ответил: неразумен
ты. Неужели не понимаешь, что печаль самый злой из всех духов и самый
вредный для рабов Божьих?
\vs 2Er 16:4
Она губит человека как
ничто другое и изгоняет из него Святого Духа и опять спасает.
\vs 2Er 16:5
Господин, не могу я
постичь смысла этих притчей и не понимаю, каким образом печаль может погубить
и опять спасти.
\vs 2Er 16:6
Слушай, сказал он, и
разумей. Кто никогда не изыскивал истины и не исследовал Божество, но только
уверовал и потом предался разным языческим занятиям и другим делам сего мира,
\vs 2Er 16:7
тот не понимает притчей
божественных, потому что помрачается от таких дел, повреждается и загрубевает
разумом.
\vs 2Er 16:8
Как хорошие виноградные
лозы, оставленные без ухода, подавляются и заглушаются разными сорняками и
терниями,
\vs 2Er 16:9
так и люди, которые только
уверовали и вдались в дела этого мира, лишаются своего смысла и, думая о
богатствах, совершенно ничего не понимают, и разум их, занятый мирской суетой,
глух к Господу.
\vs 2Er 16:10
Но те, которые живут в
страхе Божьем, тщательно исследуют истину и божественное и сердцем обращены к
Господу, они легко принимают и разумеют все, что говорится им.
\vs 2Er 16:11
Ибо, где обитает Господь,
там много разума.
\vs 2Er 16:12
Поэтому прилепись к
Господу и все поймешь и уразумеешь.

\vs 2Er 17:1
Послушай теперь,
неразумный, каким образом печаль изгоняет Духа Святого и как опять спасает.
\vs 2Er 17:2
Когда сомневающийся не
обретает успеха в каком-либо деле из-за своего сомнения, то печаль входит в
сердце такого человека, омрачает Духа Святого и изгоняет его.
\vs 2Er 17:3
И когда охватывают
человека гнев и сильное раздражение по какому-нибудь поводу, то опять печаль
входит в сердце, он скорбит о своем поступке, раскаивается, что разгневался.
\vs 2Er 17:4
Эта печаль кажется
спасительною, потому что влечет раскаянье.
\vs 2Er 17:5
Но и в том и в другом
случае печаль оскорбляет Святого Духа.
\vs 2Er 17:6
Печаль, вызванная
сомнением или тем, что не удалось человеку его дело, печаль неправедная.
\vs 2Er 17:7
Печаль же от досады на
дурной поступок не плохая печаль, но и она оскорбляет Святого Духа.
\vs 2Er 17:8
Посему удаляй от себя
печаль и не оскорбляй Святого Духа, в тебе живущего, чтобы он не возроптал на
тебя к Господу и не удалился от тебя.
\vs 2Er 17:9
Ибо Дух Божий, обитающий в
этом теле, не терпит печали.
\vs 2Er 17:10
Итак, облекись ты в
радость, которая всегда имеет благодать пред Господом и угодна Ему и утешайся
ею.
\vs 2Er 17:11
Всякий радующийся человек
совершает добро и помышляет о добре, презирая печаль.
\vs 2Er 17:12
А человек печальный
всегда зол, во-первых, потому, что оскорбляет Святого Духа, который дан
человеку радостным;
\vs 2Er 17:13
и, во-вторых, потому, что
он творит беззаконие, не обращаясь к Господу и не исповедуясь перед Ним.
\vs 2Er 17:14
Молитва печального
человека никогда не достигает престола Божьего.
\vs 2Er 17:15
И я спросил его: почему
же, господин, молитва печального человека не восходит к престолу Господню?
\vs 2Er 17:16
Потому, ответил он,
что печаль пребывает в его сердце.
\vs 2Er 17:17
Печаль, смешанная с
молитвою, не допускает молитву чистою взойти к престолу Божьему.
\vs 2Er 17:18
Как вино с добавлением
уксуса уже не имеет прежней приятности, так и печаль, примешанная к Святому
Духу, не имеет той же чистой молитвы.
\vs 2Er 17:19
Посему очищайся от злой
печали и будешь жить с Богом,
\vs 2Er 17:20
и все будут жить с Богом,
если только отбросят от себя печаль и облекутся в радость.

\chhdr{Заповедь 11-я.}
\vs 2Er 18:1
Пастырь показал мне людей, сидящих на скамьях, и одного стоящего на кафедре,
\vs 2Er 18:2
и сказал: посмотри на них.
Те, которые сидят на скамьях, верующие, а стоящий на кафедре лжепророк,
погубляющий смысл рабов Божьих тех, которые двоедушествуют, а не истинно
верующих.
\vs 2Er 18:3
Эти двоедушные приходят к
нему как к пророку и спрашивают его о том, что станет с ними,
\vs 2Er 18:4
и он, не имея в себе силы
Духа Божественного, отвечает им, говоря то, что хотят они услышать, и
наполняет души их лживыми обещаниями.
\vs 2Er 18:5
Будучи суетен, он суетно и
отвечает суетным людям.
\vs 2Er 18:6
Впрочем, он говорит и
кое-что справедливое, потому что дьявол вселяет в него свой дух, дабы привлечь
кого-либо из праведных.
\vs 2Er 18:7
Но сильные в вере,
облеченные в истину не присоединяются к таким духам, но удаляются от них.
\vs 2Er 18:8
Двоедушные же и часто
кающиеся обращаются за прорицаниями, как и народы, и навлекают на себя великий
грех своим идолопоклонством,
\vs 2Er 18:9
потому что спрашивающий
лжепророка является идолопоклонником, он чужд истины и неразумен.
\vs 2Er 18:10
А всякий дух, Богом
данный, не дожидается расспросов, но, имея силу Божественную, говорит все сам,
потому что он свыше, от силы Духа Божьего.
\vs 2Er 18:11
Дух, который отвечает на
вопросы согласно желаниям человеческим, есть дух земной, легкомысленный, не
имеющий силы: он совсем не говорит, если его не спрашивают.
\vs 2Er 18:12
И я сказал: как же можно
распознать, кто истинный пророк и кто лжепророк?
\vs 2Er 18:13
Выслушай, говорит, об
обоих пророках; и по тому, что я скажу тебе, отличишь пророка Божьего от
ложного пророка.
\vs 2Er 18:14
По делам узнавай
человека, который имеет Дух Божий.
\vs 2Er 18:15
Во-первых, он спокоен,
кроток и смирен, удаляется от всякого зла и суетного желания этого века,
\vs 2Er 18:16
ставит себя ниже всех
людей и никому не отвечает на вопросы, не говорит наедине;
\vs 2Er 18:16
Дух Божий говорит не
тогда, когда человек того желает, но когда угодно Богу.
\vs 2Er 18:17
Поэтому когда человек,
имеющий Дух Божий, придет в церковь праведных, имеющих веру, там совершается
молитва к Господу;
\vs 2Er 18:18
тогда ангел пророческого
духа, приставленный к нему, исполняет этого человека Духом Святым, и он
говорит к собранию, как угодно Богу. Так проявляется Дух Божественный и сила
его.
\vs 2Er 18:19
Слушай теперь и о духе
земном, суетном, неразумном и не имеющем силы.
\vs 2Er 18:20
Прежде всего человек,
кажущийся исполненным духа, возвышает себя, стремится к власти, нагл и
многословен,
\vs 2Er 18:21
живет среди роскоши и
многих удовольствий, берет мзду за свое прорицание, без вознаграждения не
пророчествует.
\vs 2Er 18:22
Может ли Дух Божий брать
мзду и пророчествовать?
\vs 2Er 18:23
Это не свойственно
пророку Божьему, и в поступающих таким образом обитает дух земной.
\vs 2Er 18:24
Далее, он не входит в
собрание мужей праведных, но избегает их
\vs 2Er 18:25
и, наоборот, общается с
людьми двоедушными и пустыми, пророчествует в местах потаенных и обманывает
речами, которые хотят услышать, и говорит суетное людям суетным:
\vs 2Er 18:26
так пустая посуда, когда
складывается с другими пустыми же, не разбивается, но они хорошо приходятся
одна к другой.
\vs 2Er 18:27
А когда он оказывается
среди людей праведных, исполненных Духа Божественного, возносящих молитву,
тогда и обнаруживается его пустота:
\vs 2Er 18:28
земной дух от страха
покидает его, и он, совершенно поверженный, ничего не может говорить.
\vs 2Er 18:29
Если в кладовую поместить
вино или масло и туда же поставить пустой сосуд, а после брать запасы из
кладовой, то сосуд, который поставил пустым, пустым и найдешь.
\vs 2Er 18:30
И пустые пророки, какими
приходят к людям, имеющим Святого Духа, такими и остаются. Вот образ пророка
истинного и ложного.
\vs 2Er 18:31
Итак, испытывай по делам
и по жизни того человека, который говорит, что он имеет Святого Духа.
\vs 2Er 18:32
Верь Духу, приходящему от
Бога и имеющему силу; духу же земному и пустому, в котором нет силы, не верь,
ибо он приходит от дьявола.
\vs 2Er 18:33
Задумайся над примером,
который приведу я тебе. Если взять камень и бросить в небо, то сможешь ли
докинуть до него?
\vs 2Er 18:34
Или же если взять трубу с
водою, направить струю в небо, то сможешь ли ты пробить небо?
\vs 2Er 18:35
Что ты, господин,
воскликнул я, все это невозможно!
\vs 2Er 18:36
Вот, сказал он, как
этого не может быть, так точно дух земной бессилен и недейственен.
\vs 2Er 18:37
Осознай теперь силу,
свыше приходящую. Град крупинка очень малая, но, попадая в голову человека,
какую причиняет боль?
\vs 2Er 18:38
Или еще пример: дождевая
капля, которая, с крыши скатываясь вниз, источает камень.
\vs 2Er 18:39
Видишь, и самое малое,
что сверху падает на землю, имеет великую силу: так силен и Дух Божественный,
приходящий свыше.
\vs 2Er 18:40
Этому Духу ты верь, а от
другого удаляйся.

\chhdr{Заповедь 12-я.}
\vs 2Er 19:1
Пастырь сказал мне: удали от себя всякую похоть злую и облекись в желание
доброе и святое.
\vs 2Er 19:2
Ибо, облекшись в желание
доброе, ты возненавидишь зло и будешь управлять им, как захочешь.
\vs 2Er 19:3
Похоть злая люта и с
трудом усмиряется: она страшна и своею лютостью сокрушает людей.
\vs 2Er 19:4
Но сокрушает тех людей,
которые не имеют стремления доброго и погрузились в дела этого века: их-то она
предает смерти.
\vs 2Er 19:5
Какие действия, господин,
спросил я, злой похоти обрекают людей на смерть? Объясни мне, чтобы я мог
избегать их.
\vs 2Er 19:6
Послушай, посредством
каких действий злая похоть умерщвляет рабов Божьих.

\vs 2Er 20:1
Злая похоть состоит в том,
чтобы желать чужой жены, или жене желать чужого мужа, желать великого
богатства, множества роскошных яств и питий и других наслаждений:
\vs 2Er 20:2
ибо всякое наслаждение
бессмысленно и суетно для рабов Божьих.
\vs 2Er 20:3
Таковы пожелания злые,
умерщвляющие рабов Божьих.
\vs 2Er 20:4
Злая похоть есть дочь
дьявола. Поэтому должно удаляться злой похоти, чтобы жить с Богом.
\vs 2Er 20:5
А те, которые поддадутся
злой похоти и не воспротивятся ей, погибнут, потому что она смертоносна.
\vs 2Er 20:6
Итак, ты стремись к правде
и, вооружившись страхом Господним, противостой злой похоти. Ибо страх Божий
обитает в добрых пожеланиях.
\vs 2Er 20:7
И злая похоть, видя тебя
вооруженным страхом Господним и противящимся ей, убежит от тебя далеко и не
явится к тебе, боясь твоего оружия;
\vs 2Er 20:8
и одержавши победу и
увенчанный за нее, предайся стремлению к правде и, воздавши Ему за полученную
тобою победу, служи Ему по Его воле.
\vs 2Er 20:9
И если послужишь доброму
началу и покоришься Ему, то можешь владычествовать над злою похотью и
управлять ею, как тебе угодною.
\vs 2Er 20:10
Желал бы я услышать,
господин, сказал я, как должно служить доброму желанию?
\vs 2Er 20:11
Слушай. Имей страх Божий
и веру в Бога, люби истину, твори правду и подобные добрые дела.
\vs 2Er 20:12
Делая это, ты будешь
угодным рабом Божьим и будешь жить с Богом; и все, которые будут служить
стремлению доброму, будут жить с Богом.

\vs 2Er 21:1
И так окончил он
двенадцать заповедей и сказал мне:
\vs 2Er 21:2
вот тебе заповеди,
поступай по ним и к тому же убеждай людей слушать тебя, чтобы покаяние их было
чисто в остальные дни их жизни.
\vs 2Er 21:3
И это служение, которое
поручаю тебе, исполняй тщательно и получишь великий плод,
\vs 2Er 21:4
ибо найдешь любовь у всех,
которые покаются и послушаются слов твоих.
\vs 2Er 21:5
Я буду с тобою и буду
побуждать их слушаться тебя.
\vs 2Er 21:6
И я сказал ему: господин,
эти заповеди величественны, прекрасны и могут возвеселить сердце человека,
который исполнит их.
\vs 2Er 21:7
Но не знаю, господин,
способен ли человек соблюдать эти заповеди, потому что они очень трудны.
\vs 2Er 21:8
Он отвечал мне: эти
заповеди легко соблюсти, и не покажутся они трудными, если будешь убежден, что
их можно соблюсти;
\vs 2Er 21:9
но если закралось в сердце
твое сомнение, что не по силам человеку, то не соблюдешь их.
\vs 2Er 21:10
Теперь же говорю тебе:
если не соблюдешь этих заповедей и пренебрежешь ими, то не спасешься ты и дети
твои, и весь дом твой,
\vs 2Er 21:11
потому что ты сам себе
присудил, что этих заповедей нельзя соблюсти человеку.

\vs 2Er 22:1
Произносил он это с
большим гневом, и я очень смутился и испугался.
\vs 2Er 22:2
Лицо его изменилось так,
что вид его стал невыносим для человека.
\vs 2Er 22:3
Но, видя, что я весь в
смущении и страхе, начал он говорить умереннее и ласковее:
\vs 2Er 22:4
неразумный и непостоянный,
не видишь ли славу Божью, не понимаешь, как велик и дивен Тот, который
сотворил мир для человека,
\vs 2Er 22:5
и все творение покорил
человеку, и дал ему всю власть господствовать над всем поднебесным?
\vs 2Er 22:6
Если человек есть владыка
тварей Божьих и над всем господствует, то ужели он не может господствовать и
над этими заповедями?
\vs 2Er 22:7
Это по силам человеку,
имеющему Господа в сердце своем.
\vs 2Er 22:8
Кто же имеет Господа
только в устах своих, огрубел сердцем и далек от Господа, для того эти
заповеди тяжки и неисполнимы.
\vs 2Er 22:9
Итак вы, слабые и
нетвердые в вере, положите себе Господа вашего в сердце и узнаете, что ничего
нет легче этих заповедей, ничего приятнее и доступнее их.
\vs 2Er 22:10
Обратитесь к Господу,
оставьте дьяволу его удовольствия, которые злы и горьки, и не бойтесь дьявола,
потому что над вами он не имеет силы.
\vs 2Er 22:11
Ибо я с вами, ангел
покаяния, и я господствую над ним.
\vs 2Er 22:12
Дьявол наводит страх, но
страх его не имеет силы.
\vs 2Er 22:13
Посему не бойтесь его, и
он покинет вас.

\vs 2Er 23:1
И я попросил его:
господин, выслушай несколько слов моих.
\vs 2Er 23:2
Говори, разрешил он.
\vs 2Er 23:3
Всякий человек желает
исполнять Божьи заповеди, и нет такого, который бы не просил у Бога силы
соблюдать Его заповеди;
\vs 2Er 23:4
но дьявол упорен и своею
силою противодействует рабам Божьим.
\vs 2Er 23:5
Не может дьявол,
возразил он, пересилить рабов Божьих, которые веруют в Господа от всего
сердца.
\vs 2Er 23:6
Дьявол может
противоборствовать, но победить не может.
\vs 2Er 23:7
Если воспротивитесь ему,
то, побежденный, он с позором покинет вас.
\vs 2Er 23:8
Боятся дьявола, как будто
имеющего власть, те, которые не тверды в вере.
\vs 2Er 23:9
Дьявол искушает рабов
Божьих и, если найдет слабых, губит их.
\vs 2Er 23:10
Когда человек наполняет
сосуды хорошим вином и между ними ставит несколько сосудов неполных,
\vs 2Er 23:11
то, приходя попробовать
вино, не думает о полных, ибо знает, что они хороши, а отведывает из неполных,
не скисло ли в них вино,
\vs 2Er 23:12
потому что в неполных
сосудах вино скоро скисает и теряет вкус.
\vs 2Er 23:13
Так и дьявол приходит к
рабам Божьим, чтобы искусить их.
\vs 2Er 23:14
И все те, которые полны
веры, мужественно противятся ему; и он удаляется от них, потому что негде
войти ему.
\vs 2Er 23:15
Тогда он подступает к
тем, которые не полны веры, и, имея возможность, вселяется в них, делает с
ними что хочет, и они становятся его рабами.

\vs 2Er 24:1
Но, говорю вам я, ангел
покаяния: не бойтесь дьявола, ибо я послан для того, чтобы быть с вами,
кающимися от всего сердца, и утвердить вас в вере.
\vs 2Er 24:2
Посему верьте вы, которые
по грехам своим отчаялись в спасении, и, прилагая грехи к грехам, отягощаете
жизнь свою:
\vs 2Er 24:3
если обратитесь к Господу
от всего сердца вашего и будете творить правду в остальные дни своей жизни и
служить Ему по воле Его,
\vs 2Er 24:4
то Он простит прежние
грехи ваши, и обретете власть над делами дьявола.
\vs 2Er 24:5
Угроз же дьявола вовсе не
бойтесь, потому что они бессильны, как нервы человека мертвого.
\vs 2Er 24:6
Итак, слушайте меня и
бойтесь Господа, Который может спасти и погубить: соблюдайте заповеди Его и
будете жить с Богом.
\vs 2Er 24:7
И я сказал ему: господин,
теперь я проникся всеми заповедями Господа, потому что ты со мною;
\vs 2Er 24:8
знаю, что сокрушишь всю
силу дьявола, и мы восторжествуем над ним;
\vs 2Er 24:9
и надеюсь, что могу
соблюсти при помощи Божьей заповеди, которые ты передал.
\vs 2Er 24:10
Соблюдешь, сказал он,
если сердце твое будет чисто пред Господом, и все соблюдут, которые очистят
сердца свои от суетных похотей этого века, и будут жить с Богом.

\bibbookdescr{3Er}{
  inline={Пастырь Ермы. Книга 3. Подобия},
  toc={3-я Ермы},
  bookmark={3-я Ермы},
  header={3-я Ермы},
  abbr={3~Ермы}
}
\chhdr{Подобие 1-е.}
\vs 3Er 1:1
Мы, не имея в этом мире постоянного города, должны искать будущего.
\vs 3Er 1:2
Пастырь сказал мне: знаете
ли, что вы, рабы Божьи, находитесь в странствии? Ваш город далеко отсюда. Если
знаете ваше отечество, в котором надлежит вам жить, то зачем здесь покупаете
поместья, строите великолепные здания и ненужные жилища?
\vs 3Er 1:3
Ибо кто занимается подобными приготовлениями в этом городе, тот не помышляет о
возвращении в свое отечество. Несмысленный, двоедушный и жалкий человек, разве
не понимаешь, что всё это чужое и под властью другого?
\vs 3Er 1:4
Ибо господин этого города
говорит: или следуй моим законам, или убирайся вон из моих пределов. Что же
поэтому сделаешь ты, имея собственный закон в твоем отечестве? Ужели ради
полей или других стяжаний своих откажешься от отечественного закона?
\vs 3Er 1:5
Если же ты откажешься, а
потом пожелаешь возвратиться в свое отечество, то не будешь принят, но изгнан
оттуда.
\vs 3Er 1:6
Итак, смотри, подобно
страннику на чужой стороне, не приготовляй себе ничего более того, сколько
тебе необходимо для жизни;
\vs 3Er 1:7
и будь готов к тому,
чтобы, когда господин этого города захочет изгнать тебя за то, что не
повинуешься закону его,~--- идти тебе в своё отечество и жить по своему закону
беспечально и радостно.
\vs 3Er 1:8
Итак, вы, служащие Богу и
имеющие Его в сердцах своих, смотрите: делайте дела Божьи, помня о заповедях
Его и обетованиях, Им данных, и веруйте Ему, что Он исполнит их, если будут
соблюдены Его заповеди.
\vs 3Er 1:9
Вместо полей искупайте
души от нужд, сколько кто может, помогайте вдовам и сиротам; богатство и все
стяжания ваши употребляйте на такого рода дела, ради которых вы и получили их
от Бога.
\vs 3Er 1:10
Ибо Господь обогатил вас
для того, чтобы вы исполняли такое служение Ему.
\vs 3Er 1:11
Гораздо лучше делать это,
нежели покупать дома и поместья, ибо имущество тленно, тогда как то, что
сделаешь во имя Божье, обретешь в своём городе и будешь иметь радость без
печали и страха.
\vs 3Er 1:12
Итак, не желайте богатств
народов, ибо несвойственны они рабам Божьим; избытком же своим распоряжайтесь
так, чтобы могли вы получить радость.
\vs 3Er 1:13
И не делайте фальшивой
монеты, не касайтесь и не желайте чужого. Делай своё дело~--- и спасешься.

\chhdr{Подобие 2-е.}
\vs 3Er 2:1
Однажды, когда я, прогуливаясь по полю, увидел вяз и
виноградное дерево и размышлял о плодах их~--- пастырь явился мне и спросил: что
ты думаешь об этом виноградном дереве и вязе?
\vs 3Er 2:2
Думаю, что они пригодны
друг для друга.
\vs 3Er 2:3
И сказал он мне: эти два
дерева являют рабам Божьим глубокий смысл.
\vs 3Er 2:4
Желал бы я познать,
господин, этот смысл.
\vs 3Er 2:5
Смотрите же,~--- сказал он.
Это виноградное дерево имеет плод, а вяз~--- дерево бесплодное; но виноградное
дерево не может приносить обильных плодов, если не будет опираться на вяз.
\vs 3Er 2:6
Ибо, лёжа на земле, оно
дает гнилой плод; но если виноградная лоза будет висеть на вязе, то дает плод
и за себя, и за вяз.
\vs 3Er 2:7
Итак, видишь, что вяз дает
плод не меньший, а гораздо больший, нежели виноградная лоза, потому что
виноградная лоза, поддерживаемая вязом, дает плод и обильный и хороший, но,
лёжа на земле, дает плод плохой и малый.
\vs 3Er 2:8
Это служит притчею для
рабов Божьих, для бедного и богатого.
\vs 3Er 2:9
Каким образом, объясни
мне?
\vs 3Er 2:10
Слушай,~--- говорит он.
Богатый имеет много сокровищ, но беден перед Господом. Занятый своими
богатствами, он очень мало молится Господу и если имеет какую молитву, то
скудную и не имеющую силы.
\vs 3Er 2:11
Но когда богатый подает
бедному то, в чем он нуждается, тогда бедный молит Господа за богатого, и Бог
подает богатому все блага, потому что бедный богат в молитве и молитва его
имеет великую силу пред Господом.
\vs 3Er 2:12
Богатый подает бедному,
веруя, что ему внимает Господь, и охотно и без сомнения подает ему всё,
заботясь, чтобы у него не было в чем-нибудь недостатка.
\vs 3Er 2:13
Бедный благодарит Бога за
богатого, дающего ему.
\vs 3Er 2:14
Так люди, думая, что вяз
не дает плода, не понимают того, что во время засухи вяз, имея в себе влагу
питает виноградную лозу, и виноградная лоза благодаря этому дает двойной плод
и за себя, и за вяз.
\vs 3Er 2:15
Так и бедные, моля
Господа за богатых, бывают услышаны и умножают богатства их, а богатые,
помогая бедным, ободряют их души. Те и другие участвуют в добром деле.
\vs 3Er 2:16
Итак, кто поступает таким
образом, не будет оставлен Господом, но будет вписан в книгу жизни.
\vs 3Er 2:17
Блаженны те, которые,
имея богатство, сознают, что они обогащаются от Господа, ибо кто почувствует
это, тот может совершать добро.

\chhdr{Подобие 3-е.}
\vs 3Er 3:1
Пастырь показал мне много деревьев без листьев, казавшихся
иссохшими: все они были похожи.
\vs 3Er 3:2
Видишь эти деревья?
\vs 3Er 3:3
Вижу,~--- говорю я.~--- Они
похожи друг на друга и сухи.
\vs 3Er 3:4
Эти деревья служат образом
людей, живущих в этом мире.
\vs 3Er 3:5
Почему же, господин,~--- спросил я,~--- они как бы засохли и похожи друг на друга?
\vs 3Er 3:6
Потому,~--- отвечал он,~--- что в этом веке не различимы
ни праведные, ни нечестивые люди: одни походят на других.
\vs 3Er 3:7
Ибо настоящий век есть
зима для праведных, которые, живя с грешниками, по виду не отличаются от них.
\vs 3Er 3:8
Как во время зимы все
деревья с облетевшими листьями сходны между собою, и не видно, которые из них
действительно засохли, а которые живы, так точно в настоящем веке нельзя
распознать праведников и грешников, но все похожи одни на других.

\chhdr{Подобие 4-е.}
\vs 3Er 4:1
Снова показал мне пастырь многие деревья, из которых одни
расцвели, а другие были иссохшие.
\vs 3Er 4:2
Видишь ли эти деревья?
\vs 3Er 4:3
Вижу, господин,~--- отвечал
я,~--- одни засохли, а другие покрыты листьями.
\vs 3Er 4:4
Эти зеленеющие деревья,~--- сказал он,~--- означают праведных, которые будут жить в грядущем веке.
\vs 3Er 4:5
Ибо будущий век есть лето
для праведных и зима для грешников.
\vs 3Er 4:6
Итак, когда воссияет
благость Господа, тогда явятся служащие Богу и все будут видимы.
\vs 3Er 4:7
Ибо как летом созревает
плод всякого дерева, и становится понятно, каково оно, так точно обнаружится и
будет видим и плод праведных, и все они явятся радостными в том веке.
\vs 3Er 4:8
Народы же и грешники суть
сухие деревья, которые ты видел. Они обретутся в будущем веке сухими и
бесплодными, и будут преданы огню, как дрова, и обнаружится, что во время их
жизни дела их были злы.
\vs 3Er 4:9
Грешники будут преданы
огню, потому что согрешили и не раскаялись в грехах своих, народы же потому,
что не познали Бога~--- Творца своего.
\vs 3Er 4:10
Посему ты приноси плод
добрый, чтобы он явился во время того лета. Воздерживайся от многих попечений
и никогда не согрешишь.
\vs 3Er 4:11
Ибо имеющие многие заботы
согрешают во многом, потому что озабочены своими делами и не служат Богу.
\vs 3Er 4:12
Каким же образом человек,
не служащий Богу, может просить и получить что-либо от Бога?
\vs 3Er 4:13
Те, которые служат Богу,
просят и получат свои прошения, а не служащие Богу~--- не получат.
\vs 3Er 4:14
Кто занимается одним
делом, тот может и служить Богу; потому что дух его не отчуждается от Господа,
но чистою мыслию служит Богу.
\vs 3Er 4:15
Итак, если исполнишь это,
будешь иметь плод в грядущем веке; равно как и все, которые исполнят это,
будут иметь плод.

\chhdr{Подобие 5-е.}
\vs 3Er 5:1
Однажды во время поста сидел я на горе, благодарил Господа за
то, что сделал Он со мною, и увидел вдруг пастыря рядом с собою.
\vs 3Er 5:2
И спрашивает он у меня: что так рано пришел ты сюда?
\vs 3Er 5:3
Потому, господин, что нахожусь на стоянии.
\vs 3Er 5:4
А что такое стояние?
\vs 3Er 5:5
То есть пощусь, господин,~--- объяснил я.
\vs 3Er 5:6
Каким же образом,~--- спросил он,~--- постишься ты?
\vs 3Er 5:7
Как постился по обыкновению, так и пощусь.
\vs 3Er 5:8
Не умеете вы,~--- сказал он,
поститься Богу; и пост, который совершаете, бесполезен.
\vs 3Er 5:9
Почему, господин, говоришь так?
\vs 3Er 5:10
То, как вы думаете
поститься, не есть пост, но я научу тебя, какой пост есть совершенный и
угодный Богу.
\vs 3Er 5:11
Слушай: Бог не хочет
такого суетного поста, ибо, постясь таким образом, ты не совершаешь правды.
\vs 3Er 5:12
Постись же Богу следующим
постом: не лукавствуй в жизни, но служи Богу чистым сердцем; соблюдай Его
заповеди, ходи в Его повелениях и не допускай никакой злой похоти в сердце
своем.
\vs 3Er 5:13
Веруй в Бога, и если
исполнишь это и будешь иметь страх Божий и удержишься от всякого злого дела,
то будешь жить с Богом.
\vs 3Er 5:14
И таким образом ты
совершишь великий и угодный Богу пост.

\vs 3Er 6:1
Послушай притчу
относительно поста, которую я намерен поведать тебе.
\vs 3Er 6:2
Некто имел поместье и
много рабов. В одной части земли своей он насадил виноградник, и потом,
отправляясь в дальнее путешествие, избрал раба, самого верного и честного, и
поручил ему виноградник с тем, чтобы он к виноградным лозам приставил
подпорки, обещая за исполнение этого приказания дать ему свободу.
\vs 3Er 6:3
Только это хозяин приказал
рабу сделать в винограднике и с тем отправился.
\vs 3Er 6:4
Раб тщательно сделал, что
господин повелел: он расставил подпорки в винограднике, но, приметив в нём
много сорных трав, стал рассуждать сам с собою: я исполнил приказание
господина, вскопаю теперь виноградник, и он будет красивее; а если выполоть
сорную траву, он, не заглушаемый сорняками, даст больше плода.
\vs 3Er 6:5
И принялся за работу;
вскопал виноградник и выполол в нём все сорняки, и стал виноградник красивым и
цветущим, не засоренным травами.
\vs 3Er 6:6
Через некоторое время
возвратился господин его и пришел в виноградник. Когда он увидел, что
виноградник хорошо обставлен и сверх того вскопан, прополот, и лозы обильны
плодами, то был весьма доволен поступком раба своего.
\vs 3Er 6:7
Итак, пригласил он
любимого сына, своего наследника, и друзей, своих советников, и рассказал им,
что приказал он сделать рабу своему и что тот сверх этого сделал.
\vs 3Er 6:8
Они тотчас приветствовали
раба с тем, что он получил столь высокую похвалу от своего господина.
\vs 3Er 6:9
Господин же говорит им:
<<Я обещал свободу этому рабу, если он исполнит данное приказание, он исполнил его
и сверх того приложил к винограднику добрый труд, который мне весьма
понравился.
\vs 3Er 6:10
Поэтому за его усердие я
хочу сделать его сонаследником моего сына, потому что, помысливши доброе, он
не оставил его, но исполнил.>>
\vs 3Er 6:11
Это намерение господина,
то есть чтобы раб был сонаследником сыну, одобрили и сын, и друзья его.
\vs 3Er 6:12
Потом, спустя несколько
дней, когда созваны были гости, господин со своего пира посылал тому рабу
много яств.
\vs 3Er 6:13
Получая их, раб брал из
них то, что было для него достаточно, остальное же делил между товарищами
своими.
\vs 3Er 6:14
Они, обрадованные, начали
желать ему, чтобы он еще большую любовь нашел у хозяина за свою доброту и
щедрость.
\vs 3Er 6:15
Когда обо всем этом узнал
господин его, он опять весьма обрадовался и снова рассказал друзьям и сыну о
поступке своего раба, и они еще более одобрили мысль господина, чтобы раб этот
был сонаследником сына.

\vs 3Er 7:1
Я сказал: господин, не
знаю этих притчей и не смогу понять, если ты не объяснишь мне их.
\vs 3Er 7:2
Всё,~--- обещал он,~--- объясню, что только скажу и покажу тебе. Соблюдай заповеди Господа, и будешь
угоден Богу и включен в число тех, которые соблюли Его заповеди.
\vs 3Er 7:3
Если же сделаешь что-либо
доброе сверх заповеданного Господом, то приобретешь себе еще большее
достоинство и будешь пред Господом славнее, нежели мог быть прежде.
\vs 3Er 7:4
Итак, если соблюдешь
заповеди Господа и к ним присоединишь эти стояния, то получишь великую
радость, особенно если будешь исполнять их согласно с моим внушением.
\vs 3Er 7:5
Господин,~--- говорю,~--- я
исполню все, что ни повелишь мне, ибо я знаю, что ты будешь со мною.
\vs 3Er 7:6
Буду,~--- сказал он,~--- с
тобою, потому что имеешь такое доброе намерение; буду также и со всеми
имеющими такое намерение.
\vs 3Er 7:7
Этот пост,~--- продолжал он,
при исполнении заповедей Господа очень хорош, и соблюдай его таким образом:
прежде всего воздерживайся от всякого дурного слова и злой похоти и очисти
сердце своё от всех сует века сего.
\vs 3Er 7:8
Если соблюдать это, пост у
тебя будет праведный.
\vs 3Er 7:9
Поступай же так: исполнив
вышесказанное, в тот день, в который постишься, ничего не вкушай, кроме хлеба
и воды; а то из пищи, что ты в этот день сбережешь таким образом, отложи и
отдай вдове, сироте или бедному;
\vs 3Er 7:10
таким образом ты смиришь
свою душу; а получивший от тебя насытит свою душу и будет за тебя молиться
Господу.
\vs 3Er 7:11
Если будешь совершать
пост так, как я повелел тебе, то жертва твоя будет приятна Господу, и этот
пост будет написан, и дело, таким образом совершаемое, прекрасно, радостно и
угодно Господу.
\vs 3Er 7:12
Если ты соблюдешь это с
детьми своими и со всеми домашними твоими, то будешь блажен; и все, кто только
соблюдут это, будут блаженны и что ни попросят у Господа, всё получат.

\vs 3Er 8:1
И упрашивал я его, чтобы
объяснил мне эту притчу о поместье и господине, о винограднике и рабе,
поставившем подпорки в нем, о травах, выполотых в винограднике, о сыне и
друзьях, призванных для совета: ибо я понял, что все это~--- притча.
\vs 3Er 8:2
Он сказал мне: очень смел
ты на вопросы. Ты ни о чем не должен спрашивать; что должно быть объяснено, то
объяснится тебе.
\vs 3Er 8:3
Господин, я напрасно буду
видеть то, что ты покажешь мне, не истолковав, что это значит; напрасно буду
слушать и притчи, если ты будешь предлагать их мне без объяснения.
\vs 3Er 8:4
Он сказал мне снова: кто
раб Божий и в сердце своем имеет Господа, тот просит у Него разума и получает,
и постигает всякую притчу, и понимает слова Господа, сказанные приточно.
\vs 3Er 8:5
А беспечные и ленивые к
молитве колеблются просить Господа, тогда как Господь многомилостив и
непрестанно дает всем просящим у Него.
\vs 3Er 8:6
Ты же утвержден тем
достопоклоняемым ангелом и получил от Него столь могущественную молитву.
Почему, если не ленив ты, не просишь разума и не получаешь от Господа?
\vs 3Er 8:7
Если ты при мне,~--- сказал
я ему~--- надлежит мне тебя обо всем просить и спрашивать, ибо ты всё мне
показываешь и говоришь со мною. Если бы без тебя я видел это или слышал, тогда
бы Господа просил, чтобы было мне объяснено.

\vs 3Er 9:1
И он отвечал: я и прежде
говорил тебе, что ты искусен и смел на то, чтобы спрашивать смысл притчей.
\vs 3Er 9:2
Так как ты настойчив, то
объясню тебе притчу о поместье и о прочем, чтобы ты рассказал всем.
\vs 3Er 9:3
Слушай же и разумей.
Поместье, о котором говорится в притче, означает мир. Владелец поместья есть
Творец, который всё создал и утвердил. Сын есть Дух Святой. Раб~--- Сын Божий.
\vs 3Er 9:4
Виноградник означает
народ, который насадил Господь. Подпорки суть ангелы, приставленные Господом
для сохранения Его народа.
\vs 3Er 9:5
Травы, уничтоженные в
винограднике, суть преступления рабов Божьих. Яства, которые с пира посылал
господин рабу, суть заповеди, которые через Сына своего дал Господь своему
народу.
\vs 3Er 9:6
Друзья, призванные на
совет, суть святые ангелы первозданные.
\vs 3Er 9:7
Отсутствие же господина
означает время, остающееся до Его Пришествия.

\vs 3Er 10:1
Я сказал тогда: господин,
величественно, дивно и славно всё, что ты поведал, но мог ли я, господин,
понять это?
\vs 3Er 10:2
Да ни один человек, хотя
бы и очень разумный, не может постичь этого. Теперь же спрошу тебя вот о чем.
\vs 3Er 10:3
Спрашивай, что хочешь.
\vs 3Er 10:4
Почему Сын Божий в этой
притче представляется рабом?
\vs 3Er 10:5
Слушай,~--- сказал он. Сын
Божий предстает в рабском положении, но имеет великое могущество и власть.
\vs 3Er 10:6
Каким образом, господин,
не понимаю?
\vs 3Er 10:7
Бог насадил виноградник,
то есть создал народ и поручил Сыну своему; Сын же приставил ангелов для
сохранения каждого из людей и сам усердно трудился и изрядно пострадал, чтобы
искупить грехи их.
\vs 3Er 10:8
Ибо никакой виноградник не
может быть очищен без труда и подвига.
\vs 3Er 10:9
Итак, очистив грехи народа
Своего, Он показал им путь жизни и дал им закон, принятый Им от Отца.
\vs 3Er 10:10
Видишь, что Он есть
Господь народа со всею властью, полученною от Отца.
\vs 3Er 10:11
А вот почему Господь
держал совет о наследстве с Сыном Своим и славными ангелами. Дух Святой,
прежде Сущий, создавший всю тварь, Бог поселил в плоть, какую Он пожелал.
\vs 3Er 10:12
И эта плоть, в которую
вселился Дух Святой, хорошо послужила Духу, ходя в чистоте и святости и ничем
не осквернив Духа.
\vs 3Er 10:13
И так как жила она
непорочно, и подвизалась вместе с Духом, и мужественно содействовала Ему во
всяком деле, то Бог принял её в общение, ибо Ему угодно было житие плоти,
которая не осквернилась на земле, имея в себе Дух Святой.
\vs 3Er 10:14
И призвал Он в совет Сына
и добрых ангелов, чтобы и эта плоть, непорочно послужившая Духу, обрела место
успокоения, чтобы не оказалась без награды непорочная и чистая, в которой
поселился Святой Дух. Вот тебе объяснение этой притчи.

\vs 3Er 11:1
Возрадовался я, господин,
сказал я, услышав такое объяснение.
\vs 3Er 11:2
Слушай далее. Эту плоть
храни неоскверненною и чистою, чтобы дух, живущий в ней, был доволен ею и
спаслась твоя плоть.
\vs 3Er 11:3
Смотри также, никогда не
допускай мысли, что эта плоть погибнет, и не злоупотребляй ею в какой-либо
похоти.
\vs 3Er 11:4
Ибо если осквернишь плоть
свою, то осквернишь и Духа Святого, если же осквернишь Духа Святого, не будешь
жить.
\vs 3Er 11:5
И спросил я: что же, если
кто по неведению, до того, как услышать эти слова, осквернил свою плоть, каким
образом получит он спасение?
\vs 3Er 11:6
Прежние грехи неведения,~--- сказал он,~--- исцелить может один Бог, ибо Ему принадлежит всякая власть.
\vs 3Er 11:7
Но теперь храни себя; и
Господь Всемогущий и милостивый даст искупление для прежних грехов, если
впредь не осквернишь плоти своей и духа. Ибо они взаимопричастны, и одна без
другого не оскверняется.
\vs 3Er 11:8
Итак, и то и другое
сохраняй чистым и будешь жить с Богом.

\chhdr{Подобие 6-е.}
\vs 3Er 12:1
Когда я, сидя дома, прославлял Господа за всё то, что видел,
и размышлял о заповедях, как они прекрасны, тверды, почтенны и сладостны и
могут спасти душу человека,
\vs 3Er 12:2
то я говорил сам себе:
<<Блажен буду, если стану поступать по этим заповедям; и всякий поступающий по
ним, будет блажен!>>
\vs 3Er 12:3
Когда рассуждал таким
образом, вдруг пастырь появился возле меня
\vs 3Er 12:4
и сказал: что раздумываешь
о заповедях моих, которые я тебе преподал? Они прекрасны, нисколько не
сомневайся; но облекись верою в Господа и будешь исполнять их, ибо наделю тебя
для этого силой.
\vs 3Er 12:5
Заповеди эти полезны для
тех, которые хотят покаяться; если не будут исполнять их, то тщетным будет их
покаяние.
\vs 3Er 12:6
Итак, вы, кающиеся,
отриньте от себя лукавства этого века, губящие вас. Облекитесь же во всякую
добродетель, чтобы вы могли соблюсти эти заповеди, и ничего не прибавляйте к
грехам вашим.
\vs 3Er 12:7
Ибо если снова не будете
грешить, то загладите прежние грехи. Поступайте по заповедям моим и будете
жить с Богом. Все это мною наказано вам.
\vs 3Er 12:8
После этих слов он
продолжал: пойдем в поле, и я покажу тебе пастухов овец.
\vs 3Er 12:9
Пойдем, господин,~--- согласился я.
\vs 3Er 12:10
Пошли мы и в поле увидели
молодого пастуха, одетого в богатые одежды багряного цвета; стадо его было
многочисленно, и ухоженные овцы весело резвились в травах. И сам пастух
радовался на свое стадо и с довольным лицом ходил около овец.

\vs 3Er 13:1
Ангел указал мне на
пастуха и сказал: это~--- ангел наслаждения и лжи, он изводит души рабов Божьих,
отвращая их от истины, обольщая злыми пожеланиями;
\vs 3Er 13:2
и они забывают заповеди
живого Бога и живут в роскоши и суетных удовольствиях, и этот злой ангел губит
их~--- некоторых до смерти, а некоторых до растления.
\vs 3Er 13:3
Господин,~--- спросил я,~--- как понять <<до смерти>>
и что значит <<до растления>>?
\vs 3Er 13:4
Слушай. Овцы, которых ты
видел резвящимися, это те, которые навсегда отреклись от Бога и предались
удовольствиям этого века.
\vs 3Er 13:5
Поэтому им нет возврата к
жизни через покаяние, ибо они к другим своим преступлениям прибавили еще
больше~--- нечестиво хулили имя Господа. Жизнь таких людей подобна смерти.
\vs 3Er 13:6
А овцы, которые не скакали
по полю, а скучно паслись, означают тех, которые хоть и предавались
наслаждениям и удовольствиям, но не возводили хулы против Господа: они не
отошли от истины, и для них есть еще покаяние, посредством которого они спасут
жизнь.
\vs 3Er 13:7
В растлении есть некоторая
надежда на восстановление; а смерть имеет окончательную погибель.
\vs 3Er 13:8
Еще прошли мы немного, и
он показал мне большого пастуха, дикого на вид, одетого в белую козью шкуру, с
сумой на плечах, сучковатой и крепкой палкой и большим бичом в руках; лицо его
было суровое и грозное, так что становилось страшно.
\vs 3Er 13:9
Он принимал от юного
пастуха овец, которые жили в неге и наслаждении, но не скакали; он отгонял их
в местность скалистую и тернистую, и овцы, запутавшись в колючках, сильно
страдали, а пастух осыпал их ударами, гонял туда и сюда, не давая им покоя и
не позволяя где-либо остановиться.

\vs 3Er 14:1
Видя, что овцы
подвергаются побоям, терпят такие мучения и не находят покоя, я пожалел их и
спросил пастыря, кто этот безжалостный и жестокий пастух, не имеющий ни
малейшего сострадания к овцам.
\vs 3Er 14:2
Это,~--- ответил пастырь,~--- ангел наказания; он из праведных ангелов, но приставлен для наказания. Ему
вверяются те, которые уклонились от Бога и предались похотям и удовольствиям
этого века; и он наказывает их, как они того заслуживают, различными жестокими
мучениями.
\vs 3Er 14:3
Расскажи мне, господин,~---
попросил я,~--- что это за мучения, какого рода они?
\vs 3Er 14:4
Слушай: эти различные
наказания и мучения~--- те, которые люди терпят в своей ежедневной жизни. Одни
терпят убытки, другие~--- бедность, иные~--- различные болезни,
некоторые~--- непостоянство в жизни, другие подвергаются обидам от людей недостойных и
многим иным неприятностям.
\vs 3Er 14:5
Очень многие с
непостоянными намерениями принимаются за различные дела, но ничто им не
удается, и жалуются они, что не имеют успеха в своих начинаниях; не приходит
им мысль, что они творят худые дела, но жалуются на Господа.
\vs 3Er 14:6
После того, как натерпятся
они всякой скорби, они предаются мне для доброго увещевания, укрепляются в
вере в Господа и в остальные дни жизни своей служат Господу чистым сердцем.
\vs 3Er 14:7
И когда начнут они каяться
в преступлениях, тогда на сердце их приходят беззаконные дела их и они воздают
славу Господу, говоря, что Он~--- Судия праведный и что они всё претерпели
достойно по делам своим.
\vs 3Er 14:8
И в остальное время служат
Богу чистым сердцем и имеют успех во всех делах своих, получая от Бога всё,
чего ни попросят; и тогда благодарят Бога, что вручены мне, и уже не
подвергаются более никакой жестокости.

\vs 3Er 15:1
И захотел я узнать,
столько ли времени мучаются оставившие страх Божий, сколько наслаждались
удовольствиями, и спросил пастыря об этом.
\vs 3Er 15:2
Столько же времени и
мучаются,~--- ответил он.
\vs 3Er 15:3
Мало они мучаются, надобно
бы предавшимся удовольствиям и забывшим Бога терпеть наказания в семь раз
более.
\vs 3Er 15:4
Неразумен ты,~--- сказал он,
и не понимаешь силы наказания.
\vs 3Er 15:5
Господин, если бы я
понимал, то и не просил бы тебя объяснить мне.
\vs 3Er 15:6
Слушай,~--- сказал он,~--- какова сила того и другого~--- наслаждения и наказания. Один час наслаждения
ограничивается своим протяжением, а один час наказания имеет силу тридцати
дней.
\vs 3Er 15:7
Кто один день предавался
наслаждению и удовольствию, тот будет мучиться один день, но день мучения
будет стоить целого года.
\vs 3Er 15:8
Следовательно, сколько
дней кто наслаждается, столько лет мучится. Видишь,~--- заключил он,~--- что время
мирского наслаждения и обольщения очень кратко, а время наказания и мучения
велико.

\vs 3Er 16:1
Я сказал ему: не совсем
понимаю относительно времени наслаждения и наказания, объясни мне лучше.
\vs 3Er 16:2
Он ответил: неразумие твоё
упорно остается с тобою, и ты не хочешь очистить сердце своё и служить Богу.
Смотри, чтобы не оказаться тебе неразумным, когда исполнится время.
\vs 3Er 16:3
А теперь слушай, если
желаешь понять. Кто один день предавался удовольствиям и делал, что было
угодно душе его, тот исполняется великим неразумием и наутро не понимает своих
действий и не помнит, что делал накануне, ибо наслаждение и обольщение не
имеют никакой памяти по причине неразумия, которым человек исполняется.
\vs 3Er 16:4
Но когда на один день
придет человеку наказание и мучение, то он страдает целый год, потому что
наказание и мучение имеют великую память.
\vs 3Er 16:5
Страдающий в течение
целого года вспоминает и о суетном наслаждении и сознаёт, что за него он
терпит зло.
\vs 3Er 16:6
Таким-то образом
наказываются те, которые предались наслаждению и обольщению; потому что,
наделенные жизнью, сами себя предали смерти.
\vs 3Er 16:7
Я спросил: господин, какие
удовольствия вредны?
\vs 3Er 16:8
Любое дело,~--- ответил он,
доставляет удовольствие человеку, если он выполняет его с приятностью.
\vs 3Er 16:9
Ибо и гневливый, исполняя
свое дело, получает удовольствие, и расово смешивающийся, и пьяница, и
клеветник, и лжец, и любостяжательный человек, и хищник, и всякий совершающий
что-либо подобное удовлетворяет свою страсть и наслаждается своим делом. Все
эти наслаждения вредны рабам Божьим, и за них-то они страдают и терпят
наказания.
\vs 3Er 16:11
Но есть также
удовольствия, спасительные для людей: многие, совершая добрые дела, получают
удовольствие, находя в них для себя сладость. Это удовольствие полезно рабам
Божьим и приготовляет жизнь таким людям.
\vs 3Er 16:12
А те, о которых сказано
прежде, заслуживают наказания и мучения, и те, которые будут нести их и не
покаются в своих преступлениях, обрекут себя на смерть.

\chhdr{Подобие 7-е.}
\vs 3Er 17:1
Спустя несколько дней я встретил пастыря на том поле, на
котором прежде видел пастухов,
\vs 3Er 17:2
и спросил он меня: чего ты
ищешь?
\vs 3Er 17:3
Я пришел, господин,
просить тебя, чтобы ты приказал удалиться из моего дома пастырю,
приставленному для наказания, потому что он сильно поражает меня.
\vs 3Er 17:4
Он сказал мне в ответ:
необходимо пережить тебе бедствия и скорби, потому что так заповедал тебе тот
славный ангел, который хочет испытать тебя.
\vs 3Er 17:5
Какое же зло, господин,
сделал я, что предан этому ангелу?
\vs 3Er 17:6
Слушай,~--- сказал он. Ты
имеешь очень много грехов, но не столь много, чтобы следовало тебя предать
этому ангелу;
\vs 3Er 17:7
но домочадцы твои
совершили великие грехи и преступления, и тот славный ангел прогневался на их
дела и повелел понести тебе наказание некоторое время, чтобы и они покаялись в
своих прегрешениях и очистились от всякой скверны этого века. И когда они
покаются и очистятся, тогда удалится от тебя ангел наказания.
\vs 3Er 17:8
Я сказал ему: господин,
если они так вели себя, что рассердили славного ангела, в чем же моя вина?
\vs 3Er 17:9
Он отвечал: они не могут
быть наказаны, если ты, глава всего дома, не подвергнешься наказанию. Ибо всё,
что претерпишь ты, неизбежно претерпят и они, а при твоем благополучии они не
могут испытать никакого мучения.
\vs 3Er 17:10
Но теперь, господин,~--- сказал я,~--- они уже покаялись от всего сердца своего.
\vs 3Er 17:11
Знаю, что они покаялись
от всего сердца. Но не думаешь ли ты, что тотчас отпускаются грехи кающихся?
Нет, кающийся должен помучить свою душу, смириться во всяком деле своем и
перенести многие и различные скорби.
\vs 3Er 17:12
И когда перенесет всё,
что ему назначено, тогда, конечно, Тот, Который всё сотворил и утвердил,
подвигнется к нему Своею милостью и даст ему спасительное врачевание, и лишь
тогда, когда увидит, что сердце кающегося чисто от всякого злого дела.
\vs 3Er 17:13
А тебе и семейству твоему
пострадать теперь полезно. Нужно пострадать так, как повелел тот ангел
Господа, который мне предал тебя.
\vs 3Er 17:14
А ты лучше благодари
Господа, что Он удостоил предварительно открыть тебе наказание, чтобы, наперед
зная о нём, ты стойко перенёс его.
\vs 3Er 17:15
И я просил его: господин,
будь со мною, и я легко перенесу всякое бедствие.
\vs 3Er 17:16
Я буду с тобою и даже
попрошу ангела наказания, чтобы он легче поражал тебя; впрочем, не долго ты
потерпишь бедствие и снова возвратишься в свое благосостояние, только пребывай
в смиренномудрии и повинуйся Господу от чистого сердца.
\vs 3Er 17:17
Пусть и дети твои, и весь
дом твой живут по заповедям, которые я тебе преподал,~--- и покаяние ваше может
сделаться твердым и чистым.
\vs 3Er 17:18
И если ты с семьей своей
соблюдешь мои заповеди, то удалится от тебя всякое бедствие; и от всех тех,
которые будут придерживаться этих заповедей, удалится всякое бедствие.

\chhdr{Подобие 8-е.}
\vs 3Er 18:1
Пастырь показал мне заросли ивы, покрывшие поля и горы, в
тень которых пришли все призванные в имени Господа.
\vs 3Er 18:2
И подле этой ивы стоял славный, весьма высокий ангел, он большим серпом срезал
с ивы ветки и раздавал их народу.
\vs 3Er 18:3
После того, как все получили ветки, ангел положил серп, но дерево осталось
таким же целым, каким я видел его прежде. Очень я удивился этому,
\vs 3Er 18:4
а пастырь сказал: не удивляйся, что дерево осталось невредимо после того, как
срезано было с него столько веток. Подожди, что будет дальше, и станет
понятным тебе, что всё это означает.
\vs 3Er 18:5
Ангел, раздававший ветки,
потребовал их назад. В том же порядке, в каком получали, он подзывал людей:
они подходили и возвращали ветки.
\vs 3Er 18:6
Ангел Господень принимал
их и рассматривал. От некоторых он получал сухие, как бы изъеденные молью
ветки, и тем он повелел стать отдельно; те, которые вернули ветки сухие, но не
тронутые молью, тоже стали отдельно.
\vs 3Er 18:7
Особо стали и те, кто
принес ветки полусухие и с трещинами, и те, чьи ветки были наполовину сухие,
наполовину зеленые.
\vs 3Er 18:8
Некоторые возвращали ветки
на две трети сухими, а на треть~--- зелеными; а некоторые~--- наоборот: на две
трети зелеными и на треть~--- сухими. Ангел их также поставил отдельно.
\vs 3Er 18:9
Иные подавали ветки
полностью зеленые, и только малая часть их, самая верхушка, была сухая, и они
были потрескавшиеся.
\vs 3Er 18:10
А в других ветках было
совсем мало зеленого.
\vs 3Er 18:11
А у большинства людей
были такие же зеленые ветки, какими они их и получили; ангел весьма радовался
им.
\vs 3Er 18:12
Иные отдавали ветки
зелеными и с молодыми побегами, ангел принимал их также с большим
удовольствием.
\vs 3Er 18:13
У некоторых зеленые ветки
были и с новыми отростками, и с плодами на них. Мужи, возвращающие такие
ветки, приходили с очень довольным видом, и сам ангел был весьма весел, и
пастырь тоже радовался.

\vs 3Er 19:1
Потом ангел Господа велел
принести венцы.
\vs 3Er 19:2
Принесены были венцы,
словно сплетенные из пальмовых листьев, и ангел надел их на тех мужей, ветки
которых были с отростками и плодами, и велел им идти в башню;
\vs 3Er 19:3
и других мужей, ветки
которых были зелены и с побегами, но без плодов, послал туда же, дав им
печать.
\vs 3Er 19:4
На всех входивших в башню
была одежда, белая как снег.
\vs 3Er 19:5
В ту же башню послал он и
тех, которые возвратили свои ветки такими же зелеными, как приняли, дав им
печать и белую одежду.
\vs 3Er 19:6
По окончании этого он
обратился к пастырю: я пойду, а ты впусти их внутрь стен, на то место, какое
каждый заслужил, но прежде рассмотри внимательно их ветки; следи, чтобы
кто-нибудь не миновал тебя; если же кто пройдет мимо, я обличу их перед
алтарем.
\vs 3Er 19:7
Он удалился, после чего
пастырь сказал мне: возьмем у них ветки и посадим их в землю, может быть,
некоторые из них зазеленеют снова?
\vs 3Er 19:8
Я удивился: господин,
каким образом могут снова зазеленеть ветки, которые уже засохли?
\vs 3Er 19:9
Он ответил мне: это дерево
ива, и оно всегда любит жизнь: поэтому, если эти ветки будут посажены и
получат чуть-чуть влаги, очень многие из них опять зазеленеют.
\vs 3Er 19:10
Попробую полью их водой,
и если какая из них сможет ожить, порадуюсь за неё; если же нет, по крайней
мере, видно будет, что я не был небрежен.
\vs 3Er 19:11
Потом пастырь приказал
мне позвать их в том порядке, в каком они стояли; подошли они и передали свои
ветки. Получив их, пастырь каждую посадил по порядку.
\vs 3Er 19:12
И, рассадив, так обильно
поливал их водою, что вода полностью покрыла их.
\vs 3Er 19:13
Полив, он сказал: пойдем,
а через несколько дней возвратимся и осмотрим все ветки. Ибо Сотворивший это
дерево хочет, чтобы были живы все происшедшие от него ветки.
\vs 3Er 19:14
А я надеюсь, что после
того, как эти ветки политы водою, очень многие из них оживут, напоенные
влагою.

\vs 3Er 20:1
Я попросил: господин,
объясни мне, что означает это дерево; я недоумеваю, почему оно остается целым:
ведь срезано с него столько веток, но не видно, чтобы от него что-нибудь
убавилось?
\vs 3Er 20:2
Слушай,~--- сказал он. Это
большое дерево, покрывающее поля и горы и всю землю, означает Закон Божий,
данный всему миру.
\vs 3Er 20:3
Закон этот есть Сын Божий,
проповеданный во всех концах земли. Люди, стоящие под сенью его, означают тех,
которые услышали проповедь и уверовали в Него.
\vs 3Er 20:4
Величественный и сильный
ангел есть Михаил, который имеет власть над этим народом и управляет им: он
насаждает Закон в сердцах верующих и наблюдает за теми, которым дал Закон,
соблюдают ли они его.
\vs 3Er 20:5
У каждого есть ветки:
ветки означают также Закон Господа.
\vs 3Er 20:6
Видишь, многие из них
сделались негодными, и ты узнаешь всех тех, которые не соблюли Закона, и
увидишь место каждого из них.
\vs 3Er 20:7
Почему же, господин, одних
Он отослал в башню, а других здесь оставил, при тебе?
\vs 3Er 20:8
Те, которые преступили
Закон, от Него принятый, оставлены в моей власти, чтобы покаялись в своих
преступлениях; а которые удовлетворили Закону и его соблюли, находятся под
собственною Его властью.
\vs 3Er 20:9
Кто же, господин, те,
которые увенчаны и вошли в башню?
\vs 3Er 20:10
Он ответил: это те,
которые вели борьбу с дьяволом и победили его; те, которые, соблюдая Закон,
пострадали за него;
\vs 3Er 20:11
другие, которые
возвратили ветки зелеными и с отростками, но без плодов,~--- это те, которые,
хотя и потерпели мучение за тот Закон, но не вкусили смерти и не отреклись от
своего Закона;
\vs 3Er 20:12
те же, которые возвратили
зелеными, какими и взяли, суть кроткие и праведные, которые жили с чистым
сердцем и соблюли заповеди Божии.
\vs 3Er 20:13
Остальное ты узнаешь
тогда, когда пересмотрю ветки, которые я посадил в землю и полил.

\vs 3Er 21:1
Через несколько дней мы
возвратились туда, и пастырь сел на месте того ангела, а я стал подле него, и
он велел мне подпоясаться полотенцем и помогать ему.
\vs 3Er 21:2
Я подпоясался чистым
платом, сделанным из мешка. Видя, что я готов служить ему,
\vs 3Er 21:3
он сказал: зови тех мужей,
ветки которых посажены в землю, в том порядке, в каком каждый их подавал.
\vs 3Er 21:4
И отправился я в поле,
созвал всех, и они стали на свои места. Пусть каждый вынет свою ветку и подаст
мне,~--- указал он.
\vs 3Er 21:5
Прежде всего подали те, у
которых были ветки сухие и гнилые. И так как они опять оказались загнившими и
сухими, то он повелел им стать отдельно.
\vs 3Er 21:6
После подали те, у которых
ранее они были сухие, но не гнилые. Одни из них подали ветки зеленые, а другие
сухие и загнившие, как бы тронутые молью.
\vs 3Er 21:7
Тем, которые подали
зеленые, велел он стать отдельно; а тем, которые подали сухие и загнившие,
велел стать вместе с первыми.
\vs 3Er 21:8
Потом подали те, чьи были
полузасохшие и с трещинами; многие из них принесли ветки зеленые и без трещин;
а некоторые~--- зеленые, имеющие побеги и даже плоды~--- как те, которые
увенчанные вошли в башню;
\vs 3Er 21:9
другие подали сухие и
поврежденные, иные сухие, но не гнилые, а некоторые полусухие и с трещинами,
какими и прежде были.
\vs 3Er 21:10
И всех их пастырь
разделил на группы, повелел каждой стать отдельно.

\vs 3Er 22:1
Потом принесли ветки те, у
которых они были хотя зеленые, но с трещинами: все они подали их теперь
зелеными и стали на своем месте, и пастырь радовался за них, что все они
оправились и заживили свои трещины.
\vs 3Er 22:2
Подали и те, которые
прежде имели ветки наполовину сухие; ветки некоторых из них оказались все
зелеными, других~--- полусухими, иных~--- сухими и поврежденными,
а иных~--- зелёными и с отростками.
\vs 3Er 22:3
Потом подали те, у которых
ветки на две трети были зеленые и на треть сухие; многие из них подали ветки
зеленые, многие полусухие, прочие же сухие и гнилые.
\vs 3Er 22:4
Далее подали те, у которых
до того ветки на две трети были сухие, а на треть зеленые; из них многие
подали полусухие, некоторые сухие и гнилые, другие полусухие и с трещинами, а
иные зеленые.
\vs 3Er 22:5
Потом подали те, у которых
ветки были зелены, но немного и сухи и с трещинами; из них некоторые
возвратили ветки зеленые, другие же зеленые и с побегами; и они отошли на свое
место.
\vs 3Er 22:6
Наконец, у тех, у которых
в ветках было немного зелени, а остальное засохло, ветки большею частью
оказались зелеными, с отростками и даже с плодом на них, а остальные были
зеленые. Этими ветками пастырь весьма был доволен.
\vs 3Er 22:7
И каждого он отправлял на
своё место.

\vs 3Er 23:1
Пересмотрев все ветки,
сказал мне пастырь: я говорил тебе, что дерево это любит жизнь. Видишь, многие
покаялись и получили спасение.
\vs 3Er 23:2
Вижу, господин.
\vs 3Er 23:3
Знай же,~--- продолжал он,~--- велики и славны благость и милость Господа, Который дал дух, способный
покаяться.
\vs 3Er 23:4
Почему же, господин,~--- спросил я,~--- не все покаялись?
\vs 3Er 23:5
Он ответил: Господь дал
покаяние тем, чьи сердца, он видел, будут чисты и кто будет служить Ему
усердно и праведно.
\vs 3Er 23:6
А тем, у которых
чувствовал лукавство, и неправду, и притворное к Нему обращение, не дал
покаяния, чтобы они снова не осквернили имени Его.
\vs 3Er 23:7
Теперь, господин, объясни
мне, что означает каждый из тех, кто возвратил ветки, и где его место, чтобы
узнали об этом уверовавшие, которые получили печать, но сокрушили её и не
сохранили в целости
\vs 3Er 23:8
и, дабы, познав дела свои,
покаялись и, приняв от тебя печать, воздали славу Господу, что подвигся Он к
ним Своею милостью, и послал тебя для обновления душ их.
\vs 3Er 23:9
Слушай,~--- сказал он. У
кого ветки найдены сухими и гнилыми, как бы поврежденными тлёю,~--- это суть
отступники и предатели Церкви, которые во грехах своих хулили Господа и
постыдились имени Его, на них призванного: все они умерли для Бога.
\vs 3Er 23:10
И ты видишь, что никто из
них не покаялся, и они презрели слова Божьи, которые я заповедал тебе; от этих
людей отступила жизнь.
\vs 3Er 23:11
Равным образом недалеко
от них те, которые возвратили ветки сухими, хотя не гнилыми, ибо они были
лицемеры, вводили чуждые учения и совращали рабов Божьих, особенно тех,
которые согрешили, не дозволяя им возвращаться к покаянию, но внушая им
вредные мысли.
\vs 3Er 23:12
Они имеют надежду
покаяния; и ты видишь, что многие из них уже покаялись после того, как я
возвестил им мои заповеди, и еще покаются.
\vs 3Er 23:13
Те, которые не покаются,
потеряли жизнь свою; те же, которые покаялись, сделались добрыми и
местопребыванием их стали первые стены, а некоторые вошли даже внутрь башни.
\vs 3Er 23:14
Итак, видишь, покаяние
грешников несет в себе жизнь, а нераскаянность~--- смерть.

\vs 3Er 24:1
Послушай и о тех, которые
вернули ветки полусухие и с трещинами,~--- говорил пастырь далее.
\vs 3Er 24:2
Те, у которых ветки были
только полусухие,~--- это сомневающиеся: они ни живы, ни мертвы; а те, которые
подали ветки полусухие и с трещинами,~--- это сомневающиеся и вместе с тем
злоязычные, которые поносят отсутствующих, никогда не живут в мире, но
постоянно находятся в раздоре.
\vs 3Er 24:3
Впрочем, и им есть
покаяние. Видишь, и из них некоторые покаялись.
\vs 3Er 24:4
Из них немедленно
покаявшиеся найдут себе место в башне, а те, которые позднее покаялись, будут
обитать на стенах.
\vs 3Er 24:5
Те же, которые не
покаялись, но остались при своих делах, обретут погибель.
\vs 3Er 24:6
Те, которые подали ветки
зеленые, но с трещинами, всегда были верными и добрыми, хотя имеют между собою
зависть и соперничество о первенстве и достоинстве: только глупы люди,
спорящие между собою о первенстве.
\vs 3Er 24:7
Впрочем, они были добры,
послушались моих заповедей, исправились и скоро покаялись, потому и место их в
башне.
\vs 3Er 24:8
Если же кто-нибудь из них
возвратится к раздору, будет изгнан из башни и погубит жизнь свою.
\vs 3Er 24:9
Ибо жизнь званных Богом
состоит в соблюдении заповедей Господа: в этом жизнь, а не в первенстве или
каком-либо достоинстве.
\vs 3Er 24:10
Чрез терпение и смирение
духа люди получат жизнь от Господа, а пренебрегающие Законом приобретут себе
смерть.

\vs 3Er 25:1
Те, у которых ветки
наполовину сухи, наполовину зелены,~--- это привязанные к мирским занятиям и
отчуждавшиеся от общения со святыми, и потому половина их жива, половина
мертва.
\vs 3Er 25:2
И из них многие,
послушавшись заповедей моих, покаялись и получили место в башне; некоторые же
вовсе отпали.
\vs 3Er 25:3
Для них нет покаяния,
потому что они хулили Господа и наконец отвергли Его, и за это нечестие они
потеряли жизнь свою.
\vs 3Er 25:4
Но многие из них
двоедушествовали: этим еще есть покаяние, и если вскоре покаются, будут иметь
жилище в башне; если позднее~--- будут обитать на стенах; если же совсем не
покаются~--- потеряют жизнь свою.
\vs 3Er 25:5
Те, у которых ветки на две
трети были зеленые, а на треть сухие, означают тех, которые, будучи различным
образом совращены, отреклись от Господа:
\vs 3Er 25:6
из них многие покаялись и
уже получили место в башне; а иные навсегда отпали от Бога и совсем потеряли
жизнь.
\vs 3Er 25:7
А некоторые из них
двоедушествовали и возбуждали раздоры: им еще есть покаяние, если вскоре
покаются и откажутся от своих удовольствий; если же останутся при своих делах,
то приготовят себе смерть.

\vs 3Er 26:1
Подавшие свои ветки на две
трети сухими, а на треть зелеными суть верные, но, обогатившись и обретя славу
среди народов, они впали в большую гордость, стали высокомерными, оставили
истину и не имели общения с праведными, но жили вместе с народами, и эта жизнь
казалась им приятнее; от Бога, впрочем, они не отпали и сохраняли веру; только
не творили дела веры.
\vs 3Er 26:2
Многие из них уже
покаялись и стали обитать в башне.
\vs 3Er 26:3
Другие, живя с народами и
набравшись надменного тщеславия у них, совершенно отошли от Бога, предавшись
делам народов: такие люди причислились к народам.
\vs 3Er 26:4
Некоторые же из них начали
колебаться, не надеясь спастись по делам, ими совершаемым; другие пришли в
сомнение и стали возбуждать несогласия.
\vs 3Er 26:5
И тем и другим еще есть
покаяние, но покаяние их должно быть немедленным, чтобы осталось для них место
в башне.
\vs 3Er 26:6
А тем, которые не
раскаются, пребывая в своих удовольствиях, скоро предстоит смерть.

\vs 3Er 27:1
Те, которые подали ветки
зеленые, за исключением их сухих верхушек, и с трещинами, те всегда были
добрыми, верными и славными у Бога, но согрешили несколько раз по причине
небольших удовольствий и мелких несогласий, которые имели между собою.
\vs 3Er 27:2
Услышав слова мои, очень
многие тотчас покаялись, и место их стало в башне.
\vs 3Er 27:3
Некоторые из них пришли в
сомнение, а некоторые, сверх того, произвели большой раздор. Для таких есть
надежда покаяния, потому что всегда были добрыми и едва ли кто из них умрет.
\vs 3Er 27:4
Те же, которые подали
сухие ветки с зелеными верхушками, они только уверовали в Бога, но творили
беззаконие; впрочем, они никогда не отступали от Бога, но всегда охотно носили
Его имя и с любовью принимали рабов Божьих в дома свои.
\vs 3Er 27:5
Услышав о покаянии, они
немедленно покаялись и делают всякую добродетель и правду.
\vs 3Er 27:6
Некоторые из них
претерпели смерть, а другие охотно перенесли несчастия,
помня о делах своих,~--- всем таковым место будет в башне.

\vs 3Er 28:1
Окончив объяснение всех
веток, он повелел мне: пойди и скажи всем, чтобы покаялись и жили для Бога,
потому что Господь по Своему милосердию послал меня дать всем покаяние, даже и
тем, которые по делам своим не заслуживают спасения. Но терпелив Господь и
хочет, чтобы спаслись призванные Его Сыном.
\vs 3Er 28:2
Я надеюсь, господин,~--- ответил я,~--- что все услышавшие это покаются. Ибо я убежден, что всякий
обратится к покаянию, познав дела свои и убоявшись Бога.
\vs 3Er 28:3
Все те, которые от всего
сердца покаются и очистятся от всех неправедных дел, о которых говорилось
прежде, и не приумножат еще чем-либо свои преступления, получат от Господа
прощение прежних грехов своих, если не усомнятся в этих заповедях моих и будут
жить с Богом.
\vs 3Er 28:4
И ты ходи в этих заповедях
и будешь жить с Богом; и все, кто только будет верно исполнять их, будут жить
с Богом.
\vs 3Er 28:5
Показав мне всё это, он
пообещал: остальное я покажу тебе спустя несколько дней.

\chhdr{Подобие 9-е.}
\vs 3Er 29:1
После того как я написал заповеди и притчи пастыря, ангела
покаяния, он пришел ко мне и сказал: я хочу показать тебе всё, что показал
тебе Дух Святой, Который беседовал с тобою в образе Церкви: Дух тот есть Сын
Божий.
\vs 3Er 29:2
И так как ты был слаб телом, то не было открываемо тебе через ангела, доколе
ты не утвердился духом и не укрепился силами, чтобы мог видеть ангела.
\vs 3Er 29:3
Тогда Церковью показано было тебе строение башни хорошо и величественно; но ты
видел, как было показано тебе всё девою.
\vs 3Er 29:4
А теперь ты получишь откровение через ангела, но от того же Духа. Ты должен
тщательно всё узнать от меня; ибо для того и послан я тем досточтимым ангелом
обитать в доме твоём, чтобы ты рассмотрел всё хорошо, ничего не страшась, как
прежде.
\vs 3Er 29:5
И повел он меня в Аркадию,
на гору, имеющую форму груди, и сели мы на её вершине. И показал он мне
большое поле, которое окружали двенадцать гор, не похожих одна на другую.
\vs 3Er 29:6
Первая из них была черная
как сажа. Вторая была голая, без растений. Третья заросла сорняками и
терниями. На четвертой были растения полузасохшие, с зеленой верхушкой и
мёртвым стеблем, а некоторые растения совсем засохли от солнечного жара.
\vs 3Er 29:7
Пятая гора была скалистая,
но на ней зеленели растения. Шестая гора была с расселинами, в иных местах
малыми, в других большими; в этих расселинах были растения, но не цветущие, а
слегка увядшие.
\vs 3Er 29:8
На седьмой горе цвели
растения, и была она плодородна: всякий скот и птицы небесные собирали там
корм, и чем более питались они на ней, тем обильнее росли растения.
\vs 3Er 29:9
Восьмую гору сплошь
покрывали источники, и из этих источников утоляли жажду твари Божьи. Девятая
гора вовсе не имела никакой воды и вся была обнажена: на ней обитали ядовитые
змеи, гибельные для людей.
\vs 3Er 29:10
Десятая гора вся была
затенена огромными деревьями, на ней растущими, и в тени лежал скот, отдыхая и
пережевывая жвачку.
\vs 3Er 29:11
На одиннадцатой горе тоже
во множестве росли деревья, и они изобиловали разными плодами, и видевший их
желал вкусить этих плодов. Двенадцатая гора, вся белая, имела вид самый
приятный, всё было на ней прекрасно.

\vs 3Er 30:1
В середине поля он показал
мне огромный белый камень; камень этот, квадратный по форме, был выше тех гор,
так что мог бы держать всю землю.
\vs 3Er 30:2
Он был древний, но имел
высеченную дверь, которая казалась недавно сделанною. Дверь эта сияла ярче
солнца, так что я поразился ее блеску
\vs 3Er 30:3
Двенадцать дев стояли
возле двери, по четырем сторонам её, в середине попарно.
\vs 3Er 30:4
Четверо из них, стоявшие
по углам двери, показались мне самыми великолепными, но и остальные были
прекрасны.
\vs 3Er 30:5
Веселые и радостные, эти
девы одеты были в полотняные туники, красиво подпоясанные; их правые плечи
были обнажены, словно девы намеревались нести какую-то ношу.
\vs 3Er 30:6
Я залюбовался этим
величественным и дивным зрелищем, но в то же время недоумевая, что девы,
будучи столь нежны, стояли мужественно, будто готовясь понести на себе целое
небо.
\vs 3Er 30:7
И когда размышлял я так,
пастырь сказал мне: что размышляешь ты и недоумеваешь и сам на себя навлекаешь
заботу? Чего не можешь понять, за то не берись, но проси Господа, чтобы
вразумил понять это.
\vs 3Er 30:8
Что за тобою, того не
можешь видеть; а видишь, что перед тобою. Чего не можешь видеть, то оставь и
не мучь себя.
\vs 3Er 30:9
Владей тем, что видишь, о
прочем же не беспокойся. Я объясню тебе всё, что покажу; а теперь смотри, что
будет дальше.

\vs 3Er 31:1
И вот увидел я, что пришли
шесть высоких и почтенных мужей, и все были похожи один на другого; они
призвали множество других мужей, которые также были высоки, красивы и сильны.
\vs 3Er 31:2
И те шесть мужей приказали
строить башню над дверью.
\vs 3Er 31:3
Тогда мужи, которые пришли
для строительства башни, подняли великий шум и беготню около двери.
\vs 3Er 31:4
Девы, стоявшие при двери,
сказали им поспешить со строительством и сами протянули свои руки, как бы
готовясь что-нибудь брать у них.
\vs 3Er 31:5
Те шестеро приказали
доставать камни со дна и подносить их к башне. И подняты были десять камней
белых, квадратных, обтесанных.
\vs 3Er 31:6
Те шесть мужей подозвали
дев и приказали им носить все камни, которые должны были идти на
строительство, проходить через дверь и передавать камни строителям башни.
\vs 3Er 31:7
И тотчас же девы начали
возлагать друг на друга первые камни, извлеченные со дна, и носить их вместе
по одному камню.

\vs 3Er 32:1
Как стояли девы около
двери, так они и носили: те, которые казались сильнее, брались за углы камня,
а другие держали по бокам.
\vs 3Er 32:2
И таким образом носили они
все камни, проходили через дверь, как было велено, и передавали строителям
башни; а те, принимая их, строили.
\vs 3Er 32:3
Башня строилась на большом
камне, над дверью. Те десять камней были положены в основание башни: камень же
и дверь держали на себе всю башню.
\vs 3Er 32:4
После извлекли со дна
другие двадцать пять камней, и они были принесены девами и использованы для
строительства башни.
\vs 3Er 32:5
После них подняли другие
тридцать пять, которые подобным же образом уложили в башню.
\vs 3Er 32:6
Затем подняли еще сорок
камней, и они все пошли на строительство этой башни.
\vs 3Er 32:7
Таким образом в основание
башни легло четыре ряда камней.
\vs 3Er 32:8
Когда закончились все
камни, которые брали со дна, немного отдохнули строители.
\vs 3Er 32:9
Потом те шесть мужей
приказали народу приносить для башни камни с двенадцати гор.
\vs 3Er 32:10
И стали мужи приносить со
всех гор камни обсеченные, различных цветов, и подавали их девам, а те
проносили их через дверь и подавали строителям.
\vs 3Er 32:11
И когда эти разнообразные
камни были положены в здание, то изменили свои прежние цвета и сделались
белыми и одинаковыми.
\vs 3Er 32:12
Но некоторые камни не
были передаваемы девами и не проносились через дверь, а подавались самими
мужами прямо в строение и не делались светлыми, а оставались такими, какими
клались.
\vs 3Er 32:13
Эти камни безобразно
смотрелись в здании башни. Увидев их, те шесть мужей приказали вынуть и
положить на то место, откуда их взяли.
\vs 3Er 32:14
И сказали они тем,
которые приносили эти камни: вы совсем не подавайте камней для строения, но
кладите их возле башни, чтобы девы проносили через дверь и подавали их, иначе
камни не смогут изменить цветов своих, так что не трудитесь понапрасну.

\vs 3Er 33:1
И кончились в тот день
работы, но башня не была завершена; строительство её должно было опять
возобновиться, и только на время сделана некоторая остановка.
\vs 3Er 33:2
Те шесть мужей приказали
строившим удалиться и отдохнуть немного; девам же повелели не отходить от
башни, чтобы охранять её.
\vs 3Er 33:3
После того как ушли все, я
спросил пастыря, почему не окончено здание башни.
\vs 3Er 33:4
Не может оно быть
завершено прежде, нежели придет господин башни и испытает это строение, чтобы,
если окажутся некоторые камни негодными, заменить их, ибо по его воле строится
эта башня,~--- отвечал он.
\vs 3Er 33:5
Господин,~--- попросил я,~---
я желал бы знать, что означает строение башни, а также узнать и об этом камне,
и о двери, и о горах, и о девах, и о камнях, извлеченных со дна и не
отёсанных, но сразу положенных в здание;
\vs 3Er 33:6
и почему сперва положены в
основание десять камней, потом двадцать пять, затем тридцать пять и, наконец,
сорок;
\vs 3Er 33:7
равно и о тех камнях,
которые положены были в строение, но потом вынуты и отнесены на свое место;
всё это, господин, объясни и успокой душу мою.
\vs 3Er 33:8
И сказал он мне: если не
будешь попусту любопытен, то всё узнаешь и увидишь, что дальше будет с этой
башней, и все притчи обстоятельно узнаешь.
\vs 3Er 33:9
Через несколько дней
пришли мы на то же самое место, где сидели прежде, и позвал он меня: пойдем к
башне, ибо господин её придет, чтобы испытать её.
\vs 3Er 33:10
И пришли мы к башне и
никого другого не нашли, кроме дев.
\vs 3Er 33:11
Пастырь спросил их, не
прибыл ли господин башни. И они ответили, что он скоро придет осмотреть это
здание.

\vs 3Er 34:1
И вот, спустя немного
времени, увидел я, что идет великое множество мужей, и в середине муж такого
величайшего роста, что он превышал саму башню;
\vs 3Er 34:2
окружали его шесть мужей,
которые распоряжались строительством, и все те, которые строили эту башню, и
сверх того еще очень многие славные мужи.
\vs 3Er 34:3
Девы, охранявшие башню,
поспешили к нему навстречу; облобызали его, и стали они вместе ходить вокруг
башни.
\vs 3Er 34:4
И он так внимательно
осматривал строение, что испытал каждый камень: по каждому камню он ударил
трижды тростью, которую держал в руке.
\vs 3Er 34:5
Некоторые камни после его
ударов сделались черны как сажа, некоторые шероховаты, другие потрескались,
иные стали коротки, некоторые ни черны, ни белы, другие неровны и не подходили
к прочим камням, иные покрылись множеством пятен. Так разнообразны были камни,
найденные негодными для здания.
\vs 3Er 34:6
Господин повелел убрать
все их из башни и оставить подле неё, а на место их принести другие камни.
\vs 3Er 34:7
И спросили его строившие:
с какой горы прикажешь принести камни и положить на место выброшенных?
\vs 3Er 34:8
Он запретил приносить с
гор, но велел носить с ближайшего поля.
\vs 3Er 34:9
Взрыли поле и нашли камни
блестящие, квадратные, а некоторые и круглые.
\vs 3Er 34:10
И все камни, сколько их
было на этом поле, были принесены и девами пронесены через дверь;
\vs 3Er 34:11
из них квадратные были
обтёсаны и положены на место выброшенных, а круглые не употреблены в здание,
ибо трудно и долго было их обсекать.
\vs 3Er 34:12
Их оставили около башни,
чтобы после обсечь и употребить в здание, потому как они были очень блестящи.

\vs 3Er 35:1
Окончив это,
величественный муж, господин этой башни, призвал пастыря и поручил ему камни,
не одобренные для здания и положенные около башни.
\vs 3Er 35:2
Тщательно очисти эти
камни,~--- велел он,~--- и положи в здание башни те, которые могут приладиться к
прочим, а неподходящие отбрасывай далеко в сторону.
\vs 3Er 35:3
Приказав это, он удалился
со всеми, с кем пришел к башне. Девы же остались около башни охранять её.
\vs 3Er 35:4
И спросил я пастыря: каким
образом эти камни могут снова пойти в здание башни, когда они уже найдены
негодными?
\vs 3Er 35:5
Он отвечал: я из этих
камней большую часть обсеку и использую для строения, и они придутся к прочим.
\vs 3Er 35:6
Господин,~--- сказал я,~--- каким образом, обсечённые, они могут занять то же самое место?
\vs 3Er 35:7
Те, которые кажутся
малыми, пойдут в середину здания; а большие лягут снаружи и будут их
удерживать.
\vs 3Er 35:8
Потом он сказал: пойдем и
через два дня возвратимся и, очистив эти камни, положим в здание. Ибо всё, что
находится около башни, должно быть очищено, а то вдруг случайно явится
господин, увидит, что нечисто около башни, и прогневается; тогда эти камни не
пойдут на строительство башни, и сочтет он меня нерадивым.
\vs 3Er 35:9
Спустя два дня, когда
пришли мы к башне, он сказал мне: рассмотрим все эти камни и узнаем, которые
из них могут идти в здание.
\vs 3Er 35:10
Рассмотрим, господин,~--- ответил я.

\vs 3Er 36:1
Сначала мы рассмотрели
черные камни. Они оказались такими же, какими были отложены от здания.
\vs 3Er 36:2
Он приказал отнести их от
башни и положить отдельно.
\vs 3Er 36:3
Потом он рассмотрел камни
шероховатые и многие из них велел обсечь и девам взять их и положить в здание;
\vs 3Er 36:4
и они, взяв их, положили в
середину башни.
\vs 3Er 36:5
Остальные же он велел
положить с черными камнями, потому что и они оказались черными.
\vs 3Er 36:6
Затем он рассмотрел камни
с трещинами и из них многие обсек и велел чрез дев отнести в здание: они были
положены снаружи, как более крепкие;
\vs 3Er 36:7
остальные же, по множеству
трещин, не могли быть обработанными и потому были удалены от здания башни.
\vs 3Er 36:8
Далее он рассмотрел камни,
которые были коротки: многие из них оказались черными, а некоторые с большими
трещинами, и он велел положить их с теми, которые были отброшены;
\vs 3Er 36:9
остальные же, очищенные и
обработанные, он велел использовать, и девы, взяв их, положили в середину
здания башни, потому что они были не так крепки.
\vs 3Er 36:10
Потом он рассмотрел камни
наполовину белые и наполовину черные: многие из них оказались черными, и он
велел их перенести к отброшенным.
\vs 3Er 36:11
Остальные же все были
найдены белыми и взяты девами и положены снаружи, будучи крепкими, так что
могли удерживать камни, помещенные в середине, ибо в них ничего не было
отсечено.
\vs 3Er 36:12
Затем он рассмотрел камни
неровные и крепкие: некоторые из них отбросил, потому что по причине твердости
нельзя было обработать их;
\vs 3Er 36:13
остальные же были
обсечены и положены девами в середину здания башни, как более слабые.
\vs 3Er 36:14
Далее он рассмотрел камни
с пятнами, и из них немногие оказались черными и были отброшены к прочим;
остальные же оказались белыми~--- они в целости были использованы девами для
строительства и уложены снаружи по причине их твердости.

\vs 3Er 37:1
Потом стал он
рассматривать камни белые и круглые и спросил меня, что делать с ними.
\vs 3Er 37:2
Не знаю, господин,~--- ответил я.
\vs 3Er 37:3
Значит, ты ничего не
можешь придумать насчет них?
\vs 3Er 37:4
Господин,~--- сказал я,~--- не
владею этим искусством, я не каменщик и ничего не могу придумать.
\vs 3Er 37:5
И сказал он: разве не
видишь, что они круглы? Если я захочу сделать их квадратными, то нужно очень
много от них отсекать, но необходимо, чтобы некоторые из них вошли в здание
башни.
\vs 3Er 37:6
Если необходимо,~--- сказал
я,~--- что же ты затрудняешься, не выбираешь, что хочешь, и не подгоняешь в это
здание?
\vs 3Er 37:7
И он выбрал камни большие
и блестящие и обсек их; а девы, взяв их, положили во внешних частях здания.
\vs 3Er 37:8
Остальные же были отнесены
на то же поле, откуда взяты, но не отброшены. Потому что,~--- объяснил пастырь,
несколько еще недостает башне для окончания; господину угодно, чтобы эти
камни пошли в здание башни, так как они очень белы.
\vs 3Er 37:9
Потом призваны были
двенадцать очень красивых женщин, одетых в черное, с обнаженными плечами и
распущенными волосами. Эти женщины казались деревенскими.
\vs 3Er 37:10
Пастырь приказал им взять
отброшенные от здания камни и отнести их на горы, откуда они были принесены.
\vs 3Er 37:11
И они с радостью подняли,
отнесли все камни и положили туда, откуда они взяты.
\vs 3Er 37:12
Когда же не осталось
возле башни ни одного камня, он сказал: обойдем башню и посмотрим, нет ли в
ней какого изъяна.
\vs 3Er 37:13
Обойдя башню, пастырь
увидел, что она прекрасна и построена безукоризненно, и очень развеселился.
\vs 3Er 37:14
И всякий залюбовался бы
постройкою, потому что не было видно ни одного соединения и башня казалась
высеченною из единого камня.

\vs 3Er 38:1
И я, ходя вместе с
пастырем, весьма был доволен таким прекрасным зрелищем.
\vs 3Er 38:2
И повелел он мне: принеси
известь и мелкие черепицы, чтобы мне исправить вид тех камней, которые опять
вынули из здания, ибо всё вокруг башни должно быть ровно и гладко.
\vs 3Er 38:3
И я всё принес, как
приказал он мне, и он добавил: послужи мне: это дело скоро окончится.
\vs 3Er 38:4
Он исправил вид тех камней
и приказал навести порядок около башни.
\vs 3Er 38:5
Тогда девы, взяв веники,
убрали всю грязь и полили водою~--- и место около башни стало красивым и
веселым.
\vs 3Er 38:6
Пастырь сказал мне: всё
очищено; если Господь придет посмотреть эту башню, не найдет ничего, за что бы
укорить нас,
\vs 3Er 38:7
и он хотел удалиться, но я
схватил его за суму и начал умолять его Господом, чтобы объяснил мне
показанное.
\vs 3Er 38:8
Мне нужно отдохнуть
немного, потом я всё объясню тебе,~--- пообещал он.~--- Дожидайся меня здесь.
\vs 3Er 38:9
Господин, что я здесь буду
один делать?
\vs 3Er 38:10
Ты не один,~--- отвечал он,
все девы с тобою.
\vs 3Er 38:11
Господин,~--- попросил я,~--- передай им меня. И он позвал их и сказал: поручаю вам его, пока не вернусь.
\vs 3Er 38:12
И так я остался один с
теми девами. И они были веселы и ласковы со мною, особенно же четыре из них,
превосходнейшие.

\vs 3Er 39:1
Девы сказали: сегодня
пастырь сюда не придет.
\vs 3Er 39:2
Что же я буду делать?
\vs 3Er 39:3
Подожди до вечера, может
быть, придет и будет говорить с тобою, если же не придет, пробудешь с нами,
доколе придет.
\vs 3Er 39:4
Буду дожидаться его до
вечера,~--- решил я,~--- если же не придет, пойду домой и возвращусь поутру.
\vs 3Er 39:5
Но они воспротивились: ты
нам перепоручен и не можешь уйти от нас.
\vs 3Er 39:6
Я спросил тогда: где я
останусь?
\vs 3Er 39:7
С нами,~--- ответили они,~--- ты уснешь, как брат, а не как муж, ибо ты~--- брат наш и после мы будем обитать
с тобою, потому что очень тебя полюбили.
\vs 3Er 39:8
Мне же стыдно было
оставаться с ними. Но та, которая из них казалась главною, обняла меня и
начала лобызать. И прочие, увидев это, тоже начали лобызать меня, как брата,
водить около башни и играть со мною.
\vs 3Er 39:9
Некоторые из них пели
псалмы, а иные водили хороводы. А я в молчании ходил с ними около башни, и
казалось мне, что я помолодел.
\vs 3Er 39:10
С наступлением вечера я
хотел уйти домой, но они удержали меня и не позволили уйти.
\vs 3Er 39:11
И так я провел с ними эту
ночь около башни. Они постлали на землю свои полотняные туники и уложили меня
на них, сами же ничего другого не делали, только молились.
\vs 3Er 39:12
И я с ними молился
непрерывно и столь же усердно, и девы радовались моему усердию. Так оставался
я с девами до следующего дня.
\vs 3Er 39:13
Потом пришел пастырь и
спросил их: вы не причинили ему никакой обиды?
\vs 3Er 39:14
И отвечали они: спроси
его самого.
\vs 3Er 39:15
Господин,~--- сказал я,~--- я
получил великое удовольствие оттого, что остался с ними.
\vs 3Er 39:16
Что ты ужинал?~--- спросил
он.
\vs 3Er 39:17
Я ответил: всю ночь,
господин, я питался словами Господа.
\vs 3Er 39:18
Хорошо ли они тебя
приняли?
\vs 3Er 39:19
Хорошо, господин.
\vs 3Er 39:20
Теперь что прежде всего
желаешь услышать?
\vs 3Er 39:21
Чтобы ты, господин,
объяснил мне, всё, что до этого показал.
\vs 3Er 39:22
Как желаешь,~--- сказал он,
так и буду объяснять тебе и ничего от тебя не скрою.

\vs 3Er 40:1
Прежде всего, господин,~--- попросил я,~--- объясни мне, что означают камень и дверь.
\vs 3Er 40:2
Камень и дверь,~--- сказал
он,~--- это Сын Божий.
\vs 3Er 40:3
Как же так, господин,~--- удивился я,~--- ведь камень древний, а дверь новая?
\vs 3Er 40:4
Слушай, неразумный, и
понимай. Сын Божий древнее всякой твари, так что присутствовал на совете Отца
Своего о создании твари.
\vs 3Er 40:5
А дверь новая потому, что
Он явился в последние дни, сделался новою дверью для того, чтобы желающие
спастись через неё вошли в царство Божье.
\vs 3Er 40:6
Ты видел, что камни через
дверь были пронесены в здание башни, а те, которые не пронесены через неё,
были возвращены на своё место.
\vs 3Er 40:7
Так,~--- продолжал он,~--- никто не войдет в царство Божье, если не примет имени Сына Божьего.
\vs 3Er 40:8
Ибо если бы ты захотел
войти в какой-либо город, окруженный стеною с одними только воротами, не мог
бы ты проникнуть в этот город иначе как только через эти ворота.
\vs 3Er 40:9
По-другому и быть не
может, господин,~--- согласился я.

\vs 3Er 41:1
Итак, как в этот город
можно войти только через ворота его, так и в царство Божье не попадет человек
иначе как только через имя Сына Божьего возлюбленного.
\vs 3Er 41:2
Видел ли ты множество
строящих этими духовными силами? Будут один дух и одно тело, и будет один цвет
одежд их; тот именно заслужит место в башне, кто будет носить имена этих дев.
\vs 3Er 41:3
Почему же, господин,~--- спросил я,~--- отброшены и забракованы были некоторые камни, тогда как и их
пронесли через дверь и передали через руки дев в здание башни?
\vs 3Er 41:4
Так как у тебя есть
обыкновение всё тщательно исследовать, то слушай и об отброшенных камнях.
\vs 3Er 41:5
Все они приняли имя Сына
Божьего и силу этих дев. Приняв эти дары Духа, они укрепились и были в числе
рабов Божьих, и стали у них один дух, одно тело и одна одежда, потому что они
были единомысленны и делали правду.
\vs 3Er 41:6
Но спустя некоторое время
они увлеклись теми красивыми женщинами, которых ты видел одетыми в черную
одежду с обнаженными плечами и распущенными волосами;
\vs 3Er 41:7
увидев их, они возжелали
их и облеклись их силою, а силу дев свергли с себя.
\vs 3Er 41:8
Поэтому они изгнаны из
дома Божьего и преданы тем женщинам. А не соблазнившиеся красотою их остались
в доме Божьем.
\vs 3Er 41:9
Вот тебе,~--- заключил он,~--- значение камней отброшенных.

\vs 3Er 42:1
Что если, господин,~--- продолжал я расспросы,~--- такие люди покаются, отринут пожелания тех женщин и,
вновь обратившись к девам, облекутся их силою,~--- то войдут ли они в дом Божий?
\vs 3Er 42:2
Войдут, если отвергнут
дела тех женщин и снова приобретут силу дев и будут ходить в делах их.
\vs 3Er 42:3
Для того и остановлено
строительство, чтобы они покаялись и вошли в здание башни; если же не
покаются, то другие займут их место, а они будут отвержены навсегда.
\vs 3Er 42:4
За всё это я возблагодарил
Господа, что Он, подвигнутый милостью ко всем призывающим Его имя, послал
ангела покаяния к ним, согрешившим против Него, и обновил души наши, уже
ослабевшие и не имеющие надежды на спасение, восстановив нас к жизни.
\vs 3Er 42:5
Теперь, господин,~--- сказал
я,~--- объясни мне, почему башня строится не на земле, но на камне и двери?
\vs 3Er 42:6
Ты спрашиваешь, потому что
неразумен.
\vs 3Er 42:7
Господин, я вынужден обо
всем тебя спрашивать, потому что совершенно не могу ничего понять, ведь всё
это так величественно и дивно, что людям трудно постичь.
\vs 3Er 42:8
Слушай,~--- сказал мне
пастырь.~--- Имя Сына Божьего велико и неизмеримо, и оно держит весь мир.
\vs 3Er 42:9
Если всё творение держится
Сыном Божьим,~--- спросил я,~--- то как думаешь, поддерживает ли Он тех, которые
призваны Им, носят имя Его и живут по Его заповедям?
\vs 3Er 42:10
Видишь, Он поддерживает
тех, которые от всего сердца носят Его имя. Он Сам служит для них основанием и
с любовью держит их, потому что они не стыдятся носить Его имя.

\vs 3Er 43:1
Открой мне, господин,~--- попросил я,~--- имена дев и тех женщин, облеченных в черную одежду
\vs 3Er 43:2
Слушай. Из тех, которые
могущественнее и стоят по углам двери, первая зовется Верою, вторая~--- Воздержанием, третья~--- Мощью, четвертая~--- Терпением.
\vs 3Er 43:3
Прочие же, которые в
середине, имеют следующие имена: Простота, Невинность, Целомудрие, Радость,
Правдивость, Разумение, Согласие и Любовь.
\vs 3Er 43:4
Носящие эти имена и имя
Сына Божьего могут войти в царство Божье.
\vs 3Er 43:5
Слушай теперь имена
женщин, одетых в черную одежду. Четыре самые могущественные: первую зовут
Вероломством, вторую~--- Неумеренностью, третью~--- Неверием,
четвертую~--- Сластолюбием.
\vs 3Er 43:6
Имена следующих за ними:
Печаль, Лукавство, Похоть, Гнев, Ложь, Неразумие, Злословие, Ненависть.
\vs 3Er 43:7
Раб Божий, носящий такие
имена, хоть и увидит царство Божье, но не войдет в него!
\vs 3Er 43:8
Тогда решил узнать я у
пастыря, что означают камни, которые со дна подняты для здания.
\vs 3Er 43:9
Первые 10,~--- ответил он,~--- положенные в основание,
означают первый век,
следующие 25~--- второй век мужей праведных;
\vs 3Er 43:35
означают пророков и служителей
Господа; 40 же означают апостолов и учителей Евангелия Сына Божьего.
\vs 3Er 43:10
Почему же, господин, девы
подавали и эти камни в здание башни, пронеся их через дверь?
\vs 3Er 43:11
Потому, что они первые
имели силы этих дев, и те и другие не отступали~--- ни духовные силы от людей,
ни люди от сил; но эти силы пребывали с ними до дня упокоения;
\vs 3Er 43:12
если бы они не имели этих
сил духовных, то не годились бы для здания башни.

\vs 3Er 44:1
И снова я попросил: еще,
господин, объясни мне, почему эти камни были извлечены со дна и положены в
здание башни, тогда как они уже имели этих духов?
\vs 3Er 44:2
Им было необходимо пройти
через воду, чтобы оживотвориться; не могли они иначе войти в царство Божие,
как отринув мертвость прежней жизни.
\vs 3Er 44:3
Посему эти почившие
получили печать Сына Божьего и вошли в царство Божье.
\vs 3Er 44:4
Ибо человек до принятия
имени Сына Божьего мертв; но как скоро примет эту печать, он отлагает
мертвость и воспринимает жизнь.
\vs 3Er 44:5
Печать же эта есть вода, в
неё сходят люди мертвыми, а восходят из неё живыми; посему и им проповедана
была эта печать, и они воспользовались ею, чтобы войти в царство Божье.
\vs 3Er 44:6
Почему же,~--- спросил я,~--- вместе с ними взяты со дна и те сорок камней, уже имеющие эту печать?
\vs 3Er 44:7
Потому, что эти апостолы и
учители, проповедовавшие имя Сына Божьего, скончавшись с верою в Него и с
силою, проповедовали Его и прежде почившим, и сами дали им эту печать; они
вместе с ними нисходили в воду и с ними опять восходили.
\vs 3Er 44:8
Но они нисходили живыми, а
те, которые почили прежде, нисходили мертвыми, а вышли живыми; через апостолов
они восприняли жизнь и познали имя Сына Божьего и потому взяты вместе с ними и
положены в здание башни;
\vs 3Er 44:9
они употреблены в строение
не обсеченные, потому что они скончались в праведности и чистоте, только не
имели этой печати. Вот тебе объяснение этих камней.

\vs 3Er 45:1
Теперь, господин,~--- сказал
я,~--- объясни мне значение тех гор: почему они такие разные?
\vs 3Er 45:2
Слушай. Эти двенадцать
гор, которые ты видишь, означают двенадцать племен, населяющих весь мир; среди
них был проповедан Сын Божий через апостолов.
\vs 3Er 45:3
Почему же они различны и
вид имеют неодинаковый?
\vs 3Er 45:4
Эти двенадцать племен,
населяющие весь мир, суть двенадцать народов; и как различны, ты видел, горы,
так различны мысль и внутреннее настроение этих народов. Я поясню тебе смысл
каждого из них.
\vs 3Er 45:5
Прежде всего, господин,
скажи мне вот что: если эти горы так различны, то каким образом камни с них,
будучи положены в здание башни, сделались одноцветными и блестящими, как и
камни, поднятые со дна?
\vs 3Er 45:6
Потому, что все народы под
небом, услышав проповедь, уверовали и нареклись одним именем Сына Божьего и,
приняв печать Его, все получили один дух и один разум, и стала у них одна вера
и одна любовь, и вместе с именем Его они облеклись духовными силами дев.
\vs 3Er 45:7
Потому-то здание башни
сделалось одноцветным и сияющим, подобно солнцу.
\vs 3Er 45:8
Но после того как они
сошлись воедино и стали одним телом, некоторые из них осквернили себя и были
извергнуты из рода праведных; опять возвратились к прежнему состоянию и даже
сделались хуже.

\vs 3Er 46:1
Каким образом, господин,~--- говорю я,~--- они, познав Господа, сделались худшими?
\vs 3Er 46:2
Если не познавший Господа,
сказал он,~--- сделает зло, он подлежит наказанию за свою неправду. Но кто
познал Господа, тот уже должен удерживаться от зла и делать добро.
\vs 3Er 46:3
И если тот, который должен
совершать добро, вместо этого причиняет зло, то не более ли он преступен,
нежели не ведающий Бога?
\vs 3Er 46:4
Посему хотя и не познавшие
Бога и делающие зло обречены на смерть; но те, которые познали Господа и
видели дивные дела Его, делая зло, будут вдвойне наказаны и умрут навеки.
\vs 3Er 46:5
Так очистится Церковь
Божья.
\vs 3Er 46:6
Ты видел: забракованные
камни были выброшены из башни и преданы злым духам, и башня так очистилась,
что казалась вся высеченною из одного камня;
\vs 3Er 46:7
такою будет и Церковь
Божья, когда она очистится и будут изринуты из неё злые, лицемеры,
богохульники, двоедушные и все виновные в различной неправде;
\vs 3Er 46:8
она будет единое тело,
единый дух, единый разум, единая вера и единая любовь, и тогда Сын Божий будет
торжествовать между ними и радоваться, приняв Свой народ чистым.
\vs 3Er 46:9
Господин,~--- сказал я,~--- всё это величественно и славно. Теперь объясни мне значение каждой из гор,
чтобы всякая душа, уповающая на Господа, услышав это, прославляла великое,
дивное и славное имя Его.
\vs 3Er 46:10
Слушай и об этих
различных горах, то есть двенадцати народах.

\vs 3Er 47:1
Первая гора черная
означает верующих отступников, хулителей Господа и предателей рабов Божиих: им
назначена смерть и нет покаяния, и они черны потому, что род их беззаконен.
\vs 3Er 47:2
Вторая гора голая~--- это
верующие лицемеры и учители неправды; они весьма близки к первым и не имеют
плода правды.
\vs 3Er 47:3
Ибо как гора их пуста и
бесплодна, так и эти люди, хотя и имеют имя, но не имеют веры и нет в них
никакого плода истины.
\vs 3Er 47:4
Им, впрочем, есть
покаяние, если только немедленно покаются; а если замедлят, то и им будет
смерть вместе с первыми.
\vs 3Er 47:5
Почему же, господин,
последним есть доступ к покаянию, а первым нет? Ведь дела их почти те же
самые.
\vs 3Er 47:6
Потому для них есть
покаяние, что они не хулили Господа своего и не были предателями рабов Божьих;
но, стремясь к корысти, они обольщали людей, и каждый потворствовал похотям
грешных;
\vs 3Er 47:7
за это дело они понесут
наказание, но так как не были хулителями и предателями Господа, есть у них
возможность покаяния.

\vs 3Er 48:1
Третья гора,~--- продолжал
пастырь,~--- покрытая терниями и сорняками, знаменует верующих, из которых одни
богаты, а другие занялись множеством дел,
\vs 3Er 48:2
ибо сорняки означают
богатых, а тернии~--- тех, которые предались многим попечениям.
\vs 3Er 48:3
Таковые не имеют общения с
рабами Божьими, но удаляются от них, увлекаемые делами своими.
\vs 3Er 48:4
А богатые с трудом
вступают в общение с рабами Божьими, опасаясь, чтобы у них не попросили
чего-либо.
\vs 3Er 48:5
И как разутыми ногами
трудно ходить по колючкам, так и людям такого рода трудно попасть в царство
Божье.
\vs 3Er 48:6
Но и им есть покаяние,
только они должны немедленно обратиться к нему, чтобы упущенное ими прежде
вознаградить в остающиеся дни и делать добро.
\vs 3Er 48:7
Покаявшись и творя добрые
дела, они будут жить с Богом; если же пребудут в своих делах, то будут преданы
тем женщинам, которые лишат их жизни.

\vs 3Er 49:1
И далее повествовал
пастырь: четвертая гора, на которой очень много растений, в верхней части
зеленых, а к корням сухих и даже увядших от солнечного зноя, означает
верующих, которые колеблются или же имеют Господа только на устах, но не в
сердце.
\vs 3Er 49:2
Потому они в основании
сухи и лишены силы, и только слова их живы, а дела мертвы~--- и сами они ни
мертвы, ни живы.
\vs 3Er 49:3
Подобным образом и
колеблющиеся~--- ни зелены, ни сухи, то есть ни живы и ни мертвы.
\vs 3Er 49:4
Как те растения засохли,
едва лишь показалось солнце, так точно и двоедушные, услышав о гонении, по
малодушию поклоняются идолам и стыдятся имени своего Господа;
\vs 3Er 49:5
такие люди ни живы и ни
мертвы; но и они могут жить, если скоро покаются; если же не покаются, то
будут преданы тем женщинам, которые лишат их жизни.

\vs 3Er 50:1
Пятая гора скалистая, но
поросшая зелеными травами, означает верующих таких, которые хоть и веруют, но
мало учатся, дерзки и самодовольны, желают казаться всезнающими, но ничего не
знают.
\vs 3Er 50:2
За эту дерзость разум
отступил от них и вошло в них тщеславное безрассудство.
\vs 3Er 50:3
Они выдают себя за умных
и, будучи глупы, желают быть учителями.
\vs 3Er 50:4
За это высокомудрие многие
из них уничижены, ибо великое беснование~--- дерзость и суетная самонадеянность.
\vs 3Er 50:5
Из них многие отвержены,
другие же, осознав свое заблуждение, покаялись и покорились имеющим разум.
\vs 3Er 50:6
Но и прочим, подобным им,
есть покаяние, потому что они не столько были злы, сколько неразумны и глупы.
\vs 3Er 50:7
Посему, если покаются, они
будут жить с Богом; если же не покаются, будут обитать с женщинами,
коварствующими над ними.

\vs 3Er 51:1
Шестая гора, с большими и
малыми расселинами и с сухими растениями в них, означает верующих.
\vs 3Er 51:2
Малые расселины~--- тех,
которые имели между собою распри и от взаимных пререканий притупилась их вера;
\vs 3Er 51:3
многие из них покаялись,
то же сделают и прочие, услышав мои заповеди, потому что незначительны их
распри и легко они обратятся к покаянию.
\vs 3Er 51:4
Большие~--- это те, что
упорствуют в распрях, злопамятны и гневливы;
\vs 3Er 51:5
они отброшены от башни и
не годятся для здания, трудно им жить с Богом.
\vs 3Er 51:6
Если Бог и Господь наш,
владычествующий над всем Своим творением, не помнит зла на исповедующих грехи
свои, но умилостивляется, то пристало ли человеку, смертному и исполненному
грехов, упорно гневаться на другого, словно он может спасти или погубить его?
\vs 3Er 51:7
Я, ангел покаяния, убеждаю
вас, склонных к этому: одумайтесь и обратитесь к покаянию~--- и Господь уврачует
прежние ваши прегрешения, если очиститесь от этого бесовского зла, если же нет
будете преданы смерти.

\vs 3Er 52:1
Седьмая гора,~--- продолжал
пастырь свои объяснения,~--- на которой растительность зеленая, цветущая и
обильная, так что всякий скот и птицы небесные питаются ею, и она, будучи
срываема, растет еще лучше,
\vs 3Er 52:2
знаменует верующих,
которые просты и добры, не враждуют между собою, но всегда радуются за рабов
Божьих, исполнены духом дев, милосердны к любому человеку и плодами от трудов
своих делятся со всяким немедленно и без колебания.
\vs 3Er 52:3
Посему Господь, видя
простоту и доброту их, благопоспешал трудам рук их и даровал успех во всяком
деле.
\vs 3Er 52:4
Я, ангел покаяния,
убеждаю вас пребывать в таком расположении, и семя ваше не искоренится вовек.
\vs 3Er 52:5
Господь одобрил вас и
вписал в наше число, и всё семя ваше будет обитать с Сыном Божьим, потому что
вы~--- от Духа Его.

\vs 3Er 53:1
Восьмая гора, со многими
источниками, которые поили всякую тварь Божью, означает апостолов и учителей,
\vs 3Er 53:2
которые проповедовали по
всему миру, и свято и чисто учили слову Господню, и не склонялись к дурным
желаниям, но постоянно пребывали в правде и истине, приняв Святого Духа.
\vs 3Er 53:3
Посему они обитают с
ангелами.

\vs 3Er 54:1
Камни с пятнами на девятой
горе, пустынной и населенной вредоносными змеями, означают дьяконов,
\vs 3Er 54:2
которые плохо проходили
служение, расхищая блага вдов и сирот и обогащаясь от своего служения.
\vs 3Er 54:3
Если останутся в своем
пороке, то они мертвы и нет в них никакой надежды жизни; если же обратятся и
будут непорочно исполнять свое служение, то смогут жить.
\vs 3Er 54:4
А камни шероховатые
означают тех, которые отреклись и не обратились к Господу, одичали и
уподобились пустыне, не общаются с рабами Божьими, но, живя одиноко, губят
свои души.
\vs 3Er 54:5
Как виноградная лоза,
оставленная без всякого ухода, пропадает, заглушается травами, со временем
делается дикой и бесполезной для хозяина, так и эти люди, отчаявшись в себе
самих, одичали и стали бесполезны для своего Господа.
\vs 3Er 54:6
Для них возможно покаяние,
если отреклись они не от сердца; если же кто сделал это от сердца, не знаю,
сможет ли он возродиться.
\vs 3Er 54:7
Я не о настоящих днях
говорю, чтобы отрекшийся мог покаяться;
\vs 3Er 54:8
невозможно обрести
спасение тем, кто намерен отречься от своего Господа; но покаяние дается
тем, кто отрекся в прошлом.
\vs 3Er 54:9
Итак, кто намерен
покаяться, пусть сделает это немедленно, прежде чем закончится строительство
башни.
\vs 3Er 54:10
Если же кто не поспешит,
то будет предан смерти теми женщинами.
\vs 3Er 54:11
Камни короткие означают
людей коварных и клеветников: они подобны змеям, которых ты видел на девятой
горе.
\vs 3Er 54:12
Ибо как яд змеи
смертоносен для человека, так и слова таких людей губительны для других.
Несовершенны они в своей вере по причине их образа действий.
\vs 3Er 54:13
Впрочем, некоторые из
них покаялись и спаслись.
\vs 3Er 54:14
Равным образом прочие
таковые получат спасение, если покаются; если же не покаются, то погибнут от
тех женщин, силою и властью которых они обладают.

\vs 3Er 55:1
Деревья на десятой горе,
которые служат кровом для скота, означают епископов и верующих страннолюбцев,
\vs 3Er 55:2
которые всегда непритворно
и радушно принимали в домах своих рабов Божьих;
\vs 3Er 55:3
епископов, которые
беспрестанно покровительствовали бедным и вдовствующим и жили всегда
непорочно.
\vs 3Er 55:4
Таким людям
покровительствует сам Господь: они почтенны у Бога, и им место среди ангелов,
если пребудут до конца в служении Господу.

\vs 3Er 56:1
Одиннадцатая гора, деревья
на которой обильны разными плодами, означает верующих, пострадавших за имя
Сына Божьего, пострадавших с любовью и от всего сердца своего.
\vs 3Er 56:2
Я спросил: почему же,
господин, все деревья имеют плоды, но на некоторых плоды менее приятны?
\vs 3Er 56:3
И это объясню тебе.
Пострадавшие за имя Господне почтенны у Бога, и всем им отпущены грехи, потому
что пострадали за имя Сына Божьего.
\vs 3Er 56:4
Но некоторые, будучи
допущены ко властям и спрошены, не отреклись от Господа, но с готовностью
пострадали,~--- они почтенны у Бога, и плод их превосходнее.
\vs 3Er 56:5
А некоторые, охваченные
страхами и смущением, колебание имели в своем сердце, проповедать ли Бога или
отречься, и пострадали~--- их плоды хуже, потому что в сердце их был лукавый
помысел раба отречься от своего господина.
\vs 3Er 56:6
Смотрите вы, помышляющие
так, чтобы эта мысль не утвердилась в ваших сердцах и чтобы не умереть вам для
Бога.
\vs 3Er 56:7
А вы, страдающие за имя
Божье, должны прославлять Господа, что удостоил вас носить Его имя, ибо
исцелятся все грехи ваши.
\vs 3Er 56:8
Ужели вы не почитаете себя
более других блаженными? Вы думаете, что совершили великое дело, если кто из
вас пострадал?
\vs 3Er 56:9
Но Господь дарует вам
жизнь, и вы об этом не помышляете. Вас отягощали грехи ваши, и если бы не
пострадали вы за имя Господне, то вы умерли бы для Бога за грехи свои.
\vs 3Er 56:10
Это я говорю вам,
сомневающимся, исповедать ли Бога или отречься.
\vs 3Er 56:11
Исповедуйте, что вы
имеете Господа, и, не отрекаясь, отдавайте себя в оковы.
\vs 3Er 56:12
Если все народы
наказывают рабов за отречение от своего хозяина, то что, думаете вы, сделает с
вами Господь, имеющий власть над всеми?
\vs 3Er 56:13
Итак, удалите из сердец
своих такие помыслы, чтобы вовеки жить вам с Богом.

\vs 3Er 57:1
Двенадцатая гора, белая,
означает верующих, подобных младенцам, коим не всходила на сердце никакая
злоба, которые не знают, что такое лукавство, но всегда пребывают в простоте.
\vs 3Er 57:2
Такие люди, без сомнения,
будут обитать в царстве Божьем, потому что они ни в одном деле не преступили
заповедей Божьих, но с простотою пребывали в том же расположении все дни своей
жизни.
\vs 3Er 57:3
Те, которые останутся как
младенцы, не имеющие злобы, будут почетнее всех, о которых сказано выше: все
младенцы славны у Господа и почитаются у Него первыми.
\vs 3Er 57:4
Итак, блаженны вы, которые
удалили от себя лукавство и облеклись в невинность, потому что вы первые
будете жить с Богом.
\vs 3Er 57:5
После того как пастырь
истолковал мне все горы, я сказал ему:
\vs 3Er 57:6
<<Господин, теперь поведай о тех камнях, которые принесены с поля и заложены в
башню вместо вынутых, а также о тех круглых камнях, которые вошли в здание
башни, и о тех, которые доселе остаются круглыми.>>

\vs 3Er 58:1
Слушай и об этом. Камни,
которые были принесены с поля и заложены в здание башни
вместо отвергнутых,~--- это суть отроги белой горы.
\vs 3Er 58:2
Поскольку верующие с этой
горы оказались невинными, то господин башни поместил их в здание башни, ибо
знал, что, войдя в здание, они останутся белыми и ни один из них не почернеет.
\vs 3Er 58:3
А если бы он приказал
положить в здание башни камни и с прочих гор, то нужно было бы ему снова
осматривать эту башню и очищать.
\vs 3Er 58:4
Эти белые камни суть
новообращенные, которые уверовали и уверуют, ибо они веруют от сердца. Блажен
этот род, потому что невинен.
\vs 3Er 58:5
Слушай теперь и о круглых
блестящих камнях. И они все от белой горы.
\vs 3Er 58:6
Круглыми же они оказались
потому, что богатство немного омрачило их, но они не отступили от Бога и ни
единое слово хулы не сошло с языка их~--- только правда, добродетель и истина.
\vs 3Er 58:7
Посему Господь, зная душу
их и то, что они родились и остаются добрыми, повелел отсечь их богатства, но
не совсем отнять их, чтобы из оставшегося они могли делать добро и жить с
Богом, ибо и они из доброго рода.
\vs 3Er 58:8
Посему их несколько
отесали и положили в здание башни.

\vs 3Er 59:1
А прочие камни, которые
остались круглыми и были негодны для здания, еще не получили печати и
возвращены на свое место, ибо оказались слишком круглыми.
\vs 3Er 59:2
Должно лишить их благ
настоящего века и суетного богатства~--- и тогда они будут годны в царстве
Божьем.
\vs 3Er 59:3
Они должны войти в царство
Божье, ибо Господь благословил этот род, и из него никто не погибнет;
\vs 3Er 59:4
может быть, кто из них,
искушенный злым дьяволом, и согрешит в чем-либо, но скоро вновь обратится к
Господу своему.
\vs 3Er 59:5
Я, ангел покаяния, почитаю
счастливыми вас, которые невинны, как дети, потому что ваша участь благая и
почтенная перед Богом.
\vs 3Er 59:6
И всем, которые приняли
печать Сына Божьего, говорю: имейте простоту, не помните обид, не пребывайте в
злобе, да не будет в душе кого-либо из вас горечи злопамятства;
\vs 3Er 59:7
врачуйте и удаляйте от
себя злые раздоры, чтобы господин стада пришел и возрадовался, найдя целыми
овец своих.
\vs 3Er 59:8
Если же какая овца будет
потеряна пастырями или самих пастырей господин найдет дурными, что ответят
ему? Ужели скажут, что они измучены стадом?
\vs 3Er 59:9
Не поверят им, ибо не
может пастырь потерпеть что от овец и еще более будет наказан за ложь свою.
\vs 3Er 59:10
И я~--- пастырь и должен
дать Всевышнему отчет за вас.

\vs 3Er 60:1
Итак, позаботьтесь о себе,
пока еще строится башня.
\vs 3Er 60:2
Господь обитает в людях,
любящих мир, ибо Он Сам любит мир и далек от сварливых и развращенных злобою.
\vs 3Er 60:3
Возвратите Ему дух целым,
какой приняли от Него.
\vs 3Er 60:4
Ибо если ты отдашь
валяльщику одежду целую, то желаешь и получить ее обратно целою, а если
валяльщик возвратит тебе её изодранною, возьмешь ли ты ее?
\vs 3Er 60:5
Не прогневаешься ли и не
будешь ли бранить его, говоря: я дал тебе одежду целою, а ты изодрал её, и
теперь она из-за дыр, которые ты на ней сделал, стала непригодна. Разве не так
будешь пенять ты валяльщику и скорбеть о своей одежде?
\vs 3Er 60:6
Так что же, думаешь,
сделает тебе Господь, который вручил тебе дух чистый, а ты повредил его и
привел в негодность, так что он никак не может служить Господу?
\vs 3Er 60:7
И за это Господь предаст
тебя смерти.
\vs 3Er 60:8
Так накажет Он всех тех,
которых найдет упорно помнящими обиды.
\vs 3Er 60:9
Не пренебрегайте Его
милосердием, но лучше прославляйте Его за то, что Он, не в пример вам, столь
терпим к вашим преступлениям.
\vs 3Er 60:10
Покайтесь, ибо это
полезно для вас.

\vs 3Er 61:1
Всё, что описано выше,
показал я, пастырь, ангел покаяния, ради покаяния.
\vs 3Er 61:2
Я всегда говорил и теперь
говорю рабам Божьим: если поверите и послушаетесь слов моих, будете поступать
по ним и исправите пути ваши, то сможете спастись.
\vs 3Er 61:3
Если же будете
упорствовать в лукавстве и злопамятстве, ни один из таких грешников не будет
жить с Богом: ибо всё это мною наперед сказано вам.
\vs 3Er 61:4
И после этих слов пастырь
спросил меня: всё ли ты проведал у меня?
\vs 3Er 61:5
Я ответил, что всё.
\vs 3Er 61:6
Почему же ты не спросил
меня,~--- сказал тогда он,~--- о камнях, положенных в здание, вид которых мы
исправили?
\vs 3Er 61:7
Забыл, господин.
\vs 3Er 61:8
Выслушай и о них. Это те,
до которых дошли теперь мои заповеди, и они от всего сердца покаялись, и
Господь, видя, что покаяние их доброе и чистое и что пребудут они в нем,
повелел загладить прежние грехи их.
\vs 3Er 61:9
Так грехи их изглажены,
чтобы после они не были видны.

\chhdr{Подобие 10-е.}
\vs 3Er 62:1
После того как я написал эту книгу, тот ангел, который
вручил меня пастырю, пришел в дом мой и сел на ложе, а справа от него стал
пастырь.
\vs 3Er 62:2
Позвал ангел меня и сказал: я поручил тебя и дом твой этому пастырю под его
покровительство.
\vs 3Er 62:3
Так, господин,~--- подтвердил я.
\vs 3Er 62:4
Итак, если хочешь быть
защищен от всякого бедствия и злополучия, иметь успех во всяком благом деле и
слове и во всякой истинной добродетели, то поступай по тем заповедям, которые
он дал тебе, и будешь господствовать над всякою неправдою.
\vs 3Er 62:5
Ибо, если будешь соблюдать
эти заповеди, покорятся тебе всякое пожелание и сладость этого века и будет
сопровождать тебя удача во всяком добром деле.
\vs 3Er 62:6
Почитай его достоинство и
святость и скажи всем, что он в великой чести и славе у Бога и имеет великую
власть и силу.
\vs 3Er 62:7
Ему одному во всей
вселенной вручена власть покаяния. Разве он не кажется тебе могущественным?
\vs 3Er 62:8
Но вы пренебрегаете его
достоинством и властью, которую он имеет над вами.

\vs 3Er 63:1
Я сказал: спроси,
господин, самого его, сделал ли я что дурное или оскорбил его чем-нибудь за то
время, что он находится в доме моем.
\vs 3Er 63:2
И я знаю, что ты не сделал
и не сделаешь ничего дурного, потому я и говорю это тебе, чтобы ты всегда был
таков. Ибо он предо мною хорошо засвидетельствовал о тебе.
\vs 3Er 63:3
Скажи это и прочим, чтобы
и они, если покаялись или намерены покаяться, чувствовали то же, что и ты,~--- и
он засвидетельствует доброе о них предо мною, а я пред Господом.
\vs 3Er 63:4
Господин,~--- ответил я,~--- я
всякому человеку возвещу великие дела Божьи и надеюсь, что все прежде
согрешившие, услышав это, покаются, чтобы получить жизнь.
\vs 3Er 63:5
Итак, совершай неуклонно
это служение и впредь.
\vs 3Er 63:6
Кто исполнит заповеди Его,
будет иметь жизнь и великую честь у Господа.
\vs 3Er 63:7
А кто не соблюдет Его
заповедей, бежит от своей жизни; кто не чтит Его, теряет свою честь у Господа.
\vs 3Er 63:8
Презирающие Его и не
соблюдающие Его заповедей обрекают себя на смерть, и любой из них виновен в
крови своей.
\vs 3Er 63:9
Тебе же наказываю
соблюдать эти заповеди~--- и получишь искупление всех грехов своих.

\vs 3Er 64:1
Я послал к тебе также и
этих дев, чтобы они жили с тобою, ибо я видел, что они очень ласковы к тебе.
\vs 3Er 64:2
Они станут тебе
помощниками, чтобы усерднее ты мог соблюдать заповеди, ибо без этих дев
невозможно соблюсти заповеди.
\vs 3Er 64:3
Я вижу, что им приятно
быть с тобою, и я прикажу, чтобы они вовсе не выходили из твоего дома.
\vs 3Er 64:4
Ты только очисти дом свой:
в чистом доме они живут охотно.
\vs 3Er 64:5
Они сами чисты, непорочны
и рачительны и весьма угодны Господу.
\vs 3Er 64:6
Итак, если будет чист дом
твой, они останутся с тобою. Если же чем осквернится дом твой, они совсем
удалятся из него, ибо не любят никакой нечистоты.
\vs 3Er 64:7
Я надеюсь угодить им, так
что они охотно и безотлучно будут жить в доме моем. И как тот, которому ты
передал меня, ни в чем на меня не жалуется, так и они не будут жаловаться.
\vs 3Er 64:8
Ангел сказал пастырю: я
вижу, что раб Божий хочет соблюдать эти заповеди и поместить дев в чистом
жилище.
\vs 3Er 64:9
Произнеся это, он опять
поручил меня пастырю и обратился к девам: так как я вижу, что вам приятно жить
в этом доме, то вручаю вам Ерму и семью его с тем, чтобы вы не покидали этого
дома.
\vs 3Er 64:10
И они с удовольствием
вняли этим словам.

\vs 3Er 65:1
Потом он сказал мне:
мужественно проходи это служение и поведай всякому человеку величие Божье~--- и
будешь иметь благодать в своем служении.
\vs 3Er 65:2
Всякий, кто исполнит эти
заповеди, будет жить и будет блажен; а кто пренебрежет ими, не будет жить и
будет несчастлив в своей жизни.
\vs 3Er 65:3
Скажи всем, чтобы не
переставали, кто может, благотворить, ибо благотворение полезно им.
\vs 3Er 65:4
Говорю о том, что должно
всякого человека вызволять из бедствия. Неимущий в ежедневной жизни терпит
великое мучение и скорбь.
\vs 3Er 65:5
Кто вырвет из нужды душу
такого человека, тот обретет великую радость, ибо терпящий подобное бедствие
испытывает страдания сродни заключенному в узах.
\vs 3Er 65:6
Многие, не вынеся
бедственного положения, причиняют себе смерть. Посему кто знает о бедствии
такого человека и не избавляет его, тот совершает великий грех и принимает
вину за кровь его.
\vs 3Er 65:7
Итак, благотворите,
сколько кто получил от Господа. Не медлите, пока не окончилось строительство
башни, ибо ради вас приостановлено оно.
\vs 3Er 65:8
Если не поспешите
исправиться, будет достроена башня и вы не попадете в неё.
\vs 3Er 65:9
После этих слов он встал с
ложа и, взяв пастыря и дев, удалился, но обещал мне, что пастыря и дев
отпустит обратно в дом мой.

\end{document}
